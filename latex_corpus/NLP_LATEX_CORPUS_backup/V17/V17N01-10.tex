    \documentclass[english]{jnlp_1.4}

\usepackage{jnlpbbl_1.2}
\usepackage[dvips]{graphicx}
\usepackage{hangcaption_jnlp}
\usepackage{udline}
\setulminsep{1.2ex}{0.2ex}
\let\underline


\newcommand{\wire}[1]{}
    \usepackage{lingmacros_crlf}
    \input{macros_re}


\Volume{17}
\Number{1}
\Month{January}
\Year{2010}

\received{2009}{7}{29}
\revised{2009}{9}{16}
\accepted{2009}{9}{24}

\setcounter{page}{221}


\etitle{Resolving Direct and Indirect Anaphora for Japanese Definite Noun Phrases}
\eauthor{Naoya Inoue\affiref{Author_1} \and Ryu Iida\affiref{Author_2}
	\and Kentaro Inui\affiref{Author_1} \and Yuji Matsumoto\affiref{Author_1}} 
\eabstract{
  An anaphoric relation can be either direct or indirect. In some
  cases, the antecedent being referred to lies outside of the
  discourse its anaphor belongs to. Therefore, an anaphora resolution
  model needs to consider the following two decisions in parallel:
  \emph{antecedent selection}--selecting the antecedent itself, and
  \emph{anaphora type classification}--classifying an anaphor into
  direct anaphora, indirect anaphora or exophora. However, there are
  non-trivial issues for taking these decisions into account in
  anaphora resolution models since the anaphora type classification
  has received little attention in the literature. In this paper, we
  address three non-trivial issues: (i) how the antecedent selection
  model should be designed, \wire{(ii) what information helps with
    anaphora type classification, (iii) how the antecedent selection
    and anaphora type classification should be carried out}, taking
  Japanese as our target language. Our findings are: \wire{first, an
    antecedent selection model should be trained separately for each
    anaphora type using the information useful for identifying its
    antecedent.} Second, the best candidate antecedent selected by an
  antecedent selection model provides contextual information useful
  for anaphora type classification. Finally, the antecedent selection
  should be carried out before anaphora type classification.
}
\ekeywords{Anaphora resolution, antecedent selection, anaphora type classification, direct anaphora, indirect anaphora, exophora}

\headauthor{Inoue et al.}
\headtitle{Anaphora Resolution for Japanese Definite Noun Phrases}

\affilabel{Author_1}{}{Graduate School of Information Science, Nara Institute of Science and Technology}
\affilabel{Author_2}{}{Department of Computer Science, Tokyo Institute of Technology}


\begin{document}

\maketitle

\section{Introduction}

Anaphora resolution has been studied intensively in recent years
because of its significance in many natural language processing (NLP)
applications such as information extraction and machine
translation. In nominal anaphora, an anaphor (typically a definite
noun phrase) and its antecedent in the preceding discourse holds
either a \emph{direct} anaphoric relation (e.g. coreference) or an
\emph{indirect} relation (e.g., bridging reference
~\shortcite{Clark77bridging}). \emph{Direct anaphoric relation} refers
to a link in which an anaphor and an antecedent are in such a relation as
\emph{synonymy} and \emph{hypernymy/hyponymy}, as in
\emph{house}--\emph{building}. \emph{Indirect anaphoric relation}, on
the other hand, refers to a link in which an anaphor and an antecedent
have such relations as \emph{meronymy/holonymy} and
\emph{attribute/value} as in \emph{ticket}--\emph{price}. For the
other case, a noun phrase occasionally holds an \emph{exophoric} relation
to an antecedent that lies outside the discourse that the noun
phrase presents. Recent studies in anaphora resolution have 
proposed the resolution frameworks for each direct and indirect
anaphoric case
respectively~\shortcite{soon01_coref,iida05_coref,poesio04_lbd}, placing
the main focus on the \emph{direct} anaphoric case. The identification of
exophoric relations, in contrast, has been paid little attention in
the literature. Anaphoricity determination, which is the task of
determining whether an anaphor has an antecedent in the preceding
discourse or not, is related to identifying exophoric relations, but
the methods for anaphoricity determination are not designed to
explicitly capture exophoric relations because they are tuned for
finding NP coreference chains in discourse.

However, for the practical use of anaphora resolution, we need to
solve the following non-trivial problem: in a real text, anaphors such
as definite noun phrases can occur as either direct anaphoric,
indirect anaphoric or exophoric relations, which is not easy to 
disambiguate from its surface expression. That is, in anaphora
resolution, it is necessary to judge what kind of anaphoric relation
is used to tie an anaphor and its (potential) antecedent (henceforth,
we call this task \emph{anaphora type classification}). In fact, our
corpus analysis (detailed in Section \ref{sec:dataset}) shows that
more than 50\% of noun phrases modified by a definiteness modifier
have non-trivial ambiguity in terms of the anaphora types that have
to be classified for each given text. Given these issues, we decompose
the task of nominal anaphora resolution as a combination of two
distinct but arguably interdependent subtasks.
\begin{itemize}
\item \emph{Antecedent selection}: the task of identifying the
  antecedent of a given anaphor, and
\item \emph{Anaphora type classification}: the task of judging what
  kind of \emph{anaphora type} is used for a given anaphor, i.e.,
  classifying a given anaphor into \emph{direct anaphoric},
  \emph{indirect anaphoric} or \emph{exophoric}.
\end{itemize}
Given this task decomposition, three unexplored issues immediately come up:
\begin{description}
\item[ Issue 1.]Whether the model for antecedent selection should be designed
  and trained separately for direct anaphora and indirect anaphora or
  whether it can be trained as a single common model;
\item[ Issue 2.]\wire{What contextual information is useful for determining each
  anaphora type;}
\item[ Issue 3.]\wire{How the two subtasks can be best combined (e.g. which subtask
  should be carried out first).}
\end{description}
In this paper, we explore these issues taking Japanese as our target
language. Specifically, we focus on anaphora resolution for noun
phrases modified by a definiteness modifier, as detailed in
the next section.

This paper is organized as follows. In the next section, we describe
our motivation for this work more specifically. In Section 3, we
review previous work of antecedent selection and anaphora type
classification. In Section 4, we give a detailed explanation of our
investigations. In Section 5, the dataset for our experiments is
described. We then show the experimental setup and results of our
investigations and discussion in Section 6. \wire{Finally, the
  conclusion is presented along with future work.}



\section{Motivation for our approach} 
\label{sec:motivation}

As mentioned, an anaphor can hold a direct or indirect relation
with its antecedent. Occasionally, an anaphor refers to an antecedent
that is not in the same discourse. The terms \emph{direct anaphora} and
\emph{indirect anaphora} have been used to denote some different
anaphoric phenomena in previous works, e.g. \emph{direct anaphora} in
~\shortcite{vieira00_defdesc} indicates only the reference that an
anaphor and its antecedent have identical head words, whereas
\emph{direct anaphora} in~\shortcite{mitkov00_ana} includes a
synonymous or generalization/specialization link of an anaphor and its
antecedent. As a result, we redefine the following three \emph{anaphora types}
to denote the use of anaphoric expressions in our classification task:

\begin{itemize}
\item \emph{direct anaphora}: An anaphor refers to its antecedent
  directly. In example (\ref{ex:direct_anaphora}), \emph{The CD}
  refers to \emph{the new album} directly.
  \EX{ex:direct_anaphora}{ \emph{\underline{Her new album} was
      released yesterday.  I want to get \underline{the CD} as soon as
      possible. }} 

\item \emph{indirect anaphora}: An anaphor has an antecedent related
  with the anaphor rather than referred to, as in example
  (\ref{ex:indirect_anaphora}).
  \EX{ex:indirect_anaphora}{ \emph{ The artist announced
      \underline{\mbox{her new song}}. I want to get \underline{the CD} as
      soon as possible. }}
  \emph{The CD} refers to \emph{her new song} indirectly. The
  discourse entity that directly corresponds to \emph{the CD} is not
  in the preceding sentence; \wire{ instead \emph{her new song} is
  considered as an antecedent of \emph{the CD} because it is
  associated with \emph{the CD}. }

\item \emph{exophora}: An anaphor that has no antecedent in a text is
  regarded as exophoric. An exophoric expression is typically used in
 newspaper articles; for instance, \emph{the day} refers to the
  date of the post.
\end{itemize}
For our target language, Japanese, noun phrases (NP) behave similarly
to those in English; that is, a definite NP may bear a direct
anaphoric relation but may also bear an indirect anaphoric relation to
its antecedent as shown in examples (\ref{ex:jp_direct}) and
(\ref{ex:jp_indirect}).

\EX{ex:jp_direct}{ \underline{新しいミニバン}$_{(i')}$が発売され
  た。\underline{この新型車}$_{(i)}$は燃費が非常によい。\\
  \emph{\underline{A new minivan}$_{(i')}$ was released. \underline{The
      vehicle}$_{(i)}$ has good gas mileage. }}
\EX{ex:jp_indirect}{ 家
  具屋で\underline{机}$_{(i')}$を見た。\underline{そのデザ
    イン}$_{(i)}$ は素晴らしかった。\\
  \emph{I saw \underline{a desk}$_{(i')}$ in a furniture
    shop. \underline{\mbox{The design}}$_{(i)}$ was marvelous. }}
``この新型車(\textit{the vehicle})'' refers to ``新しいミニバン(\textit{a
  new minivan})'' directly in (\ref{ex:jp_direct}), while ``そのデザイ
ン(\textit{the design})'' refers to ``机(\textit{a desk})'' indirectly
in (\ref{ex:jp_indirect}). As seen from the above examples
\NUM{ex:direct_anaphora}, \NUM{ex:indirect_anaphora} and reported in
Section 1, the anaphora type can be different for a unique
expression. In other words, the anaphora type has to be disambiguated
taking its appearing context into account.

In Japanese, however, the problem can be even more complex because a
definite NP is not always marked by a definiteness modifier, such as
\emph{this}(この),\emph{the}(その),or \emph{that}(あの).For example,
bare NP 大統領 (\textit{president}) refers to \textit{韓国大統
  領} (\textit{Korean President}) in text \NUM{ex:japanese}.

\EX{ex:japanese}{ 今月4日、\underline{韓国大統領}$_{(i')}$が来日し
  た。\underline{大統領}$_{(i)}$は翌日の記者会見で新プランの詳細を語っ
  た。\\ \emph{\underline{Korean President}$_{(i')}$ visited Japan on the
    4th this month. \underline{\mbox{{\upshape (}The{\upshape )} president}}$_{(i)}$ talked about the
    details of his new plan at the news conference next day.}}
For this reason, it is sometimes difficult even for human annotators
to determine the definiteness of a bare NP. As the first step toward
complete understanding of Japanese NP anaphora, we focus on anaphora
resolution for NPs marked with either \emph{this NP}(この+NP),
\emph{the NP}(その+NP) or \emph{that NP}(あの+NP),which account for
a large proportion of occurrences of nominal anaphora in Japanese
texts.



\section{Related work}

In this section, we review previous research on anaphora resolution
for antecedent selection and anaphora type classification
respectively. In Section \ref{sec:related_as}, we look over how the
previous work had taken the approaches to antecedent selection for
direct anaphora and indirect anaphora. In Section
    \ref{sec:related_atc}, we discuss Vieira and Poesio's work
and Nakaiwa's work on anaphora type
classification.


\subsection{Antecedent selection} \label{sec:related_as}

A wide range of approaches to anaphora resolution has been proposed in
earlier work. There exist two main approaches: rule-based
approaches and machine learning-based approaches. In contrast to the
rule-based approaches such as
~\shortcite{Brennan,Lappin,Baldwin,Nakaiwa,Okumura,Mitkov}, empirical,
or machine learning-based approaches have been shown to be a cost-efficient
solution achieving performance that is comparable to the best
performing rule-based
systems \cite[etc.]{McCarthy,Ge:98,soon01_coref,1073102,Strube:03,iida05_coref,1075119}. Most
of these studies focus only on the coreference resolution
task, particularly in the context of evaluation-oriented research
programs such as Message Understanding Conference
(MUC)\footnote{http://www-nlpir.nist.gov/related\_projects/muc/index.html}
and Automatic Content Extraction
(ACE)\footnote{http://www.nist.gov/speech/tests/ace/}. To the
contrary, the methods for indirect anaphora resolution have been
relatively unexplored compared with direct anaphora. Those works are
implemented by rule-based
approaches~\shortcite[etc.]{poesio97_bd,murata99_xnoy,razvan03_asocweb}
and learning-based approaches~\shortcite{poesio04_lbd}, encoding the
centering theory~\shortcite{grosz95_centering}, lexical resources such
as WordNet~\shortcite{fellbaum98_wn} and web-based knowledge. In
comparison to direct anaphora, the resolution of indirect anaphora is
still a much more difficult task because it is required to capture the
wide variety of semantic relations (e.g. \emph{store}--\emph{the
  discount}, \emph{drilling}--\emph{the activity}). For example,
\shortcite{poesio02_aclex} proposed acquiring the lexical knowledge of
the meronymy relations for resolving bridging descriptions by using
syntactic patterns such as \emph{the NP of NP} and \emph{NP's NP}.

Recall that these works are based on the assumption that the system
knows that the given anaphor is direct anaphora or indirect anaphora,
which motivates us to explore the design of the antecedent selection
model.



\subsection{Anaphora type classification} \label{sec:related_atc}

As mentioned in Section 1, there has been little attention paid to the
issue of anaphora type classification. Exceptions can be seen in
~\shortcite{nakaiwa95_extra} and
~\shortcite{vieira00_defdesc}. Nakaiwa's work focuses on the
extra-sentential resolution of Japanese zero pronouns in machine
translation. They identify zero pronouns whose referent is the
extra-sentential element such as \emph{I, we} and \emph{you} by using
the semantic constraints such as modal expressions, verbal semantic
attributes. In their classification, the verbs depended on by pronouns
are important clues, whereas the contextual information is important in
anaphora type classification as mentioned in Section
\ref{sec:motivation}.

    Vieira and Poesio's work (2000) is motivated by corpus
study for the use of definite descriptions\footnote{Noun phrases with
  the definite article \emph{the}.}. Their system does not only find
an antecedent but classifies a given definite description into the
following three categories.

\begin{itemize}
\item \emph{direct anaphora}: subsequent-mention definite descriptions that
  refer to an antecedent with the same head noun as the description;
\item \emph{bridging descriptions}: definite descriptions that either
  (i) have an antecedent denoting the same discourse entity, but using
  a different head noun (as in \emph{house} ... \emph{building}), or
  (ii) are related by a relation other than identity to an entity
  already introduced in the discourse;
\item \emph{discourse-new}: first-mention definite descriptions that denote
  objects not related by shared associative knowledge to entities
  already introduced in the discourse.
\end{itemize}
Compared with our taxonomy, their definition of \emph{direct anaphora}
is restricted to the case where an anaphor and its antecedent have an
identical head. Therefore, the other cases (e.g. a pair of \emph{new
  album} and \emph{the CD}) are not regarded as direct anaphora but
such cases are classified into bridging descriptions. The definition
of \emph{discourse-new}, on the other hand, refers to the same notion
as our definition of \emph{exophora} except that the generic use of
the definite article \emph{the} as in \emph{play \underline{the}
  piano} is classified into \emph{discourse-new}. Note that Japanese
definiteness modifiers are not used in such a way.

In their work, the system chooses the correct anaphora type of a given
definite NP and if possible, finds its antecedent following a set
of hand-coded rules on the basis of the lexical and syntactic
features. The process can be regarded as four notable steps.

\begin{enumerate}
\renewcommand{\labelenumi}{}
\item The system applies some heuristics exploiting lexical and
  syntactic features based on \shortcite{hawkins78} to detect
  non-anaphoric cases (`unfamiliar use' or `larger situation use'
      in Hawkins's work) to an anaphor. If the test
  succeeds, it interprets the anaphor as discourse-new.
\item The system tries to find a same-head antecedent (i.e., an
  antecedent as direct anaphora) from a set of potential candidates
  appearing in the preceding discourse. If a suitable candidate is
  found, the system classifies an anaphor as \emph{direct anaphora}
  and returns the candidate as its antecedent.
\item The rules to recognize discourse-new, such as `pre-modifier use'
  and `proper noun use' (e.g. \emph{the United States}), are applied
  to an anaphor. If the test succeeds, the anaphor is classified as
  discourse-new.
\item The system tries to find an NP associated with an anaphor (which
  is called \emph{an anchor} in their work) in the preceding
  discourse. If such an NP is found, the anaphor is classified as
  bridging description and judges the NP as its anchor. Otherwise, the
  system does not output anymore.
\end{enumerate}

The heuristics to detect non-anaphoric or discourse-new anaphors are
based on the syntactic and lexical features, while the rules for
direct anaphora and bridging descriptions simply try to find an
antecedent. Consequently, their work can be said to focus on
detecting discourse-new descriptions compared to our work. They
reported their system achieved 57\% recall and 70\% precision in their
empirical evaluation.


Note that their system carries out anaphora type classification before
antecedent selection. However, it remains unexplored how to integrate
antecedent identification and anaphora type classification into
anaphora resolution, which is to be investigated as \emph{issue 2}
and \emph{issue 3}, which we addressed in Section 1.



\section{Model}

The purpose of our work is to investigate the three unexplored issues
shown in Section 1. First of all, we explain our learning-based
antecedent selection models and anaphora type classification models.


\subsection{Antecedent selection} \label{sec:antecedent_selection}

One issue to explore in antecedent selection is whether a single
common model should be built for both direct and indirect anaphora or
a separate model should be built for each. In this section, in order
to explore \emph{issue 1}, we design two different models for
selecting antecedents.

From the point of view in which we consider both anaphora types in parallel
in an antecedent identification, we can consider the following two
strategies.
\begin{itemize}
\item \emph{Single model}: Designing the model for the resolution of
  both direct and indirect anaphora. The information to capture an
  direct-anaphoric antecedent and indirect-anaphoric antecedent is
  jointly incorporated into a single common model. The model is trained
  with labeled examples of both direct and indirect anaphora.
\item \emph{Separate model}: Preparing two distinct models for each
  anaphora type separately; i.e., the selection model for direct
  anaphora and the model for indirect anaphora. Unlike the
  \emph{single model}, each model incorporates the information to
  capture an antecedent for each anaphora type separately. In the
  \emph{direct antecedent selection model}, only the information that
  captures a direct-anaphoric antecedent is used. In the \emph{indirect
    antecedent selection model}, on the other hand, only the
  information for the indirect-anaphoric antecedent is used. For the
  training, labeled examples of direct anaphora are only used in the
  \emph{direct antecedent selection model} and labeled examples of
  indirect anaphora are only used in the \emph{indirect antecedent
    selection model}.
\end{itemize}

The separate model approach is expected to be advantageous because
useful information for detecting direct-anaphoric antecedents is
different from one for indirect-anaphoric antecedents. For example,
synonymous relations between anaphor and antecedent are
important for selecting direct-anaphoric antecedents. In example
(\ref{ex:direct_anaphora}), an antecedent selection model has to know
that \textit{ CD} and \textit{album} are synonymous. For indirect
anaphora, on the other hand, it is required to recognize such semantic
relations as \emph{part-whole} and \emph{attribute-value} as shown in
example (\ref{ex:indirect_anaphora}), where it is essential that
\emph{CD} is semantically related with \emph{song}.

There are a variety of existing machine learning-based methods
designed for coreference resolution ranging from classification-based
models~\shortcite[etc.]{soon01_coref} and preference-based
models~\shortcite[etc.]{1073102} to comparison-based
models~\shortcite[etc.]{iida05_coref,1075119}. Among them, we adopt a
state-of-the-art model for coreference resolution in
Japanese~\shortcite{iida05_coref}, called the \emph{tournament model}
because it achieved the best performance for coreference resolution in
Japanese. The tournament model selects the best candidate antecedent
by conducting one-on-one games in a step-ladder tournament. More
specifically, the model conducts a tournament consisting of a series
of games in which candidate antecedents compete with each other and
selects the winner of the tournament as the best candidate
antecedent. The model is trained with instances, each created from an
antecedent paired with one other competing candidate.



\subsection{Anaphora type classification}
\label{sec:anaphora_type_classification}

In this section, we elaborate \emph{issue 2} and \emph{issue 3} for
anaphora type classification. An interesting question for this subtask
is whether anaphora type classification should be carried out before
antecedent selection or after because the available information
differs depending on the order of those two subtasks. To reflect this,
we consider two kinds of configurations: \emph{Classify-then-Select}
and \emph{Select-then-Classify} as follows. The difference between the
clues that each classifier uses is summarized in Table
\ref{tb:info_model}. The classifiers are trained in a supervised
fashion.

\begin{table}[b]
\caption{Summary of the information used in each anaphora type classifier}
\label{tb:info_model}
\input{11table01.txt}
\end{table}


\subsubsection{Classify-then-Select (C/S) model}

Given an anaphor, an anaphora type classifier first determines whether
\pagebreak
the anaphor bears either direct anaphora, indirect anaphora or
exophora. \wire{If the anaphora type is judged as direct anaphora,
  then the direct antecedent selection model is called. If the
  anaphora type is judged as indirect anaphora, on the other hand,
  then \wire{the indirect antecedent selection model is called}. There
  is no antecedent selection model called if exophora is selected.}

By altering the choice of information used in anaphora type
classification, the following two alternative models are available for
the Classify-then-Select configuration, each of which is illustrated
in Figure \ref{fig:cs_models}.
\begin{itemize}
\item \emph{a-Classify-then-Select (aC/S) Model}: Classify anaphora
  type of a given anaphor by using the anaphor and its properties
  before selecting the antecedent.
\item \emph{c-Classify-then-Select (cC/S) Model}: Classify anaphora
  type of a given anaphor by using the anaphor, its properties and the
  lexical and syntactic information from all potential antecedents
  before selecting the antecedent.
\end{itemize}
By comparing the cC/S model with the aC/S model, we can see the effect
of using contextual information in anaphora type classification. The
feature set used in the models is detailed in Section
\ref{sec:feature_set}.

\begin{figure}[b]
\begin{center}
  \includegraphics{17-1ia11f1.eps}
\end{center}
\vspace{0.5zw}
  \hangcaption{Classify-then-Select Anaphora Resolution Models. $ASM$
    denotes Antecedent Selection Model and $ATC$ denotes Anaphora
    Type Classifier. $ant_d$ and $ant_i$ denote an antecedent
    selected by Direct-ASM and Indirect-ASM respectively}
\label{fig:cs_models}
\end{figure}


\subsubsection{Select-then-Classify (S/C) model} 
\label{sec:sc_models}

Given an anaphor, an antecedent selection model first selects the most
likely antecedent and an anaphora type classifier determines the
anaphora type by utilizing information from both the anaphor and the
selected candidate antecedent(s). This way of configuration has an
advantage over the Classify-then-Select models in that it determines
the anaphora type of a given anaphor taking into account the
information of its most likely candidate antecedent. The candidate
antecedent selected in the first step can be expected to provide
contextual information useful for anaphora type classification: for
example, if \emph{her new song} is selected as the best candidate
antecedent in example \NUM{ex:atc_sample}, the anaphora type will be
easily identified by using the lexical knowledge that \emph{CD} is
the semantically related object of \emph{song}.

\EX{ex:atc_sample}{\emph{
The artist announced
      \underline{\mbox{her new song}}. I want to get \underline{the CD} as
      soon as possible. }
}

\begin{figure}[b]
\vspace{-0.5\baselineskip}
\begin{center}
\includegraphics{17-1ia11f2.eps}
\end{center}
\vspace{0.5zw}
 \hangcaption{Select-then-Classify Anaphora Resolution Models. $ASM$
   denotes Antecedent Selection Model and $ATC$ denotes Anaphora
   Type Classifier. $ant_s$, $ant_d$ and $ant_i$ denote an
   antecedent selected by Single-ASM, Direct-ASM and Indirect-ASM
   respectively}
 \label{fig:sc_models}
\end{figure}

Since we have two choices of antecedent selection models (i.e., the
single and separate models) as shown in
\ref{sec:antecedent_selection}, finally at least the following four
models are available for anaphora type classification, each of which
is illustrated in Figure \ref{fig:sc_models}.
\begin{itemize}
\item \emph{s-Select-then-Classify (sS/C) Model}: Select the best
  candidate antecedent with the single model and then classify the
  anaphora type.
\item \emph{d-Select-then-Classify (dS/C) Model}: Select the best
  candidate antecedent by the direct anaphora model and then classify
  the anaphora type. If the candidate is classified as indirect
  anaphora, search for the antecedent with the indirect anaphora
  model.
\item \emph{i-Select-then-Classify (iS/C) Model}: Analogous to the 
  d-Select-then-Classify (dS/C) model with the steps reversed.
\item \emph{p-Select-then-Classify (pS/C) Model}: Call the direct
  anaphora and indirect anaphora models in parallel to select the
  best candidate antecedent for each case and then classify the
  anaphora type referring to both candidates. 
\end{itemize}
The pS/C configuration provides richer contextual information for
classifying anaphora type than any other configuration because it can
always refer to the most likely candidate antecedents of direct
anaphora and indirect anaphora, which may be useful for determining
anaphora type.

We adopt the \emph{one-versus-rest} method for the three-way
classification in our experiments. In other words, we recast the
multi-class classification problem as combinations of a binary
classification. Given an anaphor, each anaphora type classifier
outputs a score that represents the likelihood of its anaphora
type. According to these three scores, we select the anaphora type that
achieves the maximum score.

The training procedure of each model depends on which kinds of
information is needed. To exemplify how to create training instances,
assume that we have the following text to create training instances.

\EX{training_example}{\texttt{ \underline{\mbox{Mariah Carey}}$_{(3')}$ is an
    artist who comes from \underline{the USA}$_{(1)}$. \\ She
    announced \underline{\mbox{her new song}}$_{(2')}$ yesterday. \\I'm
    looking forward to hearing \underline{\mbox{the new song}}$_{(2)}$. \\\underline{The beautiful voice}$_{(3)}$ will
    attract me.}

  with annotated as the following:

  \texttt{the USA}$_{(1)}$ :: exophora \\
  \texttt{the new song}$_{(2)}$ --  \texttt{her new song}$_{(2')}$ :: direct anaphora \\
  \texttt{the beautiful voice}$_{(3)}$ --  \texttt{Mariah Carey}$_{(3')}$ :: indirect anaphora
}
In the aC/S model, the information of an anaphor is needed to determine
anaphora type. Since there are three instances in
\NUM{training_example}, the classifier is trained with \texttt{the
  USA}$_{(1)}$ as exophoric instance, \texttt{the new song}$_{(2)}$ as
direct-anaphoric instance and \texttt{the beautiful voice}$_{(3)}$ as
indirect-anaphoric instance. For the cC/S configuration, in addition
to the anaphor information, the classifier takes all the potential
antecedents. More specifically, the classifier is trained with the
pair of the anaphor and its potential antecedents, hence,
\{\texttt{the USA}$_{(1)}$, (\texttt{Mariah Carey}, \texttt{an
  artist})\} as exophoric instance, \{\texttt{the new song}$_{(1)}$,
(\texttt{Mariah Carey, an artist, the USA, she, her new song,
  yesterday, I})\} as direct-anaphoric instance and \{\texttt{the
  beautiful voice}$_{(3)}$, (\texttt{Mariah Carey, an artist, the USA,
  she, her new song, yesterday, I, the new song})\} as
\linebreak
indirect-anaphoric instance\footnote{ We enumerated only noun phrases
  as the potential antecedents for convenience. In our
  evaluations, we include verbal predicates in the list of potential
  antecedents for such cases as \emph{...we calculate the value in
    advance.} -- \emph{The precomputation ...}. }.

For the S/C configuration, we use the pair of an anaphor and annotated
antecedent or \emph{pseudo}-antecedent as a training instance. It
depends on the anaphora type of interested anaphor and the type of
antecedent selection model that \wire{the classifier utilizes to
determine} whether it is annotated antecedent or
\emph{pseudo}-antecedent. At first, in the sS/C configuration, the
classifier selects \emph{pseudo}-antecedent of \texttt{the
  USA}$_{(1)}$ using the single model since there is no annotated
antecedent in the training set. Suppose \texttt{an artist} is
selected; we create a training instance of exophora from \texttt{the
  USA}$_{(1)}$ paired with \texttt{an artist}. For direct-anaphoric
and indirect-anaphoric instances, we simply take the anaphor and the
annotated antecedent, i.e. $\langle$\texttt{the new song$_{(2)}$},
\texttt{her new song$_{(2')}$}$\rangle$ and $\langle$\texttt{the
  beautiful voice}$_{(3)}$, \texttt{Mariah Carey$_{(3')}$}$\rangle$ as
each training instance.


Second, the case of the dS/C configuration is slightly more
complex. Analogously to the sS/C model, for \texttt{the USA}$_{(1)}$,
we obtain the \emph{pseudo}-antecedent using the direct antecedent
selection model. Suppose \texttt{an artist} is selected; the pair
$\langle$\texttt{the USA}$_{(1)}$, \texttt{an artist}$\rangle$ is used
as an exophoric instance. For \texttt{the new song$_{(2)}$}, we use
the pair $\langle$\texttt{the new song$_{(2)}$}, \texttt{her new
  song$_{(2')}$}$\rangle$ as direct-anaphoric instance. For
\texttt{the beautiful voice$_{(3)}$}, however, since the annotated
antecedent \texttt{Mariah Carey$_{(3')}$} is unlikely to be selected
as the best candidate of \texttt{the beautiful voice$_{(3)}$} by the
\emph{direct} antecedent selection model, the classifier does not use
the annotated example. In dS/C anaphora type classification fashion,
it is required to classify \texttt{the beautiful voice$_{(3)}$} paired
with the \emph{pseudo}-best candidate selected by the \emph{direct} antecedent
selection model as the \emph{indirect} anaphora. We therefore run the
\emph{direct} antecedent selection model to select the
\emph{pseudo}-best candidate. Suppose \texttt{yesterday} is selected;
we create a training instance of indirect anaphora from \texttt{the
  beautiful voice}$_{(3)}$ paired with \texttt{yesterday}.

An analogous method applies also to the iS/C configuration; that is, we
run the \emph{indirect} antecedent selection model to select the
\emph{pseudo}-best candidate except for the indirect-anaphoric instance
\texttt{the beautiful voice}$_{(3)}$. We assume that \texttt{Mariah
  Carey} is selected as the \emph{pseudo}-best candidate for
\texttt{the USA$_{(1)}$} and \texttt{an artist} for \texttt{the new
  song$_{(2)}$}. We create the following training instances:
$\langle$\texttt{the USA$_{(1)}$}, \texttt{Mariah Carey}$\rangle$ as
exophora, $\langle$\texttt{the new song$_{(2)}$} \texttt{an
  artist}$\rangle$ as direct anaphora, $\langle$\texttt{the beautiful
  voice}$_{(3)}$, \texttt{Mariah Carey$_{(3')}$}$\rangle$ as indirect
anaphora.

Finally, in the pS/C configuration, we need \emph{triplets}
$\langle$an anaphor, a direct-anaphoric antecedent, an
indirect-anaphoric antecedent$\rangle$. For \texttt{the USA$_{(1)}$},
since we have no annotated antecedents for direct and indirect
anaphora, we run both the direct and indirect antecedent selection
model to select \emph{pseudo}-best candidates of \texttt{the
  USA$_{(1)}$}. Supposing that \texttt{an artist} is selected by the
\emph{direct} antecedent selection model and \texttt{Mariah Carey} is
selected by the \emph{indirect} model, we create a training instance
of exophora from $\langle$\texttt{the USA$_{(1)}$}, \texttt{an
  artist}, \texttt{Mariah Carey}$\rangle$. For \texttt{the new
  song$_{(2)}$}, since we have no annotated antecedent for
\emph{indirect} anaphora, the indirect antecedent selection model is
chosen to select the \emph{pseudo}-best candidate. Suppose \texttt{an
  artist} is selected; we create a direct-anaphoric training instance
$\langle$\texttt{the new song$_{(2)}$}, \texttt{her new
  song$_{(2')}$}, \texttt{an artist}$\rangle$. For \texttt{the
  beautiful voice}$_{(3)}$, analogous to \texttt{the new
  song$_{(2)}$}, supposing \texttt{yesterday} is selected by the
direct antecedent selection model, we create an indirect-anaphoric
training instance $\langle$\texttt{the beautiful voice$_{(3)}$},
\texttt{yesterday}, \texttt{Mariah Carey$_{(3')}$}$\rangle$.



\section{Dataset} \label{sec:dataset}

For training and testing our models, we created an annotated corpus
that contains 2,929 newspaper articles consisting of 19,669 sentences
for 2,320 broadcasts, 18,714 sentences for 609 editorials, which is
the same articles as in the NAIST Text Corpus~\shortcite{iida07_ntc}. The
NAIST Text Corpus also contains anaphoric relations of noun phrases,
but they are strictly restricted as coreference relations (i.e. two NPs
must refer to the same entity in the world). For this reason, most NPs
marked with a definiteness modifier that we need are not annotated
even when two NPs have a direct-anaphoric relation. Therefore, we
re-annotated (i) direct anaphoric relations, (ii) indirect anaphoric
relations and (iii) exophoric noun phrases of noun phrases marked by
one of the three definiteness modifiers, that is \emph{this}(この),
\emph{the}(その),and \emph{that}(あの).In the specification of our
corpus, not only noun phrases but verb phrases are chosen as
antecedents. For example, the verbal predicate \emph{calculates in
  advance} is selected as an antecedent of \emph{the precomputation}
in example \NUM{verbal}.

\EX{verbal}{システムは前もって値を\underline{計算する}$_{(i')}$。
    \underline{その前計算}$_{(i)}$はシステムの性能を大幅に向上させている。\\
    \emph{
    The system \underline{calculates}$_{(i')}$ the value in
    advance. \underline{\mbox{The precomputation}}$_{(i)}$ significantly improves its
    performance. }}
We also annotated anaphoric relations in the case where an anaphor is
anaphoric with more than two antecedents. For example, we label
anaphoric relations for the two pairs of NPs \emph{mouse devices--the
  other items} and \emph{keyboards--the other items} as seen in
example \NUM{multiple_antecedent}.

\EX{multiple_antecedent}{
    ABCコンピュータは\underline{マウス}$_{(i')}$と\underline{キーボード}$_{(j')}$の値下げを発表した。
    \underline{その他の商品}$_{(i,j)}$については値下げをしないと主張した。\\
    \emph{ ABC computer announced that they
    reduced the price of \underline{mouse devices}$_{(i')}$ and
    \underline{\mbox{keyboards}}$_{(j')}$. They claimed that they would not cut the
    price of \underline{the other items}$_{(i,j)}$. }}

\begin{table}[b]
\caption{Distribution of anaphoric relations in the annotated corpus}
\label{tb:result_annotation}
\input{11table02.txt}
\end{table}

Finally, we obtained 1,264 instances of direct anaphora, 2,345
instances of indirect anaphora, and 470 instances of exophora. The
detailed statistics are shown in Table~\ref{tb:result_annotation}.  To
assess the reliability of the annotation, we estimated its agreement
rate with the two annotators from 418 examples\footnote{These examples
  are randomly sampled from our corpus, and account for 10\% of all
  the examples.} in terms of $K$
statistics~\shortcite{siegel88_kappa}. It resulted in $K$ = 0.73,
which indicates good reliability. For measuring the agreement ratio of
antecedent selection, we used 322 examples (109 for direct anaphora
and 213 for indirect anaphora) whose anaphora types are identically
identified by both two annotators. The agreement ratio was
calculated\footnote{ \wire{We regarded the matching of the rightmost
    offset} as the agreement. When multiple antecedents are annotated,
  the criterion of matching is that one of the antecedents is at least
  identical with one of the antecedents annotated by the other
  annotator.}  according to the following equation:
$$ Agreement = \frac{\mbox{\# of instances which both two annotators
    identified the same antecedent}}{\mbox{\# of all instances}}.$$
The agreement ratio for annotating direct-anaphoric relation obtained
80.7\% (88/109). However, for 21 examples whose antecedents are not
identically selected by the annotators, our analysis revealed that
52.4\% (11/21) of these examples are cases where the antecedents
annotated by the two annotators are different but in anaphoric
relation, which should be regarded as an agreement. Therefore, the
inter-annotator agreement ratio of direct-anaphoric relation achieves
90.8\% (99/109), which indicates good reliability but it is required
to consider anaphoric chains in the annotation procedure. The
agreement ratio of indirect-anaphoric relation, on the other hand,
obtained a comparatively lower ratio of 62.9\% (134/213). One of the
typically non-matching cases is shown in example
\NUM{ex:unmatched_indirect}.


\EX{ex:unmatched_indirect}{\underline{政府}$_{(i)}$は明日まで
  に\underline{委員}$_{(j)}$を決める方針だ。\underline{その人選}$_{(k)}$は
  我々にも影響が及ぶだろう。\\
  \emph{\underline{\mbox{The government}}$_{(i)}$ is going to determine
    \underline{\mbox{the member of the committee}}$_{(j)}$ by tomorrow.\\
    Probably \underline{the election}$_{(k)}$ will also affect us. }
}
In this example, both \emph{the government} and \emph{the member
  of the committee} are considered to be associated objects of
\emph{the election}, which indicates that multiple discourse elements
are often associated with one anaphor in various semantic relations in
indirect anaphora. We should reflect on such problems when the
annotation scheme and task definition of indirect anaphora resolution
are argued, including bridging reference resolution.


\section{Evaluation} \label{sec:evaluation}

We conduct empirical evaluations in order to investigate the three
issues shown in Section 1. First, we compare two antecedent selection
models, the single and separate models described in Section
\ref{sec:antecedent_selection} in order to find out \emph{issue 1},
i.e., whether an antecedent selection model should be trained
separately for direct anaphora and indirect anaphora. Second, the
anaphora type classification models described in Section
\ref{sec:anaphora_type_classification} are evaluated to explore what
information helps with the anaphora type classification (\emph{issue
  2}). Finally, we evaluate the overall accuracy of the entire
anaphora resolution task to explore how the models can be best
configured (\emph{issue 3}).

\wire{ In our experiments, we used anaphors whose antecedent is a head
  of NP that appears in the preceding context of the anaphor (i.e.,
  cataphora is ignored), only taking articles in the broadcast domain
  into account. Therefore, we used 572 instances of direct anaphora,
  878 instances of indirect anaphora and 248 instances of
  exophora. The evaluation was carried out by 10-fold
  cross-validation. In our evaluation of antecedent selection, if a
  selected antecedent is in the same direct-anaphoric chain as the
  labeled antecedent, this selected antecedent is evaluated as
  correct\footnote{\wire{We manually checked our results because of the lack
    of annotation of anaphoric chains as noted in Section
    \ref{sec:dataset}. Due to the cost of this manual
      checking, we took only the broadcast articles into account in
      our experiments, leaving the editorials out.} }. }


For creating binary classifiers used in antecedent selection and
anaphora type classification, we adopted Support Vector
Machines~\shortcite{vapnik95_svm}\footnote{$SVM^{light}$
  http://svmlight.joachims.org/}, with a polynomial kernel of degree 2
and its default parameters.


\subsection{Feature set} \label{sec:feature_set}

\begin{table}[b]
\caption{Feature set for antecedent selection and the S/C models}
 \label{tb:features_antident}
\input{11table03.txt}
\end{table}

The feature set for antecedent selection is designed based on the
\pagebreak
literature of coreference
resolution~\shortcite[etc.]{iida05_coref,1073102,soon01_coref,denis-baldridge:2008:EMNLP,1075119}
as summarized in Table~\ref{tb:features_antident}. In addition, we
introduce the following lexical semantic features:

\begin{itemize}
\item WN\_SEMANTIC\_RELATION: In order to capture various semantic
  relations between an anaphor and its antecedent, we incorporate the
  binary features that represent the semantic relation found in the
  Japanese WordNet
  0.9~\shortcite{isahara08_wnja}\footnote{http://nlpwww.nict.go.jp/wn-ja/}.
\item SYNONYMOUS and IS\_HYPONYM\_OF\_ANAPHOR: We recognize synonymous
  and hyper-hyponym relations by using a very large amount of synonym
  and hypernym-hyponym relations (about three million hypernymy
  relations and two hundred thousand synonymy relations)
  automatically created from Web texts and
  Wikipedia~\shortcite{sumi08_hypowiki}.
\item BGH\_ID, BGH\_COMMON\_ANC: We incorporate the lexical
  information obtained from the \textit{Bunrui Goi Hyo}
  thesaurus~\shortcite{nlsi64_bgh}. We encode the information as two
  types: (i) binary features that represent the semantic class ID,
  and (ii) a real-valued feature that indicates the depth of the lowest
  common ancestor of an anaphor and its candidate.
\item SIMILARITY: To robustly estimate semantic similarities between
  an anaphor and its candidate antecedent, we adopt the cosine
  similarity between an anaphor and candidate antecedent, which is
  calculated from a cooccurrence matrix of $(n, \bracket{c,v})$, where
  $n$ is a noun phrase appearing in an argument position of a verb $v$
  marked by a case particle $c$. \wire{The cooccurrences are counted from
  two decades worth of news paper articles, and their distribution
  $P(n,\bracket{c,v})$ is estimated by pLSI~\shortcite{hofmann99_plsi}
  with 1,000 hidden topic classes to overcome the data sparseness
  problem. }
\item PMI: The degree of indirect-anaphoric association between an
  anaphor \emph{ANA} and candidate \emph{CND} is calculated
  differently depending on whether \emph{CND} is a noun or
  predicate. For the case of a noun, we follow the literature of
  indirect anaphora
  resolution~\shortcite[etc.]{poesio04_lbd,murata99_xnoy} to capture
  such semantic relations as \emph{part-whole}. The associativeness is
  calculated from the cooccurrences of \emph{ANA} and \emph{CND} in
  the pattern of ``\emph{CND} の \emph{ANA} (\emph{ANA of
    CND})''. Frequencies of cooccurrence counts are obtained from the
  Web Japanese N-gram Version 1~\shortcite{kudo07_gng}.  For the case
  of a predicate, on the other hand, the associativeness is calculated
  from the cooccurrences of \emph{ANA} and \emph{CND} in the pattern
  where \emph{CND} syntactically depends on (i.e. modifies) \emph{ANA}
  (in English, the pattern like ``\textit{ANA that \textup{(}subj\textup{)} CND}''). If we
  find many occurrences of, for example, ``闘う (\emph{to fight})''
  modifying ``夢 (\emph{a dream})'' in a corpus, then ``夢 (\emph{a
    dream})'' is likely to refer to an event referred to by
  ``闘う (\emph{to fight})'' as in \NUM{dream}.
  \EX{dream}{
    チャンピオンと\ul{闘い}$_{(i')}$たい。\ul{その夢}$_{(i)}$は実現する
    と信じている。\\
    \emph{I want to \ul{\mbox{fight}}$_{(i')}$ the champion. I believe
    \ul{the dream}$_{(i)}$ will come true.}}
\end{itemize}

For anaphora type classification, we use a different feature set
depending on the configuration described in
\ref{sec:anaphora_type_classification}. For the Classify-then-Select
configuration, as summarized in Table~\ref{tb:features_prior}, it
includes such features as HAS\_SYNONYM\_OF\_ANAPHOR and
HAS\_STRING\_MATCHED, which capture contextual information encoded from
all potential antecedents,  based on the
literature~\shortcite[etc.]{vieira00_defdesc}. For the
Select-then-Classify configurations, on the other hand, an anaphora
type classifier uses the best candidate(s) selected in antecedent
selection phase as its contextual information, instead of the
information encoded from all the potential antecedents. This sort of
information is encoded as features analogous to that for antecedent
selection as summarized in Table~\ref{tb:features_antident}.

\begin{table}[t]
\caption{Feature set for the C/S models}
\label{tb:features_prior}
\input{11table04.txt}
\end{table}



\subsection{Results of antecedent selection} \label{sec:result_as}

The results of antecedent selection are shown in Table
\ref{tb:result_as}. The results\footnote{The accuracy of the separate
  model is better than the single model with statistical significance
  ($p<0.01$, McNemar test).} indicate that the Separate Model
outperforms the Single Model on two anaphora types. As for \emph{issue
  1}, we conclude that the information used for antecedent selection
should be separated for each anaphora type and the selection models
should be trained for each anaphora type. We therefore discard the
Single Model for the further experiments (i.e. discarding sS/C model).

\begin{table}[b]
\vspace{-0.5\baselineskip}
  \caption{Results of antecedent selection}
 \label{tb:result_as}
\input{11table05.txt}
\end{table}
\begin{figure}[b]
\vspace{0.5\baselineskip}
\begin{center}
\includegraphics{17-1ia11f3.eps}
\end{center}
\caption{Learning curve for Separate Models} 
\label{fig:learning_curve_as}
\end{figure}

We also illustrate the learning curves of each model, shown in Figure
\ref{fig:learning_curve_as}. Reducing the training data to 50\%, 25\%,
12.5\%, 6.25\% and 3.13\%, we conducted the evaluation over three
random trials for each size and averaged the accuracies. Figure
\ref{fig:learning_curve_as} indicates that in the direct antecedent
selection model the accuracy becomes better as the training data
increase, whereas the increase of the indirect one looks difficult to
improve although our data set included more instances for indirect
anaphora than for the direct one. These results support the finding in
previous work that an indirect anaphora is harder to resolve than
direct anaphora and suggest that we need a more sophisticated
antecedent selection model for indirect anaphora.

Our error analysis revealed that a majority (about 60\%) of errors in
direct anaphora were caused by the fact that both correct and incorrect
candidates belong to the same semantic category. Example \NUM{d_error}
shows a typical selection error:
\EX{d_error}{
  私は\ul{映画}$_{(j)}$の知識がないが、\ul{『フランケンシュタイ
    ン』}$_{(i')}$ぐらいは知っている。\\\ul{この映画}$_{(i)}$は、本当に名作だ。\\
  \emph{I don't have good knowledge of \ul{movies}$_{(j)}$ but still know
    of \ul{``Frankenstein''}$_{(i')}$.\\I think \ul{this movie}$_{(i)}$ is
    indeed a great masterpiece.}}
where the wrong candidate ``映画$_{(j)}$ (\emph{movies$_{(j)}$})'' was
selected as the antecedent of ``この映画$_{(i)}$ (\emph{this
  movie$_{(i)}$})''\footnote{In Japanese, the plural form of a noun
  is not morphologically distinguished from its singular form.}.  As
can be imagined from this example, there is still room for
improvement by carefully taking into account this kind of error using
other clues such as information from salience. For indirect anaphora,
we analyzed our resource to capture the associativeness between an
anaphor and its antecedent, encoded as PMI in the feature set. Our
analysis indicated that about half of the pattern `$ANT$ of $ANA$',
which occurred in the test data, had been assigned a minus value, i.e., no
positive association found between an anaphor and its antecedent for
the resource when applying PMI. To evaluate the contribution to our
model, we conducted an evaluation where the PMI feature set was
disabled. As a result of this additional evaluation, the model
obtained 51.4\% (451/878), which is no significant difference compared
with the original accuracy. We need to find more useful clues to
capture the associativeness between an anaphor and the related object
in indirect anaphora. The low quality of our annotating data of
indirect-anaphoric relation, as mentioned in
Section~\ref{sec:dataset}, might be also one of the reasons for the
low accuracy of indirect anaphora resolution.


\subsection{Results of anaphora type classification}

\begin{table}[b]
\caption{Results of anaphora type classification}
  \label{tb:result_tc}
\input{11table06.txt}
\end{table}

Now, we move on to \emph{issue 2} and \emph{issue 3}. The results of
anaphora type classification are shown in Table
\ref{tb:result_tc}. The cC/S model obtained the lowest accuracy of
73.6\%, which indicates that contextual information features proposed
in the literature~\shortcite[etc.]{vieira00_defdesc}, such as
HAS\_STRING\_MATCHED, were not actually informative. Note that the
performance of the cC/S model is lower than the aC/S
model\footnote{\wire{The difference is statistically significant
    ($p<0.06$, McNemar test).}}, which identifies an anaphora type by
using only the information of an anaphor. On the other hand, the dS/C
model successfully improved its performance by using the information
of selected candidate antecedent as the contextual information. The
dS/C model achieved the best accuracy of 78.7\%, which indicates that
the selected best candidate antecedent provides useful contextual
information for anaphora type classification\footnote{\wire{The dS/C
    model outperformed the aC/S, cC/S models with statistical
    significance using $p<0.03$, $p<0.01$, as McNemar test parameters
    respectively.}}. The iS/C and pS/C models, however, do not improve
their performance as well as the dS/C model although it uses the
selected best candidate(s) information. It is considered that the
fundamental reason is the poor performance of the \emph{indirect}
antecedent selection model as shown in Table \ref{tb:result_as}, i.e.,
the \emph{indirect} antecedent selection model does not provide
correct contextual information to anaphora type classification.  It is
expected that all the S/C models get better performance when the
antecedent selection model improves.

\begin{table}[b]
\caption{The majority of misclassified-exophoric instances}
  \label{tb:misclassified}
\input{11table07.txt}
\end{table}

The identification of exophora is a more difficult task than the other
anaphora types as shown in the low F-measure and recall in Table
\ref{tb:result_tc}. Our analysis for the exophoric instances
misclassified by the dS/C model revealed that the typical errors were
temporal expressions such as 年 (\emph{year}), 日 (\emph{day})
and 時期 (\emph{period}). We observed that such expressions occurred
as not only exophora but also as the other anaphora types many times,
as summarized in Table \ref{tb:misclassified}, which indicates that
the interpretation of temporal expression is also important for
identifying the other anaphora types. In our current framework,
however, it is hard to recognize such expressions accurately since the
precise recognition of temporal expressions is required to identify a
relation between an event specified by the expression and the other
events. \wire{We consider integrating the framework of temporal
  relation identification, which has been proposed in the
  evaluation-oriented studies such as
  TempEval\footnote{http://www.timeml.org/tempeval/}, with anaphora
  type classification framework, which will be our future work}.


\subsection{Results of overall anaphora resolution}

Finally, we evaluated the overall accuracy of the entire anaphora
resolution task given by:
$$ Accuracy = \frac{\mbox{\# of instances whose antecedent and anaphora type is identified correctly}}{\mbox{\# of all instances}}.$$
The results are shown in Table~\ref{tb:result_overall}. Again, the
dS/C model achieved the best accuracy, which is significantly better
than the Classify-then-Select models.

\begin{table}[t]
\caption{Overall results of anaphora resolution}
 \label{tb:result_overall}
\input{11table08.txt}
\end{table}



\section{Conclusion}

We have addressed the three issues of nominal anaphora resolution for
Japanese NPs marked by a definiteness modifier under two subtasks,
i.e., \emph{antecedent selection} and \emph{anaphora type
  classification}. The issues we addressed were: (i) how the
antecedent selection model should be designed, \wire{(ii) what
  information helps anaphora type classification, and (iii) how the
  antecedent selection and anaphora type classification should be
  carried out.} Our empirical evaluations showed that the separate
model achieved better accuracy than the single model, and the
d-Select-then-Classify and p-Select-then-Classify models give the best
results. We have made several findings through the evaluations: (i) an
antecedent selection model should be trained separately for each
anaphora type using the information useful for identifying its
antecedent, (ii) the best candidate antecedent selected by an
antecedent selection model provides contextual information useful for
anaphora type classification, \wire{and (iii) the antecedent selection
  should be carried out before anaphora type classification.}

However, there is still considerable room for improvement in both
subtasks. Our error analysis for antecedent selection reveals that
the wrong antecedent, which belongs to the same semantic category as
correct antecedent, is likely to be selected while selecting
direct-anaphoric antecedent, and the association measure of
indirect-anaphoric relatedness does not contribute to selecting
the indirect-anaphoric antecedent. We will incorporate the information
that captures salience and various noun-noun relatedness into
antecedent selection in future work. For anaphora type
classification, our analysis reveals that temporal expressions
typically cause error in the identification of exophora. To recognize
such expressions precisely, we will consider integrating temporal
relation identification with anaphora type classification. Our future
work also includes taking general noun phrases into account in
anaphora resolution.





\acknowledgment

I would like to thank the reviewers of this
  paper for their helpful comments.

\bibliographystyle{jnlpbbl_1.4}
\begin{thebibliography}{}

\bibitem[\protect\BCAY{Baldwin}{Baldwin}{1995}]{Baldwin}
Baldwin, F.~B. \BBOP 1995\BBCP.
\newblock {\Bem Cogniac: a discourse processing engine}.
\newblock Ph.D.\ thesis.

\bibitem[\protect\BCAY{Brennan, Friedman, \BBA\ Pollard}{Brennan
  et~al.}{1987}]{Brennan}
Brennan, S.~E., Friedman, M.~W., \BBA\ Pollard, C.~J. \BBOP 1987\BBCP.
\newblock \BBOQ A centering approach to pronouns.\BBCQ\
\newblock In {\Bem Proceedings of the 25th annual meeting on Association for
  Computational Linguistics}, \mbox{\BPGS\ 155--162}.

\bibitem[\protect\BCAY{Bunescu}{Bunescu}{2003}]{razvan03_asocweb}
Bunescu, R. \BBOP 2003\BBCP.
\newblock \BBOQ Associative anaphora resolution: A web-based approach.\BBCQ\
\newblock In {\Bem Proceedings of the EACL 2003 Workshop on The Computational
  Treatment of Anaphora}, \mbox{\BPGS\ 47--52}.

\bibitem[\protect\BCAY{Clark}{Clark}{1977}]{Clark77bridging}
Clark, H.~H. \BBOP 1977\BBCP.
\newblock \BBOQ Bridging.\BBCQ\
\newblock In {\Bem Thinking: Readings in Cognitive Science}.

\bibitem[\protect\BCAY{Denis \BBA\ Baldridge}{Denis \BBA\
  Baldridge}{2008}]{denis-baldridge:2008:EMNLP}
Denis, P.\BBACOMMA\ \BBA\ Baldridge, J. \BBOP 2008\BBCP.
\newblock \BBOQ Specialized Models and Ranking for Coreference
  Resolution.\BBCQ\
\newblock In {\Bem Proceedings of Empirical Methods in Natural Language
  Processing}, \mbox{\BPGS\ 660--669.}\ Honolulu, Hawaii. Association for
  Computational Linguistics.

\bibitem[\protect\BCAY{Fellbaum}{Fellbaum}{1998}]{fellbaum98_wn}
Fellbaum, C. \BBOP 1998\BBCP.
\newblock {\Bem WordNet: An Electronic Lexical Database}.
\newblock MIT Press.

\bibitem[\protect\BCAY{Ge, Hale, \BBA\ Charniak}{Ge et~al.}{1998}]{Ge:98}
Ge, N., Hale, J., \BBA\ Charniak, E. \BBOP 1998\BBCP.
\newblock \BBOQ A statistical approach to anaphora resolution.\BBCQ\
\newblock In {\Bem Proceedings of the 6th Workshop on Very Large Corpora},
  \mbox{\BPGS\ 161--170}.

\bibitem[\protect\BCAY{Grosz, Weinstein, \BBA\ Joshi}{Grosz
  et~al.}{1995}]{grosz95_centering}
Grosz, B.~J., Weinstein, S., \BBA\ Joshi, A.~K. \BBOP 1995\BBCP.
\newblock \BBOQ Centering: a framework for modeling the local coherence of
  discourse.\BBCQ\
\newblock {\Bem Computational Linguistics}, {\Bbf 21}  (2), \mbox{\BPGS\
  203--226}.

\bibitem[\protect\BCAY{Hawkins}{Hawkins}{1978}]{hawkins78}
Hawkins, J.~A. \BBOP 1978\BBCP.
\newblock {\Bem Definiteness and Indefiniteness: A Study in Reference and
  Grammaticality Prediction}.
\newblock Croom Helm Linguistic Series. Taylor \& Francis.

\bibitem[\protect\BCAY{Hofmann}{Hofmann}{1999}]{hofmann99_plsi}
Hofmann, T. \BBOP 1999\BBCP.
\newblock \BBOQ Probabilistic Latent Semantic Indexing.\BBCQ\
\newblock In {\Bem Proceedings of the Twenty Second Annual International SIGIR
  Conference on Research and Development in Information Retrieval},
  \mbox{\BPGS\ 50--57}.

\bibitem[\protect\BCAY{Iida, Inui, Matsumoto, \BBA\ Sekine}{Iida
  et~al.}{2005}]{iida05_coref}
Iida, R., Inui, K., Matsumoto, Y., \BBA\ Sekine, S. \BBOP 2005\BBCP.
\newblock \BBOQ Noun Phrase Coreference Resolution in Japanese based on Most
  Likely Candidate Antecedents Noun Phrase Coreference Resolution in Japanese
  Based on Most Likely Antecedent Candidates.\BBCQ\
\newblock In {\Bem Journal of Information Processing Society of Japan},
  \mbox{\BPGS\ 831--844}.

\bibitem[\protect\BCAY{Iida, Komachi, Inui, \BBA\ Matsumoto}{Iida
  et~al.}{2007}]{iida07_ntc}
Iida, R., Komachi, M., Inui, K., \BBA\ Matsumoto, Y. \BBOP 2007\BBCP.
\newblock \BBOQ Annotating a Japanese Text Corpus with Predicate-Argument and
  Coreference Relations.\BBCQ\
\newblock In {\Bem Proceedings of the ACL 2007 Workshop on Linguistic
  Annotation Workshop}, \mbox{\BPGS\ 132--139}.

\bibitem[\protect\BCAY{Isahara, Bond, Uchimoto, Utiyama, \BBA\ Kanzaki}{Isahara
  et~al.}{2008}]{isahara08_wnja}
Isahara, H., Bond, F., Uchimoto, K., Utiyama, M., \BBA\ Kanzaki, K. \BBOP
  2008\BBCP.
\newblock \BBOQ Development of the Japanese WordNet.\BBCQ\
\newblock In {\Bem Proceedings of the 6th International Language Resources and
  Evaluation}.

\bibitem[\protect\BCAY{Kudo \BBA\ Kazawa}{Kudo \BBA\ Kazawa}{2007}]{kudo07_gng}
Kudo, T.\BBACOMMA\ \BBA\ Kazawa, H. \BBOP 2007\BBCP.
\newblock {\Bem Web Japanese N-gram Version 1}.
\newblock Gengo Shigen Kyokai.

\bibitem[\protect\BCAY{Mccarthy \BBA\ Lehnert}{Mccarthy \BBA\
  Lehnert}{1995}]{McCarthy}
Mccarthy, J.~F.\BBACOMMA\ \BBA\ Lehnert, W. \BBOP 1995\BBCP.
\newblock \BBOQ Using decision trees for coreference resolution.\BBCQ\
\newblock In {\Bem Proceedings of the 14th International Joint Conference on
  Artificial Intelligence}, \mbox{\BPGS\ 1050--1555}.

\bibitem[\protect\BCAY{Mitkov}{Mitkov}{1997}]{Mitkov}
Mitkov, R. \BBOP 1997\BBCP.
\newblock \BBOQ Factors in anaphora resolution: they are not the only things
  that matter. A case study based on two different approaches.\BBCQ\
\newblock In {\Bem Proceedings of the 36th Annual Meeting of the Association
  for Computational Linguistics and the 8th Conference of the European Chapter
  of the Association for Computational Linguistics Workshop on Operational
  Factors in Practical}.

\bibitem[\protect\BCAY{Mitkov, Evans, Orasan, Barbu, Jones, \BBA\
  Sotirova}{Mitkov et~al.}{2000}]{mitkov00_ana}
Mitkov, R., Evans, R., Orasan, C., Barbu, C., Jones, L., \BBA\ Sotirova, V.
  \BBOP 2000\BBCP.
\newblock \BBOQ Coreference and anaphora: developing annotating tools,
  annotated resources and annotation strategies.\BBCQ\
\newblock In {\Bem Proceedings of the Discourse Anaphora and Anaphora
  Resolution Colloquium}.

\bibitem[\protect\BCAY{Murata, Isahara, \BBA\ Nagao}{Murata
  et~al.}{1999}]{murata99_xnoy}
Murata, M., Isahara, H., \BBA\ Nagao, M. \BBOP 1999\BBCP.
\newblock \BBOQ Resolution of Indirect Anaphora in Japanese Sentences Using
  Examples ``XnoY(Yof X)''.\BBCQ\
\newblock In {\Bem Proceedings of the ACL 1999 Workshop on Coreference and Its
  Applications}.

\bibitem[\protect\BCAY{Nakaiwa, Shirai, Ikehara, \BBA\ Kawaoka}{Nakaiwa
  et~al.}{1995a}]{Nakaiwa}
Nakaiwa, H., Shirai, S., Ikehara, S., \BBA\ Kawaoka, T. \BBOP 1995a\BBCP.
\newblock \BBOQ Extrasentential Resolution of {Japanese} Zero Pronouns using
  Semantic and Pragmatic Constraints.\BBCQ\
\newblock In {\Bem Proceedings of AAAI 1995 Spring Symposium Series, Empirical
  Methods in Discourse Interpretation and Generation}. AAAI.

\bibitem[\protect\BCAY{Nakaiwa, Shirai, Ikehara, \BBA\ Kawaoka}{Nakaiwa
  et~al.}{1995b}]{nakaiwa95_extra}
Nakaiwa, H., Shirai, S., Ikehara, S., \BBA\ Kawaoka, T. \BBOP 1995b\BBCP.
\newblock \BBOQ Extrasentential Resolution of {Japanese} Zero Pronouns using
  Semantic and Pragmatic Constraints.\BBCQ\
\newblock In {\Bem Proceedings of AAAI 1995 Spring Symposium Series, Empirical
  Methods in Discourse Interpretation and Generation}. AAAI.

\bibitem[\protect\BCAY{Ng \BBA\ Cardie}{Ng \BBA\ Cardie}{2001}]{1073102}
Ng, V.\BBACOMMA\ \BBA\ Cardie, C. \BBOP 2001\BBCP.
\newblock \BBOQ Improving machine learning approaches to coreference
  resolution.\BBCQ\
\newblock In {\Bem Proceedings of the 40th Annual Meeting on Association for
  Computational Linguistics}, \mbox{\BPGS\ 104--111}.

\bibitem[\protect\BCAY{NLRI}{NLRI}{1964}]{nlsi64_bgh}
NLRI\BED\ \BBOP 1964\BBCP.
\newblock {\Bem Bunrui Goi Hyo (in Japanese)}.
\newblock Shuei Shuppan.

\bibitem[\protect\BCAY{Okumura \BBA\ Tamura}{Okumura \BBA\
  Tamura}{1996}]{Okumura}
Okumura, M.\BBACOMMA\ \BBA\ Tamura, K. \BBOP 1996\BBCP.
\newblock \BBOQ Zero pronoun resolution in Japanese discourse based on
  centering theory.\BBCQ\
\newblock In {\Bem Proceedings of the 16th conference on Computational
  linguistics}, \mbox{\BPGS\ 871--876}.

\bibitem[\protect\BCAY{Poesio, Ishikawa, Walde, Vieira, \BBA\ Sinos}{Poesio
  et~al.}{2002}]{poesio02_aclex}
Poesio, M., Ishikawa, T., Walde, S. S.~I., Vieira, R., \BBA\ Sinos, R. \BBOP
  2002\BBCP.
\newblock \BBOQ Acquiring lexical knowledge for anaphora resolution.\BBCQ\
\newblock In {\Bem Proceedings of the 3rd Conference on Language Resources and
  Evaluation}, \mbox{\BPGS\ 1220--1224}.

\bibitem[\protect\BCAY{Poesio, Mehta, Maroudas, \BBA\ Hitzeman}{Poesio
  et~al.}{2004}]{poesio04_lbd}
Poesio, M., Mehta, R., Maroudas, A., \BBA\ Hitzeman, J. \BBOP 2004\BBCP.
\newblock \BBOQ Learning to resolve bridging references.\BBCQ\
\newblock In {\Bem Proceedings of the 42nd Annual Meeting on Association for
  Computational Linguistics}, \mbox{\BPGS\ 144--151}.

\bibitem[\protect\BCAY{Poesio, Vieira, \BBA\ Teufel}{Poesio
  et~al.}{1997}]{poesio97_bd}
Poesio, M., Vieira, R., \BBA\ Teufel, S. \BBOP 1997\BBCP.
\newblock \BBOQ Resolving bridging references in unresricted text.\BBCQ\
\newblock In {\Bem Proceedings of the 42nd Annual Meeting on Association for
  Computational Linguistics}, \mbox{\BPGS\ 1--6}.

\bibitem[\protect\BCAY{Shalom \BBA\ J.}{Shalom \BBA\ J.}{1994}]{Lappin}
Shalom, L.\BBACOMMA\ \BBA\ J., L.~H. \BBOP 1994\BBCP.
\newblock \BBOQ An algorithm for pronominal anaphora resolution.\BBCQ\
\newblock {\Bem Computational Linguistics}, {\Bbf 20}  (4), \mbox{\BPGS\
  535--561}.

\bibitem[\protect\BCAY{Sidney \BBA\ Castellan}{Sidney \BBA\
  Castellan}{1988}]{siegel88_kappa}
Sidney, S.\BBACOMMA\ \BBA\ Castellan, N.~J. \BBOP 1988\BBCP.
\newblock {\Bem Nonparametric statistics for the Behavioral Sciences}.
\newblock McGraw Hill.

\bibitem[\protect\BCAY{Soon, Ng, \BBA\ Lim}{Soon et~al.}{2001}]{soon01_coref}
Soon, W.~M., Ng, H.~T., \BBA\ Lim, C.~Y. \BBOP 2001\BBCP.
\newblock \BBOQ A Machine Learning Approach to Coreference Resolution of Noun
  Phrases.\BBCQ\
\newblock {\Bem Computational Linguistics}, {\Bbf 27}  (4), \mbox{\BPGS\
  521--544}.

\bibitem[\protect\BCAY{Strube \BBA\ Muller}{Strube \BBA\
  Muller}{2003}]{Strube:03}
Strube, M.\BBACOMMA\ \BBA\ Muller, C. \BBOP 2003\BBCP.
\newblock \BBOQ A machine learning approach to pronoun resolution in spoken
  dialogue.\BBCQ\
\newblock In {\Bem Proceedings of the 41st Annual Meeting on Association for
  Computational Linguistics}, \mbox{\BPGS\ 168--175}.

\bibitem[\protect\BCAY{Sumida, Yoshinaga, \BBA\ Torisawa}{Sumida
  et~al.}{2008}]{sumi08_hypowiki}
Sumida, A., Yoshinaga, N., \BBA\ Torisawa, K. \BBOP 2008\BBCP.
\newblock \BBOQ Boosting Precision and Recall of Hyponymy Relation Acquisition
  from Hierarchical Layouts in Wikipedia.\BBCQ\
\newblock In {\Bem Proceedings of the 6th Language Resources and Evaluation
  Conference}.

\bibitem[\protect\BCAY{Vapnik}{Vapnik}{1995}]{vapnik95_svm}
Vapnik, V.~N. \BBOP 1995\BBCP.
\newblock {\Bem The Nature of Statistical Learning Theory}.
\newblock Wiley.

\bibitem[\protect\BCAY{Vieira \BBA\ Poesio}{Vieira \BBA\
  Poesio}{2000}]{vieira00_defdesc}
Vieira, R.\BBACOMMA\ \BBA\ Poesio, M. \BBOP 2000\BBCP.
\newblock \BBOQ An Empirically Based System for Processing Definite
  Descriptions.\BBCQ\
\newblock {\Bem Computational Linguistics}, {\Bbf 26}  (4), \mbox{\BPGS\
  539--593}.

\bibitem[\protect\BCAY{Yang, Zhou, Su, \BBA\ Tan}{Yang et~al.}{2003}]{1075119}
Yang, X., Zhou, G., Su, J., \BBA\ Tan, C.~L. \BBOP 2003\BBCP.
\newblock \BBOQ Coreference resolution using competition learning
  approach.\BBCQ\
\newblock In {\Bem Proceedings of the 41st Annual Meeting on Association for
  Computational Linguistics}, \mbox{\BPGS\ 176--183}.

\end{thebibliography}

\begin{biography}

  \bioauthor[:]{Naoya Inoue}{
    He received his bachelor's in economics from Musashi University in
    2008. He is currently a student in the Graduate School of
    Information Science at Nara Institute of Science and
    Technology. His research interests are in anaphora resolution,
    discourse understanding and spoken dialogue systems. }

\bioauthor[:]{Ryu Iida}{
  He received his master and Ph.D degrees of engineering from Nara
  Institute of Science and Technology in 2004 and in 2007.  He is
  currently an assistant professor in the School of Information
  Science and Engineering at Tokyo Institute of Technology. His
  research interests include anaphora resolution and information
  extraction.  }

\bioauthor[:]{Kentaro Inui}{
  He received his doctoral degree of engineering from Tokyo Institute
  of Technology in 1995. Having experienced Assistant Professor at
  Tokyo Institute of Technology and Associate Professor at Kyushu
  Institute of Technology, he has been Associate Professor of Graduate
  School of Information Science at Nara Institute of Science and
  Technology since 2002. His research interests include natural
  language understanding and knowledge engineering.  }

\bioauthor[:]{Yuji Matsumoto}{
  He is currently a Professor of Information Science, Nara Institute
  of Science and Technology.  He received his M.S. and Ph.D. degrees
  in information science from Kyoto University in 1979 and in 1989.
  He joined Machine Inference Section of Electrotechnical Laboratory
  in 1979.  He has then experienced an academic visitor at Imperial
  College of Science and Technology, a deputy chief of First
  Laboratory at ICOT, and an associate professor at Kyoto University.
  His main research interests are natural language understanding and
  machine learning.  }

\end{biography}

\biodate




\end{document}
