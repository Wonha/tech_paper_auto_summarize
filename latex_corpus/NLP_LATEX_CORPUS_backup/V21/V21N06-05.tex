    \documentclass[japanese]{jnlp_1.4}

\usepackage{jnlpbbl_1.3}
\usepackage[dvipdfm]{graphicx}
\usepackage{amsmath,amssymb}
\usepackage{udline}
\setulminsep{1.2ex}{0.2ex}
\let\underline
\settensen
\usepackage{gb4e}
\usepackage{cgloss4e}
\noautomath
\setcounter{secnumdepth}{4}
\makeatletter
\renewcommand{\paragraph}{}
\renewcommand{\theparagraph}{}
\makeatother

\usepackage{arydshln}
\renewcommand{\theequation}{}
\renewcommand{\labelenumi}{}


\Volume{21}
\Number{6}
\Month{December}
\Year{2014}

\received{2014}{6}{6}
\revised{2014}{8}{25}
\accepted{2014}{10}{6}

\setcounter{page}{1207}

\jtitle{受身・使役形と能動形間の格交替に関する語彙知識の自動獲得}

\jauthor{笹野 遼平\affiref{TITECH} \and 河原 大輔\affiref{KU} \and 黒橋 禎夫\affiref{KU} \and 奥村  学\affiref{TITECH}}

\jabstract{日本語において受身文や使役文を能動文に変換する際,格交替が起こ
る場合がある.本論文では,対応する受身文・使役文と能動文の格の用例や分布
の類似性に着目し,Webから自動構築した大規模格フレームと,人手で記述した少
数の格の交替パターンを用いることで,受身文・使役文と能動文の表層格の対応
付けに関する知識を自動獲得する手法を提案する.さらに,自動獲得した知識を
受身文・使役文の能動文への変換における格交替の推定に利用することによりそ
の有用性を示す.}

\jkeywords{格交替,受身,使役,格フレーム}

\etitle{Automatic Knowledge Acquisition for Case Alternation between the Passive/Causative and Active Voices}

\eauthor{Ryohei Sasano\affiref{TITECH} \and Daisuke Kawahara\affiref{KU} \and Sadao Kurohashi\affiref{KU} \and Manabu Okumura\affiref{TITECH}}

\eabstract{We propose a method for automatically acquiring knowledge
 about case alternations between the passive/causative and active
 voices. Our method leverages large lexical case frames obtained from a
 large Web corpus, and several alternation patterns. We then use the
 acquired knowledge to a case alternation task and show its usefulness.}

\ekeywords{Case Alternation, Passive, Causative, Case Frame}

\headauthor{笹野,河原,黒橋,奥村}
\headtitle{受身・使役形と能動形間の格交替に関する語彙知識の自動獲得}

\affilabel{TITECH}{東京工業大学精密工学研究所}{Precision and Intelligence Laboratory, Tokyo Institute of Technology}
\affilabel{KU}{京都大学大学院情報学研究科}{Graduate School of Informatics, Kyoto University}



\begin{document}
\maketitle

 \section{はじめに}
 \label{SEC::INTRO} 

テキスト中に出現する述語の格構造を認識する処理は述語
 項構造解析や格解析などと呼ばれ,計算機によるテキスト理解のための重要な1
 ステップである.しかし,格構造を表現する際に使用される``格''には,述語の
 出現形\footnote{本論文では,能動形,受身形,使役形など,述語が実際にテキ
 ストにおいて出現した形のことを出現形と呼ぶ.}に対する表層格や,能動形に
 対する表層格,さらには深層格など複数の表現レベルが存在し,どの表現レベル
 を用いるべきかは使用するコーパス\footnote{京都大学テキストコーパス
 \cite{TAG}では出現形の表層格情報,NAISTテキストコーパス\cite{Iida2007}で
 は能動形の表層格情報が付与されている.}やタスクにより異なっている.

 格構造を表層格で扱う利点としては,表層格はテキスト中に格助詞として明示的
 に出現することから``格''を定義する必要がないこと,述語ごとに取りうる格を
 コーパスから自動獲得することが可能なことなどが挙げられる.さらに,出現形
 に対する表層格で扱う利点としては,能動形では現れない使役文におけるガ格や
 一部の受身文のガ格を自然に扱えること,先行する述語のガ格の項が後続する述
 語でもガ格の項となりやすい\cite{Kameyama1986s,Nariyama2002s}などといった
 談話的な情報が自然に利用できることなどが挙げられる.特に後者はゼロ照応解
 析において重要な手掛りになることが知られており
 \cite{Iida2007T,Sasano2011},ゼロ照応の解決も含む述語項構造解析の高精度
 化のためには,格構造を出現形の表層格で扱うのが望ましいと考えられる.

 一方,テキストの意味を考える上では,出現形に対する表層格解析では不十分な
 場合がある.
\begin{exe}
 \ex\label{EX::FRIEND} 私が知り合いに誘われた.
 \ex\label{EX::PARTY} 私がパーティーに誘われた.
\end{exe}
たとえば,(\ref{EX::FRIEND}),(\ref{EX::PARTY})のような文を考
 えると,出現形の表層格としては(\ref{EX::FRIEND})の「知り合い」と
 (\ref{EX::PARTY})の「パーティー」は同じニ格となっているが,前者は能動主
 体を表しており能動形ではガ格となるのに対し,後者は誘致先を表しており能動
 形においてもニ格となる.このような違いを認識することは情報検索や機械翻訳
 などといった多くの自然言語処理のアプリケーションにおいて重要となる
 \cite{Iida2007}.実際に,Google翻訳
 \footnote{http://translate.google.co.jp/, 2014年5月10日実施.}を用いてこ
 れらの文を英訳すると,(\ref{EX::FRIEND}$'$),(\ref{EX::PARTY}$'$)に示すよ
 うにいずれの文もニ格が誘致先を表すものとして翻訳される.このう
 ち,(\ref{EX::FRIEND}$'$)に示した翻訳は誤訳であるが,これは
 (\ref{EX::FRIEND})の文と(\ref{EX::PARTY})の文におけるニ格の表す意味内容
 の違いを認識できていないため誤って翻訳されたと考えられる.

\begin{exe}
 \exp{EX::FRIEND} I was invited to an acquaintance. \label{EXE::FRIEND}
 \exp{EX::PARTY} I was invited to the party. \label{EXE::PARTY}
\end{exe}

 また,文(\ref{EX::BOTH})は(\ref{EX::FRIEND}),(\ref{EX::PARTY})の2文が表
 す内容を含意していると考えられるが,出現形に対する表層格解析だけではこれ
 らの含意関係を認識することはできない.このため,含意関係認識や情報検索な
 どのタスクでは,能動形に対する表層格構造や深層格構造といった,より深い格
 構造を扱うことが望ましいと言える.

\begin{exe}
 \ex 知り合いが私をパーティーに誘った.\label{EX::BOTH}
\end{exe}

 そこで,まず出現形における表層格の解析を行い,その結果をより深い格構造に
 変換することを考える.このような手順を用いることで,談話的な情報を自然に
 取り入れながら,含意関係認識や情報検索などのタスクにも有用な能動形格構造
 を扱うことができると考えられる.本研究ではこのうち特に受身形・使役形から
 能動形への格構造変換に焦点を当てる.

 受身形・使役形から能動形への格構造変換における格交替パターンの数は限定的
 であり人手で列挙することは容易である.しかし,文
 (\ref{EX::FRIEND}),(\ref{EX::PARTY})からも分かるように,述語と格が同じ
 であっても同一の格交替パターンとなるとは限らない.同様に,項とその格が同
 じであっても同一の格交替パターンとなるとは限らない.たとえば,文
 (\ref{EX::FRIEND})と(\ref{EX::AWARD})のニ格はいずれも「知り合い」である
 が,これらの文を能動形に変換した場合,文(\ref{EX::FRIEND})のニ格はガ格と
 なるのに対し,文(\ref{EX::AWARD})のニ格は能動形においてもニ格のままであ
 る.

\begin{exe}
 \ex 奨励賞が知り合いに贈られた.\label{EX::AWARD}
\end{exe}

 このため,受身形・使役形から能動形への格構造の変換を高精度に行うために
 は,述語・項・格の組み合わせごとに,どのような格交替パターンとなるかを記
 述した大規模な語彙知識が必要となると考えられる.そこで,本研究ではこのよ
 うな語彙知識を大規模コーパスから自動獲得する手法を提案する.具体的には,
 格交替のパターンの数が限定的であること,および,対応する受身文・使役文と
 能動文の格の用例や分布が類似していることに着目し,人手で記述した少数の格
 交替パターンとWebから自動構築した大規模格フレームを用いることで,受身文・
 使役文と能動文の表層格の対応付けに関する知識の自動獲得を行う.また,自動
 獲得した知識を受身文・使役文の能動文への変換における格変換タスクに適用す
 ることにより,その有用性を示す.

 本論文の構成は以下の通りである.まず,2節で関連研究について概観した後,3節
 で受身・使役形と能動形間の格の交替パターンについて,4節でWebから自動構築
 した大規模格フレームについてそれぞれまとめる.続いて5節で提案する格フレー
 ムの対応付け手法について説明し,6節では実験を通してその有効性を示す.最
 後に7節で本論文のまとめを記す.


 \section{関連研究} 

 Levin \citeyear{Levin1993}は態の交替現象に着目し,3,000以上の英語の動詞を共有
 する意味構成と構文的な振る舞いに基づくクラスに分類した.コーパスを用いた
 英語の動詞の自動分類に関する研究においても態の交替を手掛かりとして利用し
 ている研究が数多く存在している
 \cite{Lapata2004CL,Walde2008,Joanis2008,Li2008,Sun2009,Sun2013}.また,
 態の交替に関連してコーパスから得られる用例の分布類似度を利用した研究とし
 てBaroniらの研究\cite{Baroni2010}がある.Baroniらはコーパスに基づく意味
 論の研究において,使役・能動交替を起こす動詞と起こさない動詞の分類にコー
 パスから得られる用例の分布類似度が有用であることを示している.

 日本語における受身形・使役形と能動形の格の変換を扱った研究として
 は,Baldwinらの研究\cite{Baldwin2000},近藤らの研究\cite{Kondo2001},村
 田らの研究\cite{Murata2002,Murata2008}が挙げられる.Baldwinら
 \cite{Baldwin2000}は日本語における受身・使役と能動形の交替を含む動詞交替
 の種類と頻度の定量的な分析を行っている.具体的には人手で記述した格フレー
 ム辞書である日本語語彙大系\cite{NTT}の結合価辞書を解析し,格スロット間の
 選択制約を比較して動詞交替の検出を行っている.しかし,Baldwinらの研究の
 目的は動詞交替の定量的な分析であり,受身形・使役形の能動形への変換は行っ
 ていない.

 近藤ら\citeyear{Kondo2001}は単文の言い換えの1タイプとして,受身形から能動形
 の格の変換,および,使役形から能動形の格の変換を扱っており,動詞のタイプ
 や格パターンなどをもとに作成したそれぞれ7種類,6種類の交替パターンを用い
 て格の変換を行っている.動詞のタイプとしては,「比較動詞」,「授受動
 詞」,「対称動詞」,「一般動詞」の4種類を定義しており,IPAL基本動詞辞書
 \cite{IPAL}をもとに1,564エントリからなる動詞辞書(VDIC辞書)を作成し使用し
 ている.また,村田ら\cite{Murata2002,Murata2008}は京都大学テキストコーパ
 ス\cite{TAG}の社説を除く約2万文において,受身形・使役形で出現した述語に
 係る格助詞を対象に,述語を能動形に変換した場合の格を付与した学習データを
 作成し,SVM\cite{SVM}を用いた機械学習により受身形・使役形と能動形の格を
 変換する手法を提案している.学習に使用する素性には,関係する動詞や体言,
 格助詞の出現形や品詞情報などといった情報に加え,IPAL基本動詞辞書やVDIC辞
 書から得られる情報を使用している.

 このように日本語文における格交替に関する研究では,人手で整備された大規模
 な語彙的リソースや人手で作成した大規模な学習データが利用されてきた.しか
 しながら,文(\ref{EX::FRIEND})と(\ref{EX::PARTY})のように述語と表層格が
 一致していても能動形における格が異なる場合があることからも分かるように,
 格の対応は述語ごと,用法ごとに異なっており,網羅的な対応付けに関する知識
 を人手で記述することは現実的ではないと言える.そこで本研究では格交替に関
 する大規模な語彙知識の自動獲得に取り組む.

 また,NAISTテキストコーパス\cite{Iida2007}では能動形の表層格情報が付与さ
 れていることから,NAISTコーパスを対象とした述語項構造解析やゼロ照応解析
 に関する研究
 \cite{Taira2008s,Iida2009,Imamura2009s,Yoshikawa2010,Hayashibe2014}は,
 受身・使役形で出現した述語の解析を行う際には格交替の解析も行っているとみ
 なすことができる.しかし,これらの研究では素性の1つとして態に関する情報
 を考慮しているものの,格交替に関する語彙知識は使用していない.このため,
 能動形以外の態で出現した述語に対しては相対的に低い解析精度である可能性が
 高く\footnote{実際に,Iidaらからゼロ照応解析システム\cite{Iida2009}の提
 供を受け,NAISTテキストコーパスの社説を除く記事に適用した場合の精度
 (F値)は,\pagebreak 能動形で出現した述語に対してはガ格,ヲ格,ニ格で,それぞれ
 0.358,0.110, 0.014であったのに対し,受身・使役形で出現した述語に対して
 は0.113,0.034,0.021であった.ニ格に対しては後者の方が僅かに高い精度で
 あったものの,全体の80\%以上を占めるガ格に対しては後者の方が大幅に低い精
 度であり,全体として能動形以外の態で出現した述語に対しては低い解析精度で
 あった.},本研究で獲得を行う格交替に関する知識は有用な情報になると考え
 られる.


 \section{格の交替パターン}
 \label{SEC::PATTERN}

受身文・使役文における表層格と,対応する能動文の表層
 格の対応はいくつかのパターンで記述できる.本節では受身文と能動文,使役文
 と能動文,それぞれの表層格の交替パターンについて述べる.


  \subsection{受身文と能動文の格の交替パターン}

  受身文は,何を主語として表現するかによって,直接受身文,間接受身文,持
  ち主の受身文の3つのタイプに分けられる\cite{GendaiNihongo4}.

  直接受身文とは,対応する能動文でヲ格やニ格で表される人や物を主語として
  表現する受身文であり,以下の(\ref{EX::SIRI})〜(\ref{EX::SIMERU})に示す
  ように能動主体は基本的に「に」,「によって」,「から」,「で」のいずれ
  かにより表される.また,直接受身文でガ格として表される名詞は,基本的に
  能動文のヲ格,または,ニ格のいずれかに対応している\footnote{直接受身文
  「私が彼に本を取られた.」と能動文「彼が私から本を取った.」のように,
  直接受身文でガ格として表される名詞が,能動文のカラ格に対応している場合
  も考えられる.しかし,このような事例は少ないと考えられることから本研究
  では考慮しない.実際に,本研究における実験で使用したNICT格助詞変換デー
  タ Version 1.0にはこのような用例は1つも含まれていなかった.}.

\begin{exe}
\ex {\footnotesize\textbf{[受身形]}\hspace{-0.5zw}:}私\underline{が}知り合い\unc{に}誘われた.
	\label{EX::SIRI}
\sn {\footnotesize\textbf{[能動形]}\hspace{-0.5zw}:}知り合い\unc{が}私\underline{を}誘った.
\ex {\footnotesize\textbf{[受身形]}\hspace{-0.5zw}:}原因\underline{が}研究\unc{によって}解明された.
	\label{EX::SHOUMEI}
\sn {\footnotesize\textbf{[能動形]}\hspace{-0.5zw}:}研究\unc{が}原因\underline{を}解明した.
\ex {\footnotesize\textbf{[受身形]}\hspace{-0.5zw}:}私\underline{が}彼\unc{から}頼まれた.
	\label{EX::TANOMI}
\sn {\footnotesize\textbf{[能動形]}\hspace{-0.5zw}:}彼\unc{が}私\underline{に}頼んだ.
\ex {\footnotesize\textbf{[受身形]}\hspace{-0.5zw}:}大半\underline{が}推進派\unc{で}占められた.
	\label{EX::SIMERU}
\sn {\footnotesize\textbf{[能動形]}\hspace{-0.5zw}:}推進派\unc{が}大半\underline{を}占めた.
\end{exe}
  
  間接受身文とは,対応する能動文の表す事態には直接的に関わっていない人物
  を主語とし,その人物が事態から何らかの影響を被っていることを表現する受
  身文であり,迷惑の受身文とも呼ばれる.(\ref{EX::SAWAGU})に示すように間
  接受身文の能動主体は基本的に「に」によって表され,間接受身文でガ格とし
  て表される名詞は能動文では出現しない.

\clearpage
\begin{exe}
 \ex {\footnotesize\textbf{[受身形]}\hspace{-0.5zw}:}太郎\underline{が}雨\unc{に}降られた.
	\label{EX::SAWAGU}
 \sn {\footnotesize\textbf{[能動形]}\hspace{-0.5zw}:}雨\unc{が}降った.
\end{exe}
  
  持ち主の受身文とは,ヲ格やニ格などで表されていた物の持ち主を主語とし,
  能動文で主語として表されていた名詞を主語でない項として表現する受身文で
  ある.(\ref{EX::NUSUMU})に示すように持ち主の受身文の能動主体は基本的に
  「に」によって表され,持ち主の受身文でガ格として表される名詞は能動文で
  はヲ格やニ格の名詞句にノ格で係る名詞句として出現する.

\begin{exe}
 \ex {\footnotesize\textbf{[受身形]}\hspace{-0.5zw}:}友人\underline{が}泥棒\unc{に}カードを盗まれた.
	\label{EX::NUSUMU}
 \sn {\footnotesize\textbf{[能動形]}\hspace{-0.5zw}:}泥棒\unc{が}友人\underline{の}カードを盗んだ.
\end{exe}

  以上のように,受身文には3つのタイプがあるものの,いずれの場合も格交替が
  起こるのは能動文においてガ格で表現される要素と受身文においてガ格で表現
  される要素のたかだか2つである.また,前者は受身文において「に」,「によっ
  て」,「から」,「で」のいずれかによって表され,後者は能動文において
  「を」,「に」,「の」のいずれかによって表されるか出現しないかである.
  したがって,受身文と能動文の格の対応付けを行う際は,これらの組み合わせ
  からなる格の交替パターンを考えれば十分であると言える.


  \subsection{使役文と能動文の格の交替パターン}

  使役文と能動文の格の対応付けも述語と項の組み合わせを考慮して行う必要が
  ある.たとえば,(\ref{EX::STUDENT}),(\ref{EX::SCHOOL})のような文を考え
  ると,出現形の表層形としては(\ref{EX::STUDENT})の「生徒」と
  (\ref{EX::SCHOOL})の「学校」は同じニ格となっているが,前者は能動主体を
  表しており能動形ではガ格となるのに対し,後者は目的地を表しており能動形
  においてもニ格のままである.このため,使役文と能動文の間の格の交替パター
  ンについても,どのような格の交替パターンを考えれば良いか考察を行う.

\begin{exe}
 \ex 先生が生徒に行かせた.\label{EX::STUDENT}
 \ex 先生が学校に行かせた.\label{EX::SCHOOL}
\end{exe}

  まず,一般的な使役文では,対応する能動文に含まれていない人や物がガ格と
  なり,能動文の表す事態の成立に影響を与える主体として表現され,能動主体
  は「を」,または,「に」で表される\cite{GendaiNihongo4}.たとえば,以下
  の例では,能動文に含まれていない「先生」が,使役文におけるガ格として出
  現しており,能動主体である「生徒」は使役文ではそれぞれ「を」,「に」に
  よって表されている.

\begin{exe}
 \ex {\footnotesize\textbf{[使役形]}\hspace{-0.5zw}:}先生\underline{が}生徒\unc{を}行かせる.
	\label{EX::SEITO-WO}
 \sn {\footnotesize\textbf{[能動形]}\hspace{-0.5zw}:}生徒\unc{が}行く.
 \ex {\footnotesize\textbf{[使役形]}\hspace{-0.5zw}:}先生\underline{が}生徒\unc{に}行かせる.
	\label{EX::SEITO-NI}
 \sn {\footnotesize\textbf{[能動形]}\hspace{-0.5zw}:}生徒\unc{が}行く.
\end{exe}

  ただし,頻度は多くないものの,以下のように感情や思考を表す動詞から使役
  文が作られる場合には,能動文においてニ格やヲ格で表される原因が使役文で
  ガ格として表現される場合がある.しかし,このような用例は少ない
  \footnote{本研究における実験で使用したNICT格助詞変換データ Version 1.0
  ではこのような用例は1例のみであった.}ことから,本研究ではこのような対
  応付けは考慮しない.

\begin{exe}
 \ex {\footnotesize\textbf{[使役形]}\hspace{-0.5zw}:}彼の発言\underline{が}社長\unc{を}喜ばせた.
	\label{EX::STUDY}
 \sn {\footnotesize\textbf{[能動形]}\hspace{-0.5zw}:}社長\unc{が}彼の発言\underline{に}喜んだ./社長\unc{が}彼の発言\underline{を}喜んだ.
\end{exe}

  以上のように,使役文と能動文の格の対応付けも複数の格の交替パターンが考
  えられるが,受身文から能動文の格の対応付けの場合と同様に,格交替が起こ
  るのは能動文においてガ格で表現される要素と使役文においてガ格で表現され
  る要素のたかだか2つである.さらに,能動文でガ格として表される要素は使役
  文においては「に」,「を」のいずれかによって表され,使役文においてガ格
  で表現される要素は基本的に能動文には出現しないことから,能動文でガ格と
  して表される要素が使役文において「に」で表されるか「を」で表されるかの
  曖昧性を考慮するだけで十分であると言える.


 \section{Webから自動構築した大規模格フレーム}
 \label{SEC::FRAME}

本研究では述語ごとの格構造に関する大規模語彙知識とし
 て,河原らの手法\cite{Kawahara2005}を用いてWebテキスト69億文から自動構築
 した格フレームを使用する.このWebテキストは,約10億のWebページから日本語
 文を抽出し,重複する文を除いた結果得られたものである.
 
 河原らの手法では格フレームは述語ごとに,また,能動形,受身形,使役形など
 の出現形ごとに,さらに用法ごとに別々に構築され,それぞれ取りうる格とその
 用例,および,各用例の出現回数がまとめられる.この際,河原らは述語の直前
 の格の用例が同じである場合,多くは同じ用法であることを利用し,用法の曖昧
 性に対処している.具体的には,まず,「荷物を積む」や「物資を積む」,「経
 験を積む」などのように述語とその直前の格の用例の組を単位として個別に格フ
 レームを構築し,続いて「荷物を積む」と「物資を積む」のように類似する格フ
 レームをマージすることにより,用法ごとに別々の格フレームを構築している.

 たとえば「誘われる」という述語・出現形に対しては47個の格フレームが構築さ
 れており,その中には(\ref{EX::FRAME1})のようにニ格が能動主体を表す格フレー
 ム\footnote{本論文では格フレームを示す場合,主要な格,用例のみを抜粋して
 示す.また,用例の後の数字はその用例の出現回数を表
 す.}と,(\ref{EX::FRAME2})のようにニ格が誘致先を表す格フレームが含まれ
 ている\footnote{実際には複数の用法が混ざった格フレームも構築される.たと
 えば,接尾辞「れる/られる」には受身,尊敬,自発,可能など複数の意味が考
 えられるが,これら複数の用法が混ざった格フレームも構築されている.}.
 
\begin{exe}
 \ex {\footnotesize\textbf{「誘われる」の格フレーム1:}} 
	\label{EX::FRAME1}
 \sn \{私:5, 自分:3, 母親:2, 女性:2, 彼氏:2, …\}\textbf{が}
 \sn \{食事:49, 御飯:36, ランチ:21, 映画:17, …\}\textbf{を}
 \sn \{友達:9536, 友人:5856, 人:2443, 先輩:1695, …\}\textbf{に} 誘われる
 \ex {\footnotesize\textbf{「誘われる」の格フレーム2:}}
	\label{EX::FRAME2}
 \sn \{俺:10, 私:9, 友達:7, 人:5, 彼氏:5, …\}\textbf{が}
 \sn \{友達:300, 人:221, 男性:211, 友人:168, …\}\textbf{から}
 \sn \{食事:3710, デート:3180, 飲み:3159, パーティー:1948, …\}\textbf{に} 誘われる
\end{exe}

 同様に,「誘う」という述語・出現形に対しては(\ref{EX::FRAME3})に示す格フ
 レームを含む9個の格フレームが構築されており,たとえば(\ref{EX::FRAME2})
 に示した「誘われる」の格フレームのガ格,カラ格,ニ格を,それぞれヲ格,ガ
 格,ニ格に対応付けることができれば,格構造の変換に有用な知識になると考え
 られる.

\begin{exe}
 \ex {\footnotesize\textbf{「誘う」の格フレーム1:}}
	 \label{EX::FRAME3}
 \sn \{私:50, 男性:48, 彼女:43, セラピスト:43, 彼:41, …\}\textbf{が}
 \sn \{友達:16019, 私:8908, 人:7898, 友人:6622, …\}\textbf{を}
 \sn \{デート:1325, 食事:804, 世界:822, 遊び:502, …\}\textbf{に}誘う
\end{exe}

 また,格として収集する対象としては格助詞を伴って直接述語に係る要素に加え
 て,「によって」などの一部の複合辞や,持ち主の受身文のガ格になりうること
 から,述語の直前項にノ格で係る要素も収集の対象としている.このためこ
 れらの表現も格と同等に扱っており,(\ref{EX::FRAME4}),(\ref{EX::FRAME5})の
 ように,これらの表現に相当する格スロットが生成される.本論文ではこれらの
 格を便宜上,ニヨッテ格,ノ格と呼ぶ.
 
\begin{exe}
 \ex{\footnotesize\textbf{「解明される」の格フレーム1:}}
\label{EX::FRAME4}
 \sn \{謎:1998, メカニズム:804, 原因:734, …\}\textbf{が}
 \sn \{研究:29, 生物学:27, 進歩:15, …\}\textbf{によって}解明される
\pagebreak
 \ex {\footnotesize\textbf{「盗む」の格フレーム3:}} 
	\label{EX::FRAME5}
 \sn \{子供:23, 誰:16, 泥棒:8, …\}\textbf{が}
 \sn \{親:165, 人:98, 家:58, …\}\textbf{の}
 \sn \{金:4163, 現金:951, 金品:681, …\}\textbf{を}盗む
\end{exe}


 \section{格フレームの対応付け}
 \label{SEC::PROPOSED} 

本研究では,受身形・使役形の格フレームを対応する能
 動形の格フレームと適切に対応付けることを目的とする.1つの述語に対し複数
 の格フレームが構築されることから,ある受身形・使役形の格フレームが与えら
 れた場合,複数ある能動形の格フレームから最適な格フレームを選択した上で,
 それぞれの格フレームに含まれる格同士を適切に対応付ける必要がある.図
 \ref{FIG::TaskDef}に対応付けの例を示す.図\ref{FIG::TaskDef}の例では,受
 身形「誘われる」の格フレーム2が入力として与えられた結果,能動形「誘う」
 の格フレームの中から格フレーム1が選択され,「誘われる」の格フレーム2のガ
 格,カラ格,ニ格はそれぞれ「誘う」の格フレーム1のヲ格,ガ格,ニ格に対応
 付けられている.

  \begin{figure}[b]
   \begin{center}
    \includegraphics{21-6ia5f1.eps}
    \caption{受身形・使役形と能動形格フレームの対応付けの例}
    \label{FIG::TaskDef}
   \end{center}
  \end{figure}


  \subsection{対応付けアルゴリズム}

  出現形格フレームが与えられた場合に,
それを能動形格フレームに対応付ける
  アルゴリズムを表\ref{TABLE::ALG}に示す.
まず,能動形格フレーム
  $cf_{active}$と,出現形のガ格が対応付けられる能動形の格$c_{ga\_to}$,能
  動形のガ格に対応付けられる出現形の格$c_{to\_ga}$の考えうるすべての組み
  合わせ$A=$\{$cf_{active},c_{ga\_to},c_{to\_ga}$\}を生成し,その中から以
  下の式で定義されるスコアが最大となる組み合わせ$A$を出力する.この
  際,$c_{ga\_to}$と$c_{to\_ga}$が事前に作成した格交替パターンを満たすよ
  うな組み合わせのみを考慮する.また,対応付けられる格がない場合は
  $c_{ga\_to}$,$c_{to\_ga}$としてNILを与える.\ \\\vspace{-5ex}
  \begin{equation}
   \mathit{score} = \operatorname{sim}_{_\mathit{SEM}}(A)\times
	\operatorname{sim}_{_\mathit{DIST}}(A)^\alpha\times f_{_\mathit{PP}}(A) 
	\label{EQ::SCORE}
  \end{equation} 

  \begin{table}[t]
  \caption{能動形格フレームへの対応付けアルゴリズム} \label{TABLE::ALG}
\input{05table01.txt}
  \end{table}

  式(\ref{EQ::SCORE})におい
  て,$\operatorname{sim}_{_\mathit{SEM}}(A)$,$\operatorname{sim}_{_\mathit{DIST}}(A)$,
$f_\mathit{pp}(A)$は
  それぞれ,対応する格の用例集合間の意味的な類似度,対応する格の出現頻度
  の分布の類似度,格交替パターンの起こりやすさを表しており,本研究ではこ
  れら3つの手掛かりを利用し格の対応付けを行う.各指標をどのように計算する
  かについては次節で述べる.$\alpha$は出現頻度の分布の類似度
  $\operatorname{sim}_{_\mathit{DIST}}(A)$を用例集合間の意味的な類似度
  $\operatorname{sim}_{_\mathit{SEM}}(A)$に対してどのくらい重視するかを決めるパラメータ
  であり,その値は開発データを用いて決定する.格交替パターンの起こりやす
  さを表す$f_\mathit{pp}(A)$に関しては同様のパラメータが出現していないのは
  $f_\mathit{pp}(A)$自体が開発データを用いて決定される重みによって構成される関数
  であり,他の指標に対してどのくらい$f_\mathit{pp}$を重視するかは既に考慮されて
  いるためである.

  また,計算時間を短縮するため,事前に作成した格交替パターンに含まれない
  ような不適切な格対応の組み合わせはスコア計算前にフィルタリングする(表
  \ref{TABLE::ALG}のアルゴリズム中の5行目).具体的には,格変換の結果,同
  一の格が重複してしまう場合や,受身形格フレームにヲ格が存在しない場合に
  受身形ガ格の変換先としてノ格が選択された場合などはここでフィルタリング
  される.さらに,能動形の格フレームは出現頻度順にソートし,頻度の大きい
  ものから順に対応付けを行っていき,全体の80\%をカバーした時点でもっとも
  スコアが大きくなる組み合わせを出力する.対応する格の意味的な類似度
  $sim_{SEM}$を計算する際も,それぞれの格の用例のうち頻度上位40用例のみか
  ら類似度を計算することで計算時間を短縮する.
 


  \subsection{対応付けの手掛かり}

  本節では格フレームの対応付けの手掛かりとして使用する3つの指標について,
  それらの指標を用いる理由,および,その計算方法を説明する.
  

  \subsubsection*{1. 対応する格の用例集合間の意味的な類似度:$\boldsymbol{\operatorname{sim}_\mathit{SEM}}$}

出現形と能動形格フレームの間で対応する格の用例は類
   似していると考えられる.そこで,対応する格の用例集合間の意味的な類似度
   を対応付けの手掛かりの1つとして利用する.

   まず,格の用例集合$C_1$,$C_2$間の意味的な類似度
   $\operatorname{sim}_s(C_1,C_2)$を,ある単語と共起する単語の分布の類似度から計
   算された単語間の分布類似度$\operatorname{sim}(w_1, w_2)$を用い,以下の式により
   計算する.
   \begin{gather*}
    \operatorname{sim}_s(C_1,C_2)=\frac{1}{2}(\operatorname{sim}_a(C_1,C_2)+
	\operatorname{sim}_a(C_2,C_1)) \\
    \text{ただし}\quad \operatorname{sim}_a(C_1, C_2)
	=\frac{1}{|C_1|}\sum_{w_1\in {C}_1} \max_{w_2\in C_2}(\operatorname{sim}(w_1, w_2))
   \end{gather*}
   本研究では,単語間の分布類似度として,柴田らの手法\cite{Shibata2009}に
   基づき,格フレーム構築に使用したWebテキスト69億文から計算した類似度を
   使用した.また,$\operatorname{sim}_a(C_1, C_2)$は,用例集合$C_1$に含まれる用
   例$w_1$ごとに,用例集合$C_2$中でもっとも類似している用例$w_2$との類似
   度の平均を表しており\footnote{用例集合$C_1$,$C_2$は単語の種類ではなく
   単語の各出現を要素としている.したがって類似度$\operatorname{sim}_a(C_1,C_2)$
   は単語の種類単位で考えると出現頻度で重み付けされていることになる.},
   引数$C_1$,$C_2$に関して非対称な(asymmetric)式となっている.一
   方,$\operatorname{sim}_s(C_1,C_2)$は$\operatorname{sim}_a(C_1,C_2)$の引数を入れ替え
   て,その平均を取ったものであり,引数$C_1$,$C_2$に関して対称な
   (symmetric)式となっている.

   続いて,格フレームの対応付け
   $A=$\{$cf_{active},c_{ga\_to},c_{to\_ga}$\}に対する意味的な類似度
   $\operatorname{sim}_{SEM}$を,以下に示すように対応付けられた格の用例集合間の類
   似度の平均として定義する.
   \[
     \operatorname{sim}_{SEM}(A)= \frac{1}{N}\sum_{i=1}^{N}\operatorname{sim}_s(C_{1,i},C_{2,a(i)})
   \]
   ここで,$C_{1,i}$は対応元格フレームにおける$i$番目の格の用例集
   合,$C_{2,a(i)}$は$i$番目の格が対応付けられた対応先格フレームの格の用
   例集合を表している.すなわち,$i$番目の格がガ格である場合は
   $C_{2,a(i)}$は対応先格フレームの格$c_{ga\_to}$の用例集合,$i$番目の格
   が$c_{to\_ga}$である場合は$C_{2,a(i)}$は対応先格フレームのガ格の用例集
   合,それ以外の場合は同一の格が対応先格フレームに存在すればその格の用例
   集合となる.ただし,$c_{ga\_to}$がNILである場合など,対応する格が対応
   先格フレームに存在しない場合は,$C_{2,a(i)}$は空集合と
   し,$\operatorname{sim}_s(C_{1,i},C_{2,a(i)})=0$として計算する.

   たとえば,図\ref{FIG::TaskDef}に示したような格フレームの対応に対して
   は,「誘われる」の格フレーム2のガ格,カラ格,ニ格の用例集合に対し,そ
   れぞれ「誘う」の格フレーム1のヲ格,ガ格,ニ格の用例集合との
   $\operatorname{sim}_s$を求め,その平均を$\operatorname{sim}_{SEM}$として使用す
   る.


\subsubsection*{2. 対応する格の出現頻度の分布の類似度:$\boldsymbol{\operatorname{sim}_\mathit{DIST}}$}

   類似度$\operatorname{sim}_{DIST}$は対応する格同士の出現頻度の分
   布は似ているという仮定に基づく指標であり,以下のようにベクトル
   $(|C_{1,1}|, |C_{1,2}|, \dots, |C_{1,N}|)$とベクトル$(|C_{2,a(1)}|,
   |C_{2,a(2)}|, \dots,|C_{2,a(N)}|)$の余弦類似度として定義する.
\[
  \operatorname{sim}_{_{DIST}}(A)=\cos((|C_{1,1}|, \dots, |C_{1,N}|),
    (|C_{2,a(1)}|, \dots, |C_{2,a(N)}|))
\]
\begin{exe}
 \ex {\footnotesize\textbf{「選ばれる」の格フレーム1:}}
	\label{EX::ERABA}
 \sn \{選手:1119, 作品:983, 私:232, …\}{\footnotesize [用例数合計:17722]}\textbf{が}
 \sn \{代表:18295, 選手:9661,百選:7024, …\}{\footnotesize [用例数合計:122273]}\textbf{に}
 \sn \{作品:5, 市長:3, 選手:2, …\}{\footnotesize [用例数合計:96]}\textbf{を}選ばれる
 \ex {\footnotesize\textbf{「選ぶ」の格フレーム13:}}
	\label{EX::ERABU}
 \sn \{私:22, 先生:18, 誰:14, …\}{\footnotesize [用例数合計:382]}\textbf{が}
 \sn \{優秀賞:42, シングル:17, 自由曲:17,  …\}{\footnotesize [用例数合計:800]}\textbf{に}
 \sn  \{曲:16666, 作品:9967, 漫画:3820, …\}{\footnotesize [用例数合計:33338]}\textbf{を}選ぶ
\end{exe}

   例として,(\ref{EX::ERABA})に示す「選ばれる」の格フレーム1を
   (\ref{EX::ERABU})に示す「選ぶ」の格フレーム13に対応付ける場合を考え
   る.対応する格の用例集合間の意味的な類似度$\operatorname{sim}_{SEM}$を計算する
   と\mbox{\{ガ格$\rightarrow$}ニ格,ニ格$\rightarrow$ガ格,ヲ格
   $\rightarrow$ヲ格\}という対応付け$A_1$の方が, \{ガ格$\rightarrow$ヲ格,
   ニ格$\rightarrow$ニ格,ヲ格$\rightarrow$NIL, NIL$\rightarrow$ガ格\}と
   いう対応付け$A_2$よりも高いスコアとなる.しかし,対応する格の出現頻度
   の分布の類似度を計算すると,前者は$(17722,122273,96)$と
   $(800,382,33338)$の余弦類似度,後者は$(17722,122273,96,0)$と
   $(33338,800,0,382)$の余弦類似度となり,以下に示すとおり後者の方がはる
   かに大きな値となることから,$\operatorname{sim}_{SEM}$だけでなく
   $\operatorname{sim}_{DIST}$も考慮することで,最終的に後者の対応付けの方が優先
   して選択されるようになる\footnote{ただし,「選ばれる」の格フレーム1の
   ヲ格は尊敬の意味で使用された「選ばれる」の用例から生成されたものであ
   り,「選ぶ」の格フレーム13のヲ格と対応付けることは必ずしも誤りとは言え
   ない.この点については\ref{SEC::EVAL}節で考察する.}.
\newpage
\begin{gather*}
  \operatorname{sim}_{_{DIST}}(A_1)=\cos((17722,122273,96)
     (800,382,33338))\approx 0.016 \\
  \operatorname{sim}_{_{DIST}}(A_2)=\cos((17722,122273,96,0),
     (33338,800,0,382))\approx 0.167
\end{gather*}


\subsubsection*{3. 格交替パターンの起こりやすさ:$\boldsymbol{f_{pp}}$}

   格交替パターンの起こりやすさも重要な手掛りとなると考えられる.たとえ
   ば,村田ら\cite{Murata2008}が受身文から能動文への変換における格助詞の
   変換実験に使用したデータ\footnote{NICT格助詞変換データ Version 1.0:
   http://alaginrc.nict.go.jp/case/src/kaku1.0.tar.gz}では,受身形におけ
   るガ格の96.47\%が能動形ではヲ格となっているのに対し,受身形におけるニ
   格のうち能動形でガ格となるものは27.38\%となっており,交替パターンによ
   り起こりやすさが異なることが分かる.そこで本研究では,以下の式により定
   義される格交替パターンの選好$f_{pp}$を対応付けの手掛かりとして考慮する.
   \begin{equation}
    f_{_\mathit{PP}}(A)=w(ガ\rightarrow c_\mathit{ga\_to})\times w(c_\mathit{to\_ga}
     \rightarrow ガ), \label{EQ::SP}
   \end{equation}
   ここで,$c_\mathit{ga\_to}$は出現形のガ格が対応付けられる能動形の
   格,$c_\mathit{to\_ga}$は能動形のガ格に対応付けられる出現形の格を表してお
   り,$w(c_1 \rightarrow c_2)$は出現形の$c_1$格が能動形で格が交替し
   $c_2$格となる度合いを表す.$w(c_1 \rightarrow  c_2)$は大きいほどそ
   の格交替が発生しやすいことを表し,その値は格交替の正解がタグ付けされた
   データを開発データとして用い決定する.ここで,出現形のガ格,または,能
   動形のガ格を含む格交替のみを考慮しているのは,\ref{SEC::PATTERN}節で述
   べたように,本研究で対象とする受身形・使役形から能動形への格構造変換で
   はこれら以外の格が交替することは基本的にないためである.
  


 \section{自動獲得した対応付け知識の評価}
 \label{SEC::EVAL}

本節では実験により提案手法の有効性を確認する.具体的に
 は,提案手法を用いて自動獲得した受身・使役形の格フレームと能動形の格フレー
 ムの対応付け知識が,受身・使役文から能動文への変換における格交替を推定す
 るタスクにおいて有用であることを示すことにより,提案手法の有効性を確認す
 る.


  \subsection{評価方法の概要}
  \label{SEC::EVA_ABS} 

\ref{SEC::FRAME}節で述べたように,本研究で使用する
  大規模格フレームはコーパスから自動構築されたものであるため,同じ用法の
  格フレームが複数構築されている場合や,複数の用法が混在した格フレームが
  含まれている場合がある.このため,対応付け知識そのものを定量的に評価す
  るのは難しい.たとえば,(\ref{EX::FRAME6})に示す「誘われる」の格フレー
  ム4のニ格には能動主体を表す用例と招致先を表す用例が混在しているため,仮
  にこの格フレームのニ格が\ref{SEC::FRAME}節に示した「誘う」の格フレーム
  1のガ格に対応付けられたとしても,それが正しいかどうかは一概には言えな
  い.同様に,\ref{SEC::PROPOSED}節の(\ref{EX::ERABA})に示した「選ぶ」の
  格フレーム1のヲ格は,尊敬の意味で使用された「選ばれる」の用例から生成さ
  れたものであり,仮にこの格が(\ref{EX::ERABU})に示した「選ぶ」の格フレー
  ム13のヲ格と対応付けられたとしても必ずしも誤りとは言えない.そこで本研
  究では,格フレームの対応付け結果そのものを評価するのではなく,自動獲得
  された対応付け知識の実タスクにおける有用性を示すことで,提案手法の有効
  性を確認する.

\begin{exe}
 \ex {\footnotesize\textbf{「誘われる」の格フレーム4:}}
	\label{EX::FRAME6}
 \sn \{私:2, 主人公:1, 友達:1, 妹:1, 女の子:1,  …\}\textbf{が}
 \sn \{姉:8, 展示会:7, 娘:6, 説明会:6, 皆:6, …\}\textbf{に}誘われる
\end{exe}

  具体的には,受身・使役文から能動文への変換における格交替を推定するタス
  クを考える.すわなち,たとえば(\ref{EX::TOMO})のような受身文が入力され
  た場合に,受身文におけるガ格,ニ格がそれぞれ能動文ではヲ格,ニ格として
  表されることを推定するタスクを考え,格フレームの対応付け知識を用いるこ
  とで推定精度が向上することを示す.

\begin{exe}
 \ex 友達\underline{が}食事\underline{に}誘われた.
	\label{EX::TOMO}
\end{exe}


  \subsection{実験に使用するデータ}
  \label{SEC::DATA} 

実験にはNICT格助詞変換データ Version 1.0を使用する.
  このデータは村田ら\cite{Murata2002,Murata2008}が実験に使用したデータで
  あり,京都大学テキストコーパス\cite{TAG}の社説を除く約2万文において,受
  身形または使役形で出現した述語に係る格助詞を1事例として,述語を能動形に
  変換した場合の格を人手で付与したデータとなっている.ただし,受身形と能
  動形の格フレームの対応付け知識,使役形と能動形の格フレームの対応付け知
  識をそれぞれ適切に評価できるように,以下に述べるような変更・抜粋を行っ
  た上で使用する.

  まず,受身文と能動文の変換実験のデータとしては,基本的にNICT格助詞変換
  データに含まれる``受身文の能動文への変換における格助詞変換デー
  タ''\cite{Murata2008}を使用する.ただし,村田らの実験設定では持ち主の受
  身文においてガ格で表される名詞の能動文における格としてノ格を認めておら
  ず,カラ格またはヲ格となっているため,持ち主の受身文のガ格と考えられる
  5事例の変換後の格をノ格に変更した.また,それ以外にも誤っていると考えら
  れる16事例に修正を加えて使用した.本データは,受身文に出現した格助詞を
  それぞれ1つの事例とし全部で3,576事例からなっている.村田らはこのデータ
  を1,788個ずつに分け,それぞれクローズドデータ,オープンデータと呼んでい
  る.本研究でも村田らと同様に分割し,評価の際には2分割交差検定を行う.ま
  た,このデータの一部には複数の格が正解として付与されている事例がある
  が,本研究では正解として付与されている格のうち1つ,または,その両方を出
  力できれば正解とみなすという評価基準を採用する.これは村田らの論文
  \cite{Murata2008}における評価Bに相当する.

  使役文と能動文の変換実験のデータとしては,基本的にNICT格助詞変換データ
  に含まれる\mbox{``使}役文・受身文の能動文への変換における格助詞変換デー
  タ''\cite{Murata2002}の全4,671事例から使役文に出現した524事例を抜き出し
  て使用する.ただし,「退学させられた」などの使役受身文については,通常
  の使役文と格交替の起こり方が異なるため524事例には含めなかった.受身文と
  能動文の変換実験のデータの場合と同様に,もともと付与されている格が誤っ
  ていると考えられる39事例に修正を加え,評価の際には2分割交差検定を行う.


  \subsection{実験設定}

   \subsubsection{考慮する格の交替パターン}

   \ref{SEC::PATTERN}節で行った分析に基づき,受身形格フレームと能動形格フ
   レームの格交替のパターンとしては以下の組み合わせのみを考慮する.

   \begin{itemize}
    \item 受身形のガ格の対応先の候補:ヲ格,ニ格,ノ格,対応なし(NIL)
    \item 能動形のガ格の対応先の候補:ニ格,ニヨッテ格,カラ格,デ格,対応なし(NIL)
   \end{itemize}

   これらの各候補は表\ref{TABLE::ALG}の3行目の$c_\mathit{ga\_to}$,および,4行目
   の$c_{to\_ga}$の候補となる.ただし,一方の格フレームの複数の格が,もう
   一方の格フレームの1つの格に対応付けられるような交替パターンは認めな
   い.同様に,使役形格フレームと能動形格フレームの格の交替パターンとして
   は以下の組み合わせのみを考慮する.

   \begin{itemize}
    \item 使役形のガ格の対応先の候補:対応なし(NIL)
    \item 能動形のガ格の対応先の候補:ニ格,ヲ格
   \end{itemize}
   

   \subsubsection{正解データの使用方法}

   正解が付与されたデータを格交替推定時にどのように使用するかについては,
   正解データを使用しない,開発データとして使用する,開発データ・学習デー
   タとして使用する,という3つの設定を用いる.以下ではそれぞれの設定につ
   いて詳述する.

\paragraph{
正解データを使用しない}

    格交替推定に正解データを使用しない場合は,
使用できる開発データがない
    ことになるため\ref{SEC::PROPOSED}節の式(\ref{EQ::SCORE})の$\alpha$,
    および,$f_\mathit{pp}$はいずれも1に固定し,
以下の式が最大となる組み合わせ
    を,格フレームの対応付け結果として出力する.
\[
   \mathit{score} = \operatorname{sim}_{_\mathit{SEM}}(A)
	\times \operatorname{sim}_{_\mathit{DIST}}(A)
\]
    その上で,得られた対応付け知識を用い,以下の手順で能動形における格の
    推定を行う.

    \begin{enumerate}
     \item 格フレームに基づく構文・格解析器である
	   KNP\footnote{http://nlp.ist.i.kyoto-u.ac.jp/index.php?KNP}
	   \hspace{-0.2em}を用いて入力文の格解析を行う\footnote{KNPで使用
	   する格フレームには受身・使役形と能動形の対応付けに用いたものと
	   同一の格フレームを使用した.}.KNPは格解析を行う際,入力文中の
	   各動詞に対し,その出現形の格フレームの集合の中から適切な格フレー
	   ムを1つ選択し,その動詞に係る項と格スロットの対応付けを行う.
     \item 格フレームの対応付け情報を利用し,受身・使役文の格を能動文にお
	   ける格に変換する.この際,出現格がニ格であった場合でも,格解析
	   の結果,時間格や修飾格に対応付けられている場合は,格変換を行わ
	   ない.
    \end{enumerate} 

    本論文ではこのモデルを\textbf{モデル1}と呼ぶ.たとえ
    ば,\ref{SEC::EVA_ABS}節に示した文(\ref{EX::TOMO})が入力され,さら
    に,「誘われた」のガ格,および,ニ格がそれぞれ\ref{SEC::PROPOSED}節の
    図\ref{FIG::TaskDef}に示した受身形「誘われる」の格フレーム2のガ格,ニ
    格にそれぞれ割り当てられたとすると,これらの格はそれぞれ能動形「誘う」
    の格フレーム1のヲ格,ニ格に対応付けられていることから,能動文における
    格はそれぞれヲ格,ニ格であると出力される.


\paragraph{
開発データとして使用}

    続いて\ref{SEC::PROPOSED}節の式(\ref{EQ::SCORE})の$\alpha$,およ
    び,$f_{pp}$の値の調整に正解データを使用する場合について説明する.こ
    の実験設定では,\ref{SEC::DATA}節で説明したように正解データを2分割
    し,一方を開発データ,もう一方をテストデータとした実験を,データの役
    割を入れ替え2度繰り返す.
    
    式(\ref{EQ::SP})中の$w(ga\rightarrow c_\mathit{ga\_to})$,
$w(c_\mathit{to\_ga} \rightarrow  ga)$,および,式
    (\ref{EQ::SCORE})中の$\alpha$は山登り法により決定する.具体的には,た
    とえば,受身形と能動形の格フレームの対応付けを行う際は,$c_\mathit{ga\_to}$
    の候補はニ格,ニヨッテ格,カラ格,または,対応なし
    (NIL),$c_\mathit{to\_ga}$の候補は,ヲ格,ニ格,ノ格,対応なし(NIL)であるこ
    とから,以下のようなパラメータベクトル $\mathbf{x}$ を定義し,表
    \ref{TABLE::TUNE}に示すアルゴリズムにより値を決定する.
\begin{align*}
  \mathbf{x} & = (w(ガ \rightarrow ニ), w(ガ \rightarrow ニヨッテ),
       w(ガ\rightarrow カラ), w(ガ \rightarrow デ), w(ガ \rightarrow \operatorname{NIL}), \\
     & \quad w(ヲ \rightarrow ガ), w(ニ \rightarrow ガ), w(ノ \rightarrow ガ), 
	w(\operatorname{NIL}\rightarrow ガ), \alpha)
\end{align*}
    
    ここで,表\ref{TABLE::TUNE}中の$f_{accuracy}(\mathbf{x})$は,あるパラメー
    タベクトル$\mathbf{x}$が与えられた場合に,その$\mathbf{x}$の値を用いて格フレー
    ムの対応付けを行い,その結果を開発データに適用し得られた格変換の精度
    を返す関数である.このアルゴリズムはパラメータを1つずつ順に0.1刻みで
    更新していき,$f_\mathit{accuracy}(\mathbf{x})$が大きくなるようにパラメータを更
    新していくという手順を,$f_\mathit{accuracy}(\mathbf{x})$の値に変化がなくなるま
    で繰り返す山登り法に基づくアルゴリズムとなっている.
    
    本実験設定では,開発データを利用し最終的に得られたパラメータを用いて
    格フレームの対応付けを行い,得られた対応付け知識を用いてモデル1と同様
    の手順で能動形における格の推定を行う.本論文では,このモデルを
    \textbf{モデル2}と呼ぶ.
  
    \begin{table}[b]
     \caption{パラメータベクトルの調整アルゴリズム}
     \label{TABLE::TUNE}
\input{05table02.txt}
    \end{table}
    

\paragraph{
開発データ・学習データとして使用}

    開発データとして使用したデータを学習データとしても使用し格交替の推定
    モデルを生成する.学習の方法は基本的にSVMに基づく村田ら
    \cite{Murata2002,Murata2008}と同様の方法で行い,モデル2の手法による格
    の推定結果を新たに素性として追加する.使用した素性の詳細については次
    節以降で説明する.本論文では,このモデルを\textbf{モデル3}と呼ぶ.


  \subsection{受身文と能動文の変換実験}
  
   \subsubsection{学習データを使用しない場合(モデル1・モデル2)}

   \begin{table}[b]
    \caption{受身文から能動文への変換における格交替推定実験の結果(学習データを使用しない場合)}
\label{TABLE::RESULT1}
\input{05table03.txt}
   \end{table}

   正解データを学習データとして使用しない設定における受身文から能動文への
   変換における格交替推定実験の結果を表\ref{TABLE::RESULT1}に示す.ベース
   ラインとしては村田ら\citeyear{Murata2008}と同様に,各格助詞ごとに最も頻度
   の高い変換後の格を出力する方法(最頻変換)を使用した.最頻変換の結果が村
   田らの報告にある0.882より高くなっているが,これはデータに修正を加えた
   ためである.また,対応する格の用例集合の意味的な類似度
   $\operatorname{sim}_{_\mathit{SEM}}$と,対応する格の出現頻度の分布の類似度
   $\operatorname{sim}_{_\mathit{DIST}}$,それぞれの効果を確認するため,それぞれ片方だけ
   使用した場合の実験も行った.表~\ref{TABLE::RESULT1}では,モデル1,モデ
   ル2の設定でそれぞれ$\operatorname{sim}_{_\mathit{SEM}}$だけを用いたモデルをモデル
   1$_S$,モデル2$_S$,$\operatorname{sim}_{_\mathit{DIST}}$だけを用いたモデルをモデル
   1$_D$,モデル2$_D$として示している.
  
   モデル1$_S$,モデル1はいずれも開発データも学習データも使用しないモデル
   であるが,マクネマー検定\cite{McNemar1947}の結果,最頻変換の精度との間
   には有意な差\footnote{本論文における実験では特に断りがない限り有意水準
   として0.01を用いた.}があることが確認できた.一方,モデル1$_S$とモデル
   1の精度の間には有意な差は確認できず,これらの結果からは対応する格の出
   現頻度の分布の類似度$\operatorname{sim}_{_\mathit{DIST}}$を対応付けの手掛りとして利用
   することの有効性は確認できなかった.

   モデル2$_S$,モデル2$_D$,モデル2はいずれも開発データを用いてパラメー
   タ調整を行うモデルである.モデル1とモデル2の精度の差は有意であり,パラ
   メータ調整を行うことが有効であることが確認できた.また,モデル2$_S$と
   モデル2,モデル2$_D$とモデル2の差もそれぞれ有意であ
   り,$\operatorname{sim}_{_\mathit{SIM}}$と$\operatorname{sim}_{_\mathit{DIST}}$,
いずれの手掛かりも
   格フレームの対応付けの手掛かりとして有用であることが確認できた.モデル
   2を用いた場合の$\operatorname{sim}_{_\mathit{SIM}}$と$\operatorname{sim}_{_\mathit{DIST}}$の寄与度を
   制御するパラメータである$\alpha$の値は2分割交差検定のいずれに対しても
   0.3であった.

   \subsubsection{学習データを使用する場合(モデル3)}

   続いて,正解データを学習データとして使用した場合の格交替推定精度を村田
   らの手法\cite{Murata2008}を用いた場合の精度とともに表
   \ref{TABLE::RESULT2}に示す.また,実際に使用した素性を表
   \ref{TABLE::FEATURES}に示す.F1からF32までは村田らが使用した素性
   \cite{Murata2008}と同じであり,F33のみが新たに追加した素性である.表
   \ref{TABLE::RESULT2}に示した結果からモデル2の出力を素性として追加する
   ことで格の推定精度が向上することが確認できる.また,検定の結果この差は
   有意なものであることが確認された.

  \begin{table}[b]
   \caption{受身文から能動文への変換における格交替推定実験の結果(学習データを使用する場合)}
\label{TABLE::RESULT2}
\input{05table04.txt}
  \end{table}

  表\ref{TABLE::FEATURES}に示した素性の一部は人手で作成された語彙知識に基
  づいている.具体的には,F15,F22,F23,F24,F26は近藤ら
  \cite{Kondo2001}によって作成されたVDIC辞書に基づく素性(以下ではVDIC素性
  と呼ぶ),F16,F17,F18,F19,F20,F21はIPAL基本動詞辞書\cite{IPAL}に基
  づく素性(以下ではIPAL素性と呼ぶ),F4,F7,F9,F12は分類語彙表
  \cite{BGH1964}に基づく素性(以下ではBGH素性と呼ぶ)となっている.本研究で
  獲得した語彙知識はこれらの語彙知識に相当する知識となっていると考えられ
  ることから,これらの素性を除いた場合の精度の調査も行った.結果を表
  \ref{TABLE::RESULT3}に示す.

  \begin{table}[t]
   \caption{受身文から能動文への変換における格交替推定に使用した素性}
   \label{TABLE::FEATURES}
\input{05table05.txt}
  \end{table}

  VDIC素性,IPAL素性については,村田らのモデル\cite{Murata2008},対応付け
  知識に基づくモデル3,いずれのモデルに対しても,使用しないことによる精度
  の低下は確認できなかった.一方,BGH素性については,村田らのモデル
  \cite{Murata2008}から除いた場合は精度が低下したのに対し,対応付け知識に
  基づくモデル3から除いても精度の低下は確認できなかった.このことから,格
  フレームの対応付けによって得られる語彙知識は分類語彙表から得られる知識
  をカバーしていると考えられる.以上の分析から,自動獲得した語彙知識が使
  用できる場合,人手で作成した語彙知識の有用性は限定的であると言える.実
  際,人手で作成した語彙知識に基づく素性をすべて除いて格交替の推定実験を
  行ったところ0.960という高い精度が得られた.

  \begin{table}[b]
   \caption{人手で作成された語彙知識を用いなかった場合の精度}
\label{TABLE::RESULT3}
\input{05table06.txt}
  \end{table}


  \subsection{使役文と能動文の変換実験}

   \subsubsection{学習データを使用しない場合(モデル1・モデル2)}

   正解データを学習データとして使用しない設定における使役文から能動文への
   変換における格交替推定実験の結果を表\ref{TABLE::RESULTS1}に示す.ベー
   スラインとしては,受身文からの変換の場合と同様に,各格助詞ごとに最も頻
   度の高い変換後の格を出力する方法(最頻変換)を使用し,対応する格の用例集
   合の意味的な類似度$\operatorname{sim}_{_\mathit{SEM}}$と,対応する格の出現頻度の分布
   の類似度$\operatorname{sim}_{_\mathit{DIST}}$,それぞれ片方だけを使用した実験も行った.

\begin{table}[b]
    \caption{使役文から能動文への変換における格交替推定実験の結果(学習データを使用しない場合)}
\label{TABLE::RESULTS1}
\input{05table07.txt}
   \end{table}
   
   受身文の変換の場合と異なりパラメータ調整を行わなかった場合は最頻変換と
   同等または低い精度となった.一方,パラメータ調整を行った場合は,モデル
   2$_S$,モデル2$_D$,モデル2,いずれについても最頻変換より良い精度となっ
   た.ただし,マクネマー検定におけるp値がもっとも小さくなった最頻変換と
   モデル2$_D$の間の差に対しても,有意水準0.05では有意性を確認できたもの
   の,有意水準0.01では有意性を確認できなかった.しかし,受身形の変換の場
   合とほぼ同様に格交替の推定精度は上昇幅は約0.05であり,有意水準0.01で有
   意性が確認できなかったのは事例数が少なかったことが主な要因であると考え
   られる.

   本実験結果において,受身文の変換の場合と大きく異なる点は,対応する格の
   用例集合の意味的な類似度$\operatorname{sim}_{_\mathit{SEM}}$を用いることの効果が確認
   できなかった点である.実際,もっとも高い精度となったのは類似度として
   $\operatorname{sim}_{_\mathit{DIST}}$のみを用い,パラメータ調整を行ったモデル2$_D$で
   あった.


  \subsubsection{学習データを使用する場合(モデル3)}

  続いて,正解データを学習データとして使用する設定における使役文から能動
  文への変換における格交替推定実験の結果を村田らの手法\cite{Murata2002}を
  用いた場合の精度とともに表\ref{TABLE::RESULTS2}に示す.また,実際に使用
  した素性を表\ref{TABLE::FEATURES2}に示す.
  
  \begin{table}[b]
   \caption{使役文から能動文への変換における格交替推定実験の結果(学習データを使用する場合)}
\label{TABLE::RESULTS2}
\input{05table08.txt}
  \end{table}
  \begin{table}[b]
    \caption{使役文から能動文への変換における格交替推定に使用した素性}
    \label{TABLE::FEATURES2}
\input{05table09.txt}
  \end{table}

  本実験では基本的に村田らが使用した素性\cite{Murata2002}を使用し,獲得し
  た語彙知識の有用性を確認するため新たに素性F9を追加している.ただし,受
  身形の変換の場合に倣い,動詞の単語の分類語彙表の分類番号の1, 2, 3, 4, 5, 7桁
  までの数字を素性に追加したところ変換精度が向上したことから,動詞の単語
  の分類語彙表の分類番号に関する情報だけは新たに素性F3として追加しており
  \footnote{受身文の変換で使用した他の素性も有効である可能性があるが,本
  実験の目的は基本的に素性F9の効果を確認することであることから,他の素性
  が本実験において有効かどうかの確認は行っていない.},表
  \ref{TABLE::RESULTS2}に示した精度は村田らの手法も含めすべて素性F3を使用
  した場合の精度である.また,表\ref{TABLE::RESULTS2}中のモデル3$_D$は,
  学習データを使用しない設定ではモデル2$_D$がもっとも高い精度であったこと
  から,モデル2$_D$の出力結果を素性F9として使用したモデルである.

  モデル3,モデル3$_D$ともに,自動獲得した語彙知識を用いない村田らのモデ
  ルより有意に高い精度を達成しており,使役文から能動文への変換における格
  交替推定においても,格フレームの対応付けによって得られた語彙知識が有用
  であることが確認できた.ただし,受身文の変換の場合とは異なり,分類語彙
  表に基づく素性であるF3,F6を除いた場合,格交替の推定精度は低下した.


  \subsection{格フレームの対応付けの例と既知の問題点}

  表\ref{TABLE::NAGURU}に正しく受身形格フレームと能動形格フレームが対応付
  けられた例を示す.この例では持ち主の受身文から生成されたと考えられる受
  身形「殴られる」の格フレーム2のニヨッテ格,ガ格,ヲ格,デ格がそれぞれ,
  能動形「殴る」の格フレーム2のガ格,ノ格,ヲ格,デ格に対応付けられてお
  り,この知識を用いることにより,入力テキスト中の「殴られ」のガ格は能動
  文ではノ格となり,デ格,ヲ格については能動文においてもデ格,ヲ格のまま
  であると解析できるようになる.

  \begin{table}[b]
  \caption{受身形格フレームと能動形格フレームの対応付けの例}
\label{TABLE::NAGURU}
\input{05table10.txt}
  \end{table}

  同様に,表\ref{TABLE::HOSSOKU}に正しく使役形格フレームと能動形格フレー
  ムが対応付けられた例を示す.この例では能動主体がヲ格となっていると考え
  られる使役形「発足させる」の格フレーム1のヲ格,ニ格がそれぞれ,能動形
  「発足する」の格フレーム2のガ格,ニ格に対応付けられ,ガ格は対応なしとなっ
  ている.この知識を用いることにより,入力テキスト中の「発足させ」のガ格
  は能動文では出現せず,ヲ格については能動文においてはガ格となると解析で
  きるようになる.

  \begin{table}[t]
   \caption{使役形格フレームと能動形格フレームの対応付けの例}
   \label{TABLE::HOSSOKU}
\input{05table11.txt}
  \end{table}

  一方,格フレームの対応付けおよび格交替推定に関する既知の問題点としては
  以下の4つが挙げられる.

  まず,格フレーム構築法に関する問題点として,複数のニ格を取る受身形格構
  造・使役形格構造を考慮していないという点が挙げられる.受身形および使役
  形の場合,それぞれ(\ref{EX::NINI1}),(\ref{EX::NINI2})のように,能動主
  体を表すニ格と,能動形ニ格の2つのニ格を取る場合があるが,このような格構
  造を考慮して格フレームを構築していないため,このような文が入力された場
  合に適切な解析を行えない.
  
\begin{exe}
  \ex 彼\underline{に}家\underline{に}帰られる.\label{EX::NINI1}
  \ex 彼\underline{に}東京\underline{に}行かせる.\label{EX::NINI2}
\end{exe}
  
  また,受身形格フレームの構築法に関する問題点として,受身とそれ以外の意
  味,すなわち,尊敬,自発,可能の意味で使用された「れる/られる」を区別し
  ていないという点が挙げられる.その結果,たとえば(\ref{EX::KOKU})のよう
  に尊敬の意味で「れる」が使用された文から,本来,ヲ格を持たないはずの
  「選ばれる」の受身形格フレームにヲ格が生成されてしまい,格フレームの対
  応付けを行おうとした際に,適切な対応付けが行えなくなってしまう場合があ
  る.

\begin{exe}
  \ex 国王がこの作品を選ばれた.\label{EX::KOKU}
\end{exe}

  格フレームの対応付け法の問題点としては,格フレーム中の複数の格スロット
  が同一の意味内容を表している場合を考慮していないことが挙げられる.たと
  えば,(\ref{EX::SAIYO})に示すように「採用される」の能動主体はニ格,ニヨッ
  テ格,カラ格の3つの格で表すことができるため,「採用される」の格フレーム
  はこれらの格を持っている.しかし,受身形格フレームの複数の格が能動主体
  を表す場合であっても,受身形と能動形の格フレームの対応付けを行う際はそ
  れぞれの格フレームの格の1対1の対応付けしか考慮していないため,これらの
  格すべてを能動形のガ格と対応付けることができない.

\begin{exe}
  \ex 福岡県[に/によって/から]採用された.\label{EX::SAIYO}
\end{exe}

  さらに,KNPによる解析において適切な格フレームが選択されなかったために,
  正しい格交替を推定できない場合がある.たとえば,(\ref{EX::NOMIYA})のよ
  うな文が入力されると,\ref{SEC::FRAME}節の(\ref{EX::FRAME1})に示した
  「誘われる」の格フレーム1のようにニ格が能動主体を表している格フレームが
  選択されるべきであると考えられるが,実際には(\ref{EX::FRAME2})に示した
  「誘われる」の格フレーム2のようにニ格が誘致先を表す格フレームが選択され
  てしまうため,本来,能動文においてはガ格となるべき格が,ニ格のままであ
  ると出力されてしまう.

\begin{exe}
 \ex 花見の席で野宮\underline{に}誘われた.\label{EX::NOMIYA}
\end{exe}


 \section{おわりに}

 本論文では,Webから自動構築した大規模格フレームと,人手で記述した少数の
 受身形・使役形と能動形の格の交替パターンを組み合わせることで,受身形・使
 役形と能動形の表層格の対応付けに関する知識を自動獲得する手法を提案した.
 また,獲得した知識を受身文・使役文の能動文への変換における格交替推定に利
 用することにより,その有用性を示した.今後の方向性としては,他の対応する
 格フレーム間,たとえば,授受動詞間や自他動詞間の格フレームの対応付けを行
 うことが考えられる.本論文で提案した手法は基本的に学習データを必要としな
 いことから,考えうる格の交替パターンさえ記述できれば自動的に対応を取るこ
 とが可能である.また,本論文では受身文・使役文において格助詞が明示された
 項のみを格変換の対象としているが,今後は提題助詞の使用や,被連体修飾要素
 としての出現,ゼロ代名詞化などにより格が明示されていない場合も解析の対象
 とすることが考えられる.


\acknowledgment

ゼロ照応解析システム\cite{Iida2009}を提供していただいた情報通信研究機構の
飯田龍氏,格助詞変換データを公開してくださいました鳥取大学の村田真樹氏,
情報通信研究機構の鳥澤健太郎氏に感謝いたします.また,本研究の一部はJSPS
科研費23800025,25730131の助成を受けたものです.


\bibliographystyle{jnlpbbl_1.5}
\begin{thebibliography}{}

\bibitem[\protect\BCAY{Baldwin \BBA\ Tanaka}{Baldwin \BBA\
  Tanaka}{2000}]{Baldwin2000}
Baldwin, T.\BBACOMMA\ \BBA\ Tanaka, H. \BBOP 2000\BBCP.
\newblock \BBOQ Verb Alternations and {J}apanese -- How, What and Where?\BBCQ\
\newblock In {\Bem Proceedings of PACLIC 14}, \mbox{\BPGS\ 3--14}.

\bibitem[\protect\BCAY{Baroni \BBA\ Lenci}{Baroni \BBA\
  Lenci}{2010}]{Baroni2010}
Baroni, M.\BBACOMMA\ \BBA\ Lenci, A. \BBOP 2010\BBCP.
\newblock \BBOQ Distributional Memory: A General Framework for Corpus-Based
  Semantics.\BBCQ\
\newblock {\Bem Computational Linguistic}, {\Bbf 36}  (4), \mbox{\BPGS\
  673--721}.

\bibitem[\protect\BCAY{林部祐太\JBA 小町守\JBA 松本裕治}{林部祐太 \Jetal
  }{2014}]{Hayashibe2014}
林部祐太\JBA 小町守\JBA 松本裕治 \BBOP 2014\BBCP.
\newblock 述語と項の位置関係ごとの候補比較による日本語述語項構造解析.\
\newblock \Jem{自然言語処理}, {\Bbf 21}  (1), \mbox{\BPGS\ 3--26}.

\bibitem[\protect\BCAY{Iida, Inui, \BBA\ Matsumoto}{Iida
  et~al.}{2007a}]{Iida2007T}
Iida, R., Inui, K., \BBA\ Matsumoto, Y. \BBOP 2007a\BBCP.
\newblock \BBOQ Zero-Anaphora Resolution by Learning Rich Syntactic Pattern
  Features.\BBCQ\
\newblock {\Bem ACM Transactions on Asian Language Information Processing
  (TALIP)}, {\Bbf 6}, \mbox{Article 12}.

\bibitem[\protect\BCAY{Iida, Inui, \BBA\ Matsumoto}{Iida
  et~al.}{2009}]{Iida2009}
Iida, R., Inui, K., \BBA\ Matsumoto, Y. \BBOP 2009\BBCP.
\newblock \BBOQ Capturing Salience with a Trainable Cache Model for
  Zero-anaphora Resolution.\BBCQ\
\newblock In {\Bem Proceedings of ACL-IJCNLP'09}, \mbox{\BPGS\ 647--655}.

\bibitem[\protect\BCAY{Iida, Komachi, Inui, \BBA\ Matsumoto}{Iida
  et~al.}{2007b}]{Iida2007}
Iida, R., Komachi, M., Inui, K., \BBA\ Matsumoto, Y. \BBOP 2007b\BBCP.
\newblock \BBOQ Annotating a {J}apanese Text Corpus with Predicate-Argument and
  Coreference Relations.\BBCQ\
\newblock In {\Bem Proceedings of ACL'07 Workshop: Linguistic Annotation Workshop},
  \mbox{\BPGS\ 132--139}.

\bibitem[\protect\BCAY{im~Walde, Hying, Scheible, \BBA\
  Schmid}{im~Walde et~al.}{2008}]{Walde2008}
im~Walde, S. S., Hying, C., Scheible, C., \BBA\ Schmid, H. \BBOP
  2008\BBCP.
\newblock \BBOQ Combining {EM} Training and the {MDL} Principle for an
  Automatic Verb Classification Incorporating Selectional Preferences.\BBCQ\
\newblock In {\Bem Proceedings of ACL-HLT'08}, \mbox{\BPGS\ 496--504}.

\bibitem[\protect\BCAY{Imamura, Saito, \BBA\ Izumi}{Imamura
  et~al.}{2009}]{Imamura2009s}
Imamura, K., Saito, K., \BBA\ Izumi, T. \BBOP 2009\BBCP.
\newblock \BBOQ Discriminative Approach to Predicate-Argument Structure
  Analysis with Zero-Anaphora Resolution.\BBCQ\
\newblock In {\Bem Proceedings of ACL-IJCNLP'09}, \mbox{\BPGS\ 85--88}.

\bibitem[\protect\BCAY{Joanis, Stevenson, \BBA\ James}{Joanis
  et~al.}{2008}]{Joanis2008}
Joanis, E., Stevenson, S., \BBA\ James, D. \BBOP 2008\BBCP.
\newblock \BBOQ A General Feature Space for Automatic Verb
  Classification.\BBCQ\
\newblock {\Bem Natural Language Engineering}, {\Bbf 14}  (3), \mbox{\BPGS\
  337--367}.

\bibitem[\protect\BCAY{情報処理振興事業協会技術センター}{情報処理振興事業協会
技術センター}{1996}]{IPAL}
情報処理振興事業協会技術センター \BBOP 1996\BBCP.
\newblock 計算機用日本語基本動詞辞書 {IPAL}.

\bibitem[\protect\BCAY{Kameyama}{Kameyama}{1986}]{Kameyama1986s}
Kameyama, M. \BBOP 1986\BBCP.
\newblock \BBOQ A Property-sharing Constraint in Centering.\BBCQ\
\newblock In {\Bem Proceedings of ACL'86}, \mbox{\BPGS\ 200--206}.

\bibitem[\protect\BCAY{河原\JBA 黒橋}{河原\JBA 黒橋}{2005}]{Kawahara2005}
河原大輔\JBA 黒橋禎夫 \BBOP 2005\BBCP.
\newblock 格フレーム辞書の漸次的自動構築.\
\newblock \Jem{自然言語処理}, {\Bbf 12}  (2), \mbox{\BPGS\ 109--131}.

\bibitem[\protect\BCAY{河原\JBA 笹野\JBA 黒橋\JBA 橋田}{河原 \Jetal
  }{2005}]{TAG}
河原大輔\JBA 笹野遼平\JBA 黒橋禎夫\JBA 橋田浩一 \BBOP 2005\BBCP.
\newblock \Jem{格・省略・共参照タグ付けの基準}.

\bibitem[\protect\BCAY{国立国語研究所}{国立国語研究所}{1964}]{BGH1964}
国立国語研究所 \BBOP 1964\BBCP.
\newblock \Jem{分類語彙表}.
\newblock 秀英出版.

\bibitem[\protect\BCAY{近藤\JBA 佐藤\JBA 奥村}{近藤 \Jetal }{2001}]{Kondo2001}
近藤恵子\JBA 佐藤理史\JBA 奥村学 \BBOP 2001\BBCP.
\newblock 格変換による単文の言い換え.\
\newblock \Jem{情報処理学会論文誌}, {\Bbf 42}  (3), \mbox{\BPGS\ 465--477}.

\bibitem[\protect\BCAY{Lapata \BBA\ Brew}{Lapata \BBA\
  Brew}{2004}]{Lapata2004CL}
Lapata, M.\BBACOMMA\ \BBA\ Brew, C. \BBOP 2004\BBCP.
\newblock \BBOQ Verb Class Disambiguation using Informative Priors.\BBCQ\
\newblock {\Bem Computational Linguistics}, {\Bbf 30}  (1), \mbox{\BPGS\
  45--73}.

\bibitem[\protect\BCAY{Levin}{Levin}{1993}]{Levin1993}
Levin, B. \BBOP 1993\BBCP.
\newblock {\Bem English Verb Classes and Alternations: A Preliminary
  Investigation}.
\newblock University of Chicago Press.

\bibitem[\protect\BCAY{Li \BBA\ Brew}{Li \BBA\ Brew}{2008}]{Li2008}
Li, J.\BBACOMMA\ \BBA\ Brew, C. \BBOP 2008\BBCP.
\newblock \BBOQ Which Are the Best Features for Automatic Verb
  Classification.\BBCQ\
\newblock In {\Bem Proceedings of ACL-HLT'08}, \mbox{\BPGS\ 434--442}.

\bibitem[\protect\BCAY{McNemar}{McNemar}{1947}]{McNemar1947}
McNemar, Q. \BBOP 1947\BBCP.
\newblock \BBOQ Note on the Sampling Error of the Difference between Correlated
  Proportions or Percentages.\BBCQ\
\newblock {\Bem Psychometrika}, {\Bbf 12}, \mbox{\BPGS\ 153--157}.

\bibitem[\protect\BCAY{村田\JBA 井佐原}{村田\JBA 井佐原}{2002}]{Murata2002}
村田真樹\JBA 井佐原均 \BBOP 2002\BBCP.
\newblock 受け身/使役文の能動文への変換における機械学習を用いた格助詞の変換.\
\newblock \Jem{情報処理学会研究報告,自然言語処理研究会報告 2002-NL-149}, \mbox{\BPGS\
  39--44}.

\bibitem[\protect\BCAY{村田\JBA 金丸\JBA 白土\JBA 井佐原}{村田 \Jetal
  }{2008}]{Murata2008}
村田真樹\JBA 金丸敏幸\JBA 白土保\JBA 井佐原均 \BBOP 2008\BBCP.
\newblock
  入力文の格助詞ごとに学習データを分割した機械学習による受身文の能動文への変換
における格助詞の変換.\
\newblock \Jem{システム制御情報学会論文誌}, {\Bbf 21}  (6), \mbox{\BPGS\
  165--175}.

\bibitem[\protect\BCAY{Nariyama}{Nariyama}{2002}]{Nariyama2002s}
Nariyama, S. \BBOP 2002\BBCP.
\newblock \BBOQ Grammar for Ellipsis Resolution in Japanese.\BBCQ\
\newblock In {\Bem Proceedings of TMI'02}, \mbox{\BPGS\ 135--145}.

\bibitem[\protect\BCAY{日本語記述文法研究会}{日本語記述文法研究会}{2009}]{Gend
aiNihongo4}
日本語記述文法研究会\JED\ \BBOP 2009\BBCP.
\newblock \Jem{現代日本語文法2, 第4部 ヴォイス}, \mbox{\BPGS\ 207--298}.
\newblock くろしお出版.

\bibitem[\protect\BCAY{NTTコミュニケーション科学研究所}{NTTコミュニケーション科学研究所}{1997}]{NTT}
NTTコミュニケーション科学研究所 \BBOP 1997\BBCP.
\newblock \Jem{日本語語彙大系}.
\newblock 岩波書店.

\bibitem[\protect\BCAY{笹野\JBA 黒橋}{笹野\JBA 黒橋}{2011}]{Sasano2011}
笹野遼平\JBA 黒橋禎夫 \BBOP 2011\BBCP.
\newblock 大規模格フレームを用いた識別モデルに基づく日本語ゼロ照応解析.\
\newblock \Jem{情報処理学会論文誌}, {\Bbf 52}  (12), \mbox{\BPGS\ 3328--3337}.

\bibitem[\protect\BCAY{柴田\JBA 黒橋}{柴田\JBA 黒橋}{2009}]{Shibata2009}
柴田知秀\JBA 黒橋禎夫 \BBOP 2009\BBCP.
\newblock 超大規模ウェブコーパスを用いた分布類似度計算.\
\newblock \Jem{言語処理学会第15回年次大会発表論文集}, \mbox{\BPGS\ 705--708}.

\bibitem[\protect\BCAY{Sun \BBA\ Korhonen}{Sun \BBA\ Korhonen}{2009}]{Sun2009}
Sun, L.\BBACOMMA\ \BBA\ Korhonen, A. \BBOP 2009\BBCP.
\newblock \BBOQ Improving Verb Clustering with Automatically Acquired
  Selectional Preferences.\BBCQ\
\newblock In {\Bem Proceedings of EMNLP'09}, \mbox{\BPGS\ 638--647}.

\bibitem[\protect\BCAY{Sun, McCarthy, \BBA\ Korhonen}{Sun
  et~al.}{2013}]{Sun2013}
Sun, L., McCarthy, D., \BBA\ Korhonen, A. \BBOP 2013\BBCP.
\newblock \BBOQ Diathesis Alternation Approximation for Verb Clustering.\BBCQ\
\newblock In {\Bem Proceedings of ACL'13}, \mbox{\BPGS\ 736--741}.

\bibitem[\protect\BCAY{Taira, Fujita, \BBA\ Nagata}{Taira
  et~al.}{2008}]{Taira2008s}
Taira, H., Fujita, S., \BBA\ Nagata, M. \BBOP 2008\BBCP.
\newblock \BBOQ A {J}apanese Predicate Argument Structure Analysis using
  Decision Lists.\BBCQ\
\newblock In {\Bem Proceedings of EMNLP'08}, \mbox{\BPGS\ 523--532}.

\bibitem[\protect\BCAY{Vapnik}{Vapnik}{1995}]{SVM}
Vapnik, V. \BBOP 1995\BBCP.
\newblock {\Bem The Nature of Statistical Learning Theory}.
\newblock Springer.

\bibitem[\protect\BCAY{吉川克正\JBA 浅原正幸\JBA 松本裕治}{吉川克正 \Jetal
  }{2010}]{Yoshikawa2010}
吉川克正\JBA 浅原正幸\JBA 松本裕治 \BBOP 2010\BBCP.
\newblock Markov Logicによる日本語述語項構造解析.\
\newblock \Jem{情報処理学会研究報告,自然言語処理研究会報告 2010-NL-199}, \mbox{\BPGS\
  1--7}.

\end{thebibliography}


\begin{biography}
 \bioauthor{笹野 遼平}{2009年東京大学大学院情報理工学系研究科博士課程修
 了.博士(情報理工学).京都大学大学院情報学研究科特定研究員を経て2010年よ
 り東京工業大学精密工学研究所助教.自然言語処理,特に照応解析,述語項構造
 解析の研究に従事.言語処理学会,情報処理学会,人工知能学会,ACL各会員.}

 \bioauthor{河原 大輔}{1997年京都大学工学部電気工学第二学科卒業.1999年同
 大学院修士課程修了.2002年同大学院博士課程単位取得認定退学.東京大学大学
 院情報理工学系研究科学術研究支援員,独立行政法人情報通信研究機構研究員,
 同主任研究員を経て,2010年より京都大学大学院情報学研究科准教授.自然言語
 処理,知識処理の研究に従事.博士(情報学).情報処理学会,言語処理学会,人
 工知能学会,電子情報通信学会,ACL各会員.}

 \bioauthor{黒橋 禎夫}{1994年京都大学大学院工学研究科電気工学第二専攻博士
 課程修了.博士(工学).2006年4月より京都大学大学院情報学研究科教授.自
 然言語処理,知識情報処理の研究に従事.言語処理学会10周年記念論文賞,同
 20周年記念論文賞,第8回船井情報科学振興賞,2009 IBM Faculty Award等を受
 賞.2014年より日本学術会議連携会員.}

 \bioauthor{奥村  学}{1962年生.1984年東京工業大学工学部情報工学科卒
 業.1989年同大学院博士課程修了.同年,東京工業大学工学部情報工学科助
 手.1992年北陸先端科学技術大学院大学情報科学研究科助教授,2000年東京工業
 大学精密工学研究所助教授, 2009年同教授,現在に至る.工学博士.自然言語
 処理,知的情報提示技術,語学学習支援,テキスト評価分析,テキストマイニン
 グに関する研究に従事.情報処理学会,電子情報通信学会,人工知能学会,
 AAAI,言語処理学会,ACL, 認知科学会,計量国語学会各会員.}
\end{biography}


\biodate



\end{document}
