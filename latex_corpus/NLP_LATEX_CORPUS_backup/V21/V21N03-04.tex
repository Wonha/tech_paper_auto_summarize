    \documentclass[english]{jnlp_1.4}

\usepackage{jnlpbbl_1.3}
\usepackage[dvipdfm]{graphicx}
\usepackage{amsmath}
\usepackage[multi]{otf}

    \usepackage[small,hang,labelsep=quad]{caption}
\usepackage{subcaption}


\Volume{21}
\Number{3}
\Month{June}
\Year{2014}

\received{2013}{12}{10}
\revised{2014}{2}{18}
\accepted{2014}{4}{8}

\setcounter{page}{485}

\etitle{Unlabeled Dependency Parsing Based Pre-reordering \\
	for Chinese-to-Japanese SMT}
\eauthor{Dan Han\affiref{Author_1} \and Pascual Mart\'inez-G\'omez\affiref{Author_2} \and Yusuke Miyao\affiref{Author_1} \and \\
	 Katsuhito Sudoh\affiref{Author_3} \and Masaaki Nagata\affiref{Author_3}} 
\eabstract{
In statistical machine translation, Chinese and Japanese is a well-known 
long-distance language pair that causes difficulties to word alignment techniques. 
Pre-reordering methods have been proven efficient and effective; however, they 
need reliable parsers to extract the syntactic structure of the source sentences. 
On one hand, we propose a framework in which only part-of-speech (POS) tags and 
unlabeled dependency parse trees are used to minimize the influence of parse 
errors, and linguistic knowledge on structural difference is encoded in the form 
of reordering rules. We show significant improvements in translation quality of 
sentences in the news domain over state-of-the-art reordering methods. On the 
other hand, we explore the relationship between dependency parsing and our 
pre-reordering method from two aspects: POS tags and dependencies. We observe 
the effects of different parse errors on reordering performance by combining 
empirical and descriptive approaches. In the empirical approach, we quantify the 
distribution of general parse errors along with reordering quality. In the descriptive 
approach, we extract seven influential error patterns and examine their correlations 
with reordering errors.
}
\ekeywords{Statistical Machine Translation, Chinese-to-Japanese, Dependency Parsing, Error Analysis}

\headauthor{Han et al.}
\headtitle{Unlabeled Dependency Parsing Based Pre-reordering for C2J SMT}

\affilabel{Author_1}{}{The Graduate University For Advanced Studies, National Institute of Informatics}
\affilabel{Author_2}{}{Ochanomizu University, National Institute of Informatics}
\affilabel{Author_3}{}{NTT Communication Science Laboratories, NTT Corporation}


\begin{document}

\maketitle


\section{Introduction}
\label{sec:introduction}

The statistical machine translation (SMT) community has developed methods to 
obtain useful translations in some domains. Language pairs that have provided 
the highest translation quality are languages with similar sentence structure, 
such as English, French, and Spanish. Such languages are typical subject-verb-object (SVO) 
languages, and unsupervised word alignment methods \cite{Brown1993,Yamada2001,Zens2002,Koehn2003} 
have performed reasonably well. Although these languages have similar sentence 
structures, they may show slight local differences in word order, making word 
alignments nonmonotonic. There have been efforts to address the issue of local 
nonmonotonic word alignments; lexicalized reordering models are an example \cite{Tillmann2004,Koehn2005,Galley2008}.

However, language pairs such as English-Japanese and Chinese-Japanese have larger 
differences in sentence structure. Japanese is a typical subject-object-verb (SOV) 
language, and the translation quality usually drops when translating between SVO 
and SOV languages. The main reason behind this drop is the difficulty in finding 
word-to-word alignments between words in sentences from these language pairs. 
Methods addressing local nonmonotonic word alignments have proven ineffective 
because words in SVO sentences usually should be aligned to those in SOV languages 
that are in a very different position in the sentence.

Estimating the likelihood of all possible word-to-word alignments in a bilingual 
sentence pair is a combinatorial problem, and its complete exploration is unfeasible 
for medium-length sentences. Therefore, huge computational cost when reordering 
during decoding is a difficulty faced by SMT. Moreover, the introduction of weak 
constraints by existing language-independent reordering models is another difficulty. 
As will be discussed later in greater detail, using knowledge of structural differences 
between SVO and SOV languages can contribute to overcome these problems. A popular 
approach is to extract the syntactic structure of sentences from the source language, 
and reorder the source words to imitate the word order of the target language. This 
strategy is called pre-reordering; parsers play an important role in extracting 
the syntactic structure of source sentences in this strategy.

There have been important advances in syntactic parsers, and different types of 
parsing technologies have been developed for several languages such as English and 
Chinese. In general, two types of parsers have been used to pre-reorder sentences 
in SMT: those following the paradigm of head-driven phrase structure grammar (HPSG) \cite{HPSG1994} 
and dependency grammars. Both types of parsers infer the structure of sentences, 
but they can recover different information from the sentence, such as phrase 
constituents or dependency relations between words.

Despite these considerable advances in parsing technology, current parsers are 
still not perfect and may produce some errors with a certain frequency. The issue 
of parse errors in syntax-based pre-reordering is crucial, as it may affect the 
performance of reordering methods and the overall translation quality. Although 
there has been extensive research in pre-reordering methods for SMT, there has 
been little focus on the influence of parse errors to pre-reordering performance 
and overall machine translation quality.

In this study, we propose a novel pre-reordering framework with a hybrid approach 
for SMT from Chinese (SVO) to Japanese (SOV). We use only part-of-speech (POS) 
tags and word dependencies produced by an unlabeled dependency parser to determine 
sentence reordering for releasing pre-reordering from constraints of 
phrase structure grammar based parsing and reducing the effects of parse errors 
on it. Moreover, since in our framework reorderings do not only occur inside chunks, 
our method is more flexible and easily addresses the typical long-distance reordering 
of speech used in news reports. The evaluation results show superior performance 
of our method compared to a baseline system.

Related studies are discussed in Section 2 and a detailed description of our 
pre-reordering framework is included in Section 3. In Section 4, we evaluate the 
performance of this reordering method and compare it with state-of-the-art 
pre-reordering methods on the same corpora. Although our method is designed to be
robust against parse errors, it is meaningful to quantitatively and qualitatively analyze 
the influence of parse errors on this reordering method; this is discussed in 
Section 5. We conclude this study in Section 6 with a summary of our findings.


\section{Related Work}
\label{sec:rw}

Our study is not the first attempt to use pre-reordering methods for SMT.
There are many studies on pre-reordering methods for translation between several languages into English 
or vice versa \cite{FX2004,NH2007,CW2007,CHL2007,XCW2011}.
However, as a representative of long-distance language pairs, translation from
Chinese to Japanese has received little attention. Few previous studies
focused on translation of this language pair, but used English as a pivot language \cite{Wu2007,Tsunakawa2009}.

The most relevant study to ours is by \citeA{DH2012}, in which the authors 
refined an existing reordering method (HFC) proposed by \citeA{HI2010}. 
These reordering methods are based on HPSG \cite{HPSG1994}. 
They used HPSG parsers \cite{YM2008,KY2011} to extract the structure of 
sentences as binary trees and swap the head branches and their dependents 
according to certain heuristics. However, since Chinese has a loose word order, 
the binary structure of the parse trees imposes hard constraints in reordering. \shortciteA{DH2012} 
also noted that reordering strategies that are derived from the HPSG theory 
may not perform well when the head definition is inconsistent in the language 
pair. As an exemplifying phenomenon, the word ``等 (Et cetera)'' would be the dependent 
of its constituent in Chinese, but it would be the head of its constituent in Japanese.

\citeA{PX2009} aimed at reordering sentences from English 
to resemble the order of SOV languages by using dependency trees, dependency 
labels, POS tags, and weights in order to perform local and global reorderings. 
It performed well when translating English to other languages, and the idea of 
using dependency trees instead of head-driven phrase trees is also advantageous 
to reorder Chinese due to the simplicity and robustness of dependency trees. 
However, this method has some limitations that arise from the need to use dependency 
labels in Chinese, because those labels are complex and error-prone.

It is well known that syntax-based pre-reordering methods suffer from parse errors, 
regardless of whether handcrafted reordering rules \shortcite{CW2007,DH2012} or 
automatically extracted rewriting patterns \shortcite{FX2004,NH2007,CHL2007,XCW2011} are 
applied.
In our method, we use a more straightforward approach that only uses information on
POS tags and dependency relations to move all Vs (verbs) in SVO to the right-hand side of all 
their objects to resemble the SOV word order.
Thus, our method is not constrained by the binary structure of constituent trees
nor influenced by errors in dependency labels (as our method does not use them).
Furthermore, unlike 
previous methods that forbid reordering from crossing punctuations and coordination 
structures, our method does not prevent it, which is an important requirement for 
long-distance reordering of reported speech, especially in the news domain.

Although parse errors are influential on linguistically motivated pre-reordering 
methods, there have not been studies related to observing the relationship between 
parse errors and reordering performance. \shortciteA{CQ2006} observed the impact 
of parsing accuracy on an SMT system. They presented a general idea of the sensitivity 
of syntax-based SMT models. However, they did not further analyze concrete parse 
error types that affect task accuracy. \shortciteA{NG2011} explored the effects 
of noun phrase bracketing in dependency parsing in English, and examined English 
to Czech SMT. However, the study focused on using noun phrase structures to improve 
a machine translation framework. In another study, \citeA{JK2011} proposed 
a method to train parsers with the objective to maximize an extrinsic metric, namely reordering performance.
However, they did not study the direct relationship between the reordering quality and
parse quality. Other studies have solely focused on parse error analysis \cite{RM2007,MD2007,TH2009,KY2011},
regardless of machine translation.


\section{Unlabeled Dependency Parsing based Pre-reordering (DPC)}
\label{sec:met}

Since the objective is to reorder words in Chinese to resemble the word order of 
Japanese, namely from an SVO language to an SOV language, an intuitive way is to 
move the V in SVO to the right-hand side of the O to produce SOV. Therefore, our method 
consists of moving verbs to the right-hand side of their objects. For this purpose, 
there are two prominent challenges: identifying the appropriate verbs to 
trigger reordering and determining the precise reordering destination. 
Nonetheless, based on linguistic studies and findings on conventional methods, 
in most sentences, it is obvious that the reordering candidate V not only is the verb 
but also may include surrounding words, such as particles, adverbs, or other 
verbs in coordination structures. Accordingly, we refer to them as a verbal block 
(Vb) in our pre-reordering method and the combination of POS tags and dependencies 
produced by an unlabeled dependency parser can be used to identify a Vb. Consequently, 
the first step of the pre-reordering framework is to correctly identify the verb 
and its companion words that need to be reordered together. 
After that, we need to identify the Vb's object in order to orientate the reordering. 
Since object is the dependent of the Vb, certain types of right-most dependent (RM-D) 
words (such as nouns and pronouns) encode the potential position of the Vb. Then, the second 
step is to move the Vb to the right-hand side of its RM-D. In the case of multiple 
objects for a Vb, the RM-D should be obtained from the right-most object.
There are some other Chinese particles that need to be relocated when
changing the order of Chinese words to resemble the word order of Japanese sentences; these particles will be 
reordered relative to the verbs they accompany. In this study, we follow the POS tag guideline of 
the Penn Chinese Treebank v3.0 \shortcite{POS2000}. Table~\ref{tab:posdef} in 
Appendix~\ref{apx:posdef} presents the detailed definitions.

Figure~\ref{fig:vbtree} demonstrates the reordering procedure with an example. 
In Figure~\ref{fig:vbtreeB}, the outer box indicates the entire Vb of the sentence 
including five words ``\UTFC{5DF2}\UTFC{7ECF} (has already) \UTFC{7F16}\UTFC{8F91} (edit) 和 (and) 出版 (publish) 
了 (-ed)'' that must be reordered. According to the parse tree, since 
the Vb's object is ``一 (a) 本 \UTFC{4E66} (book)'' and ``\UTFC{4E66} (book)'' is the RM-D, the Vb is moved to its
right-hand side, as shown in Figure~\ref{fig:vbtreeC}. The word order becomes an SOV as in Japanese.

It is well known that linguistically motivated pre-reordering methods are affected by
parse errors. To reduce their effects, we minimize the use of parsing information 
to trigger reordering by only considering POS tags and unlabeled dependency structures. 
POS tags are used to categorize words (Table~\ref{tab:dpc}) and the dependency 
structure provides clues for identifying Vbs and RM-D.

\begin{figure}[p]
\begin{subfigure}{1\textwidth}
\begin{center}
\includegraphics{21-3ia1004f1a.eps}
\end{center}
\caption{Original dependency tree}
\label{fig:vbtreeA}
\end{subfigure}
\vspace{0.5\Cvs}

\begin{subfigure}{1\textwidth}
\begin{center}
\includegraphics{21-3ia1004f1b.eps}
\end{center}
\caption{Vbs in rectangular boxes}
\label{fig:vbtreeB}
\end{subfigure}
\vspace{0.5\Cvs}

\begin{subfigure}{1\textwidth}
\begin{center}
\includegraphics{21-3ia1004f1c.eps}
\end{center}
\caption{Merged and reordered Vb}
\label{fig:vbtreeC}
\end{subfigure}
\caption{Example of unlabeled dependency parsing based Chinese pre-reordering 
method (DPC) shows how to detect and reorder a Vb in a 
Chinese SVO sentence. In each subfigure, Chinese Pinyin, Chinese token, 
and token-to-token English translations are listed in three lines.
Figure~\ref{fig:vbtreeA} presents the original dependency tree.
Figure~\ref{fig:vbtreeB} shows nested Vbs in two rectangular 
boxes, and Figure~\ref{fig:vbtreeC} shows the reordered sentence and word 
alignment between a reordered Chinese sentence and its Japanese counterpart.}
\label{fig:vbtree}
\end{figure}

\begin{table}[t]
\caption{List of Chinese POS tags (Xia 2000) for identifying words as different candidates}
\label{tab:dpc}
\vspace{8pt}
\input{1004table01.txt}
\vspace{4pt}
\small
For completeness, the detailed description of the Chinese POS tags can be found in the appendix.
\par
\end{table}

{\bf Identifying verbal block (Vb)}\quad Verbal blocks comprise a head (Vb-H)
and possibly accompanying dependents (Vb-D). For the example in Figure~\ref{fig:vbtreeB},
for the outer Vb, ``\UTFC{7F16}\UTFC{8F91} (edit)'' is the main verb, and therefore Vb-H. 
Other dependents, such as the adverb ``\UTFC{5DF2}\UTFC{7ECF} (has already)'' that add a preterit 
tense to the verb, is considered as a Vb-D. As for the inner Vb, ``出版 (publish)''
is the Vb-H while the aspect particle ``了 (-ed)'' is its Vb-D. The Vb-H entry 
in Table~\ref{tab:dpc} lists all POS tags of words that can be Vb-H candidates.
These POS tags indicate that Vb-H candidates consist of verbs (V*) and prepositions (P)\footnote{Similarly to verbs, prepositions
in Chinese appear to the left of their dependents, while prepositions in Japanese appear to the right.
Figure~\ref{fig:rs} illustrates an example of the preposition ``随着 (with)'' being considered as a Vb-H candidate.}.
There are three conditions for a word to be considered as a Vb-H:

\begin{itemize}
\item Its POS tag is in the set of Vb-H in Table~\ref{tab:dpc}. 
\item It is a dependency head, indicating that it may have an object as a dependent.
\item It has no dependent whose POS tag is in the set of BEI in Table~\ref{tab:dpc}.
BEI particles indicate that the verb is in the passive voice and
should not be reordered since it already resembles the Japanese order. 
\end{itemize}

A bei-construction is a special structure that is commonly used to create
passive voice in Chinese sentences.
In order to compensate the lack of verb inflection in Chinese,
particles are introduced to indicate the occurrence of a passive voice. 
These particles have a POS tag LB or SB, and are dependents of the verb.
In Chinese, bei-constructions follow the OV word order,
which is the same as the Japanese word order.
That is, the verb is on the right-hand side of its object.
For this reason, reordering is not required and we exclude Vb-H candidates that are involved in 
a bei-construction. Figure~\ref{fig:bei} illustrates this linguistic phenomenon.
In the Chinese sentence, the main verb ``批\UTFC{8BC4} (criticize)''
is already on the right-hand side of its object, ``学生 (student)'' in this case.
This word order is the same as in the Japanese sentence.
For this reason, no reorder is necessary between the main verb and its object,
which motivates our condition on bei-constructions presented above.

\begin{figure}[t]
\begin{center}
\includegraphics{21-3ia1004f2.eps}
\end{center}
\caption{Example of bei-construction}
\label{fig:bei}
\end{figure}

Since Chinese does not have inflection, conjugation, or case markers~\cite{Li1989},
adverbs (AD), aspect particles (AS) or sentence-final particles (SP) are used 
to signal modality, and indicate grammatical tense or add aspectual value to verbs.
Coordinating conjunctions (CC) connect multiple verbs. They should thus be 
reordered together with verbs as part of the Vb, and we refer to them as Vb-D. 
Similarly, to be a Vb-D, there are three necessary conditions:

\begin{itemize}
\item Its POS tag is in the Vb-D entry in Table~\ref{tab:dpc}. 
\item It is a dependent of a word that is already in the Vb. 
\item It is next to its dependency head or only a coordination conjunction is in between them.
\end{itemize}

Accordingly, to build a Vb, we first examine words and confirm 
the Vb-H to initialize the Vb. Then, we test the Vb-D conditions on words adjacent 
to Vb-H and extend the Vb by including qualified ones. This process is iteratively 
applied to adjacent words of the Vb, possibly nesting other Vbs if necessary. 
Therefore, in Figure~\ref{fig:vbtreeB}, according to the definition introduced 
above, ``\UTFC{7F16}\UTFC{8F91} (edit)'' and ``出版 (publish)'' are both Vb-Hs that form two separate 
incipient Vbs simultaneously. By analyzing their adjacent words which are their 
dependents, the words ``\UTFC{7F16}\UTFC{8F91} (has already)'', ``和 (and)'', ``出版 (publish)'' are 
qualified Vb-Ds for the Vb started by ``\UTFC{7F16}\UTFC{8F91} (edit)'', and the word ``了 (-ed)'' 
is the only qualified Vb-D for the Vb started by ``出版 (publish)''. We thus 
obtain a nested Vb and merge nested Vbs at the end, as shown in Figure~\ref{fig:vbtreeC}. 

{\bf Identifying right-most dependent (RM-D)}\quad Objects may be a single word 
(e.g., noun, pronoun) or comprise noun phrases or clauses \cite{Downing2006} 
such as nominal groups, finite clauses (e.g., \emph{that} clauses, \emph{wh-}clauses), 
and non-finite clauses (e.g., \emph{-ing} clauses). It is a difficult task to recognize
objects in a sentence. However, since an object is the dependent of a verb and our objective 
is to move V to the right-hand side of O in an SVO language, we simplify our task 
to identify the right-most dependent (RM-D) of the Vb instead of finding the entire object.
We define RM-D as the word:

\begin{itemize}
\item whose POS tag is in the RM-D entry of Table~\ref{tab:dpc},
\item whose dependency head is inside of the Vb, and
\item belongs to the right-most object among all objects of the Vb.
\end{itemize}

Consequently, in Figure~\ref{fig:vbtree}, the RM-D of the Vb ``\UTFC{7F16}\UTFC{8F91} (has already) 
\UTFC{7F16}\UTFC{8F91} (edit) 和 (and) 出版 (publish) 了 (-ed)'' is ``\UTFC{4E66} (book)''. Two more complex 
examples are shown in Figures~\ref{fig:rmdb} and~\ref{fig:rmda}. In these
figures, alignment lines between the original Chinese sentence and the reordered 
Chinese sentence indicate the reordering trace of Vb, while those 
between the reordered Chinese sentence and its Japanese counterpart show the reordering result.
In Figure~\ref{fig:rmdb}, there is a coordination structure of verb phrases, and 
the dependency tree shows that the first verb ``吃 (eat)'' appears as the dependency 
head of the second verb ``去 (go to)''. Thus, ``午\UTFC{996D} (lunch)'' and ``学校 (school)'' are 
the RM-Ds of ``吃 (eat)'' and ``去 (go to)'', respectively.
Figure~\ref{fig:rmda} illustrates the case of a subordinate clause. Vb of 
``鼓励 (encourage)'' has no dependent that can be considered as its RM-D. Instead, 
an embedded clause appears as its object argument. We thus move Vb of 
``鼓励 (encourage)'' to the end of the sentence. Meanwhile, ``\UTFC{5B9E}践 (practice)'' 
is the RM-D of the Vb of ``参与 (participate)''. In the news domain, reported 
speech is a frequent example that follows this pattern. In our method, if
Vb that comprises the main verb (labeled as ROOT) has dependents but no 
RM-D, we move the Vb to the end of the sentence. 

\begin{figure}[p]
\begin{subfigure}{1\textwidth}
\begin{center}
\includegraphics{21-3ia1004f3a.eps}
\end{center}
\caption{Example of a Chinese sentence with a coordination of verb phrases as the predicate}
\label{fig:rmdb}
\end{subfigure}
\vspace{0.5\Cvs}

\begin{subfigure}{1\textwidth}
\begin{center}
\includegraphics{21-3ia1004f3b.eps}
\end{center}
\caption{Example of a Chinese sentence, where a subordinate clause is an object}
\label{fig:rmda}
\end{subfigure}
\caption{Subject (S), verbs (V), and objects (O) are displayed for main and subordinate 
clauses in the Japanese translation} 
\label{fig:rmd}
\end{figure}


{\bf Identifying particles (Oth-DEP)}\quad To resemble Japanese, the location 
of certain particles in Chinese should be reordered relative to their dependency 
heads. Typically, particles are similar to the bei4(被)particle (LB or SB) and subordinating 
conjunctions (CS). These particles appear on the left-hand side of their dependency 
head in Chinese and the right-hand side in Japanese. Reordering these particles 
in our framework can be summarized as follows:

\begin{enumerate}
\item Find dependents of a head whose POS tags are in the Oth-DEP entry of Table~\ref{tab:dpc}.
\item Move those particles to the direct right-hand side of the head.
\item For multiple such particles, maintain the relative order among them while moving.
\end{enumerate}

Figure~\ref{fig:bei} shows an example of a Chinese sentence that includes the particle
``被'' whose POS tag is SB. This particle aligns to ``られ'' in Japanese, which follows 
its dependency head. Thus, if we reorder it according to our rules described above,
the word alignment between the reordered Chinese sentence and
the corresponding Japanese sentence becomes monotonic.

{\bf Summary:}\quad Based on the definitions above, our dependency parsing based 
pre-reordering framework is summarized by the following steps:

\begin{enumerate}
\item Obtain POS tags and an unlabeled dependency tree of a Chinese sentence.
\item Obtain reordering candidates: Vbs.
\item Obtain the object (RM-D) of each Vb.
\item Reorder each Vb in two exclusive cases by following the order:
\begin{enumerate}
\item If RM-D exists, reorder Vb to be the right-hand side of RM-D.
\item If Vb-H is a ROOT and its RM-D does not exist, reorder Vb to the end of the sentence.
\item If none of the above conditions is met, reordering does not occur. 
\end{enumerate}
\item Reorder Oth-DEPs to the right-hand side of their corresponding Vbs.
\end{enumerate}

\begin{figure}[t]
\begin{center}
\includegraphics{21-3ia1004f4.eps}
\end{center}
\caption{Example of reported speech sentences}
\label{fig:rs}
\end{figure}

Unlike other reordering frameworks for distant language pairs \shortcite{PX2009,HI2010,DH2012},\footnote{
Note that the reordering model of HF \cite{HI2010} prevents words from crossing colons and semicolons,
but it does not prevent reorderings from crossing other types of punctuations such as commas or quotes.
}
we do not prevent chunks from crossing punctuations or coordination structures.
We illustrate an example in Figure~\ref{fig:rs}.
According to our reordering model, ``\UTFC{62A5}道 (report)'' is reordered to the end of the sentence,
regardless of a comma occurring after it.
However, in a previous reordering model for Chinese \shortcite{DH2012},
``\UTFC{62A5}道 (report)'' would not be reordered since they incorporated punctuation exception rules
to block reordering over punctuations.
Therefore, our method allows us to achieve a global inter-clause reordering.


\section{Experiments}
\label{sec:exp}

To evaluate the performance of the pre-reordering method for Chinese 
to Japanese machine translation, we used two corpora from the news domain. We 
collected an in-house Chinese-Japanese corpus of news articles that we call News. 
Then, we augmented this corpus with the CWMT corpus \cite{CWMT2011} to use it as an 
extended training set. We extracted disjoint development and test sets from the News 
corpus containing 1,000 and 2,000 sentences, respectively. 
Table \ref{tab:corpus} shows the corpora statistics.

\begin{table}[b]
\caption{Basic statistics of the corpora}
\label{tab:corpus}
\vspace{8pt}
\input{1004table02.txt}
\vspace{4pt}
\small
News Devel. and News Test were used to tune and 
test the systems trained with both training corpora. Data statistics were collected 
after tokenizing and filtering sentences longer than 64 tokens.
\par
\end{table}

We first segmented Japanese and Chinese sentences using 
MeCab\footnote{http://mecab.googlecode.com/svn/trunk/mecab/doc/index.html} \cite{TK2000}
and the Stanford Chinese segmenter\footnote{http://nlp.stanford.edu/software/segmenter.shtml} \cite{PCC2008}, respectively.
POS tags of Chinese words were automatically extracted from the results of Berkeley 
parser\footnote{http://nlp.cs.berkeley.edu/Software.shtml} \cite{SP2006}.
Unlabeled dependency trees were constructed using Corbit \cite{Corbit2011}.
In addition to the baseline system, we re-implemented the HFC \cite{DH2012} 
using the Chinese Enju\footnote{http://www.nactem.ac.uk/enju/} \cite{KY2011} to obtain HPSG parse trees for comparison.

Previous studies on translating English to Japanese \cite{HI2010}
used artificial particle insertions in the source language to align them with
Japanese particles that do not have a counterpart on the English side.
In Chinese to Japanese translation, Chinese counterparts of Japanese particles are not present.
Since the method for particle seed word insertion in \cite{Isozaki2012} proved to be effective,
we have adapted the same method to HFC and DPC.
Therefore, we can not entirely disclaim the use of constituent parsing at 
the current stage. However, we believe that dependency parsing can provide sufficient
information for inserting particles and this is considered as one of our
future works.

The described pre-reordering methods were applied at a pre-processing stage
to increase the chances of monotonic alignments at a later stage.
The standard Moses baseline was used \cite{MOSES2007}, where reordered 
Chinese sentences were paired with their Japanese counterparts and word-to-word 
alignments were estimated using MGIZA++ \cite{MGIZA2008}.
Following standard practice, we used the default distance reordering model
and ``msd-bidirectional-fe'' as a lexicalized reordering model \shortcite{Koehn2005,Galley2008}
with the default distortion limit of $6$.
We estimated a $5$-gram language model using SRILM \shortcite{SRILM2002} on 
the target side of the corresponding training corpus.
Then, the scaling factors of the log-linear combination of models were estimated
using MERT \shortcite{MERT2003}.

\begin{table}[b]
\caption{Evaluation of translation quality of two test sets when CWMT, News, and the 
combination of both corpora were used for training}
\label{tab:results}
\vspace{8pt}
\input{1004table03.txt}
\end{table}

Translation quality can be evaluated from different perspectives, and two metrics 
have been used in literature for this purpose. BLEU \cite{BLEU2002} is 
a widely used metric that computes the geometric average between uni-gram, bi-gram, 
tri-gram, and four-gram precisions, and emphasizes the importance of local fluency.
When translating between language pairs with different sentence structure, such 
as Chinese and Japanese, long-distance phrase mis-reorderings are a major source 
of translation quality drop. RIBES \cite{RIBES2010} is a score based on rank 
correlations of word orders that was specifically designed to evaluate translation 
quality between distant language pairs.
BLEU and RIBES are precision metrics, and higher scores suggest higher performance.
Table~\ref{tab:results} summarizes the results of the baseline system (no pre-reordering 
or particle word insertion), baseline system plus particle insertion ($\text{Base.} + \text{Part.}$), 
HFC, and our DPC method.

As shown, DPC obtains approximately $0.7$ BLEU points of improvement and $0.3$ 
or $0.2$ RIBES points when compared to the second-best system in both corpora.
We suspect that corpus diversity is a reason why HFC does not show any advantage in this setting. 
By comparing the baseline system and the baseline system with pseudo particles
inserted, we found that without reordering, particle insertion did not influence 
much on the results, especially after increasing the size of the training data. We suspect 
that the inserted pseudo particles are also suffering long distant word alignment. 
For instance, we insert a pseudo particle for ``を'' after the object. 
However, since objects are at the end of Chinese sentences, these pseudo particles
are also at the end of sentences, far from their Japanese counterparts.
We tested the significance of BLEU improvement for HFC and DPC when compared to 
the baseline phrase-based system. HFC tests obtained p-values of $0.355$ and $0.135$ 
on News and CWMT + News corpora, while our proposed DPC method obtained p-values of
$0.002$ and $0.0$, which indicate significant improvements over the phrase-based 
system.


\section{Analysis on Causes of Reordering Errors}
\label{sec:dis}

In the previous section, we assessed the impact of reordering performance on machine 
translation quality. In this section, we assess the influence 
of parse errors on reordering quality. Because DPC is a parsing based reordering method, 
to reveal the relationship between specific parse errors and reordering issues, 
a straightforward method is to contrast the reordering based on error-free parse trees 
(Gold-Trees) with the reordering based on automatic parse trees (Auto-Trees) that were obtained using Corbit.
We refer to the former as Gold-DPC and the 
latter as Auto-DPC. We first identify the most influential parse errors by using POS 
tags, and then define concrete parse errors by comparing Gold-Trees with Auto-Trees 
to quantify the effects of certain parse errors from the aspect of dependencies, 
since the syntactic information that guides reordering in DPC is limited 
to these two linguistic categories.

The value of Kendall's tau ($\tau$) rank correlation coefficient \cite{Kendall1938} is used
to measure word order similarities between reordered Chinese sentences and their Japanese counterparts, for both
Gold-DPC and Auto-DPC. We use Equation~\ref{eq:tau}
introduced in \shortcite{HI2010} that uses the alignment file ch-ja.A3.final
generated by MGIZA++. In ch-ja.A3.final, parallel sentence pairs (Chinese and Japanese)
are aligned to each other as follows:

\begin{tabular}{ll}
Chinese: & 我 去 \UTFC{4E1C}京 和 横\UTFC{6EE8} 。\\
Japanese: & NULL ({ }) 私 ({ 1 }) は ({ }) 東京 ({ 3 }) と ({ 4 }) 横浜 ({ 5 }) へ ({ }) 行く ({ 2 }) 。 ({ 6 }) 
\end{tabular}
\begin{equation}
\mathbf{\tau} = \frac{\# \mathrm{of~concordant~pairs}}{\# \mathrm{of~all~pairs}}\times 2 - 1
\label{eq:tau}
\end{equation}

The alignment order of the previous example is ``1 3 4 5 2 6'', where the total 
number of position pairs is $6\choose 2$. Pairs of the form ``1 3'', ``1 4'', 
or ``4 5'' are concordant pairs, since the value of the first position is lower 
than the second one. Pairs of the form ``3 2'' or ``5 2'' are not concordant, 
since the first position is greater than the second one. Therefore, the $\tau$ 
value of this Chinese sentence is $11/{6 \choose 2} \times 2 - 1 \approx 0.47$.


\subsection{Gold Data}

Human annotated sentences from Chinese Penn Treebank ver. 7.0 (CTB-7) are considered 
as the source of building up gold data. CTB-7 is a widely used corpus that comprises 
parsed sentences in Chinese from five genres: broadcast news (BN), broadcast 
conversations (BC), news magazines (NM), newswire (NS), and newsgroup weblogs 
(NW). We divide the corpus in the same manner as that introduced in \cite{YW2011} 
and use the development set to obtain the gold data. 
However, sentences from BC and NW tend to have faults such as repetitions, incomplete 
sentences, corrections, and incorrect sentence segmentation. We thus randomly 
sampled $2,643$ unique sentences from all five genres in the development set, 
but with different numbers of sentences in each genre. Table~\ref{tab:ctb7} shows 
the statistics of all selected sentences.
In order to automatically align reordered Chinese sentences with their Japanese translation,
we need the Japanese translations of the Chinese sentences of our gold data set.
Since there is not any official Japanese translation for sentences from CTB-7,
professional translators manually translated these Chinese sentences into Japanese.

\begin{table}[b]
\caption{Statistical characteristics of gold parsed and reordered corpus} 
\label{tab:ctb7}
\vspace{8pt}
\input{1004table04.txt}
\end{table}

After obtaining the gold data set, it is necessary to convert the CTB-7 parsed text 
to dependency Gold-Trees. An open utility Penn2Malt\footnote{http://stp.lingfil.uu.se/{\textasciitilde}nivre/research/Penn2Malt.html} 
is used to convert Penn trees into MaltTab format containing dependency information. 
However, CTB-$7$ contains three new phrase labels that are not covered in the head rules of Penn2Malt.
These new phrase labels correspond to fillers (FLR), disfluencies (DFL), and incomplete sentences (INC).
For this reason, we introduced three new rules to identify heads of FLR, DFL and INC phrases.
For FLR and DFL phrases, we select the right-most branch as a head.
For INC phrases, we follow the same rule as FRAG phrases (which are already defined in Penn2Malt tool).
The complete list of rules of Penn2Malt tool can be found in Table~\ref{tab:hrdef},
together with our newly introduced rules (at the bottom of the table).


\subsection{Dependency Parse Errors by Part-of-Speech}
\label{subsec:pos}

\begin{figure}[b]
\begin{subfigure}{1\textwidth}
\begin{center}
\includegraphics{21-3ia1004f5a.eps}
\end{center}
\caption{Gold dependency tree}
\label{fig:errex}
\end{subfigure}
\vspace{0.5\Cvs}

\begin{subfigure}{1\textwidth}
\begin{center}
\includegraphics{21-3ia1004f5b.eps}
\end{center}
\caption{Possible wrong dependency parse tree}
\label{fig:poserr}
\end{subfigure}
\caption{Example for calculating parsing errors in terms of POS tag}
\label{fig:poserror}
\end{figure}


We construct two types of parse errors in terms of POS tags to quantify the most influential
parse errors in dependency parse trees: dependent error, where a word points to a wrong head, 
and head error, where a word is wrongly recognized as a head for other words. We collect and calculate the 
proportions of POS tags from those words. For example, Figure~\ref{fig:poserr} 
presents a possible wrong parse tree of the example shown in Figure~\ref{fig:errex}.
By comparing with the Gold-Tree in Figure~\ref{fig:errex}, Chinese characters 
(English word, POS tag) of ``他 (he, PN)'', ``去 (went, VV)'', ``\UTFC{4E66}店 (bookstore, NN)'', 
``\UTFC{4E70} (buy, VV)'', and ``。(., PU)'' in the dependency tree in Figure~\ref{fig:poserr}
point to different wrong heads, and are dependent errors. Meanwhile, Chinese 
characters (English word, POS tag) of ``去 (went, VV)'' and ``\UTFC{4E70} (buy, VV)'' are 
wrongly recognized as heads for other words(e.g., ``他 (he)'', ``\UTFC{4E66}店 (bookstore)''),
and are head errors. Therefore, every head error has at least one corresponding 
dependent error, whereas a dependent error may correspond with no head error as in 
the case where a word is not the root in a Gold-Tree but is a root in the wrong tree.
An example is the dependent-error ``去 (went, VV)'' in Figure~\ref{fig:poserr}.

We use an $\tau$ value of Gold-DPC and Auto-DPC within the range of $0.1$ to group 
sentences. For example, all sentences that gained $\tau$ in the set of $[0.7, 0.8)$ 
are in one group. In each group, we collect all POS tags that are involved with 
these two types of errors separately and calculate the frequency for each POS tag.
Figure~\ref{fig:poserrortau} presents dependency parse errors and head errors segregated
by three and two POS tags of highest frequency, respectively;
the rest of the POS tags
are considered negligible.\footnote{Although the range of $\tau$ value is [-1, 1], 
there were no reordered sentences that fell into the interval [-1, 0.1) in our analysis
results. For this reason, we only show the region of [0.1, 1] in all the figures.}

\begin{figure}[b]
\begin{minipage}[t]{196pt}
\includegraphics{21-3ia1004f6a.eps}
\subcaption{Distribution of dependency errors segregated by the three most frequent POS tags and their trends}
\label{fig:deptau}
\end{minipage}
\hfill
\begin{minipage}[t]{196pt}
\includegraphics{21-3ia1004f6b.eps}
\subcaption{Distribution of head errors segregated by the two most frequent POS tags and their trends}
\label{fig:headtau}
\end{minipage}
\caption{Distribution of dependency and head errors segregated by the two and three most frequent POS tags, respectively}
\label{fig:poserrortau}
\end{figure}

Figure~\ref{fig:deptau} shows the distribution of dependency parse errors segregated by
the top three most frequent POS tags.
That is, ``VV'', ``PU'', and ``NN'' are the three most frequent POS tag of words that 
point to a wrong head in Auto-Trees. VV represents all verbs except predicative 
adjectives (VA), copula (VC), and you3(有)\footnote{A Chinese character expresses 
possession and existence.} as the main verbs (VE). PU represents punctuation and 
NN represents all nouns except proper nouns (NR), temporal nouns (NT), and those 
for locations that cannot modify verb phrases with or without de0(地)\footnote{A 
Chinese character is specially used to connect the verb phrase and its modifier.}. 
The dependent error on VV accounts for a larger proportion in low reordering accuracy 
sentences, revealing that the case of a VV pointing to a wrong head tends to 
largely reduce the reordering performance.

Figure~\ref{fig:headtau} shows the distribution of head errors segregated by the most frequent POS tags.
As expected, ``VV'' and ``NN'' are the two most frequent POS tag of words that have been 
wrongly recognized as heads in Auto-Trees. Similarly to Figure~\ref{fig:deptau}, 
VV appears to be dominant in the groups of sentences that display low reordering quality,
whereas head errors that involve NN POS tags do not have an impact on reordering quality in general.


\subsection{Dependency Parse Errors by Dependency Structure}
\label{subsec:dp}

The analysis on parse errors in terms of POS tags provides the general idea that parse
errors on verbs are primary in low reordering performance.
However, such analysis does not take into account the role of those verbs
(or other words with different POS tags) in the structure of the sentence.
We hence follow a descriptive approach to analyze dependency parse errors
in terms of their structural role (whether errors occur on heads or dependents),
to further explore the effects of more concrete parse errors.

As introduced in Section~\ref{sec:met}, DPC first identifies Vb and RM-D, and 
then reorders necessary words.
The correct identification of Vbs and RM-Ds is essential for pre-reordering performance.
For this reason, parsing errors that lead to misrecognitions of dependencies
may impact on pre-reordering performance if such misrecognitions
affect the identification of Vbs or RM-Ds.
To discover the effects of more concrete parse 
errors, we define seven parse error pattens that will mislead reordering
and classify those errors into three categories of dependencies, i.e., {\bf ROOT}, {\bf RM-D}, 
and {\bf BEI}. ROOT denotes whether the Vb-H candidate\footnote{Note that a Vb-H candidate is not an authorized Vb-H.
A word is a Vb-H candidate if its POS tag appears in the Vb-H entry of Table~\ref{tab:dpc}.
However, a Vb-H candidate needs to meet two more conditions to become an actual Vb-H, namely
1) to be a dependency head, and 2) not to be involved in a bei-construction.}
is the root of the sentence, 
RM-D is the right-most dependent that belongs to the object of the Vb-H candidate if it has 
one, and BEI denotes whether the Vb-H candidate is involved in a bei-construction. 

\begin{figure}[t]
\begin{subfigure}{1\textwidth}
\begin{center}
\includegraphics{21-3ia1004f7a.eps}
\end{center}
\caption{Gold dependency tree}
\label{fig:rmddrootg}
\end{subfigure}
\vspace{0.5\Cvs}

\begin{subfigure}{1\textwidth}
\begin{center}
\includegraphics{21-3ia1004f7b.eps}
\end{center}
\caption{Possible wrong dependency parse tree}
\label{fig:rmddrootc}
\end{subfigure}
\caption{Example for parse error patterns of Root-C and RM\_D-D}
\label{fig:rmddrootgc}
\end{figure}

According to the methodology of DPC in Section~\ref{sec:met}, a Vb-H candidate
will be reordered to the end of the sentence if it meets the following three conditions,
1) it is the ROOT in the parse tree; 2) it is not
involved in a bei-construction; and 3) its RM-D does not exist.
In the ROOT parse error, there are
two types of parse error patterns, namely Root-C and Root-G. Root-C denotes the case
where a Vb-H candidate matches the preceding three conditions in the automatically parsed tree (Auto-Tree), but not
in the Gold-Tree. In such case, the Vb-H candidate would be reordered if the input
were an Auto-Tree, but it would not be reordered if we had a Gold-Tree. Root-G is the opposite
case of Root-C, where a Vb-H candidate would be reordered if we had a Gold-Tree,
but it would not be reordered based on an Auto-Tree.
Figure~\ref{fig:rmddrootgc} illustrates an example of error pattern Root-C. The Vb-H candidate ``\UTFC{5E94}\UTFC{8BE5} (should)''
is the root in the Auto-Tree of Figure~\ref{fig:rmddrootc} whereas is not in the Gold-Tree
of Figure~\ref{fig:rmddrootg}. Moreover, the Vb-H candidate ``\UTFC{5E94}\UTFC{8BE5} (should)'' does not have any RM-D dependent and is not involved
in a bei-construction in both the Gold-Tree and the Auto-Tree.
For this reason, ``\UTFC{5E94}\UTFC{8BE5} (should)'' will be reordered
incorrectly to the end of the sentence given the Auto-Tree, but will remain in the correct position
given the Gold-Tree.
As an illustration of error pattern Root-G, the Vb-H candidate ``同意 (agree)'' in Figure~\ref{fig:rootgg}
will be reordered to the end of the sentence. However, given an erroneous parse tree
as in Figure~\ref{fig:rootgc}, ``同意 (agree)'' will not be reordered.

\begin{figure}[t]
\begin{subfigure}{1\textwidth}
\begin{center}
\includegraphics{21-3ia1004f8a.eps}
\end{center}
\caption{Gold dependency tree}
\label{fig:rootgg}
\end{subfigure}
\vspace{0.5\Cvs}

\begin{subfigure}{1\textwidth}
\begin{center}
\includegraphics{21-3ia1004f8b.eps}
\end{center}
\caption{Possible wrong dependency parse tree}
\label{fig:rootgc}
\end{subfigure}
\vspace{0.5\Cvs}

\begin{subfigure}{1\textwidth}
\begin{center}
\includegraphics{21-3ia1004f8c.eps}
\end{center}
\caption{Another possible erroneous parse tree}
\label{fig:rmdcc}
\end{subfigure}
\caption{Example for parse error patterns of Root-G and RM\_D-C}
\label{fig:rmdcrootg}
\end{figure}

In the RM-D category, there are three types of parse error patterns: RM\_D-C,
RM\_D-G, and RM\_D-D. RM\_D-C is the case where the RM-D of a Vb-H candidate exists 
in an Auto-Tree but not in a Gold-Tree. In other words, an RM-D candidate was 
parsed wrongly on its head and may lead to incorrect reordering of its 
corresponding Vb-H candidate. 
Figure~\ref{fig:rmdcc} shows another possible erroneous parse of the Gold-Tree 
in Figure~\ref{fig:rootgg}. Based on the wrong parse tree, our algorithm will 
identify ``\UTFC{8BB0}者 (journalist)'' as the RM-D of the Vb-H candidate ``同意 (agree)'' 
and thus ``同意 (agree)'' will be reordered incorrectly to the right-hand side 
of ``\UTFC{8BB0}者 (journalist)'' instead of to the end of the sentence. Similar to the ROOT 
category, RM\_D-G is the opposite case of RM\_D-C in that the RM-D of a Vb-H candidate 
exists in the Gold-Tree instead of the Auto-Tree. The Vb-H candidate ``去 (went)'' 
in the previous parse error example of Figure~\ref{fig:poserror} is an example 
of such a case by the reason of reordering differently according to the Gold-Tree and 
the possible wrong tree. RM\_D-D indicates that the RM-D of a Vb-H candidate exists 
in both the Gold-Tree and the Auto-Tree but is different. For example, the Vb-H candidate 
``加入 (join)'' in Figure~\ref{fig:rmddrootgc} has received different RM-Ds 
in the two trees, causing a reordering problem by comparing the reordered results
(R-Chinese) shown in the figure.

The bei-construction is one necessary constraint to determine a Vb-H (see Section~\ref{sec:met}).
For this reason, incorrect reorderings may occur if the head of a bei-construction
differs between the Auto-Tree and the Gold-Tree.
Therefore, we also defined two types of error patterns for the BEI 
category: BEI-C and BEI-G. BEI-C is the case where a Vb-H candidate received a 
wrong BEI dependent in the Auto-Tree; this will prevent reordering independently 
on whether the Vb-H candidate has an RM-D or is the root. Inversely, BEI-G is the 
case where a Vb-H candidate is missing its BEI dependent in the Auto-Tree.

After defining seven parse error patterns in three categories that influence 
reordering, we use the same analysis method as in Section~\ref{subsec:pos} to 
distribute the average frequency proportions of each type in different $\tau$ 
value groups of sentences.

\begin{figure}[b]
\begin{center}
\includegraphics{21-3ia1004f9.eps}
\end{center}
\caption{Distribution of three types of parse errors in terms of dependency and their trend lines}
\label{fig:error}
\end{figure}

Figure~\ref{fig:error} presents the distribution of the three categories of parse 
errors and their tendencies. ROOT errors show high relativity with low $\tau$ 
value sentences, whereas RM-D errors are dominant in high $\tau$ value 
sentences. This implies that the effects of ROOT errors on reordering are more 
serious than the effects from RM-D errors. The reason could be that ROOT errors 
cause long-distance reordering failure whereas RM-D errors lead to local reordering 
errors. Due to the size limitation of the analysis corpus, there are very few BEI 
errors and it is difficult to completely capture their trends.

Figure~\ref{fig:rootrmderr} shows the correlations between parse error patterns 
and reordering accuracy. In the ROOT errors of Figure~\ref{fig:rooterr}, Root-C errors
occur more frequently than Root-G errors in poorly reordered sentences (low Kendall's tau).
This is consistent with the distribution results 
that are shown in Figure~\ref{fig:rmderr}. The error pattern of RM\_D-G has a higher
percentage than the other two patterns, also implying that a Vb-H candidate 
in an Auto-Tree tends to have less or no dependent as its object.

\begin{figure}[t]
\begin{subfigure}{195pt}
\begin{center}
\includegraphics{21-3ia1004f10a.eps}
\end{center}
\caption{Distribution of different patterns of ROOT errors and their trend lines}
\label{fig:rooterr}
\end{subfigure}
\hfill
\begin{subfigure}{195pt}
\begin{center}
\includegraphics{21-3ia1004f10b.eps}
\end{center}
\caption{Distribution of different patterns of RM-D errors and their trend lines}
\label{fig:rmderr}
\end{subfigure}
\caption{Distribution of ROOT and RM-D errors}
\label{fig:rootrmderr}
\end{figure}


\section{Conclusion}
\label{sec:con}

We have described a novel syntax-based pre-reordering method for Chinese to 
Japanese SMT; the unlabeled dependency parsing based pre-reordering for Chinese (DPC). 
This method uses linguistically motivated rules to reorder Chinese words to 
resemble the word order in Japanese sentences. It relies on parsing technologies 
to extract the syntactic structure of Chinese sentences, and the reordering 
performance might be sensitive to parse errors.

We first evaluated the impact on translation quality of the reordering method 
on typical sentences used in the news domain and compared it with a state-of-the-art 
reordering method (a refined Head finalization reordering method for Chinese). 
DPC performed consistently better than both the baseline and the HFC methods 
in terms of the BLEU and RIBES performance evaluators. 
The significance test also confirmed the improvement obtained from DPC.

We then assessed the influence of parse errors on reordering quality, as measured 
by Kendall's $\tau$ rank correlation. For this purpose, we conducted linguistically 
motivated analysis methods by combining empirical and descriptive approaches 
from two aspects. Moreover, we not only profiled general parse errors but also examined 
the effects of specific parse errors on reordering.

Based on our findings, we believe that future developments of syntax-based pre-reordering 
methods would improve the preliminary analysis of POS tagging and parsing accuracies. 
On one hand, to develop more robust linguistically motivated pre-reordering 
methods, reordering rules could be designed to suppress or reduce unreliable POS tags or 
parsing relations. For example, those systems could be designed to make use of 
N-best lists of certain POS tags or dependencies. 
On the other hand, as for the POS tag or parser development, with the objective 
to aid pre-reordering, we could attempt to maximize the accuracy of POS tags or dependencies 
that are relevant to the reordering task, maybe at the expense of lower accuracies 
on other elements.

Therefore, to improve the robustness of DPC against parse errors is one of 
our future goals. Since the current analysis on exploring influential parse errors 
is not exhaustive, it should be meaningful to investigate the parse error patterns that 
cause unexpected reordering on these words as well as the correlations between them. 
In addition, increasing the number of sentences in the analysis corpus is interesting for further research.

Currently, DPC is designed to use only information on dependency structure and
POS tags that are defined in Chinese Penn Treebank
v3.0. DPC has three limitations. The first one is the ordering problem between
words in different languages that have the same function but different POS tag.
``不 bu4'' in Chinese is an adverb that precedes and modifies its dependency head.
Its English counterpart is ``not'' and its Japanese counterpart is ``ない''.
Both of them modify the dependency heads by following them.
For this reason, ``不 bu4'' should be moved to the right of its head.
However, most English or Japanese words that correspond to other Chinese adverbs behave differently,
and the same reordering rule does not apply to them, rendering ``不 bu4'' as an exception.
Since there is not a special POS tag for ``不 bu4'' and we only use POS tags to distinguish Chinese words,
DPC has a limitation on solving these cases. A more detailed POS tag definition may help on dealing
with this problem. The second limitation is the lack of generalization to other sets of POS tags or definitions.
For example, there are more and different POS tags
defined by the Institute of Computing Technology (ICT) for Chinese,
but our method does not guarantee a good performance given such different set of POS tags.
The third limitation derives from the difficulty in the direct application of
DPC to other language pairs since the definition of POS tags for other languages
may be different. For this reason, we would like to improve our reordering model to
automatically learn reordering rules based on POS tag and dependencies from the data.

\acknowledgment

The authors would like to thank the reviewers for their helpful comments 
and constructive suggestions.

\bibliographystyle{jnlpbbl_1.5}
\begin{thebibliography}{}

\bibitem[\protect\BCAY{Brown, Pietra, Pietra, \BBA\ Mercer}{Brown
  et~al.}{1993}]{Brown1993}
Brown, P.~F., Pietra, V. J.~D., Pietra, S. A.~D., \BBA\ Mercer, R.~L. \BBOP
  1993\BBCP.
\newblock \BBOQ The Mathematics of Statistical Machine Translation: Parameter
  Estimation.\BBCQ\
\newblock {\Bem Computational linguistics}, {\Bbf 19}  (2), \mbox{\BPGS\
  263--311}.

\bibitem[\protect\BCAY{Chang, Galley, \BBA\ Manning}{Chang
  et~al.}{2008}]{PCC2008}
Chang, P.-C., Galley, M., \BBA\ Manning, C.~D. \BBOP 2008\BBCP.
\newblock \BBOQ Optimizing {Chinese} Word Segmentation for Machine Translation
  Performance.\BBCQ\
\newblock In {\Bem Proceedings of the 3rd Workshop on SMT}, \mbox{\BPGS\
  224--232}.

\bibitem[\protect\BCAY{Downing \BBA\ Locke}{Downing \BBA\
  Locke}{2006}]{Downing2006}
Downing, A.\BBACOMMA\ \BBA\ Locke, P. \BBOP 2006\BBCP.
\newblock {\Bem English Grammar: A University Course}.
\newblock Second Edition. Prentice Hall International English Language
  Teaching. Abingdon, Oxon: Routledge.

\bibitem[\protect\BCAY{Dredze, Blitzer, Talukdar, Ganchev, Graca, \BBA\
  Pereira}{Dredze et~al.}{2007}]{MD2007}
Dredze, M., Blitzer, J., Talukdar, P.~P., Ganchev, K., Graca, J., \BBA\
  Pereira, F. \BBOP 2007\BBCP.
\newblock \BBOQ Frustratingly Hard Domain Adaptation for Dependency
  Parsing.\BBCQ\
\newblock In {\Bem Proceedings of the CoNLL Shared Task Session of
  EMNLP-CoNLL}, \mbox{\BPGS\ 1051--1055}.

\bibitem[\protect\BCAY{Galley \BBA\ Manning}{Galley \BBA\
  Manning}{2008}]{Galley2008}
Galley, M.\BBACOMMA\ \BBA\ Manning, C.~D. \BBOP 2008\BBCP.
\newblock \BBOQ A Simple and Effective Hierarchical Phrase Reordering
  Model.\BBCQ\
\newblock In {\Bem Proceedings of the Conference on Empirical Methods in
  Natural Language Processing (EMNLP)}, \mbox{\BPGS\ 848--856}. Association for
  Computational Linguistics.

\bibitem[\protect\BCAY{Gao \BBA\ Vogel}{Gao \BBA\ Vogel}{2008}]{MGIZA2008}
Gao, Q.\BBACOMMA\ \BBA\ Vogel, S. \BBOP 2008\BBCP.
\newblock \BBOQ Parallel Implementations of Word Alignment Tool.\BBCQ\
\newblock In {\Bem Software Engineering, Testing, and Quality Assurance for
  Natural Language Processing}, \mbox{\BPGS\ 49--57}.

\bibitem[\protect\BCAY{Green}{Green}{2011}]{NG2011}
Green, N. \BBOP 2011\BBCP.
\newblock \BBOQ Effects of Noun Phrase Bracketing in Dependency Parsing and
  Machine Translation.\BBCQ\
\newblock In {\Bem Proceedings of ACL-HLT, Student Session}, \mbox{\BPGS\
  69--74}.

\bibitem[\protect\BCAY{Habash}{Habash}{2007}]{NH2007}
Habash, N. \BBOP 2007\BBCP.
\newblock \BBOQ Syntactic Preprocessing for Statistical Machine
  Translation.\BBCQ\
\newblock In {\Bem Proceedings of Machine Translation Summit XI}, \mbox{\BPGS\
  215--222}.

\bibitem[\protect\BCAY{Han, Sudoh, Wu, Duh, Tsukada, \BBA\ Nagata}{Han
  et~al.}{2012}]{DH2012}
Han, D., Sudoh, K., Wu, X., Duh, K., Tsukada, H., \BBA\ Nagata, M. \BBOP
  2012\BBCP.
\newblock \BBOQ Head Finalization Reordering for {Chinese-to-Japanese} Machine
  Translation.\BBCQ\
\newblock In {\Bem Proceedings of the 6th Workshop on Syntax, Semantics and
  Structure in Statistical Translation}, \mbox{\BPGS\ 57--66}.

\bibitem[\protect\BCAY{Hara, Miyao, \BBA\ Tsujii}{Hara et~al.}{2009}]{TH2009}
Hara, T., Miyao, Y., \BBA\ Tsujii, J.-i. \BBOP 2009\BBCP.
\newblock \BBOQ Descriptive and Empirical Approaches to Capturing Underlying
  Dependencies among Parsing Errors.\BBCQ\
\newblock In {\Bem Proceedings of EMNLP}, \mbox{\BPGS\ 1162--1171}.

\bibitem[\protect\BCAY{Hatori, Matsuzaki, Miyao, \BBA\ Tsujii}{Hatori
  et~al.}{2011}]{Corbit2011}
Hatori, J., Matsuzaki, T., Miyao, Y., \BBA\ Tsujii, J.-i. \BBOP 2011\BBCP.
\newblock \BBOQ Incremental Joint {POS} Tagging and Dependency Parsing in
  {C}hinese.\BBCQ\
\newblock In {\Bem Proceedigns of the 5th International Joint Conference on
  Natural Language Processing}, \mbox{\BPGS\ 1216--1224}.

\bibitem[\protect\BCAY{Isozaki, Hirao, Duh, Sudoh, \BBA\ Tsukada}{Isozaki
  et~al.}{2010a}]{RIBES2010}
Isozaki, H., Hirao, T., Duh, K., Sudoh, K., \BBA\ Tsukada, H. \BBOP 2010a\BBCP.
\newblock \BBOQ Automatic Evaluation of Translation Quality for Distant
  Language pairs.\BBCQ\
\newblock In {\Bem Proceedings of the 2010 Conference on Empirical Methods in
  Natural Language Processing}, \mbox{\BPGS\ 944--952}. Association for
  Computational Linguistics.

\bibitem[\protect\BCAY{Isozaki, Sudoh, Tsukada, \BBA\ Duh}{Isozaki
  et~al.}{2010b}]{HI2010}
Isozaki, H., Sudoh, K., Tsukada, H., \BBA\ Duh, K. \BBOP 2010b\BBCP.
\newblock \BBOQ Head Finalization: A Simple Reordering Rule for {SOV}
  Languages.\BBCQ\
\newblock In {\Bem Proceedings of WMTMetricsMATR}, \mbox{\BPGS\ 244--251}.

\bibitem[\protect\BCAY{Isozaki, Sudoh, Tsukada, \BBA\ Duh}{Isozaki
  et~al.}{2012}]{Isozaki2012}
Isozaki, H., Sudoh, K., Tsukada, H., \BBA\ Duh, K. \BBOP 2012\BBCP.
\newblock \BBOQ {HPSG}-Based Preprocessing for {E}nglish-to-{J}apanese
  Translation.\BBCQ\
\newblock {\Bem ACM Transactions on Asian Language Information Processing
  (TALIP)}, {\Bbf 11}  (3), \mbox{\BPGS\ 8:1--8:16}.

\bibitem[\protect\BCAY{Katz-Brown, Petrov, McDonald, Och, Talbot, Ichikawa,
  \BBA\ Seno}{Katz-Brown et~al.}{2011}]{JK2011}
Katz-Brown, J., Petrov, S., McDonald, R., Och, F., Talbot, D., Ichikawa, H.,
  \BBA\ Seno, M. \BBOP 2011\BBCP.
\newblock \BBOQ Training a Parser for Machine Translation Reordering.\BBCQ\
\newblock In {\Bem Proceedings of the 2011 Conference on EMNLP}, \mbox{\BPGS\
  183--192}.

\bibitem[\protect\BCAY{Kendall}{Kendall}{1938}]{Kendall1938}
Kendall, M.~G. \BBOP 1938\BBCP.
\newblock \BBOQ A New Measure of Rank Correlation.\BBCQ\
\newblock {\Bem Biometrika}, {\Bbf 30}  (1/2), \mbox{\BPGS\ 81--93}.

\bibitem[\protect\BCAY{Koehn, Axelrod, Mayne, Callison-Burch, Osborne, \BBA\
  Talbot}{Koehn et~al.}{2005}]{Koehn2005}
Koehn, P., Axelrod, A., Mayne, A.~B., Callison-Burch, C., Osborne, M., \BBA\
  Talbot, D. \BBOP 2005\BBCP.
\newblock \BBOQ Edinburgh System Description for the 2005 {IWSLT} Speech
  Translation Evaluation.\BBCQ\
\newblock In {\Bem Proceedings of International Workshop on Spoken Language
  Translation (IWSLT)}, \mbox{\BPGS\ 68--75}.

\bibitem[\protect\BCAY{Koehn, Hoang, Birch, Callison-Burch, Federico, Bertoldi,
  Cowan, Shen, Moran, Zens, Dyer, Bojar, Constantin, \BBA\ Herbst}{Koehn
  et~al.}{2007}]{MOSES2007}
Koehn, P., Hoang, H., Birch, A., Callison-Burch, C., Federico, M., Bertoldi,
  N., Cowan, B., Shen, W., Moran, C., Zens, R., Dyer, C., Bojar, O.,
  Constantin, A., \BBA\ Herbst, E. \BBOP 2007\BBCP.
\newblock \BBOQ Moses: Open Source Toolkit for Statistical Machine
  Translation.\BBCQ\
\newblock In {\Bem Proceedings of the Association for Computational Linguistics
  '07, Demonstration Sessions}, \mbox{\BPGS\ 177--180}.

\bibitem[\protect\BCAY{Koehn, Och, \BBA\ Marcu}{Koehn et~al.}{2003}]{Koehn2003}
Koehn, P., Och, F.~J., \BBA\ Marcu, D. \BBOP 2003\BBCP.
\newblock \BBOQ Statistical Phrase-based Translation.\BBCQ\
\newblock In {\Bem Proceedings of the 2003 Conference of the North American
  Chapter of the Association for Computational Linguistics on Human Language
  Technology-Volume 1}, \mbox{\BPGS\ 48--54}. Association for Computational
  Linguistics.

\bibitem[\protect\BCAY{Kudo \BBA\ Matsumoto}{Kudo \BBA\
  Matsumoto}{2000}]{TK2000}
Kudo, T.\BBACOMMA\ \BBA\ Matsumoto, Y. \BBOP 2000\BBCP.
\newblock \BBOQ {J}apanese Dependency Structure Analysis Based on Support
  Vector Machines.\BBCQ\
\newblock In {\Bem Proceedings of the EMNLP/VLC-2000}, \mbox{\BPGS\ 18--25}.

\bibitem[\protect\BCAY{Li \BBA\ Thompson}{Li \BBA\ Thompson}{1989}]{Li1989}
Li, C.~N.\BBACOMMA\ \BBA\ Thompson, S.~A. \BBOP 1989\BBCP.
\newblock {\Bem Mandarin {C}hinese: A Functional Reference Grammar}.
\newblock Linguistics-Asian Studies. University of California Press.

\bibitem[\protect\BCAY{Li, Li, Zhang, Li, Zhou, \BBA\ Guan}{Li
  et~al.}{2007}]{CHL2007}
Li, C.-H., Li, M., Zhang, D., Li, M., Zhou, M., \BBA\ Guan, Y. \BBOP 2007\BBCP.
\newblock \BBOQ A Probabilistic Approach to Syntax-based Reordering for
  Statistical Machine Translation.\BBCQ\
\newblock In {\Bem Proceedings of the 45rd Annual Meeting on Association for
  Computational Linguistics (ACL)}, \mbox{\BPGS\ 720--727}. Association for
  Computational Linguistics.

\bibitem[\protect\BCAY{McDonald \BBA\ Nivre}{McDonald \BBA\
  Nivre}{2007}]{RM2007}
McDonald, R.\BBACOMMA\ \BBA\ Nivre, J. \BBOP 2007\BBCP.
\newblock \BBOQ Characterizing the Errors of Data-driven Dependency \linebreak
  Parsing Models.\BBCQ\
\newblock In {\Bem Proceedings of the 2007 Joint Conference on EMNLP-CoNLL},
  \mbox{\BPGS\ 122--131}.

\bibitem[\protect\BCAY{Miyao \BBA\ Tsujii}{Miyao \BBA\ Tsujii}{2008}]{YM2008}
Miyao, Y.\BBACOMMA\ \BBA\ Tsujii, J. \BBOP 2008\BBCP.
\newblock \BBOQ Feature Forest Models for Probabilistic {HPSG} Parsing.\BBCQ\
\newblock {\Bem Computational linguistics}, {\Bbf 34}  (1), \mbox{\BPGS\
  35--80}.

\bibitem[\protect\BCAY{Nivre}{Nivre}{2004}]{add01}
Nivre, J. \BBOP 2004\BBCP.
\newblock {\Bem {\upshape ``Inductive Dependency Parsing.'' MSI Report 04070}}.
\newblock V{\"{a}}xj{\"{o}} University: School of Mathematics and Systems
  Engineering.

\bibitem[\protect\BCAY{Och}{Och}{2003}]{MERT2003}
Och, F.~J. \BBOP 2003\BBCP.
\newblock \BBOQ Minimum Error Rate Training for Statistical Machine
  Translation.\BBCQ\
\newblock In {\Bem Proceedings of the Association for Computational
  Linguistics}, \mbox{\BPGS\ 160--167}.

\bibitem[\protect\BCAY{Papineni, Roukos, Ward, \BBA\ Zhu}{Papineni
  et~al.}{2002}]{BLEU2002}
Papineni, K., Roukos, S., Ward, T., \BBA\ Zhu, W.-J. \BBOP 2002\BBCP.
\newblock \BBOQ Bleu: A Method for Automatic Evaluation of Machine
  Translation.\BBCQ\
\newblock In {\Bem Proceedings of the Association for Computational
  Linguistics}, \mbox{\BPGS\ 311--318}.

\bibitem[\protect\BCAY{Petrov, Barrett, Thibaux, \BBA\ Klein}{Petrov
  et~al.}{2006}]{SP2006}
Petrov, S., Barrett, L., Thibaux, R., \BBA\ Klein, D. \BBOP 2006\BBCP.
\newblock \BBOQ Learning Accurate, Compact, and Interpretable Tree
  Annotation.\BBCQ\
\newblock In {\Bem Proceedings of the 21st COLING and the 44th the Association
  for Computational Linguistics}, \mbox{\BPGS\ 433--440}.

\bibitem[\protect\BCAY{Pollard \BBA\ Sag}{Pollard \BBA\ Sag}{1994}]{HPSG1994}
Pollard, C.~J.\BBACOMMA\ \BBA\ Sag, I.~A. \BBOP 1994\BBCP.
\newblock {\Bem Head-driven Phrase Structure Grammar}.
\newblock The University of Chicago Press and CSLI Publications.

\bibitem[\protect\BCAY{Quirk \BBA\ Corston-Oliver}{Quirk \BBA\
  Corston-Oliver}{2006}]{CQ2006}
Quirk, C.\BBACOMMA\ \BBA\ Corston-Oliver, S. \BBOP 2006\BBCP.
\newblock \BBOQ The Impact of Parse Quality on Syntactically-informed
  Statistical Machine Translation.\BBCQ\
\newblock In {\Bem Proceedings of EMNLP}, \mbox{\BPGS\ 62--69}.

\bibitem[\protect\BCAY{Stolcke}{Stolcke}{2002}]{SRILM2002}
Stolcke, A. \BBOP 2002\BBCP.
\newblock \BBOQ {SRILM}---an Extensible Language Modeling Toolkit.\BBCQ\
\newblock In {\Bem Proceedings of the 7th International Conference on Spoken
  Language Processing}, \mbox{\BPGS\ 901--904}.

\bibitem[\protect\BCAY{Tillmann}{Tillmann}{2004}]{Tillmann2004}
Tillmann, C. \BBOP 2004\BBCP.
\newblock \BBOQ A Unigram Orientation Model for Statistical Machine
  Translation.\BBCQ\
\newblock In {\Bem Proceedings of the Annual Meeting of Human Language
  Technology Conference/North American Chapter of the Association for
  Computational Linguistics (HLT-NAACL)}, \mbox{\BPGS\ 101--104}. Association
  for Computational Linguistics.

\bibitem[\protect\BCAY{Tsunakawa, Okazaki, Liu, \BBA\ Tsujii}{Tsunakawa
  et~al.}{2009}]{Tsunakawa2009}
Tsunakawa, T., Okazaki, N., Liu, X., \BBA\ Tsujii, J. \BBOP 2009\BBCP.
\newblock \BBOQ A {Chinese-Japanese} Lexical Machine Translation through a
  Pivot Language.\BBCQ\
\newblock {\Bem ACM Transactions on Asian Language Information Processing
  (TALIP)}, {\Bbf 8}  (2), \mbox{\BPGS\ 9:1--9:21}.

\bibitem[\protect\BCAY{Wang, Collins, \BBA\ Koehn}{Wang et~al.}{2007}]{CW2007}
Wang, C., Collins, M., \BBA\ Koehn, P. \BBOP 2007\BBCP.
\newblock \BBOQ {C}hinese Syntactic Reordering for Statistical Machine
  Translation.\BBCQ\
\newblock In {\Bem Proceedings of the 2007 Joint Conference on EMNLP-CoNLL},
  \mbox{\BPGS\ 737--745}.

\bibitem[\protect\BCAY{Wang, Kazama, Tsuruoka, Chen, Zhang, \BBA\
  Torisawa}{Wang et~al.}{2011}]{YW2011}
Wang, Y., Kazama, J.-i., Tsuruoka, Y., Chen, W., Zhang, Y., \BBA\ Torisawa, K.
  \BBOP 2011\BBCP.
\newblock \BBOQ Improving {C}hinese Word Segmentation and {POS} Tagging with
  Semi-supervised Methods using Large Auto-analyzed Data.\BBCQ\
\newblock In {\Bem Proceedings of 5th IJCNLP}, \mbox{\BPGS\ 309--317}.

\bibitem[\protect\BCAY{Wu \BBA\ Wang}{Wu \BBA\ Wang}{2007}]{Wu2007}
Wu, H.\BBACOMMA\ \BBA\ Wang, H. \BBOP 2007\BBCP.
\newblock \BBOQ Pivot Language Approach for Phrase-based Statistical Machine
  Translation.\BBCQ\
\newblock {\Bem Machine Translation}, {\Bbf 21}  (3), \mbox{\BPGS\ 165--181}.

\bibitem[\protect\BCAY{Wu, Sudoh, Duh, Tsukada, \BBA\ Nagata}{Wu
  et~al.}{2011}]{XCW2011}
Wu, X., Sudoh, K., Duh, K., Tsukada, H., \BBA\ Nagata, M. \BBOP 2011\BBCP.
\newblock \BBOQ Extracting Pre-ordering Rules from Predicate-Argument
  Structures.\BBCQ\
\newblock In {\Bem Proceedings of 5th International Joint Conference on Natural
  Language Processing}, \mbox{\BPGS\ 29--37}.

\bibitem[\protect\BCAY{Xia}{Xia}{2000}]{POS2000}
Xia, F. \BBOP 2000\BBCP.
\newblock {\Bem The Part-of-Speech Tagging Guidelines for the {Penn Chinese
  Treebank} 3.0}.
\newblock University of Pennsylvania Institute for Research in Cognitive
  Science Technical Report No.~IRCS-00-07.

\bibitem[\protect\BCAY{Xia \BBA\ McCord}{Xia \BBA\ McCord}{2004}]{FX2004}
Xia, F.\BBACOMMA\ \BBA\ McCord, M. \BBOP 2004\BBCP.
\newblock \BBOQ Improving a Statistical {MT} System with Automatically Learned
  Rewrite Patterns.\BBCQ\
\newblock In {\Bem Proceedings of the 20th International Conference on
  Computational Linguistics (COLING)}, \mbox{\BPGS\ 508--514}. Association for
  Computational Linguistics.

\bibitem[\protect\BCAY{Xu, Kang, Ringgaard, \BBA\ Och}{Xu
  et~al.}{2009}]{PX2009}
Xu, P., Kang, J., Ringgaard, M., \BBA\ Och, F. \BBOP 2009\BBCP.
\newblock \BBOQ Using a Dependency Parser to Improve {SMT} for
  Subject-Object-Verb Languages.\BBCQ\
\newblock In {\Bem Proceedings of HLT: NA-ACL2009}, \mbox{\BPGS\ 245--253}.

\bibitem[\protect\BCAY{Yamada \BBA\ Knight}{Yamada \BBA\
  Knight}{2001}]{Yamada2001}
Yamada, K.\BBACOMMA\ \BBA\ Knight, K. \BBOP 2001\BBCP.
\newblock \BBOQ A Syntax-based Statistical Translation Model.\BBCQ\
\newblock In {\Bem Proceedings of the 39th Annual Meeting on Association for
  Computational Linguistics}, \mbox{\BPGS\ 523--530}, Stroudsburg, PA, USA.
  Association for Computational Linguistics.

\bibitem[\protect\BCAY{Yu, Miyao, Matsuzaki, Wang, \BBA\ Tsujii}{Yu
  et~al.}{2011}]{KY2011}
Yu, K., Miyao, Y., Matsuzaki, T., Wang, X., \BBA\ Tsujii, J. \BBOP 2011\BBCP.
\newblock \BBOQ Analysis of the Difficulties in {C}hinese Deep Parsing.\BBCQ\
\newblock In {\Bem Proceedings of the 12th International Conference on Parsing
  Technologies}, \mbox{\BPGS\ 48--57}.

\bibitem[\protect\BCAY{Zens, Och, \BBA\ Ney}{Zens et~al.}{2002}]{Zens2002}
Zens, R., Och, F.~J., \BBA\ Ney, H. \BBOP 2002\BBCP.
\newblock \BBOQ Phrase-based Statistical Machine Translation.\BBCQ\
\newblock In {\Bem Proceedings of the German Conference on Artificial
  Intelligence (KI2002)}, \mbox{\BPGS\ 18--32}. Springer.

\bibitem[\protect\BCAY{Zhao, Lv, Ben, Huang, \BBA\ Liu}{Zhao
  et~al.}{2011}]{CWMT2011}
Zhao, H.-M., Lv, Y.-J., Ben, G.-S., Huang, Y., \BBA\ Liu, Q. \BBOP 2011\BBCP.
\newblock {\Bem Evaluation Report for The 7th {C}hina Workshop on Machine
  Translation ({CWMT}2011)}.

\end{thebibliography}


\clearpage
\appendix

\section{Summary of Part-of-Speech Tag Set in Penn Chinese Treebank}
\label{apx:posdef}

                             

\begin{table}[h]
\vspace{-3\Cvs}
\caption{POS tags defined in Penn Chinese Treebank v3.0 (Xia 2000)}
\label{tab:posdef}
\vspace{8pt}
\input{1004table05.txt}
\end{table}


\clearpage
\section{Head Rules for Penn2Malt to Convert the Penn Chinese Treebank}
\label{apx:hrdef}

                             
\begin{table}[h]
\vspace{-3\Cvs}
\caption{Rules for converting trees in the Penn Chinese Treebank format into MaltTab format using Penn2Malt tool (Joakim Nivre 2004)}
\nocite{add01}
\label{tab:hrdef}
\vspace{8pt}
\input{1004table06.txt}
\vspace{4pt}
\small
These rules were originally compiled by Yuan Ding$^{12}$, and were used to identify head branches of phrase structures.
As an example, in an ADJP branch (first row), in order to discover the head branch we scan from right (r) to left
all branches. If we find an ADJP or JJ branch, then we select it as a head.
If we do not find them, then we scan again the branches from right (r) to left, searching for AD, NN or CS.
If we do not find them, then we select the right-most (r) branch.
In this work, we introduced new rules to identify head branches for FLR, INC and DFL phrases,
which are not originally covered in Penn2Malt tool.
\par
\end{table}

{\hrule width .4\columnwidth}
\kern 2.6pt
\noindent{\footnotesize$^{12}$\ http://stp.lingfil.uu.se/{\textasciitilde}nivre/research/Penn2Malt.html}

\clearpage

\begin{biography}

\bioauthor[:]{Dan Han}{
She received her M.S. degree of software engineering from Sichuan University, China,
in 2008. She is currently a Ph.D student at the Graduate University For Advanced Studies, 
and research assistant at National Institute of Informatics.
Her research interest is in natural language processing, especially statistical 
machine translation and reordering models.}

\bioauthor[:]{Pascual Mart\'inez-G\'omez}{
He received his Ph.D. from the University of Tokyo in 2014.
Currently, he is assistant professor at Ochanomizu University and visiting researcher
at National Institute of Informatics.
He is interested in pattern recognition, natural language processing,
eye-tracking and multi-modal interfaces.
His latest research focus is on recognizing textual entailment.
}

\bioauthor[:]{Yusuke Miyao}{
He received his Ph.D. from the University of Tokyo in 2006. 
He has been Assistant Professor at the University of Tokyo from 2001 to 2010, 
and Associate Professor at National Institute of Informatics since 2010. 
He has been engaged in research on natural language processing, in particular 
on syntactic parsing and its applications.}

\bioauthor[:]{Katsuhito Sudoh}{
He received a bachelor's degree in engineering and a master's degree in informatics 
from Kyoto University, Kyoto, Japan in 2000 and 2002 respectively. In 2002 he joined 
NTT Corporation, and he works at NTT Communication Science Laboratories. His research 
interest is machine translation.}

\bioauthor[:]{Masaaki Nagata}{
He received the B.E., M.E., and Ph.D. degrees in information science
from Kyoto University, Kyoto, in 1985, 1987, and 1999,
respectively. He joined NTT in 1987. He was with ATR Interpreting
Telephony Research Laboratories from 1989 to 1993.  He is the leader
of the Linguistic Intelligence Research Group at the NTT Communication
Science Laboratories. His research interests include natural language
processing, especially morphological analysis, named entity
recognition, parsing, and machine translation.
}

\end{biography}

\biodate




\end{document}
