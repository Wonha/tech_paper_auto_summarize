    \documentclass[japanese]{jnlp_1.4}
\usepackage{jnlpbbl_1.1}
\usepackage[dvips]{graphicx}
\usepackage{amsmath}
\usepackage{hangcaption_jnlp}
\usepackage{udline}
\setulminsep{1.2ex}{0.2ex}
\let\underline

\def\maru#1{}

\Volume{15}
\Number{3}
\Month{July}
\Year{2008}
\received{2006}{12}{22}
\revised{2007}{12}{20}
\accepted{2008}{3}{7}

\setcounter{page}{3}


\jtitle{看護学分野の専門用語抽出方法の研究}
\jauthor{木浪 孝治\affiref{Author_1} \and 池田 哲夫\affiref{Author_2} 
	\and 村田 嘉利\affiref{Author_3} \and 高山  毅\affiref{Author_3} 
	\and 武田 利明\affiref{Author_4}}
\jabstract{
今日,大学は産学連携の一層の活性化が求められており,これを可能にするためには大学側のシーズを簡単に検索できるシステムが望まれる.そこで著者らは,産学連携の専門家が研究のシーズを専門用語によって簡単に検索することができるシステムの構築を狙いとし,その第一段階として専門用語抽出の研究を行っている.本研究ではこれまで研究されていない看護学分野を対象分野とした.予備研究によって,病気の症状や治療法を表す専門用語が情報検索分野における代表的な専門用語の抽出方法では抽出が難しいことが判明した.そこで,専門用語になりうる品詞の組合せの拡張と一般的な語を除去することで専門用語抽出の性能改善を図った.その結果,品詞の組合せを拡張することで再現率は83{\%}から99{\%}と専門用語をほぼもれなく抽出可能となった.更に,単語親密度に関する研究成果を活用することで適合率は42{\%}から55{\%}となり大幅に向上した.
}
\jkeywords{用語抽出,専門用語,看護学}

\etitle{Research on Technical Term Extraction \\
	in the Nursing Domain}
\eauthor{Koji Kinami\affiref{Author_1} \and Tetsuo Ikeda\affiref{Author_2}
	\and Yoshitoshi Murata\affiref{Author_3} \and \\
	Tsuyoshi Takayama\affiref{Author_3}
	\and Toshiaki Takeda\affiref{Author_4}} 
\eabstract{
This paper presents a research for extracting technical terms from documents 
in the nursing domain. An exploratory study showed that a well-known term 
extraction method, which has proven to be effective in extracting technical 
terms specified to the computing domain, can not effectively extract 
technical words representing symptoms or treatments of diseases. We propose 
a new technical term extraction method to improve extraction performance. 
Its main characteristics are enhancing permissible combinations of 
word-class; and excluding fundamental vocabulary. Experimental results 
showed that our extraction method attained 99{\%} recall, which is 
16{\%} better than the recall attained by the base method. In addition, 
13{\%} of precision rate is improved by excluding fundamental vocabulary.
}
\ekeywords{Term recognition, Domain specific terms, Nursing domain}

\headauthor{木浪,池田,村田,高山,武田}
\headtitle{看護学分野の専門用語抽出方法の研究}

\affilabel{Author_1}{日立製作所}{HITACHI, LTD.}
\affilabel{Author_2}{静岡県立大学経営情報学部}{School of Administration and Infomatics, University of Sizuoka}
\affilabel{Author_3}{岩手県立大学ソフトウェア情報学部}{Faculty of Software and Information Science, Iwate Prefectural University}
\affilabel{Author_4}{岩手県立大学看護学部}{Faculty of Nursing, Iwate Prefectural University}



\begin{document}
\maketitle




\section{はじめに}

今日,大学は社会に貢献することが求められているようになっている.特に,産業界と関係の深い学部においては産学連携が強く求められるようになってきている.そのような産学連携を活性化するためには大学側のシーズを専門用語によって簡単に検索できるシステムが望まれる.そこで,著者らは産学連携マッチングを支援する研究情報検索システムの研究を開始した.本研究では研究情報検索システムの主要要素である専門用語の抽出に取り組んでいる.対象分野としては専門用語による研究情報検索システムのニーズが高く,これまで研究がなされていない分野の1つである看護学分野を選択した.

専門用語抽出の研究は情報処理分野を対象にした研究は盛んに行われている.しかしながら,一部の医学・基礎医学分野以外には他分野の専門用語抽出の研究は見当たらない.予備研究によって,病気の症状や治療法を表す専門用語が情報検索分野における代表的な専門用語の抽出方法では抽出が難しいことが判明した.そこで,専門用語になりうる品詞の組合せの拡張と一般的な語を除去することで専門用語抽出の性能改善を図った.

以下,2章で従来研究とアプローチについて述べ,3章で提案手法,4章で実験及び評価,5章で考察と今後の課題について述べる.

\section{従来研究とアプローチ}

\subsection{従来研究}

用語には1単語から構成されるものもあれば複数の単語から構成される複合語のものも存在する.例えば「専門用語抽出」は「専門」「用語」「抽出」の3つの単語から構成されている.多くの専門用語は,例のように複合語で構成されていることが多い.

このような複合語を考慮した情報処理分野における専門用語抽出の研究の代表的なものに中川らの研究(中川 2003)がある.中川らは名詞と一部の特殊な形容詞を単名詞として扱い,それら単名詞の出現頻度と連接頻度を用いた専門用語抽出方法とスコア付け方法を提案している.この提案手法は情報処理分野における専門用語抽出では高い性能を示している.しかしながら,著者らの予備実験(木浪 2006)では,看護学分野における専門用語に中川らが専門用語の構成要素として許した形態素以外の形態素を含むものが多数存在するため,中川らの手法では良い性能を示せていない.

複合語を構成する品詞の組合せに着目した従来研究として辻河らの研究(辻河 2003)がある.辻河らは品詞の組合せを用いて専門用語を構築する場合に,名詞だけではなく接頭語・接尾語も対象とすることが抽出性能の向上に有効であるという結論を導き出している.

複合語に着目した他の研究として,単語n-gramに対して連接コストを割り当てることで専門用語抽出を行う相澤らの研究(相澤 2005)がある.相澤らの手法では以下の手順によって専門用語の抽出を行っている.はじめに,対象分野の既知の用語集合から構成語間の連接コストを求める.連接コストは,ある単語が専門用語の先頭に位置する確率,中間に位置する確率,末尾に位置する確率に基づき求められる.次に,算出された連接コストを元に値が小さい順に構成語を葉として追加して2分木を構成する.これら単語と連接コストによって構成され2分木の集合が依存木を構成し,この依存木から専門用語候補の抽出を行う.最後に,抽出された専門用語候補の非終端確率(門用語候補の前後に語が必要である確率)を求め,算出された確率が一定の閾値以下であれば専門用語であると判断するという手法である.相澤らの手法は非終端確率を用いることで専門用語ではない一般名詞や単名詞の除去を可能とする特徴を持つ.他方連接コストの計算に用いる重みを求めるため既知の用語集合を用いていることから,その用語集合に収録されていない新語が現れた場合,新語をどのように扱うのか不明確であるという問題がある.

文書のテキストだけではなく作者情報を考慮することを特徴とする専門用語抽出の研究に立石らの研究(立石 2006)がある.この研究は立石らの論文でも述べられているように,中川らの研究と比較してどちらが優れているという関係ではなく,相互に補完的な研究であると言える.

他に,「テンプレート」を用いた情報抽出を行う研究として井上らの研究(井上 2001)がある.井上らの手法では病名や診断機器,診断症例などに対して記述パターンや文中に共起する文字列について分析を行い,その分析結果を元に抽出すべき情報とその周辺の文字列の関係を記した「テンプレート」を用いて情報抽出を行っている.井上らの手法は単純な構文の文書中に表れる病名や診断機器に関する情報は良い抽出結果を得ているが,「抗癌剤と放射線を用いた…」や「化学療法や食事療法といった…」といった「AとB」「AやB」のように並列構造を持った文書など,テンプレートの抽出能力を超えた複雑な文書構造の場合の情報抽出は,再現率・適合率ともに50〜60{\%}と良い結果を得ていない.

上記研究から,これまで現れていない新語や複雑な文書構造に対応するためには,複合語を専門用語の候補とみなし出現頻度と連接頻度を用いたランキング手法を用いた抽出方法である中川らの手法及び辻河らの手法が有力であることがわかる.但し,中川ら及び辻河らの手法はいずれも情報処理分野を対象としたものであり,他分野への適用可能性は不明である.

\subsection{アプローチ}

著者らは,上記研究の手法をベースに看護学分野の専門用語抽出方法を考案することとした.

看護学分野の文献から専門用語を抽出する予備実験(木浪 2006)を行った結果,以下の3つの問題が判明した.1つ目の問題は,看護学分野において従来研究で前提条件としている品詞の組合せでは抽出できない専門用語が多数存在している.例えば「破(動詞)骨(名詞)細胞(名詞)」のように動詞を含んだ専門用語などが存在する.2つ目の問題は,誤って抽出された一般的な用語が多数存在していることである.ここで言う「一般的な用語」とは,看護学分野で使用される用語ではあるが,看護学分野固有の用語ではなく日常生活においても広く利用される用語をいう.このような用語には「積極的,人間関係,価値観」などが含まれる.

以上の事を踏まえ,本研究はそれぞれの問題別にアプローチを検討した.1つ目の問題については,専門用語の候補となりうる品詞の組合せを拡張することで専門用語抽出の再現率向上を図る.2つ目の問題については,一般的な語を除去することで専門用語抽出の適合率向上を図る.

本論文では,再現率向上を優先し,それが低下しない範囲で適合率の向上を目指す.研究情報検索システムでは,検索キーワードが専門用語と判定された場合にその重みを大きくすることにより専門性の高い論文として選択することを考えている.それ故,専門用語をもれなく抽出すること,すなわち再現率の向上が重要となる.その一方,専門用語以外を誤って専門用語と判定してもその論文が選択候補から除去されることはなく,弊害はそれほど大きくないといえる.以上のことから,まず初めに再現率向上のルール導出によって再現率向上を行い,次に再現率を低下させずに適合率向上のためのルール導出を行うこととした.

予備研究の結果,看護学分野の専門用語は日本語形態素のみで構成される専門用語の割合が多いことが判明した.英語形態素と日本語形態素の両方が含まれている専門用語(例:情報処理分野で言うと,ACID特性など),あるいは英語形態素のみで構成される専門用語を正しく抽出するには日本語のみに着目した提案手法は不適切となるが,そのような例は現時点ではそれほど多くない.従って,本研究では日本語の専門用語に着目して研究を行った.

なお,3章以降で述べる「ルール」とは,専門用語になりうる品詞の組合せと,それら品詞を連接する条件を表す.

\section{提案手法}

\subsection{前提条件}

\subsubsection{専門用語抽出環境}

本章以降で用いる専門用語抽出対象となる文献(以降データセットと呼ぶ)の提供,正解となる専門用語集合(以降正解セットと呼ぶ)の作成は看護学の専門家である本学看護学研究科の社会人大学院生に依頼した.正解セットの作成方法について説明する.正解セットはもれなく全ての専門用語を含んでいることが望ましいことから,1つの文献に対して2人が専門用語の選択作業を行い,選択された2人分の専門用語の和集合を正解セットとした.

依頼した文献(30文献)のうち,前半部分(16文献)を用いて提案手法によるルールの導出と洗練を行い,残り後半(14文献)を用いて提案手法の評価を行った.これ以降,前半を学習用データセット,後半を評価用データセットと呼ぶ.以下にデータセットの詳細を示す.学習用データセットは,1文献あたりの単語数は約9,000語,全正解単語数(専門用語)は2,587語である.評価用データセットは,1文献あたりの単語数は約5,600語,全正解単語数(専門用語)は4,711語である.

なお,本研究に必要となる形態素解析器には「茶筌」version 2.3.3,形態素解析辞書にはIPADIC version 2.6.3を用いた.

\subsubsection{専門用語抽出処理}

\begin{figure}[b]
\begin{center}
\includegraphics{15-3ia1f1.eps}
\end{center}
\caption{専門用語抽出処理の流れと実行例}
\end{figure}

専門用語の抽出は形態素解析結果とルールを用いて行う.専門用語抽出処理の流れと実行例を図1に示す.図 1の専門用語抽出処理の流れを説明する.文章入力(図 1(a))として「看護学分野での専門用語抽出」という句が入力されたとする(図 1(a$'$)).次に形態素解析(図 1(b))が実行される.例では8つの形態素と品詞情報が得られる(図 1(b$'$)).その次に形態素解析の結果として得られた品詞情報と専門用語を抽出するためのルールを用いて専門用語抽出を行う(図 1(c)).例では,「看護」「学」「分野」の組合せである「看護学分野」(図 1(c1$'$))と,「専門」「用語」「抽出」の組合せである「専門用語抽出」(図 1(c2$'$))の2つが得られる.最後に専門用語が出力される(図 1(d),(d$'$)).

基本となるルールは,中川らの研究成果を実装したTermExtract (TermExtract 2006)モジュールの品詞に,辻河らの研究で述べられている接頭語,接尾語を例外的に単名詞として扱うルールを追加したものを用いる.基本となるルールの全てを表 1に示す.例における下線部は対応する品詞を示す.なお,ここで用いる品詞形態はIPA品詞形態に準拠している.

\begin{table}[b]
\caption{ルール一覧}
\input{01table1.txt}
\end{table}

以下,連接条件について説明する.「無条件連接」とは「表 1にあるいずれかの品詞が連続して現れなくても連接可能である」ことを表す.無条件連接以外の連接条件では,条件を満たす場合に形態素の前(あるいは後)の形態素と連接されて用語を構成する.条件を満たさない形態素は破棄される.なお,無条件連接と条件付連接の両方が適用可能である場合は,条件付連接を優先して形態素の連接を行う.以下に例を示す.

\noindent
例1)無条件連接:名詞--一般,名詞--サ変接続と連続した場合

\vspace{1zw}
\fbox{\parbox{33zw}{
用語(名詞--一般) 抽出(名詞--サ変接続);\par
 → どちらも無条件連接なので「用語抽出」という用語が構成される.
}}
\vspace{1zw}

\noindent
例2)条件付連接:名詞--一般,名詞--形容動詞語幹と連続した場合

\vspace{1zw}
\fbox{\parbox{39zw}{
円錐(名詞--一般) 小体(名詞--形容動詞語幹);\par
\hangafter=1\hangindent=3zw
 → 名詞--形容動詞語幹の次には連接対象が必要だが,連続していないため「小体」が破棄され「円錐」までが用語として抽出される
}}
\vspace{1zw}




\subsection{再現率向上のためのルール導出手順}

以下のサイクルの繰り返しにより,再現率を向上させるルールを導出する.

\begin{itemize}
\item[(i)]
専門用語抽出システムにより専門用語抽出を行う.

\item[(ii)]
抽出できなかった専門用語を人手によって抽出する.

\item[(iii)]
抽出できなかった専門用語を形態素解析し,その用語を抽出可能とするルールの導出を行う

\item[(iv)]
導出されたルールの妥当性を後述する手順で評価する.

\item[(v)]
妥当性評価の結果が,品詞レベルで妥当なルールあるいは語レベルでは妥当なルールの場合は,それらのルールをルール集合に追加する.
\end{itemize}

例外を除いた全ての専門用語に関して (ii) から (v) の処理が行われるまで繰り返す.ここで言う例外とは形態素解析器の限界によって正しく形態素解析を行うことができない形態素(同綴異品詞)を含む専門用語のことを言う.例えば,「うつ病」の形態素解析を行うと「うつ(動詞)」「病(名詞)」となり,誤った形態素解析結果となる.

ルールの妥当性は以下の手順で評価する.

\begin{itemize}
\item[(i)]
品詞レベルでルールを適用した結果,再現率が向上し適合率が低下しないか評価する.再現率が向上し,適合率が低下しない場合,このルールは妥当とする.

\item[(ii)]
再現率が向上し,適合率が低下した場合,品詞レベルでルールを適用するのではなく特定の語を連接対象とする.特定の語は,その意味分類が看護学分野に関連した分野に属するものを選択する.具体的には,分類語彙表(国立国語研究所 2004)の「医療・看護」「生理・病気など」「救護・救援」など看護学分野に関連した分類に属するものを選択する.
\end{itemize}


\subsection{適合率向上のためのルール導出手順}

適合率の低下は一般用語を専門用語と認識することに起因することから,適合率向上のためには専門用語ではありえない用語の組み合わせルールを導出すると共に,一般用語を取り除くことが有効といえる.

適合率を向上させるルールの導出は,以下のサイクルを繰り返すことにより行う.

\begin{itemize}
\item[(i)]
専門用語抽出システムにより専門用語の抽出を行う.

\item[(ii)]
誤って抽出された語を人手によって抽出する.

\item[(iii)]
誤って抽出された語を形態素解析し,追加するルールの導出を行う.追加ルールは次の2種類のルールのいずれかである.専門用語の構成要素から特定の形態素を除外するルール,あるいは専門用語から特定の形態素の組合せを除外するルールである.

\item[(iv)]
仮にそのルールを追加した場合の再現率・適合率を評価し,再現率が低下せずに適合率が向上するのならば,ルール集合へのルールの追加を行う.
\end{itemize}

一般用語の除去については,専門用語は一般の人になじみが無い,つまり親密度が低いとの仮説に基づき,親密度の高い語を除去することにより実施した.具体的には,佐藤らの研究成果(佐藤 2004)で得られた単語親密度が付与された語彙集合を用いて,一般用語を除去した.以下に佐藤らの研究について説明する.

佐藤らは基本となる語彙集合から単語親密度を用いて基本語彙を選定する研究を行っている.ここで言う基本語彙とは,コミュニケーションや日常生活でもっとも普通に使用され,使用頻度が高い語彙のことを言う.単語親密度とは,単語に対する主観的ななじみの程度を示した尺度で,複数の被験者が1から7の7段階で評定した結果を平均化したものを言う.評定に用いる値はそれぞれ「1はなじみがなく,7はなじみがある」を示している.基本となる語彙の母集合には,時代や性差に左右されにくく普遍的な語彙が多数収録されている国語大辞典の見出し語を選択している.収録されている語彙数は成人の理解語数とされている48,000語の2倍程度である94,928語と十分な数が収録されている.基本的な語彙の選択にはこれら母集合に対し実験により単語親密度を付加し,得られた単語親密度が5以上の語彙を選択している.これは,従来研究によって成人の過半数が知っていると推定される理解語彙数,小学校修了時の理解語彙数,単語親密度が5以上の語彙数の3つにおいて語彙数の整合が取れていることから単語親密度が5以上の語彙を基本語彙候補としている.

除去対象となる一般的な語は,単語親密度が5以上の語であれば「人間関係,価値観,医薬,コンピュータ」など28,445語,単語親密度が6以上の語は「積極的,時間,不安,性格」など4,523語である.

\subsection{形態素解析手法の改善}

形態素解析の結果得られた片仮名に関して,連続する片仮名を連接して1語とすることで形態素解析手法の改善を図る.

\subsection{導出したルール}

\subsubsection{再現率向上のためのルール}

全部で8つのルールを導出した.特定の品詞を連接するルール,特定の語を連接するルール,変更したルールの3つに分類して説明する.

\noindent
1) 特定の品詞を連接するルール

\noindent
1-1) 名詞--副詞可能

品詞が「名詞--副詞可能」で連接条件が「無条件連接」であるルールを追加した.ただし,連接対象として不要と判断した時相名詞などを含む形態素は連接対象から除外した.除外した全ての形態素を表 2に示す.以下に本ルールが適用される専門用語の例を示す.下線部が対象の形態素である.他のルールに関しても対象の形態素に下線を付記する.

\vspace{1zw}
\fbox{\parbox{39zw}{
急性腎\ul{前}性腎不全,鼓室形成\ul{術後}後遺症,\ul{産後}脚気,\ul{絶対}好気性菌,\ul{前後}十字靱帯損傷,\ul{時間}薬理学}}
\vspace{1zw}

\begin{table}[t]
\caption{除外する名詞--副詞可能}
\input{01table2.txt}
\end{table}

\noindent
1-2) 形容詞--自立 アウオ段 ガル接続

品詞が「形容詞--自立」,細分類が「アウオ段--ガル接続」,連接条件が「無条件連接」であるルールを導出した.以下に本ルールが適用される専門用語の例を示す.

\vspace{1zw}
\fbox{\ul{暗}視野照明,炎症性\ul{硬}結,\ul{緩}速導入,\ul{狭}隅角緑内障,\ul{硬}膜下出血,\ul{多}剤耐性}
\vspace{1zw}

\noindent
1-3) 動詞--自立 五段・ラ行体言接続特殊2

品詞が「動詞--自立」,細分類が「五段・ラ行体言接続特殊2」,連接条件が「無条件連接」であるルールを導出した.以下に本ルールが適用される専門用語の例を示す.

\vspace{1zw}
\fbox{下顎\ul{切}創,外旋\ul{拘}縮,眼位性眼\ul{振},\ul{駆}散薬,\ul{散}腫,\ul{殺}真菌薬,\ul{粘}膿性,\ul{破}骨細胞}
\vspace{1zw}

\noindent
2) 特定の語を連接するルール

\noindent
2-1) 副詞--一般

品詞が「副詞--一般」で連接条件が「無条件連接」であるルールを導出した.ただし,語彙分類が生理・病気などに分類されている語と「的」で終わる語を連接対象とした.以下に本ルールが適用される専門用語の例を示す.

\vspace{1zw}
\fbox{\ul{極}低産体重児,\ul{早}発症,\ul{早}成,\ul{漸}深帯,\ul{漸}加}
\vspace{1zw}

\noindent
2-2) 名詞--非自立 副詞可能

品詞が「名詞--非自立」,細分類が「副詞可能」で連接条件が「無条件連接」であるルールを導出した.ただし,連接対象は「間」とした.ここで「間」の語彙分類が看護学分野に属するものではないにも関わらず連接対象とした理由を説明する.辻河らの研究(辻河 2003)によって接辞を専門用語の構成要素としてみなすのが妥当であることが判明していること,「間」は接辞としての性質を有する語であることから「間」を連接対象とした.

以下に本ルールが適用される専門用語の例を示す.

\vspace{1zw}
\fbox{\ul{間}質性肺炎,\ul{間}入性,\ul{間}擦疹,\ul{間}擦性湿疹}
\vspace{1zw}

\noindent
2-3) 動詞--自立 一段 連用形

品詞が「動詞--自立」,細分類が「一段連用形」,連接条件が「前または後ろに連接対象が続いた場合のみ連接」であるルールを導出した.ただし,語彙分類が「医療」「救護・救援」に分類されている語のみを連接対象とした.以下に本ルールが適用される専門用語の例を示す.

\vspace{1zw}
\fbox{\ul{病診}連携,脈\ul{診},\ul{視}紫紅,\ul{視}束前核,低拍\ul{出}性,拍\ul{出}量}
\vspace{1zw}

\noindent
2-4) 名詞--数

単体では専門用語としての意味を成さないため,連接条件を「前または後に連接対象が続いた場合のみ連接」であるルールを導出した.以下に本ルールが適用される専門用語の例を示す.

\vspace{1zw}
\fbox{\ul{二}次性高血圧症,\ul{四}段脈,\ul{一}次性脳幹外傷,膝蓋骨\ul{一}次中枢若年性骨軟骨症,\ul{三}色性色覚}
\vspace{1zw}

\noindent
3) 変更したルール

\noindent
3-1) 名詞--形容動詞語幹,名詞--ナイ形容詞語幹

該当品詞の次に連接対象が続かない専門用語が存在したため,連接条件を「無条件連接」へ変更した.以下に本ルールが適用される専門用語の例を示す.

\vspace{1zw}
\fbox{眼部外傷性色素\ul{沈着},胃腸機能\ul{異常},喀痰喀出\ul{困難},膵硬\ul{変},アウエル\ul{小体}}\vspace{1zw}


\subsubsection{適合率向上のためのルール}

適合率向上に寄与するルールを全部で11導出した.専門用語の構成要素から除外する形態素の導出,専門用語候補から除外する語の組み合わせの導出,一般的な語の除去,の3つに分類して示す.

\noindent
1) 専門用語の構成要素から除外する形態素の導出

専門用語の一部になりえない語を連接対象から除外するルールを導出した.

\noindent
1-1) 名詞--一般に分類される「一つ〜九つ」を連接対象から除外した.

\noindent
1-2) 名詞--接尾の中で専門用語として不要と判断された形態素「ごと」を除外した.

\noindent
\hangafter=1\hangindent=1zw
1-3) 専門用語の一部になりえない数詞と特定の助数詞(ヶ月,週間など)の組み合わせを除外した.以下に本ルールによって除外された一般用語例を示す.下線部が除外対象となった語である.以下のルールにおいても同様にルール適用箇所に下線を付与した.

\vspace{1zw}
\fbox{術後 \ul{\mbox{1ヶ月}},\ul{\mbox{1週間}}服用,胸部食道癌 \ul{\mbox{275例}}}
\vspace{1zw}

\noindent
\hangafter=1\hangindent=1zw
1-4) 専門用語の一部になりえない特定の未知語(章節番号や箇条書きに使われる記号など)を連接対象から除外した.以下に本ルールによって除外された一般用語例を示す.

\vspace{1zw}
\fbox{\ul{\mbox{\maru{1}}} 急性盲腸炎,\ul{\mbox{\maru{2}}} 異染体,\ul{III} 破骨細胞,\ul{iv} 膵硬変}
\vspace{1zw}

\noindent
2) 専門用語候補から除外する語の組み合わせの導出

抽出された複合語のうち,専門用語になりえない語の組み合わせを専門用語候補から除外した.各項目に例を示す.

\noindent
2-1) 年代や区間を示すものを除外した.

\vspace{1zw}
\fbox{\ul{\mbox{0.01--9.95}},\ul{\mbox{1999--2006}}}
\vspace{1zw}

\noindent
2-2) 小数点を含む数値のみを除外した.

\vspace{1zw}
\fbox{\ul{10},\ul{0.1},\ul{1999},\ul{20061210}}
\vspace{1zw}

\noindent
2-3) 数値と特定の単位(kg,歳,回など)で構成されている語を除外した.

\vspace{1zw}
\fbox{\ul{\mbox{50\,kg}},\ul{\mbox{24歳}},\ul{\mbox{20回}}}
\vspace{1zw}

\noindent
2-4) 数式を表す語を除外した.

\vspace{1zw}
\fbox{\ul{\mbox{$0.1 < x < 9.1$}},\ul{\mbox{$y = 2x + b$}}}
\vspace{1zw}

\noindent
2-5) 図表番号を除外した.

\vspace{1zw}
\fbox{\ul{\mbox{図1}},\ul{\mbox{表2-1}},\ul{\mbox{Fig.~a}},\ul{Table.~b-1}}
\vspace{1zw}

\noindent
2-6) 1文字で構成されている語は専門用語ではないとして除外した.

\noindent
3) 単語親密度に基づく一般用語の除去

抽出された複合語のうち,単語親密度が5以上の語を除去した.以下に除去した語の例を示す.

\vspace{1zw}
\fbox{\parbox{39zw}{\ul{国際的},\ul{積極的},\ul{プライバシー},\ul{コミュニケーション},\ul{ガイドライン},\ul{困難さ},\ul{価値観},\ul{不十分},\ul{人間関係}}}
\vspace{1zw}



\section{実験及び評価}

\subsection{データセット}

本学の看護学研究科の社会人大学院生から提供された看護学に関する文献の後半を評価用データセットとした.正解データセットは上記社会人大学院生が人手で作成した.

\subsection{評価方法}

中川らの手法である従来手法と,ルールの拡張と一般的な語の除去を行った提案手法の2つについて,再現率と適合率の観点から比較・評価を行った.

ここで,再現率・適合率の計算に用いる「完全一致」「部分一致」という概念について説明する.「完全一致」とは抽出された専門用語に不要な語が連接されていないことであり,「部分一致」とは不要な語が連接されていることを言う.例として「情報検索数を数える」という文から専門用語を抽出する場合を考えてみる.上記句から専門用語を抽出する場合「\ul{情報検索}」が完全一致の専門用語であるのに対して「数」という名詞が誤って余分に連接された「\ul{情報検索}数」は部分一致の専門用語であるといえる.抽出された専門用語は研究情報検索システムにおいて検索キーワードの重み付けに使用される.不要な語が連接されている部分一致の専門用語であっても,検索キーワードと一致する部分を見つけ出し重み付けできることから,完全一致の専門用語だけではなく部分一致である専門用語も重要であるといえる.従って,再現率・適合率の計算においては完全一致,部分一致の両方について評価する.

\subsection{実験結果}

\begin{table}[b]
\caption{実験結果}
\input{01table3.txt}
\end{table}

表 3に各手法の部分一致,完全一致における再現率・適合率を示す.全ての評価指標において提案手法が従来手法を上回っている.部分一致における再現率は83{\%}から96{\%}となり,ほぼもれなく専門用語を抽出可能となったと言える.部分一致における適合率は42{\%}から55{\%}となり,不要な語を大幅に除去可能となったと言える.


\section{考察と今後の課題}

\subsection{考察}

従来研究と提案手法を比較する実験を行った結果,再現率と適合率の双方において部分一致の場合10{\%}以上の向上していることを確認した.

次に各アプローチの分析結果と考察について述べる.

\subsubsection{品詞の組合せの拡張に関する考察}

まず再現率向上ルールを適用した結果の分析結果について述べる.表 4のNo.2に再現率向上ルールを適用した結果を示す.ルールを適用することで部分一致では83{\%}から99{\%},完全一致では76{\%}から90{\%}となり,部分一致において専門用語をほぼもれなく抽出可能となったといえる.部分一致で抽出できなかった原因は,該当用語の構成要素に除去対象が含まれていたためである.具体的には連接対象外となっている語が含まれている「\ul{I} 期腺癌」が抽出できなかった.

\begin{table}[b]
\caption{従来研究とルール拡張の比較}
\input{01table4.txt}
\end{table}

完全一致において,約10{\%}抽出できていない専門用語が残存しているが,これは誤って不要な語が連接されたことに起因する.例として,「\ul{女性}レシピエント」「\ul{広範囲}孔脳症」のように不要な語が誤って連接されてしまうために完全一致の専門用語として抽出されていないものが存在した.このような語が完全一致における再現率低下要因の大部分を占めていた.この点に関しては,4.2で述べたように研究者情報検索では部分一致の専門用語でも十分対応可能であること,完全一致の再現率の向上を行うことで部分一致の再現率を低下させる可能性が高いことから,提案手法によって十分な再現率を得られていると言える.

次に適合率向上ルールの適用結果の分析結果について述べる.表 4のNo.3に再現率向上ルールと一般的な語の除去を除いた適合率向上ルールを適用した結果を示す.ルールを適用することで部分一致では37{\%}から41{\%},完全一致では27{\%}から30{\%}と不要な語が除去可能となっている.しかしながら,再現率を低下させずに適合率を向上させるという制約の下にルールを導出したにも関わらず,再現率が約0.1{\%}低下した.これは,適合率向上のためのルールを適用することで専門用語の構成要素が除去対象となり,抽出不可能となる専門用語が存在することに起因する.例として,専門用語である「10年生存率」では年月日を表す「10年」が除去され「生存率」だけが抽出されるため再現率が低下する.このような語は5語存在した.

一方,完全一致における再現率では,約0.3{\%}向上した.これは誤って連接されていた不要な語が適合率向上ルールによって除去されることにより完全一致の専門用語が増加したことに起因する.例として,「3時間」という不要な語が連接された「\ul{\mbox{3時間}}急速静脈内投与」から時間を表す「3時間」が除去され,完全一致の専門用語「急速静脈内投与」が得られる.このような語が7語存在した.


\subsubsection{単語親密度に基づく一般的な語の除去に関する考察}

評価用データセット(品詞によるルール適用済み)に対して本処理を行った時の専門用語候補数の推移を図2に示す.同様に再現率および適合率の変化を図3に示す.

\begin{figure}[b]
\begin{minipage}[b]{176pt}
\begin{center}
\includegraphics{15-3ia1f2.eps}
\end{center}
\caption{専門用語候補の抽出数推移}
\end{minipage}
\hfill
\begin{minipage}[b]{224pt}
\begin{center}
\includegraphics{15-3ia1f3.eps}
\end{center}
\caption{再現率--適合率グラフ}
\end{minipage}
\end{figure}

ここで,図2および図3における親密度の境界値がnであるとは,「一般的な語」として用いた語彙集合における単語親密度が $n+1$ 以上7以下である語彙を専門用語候補から除去したことを意味する.言い換えれば,専門用語候補には単語親密度が1からnの単語しか残存しないということである.なお,$n=7$ の場合は,専門用語候補から除去する「一般的な語」は無いものとする.図2から分かるように,境界値が5においてはほとんど専門用語しか残存していないことが分かる.

図3において,単語親密度の研究結果を導入することにより適合率が向上したものの,部分一致における適合率は55{\%}とまだ改善の余地が残されている.除去できなかった1,933語のうち,統計用語が141語,「〜病院,〜先生,〜大学」といった固有名詞が127語,存在した.それ以外では,「医療水準,自己申告,謝辞,本論分」のような複合名詞や単名詞も看護関係では専門用語といえない用語が残ってしまった.これらの原因は,以下に起因する.

・統計用語といった他分野の専門用語が含まれていた.

・固有名詞は全て専門用語候補としていたが,専門用語と言えない固有名詞が存在した.

・佐藤らが用いた語彙集に含まれていない一般的な語が存在した.

統計用語に関しては統計用語集を利用することにより除去する.あるいは他の領域における専門用語抽出方法が確立されていれば,それを利用してその領域の専門用語を除去する.固有名詞に関しては,看護学分野に関係しない一般的な語に固有名詞が連接されている語は一般的な語として除去するルールを追加することが考えられる.佐藤らが用いた語彙集に含まれていない用語に関しては,語彙集の充実が待たれるところである.

再現率に関しては,完全一致における再現率は82{\%}と改善の余地があるものの,部分一致における再現率は96{\%}と高い値を示しており,適合率を向上させながらほぼ全ての専門用語を抽出可能になったといえる.誤って除去してしまった4{\%}の語について述べる.親密度の境界を5とした場合,68種類78語の専門用語が誤って除去された.以下に誤って除去された専門用語の一部を示す.

\vspace{1zw}
\fbox{\parbox{39zw}{ぜん息,アレルギー,医師,炎症,下痢,解毒,患者,吸引,救命,血小板,抗生物質,更年期,高血圧,採血,酸素吸入,止血,治療,食生活,心電図,蛋白,肺がん,肺炎,貧血,副作用}}
\vspace{1zw}

親密度の境界を6とした場合,10種類16語の専門用語が誤って除去された.以下に誤って除去された専門用語の一部を示す.

\vspace{1zw}
\fbox{アレルギー,医師,患者,高血圧,死亡,治療,食生活,診断,入院,輸血}
\vspace{1zw}

専門用語が誤って除去された事は,全ての看護専門用語が馴染みのない親密度の低い語ではないことを意味している.つまり,看護は生活の一部であり,一部の専門用語は日常生活の中で利用されているということである.図3において,適合率は親密度境界値が1から5までほとんど低下しないが,再現率は親密度境界が低くなるに従い少しずつ低下しており,適合率のような急激な変化は見られない.この違いも,専門用語が全て馴染みのない親密度の低い語ではないことに起因していると考えられる.親密度が一定以下(ここでは親密度境界=5)の語は,前述の統計用語や固有名詞などを除けば専門用語である確率が高い.一方,専門用語が全て親密度の低い訳ではないことから,親密度境界値が低くなるに従い少しずつ看護の専門家だけに通用する用語になっていくためと考えられる.



\subsection{今後の課題}

日本語に着目して専門用語抽出法では,英語形態素を含んだ専門用語を抽出できない.英語形態素を含んだ専門用語数を調査したところ,学習用データセットにおいて12.2{\%}(317語/2,587語),評価用データセットにおいて13.7{\%}(290語/2,123語)と専門用語全体の10{\%}を超えており,多いとは言えないが英語形態素を含んだ専門用語についても抽出技法を確立することが望まれる.

また,専門用語といえない固有名詞が存在することが確認された.これに関しては,新たなルールを追加することが望ましい.



\section{まとめ}

本論文では専門用語になりうる品詞の組合せを拡張することにより看護学分野における専門用語抽出の再現率の向上を図った.更に単語親密度の研究と組み合わせることで適合率の向上を図った.

再現率向上においては,専門用語抽出のルールに連接可能な品詞の追加および特定の語の追加と,連続する片仮名を連接して1語とする形態素解析手法の改善によって看護学分野における専門用語抽出の再現率が99{\%}とほぼ全ての専門用語を抽出可能となった.適合率向上においては,専門用語の構成要素としない語を除去するルールを追加することで再現率を低下させずに適合率を向上させることができた.更に単語親密度5以上の語彙(基本語彙)を除去するルールにより,再現率が99{\%}から96{\%}と僅かに低下したものの適合率は41{\%}から55{\%}と大幅に向上した.

今後の課題として,英語を含んだ専門用語抽出技法,専門用語でない固有名詞を除去する技法を確立することが望まれる.

\acknowledgment

本研究は,
岩手県学術研究振興財団研究費補助金及び岩手県立大学全学プロジェクト等研究費の助成を受けて行ったものである.本研究で用いた実験データの提供や専門用語判定作業を行って頂いた岩手県立大学 
看護学研究科の大学院生の皆様にはこの場を借りて深く感謝する.




\begin{thebibliography}{}

\item
相澤彰子,野末道子,今尚之,坂本真至,中渡瀬秀一 (2005). 土木関連用語辞書の見出し語の分析と検索システムにおける活用に関する考察. 情報処理学会研究報告,自然言語処理研究会,\textbf{169} (19),pp.~131--138.

\item
井上大悟,永井秀利,中村貞吾,野村浩郷,大貝晴俊 (2001).“医療論文抄録からのファクト情報抽出を目的とした言語分析.”自然言語処理,\textbf{141} (17),pp.~103--110.

\item
木浪孝治,池田哲夫,高山毅,武田利明 (2006). 品詞の組合せの拡張による看護学分野での専門用語抽出再現率の改善. 情報処理学会 データベースシステム研究会 電子情報通信学会データ工学専門委員 日本データベース学会共催夏のデータベースワークショップDBWS2006,Vol.~2006,No.~78,pp.~313--320.

\item
国立国語研究所 (2004). 国立国語研究所資料集14「分類語彙表 増補改訂版」,大日本図書.

\item
中川裕志,森辰則,湯本紘彰 (2003).“出現頻度と連接頻度に基づく専門用語抽出.”自然言語処理,\textbf{10} (1),pp.~27--45.

\item
奈良先端科学技術大学院大学自然言語処理学講座,日本語形態素解析器 ChaSen,http://ChaSen.naist.jp/hiki/ChaSen/.

\item
佐藤浩史,笹原要,金杉友子,天野成昭 (2004).“単語親密度に基づく基本的語彙の選定.”人工知能学会論文誌,\textbf{19} (6),pp.~502--510.

\item
立石健二,久寿居大 (2006). 複数の作成者情報付き文書から専門用語抽出. 情報処理学会論文誌データベース,Vol.~47,No.~SIG8,pp.~24--32.

\item
TermExtract (2006). 「茶筅」用モジュール ``Chasen.pm'' の説明,http://gensen.dl.itc.u-tokyo.ac.jp/doc/Chasen.html.

\item
辻河亨,吉田稔,中川裕志 (2003).“語彙空間の構造に基づく専門用語抽出.”情報処理学会研究報告,自然言語処理研究会,\textbf{159} (22),pp.~155--162.

\end{thebibliography}



\begin{biography}
\bioauthor{木浪 孝治}{
2005年岩手県立大学ソフトウェア情報学部卒業.2007年同大学院修士課程修了.修士(ソフトウェア情報学).同年,(株)日立製作所入社.DBMSの設計開発に従事.}

\bioauthor{池田 哲夫(正会員)}{
1981年東京大学大学院情報科学専攻修士課程修了.同年日本電信電話公社入社.
岩手県立大学教授を経て2006年静岡県立大学教授.情報検索,GIS等の研究に従
事.博士(工学)(東京大学).}

\bioauthor{村田 嘉利}{
1979年名古屋大学大学院修了.同年日本電信電話公社入社.2003年静岡大学理工学研
究科後期博士課程修了.博士(工学).2006年から岩手県立大学ソフトウェア情報学
部教授ならびにNiCT研究員.データベース応用の研究に従事.}

\bioauthor{高山  毅}{
1966年生.1995年筑波大学大学院博士課程工学研究科電子・情報工学専攻修了.博士(工学).現在,岩手県立大学ソフトウェア情報学部准教授.データベース応用システムの研究開発に従事.情報処理学会会員.}

\bioauthor{武田 利明}{
1979年千葉大学看護学部看護学科卒業.1982年同大学院修士課程修了.同年,帝人(株)入社,1994年医薬開発研究所主任研究員,獣医学博士,現在,岩手県立大学看護学部教授.専門は基礎看護学,看護技術に関する実証的研究に従事.}
\end{biography}


\biodate

\end{document}

