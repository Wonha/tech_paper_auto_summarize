


\documentstyle[eclepsf,nlpbbl]{jnlp_e}

\setcounter{page}{59}
\setcounter{巻数}{8}
\setcounter{号数}{3}
\setcounter{年}{2001}
\setcounter{月}{7}
\受付{October}{18}{2000}
\採録{April}{13}{2001}

\setcounter{secnumdepth}{3}

\newcommand{\deflabel}[1]{}

\newenvironment{deflist}[1]{}{}

\begin{document}


\title{}
\etitle{A trainable method for pronominal anaphora \\ resolution using shallow information}

\author{
         パウル ミヒャエル\affiref{ATR} \and     
         隅田  英一郎\affiref{ATR}
}
\eauthor{
         Michael Paul\affiref{ATR}  \and     
         Eiichiro Sumita\affiref{ATR}
         }

\headtitle{A trainable method for pronominal anaphora resolution $\ldots$} 
\headauthor{Paul,~M.~and~Sumita,~E.}

\affilabel{ATR}
    {ATR音声言語通信研究所}
    {ATR Spoken Language Translation Research Laboratories}

\jkeywords{照応解析, 機械学習, 優先度}
\ekeywords{Anaphora Resolution, Machine Learning, Saliency Measure}


\jabstract{}

\eabstract{
         We propose a corpus-based approach to anaphora resolution of Japanese pronouns
         combining a machine learning method and statistical information. First,
         a decision tree trained on an annotated corpus determines the 
         coreference relation of a given anaphor and antecedent candidates
         and is utilized as a filter in order to reduce the number of potential
         candidates. In the second step, preference selection is achieved by taking into
         account the frequency information of coreferential and non-referential pairs
         tagged in the training corpus as well as distance and counting features
         within the current discourse.
}


\maketitle

\section{Introduction}
\label{intro}

Coreference information is relevant for numerous NLP systems.
Our interest in anaphora resolution is based on the demand for machine translation
systems to be able to translate anaphoric expressions in agreement with
the morphosyntactic characteristics of the referred object in order to prevent contextual
misinterpretations.

So far various approaches to anaphora resolution have been proposed.
In this paper a {\em machine learning approach} (decision tree) is combined with a preference
selection method based on the {\em frequency} information of non-/coreferential pairs tagged
in the corpus as well as {\em counting} and {\em distance} features within the current discourse.

The advantage of machine learning approaches is that they result in modular anaphora
resolution systems automatically trainable from a corpus with no or only a minimal
amount of human intervention. In the case of decision trees, we do have to provide information
about possible antecedent indicators (syntactic, semantic, and pragmatic features) contained
in the corpus, but the relevance of features for the resolution task is extracted
automatically from the training data.

The machine learning approaches using decision trees proposed so far have focused on preference
selection criteria directly derived from the decision tree results.
The work described in \cite{Conolly94} utilized a decision tree capable of judging which one
of two given anaphor-antecedent pairs is ``better''. Due to the lack of a strong assumption on
``transitivity'', however, this sorting algorithm is more like a greedy heuristic search as it
may be unable to find the ``best'' solution.

The preference selection for a single antecedent in \cite{Aone95} is based on the maximization
of confidence values returned from a pruned decision tree for given anaphor-candidate pairs.
However, relations between single attributes cannot be obtained automatically within the
decision tree learning phase.
Moreover, these confidence values do not take into account any contextual information from the discourse 
the decision tree is applied to.

The preference selection in our approach is based on the combination of statistical
frequency information as well as distance and counting features in the discourse. Therefore, our decision
tree is not applied directly to the task of preference selection, but aims at the elimination
of irrelevant candidates based on the knowledge obtained from the training data.

The decision tree is trained on syntactic (lexical word attributes), semantic, and
primitive discourse (distance, counting features) information and determines the coreferential relation
between an anaphor and antecedent candidates in the given context. Irrelevant antecedent
candidates are filtered out, achieving a noise reduction for the preference selection
algorithm. A preference value is assigned to each potential anaphor-candidate pair depending
on the proportion of non-/coreferential occurrences of the pair in the training corpus
({\em frequency ratio}), the relative position of both elements in the discourse
({\em distance}), and the number of previous occurrences in the discourse ({\em count}).
The candidate with the maximal preference value is resolved as the antecedent of the anaphoric expression.

\section{Corpus-based anaphora resolution}
\label{res}

In this section we introduce a new approach to anaphora resolution based on 
coreferential properties automatically extracted from a training corpus.
The correct antecedents of anaphoric expressions ($A$) are tagged in our annotated corpus.
Antecedent candidates ($C$) are all nouns preceding the anaphor identified according to part-of-speech tags.

In the first step, the decision tree filter is trained on the linguistic, discourse, and coreference
information annotated in the training corpus, which is described in Section~\ref{data}.

\begin{figure}[htb]
  \vspace*{-1\baselineskip}
  \begin{center}

    \atari(105,34)

    \caption{System outline}\label{pic-outline}
  \vspace*{-0.5\baselineskip}
  \end{center}
\end{figure}

The resolution system in Figure~\ref{pic-outline} applies the coreference filter 
(cf. Section~\ref{coref-analysis}) to all anaphor-candidate pairs ($A_{i}+C_{ij}$) found in the history. 
The reduced set ($A_{i}+C_{ik}$) consists of all potential candidates and forms the input of the preference
algorithm, which selects the most salient anaphor-candidate pair ($A_{i}+C_{ip}$) as described in Section~\ref{pref}.
Preliminary experiments are conducted and the performance of our system is evaluated in Section~\ref{eval}. \\[-0.5em]

\subsection{Tagged corpus}
\label{data}

For our experiments we use 500 Japanese spoken-language dialogues annotated with coreferential tags
out of the {\em ATR-ITL Speech and Language Database} \cite{Takezawa98}.
They include 1745 pronominal annotations, whereby the anaphoric expressions
used in our experiments are limited to those referring to nominal antecedents.

Besides the anaphor type, we also include morphosyntactic information like stem form and inflection
attributes for each surface word as well as semantic codes \cite{Ohno81} for content words in this corpus.

\begin{figure}[htb]
  \vspace*{-1\baselineskip}
  \begin{center}
\atari(108,54)
    \caption{Example dialogue}\label{pic-data}
  \vspace*{-1.5\baselineskip}
  \end{center}
\end{figure}

In the example dialogue between a hotel receptionist (r) and a customer (c) listed in Figure~\ref{pic-data},
anaphora are marked with a box, whereas all underlined nouns preceding the respective anaphor form the
list of possible candidates.
The proper noun (r1)``シティホテル[City Hotel]'' is tagged as the antecedent of the
pronoun (c1)``そちら[there]''.

According to the tagging guidelines used for our corpus, an anaphoric tag refers to the most recent
antecedent found in the dialogue. However, this antecedent might also refer to a previous one, e.g.,
(r3)``こちら[here]'' refers to (c1)''そちら[there]'', which refers to (r1) ''シティホテル [City Hotel]''.
Thus, the {\em transitive closure} between the anaphora and the first mention of the
antecedent in the history defines the set of {\em positive examples}, whereas
the nominal candidates outside the transitive closure are considered {\em negative examples}
for coreferential relationships. In our example the sets consist of the following examples.

\renewcommand{\arraystretch}{}

\begin{center}
  \begin{scriptsize}
  \begin{tabular}{cccccc}
    positive examples & \hspace*{1cm} & \multicolumn{3}{c}{negative examples} \\
    \cline{1-1} \cline{3-5}
                          & &                     &                     & \\[-0.5em]
    (こちら,シティホテル) & & (こちら,田中)       & (こちら,お客様)     & (こちら,十日)       \\[-0.5em]
    [here,City Hotel]     & & [here,Tanaka]       & [here,guest]        & [here, 10${}^{th}$] \\
    (そちら,シティホテル) & & (こちら,弘子)       & (こちら,お名前)     & (そちら,田中)       \\[-0.5em]
    [there,City Hotel]    & & [here,Hiroko]       & [here,name]         & [there,Tanaka]      \\
                          & & (こちら,ホテル)     & (こちら,スペル)     &  (そちら,弘子)       \\[-0.5em]
                          & &  [here,hotel]       & [here,spelling]     & [there,Hiroko]      \\
                          & & \multicolumn{3}{l}{(こちら,ティーエーエヌエーケーエー)} \\[-0.5em]
                          & & \multicolumn{3}{l}{[here, TANAKA]} \\[0.5em]
  \end{tabular}
  \end{scriptsize}
\end{center}

The difficulty of our resolution task can be verified according to the average number of antecedent
candidates, i.e. the sum of positive and negative examples, for a given pronoun.
In our corpus, the average number is 36.7 (こちら: 41.0, そちら: 28.8, etc.).

Table~\ref{tab-tag} summarizes the corpus annotations for anaphoric expressions and antecedent
candidates of the example dialogue given in Figure~\ref{pic-outline}.

\setlength{\tabcolsep}{5pt}
\begin{table}[hbt]
  \vspace*{-0.5\baselineskip}
   \begin{center}
   \begin{scriptsize}
    \caption{Corpus annotations}\label{tab-tag} 
    \vspace*{0.5em}
    \begin{tabular}{|r|c|c|c|c|c|}
      \hline
      ID & word         & stem          & pos           & sem                  & ant \\ \hline \hline
      2  & シティホテル & シティホテル  & 普通名詞      &  \{shop\}            & --    \\ [-0.5em]
         & [City Hotel] &  [City Hotel] & [common noun] &                      &       \\ \hline
      6  & 田中         & 田中          & 普通名詞      &  \{name\}            & --    \\ [-0.5em]
         & [Tanaka]     & [Tanaka]      & [common noun] &                      &       \\ \hline
      7  & 弘子         & 弘子          & 普通名詞      &  \{name\}            & --    \\ [-0.5em]
         & [Hiroko]     &[Hiroko]       & [common noun] &                      &       \\ \hline
      12 & そちら       & そちら        & 代名詞        &  \{demonstratives\}  & (2,2) \\ [-0.5em]
         & [there]      & [there]       & [pronoun]     &  \{directions\}      &       \\ [-0.5em]
         &              &               &               &  \{third person\}    &       \\ \hline
      14 & ホテル       & ホテル        & 普通名詞      &  \{shop\}            & --    \\ [-0.5em]
         & [hotel]      & [hotel]       & [common noun] &                      &       \\ \hline
      21 & お客様       & 客様          & 普通名詞      &  \{guest\}           & --    \\ [-0.5em]
         & [guest]      & [guest]       & [common noun] &                      &       \\ \hline
      23 & お名前       & 名前          & 普通名詞      &  \{name\}            & --    \\ [-0.5em]
         & [name]       & [name]        & [common noun] &                      &       \\ \hline
      25 & スペル       & スペル        & 普通名詞      &  \{notation\}        & --    \\ [-0.5em]
         & [spelling]   & [spelling]    & [common noun] &                      &       \\ \hline
      32 & ティーエーエヌ  & AA列     & ローマ字      &  \{letter\}          & --    \\ [-0.5em]
         & エーケーエー    & [letter    & [letter]      &                      &       \\ [-0.5em]
         & [TANAKA]     & sequence]     &          &                      &       \\ \hline
      35 & 十日         & N日          & 普通名詞      &  \{day\}             & --    \\ [-0.5em]
         & [10${}^{th}$]& [day]         & [common noun] &  \{period\}          &       \\ [-0.5em]
         &              &               &               &  \{unit\}            &       \\ \hline
      37 & こちら       & こちら        & 代名詞        &  \{demonstratives\}  & (12,12) \\ [-0.5em]
         & [here]       & [here]        & [pronoun]     &  \{directions\}      &       \\ [-0.5em]
         &              &               &               &  \{oneself\}         &       \\ \hline
     \end{tabular}
   \end{scriptsize}
   \end{center}
   \vspace*{-1\baselineskip}
\end{table}

They consist of tags for a unique identifier for each morpheme ({\em ID}), the surface word ({\em word}),
the stem form ({\em stem}), the part-of-speech ({\em pos}), the semantic code ({\em sem}),
and the antecedent ({\em ant}).
The expression ``\{$\ldots$\}'' denotes the semantic class assigned to the respective word.
``[$\ldots$]'' gives the English equivalent of the respective Japanese expression.

The antecedents tagged in our corpus consist of either single nouns or compound nouns, i.e. a 
sequence of single nouns. The tags used to mark coreferential relationships ({\em ant}) consist
of a range {\em (startID,endID)} of morphological identifiers. {\em startID} is the ID of the
first morpheme in the tagged sequence of single nouns, whereas {\em endID} is the ID of the
last morpheme of this sequence. For example, a coreferential reference towards the compound noun
「田中弘子」 [Hiroko Tanaka] given in Table~\ref{tab-tag} would be specified as {\em ant=(6,7)}.
In the case of a single noun, {\em startID} and {\em endID} are identical.

Based on the corpus annotations illustrated in Table~\ref{tab-tag}, we can obtain additional
statistical information describing the characteristics of the underlying data.

First, we extract the frequency information of coreferential anaphor-antecedent pairs ($freq^+$)
and non-referential pairs ($freq^-$) from the training data.
$freq^+$ denotes the number of times the anaphor-candidate pair is tagged coreferentially,
whereas $freq^-$ is the number of times the anaphor-candidate pair does not occur coreferential
in the training corpus.
In Table~\ref{tab-freq} some examples are given for surface form ({\em word}) and
semantic code ({\em sem}) combinations.

\setlength{\tabcolsep}{2pt}
\begin{table}[hbt]
  \vspace*{-1\baselineskip}
   \begin{center}
   \begin{footnotesize}
    \caption{Frequency data}\label{tab-freq} 
    \vspace*{0.5em}
    \begin{tabular}{|p{0.5em}lccrrp{0.5em}|}
      \hline
      & \multicolumn{1}{c}{type} & anaphor & candidate    & $freq^+$ & $freq^-$   & \\
      \hline \hline
      & word-word & そちら[there]  & シティホテル[City Hotel] &     6    &     0 & \\
      &           & そちら[there]  & 田中[Tanaka]             &     0    &    11 & \\
      &           & こちら[here]   & 十日[10${}^{th}$]               &     0    &     0 & \\
      \hline
      & word-sem  & こちら[here]   & \{shop\}                 &    33    &    33 & \\
      \hline
      & sem-sem   & \{demonstratives\} & \{shop\}             &    51     &   18 & \\
      \hline
    \end{tabular}
   \end{footnotesize}
   \end{center}
   \vspace*{-1\baselineskip}
\end{table}

Moreover, each dialogue is subdivided into {\em utterances} consisting of
one or more {\em chunks} divided by conjunctions.
Therefore, distance features are available on the utterance ($d_{uttr}$), chunk ($d_{chunk}$),
candidate ($d_{cand}$), and morpheme ($d_{morph}$) levels.
For example, the distance values for the pronoun (r3)``こちら[here]'' and
the antecedent (r1)''シティホテル[City Hotel]'' as well as those
for the most recent candidate (r3)``十日[10${}^{th}$]'' in our sample dialog
in Figure~\ref{pic-data} are given in Table~\ref{tab-dist}.

\setlength{\tabcolsep}{10pt}
\begin{table}[hbt]
   \vspace*{-0.5\baselineskip}
   \begin{center}
   \begin{footnotesize}
    \caption{Distance features}\label{tab-dist} 
    \vspace*{0.5em}
    \begin{tabular}{|ccl@{=}r|}
      \hline
      anaphor (ID) & candidate (ID) & \multicolumn{2}{c|}{distance} \\
      \hline \hline
                  &                  & $d_{uttr}$   &  4  \\
      こちら (37) & シティホテル (2) & $d_{chunk}$  &  7  \\
      $[$here]    &  [City Hotel]    & $d_{cand}$   &  10  \\
                  &                  & $d_{morph}$  &  35 \\
      \hline
                  &                  & $d_{uttr}$   &  0  \\
      こちら (37) & 十日 (35)        & $d_{chunk}$  &  0  \\
      $[$here]    & [10${}^{th}$]      & $d_{cand}$   &  1  \\
                  &                  & $d_{morph}$  &  2  \\
      \hline
     \end{tabular}
   \end{footnotesize}
   \end{center}
   \vspace*{-1\baselineskip}
\end{table}

Additional statistics can be extracted from the dialogue by counting the occurrences of candidate features
in the previous discourse. Table~\ref{tab-count} lists the values of the counting feature for
the stem form ($c_{stem}$), the part-of-speech ($c_{pos}$), and
the semantic code ($c_{sem}$) of the respective candidates.

\vspace*{8mm}

\setlength{\tabcolsep}{10pt}
\begin{table}[hbt]
   \vspace*{-0.75\baselineskip}
   \begin{center}
   \begin{footnotesize}
    \caption{Count features}\label{tab-count} 
    \vspace*{0.5em}
     \begin{tabular}{|l|ccc|}
      \hline
      \multicolumn{1}{|c|}{count}  & シティホテル (2) &  ホテル (14)      & 十日 (35)  \\
      \hline \hline
      $c_{stem}$ & 0               &  0               & 0  \\
      $c_{pos}$  & 0               &  3               & 7   \\
      $c_{sem}$  & 0               &  1               & 0  \\
     \hline
     \end{tabular}
   \end{footnotesize}
   \end{center}
   \vspace*{-0.75\baselineskip}
\end{table}

\subsection{Coreference analysis}
\label{coref-analysis}

The resolution of anaphoric expressions is important in numerous
natural language processing applications and has been intensively
studied in the past. Various features indicating coreferential
relationships have already been proposed. But most of the previous
approaches make use of quite sophisticated linguistic knowledge, such as,
for example, dependency structures or discourse markers.
The richness of this knowledge, however, has its drawbacks
with respect to the scalability of a resolution system to new tasks or different languages.

Therefore, we are proposing a trainable resolution approach using shallow information,
i.e., syntactic and semantic word attributes as well as primitive discourse information
extracted from the morphological analysis of the input utterances.

To learn the coreferential relationships from our corpus we have chosen
a C4.5-like machine learning algorithm \cite{Quinlan93}. 
The set of attributes used for the decision tree learning consists of discrete
and continuous values extracted from the training corpus (cf. Section~\ref{dt-attr}).
Two decision tree classes are used to determine whether there is a coreferential
relationship or not (cf. Section~\ref{dt-learn}).

During our anaphora resolution algorithm, the learned decision tree is applied to all
possible anaphor-candidate pairs  (cf. Section~\ref{dt-apply}).
If the coreferential relationship can be verified, the respective candidate is taken
into further consideration by the resolution algorithm. Otherwise, the candidate is
judged as unrelated and deleted from the list of potential candidates,
thus achieving a noise reduction for the preference selection scheme described
in Section~\ref{pref}.

\subsubsection{Training attributes}
\label{dt-attr}

For the learning of the decision tree we use the word attributes and the discourse
information described in Table~\ref{tab-dt-attr}.

\setlength{\tabcolsep}{5pt}
\begin{table}[hbt]
   \vspace*{-1\baselineskip}
   \begin{center}
    \caption{Training attributes}\label{tab-dt-attr} 
     \begin{tabular}{|ll|l|}
      \hline
      \multicolumn{2}{|c|}{category}        & \multicolumn{1}{c|}{sample}     \\
      \hline \hline
      \multicolumn{2}{|l|}{content word:}   & ``そちら'', ``行く''             \\
                                            & {\footnotesize semantic code}     & \{name\}, \{shop\}              \\
                                            & {\footnotesize part-of-speech}    & 代名詞, 普通名詞, 本動詞        \\
      \hline
      functional word: & {\footnotesize particle}    & ``は'', ``を''; ``と'', ``や''        \\ 
                       & {\footnotesize conjunction} & ``ので'', ``たら''                    \\
                       & {\footnotesize conjugation} & ``ない'', ``れる'', ``た''            \\
      \hline
      discourse:       & {\footnotesize distance}    & ({\em continuous values})       \\
                       & {\footnotesize count}       & ({\em continuous values})       \\
      \hline
     \end{tabular}
   \end{center}
   \vspace*{-1\baselineskip}
\end{table}

According to the category, we distinguish attributes concerning the stem form of specific content words,
the semantic classification of words and their part-of-speech as well as inflectional attributes.
Moreover, information about syntactical markers like particles or sentence conjunctions as well as
discourse information about distance and number of occurrences are used for the determination of coreferential relationships.

\begin{deflist}{}
 \item[{\bf content word}] \hfill \\
   For the resolution of pronouns, we have to check not only which anaphoric expressions are involved,
   but also the existence of other content words, like, for example, the sentence predicate, are verified for the respective input sentence.
 \item[{\bf semantic code}] \hfill \\
   For the semantic classification of content words, we use the {\em Ruigo-Shin-Jiten}, a 3-layered semantic hierarchy
   distinguishing 1000 semantic classes \cite{Ohno81}.
 \item[{\bf part-of-speech}] \hfill \\
   We distinguish 33 parts-of-speech for verbs (e.g., 本動詞, 助動詞, 判定詞), nominal expressions (e.g., 普通名詞, 代名詞),
   adjectives (e.g., 形容詞, 数詞) and functional words (e.g., 格助詞, 接続詞). 
 \item[{\bf functional word}] \hfill \\
   In Japanese, the grammatical role of specific content words is marked according to particles succeeding the expression.
   We distinguish {\em case particle} (e.g., は, が, を, に), conjunction particle (e.g., と, や), compound particle
   (e.g., の, と言う), and adverbial particle (e.g., とか, など).
   Moreover, the existence of specific conjunctions (e.g., ながら, ので) as well as the conjugation form of the sentence predicate
   is verified for the determination of coreferential relationships.
 \item[{\bf discourse}] \hfill \\
   The attributes extracted from the respective discourse are described in more detail in Section~\ref{data}.
   In our experiments described in Section~\ref{eval}, we use information about the occurrence of specific
   content words and its distance in the discourse. \\[-0.5em]
\end{deflist}

\noindent
The scope of a specific attribute is not limited to the specified expression, e.g., the anaphor, but can also be applied to
preceding or succeeding expressions using the keywords listed below. \\

\begin{deflist}{}
 \item[{\bf :backward}] \hfill \\
   check all elements preceding the specified expression,
   e.g., {\em ``Which semantic classes are assigned to the content word preceding the sentence predicate?''} \\
         {\tt (:predicate :backward :semcode)}
 \item[{\bf :latest-cont}]  \hfill \\
   check the most recent content word preceding the specified expression,
   e.g., {\em ``Is the candidate modified by a verbal phrase?''} 
        {\tt (:candidate :latest-cont (:pos  "本動詞"))}
 \item[{\bf :latest}]  \hfill \\
   check the element directly preceding the specified expression,
   e.g., {\em ``Is the candidate part of a compound noun phrase?''}
        {\tt (:candidate :latest ("と" "並立助詞"))}
 \item[{\bf :here}]  \hfill \\
   check the attribute of the specified expression,
   e.g., {\em ``Is そちら the stem form of the anaphor?''} 
        {\tt (:anaphor :here ("そちら" "代名詞"))}
 \item[{\bf :next}]  \hfill \\
   check the element directly succeeding the specified expression,
   e.g., {\em ``Is the anaphor marked as the subject?''} 
        {\tt (:anaphor :next ("が" "格助詞"))}
 \item[{\bf :next-cont}]  \hfill \\
   check the next content word succeeding the specified expression,
   e.g., {\em ``How often did the content word succeeding the anaphor occurred previously?''} \\
        {\tt (:anaphor :next-cont :count)}
 \item[{\bf :forward}]   \hfill \\
   check all elements succeeding the specified expression, 
   e.g., {\em ``Does the utterance contains a specific conjunction?''}
        {\tt (:predicate :forward ("ながら" "接続詞"))}\\[-0.5em]
\end{deflist}

\noindent
For the training of the decision tree we provide the complete set of attributes described above.
Besides these attributes, various coreference indicators have been proposed
in previous resolution systems which are not included in our approach because they require
a more sophisticated linguistic analysis of the input data. 
These information sources include structural parsing of the input sentence \cite{Lappin94},
semantic constraints on verb case and verbal semantic attributes \cite{Nakaiwa96},
the analysis of discourse marker \cite{Aone95} as well as topic and focus information
\cite{Murata97}.
\vspace*{-0.25em}

\subsubsection{Learning phase}
\label{dt-learn}

During the iterative analysis of each dialog, anaphoric expressions are identified
according to the assigned coreference tags. Previously mentioned nouns
are considered as possible antecedent candidates.

Questions are applied to each anaphor-candidate pair by either matching specified
expressions in the respective utterances (discrete values) or calculating attribute values
in the given context (continuous values).
We distinguish four groups of question sets against which these attributes are tested:

\begin{itemize}
\item unary features of the anaphor
      (e.g., lexical word attributes, grammatical role marker)
\item unary features of the candidate
\item features of expressions other than the anaphor and the candidate in the respective utterances 
      (e.g., modifying constituents, sentence predicate)
\item features concerning anaphor and candidate alike 
      (e.g., attribute agreement, relative distance)
\end{itemize}

The application of the four question sets to each anaphor-candidate pair
yields a single attribute vector classifying the characteristics of the given reference.
In the case of antecedents, this vector is assigned to the coreference class, whereas vectors
of the non-antecedent candidates form a separate class determining the non-referential relationship.

The amount of attribute vectors for all training samples forms the input of the learning
method. By optimizing the entropy value for each subset, the automatic classifier algorithm
produces a decision tree ranking important attributes higher in the tree in order to achieve
an early decision about the classification of the specified input \cite{Quinlan93}.
\vspace*{-0.25em}

\subsubsection{Application phase}
\label{dt-apply}

During anaphora resolution, the decision tree is used as a module determining the
coreferential property of each anaphor-candidate pair (cf. Figure~\ref{pic-outline}).
For each anaphoric expression, a candidate list, i.e., a list of the nominal candidates preceding
the anaphor element in the current discourse, is created.
The decision tree filter is then successively applied to all anaphor-candidate pairs.

\begin{figure}[ht]
  \vspace*{-1.75\baselineskip}
  \begin{center}
\atari(100,46)
    \caption{Decision tree filter}\label{pic-dt}
  \vspace*{-1.5\baselineskip}
  \end{center}
\end{figure}

Starting with the top node of the decision tree, the question assigned to this node is
tested against the input, i.e., the respective anaphor-candidate attribute vector.
Depending on the truth value of the question, the procedure descends to the respective sub-branch.
The verification procedure is continued until a leaf containing the classification result 
({\em coreference} vs. {\em no-relation}) is reached (cf. Figure~\ref{pic-dt}).

If the decision tree application results in the {\em no-relation} class, the candidate is
eliminated from the list of potential antecedents forming the input of the
preference selection algorithm.

\subsection{Preference selection}
\label{pref}

The primary order of candidates is given by their word distance from the anaphoric expression.
A straightforward preference strategy we could choose is the selection of the most recent
candidate as the antecedent, i.e., the first element of the candidate list.
However, the experimental results described in Section~\ref{eval} show that it is not sufficient
to rely only on single indicators like {\em recency} in order to determine the saliency of potential
anaphor-candidate pairs.

We propose a preference selection scheme (Section~\ref{pref-pref}) which is based on the combination
of statistical information about the ratio of coreferential occurrences extracted from our training
corpus (Section~\ref{pref-freq}), discourse information about the number of candidate occurrences
in the current conversation (Section~\ref{pref-count}), and recency
information (Section~\ref{pref-dist}).

\subsubsection{Frequency ratio}
\label{pref-freq}

An examination of our corpus indicates that similarities to references in our training
data might be useful for the identification of those antecedents. {\em The more frequently
an anaphor-candidate pair is tagged coreferential in the training corpus, the more
likely is a coreferential relationship for this pair}.

In order to measure the similarity of the input data to those of the training corpus,
we define the {\em ratio} of a given anaphor-candidate pair based on the frequency
information described in Section~\ref{data} as follows:

\vspace*{-1em}
\setlength{\arraycolsep}{1pt}
\begin{eqnarray*}
  ratio & = & \left\{
                \begin{array}{l@{\ :\ }l}
                  -\delta & (freq^+ = freq^- = 0) \\
                  \frac{freq^+\ -\ freq^-}{freq^+\ +\ freq^-}
                         & otherwise
                \end{array}
              \right. \\[-0.5em]
\end{eqnarray*}

The value of $ratio$ is in the range of $[-1,+1]$, whereby $ratio=-1$ is the case of exclusive
non-referential relationships and $ratio=+1$ is the case of exclusive coreferential relationships.
In order for referential pairs occurring in the training corpus with $ratio=0$ to be preferred
to those without frequency information, we slightly decrease the $ratio$ value of the latter
by a factor of $\delta$.

The {\em ratio} values of the sample frequency data given in Table~\ref{tab-freq} are summarized in
Table~\ref{tab-ratio}.

\setlength{\tabcolsep}{5pt}
\begin{table}[hbt]
  \vspace*{-0.75\baselineskip}
   \begin{center}
   \begin{footnotesize}
    \caption{Frequency ratio}\label{tab-ratio} 
    \begin{tabular}{|lccc|}
      \hline
      \multicolumn{1}{|c}{type} & anaphor & candidate & $ratio$ \\
      \hline \hline
      word-word & そちら[there]  & シティホテル      &    1   \\[-0.5em]
                &                & [City Hotel]      &        \\
                & そちら[there]  & 田中[Tanaka]      &   -1   \\
                & こちら[here]   & 十日[10${}^{th}$]   & -$\delta$ \\
      \hline
      word-sem  & こちら[here]   & \{shop\}          &    0   \\
      \hline
      sem-sem   & \{demonstratives\} & \{shop\}      & 0.48   \\
      \hline
    \end{tabular}
   \end{footnotesize}
   \end{center}
   \vspace*{-1.5\baselineskip}
\end{table}

The domain dependency of this saliency feature can be varied according to the selection
of the frequency type used for the calculation of the {\em ratio} value.
The most {\em domain-specific} frequency information is given by the word form. Less
specific is the frequency information of stem forms.
In contrast, the frequency ratio of semantic classes and parts-of-speech are more
{\em general}. According to the experimental results described in Section~\ref{pref-freq-type}
we are using the frequency information of the stem forms of the anaphor-candidate pair as the {\em ratio} value.

\subsubsection{Count}
\label{pref-count}

Other important indicators for the determination of saliency can be obtained from
the context in which the respective anaphor-candidate pair occurs.
{\em The larger the number of previous occurrences of the respective candidate
in the current conversation, the more likely the candidate will be referred to}.

Similar to {\em ratio}, we could define various types of {\em count} features influencing
the degree of generality of this feature. However, in the case of short conversations,
the {\em count} value of specific word forms will be almost the same (i.e., close to zero)
for most of the candidates. A more meaningful distinction can be achieved by relying on the
semantic classification of the respective elements.
Therefore, we define the {\em count} feature as the number of nouns $N$ with
the same semantic code as the candidate. The utilization of different {\em count} types
is evaluated in Section~\ref{pref-count-type}.
\vspace*{-1em}

\subsubsection{Distance}
\label{pref-dist}

Even if the {\em recency} information proves to be insufficient as a single indicator
for saliency, the distance feature plays a crucial role in our selection method.
{\em The smaller the distance between the anaphor and the candidate is, the more likely
is a coreferential relationship}.

As mentioned in Section~\ref{data} recency can be measured on various granularity
(utterance, chunk, candidate, morph) levels. However, measuring the distance
according to utterances or chunks, we might end up with multiple candidates in the same
chunk, thus not being able to make a reliable distinction. On the other hand, the
number of morphemes between the anaphor and the candidate depends on the length of 
the utterances, which might vary depending on the domain. Therefore, we define
the {\em dist} feature as the number of nouns occurring between the anaphor
and the candidate. The utilization of different {\em dist} types
is evaluated in Section~\ref{pref-dist-type}.
\vspace*{-1em}

\subsubsection{Saliency measure}
\label{pref-pref}

Given the saliency indicators described in the previous sections, we will prefer
those anaphor-candidate pairs with the features:
\vspace*{-0.1em}

\begin{center}
\parbox{4.5cm}{
 \begin{enumerate}
  \item large {\em ratio} value
  \item large {\em count} value
  \item small {\em dist} value \\[-1.2em]
 \end{enumerate}
 }
\end{center}

However, the examination of our corpus reveals that frequently occurring candidates
which are mentioned at the beginning of the conversation (i.e., {\em large} ratio/count + {\em large} dist)
as well as rarely occurring candidates mentioned in the last utterance (i.e., {\em small} ratio/count + {\em small} dist)
are tagged as antecedents, thus leading to a conflict concerning the proposed saliency indicators.

As a solution to this problem, we define the preference value $pref$ by dividing the $ratio$ and the $count$ values
by the $dist$ value, thus enabling an appropriate decision for conflicting candidates.

\begin{eqnarray*}
  pref & = & \frac{ratio + count}{dist}
\end{eqnarray*}

The $pref$ value is calculated for each candidate and the list of
potential candidates is sorted towards the maximization of the preference factor.
The first element is chosen as the antecedent.
The precedence order between candidates with the same {\em pref} value continues to remain so and
thus a final decision is made in the case of a draw.

\section{Evaluation}
\label{eval}

In order to prove the feasibility of our approach we compare the four
preference selection methods listed in Figure~\ref{pic-exp}.

\begin{figure}[htb]
  \vspace*{-1.5\baselineskip}
  \begin{center}
\atari(86,30)
    \vspace*{-0.25\baselineskip}
    \caption{Comparison of four methods}\label{pic-exp}
  \vspace*{-1.5\baselineskip}
  \end{center}
\end{figure}

(1) The {\small\em MRC} method selects the most recent candidate as the antecedent
of an anaphoric expression and shows the baseline performance of the task.
The necessity of the filter and preference selection components
is shown by comparing (2) the decision tree filter scheme {\small\em DT}\, (i.e., select the first
element of the filtered candidate list) and (3) preference scheme {\small\em PREF}\, (i.e., sort
the full candidate list and select the first element) against (4) our combined method 
{\small\em DT+PREF}\, (i.e., sort the filtered candidate list and select the first element).

Five-way cross-validation experiments are conducted for the resolution of 1745 pronominal
input samples. The performance of the system is calculated by the percentage (\%) of correctly
resolved antecedents.

As described on Section~\ref{data} the antecedents tagged in our corpus consist
of either single nouns or compound nouns, i.e. a sequence of single nouns.
However, the system selects only a single noun from the list of antecedent candidates. 
In the case that the tagged antecedent is a compound noun, the correctness of the coreference
analysis is judged according to whether the selected candidate is contained in the set
of tagged single nouns or not. For example, in the case of a coreferential reference towards
the compound noun 「田中弘子」 [Hiroko Tanaka] given in Table~\ref{tab-tag}, the selection of both,
「田中[6]」and「弘子[7]」, would be judged as correct.

\subsection{Resolution method}
\label{size}

We use varied numbers of dialogues (50-400) for the training of the decision tree and
the extraction of frequency information from the corpus. {\em Open tests} are conducted on
100 non-training dialogues whereas {\em closed tests} use the training data itself for evaluation.
The results of the four different preference selection methods are shown
in Figure~\ref{pic-size}.

\begin{figure}[htb]
  \begin{center}
    \vspace*{-1\baselineskip}

\begin{minipage}{.5\linewidth}
        \begin{center}
                 \atari(64,44)
        \end{center}
\end{minipage}
\begin{minipage}{.4\linewidth}
        \begin{center}
                 \atari(64,44)
        \end{center}
\end{minipage}

    \vspace*{-0.25\baselineskip}
   \caption{Training size}\label{pic-size}
  \vspace*{-1.5\baselineskip}
  \end{center}
\end{figure}

The baseline {\small\em MRC}\, succeeds in resolving only 57.1\%. The best performance of 
{\small\em PREF}\, applied to the open data is 74\%. 
{\small\em DT}\, performs only slightly better resulting in an accuracy of 74.5\%.

A side-effect of {\small\em DT}\, is the filtering out of correct
antecedents. In our experiments, 7.9\% of correct antecedents are classified
as non-coreferential.
Therefore, we might conclude that the decision tree is not much help for the
identification of the most salient antecedent candidate.
However, due to noise reduction\footnote{
Looking at the results of the closed test we can see a performance drop from 90.1\% to
85.7\% for {\small\em PREF}, whereas the filter mechanism used in {\small\em DT}\, 
still improves. This is due to the fact that {\small\em PREF}\, assigns a preference value
to all possible candidates in the discourse history, regardless of whether a coreferential relationship exists
or not. In contrast, the decision tree filter reduces the number of possible candidates by 22\%.
Moreover, the number of trivial selection cases (only one candidate) increases from 0.9\% (all candidates) to
4.2\% (filter; closed test: 21\%). On average, two candidates are skipped in the history
to select the correct antecedent.},
the utilization of the decision tree filter in combination with the statistical
preference selection proves to be effective. The combination of both methods achieves a success rate
of 76.6\%. Therefore, {\small\em DT+PREF}\, gains a relative improvement of 2.6\% (closed: 10.1\%)
above {\small\em PREF}\, and 2.1\% (closed: 2.3\%) above {\small\em DT}.

\subsection{Data balance}

However, the continuous improvement of the decision tree filter according to the training size
as well as the large number of antecedents filtered out in the open test ({\em OUT}) 
implies a lack of training data for the identification of potential candidates.
Moreover, an examination of the training corpus reveals an imbalance of coreferential
anaphor-antecedent pairs towards unrelated pairs.

In order to overcome these problems, we artificially increase the amount
of training data by repeating tagged anaphor-antecedent pairs $N*db$ times, where $db$ is the
average of the number of negative examples divided by the number of positive examples for each
tagged anaphor.

\begin{figure}[ht]
  \begin{center}
    \vspace*{-0.5\baselineskip}
    \hspace*{-1cm}
\atari(75,54)
    \caption{Data balance}\label{pic-balance}
  \vspace*{-1\baselineskip}
  \end{center}
\end{figure}

Figure~\ref{pic-balance} shows the influence of increasing the data balance factor $N$ versus
the system performance. 

Retraining the same coreferential relationship more frequently has only a small impact on
the performance of the {\small\em DT}\, method, but it significantly decreases the number of antecedents
filtered out from 10.5\% to 7.6\%, resulting in an improvement of the preference selection method
when used in combination with the decision tree filter. The best performance
of the {\small\em DT+PREF}\, method is achieved for the balance factor of $N$=10,
whereas a larger $N$ does not improve the system performance further.

\vspace*{-5mm}

\section{Feature dependency}
\label{feature}

In order to be able to improve the performance described above, it is necessary to go into
the particulars about the importance of the features utilized for the identification of coreferential
relationships as well as the assignment of preference values.

For the task of coreferential analysis, we can categorize the utilized feature sets into the
following groups:

\renewcommand{\arraystretch}{}
\begin{center}
\begin{tabular}{lll}
  {\em :sem}   & $\rightarrow$ & semantic codes assigned to content words \\
  {\em :word}  & $\rightarrow$ & identification of word strings\\
  {\em :pos}   & $\rightarrow$ & identification of part-of-speech\\
  {\em :conj}  & $\rightarrow$ & conjugation attributes of the respective words\\
  {\em :hdist} & $\rightarrow$ & position in discourse history (absolute distance) \\
  {\em :dist}  & $\rightarrow$ & relative distance between anaphor and candidate \\
  {\em :count} & $\rightarrow$ & number of occurrences in previous discourse \\
\end{tabular}
\end{center}
Moreover, we distinguish four groups according to which element in the discourse
({\em :ana}, {\em :cand}, {\em :agree}, {\em :sen}) the features are applied to 
and four additional groups specifying the locations of these elements
({\em :backward}, {\em :latest}, {\em :next}, {\em :forward}).

\begin{center}
\begin{tabular}{lll}
  {\em :ana}   & $\rightarrow$ & feature set applied to anaphor \\
  {\em :cand}  & $\rightarrow$ & feature set applied to candidate \\
  {\em :agree} & $\rightarrow$ & attribute agreement between anaphor/candidate \\
  {\em :sen}   & $\rightarrow$ & feature set applied to other elements in utterance \\[0.5em]
  {\em :backwards} & $\rightarrow$ & elements previous to coreferential expression \\
  {\em :latest}    & $\rightarrow$ & element directly preceding coreferential expression \\
  {\em :next}      & $\rightarrow$ & element directly succeeding coreferential expression \\
  {\em :forward}   & $\rightarrow$ & element succeeding coreferential expression
\end{tabular}
\end{center}
\renewcommand{\arraystretch}{}

The assignment of saliency values is based on information about how often a given anaphor candidate pair
is tagged coreferential vs. non-referential in the training corpus ({\em :ratio}) as well as 
the absolute position and relative distance ({\em :pdist}). Moreover, we take into account how often
the respective elements have already been mentioned in the discourse ({\em :pcount}).

\subsection{Investigation of coreferential analysis by feature omission}
\label{feature-coref}

The importance of a feature is verified by retraining the decision tree filter without 
using the respective feature information and comparing its performance with the {\em :all} method
using the complete set of features.
The results for open and closed tests are listed in Figure~\ref{pic-feature-coref}.
\begin{figure}[ht]
  \begin{center}
    \vspace*{-0.5\baselineskip}
\atari(75,54)
    \vspace*{-0.25\baselineskip}
    \caption{Feature dependency of coreferential analysis}\label{pic-feature-coref}
  \vspace*{-0.75\baselineskip}
  \end{center}
\end{figure}

The most important feature is the {\em :sem} feature, whose omission leads to a performance drop of 3.1\% in the
open test. The descriptive power of this feature is indicated by the increase in accuracy for the closed test when
{\em :sem} is omitted, i.e., the usage of semantic information leads to a generalization that is helpful
for the identification of unseen data, but has its drawbacks in the classification of the training samples.
The omission of {\em :conj} leads only to a small drop in the open test performance, but
the closed test result is 1\% below the {\em :all} method.
The features {\em :dist} and {\em :count} may not seem to be useful for the task of coreferential analysis
and might lead one to consider eliminating them from the set of training features.
However, this information proves to be very important for the preference selection task as described
in more detail in Section~\ref{feature-pref}.

Concerning the elements to which the features are applied, the biggest performance drop of 3.1\% is seen
when any information about the anaphor ({\em :ana}) itself is omitted. In the open test, only 71.6\% of
the antecedents can be resolved correctly, whereas the closed test performance goes down 1.6\% to 89.8\%. 

\subsection{Feature distribution inside the decision tree}
\label{feature-dt}

Additional information about the importance of each feature can be retrieved by
looking at how often the feature is used ({\em quantity}) and where its usage is located
in the decision tree ({\em quality}).

\subsubsection{Quantity}

Figure~\ref{pic-feature-quantity} describes the number of tree nodes using a specific feature
within decision trees trained with a varied number of dialogues.
\begin{figure}[ht]
  \begin{center}
  \vspace*{-1\baselineskip}
\atari(75,54)
    \caption{Feature frequency}\label{pic-feature-quantity}
  \vspace*{-1\baselineskip}
  \end{center}
\end{figure}

Again, the semantic feature {\em :sem} is the most important one, as it is the feature most frequently
used within a decision tree. Less frequent are the features {\em :word} and {\em :hdist},
followed by {\em :pos}, {\em :dist}, and {\em :count}.
Moreover, not only is information about the coreferential elements themselves important
for the identification of coreferential relationships, but what also prove to be quite helpful are
the features of elements directly succeeding the coreferential expressions ({\em :next}).

Comparing the decision trees for different data balance factors, we find that an artificial increase
of the training data leads to a quantitative increase of the {\em :sem} feature usage up to
a specific level, with stagnation for a data balance factor of $N>$10.

Additionally, the increasing number of nodes addressing features of elements directly preceding ({\em :latest})
and succeeding ({\em :forward}) the coreferential elements indicates a focus shift towards
local constraints around these constituents.
Furthermore, the quantitative importance of features with continuous values ({\em :hdist},{\em :dist},{\em :count})
decreases, because no new information is provided for the learning procedure.

\subsubsection{Quality}

For an investigation of the feature locations described below, we extracted the positions
of the respective nodes within a decision tree. The distribution of the most important quantitative features
{\em :sem}, {\em :hdist}, and {\em :word}
is listed in Figure~\ref{pic-feature-location}.

\begin{figure}[t]       
  \begin{center}
    \hspace*{-1cm}

\begin{minipage}{.43\linewidth}
        \begin{center}
                 \atari(66,45)
        \end{center}
\end{minipage}
\begin{minipage}{.5\linewidth}
        \begin{center}
                 \atari(66,45)
        \end{center}
\end{minipage}
\atari(66,45)

    \caption{Feature location within a decision tree}\label{pic-feature-location}
  \vspace*{-0.5\baselineskip}
  \end{center}
\end{figure}

The most important qualitative feature {\em :hdist} is located at the top-level of the
decision tree. Interestingly, this node determines  whether the antecedent candidate
is the first or second candidate in the discourse history, thus distinguishing long range
references from more recent ones. In contrast, the relative distance feature {\em :dist}
is mainly used on lower levels (level 11 to 35) to identify the correct antecedent
within a short range (1 to 3 utterances apart).

Concerning the qualitative importance, {\em :hdist} is followed by {\em :sem}
and {\em :word} on levels 2 and 4. 
However, the most frequent feature {\em :sem} (levels 2-44) is utilized on almost all levels to
determine coreferential relationships. In contrast, the features {\em :hdist} (level 4-15),
{\em :pos} (levels 5-20), and {\em :count} (levels 5-25) are mainly applied on medium levels,
whereas the feature {\em :word} is also used for the purpose of disambiguation on lower levels.
Moreover, the decision tree levels 1-3 are identical for the training size 250-400, thus
representing general attributes to be checked for the identification
of the correct antecedent in this task. 

The investigation on the feature importance in this section suggests that it is not sufficient
to simply provide information tagged in our training corpus for the decision tree learning,
but a selection of appropriate learning features beforehand can positively affect the coverage
of the decision tree performance.

\subsection{Investigation of preference selection by feature type}
\label{feature-pref-type}

The preference selection scheme described in Section~\ref{pref} uses statistical information about
the frequency ratio of anaphor-can\-didate pairs as well as distance and count features extracted
from the discourse in order to select the most salient antecedent candidate.
However, for all of these features, we can extract different types as described in Section~\ref{data}.
In this section we will investigate the dependency of the system performance to different types
of saliency features. 
For the evaluation of our system described in Section~\ref{eval}, we used the types
resulting in the best performance. Those are marked with ``{\large ${}^{*}$}'' in the descriptions below.

\subsubsection{Frequency ratio}
\label{pref-freq-type}

The types of the frequency feature are defined using the stem form ({\em r}),
part-of-speech ({\em p}), and semantic class ({\em s}) of the anaphor and the candidate alike.
Each type is specified using a two-character abbreviation, whereby the first character specifies the
type of the anaphor and the second character that of the candidate. Thus, the type {\em :rr}
uses the stem form frequency of the anaphor-candidate pair, whereas, for example, the type
{\em :rs} describes the frequency information of the anaphor stem form in combination
with the semantic classification of the candidate. 
The results for different frequency types are listed in Table~\ref{tab-freq-type}.

\setlength{\tabcolsep}{3pt}
\begin{table}[htb]
 \vspace*{-0.75\baselineskip}
 \begin{center}
    \caption{Types of frequency feature}\label{tab-freq-type} 
   \begin{small}
    \begin{tabular}{l|cc@{\hspace{1cm}}l|cc}
        \multicolumn{1}{c|}{type} & open & closed & \multicolumn{1}{c|}{type} & open & closed \\
        \hline
        :rr{\large ${}^{*}$} & 76.6 & 94.2
                          & :sp & 74.6 & 90.1 \\
        :ss & 75.4 & 90.3 & :rs & 75.1 & 91.5 \\
        :pp & 74.6 & 90.1 & :ps & 75.2 & 90.6 \\
        :rp & 74.6 & 90.2 &     &      &      \\
        \hline
    \end{tabular}
    \end{small}
    \vspace*{-1\baselineskip}
  \end{center}
\end{table}

The best performance for the open test was achieved using the {\em :rr} type.
The worst performance can be seen when using the part-of-speech frequency information of the
candidate. 
Despite the drop in performance for types other than the stem form, all frequency types
achieved a positive improvement to the coreference filter performance
(cf. Figure~\ref{pic-size}; DT open: 74.5\%, DT closed: 91.4\%)
when applied to the open test data. Concerning the closed test, however, we see an improvement
of more than 1\% only for the type {\em :rr}.

\subsubsection{Count}
\label{pref-count-type}

The types of the count feature consist of the stem form ({\em :stem}),
part-of-speech ({\em :pos}), and semantic code ({\em :sem}) of the candidate.
The results for different count types are listed in Table~\ref{tab-count-type}.

\setlength{\tabcolsep}{3pt}
\begin{table}[hbt]
 \vspace*{-1.25\baselineskip}
 \begin{center}
    \caption{Types of count feature}\label{tab-count-type} 
    \begin{tabular}{l|cc}
        \multicolumn{1}{c|}{type} & open & closed \\ 
        \hline
        :stem                 & 76.2 & 94.1 \\
        :sem{\large ${}^{*}$} & 76.6 & 94.2 \\
        :pos                  & 75.4 & 92.4 \\
        \hline
    \end{tabular}
    \vspace*{-1.5\baselineskip}
  \end{center}
\end{table}

As mentioned in Section~\ref{pref-count}, the count feature depends on the length of the dialogue.
The repetition of specific stem forms of candidates does not occur very frequently in our task.
Usually, the count factor is between 0 and 2. In our experiments, the utilization of the {\em :stem} type
results in the performance of 76.2\% (closed: 94.1\%).
The best results of 76.6\% can be achieved by counting semantic classes, because they occur more frequently
in the discourse, thus resulting in a wider variation of the count feature value.

\subsubsection{Distance}
\label{pref-dist-type}

The types of the distance feature consist of utterance ({\em :uttr}), chunk ({\em :chunk}),
candidate ({\em :cand}), and morpheme ({\em :morph}).
The results for different distance types are listed in Table~\ref{tab-dist-type}.

\begin{table}[hbt]
 \vspace*{-1\baselineskip}
 \begin{center}
    \caption{Types of distance feature}\label{tab-dist-type} 
    \begin{tabular}{l|cc}
        \multicolumn{1}{c|}{type} & open & closed \\
        \hline
        :uttr & 72.3 & 90.6 \\
        :chunk  & 73.0 & 91.4 \\
        :cand{\large ${}^{*}$} & 76.6 & 94.2 \\
        :morph & 66.0 & 88.1 \\
        \hline
    \end{tabular}
    \vspace*{-1\baselineskip}
  \end{center}
\end{table}

The results show that the distance type {\em :cand} achieves the best performance followed by
{\em :chunk} and {\em :uttr}. The reason for the decline in performance when using chunk and
utterance measures is their coarseness, i.e., the same distance factor might be assigned to multiple candidates.
In this case, the distance factor makes no contribution to the distinction of saliency of the
respective candidates and the more frequent one is preferred, even if it precedes the other candidates.

In contrast, too much stress is put on the distance factor, when using the number of morphemes as
the distance measure, which is much larger than the number of candidates in between. Using the
{\em :morph} type results in the worst performance of 66.0\% (closed: 88.1\%).

\subsection{Investigation of preference selection by feature omission}
\label{feature-pref}

In our approach, the distance and counting information do not play a crucial role
for the identification of potential candidates during decision tree filtering, but 
they are very important for the determination of the preference value for each antecedent candidate.

The preference selection method is based on the dependency between the distance ({\em :pdist}) and
counting ({\em :pcount}) values of the given anaphor-candidate pair in the context of the respective
discourse. Additionally, we use statistical knowledge about the frequency of coreferential
and non-referential pairs extracted from the training corpus ({\em :ratio}).
The relative importance of each factor is shown in Table~\ref{tab-feature}.

\setlength{\tabcolsep}{2pt}
\begin{table}[hbt]
  \begin{center}
    \vspace*{-0.5\baselineskip}
    \caption{Feature dependency of preference selection}\label{tab-feature} 
   \begin{scriptsize}
    \begin{tabular}{|c|ccc|cc|cc|}
        \multicolumn{1}{c}{} & \multicolumn{3}{c}{feature} & \multicolumn{2}{c}{open test} & \multicolumn{2}{c}{closed test}
        \cr
        \multicolumn{1}{c|}{}
          & :pdist
          & :ratio
          & :pcount
          & \hspace*{0.25em} {\footnotesize PREF\ } \hspace*{0.25em} 
          & \hspace*{0.05em} {\footnotesize DT+PREF} \hspace*{0.05em} 
          & \hspace*{0.25em} {\footnotesize PREF} \hspace*{0.25em} 
          & \hspace*{0.05em} {\footnotesize DT+PREF} \hspace*{0.05em}
        \cr \hline
        only-freq  & -- & +  & -- & 64.7 & 64.5 & 92.3 & 92.3 \cr
        only-count & -- & -- & +  & 33.7 & 38.1 & 33.7 & 44.2 \cr
        no-count   & +  & +  & -- & 74.0 & 75.9 & 86.3 & 94.2 \cr
        no-freq    & +  & -- & +  & 58.6 & 62.6 & 58.6 & 72.0 \cr
        no-dist    & -- & +  & +  & 41.4 & 41.9 & 59.9 & 63.0 \cr
        \hline
        {\em all} & + & + & + & 74.0 & 76.6 & 84.5 & 94.2 \cr
        \hline
    \end{tabular}
    \end{scriptsize}
    \vspace*{-1\baselineskip}
  \end{center}
\end{table}

We compare the performance of the experiments described in Section~\ref{eval}
({\em all}) to those methods that rely only on frequency ({\em only-freq}) or counting ({\em only-count})
information\footnote{Utilizing only distance information would not change the saliency ranking given by the primary order of candidates, thus achieving the same performance as the {\small\em DT} method.}
as well as to preference schemes that do not use counting ({\em no-count}),
frequency ({\em no-freq}), or distance ({\em no-dist}) information.

Most of the listed methods perform significantly worse than the {\em all} approach taking into account
all three saliency indicators.
Relying only on frequency information achieves an accuracy of around 64.5\% (closed: 92.3\%) 
regardless of whether the filter mechanism is used ({\small\em DT+PREF}\,) or not ({\small\em PREF}\,). Using only
counting information decreases the performance of the {\small\em PREF}\, method to 33.7\% for open as well as
closed test data.

Thus, frequency information on coreferential relationships does appear to be more relevant
for the identification of potential candidates over the counting of features within the current discourse. 
Even the combination of both features ({\em no-dist} method) does not have a positive impact on
the preference selection.

The reason for this is that the larger the {\em count} factor is, the later the respective candidate occurs
in the discourse history, thus emphasizing the selection of more recent candidates. Moreover, the {\em count} factor
is not much help for the resolution of anaphora occurring early in the discourse due to the missed mentioning
of previous similar expressions.
In contrast, due to the large amount of long range references towards dialogue-initial candidates contained
in our tagged corpus, the frequency information tends to prefer candidates occurring earlier in the discourse.

In order to find a metrics that takes into account recent candidates rarely occurring coreferentially as well
as dialogue-initial elements occurring frequently coreferentially, we divide the saliency value based on
the frequency and count information by the distance factor.

The application of the new metrics to the count information ({\em no-freq}) improves the performance
of the {\em only-count} scheme by 25\%, but the metrics is still insufficient as a saliency measure. However, 
this does not mean that the {\em count} information is completely unimportant for determining
saliency. Even if the combination of the frequency and distance information ({\em no-count}) performs
as well as the {\em all} method for the preference selection applied to all candidates ({\small\em PREF}\,),
the additional {\em count} information gains an improvement when applied to potential candidates only
({\small\em DT+PREF}\,).

Therefore, the effectiveness of our approach is not only based on the usage of single
antecedent indicators extracted from the corpus, but also on the combination of these
features for the selection of the most preferable candidate in the context of the given
discourse.

\section{Related research}
\label{relr}

Due to the characteristics of the underlying data\footnote{An approach to ellipsis resolution using the same data set is described in \cite{Yamamoto98}. However, their task is limited to the resolution of omitted subjects in utterances, i.e., extra-sentential references, and does not deal with coreferential relationships.}
used in our experiments, a comparison involving absolute numbers to previous
approaches gives us little practical use.
However, the difficulty of our task lies in frequent long range references.
Resolving pronouns in our data corpus to the most recent candidate achieves
a success rate of only 57.1\% (cf.~Section~\ref{eval}).

Whereas {\em knowledge-based} systems like \cite{Carbonell88} and \cite{Rich88} combining
multiple resolution strategies are expensive in the cost of human effort during development
and have a limited ability to scale to new domains, more recent {\em knowledge-poor} approaches
like \cite{Kennedy96} and \cite{Mitkov98} address the problem without sophisticated
linguistic knowledge.
Similar to them, we do not use any sentence parsing or structural analysis, but simply rely
on morphosyntactic and semantic word information.

Moreover, clues regarding the {\em grammatical} function of expressions are used in 
the centering theory of \cite{Grosz95} and more specifically for Japanese in
\cite{Walker94}. In contrast to them, the grammatical role markers of candidates
seem to be less important for our task, because the respective questions appear
mainly on lower levels of the decision tree.

Rule-based {\em empirical approaches} to determine the most salient referent are utilized
in \cite{Nakaiwa96} and \cite{Murata97}. These kinds of manually defined scoring heuristics,
however, involve a considerable amount of human intervention which is avoided
in machine learning approaches.

As briefly noted in Section~\ref{intro}, the learning approaches described
in \cite{Conolly94} and \cite{Aone95} differ from our work
according to the usage of the decision tree in the resolution task.
In \cite{Conolly94}, a decision tree is trained on a small number of features
concerning anaphor type, grammatical function, recency, morphosyntactic agreement
and subsuming concepts. Given two anaphor-candidate pairs, the system judges which is
``better''. However, due to the lack of a strong assumption on ``transitivity'',
this sorting algorithm may be unable to find the ``best'' solution.

Based on discourse markers extracted from lexical, syntactic, and semantic processing,
the approach of \cite{Aone95} uses unary and binary attributes (lexical, syntactic,
semantic, position, matching category, topic) during decision tree training. The confidence
values returned from the pruned decision tree are utilized as a saliency measure
for each anaphor-candidate pair in order to select a single antecedent.
However, we use dependency factors for preference selection which cannot be learned
automatically because of the independent learning of specific features during decision tree
training. Therefore, {\em our decision tree is not applied directly to the task of preference
selection, but only used as a filter to reduce the number of potential candidates for
preference selection}.

In addition to salience preference, a statistically modeled {\em lexical preference} is
exploited in \cite{Dagan95} by comparing the conditional probabilities of co-occurrence
patterns given the occurrence of candidates.
Experiments, however, are carried out on computer manual texts with mainly intra-sentential
references. This kind of data is also characterized by the avoidance of disambiguities
and only short discourse units, which prohibits almost any long-range references.
In contrast to this research, our results show that the distance factor in addition to
corpus-based frequency information is quite relevant for the selection of the most
salient candidate in our task.

\section{Conclusion}
\label{conc}

In this paper we proposed a corpus-based anaphora resolution method combining an automatic
learning algorithm for coreferential relationships with statistical preference selection.
We proved the applicability of our approach to pronoun resolution, achieving a resolution
accuracy of 76.6\% for Japanese pronouns.

In order to overcome the limitation of sparse data for the task of coreferential analysis,
we artificially increased the number of training samples by repeating positive examples.
As a result, the focus of the feature selection shifted towards constraints concerning non-anaphoric
elements directly preceding/succeeding the coreferential expressions, suggesting that a preprocess
for the selection of appropriate features for the decision tree learning is necessary to obtain
the optimal performance for coreferential analysis.

The separation of the coreferential analysis and preference selection proves quite effective.
The decision tree filter eliminates non-coreferential candidates from the discourse history.
This noise reduction increases the effectiveness of the preference selection method, which
is not only based on the usage of single antecedent indicators extracted from the corpus,
but also on the combination of statistical and discourse features for the selection of the most
preferable candidate in the given context.

Our system does not use any kind of parser to identify compound nouns, but we are only
relying on the morphological analysis of our input sentence. Therefore, the system selects
only a single noun as the antecedent. One future extension of our system will be the incorporation
of a parser to obtain additional information about the sentence structure which will enable the handling
of compound nouns in our framework.

This approach is also not limited to the resolution of pronominal anaphora. We are going to apply
this approach to other types of anaphoric reference, like ellipsis and nominal resolution, too. 
Preliminary experiments on ellipsis resolution have also shown promising results.

However, the filter mechanism proposed in this paper is automatically trained from the training corpus
using the same attributes for all anaphoric expressions. But, comparing attribute sets
used for the resolution of pronominal anaphora to those of other anaphoric expressions, like,
for example, ellipsis resolution \cite{Murata97}, we can find a discrepancy which indicates
a dependency of the attribute importance on the resolution task.
In addition, not only the attribute type, but also the scope of specific attribute values can
influence the performance of the system.
For example, \cite{Kameyama97} introduces a {\em locality assumption} which restricts the reference scope
according to the anaphor type.
Therefore, task-oriented extensions of our general approach might be necessary to improve the resolution
performance for specific applications.

Moreover, investigations into the feasibility of our approach for languages other than Japanese,
e.g., the English MUC corpus, will enable us to compare this approach more precisely 
towards related research.

Additionally,  we plan to incorporate this approach into multi-lingual machine translation,
which will enable us to handle a variety of referential relations in order to improve
the translation quality.

\section*{Acknowledgment}

The authors would like to thank Kadogawa-Shoten for providing us with Ruigo-Shin-Jiten.

\clearpage

\bibliographystyle{nlpbbl}
\bibliography{JNLPcoref}

\begin{biography}

\biotitle{Biography}

\bioauthor{Michael PAUL}
{He received the B.E. and the M.E. degrees in computer science from the 
 University of Saarland, in Saarbr\"ucken/Germany, in 1992 and 1994 
respectively.
 He is a research associate of the ATR Spoken Language Translation Research Laboratories.
 His research interests include natural language processing, 
 machine translation and context processing. He is a member of the ACL 
 and the EAMT.
}



\bioauthor{Eiichiro SUMITA}
{He received M.E in computer science from University of
 Electro-Communications in 1982 and PhD in engineering from Kyoto University
 in 1999 respectively. He is a senior researcher of ATR Spoken Language
 Translation Research Laboratories. His research interests include natural
 language processing, machine translation, information retrieval, and
 parallel processing. He is a member of the IEICE and the IPSJ, and the ACL.
}


\bioreceived{Received}
\bioaccepted{Accepted}

\end{biography}

\end{document}

