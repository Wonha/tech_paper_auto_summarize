    \documentclass[japanese]{jnlp_1.3a}

\usepackage[dvips]{graphicx}
\usepackage{url}
\usepackage{amsmath}
\usepackage{udline}
\setulsep{0pt}
    \usepackage{dcolumn}
\def\maru#1{}

\Volume{14}
\Number{3}
\Month{Apr.}
\Year{2007}
\received{2006}{4}{20}
\revised{2006}{8}{7}
\rerevised{2006}{10}{28}
\accepted{2006}{11}{22}

\setcounter{page}{165}
\setcounter{secnumdepth}{3}

\jtitle{理由に着目した感情表現の構成要素分析}
\jauthor{中山 記男\affiref{sougou}\affiref{national} \and 神門 典子\affiref{national}\affiref{sougou}}
\jabstract{
映画や書籍などの作品検索への応用を目的として,作品レビューテキスト中の感情表現の構成要素を分析した. まずWeb上の作品レビュー82件1,528文中の653組の主観表現を人手で分析し,「態度」「主体」「対象」「理由」という4つの構成要素と,その各々の下位要素を定義した.653組の主観表現中,「態度」が感情を表している感情は257組あった.次に感情表現の各構成要素の内容や働きを分析し,構成要素間の結びつきや,「主体」や「対象」が省略されるパタン,省略されない特殊なパタンなどを明らかにした.「理由」は,感情が生起した根拠や理由を述べている部分をさし,257組中66件(25.7{\kern0pt}%)に出現した.本稿では,利用者の作品選択に,より具体的な情報を提供しうる可能性がある要素として「理由」に着目し,さらに分析を進めることとした.次に,異なる文書タイプにおける「理由」の出現のしかたや分析の手がかりを調べるために,Web上の作品レビュー,オンラインショッピング内のブックレビュー,新聞記事からサンプルを選び,「理由」に着目した追加分析を行った.最後に,作品検索システム利用者の鑑賞作品選択における「理由」のあるレビューの重要性を確認するため,3つの映画それぞれ10件ずつ計30件の作品レビューを用いて,8名ずつ2グループの計16名に対し被験者実験とフォーカスグループインタビューを行った.この結果,作品レビューの読み手が第三者の書いたテキストを参考にする際,「理由」の有無がその内容を理解し,鑑賞する作品を選択するのに参考になるかどうかの判断に大きく寄与していることがわかった.作品検索では,「理由」の有無や内容をレビューの重要性の順位付け等に応用することなどが考えられる.
}
\jkeywords{感情表現,理由分析,情報抽出,情報検索,テキスト分類,主観情報}

\etitle{Reason-focused Analysis of Emotion Expression-related Components}
\eauthor{Norio Nakayama\affiref{sougou}\affiref{national} \and Noriko Kando\affiref{national}\affiref{sougou}} 
\eabstract{
As a preparatory study for using emotion in advanced retrieval systems of 
books and movies, we analyzed expressions of emotion that appeared in film 
and book reviews, and defined the related components. First, through the 
manual analysis of 653 subjective expressions in 82 book reviews selected 
randomly from the web, we defined four major components related to 
expressions of emotion and their subcomponents: ``Attitude'', ``Subject'', 
``Object'', and ``Reason''; we analyzed the existence of each component and 
their combinations in the reviews. Second, we found that different writers 
stated different ``Reasons'' for the same ``Emotion'' about the same 
``Object''; that tendency was confirmed in our additional analysis on 
emotion expressions, in which we focused on ``Reasons'' using different 
corpora consisting of film and book reviews and newspaper articles. Finally, 
we conducted experiments using 16 human subjects in two sessions of focused 
group interviews. The results showed that ``Reasons'' associated with 
emotion expressions appearing in film reviews are important and useful for 
users to select films relevant for them.
}
\ekeywords{Emotion Expression, Sentiment Reason, Information Extraction, Information Retrieval, Text Classification, Subjective Statements}

\headauthor{中山,神門}
\headtitle{理由に着目した感情表現の構成要素分析}

\affilabel{sougou}{総合研究大学院大学情報学専攻}{Department of Informatics The Graduate University for Advanced Studies}
\affilabel{national}{国立情報学研究所}{National Institute of Informatics}


\begin{document}
\maketitle




\section{はじめに}

インターネット上での商取引やブログの増加により,特定の商品や出来事についての感情や評価,意見などの個人の主観を表明したテキストが増加している.この主観の対象が特定の商品に対するものである時は,商品へのフィードバックとして企業に注目される.主観が特定のニュースや施策に対するものであれば,国民の反応を知る手がかりとして利用する用途なども考えられる.国内外で多数の主観に注目した会議が開催されていることからも,関心の高さをうかがい知ることができる(EAAT 2004, Shanahan et al. 2005,言語処理学会2005,言語処理学会2006,AAAI2006,EACL2006,ACL2006). 

本研究では,このようなテキストに現れた個人の主観の表明の中でも,特に,「うれしい」「かなしい」などの個人の感情を表す感情表現に着目し,その特性を理解するためのモデルを提案し,書籍や映画などの作品検索に応用するための方策を考察する.なお感情とは,ある対象に対する主体の気分や心の動きであり,感情表現とは,感情とその主体,対象などの構成要素をまとめて呼ぶ呼称である.態度とはテキストの中で感情や評価,意見など主観を表明した部分である.感情表現には感情表現事典(中村1993)に収録されているような感情という態度を表明している部分だけではなく,それを表明した主体や向けられた対象,その理由や根拠が関連する構成要素が存在する.我々は書評や映画評などの作品レビューが利用者にとって鑑賞する作品の選択に参考となるかどうかを判断するためには,感情表現の中の態度だけでなく他の構成要素も抽出する必要性があると考える.これは,作品レビューには参考になるもの,ならないものがあり,それを判断する手がかりとして構成要素が利用されていると仮定したことによる.さらに構成要素の中でも態度を表明した理由や根拠がその判断に大きく影響していると考えた.

そこでまず感情表現抽出の準備段階として,感情表現の構成要素をあきらかにするため,Web上の作品レビューを用いて分析を行い,感情表現のモデルを定義し,構成要素の特徴をあきらかにした.次に感情表現の理由や根拠の重要性や働きを調べるため,追加分析と被験者実験を行い,作品検索に感情表現を用いるとき,検索結果が利用者にとって参考となる情報となるためには理由という構成要素が重要な働きをしていることを示した.

\subsection{作品レビューにおける主観的な情報}

本研究で扱うレビューとは,ある対象について評論したテキストのことである.レビューには,下記のような多様なドメインが考えられる.

・作品:映画評,ブックレビューやCD,楽曲,演劇などの作品に関するレビュー

・製品:携帯電話や車などの製品についてのレビュー

・サービス:レストランや飛行機,ホテルなどのサービスに関してのレビュー

・組織:会社や団体など,組織についてのレビュー

これらドメインによって,レビュー中に表明された主観的な情報の用途,関連する構成要素と各要素の重要性,働き,評価の観点などが異なる.製品においては,使い勝手や好みなどの主観的な情報も重要であるが,その仕様や機能,価格など製品に関する事実がより重要な観点となる.同様にサービスではその特徴や利点が,組織では活動の内容などが重要な観点となる.これら製品やサービス,組織は利用するためのものであるため,それぞれが持つ機能や特徴,性質など主に具体的な事実や数値とそれが好意的なのか否定的なのかという評価がレビューとして重要視される.しかし映画や書籍のような作品は個人が味わうためのものであり,価格やあらすじ,登場人物などの事実以上に,それを利用者が読んだり鑑賞したりしてどう感じるかといった,利用者の抱く感情が重要である.

\subsection{作品検索の問題点}

現在の作品を対象とした検索では,作品のタイトルや登場人物,ジャンルなどを手がかりにして,利用者が自分の希望する作品を検索している.しかし利用者の要求には,「今日は泣ける本が読みたい」「派手な映画を見て元気を出したい」「背筋も凍るような恐怖のホラー映画が見たい」など,それらを見聞きした結果どのような感情を感じるかといったものもある.実際Web上の質問サービスである「教えてgoo\footnote{
	教えてgoo,http://oshiete.goo.ne.jp/}」や「Yahoo!知恵袋\footnote{
	Yahoo! 知恵袋,http://chiebukuro.yahoo.co.jp/}」などの質問回答サービスには,「切なくなる本を教えてほしい」「怖い映画を教えてください」などの質問が存在する.感情表現を手がかりとして作品を検索できれば,これら要求を満たすことができる.

我々は,単に作品へ向けられた感情表現中の感情という態度を表明した語句のみから作品を探すのではなく,感情の主体,対象,理由などの感情表現の他の構成要素も利用することが重要と考える.さらに構成要素の中でも理由,根拠,原因が明記された感情表現が特に利用者にとって参考となり得る重要な情報であると考えた.理由,根拠,原因の記述された感情表現を検索に利用することで,同じ「幸せな気分になれる本」を探したときでも,「笑える内容だったから幸せだった」のか「ハッピーエンドで終わったから幸せだった」のかなどを区別することができる.また,我々は,趣味嗜好が強く反映される作品レビューのようなテキストではそれを読んだ利用者がテキストに記述された内容を理解し,鑑賞する作品を選択するときに参考にすることが可能であることが重要であると考えた.具体的には,感情表現を用いた作品検索において,「理由」が記述されたものに重み付けをし,さらに結果をその作品レビューが含む理由と共に表示することなどが考えられる. 

そこで,本研究では作品レビューのテキストを対象とし,そこに出現する感情表現を分析した.なかでも感情表現の理由や根拠に注目して研究を行った.

\subsection{本論文の構成}

本論文の構成は次のとおりである.2節では,関連研究を概観し,本研究の位置づけを明確にする.3節では,書籍と映画に関するレビューを人手分析し,感情表現の構成要素を定義した.4節では,3節で定義した構成要素の特徴と働きについて考察をした.5節では,構成要素の中から理由に着目し,その重要性を分析,検討した.6節では5節での検討内容を被験者実験によって実証し,7節ではその結果を考察した.8節は本論文の結論である.

\section{関連研究}

テキストから,「意見」「評価」「感情」など主観表現の態度を抽出,または分類する研究では,処理対象とする要素が様々である.論文によって要素の名称が異なるが,今節では説明のため代表的な要素の表記を統一する.対象へ向けられた意見,評価,感情などは「態度」とする.同様にして態度が向けられた対象は「対象」,主観を表明した主体は「主体」,「態度」における肯定的,否定的,中立などの属性を「極性」とする.

\subsection{主観の極性を判定する研究}

テキスト中に表明された主観を扱う研究の中のひとつの大きなグループは,極性を判定する研究である.Turney(Turney 2002)はWeb上の映画のレビューテキストから態度を抽出し,その極性を検索エンジンを用いて得られた「excellent」または「poor」との共起しやすさから判定し,さらに抽出された態度の極性の集合から,テキストの極性を判定している.極性を正しく判定するため,態度を単語ではなく前後の文脈を追加した句として抽出している.舘野(舘野2002)は,企業のサポートセンターによせられた「お客様の声」に含まれる態度に着目し,事前に行った構文構造解析から木構造を用いて,否定的な極性を含む文を抽出している.那須川ら(那須川,金山2004)は,デジタルカメラまたは映画について書き込まれているWeb上の掲示板から,極性が既に判定されている既知の態度をもとに,新たな態度を極性付きで抽出している.極性の判定には,極性が定義されている態度と新たな態度の間に,極性を反転させる「しかし」などの表現が出現するかどうかを利用している.立石ら(立石ら2004)は評判情報検索に,態度,対象,極性を組み合わせた辞書を用いている.たとえばコンピュータの分野において「小さい」が1,000表現中8回出現し,そのうち7回が肯定的なら,コンピュータが対象のとき,「小さい」という態度は肯定的な極性であると判定される.Kim and Hovy(Kim and Hovy 2004)は,態度,主体,対象,極性に着目し,抽出を行っている.主体ごとに極性の方向(肯定的か,否定的か,中立か)とその強さを計算することで,主体がどのような極性をもっているか判定している.

\subsection{主観の極性以外の側面も扱う研究}

小林ら(小林ら2005)は態度,対象,対象の属性を抽出している.対象の属性とは,対象の要素のことである.例えば「携帯電話」における対象の属性は「液晶画面」であり,大きな枠組みである対象と,下位要素である対象の属性を区別している. 

 Wiebeら(Wiebe et al. 2005)は,手作業でコーパスにタグ付けを行うことにより,態度の構成要素を定義している.主観的な発言や明示的である率直な態度に関しては,態度,主体,対象,極性に加え,主観の強さ,表現の強さ,主観に実態があるかどうか(仮定の話か,実際の話か),の各要素を定義している.表現による主観的な態度については,態度,主体,主観の強さ,極性を定義している.発話や記述の事実に関しては,発話や記述の部分とその主体,対象を定義している. 

 Liuら(Liu et al. 2003)は,常識知から出来事と感情の組み合わせを学習することで,文を感情カテゴリに分類している.例えば「自動車事故で恐怖を感じた」という事例から「自動車事故」は「恐怖」という組み合わせを辞書に登録することにより,「自動車事故にあった」という文を「恐怖」という感情カテゴリに分類している.田中ら(田中ら2004)はテキストの情緒を推定するため,日本語語彙体系をもとに作成した結合価パターンを用いている.この研究で情緒属性と呼ばれている要素には,「前提条件」「情緒主(主体)」「情緒対象(対象)」「原因」「情緒名」がある.「太郎がコンサートのチケットを入手した」という文から情緒主「太郎」は原因「獲得」から「獲得による喜び」という情緒を導き出している. 


大塚(大塚2004)は,道路計画に対する住民への自由記述アンケートテキストが要求か否かを判定している.要求の要素として要求動機,要求内容,要求意図が定義されている.要求意図が明示的に表明されていなくても,要求動機が出現することでそこに暗黙的な要求意図が存在すると示されている.例えば「歩道がせまい」という事実を要求動機ととらえることで,「歩道を広くして欲しい」という暗黙的な要求意図を導き出している.ただしテキストが要求か否かの判定に関しては,明示的要求のみを対象としている. 


これまでの研究において,我々が「理由」と呼ぶ理由,原因,根拠,動機などは,態度を導きだしたり,テキストや文,句などを分類するための手がかりとして扱われてきた.本研究では利用者にとって作品レビューが鑑賞する作品を選択するのに参考になるかどうかを判断する手がかりとして感情表現の「理由」に着目し,その特性や働きの分析を行った.

\section{感情表現のモデル作成}

作品レビュー中の感情表現のモデルを検討するためにグラウンデッドセオリーの「絶えざる比較法」(Glaser 
and Strauss 1967)の枠組に従った.データ収集,分析,分析を通じて見えてきた問題に沿ったデータ収集のサイクルを理論的飽和まで繰り返すことにより,結論を得る(Keith 2005).分析対象にはWeb上の書籍および映画の作品レビューを取り上げた.

\subsection{コーパス}

コーパスは,杉田と江口(杉田ら2001)が2000年10月に収集した作品レビューに関するコミュニティに属する332のWebサイト上のページである.本稿では,その中から6サイトを無作為に選択し,さらに各サイトから作品レビューが記述されたファイルを1つずつ選択した「作品レビューサブコーパス(以下:BSC)」を作成し,実験に用いた.BSCは書籍64件と映画18件の計82件の作品レビューから構成されており,計1,528文である.表1にBSCの詳細を示す.選択した6サイトおよびその書き手は重複しない.書き手とは,そのサイトで作品レビューを書いた人物である.書籍に関する作品レビューは1冊の書籍,映画に関する作品レビューは1作の映画について書かれたもので,はじめと終わりが読み手に明確なひとまとまりのテキストを1件の作品レビューとした.

\begin{table}[b]
\begin{center}
\input{10t1.txt}
\end{center}
\end{table}

\subsection{タグ付け}

中山ら(中山ら2005)は,BSCに対して,感情だけではなく,感情,意見,評価などの何らかの主観を表明している箇所に着目して,人手でタグ付けを行い,主観の表明に関わる一連の構成要素を含む主観表現のモデルを提案した.BSCには,主観表現は653組あり,そのうち,「\textgt{態度}」が感情と分類された感情表現は257組であった.本稿では,この感情表現に着目して,さらに分析をすすめた.

\begin{figure}[b]
\centerline{\includegraphics{14-3ia10f1.eps}}
\caption{タグ付け作業の手順}
\end{figure}

モデル構築のためのタグ付けの作業は次のとおりである.作業者は著者である.まず,感情が表明されている箇所に注目し,その「\textgt{態度}」を1つタグ付けした後に,続けてその態度に関連する他の構成要素を検討した.タグ付けは作品レビュー単位で行い,1文の中に複数の態度がある場合や,関連要素が作品レビュー内で文を超えて存在するものも検討の対象とした.同一箇所に複数のタグを重複して付与することもできる.作業者である著者の主観により新たな構成要素が発見されるごとに,その要素をモデルに加え,既に分析済みのコーパスも新たなモデルでタグ付けをやり直した.この繰り返しによりタグ付けする構成要素を決定しながらBSC中の全てのテキストにタグ付けを行った.図1にタグ付け作業の手順を示す.その結果,構成要素として「\textgt{態度}」「\textgt{主体}」「\textgt{対象}」「\textgt{理由}」という4つとそれぞれの下位要素を得た.表2に感情表現の構成要素と下位要素の一覧を示す.タグ付けに際しては開始時と終了時の判断の差が発生しないよう,終了後にもう1度見直しを行った.見直しでは,各要素を「\textgt{\ul{主体}}は\textgt{\ul{対象}}を\textgt{\ul{理由}}によって\textgt{\ul{態度}}と感じた」と同等の文に言い換え,矛盾がないか確認した.

\begin{table}[t]
\input{10t2.txt}
\end{table}

以下,構成要素について説明する.「\textgt{態度}」は主観が表明されていると判断した部分にタグ付けした.「\textgt{態度}」であるかどうかの判断には,\maru{1}事実報告ではないこと 
\maru{2}思ったこと,感じたことであること 
などを基準とした.表明のタイプについて,態度が明示的に表明されている場合は「\textgt{態度記述}」,暗黙的に表明されている場合は「\textgt{ゼロ態度}」とした.下位要素について,「\textgt{態度}」がタグ付けされた中でも感情だと判断された場合は「\textgt{感情}」,感情かどうかの判断が難しい場合は「\textgt{保留}」とした.これ以降に説明する「\textgt{態度}」以外の上位要素は,タグ付けされている「\textgt{態度}」に関連するもののみを検討している.「\textgt{態度}」には「\textgt{態度なし}」という表明のタイプが存在するが,今回は「\textgt{態度なし}」に該当するものはタグ付けを行わず分析対象としなかった. 


「\textgt{主体}」は「\textgt{態度}」を表明した人またはモノ,「\textgt{対象}」は「\textgt{態度}」が向けられた人またはモノである.両者の表明のタイプに関して,テキスト中に要素が明示的に記述されている場合はそれぞれ「\textgt{主体記述}」「\textgt{対象記述}」,明示的に要素が記述されていないが省略されていると判断できる場合はそれぞれ「\textgt{ゼロ主体}」「\textgt{ゼロ対象}」とした.テキスト中に記述されてなく,かつ該当するものが無い場合は「\textgt{不明}」とした.それぞれの下位要素として,タグ付けされたものの属性である書き手,登場人物,ストーリーなどがある. 

\begin{table}[t]
\input{10t3.txt}
\end{table}

「\textgt{理由}」は「\textgt{態度}」を表明した原因,理由,根拠となる部分である.「\textgt{理由}」の表明のタイプに関しては「\textgt{記述されていない}」もの,語尾が(から|ため|ので|よって)または(すれば|してくれば|なら)という表現であるかこの表現に言い換えることのできる「\textgt{明確に理由が記述されている}」もの,語尾を(を理由として|を原因として)と大きく言い換えることで「\textgt{理由}」であると判断できる「\textgt{暗黙的に理由が記述されている}」ものに分類した.「\textgt{理由}」の下位要素は「主体の主観的な判断によるもの」を「主観理由」,「事実によるもの」を「客観理由」とした.「\textgt{理由}」に関しては,5節で詳しく考察をした.

タグ付けされた作品レビューの例を表3に,そこから抽出した要素の一覧を表4に示す.同一箇所に複数のタグが付与される例として,表3の作品レビューAでは,ある態度について同じ部分が「\textgt{対象}」「\textgt{理由}」になっている例(o4,r4)を,作品レビューBではある態度に関しては「\textgt{理由}」とタグ付けされている部分に別の態度についての構成要素がある例(r8,a9など)を含んでいる.

\begin{table}[t]
\input{10t4.txt}
\end{table}

\section{タグ付けの結果}

4.1節では感情表現に関するタグ付けの結果を報告し,4.2節では,「\textgt{理由}」の特性や下位要素を検証する分析をした. 
4.3節では複数分析者間の一致度調査の結果を報告する. 
4.4節では感情表現の「\textgt{理由}」の働きについて考察を行った.これら分析により,感情表現の構成要素の特徴やパタンをあきらかにし,また「\textgt{理由}」の重要性を検討した.

\subsection{感情表現の分析}

感情表現257組についての分析を行い,感情表現の構成要素の特徴や出現パタンを調べた.タグ付けされた上位要素の組み合わせ件数の一覧を表5に示す.

\begin{table}[b]
\begin{center}
\input{10t5.txt}
\end{center}
\end{table}

\subsubsection{構成要素}

表5に示すように,もっとも多くタグ付けされた上位要素の組み合わせは,

\textgt{[態度あり]+[主体あり]+[対象あり]+[理由不明]}

であり,257組の中で170組あった.「\textgt{主体}」は全ての257件あり,うち記述あり22件,ゼロ主体148件であった.「\textgt{対象}」は236件で,うち記述あり181件,ゼロ対象55件だった.「\textgt{理由}」は66件で,うち明確54件,暗黙12件だった.「\textgt{対象}」は作品レビューの対象書籍または映画であるものが50件で最も多く,他で多かったものとしては,事実が40件,登場人物が38件,場面が28件だった.

\subsubsection{構成要素の特徴と考察}

「\textgt{対象}」236件のうち,45件は記述が省略されている「ゼロ対象」であった.また,「書き手」以外の「\textgt{主体}」がタグ付けされた例は全部で23件あった.主観表現全体で24件あったうち23件が「\textgt{感情}」で見られ,書き手以外の「\textgt{主体}」がタグ付けされている部分では,ほぼ感情が表明されている.感情表現の処理において,構成要素のうち「\textgt{主体}」が省略されているかどうかは重要な要素であると考える.書き手自身が「\textgt{主体}」であるとき「\textgt{主体}」の表記が省略されることの多い日本語において,あえて「\textgt{主体}」が記述されるときは,主体が自らの意見について自信がない,もしくは自らの意見が特殊であると自認している特別な場合だった.

また,下位要素が「書き手」以外の「\textgt{主体}」がタグ付けされた例は全部で24件あった.うち22件は登場人物の感情を記述したものであり,あらすじの説明中に出現したものが多かった.作品レビューの書き手が表明した「\textgt{感情}」と,作品内で登場人物が表明した「\textgt{感情}」では意味が異なるため,区別しなくてはならない.これはレビューの書き手による「\textgt{感情}」は主観的な情報であるのに対し,登場人物の「\textgt{感情}」は作品の一部で客観的な情報だからである.同様にして,作品に向けられた「\textgt{感情}」と作品以外に向けられた「\textgt{感情}」も区別する必要がある.そのため作品レビューを処理する際には,タイトルと記事,作品レビューの地の部分とあらすじの記述箇所の切り分けなど,テキスト全体の構造に着目した処理も必要である.

また,中山ら(2005)で行った,感情,意見,評価などの多様な主観の表明に関わる表現の分析と比較しても,とくに,感情表現のみの顕著な傾向は見られなかった.

\subsection{「理由」について}

\subsubsection{明確な「理由」と暗黙的な「理由」}

感情表現の「\textgt{理由}」は66件あった.「\textgt{理由}」は表明のタイプとして明確なものと暗黙的なものに分類できた.それぞれの理由の表明のタイプは,タグが付与された部分の語尾が以下に示す3つのパタンのいずれかであるか,言い換えることができるものを指す.

(から|ため|ので|よって)+「\textgt{態度}」 明確\maru{1}

(すれば|してくれば|なら)+「\textgt{態度}」 明確\maru{2}

(を理由として|を原因として)+「\textgt{態度}」 暗黙的

明確な理由\maru{1}に当てはまるものは47件あった.明確な理由\maru{2}にあてはまるものは4件あった.暗黙的な理由は15件あり,うち6件は「\textgt{対象}」と「\textgt{理由}」のタグ付与が重なっていた.

暗黙的な理由をさらに分析すると下記のカテゴリを見出すことができた.

\noindent
I. 作品レビューのドメインに依存した対象(場面・ストーリー・登場人物)の説明が理由になっているもののうち,直接的な因果関係が成立しないため,「〜ので」「〜から」などに言い換えられないもの

Ia. 個々の場面(6件)

Ib. 全体のストーリー(2件)

Ic. 登場人物(3件)

Id. 視線移動(1件)

\noindent
II. 分析者の体験・世界知識が理由となっているもののうち,「〜ので」「〜から」など直接的な因果関係に言い換えられないもの(3件)

このうち,Iのような具体的な場面,ストーリー等に関連する理由は,「〜ので」「〜から」などには言い換えられないが,「〜を理由として」「〜を原因として」には言い換えることができ,また,後述する複数分析者間の一致度調査でも一致して判定された割合が高かった.それに対し,IIは,個々の判定者の個人的な体験や世界知識に関連するものであり,他者との共有や理解は難しい場合もある.

\subsubsection{主観的な「理由」と事実による「理由」}

「\textgt{理由}」には,下位要素として主観的なものと事実による客観的なものがあった.木戸(木戸,佐久間1989)によれば,文の機能として「根拠:判断のよりどころとなった事実を提示する機能」,「理由:判断のよりどころとなった意見を提示する機能」と定義されている.本研究が「\textgt{理由}」の下位要素として定義した「主観理由」は後者の「判断のよりどころとなった意見を提示する機能」に,「客観理由」は前者の「判断のよりどころとなった事実を述べる機能」にあたる.

「\textgt{理由}」の66件中,「主観理由」は33件,「客観理由」は30件,どちらとも判断つかず保留したものが3件あった.「主観理由」には書き手の考えが,「客観理由」には書籍の内容や書き手の状況が多かった.表6に件数の一覧を,表7に下位要素の例を示す.

\begin{table}[b]
\begin{center}
\input{10t6.txt}
\end{center}
\end{table}

\begin{table}[t]
\begin{center}
\input{10t7.txt}
\end{center}
\end{table}

\subsection{分析者間でのタグ付与一致度}

タグの信頼性を調査するため,3名の分析者によるタグ付けの一致度調査を行った.分析者1は20代女性,分析者2は20代男性,分析者3はBSC全体のタグ付けを行った著者である.いずれの分析者も大学院生である.分析対象としてBSC82件全体(1,528文)の中から10件(150文)を無作為に選んだ.タグ付けは3.2節に示した構成要素に関して行った.感情表現のタグ付与のみの一致度調査では,比較する件数が少なかったため,感情のみではなく,なんらかの主観を表明した箇所に関して,関連する構成要素をタグ付与し,その一致度を調査した.4.3.2節では,分析者間でタグ付与に揺れが現れることが予想される「\textgt{態度}」と「\textgt{理由}」に関して考察する.

\subsubsection{分析者間での態度タグ付与}

表8に各分析者が「\textgt{態度}」タグを付与した数と,その中でも3名の分析者が同じ部分に「\textgt{態度}」タグを付与した数を示す.分析者1,分析者2,分析者3は,それぞれ70件,144件,98件の「\textgt{態度}」をタグ付与し,そのうち3名が同じ部分にタグ付与したものは52件あった. 

\begin{table}[t]
\begin{center}
\input{10t8.txt}
    \vspace{\baselineskip}
\input{10t9.txt}
\end{center}
\end{table}
分析者間でのタグ付与の一致度は,2つの調査で評価した.表9に各分析者間の「\textgt{態度}」
タグ付与一致率\footnote{
	一致率は,比較する分析者同士での同じ部分にタグを付与した割合の平均で求めた.}と
コーエンの$\kappa$係数(Cohen 1960)によりもとめた各分析者間での「\textgt{態度}」タグ付与の一致度\footnote{
	タグ付与される可能性のある部分281箇所を仮定し,共通でタグ付与しなかった部分も一致として計算した.281箇所は,1行あたりに含まれる態度の数の平均値を求め,それを全行数にかけあわせて計算した.}を示す.分析者1と3の態度タグ付与には高い一致が見られたが,分析者2は,分析者1および3との一致は中程度であった.

\subsubsection{分析者間での理由タグ付与一致度}

前節にて2名以上が同じ部分に「\textgt{態度}」が付与された87件それぞれの「\textgt{理由}」について,その一致度を調査した.「\textgt{理由}」のタグ付け部分が同じかどうかは著者が判定した.複数者が「\textgt{理由なし}」とした場合も一致していると数えた.87件中で「\textgt{態度}」を付与した2名以上が同じ部分に「\textgt{理由}」を付与したものでは86.2{\kern0pt}%の75件あった.中でも2名両者または3名全員が同じ部分に「\textgt{理由}」を付与したものは66.7{\kern0pt}%の58件あった.まったく一致しないものは13.8{\kern0pt}%の12件あった.


\begin{table}[t]
\begin{center}
\input{10t10.txt}
\end{center}
\end{table}

分析者間での一致度を表10に示す.被験者間での「\textgt{理由}」付与について,7割以上の一致が確認された.
感情表現のタグ付けは個々の判定者の主観による揺れを含むやや難しいタスクであるが,共通して認識される要素も少なくなく,本稿のタグ付けも一定の範囲で信頼性が確保されたと考える.

\subsection{「理由」についての考察}

態度のように主観的な情報では「いいですね」と思った「\textgt{理由}」は様々である.我々は,この理由こそが利用者の知りたい情報となるのではないかと考える.これは同一態度であってもその理由が異なれば,まったく違った情報になる場合があるからである.例えばストーリー重視で作品を探す利用者にとって,最も有用な情報となるのは,ストーリーの良さを理由にした「この作品はいいですね」などの肯定的表現である.このことから態度がどんな理由で表明されたかが,作品レビューの感情表現をもとに作品を検索する際には重要となり,かつ利用者がそのレビューを信頼し,参考にできるかどうかに関連すると仮説を立てた.そこで我々は,「\textgt{理由}」の働きを確認するため,次節以降で追加分析を行った.

\section{「理由」の特性の分析}

5.1節では,レビューに対して評価がつけられているAmazon.co.jpにて,参考になるレビューと参考にならないレビューとで「\textgt{理由}」の出現に違いがあるのかを示した. 
5.2節では「\textgt{理由}」が同じ作品に同じ感情を感じた場合であっても様々であることを調べ,「\textgt{理由}」を提示することの必要性を示した.5.3節では異なるジャンルである新聞記事にて,「\textgt{理由}」がどのように存在するのかを調査し,そこから「\textgt{理由}」の重要性を分析した.これらの分析により,作品検索において利用される作品レビューが参考になるかどうかという面での「\textgt{理由}」の重要性を示した.

\subsection{「理由」は利用者に求められているか}

Amazon.co.jp\footnote{
	Amazon.co.jp,http://www.amazon.co.jp/}のカスタマーレビューには,そのレビューが参考になったかどうかの投票システムがある.これを用いて,感情表現の「\textgt{理由}」が明記してあるレビューとそうでないレビューでは,参考になる度合いが違うかどうかを調査した. 
2004年書籍ベストセラー上位5冊の書籍について,書かれてから1ヶ月以上経過しているレビューから,それぞれ最も参考になっているレビュー10件,最も参考になっていないレビュー10件,計100件のレビューに感情表現のタグ付けを行い,その内容を比べた.タグ付けは著者が行った.それぞれのレビューに含まれた感情表現のうち,何件が「\textgt{理由}」を持つかを調べた結果を表11に示す. 


参考になったレビューには,参考にならないものに比べ,「\textgt{態度}」と「\textgt{理由}」ともに多く含まれていた.表11で示したように参考になったレビューに「\textgt{理由}」が含まれている割合は36.9{\kern0pt}%である.参考にならないレビューで「\textgt{理由}」が含まれる割合は 
24.5{\kern0pt}%であり,両者にはカイ二乗検定により優位水準5{\kern0pt}%で優位差があった.このことからも,参考になったレビューには理由が含まれる割合が高いといえる.参考になったレビューには,好意的なもの,批判的なものも含め,なぜそう思ったのか,つまり「\textgt{理由}」をわかりやすく書いてあるものが多かった.また「\textgt{理由}」以外にも簡単なあらすじや,どんな人に最適かが書かれているものが多かった.それに対して参考にならないレビューでは,単に「面白い」や「つまらない」と書いてあるものや,参考になったものに比べ理由が不明確であるもの,さらに書籍とは関係のない内容などが多く,利用者にとって有効な情報になっていないと考える.

\begin{table}[b]
\begin{center}
\input{10t11.txt}
\end{center}
\end{table}

この結果から,参考になるレビューには参考にならないレビューよりも「\textgt{理由}」が記述されている「\textgt{態度}」が多く,参考になるものとならないものを区別するひとつの手がかりになっているのではないかと考えた.

\subsection{「理由」には多様性があるか}

同じ「\textgt{対象}」,同じ「\textgt{態度}」で「\textgt{主体}」が異なる場合において,「\textgt{理由}」に多様性があるかを確認するため,新たなテキストを用いて詳しく分析した.これはある作品に対して同様に「面白かった」という結論を示している作品レビューにも参考にできるものとできないものがあるのは,「\textgt{理由}」の多様性に一因があるのではないかと考えたためである.

\begin{table}[b]
\input{10t12.txt}
\end{table}

BSCではこの条件にあてはまる例が確認できなかったため,gooブログ検索\footnote{
	gooブログ検索,http://blog.goo.ne.jp/}を用いて検索した,
映画「交渉人真下正義」および「チャーリーとチョコレート工場」に関する作品レビューを新たなコーパスとして利用した.同じ「\textgt{対象}」に同じ「\textgt{態度}」を表すテキストとして,「楽しめた」という感情表現の記述があり,なおかつ「\textgt{理由}」の記述があるもので,「交渉人真下正義」および「チャーリーとチョコレート工場」に関する作品レビューをそれぞれ10件ずつ,計20件選択した.これを映画レビューコーパスと呼ぶ.映画レビューコーパスへの構成要素のタグ付けは著者が行った.映画レビューコーパスのテキスト中の「楽しめた」という感情表現に対応する「\textgt{理由}」を表12に示す.同じ作品を「\textgt{対象}」とし,同じ「\textgt{態度}」を感じたとしても,人によりその「\textgt{理由}」が異っている.これは同じ作品に対し 
「楽しめた」という感情を持ったとしても,人によってそれを感じる部分(場面)が違うためである.

実際に「\textgt{理由}」とされた内容を見ていくと,「交渉人真下正義」ではテンポやノリ,ドキドキ感が挙げられており,この映画の特徴の中でも楽しめた部分を指している.「チャーリーとチョコレート工場」では世界観や役者の演技などが挙げられており,前者の「交渉人真下正義」とは異なる「楽しめた」理由が示されている.このような「\textgt{理由}」の多様性が参考にできるものとそうでない作品レビューを生み,結果として「\textgt{理由}」が利用者にとって必要とされていると考える.

\subsection{異なるジャンルにおける「理由」の特徴}

作品レビューにおける「\textgt{理由}」の特徴を明確にするために,異なる文書ジャンルでの「\textgt{理由}」について比較分析を行った.異なる文書ジャンルとして新聞記事(社説を含む)をとりあげた.この分析において作品レビューはBSCを,新聞記事は,一例としてGoogleニュース日本版\footnote{
	Googleニュース日本版,http://news.google.co.jp/}で上位にランキングされていた30件を利用した.これを新聞記事コーパスと呼ぶ.新聞記事コーパスにおける構成要素のタグ付けは筆者が行った.新聞記事コーパスのテキストには感情表現がほとんど含まれないため,ここではBSCにおける主観表現の「\textgt{理由}」と,新聞記事コーパスにおける主観表現の「\textgt{理由}」を比較分析した.

BSC全体で「\textgt{理由}」が653件中151件,23.1{\kern0pt}%出現していたのに対し,新聞記事コーパスでは73.2{\kern0pt}%で出現していた.両者にはX二乗検定にて優位水準1{\kern0pt}%で有意差が見られ,新聞記事コーパスのほうが「\textgt{理由}」が出現する確率が高い.これは新聞記事では読み手に納得,理解させることや,記事の信頼性が問われるため,「\textgt{理由}」を記述することで読み手に訴えかけているのではないかと考えた.またBSCには客観理由が44.4{\kern0pt}%,主観理由が49.0{\kern0pt}%と同じ程度で出現していたのに対し,新聞記事コーパスでは主観理由が44.3{\kern0pt}%,客観理由が55.6{\kern0pt}%であった.これについては,同じくX二乗検定にて優位水準5{\kern0pt}%までは有意差があると言えず,有意水準10{\kern0pt}%ではじめて有意差があるといえた.ただし,この中で10件ある郵政民営化問題と解散総選挙問題だけを取り上げてみると,客観理由は75.0{\kern0pt}%にもなり,記事の内容によって書き手の主観が「\textgt{理由}」になりやすいニュースと,そうでないニュースがある.BSCにおける「\textgt{理由}」において,客観理由ではストーリーや登場人物の行動などの作品の内容を「\textgt{理由}」として書き手の感情が表明されたケースが多かった.主観理由では,「物語の力強さ」「まるで自分も登場人物の一人になったかのよう」など,書籍に書かれている内容や映画に映っている内容以外に対する書き手の考え,状況,様子を「\textgt{理由}」として,書き手の感情を表明していた.BSCに比べ新聞記事コーパスは小規模であるが,異なるジャンルにおける「\textgt{理由}」の特徴の違いが示唆された.

\section{被験者実験}

\subsection{「理由」の働きと重要性に関する被験者実験}

感情表現の「\textgt{理由}」が作品レビューを読む利用者にとってどのような働きをしているのかを分析するために,被験者実験を行った.被験者は,都内の大学生16名であり,8名ずつの2グループに分けて行った.グループXは文系女子大の学生8名,グループYは理工系の男子学生8名である.被験者に映画に関する作品レビューを読んでもらい,見る映画を選択する上で参考になる部分に下線を付与してもらった.実験手順は次の通りである.

I. 事前アンケート

II. 被験者による下線付与

III. フォーカスグループインタビュー

被験者実験により,5節で示した「\textgt{理由}」の重要性に関して,実際に作品レビューがどのように読まれているかという面から分析する.

\subsubsection{事前アンケート}

被験者のインターネット利用時間などを調べた.主要な回答結果を表13に示す.

\subsubsection{被験者による下線付与}

分析対象は,表14に示す3つの映画の各々について,ブログに書かれた作品レビューを10件ずつ,計30件である.作品レビューはgooブログ検索[11]を用いて各映画のタイトルで検索した結果の上位から,主に映画について書かれているものを10件選択した.30件の作品レビューの書き手は全て異なる.各映画にはM1〜M3,各作品レビューにはR1〜R30というIDを付与した.R1〜R10はM1の,R11〜R20はM2の,R21〜R30はM3の作品レビューである.この実験に用いた映画の作品レビューの集合を実験用レビューと呼ぶ.実験用レビューを分析するためのタグ付けは筆者が行った.実験用レビューの詳細を表15に示す.

16名の被験者に提示した実験用レビューは同じものであるが,提示する順番による結果への影響を考慮し,提示した順は被験者それぞれで異なっている.調査はグループごとに8名ずつまとめて行い,グループXとグループYの調査日時,調査場所は異なっている.各被験者は,3つの映画についての情報を探しており,その情報をもとに映画を見るかどうか決めようとしている状況を想定するよう求めた.

\begin{table}[p]
\begin{center}
\input{10t13.txt}
    \vspace{\baselineskip}
\input{10t14.txt}
    \vspace{\baselineskip}
\input{10t15.txt}
\end{center}
\end{table}

実験には実験用レビューを1件ずつ紙に印刷したものを用いた.被験者には,1件の作品レビューの中で被験者が参考にできると感じた部分があればそこに下線を,さらにその作品レビューの中で特に参考になった部分があれば下線と丸印を付与するよう求めた.被験者が提示された順に作品レビューを読み,下線を付与する.作業時間に制限はないが,作業は1つの作品レビューずつ行い,次の作品レビューに進んだ後で前の作品レビューに戻ることはできない.1映画に関する10件の作品レビューについて作業が終わった時点で,その10件中で最も参考となった作品レビューを選択し,その理由を回答してもらった.この作業を3つの映画に関して行った.被験者間で下線付与した部分が重なり,かつ著者が同じ内容に下線が引かれていると判断した場合,同じ箇所に対する下線とした.以下に実例を示す.

\vspace{0.25\baselineskip}
\mbox{\vtop{\hbox{やっぱり期待外れで}{\hrule height0.25pt }\kern2pt{\hrule height0.25pt}}\vtop{\hbox{ありましたよ。}\kern2.25pt{\hrule height0.25pt}}   …ア}

\vspace{0.25\baselineskip}
\mbox{\vtop{\hbox{とても}{\hrule height0.25pt}\kern4.25pt{\hrule height0.25pt}}\vtop{\hbox{ほのぼのとした良い}{\hrule height0.25pt}\kern2pt{\hrule height0.25pt}\kern2pt{\hrule height0.25pt}}\vtop{\hbox{映画。}\kern2.25pt{\hrule height0.25pt}\kern2pt{\hrule height0.25pt}}    …イ}


\vspace{0.5\baselineskip}
アは2つの下線を,イでは3つの下線を同じ部分に下線付与したものとして扱った.これは,まったく同じ部分に下線が引かれていなくとも,構成要素として関連性がある部分を分離して考え難かったためである.

\begin{table}[t]
\begin{center}
\input{10t16.txt}
\end{center}
\end{table}

\subsection{フォーカスグループインタビュー}

被験者の参考にする部分に対する考えを明らかにするために,作業終了後に休憩を挟んだ後,グループごとの被験者全員に対してフォーカスグループインタビューを行った.フォーカスグループインタビューでは,グループの8名全員にひとつの部屋に入ってもらい,表16に示す主な質問を軸にして,被験者全員に自由に発言を行ってもらった.被験者から新たな議論の種がまかれた場合,その内容について我々が用意した質問と同じように議論してもらった.モデレータは著者がつとめた.会場では机をコの字型に並べ,被験者間の発言の様子がわかるようにした.

\section{被験者実験の結果と考察}

\subsection{事前アンケート}

表13より,グループXとグループYでは趣味としてインターネットを使う時間が大きく違う.また,グループXがネット上の情報を参考として映画を見るのに対し,グループYはあまり下調べもネット上の情報を参考にもしていない.

\subsection{被験者による下線付与}

表17は,実験用レビューと下線が付与された部分の特徴である.実験用レビューへの感情表現のタグ付けは筆者が行った.表17は最も参考となる作品レビューを選んだ理由である.表17,表18中のR10,R17,R23,R28は最も参考になるとされた作品レビューのIDである.参考とする部分の下線の数には,特に参考となった部分の下線の数も含まれる.

\begin{table}[b]
\input{10t17.txt}
\end{table}

\subsection{グループ間での下線付与の差異}

表13に示したように,被験者両グループの性質は異なる.しかし被験者により下線付与された部分の性質に差はなかった.被験者にとって重要な部分は,「\textgt{態度}」や「\textgt{理由}」を含み映画の感想を記述する部分と,「\textgt{態度}」や「\textgt{理由}」を含まず映画の内容や特徴を説明する部分に大別でき,それは被験者の性質に関わらなかった.表17に示すように,実験用レビュー全体で「\textgt{態度}」と「\textgt{理由}」は高い確率で下線付与されており,これも被験者の性質に関わらず参考とすべき情報として認識されていた.また,どちらのグループも最も参考になる作品レビューとして選んだものは作品レビューR10,R17,R23およびR28で共通していた.

\subsection{被験者間での下線付与の差異}

「\textgt{態度}」と「\textgt{理由}」には多く下線付与されており,被験者間でも下線付与の傾向に顕著な差はなかった.「\textgt{態度}」と「\textgt{理由}」以外の下線付与部分では,映画のストーリーの記述が多かった.ただし,一部の被験者は下線を付与しなかった.これは後のフォーカスグループインタビューで,僅かなストーリーの記述でもネタバレとして敬遠するためであることがわかった.

映画M2の作品レビューには「虫が苦手な人は見ないほうがいいかも」との記述があり,複数の被験者から特に参考となる下線がひかれた.またM3の作品レビューでの「アメリ好きにおすすめ」といった記述も同様であった.特に参考となる部分については,キャラクター重視もあれば,ストーリー重視もあり,他にも被験者各々が気になる観点について下線が引かれることがわかった.このことから被験者の趣味趣向や知識が強く反映され,被験者それぞれにとって異なる部分が選ばれるとわかった.

実験用レビューに出現した全ての感情表現の「\textgt{態度}」337箇所中,86.6{\kern0pt}%にあたる292箇所,全ての「\textgt{理由}」146箇所中,79.4{\kern0pt}%にあたる116箇所で下線付与された.「\textgt{態度}」と「\textgt{理由}」どちらも無い部分に付与された下線は全下線338箇所中の32.2{\kern0pt}%にあたる125箇所で,映画を見に来た人の様子による事実報告や映画のストーリー,特徴的な映画の場面を実体験から解説している部分への付与が多かった.特に参考となる下線が付与された部分の特徴を調べると,「\textgt{態度}」と「\textgt{理由}」がそろっている部分に付与されたものが41.4{\kern0pt}%,「\textgt{態度}」だけが37.7{\kern0pt}%だった.「\textgt{態度}」も「\textgt{理由}」も無い部分は20.9{\kern0pt}%だったことから,特に参考になる部分を選択するとき,「\textgt{態度}」と「\textgt{理由}」が出現するかどうかも関連していると考えられた.

\subsection{最も参考となる作品レビュー}

表17に示すように,特に参考となる作品レビューとされたR10,R17,R23,R28の4件はどれも文字数,下線数,特に参考となった下線数ともに多かったが,それ以外の特徴については一定の傾向が見られなかった.これら特に参考となる作品レビューには,選んだ理由を示した表18からもわかるように,被験者個々の趣味や嗜好に合致する点がわかりやすく記述されていたと考えられる.被験者によって最も参考となる作品レビューの内容や観点は異なっていることから,参考となる作品レビューではただ単に「この映画は面白かった」ということだけでなく,映画の中のどのような部分がどのように面白かったのかの記述が必要とされた.これは我々の主張する「\textgt{理由}」の部分に該当している.

\begin{table}[p]
\input{10t18.txt}
\end{table}

\subsection{フォーカスグループインタビュー}

フォーカスグループインタビューでは,表16の質問を中心に議論してもらった.その結果,被験者が参考になる情報を選ぶ基準として主に以下の8点があげられた.

\maru{1} 映画の上映場所,日時,長さがわかる

\maru{2} 出演者がわかる

\maru{3} あらすじ,見所がわかる

\maru{4} 映画がどんな人に向けられているか

\maru{5} 映画をみてどう思ったか

\maru{6} 作品レビューの書き手の人物像がうかがえる部分

\maru{7} 決め付けではなく,筋道立った記述により共感できる/できないの判断ができる部分

\maru{8} 〜のような映画など,比較対象がわかる

これらの点は,このような部分に下線付与しなかった被験者からも支持された.\maru{1}〜\maru{3}については,作品レビューでなくとも得られる事実報告のような内容の記述である.これに対して\maru{4}〜\maru{8}は,作品レビューでなければ得られない,主観的な内容の記述である. 

フォーカスグループインタビューの内容から,我々は以下のような仮説を立てた.友人や知人からのクチコミのほうが,ネットに書き込まれた情報よりも信頼できるという意見が多かったことから,被験者にとって最も参考としやすい情報は「書き手がどんな趣向でどのような性質を持つ人かわかっていること」である.その中でも,なぜ書き手がそういった考えを記述したか理解するため,書き手の考えを理解する手がかりになる具体的な「\textgt{理由}」の記述があるかどうかは,書き手の性質がわからないときに,大きな判断材料となる.ブログなどに書かれた作品レビューに理由のない態度のみが記述されている場合,その作品レビューは理由があるものに比べて参考にし難く,利用者がその内容を参考にできるかどうかを判断することができない.それに対して,明確な比較対象や具体的な説明,さらには筋道立てた内容があれば,利用者はそれに共感できるかどうかを判断することができる.共感できれば作品への興味が増幅され,共感できなければ作品への興味が薄れる.こういった判断ができる作品レビューには理解や信頼が生じ,参考になる情報となる.

\subsection{「理由」の分析のまとめ}

実験と分析の結果から,作品レビューの利用者は「\textgt{理由}」を手がかりとして,書き手の嗜好や性質および作品の性質が強く影響する作品レビューのテキスト内容が自分にとって信頼できるのか,参考にできるのかを判断していると考えられた.「\textgt{態度}」と「\textgt{理由}」の組み合わせは,作品レビューのような作品レビューにおいて,とても重要な要素である.本稿にて行われた実験においては,被験者が作品レビューを見て映画を見るかどうか決める際には,まず「\textgt{理由}」や「\textgt{態度}」,および事実報告の記述の有無によってその作品レビューが参考になるかどうかを,さらに「\textgt{理由}」や「\textgt{態度}」の内容に共感できるかどうかにより,その映画についての最終的な判断が下されていた.

\section{おわりに}

本稿では,まずWeb上の作品レビュー82件に対し構成要素を人手でタグ付けることで,感情表現のモデルについて検討し,「\textgt{態度}」「\textgt{主体}」「\textgt{対象}」「\textgt{理由}」の上位要素からなる4つ組みのモデルを提案した.「\textgt{主体}」「\textgt{対象}」については省略されるものとされないパタンが,「\textgt{理由}」については明確なものと暗黙的なもの,さらに主観的なものと事実にもとづくものに着目した.次に作品レビューにおける「\textgt{理由}」の重要性に注目し,その特徴と働きを調べるため,別のテキストによる追加分析とフォーカスグループインタビューを行った.その結果,書籍や映画などを対象とした書き手の嗜好や状況がその感想に強く反映される作品レビューのようなテキストでは,書き手の態度だけでなく,感情表現の構成要素が重要とわかった.特に理由が嗜好や性質のわからない第三者が書いたテキストを利用者が理解し,鑑賞する作品を選択するのに参考にする上で重要だとわかった.また,あらすじのような情報も作品レビューにおいて鑑賞する作品を選択するのに参考になる情報とわかった. 


今後は,本稿にて定義した各要素の構造を分析し,自動抽出する手法について検討する.またフォーカスグループインタビューにおける質的な分析を行った点について,量的な調査から客観的な分析も必要である.今回はWeb上のテキストを中心に分析を行ったが,テキスト媒体や購読対象の違いによる特性についても分析を行う予定である.今回の調査によってあらすじなどの記述も作品レビューにおいて重要であるとわかったが,その扱いについても今後議論する必要がある.

\acknowledgment

本研究を進めるにあたり,多くの有益なコメントを下さった豊橋技術科学大学情報工学系の関洋平助手,および本論文の査読者の方に深く感謝します.

\section*{参考文献}

\noindent\hangafter=1\hangindent=2zw
AAAI (2006). ``Spring Symposium on Computational Approaches to Analyzing 
Weblogs.'' Stanford, CA, USA.

\noindent\hangafter=1\hangindent=2zw
ACL (2006). ``Sentiment and Subjectivity in Text.'' \textit{Workshop at the Annual Meeting of the Association of Computational Linguistics}, Sydney.

\noindent\hangafter=1\hangindent=2zw
Cohen, J. (1960). ``A Coefficient of Agreement for Nominal Scales.'' 
\textit{Educational and Psychological Measurement}, \textbf{20}, pp.~37--46.

\noindent\hangafter=1\hangindent=2zw
EAAT (2004). ``Proceedings of AAAI Spring Symposium on Exploring Attitude 
and Affect in Text: Theories and Applications.'' Stanford, Technical Report 
SS-04--07.

\noindent\hangafter=1\hangindent=2zw
EACL (2006). ``Proceedings of the Workshop on NEW TEXT-Wikis and blogs and 
other dynamic text sources.'' Trento, Italy.

\noindent\hangafter=1\hangindent=2zw
言語処理学会 (2005). 第11回年次大会発表論文集 テーマセッション 
感情・感性の言語学.

\noindent\hangafter=1\hangindent=2zw
言語処理学会 (2006). 
第12回年次大会ワークショップ「感情・評価・態度と言語」論文集.

\noindent\hangafter=1\hangindent=2zw
Glaser, B. G. and Strauss, A. L. (1967). ``The Discovery of Grounded 
Theory.'' Chicago, Aldine.

\noindent\hangafter=1\hangindent=2zw
Shanahan, J.G., Qu, Y. and Wiebe, J. (2005). ``Computing Attitude and Affect 
in Text: Theory and Applications (The Information Retrieval Series).'' 
Springer.

\noindent\hangafter=1\hangindent=2zw
Wiebe, J., Wilson, T. and Cardie, C. (2005). ``Annotating expressions of 
opinions and emotions in language.'' \textit{Language Resources and Evaluation}, \textbf{39}, (2--3), pp.~165--210.

\noindent\hangafter=1\hangindent=2zw
木戸光子 著, 佐久間まゆみ 編 (1989). ``文章構造と要約文の諸相 第7章 
文の機能による要約文の特徴.'' くろしお出版, pp.~112-125.

\noindent\hangafter=1\hangindent=2zw
Kim, S.M. and Hovy, E. (2004). ``Determing the sentiment of opinions.'' 
\textit{Proceeding of Conference on Computational Lin-guistics (CoNLL-2004)}, pp.~1367--1373.

\noindent\hangafter=1\hangindent=2zw
Keith, F.P. 著, 川合隆男 監訳 (2005). ``社会的調査入門 
量的調査と質的調査の活用.'' 慶應義塾大学出版会.

\noindent\hangafter=1\hangindent=2zw
小林のぞみ, 乾健太郎, 松本裕治, 立石健二, 福島俊一 (2005). 
``意見抽出のための評価表現の収集, 自然言語処理.'' \textbf{12}(2), 
pp.~203--222.

\noindent\hangafter=1\hangindent=2zw
Liu, H., Lieberman and H., Selker, T. (2003). ``A Model of Textual Affect 
Sensing using Real-World Knowledge.'' Proceedings of IUI 2003.

\noindent\hangafter=1\hangindent=2zw
Landis, J.R. and Koch, G.G. (1977). ``The measurement of observer agreement 
for categorical data.'' \textit{Biometrics}, \textbf{33}, pp.~159--174.

\noindent\hangafter=1\hangindent=2zw
森田良行, 松木正恵 (1989). ``日本語表現文型:用例中心・複合辞の意味と用法.'' 
アルク.

\noindent\hangafter=1\hangindent=2zw
中村明 編 (1993). ``感情表現辞典.'' 東京堂出版.

\noindent\hangafter=1\hangindent=2zw
中山記男, 江口浩二, 神門典子 (2005). ``感情表現のモデル.'' 言語処理学会第11回年次大会発表論文, pp.~149--152.

\noindent\hangafter=1\hangindent=2zw
那須川哲哉, 金山博 (2004). ``文脈一貫性を利用した極性付評価表現の語彙獲得.'' 
情報処理学会研究報告 ``自然言語処理 (NL162-16).'' pp.~109--116.

\noindent\hangafter=1\hangindent=2zw
大塚裕子 (2004). 
``自由記述アンケート回答の意図抽出および自動分類に関する研究.'' 
神戸大学大学院自然科学研究科博士論文.

\noindent\hangafter=1\hangindent=2zw
杉田茂樹, 江口浩二 (2001). 
``目録データベースとWebコンテンツの統合的利用方式.'' 
情報処理学会研究報告 情報学基礎 (NL142-), pp.~153--158.

\noindent\hangafter=1\hangindent=2zw
田中努, 徳久雅人, 村上仁一, 池原悟 (2004). 
``結合価パターンへの情緒生起情報の付与.'' 
言語処理学会第10回年次大会発表論文, pp.~345--348.

\noindent\hangafter=1\hangindent=2zw
Turney, P. D. (2002). ``Thumbs up? thumbs down? Semantic Orientation Applied 
to Unsupervised Classification of Reviews.'' \textit{In Proceedings of the 40th Annual Meeting of the Association for Computational Linguistics (ACL-2002)}, pp.~417--424.

\noindent\hangafter=1\hangindent=2zw
立石健二, 石黒義英, 福島俊一 (2004). ``インターネットからの評判情報検索.'' 
人工知能学会誌, \textbf{19}(3), pp.~317--323.

\noindent\hangafter=1\hangindent=2zw
舘野昌一 (2002) ``「お客様の声」に含まれるテキスト感性表現の抽出方法.'' 
情報処理学会研究報告 自然言語処理 (NL153-14), pp.~105--112. 



\begin{biography}
\bioauthor{中山 記男}{
2001年芝浦工業大学工学部工業経営学科卒業.2003年芝浦工業大学大学院工学研究科電気工学専攻修士課程修了.同年,総合研究大学院大学情報学専攻博士課程に入学,現在に至る.自然言語処理の研究に従事.
}

\bioauthor{神門 典子}{
1994年慶應義塾大学文学研究科博士課程修了.博士(図書館・情報学).1995年米国シラキウス大学情報学部客員研究員,1996〜1997年デンマーク王立図書館情報大学客員研究員.1998年学術情報センター助教授.2000年国立情報学研究所助教授,2002年より総合研究大学院大学助教授を併任,2004年より国立情報学研究所教授並びに総合研究大学院大学教授,現在に至る.テキスト構造を用いた検索と情報活用支援,言語横断検索,情報検索システムの評価等の研究に従事.ACM-SIGIR,ASIS{\&}T,言語処理学会,日本図書館情報学会各会員.
}

\end{biography}

\biodate



\end{document}





