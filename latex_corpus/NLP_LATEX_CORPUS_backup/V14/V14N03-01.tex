    \documentclass[japanese]{jnlp_1.3b}
\usepackage{jnlpbbl_1.1}

\Volume{14}
\Number{3}
\Month{Apr.}
\Year{2007}

\received{2006}{4}{20}
\accepted{2006}{7}{9}

\setcounter{page}{3}

\jtitle{話し手は言語で感情・評価・態度を表して目的を達するか?\\
	—日常の音声コミュニケーションから見えてくること—}
\jauthor{定 延 利 之\affiref{Author_1}}
\jabstract{
	「話し手は,迅速で正確な情報伝達や,円滑な人間関係の構築といった目的を果たすために,言語を使って自分の感情・評価・態度を表す」という考えは,言語の研究においてしばしば自明視され,議論の前提とされる.本稿は,話し手の言語行動に関するこの一見常識的な考え(「表す」構図)が,日常の音声コミュニケーションにおける話し手の実態をうまくとらえられない場合があることを示し,それに代わる新しい構図(「する」構図)を提案するものである.

現代日本語の日常会話の音声の記録と,現代日本語の母語話者の内観を用いた観察の結果,「表す」構図が以下3点の問題点をはらむことを明らかにする:(i)目的論的性格を持ち,目的を伴わない発話を収容できない;(ii)外部からの観察に基づいており,当事者(話し手)のきもちに肉薄し得ない;(iii)モノ的な言語観に立ち,言語を行動と見ることができない.

中心的に扱われるのは,あからさまに儀礼的なフィラー,つっかえ方,りきみである.「話し手は自分のきもちに応じて,フィラー・つっかえ方・声質を使い分けている」という「表す」考えが一見正しく思えるが,実はどのような限界を持つのかを,実際のコミュニケーションから具体的に示す.
}
\jkeywords{日常会話,目的論,非流ちょう性,フィラー,つっかえ,声質,りきみ,発話キャラクタ
}
\etitle{Do Speakers Express Their Emotion, Evaluation, 
	or Attitude by Their Speech to Achieve Their Purposes?: \\
	A View from Everyday Oral Communication}
\eauthor{Toshiyuki Sadanobu\affiref{Author_1}} 
\eabstract{
It is often assumed as self-evidential that speakers express their emotion, evaluation, or attitude by their speech to achieve their purposes. In this paper I shall show that this common view, apart from its seeming plausibility, does not always capture successfully the very nature of speaker's behavior in everyday communication, and suggest an alternative view for understanding the correlation between speaker's speech and their emotion, evaluation, and attitudes.

Close observation on everyday Japanese conversation data, especially focused on disfluent phenomena such as fillers, stuttering, and pressed voice (a kind of creaky voice), by using native speaker introspection reveals that the so-called common view has three defects. The idea that the speaker uses various fillers, various ways of stuttering, various voice qualities to express his/her emotion, evaluation, and attitudes cannot explain the detail of these disfluent phenomena because (i) it does not accept unintended speech because of its teleological nature, (ii) it does not really touch speaker's psychology because it is based on outward perspective, and (iii) it regards language as a thing to express something rather than an expressing action itself. 
}
\ekeywords{Everyday conversation, Teleology, Disfluency, Fillers, Stuttering, Voice quality, Creaky voice, Speech characters}
\headauthor{定延}
\headtitle{話し手は言語で感情・評価・態度を表して目的を達するか?}

\affilabel{Author_1}{神戸大学国際文化学部}{
	Dep. of Cross-cultural Studies, Kobe Univ.}

\begin{document}
\maketitle

\vfill
\section{まえがき}

「話し手は,迅速で正確な情報伝達や,円滑な人間関係の構築といった目的を果たすために,言語を使って自分の感情・評価・態度を表す」という考えは,言語の研究においてしばしば自明視され,議論の前提とされる.

たとえば「あのー,あなたは失格,なんです」という発言は,単に聞き手の失格(命題情報)を告げるだけのものではない.「失格は,聞き手にとってよくないことだ」という話し手の評価や,「聞き手にとってよくないことを聞き手に告げるのはイヤだ,ためらわれる」といった話し手の感情・態度をこの発言から読みとることは,多くの場合,難しくない.

また,このような話し手の評価や感情・態度を早い段階(たとえば冒頭部「あのー」の段階)で読みとることによって,聞き手は,その後に続く,つらい知らせを受け入れる(つまり迅速で正確な情報伝達を実現させる)ための心の準備ができる.さらに「話し手が発話をためらっているのは,自分に気を遣ってのことだ」と意識することは,話し手との人間関係にとってもプラスに働くだろう.

これらの観察からすれば,「話し手は,迅速で正確な情報伝達や,円滑な人間関係の構築といった目的を果たすために,言語を使って自分の感情・評価・態度を表す」という考えは,疑問の生じる余地のない,この上なく正しい考えにも見える.だが,本当にそうだろうか?

本稿は,話し手の言語行動に関するこの一見常識的な考え(便宜上「『表す』構図」と呼ぶ)が,日常の音声コミュニケーションにおける話し手の実態をうまくとらえられない場合があることを示し,それに代わる新しい構図(『する』構図)を提案するものである.

データとして用いるのは,現代日本語の日常会話の音声の記録(謝辞欄に記した3つのプロジェクトによるもの)と,現代日本語の母語話者の内観である.コントロールされていない日常会話の記録をデータとしてとりあげるのは伝統的な言語学者や多くの情報処理研究者にはなじみにくいことかもしれないし,内観の利用も情報処理研究者や会話分析者には奇異に映るかもしれないが,最善のデータをめぐる議論はかんたんには決着がつかない.\pagebreak
ここでは,両者をデータとして併用している研究は他にも見られる(たとえばChafe 1992: 234を参照)とだけ述べておく.


\section{「表す」構図の問題点}

もっともらしい印象とは異なり,「表す」構図は,日常の音声コミュニケーションにおける話し手の言語行動の実態に合わないことがある.この節ではかんたんな事例と内観を適宜用いながら,この構図がはらむ3つの問題
点を洗い出し,それに代わる新しい構図として「する」構図を提案したい.


\subsection{目的論とソバ屋の出前持ち}

「表す」構図の第1の問題点は,「表す」構図の目的論的性格である.「迅速で正確な情報伝達や,円滑な人間関係構築その他の目的を果たすために」という部分に見られるとおり,「表す」構図は「話し手の発言は何らかの目的の達成に向けられているはず」という目的論を前提としている.だが,この前提は常に妥当するわけではない.たとえば次のような,ソバ屋の出前持ちの事例を考えてみよう.

これから出前に行くソバ屋の出前持ちが,うず高く積まれたソバざるをかつぐ際に「よっ」と言ったとする.続いて,そのソバざるをかついだまま自転車に乗って進み,よろけてバランスをとる際に「おっ」と言ったとする.さらにバランスをとりきれず,倒れていくソバざるを見ながら「あーっ」と言ったとする.このような出前持ちの「よっ」「おっ」「あーっ」発言は,別段,不自然なものではないだろう.

これらの「よっ」「おっ」「あーっ」発言が,目的を達成するためになされたものとして絶対に説明できないというわけではない.たとえば,出前持ちがソバざるをかつぐ際に「よっ」と言ったのは「自分自身を鼓舞するため」であり,ソバざると自転車のバランスをとる際に「おっ」と言ったのは「きもちを引き締めるため」である.倒れていくソバざるを見ながら「あーっ」と言ったのは「自分のきもちを表現するため」である,という具合である.

だが,そのような説明は,我々の日常感覚とはまったくかけ離れたものと言わざるを得ない.たとえば,ソバざると自転車のバランスをとるだけで無我夢中のはずの出前持ちが,その瞬間にも心内では「きもちの引き締め」といった目的の達成を意図している,という説明がリアルなものとは考えにくい.そもそも,出前という任務に集中しているはずの出前持ちが,ソバざるが倒れていく最後の一瞬まで「自分のきもちの表現」という内職に余念がないとは,おかしな考えではないだろうか.

むしろ,出前持ちの発言「よっ」「おっ」「あーっ」は,目的意識を必ずしも伴わない,行動そのものの一部として考える方が実態に合っているのではないか.出前という一大任務を前にして若干の高揚状態に至った出前持ちは,出前の行動を言語も含めた全チャンネルでおこなった.「よっ」と言うのはソバざるをかつぐ行動の一部であり,「おっ」はバランスをとる行動の一部である.そして「あーっ」は倒れていくソバざるを見守る行動の一部である,と考えることが,上の「内職」問題などを生じさせないためには,必要なのではないか.話し手の言語行動を「表す」構図ではなく「する」構図でとらえるとは,このような考えを指す.

\subsection{外部観察と狩人の知恵}

「表す」構図がはらむ第2の問題点は,「表す」構図が当該のコミュニケーションの内部ではなく,外部からの観察に基づいているという点である.このことを具体的に示すために,クマを追う狩人の事例を取りあげてみよう.

「このクマの足跡は,このクマがいま最高にうまいことを表している」という発言は,地面に残ったクマの足跡と,クマの肉の味の間に結びつきを見いだした狩人の発言としてなら,十分あり得る.だが,それはクマの外部に身を置く観察者の発言でしかない.狩人が語る,クマの足跡と肉の味との結びつきがたとえ正しいとしても,当のクマはそんなことは知らずに生きている可能性が高い.その場合,クマの足跡と肉の味との結びつきは,狩人の知恵ではあるが,クマ社会の日常を生きるクマのきもちを知ろうとする者にとっては真に重要な情報ではない.

「このクマはこの足跡で,自分がいま最高にうまいことを表している」という発言についても,基本的に同じことが言える.先の発言と比べると,この発言は「このクマ」を主語に据えており,クマの意図を強く含意するのでそれだけ不自然だが,「このクマは(自分では気づかないうちに)この足跡で,自分がいま最高にうまいことを表している」のように意図性排除の語句(「自分では気づかないうちに」)を補ってやれば自然さは向上する.だが,そうした語句を補っても,この発言は狩人の発言として自然であるにすぎず,クマのきもちを述べた発言としては成り立たない.

「表す」構図には「話し手は言語で,自分のきもち(感情・評価・態度)を表す」という考えが含まれている.発言の根底に話し手の意図を常に想定する目的論についてはすでに2.1節で問題点を指摘したが,より意図性を含意しない「話し手の言語は,話し手のきもち(感情・評価・態度)を表す」という形に置き換えてもやはり問題は残る.「このクマの足跡は,このクマがいま最高にうまいことを表している」という発言が狩人の知恵でしかないように,「話し手の言語は,話し手のきもちを表す」という考えは,本来,内部から論じなければならないきもちの問題に,外部の視点を持ち込んでいるのではないか(定延 2005b).

話し手の内部に視点を置いて,話し手のきもちを考えようとする時,言語はきもちを本当に「表して」いると言えるのか.むしろ,言語は「相手の目の前でやってみせる」行動ではないだろうか.


\subsection{モノ的な言語観}

「する」構図が,言語を行動とみなしていることは言うまでもない.それに対して「表す」構図は,「話し手の感情・評価・態度を表すために,話し手に使われる」モノ,つまり記号としての言語観を内包している.

{\renewcommand{\baselinestretch}{}\selectfont
記号としての言語観が,これまでに莫大な有益な研究成果を生みだしてきたことは否定し難い.だが,日常の音声コミュニケーションにおける言語の姿が本格的に追求され始めるにつれ,この言語観がさまざまな立場から問題視されていることも事実である.「言語行動を理解する上で言語能力と言語運用の区別はさほど重要ではない」,「言語とは動的なプロセスである」=C「言語を記号としてとらえ,意味と形式の対応を前提とする考えでは,談話における主語や目的語の分析に困難が生じてしまう」(Du Bois2003: 特に51, 80)等々,モノ的な言語観から離れ,行動としての言語観に向かおうとする研究は枚挙にいとまがない.(記号的な言語観が,日常の音声コミュニケーションを離れたところでも根本的な問題をはらんでいることについてはたとえば定延2000を参照されたい.)そもそもコミュニケーションの中で「表される」モノであるはずのきもちが,実は「表情を帯びた身振り」(菅原 2002)それじたいであるとしたら,我々がモノ的な言語観にこの上さらにとどまらなければならない根拠はどこにあるだろうか.

記号的な言語観を疑問視するという点に関しては,本稿が提案する「する」構図は,いま挙げたフィルモア,チェイフ,デュ・ボワらの考え,あるいは多くの機能言語学者の考えと変わるものではないし,少なくとも現状において大きな意味を持つのは,これらの考えとの違いよりも共通性の方だと思われるが,違いがまったくないわけではない.

「ハサミの機能は?」という質問には,「紙や布を切ること」などと,たやすく答えることができる.だが,「秋の日」や「山肌」「14才」の機能は答えにくい,あるいは答えられない.機能という概念は,いつでも無条件に設定できるわけではなく,基本的には,人間が何らかの目的を果たすために用いる道具にしか設定できない.「花びらの機能」が「ハサミの機能」よりも難しく,しかし少し考えれば「虫を惹きつけること」「おしべやめしべを守ること」などと答えられるのは,「植物は子孫繁栄という目的を持つ」「植物はこの目的を果たすために,自らの身体の一部である花びらを道具として用いる」ということが,事実ではないが,1つの見立てとして成り立つからだろう.
}

このような意味で,「言語の機能」というきわめてありふれたフレーズは,言語を目的論でとらえようとするものである.(定延 2005a)Dそして,話し手自身の目ではなく,外部観察者の目からすれば,話し手の発話には,たいてい何らかのそれらしい目的を想定してしまえる以上,このような目的論の妥当性は,外部観察からすれば揺るぎないものに見える.本稿の「する」構図は,「言語の機能」概念の導入には慎重である.
つまり,目的論の導入に対する慎重な姿勢,内部からの観察にこだわろうとする姿勢を鮮明にする点は,「する」構図の特徴と言ってよいだろう.


\section{あからさまに儀礼的なフィラー}

以上で洗い出した「表す」構図の問題点を,日常的な音声コミュニケーションに見られる現象の観察を通して,具体的に示してみよう.最初に取りあげるのは,筆者が「あからさまに儀礼的なフィラー」と呼ぶ一群のフィラーである(定延 2002).

「あ,すいません,このあたりに交番ありませんか?」とxに道を聞かれて,Yが「さー」と言ったとする.この直後に続くYの発話として「交番はあそこです」「ちょっとわかりません」「このへん交番はないですねー」のどれが自然か,という質問を,100名を超える大学生や大学教員におこなったところ,1名を除いて残り全員が「交番はあそこです」は不自然で,「ちょっとわかりません」「このへん交番はないですねー」だけが自然と回答した.(「さー,あっ,交番はあそこです」のように,驚きの「あっ」の挿入で「意外な展開」が演出された発話はこの質問の対象外であることに注意されたい.)

また,実際の会話記録を調べても,「さー,交番はあそこです」に相当すると思えるデータはなかなか出てこない一方で,「さー,ちょっとわかりません」「さー,このへん交番はないですねー」に相当すると思えるデータは容易に見つかる.

以上の観察は,「さー」というフィラーがどんな検討の場合にも現れるわけではなく,検討しても答が出ない場合(「さー,ちょっとわかりません」型の場合)や,検討の結果,望ましくない答が出てくる場合(「さー,このへん交番はないですねー」型の場合)にかぎって現れることを示している.つまり,「さー」は,検討してもダメな場合専用のフィラーである.「あからさまに儀礼的なフィラー」とは,このようなフィラーを指す.

したがって,これから「さー」と言おうと口を開き舌を動かし始める段階で話し手はすでに,自分がこれからおこなう検討が,見込みのない検討だと知っている,ということになる.検討しても見込みがないと知っていることを「さー」で示しつつ,わざわざ検討することは,単純に考えれば時間の浪費か,相手に対する愚弄行為でしかないはずだが(実際,他言語を母語とする日本語学習者の中にはそのように感じる者もいる),日本語コミュニケーションの中でフィラー「さー」は丁寧な印象と結びついており,むしろ「さー」のない「ちょっとわかりません」「このへん交番はないですねー」の方がつっけんどんな印象を与えがちである.

このことを理解するには,「さー」と言いつつ交番のありかを検討することは,「私がいまやっている検討は,やってもうまい結果が出る見込みのない検討です」と言いつつ交番のありかを検討することは違う,と考える必要がある.日本語社会において丁寧と位置づけられているのは,他人から「このあたりの交番を教えてほしい」などと頼み事をされたら,たとえ見込みがないと思っても,相手の目の前でダメもとでがんばってみせるという行動であり,「さー」はその行動の一部である.「さー」の検討が見込みのない検討であることを,話し手は「表し」てなどいない.それは外部から見た,狩人の知恵である.


\section{つっかえ}

単語をしゃべっている最中につっかえてしまうということはどんな言語の話者にもある.だが,そのつっかえ方が日本語には豊富にあり,つっかえ方によって態度が違う(定延,中川 2005; 定延 2005c).
 たとえば,「最近テレビではやっているマンガあるでしょ,ほら」に続けて「どくろ仮面」と言おうとしたもののうまく思い出せずつっかえる場合なら,「ど,どくろ仮面?」(とぎれ型・語頭戻り方式),「ど,くろ仮面?」(とぎれ型・続行方式),「どーどくろ仮面?」(延伸型・語頭戻り方式),「どーくろ仮面?」(延伸型・続行方式)など,どれも自然で特に制限はない.

だが,「これで街は壊滅じゃ.うわっはっはっ……」と悪の首領が笑っているところへ「そうはさせん!」とどくろ仮面の声が聞こえてきた場合,あわてふためく悪の首領が「その声は」に続けて言うセリフとしては,「ど,どくろ仮面!」(とぎれ型・語頭戻り方式)だけが自然で,「ど,くろ仮面!」「どーどくろ仮面!」「どーくろ仮面!」は自然ではない.より現実的な例を挙げれば,相手の子供が海外(たとえばカリフォルニア)に留学すると聞いて,それはまた大変な,すばらしいところへと儀礼的に驚いてみせる場合,「カ,カリフォルニアですか!」(とぎれ型・語頭戻り方式)は自然だが,「カ,リフォルニアですか!」(とぎれ型・続行方式),「カーカリフォルニアですか!」(延伸型・語頭戻り方式),「カーリフォルニアですか!」(延伸型・続行方式)は自然ではない.これらの例が示すように,驚いてモノの名を叫ぶ際のつっかえ方は,厳しく制限されている.

また,店員が商品の在庫状況を考え考え客に語る際,「ざい,こ,は…」のようなとぎれ型は余裕のない新米店員風,「ざいーこーは…」のような延伸型は余裕があるベテラン店員風という具合に,つっかえ方は話し手の発話キャラクタの違いにも結びついている.「構造改革,うーを,進めるに,いーおいてですね」のような,とぎれ延伸型のつっかえ方も,「知識人」のキャラクタと結びついている.

このように,一見したところでは単なる非流ちょうなまちがいに過ぎないつっかえには実はさまざまな型や方式があり,それらは話し手の態度や発話キャラクタと結びついている.このことじたいはもちろん興味深いことで,今後さらに調べていく必要があるが,ここで強調しておきたいのは,この結びつきがあくまで狩人の知恵だということである.話し手がこの狩人の知恵を利用して自己を演出する場合は多いかもしれないが,いつも必ずそうだというわけではない.「話し手はつっかえ方を選ぶことによって,自分の態度や発話キャラクタを見事に表している」という考え方は常には妥当しない.たとえば,どくろ仮面の出現に驚いたからといって,悪の首領が「ど,どくろかめん!」で驚きを表すという想定は自然なものとは思えない.首領にとっては,どくろ仮面の登場に自分が動揺し,驚いていることは,何よりも隠しておきたいことのはずである.「新米店員がつっかえ方を選ぶことで,自分は新米で余裕がないと表す」という想定にも同様の不自然さがつきまとう.高度な知識を客に問われて返答に窮するという失態は何としてもさらしたくないから,自分が頼りない店員だと積極的ににおわせて,ベテラン店員に乗り換えてもらう,といった場合はもちろんあるかもしれないが,いつも必ずそうだというわけではないだろう.

単語を発音している最中につっかえてしまう多くの場合,話し手は,その単語をうまく最後まで発音したいと,それなりに一所懸命になっている.よりによってその局面で,話し手が「自分の態度や発話キャラクタの表現」という別の仕事に余念がないという想定が自然なものとは思われない.つっかえる話し手は多くの場合, つっかえたくてつっかえているのではない.話し手にとって,つっかえは「ヘタ」な「失敗」だということをはっきりさせておく必要がある.発話キャラクタや態度は,われわれがつっかえ方で「表す」ものではない.これらは我々が否応なしに日々「実践する」ものである.「言語にはスキル(うまい〜ヘタ)という概念が不可欠で,この概念を含まない言語モデルは破綻する」というデュリーの言葉の意味を(Durie 1995: 304, 注3),我々は考えてみる必要があるのではないか.


\section{りきみ}

これまでの「感情音声」研究では,音声の高低・長短・強弱ばかりが取りあげられ,他の側面はほとんど注目されてこなかった.しかし,言語によっては声質(せいしつ,voice quality)の違いが音素なみに単語の識別に貢献する(したがって声質を特異なものと見るのは偏見に過ぎない)という認識が広まり(たとえばGordon and Ladefoged 2001: 383),声質を処理する技術の開発と相俟って(Campbell and Mokhtari 2003),近年では声質と感情・態度の結びつきも積極的に考察され始めている.ここでは「りきみ」と呼ばれる(郡 1989),日本語の声質の一つに目を向けてみよう.

りきみは一般には強調表現と理解されているようだが,実例を見るとそれは必ずしも当たっていない.たとえば次の例では,下線を付けた,上司の提案に対する否定的な評価の部分がりきんで発せられている.
\vspace{\baselineskip}

なかなかでもほんとにたしかにーあの,意見を言うのは(笑),むずかしいですよねそのー上司とかがー,いてて・\underline{「いやーこれは,ちょっとやっぱりー良くないと思います」}
ってのは,すごくー勇気がー要ります(笑).
\vspace{\baselineskip}

もしもりきみが強調を表すなら,下線部は上司の提案に対する強い否定になり,それだけ失礼な物言いになるはずだが,この録音を聞かせた10人の日本語話者は全員,このりきみに丁寧な印象を受けるとアンケート調査で回答している.このことからすれば,りきみは強調ではなく,恐縮(という一種の苦しみ)と結びついているということになる—
だが,もっと重要なことは,上のりきみ発言が,「恐縮ですが」と前置きして上司の提案を朗々と批判することよりもはるかに丁寧だということである.もしもりきみが恐縮を「表す」なら,この違いは説明できない.りきみは恐縮を「表す」のではなく,恐縮という行動それじたいの一部である.

りきみが単なる強調表現と考えられないことを示す例の中には,りきみが「体験者」の特権的行動であることをよく見せてくれるものがある.たとえば次のようなものである.

\vspace{\baselineskip}
でもそのあの脳ミソの構造ってどいなってんのやろなーあの忘れていくのんでも,
    \underline{私あれ恐}\linebreak\underline{怖やわー}
\vspace{\baselineskip}

このデータはなごやかな談笑会話の断片ではあるが,下線を付けた最後の部分「私あれ恐怖やわー」で話し手(Aとする)は,脳の老化に対する恐怖を吐露しており,この部分がりきんで発せられている.たとえば久しぶりに再会した恩師のボケぶりに愕然としたこと,入院している母親に「どちらさまですか?」と言われてしまったこと,自分の血筋が代々ボケる血筋で,あと十年もしたら自分もどうなっているかと折に触れ感じていること等々,これまでの人生で感じてきた,脳の老化に対する数々の恐怖が,一つ一つ具体的に語られてこそいないが,Aの心内ではここで改めて呼び起こされていると言ってよいだろう.

ここで重要なことは,この発話を聞いた相手 (B) が後日,別のところで第三者 (C) を相手に話しても,同じところでりきめないということである.「ほら,年とってだんだん物忘れが激しくなって,ボケるってのあるじゃない.Aさんあれすごい恐怖だって」などと,BはCに対して「すごい」という語句を使って強調してしゃべることはできるが,Bは「Aさんあれ恐怖だって」をりきんで発することはできない.BはAと異なり,いま語られている恐怖の体験の当事者ではないからである (Sadanobu 2004).
では,過去の体験を語る時,体験者だけがなぜりきめるのか? この問題の解答は,「人は体験を語ることで,それをもう一度体験する」というラボフの言葉 (Labov 1972: 354) で尽きているというのが筆者の考えである.過去の体験を語る話し手は,体験を「表す」わけではない.「表す」ことなら誰にでもできるはずである.体験者は過去の体験を「表す」のではなく,体験を(もちろん脚色・演出も含めて)「もう一度相手の前でやってみせる」.

過去の苦しい体験(恐怖の体験もその一種である)を語るとは,たとえ全体としてはなごやかな談笑であっても,相手の前でもう一度苦しんでみせるということである.体験者だけがりきめるのは,りきみが苦しみ行動それじたいの一部だからである.

なお,りきみは苦しみだけでなく,感心と結びつくこともある.「カール・パーキンズのレコードって集めるの大変なんだけど,これがまたいいんだよねー」という1つの文をしゃべる話し手の声が,「集めるの大変なんだけど」の部分で苦しくりきまれ,「これがまたいいんだよんねー」の部分で明るくりきまれる,といったことは日常めずらしくない.話し手はこの声で何を表しているのか? いままで述べてきたことが正しければ,話し手はこの声で何も「表し」てはおらず,もう一度体験をしてみせている.「集めるの大変なんだけど」の部分ではレコード収集の苦しい体験をしてみせており,「これがまたいいんだよんねー」の部分では鑑賞の嬉しい感心体験をしてみせている.これらのりきみは,それぞれの体験の一部である.


\section{むすび〜個人と共同体の間}

「ちょっとわかりません」の前の「さー」のように,フィラーがある方が発話が丁寧で,フィラーがない方がつっけんどんで印象が悪いという場合がある(第3節).つっかえ方にもいろいろな型や方式あり,たとえば驚いてモノの名前を叫ぶ場合はとぎれ型・語頭戻り方式という具合に,それぞれの態度によってつっかえ方が決まっている(第4節).声質も同様で,「普通」の声質よりもりきんだ声質の方が恐縮という態度と結びつき,発話が丁寧になる場合がある\linebreak
(第5節).

これらの観察はともすれば,「話し手は自分のきもちに応じて,フィラー・つっかえ方・声質を使い分けている」という考えを正しく見せる.だが,本稿がこの考えに満足するものではないということも,これまで述べてきたことから明らかだろう.フィラーやつっかえ,声質は,行動それじたいであって,「使い分け」の対象になるようなモノではない.そもそも,「使い分け」という目的論的な行動は外部者の見立てであって,話し手が常にそのようなふるまいに出るわけではない.たとえば,どくろ仮面の思わぬ登場に動揺した悪の首領が,自らの動揺を表すために,専用のつっかえ方を選ぶといった想定は不自然である.

それはちょうど,「カメレオンの祖先は,体表を周囲の色と同化させる,保護色という進化の道を選んだ」という生物学的なレトリックを,「カメレオンの祖先の一匹一匹が進化という概念を理解しており,自分たちにどういう進化の選択肢があるかを把握した上で,その中から保護色という進化の道を『選んだ』」と受け取ることが不自然であるのと同じことである.言語共同体レベルで,きもちと,フィラー・つっかえ方・声質の間に結びつきが観察されたとしても,それを個々の状況における個々の話し手の「使い分け」と考えてよいわけではない.

しかしながら,言語共同体レベルで観察されるそれらの結びつきが,話し手一人一人の個別的行動と無関係に存在するはずもない.そして,これまでの多くの言語研究が,言語共同体レベルの結びつきばかりを重視し,個人の個別的行動を軽視〜無視する傾向にあったということは否めない(定延,中川 2005).

では,言語共同体レベルのそれらの結びつきと,個人の個別的行動とは,どのようにつながっているのだろうか? この問題を考える上で,ギヴォンやホッパーらの談話語用論 (Discourse Pragmatics) と呼ばれる学派の考えは,きもちと,フィラー・つっかえ方・声質との結びつきに特化したものではないが,有益なものである.

筆者の理解によれば,談話語用論は,個人の個別的行動こそ言語共同体における言語慣習(文法)の源だと考えている.数限りない個別の日常的談話の中で,繰り返し生じる単語列のパターンが,やがて文型になり,文法として立ち上がる(つまり「文法化する」).個別的な談話で話し手が何事かを1回しゃべるたびに,発せられた語列が文法へと近づく,という形で,談話語用論は文法を談話からとらえ直そうとしている.「文法というものはない.あるのは文法化だけだ」というホッパーの発言は (Hopper 1987),このことをよく表している.

但し,個々の談話から,どのように言語慣習が立ち上がるかについて,これまで提出されているアイデアは「頻度」1つしかない.つまり,個別的な談話において何度も繰り返し生じる言い方が共同体の言語慣習となり,あまり生じない言い方は言語慣習とならないという考えである.この考えは不自然なものではないと思うが,考えるべきことは頻度以外にもあるのではないか.

たとえば,終助詞「わ」の女性専用の用法は,実際の会話ではほとんど観察されなくなってきている(尾崎 1999を参照).だが,女性専用の「わ」がドラマや映画,小説の言語に現れることは今でも珍しくない.女性専用の「わ」でかもしだされるキャラクタが(たとえば,あまりに女性らしさを強調しているなどの理由で)魅力あるものに映らなければ,「わ」に頻繁にさらされても使わないという女性話者の「選り好み」がここには見て取れる.個人の個別的行動と言語共同体レベルの言語慣習をつなぐには,出現頻度だけでなく,たとえば「かっこよく/強そうに/かわいく/セクシーに/知的に/まじめに ふるまいたい」,逆に「かっこわるく/弱々しく/醜く/野暮ったく/馬鹿者として/不誠実に ふるまいたくない」といった,日々のコミュニケーションを生きる個々人の欲(思い,思惑,打算など)と発話キャラクタ(定延 2006)に着目する必要があるのではないだろうか.



\acknowledgment

本稿は,言語処理学会第12回年次大会併設ワークショップ(W1)「感情・評価・態度と言語」(2006年3月17日,於慶應義塾大学)での招待講演をもとにしている.講演のために金田純平氏・中川明子氏(ともに神戸大学大学院総合人間科学研究科)の技術的協力を得たこと,講演後,会場内外で多くの方から有益なコメントを頂いたことを記して謝意を表したい.なお本稿は,日本学術振興会の科学研究費補助金による基盤研究(A)「日本語・英語・中国語の対照にもとづく,日本語の音声言語の教育に役立つ基礎資料の作成」(課題番号:16202006,研究代表者:定延利之),総務省の戦略的情報通信研究開発推進制度(SCOPE,課題番号:041307003,研究代表者:ニック・キャンベル),科学技術振興機構 (JST) による戦略的創造研究推進事業 (CREST),「表現豊かな発話音声のコンピュータ処理システム」(研究代表者:ニック・キャンベル)の成果の一部である.



\nocite{*} 
\bibliographystyle{jnlpbbl_1.2}
\begin{thebibliography}{}

\bibitem[\protect\BCAY{Campbell \BBA\ Mokhtari}{Campbell \BBA\
  Mokhtari}{2003}]{Nick2003}
Campbell, N.\BBACOMMA\ \BBA\ Mokhtari, P. \BBOP 2003\BBCP.
\newblock \BBOQ Voice quality:the 4th prosodic dimension\BBCQ\
\newblock {\Bem Proceedings of the 15th International Congress of Phonetic
  Sciences'03}, \mbox{\BPGS\ 2417--2420}.

\bibitem[\protect\BCAY{Chafe}{Chafe}{1992}]{Wallace1992}
Chafe, W. \BBOP 1992\BBCP.
\newblock \BBOQ Immediacy and displacement in consciousness and language\BBCQ\
\newblock {\Bem In Stein, Dieter (ed.), Cooperating with Written Texts: The
  Pragmatics and Comprehension of Written Texts, Berlin; New York: Mouton de
  Gruyter}, \mbox{\BPGS\ 231--255}.

\bibitem[\protect\BCAY{Chafe}{Chafe}{2001}]{Wallace2001}
Chafe, W. \BBOP 2001\BBCP.
\newblock \BBOQ The analysis of discourse flow\BBCQ\
\newblock {\Bem In Schiffrin, Deborah, Tannen, Deborah, and Hamilton, Heidi
  Ehernberger (eds.), The Handbook of Discourse Analysis, Blackwell},
  \mbox{\BPGS\ 673--687}.

\bibitem[\protect\BCAY{{Du Bois}}{{Du Bois}}{2003}]{DuBois2003}
{Du Bois}, J.~W. \BBOP 2003\BBCP.
\newblock \BBOQ Discourse and grammar\BBCQ\
\newblock {\Bem In Tomasello, Michael (ed.), The New Psychology of Language:
  Cognitive and Functional Approaches to Language Structure, Mahwah, New
  Jersey: Lawrence Erlbaum}, {\Bbf 2}, \mbox{\BPGS\ 47--87}.

\bibitem[\protect\BCAY{Durie}{Durie}{1995}]{Mark1995}
Durie, M. \BBOP 1995\BBCP.
\newblock \BBOQ Towards an understanding of linguistic evolution and the notion
  ``X has a function Y''\BBCQ\
\newblock {\Bem Amraham, Werner, Giv\'on, Talmy, and Thompson, Sandra A.
  (eds.), Discourse Grammar and Typology: Papers in Honor of John W. M.
  Verhaar, Amsterdam/Philadelphia: John Benjamins}, \mbox{\BPGS\ 275--308}.

\bibitem[\protect\BCAY{Fillmore}{Fillmore}{1978}]{Charles1978}
Fillmore, C.~J. \BBOP 1978\BBCP.
\newblock \BBOQ On fluency\BBCQ\
\newblock {\Bem Fillmore, Charles J, Daniel, Kempler and William S-Y. Wang
  (eds.), Individual Differences in Language Ability and Language Behavior, New
  York: Academic Press}, \mbox{\BPGS\ 85--101}.

\bibitem[\protect\BCAY{Gordon \BBA\ Ladefoged}{Gordon \BBA\
  Ladefoged}{2001}]{Matthew2001}
Gordon, M.\BBACOMMA\ \BBA\ Ladefoged, P. \BBOP 2001\BBCP.
\newblock \BBOQ Phonation types: a cross-linguistic overview\BBCQ\
\newblock {\Bem Journal of Phonetics}, {\Bbf 29}, \mbox{\BPGS\ 383--406}.

\bibitem[\protect\BCAY{Hopper}{Hopper}{1987}]{PaulJHopper1987}
Hopper, P.~J. \BBOP 1987\BBCP.
\newblock \BBOQ Emergent grammar\BBCQ\
\newblock {\Bem BLS 13}, \mbox{\BPGS\ 139--157}.

\bibitem[\protect\BCAY{Labov}{Labov}{1972}]{William1972}
Labov, W. \BBOP 1972\BBCP.
\newblock {\Bem Language in the Inner City: Studies in the Black English
  Vernacular}.
\newblock Philadelphia: University of Pennsylvania Press.

\bibitem[\protect\BCAY{Sadanobu}{Sadanobu}{2004}]{Sadanobu2004}
Sadanobu, T. \BBOP 2004\BBCP.
\newblock \BBOQ A natural history of Japanese pressed voice\BBCQ\
\newblock {\Bem Journal of the Phonetic Society of Japan (Onsei Kenkyu)}, {\Bbf
  8}  (1), \mbox{\BPGS\ 29--44}.

\bibitem[\protect\BCAY{郡史郎}{郡史郎}{1989}]{郡1989}
郡史郎 \BBOP 1989\BBCP.
\newblock \JBOQ 強調とイントネーション\JBCQ\
\newblock \Jem{杉藤美代子(編),『日本語の音韻・音声(上)』,明治書院},
  \mbox{\BPGS\ 316--342}.

\bibitem[\protect\BCAY{定延利之}{定延利之}{2000}]{定延2000}
定延利之 \BBOP 2000\BBCP.
\newblock \Jem{認知言語論}.
\newblock 大修館書店.

\bibitem[\protect\BCAY{定延利之}{定延利之}{2005a}]{定延2005b}
定延利之 \BBOP 2005a\BBCP.
\newblock \JBOQ 「表す」感動詞から「する」感動詞へ\JBCQ\
\newblock \Jem{『言語』{\unskip}}, {\Bbf 34}  (11), \mbox{\BPGS\ 33--39}.

\bibitem[\protect\BCAY{定延利之}{定延利之}{2005b}]{定延2005c}
定延利之 \BBOP 2005b\BBCP.
\newblock \Jem{ささやく恋人,りきむレポーター—口の中の文化—}.
\newblock 岩波書店.

\bibitem[\protect\BCAY{定延利之}{定延利之}{2006}]{定延2006}
定延利之 \BBOP 2006\BBCP.
\newblock \JBOQ ことばと発話キャラクタ\JBCQ\
\newblock \Jem{『文学』,岩波書店}, {\Bbf 7}  (6), \mbox{\BPGS\ 117--129}.

\bibitem[\protect\BCAY{定延利之\JBA 中川明子}{定延利之\JBA
  中川明子}{2006}]{定延・中川2006}
定延利之\JBA 中川明子 \BBOP 2006\BBCP.
\newblock \JBOQ
  非流ちょう性への言語学的アプローチ:発音の延伸・とぎれを中心に\JBCQ\
\newblock \Jem{串田秀也・定延利之・伝康晴(編),
  『文と発話1:活動としての文と発話』,ひつじ書房}, \mbox{\BPGS\ 209--228}.

\bibitem[\protect\BCAY{定延利之}{定延利之}{2002}]{定延2002}
定延利之 \BBOP 2002\BBCP.
\newblock \JBOQ 「うん」と「そう」に意味はあるか\JBCQ\
\newblock \Jem{定延利之(編),『「うん」と「そう」の言語学』,ひつじ書房},
  \mbox{\BPGS\ 75--112}.

\bibitem[\protect\BCAY{定延利之}{定延利之}{2005}]{定延2005a}
定延利之 \BBOP 2005\BBCP.
\newblock \JBOQ 「雑音」の意義\JBCQ\
\newblock \Jem{『言語』{\unskip}}, {\Bbf 34}  (1), \mbox{\BPGS\ 30--37}.

\bibitem[\protect\BCAY{菅原和孝}{菅原和孝}{2002}]{菅原2002}
菅原和孝 \BBOP 2002\BBCP.
\newblock \Jem{感情の猿=人}.
\newblock 弘文社.

\bibitem[\protect\BCAY{尾崎喜光}{尾崎喜光}{1999}]{尾崎1999}
尾崎喜光 \BBOP 1999\BBCP.
\newblock \JBOQ 女性専用の文末形式のいま\JBCQ\
\newblock \Jem{現代日本語研究会(編),『女性のことば・職場編』,ひつじ書房},
  \mbox{\BPGS\ 33--58}.

\end{thebibliography}



\begin{biography}
\bioauthor{定延 利之(非会員){\unskip}}{
1962年生.85年京都大学法学部,87年文学部文学科卒.98年同大学大学院文学研究科博士課程修了.博士(文学).神戸大学大学院国際文化学研究科教授.言語とコミュニケーションの研究に従事.
}
\end{biography}

\biodate

\end{document}
