
\documentstyle[epsf,jnlpbbl]{jnlp_j_b5}

\setcounter{page}{73}
\setcounter{巻数}{2}
\setcounter{号数}{3}
\受付{1995}{1}{10}
\再受付{1995}{2}{7}
\採録{1995}{2}{17}
\setcounter{年}{1995}
\setcounter{月}{7}

\setcounter{secnumdepth}{2}

\unitlength=1mm

\title{片方向の共起性による述語型定型表現の自動抽出}
\author{新納 浩幸\affiref{KUEE} \and 井佐原 均\affiref{KUEE2}}

\headauthor{新納 浩幸・井佐原 均}
\headtitle{片方向の共起性による述語型定型表現の自動抽出}

\affilabel{KUEE}{茨城大学工学部システム工学科}
{Department of Systems Engineering, Faculty of Engineering, Ibaraki University}
\affilabel{KUEE2}{(投稿時)電子技術総合研究所知能情報部自然言語研究室}
{Natural Language Section, Machine Understanding Division, 
Electrotechnical Laboratory\\
(現在)郵政省通信総合研究所関西先端研究センター知的機能研究室,
Intelligent Processing Section,
Kansai Advanced Research Center,
Communications Research Laboratory,
Ministry of Posts and Telecommunications}

\jabstract{
本論文では「目を盗む」や「かたずを飲む」などの述語型定型表現を
コーパスから自動抽出することを目的に,
従来の相互情報量の条件を緩める方向で,
名詞動詞間の共起性を測る新たな基準を提案する.
概略,名詞,動詞のどちらかを固定して,
その単語と共起する集合内の各単語に,どの程度特異な頻度になっているかの
数値を与える.この数値は集合内のその単語の頻度の割合と,
集合内の単語の種類数から計算される.
この数値の上位のものを取り出すことで定型表現の抽出を行う.
本手法の特徴は,名詞を固定した場合に抽出できる表現と,動詞を固定した場合に
抽出できる表現はほとんど共通のものがなく,しかもどちらの場合も
相互情報量による抽出程度の正解率を得られることである.
このため,目的の抽出数の半数づつを
名詞固定と動詞固定の各々の場合から取り出せば,
相互情報量を用いて抽出する場合よりも高い正解率が得られる.
}

\jkeywords{定型表現,慣用表現,自動抽出,相互情報量,共起性}

\etitle{Automatic Acquisition of Predicative Frozen \\
Patterns on One Directional Association}
\eauthor{Hiroyuki Shinnou \affiref{KUEE} \and Hitoshi Isahara\affiref{KUEE}} 

\eabstract{
This paper presents an alternative method to measure word
association strength on predicative patterns in order to
automatically extract predicative frozen patterns and idioms from a
corpus.  For this aim, mutual information is traditionally used.  We
improve the method on mutual information from a view of linguistics.
The proposed method are realized by following steps.  First, a verb
(or noun) is fixed.  Next, the set of nouns (or verbs) which
associates the verb (or noun) is built up.  Last, nouns (or verbs)
which have peculiar frequency are chosen from this set.  The
peculiarity is confirmed from two characteristics, which are ratio
of the word frequency for total frequency of the set, and the number
of kind of word in the set.  Predicative frozen patterns are
constructed from chosen words and the fixed word.  The advantage of
this method is that patterns extracted by fixing a verb and patterns
extracted by fixing a noun have few common patterns, and each
extraction has equivalent ratio of correctness to extraction by
mutual information.  Therefore, extracting same number patterns,
this method can get more correct patterns than mutual information.
}

\ekeywords{frozen pattern, idiom, automatic extraction, mutual information, collocation}

\begin{document}
\maketitle



\section{はじめに}

自然言語には定型表現と呼ばれる単語間の共起性が強い表現が数多く存在する.
定型表現を収集,整理しておくことは言語学的な観点からも
機械処理の観点からも有益である.

例えば「目を盗む」や「かたずを飲む」などの慣用表現は,
その表現の意味が個々の構成語の意味からは
作り出すことができない\cite{miyaji}.
このために,
機械処理ではそれら表現に例外的な処理を施す必要がある.
また言語学的にも語の持つ意味の標準的用法と非標準的用法の
境界を考察する上で,このような表現を網羅的に収集することが
望まれる.
また慣用表現ではなくとも,「に関して」「も少なくない」「て欲しい」などの
定型表現では個々の構成語に分割して処理するよりも予め一語として捉えていた方が
機械処理の面では実用的な場合が多い.
また外国語習得の面でも,
共起性の強い表現を単語のように,
1つの概念に対応する固定した文字列として捉え,
それらを記憶しておくことが効果的である.
その他,音声認識,OCRにも,共起性の強い表現を
記憶しておくことが,そこでの曖昧性の解消に役立つことが
知られている\cite{church,kita}.

定型表現は付属語的なものと,そうでないもののに大きく分けられ,
後者の中に述語型定型表現が存在する.
述語型定型表現とは「目を盗む」のように
\begin{center}
名詞  +  格助詞  + 動詞
\end{center}
のパターンになっている定型表現である.
これら表現は定型表現の大きな部分を占め,また,
通常の名詞動詞間の共起による解析との整合性が
必要となる\cite{oku,suzuki}.
さらに「将棋を指す」「碁を打つ」のように,同じの意味の動詞(play)でも
名詞によって異なる表現を用いるコロケーションの問題を考察する上でも,
述語型定型表現の収集が望まれる.
このような理由から定型表現の中でも特に述語型定型表現を収集する
ことは重要である.

述語型定型表現を収集することは有益であるが,その収集は困難である.
なぜなら,それら表現の客観的な定義は困難なため,
個々の表現に対して人間の判断が必要となり,
その収集には膨大な時間と手間がかかるからである\cite{syudo}.
また人手による収集では,その網羅性,一貫性などの問題点もある.

これらの点から定型表現や慣用表現の自動抽出の試
みがなされているが\cite{smadja,shinnou},それら
研究の多くは相互情報量を用いて共起の強さを測ることを
基本としている\cite{church}.
相互情報量は
2つの単語がそれぞれ独立に現れる確率と同時に現れる確率との
比を基に共起の強さを測る.
基本的に相互情報量では2単語間が引き合う強さを総合して判断し,
共起の強さを定めている.
しかし言語的に考えれば,
一方の単語がもう一方を引っぱるような片方向だけの強さを持って
いる場合でも,その表現に定型性があると考えることは自然である.

本論文では上記の点を考察し,
述語型の表現における名詞動詞間の共起性を測る新たな基準を提案する.
概略述べると,まず,名詞あるいは動詞を固定して,
共起している単語の集合を作り,その集合内で特異な高頻度の単語を
取り出す.これによって,片方向から引っ張る強さの条件だけで
抽出を行なうことができる.
特異な高頻度の単語の判定法は,
基本的に集合内の頻度の割合と,集合内の単語の種類数から
判定する.判定の際に共起の強さを表す数値を与える.
最終的にこの数値の上位部分を抽出とする.

実験として,本論文で提案する基準を用いて,
朝日新聞1か月分のコーパス(テキスト部分約9Mbyte)から「AをBする」の
形の述語型定型表現の抽出実験を行ない,
本手法の有効性を確認した.
その結果,名詞を固定した場合に抽出できる表現と動詞を固定した場合に
抽出できる表現には,ほとんど共通のものがなかった.また抽出の
正解率はどちらの場合も相互情報量による抽出と同程度であった.
一方,相互情報量による抽出の正解率は抽出数を増やしてゆけば
当然下がる.このことから,同数の抽出を行なうことを考えると,
本手法の場合,その半数の抽出の場合の正解率を保つことができ,
相互情報量を用いた手法よりも広い範囲の定型表現を抽出できることがわかる.


\section{相互情報量からの共起性測定法の問題点}

相互情報量の定義は以下の式である.
\[
I(x,y) = \log_{2} \frac{p(x,y)}{p(x)p(y)}
\]
ただし \( p(x) \) はコーパス中に \( x \) が現れる確率,\( p(x,y) \) は
コーパス中に \( x \) と \( y \) がこの順に並んで現れる確率である.

相互情報量は基本的に単語 \( x \) と単語 \( y \) の共起の強さを測るものであり,
多単語間の共起の強さを測るために,どのような拡張を行なうかは未解決である.
ただし,述語型の表現
\begin{center}
名詞(\(n\)) + 格助詞(\(r\)) + 動詞(\(v\))  
\end{center}
の場合,以下の式によって相互情報量を測ることが自然である.
\[
I(r,n,v) = \log_{2} \frac{\frac{f(r,n,v)}{N}}{\frac{f(n)}{N} \frac{f(v)}{N}} \cdots (1)
\]
ここで \( N \) はコーパス中の文の総数,\( f(x) \) は単語\( x \)がコーパス中で
格助詞 \( r \) をともなって出現した頻度,
\( f(r,n,v) \) はコーパス中で名詞 \( n \) と動詞 \( v \) が格助詞\( r \) によって共起
した頻度である.

(1)式からわかるように,相互情報量は本質的に
\[
\frac{f(r,n,v)}{f(n)f(v)} \cdots (2)
\]
の値によって強さの比較が行なわれている.
\( f(r,n,v) \leq f(n) \),\( f(r,n,v) \leq f(v) \) は
明らかなので,(2)式は
\[
   f(r,n,v) = f(n) = f(v) \cdots (3)
\]
の時に最大値をとる.ここで注目すべきは,(3)式の条件は,
以下の条件よりも強いということである.

\begin{description}
\item[【条件a】] \underline{名詞 \( n \)  ,格助詞\( r \)が現れると必ずその後には動詞\( v \) が現れる.}
\end{description}

通常,[条件a] は定型性が認められる十分な条件だと考えられる.
しかし[条件a] を満たし,しかも (2) 式の値が低くなるケースは非常に多い.
例えば図1,図2の例を見てみる.


\begin{figure}[h]
\begin{center}
    
    
    
\epsfile{file=fig1and2.eps,width=133mm}
\end{center}
\end{figure}
\begin{minipage}{66mm} 
  \begin{center}
{\small {\gt 図1}\ \ 定型表現の共起の例}
  \end{center}
\end{minipage} \begin{minipage}{67mm}
  \begin{center}
{\small {\gt 図2}\ \ 一般表現の共起の例}
\normalsize
  \end{center}
\end{minipage}


\addtocounter{figure}{2}

\bigskip 

\bigskip

通常,「かたずを」という表現が現れれば「飲む」という動詞が現れる.
「かたずを」の頻度が 100 ,「かたずを飲む」の頻度も 100 となっているが,
「を飲む」自身は他の多くの名詞とも共起するために,
その頻度は 1000 となっている(図1参照).
このため(2)式の値は \( 10^{-3} \) となる.
一方,「日本語を勉強する」という表現では,
「日本語を」の頻度が 200,
「日本語を勉強する」の頻度が 10 ,
「を勉強する」の頻度が 50 ,
となっている(図2参照).この場合も(2)式の値は \( 10^{-3} \) となる.
「かたずを飲む」と「日本語を勉強する」が上記のような頻度分布を持つ場合,
それらに同じ共起の強さを与えるのは,不自然さがある.

言語的に考えると,[条件a] は順方向に読んでいくと
後ろの部分が定まる,つまりこう言えば必ずその後はこう言う,といった
表現のもつ条件である.
相互情報量はさらに逆方向の共起の強さも加味している.
つまりこの表現の前には必ず
こういう表現が現れているはずという条件も加味している.
言葉を換えてまとめれば,相互情報量は,その言葉通り,
相手の単語を\underline{相互に}引っ張る力を総合してその値を定めている(図3参照).



\begin{figure}[h]
\begin{center}
    
    
    
\epsfile{file=fig3.eps,width=95mm}
\end{center}
\caption{双方向からの共起}
\end{figure}


\section{片方向の強さのみによる共起性の測定}

相互情報量は双方向から引っ張る力を総合して考えて,
その単語間の共起の強さを数値化している.
これは共起の強さの一側面を表しており,
この判定法によってもある種の定型表現の類は抽出できる.

しかし,上記したように,言語的に考えれば片方向からの共起の強さだけ持っていた
場合でも共起性があると判断するのは自然である.
本論文ではこの点に注目して,
相互情報量の双方向という条件を片方向という条件の形に直した新たな
共起性の測定方法を提案する.

まず,ある動詞 \( v_{0} \) と格助詞\( r_{0} \) を伴って
現れた名詞 \( n \) の集合を作成する.つまり,
\begin{center}
名詞(\(n\)) + 格助詞(\(r_{0}\)) + 動詞(\(v_{0}\))  
\end{center}
の形を持つ名詞 \( n \) をコーパスから取り出す.
この名詞\( n \)の集合から特異な頻度を持つ名詞 \( n_{i} \) 
(複数の場合もあり得る)を取り出し,
\begin{center}
名詞(\( n_{i} \)) + 格助詞(\(r_{0}\)) + 動詞(\(v_{0}\))  
\end{center}
を定型表現として抽出する.

「特異な頻度」の判定方法だが,
ここでは基本的に以下の式の値を
動詞\( v \)に対する名詞 \( n \) の特異な頻度の程度を表す値とした.
\[
\frac{f(r,n,v)}{\sum f(n: r,v)}*\frac{1}{k(n:r,v) - 1} ...(5)
\]
ここで,\( k(n: r,v) \) は動詞\( v \)が格助詞 \( r \) をともなって
共起する名詞の種類数を表す.
\( k(n: r,v) \) が \( 1 \) の場合,(5)式の値は \( 1 \) と定義する.
また\(\sum f(n:r,v) \) は,
動詞\( v \) が格助詞\( r \)をともなって共起する名詞の総頻度数を表す.

すべての動詞を固定した場合に得られた表現を(5)式の値によってソートし,
その上位の部分を抽出する.
(5)式は固定した動詞と共起する名詞\( n \)の種類数が少なく(\( k(n: r,v) \) が小さい),
固定した動詞と共起する名詞の総頻度(\(\sum f(n:r,v) \))に対して
注目している名詞の頻度(\( f(r,n,v) \))の割合が高いほど
大きな値となるように設定している.

上記までの説明は動詞を固定した場合だが,同様にして
名詞を固定した場合の抽出も行なう.

最後に図1の「かたずを飲む」を例にして試して見る.
動詞「飲む」を固定して考えると,「かたずを飲む」の(5)式の値は大きくないが,
名詞「かたず」を固定して考えると,(5)式の値は 1 になる.
一方,図2の「日本語を勉強する」の場合,
動詞を固定した場合,
\[
\frac{f(を,日本語,勉強する)}{\sum f(n:勉強する,を)}*
\frac{1}{k(n:勉強する,を)-1} < 
\frac{10}{50}*\frac{1}{3} < 0.0067
\]
名詞を固定した場合,
\[
\frac{f(を,日本語,勉強する)}{\sum f(v:日本語,を)}*
\frac{1}{k(v:日本語,を)-1} < 
\frac{10}{200}*\frac{1}{4} = 0.0125
\]
となり,どちらも小さな値であり,強い共起性は認められない.

\section{実験}

本手法の有効性を確認するために,
朝日新聞1か月分のコーパス(テキスト部分約9Mbyte)
用いて,「を」格だけを対象に,述語型定型表現の抽出実験を行なう.
なお,このコーパスは生のテキストであり,単語区切りが行なわれていたり,
品詞のタグつけがされているものではないことを注記しておく.

\subsection{共起データの収集}

本手法を適用するために,コーパスから共起データを収集する必要がある.

コーパス中で名詞Aが格助詞Bを介して動詞Cと共起した場合に,
[A,B,C]の3組のペアを取り出す.
この3組ペアを共起データと呼ぶ.
例えば「雨が降っている」からは共起データとして[雨,が,降る]
が取り出せる.

コーパスから共起データを収集することは一般に困難である.
これは解析の曖昧性の問題(省略も含む)があるからである.
このため手作業により収集することや\cite{tanaka},曖昧性のない共起データ
だけを収集することが行なわれる\cite{nakajima}.


本論文でも形態素解析を行ない曖昧性のないデータだけを対象にする.
基本的に名詞,格助詞,動詞が以下のように連続して現れた場合のみを対象とする.
\begin{center}
名詞(A) + 格助詞(B) + 動詞(C)  
\end{center}
このデータからは[A,B,C] を取り出す.
ただし以下の点に注意する.

\begin{description}
\item[(1)] 副詞の挿入
\end{description}

格助詞と動詞の間に副詞が入った場合,副詞を無視して[A,B,C]を
取り出す.

\begin{description}
\item[(2)] 代名詞
\end{description}

名詞の部分が代名詞になっているものは共起データを作成しない.

\begin{description}
\item[(3)] 複合名詞
\end{description}

名詞の部分が複合名詞になっている場合は複合名詞のままで
共起データを作成し,名詞部分が「AのB」になっているものは「B」の形で
共起データを作成する.

\begin{description}
\item[(4)] 連用句の挿入
\end{description}

連用句が挿入されている以下の形の場合,
\begin{center}
名詞(A) + 格助詞(B) +  名詞(C) + 格助詞(D) + 動詞(E) + 句読点  
\end{center}
曖昧性なく[A,B,E],[C,D,E]を認識できるが,
ここでは[C,D,E]のみを取り出す.
これは実際には[A,B,E],[C,D,E]が組合わさって
意味をなすような5項関係のものも多く存在するからである.
例えば「損を覚悟で売る」,「彼女をキャリア・ウーマンと呼ぶ」,
「株式市場を研究テーマに選ぶ」などから
[損,を,売る],[彼女,を,呼ぶ],[株式市場,を,選ぶ]
を取り出すのは妥当ではない.

\begin{description}
\item[(5)] 「を」格に対する使役の助動詞
\end{description}

動詞に助動詞が付随した場合には,基本的に助動詞を取り除いた形で
共起データを作成する.ただし格助詞が「を」であり,
しかも使役の助動詞が使用されている場合には,
その助動詞は取り除かない.これは「を」格の場合,使役の助動詞を取り除くと
意味をなさないものが生じるからである.
例えば「波長を合わせる」の場合,助動詞を外して
「波長を合う」とは言えないので,[波長,を,合う]ではなく,
[波長,を,合わせる]を取り出す.

\begin{description}
\item[(6)] 数量詞移動の現象
\end{description}

数量詞移動の現象に対しては,数量詞も別個に取り出す.数量詞移動とは,
「3匹の子豚が住んでいた」という表現が
「子豚3匹が住んでいた」 「子豚が3匹住んでいた」
という表現にそれぞれ互いに置き換えることができるという
言語現象である\cite{inoue1}.
いずれの表現が現れても,置き換え可能であることが解析で判断できた場合に,
[子豚,が,住む],[N匹,が,住む]を取り出す.

以上の点を注意して,コーパスから格助詞が「を」である共起データを収集した.
その結果45,070組,31,899種類を取り出した.

\subsection{定型表現の抽出}

本手法は動詞(あるいは名詞)を固定したときに集められる名詞(あるいは動詞)
の集合の要素数が小さいと信頼性のある結果が得られない.
このためここでは要素数が 8 以下のものは対象にしない.
 8 という数字は相互情報量との比較実験を考慮して設定した値である.
相互情報量の場合,表現の頻度が少ないと信頼性のある値がでないために,
通常,頻度が高いものだけを対象にして計算する.
本実験は頻度が 5 以上のものを対象にした.
 8 という数字は,この 5 に多少のノイズがはいることを考えての値である.

まず動詞を固定した場合の実験を行う.
この場合,頻度 8 以上の動詞は 860 種類であった.
それぞれの動詞に対して,その動詞と共起する各々の名詞に
本手法で提案している共起性の数値を与えた.
最終的にこの数値の高いもの 25 種類から得られた表現を表1に示す.
(表中の○,△,×は評価の項参照).

同様にして,名詞を固定した場合,
頻度 8 以上の名詞は 977 種類であった.
それぞれの名詞に対して,その名詞と共起する各々の動詞に
本手法で提案している共起性の数値を与えた.
最終的にこの数値の高いもの 25 種類から得られた表現を表2に示す.

次に比較実験として相互情報量を用いた抽出実験を行なった.
ここでは頻度 5 以上の共起データ 831 種類を対象にした.
基準値の高い順に,取り出した表現 25 種類を表3に示す.
ただしここでの判定値は順位つけだけが目的であるので,
(2)式の値を用いている.



\noindent
\begin{minipage}{70mm} 
\vspace{5mm}
\small
  \begin{center}
    \begin{tabular}{|c|l|c|} \hline
判定値 &  表現   & 評価\\  \hline \hline
1.000000   &  首をかしげる & ○ \\ \hline
1.000000   &  神経をとがらせる  & ○ \\ \hline
1.000000   &  目を光らせる  & ○ \\ \hline
1.000000   &  首を絞める  & × \\ \hline
0.875000   &  やむをえない  & ○ \\ \hline
0.777778   &  姿を現す  & △ \\ \hline
0.700000   &  名を連ねる  & △ \\ \hline
0.416667   &  連絡を取り合う  & △ \\ \hline
0.400000   &  国境を接する  & △ \\ \hline
0.388889   &  手を染める  & ○ \\ \hline
0.375000   &  工夫をこらす  & △ \\ \hline
0.350000   &  たばこを吸う  & △ \\ \hline
0.272727   &  思いをはせる  & ○ \\ \hline
0.263889   &  一線を画す  & ○ \\ \hline
0.242424   &  頭を痛める  & ○ \\ \hline
0.208333   &  道を閉ざす  & △ \\ \hline
0.208333   &  豊かさを実感する  & × \\ \hline
0.200000   &  疑問を呈する  & △ \\ \hline
0.193182   &  道を歩む  & △ \\ \hline
0.166667   &  一歩を踏み出す  & △ \\ \hline
0.166667   &  上告を棄却する  & × \\ \hline
0.153846   &  N円を脱税する  & × \\ \hline
0.150000   &  仕事を休む  & × \\ \hline
0.140741   &  国交を樹立する  & △ \\ \hline
0.138889   &  融資を中断する  & × \\ \hline
    \end{tabular} 

\bigskip

{\gt 表 1}\ \ 実験結果(動詞固定)
  \end{center}
\end{minipage} \begin{minipage}{70mm}
\vspace{5mm}
\small
  \begin{center}
    \begin{tabular}{|c|l|c|} \hline
判定値 &  表現  & 評価 \\  \hline \hline
1.000000   &  メスを入れる  & ○ \\ \hline
1.000000   &  群を抜く  & ○ \\ \hline
1.000000   &  端を発する  & ○ \\ \hline
1.000000   &  拍車をかける  & ○ \\ \hline
1.000000   &  本腰を入れる  & ○ \\ \hline
1.000000   &  面倒を見る  & ○ \\ \hline
1.000000   &  予断を許さない  & ○ \\ \hline
1.000000   &  一体をなす  & △ \\ \hline
1.000000   &  重傷を負う  & △ \\ \hline
1.000000   &  大勢を占める  & △ \\ \hline
1.000000   &  ボタンを押す & × \\ \hline
1.000000   &  恩恵を受ける  & × \\ \hline
0.969697   &  難色を示す  & ○ \\ \hline
0.937500   &  足並みをそろえる  & ○ \\ \hline
0.928571   &  死者を出す  & × \\ \hline
0.925926   &  損害賠償を求める  & × \\ \hline
0.923077   &  歩調を合わせる  & ○ \\ \hline
0.909091   &  尾を引く  & ○ \\ \hline
0.900000   &  感銘を受ける  & × \\ \hline
0.900000   &  脚光を浴びる  & ○ \\ \hline
0.900000   &  集会を開く  & × \\ \hline
0.888889   &  禍根を残す  & × \\ \hline
0.888889   &  力点を置く  & △ \\ \hline
0.875000   &  みそを作る  & × \\ \hline
0.875000   &  症状を訴える  & △ \\ \hline
    \end{tabular}

\bigskip

{\gt 表 2}\ \ 実験結果(名詞固定)
  \end{center}
\end{minipage}
\normalsize

\addtocounter{table}{2}

\bigskip

\bigskip


\begin{table}[htbp]
  \begin{center}
    \begin{tabular}{|c|l|c||c|l|c|} \hline
判定値 &  表現  &  評価 & 判定値 &  表現  &  評価 \\  \hline \hline
0.454545  &  警鐘を鳴らす   & ×  & 0.285714  &  一歩を踏み出す   & △ \\ \hline
0.454545  &  腎臓を摘出する & ×  & 0.280788  &  武力行使を伴う   & × \\ \hline
0.395833  &  一線を画す     & ○  & 0.277778  &  事情を聴く       & △ \\ \hline
0.379310  &  端を発する     & ○  & 0.277778  &  足並みをそろえる & ○ \\ \hline
0.375000  &  臨時国会を召集する & × & 0.272727  &  和解を勧告する   & × \\ \hline
0.357143  &  神経をとがらせる   & ○ & 0.266667  &  支障を生じる     & × \\ \hline
0.350000  &  阿波丸を撃沈させる & × & 0.265306  &  汗をかく         & ○ \\ \hline
0.333333  &  たばこを吸う     & △ & 0.264151  &  耳を傾ける       & ○ \\ \hline
0.333333  &  上告を棄却する   & × & 0.261398  &  役割を果たす     & △ \\ \hline
0.300000  &  身柄を拘束する   & △ & 0.260870  &  工夫をこらす     & △ \\ \hline
0.294118  &  立候補を届け出る & × & 0.260870  &  第一歩を踏み出す & △ \\ \hline
0.292308  &  国交を樹立する   & △ & 0.258065  &  平和条約を締結する & × \\ \hline
0.288462  &  幕を閉じる       & ○ &           &                     &    \\ \hline
    \end{tabular}
  \end{center}
  \caption{実験結果(相互情報量)}
\end{table}

\subsection{評価}

(A)動詞を固定した場合,(B)名詞を固定した場合,(C)相互情報量によって
抽出した場合の各々の実験結果を評価する.
評価は各々の手法で取り出した上位 50 種類の表現を
以下の3つに分類することによっておこなう.

\begin{description}
\item[分類1] 慣用表現になっているもの(○)
\end{description}

これは市販の慣用表現辞典\cite{inoue2}を参照して,その表現が見出しとして
記載されていれば,この分類とした.
例えば,「難色を示す」「圧力をかける」「一線を画す」
「姿を消す」「足並みをそろえる」などである.


\begin{description}
\item[分類2] 定型表現になっているもの(△)
\end{description}

これは主観的に強い共起性があると判定したものである.
例えば「役割を果たす」「影響を及ぼす」「目を向ける」
「注目を集める」「話題を呼ぶ」などである.
この分類ではコロケーションの関係になっているものが多い.
また,「カギを握る」「迷惑をかける」「決着をつける」
「誤解を招く」「輪を広げる」などのように慣用表現との
区別が微妙なものも多い.

\begin{description}
\item[分類3] 上記以外(×)
\end{description}

これは通常の表現だと思われるものである.
例えば,「武力行使を伴う」「首を絞める」
「仕事を休む」「けがをする」「被害を受ける」などである.
分類2とは微妙なものも多少ある.ここらの判定は主観である.


手法(A)(B)(C)各々の 50 種類の抽出結果を分類1,2,3 により分類すると
表4のような結果になる.また参考として,
先の実験結果(表1,2,3)に
分類の記号(分類1 ○,分類2 △,分類3 ×)を与えている.


\begin{table}[htbp]
  \begin{center}
    \begin{tabular}{|l||c|c|c|c|} \hline
          &  慣用表現(○)  & 定型表現(△)& その他(×)&  正解率 \\  \hline \hline
(A)動詞固定  &    13       &       20      &   17        &   66.0 \%  \\ \hline
(B)名詞固定  &    18       &       11      &   21        &   58.0 \%  \\ \hline
(C)相互情報量 &   15       &       15      &   20        &   60.0 \%  \\ \hline
    \end{tabular}
  \end{center}
  \caption{抽出の評価}
\end{table}


分類1,分類2 を正解と考えると,どの手法も正解率に大きな差はない.
しかし手法(A)(B)つまり
動詞を固定した場合と名詞を固定した場合の抽出結果の共通部分を調べてみると,
共通しているものは「端を発する」の1種類だけであった.
そこで,動詞を固定した場合と名詞を固定した場合の各々の 50種類の抽出結果
をマージした結果 99 種類を本手法の抽出結果と考え,
相互情報量の基準から上位 99 種類の表現を取り出し,
それらを比較した結果が表5である.


\begin{table}[htbp]
  \begin{center}
    \begin{tabular}{|l||c|c|c|c|} \hline
           &  慣用表現(○)  & 定型表現(△)& その他(×)&  正解率 \\  \hline \hline
本手法     &   30       &       31      &   38        &   61.6 \%  \\ \hline
相互情報量 &   23       &       31      &   45        &   44.4 \%  \\ \hline
    \end{tabular}
  \end{center}
  \caption{抽出の評価(抽出数2倍)}
\end{table}

相互情報量と比較すると本手法の正解率が高いことがわかる.
相互情報量の場合,抽出数が増えてゆくと正解率は下がる.
しかし本手法の場合は,抽出数半数の場合の正解率が保たれるため,
より広範囲の定型表現を取り出すことができる.


\section{考察}

  本論文の実験により,述語型定型表現には少なくとも2つの種類が
あることが分かる.それは動詞が名詞を引っ張っているものと
名詞が動詞を引っ張っているものである.慣用表現を機械処理する場合には,
その慣用表現の情報を構成語のどの単語の辞書情報に記述するかという
中心語の問題がある.
本手法はその1つの対処方法を示している.つまり引っ張っている方の
単語を中心語とすれば良い.この場合,
慣用表現のチェックが少なくてすむために,
解析の効率化も図れるはずである.

  99 種類を取り出した実験では,
本手法のみに現れる慣用表現は16種類,
相互情報量のみに現れる慣用表現は9種類,
共通して現れる慣用表現は14種類であった.
つまり相互情報量で抽出でき,本手法では抽出できない表現も
少なからず存在する.
これは述語型定型表現には本手法で設定した
「動詞が名詞を引っ張るもの」,「名詞が動詞を引っ張るもの」,
以外に「お互いが適度に引き合っているもの」が存在していると考えられる.
これらをうまく切り分けて抽出することを今後考えたい.

定型表現を取り出す場合,接続の割合が大きな鍵になっているが,
その上に頻度をどのように反映させるかも重要な問題である.
機械処理の効率化の観点だけから見れば,接続の割合よりも
頻度の方が重要な要素である\cite{kita2}.
しかし本手法では基本的に頻度の大きさは考慮していない.
頻度の大きさを基準値に
反映させるような基準値の設定方法をいくつか試みたが,
どの結果も頻度の大きさを考えないものよりも良い結果が得られなかった.
この点も今後の課題である.
一つの方法として,
データの信頼度のようなものを設定し\cite{tamoto},
総頻度が少ないものは信頼度が小さくなるようにし,
その信頼度を判断基準に取り込みことが考えられる.
当然,この場合も信頼度の取り込み方が本質的に問題だが,
その信頼度は抽出システムをツールとして捉えた場合に,
最終的に行なう人間の判断の際に,有効に利用できると予想している.

定型表現を機械翻訳に利用することを考えてみる.
この場合,強い共起性があっても,一般の表現と同じ規則によって
翻訳が可能であれば,その表現を抽出する意味はない.
このため機械翻訳をアプリケーションに設定しいているのなら,
対訳コーパスを用いて,一般の規則で翻訳困難な
表現を取り出すことが正当なアプローチだと思われる\cite{matsumoto}.
ただし対訳コーパスは入手の困難性,抽出手法の複雑性などから
やや現実性が低い.
単言語からの抽出であっても,
語義の違いまで推測できるようになれば,
機械翻訳で役立つ知識の抽出は可能であると予想しているので,
この点からの考察を今後深めたい.

最後に共起データついて述べる.本論文では
格助詞と動詞の間に副詞が入ったものと入らない場合とを区別なく抽出している.
通常,構成語の間に別の単語が挿入されていれば,
その表現を構成する単語間の共起の強さは弱いと考えられる.
このため共起データを収集する際に,副詞などの挿入が
おきるものに関してはマイナスのポイントを与えておき,
そのポイントを定型表現の判定に利用すべきであった.
またこのマイナスポイントは,
共起データを収集する際に取り除いた
以下のパターンの[A,B,E]にも与えることができる.
\begin{center}
名詞(A) + 格助詞(B) +  名詞(C) + 格助詞(D) + 動詞(E) + 句読点  
\end{center}
具体的にこのマイナスポイントをどのように利用したらよいかは
未解決だが,この点から今後の改良を行ないたい.


\section{おわりに}

本論文では述語型定型表現をコーパスから自動抽出することを目的に,
従来の相互情報量の条件を緩める方向で,
新たな名詞動詞間の共起性を測る基準を提案した.
概略,名詞,動詞のどちらかを固定して,
その単語と共起する集合内の単語にどの程度,特異な頻度になっているかの
数値を与える.この数値の上位のものを取り出すことで抽出を行う.

本手法の特徴は,名詞を固定した場合に抽出できる表現と動詞を固定した場合に
抽出できる表現にはほとんど共通のものがないが,
どちらも相互情報量による抽出程度の正解率はあるという点である.
このため本手法では,目的の抽出数の半数づつを各々の場合から
取り出せば良いために,同じ数を相互情報量を用いて
抽出する場合よりも高い正解率が得られる.

今後は動詞あるいは名詞が引っ張るタイプの定型表現の他に
両者が適度に引っ張り合うようなタイプの定型表現も抽出する
方法を考察したい.
また頻度,接続割合以外の特徴を考慮した抽出法も試みたい.


\acknowledgment

本実験の形態素解析の多くの部分で,
京都大学長尾研究室から配布された
日本語形態素解析システム JUMAN を利用させて頂きました.
JUMAN を作成された関係諸氏に感謝いたします.




\bibliographystyle{jnlpbbl}
\bibliography{jpaper}


\begin{biography}
\biotitle{略歴}
\bioauthor{新納 浩幸}{
1961年生.1985年東京工業大学理学部情報科
学科卒業.1987年同大学大学院理工学研究科情報
科学専攻修士課程終了.同年富士ゼロックス,翌年松
下電器を経て,1993年4月より茨城大学工学部シ
ステム工学科助手,現在に至る.自然言語処理の研究
に従事.情報処理学会,人工知能学会,ACL各会員.
}
\bioauthor{井佐原 均}{
1954年生.1978年京都大学工学部電気工学第
2学科卒業.1980年同大学大学院工学研究科電気
工学専攻修士課程終了.同年通商産業省電子技術総合
研究所入所.1995年より郵政省通信総合研究所関
西先端研究センター知的機能研究室長.京都大学博士(工学).
主たる研究テーマは,自然言語処理,知識
表現,機械翻訳など.情報処理学会,日本認知科学会,人工知能学会,
ACLなど会員.
}


\bioreceived{受付}
\biorevised{再受付}
\bioaccepted{採録}

\end{biography}

\end{document}

