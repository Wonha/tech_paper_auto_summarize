


\documentstyle[macros,theapa]{jnlp_e_b5_old}

\setcounter{page}{59}
\setcounter{巻数}{2}
\setcounter{号数}{2}

\受付{September}{26}{1994}
\再受付{December}{22}{1994}
\採録{January}{13}{1995}

\setcounter{secnumdepth}{2}

\title{}
\author{}
\jkeywords{}

\etitle{Integration of Morphological and Syntactic Analysis\\
  based on LR Parsing Algorithm}

\eauthor{Tanaka Hozumi\affiref{TIT} \and
  Tokunaga Takenobu\affiref{TIT} \and
  Aizawa Michio\affiref{Canon}}

\headauthor{Tanaka,~H.~et~al.}
\headtitle{Integration of Morphological and Syntactic Analysis}

\affilabel{TIT}
          {Department of Computer Science, Tokyo Institute of Technology}
          {Department of Computer Science, Tokyo Institute of Technology}
\affilabel{Canon}
          {Media Technology Laboratory, Canon Inc.}
          {Media Technology Laboratory, Canon Inc.}

\eabstract{
Morphological analysis of Japanese is very different from that of
English, because no spaces are placed between words. This is also the
case in many Asian languages such as Korean, Chinese, Thai and so
forth. In the Indo-European family, some languages such as German have
the same phenomena in forming complex noun phrases. Processing such
languages requires the identification of the word boundaries in the
first place. This process is often called {\em segmentation\/}.
Segmentation is a very important process, since the wrong segmentation
causes fatal errors in the later stages such as syntactic, semantic
and contextual analysis. However, correct segmentation is not always
possible only with morphological information. Syntactic, semantic and
contextual information are also necessary to resolve the ambiguities
in segmentation.  This paper proposes a method to integrate the
morphological and syntactic analysis based on LR parsing algorithm. An
LR table derived from grammar rules is modified on the basis of
connectabilities between two adjacent words. The modified LR table
reflects both the morphological and syntactic constraints.  Using the
LR table and the generalized LR parsing algorithm, efficient
morphological and syntactic analysis is available.
}

\ekeywords{Generalized LR parsing, Morphological analysis, Syntactic analysis}

\begin{document}

\maketitle

\section{Introduction}
\label{sec:introduction}

\hspace{-0.2mm}Morphological \hspace{-0.2mm}analysis \hspace{-0.2mm}of \hspace{-0.2mm}Japanese \hspace{-0.2mm}is \hspace{-0.2mm}very \hspace{-0.2mm}different \hspace{-0.2mm}from \hspace{-0.2mm}that \hspace{-0.2mm}of
\hspace{-0.2mm}English, \hspace{-0.2mm}because \hspace{-0.2mm}no \hspace{-0.2mm}spaces\\ are \hspace{-0.2mm}placed \hspace{-0.2mm}between \hspace{-0.2mm}words. \hspace{-0.2mm}This \hspace{-0.2mm}is \hspace{-0.2mm}also \hspace{-0.2mm}the
\hspace{-0.2mm}case \hspace{-0.2mm}in \hspace{-0.2mm}many \hspace{-0.2mm}Asian \hspace{-0.2mm}languages \hspace{-0.2mm}such \hspace{-0.2mm}as \hspace{-0.2mm}Korean, \hspace{-0.2mm}Chi- nese, Thai and so
forth. In the Indo-European family, some languages such as German have
the same phenomena in forming complex noun phrases. Processing such
languages requires the identification of the word boundaries in the
first place. This process is often called {\em segmentation\/}.
Segmentation is one of the most important tasks of morphological
analysis for these languages, since the wrong segmentation causes
fatal errors in the later stages such as syntactic, semantic and
contextual analysis. However, correct segmentation is not always
possible only with morphological information. Syntactic, semantic and
contextual information are also necessary to resolve the ambiguities
in segmentation.

Over the past decades a number of studies have been made on the
morphological and syntactic analysis of Japanese. From the viewpoint
of two kinds of integration, description of constraints and
processing, they would be classified into the following three
approaches. The relation of these approaches is illustrated in
Figure~\ref{fig:map}.

\begin{center}
  \input{fig:map}
  \figcap{Relation of approaches}{fig:map}
\end{center}

\begin{description}
\item[Cascade:]Separate the morphological and syntactic analysis and
  execute them in a cascade manner. The morphological and syntactic
  constraints are represented separately.

\item[Interleave:]Separate the morphological and syntactic analysis
  and execute them interleavingly. The morphological and syntactic
  constraints are represented separately.

\item[Single Framework:]Represent both the morphological and
  syntactic constraints in a single framework such as context free
  grammars (CFGs) and make no distinction between the two analysis.
\end{description}

Representing the morphological and syntactic constraints separately
as in the first two approaches, Cascade and Interleave, makes
maintaining and extending the constraints easier. This is an advantage
of these approaches. Many natural language processing systems have
used these two approaches. For example, Mine et~al. proposed a method
to represent the morphological constraints in regular grammar and the
syntactic constraints in CFG, and interleave the morphological and
syntactic analysis~\cite{mine:91:a}. Most other systems use a
connection matrix instead of regular
grammar~\cite{miyazaki:84:a,sugimura:88:b}. The main drawbacks of
these approaches would be summarized as follows:
\begin{itemize}
    \item It may require two different algorithms for each analysis.
    \item It must retain all ambiguities from the morphological analysis
    until the syntactic analysis begins. This wastes memory space and
    computing time.
\end{itemize}

On the other hand, from a viewpoint of processing, it is preferable to
integrate the morphological and syntactic analysis into a single
framework, since some syntactic constraints are useful for
morphological analysis and vice versa. The last approach fulfills this
requirement. There have been several attempts to develop CFG that
covers both the morphological and syntactic
constraints~\cite{kita:92:a,sano:92:a}. However, it is empirically
difficult to describe both constraints by using only CFG. In order to
have CFG rules include morphological constraints, nonterminal symbols
have to bear the morphological attributes which are used for checking
connectabilities between morphemes. In other words, nonterminals
should be more precisely subcategorized. This increases the number of
nonterminals and thus that of grammar rules.

\begin{center}
    \input{fig:cfg}
    \figcap{Connectability check by CFG}{fig:cfg}
\end{center}

Using augmented context free grammar (ACFG) instead of CFG may remedy
this problem. However, this may cause the delay of connectability
checking. For example, in Figure~\ref{fig:cfg}, in order to check the
connectability between adjacent words, $w_{i}$ and $w_{i+1}$, the
morphological attributes of each word should be propagated up to their
mother nodes B and C, and the check is delayed until the application
of the rule A $\to$ B C. 

By using connection matrices for morphological analysis as in the
Cascade/Interleave approaches, connectability checks between adjacent
words is performed very easily. Therefore, it is desirable to
represent the morphological and syntactic constraints separately as in
Cascade/Interleave, \hspace{0.5mm}and \hspace{0.25mm}to \hspace{0.25mm}integrate \hspace{0.25mm}the \hspace{0.25mm}execution \hspace{0.25mm}of \hspace{0.25mm}both \hspace{0.4mm}analysis
\hspace{0.25mm}into \hspace{0.3mm}a \hspace{0.25mm}single \hspace{0.25mm}process as \hspace{0.25mm}in \\Single Framework. Our method has captured
these advantages by representing the mor- phological constraints in
connection matrices and the syntactic constraints in CFGs, then
compiling both constraints into an LR table~\cite{aho:86:a}. The
already existing efficient LR parsing algorithms would be used with
minor modifications, enabling us to utilize both the morphological and
syntactic constraints at the same time. Our approach, called MSLR
(Morpho-Syntactic LR), locates at the right bottom of the map in
Figure~\ref{fig:map}, which is the most preferable position.


In the next section, we first give a brief introduction to Japanese
morphological analysis using an example sentence. In
section~\ref{sec:LR-table}, we describe the method of generating an LR
table from a connection matrix and CFG rules, then in
section~\ref{sec:algorithm} we explain the parsing algorithm. Our
algorithm is principally the same as Tomita's generalized LR parsing
algorithm~\cite{tomita:86:a}. The only difference is that the input is
not a sequence of preterminals, but a sequence of characters.

\section{Morphological analysis of Japanese}

A simple Japanese sentence consists of a sequence of postpositional
phrases (PPs) followed by a predicate. The PP consists of a noun
phrase (NP) followed by a postposition which indicates the case role
of the NP. The predicate consists of a verb or an adjective,
optionally followed by a sequence of auxiliary verbs~\cite{morioka:87:a}).

We illustrate the Japanese morphological analysis with an example
sentence ``かおるにあいます (meet Kaoru).'' We use a simple Japanese
dictionary shown in Figure~\ref{fig:dict}, and a connection matrix
shown in Figure~\ref{fig:matrix} which gives us the connectabilities
between adjacent morphological categories (mcat). For example in
Figure~\ref{fig:matrix}, the symbol ``{\tt o}'' at the intersection of
row~2 ($p_1$) and column~3 ($vs_k$) indicates that the morphological
category $vs_k$ can immediately follow the morphological category
$p_1$.

\begin{center}
  \def\arraystretch{}
  \input{fig:dict}
  \figcap{An example of Japanese dictionary}{fig:dict}
\end{center}

\begin{center}
  \def\arraystretch{}
  \input{fig:matrix}
  \figcap{An example of a connection matrix}{fig:matrix}
\end{center}
    
Using only the dictionary, we would obtain the following twelve
candidates of segmentation for the sentence ``かおるにあいます.''
\begin{center}
  \def\arraystretch{}
  \input{fig:result}
\end{center}
By \hspace{-0.2mm}also \hspace{-0.2mm}referring \hspace{-0.2mm}to \hspace{-0.2mm}the \hspace{-0.2mm}connection \hspace{-0.2mm}matrix, \hspace{-0.2mm}we \hspace{-0.2mm}would \hspace{-0.2mm}be \hspace{-0.2mm}able \hspace{-0.2mm}to \hspace{-0.2mm}filter
\hspace{-0.2mm}out \hspace{-0.2mm}illegal \hspace{-0.2mm}segmentations. From the examples above, we find (1)--(4)
violate the connectability between ``かお ($n_1$)'' and ``る 
($ve_r^3$)'', and that (5)--(8) violate the connectability between ``
る ($ve_r^3$)'' and ``に ($p_1$).''  Also (9) and (11) violate the
connectability between ``い ($ve_k^{2i}$)'' and ``ます ($ax_1$)'', and
(11) violates the connectability between ``あ ($vs_w$)'' and ``い 
($ve_k^{2i}$).'' Thus by eliminating the illegal candidates we obtain
the morphologically correct candidate, (12). A long input sentence
generally gives ambiguities which need to be resolved in later stages
using syntactic, semantic and contextual information.

\section{Generating LR table}
\label{sec:LR-table}

Connection matrices and CFG rules have been used for morphological
analysis and syntactic analysis respectively by most Japanese
processing systems. Because CFG rules were mainly used for syntactic
analysis and connection matrices for morphological analysis, they have
been developed independently of each other. 

In this section, we propose a method to integrate morphological and
syntactic constraints in the framework of LR parsing algorithm, and
thus capturing the advantages of Cascade/Interleave and Single
Framework described in section~\ref{sec:introduction}.

In order to combine connection matrices and CFG rules, the first step
we have to take is to extend the CFG rules by relating the syntactic
categories in the CFG rules with the morphological categories in a
connection matrix. This is realized by adding CFG rules called
morphological rules each of which is a unit production rule with a
syntactic category in the LHS and a morphological category in the RHS.

\begin{center}
  \def\arraystretch{}
  \input{fig:srules}
  \figcap{An example of CFG for Japanese}{fig:srules}
\end{center}

\begin{center}
  \def\arraystretch{}
  \input{fig:mrules}
  \figcap{A morphological rules derived from the dictionary
    in Figure~\ref{fig:dict}}{fig:mrules}
\end{center}

Let us take an exmaple CFG rules shown in Figure~\ref{fig:srules}.
From the dictionary shown in Figure~\ref{fig:dict}, we would extract a
set of new CFG rules as shown in Figure~\ref{fig:mrules}, which are
simply added to the CFG rules in Figure~\ref{fig:srules} to get an
extended set of CFG rules.

We would generate an LR table as shown in Figure~\ref{fig:LR-table} from
the extended CFG rules (1) through (21) from Figure~\ref{fig:srules}
and \ref{fig:mrules}.  Note that the extended CFG rules do not include
any information about connectability represented in the connection
matrix in Figure~\ref{fig:matrix}. For example, rules (4), (10) and
(14) allow the structure ``{\tt [v [vs $vs_r$],[ve [$ve_w^2$]]]}''
which violates the connectability between $vs_r$ and $ve_w^2$ with
respect to Figure~\ref{fig:matrix}.

\begin{center}
  \def\arraystretch{}
  \input{fig:LRtable}
  \figcap{LR table generated from rules (1)--(21)}{fig:LR-table}
\end{center}

The second step is to introduce the constraints on connectability into
the LR table by deleting illegal reduce actions. This is carried out
by modifying the LR table with the procedure shown in
Figure~\ref{fig:modify}.

\begin{center}
    \input{fig:modify}
    \figcap{A procedure to modify an LR table}{fig:modify}
\end{center}

Deleting reduce actions by applying the above procedure prohibits the
application of morphological rules which violates the connectability
between two adjacent words, namely the current scanned word and its
lookahead word. Note that given an LR table and a connection matrix,
this procedure can be performed automatically without human
intervention.

It is possible to incorporate this procedure into the LR table
generation process, however, it is better to keep them separate.
Since this procedure is applicable to any type of LR table, separating
this process from LR table generation enables us to use the already
existing LR table generation program.

For example, in Figure~\ref{fig:LR-table}, the reduce action {\tt r8}
in state~7 and column $ve_r^3$ is deleted, since the connection
between $vs_k$, the RHS of rule (8), and $ve_r^3$, the lookahead
symbol, is prohibited with respect to the connection matrix in
Figure~\ref{fig:matrix}. Similarly, reduce action {\tt r8} in state~7
and column $ve_w^2$ will be deleted and so forth. The reduce actions
which should be deleted is enclosed in Figure~\ref{fig:LR-table}.  The
overview of generating a modified LR table is shown in
Figure~\ref{fig:overview}.

Generally speaking, the size of the LR table is on the exponential
order of the number of rules in the grammar. Introducing the
morphological rules into the syntactic rules can cause an increase in
the number of states in the LR table, thereby exponentially increasing
the size of the overall LR table in the worst case. In our method, the
increase of the number of states is equal to that of the morphological
rules introduced. Suppose we add a morphological rule $X \to x$ to the
grammar. Only the items in the form of $[A \to \alpha \cdot X \beta]$
would produce a single new item $[X \to \cdot x]$ from which only a
single new state $\{[X \to x \cdot]\}$ would be created. Thus the
increase of the number of the states is equal to that of the
morphological rules introduced, and the size of the LR table will not
grow exponentially.

\begin{center}
  \unitlength=1mm
  \input{fig:overview}
  \figcap{Outline of generating a modified LR table}{fig:overview}
\end{center}

\section{Algorithm}
\label{sec:algorithm}

The LR parsing algorithm with the modified LR table is principally the
same as Tomita's generalized LR parsing algorithm. The only difference
is that Tomita's algorithm assumes a sequence of preterminals as an
input, while our algorithm assumes a sequence of characters. Thus the
dictionary reference process needs to be slightly modified.
Figure~\ref{fig:parse} illustrates the outline of our parsing
algorithm.

In Figure~\ref{fig:parse} the stage number (CS) indicates how many
characters have been processed. The procedure begins at stage 0 and
ends at stage N, the character length of an input sentence. In stage
0, the stack is initialized and only the node with state 0 exists
(step (1)). In the outer-most loop (2)--(14), each stack top in the
current stage is selected and processed.  In step (4), the dictionary
is consulted and look-ahead symbols are obtained. An important
point here is that look-ahead symbols may have different
character lengths.  A new node is introduced by a shift action at step
(8) and is placed into a stage which is ahead of the current stage by
the length of the look-ahead word.

\begin{center}
    \input{fig:parse}
    \figcap{Outline of the parsing algorithm}{fig:parse}
\end{center}

\section{A worked example}

The \hspace{0.3mm}following \hspace{0.3mm}example \hspace{0.3mm}well \hspace{0.3mm}illustrates \hspace{0.3mm}the \hspace{0.3mm}algorithm \hspace{0.3mm}in
\hspace{0.3mm}Figure~\ref{fig:parse}. \hspace{0.3mm}The \hspace{0.3mm}input \hspace{0.3mm}sentence \hspace{0.3mm}is ``かおるにあいます\$ ''
(meet Kaoru).  We assign position numbers between adjacent characters.
\begin{center}
    \begin{tabular}{l*{19}{c@{}}}
        Input:& &     か&&  お&& る&& に&& あ&& い&& ま&& す&& $\\
        Position:& 0&& 1&& 2&&  3&&  4&& 5&& 6&&  7&&  8&& 9\\
    \end{tabular}
\end{center}

In the following trace, the numbers in circles denote state numbers,
and the numbers in squares denote the subtree number. The symbols
enclosed by curly brackets denote a look ahead symbol followed by
the next applicable action, separated by a slash (/). The stage
numbers are shown below the stacks.

\fboxsep=0.5mm
\leftline{\bf Stage: 0}
\begin{quote}
  \begin{tabular}{ll}
    Dictionary reference:\\
    {[$n_1$, ``かお'']}&  at 0--2\\
    {[$vs_r$, ``かお'']}& at 0--2\\
    {[$n_1$,  ``かおる'']}& at 0--3\\
  \end{tabular}
\end{quote}
We find three look ahead symbols, $n_1$, $vs_r$, and $n_1$ by
consulting the dictionary in Figure~\ref{fig:dict}. A shift actions is
applied for each of them according to the LR table in
Figure~\ref{fig:LR-table}.
\begin{quote}
    \small
    \input{fig:trace1}
\end{quote}
After the shift actions, three new nodes are created at stage~2 or
stage~3 depending on the length of look ahead words. At the same time
subtrees \fbox{\tt 1}--\fbox{\tt 3} are constructed. The current stage
is updated from 0 to 2, since there is no node in stage~1. The look
ahead symbols are unknown at this moment.
\begin{quote}
    \small
    \input{fig:trace2}~
    \begin{tabular}[b]{l@{~:~}l}
      \fbox{\tt 1}& [$n_1$, ``かお'']\\
      \fbox{\tt 2}& [$vs_r$, ``かお'']\\
      \fbox{\tt 3}& [$n_1$, ``かおる'']\\
    \end{tabular}
\end{quote}

\leftline{\bf Stage: 2}
\begin{quote}
  \begin{tabular}{ll}
    Dictionary reference:\\
    {[$ve_r^3$, ``る'']}&  at 2--3\\
  \end{tabular}
\end{quote}
Dictionary reference gives one look ahead symbol from position~2. 
Since the current stage number is 2, only the first two stack tops are
concerned at this stage. No action is taken of the first stack,
because the LR table has no action in the entry for state~6 and a look
ahead symbol $ve_r^3$. As the result, the first stack is
rejected. The reduce action ({\tt r10}) is taken for the second stack.
\begin{quote}
    \small
    \input{fig:trace3}
\end{quote}
After {\tt r10}, a shift action ({\tt s23}) is carried out for the
first stack.
\begin{quote}
  \small
  \input{fig:trace4}~
  \begin{tabular}[b]{l@{~:~}l}
    \fbox{\tt 4}& [{\tt vs}, \fbox{\tt 2} ]\\
  \end{tabular}
\end{quote}
After {\tt s23}, we would proceed to stage~3.
\begin{quote}
  \small
  \input{fig:trace5}~
  \begin{tabular}[b]{l@{~:~}l}
    \fbox{\tt 5}& [$ve_r^3$, ``る'']\\
  \end{tabular}
\end{quote}

\leftline{\bf Stage: 3}
\begin{quote}
  \begin{tabular}{ll}
    Dictionary reference:\\
    {[$p_1$, ``に'']}&  at 3--4\\
  \end{tabular}
\end{quote}
We obtain symbol $p_1$ by consulting the dictionary. Because the
first stack can take no more action, it is rejected. The reduce action
({\tt r6}) is then applied to the second stack.
\begin{quote}
    \small
    \input{fig:trace6}
\end{quote}
The shift action ({\tt s25}) is applied to the following stack.
\begin{quote}
    \small
    \input{fig:trace7}~
    \begin{tabular}{l@{ : }l}
      \fbox{\tt 6}& [{\tt n}, \fbox{\tt 3} ]\\
    \end{tabular}
\end{quote}
After the shift action ({\tt s25}), new nodes are created in stage~4.
\begin{quote}
    \small
    \input{fig:trace8}~
    \begin{tabular}{l@{ : }l}
      \fbox{\tt 7}& [$p_1$, ``に'']\\
    \end{tabular}
\end{quote}

\leftline{\bf Stage: 4}
\begin{quote}
  \begin{tabular}{ll}
    Dictionary reference:\\
    {[$vs_k$, ``あ'']}& at 4--5\\
    {[$vs_w$, ``あ'']}& at 4--5\\
  \end{tabular}
\end{quote}
Dictionary reference provides two look ahead symbols for the next
word.
\begin{quote}
    \small
    \input{fig:trace9}
\end{quote}
After the two reduce actions ({\tt r7}), we get two nodes with the
same state~24, and they would be merged. This is possible because
these two reduce actions give the same structure as well. If the
structures are different, we would not able to merge the stacks. We
would see such an example later at stage~5.
\begin{quote}
  \small
  \input{fig:trace10}~
  \begin{tabular}[b]{l@{ : }l}
    \fbox{\tt 8}& [{\tt p}, \fbox{\tt7} ]\\
  \end{tabular}
\end{quote}

The process in stage 4 continues as follows.
\begin{quote}
  \small
  \input{fig:trace11}~
  \begin{tabular}[b]{l@{ : }l}
    \fbox{\tt 9}& {[pp, \fbox{\tt 6}, \fbox{\tt 8} ]}\\
  \end{tabular}
\end{quote}
\begin{quote}
    \small
    \input{fig:trace12}~
    \begin{tabular}[b]{l@{ : }l}
      \fbox{\tt 10}& [$vs_k$, ``あ'']\\
      \fbox{\tt 11}& [$vs_w$, ``あ'']\\
    \end{tabular}
\end{quote}

\clearpage
\leftline{\bf Stage: 5}
\begin{quote}
  \begin{tabular}{ll}
    Dictionary reference:\\
    {[$ve_k^{2i}$, ``い'']}& at 5--6\\
    {[$ve_w^2$, ``い'']}& at 5--6\\
  \end{tabular}
\end{quote}
We have two look ahead symbols for each stack top.  The reduce
actions ({\tt r8} and {\tt r9}) are performed.
\begin{quote}
    \small
    \input{fig:trace13}
\end{quote}
\begin{quote}
    \small
    \input{fig:trace14}~
    \begin{tabular}[b]{l@{ : }l}
      \fbox{\tt 12}& [{\tt vs}, \fbox{\tt 10} ]\\
      \fbox{\tt 13}& [{\tt vs}, \fbox{\tt 11} ]\\
    \end{tabular}
\end{quote}
Note that we are not able to merge the stack tops even with the same
state~4 since the structure of \fbox{\tt 12} and \fbox{\tt 13} are
different. If two stack tops are merged here and then different
shift actions ({\tt s16} and {\tt s18}) are carried out, we might
have invalid combinations of structure such as (\fbox{\tt 12},
\fbox{\tt 15}) and (\fbox{\tt 13}, \fbox{\tt 14}).
\begin{quote}
    \small
    \input{fig:trace15}~
    \begin{tabular}[b]{l@{ : }l}
      \fbox{\tt 14}& [$ve_k^{2i}$, ``い'']\\
      \fbox{\tt 15}& [$ve_w^2$, ``い'']\\
    \end{tabular}
\end{quote}
After the shift actions ({\tt s16} and {\tt s18}), we proceed to stage~6.

\leftline{\bf Stage: 6}
\begin{quote}
  \begin{tabular}{ll}
    Dictionary reference:\\
    {[$ax_1$, ``ます'']}& at 6--8\\
  \end{tabular}
\end{quote}
The process in stage~6 proceeds as follows.
\begin{quote}
    \small
    \input{fig:trace16}
\end{quote}
\begin{quote}
    \small
    \input{fig:trace17}~
    \begin{tabular}[b]{l@{~:~}l}
      \fbox{\tt 16}& [{\tt ve}, \fbox{\tt 15} ]\\
    \end{tabular}
\end{quote}
\begin{quote}
    \small
    \input{fig:trace18}~
    \begin{tabular}[b]{l@{~:~}l}
      \fbox{\tt 17}& [{\tt v}, \fbox{\tt 12}, \fbox{\tt 16} ]\\
    \end{tabular}
\end{quote}
\begin{quote}
    \small
    \input{fig:trace19}~
    \begin{tabular}[b]{l@{~:~}l}
      \fbox{\tt 18}& [$ax_1$, ``ます'']\\
    \end{tabular}
\end{quote}

\leftline{\bf Stage: 8}
\begin{quote}
  \begin{tabular}{ll}
    Dictionary reference:\\
    ``\$''& at 8--9\\
  \end{tabular}
\end{quote}
\begin{quote}
    \small
    \input{fig:trace20}
\end{quote}
\begin{quote}
    \small
    \input{fig:trace21}~
    \begin{tabular}[b]{l@{~:~}l}
      \fbox{\tt 19}& [{\tt ax}, \fbox{\tt 18}]\\
    \end{tabular}
\end{quote}
\begin{quote}
    \small
    \input{fig:trace22}~
    \begin{tabular}[b]{l@{~:~}l}
      \fbox{\tt 20}& [{\tt s}, \fbox{\tt 17}, \fbox{\tt 19} ]\\
    \end{tabular}
\end{quote}
The input sentence is automatically segmented and accepted, giving a
final parse result 21 as shown in Figure~\ref{fig:tree}.
\begin{quote}
    \small
    \input{fig:trace23}~
    \begin{tabular}[b]{l@{~:~}l}
      \fbox{\tt 21}& [{\tt s}, \fbox{\tt 9}, \fbox{\tt 20} ]\\
    \end{tabular}
\end{quote}
    
\begin{center}
  \unitlength=0.7mm
  \input{fig:tree}
  \figcap{An analysis of ``かおるにあいます''}{fig:tree}
\end{center}

\section{Conclusion}

We have proposed a method representing the morphological constraints
in connection matrices and the syntactic constraints in CFGs, then
compiling both constraints into an LR table. The compiled LR table
enables us to make use of the already existing, efficient generalized
LR parsing algorithms through which integration of both morphological
and syntactic analysis is obtained.

Advantages of our approach would be summarized as follows:
\begin{itemize}
    \item Morphological and syntactic constraints are represented
    separately, and it makes easier to maintain and extend them.

    \item The morphological and syntactic constraints are compiled into a
    uniform representation, an LR table. We can use the already
    existing efficient algorithms for generalized LR parsing for the
    analysis. 

    \item Both the morphological and syntactic constraints can be used
        at the same time during the analysis. 
\end{itemize}

We have implemented our method using the EDR dictionary with 300,000
words~\cite{edr:93:a} from which 437 morphological rules are derived. 
This means only 437 new states are introduced to LR table and this
does not cause an explosion in the size of the LR table.  The method
proposed in this paper is also applicable to integrate phonological
and syntactic analysis. The detail is described
elsewhere~\cite{tanaka:94:e}.


\bibliography{nlp-str,ref,LR}
\bibliographystyle{theapa}

\begin{biography}

\biotitle{}

\bioauthor{Tanaka Hozumi} {He is a professor of Department of Computer
  Science, Tokyo Institute of Technology. He received the B.S. degree
  in 1964 and the M.S. degree in 1966 from Tokyo Institute of
  Technology. In 1966 he joined in the Electro Technical Laboratories,
  Tsukuba. He received the Dr. Eng. degree in 1980. He joined in Tokyo
  Institute of Technology in 1983. He has been engaged in artificial
  intelligence and natural language processing research.  }

\bioauthor{Tokunaga Takenobu} {He is an associate professor of
  Department of Computer Science, Tokyo Institute of Technology. He
  received the B.S.  degree in 1983 from Tokyo Institute of
  Technology, the M.S. and the Dr. Eng. degrees from Tokyo Institute
  of Technology in 1985 and 1991, respectively. His current interests
  are natural language processing, information retrieval.}

\bioauthor{Aizawa Michio}{He is a researcher at Media Technology
  Laboratory, Canon Inc.  He received the B.S. degree in 1991 from
  Tokyo Institute of Technology, the M.S. degrees from Tokyo
  Institute of Technology in 1993. His current interest is natural
  language processing. }

\bioreceived{Received}
\biorevised{Revised}
\bioaccepted{Accepted}

\end{biography}

\end{document}

