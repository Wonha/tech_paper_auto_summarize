    \documentstyle[epsf,jnlpbbl,supertabular,ascmac]{jnlp_j_b5_2e}

\setcounter{page}{177}
\setcounter{巻数}{13}
\setcounter{号数}{3}
\setcounter{年}{2006}
\setcounter{月}{7}
\受付{2006}{2}{1}
\再受付{2006}{3}{27}
\採録{2006}{4}{4}

\setcounter{secnumdepth}{2}

\newcommand{\MARU}[1]{}

\newenvironment{ex}{}{}
\newenvironment{exB}{}{}



\title{日本語を援用した日本手話表記法の試み}
\author{松本 忠博\affiref{GUIS} \and 原田 大樹\affiref{GUIS}
\and 原 大介\affiref{AMUCN} \and 池田 尚志\affiref{GUIS}}

\headauthor{松本,原田,原,池田}
\headtitle{日本語を援用した日本手話表記法の試み}

\affilabel{GUIS}{岐阜大学工学部応用情報学科}
{Department of Information Science, Gifu University}
\affilabel{AMUCN}{愛知医科大学看護学部}
{College of Nursing, Aichi Medical University}

\jabstract{日本手話をテキストとして表現するための表記法を提案する.本表
  記法の検討に至った直接の動機は,日本語-日本手話機械翻訳を,音声言語間
  の機械翻訳と同様,日本語テキストから手話テキストへの翻訳(言語的な変
  換)と,翻訳結果の動作への変換(音声言語におけるテキスト音声合成と同
  様に手話動画の合成)とに分割し,翻訳の問題から動作合成の問題を切り離
  すことにある.この翻訳過程のモジュール化により,問題が過度に複雑化す
  るのを防ぐことをねらいとする.同時に,手話を書き取り,保存・伝達する
  手段としての利用も念頭に置いている.本表記法で記述される手話文は,手
  話単語,および,複合語等の単語の合成,句読点,非手指要素による文法標
  識で構成される.手話単語は,単語名とそれに付加する語形変化パラメータ
  (方向や位置,その他の手話動作によって付加される語彙的,文法的情報を
  表す)で表す.我々の表記法は,基本的に手話の動作そのものを詳細に記述
  するのではなく,動作によって表される意味内容を記述することをめざした.
  ただし,機械翻訳を念頭に置いているため,動作への変換のための便宜にも
  若干の考慮を払った.本表記法の記述力を検証するため,手話を第一言語と
  する手話話者による手話映像720文を解析し,この表記法での記述を試みた.
  全体で671文を記述することができた.十分表記できないと判断した49文
  (51表現)を分析し,問題点について考察した.}

\jkeywords{手話,機械翻訳,日本語,日本手話,表記法}

\etitle{A Japanese Gloss-based Written Notation\\
  for Japanese Sign Language}
\eauthor{Tadahiro Matsumoto\affiref{GUIS} \and
  Daiki Harada\affiref{GUIS} \and Daisuke Hara\affiref{AMUCN}
  \and Takashi Ikeda\affiref{GUIS}}

\eabstract{In this paper we propose a notation system for Japanese
  Sign Language (JSL).  This notation system is aimed to help
  modularize the Japanese-JSL machine translation process and to bring
  the JSL generation problem closer to that of traditional oral
  languages.  Accordingly, the main concern of this notation is not
  detailed motions of signs themselves but linguistic structures
  (i.e., lexical and grammatical information) expressed through such
  motions.  JSL sentences in our notation include signs, compounds of
  signs, punctuation marks, and non-manual syntactic markers.  A sign
  is represented by the sign identifier (a Japanese word or phrase)
  and its inflection parameters.  JSL sentences are transcribed in the
  text format with JIS characters.  This makes existing text tools
  available for reading, writing and processing JSL sentences.  We
  conducted a transcribing experiment to evaluate our notation system
  with 720 JSL sentences performed by native JSL signers, and found
  that 51 JSL expressions in the 49 sentences could not be
  sufficiently transcribed.  We classify and investigate those
  expressions.}

\ekeywords{Sign language, Machine translation, Japanese, JSL, Notation
  system}

\begin{document}
\thispagestyle{empty}
\maketitle


\section{はじめに}
近年,手話は自然言語であり,ろう者の第一言語である\cite{Yonekawa2002}と
いうことが認知されるようになってきた.しかし,これまで手話に関する工学
的研究は,手話動画像の合成や手話動作の認識といった画像面からの研究が中
心的であり,自然言語処理の立場からの研究はまだあまり多くは行われていな
い.言語処理的な研究が行われていない要因として,自然言語処理における処
理対象はテキストであるのに,手話には広く一般に受け入れられた文字による
表現(テキスト表現)がないことがあげられる.言語処理に利用できる機械的
に可読な大規模コーパスも手話にはまだ存在していないが,これもテキスト表
現が定まっていないためである.

本論文では,手話言語を音声言語と同様,テキストの形で扱えるようにするた
めの表記法を提案する.また,ろう者が表現した手話の映像を,提案した表記
法を使って書き取る実験により行った表記法の評価と問題点の分析について述
べる.

現在我々は,日中機械翻訳など音声言語間の機械翻訳と同じように,日本語テ
キストからこの表記法で書かれた手話テキストを出力する機械翻訳システムの
試作を行なっている.一般に翻訳は,ある言語のテキストを別の言語の等価な
テキストに置き換えることと定義されるが,手話にはテキスト表現がないため,
原言語のテキストから目的言語のテキストへの言語的な変換(翻訳)と,同一
言語内での表現の変換(音声言語ではテキスト音声合成,手話では動作合成/
画像合成)とを切り離して考えることができなかった.我々は,テキスト表現
の段階を置かずに直接手話画像を出力することは,広い範囲の日本語テキスト
を対象として処理していくことを考えると,機械翻訳の問題を複雑にし困難に
すると考え,音声言語の機械翻訳の場合と同じように,日本語テキストから手
話テキストへ,手話テキストから手話画像へと独立した二つのフェーズでの機
械翻訳を構想することとした(図\ref{fig:sltext}).

\begin{figure}[tb]
  \centering
  \epsfxsize=11cm
  \epsfbox{sltext.eps}
  \caption{日本語-手話機械翻訳における手話テキストの位置付け}
  \label{fig:sltext}
\end{figure}

現在,日本語から他の諸言語への翻訳を行うためのパターン変換型機械翻訳エ
ンジンjawとそれに基づく翻訳システムの開発が進められているが(謝他2004;
ト他2004;Nguyen et al. 2005;Thelijjagoda et al. 2004;マニンコウシン
他2004),本表記法を用いた日本語-手話翻訳システムも,それらと全く同じく
枠組みで試作が行なわれている(松本他2005; Matsumoto et al.2005).jawに
よる翻訳は次のように行われる.jawは入力日本語文を形態素/文節/構文解析し
て得られた日本語内部表現(文節係り受け構造,文節情報)の各部を,DBMS上
に登録された日本語構文パターンと照合する.パターンの要素には階層的な意
味カテゴリが指定できる.各パターンは,それを目的言語の表現に変換する翻
訳規則と対応しており,その規則の適用により目的言語の内部表現が生成され
る.目的言語の内部表現は,各形態素の情報を属性として持つC++オブジェクト
と,それらの間のリンク構造として実現される.目的言語の内部表現から目的
言語テキストへの変換(語順の決定,用言後接機能語の翻訳など)は,各形態
素オブジェクトが持つ線状化メンバ関数,および,目的言語ごとに用意された
別モジュールによって行われる.
\nocite{Bu2004,Shie2004,Nguyen2005,Thelijjagoda2004,Ngin2004}
\nocite{Matsumoto2005a,Matsumoto2005b}

本論文はこのような枠組みにおいて,翻訳システムが出力する手話のテキスト
表現方法について述べるものである.機械翻訳システム,すなわち翻訳の手法
については稿を改めて詳しく論じたい.

手話の表記法は従来からいくつか提案されている
\cite{Prillwitz2004,Sutton2002,Ichikawa2001,Honna1990}.
しかしその多くは,音声言語における発音記号のように,手話の動作そのもの
を書き取り,再現するための動作記述を目的としている.このため,言語的な
変換処理を,動作の詳細から分離するという目的には適していない.

\renewcommand{\thefootnote}{}
日本語から手話への機械翻訳の研究としては黒川らの研究があり
\cite{Fujishige1997,Ikeda2003,Kawano2004},日本語とほぼ同じ語順で
(日本語を話しながら)手指動作を行なう中間型手話\footnote{日本手話,
  日本語対応手話,中間型手話については次節で述べる.}を目的言語としたシステ
ムについて研究が行なわれている.そこでも手話の表記法についての提案があ
るが,機械翻訳の結果出力のためのシステムの内部表現としての面が強く,手
話をテキストとして書き取るための表記法というものではない.

徳田・奥村(1998)も日本語-手話機械翻訳の研究の中で,手話表記法を定義して
いる.しかし,主に日本語対応手話\footnotemark[1]を目的言語としているた
め,日本手話\footnotemark[1]において重要な言語情報を表す単語の語形変化
や非手指要素に対する表記は定義されていない.

テキスト表現を導入することによって,従来の音声言語間の機械翻訳と同じ枠
組みで手話への機械翻訳が行えるようになるが,その記述能力が不十分であれ
ば,逆に表記法が翻訳精度向上の隘路になる.機械翻訳を前提として提案され
た上述の既存表記法は,いずれも言語的に日本語に近い手話を対象としている
ため,日本手話を表記対象とした場合,記述能力不足が問題となる.

本論文で提案する表記法では,手話単語に対してそれを一意的に識別する名前
を付け,その手話単語名を基本として手話文を記述する.単語名としては日本
語の語句を援用する.手話辞典や手話学習書等でも,例えば[あなた 母 話
す]のように手話単語名を並べることによって手話文を書き表すことが多いが,
手話単語はその基本形(辞書形)から,手の位置や動きの方向・大小・強弱・
速さなどを変化させることによって,格関係や程度,様態,モダリティなどの
付加的な情報を表すことができる.また,顔の表情,頭の動きなどの非手指要
素にも文法的,語彙的な役割がある.したがって,これらの情報を排除した手
話単語名の並びだけでは,主語や目的語が不明確になったり,疑問文か平叙文
かが区別できなかったり,文の意味が曖昧になったりする.手話学習書等では,
写真やイラスト,説明文によってこのような情報が補われるが,本表記法では
これらの情報も,記号列としてテキストに含め手話文を記述する.基本的に動
作そのものではなく,その動作によって何が表されるかを記述する.たとえば,
「目を大きく開け,眉を上げ,頭を少し傾ける」といった情報ではなく,それ
によって表される疑問のムードという情報を記述する.ただし,手話テキスト
から手話動作記述への変換過程を考慮して,表記された内容が手話単語自体が
もつものなのか,あるいは,その単語の語形変化によって生じるものか,非手
指要素によるものかといった大まかな動作情報は表記に含める.

以下,2節で手話言語について述べ,3節で提案する表記法の定義を述べる.4
節で表記法の評価のための手話映像の書き取り実験と問題点の分析について述
べる.5節で既存の代表的な手話表記法について概観し,本論文の表記法との比
較を行う.

\section{手話について}

手話は手の形,位置,動き,そして,顔の表情や視線,頭の動きなどの非手指
要素といった複数のチャネルを使って意味を視覚的に伝達する言語である.話
し手の回りの空間上の位置を代名詞的に利用したり,手の動きの方向で格関係
を表すなど,音声言語にはない独自の文法を持つ.語彙についても,手話単語
が表す概念と日本語の単語が表す概念とは一般に一致しない.手話の述語はそ
の主体または対象,道具の情報と一体的に表現されることが比較的多い(「魚
が泳ぐ」「菓子を食べる」「ハサミで切る」など).このために,日本語では
同じ一つの述語で表される動作や性質が,手話では異なる述語(あるいは同じ
述語の異なる語形)で表現されることがある.逆に,日本語では別個の単語で
表現される事柄が,手話では同じ単語で表される場合も多い(手話単語〈暑い〉
は「夏」「南」「扇ぐ」「うちわ」などの意味でも使われる).また,手話に
は動詞名詞同形の単語が多く存在する.

手話は世界共通ではなく,国や地域によって異なる手話が使われてい
る\cite{Gordon2005}.日本で手話と呼ばれるものには「日本手話」と「日本語
対応手話」がある.この分類にはさまざまな考え方があり,人によってその定
義が異なるが,一般に,{\bf 日本手話}はろう者の間で生まれ広がった,日本
語とは別の体系を持つ手話言語をさし,{\bf 日本語対応手話}は手話単語を使
うものの,語彙や文法が日本語的な表現になっているものをさ
す\cite{Yonekawa2004}.日本語対応手話では日本語を話しながら手話が表現さ
れる場合が多いが,日本手話の場合,日本語とは異なる言語であるため,日本
語を話しながら手話を話そうとすると2つの言語を同時に話すことになり,日本
語につられて不自然な手話になったり,手話につられて不自然な日本語になっ
たりするという\cite{Yonekawa2004}.

日本語対応手話には,単に日本語の語順に沿って手話単語を並べただけのもの
から,手話的な表現を部分的に取り入れたもの({\bf 中間型手話}とも呼ばれ
る),逆に,日本語の助詞や助動詞に対する手指表現も人工的に与え,できる
限り日本語をそのまま手指動作で表そうとするものなど幅がある.中途失聴者
や健聴者にとって比較的習得が容易なため,手話サークルや手話講習会では日
本語対応手話を扱う場合が多い.その反面,日本語対応手話は日本語に堪能な
ろう者にとっても分かりづらいことが多いといわれ
る\cite{Akiyama2004,Ichida2005}.これは日本語と手話の単語の意味・用法の
違いに加え,手話で用いられる文法標識としての非手指要素が日本語対応手話
では欠落し,さらに日本語の助詞なども省かれることが多いため,個々の単語
は認識できても,文として理解しにくいものと考えられる.

本論文では,日本手話の記述に対応した手話表記法を提案する.

日本手話の基本語順はSOV(主語-目的語-動詞)といわれてい
る\cite{Kimura1995,Matsumoto2001}.ただし,話題化による語順の変化や主語
の省略が見られ,文末に,主語などを指す指差し(pronoun copy)が現れるこ
とも多い.

形容詞や副詞の語順については,修飾語が被修飾語に前置される場合と後置さ
れる場合が併存している.松本(2001)は,基本的には付随的な語が中心的な
語の後ろに置かれる,つまり修飾語が後置されるのが手話の自然な表現であり,
前置されるのは強調のための倒置であるが,現代では日本語の影響により,単
語によっては前置も一般化してきていると述べている.

一方,市田(1998)\nocite{Ichida1998}は名詞句内の語順は形容詞-名詞,関
係節-名詞,属格-名詞であり,形容詞や関係節が名詞に後置されているように
見える例は,主要部内在型関係節という構造を利用した表現である(前置の場
合とは非手指要素が異なる)と説明している.

数と単位の語順は,時刻や年齢などは「単位・数」の順で,金額や長さなどは
逆に「数・単位」の順で表される.日数($n$日間)では手の形が「数」を,そ
してその動きが「〜日間」という単位を表し,数と単位が同時に表現される.

このような同時性も,手話の特徴の一つである.音声言語では1度に1つの形
態素しか表すことができないが,手話では2本の手と非手指要素による複数の
チャネルを介して,並行して異なる形態素を組み合わせて表出することができ
る.

\renewcommand{\thefootnote}{}
\setcounter{footnote}{0}

また,手話による会話では,非手指要素が手指要素と同じように重要な役割を
持っていることが知られており\footnote{米川(2002)は,「お面をかぶって手
  話をした場合と,顔を出してグローブをはめて手話をした場合とでは,後者
  の方がよく伝わる」と述べている.},実際,ろう者は会話中,主に互いの
手ではなく顔を見ている\cite{Sutton-Spence1999}.

\section{手話表記法の提案}

ここでは本論文で提案する手話表記法について述べる.なお,表記法の詳細な
構文と表記例については付録Aに記載する.

\subsection{基本的な方針}

日本語-日本手話機械翻訳における言語的な変換の問題を,音声言語間の機械
翻訳と同様,動作合成(音声合成)の問題から切り離して扱えるようにするこ
とが表記法導入の大きな目的である.そのため,手話の動作そのものを詳細に
記述するのではなく,動作によって表される文の言語的な構造(語彙的・文法
的情報)の記述に重点を置いた表記法とする.具体的には次のような要素によっ
て手話文を記述する: 手話単語とその語形変化,複合語等の単語の合成,句
読点,および,非手指要素による文法標識.表記は汎用的で機械処理に適した
テキスト形式で行い,一般的な日本語環境で扱える範囲の文字だけを使用する.
これより,テキストエディタ等の既存のツールが手話用にそのまま流用できる
ことになる.テキストはフリーフォーマットとし,文中に空白や改行を自由に
挿入できるようにする.

\subsection{手話単語とその語形変化}
\label{sec:inflection}

手話単語は手の形,位置,動き,および,顔の表情などの非手指要素で構成さ
れる.単語の基本形(辞書形)から,これらの要素を変化させることにより,
意味を部分的に変えたり,付加することができる.

本表記法では,単語の基本形は手話単語名で表し,基本形からの変化がある要
素については,単語に対するパラメータとして,次のような形式で記述する.

\begin{ex}
  手話単語名[ 手形 ]( 空間;修飾 )
\end{ex}

手話単語名には手話単語の意味に近い日本語の語句を援用する(試作した翻訳
システムでは基本的に『日本語-手話辞典』\cite{JISLS1997}の手話イラス
ト名を手話単語名として用いた).ただし,手話単語とその単語名として使わ
れる日本語語句が表す概念とが全く同じであるとは限らないという点は注意し
なければならない.

語形変化は,手形要素,空間要素,修飾要素に分けて記述する.

\subsubsection*{(a) 手形要素}

手形の変化による語義の変化は,変化した手形を手形要素として記述すること
によって表現する.手話単語〈行く〉の手形変化による語義の変化(松本2001)
とその表記を図\ref{fig:iku}に示す.

\begin{figure}
  \centering
  \epsfxsize=7cm
  \epsfbox{iku123b.eps}
  \caption{手形変化の表記例.手話単語〈行く〉の基本形(左)とその手形
    変化(中・右)}
  \label{fig:iku}
\end{figure}

\subsubsection*{(b) 空間要素}

手話では話し手の回りの空間が人称と対応づけられており,話し手の位置が一
人称,聞き手の位置が二人称,その他の位置が三人称となっている(
図\ref{fig:personalLocation}).その場にいない人や物,場所が三人称の位
置で表現されるが,標準的には,人は話し手の斜め左右の位置に,物は話し手
の前方中央の位置に配置される(Baker-Shenk and Cokely 1980; 松本
2001)\nocite{Baker-Shenk1980}.
また,手話の動詞には,手の動きの方向や指先の向きによって格関係を表すも
のがある.動きが主語や目的語の人称(位置)や数に呼応して変化するため一
致動詞と呼ばれる(Sutton-Spence and Woll 1999; 市田1999).
\nocite{Ichida1999}

\begin{figure}
  \begin{minipage}[t]{.47\linewidth}
    \center
    \epsfxsize=5.5cm
    \epsfbox{personalLocations.eps}
    \caption{話し手の周りの空間上の位置と\\人称の対応}
    \label{fig:personalLocation}
  \end{minipage}
  \hfill
  \begin{minipage}[t]{.47\linewidth}
    \center
    \epsfxsize=5.5cm
    \epsfbox{look1221a.eps}
    \caption{一致動詞の方向変化とその表記}
    \label{fig:miru}
  \end{minipage}
\end{figure}

本表記法では格関係を表す一致動詞の方向や名詞の位置を,語形変化パラメー
タの空間要素として記述する.一人称と二人称の位置は位置定
数 ``1'' と ``2'' で表し,三人称の位置は ``3'', ``4'', ``$x$'',
``$y$'', ``$L$'', ``$R$'' などの位置変数\footnote{変数
  名 ``$L$'' と ``$R$'' はそれぞれ話者の左(Left)と右(Right)の位置を
  暗示している.},または,単語名で表す.単語名は,その単語が表現された
位置を示す.

図\ref{fig:miru}に動詞〈見る〉の方向変化とその表記を示す.〈見る〉の始
点は動作主体を,終点は対象を表している.また,次の表記例では,〈母〉の
位置と〈話す〉の始点が一致しており,「母が私に言う(=母から聞く)」と
いう意味の手話を表している.

\begin{ex}
  母($x$) 話す($x$→1)  
\end{ex}

\subsubsection*{数の一致}

動詞の動きはその主体や対象が単数か複数かによっても変化する.例えば,
「(みんなが私に)言う」という意味の手話は,動詞〈話す〉の始点を三人称の
位置の範囲で2, 3回変えて繰り返し表現される(
図\ref{fig:numberAgreement}a).このような動詞の複数変化は,位置の複数
形を用いることによって次のように記述する.

\begin{ex}
  話す(3s→1)
\end{ex}


漠然と複数の人(位置)を表すのではなく,
図\ref{fig:numberAgreement}(b)のように,具体的な数(とその人称位置)の
指定が必要な場合には,位置の集合を用いて次のように記述する.

\begin{ex}
  話す(\{$x,y$\}→1)
\end{ex}

一致動詞だけでなく,〈死ぬ〉のように方向を持たない動詞でも,表現位置を
変えられる場合は,位置を変えて繰り返し表現することで,主体が複数である
ことが表される.また,位置変化可能な名詞も,同様の表現方法によって,物
事が複数存在することを表すことができる.これらも,空間パラメータの「位
置」を複数形にすることによって,表記することができる.
\begin{figure}
  \centering
  \epsfxsize=5cm
  \epsfbox{numberAgreement.eps}
  \caption{動詞の複数変化}
  \label{fig:numberAgreement}
\end{figure}

\subsubsection*{(c) 修飾要素}

音声言語では,単語に対する修飾などの付加的な情報は,単語の前か後ろに一
次元的に追加される.手話にも独立した単語としての副詞や形容詞,助動詞に
相当する単語が存在するが,これとは別に,手話単語を表現する手の動きの速
さや大小,強弱,顔の表情などが副詞(様態・程度)や形容詞(高さ・大き
さ),アスペクト(起動・継続),モダリティ(勧誘)などを表す束縛形態素
となり,ベースとなる単語と同時に表現される場合がある.このような情報に
ついては,その語彙内容を修飾パラメータに記述する.この記述にも便宜上,
次のように日本語を援用する.

\begin{ex}
長い(;とても) & //「とても長い」 \\
\tt 木(;高い) &  //「高い木」
\end{ex}

\subsection{特殊な単語}

\subsubsection*{指差しの表記}

手話での指差しには,\MARU{1}話し手,聞き手,あるいは,手話単語(または
それが表現された空間上の位置)を指して代名詞や限定詞として使う用
法,\MARU{2}述語や文全体を指して,「〜するのは…」のように名詞化する形
式名詞的な用法,\MARU{3} 身体の一部を指して名詞として使う用法がある
\cite{Matsumoto2001,Kanda1996}.

\MARU{1}および\MARU{2}の用法での指差しは,``{\tt Pt}'' という手話単語名
で表し,指差しが指す位置は空間パラメータで{\tt Pt($x$)} のように指定す
る.手話単語〈私〉と〈あなた〉はそれぞれ{\tt Pt(1)} と{\tt Pt(2)}の別名
である.同様に〈それ〉は直前の単語への指差し,〈あれ〉は会話の場にいな
い第三者を表す左右遠方への指差しに対する別名である.\MARU{3}の用法につ
いてはその指差しによって表される名詞(「目」「耳」「鼻」など)を単語名
として記述する.

\subsubsection*{非利き手の指を代名詞的に用いた数詞の表記}

手話には1, 2, 3,...や,第1,第2,第3,...といった通常の数詞に加え,指を
代名詞的に使った手話特有の数詞があり,これらを松本(2001)は順序数詞,
順位数詞,限定数詞と呼んでいる.

順序数詞は複数の単語や文を順に列挙しながら,数え上げていく場合に用いら
れる数詞で,動作としては,握り拳をゼロとして出発し,順に指を立てたり,
立てた指を他方の手の人さし指で触れるといった表現になる.数え上げていく
過程で,それぞれの指に単語が対応づけられ,後にそれぞれの指が代名詞とし
て参照される場合もある.順序数詞は ``{\tt Enum}'' という手話単語名と手
形パラメータにより次のように記述する.

\begin{ex}
私 Enum[1],妹 Enum[2] & // 私と妹
\end{ex}

順位数詞は全体の数があらかじめ判っていて,そのうちの何番目かを指定する
数詞である.動作としては,全体の数に相当する本数の指を立て,指定する順
位の指を他方の手でつまんだり,指差して指定する.それぞれの指が代名詞と
して何を指すかは,立てた指全体が兄弟のように序列のある集合を表す場合は
自ずと明らかである(この場合,数を表現する手は横向きにして,上下になら
んだ指で代名詞間の上下関係を表現する)が,前述の順序数詞を表現すること
によって1つずつ定義される場合もある.順位数詞は ``{\tt Ref}'' という手
話単語名と手形パラメータにより次のように記述する.

\begin{ex}
Ref[1/3] & // 3人のうちの1番目
\end{ex}

限定数詞は,立てた指の内の何本かを倒すことにより,「いくつかの内のうち
いくつかについては」のように数を限定する働きを持つ.``{\tt Quant}'' と
いう手話単語名と手形パラメータにより次のように記述する.

\begin{ex}
Quant[2/4] & // 4人の内の2人  
\end{ex}

\subsection{複合語などの単語の合成}
\label{sec:compound}

複合語(単語の逐次的な合成)は次のように記述する.

\begin{ex}
  手話-サークル & // 手話サークル
\end{ex}


両手で異なる単語を表現した単語の同時的な合成は次のように,

\begin{ex}
  電話 \verb+|+ 仕事 & // 電話しながら仕事をする
\end{ex}

そして,1つの単語を表現した後,その一部を保持したまま,次の単語と同時に
表現する半同時的な合成は下のように表記する.

\begin{ex}
  家 / 帰る(→家) & // 家に帰る
\end{ex}


この例では,〈家〉を両手で表現した後,片手をそのまま残し,他方の手で
〈帰る〉を表現する.〈帰る〉の動きの終点は,その目的地を表す.

ただし,一般的な複合語については一つの単語名(別名)で表すことも許す.
例えば,「病院」は「脈{\tt -}ビル」の別名である.

\subsection{句読点}
\label{sec:punctuation}

文末は ``。\unskip'' で表す.ただし,疑問文の文末は ``?'' で表す.単語の並び
が同じでも,平叙文と疑問文では顔の表情などの非手指要素が異なっており,
実際の手話表現ではそれらを区別することができる.表記上,その違いをこれ
らの文末記号で表す.

節や句などの構文的な切れ目は ``{\tt ,}'' または ``{\tt ;}'' で表す.こ
れらは動作的には頷きや時間的な間,瞬きなどで表現される.

動詞の後ろに置かれ,モダリティ等を表す助動詞に相当する単語が手話にも存
在するが,手話の助動詞は,少数の例外を除き,動詞または形容詞としての用
法をもつ\cite{Ichida2000,Matsumoto2001}.これらは述語として用いられると
きと,助動詞として用いられるときとで表現に違いが表れる\cite{Kimura1995}.
そのため,助動詞として用いられる場合には,``\verb+~+'' を前置して助動
詞的用法であることを明示する.

\subsection{非手指要素}
\label{sec:nms}

ここでは非手指要素による文法的な標識の表記について述べる.

\subsubsection*{非手指文法標識}

木村・市田(1995)は話題化,平叙文,yes-no疑問文,wh疑問文,条件節など
の標識となる非手指要素について述べている.表\ref{tab:markers}にその例を
示す\footnote{現実の手話表現では,個人差やそのときの状況,話者の感情状
  態,ニュアンスの違いなどによって動作に変化があるものと考えられる.}.

前述の句読点も非手指要素による文法標識を表す記号であるが,その他に次の
ような形式で非手指要素を表記する.

\begin{ex}
  \{$<${\it NMS}$>$ 単語列\}
\end{ex}

これは,「単語列」に$<${\it NMS}$>$で示される非手指要素が伴うことを表す.
ただし,$<${\it NMS}$>$ 部には,基本的に非手指要素の動作そのものではな
く,それによって表現される機能を記述する.例)話題化:$<${\tt t}$>$,条
件節:$<${\tt cond}$>$,同意を求める文:$<${\tt conf}$>$,強
調:$<${\tt em}$>$.

\begin{table}[tb]
  \centering
  \caption{非手指要素による文法標識の例.
    条件節の動作説明は米川(2005b)\nocite{Yonekawa2005b}から,その他は
    \\木村・市田(1995) から抜粋}
  \begin{tabular}{|l|p{9.6cm}|}
    \hline
    \multicolumn{1}{|c|}{表現内容} &
    \multicolumn{1}{c|}{一般的な動作} \\ \hline\hline
    平叙文 & 文が終わったところ,または,最後の単語で頷く.\\
    \hline
    yes-no疑問文 &
    \begin{tabular}{@{}p{9.6cm}@{}}
      視線が聞き手に向かう.眉を上げ,最後の単語でうなず
    くか,あごを引いたまま答えを待つ.文末の単語は手の動きが終わった状
    態でしばらく保持される.
    \end{tabular}\\
    \hline
    wh疑問文 (疑問詞疑問文) &
    \begin{tabular}{@{}p{9.6cm}@{}}
      文末の単語が,相手の答えを待つように,手
      の動きが終わった状態でしばらく保持されるか,小刻みな動きが繰り返さ
      れる.眉を上げるか下げるかし,あごを前方か斜め前方に突き出すように
      し,さらにあごを左右に小刻みにふったりする.
    \end{tabular}
    \\
    \hline
    同意を求める文 &
    \begin{tabular}{@{}p{9.6cm}@{}}
      文末のうなずきに,疑問文の特徴である文末の単語と
      表情の保持と眉上げが加わる.
    \end{tabular}
    \\
    \hline
    命令文 &
    \begin{tabular}{@{}p{9.6cm}@{}}
      文全体または文末の指差しの直前の単語であごを上げる.命令
      の強さは表情で示す.
    \end{tabular}
    \\
    \hline
    修飾関係 &
    \begin{tabular}{@{}p{9.6cm}@{}}
      名詞の並びが修飾関係なら,うなずきがなく連続的に表現される.
    \end{tabular}
    \\
    \hline
    並列関係 & 名詞の並びが並列関係なら,名詞ごとにうなずきがある.\\
    \hline
    話題化・焦点化 &
    \begin{tabular}{@{}p{9.6cm}@{}}
      (話題化する語句を文頭に移動して)文頭で眉を上げ,
      語句の終わりで顎を引く.
    \end{tabular}
    \\
    \hline
    修辞疑問文 (wh分裂文) & 疑問詞の位置まではyes-no疑問文と同じで,疑
    問詞の直後で元に戻す. \\
    \hline
    できごと・行動の順序 &
    \begin{tabular}{@{}p{9.6cm}@{}}
      できごと・行動の順序どおりに,2つの文をつな
      げる場合,最初の文の動詞を少しの時間そのまま保ち,うなずいてから,
      次の文に進む.
    \end{tabular}
  \\
    \hline
    理由を述べる従属節 &
    \begin{tabular}{@{}p{9.6cm}@{}}
      述語が形容詞の場合,述語を少しの時間そのまま
      保ち,うなずいてから次の文に進む.述語が動詞の場合はまゆを上げるか
      下げるという動作が加わる.
    \end{tabular}
    \\
    \hline
    原因・目的 &
    \begin{tabular}{@{}p{9.6cm}@{}}
      原因を表す語でうなずく.原因を表す語の後に〈ため〉が
      続く場合は,〈ため〉の部分でうなずく.
    \end{tabular}
    \\
    \hline
    条件節 &
    \begin{tabular}{@{}p{9.6cm}@{}}
      条件節では眉を上げ,頭と体を少し前に傾ける.続く主節が平
      叙文の場合,間を置いてから,まゆを下げ,頭と体を元の位置に戻す.
    \end{tabular}
    \\
    \hline
  \end{tabular}
  \label{tab:markers}
\end{table}

\subsubsection*{発言・行動の引用}

他者や過去の自分の言動が直接話法的に引用される場合,引用部分では,その
言動を行なった人物が配置された位置に応じて(その人物の役を演じるように)
上体が少しシフトしたり,現実の聞き手に向かっていた視線が,引用内での聞
き手へ移るなどの非手指要素が文法標識となることがある.非手指要素だけで
なく,「あなた」を意味する聞き手への指差しや一致動詞の方向も,現実の聞
き手ではなく,引用される文中の聞き手に向かう.例えば,「彼が私に『君が
好きだ』と言う」という意味の手話において,
図\ref{fig:referentialShift}のように〈彼〉が話者の右前方に配置されると,
引用部分では上体が少し左を向き,『君』を表す指差しも左前方を指
す\cite{Yonekawa2005a}.このような引用部での非手指要素と手指動作の変化
を次のように記述する.

    \begin{exB}
彼(R) 言う(R→1),\{$<$rs(R)$>$  あなた 好き\}。  
    \end{exB}


\begin{figure}
  \centering
  \epsfxsize=7.5cm
  \epsfbox{referentialShift.eps}
  \caption{手形変化の表記例.手話単語〈行く〉の基本形(右)と}
  \label{fig:referentialShift}
\end{figure}

\section{手話映像の書き取り実験}

本手話表記法の記述力を検証するため,ネイティブの手話話者が表現した手話
映像を本表記法で書き取る実験を行った.対象としたのは,手話学習者向けビ
デオ教材「手話ジャーナル」のうち
の2巻 \cite{SignFactory1997,SignFactory1999}に含まれる720文である.ビデ
オにはそれぞれ4人のろう者が,家族・仕事・食事など日常の話題について日本
手話で話している様子が撮影されている.

\subsection{実験方法}

ビデオには手話に対する自然な日本語訳のほか,手話文の構造に即して訳され
た構造訳が付属している.その構造訳を参考にしながら,手話映像を解析し,
手話単語名とその語形変化,非手指文法標識等を,前節で定義した表記法で書
き取った.手話単語名は『日本語-手話辞典』(日本手話研究所1997)のイラスト
名を基本とし,他の手話辞書の見出しも参考にした.

\subsection{結果と考察}

実験の結果,720文のうち,本表記法で記述できたものは671文(約93\%)であっ
た.表記例とその日本語訳を表\ref{tab:examples}に示す.

\begin{table}[tb]
  \centering
  \caption{手話表記例.本表記法で記述した手話文(上段)とその日本語訳
    (下段).日本語訳の\\括弧内は構造訳を表す.}
  \label{tab:examples}
  \begin{tabular}{|l|p{12cm}|}
    \hline
    \raisebox{-1zw}{例1} &
      \tt\{$<$t$>$ 私 家族\}, 私 Enum[1], 兄 Enum[2], 両親 
      Enum[\{3,4\}]。\\
      \noalign{\vskip-.5zw}
      \cline{2-2}
    & 私の家族は,私と兄と両親の4人です。
    \\ \hline
    
    \raisebox{-1.5zw}{例2}
    & \tt\{$<$t$>$ Ref[\{1,2\}/4]\} 両親; \{$<$t$>$ Ref[3/4]\} 私;
    \{$<$t$>$ Ref[4/4]\} 妹。\\
    \noalign{\vskip-1zw}
    \cline{2-2}

    &
    \begin{tabular}{l}
      両親と私と妹です。\\
      (1番目と2番目は両親,3番目は私,4番目は妹です。)
    \end{tabular}
    \\
    \hline

    \raisebox{-2zw}{例3}
    & \tt\{$<$t$>$ 私, 父(R), 母(R), 3/人(R)\} 聾唖; \{$<$t$>$
    祖父(L), 祖母(L), 2/人(L)\} 健聴。\\
    \noalign{\vskip-.5zw}
    \cline{2-2}
    &
    \begin{tabular}{l}
      私と父,母はろう者,祖父と祖母は聴者です。\\
      (私と父,母の3人はろう者,祖父と祖母の2人は聴者です。)
    \end{tabular}
    \\
    \hline

    \raisebox{-1.5zw}{例4}
    & \tt テレビ 観る(→L), 妻(R) Pt(R) 一緒, 会話(1→R) 見る(1→
    L) 食べる; 6 50, 家 出る。\\
    \cline{2-2}
    & テレビを見ながら,また,妻と話をしながら食べて,6時50分に家を出ます。
    \\
    \hline

    \raisebox{-1zw}{例5}
    & \tt\{$<$cond$>$ 仕事 終わる 以降,必要 ない\},はや
    い 帰る。\\
    \noalign{\vskip-.5zw}
    \cline{2-2}
    & 仕事が終わって,用事がない場合は,早く帰ります。\\
    \hline

    \raisebox{-1zw}{例6}
    & \tt\{$<$em$>$ 今 いいえ\},過去(;とても) 古い(;とても) 本 読む 好き。
    \\
    \noalign{\vskip-.5zw}
    \cline{2-2}
    & 最近のものではなく,昔の古い本を読むのが好きです。\\
    \hline

    \raisebox{-1zw}{例7}
    & \tt\{$<$t$>$  意味\},日本-手話 教える,行く[男](1→3s) たくさん。
    \\
    \noalign{\vskip-.5zw}
    \cline{2-2}
    & どうしてかというと,手話を教えにあちこち行くことが多いからです。\\
    \hline
    
    \raisebox{-1zw}{例8}
    & \tt\{$<$t$>$ 姉\} 過去 結婚 終わる。\\
    \noalign{\vskip-.5zw}
    \cline{2-2}
    & 姉は既に結婚しています。(姉はもう以前に結婚
    しました。)\\
    \hline

    \raisebox{-2zw}{例9}
    & \tt\{$<$t$>$ 今\} \{$<$t$>$ 友達\} いじめる($x$→1)
    \{$<$rs($x$)$>$ あなた 
     手話 教える 趣味\} 話す($x$→1)。\\
    \noalign{\vskip-.5zw}
    \cline{2-2}
    & 最近友達からは,おまえ手話を教えるのが趣味なんだろうとからかわれます。
    (いま友達がからかって,おまえ手話を教えるのが趣味なんだろう,と言いま
    す。)\\
    \hline

    \raisebox{-1zw}{例10}
    & \tt 去年,3|月,$n$年[20]-目 とき 死ぬ。\\
    \noalign{\vskip-.5zw}
    \cline{2-2}
    & 去年3月に20年目で死んでしまいました。\\ \hline
  \end{tabular}
\end{table}

残りの49文(51表現)については,本表記法では十分表記できないと判断し
た.これら51表現の分類と表現例を表\ref{tab:problems}に示す.

\begin{table}[tb]
  \centering
  \caption{十分表記できなかった手話表現の分類とその例}
  \label{tab:problems}
  \begin{tabular}{|l|p{4.4cm}|p{4cm}|l|r|}
    \hline
    & \multicolumn{1}{c|}{分\hskip2zw 類} & \multicolumn{1}{c|}{手話動作例}
    & \multicolumn{1}{c|}{日本語訳} & \multicolumn{1}{c|}{数} \\
    \hline\hline
    1 & 語句をパントマイム的に説明 &
    \begin{tabular}{@{}p{4cm}@{}}
      2本の柱と,その間にたわんで掛かる線を描写
    \end{tabular}
    & 電線 & 11 \\ \hline
    2 & 実際の動作・反応を再現 & 「あっ」と驚いた表情 & 気がつく & 10
    \\ \hline
    3 & 大きさ・高さの実寸を手で示す & 両手で楕円を形作る & これくらい &
    3 \\ \hline
    4 &
    \begin{tabular}{@{}p{4.4cm}@{}}
      先に表現された単語の部分や相対的な位置の指定
    \end{tabular}
    &
    \begin{tabular}{@{}p{4cm}@{}}
      〈日本〉における山口県の地理的な位置を指示
    \end{tabular}
    & 西日本のこの辺り & 8 \\ \hline
    5 &
    \begin{tabular}{@{}p{4.4cm}@{}}
      手話単語(手形)を実際の動きや位置関係に即して表現
    \end{tabular}
    &
    \begin{tabular}{@{}p{4cm}@{}}
      両手で2つの〈座る〉を向かい合わせに表し,話者の右側に配置
    \end{tabular}
    &
    隣の席に向かい合って座る & 13 \\ \hline
    6 & 数量変化や時間経過の表現 &
    \begin{tabular}{@{}p{4cm}@{}}
      〈お金〉を大きく上下させながら横に移動
    \end{tabular}
    & (ボーナスが)乱高下する & 4 \\ \hline
    7 &
    \begin{tabular}{@{}p{4.4cm}@{}}
      単語を連続的に組み合わせた複雑な表現
    \end{tabular}
    & 省略(本文中に記載)&
    \multicolumn{1}{c|}{\rule{5zw}{.3pt}}&2
    \\ \hline
  \end{tabular}
\end{table}

分類1は,対応する手話単語が存在しないか,一般的でないために,語句をパ
ントマイム的な身振りで説明したり,視覚的に分りやすく補足する表現である.
「電線」,「スカッシュ」,「キャッチボール」,「給与明細」,「腰の曲がっ
たおばあさん」などの表現が見られた.本表記法では手話単語を基本に手話文
を表記するため,単語化されていない自由な動作で表現された手話文を表記す
ることができなかった.これらの表現を記述するためには,語句を説明してい
る一連の身振り(あるいは,それを構成する個々の身振り)に単語名を定義す
る必要がある.

分類2は,「疑うような目で見る」,「どきっとする」などの表情や動作をそ
のまま再現した表現で,分類1と同様,単語化されていない表現のため,表記で
きなかった.

分類3は,飼っていた亀やネコの大きさとその変化,缶ビールのサイズなどを
「これぐらい」と手で実物の大きさを示す表現である.「大きい」「小さい」
といった抽象的な情報ではなく,視覚的に表された具体的な寸法の情報をテキ
ストとして表記するのは難しく,今のところどのように表記するか定義できて
いない.

分類4は,「〈ビル〉の1階」「〈道路〉の両側」のように,単語で表現され
た物の一部やそれを基準にした相対的な位置を指差しなどで示す表現である.
このような表現に対する表記も今のところ定義しておらず,表記できなかっ
た.

分類5は,手話単語を現実世界の動きや位置関係に合わせて自由に動かす表現
である\footnote{これらの動詞は空間動詞
  (spatial verb),あるいは,類辞述語
  (classifier predicate)などと呼ばれる\cite{Sutton-Spence1999}.}.
左右の手で表した〈座る〉を向か
い合わせに配置して「向かい合わせに座る」や,〈男〉を倒して「息子が寝
る」,〈男〉を前方に傾けた手形(動物の動き表すときに用いられる)を素早
くランダムに動かして「(猫が)部屋中を荒らし回る」,などの表現が見られ
た.これらを表記するには,1) それぞれに単語名を与え,独立した単語として
扱う,あるいは,2) 例えば「向かい合う」という単語を定義し,その手形とし
て〈座る〉を指定することで「向かい合わせに座る」を表記するか,3)〈座る〉
の修飾パラメータとして「向かい合って」を指定する,といった方法が考えら
れる.ただし,「並木の間を歩いていって右に曲がる」という意味の手話文は,
〈道路〉と〈木〉で表現された並木道に沿って〈歩く〉を動かし,途中で歩く
方向を右に変えることによって表現されていた.このように,動きや配置が多
様な上,他の単語との位置関係が重要な表現(分類4)の表記は困難である.

分類6は,空間上に単語をプロットしてグラフを描くようにして金額や頻度の
変動を表したり,時計の針の動きで時間の経過を表す表現である.これは分類
5の一種と考えることもできる.グラフ的に表される変化の様子(手の動き)
は,「増加」「減少」「一定」「乱高下」「急落」などある程度限られるため,
それぞれ単語として定義し,何が変化するかをその手形変化として表記すると
いう方法が考えられる.

分類7は,単語を使っているが,その組合せ方が複雑で,表記しきれない手話
表現である.そのうちの1つは,「祖母には後継ぎがなく,祖父の面倒を見て,
その祖父が亡くなり,…」という意味の文であった.図\ref{fig:couple}に示
すように,手話単語〈結婚〉は左右の手で〈男〉と〈女〉を表現し,それらを
寄り添わせる動作によって表される.そしてこの寄り添った状態は〈夫婦〉を
表す.この文の表現ではまず,〈夫婦〉を目の高さに近い位置で表現すること
で,目上の夫婦を表現し,眉を上げながら〈女〉を表す手を小刻みに振ること
で,祖母についての話であることが示される.次に,〈女〉を表していた手で,
〈生まれる〉〈ない〉を表し(祖母には後継ぎがない),〈男〉に向かって
〈助ける〉を表現する(祖父の面倒を見る).再び〈夫婦〉を表してから,今
度は〈男〉の手で〈死ぬ〉を表現する(祖父が亡くなる).手話単語だけを使っ
た表現だが,現状では〈夫婦〉の構成要素である〈女〉側の手で単語を表現す
るという指定ができない,また,非手指文法標識のスコープを表すブロックと
(半)同時的な合成を表すブロックがオーバーラップしてしまい中括弧の対応
が曖昧になるなどの問題があり,表記することができなかった.これらに対す
る表記方法を検討する必要がある.
\begin{figure}
  \centering
  \epsfxsize=3.5cm
  \epsfbox{couple.eps}
  \caption{多形態素からなる手話単語〈結婚〉}
  \label{fig:couple}
\end{figure}

以上,書き取りきれなかった手話表現について述べたが,その他に今後検討す
べき点として,同形異義語を区別する口型(唇の動き)の表記への反映,およ
び,複数の文から成る談話を表記対象としたときの位置変数の有効範囲指定が
挙げられる.また,本表記法で書かれた手話文から動作記述を合成する過程で
は,語形変化パラメータの修飾要素の処理が大きな問題となることが予想され
る.

\section{関連研究}

日本語などの音声言語が,音声だけでなく文字による表現を持つことの重要性
を考えれば,手話を音声言語に訳さず,手話言語のままテキストとして扱える
ことは,手話の使用者(手話研究者や学習者を含む)にとっても有用であると
考えられる.このため,従来から目的に応じていくつかの表記法が考案されて
きた.その多くは,音声言語における発音記号のように,手話の動作そのもの
を書き取り,再現するのに適した表記法である.

HamNoSys (Hamburg Sign Language Notation System)は国際音声記号のように,
特定の国の手話に依存しないことを目指した表記法であ
る\cite{Prillwitz2004}.手話単語を構成する手の形や位置,動き,掌の向き,
非手指要素といった個々の要素を約200種類の単純な図形記号(文字)で表し,
それらを一定の順序で一列に並べることによって一つの手話単語の動作を表記
する.例えば,図\ref{fig:bear}(a)に示すアメリカ手話で熊を表す手話単語は,
同図(b)のような記号列により表現される.直感的には分りにくいが,研究用途
での使用を想定しており,動作の詳細な記述が可能となっている.
\begin{figure}
  \centering
  \epsfxsize=10cm
  \epsfbox{bears_in_asl.eps}
  \caption{ASLの ``熊'' を表す手話単語の表記:(a) イラストによる描写
    \cite{Fant1994},\\(b) HamNoSysによる表記\cite{Bentele1999},(c)
    SignWritingによる表記\\\cite{Sutton2006}}
  \label{fig:bear}
\end{figure}

SignWritingはダンスの振り付け表記法 (DanceWriting) をもと
に,1974年Suttonによって考案された手話表記法であ
る\cite{Sutton2002}.HamNoSysと同様,基本となる記号(International
Movement Writing Alphabet, IMWA)を組み合わせて手話単語を表現するが,基
本記号を一列に並べるのではなく,図\ref{fig:bear}(c)のように,2次元的に
配置することにより,直感的に分りやすい表記になっている.基本記号は手の
形,動き,顔など8つカテゴリ,約450種類が定義されている.HamNoSysとは対
照的に,手紙や新聞,文学,教育など,主に日常生活で使用されることを想定
している.SignWritingを日本手話用に拡張する研究も行なわれてい
る\cite{Honna1990} .

sIGNDEX\cite{Ichikawa2001,SILE2005}は手話単語をローマ字表記の日本語ラベ
ルで表し,同時表現や非手指要素を表す記号を付加して手話文を表記する.個々
の単語における詳しい手指動作についてはビデオ画像によって別途与えている
(sIGNDEX V.1 の動画語彙数は545語).眉の上げ下げ(eBU, eBD),目の開閉
(eYO, eYS),口角の動き(cLD,cLP)など,目に見える動作を現象的に捉え,
記号化することを基本としている.図\ref{fig:signdex}にsIGNDEXによる手話
表記例を示す.


以上は,手話の動作そのものを書き取ることを目的とした表記法であった.一
方,徳田・奥村(1998)\nocite{Tokuda1998}は,計算機上で手話を自然言語と
して処理することを目的とした表記法を提案している(図\ref{fig:tokuda}).
手話単語には手話単語辞書に登録された日本語見出しを使用し,指文字表記や,
代名詞に対する働きかけを表す動作の表現(左手で代名詞,右手で動詞),単
語の繰り返しなどの表記を定義した.しかし,基本的に日本語対応手話を表記
対象としているため,非手指要素や語形変化を表記する方法については定義さ
れていない.

\begin{figure}[tb]
  \centering
  \begin{minipage}{.65\linewidth}
    \begin{screen}[4]
      \small
      \setlength{\baselineskip}{12pt}
\begin{verbatim}
    pT2dOCHIRA+@eBU+@eYO+hDN+mOS-DOCCHIkOOCHA+
    hDN+mOS-KOOCHAkOOHII+hDN+mOS-KOOHIIdOCHIRA+
    eS2+hDF+mOS-DOCCHI+@@eBU+@@eYO+eYB//
\end{verbatim}
    \end{screen}
  \end{minipage}
  \caption{sIGNDEXによる手話表記例.「コーヒーと紅茶,
    どちらがよい\\です
    か?」に対する手話表記.\cite{Ichikawa2001}から引用}
  \label{fig:signdex}

  \vspace*{2ex}

  \begin{minipage}{.65\linewidth}
    \small
  \begin{screen}[4]
\tt\hfil 今日/本/買う
  \end{screen}
  \end{minipage}
  \caption{文献\cite{Tokuda1998}から引用した手話表記例.「今日,\\
本を買った。」に対する手話表記}
  \label{fig:tokuda}

  \vspace*{2ex}

  \begin{minipage}{.65\linewidth}
  \begin{screen}[4]
    \small
    \setlength{\baselineskip}{12pt}
\begin{verbatim}
    [[[],[[主格,[[[[[imoto.t,[rm,yes],[],[]],
    [[が,助詞,jyosi_ga.t], [],[],[]]],_]]]],
    [奪格,[[[[[kyoto.t,[rm,yes],[],[]],
    [[から,助詞,jyosi_kara.t],[],[],[]]],_]]]],
    [対格,[[[[[tokyo.t,[lm,yes],[],[]],
    [[に,助詞,yubi_ni.t],[],[],[]]],
    _G454]]]]],[[iku.t,[終始可変,rm,lm],[],[]],
    [ます。],_]]]$
\end{verbatim}
  \end{screen}
  \end{minipage}
  \caption{文献(池田・岩田・黒川2003)から引用した手話表記例.\\
    「妹が京都から東京に行きました。」に対する手話表記}
  \label{fig:ikeda}
\end{figure}

池田・岩田・黒川(2003)は,中間型手話を対象とした日本語-手話翻訳システム
において,手話文の格フレーム(入力日本語文の格フレーム中の日本語形態素
を手話形態素に置き換えたもの)からトークファイルと呼ばれる手話動作記述
ファイルを生成する際の中間形式として,手話表記法を定義して用いている.
各形態素での手の位置情報が記述可能となっており,格関係が,名詞や動詞の
位置情報として記述される.表記例を図\ref{fig:ikeda}に示す.表記には,入
力日本語文中の機能語情報が残され,組み込まれている.テキストという形式
はとっているが,他の表記法のように手話を記号化して読み書きするためのも
のではなく,翻訳過程(図\ref{fig:sltext})における中間表現(中間言語)
に相当するものと考えられる.このため,手話を書き取り,記録し,コーパス
を構築するような用途には適していない.一方,我々は,音声言語に対する文
字表現に相当するものとして手話テキストを捉え,手話文を読み書きすること
を念頭に置いた上で,計算機でも処理しやすい表記法を目指した.

前述のように手話には複数のチャネルを使って複数の形態素を組み合わせた表
現が見られる.しかし,池田・岩田・黒川(2003)ではこのような同時的な語順
に対する表記は定義されていない.同研究は,ほぼ日本語の語順に沿って表現
される中間型手話を目的言語としているため,手話と日本語との語順の違いに
ついては重視されていないのかもしれない.あるいは,表記中に残された日本
語情報から,手話の語順を決定することが可能かもしれない.しかし,手話へ
の翻訳過程を「言語的な変換(テキスト間の翻訳)」と「表現の変換」に分け
たとき,どのような単語をどのような語順で表出するかは前者の段階で決定さ
れるべき問題であり,そのためには手話表記法が語順を記述できる必要がある.
本論文で提案した表記法では,手形と動作による同時的表現は語形変化
(\ref{sec:inflection}節)として,左右の手による同時表現は単語の合成
(\ref{sec:compound}節)として記述可能である.

非手指要素が手指要素と同時的に表現される場合も多いが,徳田・奥村(1998),
池田・岩田・黒川(2003)ともに,非手指要素の記述方法は定義していない.い
ずれも音声日本語を伴う手話を表記対象としており,そのような手話では日本
語の口話と手指による表現が互いに情報を補完し合うために,非手指要素の役
割が小さくなり,表記する必要性が低いと考えられる.しかし,音声日本語を
伴わない日本手話では,非手指要素が話題化,疑問などの文法標識となるなど,
文法的にも重要な役割も持ち,非手指要素なしでは正しく意味が伝わらないた
め,本表記法では,非手指文法標識(\ref{sec:nms}節)や句読点
(\ref{sec:punctuation}節)として記述できるようにした.

語形変化に関しては,既存の表記法でも一致動詞の方向を記述できるものはあ
るが,数の一致に伴う動作や手形の変化については,表記を定めたものは見あ
たらない.本論文では,手形変化パラメータ,位置の複数形,位置集合
(\ref{sec:inflection}節)を定義することにより,手話の言語的な構造に沿
う形で,記述できるようになった.

\section{おわりに}

本論文では日本手話をテキストとして表現するための表記法を提案した.従来
の多くの手話表記法のように,手話の動作を正確に書き取ることを目的とする
のではなく,動作によって表される意味的・文法的な情報,言語的な構造の記
述に重点を置くことにより,個々の単語や単語間の動作の遷移など,動作の詳
細に立ち入らずに手話文を記述することができ,日本語-日本手話機械翻訳の
問題から動作合成の問題を切り離すことに貢献できる表記法となった.テキス
ト化によって,微妙なニュアンスなど失われる部分もあるが,文の構造や基本
的な意味は正しく伝えられるものと考えている.

表記法の表現力検証のため,手話を母語とする手話話者によって表現され
た720文の手話映像を対象に,書き取り実験を行なった.その結果,約93%の文
については表記することができたと考えている.十分表記できなかった51表現
を分析し,問題点について考察した.

日本語-手話機械翻訳システムの構築を進めていく上での今後の課題として,手
話の語彙の範囲で,手話の構造に沿った自然な手話テキストを生成するために
必要となる,入力日本語テキストに対する換言処理があげられる.また,次の
段階では手話テキストから動作記述を生成するという大きな課題がある.

現在我々は,日本語テキスト(構造訳レベル)から本表記法での手話テキスト
への機械翻訳システムの試作を行なっている.さらに,SignWritingで書かれた
手話への機械翻訳についても検討している.


\acknowledgment

本研究を進めるにあたり,岐阜県立岐阜聾学校教諭鈴村博司氏・長瀬さゆり氏,
岐阜大学教育学部池谷尚剛教授から貴重なご助言・コメントをいただきまし
た.ここに記して感謝いたします.


\bibliographystyle{jnlpbbl}
\begin{thebibliography}{}

\bibitem[\protect\BCAY{秋山\JBA 亀井}{秋山\JBA 亀井}{2004}]{Akiyama2004}
秋山なみ\JBA 亀井伸孝 \BBOP 2004\BBCP.
\newblock \Jem{手話でいこう---ろう者の言い分 聴者のホンネ}.
\newblock ミネルヴァ書房.

\bibitem[\protect\BCAY{Baker-Shenk \BBA\ Cokely}{Baker-Shenk \BBA\
  Cokely}{1980}]{Baker-Shenk1980}
Baker-Shenk, C.\BBACOMMA\ \BBA\ Cokely, D. \BBOP 1980\BBCP.
\newblock {\Bem American {S}ign {L}anguage, A Teacher's Resource Text on
  Grammar and Culture}.
\newblock Clerc Books, Gallaudet University Press.

\bibitem[\protect\BCAY{Bentele}{Bentele}{1999}]{Bentele1999}
Bentele, S. \BBOP 1999\BBCP.
\newblock \BBOQ Goldilocks \& the three bears in HamNoSys\BBCQ\
\newblock http://signwriting.org\allowbreak /forums/linguistics/ling007.html.

\bibitem[\protect\BCAY{ト\JBA 池田}{ト\JBA 池田}{2004}]{Bu2004}
ト朝暉\JBA 池田尚志 \BBOP 2004\BBCP.
\newblock \JBOQ 日中機械翻訳における否定文の翻訳\JBCQ\
\newblock \Jem{自然言語処理}, {\Bbf 11}  (3), \mbox{\BPGS\ 97--122}.

\bibitem[\protect\BCAY{Fant}{Fant}{1994}]{Fant1994}
Fant, L. \BBOP 1994\BBCP.
\newblock {\Bem The American Sign Language Phrase Book}.
\newblock Contemporary Books.

\bibitem[\protect\BCAY{藤重\JBA 黒川}{藤重\JBA 黒川}{1997}]{Fujishige1997}
藤重栄一\JBA 黒川隆夫 \BBOP 1997\BBCP.
\newblock \JBOQ
  意味ネットワークを媒介とする日本語・手話翻訳のための日本語処理\JBCQ\
\newblock \Jem{計測自動制御学会ヒューマン・インタフェース部会Human Interface
  News and Report}, {\Bbf 12}  (1), \mbox{\BPGS\ 45--50}.

\bibitem[\protect\BCAY{Gordon}{Gordon}{2005}]{Gordon2005}
Gordon, Jr., R.~G.\BED\ \BBOP 2005\BBCP.
\newblock {\Bem Ethnologue: Languages of the World, Fifteenth edition}.
\newblock SIL International.
\newblock Online version: http://www.ethnologue.com/.

\bibitem[\protect\BCAY{本名\JBA 加藤}{本名\JBA 加藤}{1990}]{Honna1990}
本名信行\JBA 加藤三保子 \BBOP 1990\BBCP.
\newblock \JBOQ 手話の表記法について\JBCQ\
\newblock \Jem{日本手話研究所所報}, {\Bbf \rule{0pt}{1pt}}  (4), \mbox{\BPGS\
  2--9}.

\bibitem[\protect\BCAY{市田}{市田}{1998}]{Ichida1998}
市田泰弘 \BBOP 1998\BBCP.
\newblock \JBOQ 日本手話の名詞句内の語順について\JBCQ\
\newblock \Jem{日本手話学会 第24回大会論文集}, \mbox{\BPGS\ 50--53}.

\bibitem[\protect\BCAY{市田}{市田}{1999}]{Ichida1999}
市田泰弘 \BBOP 1999\BBCP.
\newblock \JBOQ 日本手話一致動詞パラダイムの再検討---「順向・反転」「4
  人称」の導入から見えてくるもの---\JBCQ\
\newblock \Jem{日本手話学会 第25回大会論文集}, \mbox{\BPGS\ 34--37}.

\bibitem[\protect\BCAY{市田\JBA 川畑}{市田\JBA 川畑}{2000}]{Ichida2000}
市田泰弘\JBA 川畑裕子 \BBOP 2000\BBCP.
\newblock \JBOQ 日本手話の助動詞について\JBCQ\
\newblock \Jem{日本手話学会 第26回大会論文集}, \mbox{\BPGS\ 6--7}.

\bibitem[\protect\BCAY{市田}{市田}{2005}]{Ichida2005}
市田泰弘 \BBOP 2005\BBCP.
\newblock \JBOQ 自然言語としての手話\JBCQ\
\newblock \Jem{月刊言語}, {\Bbf 34}  (1), \mbox{\BPGS\ 90--97}.

\bibitem[\protect\BCAY{市川}{市川}{2001}]{Ichikawa2001}
市川熹 \BBOP 2001\BBCP.
\newblock \JBOQ 手話表記法sIGNDEX\JBCQ\
\newblock \Jem{手話コミュニケーション研究}, {\Bbf \rule{0pt}{1pt}}  (39),
  \mbox{\BPGS\ 17--23}.

\bibitem[\protect\BCAY{池田\JBA 岩田\JBA 黒川}{池田\Jetal }{2003}]{Ikeda2003}
池田隆二\JBA 岩田圭介\JBA 黒川隆夫 \BBOP 2003\BBCP.
\newblock \JBOQ
  日本語手話翻訳のための言語変換とそこにおける語形変化規則の処理\JBCQ\
\newblock \Jem{ヒューマンインタフェース学会研究報告集}, {\Bbf 5}  (1),
  \mbox{\BPGS\ 19--24}.

\bibitem[\protect\BCAY{神田\JBA 藤野}{神田\JBA 藤野}{1996}]{Kanda1996}
神田和幸\JBA 藤野信行\JEDS\ \BBOP 1996\BBCP.
\newblock \Jem{基礎からの手話学}.
\newblock 福村出版.

\bibitem[\protect\BCAY{河野\JBA 黒川}{河野\JBA 黒川}{2004}]{Kawano2004}
河野純大\JBA 黒川隆夫 \BBOP 2004\BBCP.
\newblock \JBOQ 日本語手話翻訳システムの開発\JBCQ\
\newblock \Jem{知能と情報 (日本知能情報ファジィ学会誌)}, {\Bbf 16}  (6),
  \mbox{\BPGS\ 485--491}.

\bibitem[\protect\BCAY{日本手話研究所}{日本手話研究所}{1997}]{JISLS1997}
日本手話研究所\JED\ \BBOP 1997\BBCP.
\newblock \Jem{日本語-手話辞典}.
\newblock 全日本ろうあ連盟.

\bibitem[\protect\BCAY{木村\JBA 市田}{木村\JBA 市田}{1995}]{Kimura1995}
木村晴美\JBA 市田泰弘 \BBOP 1995\BBCP.
\newblock \Jem{はじめての手話---初歩からやさしく学べる手話の本}.
\newblock 日本文芸社.

\bibitem[\protect\BCAY{マニンコウシン\JBA 福本\JBA
  池田尚志}{マニンコウシン\Jetal }{2004}]{Ngin2004}
マニンコウシン\JBA 福本真哉\JBA 池田尚志 \BBOP 2004\BBCP.
\newblock \JBOQ
  日本語-ミャンマー語機械翻訳システムjaw/Myanmarにおける述語構造の翻訳について
\JBCQ\
\newblock \Jem{第3回情報科学技術フォーラムFIT2004講演論文集}, \mbox{\BPGS\
  139--142}.

\bibitem[\protect\BCAY{松本}{松本}{2001}]{Matsumoto2001}
松本晶行 \BBOP 2001\BBCP.
\newblock \Jem{実感的手話文法試論}.
\newblock 全日本ろうあ連盟.

\bibitem[\protect\BCAY{松本\JBA 谷口\JBA 吉田\JBA 田中\JBA 池田}{松本\Jetal
  }{2005}]{Matsumoto2005a}
松本忠博\JBA 谷口真代\JBA 吉田鑑地\JBA 田中伸明\JBA 池田尚志 \BBOP 2005\BBCP.
\newblock \JBOQ 日本語-手話機械翻訳システムに向けて
  ---テキストレベルの翻訳系の試作と簡単な例文の翻訳---\JBCQ\
\newblock \Jem{信学技報}, {\Bbf 104}  (637), \mbox{\BPGS\ 43--48}.

\bibitem[\protect\BCAY{Matsumoto, Taniguchi, Yoshida, Tanaka, \BBA\
  Ikeda}{Matsumoto et~al.}{2005}]{Matsumoto2005b}
Matsumoto, T., Taniguchi, M., Yoshida, A., Tanaka, N., \BBA\ Ikeda, T. \BBOP
  2005\BBCP.
\newblock \BBOQ A proposal of a notation system for {J}apanese {S}ign
  {L}anguage and machine translation from Japanese text to sign language
  text\BBCQ\
\newblock In {\Bem Proceedings of the Conference Pacific Association for
  Computational Linguistics (PACLING 2005)}, \mbox{\BPGS\ 218--225}.

\bibitem[\protect\BCAY{{Nguyen,}~M.\JBA 池田}{{Nguyen,}~M.\JBA
  池田}{2005}]{Nguyen2005}
{Nguyen,}~M.C.\JBA 池田尚志 \BBOP 2005\BBCP.
\newblock \JBOQ 日本語-ベトナム語機械翻訳における連体修飾構造の翻訳\JBCQ\
\newblock \Jem{自然言語処理}, {\Bbf 12}  (3), \mbox{\BPGS\ 145--182}.

\bibitem[\protect\BCAY{Prillwitz et~al.}{Prillwitzet~al.
  }{2004}]{Prillwitz2004}
Prillwitz, S. et.~al \BBOP 2004\BBCP.
\newblock \BBOQ Sign Language Notation System\BBCQ\
\newblock http://www. sign-lang. unihamburg.de/Projects/HamNoSys.html.

\bibitem[\protect\BCAY{謝\JBA 今井\JBA 池田}{謝\Jetal }{2004}]{Shie2004}
謝軍\JBA 今井啓允\JBA 池田尚志 \BBOP 2004\BBCP.
\newblock \JBOQ 日中機械翻訳システムjaw/Chineseにおける変換・生成の方式\JBCQ\
\newblock \Jem{自然言語処理}, {\Bbf 11}  (1), \mbox{\BPGS\ 43--80}.

\bibitem[\protect\BCAY{手話情報学研究会}{手話情報学研究会}{2005}]{SILE2005}
手話情報学研究会 \BBOP 2005\BBCP.
\newblock \JBOQ sIGNDEX表記法\JBCQ\
\newblock http://www.ns.kogakuin.ac.jp/~wwc1015/sig-sile/signdex/top.html.

\bibitem[\protect\BCAY{SignFactory}{SignFactory}{1997}]{SignFactory1997}
SignFactory \BBOP 1997\BBCP.
\newblock \Jem{手話ジャーナル初級教材 No.1 (VHSビデオ)}.
\newblock ワールドパイオニア.

\bibitem[\protect\BCAY{SignFactory}{SignFactory}{1999}]{SignFactory1999}
SignFactory \BBOP 1999\BBCP.
\newblock \Jem{手話ジャーナル初級教材 No.2 (VHSビデオ)}.
\newblock ワールドパイオニア.

\bibitem[\protect\BCAY{Sutton}{Sutton}{2002}]{Sutton2002}
Sutton, V. \BBOP 2002\BBCP.
\newblock {\Bem Lessons In SignWriting, Textbook \& Workbook, 3rd ed.}
\newblock The Deaf Action Committee for SignWriting.
\newblock
  http://www.signwriting.org/archive/docs2/sw0116-Lessons-SignWriting.pdf.

\bibitem[\protect\BCAY{Sutton}{Sutton}{2006}]{Sutton2006}
Sutton, V. \BBOP 2006\BBCP.
\newblock {\Bem SignWritingSite}.
\newblock The Deaf Action Committee for SignWriting.
\newblock http://www.SignWriting.org/.

\bibitem[\protect\BCAY{Sutton-Spence \BBA\ Woll}{Sutton-Spence \BBA\
  Woll}{1999}]{Sutton-Spence1999}
Sutton-Spence, R.\BBACOMMA\ \BBA\ Woll, B. \BBOP 1999\BBCP.
\newblock {\Bem The Linguistics of British Sign Language---An Introduction}.
\newblock Cambridge University Press.

\bibitem[\protect\BCAY{Thelijjagoda, Imai, Elikewala, \BBA\ Ikeda}{Thelijjagoda
  et~al.}{2004}]{Thelijjagoda2004}
Thelijjagoda, S., Imai, Y., Elikewala, N., \BBA\ Ikeda, T. \BBOP 2004\BBCP.
\newblock \BBOQ Japanese-Sinhalese MT system (jaw/Sinhalese)\BBCQ\
\newblock In {\Bem Proceedings of Asian Symposium on Natural Language
  Processing to Overcome Language Barriers, IJCNLP-04 Satellite Symposium},
  \mbox{\BPGS\ 73--78}.

\bibitem[\protect\BCAY{徳田\JBA 奥村}{徳田\JBA 奥村}{1998}]{Tokuda1998}
徳田昌晃\JBA 奥村学 \BBOP 1998\BBCP.
\newblock \JBOQ
  日本語から手話への機械翻訳における手話単語辞書の補完方法について\JBCQ\
\newblock \Jem{情報処理学会論文誌}, {\Bbf 39}  (3), \mbox{\BPGS\ 542--550}.

\bibitem[\protect\BCAY{米川}{米川}{2002}]{Yonekawa2002}
米川明彦 \BBOP 2002\BBCP.
\newblock \Jem{手話ということば---もう一つの日本の言語}.
\newblock PHP新書.

\bibitem[\protect\BCAY{米川}{米川}{2004}]{Yonekawa2004}
米川明彦 \BBOP 2004\BBCP.
\newblock \Jem{NHKみんなの手話4--6月号}.
\newblock 日本放送出版協会.

\bibitem[\protect\BCAY{米川}{米川}{2005a}]{Yonekawa2005a}
米川明彦 \BBOP 2005a\BBCP.
\newblock \Jem{NHKみんなの手話4--6月号}.
\newblock 日本放送出版協会.

\bibitem[\protect\BCAY{米川}{米川}{2005b}]{Yonekawa2005b}
米川明彦 \BBOP 2005b\BBCP.
\newblock \Jem{NHKみんなの手話7--9月号}.
\newblock 日本放送出版協会.

\end{thebibliography}

\appendix
\section{表記法の構文と表記例}

\begin{center}
  
  
  {\bf 表4} 手話表記法の構文 \\ \vskip10pt
  \tablefirsthead{\hline}
  \tablehead{ \multicolumn{1}{l}
    {\small 前頁からの続き} \\ \hline}
  \tabletail{\hline \multicolumn{1}{r}{\small 次頁へ続く} \\}
  \tablelasttail{\hline}
  

\begin{supertabular}{|p{13.6cm}|}
    手話文 ::= 手話表現列 文末記号 \\ 
    手話表現列 ::= 手話表現{[ 区切り記号 ]手話表現 }\\
    文末記号 ::= ``。\unskip'' | ``?'' \\ 
    区切り記号 ::= ``{\tt ,}''	\hfill // 節,句など文法的な切れ目 \\
    \hspace{6zw}|``{\tt ;}'' 	\hfill	// 相対的に大きな切れ目 \\
    \hspace{6zw}|``{\tt \verb+~+}''  	\hfill	// 動詞と助動詞間の区切り \\
    手話表現 ::= 手話単語 | 複合表現 | ブロック \\ 
    手話単語 ::= 単語名 語形変化 | 指文字表現 \\
    語形変化 ::= [ ``{\tt [}'' 手形 ``{\tt ]}'' ][ ``{\tt (}''[ 空
    間][ ``{\tt ;}'' 修飾 ]``{\tt )}'' ] \\

    手形 ::= 手形名 | 指代名詞 \\
    手形名 ::= 手話単語名 | 指文字 \hfill // 手話単語や指文字${}^{*}$の手形 \\
    \hfill // ${}^{*}$指文字とは,手指で表された音声言語の文字(かな,
    英字,数字)\\

    指代名詞 ::= ゆび指定[ ``{\tt /}'' 手形名 ] \hfill // 指を代名詞的に用い
    る表現 \\
    ゆび指定 ::= ゆび名| ``{\tt\{}'' ゆび名{ ``{\tt ,}''
    ゆび名 }``{\tt\}}''\\
    ゆび名 ::= ``1'' | ``2'' | ``3'' | ``4'' | ``5'' | ``男'' | ``女''
    \hfill // 指を表す数や文字 \\
    空間 ::= 空間指定[ ``\verb+|+'' 空間指定 ]\\
    空間指定 ::= 位置 | 方向 \\ 
    位置 :: = 単一位置 | 複数位置 \\
    単一位置 ::= 位置指定[ 位置修飾子 ] \hfill // 人称位置 \\
    位置指定 ::= 位置定数	\hfill //1人称,2人称の位置 \\
    \hspace{5zw}| 位置変数	\hfill // 3人称(人・物・場所)の位置 \\
    \hspace{5zw}| 単語名	\hfill // 最後に現れた単語の位置 \\
    位置定数 ::= ``1'' | ``2'' \\
    位置変数 ::= ``3'' | ``4'' | ``$x$'' | ``$y$'' | ``$L$'' | ``$R$'' |
    ``$C$'' | $\cdots$ \\
    位置修飾子 ::= ``$\hat{ }$'' | ``\_''
    \hfill // 相対的な上下 (社会的上下関係)\\
    \hspace{6zw}| ``{\tt '}''
    \hfill // 少しずらした別の位置 (関連のある別の個体) \\
    複数位置 ::= ``\{''  単一位置{ ``,'' 単一位置 }``\}'' \hfill // 位置
    集合 \\
    \hspace{5zw}| 単一位置 ``{\tt s}''
    \hfill // 位置の複数形(彼ら,あちこち,…) \\
    方向 ::= [ 位置 ]``→'' 位置 \hfill // 始点と終点,または \\
    \hspace{5zw}| 位置 ``→'' [ 位置 ]\hfill // そのどちらかを指定 \\
    修飾 ::= 修飾指定{ ``{\tt ,}''  修飾指定 }
     \hfill // 動作の変化によって表される修飾語・機能語 \\
    修飾指定 ::= 修飾内容   \hfill // 修飾内容を日本語の語句で表す \\
    \hspace{5zw}|反復指定 \hfill // 動作の反復によって表される修飾内容 \\
    反復指定 ::= ``*''  反復回数    \hfill // n回〜する \\
    \hspace{5zw}|``**'' 	    \hfill // よく〜する (漠然と複数回) \\
    指文字表現 ::= ``{\tt '}''  指文字{ ``・''  指文字 } ``{\tt '}''  \\
    複合表現 ::= 逐次複合語 | 半同時表現 | 同時表現 \\
    逐次複合語 ::= 手話単語 ``{\tt -}''  手話単語{``{\tt -}''  手話単語} \\
    半同時表現 ::= 手話表現 ``{\tt /}'' (手話単語 | ブロック) \\ 
    同時表現 ::= 手話表現 ``\verb+|+'' (手話単語 | ブロック) \\
    ブロック ::= ``{\tt\{}''[ NMS列 ]手話表現列 ``{\tt\}}'' \\
    NMS列 ::= ``$<$'' {\it NMS} \{ ``{\tt ,}'' {\it NMS} \} ``$>$'' \\
    {\it NMS} ::= ``{\tt t}'' |``{\tt q}''|``{\tt whq}''|
    ``{\tt cond}'' | ``{\tt neg}'' 
     | ``{\tt em}''
    |``{\tt rs}''[ ``{\tt (}'' 位置 ``{\tt )}'' ]| $\cdots$ \\
\end{supertabular}
\end{center}

\newlength{\elem}
\setlength{\elem}{8zw}
\newlength{\example}
\setlength{\example}{10zw}
\newlength{\note}
\setlength{\note}{6.8cm}

\newcommand{\tbsp}{}
\begin{center}
  \vskip10pt
  {\bf 表5}   表記例 \\ \vskip10pt
  \tablefirsthead{\hline \multicolumn{1}{|c}{手話文の要素}
    & \multicolumn{1}{|c}{表記例}
    & \multicolumn{1}{|c|}{意味・説明} \\ \hline\hline}
  \tablehead{ \multicolumn{3}{l}
    {\small 前頁からの続き} \\ \hline
    \multicolumn{1}{|c}{手話文の要素}
    & \multicolumn{1}{|c}{表記例}
    & \multicolumn{1}{|c|}{意味・説明} \\ \hline}
  \tabletail{\hline \multicolumn{3}{r}{\small 次頁へ続く} \\}
  \tablelasttail{\hline}
  \begin{supertabular}{|p{\elem}|p{\example}|p{\note}|}
    単語(基本形)& 都合 & 「都合」「運」「偶然」など \\
    \hline
    語形変化(手形)& \tt 人[2] & 「二人」. 手形〈2〉で〈人〉を表現 \\
    \hline
    語形変化(位置)& \tt 東京(L) & 〈東京〉をLの位置で表現 \\
    \hline
    語形変化(方向)& \tt 言う(2→1) & 「あなた(2人称)が私(1人称)に言
    う」
    \\
    \hline
    \raisebox{-1zw}{動詞の複数変化} & \tt 言う(3s→1)
    & 「彼ら(3人称複数)が私に言う」\\
      \noalign{\vskip-.5zw}\cline{2-3}
    & \tt 行く(; **) & 「よく行く」 〈行く〉を2,3回繰り返す \\
    \hline
    語形変化(修飾)& \tt 風(; 強い) & 「強い風」「風が強い(強く吹く)」 \\
    \hline
    指差し & \tt Pt($x$) &
    \begin{tabular}{@{}p{\note}@{}}
      位置$x$をさして,「それ」「その」など代名詞,限定詞
    \end{tabular}
    \\    \hline
    順序数詞 & \tt Enum[2] &
    \begin{tabular}{@{}p{\note}@{}}
      「2つ目は…」単語や文を順に列挙しながら指を起こしてゆく表現
    \end{tabular}\\
    \hline
    順位数詞 & \tt Ref[\{1,2\}/4] & 「4つのうちの1番目と2番目」 \\
    \hline
    限定数詞 & \tt Quant[2/4] & 「4つのうち2つ」 \\
    \hline
    指文字による表現 & `パ・テ・ィ・シ・エ' & 「パティシエ」.  固有名詞
    や外来語 \\
    \hline
    逐次的な合成 & \tt 使う-税金 & 「消費税」. 複合語 \\
    \hline
    同時的な合成 & \tt 電話 \verb+|+ 仕事 &
    \begin{tabular}{@{}p{\note}@{}}
      「電話をしながら仕事をする」 左右の手で異なる単語を同時に表現
    \end{tabular}
    \\
    \hline
    半同時的な合成 & \tt 家/帰る(→家) &
    \begin{tabular}{@{}p{\note}@{}}
      「家へ帰る」 両手で〈家〉を表現した後,
      片手を残したまま,〈帰る〉を表現
    \end{tabular}
    \\
    \hline
    文法的な区切り, &   私{\tt ,} 妹 & 「私と妹」 \\
    \cline{2-3}
    名詞の並列 & 食べる{\tt ,} 寝る &「食べて,寝る」 \\
    \hline
    名詞による修飾 & あなた 母 & 「あなたのお母さん」 \\
    \hline
    \begin{tabular}{@{}p{\elem}@{}}
      手話単語の助動詞的用法
    \end{tabular}
    & 買う \verb+~+好き & 「買いたい」「買って欲しい」 \\
    \hline
    \raisebox{-1zw}{文末記号} & 私 聾唖。& 「私はろう者です。」\\
    \noalign{\vskip-.5zw}
    \cline{2-3}
    & 聾唖 あなた? & 「あなたはろう者ですか?」 \\
    \hline
    直接話法 &
    \begin{tabular}{@{}p{\example}@{}}
      \parbox{\example}{
        \tt
        彼 / 言う(彼→1)\\
        \{$<$rs(彼)$>$ あなた \\
        美しい\}}
    \end{tabular}
    &
    \begin{tabular}{@{}p{\note}@{}}
      「彼が『君はきれいだ』と私に言う」.引用部では,視線や2人称への指
    差し,体の向きがシフトする 
    \end{tabular}\\
    \hline
    非手指文法標識 & \tt \{$<$t$>$ 本\} 私 買う & 「本は私が買う」.話題化
    \\
  \end{supertabular}
\end{center}

\begin{biography}
  \biotitle{略歴}
  \bioauthor{松本 忠博}{
    1985年岐阜大学工学部電子工学科卒業.1987年同大学院修士課程修了.現
    在,同大学工学部応用情報学科助手.自然言語処理の研究に従事.言語処
    理学会,情報処理学会,電子情報通信学会,日本ソフトウェア科学会,日
    本手話学会,各会員.}
  \bioauthor{原田 大樹}{
    2006年岐阜大学工学部応用情報学科卒業.現在,同大学院工学研究科
    博士前期課程在学中.日本語から手話への機械翻訳の研究に従事.}
  \bioauthor{原 大介}{
    1991年教育学修士(国際基督教大学).2003年Ph.D.(言語学)(シカゴ大
    学).2000年4月愛知医科大学看護学部専任講師.2004年10月同学部助教
    授.専門は手話言語学.日本手話学会副会長.}
  \bioauthor{池田 尚志}{
    1968年東京大学教養学部基礎科学科卒業.同年工業技術院電子技術総合研
    究所入所.制御部情報制御研究室,知能情報部自然言語研究室に所属.
    1991年岐阜大学工学部電子情報工学科教授.現在,同応用情報学科教授.
    工博.自然言語処理の研究に従事.情報処理学会,電子情報通信学会,人
    工知能学会,言語処理学会,各会員.}
  
\bioreceived{受付}
\biorevised{再受付}
\bioaccepted{採録}

\end{biography}

\end{document}

