\documentclass[japanese]{jnlp}
\type{論文}
\etype{Paper}


\setcounter{page}{1}
\setcounter{巻数}{2}
\setcounter{号数}{3}
\setcounter{年}{2004}
\setcounter{月}{7}
\受付{2004}{1}{6}
\再受付{2004}{7}{8}
\採録{2004}{9}{10}

\setcounter{secnumdepth}{2}



\newcommand{\eng}[1]{}

\newcommand{\tx}[1]{}
\newcommand{\com}[1]{}
\newcommand{\coml}[1]{}
\newcommand{\comll}[1]{}
\newcommand{\comrr}[1]{}

\usepackage[dvips]{graphicx,color} 


\newcommand{\gm}[1]{}
\newcommand{\iz}[1]{}
\newcommand{\izs}[1]{}
\newcommand{\izj}[1]{}
\newcommand{\yizj}[1]{}
\newcommand{\jita}[1]{}
\newcommand{\alts}{}
\newcommand{\abs}{} 
\newcommand{\erg}{} 
\newcommand{\sbj}{}
\newcommand{\obj}{}
\newcommand{\ix}{}
\newcommand{\vrb}{}
\newcommand{\vt}{}
\newcommand{\vi}{}
\newcommand{\vp}{}
\newcommand{\bea}{}
\newcommand{\typeSO}{}

\newcommand{\exo}{}
\newcommand{\skel}{}
\newcommand{\flesh}{}
\newcommand{\dtr}{}
\newcommand{\grn}{}
\newcommand{\emp}{}
\newcommand{\ul}[1]{}
\newcommand{\border}{}
\newcommand{\soalt}{}
\newcommand{\saalt}{}

\usepackage{mygb4e} 
\let\eachwordone
\newcommand{\refs}[1]{}
\newcommand{\refsec}[1]{}
\newcommand{\exs}[1]{}

\usepackage{jnlpbbl}
\newcommand{\citep}{}
\newcommand{\citet}{}
\usepackage{relsize}
\usepackage{xspace}
\newcommand{\altje}{}
\usepackage{url}

\def\mpt#1{}
\def\smpt#1{}
\renewcommand{\mpt}[1]{}
\renewcommand{\smpt}[1]{}


\newcommand{\tot}{}
\newcommand{\rto}{}
\newcommand{\lto}{}

\usepackage{avm}
\avmfont{\scshape}
\avmvalfont{\normalfont}

\usepackage{fancyvrb} 

\usepackage[dvips]{graphicx} 
\usepackage{pstricks}
\usepackage{tree-dvips} 

                                
\newcommand{\sa}[2]{}


\begin{document}

\title{交替の特徴調査と二言語結合価エントリ獲得への利用}

\author{藤田 早苗 \affiref{NTT} \and
Francis Bond\affiref{NTT}}



\affilabel{NTT}{日本電信電話株式会社 NTT コミュニケーション科学基礎研究所}
{\\ NTT Communication Science Laboratories, Nippon Telegraph and Telephone Corporation}

\headauthor{}
\headtitle{交替の特徴調査と二言語結合価エントリ獲得への利用}

\subnames{\noindent
}



\jabstract{
  本稿では、自動詞の主語が他動詞の目的語となる動詞の交替を対象とし、既
  存の結合価辞書における交替の選択制限の対応関係の調査や、2言語間の交替の
  比較などを行なう。更に、これらの調査結果に基づき、
  交替データを用いて比較的単純な置き換えにより既存の結合価辞書に新しい
  エントリを追加する方法を提案する。本稿では、交替の片側に対応するエン
  トリから、もう片側のエントリを獲得する。また、本提案手法では2言語の
  結合価エントリを同時に作成する。作成したエントリは、下位範疇化構造や
  選択制限、交替情報等の詳細な情報を持っている。本稿の実験の結果、対象
  とした交替を85.4\%カバーすることができた。また、翻訳評価の結果、本手
  法で作成したエントリによって、翻訳結果が32\%改善された。}

\jkeywords{交替、自動詞、他動詞、結合価情報、辞書、日本語、英語}

\etitle{An Investigation into  the Nature of \\ 
Verbal Alternations and their Use in \\
the Creation of Bilingual Valency Entries}
\eauthor{ Sanae Fujita \affiref{NTT} \and
  Francis Bond\affiref{NTT} }

\eabstract{
  In this paper we investigate the properties of Japanese and English
  transitive-intransitive alternations.  For Japanese alternations, we
  show that the selectional restrictions of alternating arguments are
  more similar than those for non-alternating arguments.  Across
  languages we show that there are four major strategies for
  translating alternating verbs.  Finally, we present a method that
  uses alternation data to add new entries to an existing bilingual
  valency lexicon. If the existing lexicon has only one half of the
  alternation, then our method constructs the other half.  The new
  entries have detailed information about argument structure and
  selectional restrictions.  In this paper we focus on one class of
  alternations, but our method is applicable to any alternation.  We
  were able to increase the coverage of the causative alternation to
  85.4\%, and the new entries gave an overall improvement in
  translation quality of 32\%.
}

\ekeywords{Alternation, Transitive, Intransitive, Valency information,
  Dictionary, Japanese, English}




\maketitle


\section{はじめに}\label{sec:introduction}

\smpt{結合価辞書の重要性}

用言の下位範疇化構造や選択制限などの詳細な情報は、自然言語処理の様々な
分野で利用されている。本稿では、これらの詳細な情報を結合価情報と呼び、
結合価情報を持つ辞書を結合価辞書と呼ぶ。また、結合価辞書のエントリを結
合価エントリ、あるいは単にエントリと呼ぶ。結合価辞書を用いたシステムに
は、機械翻訳システム(\altje \citep{Ikehara:1991}、PIVOT
\citep{Nomura:2002j})や
自動要約システム(CB summarizer \citep{Nomura:2002j})、言い換えシステム
(蔵 \citep{Takahashi:01})、ゼロ代名詞照応システム(Zero Checker
\citep{Yamura-Takei:Fujiwara:Yoshie:Aizawa:2002})、質問応答システム
(SAIQA-II \citep{Sasaki:2004})などがあり、多岐に渡っている。


また、近年では、結合価辞書等の詳細な辞書情報とコーパスなどを利用した統
計的手法を融合させる研究も行なわれている
\citep{Uszkoreit:2002,Copestake:Flickinger:Pollard:Sag:1999}。
例えば、\citet{Carroll:Minnen:Briscoe:1998}は、統計的統語解析器に下位
範疇化構造の情報を持つ辞書を利用することで、解析精度をあげられることを
示している。


\smpt{言語現象を調べることに利用できる}

このように、詳細な情報を持つ結合価辞書は非常に有用なため、様々な自然言
語処理システムで利用されている。また、交替などの言語現象の量的な調査に
も利用できる。ここで交替関係とは、異なる表層的構造によって、ほぼ同じ意
味関係を表すことができるような関係である。例えば、「(店の)製品が\ul{完
  売する}」と「(店が)製品を\ul{完売する}」は表層的構造は異なるが、ほぼ
同じ意味関係を持ち、交替している。このような交替は、英語では、
\citet{Levin:1993} によって80種類以上提示されている。日本語では、
\citet{Bond:Baldwin:Fujita:2002j}  によって大規模に調査が行なわれてい
る。\citet{Bond:Baldwin:Fujita:2002j}によると、最も多い交替タイプは
「砂糖が\ul{溶ける}」 \tot 「私が砂糖を\ul {溶く}」などのように、自動
詞の主語 (\sbj)が他動詞の目的語 (\obj)となる交替(以降、\soalt と呼ぶ)
であると報告されている。\soalt は全交替の34\%を占めており、最も一般的
な交替タイプであるといえる。本稿では、この、最も一般的な交替タイプであ
る\soalt を対象とし、既存の結合価辞書を用いて交替の選択制限の対応関係
等の調査を量的に行なう。また、その調査結果に基づき、交替情報を用いて新
たな結合価エントリを獲得する方法を提案する。



\smpt{結合価辞書の構築方法の先行研究}


結合価辞書の構築方法は多く提案されており、これらの構築方法は大別して3種類
に分類できる。第一に人手で作成する方法がある\citep{Shirai:1999zj}。人
手で作成する方法の利点は、質の高い言語資源が獲得できるという点である。
しかし、その作成にはコストと時間がかかるという問題や、作成するエントリ
が網羅性に欠けるという問題がある。
また、内省による作成の場合、作成者や作成時期の異なりによる判断の揺れが
起こり、辞書の一貫性を保つことが難しいという問題もある。


第二に、コーパスから情報を学習する方法が提案されている
\citep{Li:Abe:1998,Manning:1993,Utsuro:1997,Kawahara:Kurohashi:2001}。
しかし、\citet{Korhonen:2002} は、コーパスからの単言語の下位範疇化構造
を自動的に獲得する場合、精度は約80\%が上限である事を示している。また、
\citet{Utsuro:1997} や\citet{Korhonen:2002} は、下位範疇化構造を自動的
に獲得する場合でも、人手による修正が必要であると述べている。このように、自
動学習では、必然的にエラーが含まれ精度が保証できないため、完全に自動構
築された結合価辞書はほとんどない。

第三に、言語資源を統合する方法が提案されている。例えば、既存の結合価辞
書を半自動的に拡張する方法
\citep{Fujita:Bond:2002a,Bond:Fujita:2003,Hong:Kim:Park:Lee:2004}、コー
パスからの学習データを用いて拡張する方法\citep{Korhonen:2002}、多言語
辞書を用いて単言語データを豊かにする方法\citep{Probst:2003}が提案され
ている。このように、言語資源を統合する方法は多様であるが、全般に人手で
全て作成するよりコストが安く、コーパスから自動的に獲得するより信頼性が
高いという利点がある。
また、こうした方法では、様々な研究者や組織により構築され
ている言語資源を有効利用できるという利点もある。


\smpt{提案手法} 


本稿で提案する結合価エントリの獲得方法は、第三の言語資源を統合する
方法に分類できる。
本提案手法では、交替を起こす動詞に対し、
交替の片側に対応する結合価エントリが不足している場合、不足しているエン
トリを自動的に獲得する手法を提案する。
本提案手法では、見出し語レベルでの交替情報、すなわち、「溶ける」と「溶
く」は交替する、という情報と、交替の片側に対応する既存の結合価エントリ
を種として用いる。これらから、交替のもう一方に対応する新たな結合価エン
トリを獲得し、両エントリ間の対応関係を辞書に付与する。すなわち、本提案
手法は、交替を起こす動詞で不足している結合価エントリを補うと共に、結合
価エントリ間の交替関係の情報を付与することで結合価辞書をより豊かにする
ことができる。また、既存の結合価辞書が2言語の結合価情報を持つ場合、両
方の言語の結合価情報も同時に獲得できる。そのため、本提案手法は特に機械
翻訳において利用価値が高い。





以下、\ref{sec:resource} 章では、本稿で利用する言語資源を紹介する。
\ref{sec:exam} 章では、\soalt の調査を行なう。\ref{sec:create-method} 
章では、\ref{sec:exam} 章の調査に基づき、交替情報を用いた結合価エント
リの作成方法を提案する。\ref{sec:eva} 章では本提案手法で作成した結合価
エントリの評価について報告する。\ref{sec:discuss} 章では、本提案手法の
改良や展開について議論し、\ref{sec:conclusion} 章はまとめである。

\section{利用する言語資源} \label{sec:resource}


本章では、交替を起こす動詞の特徴調査に利用する言語資源について述べる。
調査に利用する言語資源は、交替を起こす動詞のリスト(以下、交替動詞リス
トと呼ぶ)と、既存の2言語の結合価情報を持つ結合価辞書である。また、こ
れらの言語資源は、\ref{sec:create-method} 章で提案するの結合価エン
トリの獲得方法でも利用する。


\smpt{交替動詞リストの説明}

まず、交替動詞リストとは、\soalt を起こす動詞の組み合わせのリストであ
る。このリストは、\citet{Jacobsen:1981}と\citet{Bullock:1999}の日本語
交替のデータと、日英辞書である EDICT \citep{Breen:1995,Breen:2004}を元
に我々が作成したものであり、日本語の和語動詞の自動詞と他動詞のペアと、各動詞につき
一つ以上の英訳(gloss)で構成される。
\citet{Jacobsen:1981}のリストには370組、\citet{Bullock:1999}の
リストには190組の和語動詞が登録されている。我々がEDICTを用いて
抽出したのは434組である。これらのリストから、
「俯向く」「俯く」などの表記揺れを吸収し、重複を除くと571組になる。
更にこの中から、\soalt{}として不適切な動詞の組み合せ、例えば、
「混む」「込める」のように語義の異なる組み合わせや、
「漏れる」「漏る」のような自動詞同士の組み合わせなどを人手で削除し
\footnote{\citet{Jacobsen:1981}と\citet{Bullock:1999}のリストからも、
  主に、語義が異なるという理由で約70組が削除された。}、
最終的に460組を残した。
このリストの例を表\ref{tab:list}  に示す。




\begin{table}[htbp]
  \centering 
  \begin{tabular}{ll|ll}
    \multicolumn{2}{c|}{\jita{自動詞}} & 
    \multicolumn{2}{c}{\jita{他動詞}} \\ 
    \hline
    日本語 &  英訳 & 日本語 & 英訳\\
    \hline
    溶ける &  dissolve & 溶く  & dissolve \\
    泣く  &  cry & 泣かす &  make cry \\
    上がる &  rise & 上げる &  lift \\
  \end{tabular}
  \caption{\soalt のリストの例}
  \label{tab:list}
\end{table}



\smpt{結合価辞書(ALT)の説明}



また、本稿で用いた結合価辞書は、NTTで日英機械翻訳システム\altje 用に開
発してきた結合価辞書\citep{GoiTaikeij}である。この結合価辞書は日本語と
英語の両方の結合価情報を持っている。\altje{}の結合価辞書は、慣用表現辞
書と形容詞を除き、5,062の日本語動詞と11,214の結合価エントリで構成され
ている。
日本語側の結合価情報には項構造と選択制限の情報が含まれる。
\mpt{項構造の説明}
ここで項とは、動詞の意味を完結させるために必要な情報であり、
用いられる表層的な格助詞、意味役割の情報を含んでいる。
また多くのエントリには、必須格に加えて、曖昧性を減らすため、随格も付与
されている。



\begin{figure*}[htb]
\begin{center}
\begin{tabular}{ll}
\framebox{\begin{minipage}[t]{0.40\textwidth}
    \begin{tabular}[t]{ll}
      \multicolumn{2}{l}{J-E Entry: 302116}\\ 
      \\
      \sbj & ┌ \textbf{N1}:\izj{具体物} が \node{sbj}{\gm{nom}} \\     
      \ix  & ├ \textbf{N3}:\izj{具体物} に \node{ix1}{\gm{dat}} \\
      & └ \textbf{Vi}: 溶ける \\
      \\[-1ex]
      \\
      \sbj &┌ \ul{\textbf{N1}} \node{Esbj}{\gm{subject}}\\
      &├ \textbf{Vi} \ul{\eng{dissolve}}\\
      \ix &└ \textbf{PP} \node{Eix1}{\ul{\eng{in} \textbf{N3}}}\\
    \end{tabular}
  \end{minipage}}
  &
  \framebox{\begin{minipage}[t]{0.45\textwidth}
      \begin{tabular}[t]{ll}
      \multicolumn{2}{l}{J-E Entry: 508661}\\ 
      \node{erg}{\abs} &┌ \textbf{N1}:\izj{人 人工物} が   \gm{nom}   \\
      \node{obj}{\obj} &├ \textbf{N2}:\izj{具体物} を \gm{acc} \\ 
                                
      \node{ix2}{\ix} &├ \textbf{N3}:\izj{無生物} に \gm{dat}  \\
      &└ \textbf{Vt}: 溶く \\
      \\[-1ex]
      \node{Eerg}{\abs} &┌ \ul{\textbf{N1}}  \gm{subject}\\
      &├ \textbf{Vt} \ul{\eng{dissolve}}\\
      \node{Eobj}{\obj} &├ \ul{\textbf{N2}} \gm{direct object}\\
      \node{Eix2}{\ix}  &└ \textbf{PP} \ul{\eng{in} \textbf{N3}} \\
    \end{tabular}
  \end{minipage}}
\end{tabular}
 \caption{\mbox{\soalt の例: \textbf{Vi} 溶ける \protect\eng{dissolve}} \tot \textbf{Vt} 溶く \protect\eng{dissolve}}
  \label{fig:toku-tokeru}
 \vspace*{-5mm}
\end{center}
\aanodecurve[r][0]{ix1}[l][0]{ix2}{10mm}
\aanodecurve[r][0]{Eix1}[l][0]{Eix2}{10mm}
\aanodecurve[r][0]{sbj}[l][0]{obj}{10mm}
\aanodecurve[r][0]{Esbj}[l][0]{Eobj}{9mm}
\end{figure*}


図\ref{fig:toku-tokeru} に、\altje の結合価辞書の例を示す。
図\ref{fig:toku-tokeru} の「溶く」 \tot 「溶ける」は\soalt の関係を持
ち、図のように項構造をリンクできる。
但し図\ref{fig:toku-tokeru} では、
\abs{}は他動詞側だけに現れ、自動詞側では対応する意味役割がない。
また、図\ref{fig:toku-tokeru}で、N1は主格を、N2、N3は目的格を表す変数であり、
本稿ではこれらN1、N2、N3等を格役割と呼ぶ。
また、図\ref{fig:toku-tokeru}で、\izj{ }で示したのは格の選択制限であり、
意味属性か字面、あるいは、\iz{*}のリスト形式で定義されている。意味属性
は2,710 カテゴリを持つ日本語語彙大系\citep{GoiTaikeij}のシソーラスで定
義されている。このシソーラスの上位4レベルを図\ref{fig:iz} に示す。この
シソーラスは最大12レベルまでの深さを持つ非平衡型階層構造である。レベル
1は\iz{名詞}であり、レベル12は\iz{農作業}, \iz{出演}などの細かな意味属
性を含む。レベルが深くなるにつれ、意味はより特殊化されているため、選択
制限はより厳しくなる。また、字面は、特定の語とだけ一致し得ることを、
\iz{*}は、あらゆる語や節を取り得ることを示している。


なお、本稿では、\soalt{}における他動詞側の主格を\abs (ergative)、目的格を\obj{}、
自動詞側の主格を\sbj{}と呼ぶ。また、\abs{}、\obj{}、\sbj{}以外の格は、
\ix{}と呼ぶ\citep[p 11]{Dixon:1991}。


\smpt{調査対象と実験対象}


本稿で用いる交替動詞リスト(460組)のうち、315組(68.5\%)は、図
\ref{fig:toku-tokeru} のように自動詞と他動詞の両方のエントリが既存の結
合価辞書に存在する。また、79組(17.2\%)は自動詞か他動詞のどちらかのみが
既存の結合価辞書に存在する(図\ref{fig:venn} 参照)。自動詞、他動詞共に
結合価辞書に存在しないのは、66組合せ(14.3\%)である。ここで、両方のエン
トリが存在する315組をエントリに展開すると、381組の結合価エントリとなる。
この381組の結合価エントリを利用して、\soalt の交替の特徴調査を行なう
(\ref{sec:AOS-compare}章)。また、片側のエントリのみが結合価辞書に存在
する79組の動詞について、欠如している結合価エントリの獲得実験を行なう
(\ref{sec:create-method}章)。

\begin{figure}[htbp]
  \fbox{\begin{pspicture}(12,4)
      
      \psset{fillstyle=solid} \psset{fillcolor=white}
      \psellipse(7,2)(5,1.7) \psset{fillcolor=white}
      \psclip{\psset{fillcolor=white}\psellipse(5,2)(4.75,1.7)}
      \psset{fillcolor=lightgray} \psellipse(7,2)(5,1.7) \endpsclip
      \rput(0.75,3.75){\shortstack{全460組}}
      \rput(6,2){\shortstack{自/他動詞共に存在\\315組}}
      \rput(1.2,2){\shortstack{他動詞のみ\\27組}}
      \rput(11,2){\shortstack{自動詞のみ\\52組}}
       \rput(6,0.05){\shortstack{自/他動詞共に存在せず\hspace{0.5cm}66組}}
   \end{pspicture}}\centering
  \caption{\soalt の交替動詞リストに対応する結合価エントリの有無}
  \label{fig:venn}
\end{figure}



{
\setlength{\tabcolsep}{4.7pt} 
\begin{figure*}[hbtp]
  \begin{center}
  \begin{tabular}{llllllllllllllllllllllll}
&&\multicolumn{10}{c}{\sa{1}{名詞}}\\[2ex]
\multicolumn{7}{c}{\sa{2}{具体}} & \multicolumn{14}{c}{\sa{1000}{抽象}}\\[2ex]
\multicolumn{2}{c}{\sa{3}{主体}} &
\multicolumn{3}{c}{\sa{388}{場所}} &
\multicolumn{2}{c}{\sa{533}{具体物}} &
\multicolumn{2}{c}{\sa{1001}{抽象物}} &
\multicolumn{3}{c}{\sa{1235}{事}}  &
\multicolumn{9}{c}{\sa{2422}{抽象的関係}}    \\[2ex]
    \sa{4}{人\\ ~} 
 &  \sa{362}{組\\織}
 &  \sa{389}{施\\設}
 &  \sa{458}{地\\域}
 &  \sa{468}{自\\然}
 &  \sa{534}{生\\物\\ ~}
 &  \sa{706}{無\\生\\物}
 &  \sa{1002}{抽象物\\(精神)} 
 &  \sa{1154}{\hspace*{-1em}抽象物\\ \hspace*{-1em}(行為)}
 &  \sa{1236}{人\\間\\活\\動}
 &  \sa{2054}{事\\象}
 &  \sa{2304}{自\\然\\現\\象}
 &  \sa{2423}{存\\在}
 &  \sa{2432}{類\\・\\系}
 &  \sa{2443}{関\\連}
 &  \sa{2483}{性\\質}
 &  \sa{2507}{状\\態}
 &  \sa{2564}{形\\状}
 &  \sa{2585}{数\\量}
 &  \sa{2610}{場}
 &  \sa{2670}{時\\間}
  \end{tabular}
\nodeconnect[b]{c1}[t]{c2}
\nodeconnect[b]{c1}[t]{c1000}
\nodeconnect[b]{c2}[t]{c3}
\nodeconnect[b]{c2}[t]{c388}
\nodeconnect[b]{c2}[t]{c533}
\nodeconnect[b]{c3}[t]{c4}
\nodeconnect[b]{c3}[t]{c362}
\nodeconnect[b]{c388}[t]{c389}
\nodeconnect[b]{c388}[t]{c458}
\nodeconnect[b]{c388}[t]{c468}
\nodeconnect[b]{c533}[t]{c534}
\nodeconnect[b]{c533}[t]{c706}
\nodeconnect[b]{c1000}[t]{c1001}
\nodeconnect[b]{c1000}[t]{c1235}
\nodeconnect[b]{c1000}[t]{c2422}
\nodeconnect[b]{c1001}[t]{c1002}
\nodeconnect[b]{c1001}[t]{c1154}
\nodeconnect[b]{c1235}[t]{c1236}
\nodeconnect[b]{c1235}[t]{c2054}
\nodeconnect[b]{c1235}[t]{c2304}
\nodeconnect[b]{c2422}[t]{c2423}
\nodeconnect[b]{c2422}[t]{c2432}
\nodeconnect[b]{c2422}[t]{c2443}
\nodeconnect[b]{c2422}[t]{c2483}
\nodeconnect[b]{c2422}[t]{c2507}
\nodeconnect[b]{c2422}[t]{c2564}
\nodeconnect[b]{c2422}[t]{c2585}
\nodeconnect[b]{c2422}[t]{c2610}
\nodeconnect[b]{c2422}[t]{c2670}
     \caption{日本語語彙大系の上位4階層(一般名詞シソーラス)}
     \label{fig:iz}
\end{center}
\end{figure*}
}




\section{\soalt{} の調査} \label{sec:exam}


\subsection{\abs \obj \sbj の選択制限の調査} \label{sec:AOS-compare}


交替では、同じ意味役割が、異なる表層格(syntactic position)に出現し得る
\citep[pp 118--123]{Gunji:2002}。
「溶く」 \tot 「溶ける」の交替を例にあげると、溶か
される役
は自動詞「溶ける」の主語(\sbj{})であり、かつ、他動
詞「溶く」の目的語(\obj{})でもある。\citet{Baldwin:1999b}は、異なる表
層格に対応する選択制限が出現すると仮定している。\citet{Dorr:1997}は、
一つの表記により両方の交替のエントリを生成しており、この仮定を支持して
いるようである。
この仮定のように、異なる表層格に
同じ選択制限が用いられるのであれば、交替の片側のエントリからもう片側の
エントリを作成する場合に、対応する表層格では同じ選択制限が利用できる。

但し、\abs{}は空の項と交替するため、他動詞側のエントリを作成する場合
の選択制限を、対応する表層格から得ることができない。
しかし、\citet{Kilgarriff:1993}は、\abs{}は\izs{意識(sentient)}と\izs{意
  志性(volition)}を持ち、\obj{}は\izs{状態変化(changes-state)}と、
\izs{影響(causally affected)}を受けるという特徴を持つとしている。
Kilgarriffの主張のように、特に\abs{}の意味属性が特徴的であれば、
他動詞側のエントリを作成する場合に、最も典型的な意味属性を用いて\abs{}
を作成することができる。

そこで本章では、\soalt における意味役割、具体的には\abs{}、\obj{}、
\sbj{}の選択制限として用いられている意味属性の同一性や性質を調査する。
特に\sbj{}と\obj{}に同一の選択制限が利用できるかどうか、また、\abs に
頻出する選択制限を調査する。


まず、選択制限として用いられている意味属性の同一性を検討するため、
\abs{}、\obj{}、\sbj{}の意味属性間の距離を調査する。選択制限は意味属性
のリストで表されるため、2つの選択制限に含まれる意味属性間の親等
\footnote{親等とは、「親族関係の親疎を測る単位。直系親では、親子の間を
  一世とし、その世数によって定める。〜中略〜 傍系親では、それぞれの共
  通の祖先までの世数を合計して算出する。〜後略〜」(広辞苑 第四版
  CD-ROM版\citep{koujien}より)}のうち、最少のものを最近距離として用いる。親等は例
えば、図\ref{fig:iz} より、\izj{名詞}と\izj{具体}は1親等、\izj{名詞}と
\izj{主体}は2親等、のようになる。そのため、最近距離は近ければ近いほど、
それぞれの意味属性が近い事を示している。
但し、利用する結合価辞書は人手で作成されたものであるため、1親等程度の
差は、作成者や作成時期の異なりによる揺れの可能性もある。
例えば、図\ref{fig:toku-tokeru} の\ix{}の選択制限は
それぞれ「溶ける」では\izj{具体物}、「溶く」では\izj{無生物}であり、
最近距離が1となるが、これは有意な差ではないと考えられる。
しかしこの場合も、最近距離は高々1であり、
最近距離が近ければ近いほど、意味属性が近いことに変わりはない。

図\ref{fig:sr-diff} は、意味役
割が対応する\abs{}と\sbj{}、文法役割が対応する\obj{}と\sbj{}の最近距離
の分布の調査結果を示している。但し、最近距離0の組み合わせのうち、選択
制限が完全に同一になったものを「0(同一)」に分類し、その他のものを「0」
に分類している。例えば、図\ref{fig:toku-tokeru} の「溶く」\tot{}「溶け
る」では、\sbj{}と\obj{}の選択制限は共に\izj{具体物}なので、最近距離は
「0(同一)」である。また、\abs{}の選択制限は\izj{人 人工物}であり、図
\ref{fig:iz} より\izj{人}と\izj{具体物}は3親等、\izj{人工物}と\izj{具
  体物}は2親等\footnote{\izj {人工物}は\izj{無生物}(図\ref{fig:iz})の
  子供なので、\izj{具体物}の孫である。}なので、\abs{}と\sbj{}の最近距
離は2である。


図\ref{fig:sr-diff} から、\obj{}と\sbj{}の選択制限は最近距離0(同一)が
30.1\%、最近距離0が27.5\%であり、ここまでで、全体の57.6\%を占める。
対して、\abs{}と
\sbj{}では、最近距離1が26.7\%と最も多く、次が最近距離2の21.5\%である。つまり、
\obj{}と\sbj{}は文法的には異なる位置にあるが、
\citet{Bond:Baldwin:Fujita:2002j} が主張しているように、選択制限の一致
率は高い。特に、完全に同じ選択制限でなくとも、少なくとも一部は同じ意味属性を
含んでいる割合が非常に高い。一方、\abs{}と\sbj{}は文法役割は共に主語だ
が、\obj{}と\sbj{}に比べ、一致率は低く、選択制限は異なっている。


\begin{figure}[h]
\begin{center}
\includegraphics[angle=0,width=100mm]{SR-taiou.eps}
\caption{選択制限の最近距離}
\label{fig:sr-diff}
\end{center}
\end{figure}


次に、\abs{}、\obj{}、\sbj{}、特に\abs{}が\izs{意識}と\izs{意志性}を持
つかどうかを調べる。日本語語彙大系の階層の中で、\izj{主体}配下の意味属
性は\izs{意識}と\izs{意志性}を持ち、\izj{主体}配下の意味属性が含まれる
割合が高いほど、その格の動作主性(agentivity)が高いといえる。\abs{}、
\obj{}、\sbj{}の選択制限に、\izj{主体}配下の意味属性が含まれる割合は、
\abs{}で60.1\%、\obj{}で14.1\%、\sbj{}で13.9\%であった。つまり、
\citet{Kilgarriff:1993} が主張しているように、\abs{}は非常に動作主性が
高いが、\sbj{}は文法的には同じ位置にあるが動作主性は低い。なお、\abs{}
の選択制限として最も出現頻度が高かったのは、\izj{主体}であり、全\abs{}
の41.4\%を占めた。



これらの結果をまとめると、\sbj と \obj は完全に一致するわけではないが、
少なくとも一部の意味制限が一致する確率が高く、\sbj と \obj の選択制限
として同一の選択制限が利用できるといえる。また、\abs{}は\izs{意識}と
\izs{意志性}を持つ割合が非常に高く、\abs{}の選択制限として最もよく利用
されるのは\izj{主体}である。



\subsection{日本語と英語の交替の比較調査}\label{sec:alternations}


本節では、2言語間の交替比較を行なう。特に、日本語が\soalt を起こす場合
に、対応する英訳の結合価情報の変化を調査する。結合価エントリ獲得の観点
からすると、日本語が\soalt を起こす場合に、英語側の結合価の変化も規則
的であれば、英語側の結合価情報も日本語側と同時に獲得が可能であると予測
できる。

この調査には、交替動詞リストを用いる。交替動詞リストは日本語は460組み
合わせだが、多くの動詞は英訳が複数あるため、英訳の異なりを考慮すると、
全部で839の組み合わせからなる。このリストの英訳組み合わせを分類した結
果を、表\ref{tb:alternation-type}に示す。


{
\setlength{\tabcolsep}{2.2pt} 
\begin{table*}[h]
\small
\begin{tabular}{lllllllrr}
 \multicolumn{2}{c}{日本語} & 
 \multicolumn{2}{c}{英訳} &
 \multicolumn{2}{c}{英語構造} & タイプ & 数 & (\%)\\
\jita{自動詞} & \jita{他動詞} &
\jita{自動詞} & \jita{他動詞} &
\jita{自動詞} & \jita{他動詞} & \\
\hline
\hline
弱まる &  
弱める &
\sbj \ul{weaken} &
\abs \ul{weaken} \obj  &
\eng{\sbj Vi} &
\eng{\abs Vt \obj} & 
\typeSO &
138 & 30.0\\
漏れる &
漏らす &
 \sbj be \ul{omitted} & 
 \abs \ul{omit} \obj    &
  \eng{\sbj be Vt-ed} &
 \eng{\abs Vt \obj} & 
\iz{passive} &
91 & 19.8 \\
泣く & 泣かす &
 \sbj \ul{cry}  &
 \abs make \obj \ul{cry}  &
  \eng{\sbj Vi/be Adj} &
\eng{\abs Vc \obj Vi/Adj} & 
\iz{synthetic} &
30 & 6.5 \\
\hline
亡くなる &亡くす &
 \sbj \ul{pass away} &
 \abs \ul{lose} \obj  &
  \eng{\sbj Vi} &
 \eng{\abs Vt \obj} 
&
主辞が異なる
& 197 & 42.8 \\% \eng{\abs Vt \obj} &
じゃれる & じゃらす &
  \sbj \ul{play} & 
  \abs \ul{play} with  \obj &
   \eng{\sbj Vi} &
  \eng{\abs Vt prep \obj} & 
構造が異なる &
4 & 0.9 \\
\end{tabular}
\\[1ex] \footnotesize{Vc は\eng{make,get,let,become}等の制御動詞
 (control verb)。
 実際のエントリには随格が含まれることもある。}
\caption{英語側交替タイプ分類} 
\label{tb:alternation-type}
\end{table*}
}



表\ref{tb:alternation-type} の通り、英訳を5タイプに分類した。線より上
は、自動詞側も他動詞側も英語の主辞が同じものである。これらは、形態変化
を伴わない\typeSO タイプ、形態変化を伴う\iz{passive}タイプ、合成的に訳
される\iz{synthetic}タイプの3種類からなる。\typeSO{}タイプでは、英語側
も\soalt を許す非対格動詞が使われている。このタイプに分類されたものは
最も多く、30.0\%を占める。\iz{passive}タイプでは、日本語の自動詞側に対
応する英訳が、他動詞の受身として訳されている。このタイプに分類されたも
のは19.8\%を占める。\iz{synthetic}タイプでは、日本語の他動詞側に対応す
る英訳が、制御動詞 (control verb) が自動詞か形容詞を補語として持つ形で
訳されている。用いられる制御動詞は\eng{make}が多いが、
\eng{get,let,become}などの場合もある。このタイプに分類されたものは
6.5\%を占める。

線より下の2タイプは、英語の主辞が異なっている(42.8\%)か、\eng{\sbj play
  \tot \abs play with \obj }のように、英語の主辞は同じだが結合価の変化
が上記のクラスに当てはまらないものである(0.9\%)。ただし、同じ日本語ペア
に対し、英訳組み合わせが複数ある場合は「主辞が同じ」組み合わせが一つでもあれば
そちらに分類している。例えば、「集まる」\tot 「集める」の英訳は
\eng{gather \tot collect} と、\eng{be gathered \tot
gather} の組み合わせがあるため、\iz{passive}タイプに分類している。


結合価辞書構築の観点からすると、最初の3タイプのように英語の主辞が同じ
で、その結合価の変化を規則化できる場合、日本語の交替を作成すると同時に、
英訳も自動的に生成できる。つまり、交替の片側の英訳からもう片側の交替の
英訳を作成できる可能性は、表\ref{tb:alternation-type} から56.3\% (30.0
\% $+$ 19.8\% $+$ 6.5\%)と見積もることができる。但し、最初の3タイプに
分類されなかったものでも、他の辞書等から異なる英訳を抽出すれば、このタ
イプに分類される可能性がある。また、逆に、ほとんどの動詞が複数の英訳を
持つ事からもわかるように、同じ主辞を用いて英訳が生成できたとしても、そ
の英訳が最適とは限らない。


\smpt{日マ}

さて、本章では日英の交替の比較を行なったが、他の言語対の場合でも、多く
はこのような分類になると思われる。例えば、日本語とマレー語の場合、日本
語側が\soalt なら、マレー語では一般に、同じ語幹に自動詞と他動詞で違う
接辞をとる交替で翻訳できる。例えば、「(砂糖が水に)溶ける」 \tot 「(私
が砂糖を水に)溶く」における「溶ける」 \tot{} 「溶く」はマレー語では
\eng{larut} \tot \eng{\ul{me}larut\ul{kan}}である。しかし、マレー語に
おいてもすべての日本語の\soalt をこのように翻訳できるわけではなく、合
成的に翻訳したり、違う動詞で翻訳することもある。

このように、ある言語の交替が目的言語側でも似た交替を用いて翻訳できると
は限らない。そのため、目的言語の結合価情報の翻訳方法として、以下の4通
りの方法が考えられる。

\begin{enumerate}
\item 目的言語でも交替として翻訳できる場合。但し、一つの交替になるとは
  限らない。日英の場合、\typeSO{}タイプと\iz{passive}タイプがこれにあ
  たる。
  
\item 目的言語では、交替の意味の差を合成的に翻訳する場合。日英の場合、
  \iz{synthetic}タイプがこれにあたる。

\item 目的言語では、違う語として翻訳する場合。
  日英の場合、「主辞が異なる」としたタイプがこれにあたる。

\item 目的言語では交替によるニュアンスの差を表すことができず、両方同じ
  翻訳になる場合。
  日英の場合、表\ref{tb:alternation-type} には、ここに分類されるものはな
  かった。しかし、例えば、「私に英語がわかる」「私が英語をわかる」
  という交替を考えると、英語では共に\eng{I can understand English}
  と訳され、日本語側の交替に対応した英訳の変化がない。
\end{enumerate}




\section{結合価エントリの作成方法} \label{sec:create-method}

\ref{sec:exam} 章では、\soalt{} を起こす動詞の特徴を調査した。
その結果、互いに交替する項 (\sbj{}と \obj{})では、選択制限は完全に一致
するわけではないが、少なくとも一部の意味制限が一致する確率が高く、また、
\abs{}は自動詞側には対応する項がないが、その選択制限の出現傾向には非常
に偏りがあることがわかった。そこで本章では、これらの調査結果に基づいた
新しい結合価エントリの作成方法を提案する。


本提案手法では、交替動詞リストに載っており、
交替の片側にしか対応する結合価エントリが存在しない動詞に対し、存在する
方の既存のエントリを種 (seed)として用い、欠如している結合価エントリを
自動的に作成する。
本手法で自動的に作成された結合価エントリは、最終的には人手で修正する必
要があるにせよ、ベースとなる結合価エントリを獲得できると考えられる。




\subsection{結合価エントリの基本的作成方法} \label{sec:basic-method} 


基本的な結合価エントリの作成方法は下記の通りである。

\begin{enumerate}
\item 元の結合価エントリの、各項 N$_i$に対して、
  \begin{description}
   \item [\ if]\hspace{3mm} N$_i$ が交替する場合
     \begin{description}
     \item [\ \ if]\hspace{3mm} 他の項に対応する \textbf{then}
       対応する項に変更する
     \item [\ \ elseif]\hspace{3mm} 対応する項がない \textbf{then}
       削除する
       \end{description}
   \item [\ else]\hspace{3mm}
     そのまま複製する
     \end{description}
   \item 新しい結合価エントリに該当する交替における必須格が不足してい
     れば、デフォルトの必須格を追加する
 \end{enumerate}
 
 ここで、デフォルトとして用いる必須格の情報は、既存の同じ交替を取る結
 合価エントリの組合せの中で、該当する格として最も出現頻度が高いものを
 利用する。また、交替における必須格とは、\soalt{}の場合、\abs{}、\obj{}、
 \sbj{}である。
 




\subsection{日本語側: 結合価エントリ作成方法} \label{sec:Experimental_Method}


\paragraph{自動詞側作成方法} \label{sec:jap-int}



他動詞側エントリから自動詞側エントリを作成する方法を述べる。\obj{}と
\sbj{}が交替するので、\obj の選択制限を\sbj の選択制限として複製し、格
助詞を「ヲ格」から「ガ格」に変更する。\abs は対応する格役割がないため
削除し、それ以外の項は全てそのまま複製する。図
\ref{fig:kizutuku-kizutukeru} に、自動詞側エントリの作成例を示す。なお、
図\ref{fig:kizutuku-kizutukeru} から\ref{fig:odoroku-odorokasu} で、
[Seed Entry]は作成元の、[New Entry]は新しく作成する結合価エントリを示
している。また、図\ref{fig:kizutuku-kizutukeru}から
\ref{fig:odoroku-odorokasu}には、英語側の結合価情報作成例も併記してい
る(\ref{sec:create_eng} 章参照)。




\begin{figure*}[htb]
\begin{center}
  \begin{tabular}{ll}
    \framebox{\begin{minipage}[t]{0.40\textwidth}
      \begin{tabular}{ll}
        \multicolumn{2}{l}{New Entry ID : 700030}\\
        \\
        \sbj &┌ \textbf{{N1}}:\izj{人 動物} \node{sbj8}{が}\\           
        \ix &{├} \textbf{{N12}}:\izj{争い} \node{de8}{で}\\
        &└ \textbf{Vi}: 傷付く \\
 \\[-1ex]
      \sbj &{┌} \ul{\textbf{N1}} \gm{subject} \node{Esbj8}{} \\
      &├ \textbf{Cop} \eng{be} \textbf{Vp} {\ul{\eng{injured}}}\\
      \ix&{└} \textbf{PP} \ul{\eng{in} \textbf{N12}}  \node{Ede8}{}\\
    \end{tabular}
  \end{minipage}}
  &
  \framebox{\begin{minipage}[t]{0.45\textwidth}
      \begin{tabular}{ll}
        \multicolumn{2}{l}{Seed Entry ID : 760038}\\
 
\node{erg8}{\abs}&┌ \textbf{{N1}}: \izj{主体} が \\
\node{obj8}{\obj}&├ \textbf{{N2}}: \izj{人 動物} を \\
\node{de9}{\ix}&{├} \textbf{{N12}}:\izj{争い} で \\
&└ \textbf{Vt}: 傷付ける 
\\[-1ex]

\node{Eerg8}{\abs}&{┌} \ul{\textbf{N1}} \gm{subject}\\
&├ \textbf{Vt} \ul{\eng{injure}}\\
\node{Eobj8}{\obj}&{├} \ul{\textbf{N2}} \gm{direct object}\\
\node{Ede9}{\ix}&{└} \textbf{PP} \ul{\eng{in} \textbf{N12}}\\
\end{tabular}
  \end{minipage}}\\

  \end{tabular}
  \caption{自動詞側作成例 \iz{[passive]} 
    [New Entry] 傷付く\protect\eng{\sbj be injured in \ix} \lto 
    [Seed Entry] 傷付ける \protect\eng{\abs injure\obj in \ix}}
  \label{fig:kizutuku-kizutukeru}
\anodecurve[l][0]{de9}[r][0]{de8}{12mm}
\anodecurve[l][0]{obj8}[r][0]{sbj8}{15mm}
\anodecurve[l][0]{Ede9}[r][0]{Ede8}{10mm}
\anodecurve[l][0]{Eobj8}[r][0]{Esbj8}{12mm}
\end{center}


\begin{center}
  \begin{tabular}{ll}
    \framebox{\begin{minipage}[t]{0.40\textwidth}
  \begin{tabular}{ll}
    \multicolumn{2}{l}{Seed Entry ID : 504952}\\
    \\
    \sbj &┌ \textbf{{N1}}:\izj{食料 水} \node{sbj9}{が} \\           
    &└ \textbf{Vi}: 腐る \\
\\[-1ex]
    \sbj &{┌} \ul{\textbf{N1}} \gm{subject} \node{Esbj9}{}\\
    &└ \textbf{Vi} \node{surp1}{\ul{\eng{spoil}}}\\
  \end{tabular}
\end{minipage}}

\framebox{\begin{minipage}[t]{0.40\textwidth}
    \begin{tabular}{ll}
    \multicolumn{2}{l}{New Entry ID : 750039}\\
    \node{erg9}{\abs}&┌ \textbf{{N1}}: \izj{主体} が \\
    \node{obj9}{\obj}&├ \textbf{{N2}}: \izj{食料 水} を \\
    &└ \textbf{Vt}: 腐らす 
\\[-1ex]
    \node{Eerg9}{\abs} &{┌} \ul{\textbf{N1}} \gm{subject}\\
    &├ \textbf{Vt} \ul{\eng{spoil}}\\
    \node{Eobj9}{\obj} &{└} \ul{\textbf{N2}} \gm{direct object} \\
  \end{tabular}
\end{minipage}}
\end{tabular}

  \caption{他動詞側作成例  \iz{[\typeSO]}
    [Seed Entry] 腐る \protect\eng{\sbj spoil} \rto 
    [New Entry] 腐らす \protect\eng{\abs spoil \obj}}
  \label{fig:kusaru-kusarasu}
\anodecurve[r][0]{sbj9}[l][0]{obj9}{15mm}
\anodecurve[r][0]{Esbj9}[l][0]{Eobj9}{12mm}

\vspace{5mm}

  \begin{tabular}{ll}
    \framebox{\begin{minipage}[t]{0.40\textwidth}
  \begin{tabular}{ll}
\multicolumn{2}{l}{Seed Entry ID : 202204}\\
 
\sbj &┌ \textbf{{N1}}:\izj{主体 動物} が  \node{sbj2}{}   \\           
&{├} \textbf{{N3}}:\izj{*} \ul{に}    \node{ni}{}   \\
&└ \textbf{Vi}: 驚く 
\\[-1ex]
\sbj &{┌} \ul{\textbf{N1}} \gm{subject} \node{Esbj2}{}\\
&├ \textbf{Cop} \eng{be} \\
&| \textbf{Particle} \node{surp1}{\ul{\eng{surprised}}}\\
&{└} \textbf{PP} \eng{at/\ul{by} \textbf{N3}} \node{Eni}{}\\
\end{tabular}
  \end{minipage}}


\framebox{\begin{minipage}[t]{0.40\textwidth}
  \begin{tabular}{ll}
\multicolumn{2}{l}{New Entry ID : 760038}\\
 
\node{erg2}{\abs}&┌ \textbf{{N1}}: \izj{*} が                  \\
\node{obj2}{\obj}&├ \textbf{{N2}}: \izj{主体 動物} を \\
&└ \textbf{Vt}: 驚かす 
\\[-1ex]
\node{Eerg2}{\abs}&{┌} \ul{\textbf{N1}} \gm{subject}\\
&├ \textbf{Vt} \ul{\eng{surprise}}\\
\node{Eobj2}{\obj}&{└} \ul{\textbf{N2}} \gm{direct object}\\
\\
\end{tabular}
  \end{minipage}}\\
\end{tabular}
  \caption{他動詞側作成例2 \iz{[synthetic]}
    [Seed Entry] 驚く \protect\eng{\sbj be surprised at/by \ix} \rto 
    [New Entry] 驚かす \protect\eng{\abs (= \ix) surprise \obj}
    : 自動詞の「ニ格」を他動詞の\obj の選択制限にする}
  \label{fig:odoroku-odorokasu}
\end{center}
\anodecurve[r][0]{ni}[l][0]{erg2}{12mm}
\anodecurve[r][0]{sbj2}[l][0]{obj2}{15mm}
\anodecurve[r][0]{Eni}[l][0]{Eerg2}{10mm}
\anodecurve[r][0]{Esbj2}[l][0]{Eobj2}{12mm}
\end{figure*}



\paragraph{他動詞側作成方法} \label{sec:jap-trn}

自動詞側エントリから他動詞側エントリを作成する方法を述べる。自動詞の
\sbj の選択制限を\obj の選択制限として複製し、格助詞を「ガ格」から「ヲ
格」に変更する。それから、
\abs としてデフォルトの「\izj{主体} が」を加える。
これは、\ref{sec:AOS-compare} 章の調査で、「\izj{主体} が」が
\abs{}として、最も出現頻度が高かったためである。
図\ref{fig:kusaru-kusarasu} に作成例を示す。

\mpt{前提は? 対象とする結合価情報で必須な物は? TT}

ここまでは、日本語の格助詞と選択制限の情報のみを利用して新エントリを作
成しているが、既存の結合価辞書が英語の結合価情報も持っている場合には、
格役割 N$_i$が新エントリ側の他の格役割に対応するかどうかの判断に、
英語の結合価情報も利用できる。


例えば、自動詞「N1\izj{主体 動物}が \ul{N3\izj{*}に} 驚く」\eng{N1 be
  surprised at/\ul{by N3} }からは、「\ul{N1\izj{*}が} N2\izj{主体 動物}
を 驚かす」という他動詞が作成でき、自動詞の「ニ格」と他動詞の「ガ格」
が交替する(図\ref{fig:odoroku-odorokasu})。よって、自動詞側の格助詞が
「ニ格」に対応する英語の前置詞が\eng{by} の場合、自動詞側の「ニ格」の
選択制限を\abs{}の選択制限として複製し、「ガ格」とすることができる。こ
の場合は、他動詞側で不足する必須格がなくなるため、デフォルトの必須格を
追加する必要はない。


\subsection{英語側: 結合価エントリ作成方法} \label{sec:create_eng} 


{
\begin{figure}[hb]
  \begin{minipage}[t]{1.0\textwidth}
\border\\
\textbf{自動詞側作成方法:}
  \begin{itemize}
    \item 元の項構造が制御動詞(\eng{make,have,get,cause})を取る場合\footnote{例
  外として、主辞が制御動詞でない\eng{have} の場合、自動詞側は
  \eng{There is} 構文にする。例えば、「及ぼす」\eng{\abs have \obj on
    X} \rto  「及ぶ」  \eng{There be \sbj on X}。} 
        \begin{itemize}
        \item \abs Vc \obj Vi/Adj  \rto \sbj Vi/be Adj
        \iz{[synthetic]} \coml{\eng{\abs make \obj cry \rto \sbj cry}}

        \end{itemize} 

    \item 制御動詞を取らない場合 (元の主辞が Vt)
      \begin{itemize}
      \item 他動詞の主辞が\soalt を起こす場合
        \begin{itemize}
        \item \abs Vt \obj \rto \sbj Vi \iz{[\typeSO]} 
          \coml{\eng{\abs turn \obj \rto \sbj turn}}
          

        \end{itemize} 
      \item 起こさない場合
        \begin{itemize}
        \item \abs Vt \obj \rto \sbj be Vt-ed \iz{[passive]}
          \comll{(\eng{\abs injure \obj in \ix }}
            \comrr{\rto \sbj be injured in \ix )} 
            (図~\ref{fig:kizutuku-kizutukeru} 参照)
                                
       \end{itemize} 
      \end{itemize} 
    \end{itemize} 
\normalsize
\textbf{他動詞側作成方法:}  
  \begin{itemize}

  \item 元の項構造がbe + 形容詞の場合
    \begin{itemize}
    \item \sbj be Adj \rto 
    \abs Vc \obj Adj  \iz{[synthetic]} 
    \comll{(\eng{\sbj be surprised at/by \ix} }\\
     \comrr{\rto \eng{\abs (= \ix) make \obj surprised)}} 
     (図~\ref{fig:odoroku-odorokasu} 参照)
    \end{itemize}

  \item 元の項構造が他動詞の受身形の場合
    \begin{itemize}
    \item \sbj be Vt-ed \rto 
    \abs Vt \obj  \iz{[passive]} 
    \comll{(\eng{\sbj be defeated by \ix} }\\
    \comrr{\rto \eng{\abs (= \ix) defeat \obj)}} 
    \end{itemize}

  \item 元の項構造の主辞が自動詞の場合
  \begin{itemize}
  \item  自動詞の主辞が\soalt を起こす場合
    \begin{itemize}
    \item \sbj Vi 
      \rto \abs Vt \obj \iz{[\typeSO]}
       \coml{\eng{\sbj spoil \rto \abs spoil \obj}}
     \end{itemize} 

    \item 起こさない場合
    \begin{itemize}
    \item \sbj Vi  
      \rto \abs Vc \footnote{但し、制御動詞 Vc として\eng{make} を利用}
      \obj Vi  \iz{[synthetic]}
      \coml{\eng{\sbj rot \rto \abs make \obj rot}} 
      (図~\ref{fig:kusaru-kusarasu} 参照)
      
    \end{itemize}
  \end{itemize} 
  \end{itemize} 

\end{minipage}
\border
\caption{英語側作成方法}
\label{fig:mk-eng}
\end{figure}
}



英語側の結合価エントリの作成方法も、基本的には\ref{sec:basic-method}章
で述べた方法と同じである。但し、\altje の結合価辞書には、英語側には選
択制限の情報は付与されていない\footnote{但し、格役割が日本語側と対応す
  るので、格役割をキーとして日本語側に付与されている選択制限を参照でき
  る。}。そのため、英語側の結合価エントリの作成は、主辞にあわせた項構
造の変更が中心である。ここで、\ref{sec:alternations} 章の調査結果から、
英語側は\typeSO{}, \iz{passive}, \iz{synthetic}の3タイプで作成できるた
め、作成対象の英訳がどのタイプに分類されるかの判断が重要である。この判
断は、図\ref{fig:mk-eng} に示す場合わけにより行なっている。また、図
\ref{fig:mk-eng} の場合わけで、英語の主辞が\soalt を起こすかどうかの判
断にはLCS  データベース (EVCA+)
\citep[\url{http://www.umiacs.umd.edu/~bonnie/LCS_Database_Documentation.html}]{Dorr:1997}
を利用した。


EVCA+ とは、\cite{Levin:1993} によって行なわれた英
語の動詞分類 (EVCA) を元に拡張されたもので、4,432 動詞が492クラスに分
類されている。EVCA+から、\soalt{}を起こす動詞として、\eng{dissolve, spoil} 
など659動詞を抽出し\footnote{\soalt{}を起こす動詞としては、
Suffocate Verbs (40.7.ii),
Verbs of Light Emission (43.1.c),
Verbs of Sound Emission (43.2.d),
Verbs of Substance Emission (43.4.e),
Break Verbs (45.1.a, 45.1.b, 45.1.c),
Bend Verbs (45.2.a, 45.2.b, 45.2.c), 
Other Change of State Verbs (45.4.a, 45.4.b, 45.4.c),
Verbs of Entity-Specific Change of State (45.5),
Roll Verbs (51.3.1.a.i), 
Run Verbs (51.3.2.b.i) のクラスに分類されている動詞を用いた。}、
作成対象の英語の主辞がこれらの動詞に含まれるならば\soalt を起こすとし、
含まれないならば\soalt を起こさないとした。

例えば、自動詞作成例の図\ref{fig:kizutuku-kizutukeru} では、
元の項構造が制御動詞を取らず、かつ、主辞である \eng{injure} が
EVCA+ から抽出した\soalt{}を起こす動詞に含まれていないので、
図\ref{fig:mk-eng} の場合わけに基づき、
\iz{[passive]} タイプと判断する。


\clearpage

\section{交替情報に基づく結合価エントリの獲得実験と評価} \label{sec:eva}


\subsection{対象} \label{sec:experiment_target}

\mpt{Open textかどうか? NH}

本実験では、\soalt{}のみを対象とする。従って、実験対象となる動
詞は、他動詞側の結合価エントリがない自動詞、あるいは、自動詞側の結合価
エントリがない他動詞である。\soalt を起こす動詞の組み合わせは交替動詞
リストから抽出する。実験対象のエントリは、\ref{sec:AOS-compare} 章の調
査で用いたエントリとは別である。
一般に、語は複数の語義を持ち、同じ語であっても、語義によって交替を許す
場合と許さない場合がある。つまり、同じ語でも、語義によっては交替しない
エントリもある。
そこで、種として不適切なエントリを少しでも取り除くため、
本実験では、自動詞側の見出し語としてリストに登録されているにも関わらず
「ヲ格」を持つエントリと、「象は鼻が長い」のような「ハ格」と「ガ格」を
両方含むエントリは対象外とする。これにより4エントリが対象外となった。


本実験の対象エントリは、他動詞の81エントリ(25見出し語)と、自動詞の115
エントリ(37見出し語)、合計196 エントリ(62 見出し語)となった
\footnote{\altje の結合価辞書の構築の初期段階において、和語動詞は集中
  的に登録されたため、和語動詞の未登録語は非常に少ない。よって、一般的
  な辞書であれば、より多くのエントリが対象となると思われる。}。これは、
交替動詞リストの78組合せを占める。本実験では、他動詞のエントリから自
動詞のエントリを、自動詞のエントリから他動詞のエントリを作成した。


\subsection{翻訳による評価} \label{sec:eva-trans}


本章では、\ref{sec:create-method} 章で述べた方法で作成した結合価エント
リを翻訳によって評価した。新規に作成したエントリの動詞を対象に、新聞デー
タとWebページから1動詞につき2文を抽出し、評価対象文とした。評価対象文
は、自動詞作成側50文、他動詞作成側74文、合計124文である。翻訳は、日英
機械翻訳システム\altje で行なった。


作成したエントリを含む結合価辞書を利用した場合の翻訳結果(有)と、
作成したエントリを含まない結合価辞書を利用した場合の翻訳結果(無)を比較
し、(有)と(無)が全く同じ翻訳結果になった場合は「変化なし」に分類した。
それ以外の場合は、両言語に堪能な評価者によりどちらの翻訳結果がより良い
かの評価を行なっている。
但し、英訳のどちらが(有)か(無)か、評価者にはわからないようランダムに表示
し、\iz{A}、\iz{B}のラベルを張っている。
評価者は、翻訳結果を
(i) \iz{A}が\iz{B}より良い、(ii) \iz{A}と\iz{B}の翻訳品質は同等、
(iii) \iz{A}が\iz{B}より悪い、の3段階に評価した。
(\ref{s:70002})に評価例を示す。


\begin{exe}
\ex \label{s:70002}
 塩田喜代子さんは、毛布にくるまりながら。
\trans (\iz{A}) \texttt{Ms.\ Kiyoko Shioda is wrapped \ul{up to} a blanket.}
\trans (\iz{B}) \texttt{Ms.\ Kiyoko Shioda is wrapped \ul{in} a blanket.}
\end{exe}


(\ref{s:70002})では、評価は(iii)の「\iz{A}が\iz{B}より悪い」
になる。実際には、(\ref{s:70002})の\iz{A}は(無)、\iz{B}は(有)
なので、(有)の翻訳結果は(無)より良くなっている。


表\ref{tb:eva-trans} は評価結果である。
表\ref{tb:eva-trans} から、評価で最も割合が高いのは
「(有)が(無)より良くなった」の46.0\%である。
それに対し「(有)が(無)より悪くなった」は14.5\%であり、「良くなった」から
「悪くなった」を引くと31.5\%の改善となる。
すなわち、本提案手法によって作成した結合価エントリは、人手による修正を
全く行なわなくとも、機械翻訳システムにとって非常に有効である。


\begin{table*}[htbp]
\begin{center}
\begin{tabular}{l|rr|rr|rr}
 & \multicolumn{2}{|c|}{自動詞作成側} 
 & \multicolumn{2}{|c|}{他動詞作成側}
 & \multicolumn{2}{|c}{合計}\\
 & No. & \%  & No. & \%  &  No. & \% \\
\hline
(有)が(無)より良くなった   & 19 & 38.0 & 38 & 51.4 & 57 & 46.0 \\
同等     & 5  & 10.0 & 12 & 16.2 & 17 & 13.7 \\
変化なし  & 18 & 36.0 & 14 & 18.9 & 32 & 25.8 \\
(有)が(無)より悪くなった   & 8 & 16.0  & 10 & 13.5 & 18 & 14.5 \\
\hline
差分 (良$-$悪)      &  & +22.0  &  & +37.9 &  & +31.5 \\
\hline
合計 & 50 & 100.0 & 74 & 100.0 & 124 & 100.0 \\
\end{tabular}
\caption{翻訳による評価}
\label{tb:eva-trans}
\end{center}
\end{table*}


\subsection{問題事例の分析}\label{sec:eva-lex}


本節では、(有)が(無)より悪くなった場合の原因を分析する。
原因は大きく分類し、日本語作成側の問題と、英語作成側の問題に分類できる。

日本語作成側の問題では、
\mpt{1.全ての語義が交替するわけではない}
全ての語義が交替するわけではないという問題があげられる。例えば、「溶け
る」は次の二つの語義、(1)溶解する。固形物が液体になる。(2)液体に他の物
がまざって均一な液体になる。を持つ(広辞苑 第四版 CD-ROM版
\citep{koujien}より)。ここで、(2)の語義では、「溶ける」\tot{}「溶く」
の間で、「砂糖が\ul{溶ける}」\tot{}「私が砂糖を\ul{溶く}」のように
\soalt{}を起こす。しかし、(1)の語義では、「雪が\ul{溶ける}」\tot{}「*
私が雪を\ul{溶く}」のように他動詞側は非文となり、交替しない。また、
「雨が\ul{降る}」から「雨を\ul{降らす}」を作成する場合、\abs{}としては、
デフォルトとして用いた「\izj{主体} が」より、「\izj{空 雲} が」の方が
適切である。詳しい議論は、\ref{sec:eva-f} 章で行なう。

\mpt{交替する語義か交替しない語義かのふるい分けをどうすべきか?}



\mpt{英語の主辞が異なる}

英語作成側の問題では、
交替で作成した英訳より、異なる英語主辞が適切な場合があるという
問題がある。これには、元々同じ英語主辞では翻訳できない場合と、同じ英語
主辞でも翻訳はできるが、より適切な英訳が存在する場合とがある。英訳の改
良に関しては、\ref{sec:eng-alter-disccussion}章で詳しく議論する。




\section{議論と今後の課題}\label{sec:discuss}


本稿では、\soalt{} の特徴を量的に調査し、その結果に
基づき、交替を利用して詳細な結合価エントリを作成する手法を提案した。
\ref{sec:eva} 章までの結果から、本提案手法は有効であることがわかった。
さらに、2言語同時に結合価エントリを獲得した場合、翻訳に対して有効であ
ることがわかった。本章では、本提案手法の改良方法と展開方法等について議
論し、今後の課題について述べる。

\subsection{不適切な候補の削除}\label{sec:eva-f}
 


本手法の精度をあげるためには、語義単位で交替が行なえるかどうか、また、
特にデフォルトとして利用した選択制限が各エントリで適切かどうか、の判断
を自動的に行なう必要がある。このためには、まずフィルターとしてコーパス
を利用する方法が考えられる。この場合、もし、そのエントリに対応する文が
コーパス中になければ、そのエントリを取り除く。あるいは、頻出する意味カ
テゴリを用いて選択制限を修正する。この手法には、我々が対象とした和語動
詞は出現頻度が低いものが多いため、正しいエントリでも必ずしもコーパス中
に出現するとは限らないという問題点がある。本稿で実験対象とした和語動詞
の出現頻度は、新聞16年分で平均173回だけである。また、22動詞は日本語が
母国語の人にとっては馴染みのある語であるにもかかわらず、新聞16年分で一
度も出現していない。これらの動詞には、「吹き飛ぶ」「貼り付く」など複合
動詞が多く含まれている。もちろん、Webを利用できればこの問題に対応でき
る。例えば、検索エンジンgoogle\footnote{http://www.google.co.jp/}で検
索したところ、「吹き飛ぶ」は26,800 件、「貼り付く」は3,800 件検索され
た\footnote{検索日は2004/9/14。4,285,199,774 のウェブページから検索した。}。これら
を利用できれば十分なデータが得られると思われる。

\mpt{吹き飛ぶ
10,600 (gooq)
23,800 (google),
貼り付く1,460 (goo)
3,420(google)}

\mpt{22動詞の親密度チェック}

もう一つの対策としては、既存の交替ペアを正例とし、訓練データとして学習
することも考えられる。



\subsection{英訳の分類と改良方法} 
\label{sec:eng-alter-disccussion}


本章では、表\ref{tb:alternation-type} と同様に、自動作成した英訳
(以下、「作成後」という)の分類を行なった。
分類結果を表\ref{tb:alternation-type-compare} に示す。
表\ref{tb:alternation-type-compare} には、更に比較対象として、表
\ref{tb:alternation-type} で示した分類結果を再掲する。

{
\setlength{\tabcolsep}{4pt} 
\begin{table*}[htb]
\begin{tabular}{lll|rr|rr|rr}
& \multicolumn{2}{c|}{英語構造} 
& \multicolumn{2}{c|}{参考データ (表\ref{tb:alternation-type})} 
& \multicolumn{2}{c|}{\jita{自動詞作成側}} 
& \multicolumn{2}{c}{\jita{他動詞作成側}} \\

タイプ & \jita{自動詞} & \jita{他動詞} &
 数 & (\%) &
 数 & (\%) &
 数 & (\%)\\
\hline
\hline
\typeSO{} &
\eng{\sbj Vi} &
\eng{\abs Vt \obj} & 
138 & 30.0 &
9 & 11.1 &
24 & 21.7 \\
\iz{passive} &
  \eng{\sbj be Vt-ed} &
 \eng{\abs Vt \obj} & 
91 & 19.8 &
71 & 87.7 & 
14 & 12.2 \\
\iz{synthetic} &
  \eng{\sbj Vi/be Adj} &
\eng{\abs Vc \obj Vi/Adj} & 
30 & 6.5 &
0 & 0 & 
76 & 66.1 \\
\hline
\multicolumn{3}{l|}{主辞が異なる} & 
191 & 41.5 &
0 & 0.0 &
0 & 0.0 \\
\hline
\multicolumn{3}{l|}{構造が異なる} & 
10 & 2.2 & 
1 & 1.2 & 
0 & 0.0 \\
\hline
\hline
合計 & & &
460 & 100 &
81 & 100 & 
115 & 100 \\
\end{tabular}
\caption{作成後/修正後の英語側結合価エントリと参考データの英語構造の比較} 
\label{tb:alternation-type-compare}
\end{table*}
}


まず、表\ref{tb:alternation-type-compare}から、\iz{synthetic}タイプは、
他動詞から自動詞を作成する場合では皆無だが、自動詞から他動詞を作成する
場合では非常に多いことがわかる。他動詞から自動
詞を作成する場合、作成元の他動詞エントリでは制御動詞は全く使われていなかっ
た。一般に、辞書編集者がエントリを作成する場合、合成的なエントリよりも
簡単なエントリを作成する傾向がある。交替動詞リストは言語学者と辞書から
作成したデータだが、そのデータでは\iz{synthetic}タイプは約6.5\%であり、
皆無ではないが多くもない。一方、自動作成では、\iz{synthetic}タイプは76エ
ントリ(66.1\%)を占め、他のどのタイプよりも多い。
この内、約36\%程度は修正が必要だと思われる。例えば、作成元の自動詞側の
英訳が \eng{N1 be exhausted} で、既存の辞書では\eng{exhausted} が形容
詞として定義されている場合、作成する英訳は\eng{N1 make N2
  exhausted$_{\textnormal{adj}}$} となる。しかし実際は、他動詞の
\eng{exhaust} があるので、\eng{N1 exhaust N2} の方が好ましい。このため、
本稿で述べた手法に、形容詞と過去分詞が同じ形なら、形容詞を動詞に変換し
て利用できないかを調べるアルゴリズムを追加することが考えられる。


最後に、日本語の他動詞と自動詞が交替するが、英語側では異なる主辞を必要
とする場合について考察する。表\ref{tb:alternation-type-compare}の参考
データでは、交替の41.5\%で異なる英語主辞が用いられているが、本提案手法
では、異なる英語主辞は利用できない。しかし、作成したエントリの
14\%程度は異なる英語主辞を用いた方が適切だと考えられる。
これには例えば、「\sbj が亡くなる」  \eng{\sbj pass away} \tot  「\abs 
が \obj  を亡くす」 \eng{\abs  lose \obj} (\eng{My friend passed away}
\tot \eng{I lost my friend})があげられる。他動詞としての\eng{pass
  away} も、自動詞としての\eng{lose} も存在するが、語義が異なり、「亡
くす」「亡くなる」の訳としては不適切である。この問題には本提案手法では
対応できない。信頼性の高い英語の統語データがあったとしても、他動詞とし
ての\eng{pass away} や自動詞としての\eng{lose} を規則的に排除すること
は難しい。これを自動化するには、意味によって項構造をリンクしていて、か
つ、日本語の動詞の意味にもリンクされているようなデータが必要である。こ
れは、Papillonプロジェクト
\footnote{\url{http://www.papillon-dictionary.org/}}で構築されているよ
うなより大規模な多言語辞書を用いることで解決できると期待される。


まとめると、英語側の作成精度をより高くするには、形容詞と過去分詞が同形
でないかのチェック、語義による用法の異なり、特に交替についてのより詳し
い英語側の情報が必要である。


\subsection{語彙規則としての利用} \label{sec:lr}

本稿では、辞書構築における交替の利用について詳細な調査を行なった。本稿
で提案した手法は、語彙規則や翻訳規則としてシステムで直接利用できる。例
えば、\citet{Shirai:Bond:Nozawa:Sasaki:Ueda:1999j}は、結合価辞書に登録
されている結合価エントリと、各エントリを様々な形に展開する規則を用い、
使役の受身や被害の受身の翻訳を行なっている。また、
\citet{Trujillo:1995} は、語彙の翻訳に語彙規則を利用することを提案して
いる。つまり、個々の言語に対して用意した語彙規則により、語彙の展開を行
ない、更に、2言語間の語彙規則同士のリンクを作成して翻訳に利用する。本
稿で提案した手法も、同様に語彙規則や翻訳規則として利用できる。


しかし、同じ見出し語でもすべての語義で交替するわけではない
(\ref{sec:eva-lex} 章参照)。
また、目的言語側も必ずしも同じ主辞を用いて規則的に翻訳できるわけではない
(\ref{sec:alternations}、\ref{sec:eng-alter-disccussion} 章参照)。
そのため、本稿で述べてきたように、辞書のエントリとして作成・保存し、
そのエントリを修正して利用する方が、より精度の高い処理が行なえる。
更に、本手法による結合価エントリの獲得には、不要な揺れや
矛盾を減らすことができるという利点がある。
つまり、図\ref{fig:toku-tokeru} の\ix{}の
選択制限(\izj{具体物}\tot \izj{無生物})
に見られるような、
不必要な選択制限の揺れや矛盾を減らし、その上で、必要な修正を加えることで、
より一貫した辞書構築を行なうことができる(\ref{sec:AOS-compare} 章参照)。
また、辞書の形式にすることで、他のシステムにも適用しやすくなる。
しかし一方、全ての交替のエントリを網羅的に結合価辞書に登録できているわけ
ではないため、交替の片側の結合価エントリが辞書にない場合や、
新しい見出し語の結合価エントリを追加した場合などには、
語彙規則や翻訳規則で対応できるようにすることが必要である。


\subsection{提案手法の展開} 

\mpt{作成した数が少なかった原因}


本稿では和語動詞のみを対象として、結合価エントリを作成した。しかし、
\altje の結合価辞書は、構築の初期段階において、和語動詞を集中的に登録
したため、未登録の和語動詞は非常に少なかった。しかし、この辞書に対して
も、交替の片側にのみ対応するエントリが存在する79組の\soalt を起こす動
詞組み合わせのうち、78組に対してエントリを獲得できた。そのため、\soalt 
の組合せに対するカバー率を、68.5\%から85.4\% (315+78/460)へと増やすこ
とができた。他の多くの結合価辞書や新しい言語対の結合価辞書では、よりカ
バー率が低いことが予想されるため、より多くのエントリが対象となると思わ
れる。また、交替の片側を人手で作成し、本手法により自動的に交替の残りの
エントリを作成することも考えられる。


\mpt{対象を増やすための手段}

また、\soalt にはサ変動詞も多く含まれる。例えば「製品が\ul{完売した}」
\tot{}「店が製品を\ul{完売した}」 などである。\altje の日本語辞書には
自他動詞の品詞を付与されたサ変動詞が約2,400見出し語載っており、
そのうち、約400見出し語の結合価エントリが結合価辞書に登録されている。
但し、これらのサ変動詞に関する交替動詞リストはなく、人手で作成するとコ
ストがかかる。しかし、サ変動詞は和語動詞と異なり、自動詞と他動詞で形態
変化をしないので、\soalt{}を起こすかどうかを自動、あるいは、半自動的的
に判断できると考えられる。

まず、サ変動詞が自動詞と他動詞の両方の用法を持っているかどうかは、
\altje の日本語辞書や茶筌\cite{chasen:2.3.3j}
等の品詞情報から判
断できる。但し、自動詞と他動詞の両方の用法を持つサ変動詞であっても、
\soalt{}以外の交替、例えば、「私が晩ご飯を\ul{料理した}」 \tot 「私が
\ul{料理した}」 を起こすものも多い(この交替を、\soalt{}に対して
\saalt{}と呼ぶ)。
そこで、これらのサ変動詞が\soalt{}を起こすかどうかの自動的な判断方法を以下に2つあげる。

一つは、英訳を用いる方法である。まず、サ変動詞の英訳を日英の対訳辞書か
ら取り出し、その英訳がEVCA+等のデータベースで\soalt{}を起こす動詞とし
て分類されていれば(\ref{sec:create_eng} 章参照)、日本語側のサ変動詞も
\soalt{}を起こすと判断する方法である。もう一つの方法は、コーパスの解析
結果を利用する方法である。この場合、\soalt{}と\saalt{}の両交替を起こす
と仮定した場合の両タイプの結合価エントリを作成し、コーパスの解析に利用
する。その結果、作成した結合価エントリのうち、より多く実際の解析に用い
られた方の交替を起こすと判断する方法である。つまり、\soalt{}を起こすと
仮定して作成した結合価エントリばかりが解析に利用されており、\saalt{}を
起こすと仮定して作成した結合価エントリは全く利用されていないならば、そ
のサ変動詞は\soalt{}を起こすと判断できる。こうした判断方法を用いて、
今後は、サ変動詞を中心に本手法を適用していきたい。

なお、和語動詞に対しても、同じ語幹を持つ自動詞と他動詞に対して上記の判
断方法を利用すれば、最終的に人手による確認が必要であるとしても、始めか
ら人手で作成するより容易に交替動詞リストを作成できると思われる。


\mpt{他の交替や和語動詞でも、自動的に交替動詞リストを拡張できないか? NH}

また、本稿では、交替動詞リストとして\citet{Jacobsen:1981}と
\citet{Bullock:1999}の\soalt{}を起こす動詞のリストを利用した。こうした
リストは\soalt{}に限らず、言語学者等によって他にも作成されている
\citep{Oishi:Matsumoto:1997,Furumaki:Tanaka:2003,McCarthy:2000}。これ
らのリストを活用し、他の種類の交替に対しても本手法を展開してきたい。


\section{まとめ} \label{sec:conclusion}


本稿では、まず、交替についての調査・分析を行なった。また、その調査結果
に基づき、交替情報を利用して既存の結合価辞書に不足しているエントリを補
い、交替関係を付与する方法を提案した。対象とした交替は、自動詞の主語が
他動詞の目的語となる\soalt{}である。

交替についての調査・分析では、日本語の\soalt について選択制限の対応関
係と、日本語が\soalt を起こす場合の英語の交替変化を分類した。この結果、
日本語の\soalt では、\sbj と\obj の選択制限の一致率が高いことと、\abs 
の主体性が高いことがわかった。また、日本語が\soalt を起こす場合、英語は
56\%が規則的な交替変化を行ない、そのうち\soalt を起こす割合は30\%である
ことがわかった。



結合価情報の獲得実験では、上述の交替についての分析結果を踏まえ、交替情
報と既存の結合価辞書から、比較的単純な置き換えにより、新しい結合価情報
を自動的に獲得する方法を提案し、実験を行なった。具体的には、既存の結合
価辞書が交替の自動詞側のエントリのみを持つ場合、他動詞側のエントリを自
動的に作成し、他動詞側のエントリのみを持つ場合、自動詞側のエントリを作
成する実験を行なった。本実験では、日本語と英語の結合価エントリを同時に
獲得した。

これにより、和語動詞の\soalt
を83\%からほぼ100\%カバーすることができた。また翻訳評価の結
果、翻訳結果の32\%が改善できた。すなわち、本提案手法は人手による修正を
行なわなくとも、翻訳に対して有効であることを示した。



\subsection*{謝辞}
 
日頃熱心にご討論いただいている、中岩浩巳グループリーダーを始めとする
NTTコミュニケーション科学基礎研究所自然言語研究グループの皆様、および、
奈良先端科学技術大学院大学松本研究室の皆様に感謝致します。また特に、本
稿をまとめるにあたり多くのコメントをいただきました、田中 貴秋氏、
Timothy Baldwin氏、成山 重子氏に感謝致します。




\bibliographystyle{jnlpbbl}
\begin{thebibliography}{}

\bibitem[\protect\BCAY{Baldwin, Bond, \BBA\ Hutchinson}{Baldwin
  et~al.}{1999}]{Baldwin:1999b}
Baldwin, T., Bond, F., \BBA\ Hutchinson, B. \BBOP 1999\BBCP.
\newblock \BBOQ A Valency Dictionary Architecture for Machine Translation\BBCQ\
\newblock In {\Bem Eighth International Conference on Theoretical and
  Methodological Issues in Machine Translation: TMI-99}, \BPGS\ 207--217\
  Chester, UK.

\bibitem[\protect\BCAY{Bond, Baldwin, 藤田}{Bond\Jetal
  }{2002}]{Bond:Baldwin:Fujita:2002j}
Bond, F., Baldwin, T., 藤田早苗 \BBOP 2002\BBCP.
\newblock \BBOQ Detecting Alternation Instances in a Valency Dictionary\BBCQ\
\newblock \Jem{言語処理学会第8回年次大会}, 519--522.

\bibitem[\protect\BCAY{Bond \BBA\ Fujita}{Bond \BBA\
  Fujita}{2003}]{Bond:Fujita:2003}
Bond, F.\BBACOMMA\  \BBA\ Fujita, S. \BBOP 2003\BBCP.
\newblock \BBOQ Evaluation of a Method of Creating New Valency Entries\BBCQ\
\newblock In {\Bem {MT} Summit {IX}}, \BPGS\ 16--23\ New Orleans.
\newblock
  (\url{http://www.amtaweb.org/summit/MTSummit/FinalPapers/80-Bond-final.pdf}).

\bibitem[\protect\BCAY{Breen}{Breen}{1995}]{Breen:1995}
Breen, J.~W. \BBOP 1995\BBCP.
\newblock \BBOQ Building an electronic {Japanese-English} dictionary\BBCQ\
\newblock Japanese Studies Association of Australia Conference
  (\url{http://www.csse.monash.edu.au/~jwb/jsaa_paper/hpaper.html }).

\bibitem[\protect\BCAY{Breen}{Breen}{2004}]{Breen:2004}
Breen, J.~W. \BBOP 2004\BBCP.
\newblock \BBOQ {JMDict}: a {Japanese}-Mulitlingual Dictionary\BBCQ\
\newblock In {\Bem Coling 2004 Workshop on Multilingual Linguistic Resources},
  \BPGS\ 71--78\ Geneva.

\bibitem[\protect\BCAY{Bullock}{Bullock}{1999}]{Bullock:1999}
Bullock, B. \BBOP 1999\BBCP.
\newblock \BBOQ Alternative sci.lang.japan Frequently Asked Questions\BBCQ\
\newblock \url{http://www.csse.monash.edu.au/~jwb/afaq/jitadoushi.html}.

\bibitem[\protect\BCAY{Carroll, Minnen, \BBA\ Briscoe}{Carroll
  et~al.}{1998}]{Carroll:Minnen:Briscoe:1998}
Carroll, J., Minnen, G., \BBA\ Briscoe, T. \BBOP 1998\BBCP.
\newblock \BBOQ Can subcategorisation probabilities help a statistical
  parser?\BBCQ\
\newblock In {\Bem ACL/SIGDAT-1998}, \BPGS\ 118--126\ Montreal.

\bibitem[\protect\BCAY{Copestake, Flickinger, Pollard, \BBA\ Sag}{Copestake
  et~al.}{1999}]{Copestake:Flickinger:Pollard:Sag:1999}
Copestake, A., Flickinger, D., Pollard, C., \BBA\ Sag, I.~A. \BBOP 1999\BBCP.
\newblock \BBOQ Minimal Recursion Semantics: An Introduction\BBCQ\
\newblock (manuscript
  \url{http://www-csli.stanford.edu/~aac/papers/newmrs.ps}).

\bibitem[\protect\BCAY{Dixon}{Dixon}{1991}]{Dixon:1991}
Dixon, R. M.~W. \BBOP 1991\BBCP.
\newblock {\Bem A New Approach to {English} Grammar, on Semantic Principles}.
\newblock Oxford University Press, Oxford.

\bibitem[\protect\BCAY{Dorr}{Dorr}{1997}]{Dorr:1997}
Dorr, B.~J. \BBOP 1997\BBCP.
\newblock \BBOQ Large-Scale Dictionary Construction for Foreign Language
  Tutoring and Interlingual Machine Translation\BBCQ\
\newblock {\Bem Machine Translation}, {\Bbf 12}  (4), 271--322.

\bibitem[\protect\BCAY{Fujita \BBA\ Bond}{Fujita \BBA\
  Bond}{2002}]{Fujita:Bond:2002a}
Fujita, S.\BBACOMMA\  \BBA\ Bond, F. \BBOP 2002\BBCP.
\newblock \BBOQ Extending the Coverage of a Valency Dictionary\BBCQ\
\newblock In {\Bem COLING-2002 workshop on Machine Translation in Asia}, \BPGS\
  67--73\ Taipei.
\newblock
  (\url{http://acl.ldc.upenn.edu/coling2002/workshops/data/w07/w07-08.pdf}).

\bibitem[\protect\BCAY{Furumaki \BBA\ Tanaka}{Furumaki \BBA\
  Tanaka}{2003}]{Furumaki:Tanaka:2003}
Furumaki, H.\BBACOMMA\  \BBA\ Tanaka, H. \BBOP 2003\BBCP.
\newblock \BBOQ The Consideration of $<$N-suru$>$ for Construction of the
  Dynamic Lexicon\BBCQ\
\newblock In {\Bem 9th Annual Meeting of The Association for Natural Language
  Processing}, \BPGS\ 298--301.
\newblock (in Japanese).

\bibitem[\protect\BCAY{Hong, Kim, Park, \BBA\ Lee}{Hong
  et~al.}{2004}]{Hong:Kim:Park:Lee:2004}
Hong, M., Kim, Y.-K., Park, S.-K., \BBA\ Lee, Y.-J. \BBOP 2004\BBCP.
\newblock \BBOQ Semi-Automatic Construction of {Korean}-{Chinese} Verb Patterns
  Based on Translaiton Equivalency\BBCQ\
\newblock In {\Bem Coling 2004 Workshop on Multilingual Linguistic Resources},
  \BPGS\ 87--92\ Geneva.

\bibitem[\protect\BCAY{Ikehara, Shirai, Yokoo, \BBA\ Nakaiwa}{Ikehara
  et~al.}{1991}]{Ikehara:1991}
Ikehara, S., Shirai, S., Yokoo, A., \BBA\ Nakaiwa, H. \BBOP 1991\BBCP.
\newblock \BBOQ Toward an {MT} System without Pre-Editing -- Effects of New
  Methods in {{\bf ALT-J/E}}--\BBCQ\
\newblock In {\Bem Third Machine Translation Summit: MT Summit III}, \BPGS\
  101--106\ Washington DC.
\newblock (\url{http://xxx.lanl.gov/abs/cmp-lg/9510008}).

\bibitem[\protect\BCAY{Jacobsen}{Jacobsen}{1981}]{Jacobsen:1981}
Jacobsen, W. \BBOP 1981\BBCP.
\newblock {\Bem Transitivity in the Japanese Verbal System}.
\newblock Ph.D.\ thesis, University of Chicago.
\newblock (Reproduced by the Indiana University Linguistics Club, 1982).

\bibitem[\protect\BCAY{Kawahara \BBA\ Kurohashi}{Kawahara \BBA\
  Kurohashi}{2001}]{Kawahara:Kurohashi:2001}
Kawahara, D.\BBACOMMA\  \BBA\ Kurohashi, S. \BBOP 2001\BBCP.
\newblock \BBOQ Japanese Case Frame Construction by Coupling the Verb and its
  Closest Case Component\BBCQ\
\newblock In {\Bem Proceedings of First International Conference on Human
  Language Technology Research (HLT 2001)}, \BPGS\ 204--210\ San Diego.

\bibitem[\protect\BCAY{Kilgarriff}{Kilgarriff}{1993}]{Kilgarriff:1993}
Kilgarriff, A. \BBOP 1993\BBCP.
\newblock \BBOQ Inheriting Verb Alternations\BBCQ\
\newblock In {\Bem Sixth Conference of the European Chapter of the ACL
  (EACL-1993)}, \BPGS\ 213--221\ Utrecht.
\newblock (\url{http://acl.ldc.upenn.edu/E/E93/E93-1026.pdf}).

\bibitem[\protect\BCAY{Korhonen}{Korhonen}{2002}]{Korhonen:2002}
Korhonen, A. \BBOP 2002\BBCP.
\newblock \BBOQ Semantically Motivated Subcategorization Acquisition\BBCQ\
\newblock In {\Bem Proceedings of the ACL Workshop on Unsupervised Lexical
  Acquisition}\ Philadelphia, USA.

\bibitem[\protect\BCAY{Levin}{Levin}{1993}]{Levin:1993}
Levin, B. \BBOP 1993\BBCP.
\newblock {\Bem English Verb Classes and Alternations}.
\newblock University of Chicago Press, Chicago, London.

\bibitem[\protect\BCAY{Li \BBA\ Abe}{Li \BBA\ Abe}{1998}]{Li:Abe:1998}
Li, H.\BBACOMMA\  \BBA\ Abe, N. \BBOP 1998\BBCP.
\newblock \BBOQ Generalizing Case Frames Using a Thesaurus and the {MDL}
  principle\BBCQ\
\newblock {\Bem Computational Linguistics}, {\Bbf 24}  (2), 217--244.

\bibitem[\protect\BCAY{Manning}{Manning}{1993}]{Manning:1993}
Manning, C.~D. \BBOP 1993\BBCP.
\newblock \BBOQ Automatic Acquisition of a large subcategorization dictionary
  from corpora\BBCQ\
\newblock In {\Bem 31st Annual Meeting of the Association for Computational
  Linguistics: ACL-93}, \BPGS\ 235--242.

\bibitem[\protect\BCAY{McCarthy}{McCarthy}{2000}]{McCarthy:2000}
McCarthy, D. \BBOP 2000\BBCP.
\newblock \BBOQ Using Semantic Preferences to Identify Verbal Participation in
  Role Switching Alternations\BBCQ\
\newblock In {\Bem Proceedings of the first Conference of the North American
  Chapter of the Association for Computational Linguistics. (NAACL)}\ Seattle,
  WA.

\bibitem[\protect\BCAY{Oishi \BBA\ Matsumoto}{Oishi \BBA\
  Matsumoto}{1997}]{Oishi:Matsumoto:1997}
Oishi, A.\BBACOMMA\  \BBA\ Matsumoto, Y. \BBOP 1997\BBCP.
\newblock \BBOQ Detecting the Organization of Semantic Subclasses of {Japanese}
  Verbs\BBCQ\
\newblock {\Bem International Journal of Corpus Linguistics}, {\Bbf 2}  (1),
  65--89.

\bibitem[\protect\BCAY{Probst}{Probst}{2003}]{Probst:2003}
Probst, K. \BBOP 2003\BBCP.
\newblock \BBOQ Using `smart' bilingual projection to feature-tag a monolingual
  dictionary\BBCQ\
\newblock In {\Bem Proceedings of CoNLL 2003}.

\bibitem[\protect\BCAY{Sasaki, Isozaki, Suzuki, Kokuryou, Hirao, Kazawa, \BBA\
  Maeda}{Sasaki et~al.}{2004}]{Sasaki:2004}
Sasaki, Y., Isozaki, H., Suzuki, J., Kokuryou, K., Hirao, T., Kazawa, H., \BBA\
  Maeda, E. \BBOP 2004\BBCP.
\newblock \BBOQ {SAIQA-II: A Trainable Japanese QA System with SVM}\BBCQ\
\newblock {\Bem IPSJ}, {\Bbf 45}  (2), 635--646.
\newblock (in Japanese).

\bibitem[\protect\BCAY{Takahashi, Iwakura, Iida, Fujita, \BBA\ Inui}{Takahashi
  et~al.}{2001}]{Takahashi:01}
Takahashi, T., Iwakura, T., Iida, R., Fujita, A., \BBA\ Inui, K. \BBOP
  2001\BBCP.
\newblock \BBOQ \textsc{Kura}: a transfer-based lexico-structural paraphrasing
  engine\BBCQ\
\newblock In {\Bem Proceedings of the 6th Natural Language Processing Pacific
  Rim Symposium {\rm (}NLPRS{\rm )} Workshop on Automatic Paraphrasing:
  Theories and Applications}, \BPGS\ 37--46.
\newblock \url{http://cl.aist-nara.ac.jp/kura/doc/}.

\bibitem[\protect\BCAY{Trujillo}{Trujillo}{1995}]{Trujillo:1995}
Trujillo, A. \BBOP 1995\BBCP.
\newblock \BBOQ Bi-Lexical Rules for Multi-Lexeme Translation in Lexicalist
  {MT}\BBCQ\
\newblock In {\Bem Sixth International Conference on Theoretical and
  Methodological Issues in Machine Translation: TMI-95}, \BPGS\ 48--66.

\bibitem[\protect\BCAY{Uszkoreit}{Uszkoreit}{2002}]{Uszkoreit:2002}
Uszkoreit, H. \BBOP 2002\BBCP.
\newblock \BBOQ New Chances for Deep Linguistic Processing\BBCQ\
\newblock In {\Bem 19th International Conference on Computational Linguistics:
  COLING-2002}, \BPGS\ XIV--XXVII\ Taipei.

\bibitem[\protect\BCAY{Utsuro, Miyata, \BBA\ Matsumoto}{Utsuro
  et~al.}{1997}]{Utsuro:1997}
Utsuro, T., Miyata, T., \BBA\ Matsumoto, Y. \BBOP 1997\BBCP.
\newblock \BBOQ Maximum Entropy Model Learning of Subcategorization
  Preference\BBCQ\
\newblock In {\Bem Proc. of the 5th Workshop on Very Large Corpora}, \BPGS\
  246--260.

\bibitem[\protect\BCAY{Yamura-Takei, Fujiwara, Yoshie, \BBA\
  Aizawa}{Yamura-Takei et~al.}{2002}]{Yamura-Takei:Fujiwara:Yoshie:Aizawa:2002}
Yamura-Takei, M., Fujiwara, M., Yoshie, M., \BBA\ Aizawa, T. \BBOP 2002\BBCP.
\newblock \BBOQ Automatic Linguistic Analysis for Language Teachers: The Case
  of Zeros\BBCQ\
\newblock In {\Bem 19th International Conference on Computational Linguistics:
  COLING-2002}, \BPGS\ 1114--1120\ Taipei.

\bibitem[\protect\BCAY{池原, 宮崎, 白井, 横尾, 中岩, 小倉, 大山, 林}{池原\Jetal
  }{1997}]{GoiTaikeij}
池原悟, 宮崎雅弘, 白井{諭}, 横尾昭男, 中岩浩巳, 小倉健太郎, 大山芳史, 林良彦
  \BBOP 1997\BBCP.
\newblock \Jem{日本語語彙大系}.
\newblock 岩波書店.

\bibitem[\protect\BCAY{新村}{新村}{1996}]{koujien}
新村出\JED\ \BBOP 1996\BBCP.
\newblock \Jem{広辞苑 第四版 (CD-ROM)}.
\newblock 岩波書店.

\bibitem[\protect\BCAY{野村}{野村}{2002}]{Nomura:2002j}
野村直之 \BBOP 2002\BBCP.
\newblock \Jem{機械翻訳用の認知科学的辞書と情報検索・要約技術に関す る研究}.
\newblock Ph.D.\ thesis, 九州大学.

\bibitem[\protect\BCAY{白井}{白井}{1999}]{Shirai:1999zj}
白井諭 \BBOP 1999\BBCP.
\newblock \JBOQ 単文の結合価パターンの網羅的収集に向けて -日英機械翻
  訳の観点から-\JBCQ\
\newblock In {\Bem NLP Symposium}.
\newblock \url{www.kinet-tv.ne.jp/~sat/data/publications/1999/s29.html}.

\bibitem[\protect\BCAY{白井, Bond, 野沢, 佐々木~富, 上田洋美}{白井\Jetal
  }{1999}]{Shirai:Bond:Nozawa:Sasaki:Ueda:1999j}
白井諭, Bond, F., 野沢弥生, 佐々木~富子, 上田洋美 \BBOP 1999\BBCP.
\newblock \JBOQ 入力文と結合価パターン対辞書の照合に関する一手法\JBCQ\
\newblock \Jem{言語処理学会第5回年次大会}, \BPGS\ 80--83. 自然言語処理学会.

\bibitem[\protect\BCAY{松本, 北内, 山下, 平野, 松田, 高岡, 浅原}{松本\Jetal
  }{2003}]{chasen:2.3.3j}
松本裕治, 北内啓, 山下達雄, 平野善隆, 松田寛, 高岡一馬, 浅原正幸 \BBOP
  2003\BBCP.
\newblock \Jem{形態素解析システム「茶筌」 version 2.3.3 使用説明書}.
\newblock \url{http://chasen.naist.jp/hiki/ChaSen/}.

\bibitem[\protect\BCAY{郡司}{郡司}{2002}]{Gunji:2002}
郡司隆男 \BBOP 2002\BBCP.
\newblock \Jem{単語と文の構造}.
\newblock 現代言語学入門. 岩波書店.

\end{thebibliography}




\begin{biography}
\biotitle{略歴}

\bioauthor{藤田早苗}{
1997年 大阪府立大学工学部航空宇宙工学科卒業。
1999年 奈良先端科学技術大学院大学情報科学研究科博士前期課程修了。
同年4月よりNTT日本電信電話株式会社コミュニケーション科学基礎研究勤務。
以来、自然言語処理の研究に従事。
また、2003年4月より奈良先端科学技術大学院大学情報科学研究科博士後期課
程に社会人学生として在学中。
ACL,言語処理学会各会員。

{\tt email: sanae@cslab.kecl.ntt.co.jp}}


\bioauthor{Francis Bond(フランシス ボンド)}{
1988年 B.A. (University of Queensland)。
1990年 B.E. (Hons) (同大学)。
1991年 日本電信電話株式会社入社。
以来、計算機言語学、自然言語処理、特に機械翻訳の研究に従事。
1999年 CSLI, Stanford大学客員研究員。
2001年 Ph.D. (University of Queensland)。
2005年 3ヶ月間Oslo大学招聘研究員。
現在、NTTコミュニケーション科学基礎研究所主任研究員。
著書「Translating the Untranslatable」 CSLI Publications にて日英機械翻訳に
おける数・冠詞の問題を扱う。
ACL,ALS,言語処理学会各会員。

{\tt email: bond@cslab.kecl.ntt.co.jp}}

\bioreceived{受付}
\biorevised{再受付}
\bioaccepted{採録}
\end{biography}


\end{document}


