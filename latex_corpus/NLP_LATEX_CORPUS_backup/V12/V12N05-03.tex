\documentstyle[epsf,jnlpbbl]{jnlp_j}

\setcounter{page}{3}
\setcounter{巻数}{2}
\setcounter{号数}{3}
\setcounter{年}{1995}
\setcounter{月}{7}
\受付{1995}{5}{6}
\再受付{1995}{7}{8}
\採録{1995}{9}{10}

\setcounter{secnumdepth}{2}

\title{動詞待遇表現に対する丁寧さの印象に関する定量的分析\\
−接頭辞オを用いた表現と接頭辞ゴを用いた表現との比較−}
\author{丸元 聡子\affiref{IBS} \and  白土 保\affiref{NICT} \and 井佐原 均\affiref{NICT}}

\headauthor{丸元聡子・白土保・井佐原均}
\headtitle{動詞待遇表現に対する丁寧さの印象に関する定量的分析}

\affilabel{IBS}{(財)計量計画研究所}
{The Institute of Behavioral Sciences}

\affilabel{NICT}{(独)情報通信研究機構}
{National Institute of Information and Communications Technology}

\jabstract{
 日本語は,各品詞にわたって待遇表現が発達している.中でも動詞に関する待遇表現は多岐にわたるが,待遇表現:「接頭辞オ+本動詞+補助動詞(オ〜型表現)」,および「接頭辞ゴ+本動詞+補助動詞(ゴ〜型表現)」は,日本語の動詞待遇表現における主要な表現である.両表現の違いについては,オに続く本動詞が和語であり,ゴに続く本動詞が漢語であるということが,従来の言語学的研究で指摘されてきた.しかし,両表現の言語心理的な違いを定量的に調べた研究は,これまで殆どなかった.\\
 今回,我々は,Scheffeの一対比較法を用いてオ〜型表現,およびゴ〜型表現に対して人々が感じる丁寧さの程度を数値化した上で,統計的検定を行って丁寧さの印象に関する両表現の違いを定量的に分析した.その結果,ゴ〜型表現はオ〜型表現に比べ,通常表現を待遇表現に変化させた場合に,通常表現からの丁寧さの変化量がより小さいことが分かった.そして,その原因として,待遇表現としての認識に関する両表現の違いが示唆された.}

\jkeywords{敬語形式,丁寧さ,Scheffeの一対比較法,定量化,統計的検定}

\etitle{Quantitative Analysis regarding Impressions of \\
 Politeness of Verbal-Honorific Expressions: \\
 Comparison of Expressions Using Prefixes \\
 ``GO" and ``O" Honorific Expressions}
\eauthor{Satoko Marumoto \affiref{IBS} \and Tamotsu Shirado \affiref{NICT} \and Hitoshi Isahara \affiref{NICT}}

\eabstract{
Expressions of ``prefix O + main verb + auxiliary verb" and ``prefix GO + main verb + auxiliary verb" are important verbal-honorific expressions in the Japanese language. It has been pointed out in past linguistic researches that the difference between the two types of expressions is that the main verb after ``O" is a Japanese word and the one after ``GO" is a Chinese word. However, there have hardly been any quantitative researches made on the differences of the two expressions so far.\\
In this study, quantitative analyses were performed to reveal differences in the impressions of politeness between these two types of expressions by using Scheffe's paired comparison method and statistical tests.
Results suggest that in regard to difference in politeness from a plain form, ``prefix GO + verb of Chinese word + auxiliary verb," is smaller than ``prefix O + verb of Japanese word + auxiliary verb." It is suggested that these results are due to the difference between these expressions as to the recognition of honorific expressions.}

\ekeywords{honorific forms, politeness, Scheffe's paired comparison method, quantification, statistical tests
}

\begin{document}
\maketitle


\section{はじめに} \label{sc:1}

待遇表現は日本語の特徴の一つである.敬語的な表現は他の言語にも見られるが,日本語のように,待遇表現を作るための特別な語彙や形式が体系的に発達している言語はまれである\cite{水谷 1995}.

日本語の待遇表現は,動詞,形容詞,形容動詞,副詞,名詞,代名詞など,ほぼ全ての品詞に見られる.特に,動詞に関する待遇表現は他の品詞に比べて多様性がある.具体的には,動詞に関する待遇表現は,以下の4つのタイプに大別できる.1)「\underline{お}話しになる」や「\underline{ご}説明する」などのように,接頭辞オもしくは接頭辞ゴと動詞と補助動詞を組み合わせる,2)「おっしゃる」と「申す」(いずれも通常表現\footnote{いわゆる``敬語"は用いず,通常の言葉を用いた表現.}は「言う」)などのように動詞自体を交替させる,3)「話し\underline{て}頂く」「話し\underline{て}下さる」「話し\underline{て}あげる」などのように助詞テを介して補助動詞が繋がる,4)「ます」「れる」「られる」などの助詞・助動詞を動詞と組み合わせる,などがある.これらの中でも1つ目のタイプ(以下,「オ+本動詞+補助動詞」を``オ〜型表現",「ゴ+本動詞+補助動詞」を``ゴ〜型表現"と呼ぶ)は,同じ本動詞を用いた場合でも,補助動詞との組み合わせによって尊敬語になる場合と謙譲語になる場合がある,という複雑な特徴を持つ.

ここで,オ〜型表現とゴ〜型表現の違いについては,形式に関しては,原則的に,接頭辞ゴに続く本動詞が漢語動詞であり,接頭辞オに続く本動詞が和語動詞であるということが従来の言語学的研究で指摘されてきた.しかし,その機能に関しては,接頭辞の違いは考慮せずに同じ補助動詞を持つ表現をまとめて扱うことが多く,両者の違いについて言及されることは,これまで殆どなかった.

ところが,待遇表現としての自然さの印象に関してオ〜型表現とゴ〜型表現を比較した先行研究において,それが誤用である場合にも,オ〜型表現に比べてゴ〜型表現は,概して,不自然さの印象がより弱いという傾向が見られた.そしてその理由として,待遇表現としての認識に関するオ〜型表現とゴ〜型表現の違いが議論された\cite{白土他 2003}.

ここでもし,待遇表現としての認識に関して,オ〜型表現とゴ〜型表現の間で本質的な違いがあるとするならば,自然さの印象だけでなく,待遇表現に関する他のさまざまな印象の違いとしても観測できるはずである.そこで,本研究では,待遇表現の最も典型的な属性である丁寧さに注目する.すなわち,本研究は,待遇表現の丁寧さの印象に関するオ〜型表現とゴ〜型表現の違いについて定量的に調べることを目的とする.


\section{待遇表現の丁寧さの定量化} \label{sc:2}

\subsection{本研究における「待遇表現の丁寧さ」の捉え方} \label{sc:2.1}

待遇表現という言語現象は,(I)話し手が,その具体的な人物や場面に関わる社会的・心理的な諸要因を考慮した上で,問題の人物(話題となる人物や聞き手など)に,あるレベルの待遇を与えようとすることと(I\hspace{-.1em}I)ある待遇表現を使うことで,その人物に,あるレベルの待遇を与えることになる(という法則性がある)こと,に分けて考えることができる\cite{菊地 2003}.社会言語学的立場から待遇表現を体系的に整理している研究\cite{蒲谷他 1991}や,ポライトネスに関する包括的な理論としてよく知られているブラウンとレビンソンの理論\cite{Brown他 1987}も同様に,(I)と(I\hspace{-.1em}I)を分けて扱っている.ここで,(I\hspace{-.1em}I)において,問題の人物に対してより高いレベルの待遇を与えるような待遇表現は,概して,人々がより丁寧な印象を持つ表現であると考えられる.

本研究でも,(I)と(I\hspace{-.1em}I)を分けて扱うことができるという考えに立ち,(I\hspace{-.1em}I)で用いられる,いろいろな待遇表現の丁寧さを,場面の設定は行わずに定量化する.

\subsection{丁寧さの定量化の方法} \label{sc:2.2}

荻野はクロス集計表に基づく待遇表現の定量化の研究において,ほとんど全ての待遇表現の丁寧さは,一次元の値として表現出来ることを示した\cite{荻野 1986}.また,この仮定に基づいてScheffe法\cite{Scheffe 1952}を用いた実証例も報告されている\cite{白土他 2002}.

本研究では,これらの結果をふまえ,いろいろな待遇表現に対して人々が感じる丁寧さの大きさは,何らかの心理的な空間における一次元上の値として定量化できるものとする.丁寧さの定量化の方法としては,心理学的測定法として代表的な手法の一つであるScheffeの一対比較法の中屋の変法\cite{中屋 1970,三浦 1973,田中 1977}を用いる.

以下では,定量化によって得られた,表現の丁寧さを表す値を「待遇値」と呼ぶ.本研究ではScheffe法を用いているため,待遇値は間隔尺度上の値,すなわち待遇表現間の丁寧さの相対値を表す.

\section{オ〜型表現とゴ〜型表現の丁寧さに関する実験} \label{sc:3}

\subsection{本研究で注目する待遇表現のパタン} \label{sc:3.1}

本研究では,表\ref{tbl:table1}に示すオ〜型表現,およびゴ〜型表現の主なパタンに注目する\cite{林他 1974,菊地 1997,鈴木他 1984}.ただし,ここでは,オ〜型表現,およびゴ〜型表現との比較のために通常表現,および丁寧語(ただし,「ます」を用いるパタンのみ)も含めることとした.以下,通常表現,丁寧語,およびオ〜型表現を総称して「和語系表現」と呼び,通常表現,丁寧語,およびゴ〜型表現を総称して「漢語系表現」と呼ぶ.

表\ref{tbl:table1}の「〜」の部分は各パタンと組み合わされる本動詞,すなわち,オ〜型のパタンでは和語動詞,ゴ〜型のパタンでは漢語動詞である\footnote{具体的には,\ref{sec:5.3}節で述べるように,オ〜型のパタンでは和語動詞の連用形,ゴ〜型のパタンでは漢語動詞の語幹,丁寧語「〜ます」のパタンでは漢語動詞の語幹,和語動詞の連用形である.}.また丁寧語「〜ます」のパタンでは和語動詞,漢語動詞の両方である.

表\ref{tbl:table1}には,主な待遇表現のパタンとして二重敬語:「オ/ゴ〜になられる」も含めた.その理由は,近年では二重敬語に抵抗を感じる人が少なく\cite{文化庁文化部国語課 1995},かつ待遇表現としての自然さの印象に関する研究\cite{白土他 2003}においても,このパタンの表現に対しては待遇表現としての自然さの印象が強かったためである.


\begin{table}[htbp]
\begin{center} 
\caption{本研究で注目する待遇表現のパタン} 
\label{tbl:table1} 
\tabcolsep = 2em
\begin{tabular}{ll} 
\hline 
\multicolumn{1}{c}{種類} & \multicolumn{1}{c}{パタン} \\ 
\hline
謙譲語 & オ/ゴ〜する \\
& オ/ゴ〜します \\
& オ/ゴ〜できる \\
& オ/ゴ〜できます \\
& オ/ゴ〜致します \\
& オ/ゴ〜申します \\
& オ/ゴ〜申し上げます \\
& オ/ゴ〜頂く \\
& オ/ゴ〜頂きます \\
\hline 
尊敬語 & オ/ゴ〜なさる \\
& オ/ゴ〜なさいます \\
& オ/ゴ〜になる \\
& オ/ゴ〜になります \\
& オ/ゴ〜下さる \\
& オ/ゴ〜下さいます \\
& オ/ゴ〜になられる (二重敬語) \\
\hline 
丁寧語 & 〜ます \\
\hline 
通常表現 & 〜 \\
\hline 
\end{tabular} 
\end{center} 
\end{table} 

\subsection{実験刺激} \label{sc:3.2}

本実験では,複数個の発話意図に共通して見られる,オ〜型表現とゴ〜型表現の違いに関する傾向を調べる.具体的な発話意図としては,その発話意図に対応した和語動詞と漢語動詞が両方とも存在し,かつ表\ref{tbl:table1}のパタンと組み合わせることが可能なものを5種類,設定した(表\ref{tbl:table2}).なお,表\ref{tbl:table2}左端には各発話意図にほぼ対応する概念を表す英単語を示した.

\begin{table}[htbp]
\begin{center} 
\caption{発話意図とそれに対応する動詞} 
\label{tbl:table2} 
\begin{tabular}{cll} 
\hline 
発話意図 & 和語動詞 & 漢語動詞 \\
\hline 
{\it answer} & 答える & 回答する \\
{\it use} & 使う & 使用する \\
{\it explain} & 話す & 説明する \\
{\it inform} & 知らせる & 連絡する \\
{\it invite} & 招く & 招待する \\
\hline 
\end{tabular} 
\end{center} 
\end{table} 

実験に用いる表現グループは,謙譲語だけから成る表現グループと尊敬語だけから成る表現グループに分けられる.この理由は,謙譲語と尊敬語では行為主体が異なるので(例えば,謙譲語:「(私が)お答えする」,尊敬語:「(先生が)お答えになる」),両者の対が被験者に呈示された場合,比較が困難になるような状況が生じる可能性が否定できないためである.なお,通常表現および丁寧語は,謙譲語および尊敬語と行為主体が同じであると解釈することが可能であり,また,分析に用いる必要上,両方のグループに入れた.

表\ref{tbl:table2}に示した各発話意図における和語動詞,漢語動詞のそれぞれに対して,表\ref{tbl:table1}における謙譲語9パタン+丁寧語1パタン+通常表現1パタン=11パタン,および表\ref{tbl:table1}における尊敬語7パタン+丁寧語1パタン+通常表現1パタン=9パタンを組み合わせて,2つの表現グループを作った(以下,それぞれの表現グループを謙譲語グループ,および尊敬語グループと呼ぶ).

発話意図:{\it answer}に対する表現グループの例を表\ref{tbl:table3.1}(謙譲語),表\ref{tbl:table3.2}(尊敬語)に示す.

各行の左右の列には,同じ補助動詞(「する」,「なさる」など)を持つ表現が対として記されている(ただし,通常表現および丁寧語を除く).表の左側の列には和語系表現,右側の列には漢語系表現が記されている.

各表におけるNo.は,それぞれの表現グループの中での通し番号である.表\ref{tbl:table3.1}ではNo.1とNo.12が通常表現,No.2とNo.13が丁寧語,No.3〜No.11とNo.14〜No.22が謙譲語であり,表\ref{tbl:table3.2}では,No.1とNo.10が通常表現,No.2とNo.11が丁寧語,No.3〜No.9とNo.12〜No.18が尊敬語である.

表\ref{tbl:table3.1},表\ref{tbl:table3.2}と同様に,発話意図:{\it use},{\it explain},{\it inform}および{\it invite}に対して作った表現グループを付表\ref{tbl2:table1.1}〜付表\ref{tbl2:table4.2}に示す.以上のように,本実験では,発話意図:5種類×謙譲語/尊敬語:2種類=計10種類の表現グループを用いた.


\makeatletter
 \renewcommand{\thetable}{}
  \@addtoreset{table}{section}
 \makeatother

\begin{table}[htbp]
\begin{center} 
\setcounter{table}{0}
\caption{発話意図:{\it answer},謙譲語} 
\label{tbl:table3.1} 
\begin{tabular}{|c||l|c||l|} 
\hline 
No. & \multicolumn{1}{|c|}{和語系表現} & No. & \multicolumn{1}{|c|}{漢語系表現} \\
\hline 
1&	答える&			12&	回答する \\
\hline 
2&	答えます&		13&	回答します \\
\hline 
3&	お答えする&		14&	ご回答する \\
\hline 
4&	お答えします&		15&	ご回答します \\
\hline 
5&	お答えできる&		16&	ご回答できる \\
\hline 
6&	お答えできます&		17&	ご回答できます \\
\hline 
7&	お答え致します&		18&	ご回答致します \\
\hline 
8&	お答え申します&		19&	ご回答申します \\
\hline 
9&	お答え申し上げます&	20&	ご回答申し上げます \\
\hline 
10&	お答え頂く&		21&	ご回答頂く \\
\hline 
11&	お答え頂きます&		22&	ご回答頂きます \\
\hline 
\end{tabular} 
\end{center} 
\end{table} 


\begin{table}[htbp]
\begin{center} 
\caption{発話意図:{\it answer},尊敬語} 
\label{tbl:table3.2} 
\begin{tabular}{|c||l|c||l|} 
\hline 
No. & \multicolumn{1}{|c|}{和語系表現} & No. & \multicolumn{1}{|c|}{漢語系表現} \\
\hline 
1&	答える&			10&	回答する \\
\hline 
2&	答えます&		11&	回答します\\
\hline 
3&	お答えなさる&		12&	ご回答なさる\\
\hline 
4&	お答えなさいます&	13&	ご回答なさいます\\
\hline 
5&	お答えになる&		14&	ご回答になる\\
\hline 
6&	お答えになります&	15&	ご回答になります\\
\hline 
7&	お答え下さる&		16&	ご回答下さる\\
\hline 
8&	お答え下さいます&	17&	ご回答下さいます\\
\hline 
9&	お答えになられる&	18&	ご回答になられる\\
\hline 
\end{tabular} 
\end{center} 
\end{table} 

\subsection{実験手続き} \label{sc:3.3}

実験では,上記の10種類の表現グループそれぞれにおいて,グループ内の全ての表現に対する待遇値を求めた.定量化には,前述のようにScheffeの一対比較法の中屋の変法を用いた.実験における回答形式および回答の例を図\ref{fig:figure1}に示す.

被験者は関東在住の20代〜50代の男女80人(男女各40人)である.被験者は一対ずつ呈示された表現の間で丁寧さを比較し,図中のケース1〜ケース3のそれぞれに応じて回答用紙の適切な位置に○を付けるよう指示された(ケース1,ケース2と対称的なケースについても同様).ただし,回答が困難なケース(すなわち,ケース4)については×を付けることを認め,このケースについては定量化の計算に加味しないこととした.

以上の実験で得られた,各表現グループにおける各表現の待遇値の,全被験者にわたる平均値を$\mu$と記す.ただし表現E$_{i}$に対する$\mu$を特定する場合は,$\mu_{i}$と記す.ここで,$i$は各表現グループの中での通し番号(例えば,表\ref{tbl:table3.1},表\ref{tbl:table3.2}中のNo.)である.



\begin{figure}[htbp]
\begin{center}
\epsffile{./fig1.eps} 
\caption{回答形式および回答の例}
\label{fig:figure1}
\end{center}
\end{figure}


\section{実験結果} \label{sc:4}

表\ref{tbl:table3.1}および表\ref{tbl:table3.2}の表現に対して得られた$\mu$を,それぞれ図\ref{fig:figure2.1}および図\ref{fig:figure2.2}に示す\footnote{\ref{sc:2.2}節で述べたように,$\mu$は,表現間の相対値であるので,その符号の正負の間で本質的な違いはない.}.図中の数字は,各表現E$_{i}$の添数$i$の値を表す.同様にして,付表\ref{tbl2:table1.1}〜付表\ref{tbl2:table4.2}の各表現に対して得られた$\mu$を,それぞれ,付図\ref{fig2:figure1.1}〜付図\ref{fig2:figure4.2}に示す.
図において,$\mu_{i}$が大きい表現E$_{i}$程,それに対して平均的な被験者が感じる丁寧さの程度が大きいことを表す.
本研究で用いた二重敬語(すなわち,「オ/ゴ〜になられる」)の丁寧さについては,他の表現(すなわち,規範的な表現)と同様の傾向が見られた.例えば,図\ref{fig:figure2.2},付図\ref{fig2:figure1.2},\ref{fig2:figure2.2},\ref{fig2:figure3.2},\ref{fig2:figure4.2}の全てにおいて,「オ/ゴ〜になられる」(二重敬語)の待遇値は,「オ/ゴ〜になる」(「オ/ゴ〜になられる」と最も形が似ており,かつモーラ数がより短い規範的な待遇表現)の待遇値より大きかった.これは,規範的な待遇表現に関する特徴:``モーラ数がより長い待遇表現は概して,より丁寧に感じられる"\cite{荻野 1980}と一致する.
また,被験者が×を付けた表現ペア,すなわち,被験者が待遇表現としての丁寧さの比較が困難と判断したものは,4,279ペアであった.これは,回答全体の2.7\%に相当する.このうち75\%は,特定の10人に偏っていた.ただし,この10人に関して,年齢や性別に関する偏りは見られなかった.さらに,比較が困難とされた表現ペアには特定の表現への偏りは見られなかった.


\makeatletter
 \renewcommand{\thefigure}{}
  \@addtoreset{figure}{section}
 \makeatother

\setcounter{figure}{0}

\begin{figure}[htbp]
\begin{tabular}{cc}
\begin{minipage}{0.5\hsize}
\begin{center}

\epsfxsize=0.8\hsize
\epsffile{./fig21.eps}
\caption{ $\mu$ (発話意図:{\it answer},謙譲語)}
\label{fig:figure2.1}

\end{center}
\end{minipage}
\begin{minipage}{0.5\hsize}
\begin{center}

\epsfxsize=0.8\hsize
\epsffile{./fig22.eps}
\caption{ $\mu$ (発話意図:{\it answer},尊敬語)}
\label{fig:figure2.2}

\end{center}
\end{minipage}
\end{tabular}
\end{figure}  


\section{考察} \label{sc:5}

\subsection{和語系表現と漢語系表現の比較} \label{sc:5.1}

前述の通り,表\ref{tbl:table3.1},表\ref{tbl:table3.2},付表\ref{tbl2:table1.1}〜付表\ref{tbl2:table4.2}の各行には,接頭辞+本動詞,が異なり(左側の列はオ+和語動詞,右側の列はゴ+漢語動詞),補助動詞(「する」,「なさる」など)が同じである表現が対となって記されている.

ここでは,このように対となる表現の間で$\mu$の差($d$と記す)を式(\ref{eqn:eqn1})で計算することによって,同じ補助動詞を持つ和語系表現と漢語系表現の間での,平均的な被験者の待遇値の違いを調べる.

\begin{eqnarray} 
d=\mu_{i+n}-\mu_{i} (i=1,...,n) \label{eqn:eqn1}
\end{eqnarray}

式(\ref{eqn:eqn1})右辺第一項は漢語系表現の$\mu$,第二項は和語系表現の$\mu$である.ここで,$i$は表現E$_{i}$の添数$i$を表し,$n$は謙譲語に対しては11,尊敬語に対しては9である.すなわち,式(\ref{eqn:eqn1})では,表\ref{tbl:table4}に示す計算を行っている.


\makeatletter
 \renewcommand{\thetable}{}
  \@addtoreset{table}{section}
 \makeatother

\setcounter{table}{3}

\begin{table}[htbp]
\begin{center} 
\caption{$d$の計算方法} 
\label{tbl:table4} 
\begin{tabular}{ccc}
\hline 
ラベル & 謙譲語グループ & 尊敬語グループ \\
\hline
A & $\mu_{12}-\mu_{1}$ & $\mu_{10}-\mu_{1}$ \\
B & $\mu_{13}-\mu_{2}$ & $\mu_{11}-\mu_{2}$ \\
C & $\mu_{14}-\mu_{3}$ & $\mu_{12}-\mu_{3}$ \\
D & $\mu_{15}-\mu_{4}$ & $\mu_{13}-\mu_{4}$ \\
E & $\mu_{16}-\mu_{5}$ & $\mu_{14}-\mu_{5}$ \\
F & $\mu_{17}-\mu_{6}$ & $\mu_{15}-\mu_{6}$ \\
G & $\mu_{18}-\mu_{7}$ & $\mu_{16}-\mu_{7}$ \\
H & $\mu_{19}-\mu_{8}$ & $\mu_{17}-\mu_{8}$ \\
I & $\mu_{20}-\mu_{9}$ & $\mu_{18}-\mu_{9}$ \\
J & $\mu_{21}-\mu_{10}$ & - \\
K & $\mu_{22}-\mu_{11}$ & - \\
\hline 
\end{tabular} 
\end{center} 
\end{table} 


図\ref{fig:figure2.1},および図\ref{fig:figure2.2}に示した$\mu$に対し,以上の方法で得た$d$を図\ref{fig:figure3.1},および図\ref{fig:figure3.2}に示す.図中の記号(A,B,…,K)は,表\ref{tbl:table4}のラベルに対応する.


\makeatletter
 \renewcommand{\thefigure}{}
  \@addtoreset{figure}{section}
 \makeatother

\begin{figure}[htbp]
\begin{tabular}{cc}
\begin{minipage}{0.5\hsize}
\begin{center}

\epsfxsize=0.8\hsize
\epsffile{./fig31.eps}
\caption{ $d$ (発話意図:{\it answer},謙譲語)}
\label{fig:figure3.1}

\end{center}
\end{minipage}
\begin{minipage}{0.5\hsize}
\begin{center}

\epsfxsize=0.8\hsize
\epsffile{./fig32.eps}
\caption{ $d$ (発話意図:{\it answer},尊敬語)}
\label{fig:figure3.2}

\end{center}
\end{minipage}
\end{tabular}
\end{figure}  

図\ref{fig:figure3.1}および図\ref{fig:figure3.2}のいずれも,A(通常表現)およびB(丁寧語)の$d$が正,すなわち漢語系表現の$\mu$が和語系表現の$\mu$より大きいことを示す.他の全ての表現グループにおいても,これと同様の傾向が見られた(付図\ref{fig2:figure5.1}〜付図\ref{fig2:figure8.2}).[結果(1)]

一方,A(通常表現)およびB(丁寧語)以外の語(すなわち,尊敬語や謙譲語)の$d$は,値の正負に関する一貫した傾向はなかった.他の全ての表現グループにおいても,これと同様であった(付図\ref{fig2:figure5.1}〜付図\ref{fig2:figure8.2}).

\subsection{通常表現からの変化に関する,オ〜型表現とゴ〜型表現の比較} \label{sc:5.2}

前節で述べたように,同じ補助動詞を持つオ〜型表現(「オ+和語動詞+補助動詞」)とゴ〜型表現(「ゴ+漢語動詞+補助動詞」)の間では,通常表現および丁寧語を除き,両者の$\mu$の差,すなわち$d$に関する一貫した傾向は見られなかった.しかし,$d$は「オ+〜+補助動詞」と「ゴ+〜+補助動詞」の違いのみならず,「〜」の部分,すなわち,和語動詞と漢語動詞の違いを反映した指標であるため,ここでは,和語動詞と漢語動詞の違いの影響をできるだけ排除して,オ〜型表現とゴ〜型表現の間で丁寧さの印象に関する特性の違いをさらに調べることとする.このためには,表現全体としての$\mu$ではなく和語動詞単体(通常表現)から「オ+和語動詞+補助動詞」に変化させたことによる$\mu$の変化量,および漢語動詞単体(通常表現)から「ゴ+漢語動詞+補助動詞」に変化させたことによる$\mu$の変化量の間で比較を行えば良い.すなわち,各表現(通常表現を除く)の$\mu$をその通常表現の$\mu$からの変化量として補正した値(これを$\mu'$ と記す)に関して,オ〜型表現とゴ〜型表現の差($\delta$と記す)を次式で計算する.

\begin{eqnarray}
\delta=\mu'_{i+n}-\mu'_{i} (i=2,...,n) \label{eqn:eqn2}
\end{eqnarray}

ただし, $\mu'_{i+n}=\mu_{i+n}-\mu_{n+1}$(漢語動詞.$\mu_{n+1}$は通常表現の$\mu$),$\mu'_{i}=\mu_{i}-\mu_{1}$(和語動詞.$\mu_{1}$は通常表現の$\mu$),$n$は式(\ref{eqn:eqn1})と同様,謙譲語に対しては11,尊敬語に対しては9である.すなわち,式(\ref{eqn:eqn2})では,表\ref{tbl:table5}に示す計算を行っている.

なお,ラベルBの$\delta$は丁寧語であるため接頭辞オ/ゴは含まないが,オ〜型表現,およびゴ〜型表現との比較のため示した.


\begin{table}[htbp]
\begin{center} 
\caption{$\delta$の計算方法} 
\label{tbl:table5} 
\begin{tabular}{ccc}
\hline 
ラベル & 謙譲語グループ & 尊敬語グループ \\
\hline
B & $(\mu_{13}-\mu_{12})-(\mu_{2}-\mu_{1})$ & $(\mu_{11}-\mu_{10})-(\mu_{2}-\mu_{1})$ \\
C & $(\mu_{14}-\mu_{12})-(\mu_{3}-\mu_{1})$ & $(\mu_{12}-\mu_{10})-(\mu_{3}-\mu_{1})$ \\
D & $(\mu_{15}-\mu_{12})-(\mu_{4}-\mu_{1})$ & $(\mu_{13}-\mu_{10})-(\mu_{4}-\mu_{1})$ \\
E & $(\mu_{16}-\mu_{12})-(\mu_{5}-\mu_{1})$ & $(\mu_{14}-\mu_{10})-(\mu_{5}-\mu_{1})$ \\
F & $(\mu_{17}-\mu_{12})-(\mu_{6}-\mu_{1})$ & $(\mu_{15}-\mu_{10})-(\mu_{6}-\mu_{1})$ \\
G & $(\mu_{18}-\mu_{12})-(\mu_{7}-\mu_{1})$ & $(\mu_{16}-\mu_{10})-(\mu_{7}-\mu_{1})$ \\
H & $(\mu_{19}-\mu_{12})-(\mu_{8}-\mu_{1})$ & $(\mu_{17}-\mu_{10})-(\mu_{8}-\mu_{1})$ \\
I & $(\mu_{20}-\mu_{12})-(\mu_{9}-\mu_{1})$ & $(\mu_{18}-\mu_{10})-(\mu_{9}-\mu_{1})$ \\
J & $(\mu_{21}-\mu_{12})-(\mu_{10}-\mu_{1})$ & - \\
K & $(\mu_{22}-\mu_{12})-(\mu_{11}-\mu_{1})$ & - \\
\hline 
\end{tabular} 
\end{center} 
\end{table} 

図\ref{fig:figure2.1}および図\ref{fig:figure2.2}に示した$\mu$に対し,以上の方法で得た$\delta$を図\ref{fig:figure4.1}および図\ref{fig:figure4.2}に示す.


\makeatletter
 \renewcommand{\thefigure}{}
  \@addtoreset{figure}{section}
 \makeatother

\setcounter{figure}{0}

\begin{figure}[htbp]
\begin{tabular}{cc}
\begin{minipage}{0.5\hsize}
\begin{center}

\epsfxsize=0.8\hsize
\epsffile{./fig41.eps}
\caption{ $\delta$ (発話意図:{\it answer},謙譲語)}
\label{fig:figure4.1}

\end{center}
\end{minipage}
\begin{minipage}{0.5\hsize}
\begin{center}

\epsfxsize=0.8\hsize
\epsffile{./fig42.eps}
\caption{ $\delta$ (発話意図:{\it answer},尊敬語)}
\label{fig:figure4.2}

\end{center}
\end{minipage}
\end{tabular}
\end{figure}  

図\ref{fig:figure4.1}および図\ref{fig:figure4.2}を見ると,B(丁寧語)に対する$\delta$ を除き,全ての$\delta$の値は負であることが分かる.他の全ての表現グループにおいても,1つの例外(付図\ref{fig2:figure11.2}のC)を除き,全ての$\delta$ は負であった(付図\ref{fig2:figure9.1}〜付図\ref{fig2:figure12.2}).

そこで,各表現グループにおいて,帰無仮説:$\bar{\delta} =0$の検定\cite{石村 1989}を行った(表\ref{tbl:table6}).ここで,$\bar{\delta}$は,各表現グループにおいて,B(丁寧語)を除く全ての表現にわたる$\delta$の平均を表す.

表\ref{tbl:table6}は,10種類の表現グループいずれにおいても帰無仮説:$\bar{\delta} = 0$が棄却されることを示す(有意水準1\%).かつ,$\bar{\delta}$ は負である.従って,$\bar{\delta}$は有意に0より小さい.これは,通常表現からの変化量$\mu'$に関し,ゴ〜型表現の平均的な$\mu'$は,それに対応する(すなわち,同じ補助動詞を持つ)オ〜型表現の平均的な$\mu'$より有意に小さいことを示唆する.[結果(2)]


\begin{table}[htbp]
\begin{center} 
\caption{$\bar{\delta}$=0の検定結果} 
\label{tbl:table6} 
\begin{tabular}{|c|c|r|r|r|}
\hline 
発話意図 & 謙譲語/尊敬語 &  \multicolumn{1}{|c|}{$\bar{\delta}$} & \multicolumn{1}{|c|}{検定量$T$} & \multicolumn{1}{|c|}{自由度} \\
\hline
\raisebox{-.5zh}[0cm][0cm]{\it answer}	& 謙譲語 & -0.264 & 13.4 & 8 \\
\cline{2-5}
	& 尊敬語 & -0.156 &  5.8 & 6 \\
\hline
\raisebox{-.5zh}[0cm][0cm]{\it use}	& 謙譲語 & -0.194 &  7.1 & 8 \\
\cline{2-5}
	& 尊敬語 & -0.220 & 12.7 & 6 \\
\hline
\raisebox{-.5zh}[0cm][0cm]{\it explain}	& 謙譲語 & -0.110 &  5.9 & 8 \\
\cline{2-5}
	& 尊敬語 & -0.171 & 11.5 & 6 \\
\hline
\raisebox{-.5zh}[0cm][0cm]{\it inform}	& 謙譲語 & -0.103 &  8.4 & 8 \\
\cline{2-5}
	& 尊敬語 & -0.114 &  4.2 & 6 \\
\hline
\raisebox{-.5zh}[0cm][0cm]{\it invite}	& 謙譲語 & -0.271 & 13.3 & 8 \\
\cline{2-5}
	& 尊敬語 & -0.301 & 14.8 & 6 \\
\hline
\end{tabular} 
\end{center} 
\end{table} 

また,図\ref{fig:figure4.1}および図\ref{fig:figure4.2}のいずれも,B(丁寧語)の$\delta$が正であることを示す.図\ref{fig:figure4.1},図\ref{fig:figure4.2},付図\ref{fig2:figure9.1}〜付図\ref{fig2:figure12.2}と合わせ,全部で10個の表現グループのうち,5個(50\%)においてB(丁寧語)の$\delta$が正であった[結果(3)].

\subsection{オ〜型表現とゴ〜型表現に差異が生じた理由}\label{sec:5.3}

前節の結果(2)に示したように,ゴ〜型表現の平均的な$\mu'$(通常表現の$\mu$からの変化量)は,それに対応する(すなわち,同じ補助動詞を持つ)オ〜表現の平均的な$\mu'$より小さいことが示唆された.以下では,この理由について考察する.
\\
\\
解釈1) 一体感の違いに基づく解釈

オ〜型表現とゴ〜型表現の丁寧さの印象の差異は,それぞれの本動詞の活用形(和語動詞の場合は和語動詞連用形,漢語動詞の場合は漢語動詞語幹)が動詞転成名詞の語形と一致していることに起因すると考えられる.ここで,動詞転成名詞とは,和語動詞の場合は和語動詞連用形が名詞の性質を持ったものである.漢語動詞の場合は漢語動詞(サ変動詞)語幹が名詞の性質を持ったものであり,いわゆるサ変名詞に相当する.

具体的には,ゴ〜型表現は「ゴ回答」「ゴ招待」のように「ゴ+漢語動詞語幹」のみで独立した表現(すなわち,サ変名詞としての用法)として用いられることが多い.このため,「ゴ+漢語動詞語幹」が現れた時に,その表現だけで動詞転成名詞としての待遇表現が成立していると認識し,後続する表現,すなわち補助動詞への意識が低くなっている可能性がある.

一方,オ〜型表現の場合,「オ+和語動詞連用形」は動詞転成名詞であり形としては正しいが,「オ使い」「オ招き」のように,「オ+和語動詞連用形」の部分だけでは,動詞転成名詞としては,ゴ〜型表現に比べて,あまり用いられないことが考えられる.このため,「オ+和語動詞連用形」が現れた時に,その表現だけで待遇表現が成立しているとは認識せずに,後続する表現,すなわち補助動詞に意識が及びやすくなっている可能性がある.つまり,オ〜型表現では,接頭辞と本動詞だけでなく,補助動詞まで確認した上で表現としての適切さや丁寧さの程度を判断しようとしていると言える.

このことは,接頭辞オと接頭辞ゴの間には,本動詞との一体感の印象に差異がある,すなわち,ゴと漢語動詞の一体感は,オと和語動詞の一体感より強いことを示唆する.ゴ〜型表現の方が接頭辞と本動詞の一体感がより強いため,通常表現に接頭辞が付いて待遇表現になったことへの印象がオ〜型表現に比べて弱かったと考えられる.そしてその結果,通常表現からの待遇値の変化量がより小さくなったと考えられる.

しかし,この解釈の他に次の解釈も可能である.
\\
\\
解釈2) ある種の飽和現象の仮定に基づく解釈

前述のように結果(2)は,通常表現からの待遇値の変化量$\mu'$に関するオ〜型表現とゴ〜型表現の比較による.このため,ある種の飽和現象:``通常表現の待遇値$\mu$が大きい程,通常表現に接頭辞や語尾(丁寧語の場合は語尾のみ)を付加してモーラ数を長くした時の$\mu$の変化量は小さくなる(変化量が鈍化する)",が存在すると仮定するならば,今回の実験では,全ての表現グループにおいて漢語動詞は和語動詞に比べ通常表現の$\mu$の値が大きかった(結果(1))ため,通常表現の$\mu$がより大きいゴ〜型表現の$\mu'$が,通常表現の$\mu$がより小さいオ〜型表現の$\mu'$より小さくなった,という説明が可能である.

しかし,この解釈は結果(3)(10個の表現グループのうち,5個(50\%)においてB(丁寧語)の$\delta$が正であった)とは必ずしも一致しない.すなわち,結果(3)のうち,丁寧語(接頭辞を用いない表現)で$\delta$が正になった(すなわち,漢語系表現の$\mu'$が和語系表現の$\mu'$より大きかった)表現グループ(全体の50\%)に対しては,この解釈では説明できない.

以上のことから,解釈1は本実験結果に対するより妥当な解釈であると考えられる.すなわち動詞待遇表現に関しては,後続する本動詞との一体感に関する接頭辞ゴと接頭辞オの違いが,通常表現からの待遇値の変化量に関するオ〜型表現とゴ〜型表現の違いとして現れたと解釈できる.

\subsection{同じグループに属する表現間の丁寧さの印象のばらつきに関する,オ〜型表現とゴ〜型表現の比較} \label{sc:5.4}

前節に述べた解釈1からは,ゴ〜型表現グループはオ〜型表現グループに比べて,同じ表現グループに属する表現間を区別して認識する度合いがより小さいことが予測される.この時,ゴ〜型表現グループの方がオ〜型表現グループより,同じ表現グループに属する表現間の待遇値$\mu$の違いが小さくなることが予測される.この予測を確かめるため,ここでは,ゴ〜型表現グループとオ〜型表現グループの間で,グループ内の表現の$\mu$に関する不偏分散($s^{2}$と記す)を比較する.この時,$s^{2}$が大きい程,そのグループにおける表現間での$\mu$の違いが大きいことを意味する.従って,オ〜型表現にわたる$s^{2}$とゴ〜型表現にわたる$s^{2}$とを比較することによって,それぞれの表現型に属する表現の間での丁寧さの印象の違いを,オ〜型表現とゴ〜型表現との間で比較することができる.

各表現グループにおいて,$\mu$の全てのオ〜型表現にわたる $s^{2}$($s^{2}$(オ)と記す)および,$\mu$の全てのゴ〜型表現にわたる$s^{2}$($s^{2}$(ゴ)と記す)をそれぞれ求めた結果を表\ref{tbl:table7}に示す.


\begin{table}[htbp]
\begin{center} 
\caption{オ〜型表現とゴ〜型表現の比較} 
\label{tbl:table7} 
\begin{tabular}{|c|c|c|c|}
\hline 
発話意図 & 謙譲語/尊敬語 &  オ〜型/ゴ〜型 & $s^{2}$(普遍分散) \\
\hline
\raisebox{-2zh}[0cm][0cm]{\it answer}	& \raisebox{-.5zh}[0cm][0cm]{謙譲語} & オ〜型 & 0.279 \\
\cline{3-4}
	&	 & ゴ〜型 & 0.289 \\
\cline{2-4}
	& \raisebox{-.5zh}[0cm][0cm]{尊敬語} & オ〜型 & 0.110 \\
\cline{3-4}
	&	 & ゴ〜型 & 0.106 \\
\hline
\raisebox{-2zh}[0cm][0cm]{\it use}	& \raisebox{-.5zh}[0cm][0cm]{謙譲語} & オ〜型 & 0.225 \\
\cline{3-4}
	&	 & ゴ〜型 & 0.221 \\
\cline{2-4}
	& \raisebox{-.5zh}[0cm][0cm]{尊敬語} & オ〜型 & 0.078 \\
\cline{3-4}
	&	 & ゴ〜型 & 0.082 \\
\hline
\raisebox{-2zh}[0cm][0cm]{\it explain}	& \raisebox{-.5zh}[0cm][0cm]{謙譲語} & オ〜型 & 0.353 \\
\cline{3-4}
	&	 & ゴ〜型 & 0.303 \\
\cline{2-4}
	& \raisebox{-.5zh}[0cm][0cm]{尊敬語} & オ〜型 & 0.130 \\
\cline{3-4}
	&	 & ゴ〜型 & 0.109 \\
\hline
\raisebox{-2zh}[0cm][0cm]{\it inform}	& \raisebox{-.5zh}[0cm][0cm]{謙譲語} & オ〜型 & 0.328 \\
\cline{3-4}
	&	 & ゴ〜型 & 0.334 \\
\cline{2-4}
	& \raisebox{-.5zh}[0cm][0cm]{尊敬語} & オ〜型 & 0.118 \\
\cline{3-4}
	&	 & ゴ〜型 & 0.116 \\
\hline
\raisebox{-2zh}[0cm][0cm]{\it invite}	& \raisebox{-.5zh}[0cm][0cm]{謙譲語} & オ〜型 & 0.342 \\
\cline{3-4}
	&	 & ゴ〜型 & 0.332 \\
\cline{2-4}
	& \raisebox{-.5zh}[0cm][0cm]{尊敬語} & オ〜型 & 0.118 \\
\cline{3-4}
	&	 & ゴ〜型 & 0.103 \\
\hline
\end{tabular} 
\end{center} 
\end{table} 

表\ref{tbl:table7}を見ると,10個の表現グループのうち7個の表現グループで,$s^{2}$(オ)>$s^{2}$(ゴ)であることが分かる.

この結果は,概して,オ〜型表現に比べてゴ〜型表現は,同じグループに属する表現間の待遇値$\mu$の違いの差が小さいことを示唆する.これは先の予測と一致する.従って,この結果は解釈1の妥当性を支持する.

このことから,同じ表現グループに属する表現間での丁寧さの印象(待遇値)に関する比較によっても,オ〜型表現とゴ〜型表現の間には,接頭辞と後続の語との一体感の違いに起因する心理的印象の違いが生じていることが示唆された.

\section{おわりに} \label{sc:6}

待遇表現に対する丁寧さの印象に関し,オ〜型表現(「オ+和語+補助動詞」)とゴ〜型表現(「ゴ+漢語+補助動詞」)の違いを定量的に調べた.その結果,丁寧さの大きさに関し,オ〜型表現に比べてゴ〜型表現は,
\begin{itemize}
\item{通常表現からの変化量がより小さいこと}
\item{その表現グループに属する表現間の違いがより小さいこと}
\end{itemize}
が示唆された.その原因として,待遇表現としての認識に関する両者の違いが示唆された.すなわち,従来,同じ補助動詞の場合には,まとめて扱われることの多かったオ〜型表現とゴ〜型表現の間には,接頭辞と後続の語との一体感の違いに起因する心理的印象の違いが生じていることが示唆された.

待遇表現の自然さの印象\cite{白土他 2003}だけでなく,待遇表現の丁寧さの印象に関しても,オ〜型表現とゴ〜型表現の間に差異が見られたことから,両者の間には,本質的な違いがあると考えられる.これが適切だとすると,待遇表現に関する教育においても,両者に違いがあるということを考慮する必要があるのではないか.すなわち,ゴ〜型表現はオ〜型表現に比べ,表現間の区別のしにくさに起因した誤用が多いことから,オ〜型表現とゴ〜型表現を学習させる場合には,両者を接頭辞だけが異なり,他は等価なものとして教えるのではなく,ゴ〜型表現の方が区別がしにくく間違いやすい,ということを教えた方が良い可能性がある.このように本研究での知見は,教育上の一つの指針になりうる.

本研究で対象としたようなオ/ゴ〜型表現は,``〜"の部分に様々な動詞を当てはめることができ,かつ動詞に続く様々な補助動詞と組み合わせることができるため,非常に多くのバリエーションが存在する.今回はオ/ゴ〜型表現のうち,一部の表現のみを対象としたが,本稿で述べた手法を用いて,より多くのオ/ゴ〜型表現,``〜"の部分(動詞)の終止形,および``〜"の動詞を交替した表現(例:「言う」に対する「おっしゃる」)の丁寧さの程度の数値化を行うことにより,多くの表現に関してその表現と丁寧さの程度との対応データが作成できる.このデータは,例えば文生成の研究において,様々な丁寧さを持つ様々な待遇表現を柔軟に生成する際に役に立つと考えられる.この際,本研究で得られた知見に基づくと,例えば,同じ程度の丁寧さを持った表現を数多く生成するためにはゴ〜型表現を優先的に用い,一方,丁寧さの違いが大きい表現を数多く生成するためには,オ〜型表現を優先的に用いる,などのように対処すれば良いと考えられる.

待遇表現としての認識は,被験者の年齢や性別などにも依存する可能性がある.従って今後は,これらの被験者属性への依存性について検討する予定である.



\newpage




\newcounter{appndnum}
\def\appndnum{}
\setcounter{appndnum}{1}

\renewcommand{\figurename}{}
\renewcommand{\tablename}{}

\makeatletter
 \renewcommand{\thetable}{}
  \@addtoreset{table}{section}
 \makeatother


\makeatletter
 \renewcommand{\thefigure}{}
  \@addtoreset{figure}{section}
 \makeatother

\begin{table}[htbp]
\begin{center}
{ \scriptsize
\begin{tabular}{cc}
\begin{minipage}[t]{0.5\hsize}
\begin{center}


\caption{発話意図:{\it use},謙譲語}
\label{tbl2:table1.1} 
\begin{tabular}{|c||l|c||l|} 
\hline 
No. & \multicolumn{1}{|c|}{和語系表現} & No. & \multicolumn{1}{|c|}{漢語系表現} \\
\hline 
1&	使う&			12&	使用する \\
\hline 
2&	使います&		13&	使用します \\
\hline 
3&	お使いする&		14&	ご使用する \\
\hline 
4&	お使いします&		15&	ご使用します \\
\hline 
5&	お使いできる&		16&	ご使用できる \\
\hline 
6&	お使いできます&		17&	ご使用できます \\
\hline 
7&	お使い致します&		18&	ご使用致します \\
\hline 
8&	お使い申します&		19&	ご使用申します \\
\hline 
9&	お使い申し上げます&	20&	ご使用申し上げます \\
\hline 
10&	お使い頂く&		21&	ご使用頂く \\
\hline 
11&	お使い頂きます&		22&	ご使用頂きます \\
\hline
\end{tabular} 

\end{center}
\end{minipage}
\begin{minipage}[t]{0.5\hsize}
\begin{center}



\caption{発話意図:{\it use},尊敬語} 
\label{tbl2:table1.2} 
\begin{tabular}{|c||l|c||l|} 
\hline 
No. & \multicolumn{1}{|c|}{和語系表現} & No. & \multicolumn{1}{|c|}{漢語系表現} \\
\hline 
1&	使う&			10&	使用する \\
\hline 
2&	使います&		11&	使用します\\
\hline 
3&	お使いなさる&		12&	ご使用なさる\\
\hline 
4&	お使いなさいます&	13&	ご使用なさいます\\
\hline 
5&	お使いになる&		14&	ご使用になる\\
\hline 
6&	お使いになります&	15&	ご使用になります\\
\hline 
7&	お使い下さる&		16&	ご使用下さる\\
\hline 
8&	お使い下さいます&	17&	ご使用下さいます\\
\hline 
9&	お使いになられる&	18&	ご使用になられる\\
\hline 
\end{tabular} 

\end{center}
\end{minipage}
\end{tabular}
}
\end{center}
\end{table} 

\vspace{-2.5\baselineskip} 

\addtocounter{appndnum}{1}
\setcounter{table}{0}

\begin{table}[htbp]
\begin{center}
{ \scriptsize
\begin{tabular}{cc}
\begin{minipage}[t]{0.5\hsize}
\begin{center}



\caption{発話意図:{\it explain},謙譲語} 
\label{tbl2:table2.1} 
\begin{tabular}{|c||l|c||l|} 
\hline 
No. & \multicolumn{1}{|c|}{和語系表現} & No. & \multicolumn{1}{|c|}{漢語系表現} \\
\hline 
1&	話す&			12&	説明する \\
\hline 
2&	話します&		13&	説明します \\
\hline 
3&	お話しする&		14&	ご説明する \\
\hline 
4&	お話しします&		15&	ご説明します \\
\hline 
5&	お話しできる&		16&	ご説明できる \\
\hline 
6&	お話しできます&		17&	ご説明できます \\
\hline 
7&	お話し致します&		18&	ご説明致します \\
\hline 
8&	お話し申します&		19&	ご説明申します \\
\hline 
9&	お話し申し上げます&	20&	ご説明申し上げます \\
\hline 
10&	お話し頂く&		21&	ご説明頂く \\
\hline 
11&	お話し頂きます&		22&	ご説明頂きます \\
\hline 
\end{tabular} 

\end{center}
\end{minipage}
\begin{minipage}[t]{0.5\hsize}
\begin{center}



\caption{発話意図:{\it explain},尊敬語} 
\label{tbl2:table2.2} 
\begin{tabular}{|c||l|c||l|} 
\hline 
No. & \multicolumn{1}{|c|}{和語系表現} & No. & \multicolumn{1}{|c|}{漢語系表現} \\
\hline 
1&	話す&			10&	説明する \\
\hline 
2&	話します&		11&	説明します\\
\hline 
3&	お話しなさる&		12&	ご説明なさる\\
\hline 
4&	お話しなさいます&	13&	ご説明なさいます\\
\hline 
5&	お話しになる&		14&	ご説明になる\\
\hline 
6&	お話しになります&	15&	ご説明になります\\
\hline 
7&	お話し下さる&		16&	ご説明下さる\\
\hline 
8&	お話し下さいます&	17&	ご説明下さいます\\
\hline 
9&	お話しになられる&	18&	ご説明になられる\\
\hline 
\end{tabular} 

\end{center}
\end{minipage}
\end{tabular}
}
\end{center}
\end{table} 

\vspace{-2.5\baselineskip} 

\addtocounter{appndnum}{1}
\setcounter{table}{0}

\begin{table}[htbp]
\begin{center}
{ \scriptsize
\begin{tabular}{cc}
\begin{minipage}[t]{0.5\hsize}
\begin{center}


\caption{発話意図:{\it inform},謙譲語} 
\label{tbl2:table3.1} 
\begin{tabular}{|c||l|c||l|} 
\hline 
No. & \multicolumn{1}{|c|}{和語系表現} & No. & \multicolumn{1}{|c|}{漢語系表現} \\
\hline 
1&	知らせる&			12&	連絡する \\
\hline 
2&	知らせます&		13&	連絡します \\
\hline 
3&	お知らせする&		14&	ご連絡する \\
\hline 
4&	お知らせします&		15&	ご連絡します \\
\hline 
5&	お知らせできる&		16&	ご連絡できる \\
\hline 
6&	お知らせできます&		17&	ご連絡できます \\
\hline 
7&	お知らせ致します&		18&	ご連絡致します \\
\hline 
8&	お知らせ申します&		19&	ご連絡申します \\
\hline 
9&	お知らせ申し上げます&	20&	ご連絡申し上げます \\
\hline 
10&	お知らせ頂く&		21&	ご連絡頂く \\
\hline 
11&	お知らせ頂きます&		22&	ご連絡頂きます \\
\hline 
\end{tabular} 

\end{center}
\end{minipage}
\begin{minipage}[t]{0.5\hsize}
\begin{center}



\caption{発話意図:{\it inform},尊敬語} 
\label{tbl2:table3.2} 
\begin{tabular}{|c||l|c||l|} 
\hline 
No. & \multicolumn{1}{|c|}{和語系表現} & No. & \multicolumn{1}{|c|}{漢語系表現} \\
\hline 
1&	知らせる&			10&	連絡する \\
\hline 
2&	知らせます&		11&	連絡します\\
\hline 
3&	お知らせなさる&		12&	ご連絡なさる\\
\hline 
4&	お知らせなさいます&	13&	ご連絡なさいます\\
\hline 
5&	お知らせになる&		14&	ご連絡になる\\
\hline 
6&	お知らせになります&	15&	ご連絡になります\\
\hline 
7&	お知らせ下さる&		16&	ご連絡下さる\\
\hline 
8&	お知らせ下さいます&	17&	ご連絡下さいます\\
\hline 
9&	お知らせになられる&	18&	ご連絡になられる\\
\hline 
\end{tabular} 

\end{center}
\end{minipage}
\end{tabular}
}
\end{center}
\end{table} 

\vspace{-2.5\baselineskip} 


\addtocounter{appndnum}{1}
\setcounter{table}{0}

\begin{table}[htbp]
\begin{center}
{ \scriptsize
\begin{tabular}{cc}
\begin{minipage}[t]{0.5\hsize}
\begin{center}


\caption{発話意図:{\it invite},謙譲語} 
\label{tbl2:table4.1} 
\begin{tabular}{|c||l|c||l|} 
\hline 
No. & \multicolumn{1}{|c|}{和語系表現} & No. & \multicolumn{1}{|c|}{漢語系表現} \\
\hline 
1&	招く&			12&	招待する \\
\hline 
2&	招きます&		13&	招待します \\
\hline 
3&	お招きする&		14&	ご招待する \\
\hline 
4&	お招きします&		15&	ご招待します \\
\hline 
5&	お招きできる&		16&	ご招待できる \\
\hline 
6&	お招きできます&		17&	ご招待できます \\
\hline 
7&	お招き致します&		18&	ご招待致します \\
\hline 
8&	お招き申します&		19&	ご招待申します \\
\hline 
9&	お招き申し上げます&	20&	ご招待申し上げます \\
\hline 
10&	お招き頂く&		21&	ご招待頂く \\
\hline 
11&	お招き頂きます&		22&	ご招待頂きます \\
\hline 
\end{tabular} 

\end{center}
\end{minipage}
\begin{minipage}[t]{0.5\hsize}
\begin{center}


\caption{発話意図:{\it invite},尊敬語} 
\label{tbl2:table4.2} 
\begin{tabular}{|c||l|c||l|} 
\hline 
No. & \multicolumn{1}{|c|}{和語系表現} & No. & \multicolumn{1}{|c|}{漢語系表現} \\
\hline 
1&	招く&			10&	招待する \\
\hline 
2&	招きます&		11&	招待します\\
\hline 
3&	お招きなさる&		12&	ご招待なさる\\
\hline 
4&	お招きなさいます&	13&	ご招待なさいます\\
\hline 
5&	お招きになる&		14&	ご招待になる\\
\hline 
6&	お招きになります&	15&	ご招待になります\\
\hline 
7&	お招き下さる&		16&	ご招待下さる\\
\hline 
8&	お招き下さいます&	17&	ご招待下さいます\\
\hline 
9&	お招きになられる&	18&	ご招待になられる\\
\hline 
\end{tabular} 

\end{center}
\end{minipage}
\end{tabular}
}
\end{center}
\end{table} 

\vspace{-\baselineskip} 

\setcounter{appndnum}{0}

\addtocounter{appndnum}{1}
\setcounter{figure}{0}

\begin{figure}[htbp]
\begin{tabular}{cc}
\begin{minipage}{0.5\hsize}
\begin{center}

\epsfxsize=0.8\hsize
\epsffile{./figa11.eps}
\caption{ $\mu$ (発話意図:{\it use},謙譲語)}
\label{fig2:figure1.1}

\end{center}
\end{minipage}
\begin{minipage}{0.5\hsize}
\begin{center}

\epsfxsize=0.8\hsize
\epsffile{./figa12.eps}
\caption{ $\mu$ (発話意図:{\it use},尊敬語)}
\label{fig2:figure1.2}

\end{center}
\end{minipage}
\end{tabular}
\end{figure}  


\addtocounter{appndnum}{1}
\setcounter{figure}{0}

\begin{figure}[htbp]
\begin{tabular}{cc}
\begin{minipage}{0.5\hsize}
\begin{center}

\epsfxsize=0.8\hsize
\epsffile{./figa21.eps}
\caption{ $\mu$ (発話意図:{\it explain},謙譲語)}
\label{fig2:figure2.1}

\end{center}
\end{minipage}
\begin{minipage}{0.5\hsize}
\begin{center}

\epsfxsize=0.8\hsize
\epsffile{./figa22.eps}
\caption{ $\mu$ (発話意図:{\it explain},尊敬語)}
\label{fig2:figure2.2}

\end{center}
\end{minipage}
\end{tabular}
\end{figure}  


\addtocounter{appndnum}{1}
\setcounter{figure}{0}

\begin{figure}[htbp]
\begin{tabular}{cc}
\begin{minipage}{0.5\hsize}
\begin{center}

\epsfxsize=0.8\hsize
\epsffile{./figa31.eps}
\caption{ $\mu$ (発話意図:{\it inform},謙譲語)}
\label{fig2:figure3.1}

\end{center}
\end{minipage}
\begin{minipage}{0.5\hsize}
\begin{center}

\epsfxsize=0.8\hsize
\epsffile{./figa32.eps}
\caption{ $\mu$ (発話意図:{\it inform},尊敬語)}
\label{fig2:figure3.2}

\end{center}
\end{minipage}
\end{tabular}
\end{figure}  


\addtocounter{appndnum}{1}
\setcounter{figure}{0}

\begin{figure}[htbp]
\begin{tabular}{cc}
\begin{minipage}{0.5\hsize}
\begin{center}

\epsfxsize=0.8\hsize
\epsffile{./figa41.eps}
\caption{ $\mu$ (発話意図:{\it invite},謙譲語)}
\label{fig2:figure4.1}

\end{center}
\end{minipage}
\begin{minipage}{0.5\hsize}
\begin{center}

\epsfxsize=0.8\hsize
\epsffile{./figa42.eps}
\caption{ $\mu$ (発話意図:{\it invite},尊敬語)}
\label{fig2:figure4.2}

\end{center}
\end{minipage}
\end{tabular}
\end{figure}  


\addtocounter{appndnum}{1}
\setcounter{figure}{0}

\begin{figure}[htbp]
\begin{tabular}{cc}
\begin{minipage}{0.5\hsize}
\begin{center}

\epsfxsize=0.8\hsize
\epsffile{./figa51.eps}
\caption{ $d$ (発話意図:{\it use},謙譲語)}
\label{fig2:figure5.1}

\end{center}
\end{minipage}
\begin{minipage}{0.5\hsize}
\begin{center}

\epsfxsize=0.8\hsize
\epsffile{./figa52.eps}
\caption{ $d$ (発話意図:{\it use},尊敬語)}
\label{fig2:figure5.2}

\end{center}
\end{minipage}
\end{tabular}
\end{figure}  


\addtocounter{appndnum}{1}
\setcounter{figure}{0}

\begin{figure}[htbp]
\begin{tabular}{cc}
\begin{minipage}{0.5\hsize}
\begin{center}

\epsfxsize=0.8\hsize
\epsffile{./figa61.eps}
\caption{ $d$ (発話意図:{\it explain},謙譲語)}
\label{fig2:figure6.1}

\end{center}
\end{minipage}
\begin{minipage}{0.5\hsize}
\begin{center}

\epsfxsize=0.8\hsize
\epsffile{./figa62.eps}
\caption{ $d$ (発話意図:{\it explain},尊敬語)}
\label{fig2:figure6.2}

\end{center}
\end{minipage}
\end{tabular}
\end{figure}  


\addtocounter{appndnum}{1}
\setcounter{figure}{0}

\begin{figure}[htbp]
\begin{tabular}{cc}
\begin{minipage}{0.5\hsize}
\begin{center}

\epsfxsize=0.8\hsize
\epsffile{./figa71.eps}
\caption{ $d$ (発話意図:{\it inform},謙譲語)}
\label{fig2:figure7.1}

\end{center}
\end{minipage}
\begin{minipage}{0.5\hsize}
\begin{center}

\epsfxsize=0.8\hsize
\epsffile{./figa72.eps}
\caption{ $d$ (発話意図:{\it inform},尊敬語)}
\label{fig2:figure7.2}

\end{center}
\end{minipage}
\end{tabular}
\end{figure}  


\addtocounter{appndnum}{1}
\setcounter{figure}{0}

\begin{figure}[htbp]
\begin{tabular}{cc}
\begin{minipage}{0.5\hsize}
\begin{center}

\epsfxsize=0.8\hsize
\epsffile{./figa81.eps}
\caption{ $d$ (発話意図:{\it invite},謙譲語)}
\label{fig2:figure8.1}

\end{center}
\end{minipage}
\begin{minipage}{0.5\hsize}
\begin{center}

\epsfxsize=0.8\hsize
\epsffile{./figa82.eps}
\caption{ $d$ (発話意図:{\it invite},尊敬語)}
\label{fig2:figure8.2}

\end{center}
\end{minipage}
\end{tabular}
\end{figure}  


\addtocounter{appndnum}{1}
\setcounter{figure}{0}

\begin{figure}[htbp]
\begin{tabular}{cc}
\begin{minipage}{0.5\hsize}
\begin{center}

\epsfxsize=0.8\hsize
\epsffile{./figa91.eps}
\caption{ $\delta$ (発話意図:{\it use},謙譲語)}
\label{fig2:figure9.1}

\end{center}
\end{minipage}
\begin{minipage}{0.5\hsize}
\begin{center}

\epsfxsize=0.8\hsize
\epsffile{./figa92.eps}
\caption{ $\delta$ (発話意図:{\it use},尊敬語)}
\label{fig2:figure9.2}

\end{center}
\end{minipage}
\end{tabular}
\end{figure}  


\addtocounter{appndnum}{1}
\setcounter{figure}{0}

\begin{figure}[htbp]
\begin{tabular}{cc}
\begin{minipage}{0.5\hsize}
\begin{center}

\epsfxsize=0.8\hsize
\epsffile{./figa101.eps}
\caption{ $\delta$ (発話意図:{\it explain},謙譲語)}
\label{fig2:figure10.1}

\end{center}
\end{minipage}
\begin{minipage}{0.5\hsize}
\begin{center}

\epsfxsize=0.8\hsize
\epsffile{./figa102.eps}
\caption{ $\delta$ (発話意図:{\it explain},尊敬語)}
\label{fig2:figure10.2}

\end{center}
\end{minipage}
\end{tabular}
\end{figure}  


\addtocounter{appndnum}{1}
\setcounter{figure}{0}

\begin{figure}[htbp]
\begin{tabular}{cc}
\begin{minipage}{0.5\hsize}
\begin{center}

\epsfxsize=0.8\hsize
\epsffile{./figa111.eps}
\caption{ $\delta$ (発話意図:{\it inform},謙譲語)}
\label{fig2:figure11.1}

\end{center}
\end{minipage}
\begin{minipage}{0.5\hsize}
\begin{center}

\epsfxsize=0.8\hsize
\epsffile{./figa112.eps}
\caption{ $\delta$ (発話意図:{\it inform},尊敬語)}
\label{fig2:figure11.2}

\end{center}
\end{minipage}
\end{tabular}
\end{figure}  


\addtocounter{appndnum}{1}
\setcounter{figure}{0}

\begin{figure}[htbp]
\begin{tabular}{cc}
\begin{minipage}{0.5\hsize}
\begin{center}

\epsfxsize=0.8\hsize
\epsffile{./figa121.eps}
\caption{ $\delta$ (発話意図:{\it invite},謙譲語)}
\label{fig2:figure12.1}

\end{center}
\end{minipage}
\begin{minipage}{0.5\hsize}
\begin{center}

\epsfxsize=0.8\hsize
\epsffile{./figa122.eps}
\caption{ $\delta$ (発話意図:{\it invite},尊敬語)}
\label{fig2:figure12.2}

\end{center}
\end{minipage}
\end{tabular}
\end{figure}  




\bibliographystyle{jnlpbbl}
\begin{thebibliography}{}

\bibitem[\protect\BCAY{Brown. \BBA\ Levinson.}{Brown. \BBA\
  Levinson.}{1987}]{Brown他 1987}
Brown., P.\BBACOMMA\  \BBA\ Levinson., S. \BBOP 1987\BBCP.
\newblock {\Bem Politeness - Some universals of language usage -}.
\newblock Cambridge.

\bibitem[\protect\BCAY{文化庁文化部国語課}{文化庁文化部国語課}{1995}]{文化庁文
化部国語課 1995}
文化庁文化部国語課 \BBOP 1995\BBCP.
\newblock \Jem{国語に関する世論調査 平成7年4月調査}.
\newblock 大蔵省印刷局.

\bibitem[\protect\BCAY{林\JBA 南}{林\JBA 南}{1974}]{林他 1974}
林四郎\JBA  南不二男編集 \BBOP 1974\BBCP.
\newblock \Jem{敬語講座1 敬語の体系}.
\newblock 明治書院.

\bibitem[\protect\BCAY{石村}{石村}{1989}]{石村 1989}
石村貞夫 \BBOP 1989\BBCP.
\newblock \Jem{統計解析のはなし}.
\newblock 東京図書.

\bibitem[\protect\BCAY{蒲谷\JBA 川口\JBA 坂本}{蒲谷\Jetal
  }{1991}]{蒲谷他 1991}
蒲谷宏\JBA 川口義一\JBA  坂本恵 \BBOP 1991\BBCP.
\newblock \JBOQ 待遇表現研究の構想\JBCQ\
\newblock \Jem{早稲田大学日本語研究教育センター紀要3}, \BPGS\ 1--22.

\bibitem[\protect\BCAY{菊池}{菊池}{1997}]{菊地 1997}
菊池康人 \BBOP 1997\BBCP.
\newblock \Jem{敬語}.
\newblock 講談社.

\bibitem[\protect\BCAY{菊池}{菊池}{2003}]{菊地 2003}
菊池康人 \BBOP 2003\BBCP.
\newblock \JBOQ 敬語とその主な研究テーマの概観\JBCQ\
\newblock \Jem{朝倉日本語講座8 敬語}. 朝倉書店.

\bibitem[\protect\BCAY{三浦}{三浦}{1973}]{三浦 1973}
三浦新他編 \BBOP 1973\BBCP.
\newblock \Jem{官能検査ハンドブック}.
\newblock 日科技連.

\bibitem[\protect\BCAY{水谷}{水谷}{1995}]{水谷 1995}
水谷静夫 \BBOP 1995\BBCP.
\newblock \Jem{待遇表現提要}.
\newblock 計量計画研究所.

\bibitem[\protect\BCAY{中屋}{中屋}{1970}]{中屋 1970}
中屋澄子 \BBOP 1970\BBCP.
\newblock \JBOQ Scheffe の一対比較法の一変法\JBCQ\
\newblock \Jem{第11回官能検査大会報文集}. 日本科学技術連盟.

\bibitem[\protect\BCAY{荻野}{荻野}{1980}]{荻野 1980}
荻野綱男 \BBOP 1980\BBCP.
\newblock \JBOQ 敬語表現の長さと丁寧さ 札幌における敬語調査から(3)\JBCQ\
\newblock \Jem{計量国語学}, {\Bbf 12}  (6), 264--271.

\bibitem[\protect\BCAY{荻野}{荻野}{1986}]{荻野 1986}
荻野綱男 \BBOP 1986\BBCP.
\newblock \JBOQ 待遇表現の社会言語学的研究\JBCQ\
\newblock \Jem{日本語学}, {\Bbf 5}  (12), 55--63.

\bibitem[\protect\BCAY{Scheffe}{Scheffe}{1952}]{Scheffe 1952}
Scheffe, H. \BBOP 1952\BBCP.
\newblock \BBOQ An analysis of variance for paired comparisons\BBCQ\
\newblock {\Bem J. Am. Statist. Assoc.}, {\Bbf 47}, 381--400.

\bibitem[\protect\BCAY{白土\JBA 井佐原}{白土\JBA 井佐原}{2002}]{白土他 2002}
白土保\JBA  井佐原均 \BBOP 2002\BBCP.
\newblock \JBOQ 待遇表現選択ストラテジの数値モデル\JBCQ\
\newblock \Jem{電子情報通信学会論文誌}, {\Bbf J85-A}  (3), 389--397.

\bibitem[\protect\BCAY{白土\JBA 丸元\JBA 井佐原}{白土\Jetal
  }{2003}]{白土他 2003}
白土保\JBA 丸元聡子\JBA  井佐原均 \BBOP 2003\BBCP.
\newblock \JBOQ 敬語に対する認識の混乱に関する定量的分析\JBCQ\
\newblock \Jem{計量国語学}, {\Bbf 24}  (2), 65--80.

\bibitem[\protect\BCAY{鈴木\JBA 林}{鈴木\JBA 林}{1984}]{鈴木他 1984}
鈴木一彦\JBA  林巨樹編 \BBOP 1984\BBCP.
\newblock \Jem{研究資料日本文法9 敬語法編}.
\newblock 明治書院.

\bibitem[\protect\BCAY{田中良久}{田中良久}{1977}]{田中 1977}
田中良久 \BBOP 1977\BBCP.
\newblock \Jem{心理学的測定 第2版}.
\newblock 東京大学出版会.

\end{thebibliography}


\begin{biography}
\biotitle{略歴}
\bioauthor{丸元 聡子}{
1996年東京女子大学文理学部日本文学科卒業.同年,財団法人計量計画研究所入所,
2000年--2001年,通信・放送機構/通信総合研究所(現:情報通信研究機構)に出向.
現在,財団法人計量計画研究所言語情報研究室研究員.
自然言語処理の研究に従事.言語処理学会,計量国語学会,電子情報通信学会,各会員.}

\bioauthor{白土 保}{
電気通信大計算機科学科卒.1986年電波研究所(現NICT)入所.
鹿島センター,平磯センター,関西先端研究センター,けいはんなセンター,総務省情報通信政策局勤務を経てけいはんなセンター勤務.
主任研究員.専門分野は,言語心理,音楽音響,感性情報処理.日本音響学会,電子情報通信学会,各会員.工学博士.}


\bioauthor{井佐原 均}{
1980年京都大学大学院修士課程修了.博士(工学).同年,通商産業省電子技術総合研究所入所.
1995年郵政省通信総合研究所.現在,独立行政法人情報通信研究機構けいはんな情報通信融合研究センター
自然言語グループリーダーおよびタイ自然言語ラボラトリー長.
自然言語処理,語彙意味論の研究に従事.
言語処理学会,情報処理学会,人工知能学会,日本認知科学会,ACL,各会員.}


\bioreceived{受付}
\biorevised{再受付}
\bioaccepted{採録}

\end{biography}

\end{document}

