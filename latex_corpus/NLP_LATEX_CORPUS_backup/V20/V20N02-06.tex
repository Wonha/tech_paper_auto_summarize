    \documentclass[japanese]{jnlp_1.4}
\usepackage{jnlpbbl_1.3}
\usepackage[dvips]{graphicx}
\usepackage{amsmath}
\usepackage{hangcaption_jnlp}
\usepackage{array}
\usepackage{pifont}
\usepackage{udline}
\setulminsep{1.2ex}{0.2ex}
\let\underline

\newif\ifDRAFT
\DRAFTfalse
\usepackage{ascmac}
\usepackage{multirow}
\usepackage{color}
\newcommand{\modified}[1]{}
\def\timex{}
\def\timexii{}
\def\timexiii{}
\def\tid{}
\def\type{}
\def\value{}
\def\valuefromsurface{}
\def\temporalfunction{}
\def\definite{}
\def\freq{}
\def\quant{}
\def\mod{}
\def\tlink{}

\Volume{20}
\Number{2}
\Month{June}
\Year{2013}

\received{2012}{10}{24}
\revised{2012}{12}{28}
\accepted{2013}{2}{10}

\setcounter{page}{201}

\jtitle{『現代日本語書き言葉均衡コーパス』に対する\\時間情報アノテーション}
\jauthor{小西  光\affiref{Author_1} \and 浅原 正幸\affiref{Author_1} \and 前川喜久雄\affiref{Author_1}}
\jabstract{
時間情報表現は,テキスト中に記述される事象の生起時刻を推定するための重要な手がかりである.
時間情報表現を含む数値表現の抽出は,固有表現抽出の部分問題として解かれてきた.
英語においては,評価型国際会議が開かれ,
時間情報表現のテキストからの切り出しだけではなく,曖昧性解消・正規化のための様々な手法
が提案されている.さらに,時間情報と事象とを関連づけるアノテーション(タグづけ)基準 TimeML の定義や
新聞記事にアノテーションを行ったコーパス TimeBank の整備が進んでいる.
一方,日本語においては時間情報処理に必要なアノテーション基準の定義及びコーパスの整備が進んでいない.
本稿では,TimeML の時間情報表現を表す \timexiii タグに基づいた
時間情報のアノテーション基準を日本語向けに再定義し,
『現代日本語書き言葉均衡コーパス』(BCCWJ)コアデータの一部にアノテーションを行った.
問題点を検討し,今後事象の生起時刻を推定するために必要な課題を考察する.
}
\jkeywords{時間情報処理,コーパスアノテーション}

\etitle{Temporal Information Annotation on `Balanced Corpus of Contemporary Written Japanese'}
\eauthor{Hikari Konishi\affiref{Author_1} \and Masayuki Asahara\affiref{Author_1} \and Kikuo Maekawa\affiref{Author_1}} 
\eabstract{
Temporal information is important for grounding event expressions  
on a timeline.
Temporal expression extraction has been performed as numerical representation extraction, which is a subtask of named entity extraction.
For English texts, evaluation workshops were held in which temporal expressions
were extracted and normalized.
An annotation schema, {\it TimeML}, was designed to annotate events and temporal expressions, and several annotated corpora of newswire texts were developed.
However, a schema for temporal information and normalization of Japanese texts
has not been designed.
This paper proposes an annotation schema, which is based on {\it TimeML}, for Japanese temporal information.
We annotate the temporal information in parts of the `{\it Balanced Corpus of Contemporary Written Japanese}'.
We identify several problems in the annotation and discuss the steps to be taken
to ground Japanese event expressions on a timeline.
}
\ekeywords{Temporal Information Processing, Corpus Annotation}

\headauthor{小西,浅原,前川}
\headtitle{時間情報タグつきテキストコーパスの作成}

\affilabel{Author_1}{人間文化研究機構国立国語研究所}{National Institute for Japanese Language and Linguistics}



\begin{document}
\maketitle


\section{はじめに}

情報検索や情報抽出において,テキスト中に示される事象を実時間軸上の時点もしくは
時区間に関連づけることが求められている.
Web 配信されるテキスト情報に関しては,文書作成日時(Document Creation Time: DCT)
が得られる場合,テキスト情報と文書作成日時とを関連づけることができる.
しかしながら,文書作成日時が得られない場合や,文書に記述されている事象発生日
時と文書作成日時が乖離する場合には他の方策が必要である.テキスト中に記述されている時間情報解析の精緻化が求められている.

時間表現抽出は,固有表現抽出の部分問題である数値表現抽出のタスクとして研究されてきた.
英語においては,評価型国際会議 MUC-6 (the sixth in a series of Message Understanding Conference)\cite{MUC6}で,
アノテーション済み共有データセットが整備され,
そのデータを基に各種の系列ラベリングに基づく時間表現の切り出し手法が開発されてきた.
TERN (Time Expression Recognition and Normalization)\cite{TERN} では,時間情報の曖昧性解消・正規化がタスクとして追加され,様々な時間表現解析器が開発された.
さらに,時間情報表現と事象表現とを関連づけるアノテーション基準 TimeML \cite{TimeML}が検討され,TimeML に基づくタグつきコーパス TimeBank \cite{TimeBank}などが整備された.
2007 年には,時間情報表現—事象表現間及び2事象表現間の時間的順序関係を推定する
評価型ワークショップ SemEval-2007 のサブタスク TempEval \cite{TempEval} が
開かれ,種々の時間的順序関係推定器が開発された.
後継のワークショップ SemEval-2010 のサブタスク TempEval-2 \cite{TempEval2}
では,英語だけでなく,イタリア語,スペイン語,中国語,韓国語を含めた
\modified{5}言語が対象となった.
\modified{2013年に開かれる SemEval-2013 のサブタスク TempEval-3 では,データを大規模
化した英語,スペイン語が対象となっている.}

一方,日本語においては IREX (Information Retrieval and Extraction Exercise) ワークショップ \cite{IREX} の固有表現抽出タスクの部分問題として時間情報表現抽出が定義されているのみで,時間情報の曖昧性解消・正規化に関するデータが構築されていなかった.

そこで,我々は TimeML に基づいた日本語に対する時間情報アノテーション基準を定義し,
時間情報の曖昧性解消・正規化を目的とした時間情報タグつきコーパスを構築した.
\modified{他言語のコーパスが新聞記事のみを対象としているのに対し,本研究では均衡
コーパスである『現代日本語書き言葉均衡コーパス』(Balanced Corpus of Contemporary Written Japanese;
以下 ``BCCWJ'')を対象としており,新聞記事だけでなく,一般書籍・雑誌・ブログなど
に出現する多様な時間情報表現を対象としている.
本稿ではアノテーション基準を示すとともに,アノテーションしたコーパスの詳細について示す.}

以下,2節では時間情報表現についての背景について概観する.
3節では,対象とする時間情報表現について詳しく述べる.4節では策定した日本語時間情
報表現に対するアノテーション基準を示す.
\modified{5節でアノテーションにおける日本語特有の問題について説明する.}
\modified{6節でアノテーション作業環境を示す.}
\modified{7}節で実際にアノテーションしたコーパスの分析を
行う.最後にまとめと今後の課題を示す.



\section{背景 \label{sec:previous_work}}

\subsection{\modified{時間情報表現に関する関連研究}}

テキスト中の時間情報表現を分析する研究は日本語以外の言語で進んでおり,時間表現の
文字列の切り出しや正規化のみならず,時間表現と事象表現の関連づけなどが行われている.
表 \ref{tbl:previous_work} に英語もしくは日本語を対象とした時間情報表現に関連す
る研究を示す.以下,まず英語の時間情報表現に関する代表的な研究を俯瞰する.
次に日本語の数少ない時間情報表現に関する研究を示す.

\begin{table}[b]
\caption{関連研究}
\label{tbl:previous_work}
\input{06table01.txt}
\end{table}

英語においては,評価型国際会議 MUC-6 \cite{MUC6} の1タスクである固有表現抽出の中に時間情報表現の抽出が含まれていた.MUC-6 で定義されている時間情報表現タグ \timex は日付表現({\tt @type="DATE"}) と 時刻表現 ({\tt type="TIME"}) からなる.アノテーション対象は絶対的な日付・時刻を表す表現にのみ限定され,''last year'' などといった相対的な日付・時刻表現は含まれていない.
この MUC-6 のアノテーション基準 \timex に対し,Setzer は時間情報表現の正規化に関するアノテーション基準を提案している\cite{Setzer-2001}.
評価型国際会議 TERN \cite{TERN} では,時間情報表現検出に特化したタスクを設定している.TERN で定義された時間表現情報タグ \timexii は,相対的な日付・時刻表現,時間表現や頻度集合表現が検出対象として追加されている.
\modified{時間表現の正規化情報を記述する}ISO-8601 形式を拡張した \value 属性などが設
計され,\modified{こちらも}自動解析対象となっている.
その後,Pustejovsky らによりアノテーション基準 TimeML \cite{TimeML}が提案されている.
その中では,TERN で用いられている \timexii を拡張した \timexiii が提案され,さら
に時間情報表現と事象表現の時間的順序関係を関連づけるための情報が付加される.これらの情報は人手でアノテーションすることを目的に設計され,
TimeBank \cite{TimeBank}や Aquaint TimeML Corpusなどの人手によるタグつきコーパスの整備が行われた.
これらのコーパスに基づく時間情報表現の自動解析\cite{Boguraev-2005,Mani-2006}が試みられたが,タグの情報に不整合があったり,付与されている時間的順序関係ラベルに偏りがあったりなど扱いにくいものであった\cite{Boguraev-2006}.
2007 年に開かれた SemEval 2007 の 1 タスク TempEval \cite{TempEval} では,時間的順序関係のラベルを簡略化し,人手で見直したデータによる時間的順序関係同定のタスクが行われた.
このタスクでは,時間表現に対して正規化された \value 属性などが付与されており,事象表現の時間的順序関係同定に利用してよい.
TempEval-2 \cite{TempEval2} では,英語だけでなく,イタリア語,スペイン語,中国語,韓国語に関しても同様なデータを利用したタスクが設定された.

日本語においては,IREX \cite{IREX}の 1 タスクとして,固有表現抽出タスクが設定さ
れた.IREX の時間情報では,日付・時刻表現を対象にし,相対的な表現が定義に含められている.
また,関根らは拡張固有表現体系\fullcite{Sekine-2002}を提案し,辞書/オントロジやコー
パスの作成などを行っており,BCCWJ にも同じ体系の拡張固有表現タグが付与されている
\cite{Hashimoto-2010}.
日本語においては,表現の分類の体系化が進んでいるが,正規化のための研究は他言語と
比べて遅れをとっている.



\subsection{\modified{コーパスアノテーションの標準化}}

\modified{本研究はコーパスアノテーションにおける標準化活動の観点から重要である.}

\modified{国際標準化機構(International Organization for Standardization: ISO) の標準化技術委員会(Technical Committee) TC 37 は ``Terminology and other language and content
resources'' と題し,言語資源に関するさまざまな標準化を提案している.
そのなかに分科会(Structure of the committee) が四つ設定されているが,
TC 37/SC 4 が言語資源管理(Language resource management; LRM)に関する国際規格の規定を行っている.
TC 37/SC 4 はさらに以下に示すような作業部会を六つ設定しており,さまざまな形式・出自の一次言語
データに対するアノテーションや XML に代表される汎用マークアップ言語に基づくアノ
テーションの表現形式についての仕様記述言語を設計している.}

\begin{itemize}
\item \modified{TC 37/SC 4/WG 1 	Basic descriptors and mechanisms for language resources \\
言語資源に関する情報を記述するための作業部会}
\item \modified{TC 37/SC 4/WG 2 	Semantic annotation \\
アノテーションと表現方法を議論する作業部会}
\item \modified{TC 37/SC 4/WG 3 	Multilingual information representation \\
多言語対訳テキストに特化した作業部会} 
\item \modified{TC 37/SC 4/WG 4 	Lexical resources \\
言語資源そのものに関する作業部会}
\item \modified{TC 37/SC 4/WG 5 	Workflow of language resource management \\
言語資源管理の作業手順を議論する作業部会}
\item \modified{TC 37/SC 4/WG 6 	Linguistic annotation \\
言語情報アノテーションを議論する作業部会}
\end{itemize}

\modified{TimeML 開発者は,作業部会 TC 37/SC 4/WG 2 と連携を取りながら,
Semantic Annotation Framework(SemAF)-Time (ISO-24617-1: 2012) を2012年に正式に制
定した.}

\modified{日本の言語資源整備は,実データを作成せずに標準化活動を行うものと実デー
タを作成するが標準化活動を無視して行うものとに二極化している.
一方,国際標準の中には,実世界上の特定のモノもしくはコトに関係づけられる言語横断
的な標準化が有効なアノテーションと,言語の表現形態・表現機能のような言語横断的な
標準化がそぐわないアノテーションとが混在している.
後者の作業の失敗が顕在化しており,アノテーションの標準化作業を低く評価する傾向がある.}

\modified{我々のグループは 2006年より TimeML 開発者から TimeML 関連の情
報を得ながら時間情報表現アノテーションと事象表現アノテーションに取り組んできた.
標準化に適した時間情報表現アノテーションと,標準化に適さない事象表現アノテーションを切り分けたうえで,前者について ISO-TimeML に準拠する日本語版 \timexiii
アノテーション基準を検討し策定した.この部分が本研究の内容に相当する.
一方,後者についてはモダリティが豊かな日本
語の事象表現を国際標準に合わせてアノテーションすることが困難であり,別の方策でア
ノテーションすることを検討中である.}

\modified{日本において,国際標準に準拠している他のアノテーションデータとして,
策定中の SemAF-NE Named Entities(ISO-24617-3 制定中)に準拠している
BCCWJ に対する拡張固有表現体系アノテーション\cite{Hashimoto-2010}がある.}


\section{対象とする時間情報表現}

まず以下の例文を見て欲しい.

{\addtolength{\linewidth}{-6zw}\setlength{\leftskip}{3zw}
\begin{itembox}[l]{例文:}
{\small
彼は{\bf 2008年4月}から{\bf 週に3回}ジョギングを{\bf 1時間}行ってきたが,{\bf 昨
日}ケガをして走れなくなり,{\bf 今朝9時}に病院に行った.}
\end{itembox}
\par}

本稿の研究対象である時間情報表現\footnote{「時間情報表現」は\ding{"AC}「日付表現」
  (``DATE'')・\ding{"AD}「時刻表現」(``TIME'')・\ding{"AE}「時間表現」(``DURATION'')・
  \ding{"AF}「頻度集合表現」(``SET'')の4種類の表現を包含するものを指す.}は時間軸上の時点もしくは時区間を表現するテキスト中の文字列とする.
時間情報表現は以下の四つの分類に分けられる.
\ding{"AC}日付表現・\ding{"AD}時刻表現は「2008年4月」「昨日」「今朝9時」といった,時点もしくは時区間の時間軸上の位置を定義することを目的として用いられる表現である.
\ding{"AE}時間表現は「1時間」といった,時間軸上の位置に焦点をあてずに時区間幅を定義することを目的として用いられる表現である.
\ding{"AF}頻度集合表現は「週に3回」といった,時間軸上複数の時区間を定義することを目的として用いられる表現である.

これら時間情報表現の曖昧性を解消しながら時間軸上の特定の区間に写像することを正規
化と呼ぶ.
日付・時刻表現において,表層の情報だけで正規化ができる表現と,文脈の情報を用いな
ければ正規化ができない表現がある.前者を定時間情報表現(fully-specified temporal expression) と呼び,後者を不定時間情報表現(underspecified temporal expression) と呼ぶ.
上の例では「2008年4月」が定時間情報表現であり,「昨日」「今朝9時」が不定時間情報表現である.
時間情報表現の正規化には計算機で扱う日付や時刻を扱うための国際標準 ISO-8601 形式\footnote{日付や時刻を {\tt YYYY-MM-DDThh:mm:ss} などといった数値と記号列で表記
  する標準.{\tt YYYY} は年を表す四ケタの数字が,{\tt MM}は月を表す二ケタの数
  字が,{\tt DD}は日を表す二ケタの数字が,{\tt hh}は24時間制で時刻を表す二ケタ
  の数字が,{\tt mm}は分を表す二ケタの数字が,{\tt ss}は秒を表す二ケタの数字が
  入る.Tは日付と時刻を分割する記号.様々な略記方法が提案され,例えば「2008年4月」は ``{\tt 2008-04}''と表記する.詳細については ISO-8601 に対応する日本工業規格 JISX0301「情報交換のためのデータ要素及び交換形式—日付及び時刻の表記」参照のこと.}
への変換が一般的である.しかしながら,
自然言語では表現できるが,ISO-8601 形式では直接表現できない時間情報表現がある.
  例えば,時間表現や頻度集合表現は時間軸上不定な場合が多く ISO-8601 形式だけでは
  表現できず,\modified{TimeMLの \timexiii においてさまざまな拡張形式が提案され
  ている.}

想定する時間情報表現解析では手がかりとして,テキストが書かれた日付・時刻を表す文書作成日時を用いることを仮定している.例えば,文書作成日時が 2008年9月1日であれば,「昨日」は 2008年8月31日(ISO-8601形式では ``{\tt 2008-08-31}'')を表し,「今朝9時」は 2008年9月1日午前9時(同``{\tt 2008-09-01T09:00}'')を表す.


\section{TimeML \timexiii タグに基づいた日本語時間情報アノテーション基準}

本節では日本語時間情報表現に対するアノテーション基準の概略を示す.
アノテーション基準は,言語資源管理に関する国際標準 ISO/TC 37/SC
4\footnote{http://www.tc37sc4.org/} において,2009年に採用された
ISO-24617-1 の元になっている TimeML \cite{TimeML} \timexiii タグの仕様に準拠\footnote{2003年の TimeML と区別するために ISO-24617-1 の基準は ISO-TimeML
と呼ばれている.国際標準として,英語だけでなく他言語の時間情報表現をアノテーションす
るために仕様が拡張されている.本研究の \timexiii タグは ISO-TimeML にも準拠し
ている.}している.
以下,\timexiii のタグの\modified{日本語適応について}説明する.
細かな点で日本語に合うように変更しており,合わない部分については次節で説明する.


\subsection{アノテーション対象}

アノテーション対象は日付表現・時刻表現・時間表現・頻度集合表現の 4 種類である.
図\ref{fig:example}にアノテーション事例を示す.

\begin{figure}[b]
\input{06fig01.txt}
\caption{アノテーション例}
\label{fig:example}
\end{figure}

日付表現は「一九二九年二月」「前日」のような日暦に焦点をあてた表現である.
時刻表現は「午前十時ごろ」「午後六時ごろ」「昼」「九日昼」のような
一日のうちのある時点に焦点をあてた表現である.
日付表現と時刻表現の区別は時間軸上の粒度の区別でしかない.
便宜上不定の現在を表す「今」という表現を時刻表現に分類する.
時間表現は「その間」のような時間軸上の両端に焦点をあてておらず,
期間を表すことに焦点をあてている表現である.
頻度集合表現は「毎日」のような複数の日付・時刻・時間に焦点をあてた表現である.
この分類は,解析の方便のために導入したものである.
時間軸上一つもしくは複数の時点・時区間を表現するものをアノテーション対象である時間情報表現とする.

現在のアノテーション基準では \timexiii タグの入れ子を許さない.
日付・時刻表現の線形結合はこれを一つの日付・時刻表現として切り出す.
例えば「九日昼」のように日付表現と時刻表現が連接する場合には一つの時刻表現として切り出す.
時間を表す際に,開始時点と終了時点を明示している場合には,
開始時点と終了時点とを別々の日付・時刻表現として切り出す.
例えば「午前十時ごろから午後六時ごろまで」は一つの時間表現として切り出さず,「午
  前十時ごろ」と「午後六時ごろ」の二つの時刻表現として切り出す.
事象が起こる期間を表すために,今後,関連する事象表現に対し,
この二つの時刻表現への参照関係を付与する予定である.
頻度集合表現は,文字列上できるだけ短い単位を切り出す.
例えば「毎日」を頻度集合表現として切り出すが,
「毎日午前十時ごろから午後六時まで」は現在のところ頻度集合表現として切り出していない.


\subsection{\timexiii の属性}

\timexiii タグの属性のう
ち \tid, \type, \value, \valuefromsurface, \freq, \quant, \mod を概説する.
また,作業・分析用に導入した \definite について説明する.

\tid 属性は一文書中の各時間情報表現に付与される識別子である.
各時間情報表現を一意に同定するために用い,今後同一指示,参照,事象表現との時間的順序を表す際に用いる.

\type 属性は {\tt DATE}, {\tt TIME}, {\tt DURATION}, {\tt SET} の四つの値を持つ.
それぞれ日付表現・時刻表現・時間表現・頻度集合表現を意味する.

\value 及び \valuefromsurface 属性は時間情報表現が含意する日付・時刻・時間の値を表す.
値として ISO-8601 形式を自然言語表現向けに拡張したものを用いる.
このうち \value は文脈情報を用いて正規化を行った値を付与し,
\valuefromsurface 属性は文脈情報を用いずに文字列の表層表現のみから判定できる値を
付与する.

\modified{ここで \value と\valuefromsurface 属性の違いについて例を用いて説明する.
「2013年4月」という表現は,文脈を用いなくても表層の文字列から時間軸上に一意
に曖昧性解消ができる.
「4月」という表現に対して,文脈情報からそれが2013年の「4月」であるとわか
る場合には以下のように記述する(ここで属性にわりあてる値の詳細については \ref{subsec:value}節に示す).}
定時間情報表現は \value と \valuefromsurface の値は同じになるが,
不定時間情報表現は同じになるとは限らない.


{\addtolength{\linewidth}{-6zw}\setlength{\leftskip}{3zw}
\begin{itembox}[l]{\value と \valuefromsurface}
{\small 
{\tt $\langle$TIMEX3 @type="DATE" @value="2013-04" @valueFromSurface="2013-04"$\rangle$}
2013年4月{\tt $\langle$/TIMEX3$\rangle$}

{\tt $\langle$TIMEX3 @type="DATE" @value="2013" @valueFromSurface="2013"$\rangle$}
2013年{\tt $\langle$/TIMEX3$\rangle$}の予定ですが
{\tt $\langle$TIMEX3 @type="DATE" @value="2013-04" @valueFromSurface="XXXX-04"$\rangle$}
4月{\tt $\langle$/TIMEX3$\rangle$}は…}
\end{itembox}
\par}


\freq, \quant 属性は頻度集合表現に付与される頻度情報及び量化子情報である.
属性にわりあてる値の詳細を\ref{subsec:freqquant}節に示す.

\mod 属性は時間情報表現のモダリティを表す.
例えば 「2000年以前」をアノテーションするために
\mod 属性に {\tt ON\_OR\_BEFORE}という値をわりあてることにより「以前」というモダリティを表現する.
属性にわりあてる値の詳細を\ref{subsec:mod}節に示す.

\modified{\definite 属性は ``{\tt true}'', ``{\tt false}'' のいずれかの値を持ち,
\value 属性が,文脈情報により}定時間情報が得られる時間情報表現は ``{\tt true}''の値
を持ち,その他の時間情報表現は ``{\tt false}''の値を持つ.
\modified{言い換えると,日付・時刻表現が時間軸上の特定時区間に写像できる場合と
時間・頻度集合表現の時間幅が特定できる場合に ``{\tt true}''の値を持ち,そうでない場合
に ``{\tt false}''の値を持つ.}
\modified{例えば,以下の例はともに \temporalfunction が ``{\tt false}'' の例である.
「10日」という表現は,文脈から「4月」ということがわかるが,何年かまではわから
ないために定時間情報が得られない.なお,\definite 属性は作業・分析の便宜上導入したもので,元の ISO-TimeML の \timexiii には規定されていない.}

{\addtolength{\linewidth}{-6zw}\setlength{\leftskip}{3zw}
\begin{itembox}[l]{\temporalfunction}
{\small
{\tt $\langle$TIMEX3 @type="DATE" @value="XXXX-04" @valueFromSurface="XXXX-04" @definite="false"$\rangle$}
4月{\tt $\langle$/TIMEX3$\rangle$}の予定ですが
{\tt $\langle$TIMEX3 @type="DATE" @value="XXXX-04-10"
@valueFromSurface="XXXX-XX-10" @definite="false"$\rangle$}
10日{\tt $\langle$/TIMEX3$\rangle$}は…}
\end{itembox}
\par}

作成したコーパスに対し上記属性を付与した.
他の属性として,記事配信日時など特別な意味を表す時間情報表現に付与する {\tt @functionInDocument},同一指示を表す{\tt @anchorTimeID},時間表現の開始位置と終了位置を表す{\tt @beginPoint}, {\tt @endPoint},アノテーション時の問題点を自由記述する {\tt @comment} がある.
これらの情報は作業者が気づいた範囲で付与を行ったが完全ではない.


\subsection{\value 及び \valuefromsurface  \label{subsec:value}}

各表現に付与する \value 及び \valuefromsurface は ISO-8601 形式を基として,
自然言語が表す時間情報向けに拡張したものである.
\modified{\value は文脈情報を用いて正規化を行った値を付与し,
\valuefromsurface 属性は文脈情報を用いずに文字列の表層表現のみから判定できる値を
付与する.}
ISO-8601 の標準表記では,日付・時刻表現を {\tt XXXX-XX-XXTXX:XX:XX} の形で表す.

日付表現に対する値の事例を表\ref{tbl:date-value}に示す.
自然言語向けの拡張により,ISO-8601 では表現できない季節・四半期・年度などが表現できるようになっている.
曜日表現に対する値の事例を表\ref{tbl:date-value-dofw}に示す.
曜日表現が表す {\tt WXX} の数値部分は年内の暦週の番号を表す.
日本語でよく用いられる「第3水曜」のような月内の暦週の番号を表す方策がとられていない.
\modified{これについては 5.2節で詳説する.}

\begin{table}[b]
\begin{minipage}[t]{280pt}
\caption{日付表現に対する \value}
\label{tbl:date-value}
\input{06table02.txt}
\end{minipage}
\hfill
\begin{minipage}[t]{130pt}
\caption{曜日表現に対する \value}
\label{tbl:date-value-dofw}
\input{06table03.txt}
\end{minipage}
\end{table}
\begin{table}[b]
\caption{時刻表現に対する \value}
\label{tbl:time-value}
\input{06table04.txt}
\end{table}

時刻表現に対する値の事例を表\ref{tbl:time-value}に示す.
自然言語向けの拡張により,「朝」「昼」「夜」などが表現できるようになっている.
\modified{* が付与されている「未明」と「深夜」は日本語新聞記事に頻出したため
に \valuefromsurface の値を独自に導入した.
\value にはどちらも「夜」と同じく {\tt TNI}をわり
あてる.
詳しくは 5.3 節で説明する.}

時間表現に対する値の事例を表\ref{tbl:duration-value}に示す.
基本的に ISO-8601 の時間表現\footnote{ISO-8601 では時間を表現するために Time
  interval 形式 と Duration 形式の二つがあるが,ここでは Duration 形式を用いる.}と同じであり,
接頭辞として {\tt P} を付与し,その後に数値とともにそれぞれ年,月,日,時間,分,秒,週
を表す{\tt Y}, {\tt M}, {\tt D}, {\tt H}, {\tt M}, {\tt S}, {\tt W}を接尾辞として付与する.月(M)と分(M)を区別するために日と時間の境界に {\tt T}を付与する.

「今」「近年」「今後」など不定な表現に対する値の事例を表\ref{tbl:date-value-undef}に示す.
これらは全て自然言語向けに導入した値である.

\begin{table}[b]
\begin{minipage}[t]{0.5\textwidth}
\caption{時間表現に対する \value}
\label{tbl:duration-value}
\input{06table05.txt}
\end{minipage}
\begin{minipage}[t]{0.5\textwidth}
\caption{不定な表現に対する \value}
\label{tbl:date-value-undef}
\input{06table06.txt}
\end{minipage}
\end{table}

頻度集合表現は上記 \value 属性を流用しながら次節に示す \freq, \quant 属性を組み合わせることによって表現される.


\subsection{頻度集合表現に対する \freq 及び \quant 属性 \label{subsec:freqquant}}

頻度集合表現は \value, \freq, \quant 属性を組み合わせることにより複雑な
時間情報を表現する.

頻度情報を表すためには,期間を表す \value 属性とともに, \freq 属性に {\it n}{\tt X}をわりあてることにより,焦点をあてている期間中に事象が {\it n} 回起こることを示す.
例えば「週に2回」を表現する際には以下のようにアノテーションする\footnote{説明に不要な属性は省略して表示.以下同様.}.

{\addtolength{\linewidth}{-6zw}\setlength{\leftskip}{3zw}
\begin{itembox}[l]{「週に2回」}
{\small
{\tt $\langle$TIMEX3 type="SET" value="P1W" freq="2X"$\rangle$} 
週に2回{\tt $\langle$/TIMEX3$\rangle$}}
\end{itembox}
\par}

\quant 属性には「毎日」「毎週」「毎10月」といった表現に {\tt EACH} をわりあて,
「10日おき」「3日毎」といった表現に {\tt EVERY} をわりあてる.
この際 \value 属性には期間を表す値だけでなく,日付・時刻を表す値が入ることがある.
以下に例を示す.

{\addtolength{\linewidth}{-6zw}\setlength{\leftskip}{3zw}
\begin{itembox}[l]{「毎日」「毎10月」「10日おき」}
{\small
{\tt $\langle$TIMEX3 type="SET" value="P1D" quant="EACH"$\rangle$}
毎日{\tt $\langle$/TIMEX3$\rangle$} \\
{\tt $\langle$TIMEX3 type="SET" value="XXXX-10" quant="EACH"$\rangle$}
毎10月{\tt $\langle$/TIMEX3$\rangle$} \\
{\tt $\langle$TIMEX3 type="SET" value="P10D" quant="EVERY"$\rangle$}
10日おき{\tt $\langle$/TIMEX3$\rangle$} }
\end{itembox}
\par}

頻度集合表現は,できるだけ文字列上小さな単位で切り出しているため,
現在のところ上記定義で意味論的表示に曖昧性が生じていない.
例えば「毎日午前十時ごろから午後六時まで」
のような表現の場合,表現全体の単位で切り出すとすると,
\value, \freq, \quant 属性のみで曖昧性なく意味論的表示に落とすことは困難である.
これは,時間情報表現間の時間的順序関係のアノテーションにおいて今後対処していきたい.


\subsection{モダリティ修飾 \mod 属性 \label{subsec:mod}}

時間情報表現は接尾表現をともない,様々なモダリティを表現する.
\mod 属性は時刻・時間表現に対するモダリティ修飾情報である.
表\ref{tbl:mod}に取りうる値の一覧と例を示す.

\begin{table}[b]
\caption{\mod 属性に対する値}
\label{tbl:mod}
\input{06table07.txt}
\end{table}

日付・時刻・時間表現に共通して用いられる \mod 属性として
{\tt START}, {\tt MID}, {\tt END}, {\tt APPROX} がある.
例えば,「60年代初頭」「10月半ば」「約40年」は以下のようにアノテーションする.

{\addtolength{\linewidth}{-6zw}\setlength{\leftskip}{3zw}
\begin{itembox}[l]{「60年代初頭」「10月半ば」「約40年」}
{\small
{\tt $\langle$TIMEX3 type="DATE" value="196X" mod="START"$\rangle$}
60年代初頭
{\tt $\langle$/TIMEX3$\rangle$} \\
{\tt $\langle$TIMEX3 type="DATE" value="XXXX-10-XX" mod="MID"$\rangle$}
10月半ば{\tt $\langle$/TIMEX3$\rangle$} \\
{\tt $\langle$TIMEX3 type="DURATION" value="P40Y" mod="APPROX"$\rangle$}約40年
{\tt $\langle$/TIMEX3$\rangle$} }
\end{itembox}
\par}

日付・時刻表現に対する \mod 属性として,
{\tt BEFORE}, {\tt AFTER}, {\tt ON\_OR\_BEFORE}, {\tt ON\_OR\_AFTER}
がある.例えば「1998年以前」は以下のようにアノテーションする.

{\addtolength{\linewidth}{-6zw}\setlength{\leftskip}{3zw}
\begin{itembox}[l]{「1998年以前」}
{\small
{\tt $\langle$TIMEX3 type="DATE" value="1998" mod="ON\_OR\_BEFORE"$\rangle$}1998年以前
{\tt $\langle$/TIMEX3$\rangle$} }
\end{itembox}
\par}

時間表現に対する \mod 属性として,
{\tt EQUAL\_OR\_LESS}, {\tt EQUAL\_OR\_MORE}, 
{\tt LESS\_THAN}, {\tt MORE\_THAN}がある.
例えば「10分以内」は以下のようにアノテーションする.

{\addtolength{\linewidth}{-6zw}\setlength{\leftskip}{3zw}
\begin{itembox}[l]{「10分以内」}
{\small
{\tt $\langle$TIMEX3 type="DURATION" value="PT10M" mod="EQUAL\_OR\_LESS"$\rangle$}
10分以内
{\tt $\langle$/TIMEX3$\rangle$} }
\end{itembox}
\par}


\section{\timexiii 日本語適応上の問題点}

\modified{本節では $\langle$TIMEX3$\rangle$ の日本語適応について問題となった事例
について個別に紹介する.基本的には \valuefromsurface に日本語特有の表示を用い,
\value に TimeMLに適合した表示を用いる.
必要に応じて韓国語のデータ K-TimeML \cite{KTimeML}に準拠した Korean Timebank 1.0/2.0 でどのような扱いを行っているのかを付記する.}


\subsection{暦}

\modified{日本語に \timexiii を適応するうえで一番大きな問題として,和暦の問題が
ある.日本語では元号法に基づいた「昭和」「平成」などの年表記がテキスト中に出現す
 る.これについては \valuefromsurface に「明治」「大正」「昭和」「平成」を表
 現する ``M'', ``T'', ``S'', ``H'' の接頭子を二ケタの数字表現に付与することで記述
 し,\value に西暦に換算したものを記述することとする.上記四つ以外の元号について
 は \value 相当に西暦を記述するのみで対処する.
元号以外の時代名(例:「江戸時代」)については,時間情報表現として切り出しを行う
 のみで,\value 相当は空白とした.
 }

{\addtolength{\linewidth}{-6zw}\setlength{\leftskip}{3zw}
\begin{itembox}[l]{和暦の扱い}
{\small
{\tt $\langle$TIMEX3 type="DATE" @value="1985" @valueFromSurface="S60"$\rangle$}
昭和60年{\tt$\langle$/TIMEX3$\rangle$} }
\end{itembox}
\par}

\modified{旧暦に対しては,\valuefromsurface において日付表現の表示の末尾に ''Q''
を付与し旧暦を表し,明示的に新暦の暦日が記述されている場合にのみ \value 相当に記述する.}

\modified{皇紀に対しては,\valuefromsurface において表示の最初に ''JY''を付与し皇紀を表し,\value 相当に西暦を記述する.
}

\modified{韓国語(韓国・北朝鮮),中国語(中国・台湾)なども同様の問題が起き
うるが,公開されている文書を見る限り \value 相当に西暦を記述することで対処している.}


\subsection{月次の週番号の扱い}

\modified{欧米地域では年単位の週番号を利用する傾向がある一方,東アジア地域では「第三木曜」
など月単位の週番号を利用する傾向がある.
このため,韓国より \timexiii に対して月単位の週番号を記述可能にする拡張が提案さ
れている.具体的には以下のように\valuefromsurface の日相当部分に週番号を記述する.
\value にはカレンダーを参照することにより ISO-8601 の標準表記 {\tt XXXX-XX-XX}
形式の値をわりあてた.
 }

{\addtolength{\linewidth}{-6zw}\setlength{\leftskip}{3zw}
\begin{itembox}[l]{月次の週番号の扱い}
{\small
{\tt $\langle$TIMEX3 type="DATE" @value="2013-01-17" \\
@valueFromSurface="XXXX-XX-W3-4"$\rangle$}第三木曜{\tt$\langle$/TIMEX3$\rangle$} }
\end{itembox}
\par}


\subsection{\modified{その他日本語特有の表現}}


\modified{以下では日本語適応において問題となった雑多な事例について紹介する.基本
的に\valuefromsurface において日本語に限定した形式で正規化を行い,\value 相当
は正規表現などを用いて正規化を行う.}

\begin{itemize}
\item 旬(月の細分類)上旬,中旬,下旬 \\
\valuefromsurface においては ``1J''(上旬), ``2J''(中旬), ``3J''(下旬) などと日本
語に限定した形式を導入し,\value には ``XXXX-XX-(0[123456789]$|$10)'' (上旬) のように正規
表現を用いて正規化を行った.
\item 未明・午前・真夜中 \\
\valuefromsurface においては新聞記事などで頻出する表現に対応するため ``DW''(未
明; \value は ``NI''(夜)相当)・``FN''(午前; \value は ``MO''(朝)相当)・``MN''(真
夜中; \value は ``NI''(夜)相当)を導入した.
\item 学期 \\
\valuefromsurface においては 学期を表現する記号として ``G''を導入した.
\value は何学期制かが不明なので空白である事例が多い.
\item 国民の祝日 \\
国民の祝日は,国立天文台が官報で公表する「春分日」「秋分日」や
ハッピーマンデー制度が適用される四つの祝日も含めて \value 相当にその年の正
しい日付を記述する.
\pagebreak
\item 休日 \\
時間情報表現として切り出すが,具体的にいつなのかわからない場合が多く,
\value は空白とした.
\item お盆 \\
\value 相当に ``SU''(夏)を記述する.
\item 十干十二支 \\
時間情報表現として切り出すが,具体的にいつなのかわからない場合が多く,
\value は空白とした.
\end{itemize}


\section{\modified{作業環境と作業対象}}

\modified{本節では作業環境と作業対象であるBCCWJについて詳しく説明する.}

\subsection{\modified{作業環境}}

\begin{figure}[b]
\begin{center}
\includegraphics{20-2ia6f2.eps}
\end{center}
\caption{XML Editor oXygen によるアノテーション}
\label{fig:oxygen}
\vspace{-0.5\Cvs}
\end{figure}

\modified{
アノテーション作業には XML Editor oXygen\footnote{http://www.oxygenxml.com/} を
利用した.
DTD や XML Schema を記述することにより時間表現の切り出し部分や属性にわりあてる値
などを統制することができる.
\pagebreak
時間表現の切り出しはマウスもしくはキーボードで対象となる文字列を選択したうえで
Ctrl-e とタイプし,タグを選択することで,XML タグ(閉じタグも含む)が挿入される.
この状態で画面右の属性項目を記述することにより,XML ファイルを編集することができ
る.}

\modified{
言語資源に対するアノテーションにおいて,ある一定の基準を守ったうえで複数の作業者
の主観を尊重してそれぞれの作業者間の判断の揺れを許す場合\footnote{例えば例文「柔道
家が先輩に勝った仲間を紹介した」に対して一般的な統語的な制約基準を教示したうえで,
本質的に2通りの解釈が残る場合に片方の解釈に矯正せずに作業者間の選択選考性の差異
の判断を委ねるような場合.},基準を厳格化し作業者
間の判断の揺れを許さない場合の2通りの統制手法がある.本論文の時間情報表現
の特定の時間軸上への写像作業は後者の統制手法を取るために図\ref{fig:pair}のように
ペアプログラミングのような手法を取った.1台の PC に,キーボード二つ・マウス二つ・
ディスプレイ二つを接続し,ディスプレイはミラーリングを行う.一つの PC を共有したうえで,作業者がアノテーション作業を行い,作業監督者がアノテーション仕様の改訂を行う.}

\begin{figure}[t]
\begin{center}
\includegraphics{20-2ia6f3.eps}
\end{center}
\caption{ペアプログラミングによるアノテーション作業}
\label{fig:pair}
\vspace{-1\Cvs}
\end{figure}

\modified{
アノテーション作業は1人の作業者と1人の作業監督者により行った.1回目
のアノテーション時の作業時間は 7時間15分週2日勤務で約2か月だった.
1回目のアノテーション直後(1回目見直し)と6か月後(2回目見直し)に見直しを行った.
1回目の見直し作業はアノテーション結果を帳票形式で出力したうえで,属性に関する見
直しを行った.
2回目の見直し作業は再度 XML Editor 上で行った.
アノテーションデータは1回目の見直し作業が終わった時点で公開しており,
利用者から誤りの指摘があった場合に修正を行っている.}

\modified{このアノテーション作業の準備段階の \timexiii の切り出しの一致率は
95\% 以上であり,切り出しの不一致の多くが単純な見落としであった.属性についての
一致率は,一つの時刻・時間に複数の意味論的表示を許す \value,  \valuefromsurface
\footnote{例えば,「30分」を表現する \value として ``PT30M'' と ``PT0.5H''の両方が許されている.}を除いて90\%以上である.}


\subsection{\modified{作業対象}}

\modified{
作業対象である BCCWJ \cite{BCCWJ}について
\pagebreak
概説する.
BCCWJ のコアデータは,OW:  白書,PB: 書籍,PN: 新聞,OC:
  Yahoo! 知恵袋,PM: 雑誌,OY: Yahoo! ブログの六つのレジスタからなり,
それぞれ約5万語単位で,アノテーションすべき優先順位に基づいた部分集合が規定されている\footnote{\modified{BCCWJにおいては,均衡性を保つためにコアデータにアノテーションする際の優先順位($A>B>C>D>E$)が定義されており,その優先順位に基づいて部分集合が定義されている.}}.
上記全レジスタの部分集合 ``A'' と
比較的時間表現が多いレジスタであるPN(新聞)の部分集合 ``B''をアノテーション対象とした.}

\modified{表\ref{tbl:data}にデータの概要を示す.
表中「ファイル数」はアノテーションしたファイルの数,
ファイル数の「うち時間表現あり」は時間表現を一つ以上含むファイルの数
を表す.}

\begin{table}[t]
\caption{作業対象データ}
\label{tbl:data}
\input{06table08.txt}
\end{table}

\modified{OC, OY などのユーザー生成コンテンツはサンプリングの長さにもよるが
1ファイルに時間情報表現が一つも含まれないものがある一方,
OW, PB, PN, PM などのユーザー生成コンテンツ外のほとんどは
時間表現が必ず一つ以上含まれている.
OWの中で時間表現を一つも含まない1ファイルは「平成16年度森林・林業白書」であっ
た.
文単位では,OW, PNが時間情報表現を含む割合が最も多く,PB が時間情報表現を含む割
合が最も少なかった.}


\section{タグの分析}

\modified{本節ではアノテーションした情報について,時間情報表現の正規化の観点から分析を行う.
表\ref{tbl:result:stat}に文書作成日時を示すタグを除いた\type ごとのタグの出現数を
曖昧性解消の観点から二つの視点で四つに分割して示す.
一つ目の視点は \temporalfunction が ``{\tt true}'' か ``{\tt false}'' かである.
``{\tt true}'' の場合,時間軸上に時区間が特定可(``DURATION''と ``SET''は時間幅が特定
可)であることを意味し,
``{\tt false}'' の場合,時間軸上に時区間が特定不可であることを意味する.
二つ目の視点は \value と \valuefromsurface の値が一致する(``=''で表記)か,一致し
ない(``$\neq$''で表記)かである.
一致する場合人手による文脈を用いた正規化作業が行われていないことを意味し,
一致しない場合人手による文脈を用いた正規化作業が行われたことを意味する.}

\begin{table}[t]
\caption{\type 属性ごとの出現数と文脈による曖昧性解消可能性}
\label{tbl:result:stat}
\input{06table09.txt}
\end{table}

\modified{まず 日付表現(``DATE'') について,時区間特定可(\temporalfunction が ``{\tt true}'' )であるものの多くが,人手による曖昧性解消が行われている(\value $\neq$ \valuefromsurface)ことがわ
かる.このことから本アノテーションの目的とする時間表現の正規化作業の重要性がうか
がえる.レジスタにもよるが,OW 白書やPN 新聞など,出版年・発行年月日が明ら
かであるものほど曖昧性解消がよく行われる傾向がある.
日付表現の曖昧性解消は,和暦から西暦への換算や,西暦二ケタ表記から西暦四ケタ表記
への換算,さらに年が省略されている表現の文脈や文書作成日時に基づく年の補完による
ものがあり,白書の多くの事例がこの暦の換算作業であった.
他のレジスタは曖昧性がない表現が多いわけではなく,文書作成日時など曖昧性を解消するに足る情報がデータ中に含まれていないことにより,具体的な時間軸上の区間を指し示すことができない事例が多かった.
部分的に情報を補完すること(例えば「3日」という表現に対し9月であることまでがわ
かるが,何年であるかまではわからないので, {\tt \value="XXXX-09-03"}をわりあてること)も
含まれており,時区間特定不可(\temporalfunction が ``{\tt false}'' )であるが人手による曖
昧性解消が行われている(\value $\neq$ \valuefromsurface)ものがこれにあてはまる.
}

\modified{次に 時刻表現(``TIME'') については,PBの1件を除いて時区間特定可である
ものの殆どが人手による曖昧性解消が行われている. 
時刻表現の曖昧性解消は,日付が省略されている場合の日付の補完のほか,午前と午後の
曖昧性解消が含まれる.
日付表現と同様に出版年月日がわかる新聞記事の曖昧性解消がよく行われている一方,OC
Yahoo!\ 知恵袋やPM 雑誌は,お店の営業時間など時間軸上の特定の時刻を表現しないものが多かった.}

\modified{時間表現(``DURATION'') と頻度集合表現については,時間軸上の時区
間を特定することを目的とせず,時間幅が特定できれば \temporalfunction が
``{\tt true}'' になると定義している.実際に時間軸上の時区間に写像する際には,日付・
時刻表現や事象表現との時間的順序関係(TimeMLの \tlink)を定義することが必要になる.}


\section{おわりに}

本稿では\modified{BCCWJに対する日本語時間情報アノテーションについて}説明した.
\modified{アノテーションは各国で進められている国際標準 ISO-TimeML に定義され
た \timexiii タグに準拠している.他言語においては対象を新聞記事に限定しているの
に対し,本研究は6種類のレジスタを対象にアノテーションを行い,レジスタ横断的な分
析を行うとともに,日本語適応における問題点を調査した.}
\modified{作成したアノテーションデータはスタンドオフ形式で}公開する.
BCCWJ の DVD を入手することでタグつきテキストコーパスが復元できる.

以下,今後の展望を示す.

今回作成したテキストコーパスをベンチマークとして正規化を行う日本語時間表現解析器の開発を現在進めている.作成中の解析器では,
まず,表層文字列からわかる値をラティス上に展開し,セミマルコフモデルを用いて
曖性解消を行う.
解析対象表現—文書作成日時および
解析対象表現—隣接時間情報表現の時間的順序関係を今回作成した
タグつきコーパスを用いて機械学習器を用いて推定することにより,
不定時間情報表現に対する情報補完を行う.

今後,TimeML で行われている事象表現と時間表現間の時間的順序関係(TimeML中の \tlink)付与を
進めていきたい.
そのためには,対象となる事象表現の策定,事象表現に対する分類(TimeML 中の {\tt EVENT@type}),
テンス・アスペクト体系の整備(同 {\tt MAKEINSTANCE@tense, @aspect}),節間の関係定義(同 {\tt SLINK})など解決すべき問題は山積している.
事象表現に対する分類として工藤らの動詞分類\cite{工藤1995,工藤2004}を基にしたうえ
で,{\tt EVENT@type}を付与している.
テンス・アスペクト体系については中村らのテンス・アスペクトの解釈\cite{中村2001}
を参考にしてラベルを設計する予定である.日本語のアスペクト表現は多様であるために
{\tt MAKEINSTANCE@aspect}については,元の TimeML から拡張する予定である.
今後 TimeML に準じた事象表現に対するアノテーションを行い,最終目標である事象表現に対する時間情報付与の研究へと進んでいきたい.


\acknowledgment

本研究は,国立国語研究所基幹型共同研究「コーパスアノテーションの基礎研究」および国立国語研究所「超大規模コーパス構築プロジェクト」による補助を得ています.


\bibliographystyle{jnlpbbl_1.5}
\begin{thebibliography}{}

\bibitem[\protect\BCAY{Boguraev \BBA\ {Kubota Ando}}{Boguraev \BBA\ {Kubota
  Ando}}{2005}]{Boguraev-2005}
Boguraev, B.\BBACOMMA\ \BBA\ {Kubota Ando}, R. \BBOP 2005\BBCP.
\newblock \BBOQ {TimeML-Compliant Text Analysis for Temporal Reasoning}.\BBCQ\
\newblock In {\Bem {Proceedings of the 19th International Joint Conference on
  Artificial Intelligence (IJCAI-05)}}, \mbox{\BPGS\ 997--1003}.

\bibitem[\protect\BCAY{Boguraev \BBA\ {Kubota Ando}}{Boguraev \BBA\ {Kubota
  Ando}}{2006}]{Boguraev-2006}
Boguraev, B.\BBACOMMA\ \BBA\ {Kubota Ando}, R. \BBOP 2006\BBCP.
\newblock \BBOQ {Analysis of TimeBank as a Resource for TimeML parsing}.\BBCQ\
\newblock In {\Bem {Proceedings of the 5th International Conference on Language
  Resources and Evaluation (LREC-06)}}, \mbox{\BPGS\ 71--76}.

\bibitem[\protect\BCAY{{DARPA TIDES}}{{DARPA TIDES}}{2004}]{TERN}
{DARPA TIDES} \BBOP 2004\BBCP.
\newblock {\Bem {The TERN evaluation plan; time expression recognition and
  normalization}}.
\newblock {Working papers, TERN Evaluation Workshop}.

\bibitem[\protect\BCAY{Grishman \BBA\ Sundheim}{Grishman \BBA\
  Sundheim}{1996}]{MUC6}
Grishman, R.\BBACOMMA\ \BBA\ Sundheim, B. \BBOP 1996\BBCP.
\newblock \BBOQ {Message Understanding Conference-6: a brief history}.\BBCQ\
\newblock In {\Bem {Proceedings of the 16th International Conference on
  Computational Linguistics (COLING-96)}}, \mbox{\BPGS\ 466--471}.

\bibitem[\protect\BCAY{橋本\JBA 中村}{橋本\JBA 中村}{2010}]{Hashimoto-2010}
橋本泰一\JBA 中村俊一 \BBOP 2010\BBCP.
\newblock {拡張固有表現タグ付きコーパスの構築—白書,書籍,Yahoo!
  知恵袋コアデータ—}.\
\newblock \Jem{{言語処理学会第16回年次大会発表論文集}}, \mbox{\BPGS\ 916--919}.

\bibitem[\protect\BCAY{Im, You, Jang, Nam, \BBA\ Shin}{Im
  et~al.}{2009}]{KTimeML}
Im, S., You, H., Jang, H., Nam, S., \BBA\ Shin, H. \BBOP 2009\BBCP.
\newblock \BBOQ KTimeML: specification of temporal and event expressions in
  Korean text.\BBCQ\
\newblock In {\Bem Proceeding of ALR7 Proceedings of the 7th Workshop on Asian
  Language Resources}, \mbox{\BPGS\ 115--122}.

\bibitem[\protect\BCAY{{IREX 実行委員会}}{{IREX 実行委員会}}{1999}]{IREX}
{IREX 実行委員会} \BBOP 1999\BBCP.
\newblock \Jem{{IREX ワークショップ予稿集}}.

\bibitem[\protect\BCAY{国立国語研究所コーパス開発センター}{国立国語研究所コー
パス開発センター}{2011}]{BCCWJ}
国立国語研究所コーパス開発センター \BBOP 2011\BBCP.
\newblock \Jem{『現代日本語書き言葉均衡コーパス』利用の手引} (第1.0 \JEd).

\bibitem[\protect\BCAY{工藤}{工藤}{1995}]{工藤1995}
工藤真由美 \BBOP 1995\BBCP.
\newblock \Jem{アスペクト・テンス体系とテクスト—現代日本語の時間の表現—}.
\newblock ひつじ書房.

\bibitem[\protect\BCAY{工藤}{工藤}{2004}]{工藤2004}
工藤真由美 \BBOP 2004\BBCP.
\newblock \Jem{日本語のアスペクト・テンス・ムード体系—標準語研究を超えて—}.
\newblock ひつじ書房.

\bibitem[\protect\BCAY{Mani}{Mani}{2006}]{Mani-2006}
Mani, I. \BBOP 2006\BBCP.
\newblock \BBOQ {Machine Learning of Temporal Relations}.\BBCQ\
\newblock In {\Bem {Proceedings of the 44th Annual Meeting of the Association
  for Computational Linguistics (ACL-2006)}}, \mbox{\BPGS\ 753--760}.

\bibitem[\protect\BCAY{中村}{中村}{2001}]{中村2001}
中村ちどり \BBOP 2001\BBCP.
\newblock \Jem{日本語の時間表現}.
\newblock くろしお出版.

\bibitem[\protect\BCAY{Pustejovsky, Casta\~{n}o, Ingria, Saur\'{i}, Gaizauskas,
  Setzer, \BBA\ Katz}{Pustejovsky et~al.}{2003a}]{TimeBank}
Pustejovsky, J., Casta\~{n}o, J., Ingria, R., Saur\'{i}, R., Gaizauskas, R.,
  Setzer, A., \BBA\ Katz, G. \BBOP 2003a\BBCP.
\newblock \BBOQ {The TIMEBANK Corpus}.\BBCQ\
\newblock In {\Bem {Proceedings of Corpus Linguistics 2003}}, \mbox{\BPGS\
  647--656}.

\bibitem[\protect\BCAY{Pustejovsky, Hanks, Saur\'{i}, See, Gaizauskas, Setzer,
  Radev, Sundheim, Day, Ferro, \BBA\ Lazo}{Pustejovsky et~al.}{2003b}]{TimeML}
Pustejovsky, J., Hanks, P., Saur\'{i}, R., See, A., Gaizauskas, R., Setzer, A.,
  Radev, D., Sundheim, B., Day, D., Ferro, L., \BBA\ Lazo, M. \BBOP 2003b\BBCP.
\newblock \BBOQ {TimeML: Robust Specification of Event and Temporal Expressions
  in Text}.\BBCQ\
\newblock In {\Bem {Proceedings of the 5th International Workshop on
  Computational Semantics (IWCS-5)}}, \mbox{\BPGS\ 337--353}.

\bibitem[\protect\BCAY{Sekine, Sudo, \BBA\ Nobata}{Sekine
  et~al.}{2002}]{Sekine-2002}
Sekine, S., Sudo, K., \BBA\ Nobata, C. \BBOP 2002\BBCP.
\newblock \BBOQ {Extended Named Entity Hierarchy}.\BBCQ\
\newblock In {\Bem {The Third International Conference on Language Resources
  Evaluation (LREC-02)}}, \mbox{\BPGS\ 1818--1824}.

\bibitem[\protect\BCAY{Setzer}{Setzer}{2001}]{Setzer-2001}
Setzer, A. \BBOP {2001}\BBCP.
\newblock {\Bem {Temporal Information in Newswire Articles: An Annotation
  Scheme and Corpus Study}}.
\newblock Ph.D.\ thesis, {University of Sheffield}.

\bibitem[\protect\BCAY{Verhagen, Gaizauskas, Schilder, Hepple, Katz, \BBA\
  Pustejovsky}{Verhagen et~al.}{2007}]{TempEval}
Verhagen, M., Gaizauskas, R., Schilder, F., Hepple, M., Katz, G., \BBA\
  Pustejovsky, J. \BBOP 2007\BBCP.
\newblock \BBOQ {SemEval-2007 Task 15: TempEval Temporal Relation
  Identification}.\BBCQ\
\newblock In {\Bem {Proceedings of the 4th International Workshop on Semantic
  Evaluations (SemEval-2007)}}, \mbox{\BPGS\ 75--80}.

\bibitem[\protect\BCAY{Verhagen, Saur\'{i}, Caselli, \BBA\
  Pustejovsky}{Verhagen et~al.}{2010}]{TempEval2}
Verhagen, M., Saur\'{i}, R., Caselli, T., \BBA\ Pustejovsky, J. \BBOP
  2010\BBCP.
\newblock \BBOQ {SemEval-2010 Task 13: TempEval-2}.\BBCQ\
\newblock In {\Bem {Proceedings of the 5th International Workshop on Semantic
  Evaluations (SemEval-2010)}}, \mbox{\BPGS\ 57--62}.

\end{thebibliography}


\begin{biography}
\bioauthor{小西  光}{
2005年上智大学文学部卒業.2007年上智大学文学研究科博士前期課程修了.
2008年より国立国語研究所コーパス開発センタープロジェクト奨励研究員.現在に至る.
『日本語書き言葉均衡コーパス』『日本語話し言葉コーパス』『日本語超大規模コーパス』の整備に携わる.
}

\bioauthor{浅原 正幸}{
2003年奈良先端科学技術大学院大学情報科学研究科博士後期課程修了.
2004年より同大学助教.
2012年より国立国語研究所コーパス開発センター特任准教授.
現在に至る.博士(工学).自然言語処理・コーパス言語学の研究に従事.
}

\bioauthor{前川喜久雄}{
1956年生.1984年上智大学大学院外国語学研究科博士後期課程(言語学)
中途退学.国立国語研究所教授.言語資源系長.コーパス開発センター
長.博士(学術).専門は音声学ならびに言語資源学.}

\end{biography}

\biodate




\end{document}
