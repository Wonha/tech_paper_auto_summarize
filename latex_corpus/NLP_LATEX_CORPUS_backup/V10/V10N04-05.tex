



\documentstyle[epsf,jnlpbbl]{jnlp_j_b5}

\setcounter{page}{83}
\setcounter{巻数}{10}
\setcounter{号数}{4}
\setcounter{年}{2003}
\setcounter{月}{7}
\受付{2002}{10}{8}
\再受付{2002}{11}{20}
\再々受付{2003}{1}{27}
\採録{2003}{4}{10}

\setcounter{secnumdepth}{2}
\title{日本語--ウイグル語辞書の半自動作成と評価}
\author{ムフタル・マフスット\affiref{GSID} \and 
小川 泰弘\affiref{NUIE} \and 
杉野 花津江\affiref{NUIE} \and 
稲垣 康善\affiref{APU}}
\headauthor{マフスット,小川,杉野,稲垣}
\headtitle{日本語--ウイグル語辞書の半自動作成と評価}
\affilabel{GSID}{名古屋大学大学院国際開発研究科}
{Graduate School of International Development, Nagoya University}
\affilabel{NUIE}{名古屋大学大学院工学研究科}
{Graduate School of Engineering, Nagoya University}
\affilabel{APU}{愛知県立大学情報科学部}
{Faculty of Information Science and Technology, Aichi Prefectural University}
\jabstract{
著者らは,既存のウイグル語--日本語辞書を基にして,見出し語数約2万の
日本語--ウイグル語辞書を半自動的に作成した.
この辞書が日常よく使われる語彙をどの程度含んでいるかなどの特性を
調べるために,国立国語研究所の教育基本語彙6,104語のうちのより基本的とされている
2,071語,およびEDR日本語テキストコーパスの出現頻度上位2,056語に対し,
日本語--ウイグル語辞書の収録率を調査し,いずれについても約80\,\%の収録率
であることが分かった.未収録語について,逐一その理由を調べ,判明した種々の理由を
整理すると共に,それに基づいて未収録語を分類した.
その結果,辞書作成をする時に収録率を上げるために
注意すべき点などについていくつかの知見を得ることができた.
本論文では,それらについて述べる.}
\jkeywords{日本語,ウイグル語,辞書,自動作成,基本語彙,収録率}
\etitle{Semiautomatic Generation of Japanese-Uighur \\ Dictionary   
and Its Evaluation}
\eauthor{Muhtar Mahsut\affiref{GSID} \and
Yasuhiro Ogawa\affiref{NUIE} \and
Kazue Sugino\affiref{NUIE} \and
Yasuyoshi Inagaki\affiref{NUIE}}

\eabstract{
The authors have constructed semiautomatically a Japanese-Uighur dictionary 
consisting of about 20,000 items. This paper describes the process of generating 
our Japanese-Uighur dictionary from an available Uighur-Japanese one.
We have investigated the vocabulary of our dictionary and found it includes 
about 80\,\% of 2,000 high-priority words of Japanese. We have also investigated 
the reasons why each word of the remaining 20\,\% of 2,000 high-priority words is 
not included in our dictionary, and have classified the words not included to 
five groups according to the reasons we found through our investigation.}
\ekeywords{Japanese, Uighur, Dictionary, Automatic generation, 
	Basic vocabulary, Cover rate}
\begin{document}
\maketitle
\thispagestyle{empty}

\section{はじめに}
著者らは,実用に近い日本語--ウイグル語機械翻訳システムの実現を目指して
一連の研究をしてきた
\cite{NLC93,MSTHESIS,MUHPARAM,PRICAI94,MUH_OGA2001,OGAWA2000,MT_SUMMIT2001,MUH_NLT_2002}.
その過程で,一定の語彙数を持つ日本語--ウイグル語電子辞書の開発が
不可欠であると考え,
その開発に着手した.その時点では,日本語--ウイグル語に関する通常の辞書
さえない状況であった.最初は,
日本語--ウイグル語機械翻訳実験用の
基本的な辞書の
開発を考えて,IPAの計算機用
日本語基本動詞辞書IPAL\cite{IPAL}をベースに,名詞や形容詞などを
含め,約1,200語の日本語--ウイグル語電子辞書を作成した\cite{NLC93,MSTHESIS}.
IPAL動詞辞書には,
日本語の動詞のうちで語彙体系上ならびに使用頻度上重要であると考えられる
基本的な和語動詞861語が含まれている.両言語のなかで特に格助詞を含む
名詞接尾辞と動詞接尾辞が動詞と密接な関係にあり,日本語--ウイグル語機械
翻訳においても,動詞が重要であるため,IPAL動詞辞書を選んだ.
しかし,1,200語前後の辞書では不十分であり,実用に近い機械翻訳システムの実現
には,少なくとも日常使われる最低限の語彙を含む日本語--ウイグル語電子辞書の
開発が必要であるとの考えに至った.

そこで,我々はまずウイグル語--日本語辞書であるウイグル語辞典\cite{UJDIC}を
電子化して機械可読にし,その逆辞書を自動的に生成するという方針で本格的な
日本語--ウイグル語電子辞書の開発に着手した\cite{UJDICE,JUDICGEN}.
辞書開発は,著者らが行なったが,その内の一人は十分な日本語能力を有する
ウイグル語ネイティブ話者である.

日本語--ウイグル語電子辞書の開発作業は次のような段階に分けて行なった.\\

\begin{enumerate}
 \item ウイグル語--日本語電子辞書の作成
\begin{itemize}
 \item[1-1.] ウイグル語辞典\cite{UJDIC}のデータの電子化と項目タグの付与
 \item[1-2.] 各項目の修正および品詞の付与
\end{itemize}

 \item 日本語--ウイグル語電子辞書の作成
\begin{itemize}
 \item[2-1.] ウイグル語--日本語辞書から日本語--ウイグル語辞書を自動生成
 \item[2-2.] 各見出し語の検査および修正
 \item[2-3.] 機械翻訳システムで利用できる形式への変換
\end{itemize}       
\end{enumerate} 

各作業の詳細については,2章以降で順次説明する.
こうした一連の作業を行なった結果,
語彙数約20,000語の日本語--ウイグル語電子辞書を作成することができた.
著者らは,この辞書が日常よく使われる語彙をどの程度見出し語として採録しているかを
調べるために,\\

\begin{itemize}
\item[a.] 国立国語研究所の教育用基本語彙\cite{KOKKEN}6,104語中のより
基本的とされている2,071語に対する収録率,
\item[b.] EDRコーパス\cite{EDRCORPUS}の日本語テキスト文に含まれる単語の
上位頻度2,056語に対する収録率
\end{itemize}

\mbox{} \\
の2点に関して調査した.a.は,日本語基本語彙に対する調査で,b.は,新聞記事などからの
テキストを対象とした調査であり,それぞれの特徴はあるが,全体として見ると,
a.,b.ともに約80\,\%の収録率であった.さらに,a.とb.それぞれについて
収録されていない単語一つ一つに関して,収録されなかった,
すなわち見出し語として
採録されなかった理由について詳細な分析を行ない,その理由を大きくA〜Eの5つに分類し,
それぞれをさらに細分類して検討した.
この結果は,本論文と同様の手法で辞書作成をする際,
収録率を上げるために注意すべき点について,
いくつかの知見を与えている.

本論文は,次のような構成になっている.\ref{section:denshika}章では,
ウイグル語辞典\cite{UJDIC}を機械可読にし,それに対して一連の編集作業を行なって
ウイグル語--日本語電子辞書を作成した過程について述べる.
\ref{section:jidoseisei}章では,ウイグル語--日本語電子辞書からその
逆辞書である日本語--ウイグル語辞書の自動生成について述べる.
\ref{section:for_majo}章では,自動生成で得られた日本語--ウイグル語辞書の機械
翻訳用辞書への変換について述べる.
\ref{section:hyoka}章では,以上のようにして著者らが作成した日本語--ウイグル語
辞書の収録率,および,収録されていない単語の調査とその結果について述べ,著者らが作成した
日本語--ウイグル語辞書の評価とする.
\ref{section:owari}章は本論文のまとめである.

\section{ウイグル語--日本語辞書の電子化}\label{section:denshika}
我々の目標は,日本語--ウイグル語機械翻訳システムに使用できる辞書の作成で
ある.元となる日本語--ウイグル語辞書が存在しなかったため,
まず維漢辞典\cite{UHDIC}の和訳であるウイグル語辞典\cite{UJDIC}を
計算機で処理可能な形にするための
電算入力から作業を始めた.この作業では,計算機での処理を容易にするために,
各項目にタグを付加し,図\ref{fig:first_dic}のような形式でデータを入力した.
\begin{figure}
\begin{center}
\rule{0.9\textwidth}{0.2mm} \\ \mbox{} \\
\begin{minipage}{0.9\textwidth}
{
\small
\begin{verbatim}
\u n!etij!e \j 結果,結論,成果,効果,成績 \ 
\ue musabi!kining n!etijisi \je 試合の結果.\
\ue !uginix n!etijisi \je 学業成績.\
\ue ~ !kazanma!k \je 成績を得る.\
\ue n!etijig!e erixm!ek \je 成果を得る.\
\ue ~ b!erm!ek \je 効果があがる.\
\ue ~ qi!kma!k \je 結果が出る.\
\ue ~ qi!karma!k \je 結論を出す.\
\end{verbatim}
}
\end{minipage} \\ 
\mbox{} \\ 
\rule{0.9\textwidth}{0.2mm}
\caption{電子化されたウイグル語辞典の一部}
\label{fig:first_dic}
\end{center}
\end{figure}

ここで,\verb+\u+, \verb+\j+, \verb+\ue+および\verb+\je+は,それぞれ
ウイグル語の語彙見出し,日本語の対訳語,ウイグル語の
用例・例文およびその和訳文を
示すタグである.日本語の対訳語が複数ある場合,意味的に近いものと,
そうではないものが,それぞれ`,'と`;'で区切られ,ウイグル語例文と
和訳文の末尾に`.'が付いている.`\verb+~+'は,見出し語を表わし,`\verb+\+'は,
行末を表わす.
また,ウイグル語には32の文字があり,`c' 以外のローマ字25文字と
ウイグル語の発音に近いローマ字の前に\verb+`!'+を付した
\verb+`!g'+, \verb+`!h'+, \verb+`!k'+, \verb+`!e'+, \verb+`!o'+,
 \verb+`!u'+, \verb+`!z'+の7文字の合計32の表記を対応させた.
ウイグル文字と本論文での表記の対応は
表\ref{table:arabic_latin}の通りである.
\begin{table}[tbp]
\begin{center}
\caption{ウイグル文字とローマ字表記(本論文での表記)との対応} 
\label{table:arabic_latin}
\epsfile{file=uirmtab.eps,width=0.9\textwidth}
\end{center}
\end{table}

次に,このデータに対し,ウイグル語ネイティブ話者および日本語ネイティブ話者からなる著者らの
合議検討の共同作業により,すべての見出し語とその項目について逐一,
その内容を吟味,確認し,さらに
以下の3点に注目して修正を施した.\\

\begin{enumerate}
 \item[1)] 品詞の付与,および複数の品詞をもつ見出し語を品詞ごとに分割.
 \item[2)] 不適切な訳語の修正,および見出し語綴りの誤りの訂正.
 \item[3)] 語源・語義が異なる見出し語を異なる語義ごとに分割.
\end{enumerate}

\mbox{} \\
まず,1)の品詞の付与であるが,図\ref{fig:first_dic}に見るように,
元のウイグル語辞典\cite{UJDIC}には品詞が付与されていなかったので,
すべての見出し語に人手で品詞を付与した.

また,複数の品詞を持っている見出し語については,
品詞ごとに分けて別々の見出しとした.
日本語と違って,ウイグル語では二つ以上の品詞を持つ単語が
多く,
特に,形容詞にも名詞にもなる語が多い.
ウイグル語の`\verb+k!ok+'はその一例であるが,
図\ref{fig:arranged_dic}のように別々の見出し語に分割し,
日本語訳および例文も,それぞれの品詞ごとに分けて付加した.

\begin{figure}[tbp] 
\begin{center}
\rule{0.9\textwidth}{0.2mm} \\ \mbox{} \\
\begin{minipage}{0.9\textwidth}
\small
\begin{verbatim}
\u k!ok \j 青い\ [形容詞]
\ue ~ boya!k \je 青い顔料.\
\ue ~ k!oz \je 青い目.\
\ue ~ purqa!k \je 青豆.\

\u k!ok \j 空, 天空; 青いあざ; 青草; まだ実らない作物; 青菜, 野菜, ウマゴヤシの芽; 
           ふさ状のもの\ [名詞]
\ue B!urk!ut ~ t!e p!erwaz !kilma!kta. \je タカが空を舞っている.\
\ue b!edinig!e ~ q!uxm!ek \je 体に青いあざができる.\
\ue !Koylar ~ k!e toydi. \je 羊たちは青草を食べ飽きた.\
\end{verbatim}
\end{minipage} \\ \mbox{} \\ 
\rule{0.9\textwidth}{0.2mm} \vspace{-1.6ex}
\end{center}
\caption{人手による修正を加えたウイグル語--日本語辞書の一部} 
\label{fig:arranged_dic}
\end{figure} 


また,今回使用したウイグル語辞典\cite{UJDIC}は,
維漢辞典\cite{UHDIC}の中国語訳を日本語に翻訳して
作成されており,二次翻訳による意味のずれや欠落がかなりあった.
2)は,主にそうした点に関しての修正である.
例えば,ウイグル語の
`gilitserin'\footnote{`gilitserin'はウイグル語でも「外来語」である.}は,
維漢辞典\cite{UHDIC}での中国語への訳語が``甘油''になっており,
ウイグル語辞典\cite{UJDIC}でも日本語訳が
「甘油」になっていた.しかし,「甘油」は,日本語国語辞書の
見出し語として含まれていなかった.そこで,中国語--英語辞典\cite{EC_CE_DIC}を引くと,
``甘油''の英訳が`glycerine'であることが分かり,
ウイグル語の`gilitserin'に対して「グリセリン」という日本語訳語を当てた.
また,ウイグル語の日本語訳として与えられている単語が日本語国語辞典の見出し語に
含まれてはいるものの,中国語を介して訳付けをしたため,意味がずれていた
ものもあった.例えば,ウイグル語の`katta'の維漢辞典\cite{UHDIC}での中国語への
訳語は``大事''であるが,その意味は,「重要」という意味の「大事」ではなく,
「偉大」という意味の「大事」である.
しかし,我々が出発点として用いたウイグル語辞典\cite{UJDIC}では,中国語の「大事」を
日本語に訳すとき「大事」をそのまま日本語として解釈して`katta'の意味の一つに
「重要」が付与されていた.これは`katta'の意味としては正しくないのでそれを削除した.このような意味の取り違えは中国語の「汽車」が,日本語では「自動車」を
意味しているように,同じ文字列で表わされた単語が中国語と日本語では異なる
意味を持つことから生じている.
さらに,
こうした作業と並行して,
時代遅れになった単語や訳語を削除したり,
逆に,適切な訳語を適宜追加したりして,訳語を修正した.
また,見出し語の綴り字の誤りは気付く限り訂正した.例えば,
維漢辞典\cite{UHDIC}ならびに
ウイグル語辞典\cite{UJDIC}で`ara-tora'とあるのは誤りであり,
`ara-tura'とするのが正しいので,そのように
訂正した\footnote{飯沼「ウイグル語辞典」には,
アラビックウイグル文字綴りが独自に付加されているが,それと
ラテンウイグル文字綴りの間に不一致が見られる.
このことを指摘してくださった査読者に感謝する.
この点について,若干の説明を以下に記す.

この不一致の原因には,次の4つの場合が考えられる.

1. 『維漢辞典』のラテン文字(ラテンウイグル文字)見出し語表記と
飯沼『ウイグル語辞典』の
それとは同一であるが,飯沼版でアラビア文字(アラビックウイグル文字)の綴りを独自に
付加した際に,『維漢辞典』とは異なった規準で綴り字を決めていることが原因で生じた
不一致.

2. ウイグル語表記にラテン文字を導入した時に,中国語の発音を正確に表すために,
中国語のPINYINで使われているzh, ch, shをウイグル語の組み合わせ文字として
取り入れたが,これに対応するアラビックウイグル文字はない.そのために,それらを,
音が近いアラビックウイグル文字で表記したために生じた不一致.例えば,PINYIN式では`dad!uizhang'と表記されるものを
アラビックウイグル文字で表記し,それをラテン文字に写すと`dad!uyjang'になる.
これは,上記1の特別な場合である.

3. 『維漢辞典』の編集過程でラテン文字見出し語の綴りに間違いがあったものを
正しく修正できなかったが,アラビック文字の見出し語は正しく追加しているために
生じている不一致.
例えば,飯沼版ならびに『維漢辞典』で`ara-tora'とあるのは誤りで,`ara-tura'と
するのが正しい.

4. 飯沼版で,独自にアラビア文字綴りを追加した時にアラビア文字綴りに誤りを
犯したために生じた不一致.

ウイグル語綴り字の正書法は,
現代ウイグル語正書法辞典\cite{UIIMLA}が
1997年に出版されるまでは決まっていなかったので,上記の1, 2のようなことが
生じていた.しかし,ネイティブのウイグル語話者にとっては,そのような
表記の不一致が生じても,いずれの表記でも同一の単語として同定できるものである.
3と4の不一致は,間違いが原因で生じているので,その誤りは訂正しなければ
ならない.3の誤りは気付く限り修正した.4の修正については,今回の辞書にはアラビック
表記を入れなかったので,対処していない.
}.
2)の不適切な訳語の修正作業には,日本語の意味の確認のために,広辞苑\cite{KOJIEN},
日本語大辞典\cite{KODANSHA},大辞林\cite{SANSEIDO},
新明解国語辞典\cite{BOOKSHELF}を,ウイグル語の意味の確認のために,
ウイグル語辞典\cite{UJDIC},維漢辞典\cite{UHDIC},
中国語--英語辞典\cite{EC_CE_DIC},ウイグル語詳細辞典\cite{UILUGET}
などを参照した.

3)は複数の異なる語義を持つ単語が一つの見出し語になっている場合に,
これを別々の見出し語とした.
複数の語義を持つ単語を辞書によって
同じ一つの見出し語として扱ったり,別の見出し語として扱ったりしていることがある.
ここでも,元になった維漢辞典\cite{UHDIC}では一つの見出し語として扱われているものが
ウイグル語辞典\cite{UJDIC}では別々の見出し語になっていたり,またはその
逆であったりして,
区別する規準がはっきり
していなかったが,著者らのこの作業では,意味が類似して
いなければ,別の見出し語とした.
その例を図\ref{fig:separated_word}に示す.
\begin{figure}[t]
\begin{center}
\rule{0.9\textwidth}{0.2mm} \\ \mbox{} \\
\begin{minipage}{0.9\textwidth}
\small
\begin{verbatim}
\u k!eqm!ek \j 踏む, 渡る; 通る, 経る, 経験する\ [自動詞][他動詞]
\ue lay ~ \je 泥を踏む.\
\ue !kar ~ \je 雪を踏む.\
\ue Hiyalim!ga bir ix k!eqti. \je あることが脳裏をかすめた.\

\u k!eqm!ek \j 捨てる, あきらめる; 許す\ [他動詞]
\ue ham hiyaldin ~ \je 幻想を捨てる.\
\ue x!ehsi n!epsidin ~ \je 私欲を捨てる.\
\ue jandin ~ \je 命を捨てる.\
\end{verbatim}
\end{minipage} \\ \mbox{} \\ \rule{0.9\textwidth}{0.2mm} \vspace{-1.6ex}
\end{center}
\caption{同じ綴の語を別々の見出し語に分けた例}
\label{fig:separated_word} 
\end{figure}

\begin{table}
\begin{center}
\caption{ウイグル語--日本語電子辞書の見出し語数および平均対訳語数}
\label{mytab:ujwtab}
\footnotesize
\begin{tabular}{|l||r|r|r|r|r|r|r|r|r|r|r|r|r}
\hline
品詞 & 名詞 & 動詞 & \parbox[c]{2zw}{形容詞} & \parbox[c]{2zw}{動作名詞} & 副詞 & 
\parbox[c]{2zw}{助数詞} & \parbox[c]{2zw}{感嘆詞} & \parbox[c]{2zw}{代名詞} & 
\parbox[c]{2zw}{接続詞} & \parbox[c]{2zw}{接続助詞} & \parbox[c]{2zw}{語気助詞} & 合計\\
\hline
見出し語数 & 7,259 & 3,682 & 2,868 & 1,046 & 691 & 142 & 48 & 19 & 
21 & 10 & 2 & 15,788\\
\hline
\parbox[c]{5zw}{日本語への\\平均対訳\\語数} & 1.86 & 2.35 & 2.12 & 2.12 & 
2.09 & 1.61 & 2.14 & 1.79 & 2.48 & 2.80 & 2.50 & 2.05\\
\hline
\end{tabular}
\end{center}
\end{table}
\mbox{} \\
こうした作業の結果,最終的に約16,000語の
語彙数を持つウイグル語--日本語電子辞書を作成することができた.
この辞書の品詞ごとの見出し語数と平均対訳語数を表\ref{mytab:ujwtab}に示す.
この表の中で,形容詞としているのは体言を修飾するか,
状態・状況・様子を
表わす述語になりうる単語であり,また,動作名詞は,``-する''が付いて
動詞化する日本語のサ変名詞のように,`!kilma!k'が後に続いて動詞化する
名詞である.また,名詞,動詞,形容詞,動作名詞の合計は14,855語で,
見出し語総数15,788語の94.09\,\%になる.
ここで,動作や状況を表わす動詞,形容詞,動作名詞は,
平均2.12〜2.35語の日本語対訳語をもつのに対して,名詞に対しては
平均して1.86語であり,名詞の方が1:1対応の傾向が強いことが伺える.
これらの様子は,図\ref{myfig:ujnoun},\ref{myfig:ujverb},\ref{myfig:ujadj},
\ref{myfig:ujsverb}によく現れている.

\begin{figure}
\vspace*{-2mm}
\begin{minipage}{0.47\textwidth}
\fbox{\epsfile{file=graphujnoun.eps,width=0.96\textwidth}}
\caption{ウイグル語名詞から日本語名詞への\\対訳語数の分布}
\label{myfig:ujnoun}
\end{minipage}\hfill
\begin{minipage}{0.47\textwidth}
\fbox{\epsfile{file=graphujverb.eps,width=0.96\textwidth}}
\caption{ウイグル語動詞から日本語動詞への\\対訳語数の分布}
\label{myfig:ujverb}
\end{minipage}

\vspace{2mm}
\begin{minipage}{0.47\textwidth}
\fbox{\epsfile{file=graphujadj.eps,width=0.96\textwidth}}
\caption{ウイグル語形容詞から日本語形容詞への\\対訳語数の分布}
\label{myfig:ujadj}
\end{minipage}\hfill
\begin{minipage}{0.47\textwidth}
\fbox{\epsfile{file=graphujsverb.eps,width=0.96\textwidth}}
\caption{ウイグル語動作名詞から日本語動詞への\\対訳語数の分布}
\label{myfig:ujsverb}
\end{minipage}
\end{figure}

\section{日本語--ウイグル語電子辞書の自動生成}\label{section:jidoseisei}
前章で述べた一連の手続きによって作成したウイグル語--日本語電子辞書に対して,
機械処理を施すことによって逆辞書である日本語--ウイグル語辞書を生成した. 
{\begin{figure}[p]
\begin{center}
\rule{0.9\textwidth}{0.2mm} \\ 
\mbox{} \\
\begin{minipage}{0.9\textwidth}
\small
 \verb+\u qotka \j はけ,ブラシ\ [名詞]+\\
 \verb+\u qotkilima!k \j はけでみがく,ブラシをかける\ [他動詞]+\\
 \verb+\u qox!ka \j 豚\ [名詞]+ \\ \\
\hspace*{2cm}(a) 第\ref{section:denshika}章の手続きで生成したウイグル語--日本語辞書の一部 \\
\hspace*{4cm}$\Downarrow$ \\
\end{minipage} \\
\begin{minipage}{0.9\textwidth}
\small
 \verb+\u qotka \j はけ\ [名詞]+\\
 \verb+\u qotka \j ブラシ\ [名詞]+\\
 \verb+\u qotkilima!k \j はけでみがく\ [他動詞]+\\
 \verb+\u qotkilima!k \j ブラシをかける\ [他動詞]+\\
 \verb+\u qox!ka \j 豚\ [名詞]+ \\ \\
\hspace*{2cm}(b) ウイグル語--日本語の対の辞書 \\
\hspace*{4cm}$\Downarrow$\\
\end{minipage} \\
\begin{minipage}{0.9\textwidth}
 \verb+<はけ, はけ, 名詞, qotka>+\\
 \verb+<ブラシ, ぶらし, 名詞, qotka>+\\
 \verb+<はけでみがく, はけでみがく, 他動詞, qotkilima!k>+\\
 \verb+<ブラシをかける, ブラシをかける, 他動詞, qotkilima!k>+\\
 \verb+<豚, ぶた, 名詞, qox!ka>+ \\ \\
\hspace*{2cm}(c) \verb+<日本語見出し語, 読み, 品詞, ウイグル語訳>+の4項組辞書\\
\hspace*{4cm}$\Downarrow$\\
\end{minipage} \\
\begin{minipage}{0.9\textwidth}
\begin{tabular}{llll}
 (日本語見出し語) & (読み) & (品詞) & (ウイグル語訳語) \\
 はけ & はけ & 名詞 & \verb+qotka+ \\
 はけでみがく & はけでみがく & 他動詞 & \verb+qotkilima!k+\\
 豚 & ぶた & 名詞 & \verb+qox!ka+\\
 ブラシ & ぶらし & 名詞 & \verb+qotka+ \\
 ブラシをかける & ぶらしをかける & 他動詞 & \verb+qotkilima!k+ \\ \\
\end{tabular} \\
\hspace*{2cm}(d) 日本語--ウイグル語辞書の一部 \\
\end{minipage} \\
\rule{0.9\textwidth}{0.2mm} 
\caption{ウイグル語--日本語辞書から日本語--ウイグル語辞書への変換} 
\label{fig:convert}
\end{center}
\end{figure}
ウイグル語--日本語電子辞書において,一つのウイグル語見出し語に
複数の日本語訳が付されている場合もあるが,
まず,これらをウイグル語と日本語訳語の対にした(図\ref{fig:convert}--(b)).
これに,ウイグル語--日本語電子辞書に付加した品詞,および,
漢字かな変換プログラム KAKASI\footnote{http://kakasi.namazu.org/}を
利用して得た読み仮名を加え,
4項組$<$日本語見出し語,読み,品詞,ウイグル語訳語$>$を自動的に
作成した(図\ref{fig:convert}--(c)).
さらに,これを読みでソートして,
結果的に図\ref{fig:convert}--(d)のような形で
日本語--ウイグル語辞書の基本を作成した.

さらに,この4項組に対し,次の3点に注意しながら,人手による修正を加えた.\\ 

\begin{enumerate}
 \item[1)] 見出し語としての適切さ
 \item[2)] ウイグル語訳の妥当性
 \item[3)] 読みの正しさ
\end{enumerate}
~

1)は,訳語として適切に意味が表わされていても,それが必ずしも
見出し語として適切ではないことによって生じる問題である.
例えば,元になったウイグル語--日本語辞書においては
\begin{center}
\verb+\u rixal!e \j 卵白と砂糖で作ったお菓子.\ [名詞]+
\end{center}
という項目があったが,日本語訳の「卵白と砂糖で作ったお菓子」
は`rixal!e'の説明であり,「卵白と砂糖で作ったお菓子」の総称が
`rixal!e'であるのではない.
よって,日本語--ウイグル語辞書において
「卵白と砂糖で作ったお菓子」という見出し語を設け,
`rixal!e'をその訳語とするのは適切でない.
また,例えば,\verb+<メロンの一種, めろんのいっしゅ, [名詞], !kari!kax>+の
ように「○○の一種」といった説明的な日本語訳も日本語--ウイグル語辞書の
見出し語としては相応しくない.
さらに,\verb+<清朝時代の新疆の県知事, しんちょうじだいのしんきょうのけんちじ, [名詞], ambal>+
や宗教儀式の言葉の説明などもやはり日本語見出し語として不適切である.
こうした不適切な日本語見出し語をもつ4項組
を人手で除去した.

次に2)のウイグル語訳の妥当性については,例えば,「音」という日本語見出しを持つ4項組には,
\begin{verbatim}
<音, おと, [名詞], a!hang>   ---- (1) ×
<音, おと, [名詞], awaz>     ---- (2) 
<音, おと, [名詞], sada>     ---- (3) ×
<音, おと, [名詞], tawux>    ---- (4) ×
\end{verbatim}

\begin{table}
\begin{center}
\caption{日本語--ウイグル語電子辞書の見出し語数および平均対訳語数}
\label{mytab:juwtab}
\footnotesize
\begin{tabular}{|l||r|r|r|r|r|r|r|r|r|r}
\hline
品詞 & 名詞 & 動詞 & 形容詞 & サ変名詞 & 副詞 & 助数詞 & 感嘆詞 & 接続詞 & 合計\\
\hline
見出し語数 & 8,457 & 5,411 & 3,480 & 1,785 & 784 & 156 & 73 & 20 & 20,166\\
\hline
\parbox[c]{8zw}{ウイグル語への\\平均対訳語数} & 1.51 & 1.60 & 1.68 & 1.31 & 
1.76 & 1.47 & 1.71 & 1.55 & 1.56 \\
\hline
\end{tabular}
\end{center}
\end{table}


\begin{verbatim}
<音, おと, [名詞], tiwix>    ---- (5) 	
\end{verbatim}

\noindent
の5つが現れるが,これらは,それぞれウイグル語--日本語辞書の次の項目
\begin{verbatim}
\u a!hang \j 音,音調; 調和,協調; 口ぶり,語気\ [名詞]
\u awaz \j 声,音,音量; 叫び声; 票,投票\ [名詞]
\u sada \j 音声, 音, 声\ [名詞]
\u tawux \j 声,音\ [名詞]
\u tiwix \j 音声, 音\ [名詞]
\end{verbatim}
\noindent
から出てきている.しかし,a!hangは,ここちよい音(ye!kimli!k 〜),
きつい口調(k!eskin 〜)のように音の調子を言う語であり,sadaは,歌声(nahxa  〜si),
拍手の音(al!kix 〜si),民衆の声(ammining 〜si)のように,声ないしは特定の音を
意味する語である.また,tawuxは,母音(sozu!k 〜),子音(!uz!uk  〜),声帯(〜 p!erdisi)の
ように人の声に関連した意味を持つ. これらはいずれも「音」という日本語に対する
訳としては狭すぎて不適切であるので,上の(1),(3),(4)の4項組を削除した.
同様にして,日本語見出し語に対して付けられたウイグル語訳の意味が不適切な
場合には,日本語見出し語を修正するか,対応する4項組を人手で削除した.
なお、上記 1), 2)で削除した4項組は約1,000個である。

3)は,漢字表記の読みの曖昧さから生じる問題である.
元のウイグル語--日本語辞書においては日本語訳に読みが
付加されていなかったので,今回は漢字かな変換ソフト
KAKASI
を利用して
自動的に読みを付加した.
しかし,日本語には表記は同じでも読みが複数ある語があり,
さらにその中には,読みが異なれば意味も異なる単語が多数存在する.
そうした場合,対応するウイグル語訳も異なることになり,
例えば,「額」の場合,「ひたい」と読めばそのウイグル語訳は`pixan!e'であるが,
「がく」と読めば,額縁の意味の`jaza',あるいは金額の意味の`somma'を
そのウイグル語訳として与えねばならない.
そこで,ウイグル語訳の意味に合わせて日本語の見出し語の読みを人手で
修正した.


以上の作業の結果,語彙数約2万の日本語--ウイグル語電子辞書を作ることができた.
\mbox{表\ref{mytab:juwtab}}に,日本語--ウイグル語電子辞書の品詞ごとの見出し語数
および
平均対訳語数を示した.この表の形容詞には,いわゆる日本語文法で言う形容詞,
形容動詞,連体詞が含まれている.表\ref{mytab:ujwtab}のウイグル語--日本語辞書の
場合と同様に,名詞,動詞,形容詞,サ変名詞で19,133語で,総見出し語数20,166語の
94.88\,\%を占めている.また,日本語1単語あたりのウイグル語訳語数は,各品詞
ごとの
平均で1.31〜1.76語であり,全体の平均で1.56である.これは,本辞書の作成手続きからも
推察されるように,ウイグル語--日本語辞書のウイグル語見出し語しか,
日本語--ウイグル語辞書の訳語に出現しないことからも当然の結果とうなづかれ
よう.
また,表\ref{mytab:ujwtab}の段階では32,330組のウイグル語--日本語訳の
対があったが,
表\ref{mytab:juwtab}の段階では31,392組の日本語--ウイグル語訳の対に
減少している.
この差の938組が,上述の操作の1),2)に関連する作業で
不適切であるとして,削除されたものである.


\section{日本語--ウイグル語辞書の翻訳システム用辞書への変換}\label{section:for_majo}
前章で作成した日本語--ウイグル語電子辞書は,データが電子化されてはいるが,
そのままでは機械翻訳に用いることはできない.
その意味では,前章で作成した辞書は,電子化された人間向けの
日本語--ウイグル語辞書である.
そこで,我々の機械翻訳システムで使用できるように,以下の手順で
変換した.\\

\begin{enumerate}
 \item 動詞の語尾の削除
 \item 形容詞の細分化
 \item 動名詞の追加
 \item 漢字表記と読みの分離
\end{enumerate} \mbox{} \\

我々の派生文法に基づく日本語--ウイグル語機械翻訳\cite{MT_SUMMIT2001}は,名詞接尾辞と
動詞接尾辞の適切な変換\cite{MUH_OGA2001,MAJO}と
形態素解析結果を逐語訳することを基本としており,
この変換作業のキー・テクノロジーは,形態素解析システム用の適切な辞書を作ることにある.
この作業は,前章までの手続きで作成した4項組に対して行なうことになるが,
説明を簡単にするために,最初は読みの部分を無視した3項組で説明し,
読みの部分の処理については最後に述べる.

まず,使用する日本語形態素解析システムMAJO\cite{MAJO}の仕様に合わせ,
動詞の語尾を削除した.
その際,日本語見出し語は,MAJOで形態素解析して末尾の形態素を除去し,
ウイグル語訳語は語尾(-ma!k もしくは -m!ek)を機械的に取り除いた.
また,品詞についても,前章では自動詞,他動詞を区別したが,
MAJOではこの区分は不要であり,その代りに語幹の末尾音素が母音か子音かで
母音幹動詞か子音幹動詞かの区別を行なった.
例えば,$<$調べる, 他動詞, selixturma!k$>$を,
$<$調べ, 母音幹動詞, selixtur$>$\footnote{音韻処理に基づくMAJOでは,平仮名
部分をローマ字表記に変換するため,実際には$<$調be,母音幹動詞,selixtur$>$
と登録した.}
と変換した.
なお,「ブラシをかける」のように日本語では1形態素ではない語も
最後の語尾を取り除き,$<$ブラシをかけ, 母音幹動詞, qotkila$>$のように
登録した.

次に,形容詞と品詞付けされた単語を細分類し,それぞれに適切な品詞を与えた.
今回の日本語--ウイグル語辞書における品詞は,原則として
元となるウイグル語--日本語辞書において付加した品詞をそのまま使用した
\footnote{動詞については上で説明したように,他動詞・自動詞の区分は無視し
母音幹動詞・子音幹動詞という品詞を与えたが,動詞であることには
変わりはない.}が,
ウイグル語で形容詞と品詞付けした単語の日本語訳は,
必ずしも日本語の形容詞とはならない.
実際にウイグル語で形容詞とされた語の訳語として挙げられる日本語単語には,例えば,
「白い」「静かな」「いわゆる」「空いた」「秋の」「意味のある」「意味のない」
などの様々なパターンが存在する.

日本語の見出し語としては,これらの単語の品詞を区別する必要があるため,
動詞の場合と同様に
日本語見出し語をMAJOで形態素解析し,
その結果から以下のような品詞を付加した.

\begin{description}
 
 \item[形状動詞] 末尾の形態素が「い」で,その直前の形態素の品詞がMAJOによって形状動詞
と判定された語
 \item[形状名詞] 末尾の形態素が「な」で,その直前の形態素の品詞がMAJOによって形状名詞
と判定された語
 \item[連体詞] 1語の形態素から成り,その品詞が連体詞である語
 \item[修飾詞] 以上のいずれにも分類されない語
\end{description}

ここで,形状動詞と形状名詞という呼称であるが,これは
本研究の基礎となる派生文法における用語であり,
それぞれ,日本語文法のいわゆる形容詞,形容動詞を指す.
また,修飾詞という品詞は,機械翻訳のために追加した品詞である.

この分類によって,上に例として挙げた単語については,
「白い」「意味のない」は形状動詞,
「静か」は形状名詞,「いわゆる」は連体詞,
「空いた」「秋の」「意味のある」は修飾詞として
辞書に登録した\footnote{形状動詞と形状名詞に関しては,動詞の場合と同
様に末尾の語尾を取り除いて辞書に登録した}.

次に,ここまでの段階の辞書では,見出し語として一部のサ変名詞について次のような問題があるので
登録語を変更してそれを解決した.
例えば,「合意する」は登録されているが,「合意」は登録されていなかったので,
「合意する」は翻訳できるが「合意」は翻訳できないことになるという問題がある.
これは,ウイグル語の「合意」に相当する`kelixix'が
動詞`kelixm!ek'の語幹`kelix-'に名詞化接尾辞`-ix'が
接続することによって合成された派生語であり,
そのためベースとしたウイグル語--日本語辞書に見出し語として
掲載されていなかったからである.

そこで,こうした動詞に対しては,動詞語幹に`-ix'に後接させ,
自動的に名詞形を作成し,これに動名詞という品詞を与えて
辞書に登録した.
すなわち,上記の例では,$<$合意, 動名詞, kelixix$>$という
項目を辞書に加えた.
なお,その際に,$<$合意す, 母音幹動詞, kelix$>$の項目の方は辞書から削除したので,
辞書に登録された単語数自体は,この作業では変化しない.
しかし,「合意する」を翻訳する場合には,名詞`kelixix'を
動詞語幹`kelix-'に戻す操作が必要となるが,これは機械的に実現できる.
それについては参考文献\cite{OGAWA2000}を参照されたい.

最後に,
漢字表記と読みをそれぞれ別の単語として登録した.
形態素解析では,字面だけを見て解析するため,例えば
「物」が辞書に登録してあっても「もの」は解析できない.
そこで,見出し語の表記と読み仮名を別々に登録し,
例えば,$<$物, もの, 名詞, madda$>$に対しては
$<$物, 名詞, madda$>$と$<$もの, 名詞, madda$>$を辞書に登録した.
また,動詞・形状動詞・形状名詞の語尾も漢字表記に合わせて削除した.

以上の結果,表\ref{table:judic_for_majo}に示した数の
単語が機械翻訳システム用の辞書に登録された.
表\ref{mytab:juwtab}に比べて単語数が増えているのは,
仮名表記を登録したためであり,語彙数自体は約2万語のままである.
登録した仮名表記は表\ref{mytab:juwtab}の合計見出し語数と
表\ref{mytab:juwtab}の合計見出し語数の差の
約16,000語である.

ただし,見出し語を形態素解析する段階で,不適切な解析となったもの
については手作業で修正をした.
例えば,前章で作成した日本語--ウイグル語辞書には
$<$息苦しい,自動詞,!kisilma!k$>$という組があったが,
ここで「自動詞」という品詞が付加されているのは,ウイグル語の
`!kisilma!k'が自動詞だからである.
しかし日本語訳の「息苦しい」は動詞ではないため,そのままでは辞書に
登録できない.
そこで,`!kisilma!k'の語幹に連体修飾語とする語尾
`-idig!an'
を付加し,その品詞を「修飾詞」とした.
よって,$<$息苦しい,自動詞,!kisilma!k$>$の場合は
$<$息苦しい,修飾詞,!kisilidig!an$>$を辞書に登録した.
今回の辞書作成では,このようにして登録した単語は20語あった.
\begin{table}[tbp]
\begin{center}
\caption{機械翻訳システム用日本語--ウイグル語辞書の登録語数}
\label{table:judic_for_majo}
\footnotesize
\begin{tabular}{|l||r|r|r|r|r|r|r|r|r|r|r|r|}
\hline
品詞 & 名詞 & \parbox[c]{2em}{動名詞} & 動詞 & \parbox[c]{2em}{形状動詞} & 
\parbox[c]{2em}{形状名詞} & \parbox[c]{2em}{修飾詞} & 副詞 & \parbox[c]{2em}{助数詞} & 
\parbox[c]{2em}{感嘆詞} & \parbox[c]{2em}{接続詞} & \parbox[c]{2em}{連体詞} & 合計\\
\hline
\parbox[c]{2em}{登録語数} & 18,323& 1,629& 7,611 & 1,320 & 1293 & 4,099 & 1,375 & 294 & 
110 & 26 & 22 & 36,102\\
\hline
\end{tabular}
\end{center}
\end{table}

また,実際の機械翻訳には元のウイグル語--日本語辞書に掲載されていな
かった助詞,動詞接尾辞などが必要となるが,それらについては
ウイグル語の解説書を参考に人手で343語を登録した.
よって現在のシステム用の辞書の収録語数は表\ref{table:judic_for_majo}の36,102語に
人手で登録した単語を加えた36,445語である.
\section{日本語--ウイグル語電子辞書の評価}\label{section:hyoka}
これまでに説明してきた一連の作業によってでき上がった
日本語--ウイグル語電子辞書(以下日--ウ辞書と略記する)
が,日本語--ウイグル語機械翻訳システムの
実験用の辞書として,また日常普通に使う日本語--ウイグル語辞書として
どの程度満足できるかについて評価をしておくことは当然であろう.

本論文で,著者らは2種類の評価を行なった.一つは,国立国語研究所が発表している
「教育基本語彙」に対する収録率であり,もう一つはEDRコーパスの日本語テキスト文に
含まれる出現頻度の高い単語に対する収録率である.前者は日本語の基本語彙と
認められている語に対する収録率であり,後者は新聞記事や雑誌記事に高い頻度で出現する
単語に対する収録率である.

この評価では,第\ref{section:for_majo}章で作成した翻訳システム用の辞書
(以下,システム用辞書と略記する)
を用いて,
単語が辞書に収録されているかどうかを自動的に判定した.
この判定では,単純な字面の一致だけを見るため,
例えば「何時も」と「いつも」のように
同じ単語が別の形で記されていた場合にも,辞書に収録されていないと
判定されてしまう.
こうした語は,
機械翻訳システム用の辞書においては
未収録語として扱うべきであるが,
人間が日常使う翻訳辞書として考えれば収録されていると見なしても良いであろう.
そこで,以下では最初に翻訳システム用辞書における収録率と未収録語に
ついて検討した.一方で,未収録語のうち
人間が使用する際には収録済みと見なすことができる語を調べ,
第\ref{section:jidoseisei}章で作成した辞書を人間が
通常用いる辞書として評価する際にはそれらを収録語と見なした.

まず,国立国語研究所のデータに関する評価について述べる.

国立国語研究所の「教育基本語彙データベース」\cite{KOKKEN}に含まれている
6,104(2,071+4,033)語の内,どれぐらいが
システム用辞書に収録されているか
を調べた結果,3,552個(58.20\,\%)の単語が含まれている
ことが分かった.ここで2,071+4,033と書いているのは,より基本的な語として「◎」印が
付されたのが2,071語,その他の基本語として「○」印が付されたのが4,033語と,
国語研の語彙データベースで二つに分けているので,そのように表わした.

\begin{figure}[tbp]
\begin{center}
\fbox{
\begin{minipage}{0.9\textwidth}
\epsfile{file=graphcover.eps,width=0.9\textwidth}
\end{minipage}
} \\
\caption{EDRコーパス上位頻度単語の日本語--ウイグル語
辞書に含まれる割合}\label{myfig:cover}
\end{center}
\end{figure}

第2の評価に関しては,我々は 約21万文が入っているEDRコーパスの日本語テキスト
文中の形態素の中から出現頻度の上位6,055語\footnote{ここで,6,055に
なっているのは基本語彙とされる国語研の「教育基本語彙データベース」の
語彙数は6,103であり,それに準じて6,000前後の上位頻度語を選んだ結果である.}
を選択し,その中でどれぐらいの数の単語が収録されているかを調べた.


\mbox{} \\
その結果,3,135個の単語(51.77\,\%)が我々が作成したシステム用辞書に
収録されていた.単語の出現頻度は,コーパスに含まれる文の出典や文の
選び方によって偏りが生じる可能性もあるが,第1の評価の結果と比較して
EDRコーパスに関してはその偏りは全体の収録率に大きく影響する程ではないと言えよう.
また,EDRコーパス出現頻度上位単語を500単位ずつ増していって,対する収録率を
グラフにまとめて図\ref{myfig:cover}に示した.

これらの評価1,2の結果は,基本語彙を6,000ベースで考えた時の
単語の約半分が収録されていないが,国語研においてより基本的とされる
語彙2,000語ベースで考えると,
それぞれ国語研の2,071語(以下国語研データと略記する),
EDRの上位頻出の2,056語(以下EDRデータと略記する)に対して
システム用辞書の収録率は,
国語研データに対して79.19\,\%(1,640語), 
EDRデータに対して78.45\,\%(1,613語)である.

そこで,我々は,未収録語(国語研データ 20.81\,\%, EDRデータ 21.55\,\%)に
関して,その収録されなかった理由や現象を調査し,5種類の分類規準を決定し,
その規準に基づいて収録されなかった単語の分類を行なった.
その結果を付録\cite{JUDIC_APPENDIX}に示した.また,それを集約した結果を
表\ref{mytab:nocov}にまとめた.


\newcommand{\myitemA}[1]{}
\newcommand{\myitemB}[1]{}
\newcommand{\myitemC}[1]{}
\newcommand{\myitemD}[1]{}
\newcommand{\myitemE}[1]{}
\newcommand{\kenum}[2]{}
\newcommand{\tkenum}[2]{}
\normalsize
\begin{table}[p]
\begin{center}
\caption{未収録語の調査結果}
\label{mytab:nocov}
\small
\normalbaselineskip=17pt
\begin{tabular}{|l|l|@{}c@{}|@{}c@{}|l@{}|}
\hline
 & 種類 & \multicolumn{2}{@{}c@{}|}
{
\begin{tabular}{c|c}
\begin{minipage}[c]{0.24\textwidth}現象\end{minipage} & \begin{minipage}[c]{0.24\textwidth}例\end{minipage} \\
\end{tabular}
}
 & 対処方法 \\
\hline
\myitemA{A} & \myitemB{形式上の違いの理由で含まれなかった単語\kenum{10}{44}}
 & \multicolumn{2}{@{}c@{}|}
{
\begin{tabular}{l|l}
\myitemC{(1)表記の違いや送り仮名の違いによるもの.\kenum{9}{35}}
 & \myitemD{「何時も」,「贈り物」は日本語--ウイグル語辞書の見出しに入っていないが,
「いつも」,「贈物」はその見出しに入っている.} \\
\hline
\myitemC{(2)日本語で,形容詞の語幹が形容動詞の語幹にもなるもの.
\kenum{0}{3}}
 & \myitemD{ウイグル語の形容詞 'illi!k' の
訳語として,「暖かい」[形]は辞書にあるが,「暖かな」[形動]はない.\\
\begin{minipage}[t]{0.24\textwidth}
\tiny
\setlength{\unitlength}{1mm}
\begin{picture}(32, 18)
\put(0, 15){ウイグル語} \put(26, 15){日本語}
\put(5, 10){\oval(12, 6)}
\put(1, 9){illi!k}
\put(29, 9.8){\oval(8, 7)}
\put(26.5, 8){\shortstack{\tiny 暖かい\\{[形]}}}
\put(26.5, 0){\shortstack{暖かな\\{[形動]}}}
\put(11, 10){\vector(4,0){14}}
\put(11, 10){\vector(2,-1){14}}
\end{picture} \\ \mbox{}
\end{minipage}
} \\
\hline
\myitemC{(3)日本語の動詞で,その意味がウイグル語では形容詞で表されるもの,または逆に,
日本語では形容詞で,その意味がウイグル語では動詞で表されるもの.\kenum{1}{6}} 
& \myitemD{「込む」は,動詞であるが,そのウイグル語訳である\\`besi!k'(込んでいる)は
形容詞である.\\
「痛い」は,形容詞であるが,そのウイグル語訳である\\`a!grima!k'は動詞である.
}
\end{tabular}
} & \myitemE{これらの単語に対しては機械的処理を行なうことにより,辞書に追加
できると考えられる.} \\
\hline
\myitemA{B} & \myitemB{元のウイグル語--日本語辞書で,日本語の訳付が不充分と考えられる
単語\kenum{168}{205}}
 & \multicolumn{2}{@{}c@{}|}
{
\begin{tabular}{l|l}
\myitemC{(1)同じ概念の違った表現.\kenum{15}{26}} 
& \myitemD{「米国」と「アメリカ」,\\
「火曜」と「火曜日」など.} \\
\hline
\myitemC{(2)その単語の類似語が辞書に存在するもの.\kenum{42}{58}} 
& \myitemD{「辺り」対して「周り」,\\「危ない」に対して「危険な」など.} \\
\hline
\myitemC{(3)元のウイグル語--日本語辞書でウイグル語に語義が複数あるのに,
その一部が落ちているもの.\kenum{95}{78}} 
& \myitemD{
\tiny
\setlength{\unitlength}{1mm}
\begin{picture}(33, 16)
\put(0, 14){ウイグル語}
\put(26, 14){日本語} 
\put(4, 6.5){\oval(10, 6)}
\put(0, 6){eqilma!k}
\put(9, 7){\vector(3,0){12}}
\put(30, 6.5){\circle{6}}
\put(22, 6){(門が) 開く}
\put(22, 0){(花が) 咲く}
\put(9, 7){\vector(2,-1){12}}
\end{picture}
} \\
\end{tabular}
} & \myitemE{ウイグル語--日本語辞書の編集時にできるだけ多くの訳語を入れる.} \\
\hline
\end{tabular}
\end{center}
\end{table}
\normalsize
\begin{table}[p]
\addtocounter{table}{-1}
\begin{center}
\caption{未収録語の調査結果(続き)}
\small
\normalbaselineskip=17pt
\begin{tabular}{|l|l|@{}c@{}|@{}c@{}|l@{}|}
\hline
 & 種類 & \multicolumn{2}{@{}c@{}|}
{
\begin{tabular}{c|c}
\begin{minipage}[c]{0.24\textwidth}現象\end{minipage} & \begin{minipage}[c]{0.24\textwidth}例\end{minipage} \\
\end{tabular}
}
 & 対処方法 \\
\hline
\myitemA{
} & \myitemB{
}
 & \multicolumn{2}{@{}c@{}|}
{
\begin{tabular}{l|l}
\myitemC{(4)ウイグル語--日本語辞書の編集時に,複数の品詞を持つウイグル語に対して品詞の
付け忘れがあったもの.\kenum{15}{28}} 
& \myitemD{ウイグル語の形容詞の`!kizil'\\(赤い)は名詞の場合もあるが,[名詞]という品詞を
付けなかった.\\
\begin{minipage}[t]{3.2cm}
\tiny
\setlength{\unitlength}{1mm}
\begin{picture}(32, 18)
\put(0, 15){ウイグル語} \put(26, 15){日本語}
\put(5, 10){\oval(12, 6)}
\put(2, 9){!kizil}
\put(29, 10.4){\oval(8, 7)}
\put(27, 8){\shortstack{赤い\\{[形]}}}
\put(27, 0){\shortstack{赤\\{[名]}}}
\put(11, 10){\vector(4,0){14}}
\put(11, 10){\vector(2,-1){14}}
\end{picture}
\end{minipage} \\ \mbox{}
} \\
\hline
\myitemC{(5)日本語の丁寧語,尊敬語,謙譲語などの待遇表現.\kenum{1}{15}} 
& \myitemD{「参る」,「おっしゃる」,\\「いただく」など.
} \\
\end{tabular}
} & \myitemE{
} \\
\hline
\myitemA{C} & \myitemB{ウイグル語訳が一語で表せない単語.\kenum{191}{151}} 
& \multicolumn{2}{@{}c@{}|}
{
\begin{tabular}{l|l}
\myitemC{(1)ウイグル語訳が派生語であるもの.\kenum{28}{9}} 
& \myitemD{
\setlength{\unitlength}{1mm}
\begin{picture}(33, 16)
\put(-1, 5){「従来」} \put(18, 5){{\bf !esli}{$\cdot$}d!e} 
\put(11, 6){\vector(3,0){6}}
\end{picture}
} \\
\hline
\myitemC{(2)ウイグル語訳が2語以上の組合せからなる複合語であるもの.
\kenum{150}{130}} 
& \myitemD{
\setlength{\unitlength}{1mm}
\begin{picture}(33, 16)
\put(-2, 9){「握手」} \put(15, 6){\shortstack{!kol  \ elixma!k \\ 手  \ \ 取り合う}} 
\put(9, 10){\vector(3,0){5}}
\put(-2, 1){「彼処」} \put(15, -2){\shortstack{awu \ \ \ \ y!er \\ あの \ \ \  ところ}} 
\put(9, 2){\vector(3,0){5}}
\end{picture}
} \\
\hline
\myitemC{(3)その語をウイグル語に訳す時,言い換えや語順の変更が必要になるもの.
\kenum{10}{11}} 
& \myitemD{
「...過ぎ」[接尾]$\longrightarrow$ \\!h!eddidin oxu!k ... (限度を超えて...)
} \\
\hline
\myitemC{(4)その日本語を含む連語をウイグル語に訳すもの.
\kenum{3}{1}} 
& \myitemD{
「一緒」$\longrightarrow$ ...と{\bf 一緒}に \\
$\longrightarrow$ ... bil!en bill!e \\ \\
しれ$\longrightarrow$ かも{\bf しれ}ない \\ 
$\longrightarrow$ ... m!umkin \\ \mbox{}
} \\
\end{tabular}
} & \myitemE{辞書に新たに登録する.} \\
\hline
\end{tabular}
\end{center}
\end{table}
\normalsize
\begin{table}[p]
\addtocounter{table}{-1}
\begin{center}
\caption{未収録語の調査結果(続き)}
\small
\normalbaselineskip=17pt
\begin{tabular}{|l|l|@{}c@{}|@{}c@{}|l@{}|}
\hline
 & 種類 & \multicolumn{2}{@{}c@{}|}
{
\begin{tabular}{c|c}
\begin{minipage}[c]{0.24\textwidth}現象\end{minipage} & \begin{minipage}[c]{0.24\textwidth}例\end{minipage} \\
\end{tabular}
}
 & 対処方法 \\
\hline
\myitemA{D} & \myitemB{その日本語単語のウイグル語訳が元のウイグル語--日本語辞書から
,何らかの理由で外れた単語\kenum{65}{8}}
& \multicolumn{2}{@{}c@{}|}
{
\begin{tabular}{l|l}
\myitemC{(1)技術的用語.元の辞書ができた時点で,まだ使われていなかったか,
技術的な用語との理由で外れた単語.\kenum{26}{1}} 
& \myitemD{「テレビ」,「アプリケーション」,「エイズ」など.} \\
\hline
\myitemC{(2)現代用語に含まれると考えられる固有名詞.\kenum{31}{0}} 
& \myitemD{「リクルート」,「ゴルバチョフ」,「IBM」など.} \\
\hline
\myitemC{(3)普通に使われるウイグル語であるが,たまたま元のウイグル語--日本語辞書の
見出し語から外れたと考えられるもの.\kenum{1}{4}} 
& \myitemD{「人形(!koqa!k)」,「もしもし(w!ey)」など.} \\
\hline
\myitemC{(4)ウイグル語では接尾辞として訳されるもの.\kenum{7}{3}} 
& \myitemD{「...的」,「...性」,「...目」など.} \\
\end{tabular}
} & \myitemE{辞書に新たに登録する.} \\
\hline
\myitemA{E} & \myitemB{その概念がウイグル語の複合語や句で表すことのできない単語.\kenum{9}{23}} 
& \multicolumn{2}{@{}c@{}|}
{
\begin{tabular}{l|l}
\myitemC{(1)その概念が元々ウイグル語にない単語.\kenum{6}{18}} 
& \myitemD{「昭和」,「畳」,「神社」など.} \\
\hline
\myitemC{(2)総称としての概念が,ウイグル語にはない単語.
\kenum{1}{3}} 
& \myitemD{「親」,「親子」,「菓子」など.} \\
\hline
\myitemC{(3)対応する語法がウイグル語には存在しないもの.
\kenum{2}{2}} 
& \myitemD{「お」[接頭語],「ご」[接頭語].} \\
\end{tabular}
} & \myitemE{} 
\\
\hline\hline
合計 & \multicolumn{4}{@{}l@{}|}{\myitemB{\mbox{}\kenum{443}{431}}} \\
\hline
\end{tabular}
\end{center}
\end{table}


\subsection{分類規準}\label{subsection:kijun}
日--ウ辞書の見出し語に採録されなかった単語を大きく5種類に分類し,その理由
ないしは現象を以下に説明する.

\vspace{0.2cm}
A) 形式上の違いの理由で収録されなかった単語

これらは,形式的な違いだけの理由でシステム用辞書に含まれなかったが,
人間が利用する際には簡単に参照可能であり,
語彙としては含まれていると考えられるものである.

ここで形式的な違いというのは,表記の違いと品詞付けの違いであり,次のように
3つに細分類される.

(1) 漢字表記と仮名表記の違いと送り仮名の違い. 国語研では,「何時も」と表記され,
日--ウ辞書では「いつも」と表わされる場合である.

(2) 日本語では形容詞と形容動詞が区別されていることに起因する違い. 
国語研データでは,形容詞「暖かい」と形容動詞「暖かな」の両方が基本語彙と
されていた.
しかし,ウイグル語の形容詞`illi!k'の訳語としては「暖かい」はあっても「暖
かな」はなかったため,逆辞書においても「暖かな」の見出し語はないとされた.
しかし,この二つは語彙としては同じであり,日本語の「暖か」が形容詞の語幹にも
形容動詞の語幹にもなりえるために片方が収録されていないと判定されたので
ある.

(3) 動詞による表現と形容詞による表現の違い. 例えば,日本語の
動詞「込む」はウイグル語では形容詞`besi!k(込んでいる)'に相当し,
また,
日本語の形容詞「痛い」はウイグル語では
動詞`a!grima!k(痛む)'と表わされる.

このように,
日本語では動詞とされる語に対応するウイグル語が形容詞であったり,
また,その逆に,日本語では
形容詞とされる語に対応するウイグル語が動詞であるような場合である.
付録\cite{JUDIC_APPENDIX}では,これらの3つの場合をA1, A2, A3の種類として分類し,備考欄には,その
根拠を示す語を示している.\\

B) ウ--日辞書の日本語訳付けが不十分なことが理由で収録されなかった単語

ウ--日辞書\cite{UJDIC}は,ウイグル語--中国語辞書\cite{UHDIC}を日本語に
翻訳して作られているので,間接的な訳付けに基づく意味の欠落や歪みがあったり,
ウイグル語単語の
意味が十分理解されないまま訳付けされていたりする場合がある.また,
ウ--日辞書\mbox{\cite{UJDIC}}は,
日本語を母語とする人を対象にしたため,ウイグル語単語の日本語訳の表現が
いくつかあっても,その中の一部だけを訳語にしている場合が見られる.
例えば,ウイグル語の単語`h!et!erlik'の訳としては,「危険な」や「危ない」
が考えられるが,辞書には「危険な」だけが採録されていた.
このような理由で訳語が欠落すると,
本論文の方法で辞書を生成する場合には,見出し語として収録されないことになる.
このような場合に対処するには,ウ--日辞書を整備する段階で,できるだけ多くの日本語訳を
付しておくことである.
このクラスに属する単語は,
国語研では205語,EDRでは168語である.
しかし,人間が利用する場合を考えれば,「危険な」のウイグル語訳があれば
「危ない」という語のウイグル語訳にもそれを使うことができ,
そうした観点からは,このクラスに入る単語は,
A)に属する単語と同様に,
第\ref{section:jidoseisei}章で作成した人間用の辞書には,
実質上,含まれていると考えてもよいであろう.

このB)に属する単語はさらに次の5つに細分類される.

(1)表現は異なっているが同じ概念の語が辞書に存在する場合. 例えば,「米国」と
「アメリカ」,「火曜」と「火曜日」のように,前者は辞書にないが,それと同じことを
表わしている後者が辞書にある場合である.

(2)類似語が辞書に存在する場合. 例えば,「辺り」に対して「周り」,「危ない」に対して
「危険な」のように,前者は,辞書に入っていないが,類似語の後者が入っている場合である.

(3) ウイグル語単語に複数の語彙があるのに,ウ--日辞書ではその一部が落ちているために,
日-ウ辞書に収録できなかった場合. 例えば,`eqilma!k'には「門が開く」の「開く」と
「花が咲く」の「咲く」という語義があるが,「咲く」という語義が付されていなかったので,
「咲く」が収録されなかったというような場合である.

(4) ウ--日辞書編集時に複数の品詞を持つウイグル語に対して品詞の付け忘れがあった場合. 例えば,
`!kizil'は,「赤い」と「赤」のように形容詞と名詞の2つの品詞を持つが,[形容詞]と
だけ品詞付けをしたので,名詞の「赤」という語が収録できなかったような場合である.

(5) 日本語の丁寧語・尊敬語・謙譲語などの待遇表現で,それが意味することを表わす通常の語が
辞書に存在する場合. 例えば,「お父様」「参る」は辞書にないが,「お父さん」「行く」は
あるような場合である.

付録\cite{JUDIC_APPENDIX}の表では,これらの5つの場合をB1, B2, B3, B4, B5の種類として
分類し,その根拠になる語を備考欄に示した.\\

C) 日本語のウイグル語訳が一語で表わせないことが理由で収録されなかった単語

このクラスに属する日本語単語にウイグル語訳を付けようとすると,
派生語もしくは複合語として表現せざるを得ない場合である.
このような現象は,
日本語--ウイグル語のみならず,他言語間の辞書作成時にも当然現れる.
これらの単語については,
例えば辞書の用例部分から獲得するか,
人手によって新たに登録する必要がある.

このクラスに属する単語は,次の4つに細分類される.

(1) ウイグル語訳が派生語である場合. 例えば,日本語の「従来」が`{\bf !esli}$\cdot$d!e'と,
`!esli'の派生語と表わされるような場合.

(2)ウイグル語訳が2語以上の組み合わせからなる複合語である場合. 例えば,「握手」が
`!kol elixma!k' のように,`!kol(手)' と 
`elixma!k(握る)' の複合語として表わされるような場合である.

(3) ウイグル語訳をするときに,言い換えや語順の変更が必要になる場合. 例えば,``食べ過ぎ''
の「〜過ぎ」は`!h!eddidin oxu!k ~(yem!ek)'(限度を超えて〜する(食べる))のように訳す.

(4) その日本語単語単独でなく,それを含む連語をウイグル語に訳す場合. 例えば,「一緒」
は,「〜と{\bf 一緒}に」のような連語として考え,
それを訳して`〜 bil!en bille'とするような場合である.

付録\cite{JUDIC_APPENDIX}では,これらの4つの場合をC1, C2, C3, C4の4つの種類に細分類し,
C1の場合には,備考欄に派生語がどのように分解されるかを示している.
また,C4の備考欄には,その単語を含む連語を示している.\\

D) その日本語単語に相当するウイグル語単語が何らかの原因でウ--日辞書の見出し語から外れたために
収録されなかった単語

基礎としたウ--日辞書\cite{UJDIC},すなわち,そのもとのウイグル語--中国語辞書\cite{UHDIC}を
作った時に,その単語に相当するウイグル語単語が,何らかの原因でその見出し語から外れたために
収録されなかった場合である.このグループに属する単語の数については,国語研データ 8語,
EDRデータ 66語とはっきりした差がある.
これは,EDRデータでは,
技術用語や現代用語に現れる固有名詞が多く出現していたためである.
必要ならばこれらの単語は人手で登録するより他にない.

このクラスに属する語は,その原因別に次のように4つに細分類される.

(1) 技術用語で元のウ--日辞書が作られた時にまだ使われていなかったか,技術用語との理由で
外れた単語. 例えば,「テレビ」,「アプリケーション」,「エイズ」などである.

(2) 現代用語に含まれると考えられる固有名詞. 例えば,「リクルート」,「ゴルバチョフ」,
「IBM」,「中曽根」などである.

(3) 普通に使われるウイグル語であるが,たまたま元のウ--日辞書の見出し語から外れた単語. 例えば,
「人形(!koqa!k)」,「もしもし(w!ey)」などである.

(4) ウイグル語では接尾辞として訳される単語. 元のウ--日辞書には,接尾辞は
登録されていなかったために,
ウイグル語では接尾辞として訳される日本語単語は採録されなかった
ものである.例えば,
「君」「的」「性」「目」などである.

付録\cite{JUDIC_APPENDIX}では,これらの語をD1, D2, D3, D4と分類し,このクラスに属する単語に対して,
ウイグル語訳が付けられる場合には,それを記してある.\\

E) その概念をウイグル語の複合語や句で表わすことのできない単語

日本語単語が表わす概念がウイグル語にはなく,ウイグル語の複合語や句で表現することが
できない場合である.このように,一方の言語の単語が表わす概念を他方の言語で
簡潔に表現できない問題はどのような言語間でも存在する.このクラスの単語が,
国語研データの方で
多く見られるのは,EDRデータには
新聞や雑誌での高頻出語が含まれるのに対して,
国語研データには日本語特有の語彙を収録しているためと考えられる.
これらの単語を登録する場合,人間用の辞書であれば,例文などを添えながら
説明を書くことになる.
システム用の辞書であれば,日本語の読みをそのまま登録するか,E3の場合には
何も訳出しないことになる.

このクラスに属する単語は,次の3つに細分類される.

(1) 概念がウイグル語にない単語. 例えば,「昭和」「畳」「神社」などである.

(2) 総称として概念がウイグル語にない単語. 例えば,「親」「親子」「菓子」などである.
ウイグル語で「親」は「父と母(ata-ana)」と表わされ,
「親子」は「父子(ata-bala)」もしくは「母子(ana-bala)」としか表現できない.

(3) 対応する語法がウイグル語には存在しない場合. 例えば,「お食事」の
    「お」,
「ご挨拶」の「ご」
などである.

\subsection{分類規準の妥当性と分類結果の検討}\label{subsection:kento}
前節では,我々の日--ウ辞書の見出し語として未収録となった原因をA〜Eまで大きく5つに分類し,
さらにそれらを細分類して,A1〜A3, B1〜B5, C1〜C4, D1〜D4, E1〜E3の全部で
19の規準を与えた.
この結果は表\ref{mytab:nocov}にまとめられた通りである.

国語研の教育用基本語彙は学習者に
日本語教育をする上で基本的と見なされるもの
とされており,
一方,EDRコーパスの日本語文の高頻度出現の語彙は,新聞などから抽出されたもの
であり,以下にも述べるように,これらは性質の異なったソースと考えられる.
これらの2つの異なった語彙集合のいずれについても,未収録語は上の規準に従って無理なく分類が
でき,しかも,未収録の原因のいずれについても,それぞれどのように対処すればよいかを
それ自身が示している.この意味で,\ref{subsection:kijun}で示した分類規準は妥当であると
考えられる.

次に,この規準に従って,2,000語ベースの国語研データ,EDRデータの未収録語を分類した
結果について検討する.まず,A1〜E3までの理由ごとに,未収録語数を示すグラフを
図\ref{myfig:uncover}に示す.
この図\ref{myfig:uncover}と
表\ref{mytab:nocov}を見ると,次のような特徴が分かる.\\ \mbox{}
\begin{figure}[tbp]
\begin{center}
\epsfile{file=graphuncover.eps,width=0.9\textwidth}
\caption{日本語--ウイグル語辞書の未収録語の分布}\label{myfig:uncover} 
\mbox{}
\end{center}
\end{figure}

\noindent
{\bf 国語研データとEDRデータの共通的特徴} \\

1. 未収録語数は,国語研データでは431語(431 / 2,071 = 20.81\,\%), EDRデータでは 443語
(443 / 2,056 = 21.55\,\%)で,共に約20\,\%である.


2. 種類 B, C に属する語が未収録語の主要部分を占めており, 国語研データでは
205(B) + 151(C) = 356 語(356 / 431 = 82.60\,\%), EDRデータでは 168(B) + 191(C) = 359語
(359 / 443 = 81.04\,\%)で,共に未収録語全体の約80\,\%である.\\

\noindent 
{\bf 国語研データとEDRデータの対照的特徴} \\

3. 種類Dに属する語は,国語研データでは8語であるのに対して,EDRデータは66語で,約8倍である.
D1〜D4に細分してもD3を除いてこの傾向は変わらない.これは,EDRデータが技術的な用語や時代を
反映した現代用語的な語彙を多数含んでいることから自然な結果である.
D3は本来辞書にあるのが当然の語彙である.

4. 種類Eに属する語は,国語研データでは23語であるのに対して,EDRデータは9語であり,
3分の1余りである.これは,国語研データが,教育上基本的であるとして選定された
もので日本語特有の概念を含むことになると考えることができよう.

5. 種類Aについては,国語研データでは44語,EDRデータは10語である.種類A1, A2は,
表記の規準化(送り仮名の統一,語幹が同じである形容詞・形容動詞の処理など)
によってほとんど解消されると考えられる.

以上,未収録語の分類結果を検討し,1. 〜5. が観察された.

既に述べたように,種類Aおよび種類Bに属する語は,人間用の辞書として
は収録されていると見なせるであろう.
種類Aと種類Bに属する語を合計すると,国語研データでは249語,
EDRデータでは
178語である.これらは,2,000語ベースで収録されなかった
それぞれの語数431語(国語研), 
443語(EDR)の57.77\,\%, 40.18\,\%である.
よって,第\ref{section:jidoseisei}章で作成した辞書は
2,000語ベースで考えれば,
それぞれ91.21\,\%, 87.11\,\%の収録率をもつと見なせる.
その意味では,
本論文の自動的に生成した日--ウ辞書が
当初の目的を達成していることを示していると
考えてもよいであろう.

また,
3., 4. に見るように,
国語研データとEDRデータとでは対照的であり,これは何を基本語彙とするかに依存する
ところである.生活習慣や社会習慣が異なれば語彙も異なる.また,次々に現れる現代用語を
辞書見出し語にするのは難しい.このように見ると種類D, Eの語があることは我々が作成した
辞書のような場合には決定的な欠点にはならないとしてもよいであろう.

一方で,
種類Cの語数は,国語研データ151語,
EDRデータ191語であり,2,000語ベースで考えれば,それぞれ7.29\,\%, 9.29\,\%である.
種類C, 即ち,ウイグル訳が一語で表現できない場合には,
ウ--日辞書中の例文の中から対訳を抽出して登録したり,
人手で新たに辞書登録するなどの作業が必要であり,
今後の課題でもある.
現在,ウ--日辞書中の例文の中から
抽出した訳語を人手でチェックして
いる段階である.
\section{まとめ}\label{section:owari}
本論文では,実用に近い日本語--ウイグル語機械翻訳システムの実現の目的で,少なくとも
日常使われる最低限の語彙を含む日本語--ウイグル語辞書の開発を目標に,既存の語彙数約
16,000語のウイグル語--日本語辞書\cite{UJDIC}から,できるだけ自動的な手順,作業で
語彙数約20,000語の日本語--ウイグル語辞書を開発したプロセスについて説明し,その成果として
得られた日本語--ウイグル語辞書の収録語の分析を行なった.

その結果,機械翻訳システム用の辞書に関し,
国立国語研究所の教育基本語彙データベース\cite{KOKKEN}における,
より基本的な語彙2,071語に対する収録率が約79\,\%, 
また,EDRコーパスの約21万の日本語文から
抽出した出現頻度上位2,056語に対する収録率が約78\,\%であった.
また,\ref{subsection:kento}節で述べた観点から
人間が利用する辞書として考えれば,
それぞれ,91\,\%, 87\,\%の収録率であった.このように,
所期の目標の日本語--ウイグル語翻訳のための辞書を得ることができた.

次に,表\ref{mytab:nocov}に整理した未収録語の分類規準は,本研究の場合だけでなく,
一般に,日本語から他の言語への翻訳辞書の評価の場合にも適用できると考えられる.日本語
からの,と制限しない場合で,任意の言語間の翻訳辞書作成の場合でも,その枠組みは
利用可能である.

さらに,本研究の日本語--ウイグル語辞書の場合と同様に例えばトルコ語--日本語辞書はあるが,
日本語--トルコ語辞書がないときに,それを生成をする場合に,本研究で採用した
逆辞書を作るという手順を用いるアプローチはコストと時間の視点から有効な手段を
提供すると考えられる.

本研究で日常使われる最低限の語彙を含む日本語--ウイグル語辞書ができたので,
引き続いてこれを用いて実用に近い日本語--ウイグル語機械翻訳システムの実現を計りたい.
その際に,種々の形で辞書の充実が必要になるであろうが,語彙の追加の他に,
例えば,専門用語や外来語に対する対処,あるいは日本語固有の概念を表わす単語に
対する対処の仕方など,場合に応じた工夫も必要になろう.


\bibliographystyle{jnlpbbl}
\bibliography{judicj}

\begin{biography}
\biotitle{略歴}
\bioauthor{ムフタル・マフスット}{
1983年新疆大学数学系卒業.1996年名古屋大学大学院工学研究科情報工学専攻博士課程満了.
同年,三重大学助手.2001年より,名古屋大学助手.工学博士.
自然言語処理に関する研究に従事.人工知能学会,情報処理学会各会員.}
\bioauthor{小川 泰弘}{
1995年名古屋大学工学部情報工学科卒業.2000年同大学院工学研究科情報工学専攻博士課程後期
課程修了.同年より,名古屋大学助手.自然言語処理に関する研究に従事.
言語処理学会,情報処理学会各会員.}
\bioauthor{杉野 花津江}{
1961年愛知学芸大学数学科卒業.1965年より名古屋大学工学部助手.
1997年〜2003年3月まで同大学院工学研究科助手.現在,同大学院情報科学研究科
臨時補助員.オートマトン・言語理論,確率オートマトン,自然言語処理に関する研究に従事.
情報処理学会,電子情報通信学会各会員.}
\bioauthor{稲垣 康善}{
1962年名古屋大学工学部電子工学科卒業.
1967年同大学院博士課程修了.
同大助教授,三重大学教授を経て,1981年より名古屋大学工学部・大学院工学研究科教授.
2003年4月より同大学名誉教授,愛知県立大学情報科学部教授.工学博士.
この間,スイッチング回路理論,オートマトン・言語理論,計算論,ソフトウエア基礎論,
並列
処理
論,代数的仕様記述法,人工知能基礎論,自然言語処理などの研究に従事.
言語処理学会,情報処理学会,電子情報通信学会(現在副会長),人工知能学会,
日本ソフトウエア科学会,IEEE,ACM,EATCS各会員.}

\bioreceived{受付}
\biorevised{再受付}
\biorerevised{再々受付}
\bioaccepted{採録}

\end{biography}
\end{document}

