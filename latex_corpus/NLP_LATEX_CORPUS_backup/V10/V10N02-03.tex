


\documentstyle[epsf,jnlpbbl]{jnlp_j_b5}

\setcounter{page}{43}
\setcounter{巻数}{10}
\setcounter{号数}{2}
\setcounter{年}{2003} 
\setcounter{月}{4}
\受付{2002}{8}{13}
\再受付{2002}{11}{6}
\採録{2003}{1}{10}

\title{``名詞Aのような名詞B''表現の比喩性判定モデル}
\author{田添 丈博\affiref{SNCT} \and 椎野 努\affiref{AICHI} \and 桝井 文人\affiref{MIEU} \and 河合 敦夫\affiref{MIEU}}

\headauthor{田添,椎野,桝井,河合}
\headtitle{``名詞Aのような名詞B''表現の比喩性判定モデル}

\affilabel{SNCT}{鈴鹿工業高等専門学校電子情報工学科}
{Department of Electronic and Information Engineering, Suzuka National College of Technology}
\affilabel{AICHI}{愛知工業大学工学部情報通信工学科}
{Department of Information Network Engineering, Faculty of Engineering, Aichi Institute of Technology}
\affilabel{MIEU}{三重大学工学部情報工学科}
{Department of Information Engineering, Faculty of Engineering, Mie University}

\jabstract{
我々は文章中に現れる比喩表現,その中でも直喩・隠喩的な比喩について,その認識・抽出を目的として研究を進めている.
本論文では,``名詞Aのような名詞B''表現について,名詞の意味情報を用いたパターン分類によって比喩性を判定し,比喩表現については喩詞(喩えるもの)と被喩詞(喩えられるもの)とを正確に抽出できるモデルを提案する.
この表現には比喩(直喩)とリテラル(例示など)の2つの用法があり,また比喩であっても名詞Bが被喩詞ではない場合がある.
我々はそれらを機械的に判定するために,これまでに行ってきた構文パターンやシソーラスを用いて喩詞と被喩詞の候補を抽出する手法を発展させ,名詞Aと名詞Bの意味情報やその関係に従って``名詞Aのような名詞B''表現を6つのパターンに分類し,比喩性を判定し喩詞と被喩詞を特定するモデルを構築した.
このモデルを日本語語彙大系の意味情報を利用して実装し,新聞記事データを用いて検証したところ,得られたパターン分類結果(比喩性判定結果)と人間のそれとが一致する割合は,学習データについては82.9\,\%(未知語データを除く),評価用データについては72.7\,\%(同)であり,比喩性判定モデルの全体的な処理の流れは実際の文章中の比喩表現認識に有効であることを示した.
また,比喩語という比喩性を決定づける語についてもその効果を示すことができ,モデルへの組み込みの可能性を示唆した.
}

\jkeywords{比喩性判定,``名詞Aのような名詞B'',パターン分類,意味情報}

\etitle{The Metaphorical Judgment Model \\
for ``Noun B like Noun A'' Expressions}
\eauthor{Takehiro Tazoe\affiref{SNCT} \and Tsutomu Shiino\affiref{AICHI} \and Fumito Masui\affiref{MIEU} \and Atsuo Kawai\affiref{MIEU}}

\eabstract{
We have been studying for the automatic recognition and extraction of metaphor expressions in practical sentences.
This paper introduces our metaphorical judgment model for ``Noun B like Noun A'' expressions.
``Noun B like Noun A'' expressions are classified into two usages; simile and literality.
To automatically judge whether a phrase is simile or literality, ``Noun B like Noun A'' expressions were classified into six patterns depending on the semantic information of a noun, and the metaphorical judgment model was constructed based on these patterns.
When the ``Noun B like Noun A'' expressions from newspaper articles were judged by the model, and its judgment was compared with the correct judgment, it was approximately 80\,\% correct.
Thus, the model was found to be effective in the recognition of metaphor expressions in real-life situations.
}

\ekeywords{metaphorical judgment, ``Noun B like Noun A'', pattern classification, semantic information}

\begin{document}
\maketitle
\thispagestyle{empty}


\section{はじめに}

近年,情報分野の認知度・重要度は急速に増し,それに伴って自然言語処理分野の研究もさらに活発なものとなっている.
形態素論から構文論へと研究は進み,現在は意味論に関する研究がその中心となっている.
比喩表現はその代表的なテーマの1つであり,我々の日常的なコミュニケーションも比喩表現の雛型としての言語知識に基づいた部分が多いとされている\cite{Lakoff-1}.

比喩表現に関する研究は,近年細かく分類され,様々なアプローチによる研究が精力的に進められている.
人工知能(自然言語処理)分野における比喩処理の研究として,Barndenは,ATT-Metaと呼ばれる比喩推論システムを試作している\cite{Barnden-1}.
このシステムは,cnduit metaphorと称する意味伝達に際しての理解のずれの枠組み\cite{Reddy-1}など,比喩表現についての言語学的な研究成果をもとに構築され,喩詞と被喩詞との意味的な共通領域を定量的に示すことができる.
コンピュータに比喩を理解させるためには概念の類似性や顕現性に関する知識が必要となるが,
TverskyやOrtonyは概念の属性集合の照合によって類似性を説明する線形結合モデルを提案し,顕現性を計算する際に重要な要素として情報の強度(intensity)と診断度(diagnosticity)を提案している\cite{Tversky-1,Ortony-3}.
今井らは連想実験に基づいて構成される属性の束を用いてSD法の実験を行い,その結果を円形図上に配置し,さらに凸包という幾何学的な概念を用いて相対的に顕現性の高い属性の抽出を行っている\cite{Imai-1}.
比喩表現を大きく直喩・隠喩的な比喩と換喩的な比喩とに分類すると,換喩的な比喩の研究として,村田らは「名詞Aの名詞B」「名詞A名詞B」の形をした名詞句を利用し,それを用いて換喩を解析することを試みている\cite{Murata-1}.
内山らは換喩的な比喩を研究対象に,統計的に解釈する方法について述べている\cite{Uchiyama-1}.
また内海らは直喩・隠喩的な比喩の研究について,関連性理論を基盤とした言語解釈の計算モデルを適用し,属性隠喩を対象として文脈に依存した隠喩解釈の計算モデルを提案している\cite{Utsumi-1}.
しかしこれらの研究はいずれも比喩とわかっている表現の解釈を中心に行われており,実際の文章に現れる表現が比喩であるかどうかといった比喩認識については,あまり深い議論はなされていない.

本研究は日本語文章の比喩表現,その中でも直喩・隠喩的な比喩について,その認識・抽出を目的としている.
我々はこれまで確率的なプロトタイプモデル\cite{Iwayama-1}を利用して,コーパスから知識を取り出すことによって比喩認識に用いる大規模な知識ベースを自動構築する手法を提案し
\cite{Masui-1}
,動作に基づく属性に注目した観点からの比喩認識を提案してきた
\cite{Masui-3}
.
これにより喩詞と被喩詞とからなる表現の定量的な比喩性判断が可能となった.
しかし,この手法を実際の文章に現れる表現に対して適用するためには,比喩表現候補の喩詞と被喩詞とを正確に抽出できなければならない.
これに対しては直喩の代表的な表現形式である``名詞Aのような名詞B''を対象に,構文パターンやシソーラスを用いる手法で研究を進めてきた
\cite{Tazoe-1,Tazoe-2}
が,喩詞・被喩詞を抽出する手法は,同時に``名詞Aのような名詞B''表現が比喩であるかどうかを判定することにも密接に関連するという結論に至った.
本論文では``名詞Aのような名詞B''表現について,意味情報を用いたパターン分類によって比喩性を判定し,喩詞と被喩詞とを正確に抽出できるモデルについて提案する.

本論文の構成を示す.
\ref{sec:bunrui}章では``名詞Aのような名詞B''表現について意味情報を用いたパターン分類とそれぞれのパターンの特徴・比喩性を述べる.
\ref{sec:teian}章では我々が提案する比喩性判定モデルの処理の流れを詳細に説明する.
\ref{sec:ko-pasu}章ではコーパスを用いた判定実験結果について考察を加える.
\ref{sec:hiyugo}章では明らかに比喩性を決定づける語の存在について検証する.

\section{``名詞Aのような名詞B''表現のパターン分類}\label{sec:bunrui}

\subsection{比喩とリテラル}\label{subsec:hiyulite}

比喩表現の中の直喩の代表的な表現形式に``名詞Aのような名詞B''がある.
しかし,``名詞Aのような名詞B''表現がすべて直喩であるとは限らない.
これについては従来から議論がなされており,
中村は,
対応する2つの名詞の意味領域が互いに排斥しあっている場合は比喩表現であり,
名詞Aの意味領域が名詞Bの意味領域の中に含まれると比喩表現であることが少なくなる,としている\cite{Nakamura-1}.
またOrtonyは``A is like B(AはBのようだ)''において,
A,B共通の属性の顕現性がAにおいて低くBにおいて高ければ比喩的な類似性であり,
Aにおいて高くBにおいても高ければリテラルな(字義通りの)類似性である,
としている\cite{Ortony-2}.

以上をまとめると,
``名詞Aのような名詞B''表現の用法は大きく2つに分けることができる.
1つは「じゅうたんのような芝」を例とする直喩であり,この用法を単に『比喩』と呼ぶことにする.
もう1つは「中国のような国」を例とする比喩ではない用法であり,この場合は例示を意味する.
このような比喩ではない用法は他にも,指示(例えば「次のような点」),人の判断(例えば「当たり前のようなこと」)などがあり,それらをまとめて『リテラル』と呼ぶことにする.

我々は``名詞Aのような名詞B''表現について,これら2つの用法をコンピュータで判定するモデルを提案する.
明らかにこの判定モデルは構文情報だけでは実現できず,意味情報や概念情報を扱う必要がある.

\subsection{意味情報を用いたパターン分類}\label{subsec:imitekina}

``名詞Aのような名詞B''表現が実際の文章の中でどのような用法で使用されているのかを調べるために,日本経済新聞\cite{Nikkei-1}の1994年1月分の記事約11万文について調査した.
その結果,``のような''を含む表現はちょうど500組抽出され,構文情報を用いて分類したところ``名詞句のような名詞句''表現は311組,その中で2つとも単一名詞(修飾語句がつかない)である``名詞Aのような名詞B''表現は78組抽出された(表\ref{tab:noyouna1}参照).

\begin{table}[htbp]
	\caption{``のような''を含む表現の分類}
	\begin{center}
	\begin{tabular}{|l|r|}
		\hline
		名詞Aのような名詞B & 78 \\
		\hline
		名詞句のような名詞句(上を除く) & 233 \\
		このような & 81 \\
		そのような & 20 \\
		どのような & 61 \\
		〜かのような & 9 \\
		これまでのような & 10 \\
		かつてのような & 3 \\
		のようなのだ & 2 \\
		のような, & 1 \\
		のような) & 2 \\
		\hline
		\hline
		計 & 500 \\
		\hline
	\end{tabular}
	\end{center}
	\label{tab:noyouna1}
\end{table}

その78組について,名詞Aと名詞Bの意味情報やその関係に従って,``名詞Aのような名詞B''表現が比喩なのかリテラルなのかを判定することを考慮しながら,次の6つのパターンに分類した.

\medskip
\begin{description}
  \item[パターン1:] 名詞Aと名詞Bが直接対比され,異種概念であるもの(あとのパターン2〜6に該当しないもの) -- 11組
\end{description}
\begin{quote}
\begin{description}
  \item[例:] じゅうたんのような芝 \\ 夢のような約束
\end{description}
\end{quote}

\medskip
\begin{description}
  \item[パターン2:] 名詞Aと名詞Bが直接対比され,名詞Bが名詞Aの上位概念であるもの -- 5組
\end{description}
\begin{quote}
\begin{description}
  \item[例:] 中国のような国 \\ 雑木林のような自然
\end{description}
\end{quote}

\medskip
\begin{description}
  \item[パターン3:] 名詞Bが名詞Aの静的な属性(部分,形状など)であるもの -- 11組
\end{description}
\begin{quote}
\begin{description}
  \item[例:] (スペース)シャトルのような羽根 \\ ピラミッドのような形
\end{description}
\end{quote}

\medskip
\begin{description}
  \item[パターン4:] 名詞Bが名詞Aの動的な属性(状態,状況など)であるもの -- 5組
\end{description}
\begin{quote}
\begin{description}
  \item[例:] ニュージーランドのような自然 \\ ボスニアのような問題
\end{description}
\end{quote}

\medskip
\begin{description}
  \item[パターン5:] 名詞Bが特定の名詞(抽象名詞)であるもの -- 24組
\end{description}
\begin{quote}
\begin{description}
  \item[例:] バールのようなもの \\ 刃物のようなもの
\end{description}
\end{quote}

\medskip
\begin{description}
  \item[パターン6:] 名詞Aが特定の名詞(人称代名詞,時制名詞,文中の場所を指す名詞,事物の評価を表す名詞)であるもの -- 27組
\end{description}
\begin{quote}
\begin{description}
  \item[例:] 私のような選手 \\ 君のような人間 \\ 現在のような環境 \\ 従来のような勢い \\ 次のような点 \\ 以上のような状況 \\ 当たり前のようなこと
\end{description}
\end{quote}

\medskip
ここで,例えば「次のようなもの」のようにパターン5とパターン6に属する表現が5組あり,両方のパターンでカウントしている.

\subsection{各パターンの比喩性}\label{subsec:kakupata-n}

\ref{subsec:hiyulite}の定義をもとに,
\ref{subsec:imitekina}で分類した各パターンの比喩性について説明する.

\medskip
パターン1(直接対比,異種概念)について
\begin{quote}
名詞Aと名詞Bが異種概念で直接対比されており,異種概念の類似属性を比較しているということで,比喩といえる.
この場合,名詞Aが喩詞であり,名詞Bが被喩詞である.
\end{quote}

\medskip
パターン2(直接対比,上下関係)について
\begin{quote}
名詞Aが名詞Bの概念に含まれており,これは比喩としての意味はなく,例示を表すリテラルである.
名詞Bの例として名詞Aが挙げられているのである.
\end{quote}

\medskip
パターン3(BがAの静的な属性)について
\begin{quote}
意味の上からは``AのBのようなXのB''であり,論理的にはAとXの関係で比喩かリテラルかが決まる.
実際のところ,``AのB''が普遍性のある静的な属性を表現することから,AとXは異種概念であることが一般的であり,比喩となる.
この場合,喩詞は``AのB'',被喩詞は``XのB''である.
しかしXは文脈のどこかに記されており,``AのようなB''だけを抽出した場合,Xを特定することはできない.
喩詞として``AのB''は抽出することができる.
\end{quote}

\medskip
パターン4(BがAの動的な属性)について
\begin{quote}
意味の上からはパターン3と同様``AのBのようなXのB''であるが,``AのB''が時と場合によってゆれのある動的な属性を表現することから,XはAを普遍化したもの(上位概念)であることが一般的であり,例示を表すリテラルとなる.
\end{quote}

\medskip
パターン5(名詞Bが特定の名詞)について
\begin{quote}
名詞Bが``もの'',``こと''など特定の抽象名詞の場合で,例示を表すリテラルである可能性が高い.
ただし,「特効薬のようなもの」や「安全弁のようなもの」などのように,名詞Aが比喩性を決定づける語(\ref{sec:hiyugo}章で述べる比喩語)であるときに比喩となる.
\end{quote}

\medskip
パターン6(名詞Aが特定の名詞)について
\begin{quote}
名詞Aが人称代名詞あるいは現在過去の時制名詞であるときは例示,文中の場所を指す名詞であるときは指示,事物の評価を表す名詞であるときは人の判断と,いずれもリテラルである.
\end{quote}

\medskip
以上をまとめると,表\ref{tab:noyouna3}のようになる.

\begin{table}[htbp]
	\caption{``名詞Aのような名詞B''表現のパターン分類}
	\begin{center}
	\begin{tabular}{|l|l|l|r|}
		\hline
		パターン1 & 直接対比,異種概念 & 比喩(喩詞``A'',被喩詞``B'') & 11組 \\
		パターン2 & 直接対比,上下関係 & リテラル & 5組 \\
		パターン3 & BがAの静的な属性 & 比喩(喩詞``AのB'') & 11組 \\
		パターン4 & BがAの動的な属性 & リテラル & 5組 \\
		パターン5 & 名詞Bが特定の名詞 &ほぼリテラル & 24組 \\
		パターン6 & 名詞Aが特定の名詞 & リテラル & 27組 \\
		\hline
	\end{tabular}
	\end{center}
	\label{tab:noyouna3}
\end{table}

\section{比喩性判定モデルの提案}\label{sec:teian}

\ref{sec:bunrui}章のパターン分類をもとに,``名詞Aのような名詞B''表現を入力として,比喩あるいはリテラルを判定し,比喩については喩詞・被喩詞を抽出するモデルを提案する(図\ref{fig:hantei_moderu}参照).


\begin{figure}[htbp]
\begin{center}
 \leavevmode
\vspace*{5mm}
 \epsfxsize=14cm
 \epsfysize=14.466cm
 \epsfbox{fig1.eps}
 \caption{比喩性判定モデル}
 \label{fig:hantei_moderu}
\vspace*{5mm}
\end{center}
\end{figure}

図\ref{fig:hantei_moderu}の各ステップについて説明する.

\medskip
ステップ1 [Aが特定名詞?]:
\begin{quote}
名詞Aが特定の名詞(人称代名詞,時制名詞,文中の場所を指す名詞,事物の評価を表す名詞)であるかを調べる.
特定の名詞であれば,パターン6(リテラル)と判定する.
そうでなければ,ステップ2へ渡す.
\end{quote}

\medskip
ステップ2 [Bが特定名詞?]:
\begin{quote}
名詞Bが特定の名詞(抽象名詞)であるかを調べる.
特定の名詞であれば,パターン5(ほぼリテラル)と判定する.
そうでなければ,ステップ3へ渡す.
\end{quote}

\medskip
ステップ3 [BがAの属性?]:
\begin{quote}
名詞Bが名詞Aの属性であるかどうかを調べる.
属性であればステップ4へ,属性でなければステップ5へ渡す.
\end{quote}

\medskip
ステップ4 [静的 or 動的]:
\begin{quote}
「名詞Aの名詞B」が静的な属性か動的な属性かを判断する.
静的な属性であれば,パターン3(比喩)と判定し,喩詞は``AのB''となる.
動的な属性であれば,パターン4(リテラル)と判定する.
\end{quote}

\medskip
ステップ5 [概念比較]:
\begin{quote}
名詞Bが名詞Aの属性でなければ,名詞Aと名詞Bは直接対比できると判断し,概念比較を行う.
名詞Bが名詞Aの上位概念であれば,パターン2(リテラル)と判定する.
そうでなければ,名詞Aと名詞Bは異種概念と判断し,パターン1(比喩)と判定,喩詞は``A'',被喩詞は``B''となる.
\end{quote}

\medskip
このモデルに従えば,\ref{subsec:imitekina}のパターン5(ほぼリテラル)とパターン6(リテラル)の両方に属する表現(例えば「次のようなもの」)は,パターン6(リテラル)と分類される.

\section{コーパスでの検証}\label{sec:ko-pasu}

\subsection{比喩性判定モデルの実現}\label{subsec:jitugen}

\ref{sec:teian}章の比喩性判定モデルを実現するために,意味情報として日本語語彙大系\cite{NTT-1}を利用して,図\ref{fig:hantei_moderu}の各ステップの具体的なルールを定義した.

その手順はまず,日本経済新聞1994年1月のデータ78組に対して,日本語語彙大系からそれぞれの名詞に意味情報を付与した.
その際に,1つの名詞に複数の意味情報が付与される,いわゆる多義性が生じるが,今回は我々が妥当と考える意味情報を1つだけ付与することとした.
そして,各ステップの処理とデータに付与された意味情報を参照しながら,比喩性判定モデルを実現するための具体的なルールを定義した.
各ステップのルールを説明する.

\medskip
ステップ1のルール:
\begin{quote}
日本語語彙大系では,名詞をさらに8種類(一般名詞,用言性名詞,転生名詞,副詞型名詞,連体詞型名詞,代名詞,形式名詞,固有名詞)に細分類している.
名詞Aが用言性名詞(サ変動詞型名詞,形容動詞型名詞)か,副詞型名詞(時詞,数詞など)か,代名詞か,形式名詞であれば,名詞Aが特定の名詞とする.
\end{quote}

\medskip
ステップ2のルール:
\begin{quote}
名詞Bが形式名詞であれば,名詞Bが特定の名詞とする.
\end{quote}

\medskip
ステップ3のルール:
\begin{quote}
名詞Aと名詞Bの意味情報を調べて,名詞Aが具体(意味属性番号2--994)で名詞Bが抽象(1000--2715)か,名詞Aが抽象物(1001--1234)で名詞Bが事・抽象的関係(1235--2715)であれば,名詞Bが名詞Aの属性とする.
\end{quote}

\medskip
ステップ4のルール:
\begin{quote}
名詞Aが固有名詞であれば,「名詞Aの名詞B」は動的な属性とする.
そうでなければ静的な属性とする.
\end{quote}

\medskip
ステップ5のルール:
\begin{quote}
名詞Aと名詞Bの意味情報を日本語語彙大系の階層構造に当てはめ,名詞Bが名詞Aの上位であるか,まったく同種であるか,異種(その他)であるか判定する.
名詞Bが名詞Aの上位であれば,上下関係とする.
名詞Aと名詞Bが異種(その他)であれば,異種概念とする.
名詞Aと名詞Bの意味情報が同種であるとき,名詞Aが固有名詞で名詞Bが一般名詞であれば,厳密には名詞Bが名詞Aの上位概念であるとし,そうでなければ,厳密には異種概念であるとする.
\end{quote}

\subsection{学習データでの検証}\label{subsec:jikken1}

\ref{subsec:jitugen}のルールに基づいて比喩性判定モデルを実装し,日本経済新聞1994年1月のデータ78組を用いて実験を行った.
これは学習データを用いた実験に相当する.
比喩性判定結果を表\ref{tab:kekka1}に示す.

\begin{table}[htbp]
	\caption{比喩性判定結果(1994年1月データ78組)}
	\begin{center}
	\begin{small}
	\begin{tabular}{|lll|}
		\hline
		\multicolumn{3}{|l|}{{\bf パターン1: 直接対比,異種概念(比喩)}} \\
		\hline
		夢のような約束 & じゅうたんのような芝 & 丘のような山 \\
		悪夢のような日々 & 小春日和のような読後感 & 黒子のような組織 \\
		夢のような話 & 迷路のような路地 & 魔法のような話 \\
		屋根裏のような部屋 & & \\
		\hline
		2食中毒のような危機 & 2京都のような都 & 4ニュージーランドのような自然 \\
		5スパナのような物 & 5タイヤのような物 & 6次のような点(2組) \\
		6次のような質問 & 6右のような事態 & \\
		\hline
		\hline
		\multicolumn{3}{|l|}{{\bf パターン2: 直接対比,上下関係(リテラル)}} \\
		\hline
		中国のような国 & 雑木林のような自然 & 北朝鮮のような国 \\
		\hline
		\hline
		\multicolumn{3}{|l|}{{\bf パターン3: BがAの静的な属性(比喩)}} \\
		\hline
		ピラミッドのような形 & 大理石のような肌合い & オーケストラのような構造 \\
		杉のような感じ & 針のような形 & 雑誌のようなペース \\
		円盤のようなデザイン & 能のような動き & 航空機のような構造 \\
		渡り鳥のようなやり方 & & \\
		\hline
		\hline
		\multicolumn{3}{|l|}{{\bf パターン4: BがAの動的な属性(リテラル)}} \\
		\hline
		ボスニアのような問題 & 米国のような意識 & 東京のような混乱 \\
		\hline
		\hline
		\multicolumn{3}{|l|}{{\bf パターン5: 名詞Bが特定の名詞(ほぼリテラル)}} \\
		\hline
		バールのようなもの(2組) & 刃物のようなもの & 安全弁のようなもの \\
		特効薬のようなもの & 金づちのようなもの & ハンマーのようなもの \\
		印のようなもの & 弾のようなもの & ハサミのようなもの \\
		古典のようなもの & ジーパンのようなもの & シンポジウムのようなもの \\
		\hline
		6次のようなもの & & \\
		\hline
		\hline
		\multicolumn{3}{|l|}{{\bf パターン6: 名詞Aが特定の名詞(リテラル)}} \\
		\hline
		私のような選手 & 君のような人間 & 現在のような環境 \\
		従来のような勢い & 以上のような状況 & 当たり前のようなこと \\
		今のような話 & 今回のような不祥事 & 昔のような姿 \\
		今のような人生 & 私のようなユーザー & 私のような職業 \\
		昔日のような活気 & 前回のような調整 & 当時のような活気 \\
		今回のような決着 & 彼女のような例 & 私のような者 \\
		今回のようなこと & & \\
		\hline
		1ルネサンスのような息吹 & 5結晶のようなもの & \\
		\hline
		\hline
		\multicolumn{3}{|l|}{{\bf 未知語: }} \\
		\hline
		3シャトルのような羽根 & 4オークマのような例 & 5スタンガンのようなもの(3組) \\
		6本書のような書物 & 6同社のような企業 & 6昨夏のようなこと \\
		\hline
	\end{tabular}
	\end{small}
	\end{center}
	\label{tab:kekka1}
\end{table}

先頭に番号がついているデータは,実装モデルのパターン分類結果と我々のパターン分類結果(表\ref{tab:noyouna3})が一致しないものであり,その番号は我々の分類パターンを示す.
判定が一致する割合は,全体で74.4\,\%(58/78),未知語を除けば82.9\,\%(58/70)である.

判定が一致しないデータを中心に,モデルの各ステップについて考察を加える.

\medskip
未知語について
\begin{quote}
名詞Aあるいは名詞Bが日本語語彙大系では未知語のため,パターンを判定することができない表現である.
省略形(``シャトル''…スペースシャトル),固有名詞(``オークマ''),外来語(``スタンガン'')などは未知語となり得る.
また,データの切り出しには形態素解析器茶筅\cite{Chasen-1}を利用したため,接辞がついた表現(``昨夏'',``本書'',``同社'')などは,茶筅の辞書と日本語語彙大系とでずれが生じている.
\end{quote}

\medskip
ステップ1について
\begin{quote}
時詞の中にも顕現性が高いもの(``ルネサンス'')が存在し,比喩となり得る表現がある.
このような表現は比喩語(\ref{sec:hiyugo}章参照)で対処することも考えられる.
名詞Aがサ変動詞型名詞(``結晶'')の場合,これは用言性名詞に含まれ,パターン6(リテラル)と判定されてしまう.
名詞分類の粒度をさらに細かくし,サ変動詞型名詞と形容動詞型名詞(例えば``当たり前'')を別に処理したほうがよいのかもしれない.
``次''と``右''については,日本語語彙大系では一般名詞であるが,形式名詞扱いをすればよいと考える.
\end{quote}

\medskip
ステップ2について
\begin{quote}
``物''については,日本語語彙大系では一般名詞であるが,形式名詞扱いをすればよいと考える.
\end{quote}

\medskip
ステップ3について
\begin{quote}
現行のルールでは,名詞Bが具体(``自然'')であると,名詞Bが名詞Aの属性とは判断しない.
しかし,そのようなケースもないとは言えないので,ルールをさらに精緻化する必要がある.
これも意味属性を束ねる粒度の問題である.
\end{quote}

\medskip
ステップ5について
\begin{quote}
名詞Bと名詞Aの間に上下関係がある(``危機''--``食中毒'',``都''--``京都'')と思われるが,日本語語彙大系の意味情報がそのような上下関係にないものが存在する.
日本語語彙大系の意味情報はある視点を基に構築された階層構造となっている.
対して,本来の上下関係にはさまざまな視点があり,それらをすべて表現するならば意味ネットワーク構造になると考える.
これらの複数の視点を取り込むことは,今後の課題となる.
\end{quote}

\subsection{評価用データでの検証}

\ref{subsec:jikken1}と同じ方法で,日本経済新聞1994年2月のデータ61組を用いて実験を行った.
これはルールの評価用データを用いた実験に相当し,データの抽出には1994年1月データと同様の手法を用いている.
比喩性判定結果を表\ref{tab:kekka2}に示す.

\begin{table}[htbp]
	\caption{比喩性判定結果(1994年2月データ61組)}
	\begin{center}
	\begin{small}
	\begin{tabular}{|lll|}
		\hline
		\multicolumn{3}{|l|}{{\bf パターン1: 直接対比,異種概念(比喩)}} \\
		\hline
		魔球のような歌集 & 悪夢のような事件 & \\
		\hline
		2フォーラムのようなイベント & 2弾痕のような穴 & 2山形のような場所 \\
		2樹海のような森 & 6次のような事実 & 6以下のような行為 \\
		6次のような歌 & & \\
		\hline
		\hline
		\multicolumn{3}{|l|}{{\bf パターン2: 直接対比,上下関係(リテラル)}} \\
		\hline
		北朝鮮のような国 & 寒天のような食べ物 & \\
		\hline
		\hline
		\multicolumn{3}{|l|}{{\bf パターン3: BがAの静的な属性(比喩)}} \\
		\hline
		友達のような関係 & 戦友のような存在 & オーナーのような存在 \\
		陶器のような肌触り & ディーラーのような感覚 & ワインのような味わい \\
		貝殻のような光沢 & 子供のような思い込み & 青年のような気迫 \\
		地獄のような状況 & & \\
		\hline
		1ウソのような変化 & 1うそのような静けさ & 1糸車のような木片 \\
		2定番のような曲 & & \\
		\hline
		\hline
		\multicolumn{3}{|l|}{{\bf パターン4: BがAの動的な属性(リテラル)}} \\
		\hline
		米国のような例 & & \\
		\hline
		\hline
		\multicolumn{3}{|l|}{{\bf パターン5: 名詞Bが特定の名詞(ほぼリテラル)}} \\
		\hline
		ドライバーのようなもの & ガソリンのようなもの & ナイフのようなもの(2組) \\
		同窓会のようなもの & 制服のようなもの & 学校のようなもの \\
		ニックネームのようなもの & タンクのようなところ & イラストのようなもの \\
		\hline
		6以下のようなもの(2組) & & \\
		\hline
		\hline
		\multicolumn{3}{|l|}{{\bf パターン6: 名詞Aが特定の名詞(リテラル)}} \\
		\hline
		今回のようなケース & 昨年のような崩落 & 従来のような姿勢 \\
		昨年のような方式 & 今度のような文書 & こんどのような手法 \\
		今回のような特例 & 今回のような行動 & いまのような黒字 \\
		現在のような形 & 今回のような形式 & 昨年のような凶作 \\
		昔のような勢い & 今回のような事件 & 以上のような仕組み \\
		\hline
		3落雷のような音 & 5合併のようなもの & \\
		\hline
		\hline
		\multicolumn{3}{|l|}{{\bf 未知語: }} \\
		\hline
		1木靴のような空間 & 2北斎のような巨人 & 3レーザー光線のような光 \\
		3おわんのような形 & 3ガイドラインのような意味合い & 6今のような変革期 \\
		\hline
	\end{tabular}
	\end{small}
	\end{center}
	\label{tab:kekka2}
\end{table}

判定が一致する割合は,全体で65.6\,\%(40/61),未知語を除けば72.7\,\%(40/55)である.
1994年1月データ(表\ref{tab:kekka1})と比べると一致する割合が10\,\%ほど落ちている.
これはステップ3の,名詞Bが名詞Aの属性かどうかの判断に失敗するケースが増えていることが主な原因である.
特にステップ3については実装のためのルールにまだ精細が必要と考えられるが,これは意味属性の束の粒度の問題であり,議論・検証するためには粒度に見合うだけの用例データが必要となる.
今回は全体的な処理の流れとして,提案する比喩性判定モデルは実際の文章に現れる``名詞Aのような名詞B''表現に有効であることが確認できた.

\section{比喩語の存在}\label{sec:hiyugo}

\subsection{比喩語の定義}

典型的な比喩表現に用いられる喩詞は,属性が顕著であり,その属性を持つ代表的な事物として一般的に知識共有されているものであることが多い.
逆に,そのような語が名詞Aに使用された場合,``名詞Aのような名詞B''表現は比喩である可能性が極めて高くなる.
ここではそのような語のことを『比喩語』と呼ぶことにする.
比喩語は例えば``夢''や``魔法''などである.
これらは慣用的に喩詞として使用されている語で,典型的比喩表現を収集することによって抽出できると考えている.

\subsection{比喩語の効果}

比喩語の効果を示すために,簡単な実験を行った.

まず比喩語辞書を,比喩表現辞典\cite{Nakamura-2}を利用して作成した.
比喩表現辞典には巻末に索引があり,イメージ(喩詞に相当する)やトピック(被喩詞に相当する)という観点で語が集められている.
今回は,イメージとして複数回現れた語を収集し(1,553語),比喩語辞書とした.

データとして日本経済新聞1994年1月分78組を用い,名詞Aが比喩語であるかどうかを調べた.
その結果をパターン別に表\ref{tab:hiyugo}に示す.

\begin{table}[htbp]
	\caption{比喩語辞書との一致}
	\begin{center}
	\begin{tabular}{|l|r|r|l|}
		\hline
		& データ数 & 比喩語数 & \\
		\hline
		\hline
		パターン1(比喩) & 11 & 5 & ``夢''(2組),``悪夢'',``魔法'',``丘'' \\
		パターン2(リテラル) & 5 & 0 & \\
		パターン3(比喩) & 11 & 3 & ``大理石'',``ピラミッド'',``針'' \\
		パターン4(リテラル) & 5 & 0 & \\
		パターン5(ほぼリテラル) & 19 & 2 & ``刃物'',``ハサミ'' \\
		パターン6(リテラル) & 27 & 1 & ``彼女'' \\
		\hline
		\hline
		計 & 78 & 11 & \\
		\hline
	\end{tabular}
	\end{center}
	\label{tab:hiyugo}
\end{table}

比喩表現(パターン1,3)に比喩語の含まれる割合は36.4\,\%(8/22),リテラル(パターン2,4,6)は2.7\,\%(1/37)と,十分に利用価値のある値となっている.
このことから比喩語と呼ばれる,比喩性を決定づける語の存在は明らかであり,それらは比喩性判定に有用であると考えられる.
パターン6の``彼女''には代名詞としての用法と``恋人''という意味での用法があり,データは前者,比喩語辞書は後者でくいちがいが見られる.
パターン5の``刃物のようなもの'',``ハサミのようなもの''は通常はリテラルと考えられるが,文脈によっては比喩となる可能性もある.

\section{おわりに}

本研究は,日本語文章の比喩表現,その中でも直喩・隠喩的な比喩について,その認識・抽出を目的としている.
比喩表現に関する研究は,比喩解釈のものが多く,確率理論に基づいた計算モデルがいくつか報告されている\cite{Iwayama-1,Pattabhiraman-1}.
比喩解釈については対比する喩詞と被喩詞が既知であることが前提となり,比喩認識はその喩詞と被喩詞とを正確に抽出できなければならない.
直喩の代表的な表現形式である``名詞Aのような名詞B''を対象にしても,常に名詞Aと名詞Bが対比されるわけではない.

本論文では,``名詞Aのような名詞B''表現について,意味情報を用いたパターン分類によって比喩性を判定し,喩詞と被喩詞とを正確に抽出できるモデルを提案した.
このモデルを日本語語彙大系を利用して実装したところ,得られたパターン分類結果と人間のそれとが一致する割合は,学習データについては未知語を除けば82.9\,\%(含めれば74.4\,\%),評価用データについては72.7\,\%(同65.6\,\%)であった.
実装ルールをさらに精緻化するために,用例データを増やし細かく検証する必要はあるが,比喩性判定モデルの処理の流れは実際の文章中の比喩表現認識,喩詞・被喩詞の抽出に有効であることを示すことができた.
また,比喩語という比喩性を決定づける語についてもその効果を示すことができた.

本手法は明らかに,前後の文脈を考慮していないので,文脈に依存した比喩性の判断はできない.
「刃物のようなもの」を例にすれば,「凶器は刃物のようなものである」ではリテラル(例示)であるのに対し,「言葉は刃物のようなものである」では比喩表現となる.
``もの''が``凶器''を指すのか``言葉''を指すのか同定する手法を取り入れることによって,これらの表現を我々のモデルでも区別して判断することが可能と考えられる.
このような文脈処理の問題は,比喩性判定モデルの拡張として今後の課題となる.

自然言語において比喩表現は例外ではなくむしろ本質的なものであり\cite{Ikehara-2},これらを機械的に扱う研究を進めることで,自然言語処理研究全体に通じる重要な知見を得ることができると確信している.

\acknowledgment

鈴鹿工業高等専門学校電子情報工学科田添研究室と三重大学情報工学科人工知能研究室の学生のみなさんには,ディスカッションでの有益な意見交換,検証データの整理などで非常なお世話になった.
ここであらためて感謝の意を表したい.

\bibliographystyle{jnlpbbl}
\bibliography{jpaper}

\begin{biography}
\biotitle{略歴}
\bioauthor{田添 丈博}{
1991年三重大学工学部電子工学科卒業.
1993年同大学院工学研究科電子工学専攻修士課程修了.
同年,鈴鹿工業高等専門学校電子情報工学科助手,現在に至る.
また,2000年より三重大学大学院工学研究科システム工学専攻博士後期課程在学中.
自然言語処理の研究に従事.
言語処理学会,情報処理学会,人工知能学会各会員.}
\bioauthor{椎野 努}{
1964年名古屋大学工学部電気工学科卒業.
同年,沖電気工業(株)入社.
マイクロ波通信,データ通信,基本ソフトウェア,ソフトウェアCAD,各種エキスパートシステム,機械翻訳システム等の研究開発に従事.
1990年三重大学工学部情報工学科教授.
2002年愛知工業大学工学部情報通信工学科教授.
工学博士.
自然言語処理,画像処理,音楽情報処理等に興味をもつ.
情報処理学会,人工知能学会,日本心理学会,IEEE各会員.}
\bioauthor{桝井 文人}{
1990年岡山大学理学部地学科卒業.
同年,沖電気工業(株)入社.
2000年三重大学工学部情報工学科助手.
質問応答システム,情報抽出の研究に従事.
言語処理学会,電子情報通信学会,人工知能学会各会員.}
\bioauthor{河合 敦夫}{
1980年名古屋大学理学部卒業.
1985年同大学院工学研究科情報工学専攻博士課程修了.
工学博士.
同年,日本電信電話(株)入社.
同社情報通信網研究所主任研究員を経て,1992年より三重大学工学部情報工学科助教授.
文書添削,検索分類等の自然言語処理及び色彩画像処理の研究開発に従事.
言語処理学会,情報処理学会,電子情報通信学会各会員.}

\bioreceived{受付}
\biorevised{再受付}
\bioaccepted{採録}

\end{biography}

\end{document}
