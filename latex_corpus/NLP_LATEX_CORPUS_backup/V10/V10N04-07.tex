\documentstyle[epsf,nlpbbl]{jnlp_e_b5}

\setcounter{secnumdepth}{2}
\nonstopmode

\setcounter{巻数}{10}
\setcounter{号数}{4}
\setcounter{年}{2003}
\setcounter{月}{7}
\setcounter{page}{125}


\受付{November}{1}{2002}
\再受付{February}{10}{2003}
\採録{April}{10}{2003}

\title{}
\author{}
\jkeywords{}

\etitle{The Dynamics of Morphemes \\in Japanese Terminology}
\eauthor{Kyo Kageura\affiref{NII}}
 
\headauthor{Kageura}
\headtitle{Dynamics of Morphemes in Japanese Terminology}
 
\affilabel{NII}
          {Human and Social Informatics Division, National Institute of Informatics}
          {Human and Social Informatics Division, National Institute of Informatics}

\eabstract{
This paper quantitatively analyses the role of morphemes with respect to their
types of origin. Static quantitative analysis of a given data set is
not sufficient for this aim, as language data in general and terminological
data in particular have the specific characteristic of being ``incomplete" in the
sense that many unseen elements are expected in the theoretical population.
Thus, the quantitative structure of morphemes in terminology should be analysed
dynamically, by observing the growth pattern of morphemes. In order to allow for
that, we use binomial interpolation and extrapolation. Results of analyses
of the terminologies of six different domains follow, revealing interesting
characteristics of the role of morphemes of different types of origin
that do not manifest themselves through static quantitative analysis.
}
\jabstract{
本論文では、日本語専門用語を構成する語基の役割を、語種(外来語と漢語・和語)
の観点から分析する。専門用語データにおいては、そして言語データでは、一般に、
常に標本中には現れていない要素が存在すると仮定せざるを得ないため、与えられ
たデータを静的に記述するだけでは十分ではない。本論文では、そこで、データに
基づき、標本量に応じた語基の変化を動的に追うために有用な、二項補間・補外の
枠組みを用いて、専門用語の構成における語種ごとの語基の役割を分析する。実際
の分析では、物理学、化学、農学、植物学、計算機科学、心理学の6分野の専門語
彙データを対象とする。分析の結果、全ての分野で、語彙量が増大すると外来語の
異なり比率が増大すること、農学を除いては、将来的に外来語の比率が漢語・和語
の比率より大きくなること、与えられたデータにおいては例外的に見えた計算機科
学がその点では農学を除く他の分野と同じ傾向を単に極端なかたちで実現している
に過ぎないこと、一方で、例外的なのは、農学分野における専門語彙構成であるこ
とが示された。
}

\ekeywords{Terminology, Morphemes, Binomial Interpolation, Binomial
Extrapolation, Types of Origin}
\jkeywords{ターミノロジー,専門用語,二項補間,二項補外,語種}

\begin{document}


\maketitle

\section{Introduction}

\thispagestyle{empty}
Modelling the dynamic nature of vocabulary is important, not only 
theoretically but also practically. It is important theoretically because there 
has been little concrete work on the dynamics that underly the 
structure of vocabulary at an idiosynchronic slice of language although 
speculatively these dynamics are widely held to be important. It is important
practically because 
it can give a basic perspective from which the problem of so-called 
``lexical bottleneck" in many NLP-related applications can be diagnosed.
This task is especially important to the study of technical terminologies, due to 
their rapid growth in many different domains. This, however, is an area 
of research that has thus far gone unexplored.

Against this background, this paper analyses the role of morphemes 
in Japanese terminology --- with respect to their type of origin --- and 
clarifies the basic structural tendencies of dynamics of terminology, 
using a probabilistic method. The present study is both descriptive and 
theoretical --- descriptive because it gives a concrete 
description of the growth patterns of morphemes; theoretical because 
it is concerned with examining the underlying structural dynamics 
of terminologies in such a way that they be properly visualised and their 
basic characteristics explained. Note here that the unique position 
of lexicology requires any theoretical research in the field to be 
concretely anchored to the existing vocabulary \cite{maeda89}, which 
in turn requires the study to be descriptive \cite{kage02}. Thus, the
descriptive content, together with the basic perspective from which
the concrete descriptions are made, constitutes the theoretical 
contribution of the present study to the field of terminology.

We focus on the patterns of morphemes with regard to their type 
of origin because, in Japanese vocabulary in general and in terminology 
in particular, the roles of morphemes are said to differ according 
to origin type, i.e.\ whether they are borrowed mainly from Western 
languages ({\it gairaigo} morphemes), are of Chinese origin ({\it kango} 
morphemes) or are original Japanese morphemes ({\it wago} morphemes). Many 
studies have addressed the nature of these origin types 
\cite{saiga57,nlri58,miyaji82,nomura84} and have argued qualitatively 
that there are differences in nature among morphemes of each type.
Some have carried out quantitative analyses of Japanese 
terminologies with respect to the types of origin of the constituent 
elements or morphemes of terms \cite{ishii84,ishii87}.

Note that the quantitative analyses carried out so far are mostly static; 
they describe the quantitative characteristics of a given set of data. Static 
analyses of a given data, however, are in general not sufficient when 
dealing with language, because there are events or items which may 
not appear in the sample but do exist in the theoretical population. In 
analysing morphemes in terminology, therefore, it is necessary to use 
a method by which the nature of morphemes --- including those that do not 
appear in the data --- are properly accounted for. This is not only
technically important but also theoretically essential as it assigns 
the model a moment of dynamics and thus a basis for expectation.
Only through revealing the structural characteristics not explicit
in the data itself can the nature of vocabulary can be fully observed.

In the following, the nature of the terminological data used in this study
is first briefly described. This is followed by the introduction of a
theoretical model, that can properly treat a sample with unseen events,
together with the method of binomial interpolation and extrapolation. This gives
the basic framework for modelling the structural dynamics of morphemes
in terminology. From there descriptions of the dynamic quantitative nature of 
morphemes of different types of origin in six sets of terminologies will
ensue. This last section constitutes the central part of the present paper 
and its main contribution to the study of lexicology.


\section{The Terminological Data}

As the basic data for the analysis, we use lists of term types, as opposed
to term tokens in texts, and analyse the quantitative nature of constituent
elements or morphemes within the list of term types. There are two
reasons for this. Firstly, as terms are basically created by lexical
formation, the quantitative nature of morphemes in terminology is
independent of the token frequency of terms \cite{sager90,kage02}.
Recent psycholinguistic studies also support this claim \cite{baayen97,schreuder97}.
Secondly, as the majority of terms are complex \cite{nomura89} and
new terms are constantly formed by compounding, the quantitative nature
of morphemes in the construction of terminologies is a key element
for the modelling of terminological structure.

With the correspondences between text and terminology, sentences and
terms, and words and morphemes, the present work can be regarded as
parallel to the quantitative study of words in texts
\cite{zipf35,yule44,mandelbrot53,simon55,carrol67,sichel75}.
Terms in the field of quantitative linguistics, such as ``type", ``token",
etc., shall be used in this context.

Chose for the present study are the terminological data of the following six 
different domains: agriculture (AGR) \cite{agrterm}, botany
(BOT) \cite{botterm}, chemistry (CHM) \cite{chmterm}, computer science
(COM) \cite{aiso}, physics (PHY) \cite{phyterm} and psychology (PSY)
\cite{psyterm}. They were chosen, within the limited availability
of terminological data from roughly the same period, to cover both
``harder" and ``softer" scientific and technological domains.

Within these sets of data, the terms are identified on the basis of their
orthography and type of origin; polysemous morphemes are not 
semantically distinguished, nor are inflections stemmed (though there 
are not any in the data). The terms
are segmented into morphemes according to the criteria given in \cite{nomura89}.
Briefly, the method first defines a minimal element, the smallest unit that 
bears meaning in current Japanese. Then, according to the origin of
linguistic elements ({\it wago}, {\it kango} and {\it gairaigo}), the morphemes are defined
as follows: (i) for {\it wago} and {\it gairaigo}, a minimal element constitutes a
morpheme, e.g.\ 手 (`te': hand) and コンピュータ (computer); (ii) for
{\it kango}, a first-order combination of two minimal elements constitutes
a morpheme, while a minimal {\it kango} element attached to a morpheme is
also treated as a morpheme (e.g.\ 図書館員 has the structure [[図 書] 館] 員],
so 図書, 館 and 員 are identified as morphemes); and (iii) for
{\it kango} and {\it wago} mixture, a first-order combination of minimal
elements is identified as a morpheme, e.g.\ 係員.

In the present analysis, types of origin are classified into two, i.e.\ {\it gairaigo} 
morphemes on the one hand and {\it kango} and {\it wago} morphemes on the other
(henceforth, we will call the former ``borrowed" morphemes and the latter
``native" morphemes). {\it Kango} and {\it wago} morphemes are grouped together 
because: (i) the majority are {\it kango} and mixed morphemes (which behave roughly 
equivalent to {\it kango} and mostly written in Chinese characters), and the 
number of pure {\it wago} morphemes is very small, and (ii) we are here concerned 
more with the status of {\it gairaigo} morphemes in the recent development of 
terminologies (cf.\ \cite{ishii87}).


\begin{table*}[tb]
\caption{Basic quantities of the terminology samples of the six domains}
\label{tab:quandata}
\hbox to\hsize{\hfil
{\begin{tabular}{lrrrrrrrr} \hline
Domain & \multicolumn{1}{c}{T} & \multicolumn{2}{c}{N (\%)} & \multicolumn{2}{c}{V(N) (\%)} & \multicolumn{1}{c}{N/T} & \multicolumn{1}{c}{N/V(N)} & \multicolumn{1}{c}{$C_L$} \\ \hline \hline
AGR All    & 15067 &  29142 & (100.00\,\%) & 9093 & (100.00\,\%) & 1.93 & 3.20 & 0.256 \\
 Borrowed &       &  2610 & (8.96\,\%) & 1513 & (16.64\,\%) &      & 1.73 & 0.300 \\
 Native   &       &  26532 & (91.04\,\%) & 7580 & (83.36\,\%) &      & 3.50 & 0.247 \\ \hline
BOT All    & 10956 &  22605 & (100.00\,\%) & 5348 & (100.00\,\%) & 2.06 & 4.23 & 0.224 \\
 Borrowed &       &  3072 & (13.59\,\%) & 1678 & (31.38\,\%) &      & 1.83 & 0.283 \\
 Native   &       &  19533 & (86.41\,\%) & 3670 & (68.62\,\%) &      & 5.32 & 0.197 \\ \hline
CHM All    & 12074 &  23577 & (100.00\,\%) & 6400 & (100.00\,\%) & 1.95 & 3.68 & 0.246 \\
 Borrowed &       &  5998 & (25.44\,\%) & 2841 & (38.77\,\%) &      & 2.11 & 0.289 \\
 Native   &       &  17579 & (74.56\,\%) & 3559 & (61.23\,\%) &      & 4.94 & 0.212 \\ \hline
COM All    & 14983 &  36640 & (100.00\,\%) & 5176 & (100.00\,\%) & 2.45 & 7.08 & 0.211 \\
 Borrowed &       &  14696 & (40.11\,\%) & 2809 & (54.27\,\%) &      & 5.23 & 0.242 \\
 Native   &       &  21944 & (59.89\,\%) & 2367 & (45.73\,\%) &      & 9.27 & 0.174 \\ \hline
PHY All    & 10635 &  25095 & (100.00\,\%) & 4745 & (100.00\,\%) & 2.36 & 5.29 & 0.228 \\
 Borrowed &       &  5048 & (20.12\,\%) & 2081 & (43.86\,\%) &      & 2.43 & 0.269 \\
 Native   &       &  20047 & (79.88\,\%) & 2664 & (56.14\,\%) &      & 7.53 & 0.197 \\ \hline
PSY All    &  6272 &  14314 & (100.00\,\%) & 3594 & (100.00\,\%) & 2.28 & 3.98 & 0.235 \\
 Borrowed &       &   1541 & (10.77\,\%) &  995 & (27.69\,\%) &      & 1.55 & 0.309 \\
 Native   &       &  12773 & (89.23\,\%) & 2599 & (72.31\,\%) &      & 4.91 & 0.207 \\ \hline
\end{tabular}\hfil}
}
\end{table*}

Table \ref{tab:quandata} gives the basic quantitative data of the
six terminological data. $T$, $N$ and $V(N)$ indicate the number
of terms, the number of running morphemes (tokens), and the number of
different morphemes (types), respectively. $N/T$ indicates the average
length of a term in terms of its constituent morphemes, and $N/V(N)$
represents the average frequency of a morpheme. The meaning of $C_L$
will be explained shortly. Table 2 shows some examples of morphemes with
their frequencies. It includes the top five morphemes and five randomly 
selected five hapax for each domain (``b" and ``n" indicates ``borrowed" and ``native", 
respectively).

\begin{table*}[tb]
\caption{Some examples of morphemes}
\label{tab:bakabon}
\hbox to\hsize{\hfil
{\begin{tabular}{lp{125mm}} \hline
AGR & 性 (497, n), 機 (306, n), 体 (195, n), 土壌 (192, n), 法 (183, n),
エポキシ (1, b), インフルエンザ (1, b), こうじ (1, n), 雨害 (1, n), CEM (1, n) \\
BOT &  性 (467, n), 体 (431, n), 細胞 (337, n), 植物 (269, n), 酸 (240, n), 
アミラーゼ (1, b), アセト (1, b), つぼ (1, n), 果床 (1, n), 安全 (1, n) \\
CHM & 酸 (424, n), 性 (308, n), 剤 (282, n), 化 (251, n), 油 (188, n),
シャシ (1, b), 骨材 (1, b), りん (1, n), 剥離 (1, n), 行程 (1, n)\\
COM & システム (504, b), データ (499, b), 装置 (402, n), 制御 (368, n), の (339, n),
VLSI (1, b), BOT (1, b), スタディ (1, b), 思考 (1, n), 深度 (1, n) \\
PHY &  の (594, n), 性 (246, n), 線 (236, n), 計 (216, n), 器 (210, n), 
原色 (1, n), 誘体 (1, n), 標線 (1, n), アロイ (1, b), ストレージ (1, b)\\
PSY &  的 (491, n), の (388, n), 性 (316, n), 法 (217, n), 学 (170, n), 
付加 (1, n), 没 (1, n), 分節 (1, n), ベンダー (1, n), ホヴランド (1, n)\\ \hline
\end{tabular}\hfil}
}
\end{table*}

It is observed that, with the exception of the number of types in computer
science (COM), both the type and token numbers of borrowed morphemes are smaller
than those of native morphemes. The average frequency of the borrowed morphemes
is smaller than that of the native morphemes in all of the data sets. 
From the terminological point of view, this tendency
could be interpreted as follows: (i) The native morphemes are used to represent
core conceptual elements which appear repeatedly in terminology (in terms of average
frequency and token number), and (ii) although in terms of accommodating
new concepts, the borrowed morphemes are used relatively more frequently,
with the exception of computer science, the native morphemes still take a
major role.

However, as will be shown, this observation is too simplistic, if not 
incorrect. Technically, the problem is that, in language data in general,
most statistical measures change systematically according to sample size
\cite{tweedie98}. This makes it difficult to draw a reliable conclusion using
summary statistics based on a particular sample or a snapshot of
the target phenomena. This is related to the long-recognised fact that there
are always events that do not appear in language data but do in fact exist
\cite{yule44,herdan60,mizu83}. As we can in no way claim that our
terminological data constitutes the population of terminology for each
domain, even synchronically, we have to expect that there are morphemes that
do not appear in our data\footnote{Theoretically, whether the interpretative
framework of the present study is anchored to the synchronic state of language
or to the diachronic nature of language requires in-depth articulation. Here
we simply assume that, as far as we are dealing with the internal structure
of terminologies, this distinction is irrelevant. For further discussion, see
\cite{kage00,kage02}.}. In terms of terminology theory, this statistical
peculiarity of the data can be interpreted as a reflection of the dynamics
of terminology, in the sense that the potentiality of terminology is manifested
in the structure of a given terminological sample.


\section{Theoretical Model and the Status of Data}


We introduce here a dynamic statistical model which can treat the
terminological data properly. In the process, we also confirm that the
terminological data, like language data in general, anticipates 
unseen events, as we informally mentioned in the previous section.


\subsection{Binomial/Poisson Model}
 
The model we introduce here regards a terminology as a bag of morphemes,
without any inter-morpheme dependencies. As a distributional model of
morphemes in terminology, it offers a good and principled approximation
to the behaviour of morphemes in terminology,
for two reasons. Firstly, we can ignore the qualitative dependency of
morphemes within individual complex terms in modelling the distribution
of morphemes in terminology \cite{kage98b}. Secondly, the order of terms
in the data is basically arbitrary, unlike the order of words or sentences
in texts \cite{baayen96a,baayen96b}. We can thus safely apply the binomial
model --- which assumes no inter-event dependency and sees the data as a bag
of events --- to the distribution of morphemes in terminology.

Assume that there are $S$ different morphemes, i.e.\ $w_i$, $i = 1, 2, ... S$,
in the population of a terminology, with a population probability $p_i$
associated with each. Based on the binomial assumption, which in
turn can be approximated by the Poisson model, the expected number of different
morphemes, $E[V(N)]$, and the expected number of morphemes that appear
1, 2, 3, ... $m$ times, $E[V(m,N)]$, in a given sample of size $N$, can be
expressed as follows \cite{baayen01}:

\vspace*{-\baselineskip}
 
\begin{eqnarray}
E[V(N)] & = & S - \sum_{i=1}^S(1-p_i)^N \nonumber \\
	& = & \sum_{i=1}^S(1-e^{-Np_i}). \nonumber
\end{eqnarray}

\vspace*{-\baselineskip}
 
\begin{eqnarray}
E[V(m,N)] & = & \sum_{i=1}^S \left ( {N \atop m} \right ) p_i^m(1-p_i)^{N-m} \\
    & = &  \sum_{i=1}^S (Np_i)^me^{-Np_i}/m!. \nonumber
\end{eqnarray}

\noindent
This will be the starting point of the binomial interpolation and
extrapolation which will be introduced shortly to trace the structural
dynamics of morphemes.

Assuming this model, incidentally, we can check the statistical status of
the terminological data. As discussed, it is widely held that there
are always events that do not appear in a language sample. But do they really
exist, for example, in the data in Table 1? Should we really take into account these
morphemes that do not appear in the data? There is a convenient test to
explore this, called the coefficient of loss \cite{chita93}. The coefficient of
loss calculates the ratio of the number of events that are lost by estimating
the number of events in the original sample size, using the sample relative
frequencies to estimate the population probabilities, based on
the binomial model. Formally, the coefficient of loss ($C_L$) is defined
as follows:

\vspace*{-\baselineskip}
 
\begin{eqnarray}
C_L & = & (V(N) - \hat{E}[V(N)]) / V(N) \nonumber \\
    & = & \frac{\sum_{m \geq 1}V(m,N)(1 - p(i_{[f(i,N) = m]},N))^N}{V(N)}
    \nonumber
\end{eqnarray}

\noindent
where:

\begin{description}
\item[$f(i,N)$] : frequency of a morpheme $w_i$ in a sample of $N$.
\item[$p(i,N)$] = $f(i,N)/N$ : sample relative frequency.
\item[$m$] : frequency class or number of occurrence.
\item[$V(m,N)$] : the number of morpheme types occurring $m$ times (spectrum
elements) in a sample of $N$.
\end{description}

The column $C_L$ in Table \ref{tab:quandata} indicates the values of the coefficient
of loss for each data. The number of morpheme types is underestimated by
around 20 per cent, which means that the sample relative frequency does not give a
reliable estimate of the population probability. The data belong to what is
called the LNRE (Large Number of Rare Events) zone of the sample range
\cite{chita93,khma87}, where the population events (morpheme types) are far
from being exhausted in the sample. In this situation, not only the sample
relative frequencies but also almost all of the statistical measures as well
as the parameters of the distribution models change systematically
according to the sample size \cite{baayen01,tweedie98}. It is to overcome
this problem that binomial interpolation and extrapolation is
required.


\subsection{Binomial Interpolation and Extrapolation}

To overcome the problem of sample-size dependency among statistical
measures, Good and Toulmin \cite{good56} propose a method of interpolating
and extrapolating the sample and calculating the number of events as well
as the spectrum elements for a (theoretically) arbitrary sample size.

The number of events and the number of the spectrum elements of a sample size $N$,
conditional on the original sample of size $N_0$, can be expressed by the
following formula:
 
\begin{equation}
E[V(\lambda N)] = V(N) - \sum_{k = 0}^\infty (-1)^k (\lambda -1)^k E[V(k,N)]
\end{equation}

\begin{equation}
E[V(m,\lambda N)] = \lambda^m \sum_{k = 0}^\infty (-1)^k \left (
{{m+k} \atop m} \right ) (\lambda -1)^k E[V(m+k,N)]
\end{equation}

\noindent
Appendix A gives the derivation of (2) and (3) from (1), originally provided
in \cite{good56}.

Binomial interpolation and extrapolation provides the means of tracing
the {\it developmental profile} of the growth of morphemes as well as the
growth rate of morphemes (which will be introduced in 4.2). By employing this 
methods, we can make explicit the quantitative nature of morphemes implicit in a
given data. In other words, through binomial interpolation and extrapolation,
we can observe how the ratio between borrowed and native morphemes was,
is, and will be when the data is changed in size, as opposed to simply how it
is in a given set of data. Although the actual value diverges around $N = 2N_0$,
the formula is sufficient for observing the basic dynamic characteristics of
morphemes in a terminology within a realistic range of the terminological
phenomena. Revealing the developmental profiles, therefore, explicates
the structural characteristics of terminology.

Recall, incidentally, that, as discussed above, the randomness assumption
behind the binomial/Poisson model holds for terminological data in
general. This is confirmed to be valid in the terminological data used
in the present study, as observed in Figure \ref{fig:intext}, which shows
the developmental profiles of $E[V(N)]$ and $E[V(1,N)]$ obtained through binomial
interpolation and extrapolation for up to twice the original sample size (lines),
as well as the corresponding values obtained by 5,000 term-level (as opposed
to morpheme-level) random permutations up to the original sample size for 
20 equally-spaced intervals (dots). The
results of binomial interpolation and extrapolation based on the randomness
assumption of the distribution of morphemes are almost identical to the
empirical results obtained through the random permutation of terms\footnote{The
$z$-score is available for up to half the original sample size, using the
following formula \cite{baayen01}:
\begin{displaymath}
\frac{|V(N) - E[V(N)]|}{\sqrt{V(2N)-V(N)}}
\end{displaymath}
if we allow ourselves to estimate the variance of $V(N)$ by $V(2N) - V(N)$.
The result showed no significant difference between the two.}.


\section{The Growth of Morphemes and the Roles of Morphemes}

As the binomial model of interpolation and extrapolation provides a valid
estimation of the morphemes in a given terminology up to around twice the original
sample size, we can now observe the developmental profiles of the behaviour
of morphemes, in terms of their type of origin, in accordance with the
changes in the size of the data within and beyond the original sample size. 
This allows us to form general expectations of how the morphemes of different 
types of origin should behave.


\subsection{Patterns of the Growth of Morphemes}

Figure \ref{fig:intext} charts the developmental profiles of morphemes 
in each of the six terminological domains, according to their types of 
origin. We can observe that, in all the domains, the growth
curves of the borrowed morphemes are more ``straight" than the growth
curves of the native morphemes. The developmental curves of the native
morphemes tend to flatten out more quickly than the curves of the
borrowed ones.

We can expect that, although the number of different borrowed morphemes
is smaller in all but one domain (computer science) at the given as well as
at twice the given sample size, the relation may well be reversed
when the sample is further increased. In the domains of chemistry and physics,
borrowed morpheme types are expected to outnumber native morphemes fairly soon. 
This general estimation is informally reinforced by the fact that, in computer 
science, where the number
of different borrowed morphemes is greater at the original sample size,
there is a greater number of native morpheme types at the beginning of
the sampling range, i.e.\ $N < 14,000$.

\begin{figure*}[t]
\begin{center}
\leavevmode
\epsfile{file=fig1.eps,scale=0.7}
\end{center}
\caption{Growth of morphemes based on binomial interpolation and extrapolation}
\label{fig:intext}
\end{figure*}

To be rigorous, the actual ratio of borrowed
to native morphemes should be observed. This is shown in Figure \ref{fig:ratio1}.
A clear general pattern, irrespective of domain, is recognised in Figure
\ref{fig:ratio1}, i.e.\ the more a terminology grows, the more it depends, in terms
of type, on borrowed morphemes. In this sense, what is thought to be an exception
in terms of static quantitative measures, i.e. the status of borrowed morphemes in
computer science, follows a general pattern, the only difference being the
degree of actual manifestation of the general pattern vis-\`a-vis the size of
the terminology. This general pattern is also in accordance with the situation
concerning the diachronic development of the Japanese vocabulary in general.

\begin{figure*}[t]
\begin{center}
\leavevmode
\epsfile{file=fig2.eps,scale=0.7}
\end{center}
\caption{Transitions in the ratio of borrowed to native morphemes}
\label{fig:ratio1}
\end{figure*}

The actual ratio of borrowed to native morpheme types differs from domain
to domain, revealing the characteristics of each domain within the general
pattern of borrowed and native morphemes. In computer science, as mentioned
earlier, the number
of different borrowed morphemes is already greater than that of native
morphemes within the original sample size. In chemistry and physics, 
it is likely that the ratio will become greater than 1 within a realistic 
data size of the terminologies, say, $N=3N_0$; while in botany and
psychology, it is possible that the ratio will become bigger than 1 in due course,
but at what data size this will occur is not clear. In agriculture, 
the opposite conclusion seems to
be more reasonable. Thus it is the terminology of agriculture, not of
computer science, that is exceptional in this respect. To confirm this
informal and intuitive discussion more rigidly, it is useful to observe the
growth rate of morphemes, to which we now turn.

\subsection{Patterns of the Growth Rate}

The changing values of $E[V(N)]$ for changes in $N$ provides the growth curve of
the morpheme types, as illustrated in Figure \ref{fig:intext}. The next question
to be asked is how we can obtain the {\it growth rate} of the morphemes at
each point of observation. Interestingly, assuming the binomial/Poisson model,
the growth rate of the morpheme types can be obtained by using the number of
hapax legomena, or the morphemes that appear only once \cite{baayen91}.
Mathematically, the growth rate ${\cal P}(N)$ is defined as follows:

\begin{displaymath}
{\cal P}(N) = \frac{E[(V(1,N)]}{N}
\end{displaymath}

\noindent
A derivation of this formula from the binomial/Poisson model is explained in
\cite{baayen01}, and presented in Appendix B.

This index demonstrates the probability that new morpheme types will be encountered
when the sample size is increased. Incidentally, this equals to the
probability mass of unseen types obtained by well-known Good-Turing
estimates \cite{good53}.

\begin{figure*}[t]
\begin{center}
\epsfile{file=fig3.eps,scale=0.7}
\end{center}
\caption{Transitions of the growth rate of borrowed and native morphemes}
\label{fig:grate}
\end{figure*}

Figure \ref{fig:grate} shows the transition profile of the growth rate of
the borrowed and native morphemes in the terminologies of the six domains, in
accordance with increases in the sample size to up to twice the original size.
The transition profiles of borrowed and native morphemes take different forms, 
with the same basic pattern observed in all six domains, i.e.\ at the 
beginning of the sample range, the
growth rate of native morphemes is much higher than that of borrowed
morphemes\footnote{At the outset, i.e. $N \simeq 0$, the growth
rate of morphemes of each origin type is equal to the ratio of $N$
of each origin type to the total number of running morphemes.}, but
the former quickly decreases while the latter decreases very slowly as
the sample is increased.

From the terminological point of view, this difference can be interpreted
as follows: Native morphemes are first used to constitute the {\it core}
set of morphemes in a terminology, but as the terminology grows, it begins to
depend more and more on borrowed morphemes, in order to accommodate 
new concepts.

Within this general tendency, the actual values of the growth rates in
the six domains show the concrete nature of the terminology of each
domain\footnote{In the discussion here, we use both the absolute size
of the data and the sample scale relative to the original sample size
of each domain, though the emphasis is on the latter.}.
In computer science, the growth rate is already reversed around $N=3500$.
In chemistry and physics, the growth rate is reversed around $N=10,000$
to $15,000$, well within the original sample size.
Botany and psychology show a similar pattern, and the growth rate is
reversed or expected to be reversed around just $N=2N_0$.

Focusing on the earlier stage of the sample range, the terminology
of computer science is exceptional in its high dependency on borrowed
morphemes. If we interpret the beginning stage of the sample size to be the
stage at which core morphemes are introduced and consolidated, then computer
science can be characterised by its heavy reliance on borrowed
morphemes in the role of core morphemes.

On the other hand, when the size of a terminology becomes bigger, 
newly introduced morphemes are expected to be used to add new concepts
to the existing structure. As $N$ approaches $\infty$,
the ratio of borrowed to native morpheme types converges to the ratio
of their growth rates. From Figure \ref{fig:grate}, we can expect that,
as $N \rightarrow \infty$, there will be a greater number of different borrowed
morphemes than the number of native morphemes in all of the domains but
agriculture. In that sense, the informal observation given in 4.1 based
on Figure \ref{fig:ratio1} has been rigidly confirmed.

This leaves us with one domain, i.e.\ agriculture. In agriculture, it
is not clear whether the growth rate of the borrowed and native
morphemes will be reversed at all. In this sense, among the six different
domains we observe here, it is agriculture that is exceptional in the
use of morphemes of different types of origin in the construction of
terminology.


\subsection{Summary of the Observations}

Summarising the observations above, we can conclude, from the
developmental profiles of the morphemes and the transitions in their
growth rates, that native morphemes tend to be used to constitute
the {\it core} of a terminology. 
Because the first and main role of the native morphemes is to contribute
to expressing the core conceptual elements, it is natural that the
native morphemes are used more frequently than borrowed morphemes.
As the terminology grows, on the other hand, the use of borrowed
morphemes grows, in order to incorporate new concepts. As the new
concepts are incorporated into the existing terminological structure,
the core of which has already become stable \cite{sager90}, the average use
of a borrowed morpheme remains relatively low, as is manifested
by the low average frequency of the borrowed morphemes. This general
tendency can be observed irrespective of domain.

Turning our eyes to the differences among the domains, we can observe the
following:
\begin{enumerate}
\setlength{\itemsep}{0mm}
\item[(1)] From the point of view of the tendency of borrowed
morphemes to be used to incorporate new concepts, the terminology
of agriculture is an exception, in that the native morphemes will continue
to be used more often for incorporating new morphemes than borrowed morphemes,
even if the size of terminology becomes very large. This tendency is
expected to continue possibly for $N \rightarrow \infty$. All the other
five domains come to use, or will come to use, more borrowed morphemes than
native morphemes for incorporating new morphemes. Among these domains,
chemistry and physics show similar tendencies. Botany and psychology are
also similar.
\item[(2)] From the point of view of the basic tendency of the native
morphemes to be used to constitute the core set of morphemes in a terminology,
it is computer science that is exceptional, in light of the high presence of
borrowed morphemes from the beginning of the sample range, i.e. in the
core morpheme set.
\end{enumerate}


\section{Conclusions}

We have analysed the role of native and borrowed morphemes in the
construction of the terminologies of six different domains, tracing
the developmental patterns of the growth and the growth rates of
morphemes. A few general as well as domain-dependent patterns in
the use of morphemes were clarified. In the process, we introduced
a theoretical framework based on the binomial/Poisson assumption,
which was proved to provide a very useful and powerful method of analysing
the dynamic patterns of morphological growth in terminology.

The work reported here should be extended further, at least in
three aspects. Firstly, we should extend the observation
to the terminologies of other domains. This is not only in itself
crucial as a descriptive quantitative study of terminology but
also important for uncovering the general tendencies of terminological
structure across different domains, which in turn would lead to
the characterisation of technical terminology as a whole.

Secondly, in order to fully explore the morphological structure in terminology,
it is important to obtain reliable extrapolated values beyond $N < 2N_0$.
Chitashvili and Baayen \cite{chita93,baayen01} formulate the method
of incorporating parametric models of word frequency distributions
\cite{zipf35,yule44,simon55,carrol67,sichel75} to the framework of
binomial interpolation/extrapolation. This opens the possibility of
describing what is left open here, e.g.\ the possibility of the
reversal of the growth rate of the morphemes in agriculture.

The last point is related to the theoretical modelling of terminology.
We have focused on the general difference between borrowed and native
morphemes {\it en masse} in Japanese terminology, effectively ignoring
the differences in such factors as term length distributions among different
domains. To fully exploit the quantitative modelling of terminology, 
however, it will be necessary to take into account the wider characteristics
terms, including the intra-term dependency patterns of morphemes, in addition
to the nature of morphemes in terminology {\it en masse}.


\acknowledgment

This work is supported by the Grant-in-Aid C(2) 14580465 of the Japan
Society of the Promotion of Sciences. I am indebted to Dr. H. Baayen
of the Max Plank Institute for Psycholinguistics for guiding me to the
dynamic analysis of language data. 


\bibliographystyle{nlpbbl}
\bibliography{423}


\appendix
\subsection*{A  Derivation of the formula for binomial 
\\ interpolation/extrapolation
}


On the basis of the binomial model and under the assumption that
the population events are known, the number of morphemes that occur
exactly $m$ times in the sample of size $\lambda N$ can be given by
the equation (1):

\begin{displaymath}
E[V(m,\lambda N)]  = \sum_{j=0}^S \left ({\lambda N \atop m} \right )
    p_j^m (1-p_j)^{\lambda N -m}
\end{displaymath}

This can be transformed as follows:

\vspace*{-\baselineskip}
 
\begin{eqnarray}
E[V(m,\lambda N)] & = & \sum_{j=0}^S \left ({\lambda N \atop m} \right )
    p_j^m (1-p_j)^{\lambda N -m} \nonumber \\
& = & \sum_{j=0}^S \left ({\lambda N \atop m} \right ) p_j^m (1-p_j)^{N -m}
    \left (1 + \frac{p_j}{1 - p_j} \right )^{-(\lambda -1)N}  \nonumber \\
& = & \sum_{j=0}^S \left ({\lambda N \atop m} \right ) p_j^m (1-p_j)^{N -m} 
\cdot \sum_{k=0}^{\infty} \left ({{-(\lambda -1)N} \atop k} \right ) p_j^k (1-p_j)^{-k}
           \nonumber \\
& = & \sum_{k=0}^{\infty}  \left ({\lambda N \atop m} \right )
    \left ( {{-(\lambda -1) N} \atop k} \right )
    \sum_{j=0}^S  p_j^{m+k} (1-p_j)^{N -(m+k)}     \nonumber \\
& = & \sum_{k=0}^{\infty}  \frac{\left ({\lambda N \atop m} \right )
    \left ( {{-(\lambda -1) N} \atop k} \right )}{\left ( {N \atop {m+k}} \right )}
    E[V(m+k,N)]. \nonumber
\end{eqnarray}

\noindent
If we only use here the range $m+k \leq N$ and $k \leq (\lambda -1)N$
for actual calculation, the term for combinatorics in the last line
can be rewritten as:

\vspace*{-\baselineskip}
 
\begin{eqnarray}
\frac{\left ({\lambda N \atop m} \right ) \left ( {{-(\lambda -1) N} \atop k}
    \right )}{\left ( {N \atop {m+k}} \right )}
    & \simeq & \frac{ (\lambda N)^m (-(\lambda -1)N)^k(m+k)! }{
    m!\:k!\:N^{m+k} } \nonumber \\
& = & (-1)^k \lambda^m (\lambda -1)^k \left ( {{m+1} \atop m} \right ), \nonumber
\end{eqnarray}

\noindent
This leads to equation (3). Equation (2) immediately follows.



\subsection*{B Derivation of ${\cal P}(N)$}


Firstly, let us introduce the structural distribution of the morpheme
types, which can be expressed as follows:
 
\begin{displaymath}
G(p) = \sum_{i=1}^S I_{[p_i\geq p]}
\end{displaymath}

\noindent
where $I$ = 1 when $p_i \geq p$ and 0 otherwise. The value of $G(p)$
represents the number of morpheme types whose occurrence probability is
greater or equal to $p$. Let us then focus on the value of $p$, and
introduce the new subscript $j$, such that $p_j$ indicates, in ascending
order, the values of $p$ which at least one morpheme type takes, i.e.
$p_j < p_{j+1}$ if $j < j+1$, and there is at least one $w_i$ such
that $p_i = p_j$ if $p_j > 0$.

Using $G(p)$ with the re-indexed subscript $j$ of $p$, the equation that
estimates the number of morpheme types can be rewritten in the integral form
as follows:

\vspace*{-\baselineskip}
 
\begin{eqnarray}
E[V(N)] & = & S - \sum_{i=1}^S(1-p_i)^N \nonumber \\
 & = & \sum_{i=1}^S(1-e^{-Np_i}) \nonumber \\
 & = & \int_{0}^\infty(1-e^{-Np}) \  dG(p) \nonumber 
\end{eqnarray}

\noindent
where $dG(p)$ = $G(p_j) - G(p_{j+1})$ around $p_j$, and 0 otherwise.

As this indicates the growth curve of the vocabulary, the first derivative
of this formula expresses, in mathematical sense, the growth rate of the
vocabulary, which can be expressed as follows:

\vspace*{-\baselineskip}
 
\begin{eqnarray}
\frac{d}{dN}E[V(N)] & = & \frac{d}{dN} \int_{0}^\infty(1-e^{-Np}) \  dG(p) \nonumber \\
 & = &  \int_{0}^\infty -p \cdot -e^{-Np} \  dG(p) \nonumber \\
 & = & \frac{1}{N} \int_{0}^\infty Npe^{-Np} \  dG(p) \nonumber \\
 & = & \frac{E[V(1,N)]}{N} \nonumber
\end{eqnarray}

\begin{biography}
\biotitle{}
\bioauthor{Kyo Kageura}{
Kyo Kageura was born in 1964. He received his Ph.D. from the University
of Manchester in 1993. 
He was working at the Department of Research and
Development, the National Center for Science Information Systems (NACSIS),
Japan, since 1988. Since 2000, he has been working as an associate professor of the Human and
Social Informatics Division, the National Institute of Informatics, Japan.
His main research interest is in media studies. He is also carrying out
research in qualitative and quantitative modelling of term formation and
terminological growth as well as in the logical foundation of the theory of terminology.
He is a member of the Japan Society of Library and Information Science, the
Association for Natural Language Processing, the Mathematical Linguistic Society 
 of Japan, and the International Quantitative Linguistic Society.}

\bioreceived{Received}
\biorevised{Reviced}
\bioaccepted{Accepted}

\end{biography}

\end{document}
