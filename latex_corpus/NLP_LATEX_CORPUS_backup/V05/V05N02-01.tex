



\documentstyle[hangul,jnlpbbl]{jnlp_j}

\setcounter{page}{3}
\setcounter{巻数}{5}
\setcounter{号数}{2}
\setcounter{年}{1998}
\setcounter{月}{4}
\受付{1997}{5}{8}
\再受付{1997}{11}{11}
\採録{1998}{1}{23}

\setcounter{secnumdepth}{2}

\title{日韓機械翻訳における様相テーブルに基づいた\\韓国語述部の生成処理}
\author{金 政仁\affiref{POSTECH} \and
 李 鐘赫\affiref{POSTECH2} \and
 李 根培\affiref{POSTECH2}}

\headauthor{金政仁,李鐘赫,李根培}
\headtitle{日韓機械翻訳における様相テーブルに基づいた韓国語述部の生成処理}

\affilabel{POSTECH}{浦項工科大学校情報通信研究所}
{POSTECH Information Research Laboratories}
\affilabel{POSTECH2}{浦項工科大学校電子計算学科}
{Dept. of Computer Science and Engineering, POSTECH}

\jabstract{
日韓機械翻訳を研究している多くの研究者らは両国語の文節単位の
語順一致のような類似性を最大に生かすため,直接翻訳方式を採択
している.しかし,日本語と韓国語の述部間には,対応する品詞の
不一致,局部的な語順の不一致,活用ルールの不一致,時制表現の
不一致などが解決しにくい問題として残っている.本稿では述部表
現の不一致を解決するため“様相テーブルに基づいた韓国語の生成
方法”を提案し,それに対して体系的な評価を行なう.この方法は
述部だけを対象にする抽象的で意味記号的な様相資質をテーブル化
し,両国語の述部表現のPIVOTとして用いることにより,述部
の様相表現の効果的な翻訳を可能とする.朝日新聞と日本語の文法
本から抽出した2,338個の例文を対象に述部の翻訳処理を試み
た結果,約97.5%が自然に翻訳され,述部翻訳の際,本方法が
有効であることが確認できた.
}

\jkeywords{機械翻訳,生成,述部,様相,韓国語}

\etitle{Generation of Korean Predicates \\
Based on Modality-Feature Ordering \\
and Lexicalizing Table \\
in Japanese-Korean Machine Translation
}
\eauthor{Jung-In Kim \affiref{POSTECH} \and Jong-Hyeok Lee \affiref{POSTECH2} \and Geunbae Lee \affiref{POSTECH2}} 

\eabstract{
Both Korean and Japanese share many grammatical 
characteristics including the same word order, so that almost all 
Japanese-to-Korean machine translation systems would have 
adopted the direct translation strategy to take advantage of the 
similarities. Even in the direct translation for the very similar 
language pair, however there are still a lot of problems that have 
to be solved for high-quality translation. Out of them we only 
focus on the predicate translation, whose difficulty is caused not 
only by complex conjugation but also by the inconsistent syntactic 
category and the different relative order of modal expressions 
between two languages. To solve the difficulty, we propose a 
table-driven predicate generation in which a modality-feature 
ordering and lexicalizing table (called MFOLT) plays an 
important role to map Japanese predicates into their Korean 
equivalents via abstract pivot of symbolic modality features. 
Experimental evaluation was done with 2,338 sentences 
extracted from Asahi newspaper and some Japanese grammar 
books, which turned out that the proposed method would make a 
good effect on predicate translation, showing the success rate of 
97.5\%. 
}

\ekeywords{Machine Translation, Generation, Predicate, Modality, Korean}

\begin{document}
\maketitle


\section{はじめに}
直接翻訳方式は普通の変換翻訳方式で行なっている構文解析や意
味解析の部分を省略あるいは簡素化でき,類似性のある言語間の翻
訳によく用いられていた.現在,知られているほとんどの日韓,韓
日翻訳システムが直接翻訳方式を採用しているのも両言語の類似性
を活かすためである.最近,構文解析分野や意味解析分野など,言
語処理技術の全般的な発達とコンピュータのハードウェア性能が向
上した時点で直接翻訳方式を用いるのは,翻訳に必要な膨大な情報
の損失とつながり,比較的多くの翻訳情報が得られる変換方式や多
言語間の翻訳ができる中間言語方式を勧奨しているが(長尾 真 
1996),日韓機械翻訳においては翻訳に必要な情報があまり多くな
い.(金泰錫 1991)によると,実際になんの情報もなく両国語の単
語を1:1に対応させた場合にもある程度の理解できる訳語が出来
上がったと報告しており,我々はもう少しの追加情報を用いれば,
相当な品質の訳語が生成できると期待している.

最近,このような類似性を用いた日韓直接翻訳システムが商用化
し始めた.最初の商用システムは1983年,FUJITSUと韓
国のKAISTが共通に開発したATLAS−J/Kであり,その
後,多くの商用システムが続々登場した.日韓機械翻訳システムの
代表的な商用システムには日本の高電社が開発した“j−Seou
l”および日立情報ネットワークが開発した“HICOM/MT”
などが挙げられる.また韓国でもユニ−ソフトが開発した“5徑博
士”や創信コンピュータの“ハングルがな”などが市販されている.
しかし,(崔杞鮮 1996)によると,これらのシステムは訳語の品
質が満足できる程度まで至らず,形態素解析,多義性,対訳語選定,
品詞判定,未登録単語の処理などの部分でまだ多くの問題点を持っ
ていると報告している.

日韓直接翻訳には大きく分けて(1){\bf 多義性処理},
(2){\bf 述部の様相および活用処理}が問題と残っている(金政仁1996).
(李義東 1989;金政仁 1992;EunJa Kim 1993;朴哲済1997)は(1)
問題の先行研究であり,相当の成果を上げたが,より良い結果を目
指して現在も多義性解消のための研究が活発に進行されつつある.
(2)問題の先行研究として(李義東 1990)は,日本語述部の様
相情報に文法的な意味を付与して処理する手法を提案した.まず,
日本語述部での様相情報の意味から韓国語述部の生成に適する意味
に変換する.そのとき,意味省略,意味転移,語順調整を行ない,
意味対応テーブルを作る.意味対応テーブルには日本語述部を構成
する様相情報たちからすべての組み合わせを取り出し,日韓述部の
意味対応ペアとして記述する.だから,この手法は様相情報の組み
合わせに依存するので意味対応ペアの数が多くなる短所を持つ.そ
して,(金泰錫 1992;金政仁 1996)は韓国語の述部の様相情報
および活用形態を前後単語との意味接続関係によってあらかじめテ
ーブル形式に用意して翻訳を行なう翻訳テーブル方式を提案した.
しかし,この方式は,両国語間の活用語の対応ルールが作成しにく
いことを前提とし,複雑な活用規則を考慮せず,表層語を1:1に
対応させる宣言的な処理を選んでいる.そして,表層語が1:nに
対応するときは1:1の関係を作るため,前後単語との接続規則や
形態が異なるn個の韓国語の表層語を辞書に用意しなければならな
い.また,様相情報や活用形態を区分せず,一度に処理するので異
形態の対訳語の数が相対的に増える問題があった.

ここで,本稿では,韓国語述部を構成する広範囲な様相情報を,
抽象的で意味記号的な意味資質に表現した後,テーブル化した様相
テーブルを用いた両国語の述部処理を提案する.様相テーブルは様
相情報の意味を表わす意味記号,韓国語表層語,活性化チェックフ
ィールドで構成する.様相テーブルは日韓述部翻訳のPIVOTの
ような役割を担当し,述部生成のとき,韓国語表層語は様相テーブ
ルから取り出す.そして,様相テーブルの様相資質から韓国語表層
語を対応させた後,表層語の結合処理で音韻調和処理,音韻縮約処
理,分かち書きを行ない,自然な述部を生成する.

すなわち,様相テーブルは述部情報らの組み合わせに依存しない
ので,(李義東 1990)の意味対応テーブルより簡潔な表記ができ
る.また,(金泰錫 1992;金政仁 1996)では一遍に行なった述
部の様相情報処理や活用処理を分離して処理する.

本システムは(EunJa Kim 1993,朴哲済1997)の変換過程を
そのまま用いており,述部に生じる多義はすでに解消されているも
のとし,本稿では意味が決まった様相情報から述部の自然な生成を
目標とする.そして,本稿でのハングルに対する発音はYaleロ
ーマ字表記法に基づいて表記する.

\section{様相テーブルの導入}
日本語述部と韓国語述部はその表現方法に相違点がある.両国語
の述部は1:1の対応を基本とする直接翻訳方式では大きな問題に
なり,また,述部の翻訳が間違った場合は文法的に正しくない文が
生成されるエラー(ungrammatical error)や理解できない文が生成
される(unintelligible error)などの致命的な翻訳失敗になりやすい
(D.U.An 1995).従って韓国語の述部を正確に表現してくれるこ
とは日韓機械翻訳において大きい比重を占める.

\subsection{日本語と韓国語の述部表現の相違点}
日本語の述部は用言の語幹あるいは体言の後に補助用言や助動詞,
そして一部の助詞を用いて表現し,大部分の補助用言,助動詞およ
び助詞たちは決められた並び順に並べる.すなわち,日本語述部の
様相類は一定な表現の順序を持っている.

\vspace{3mm}
\begin{quote}
{\bf
日本語述部の構成\\
体言{\footnotesize ${^{+}}$}\ \ 
助動詞{\footnotesize ${^{*}}$}\ \ 
一部助詞{\footnotesize ${^{*}}$}\\
用言{\footnotesize ${^{+}}$}\ \ 
助動詞{\footnotesize ${^{*}}$}\ \ 
((助詞:て|で) 補助用言){\footnotesize ${^{*}}$}\ \ 
助動詞{\footnotesize ${^{*}}$}\ \ 
一部助詞{\footnotesize ${^{*}}$}\\
}
{\footnotesize ($^{+}$:1個以上の繰り返し,$^{*}$:0個以上の繰り返し)}
\end{quote}
\vspace{3mm}

韓国語述部の様相情報も一定の並び順に従って並べる.しかし,
韓国語は日本語と異なって助動詞が存在せず,語幹と補助用言,叙
述格助詞“\hg{'ida} ({\it ita})",先語末語尾,語末語尾,そして一部助詞で
構成されている.

\vspace{3mm}
\begin{quote}
{\bf 
韓国語述部の構成\\
(1)((体言  叙述格助詞{\footnotesize ${^{*}}$})|用言){\footnotesize ${^{+}}$}\ \ 
先語末語尾{\footnotesize ${^{*}}$}\ \ 語末語尾{\footnotesize ${^{*}}$}\ \ 
一部助詞{\footnotesize ${^{*}}$}\\
(2)補助用言{\footnotesize ${^{+}}$}\ \ 先語末語尾{\footnotesize ${^{*}}$}\ \ 
語末語尾{\footnotesize ${^{*}}$}\ \ 一部助詞{\footnotesize ${^{*}}$}
}
\end{quote}
\vspace{3mm}

韓国語述部の表現は,(1),(2)に分けて表記した.(1)は単独で
述部を構成するが,(2)は(1)に付けられるときだけ使用可能である.
また,(2)は(1)に複数個を付けることができ,その時は(2)の先語
末語尾はもっとも後に付く補助用言にのみ付けられる.日本語を基
準として両国語の対応関係を整理すると,{\bf 表1}のように表わすこと
ができる.

\begin{table}[htb]
\caption{両国語述部の対応}
\begin{center}
\begin{tabular}{|l|l|} \hline
{\bf 日本語} & {\bf 韓国語} \\ \hline
助動詞 \hspace{20mm}
& 先語末語尾,語末語尾,\\
& 補助用言,叙述格助詞 \\ \hline
一部助詞 & 語末語尾,一部助詞 \\ \hline
補助用言 & 補助用言 \\ \hline
\end{tabular}
\end{center}
\end{table}


両国語の述部表現は1:1の対応が難しく,日本語の助動詞の反
復表現による様相情報の順序も韓国語と一致していない.そして,
韓国語の補助用言,先語末語尾,語末語尾は前接する単語の属性に
よって異形態になったり,その上,韓国語の述部を完成させるため
には11種類の不規則音韻変化,5種類の音韻縮約が適用される.
日韓機械翻訳において韓国語述部を自然に表現するためにはこのよ
うな部分を細心に処理する必要がある.{\bf 例文(1)}に両国語の対応
関係を示す.

\atari(120,32)

両国語の述部表現は1:1対応が難しく,日本語の助動詞の繰り
返し表現による様相順序も一致しない.次の例文は日本語と韓国語
の述部での順序不一致を表わす.

\atari(90,70)
\vspace{5mm}

{\bf 例文(2)}は語尾「き」の訳語省略以外には1:1の対応によっ
て翻訳が可能なことを示している.しかし,{\bf 例文(3)}を見ると,
日本語は「丁寧+過去/終結」の順番であるのに対し,韓国語は「過
去+丁寧+終結」の順番であり,翻訳語の並びが逆になっている.
もう少し複雑な例文を見よう.

\vspace{5mm}
\atari(110,70)
\vspace{5mm}

{\bf 例文(4)}と{\bf 例文(5)}は両国語の述部情報が1:1に対応して
いないことを示す.{\bf 例文(4)}の訳文では「ませ(丁寧)」と「ん
(否定)」の様相情報が置き換わっている.また,「ん」で文が終
わる場合は終結の意味も含まれており,「ん」から「\hg{'ji 'anh} ({\it ci anh})」
や「\hg{da} ({\it ta})」を一緒に生成しなければならない.
{\bf 例文(5)}は丁寧
の意味が2回表われた場合であるが,韓国語では丁寧の意味を1ヶ
所で表現するので,「ませ」や「でし」から1個の訳語を用意して,
「ませ,でし(丁寧)」と「た(過去)」の様相情報の順序を置き
換えて並べる.そして,「た」から終結の意味まで取り出して,文
の終結部分に「\hg{da} ({\it ta})」を生成するべきである.また,日本語述部
には現在形と未来形を区別せず使っており,特定な表現では時制が
異なる述部も存在する.


\vspace{2mm}
\noindent
\hspace*{25mm}
{\bf 例文(6)} 映画を\underline{見る}ときは静かにしなさい.\\
\hspace*{42mm}
現在形) \hg{'ieoqhoalyl }\underline{\hg{bonyn}}
 \hg{Dainyn jo'ioqhi haseoi'io}. (X)\\
\hspace*{42mm}
未来形) \hg{'ieoqhoalyl }\underline{\hg{bol}}
 \hg{Dainyn jo'ioqhi haseoi'io}. \ \ \  (O)\\
\hspace*{25mm}
{\bf 例文(7)} その映画は見たほうがよい.\\
\hspace*{42mm}
過去形) \hg{gy 'ieoqhoanyn }\underline{\hg{bon}}
 \hg{pieon'i johda}.\ \ \ \  (X)\\
\hspace*{42mm}
現在形) \hg{gy 'ieoqhoanyn }\underline{\hg{bonyn}}
 \hg{pieon'i johda}. (O)
\vspace{2mm}

{\bf 例文(6)}の連体形「見る」は普段は現在時制を表わすが,文脈
の意味が「仮定,予定」の場合は,前接に未来形が来なければなら
ない.「とき,つもり,必要,予定,計画,方針」などの体言は未
来形を要求する.そして{\bf 例文(7)}の「ほうがよい」の表現は,韓
国語の述部では過去時制の代わりに現在時制が前に来る.

このように日本語と韓国語は述部の表現部分では様相情報の順序
不一致,1:n対応,n:1対応,対訳語省略,時制表現の不一致
など,様々な相違点を持っており,これらの相違点は直接翻訳方式
を用いた日韓機械翻訳で高品質の翻訳文生成に大きな障害になって
いる.

\vspace{-1mm}
\subsection{様相テーブル(MFOLT)の構成}
日韓述部の表現不一致を解決するため,我々は述部の様相情報を
言語に依存しない意味記号的な様相資質(modality feature)に表
現し,両国語の述部情報のPIVOTとして利用する方法を提案す
る.特に,本稿では様相情報と活用形を分離して処理する方法を提
案する.様相情報の処理は様相資質を集めた様相テーブルを利用し,
活用処理は韓国語述部の不規則活用および音韻縮約を規則化させ,
韓国語述部の生成に用いる.

様相資質の構成は韓国語の補助用言,先語末語尾,語末語尾,一
部助詞から抽出し,文法的には説明しにくい日韓間だけで適用でき
る特殊資質も入れ込み,より柔らかい述部の表現ができるように構
成した.この過程で,補助用言,先語末語尾,語末語尾は韓国語の
述部を生成するのに必要な最適の部分として統廃合しており,統廃
合された最適な様相資質をテーブル化したものをMFOLT
(modality feature ordering and lexicalizing table)と呼ぶ.MF
OLTは韓国語述部の言語学的な知識(高永根 1985),韓国語述
部を生成するときの経験によって作られた.最終的な様相カテゴリ
ーは{\bf 表6}に示されている.

(高永根 1985)は韓国語での補助用言を{\bf 表2}のように区分して
いる.また,これらが同時に複数使われた場合は一定の順序を持つ.
使役,受け身,強調は同じ語順であるが,その下は書かれた順番が
語順を示す.

\begin{table}[h]
\vspace{-3mm}
\caption{韓国語の補助用言の様相情報}
\begin{center}
\begin{tabular}{|r|l|l|l|}
\hline
\multicolumn{1}{|l|}{順番} & 
様相 & 韓国語の補助用言 & 対応する日本語 \\ \hline
1 & 使役 & --\hg{geoi ha} ({\it key ha})
	& せる,させる \\ \cline{2-4}
  & 受け身 & --\hg{geoi doi} ({\it key toy})/--\hg{'a ji} ({\it a ci}) 
	& れる,られる \\ \cline{2-4}
  & 強調 & --\hg{'a dai} ({\it a tay})
	& --したてる/--し散らす \\ \hline
2 & 奉仕 & --\hg{'a ju} ({\it a cwu})/--\hg{'a dyri} ({\it a tuli})
	& --てくれる/--てあげる \\ \hline
3 & 完了 & --\hg{'a nai} ({\it a nay})/--\hg{'a beri} ({\it a peli})
	& --てしまう \\ \hline
4 & 試図 & --\hg{'a bo} ({\it a po})
	& --て見る \\ \hline
5 & 中止 & --\hg{da mal} ({\it ta mal})
	& --さし \\ \hline
6 & 希望 & --\hg{go sip} ({\it ko siph})/--\hg{gi 'ueonha} ({\it ki wenha})
	& たい/--てほしい \\ \hline
7 & 意図 & --\hg{rieha} ({\it lyeha})
	& よう \\ \hline
8 & 虚飾 & --\hg{nyn ceogha} ({\it nun chekha})/ & ぶりをする \\ 
  &  & \hspace*{4mm}--\hg{nyn 'iaqha} ({\it nun yangha}) &  \\ \hline
9 & 進行 & --\hg{go 'iS} ({\it ko iss})/--\hg{'a ga} ({\it a ka})/
--\hg{'a 'o} ({\it a o})
	& ている,てある,てくる,つつある \\ \hline
10 & 可能 & --\hg{l su 'iS} ({\it l swu iss})
	& できる \\ \hline
11 & 否定 & --\hg{ji 'anh} ({\it ci anh})/--\hg{ji mosha} ({\it ci mosha})/
	& ない,ぬ,ん \\ 
   &  & \hspace*{4mm}--\hg{ji mal} ({\it ci mal}) & \\ \hline
12 & 当為 & --\hg{'eo'ia han} ({\it eya han})
	& べき,すべき \\ \hline
13 & 承諾 & --\hg{'eodo joh} ({\it eto coh})
	& てもいい \\ \hline
14 & 理由 & --\hg{gi Daimun} ({\it ki ttaemwun})/ & ため,ので \\ 
   &  & \hspace*{4mm}--\hg{n Gadarg} ({\it n kkatalk}) &  \\ \hline
15 & 推定 & --\hg{r dysha} ({\it l tusha})/
--\hg{r mo'iaq'i} ({\it l moyangi})/ & まい,はず \\
   &  & \hspace*{4mm}--\hg{r ges'i} ({\it l kesi}) &  \\ \hline
16 & 一念 & --\hg{r Bun'i} ({\it l ppwuni})
	& だけ \\ \hline
17 & 引用 & --\hg{dago han} ({\it tako han})
	& そう \\ \hline
\end{tabular}
\end{center}
\end{table}

\vspace{-6mm}
統廃合の際,「強調,奉仕,試図,虚飾,承諾」の様相情報は,
日本語で2つの述部に表現するので本用言のように扱う.「意図」
は後に「と」を付けて「ようと」の形を取って「意志」に統合する.
そして,「理由」は「原因・結果」に,「引用」は「伝聞」にそれ
ぞれ統合した.また,「進行」の様相資質は「動作進行1,動作進
行2,状態進行」に分けて3つに,「希望」は「希望1,希望2」
の2種類に区分した.

先語末語尾は語尾として分類され,補助用言の後にくるが,先語
末語尾が持つ様相情報は一定な順序を持ち,その順番を{\bf 表3}に示す.


\begin{table}[h]
\caption{韓国語の先語末語尾の様相表現}
\begin{center}
\begin{tabular}{|r|l|l|l|}
\hline
\multicolumn{1}{|l|}{順番} & 
様相 & 韓国語の先語末語尾 & 対応する日本語 \\ \hline
1 & 尊敬 & --\hg{si}--({\it si})/--\hg{sibsi}--({\it sipsi}) 
	& 尊敬表現の動詞を使う \\ \hline
2 & 時制 & 未来: --\hg{geoiS}--({\it keyss})
	& 用言の終止形 \\ \cline{3-4}
  &      & 現在: --\hg{n}--({\it n})/--\hg{nyn}({\it nun})
	& 用言の終止形,連用形 \\ \cline{3-4}
  &      & 過去: --\hg{'aS}--({\it ass})/--\hg{S}({\it ss})
	& た/だ \\ \hline
3 & 謙譲 & --\hg{'o}--({\it o})/--\hg{'ob}--({\it op})/--\hg{b}--({\it p})/
--\hg{syb}({\it sup})
	& 謙譲表現の動詞を使う \\ \hline
4 & 回想 & --\hg{de}--({\it te}) & (だった) よ/ね \\ \hline
\end{tabular}
\end{center}
\end{table}

先語末語尾からは「一部の尊敬表現,時制,回想」をMFOLT
に導入する.尊敬表現の一部は韓国語対訳語が同じ様相的な意味を
持つよう,語幹に様相情報の表層語を寄せて処理した.例えば,「お
いでになる」は「\hg{'o}(語幹) + \hg{si}(先語末語尾) + \hg{da}(語末語尾)」の形
態素で構成されるが,「\hg{'osi}(語幹) + \hg{da}(語末語尾)」に扱って,処
理を簡略化する.慣用表現「お(連用形)になる」は尊敬表現とし
てMFOLTの様相資質とする.また,敬語表現の接頭語「お,ご」
は韓国語にない表現であるので,お茶,ご飯などの慣用表現は一つ
の単語として扱い,その他「お,ご」は無視する.謙譲情報は最近
の韓国語では使っていない傾向が強いので,「いただく」は
「\hg{badda} ({\it patta})」,「いたす」は「\hg{hada}({\it hata})」
などの普段の表現と対応
させる.そして「指定,必縁,不可能,禁止,3者希望,結果・保
存」の様相情報は翻訳の際,経験によって述部生成の効率を上げる
ために様相資質化した.以上の様相資質たちを補助資質(supplement 
feature)と呼び,これらは順番に持ち,省略あるいは複数の資質が
同時に活性化されることもできる.そして,述部の終了としては使
えず,補助資質の後には必ず終了資質(terminal feature)が付く.

終了資質とは韓国語述部の終了部分が自然に生成できるよう,日
本語述部を構成する助動詞,一部助詞,活用形,複合助詞,そして
分離して処理しにくい慣用表現も様相資質化したものである.必ず
述部の最後には1つの終了資質が付かなければならない.すなわち,
韓国語述部の構成は
{\bf 「韓国語述部 = 述語 + 補助資質{\footnotesize$^{*}$} + 終了資質」}
と記述できる.

それに比べ,(李義東 1990)では日本語述部から様相情報の組
み合わせをすべて意味化し,意味対応テーブルを作成してそれぞれ
に対応する韓国語述部の意味を用意する.すなわち,日韓述部の意
味変換を行ない,変換された意味を韓国語述部の生成に用いる.し
かし,この方法は様相情報が異なるすべての述部を対象とし,意味
対応関係を記述する必要があり,様相情報の異なる表現数ほど意味
対応ペアが増えてしまう.また,この方法は未来形を要求する体言
を修飾する述部から未来時制を取り出すことができず,時制の不一
致を処理することが難しい.

{\bf 表6}にMFOLTの様相資質を示す.区分のため,'s'で始まる記
号は補助資質,'t'で始まる記号は終了資質とした.補助資質は韓国
語述部での様相情報の並べ順に,終了資質はアルファベット順に表
記した.

\vspace{5mm}
\subsection{様相資質間の干渉による活性状態の最適化}
MFOLTの様相資質らは様相資質間の干渉によって,活性化さ
れた状態を調節して,最適の韓国語述部が生成できるような形態を
取る.例えば,{\bf 例文(8)}のように可能と否定の様相資質が活性化
されると不可能の様相資質を活性化し,可能と否定の様相資質を初
期化することを行なう.

\clearpage
\vspace*{3mm}
\atari(105,85)
\vspace{5mm}

また,{\bf 例文(9)}では,ある様相資質が別の様相資質の影響を受
けて初期化されることを表わす.時の前にくる連体形は未来時制に
する.

\vspace{5mm}
\atari(105,70)
\vspace{5mm}

ここに13個の活性状態の最適化ルールを示す.{\bf 例文(8)}はル
ール6,{\bf 例文(9)}はルール2によって最適化されている.

\vspace{3mm}
\begin{quote}
{\bf 
\begin{description}
\item[ 1.] if\ \ tINTR=OFF and tDECL=OFF  then, sPLT=OFF 
\item[ 2.] if\ \ tTIME=ON and tPREN=ON  then, tPREN=OFF 
\item[ 3.] if\ \ sPAST=ON and tPREN=ON  then,\\
\hspace*{4mm}tPPREN=ON;sPAST=OFF; tPREN=OFF
\item[ 4.] if\ \ tSOTH=ON and tPREN=ON  then, tPREN=OFF
\item[ 5.] if\ \ tADVE=ON and tPREN=ON  then, tPREN=OFF
\item[ 6.] if\ \ sCAN=ON and sNEG=ON  then,\\
\hspace*{4mm}sUNCAN=ON;sCAN=OFF; sNEG=OFF
\item[ 7.] if\ \ tNOUN=ON and tREA2=ON  then, tREA2=OFF
\item[ 8.] if\ \ sPLT=ON and tEXCL=ON  then, tEXCL=OFF
\item[ 9.] if\ \ sALWS=ON and tAND1=ON  then, tAND1=OFF
\item[10.] if\ \ sSUGST=ON and sPAST=ON  then, sPAST=OFF
\item[11.] if\ \ tTIME=ON and tPAST=ON  then,\\
\hspace*{4mm}tPAST=OFF;sPAST=ON
\item[12.] if\ \ sCMTO=ON and tPREN=ON  then, tPREN=OFF
\item[13.] if\ \ sCAUS=ON and sPASS=ON  then,\\
\hspace*{4mm}sCMTO=ON; sCAUS=OFF; sPASS=OFF
\end{description}
}
\end{quote}
\vspace{3mm}

提案したMFOLTを用いた韓国語述部の生成方法は(1)MF
OLTでの補助様相資質の並べ順を生かして述部情報の異なる語順
を調節する,(2)MFOLTを活性化する際,述部の様相情報と
MFOLTの様相資質をm:nに対応させ,1:1に対応しない様
相情報を表現する,(3)時制の変異ができるように様相資質間を
調節して述部の時制不一致に対応することが可能である.

\section{MFOLTを用いた韓国語述部の生成}
自然な韓国語の述部を生成するため,いくつかの段階を用意した.
すなわち,韓国語の生成処理にはMFOLTの各様相資質の活性化
処理,受け身表現処理,否定表現処理,様相資質の表層化,用言の
不規則変化処理,補助用言や語尾の異形態処理,そして音韻縮約処
理,分かち書き処理が必要である.{\bf 図1}に述部処理の各段階を示す.

\clearpage

\begin{figure}[htb]
\atari(70,68)
\caption{MFOLTを用いた韓国語述部の生成段階}
\end{figure}

これから{\bf 図1}の各段階について詳しく説明する.

\subsection{様相資質の活性化および最適化}

入力された日本語の述部から様相情報を抽出してMFOLTの各
様相資質を活性化する過程である.MFOLTは述部単位に活性化
するので,複文の場合は何度も活性化と初期化が行われるように設
計されている.{\bf 図2}は様相資質の活性化過程を表わしている.

\begin{figure}[htb]
\atari(115,43)
\caption{様相資質の活性化}
\end{figure}

\subsection{受け身の特殊表現処理}
韓国語の受け身表現は
「\hg{'a}/\hg{'eo}/\hg{'ieo} \hg{jida} ({\it a/e/ye cita})」の一般的な
表現方法と同時に補助用言
「\hg{'i},\hg{'hi},\hg{'ri},\hg{'gi} ({\it i/hi/li/ki})」を用いて表現
する.このような表現は用言の適切な分類やMFOLTの様相資質
を用いることによって処理可能であるが,する動詞の一部は特別な
処理を必要とする.すなわち,受け身になるとき,その形態が普通
の形態とは別に表わされる.

\vspace{3mm}
\begin{center}
\begin{tabular}{ll}
保護\underline{される} & (X)\hg{bohoha}\underline{\hg{'ieo jida}}\\
	& (O)\hg{boho}\underline{\hg{badda}}\\
愛\underline{される} & (X)\hg{saraqha}\underline{\hg{'ieo jida}}\\
	&(O)\hg{saraq}\underline{\hg{badda}}\\
追跡\underline{される} & (X)\hg{cujeogha}\underline{\hg{'ieo jida}}\\
	& (O)\hg{cujeog}\underline{\hg{daqhada}}\\
\end{tabular}
\end{center}
\vspace{3mm}

このような動詞らは{\bf 図3}のようにMFOLTが活性化された後,
受け身のための特殊処理を行なって解決する.

\begin{figure}[htb]
\atari(113,62)
\caption{受け身表現の特殊処理}
\end{figure}

\subsection{否定語の特殊処理}
日本語と韓国語は否定表現の相違点が存在する.(金泰錫 1993)
によると,日本語の否定表現には形容詞または助動詞として「ない」,
そして助動詞「ん(ぬ)」と「まい」を用いる.そして接頭詞「非,
不,未,無」を付けて一単語だけを否定する漢語的な否定表現と否
定の推量または意志を表わす助動詞「まい」はその対訳語が韓国語
にも存在するので,日韓においては1:1の対応処理が可能である.
結局,日本語の否定表現から否定語処理を必要とするのは「ない」
と「ん(ぬ)」の含まれている否定文である.それに比べ,韓国語
の否定表現は「\hg{mos} ({\it mos}),\hg{'an} ({\it an})」
という否定素を用言の前に配置
する方法と否定補助用言「\hg{ji 'anhda} ({\it ci anhta})」と
「\hg{'eobssda} ({\it epsta})」を
用いて否定文を表わすことができる.
否定素「\hg{mos} ({\it mos}),\hg{'an} ({\it an})」
は使用上の制約が多くあり,否定補助用言「\hg{ji 'anhda} ({\it ci anhta})」と
「\hg{'eobssda} ({\it epsta})」でもすべての表現が可能であるから,日本語の否定
表現「ない」と「ん(ぬ)」は
韓国語の否定表現「\hg{ji 'anhda} ({\it ci anhta})」
と「\hg{'eobssda} ({\it epsta})」に翻訳しても別に問題はない.否定表現の処理は
MFOLTを活性化し,韓国語の組み立てルールを適用させること
で処理可能である.しかし,日本語の「知る,分かる,解かる」が
否定の意味で使われる場合は「\hg{'alji 'anhda}({\it alci anhta})」の代わりに
韓国語の対立肯定用言「\hg{morynda} ({\it molunta})」に,存在を表わす動詞
「在る,有る,居る」は「\hg{'iSji 'anhda} ({\it issci anhta})」の代わりに
「\hg{'eobssda} ({\it epsta})」に翻訳するのが生成された韓国語の表現が
自然であり,
そのため,否定表現が否定の意味を含む肯定表現に変わる特殊処理
を行なう必要がある.{\bf 図4}に否定表現の特殊処理を示す.

\begin{figure}[htb]
\atari(115,63)
\caption{否定表現の特殊処理}
\end{figure}

助動詞「ない」が否定の意味を表わす様相資質を活性化させるが,
述部の語幹「知」が否定の様相資質と合わせて「\hg{morynda} ({\it molunta})」
という否定の意味を含む対立語に変わるので,否定の様相資質を初
期化する.

\subsection{様相資質の表層化}
日韓直接翻訳方式に基づいた韓国語の組み立ては,変換処理から
得た結果とMFOLTの活性化された状態,そして受け身,否定語
の特殊処理を基に出来上がる.

日本語の文「彼が保護されていることは知らなかった」はMFO
LTを2回作りながら韓国語を生成する.{\bf 図5}は韓国語の組み立て
過程であり,辞書の検索による1:1変換,MFOLTを用いた様
相資質の表層化,特殊表現処理による訳語変更を表わす.{\bf 図5}の
(A)と(B)は文の前半と後半の生成処理を表わす.

\subsection{用言の不規則変化による音韻調和処理}
(金政仁 1996)には韓国語での用言の活用は語幹と語尾が一定な
形態に結合,活用する規則活用とそうでない11種類の不規則活用
に分けている.言い換えれば,11種類の不規則活用という表現よ
り12種類の活用規則が存在するとも言える.しかし,最近,韓国
語の情報処理技術は飛躍的に発達しており,特に形態素解析に関し
てはすでに効率よい多数の手法が発表されている(S.S.Kang 1995, 
D.B.Kim 1994).我々は解析手法をもとに12種類の活用に対応
できる韓国語の生成ルールを作り,本システムに反映させた.例文
には韓国語の用言「\hg{molyda} ({\it moluta})」が表われており,
これは「\hg{ly} ({\it Reu})不規則」に属し,
語尾の変化によって語幹の形まで変わる不
規則の一種類である.また,「\hg{molyda} ({\it moluta})」は陽性
\footnote{韓国語の母音は全部10個である.この中で「\hg{a,ia,o,io}」
は陽性であり,「\hg{e,ie,u,iu,y,i}」は陰性である.また,陽性は陽性,
陰性は陰性同士に結合しようとする性質(母音調和)を持っている.}
であり,母音調和の法則に従い,過去を表わす先語末語尾の中で
「\hg{'aS} ({\it ass})」と
連結される.そして,語幹「\hg{moly} ({\it molu})」は
「\hg{ly} ({\it lu})不規則」の属性により
「\hg{moly} ({\it molu}) + \hg{'aS} ({\it ass})
 $->$ \hg{morraS} ({\it mollass})」という形になる.
一方,冠形形語尾「\hg{n/'yn/nyn} ({\it n/un/nun})」は動詞と結合できる
語尾「\hg{nyn} ({\it nun})」と形容詞と結合できる「\hg{n/'yn} ({\it n/un})」
に区分できるので,
「\hg{'iS}(動詞) + \hg{n/'yn/nyn}(冠形形語尾)」は
「\hg{'iSnyn} ({\it issnun})」
の形態を取る.結局,生成された韓国語は
「\hg{gy/\{'i|ga\}/bohobad/go 'iSnyn/ges/\{nyn|'yn\}/morraS/da}」になる.


\subsection{閉鎖音素による異形態処理}
韓国語の単語は最後の音素が子音で終わる閉鎖音素であるか,
‘\hg{r}’終音素であるか,母音で終わる開放音素であるかによって結
合する助詞と語尾の形態が変わる場合がある.主な助詞の場合を{\bf 表4}に示す.

ここで,生成された韓国語の文
「\hg{gy/\{'i|ga\}/bohobad/go 'iSnyn/ges/\{nyn|'yn\}/morraS/da}」
から異形態の処理を行なうと,「\hg{gy} ({\it ku})」は開放音
素,すなわち母音で終わるので\hg{\{'i|ga\}}からは「\hg{ga}」が選択されてお
り,\hg{\{nyn|'yn\}}は閉鎖音素「\hg{ges} ({\it kes})」の影響で
「\hg{'yn}」が選ばれる.
結局,異形態処理が終わった韓国語は
「\hg{gy/ga/bohobad/go 'iSnyn/ges/'yn/morraS/da}」になる.

\clearpage

\begin{figure}[h]
\atari(120,190)
\caption{韓国語述部の組み立て}
\end{figure}

\clearpage

\begin{table}[htb]
\begin{center}
\caption{前接単語の終音素による助詞の異形態}
\begin{tabular}{|l|l|l|l|} \hline
助詞 & 閉鎖音素と結合   & `\hg{l}'終音素と & 開放音素と結合    \\ 
     & した場合         & 結合した場合     & した場合          \\ 
\hline 
が   & \hg{'i} ({\it i}) & \hg{'i} ({\it i}) & \hg{ga} ({\it ka}) \\ 
\hline
は   & \hg{'yn} ({\it un}) & \hg{'yn} ({\it un}) & \hg{nyn} ({\it nun}) \\ 
\hline
で   & \hg{'ylo} ({\it ulo}) & \hg{lo} ({\it lo}) & \hg{lo} ({\it lo})  \\ 
\hline
と   & \hg{goa} ({\it kwa}) & \hg{goa} ({\it kwa}) & \hg{'oa} ({\it wa})\\ 
\hline
を   & \hg{'yl} ({\it ul}) & \hg{'yl} ({\it ul}) & \hg{lyl} ({\it lul}) \\ 
\hline
\end{tabular}
\end{center}
\end{table}


\subsection{音韻縮約および分かち書き処理}
前段階までの処理を行なうことで,生成された訳文はだいたい読
んで解かる程度の韓国語文になる.より自然な表現と読みやすい韓
国語にするため,さらに音韻縮約と分かち書きを行なう.音韻縮約
は「\hg{'a} ({\it a})」脱落現象,「\hg{'ai} ({\it ay})」縮約現象,
「\hg{'oa} ({\it wa})」縮約現象,
「\hg{'i} ({\it i})」縮約現象,「\hg{'oi} ({\it oy})」縮約現象など
5つの現象が存在し,
本システムではそれに基づいて縮約ルールを作り,縮約処理を行な
っている.また,(黄燦鎬 1993)から11種類の分かち書きルー
ルも適用させ,完全な韓国語文に完成させる.例文には音韻縮約処
理が要らないが,普段は以上のような処理を経て,韓国語が生成さ
れるような仕組みになっている.結局,例文「彼が保護されている
ことは知らなかった」は韓国語の文
「\hg{gyga bohobadgo 'iSnyn ges'yn morraSda}
 ({\it kuka pohopatko issnun kesun mollassta})」に翻訳される.


\section{実験および評価}
本韓国語生成システムはMFOLTの最適化された様相資質から
表層語を抽出し,活用ルール12種類,縮約ルール5種類,そして
分かち書きルール11種類を適用して韓国語述部を生成する.提案
した方法を用いて生成処理を行なった場合,韓国語の述部がどのく
らい正しく生成できるかを確認するため,翻訳実験を行なった.翻
訳処理に使った辞書は日本語を専攻した辞書入力専門研究員6名に
よって市販の日韓辞書,電子辞書などを参考に作り出したものであ
り,現在,約12万単語が登録されている.

本稿での実験用の文章としては「新聞記事,日本語の文法本」2
種類を用意した.新聞記事を用いたのは,現時点で我々のシステム
が普段の文法的な誤りの少ない日本語の文章に対してどのくらいの
翻訳能力を持っているかを確かめるためである.また,日本語の文
法本を実験の対象にしたのは,普段はあまり表われないいろいろな
日本語の表現,文法などが記述された日本語の文章に対してはどの
くらいの翻訳ができるかを明らかにするためである.我々は「朝日
新聞1ヶ月分の記事」の経済,社会,国際,カラム,スポーツ面か
らランダムに1,042文,「基礎日本語文法(益岡隆志1995)」
にある1,296個のすべての例文を抽出して翻訳実験を行ない,
例文に表われた述部の総数と様相情報の総数,そして文単位の翻訳
結果として{\bf 表5},述部の様相資質単位の翻訳結果として{\bf 表6}を得た.


\begin{table}[htb]
\begin{center}
\caption{述部の翻訳結果}
\begin{tabular}{|c|r|r|r|}\hline
区分           & 新聞記事 & 日本語の文法本 & 合計 \\ \hline 
例文数         & 1,042    & 1,296          & 2,338\\ \hline
述部数         & 2,717    & 1,568          & 4,285\\ \hline 
様相情報数     & 7,698    & 4,133          & 11,831\\ \hline 
翻訳失敗文の数 & 103      & 59             & 162   \\ \hline
文の翻訳率     & 90.1     & 95.4           & 93.1  \\ \hline 
\end{tabular}
\end{center}
\end{table}

また,述部はMFOLTによって生成されており,実験文章に表
われたMFOLTの各様相資質を単位として,頻度数と翻訳失敗数
を数えた.(李義東1990)では結果データがないので直接比べる
ことはできないが,述部の生成部分は新聞記事から約96.9%,
日本語の文法本から約98.6%,合わせて約97.5%の高い成
功率が得られた.新聞記事より文法本の方が高いのは,1文当たり
の様相資質数が違い,その差から影響を受けたと思われる.文法本
の例文がほぼ短文の形式であって,様相情報の記述も相対的に少な
い数で表われたものが多かった.この結果から我々が提案したMF
OLTを用いた韓国語の生成方法は日本語文の多様な文形式にも十
分耐えることが分かった.残り2%〜3%の失敗は次のような理由
であった.

\vspace{5mm}
\noindent
{\bf ケース1)話し言葉から様相資質の抽出失敗による生成失敗}\\
\hspace*{4mm}対訳文に質問の意味がなくなってしまう.
\begin{quote}
甲:君,あの本読ん\underline{だ}.\\
\hg{gab}:\hg{janei, gy cayig 'ilg'eS}\underline{\hg{da}}. (X)\\
\hg{gab}:\hg{janei, gy cayig 'ilg'eS}\underline{\hg{ni}}? (O)
\end{quote}
\hspace*{4mm}対訳文に命令の意味がなくなってしまう.
\begin{quote}
(君が持っている本)ちょっと見\underline{せて}.\\
\hg{(janeiga gajigo 'iSnyn caig) jom bo}\underline{\hg{'igo}}. (X)\\
\hg{(janeiga gajigo 'iSnyn caig) jom bo}\underline{\hg{'iejue}}. (O)
\end{quote}
\vspace{5mm}

翻訳システムを話し言葉に適用させるためには,話し言葉の特
徴を明らかにする特別な研究や話しの流れを追跡管理する談話処
理分野の研究などが必要であると思われる(金政仁1993).

\vspace{5mm}
\noindent
{\bf ケース2)MFOLTを用いた方法論での生成失敗}
\hspace*{4mm}
\begin{quote}
彼女は結婚し\underline{ているらしかった}.\\
\hg{gynienyn gierhon}\underline{\hg{ha'ieSnyn ges gatda}}.  (X)\\
\hg{gynienyn gierhon}\underline{\hg{han ges gat'aSda}}. \ \ \ (O)\\
彼女は結婚\underline{していったらしい}.\\
\hg{gynienyn gierhon}\underline{\hg{ha'ieSnyn ges gatda}}.  (X)\\
\hg{gynienyn gierhon}\underline{\hg{haiSden ges gatda}}. \ \ \ \ (O)
\end{quote}
\vspace{5mm}

上記の2個の例文はテンスの部分で,事実の進行と完了,推測
の時点が過去と現在であることなど,その意味が微妙に違うが,
MFOLTには同じ様相資質が活性化されてしまい,述部を生成
するときにテンスの区別ができない.このケースは現在のMFO
LTを用いた生成方法では処理できない場合であり,1つの述部
から事実の様相情報と話者の様相情報が同時に表示できるように,
今後MFOLTを拡張するつもりである.

\vspace{5mm}
\noindent
{\bf ケース3)変換での多義性解消失敗による述部の生成失敗}\\
\hspace*{4mm}「れる,られる」の多義問題
\begin{quote}
この魚は刺し身では食べ\underline{られ}ない.\\
受け身)\hg{'i mulgoginyn sai'senhoilonyn}
\hg{meg}\underline{\hg{hi}}\hg{ji 'anhnynda}. (X)\\
可 能)\hg{'i mulgoginyn sai'senhoilonyn}
\hg{meg}\underline{\hg{'yl su}}\hg{ 'ebsda}. (O)
\end{quote}
\hspace*{4mm}「ている」の動作・状態進行の多義問題
\begin{quote}
高津さんは先週から神戸に\underline{来ている}.\\
動作)\hg{daGaJysa''yn jinan jubute go'obei'ei }
\underline{\hg{'ogo 'iSda}}. (X)\\
様態)\hg{daGaJysa''yn jinan jubute go'obei'ei }
\underline{\hg{'oa 'iSda}}. (O)
\end{quote}
\hspace*{4mm}「て(で)」の連続・連結・原因の多義問題
\begin{quote}
例年の夏に比べ\underline{て},今年の夏は涼しかった.\\
連続)\hg{'ieinien'yi 'ielym'ei bigioha}\underline{\hg{go}}, 
\hg{'olhai'yi 'ielym'yn si'uenha'ieSda}. (X)\\
連結)\hg{'ieinien'yi 'ielym'ei bigioha}\underline{\hg{'ie}}, 
\hg{'olhai'yi 'ielym'yn si'uenha'ieSda}. (O)
\end{quote}
\hspace*{4mm}連体形から生じる時制の多義問題
\begin{quote}
下記の学生は3時までに事務室に\underline{来る}こと.\\
現在)\hg{hagi'yi hagsai''yn 3 siGaji samusillo }\underline{\hg{'onyn}}
\hg{ ges}. (X)\\
未来)\hg{hagi'yi hagsai''yn 3 siGaji samusillo }\underline{\hg{'ol}}
\hg{ ges}. \ \ (O)
\end{quote}
\vspace{5mm}

{\bf 表6}の受け身,状態進行,連結1,連結2,冠形詞転成など,
比較的失敗の数が大きい様相資質が多義から生じた失敗であった.
多義による述部の生成失敗は,変換処理で多義が正しく解消され
なかったことから引き続いた失敗であるが,とにかく述部の対訳
語が正しく選定できなかったので,本評価には生成失敗として数
えた.しかし,このケースは変換部分での多義性解消の成功率と
緊密な関係があるので,その分野の技術が発展すれば,結果は好
転すると思われる.

\clearpage

\begin{table}[h]
\caption{様相テーブルの各資質別翻訳結果}
\atari(135,180)
\end{table}

\clearpage

\begin{table}[h]
\atari(135,52)
\end{table}

\vspace*{-7mm}

\section{結論}
日本語の述部と韓国語の述部は対応する品詞の不一致,局部的な
語順の不一致,活用語尾の不一致などの相違点を持っている.日韓
翻訳において両国語の述部の不一致を解決するため,我々は述部だ
けを対象とする意味的であり抽象的な様相資質をテーブル化して両
国語の述部表現の中間表現(PIVOT)とする方法を提案した.
すなわち,様相テーブル(MFOLT)に様々な様相資質を用意し,
生成処理のときに各資質を活性化する.そして,韓国語の接続情報
や各資質の状態を参考にして韓国語の述部を生成する.また,用言
の不規則変化処理,音韻縮約処理,活用語尾の異形態処理,そして
分かち書きを行ない,自然な韓国語の述部を生成する.新聞記事か
ら1,042文,文法本から1,296文を対象に生成テストを行
なった結果,述部に表われた様相資質の中で97.5%が正しく生
成され,日韓機械翻訳において韓国語の生成にMFOLTを用いた
方法が有効であることが分かった.




\bibliographystyle{jnlpbbl}
\bibliography{kim}

\begin{biography}
\biotitle{略歴}
\bioauthor{金 政仁}{}
\bioauthor{李 鐘赫}{}
\bioauthor{李 根培}{}


\bioreceived{受付}
\biorevised{再受付}
\bioaccepted{採録}

\end{biography}

\end{document}

