\documentstyle[epsf,jnlpbbl]{jnlp_j_b5}


\setcounter{page}{113}
\setcounter{巻数}{6}
\setcounter{号数}{6}
\setcounter{年}{1999}
\setcounter{月}{7}
\受付{1998}{9}{30}
\再受付{1998}{12}{21}
\採録{1999}{1}{21}


\setcounter{secnumdepth}{2}

\title{文章の構造化による修辞情報を\\利用した自動抄録と文章要約}
\author{比留間 正樹\affiref{IBM} \and 山下 卓規\affiref{TSB} 
\and 奈良 雅雄\affiref{HTC} \and 田村 直良\affiref{YNU}}

\headauthor{比留間,山下,奈良,田村}
\headtitle{文章の構造化による修辞情報を利用した自動抄録と文章要約}


\affilabel{IBM}{日本IBM株式会社}
{IBM Japan,Ltd.}

\affilabel{TSB}{株式会社 東芝}
{TOSHIBA Corporation}

\affilabel{HTC}{日立ソフトウェアエンジニアリング株式会社 ネットワーク本部 ネットワーク第1設計部}
{Network Systems Dept. Hitachi Software Engineering Co.,Ltd.}

\affilabel{YNU}{横浜国立大学 教育人間科学部 情報認知システム講座}
{Department of Information and Cognition Systems,
        Faculty of Education and Human Science}

\jabstract{
本論文では,
重回帰分析にもとづいた文章構造解析を利用した自動抄録手法とその評価,およ
び文章要約への展開について述べる.文章構造の解析は,文章中の様々な特徴を
パラメタとした判定式や局所的な言語知識により,文章セグメントの分割統合を
進めて構造木を作るものである.得られた文章構造上の各種特徴をもとに,さら
に文章抄録の観点から選択されたパラメタを加えて,文抽出のための判定式を作
る.本研究では被験者5人にのべ350編の新聞社説の抄録調査を実施し,これを基
準に,重回帰分析によりパラメタの重みを決定し判定式を得,また,本方式を評
価する.また,自動生成された抄録文に対して,照応情報の欠落による文章の首
尾一貫性の低下を避けるための補完や,論旨を損なわない冗長な表現の削除を行
なうことで要約文章を生成する手法を紹介する.
}

\jkeywords{文章抄録,文章要約,パラメタ,学習,文章構造,重回帰分析,修辞構造}

\etitle{Extract Generation using Rhetorical\\ Information
by Text Structuring \\
and its Extension to Summary Generation }
\eauthor{Masaki Hiruma \affiref{IBM} \and Takumi Yamashita
\affiref{TSB} \and Masao Nara \affiref{HTC}  \and
Naoyoshi Tamura\affiref{YNU}
}

\eabstract{
This paper presents a method of automatic extract generation from text
by using the rhetorical structure of the text which is built by 
discriminants trained by the multiple regression analysis.  We also show
an extension to the summary generation. The way of our method of text
structuring is to divide or to combine text segments one after another
according to the discriminant with various parameters for superficial
features of sentences. Another discriminant to extract important
sentences has parameters selected from the point of view of text
extraction along with several features on the text structure (rhetorical
structure). In the experiment, five examinees select important sentences
of newspaper editorials from  total 350 articles.  The results are used
to the calculation of weights of parameters by the multiple regression
analysis, and also to the evaluation of accuracy of our method. We also
show an extension to summary generation in which sentences or phrases
are restored not to decrease the coherency of text caused by deletion of
referred sentences, and in which redundant expressions independent to
the context are deleted.
}

\ekeywords{extract, summary, parameter, learning, text structure,
multiple regression analysis, rhetorical structure}


\begin{document}
\maketitle
\section{はじめに}

本論文では,文,文章上の特徴,および文章の解析により得られた構造上の特徴
をパラメタとして用いた判定式による文章の自動抄録手法を示す.さらに,抽出
された文の整形や照応を考慮した文章要約手法について述べる.

近年のインターネットなどの発展により,大量の電子化された文書が我々の周り
に溢れている.これら大量の文書から必要とする情報を効率良く高速に処理する
ために,キーワード抽出や文章要約,抄録といった研究が行なわれている.それ
らのためには,計算機を用い,必ずしも深い意味解析を行なわずに文章の表層的
特徴から解析を行なう方法が有効である.

文章抄録とは文章から何らかの方法で重要である文を選び出し,抽出することで
ある.山本ら\cite{Masuyama:95}は照応,省略,語彙による結束性など多くの談
話要素から重要文を選択していく論説文要約システム(GREEN)を発表している.
このシステムは談話要素を利用したものではあるが,文章の局所的な特徴を基に
文を抽出するもので,本研究の立場からすれば文章全体の構造に基づく抽出と,
電子化された大量のコーパス利用を考慮した抽出手法や手法の評価が必要と考え
る.

また,亀田\cite{Kameda:97}は重要文の抽出の際に文章の中で小さなまとまりを
示す段落や,一種の要約情報である文の見出しに着目する手法を提案,実現して
いるが,重要度計算の調整は人手により,系統的でないところが感じられる.

さて,重要文の抽出に用いられるテキスト中の表層的特徴については,
\cite{Okumura:98}にサーベイがある.これによると,Paice\cite{Paice:90}の
分類として,(1)キーワードの出現頻度によるもの,(2)テキスト,段落中の位置
情報によるもの,(3)タイトル等の情報によるもの,(4)文章の構造によるもの,
(5)手がかり語によるもの,(6)文や単語間のつながりによるもの,(7)文間の類
似性によるものがあげられている.本研究での手法は,上記のかなりの要素を組
み合わせてパラメタとして利用している.

いくつかの観点からのパラメタを組み合わせるという同様な手法として,
\cite{Watanabe:96},\cite{Nomoto:97}がある.それぞれ,重回帰分析,決定木
学習により訓練データから自動学習するものである.われわれの手法は,構造木
に関する情報を特に重視している.

人間は,目的の意見,主張を読み手に伝えるために,意識下/無意識下に文章構
成の約束に基づいて文章生成を行なっているが,それらの文章に論証性を持たせ
るためのものが文章構造である.また逆に,文章を理解し論旨を捉える際に文章
構造を活用していると考えられる.したがって,文章の抄録にあたり,論旨を捉
え,文章構造を理解した上で重要文を抽出していく手法は人間の文章抄録の流れ
に沿っており,ごく自然であると考えられる.実際,\cite{Marcu:97}では,人
間の手による生成ではあるが文間の関係を解析した修辞構造生成後の文抽出の再
現率,適合率は良好と報告されている.われわれの手法でも,修辞構造を含めた
文章構造解析による情報を利用する.

文章構造解析には田村ら\cite{Tamura:98}の分割と統合による構造解析手法を利
用する.文章抄録には,構造解析で用いたパラメタに加えて,得られた文章構造
上の情報についてのパラメタにより文抽出のための判定式を作り,それを基 
にして抄録を作成する.判定式とパラメタの重みの決定は重回帰分析に基づき, 
その訓練のため,およびシステムの評価のための基準データは,被験者に対する 
のべ350編の抄録調査による.なお,実験の対象とした文章は,均一な文章が容
易に入手可能であるとの理由から,新聞の社説を用いる.

一方,原文から単に文を選ぶだけの文章抄録では,選択された文間の隣接関係が
不自然になる場合がある.また,たとえ選択された一文でも文内には冗長な表現
が残っている場合がある.そこで,自動要約に向けては,抄録後になんらかの文
章整形過程が必要である.本研究では,抄録の整形過程としての照応処理と,一
文の圧縮処理を行なう.

以下,第2章では文章抄録,要約のための文章構造解析について述べ,第3章では 
文章の自動抄録の手法について説明する.第4章では,提案の手法について再現
率 ,適合率により評価検討を行う.最後に付録として,抄録の整形過程につい
て述べ,実際に要約した文章例を示す.

\section{分割と統合による文章構造解析} \label{structure}

文章とはある首尾一貫性を持った文の集まりである.それゆえ文章を理解し,抄
録を作成するためには,その文章がどのような構成(構造)になっているかを知る
必要がある.よってここでは文章の構造解析について述べる.

文章の構造化としてはMann\cite{Mann:87:a}の修辞構造理論を基にして論旨の展
開を木構造として表現する手法\cite{Tamura:98}を用いる.これは大量の文章を
高速に処理するために深い意味解析に立ち入らず,主に表層的な処理のみでセグ
メントの分割(トップダウン的アプローチ),およびセグメントの統合(ボトムアッ
プ的アプローチ)を行なっている.こうすることで,文章の木構造を根から葉へ,
かつ葉から根へと交互に生成してゆき,一方の欠点を他方の利点で補う効果的な
文章解析を行なうことができる.この解析処理の例を摸式的に示したのが図
\ref{fig:kaiseki}である.

\begin{figure}[hbtp]
 \begin{center}
  \epsfile{file=tb_model.eps,scale=0.7}
  \caption{分割と統合による構造解析}
  \label{fig:kaiseki}
 \end{center}
\end{figure}

トップダウン的解析では,望月ら\cite{Mochiduki:96}の文章の表層表現を情報
とした,テキストセグメンテーションの手法を利用して文章の分割を行なう.
すべての文間について判定式による評価値を求め,評価値の高い順にセグメント
の分割を行なう.また,ボトムアップ的解析でも判定式を用い,すべての文間に
ついて評価値を求め,評価値の高い順にセグメントの統合を行なう.それぞれの
判定式のパラメタの重みは,訓練データ\footnote{日本経済新聞の社説100編}の
テキスト中のすべての文と文の境界についてパラメタを評価し,重回帰分析によ
り求める.

パラメタは以下の観点から選択する\cite{Mochiduki:96}(詳細は\ref{hantei}の
1(単独の文に関するもの)参照).

\begin{itemize}
 \item 助詞「は」と「が」の出現
 \item 接続語の有無
 \item 指示詞の有無
 \item 時制の情報
 \item 文末表現の情報
 \item 語彙連鎖情報
\end{itemize} 

本手法によって新聞社説\footnote{日本経済新聞1993年1月28日(付録に示す)}を
解析してできた修辞構造木の例を図\ref{fig:tb_tree}に示す.

\begin{figure}[htb]
 \begin{center}
  {\footnotesize
  {\baselineskip=11pt
  \begin{verbatim}      
    [ ]
     |-(1,1)
     |-順接
     |-[ ]
        |-[ ]
        |  |-[ ]
        |  |  |-[ ]
        |  |  |  |-[ ]
        |  |  |  |  |-[(2,1),順接,(2,2)]
        |  |  |  |  |-逆接
        |  |  |  |  |-[[(3,1),順接,(3,2)],順接,(3,3)]
        |  |  |  |
        |  |  |  |-順接
        |  |  |  |-(4,1)
        |  |  |  
        |  |  |-順接
        |  |  |-[ ]
        |  |     |-[ ]
        |  |     |  |-[[(5,1),順接,(5,2)],順接,(5,3)]
        |  |     |  |-順接
        |  |     |  |-(5,4)
        |  |     |
        |  |     |-順接
        |  |     |-(5,5)
        |  |
        |  |-順接
        |  |-[ ]
        |     |-[[(6,1),順接,(6,2)],順接,(6,3)]
        |     |-順接
        |     |-[(7,1),順接,(7,2)]
        |
        |-順接
        |-[[(8,1),結論,(8,2)],転換,(8,3)]
  \end{verbatim}
  \caption{修辞構造解析結果の例}
  \label{fig:tb_tree}
  }}
 \end{center}
\end{figure}

この修辞構造木において\verb|(m,n)|は1文に対応し,
\verb|m|が原文章の形式段落の番号,
\verb|n|が形式段落内の文の番号を表す.

また,「順接」,「強調」,「説明」等は隣接するセグメント間の修辞関係
\footnote{それ以外に「逆接」,「換言」,「添加」,「条件」,「結論」,「一般化」,
「相反」,「提起」,「根拠」,「因果」,「並列」,「選択」,「対比」,「転換」,
「例示」がある.}を表している\cite{Mann:87:a}.修辞関係は,cue wordによ
り文頭の接続表現を18種,文末表現を14種に分類し,接続表現があればそれによ
り同定し,なければ,隣接する前文,後文の文末表現その他の組み合わせをもと
に決定している.それぞれの関係に応じて,どちらが核(nucleus),衛星
(satellite)であるかも決定する.また,セグメント間の修辞関係の同定は,核
をたどることによってセグメントを代表する1文を求め,これらの文間の関係に
より同定する.なお,「順接」をdefaultとしている.(詳細については,
\cite{Tamura:98}を参照されたい.)

\section{文章抄録}\label{shouroku}

本論文で提案する抄録手法は,まず文章の構造化を行ない,得られた修辞構造木 
の情報やその他抄録の観点から選択されたパラメタに基づく判定式により各文 
の重要度を算出し,重要度の高い文から抽出していく.判定式中のパラメタの重 
みは被験者が作成した抄録について重回帰分析により訓練する.

また,本研究では,文だけでなく構造木の部分木の削除も行なう手法についても
検討する.部分木とは,根に近い枝から順に構造木を分割したもののリスト構造
で,文章の段落に相当するものである.部分木の数は,社説の全文数を形式段落
内の文の数の平均2.69 \footnote{1993,1994年の日本経済新聞の社説1227編か
ら求めた.}で除することにより求める.この処理を摸式的に示したのが図
\ref{fig:sub_tree}であり,構造木の根に近い節点から(1,2,3)順に分割していっ
たものが部分木である.実際に図\ref{fig:tb_tree}の構造木から再構成される
部分木の例を図\ref{fig:part_tree}に示す.この例では文の数が20個なので,
それを2.69で割り,部分木の数が7となる.

\begin{figure}[hbtp]
 \begin{center}
 \epsfile{file=sub_tree.eps,scale=0.9}
  \caption{構造木の部分木}
  \label{fig:sub_tree}
 \end{center}
\end{figure}

 \begin{figure}[htbp]
  {\footnotesize
  {\baselineskip=11pt
  \begin{verbatim}			     
    [ ]
     |-(1,1)
     |
     |-[ ]
     |  |-[ ]
     |  |  |-[(2,1),順接,(2,2)]
     |  |  |-逆接
     |  |  |-[[(3,1),順接,(3,2)],順接,(3,3)]
     |  | 
     |  |-順接
     |  |-(4,1)
     |
     |-[ ]
     |  |-[[[(5,1),順接,(5,2)],順接,(5,3)],順接,(5,4)]
     |  |-順接
     |  |-(5,5)
     |
     |-[(6,1),順接,(6,2)]
     |-(6,3)
     |-[(7,1),順接,(7,2)]
     |-[[(8,1),結論,(8,2)],転換,(8,3)]
  \end{verbatim}
  }
  \vspace{-6ex}
  \caption{構造木から再構成される部分木の例(図2の構造木より)}
  \label{fig:part_tree}}
 \end{figure}


\subsection{抄録手法}\label{method}

本研究で提案する抄録手法は,基本的には\ref{structure}節で述べた文章の構
造化を行ない,その構造上の情報と\ref{hantei}節で述べるパラメタより重要文
を選択,抽出するものである.ここでは,さらに文の抽出の前に部分木を削除す
る過程を加えた手法も検討対象とする.また,構造化は行なわずに判定する手法
も加え,以下の3種の手法を検討する.

\begin{description}
 \item[手法1] (文章構造解析 + 部分木削除 + 重要文抽出による抄録) \\
	    この手法では,(i)文章構造解析により修辞構造木を作成し,作成
	    された修辞構造木をさらに部分木に分割する.
	    (ii)各部分木の重要度を後出の式(\ref{import})を用いて算出し,
	    重要度の低い部分木から削除していく.
	    (iii)残った各文の重要度を算出し,重要度の高い文から抽出して
	    いく.という3つの段階を踏む.
	    (ii),(iii)での削除,抽出の量は実験時に設定する.
 \item[手法2] (i) + (iii) \\
	    この手法では,パラメタを採集するために文章の構造化,さらに部 
	    分木の再構成を行い,その後で構造上の情報その他を用いて各文の 
	    重要度を算出し,抽出する.
 \item[手法3] (iii) \\
	    この手法では,文章の構造化は行なわずに,単独の文個々について 
	    のパラメタにより各文の重要度を算出し,抽出する.
\end{description}


\subsection{判定式を用いた自動抄録}\label{hantei}

観測点の各種パラメタを基にそれが属するクラスを判定する手法は
いくつかあるが,本研究では重回帰分析を用いる.

判定のための式は以下である.
ただし,採用,不採用の判定は,対象文章の各文を式(\ref{import})
により評価し,評価値が大きいものから,順に所定の要約率となるまで
採用することとする.

\vspace{-6ex}
\begin{center}
 \begin{equation}
  \hat{z}_{sum}=\delta_{0}+\delta_{1}y_{1}+\delta_{2}y_{2}+\cdots +\delta_{q}y_{q}
   \label{import}
   \vspace{-2ex}
 \end{equation}
 ($y_{i}$:パラメタ$i$の点数,$\delta_{i}$:パラメタ$i$の重み)

 文の重要度算出の判定式
\end{center}

各文のパラメタとして以下を用いる.

\begin{enumerate}
 \item 単独の文に関するもの
\begin{itemize}
 \item 文末表現(3種)\\
       表層表現から文を「意見」,「断定」,「陳述」の3タイプに分類し,
       該当するパラメタを1とする.

 \item 時制(1種)\\
       文を「現在」,「過去」に分類し,「現在」ならパラメタを1とする.

 \item 参照表現(1種)\\
       文に「これ」,「その」,「あれ」などの指示語による参照表現が存在
       すれば1とする.

 \item 助詞「は」と「が」の出現(2種)\\
       文に助詞「は」,「が」が存在すれば,該当するパラメタを1とする.

 \item 接続詞(2種)\\
       「補足」的,「展開」的な接続詞が存在すれば,該当するパラメタを1と
       する.

 \item 文に含まれる重要語の数(1種)\\
       キーワード抽出で用いられるtf*idf法\footnote{実験では,idf値は,
       1993,1994年の日本経済新聞の社説1227編より求めた.}により抽出され
       た重要語の出現数を値とする.

 \item 文を通過する重要語の語彙連鎖の数(1種)\\
       語彙連鎖を形成する重要語の出現数を値とする.
       ただし,同義語はまとめて1語とし,連鎖中の短いギャップに対しては重
       要語が存在しなくても数に加える.

 \item 文の位置(2種)\\
       文章の先頭からの距離,終わりからの距離を値とする.
\end{itemize}

 \item 構造木上の情報に関するもの
\begin{itemize}
 \item 修辞関係の核(nucleus)または衛星(satellite)であるかどうか(34種
       \footnote{訓練の結果,使用されなかったものもある.})
       \\
       各パラメタは「順接」を除く修辞関係17種それぞれについての核,衛星
       に対応する.文が文章構造解析で得られた修辞関係の核,衛星であれば
       対応するパラメタを1とする.

\end{itemize}

 \item 部分木(段落)の情報に関するもの

\begin{itemize}
 \item 文の位置(6種)\\
       文が含まれる部分木の位置,部分木中の位置,など.

\end{itemize}


 \item 部分木の削除についても,以下のパラメタを持つ判定式を用い,重回帰
       分析により訓練する.

\begin{itemize}
 \item 部分木内の文の文末タイプの情報(3種)\\
       部分木内で,「意見」,「断定」,「陳述」についてそれぞれのタイプ
       の文が出現する割合を値とする.

 \item 部分木内の文に含まれる重要語の数(1種)\\
       前述の重要語の数.
       
 \item 部分木内の文を通過する重要語の語彙連鎖(1種)\\
	同様に,重要語の語彙連鎖の数.

 \item 文章の開始または終りからの距離(2種)\\
       この部分木(段落)が,初めまたは終わりから幾つめの部分木であるか.

 \item 部分木内の文の数(1種)\\
       部分木(段落)内に含まれる文の数.
\end{itemize}
\end{enumerate}


\section{実験と検討}\label{experiment}

\subsection{パラメタの訓練}

実験にあたって,まず訓練と評価のために人間の被験者の抄録結果による正解を
定める.抄録は,学生5人\footnote{本学学部1年生.自然言語処理についての教
育は特に受けていない.}を被験者として,日本経済新聞CD-ROM'93〜'94版の中
の社説のべ350編について重要であると思う文を全文数の3割程度選択させる.

調査用社説は表\ref{tab:set}のようにA〜Lのセットに分け,A〜Fセットを訓練
用セット,G〜Lセットを評価用セットとする.被験者と調査させたセットの対応
を表\ref{tab:set2}に示す.

{\small
 \begin{table}[htbp]
  \begin{center}
   \begin{tabular}{|c|c||c|c|} \hline
    訓練セット & 記事数 & 評価セット& 記事数 \\ \hline \hline
    A & 20 & G & 10 \\ \cline{1-2}
    B & 30 & H & 10 \\ 
    C & 30 & I & 10 \\ 
    D & 30 & J & 10 \\ 
    E & 30 & K & 10 \\ \cline{3-4}
    F & 30 & L & 10 \\ \hline
   \end{tabular}
  \end{center}
  \caption{調査セットと記事数}
  \label{tab:set}
 \end{table}
}


{\small
 \begin{table}[htbp]
  \begin{center}
   \begin{tabular}{|c|c|} \hline
    被験者  & 調査セット \\ \hline \hline
    1 & A,B,G,L \\ \hline
    2 & A,C,H,L \\ \hline
    3 & A,D,I,L \\ \hline
    4 & A,E,J,L \\ \hline
    5 & A,F,K,L \\ \hline
   \end{tabular}
  \end{center}
  \caption{調査セットの被験者への配分}
  \label{tab:set2}
 \end{table}
}

調査の結果,全文数8164,選択された文数2172,
平均要約率26.6\%となった.

まず,Aセット20編(以下,訓練セット1),B〜Fセット合計150編(以下,まとめて
訓練セット2)の2セットについてそれぞれ重回帰分析により,式(\ref{import})
におけるパラメタの重みを決定する.それぞれは,次のような方針に基づく.

\begin{description}
 \item[訓練セット1] 
	    訓練セット1の基準値としては各文についてその文が重要であると
	    した被験者の人数を与える.つまり,各文に$0〜5$の値を与える.
	    この訓練セットは同一の文章を多人数で調査することによって判定
	    の質を重視した訓練セットである.
 \item[訓練セット2] 
	    訓練セット2の基準値としては各文について担当した被験者が重要
	    であるとした文に$1$,そうでない文に$0$の値を与える.この訓練
	    セットは標本集合を大きくすることによって訓練の規模を重視する
	    訓練セットである.
\end{description}

手法1では,部分木の選択においても重回帰分析による手法を用いるが,被験者
には文単位での重要文選択しか行なわせていないため,あらかじめ前述の手法に
より各社説の構造化を行ない,部分木を作成する.その上で,被験者が選択した
文が部分木に含まれていればその部分木の観測値を$10$,含まれていなければ
$0$とする.これを正解として重回帰分析によりパラメタの重みを決定する.

\subsection{生成された抄録文の評価}

抄録は\ref{method}節に示した3種類の抄録手法で作成し,さらに手法1について
は部分木の削除の段階で,部分木(段落)を50\%,25\%,10\%ずつ削除した場合に
ついてそれぞれ抄録を作成する.抽出する文の数は,人間の重要文選択調査にお
ける平均要約率26.6\%を目安とする.

評価セットG〜K,および評価セットLに対する実験結果から,それぞれについて
以下の式により,再現率,適合率を求める.

{\small
\[
 再現率 = \frac{自動抄録により得られた抄録文に含まれる正解文数}{正解文数(人間によって選択された文数)}
\]
\[
 適合率 = \frac{自動抄録により得られた抄録文に含まれる正解文数}{自動抄録により得られた抄録文数}
\]
}


ここで,正解とは以下のように定義する.

評価セットG〜Kについては,個別に評価するのではなくひとまとまりとして扱い,
担当した被験者が選択したかどうかによるもとのする.

評価セットLについては,被験者5人中2人以上が重要であると判断された文を抄
録の正解とする.

評価セットG〜Kに対する再現率,適合率を表\ref{tab:saigen_tekigou1}に,
評価セットLに対する再現率,適合率を表\ref{tab:saigen_tekigou2}に示す.

{\footnotesize
\begin{table}[hbtp]
 \begin{center}
  \tabcolsep=5mm
  \begin{tabular}{|c|c||c|c|c|c|} \hline
         & 部分木   & \multicolumn{2}{|c|}{訓練セット1で訓練} & \multicolumn{2}{c|}{訓練セット2で訓練} \\ \cline{3-6}
         & の削除率 & 再現率 & 適合率 & 再現率 & 適合率 \\ \hline \hline
         & 50\%     & 26.9\% & 36.9\% & 25.3\% & 36.7\% \\ \cline{2-6}
   手法1 & 25\%     & 31.9\% & 37.4\% & 30.9\% & 36.4\% \\ \cline{2-6}
         & 10\%     & 33.1\% & 36.4\% & 35.3\% & 38.8\% \\ \hline
   手法2 &  $-$     & 33.4\% & 34.5\% & 37.5\% & 38.7\% \\ \hline
   手法3 &  $-$     & 32.8\% & 33.9\% & 34.1\% & 35.2\% \\ \hline
  \end{tabular}
 \end{center}
  \caption{評価セットG〜Kに対する再現率・適合率}
  \label{tab:saigen_tekigou1}
\end{table}
}

{\footnotesize
\begin{table}[hbtp]
 \begin{center}
  \tabcolsep=5mm
  \begin{tabular}{|c|c||c|c|c|c|} \hline
         & 部分木   & \multicolumn{2}{|c|}{訓練セット1で訓練} & \multicolumn{2}{c|}{訓練セット2で訓練} \\ \cline{3-6}
         & の削除率 & 再現率 & 適合率 & 再現率 & 適合率 \\ \hline \hline
         & 50\%     & 33.3\% & 40.7\% & 34.7\% & 42.4\% \\ \cline{2-6}
   手法1 & 25\%     & 31.9\% & 39.0\% & 34.7\% & 42.4\% \\ \cline{2-6}
         & 10\%     & 33.3\% & 40.7\% & 35.4\% & 43.2\% \\ \hline
   手法2 &  $-$     & 33.3\% & 40.7\% & 34.7\% & 42.4\% \\ \hline
   手法3 &  $-$     & 40.3\% & 49.2\% & 33.3\% & 40.7\% \\ \hline
  \end{tabular}
 \end{center}
  \caption{評価セットLに対する再現率・適合率}
  \label{tab:saigen_tekigou2}
\end{table}
}

\subsection{抄録文に対する検討}

\begin{itemize}
 \item 評価セットG〜Kに対する検討 \\
       再現率に関しては手法1の場合が低い値となった.これは部分木の削除の
       際に被験者が選択した文が含まれていても,その文を含む部分木の重要
       度評価値が低かったために部分木が削除されてしまい,抄録文として抽
       出されなかったためと思われる.

       訓練セット別に見ると訓練セット1より訓練セット2で訓練した方が,高
       い値となっている.これは訓練セット1の文章の数が社説20編と少なかっ
       たからと考えられる.

       適合率に関してはすべての手法でほぼ同じ値となった.訓練セット間の
       比較では訓練セット1の方が低い値となったが,再現率のときと同様の理
       由と思われる.

       抄録手法別に見ると再現率・適合率ともに手法1(10\%)と手法2が高い値
       となった.これは本研究で提案する文章構造の情報をパラメタに含めた
       判定式による抄録手法が有効であることを示している.また,過度に部
       分木を削除することの不適切性を示している.

 \item 評価セットLに対する検討 \\
       再現率・適合率ともに評価セットG〜Kに対する結果に比べ高い値となっ
       た.これは特異的に優しいセットだったためだと思われる.あるいは,
       被験者5人の判定を重ね合わせているので判定の質がアップしていること
       も寄与していると考えられる.

       訓練セット別に関してと抄録手法別に関しては,前述の評価セットG〜K 
       に対する結果と同様の傾向を示した.

       また,訓練セット1と評価セットLの組合せでは手法3の再現率・適合率が
       他より高いが,これは訓練セット,評価セット共に少ないため信頼でき
       る評価ができないためで,より多くの評価セットで評価を行なう必要が
       あると思われる.

 \item パラメタ学習に関する検討\\
       セットAで訓練し,それを評価セットとして抄録を作成したときの再現率,
       適合率を表\ref{tab:saigen_tekigou3}に示す.ただし,正解は被験者5
       人中2人以上が重要であると判断された文とする.

{\footnotesize
\begin{table}[hbtp]
 \begin{center}
  \tabcolsep=5mm
  \begin{tabular}{|c|c||c|c|} \hline
         & 部分木   & \multicolumn{2}{|c|}{訓練セット1で訓練} \\ \cline{3-4}
         & の削除率 & 再現率 & 適合率 \\ \hline \hline
         & 50\%     & 41.1\% & 54.3\% \\ \cline{2-4}
   手法1 & 25\%     & 42.3\% & 55.9\% \\ \cline{2-4}
         & 10\%     & 42.9\% & 56.7\% \\ \hline
   手法2 &  $-$     & 42.9\% & 56.7\% \\ \hline
   手法3 &  $-$     & 39.3\% & 52.0\% \\ \hline
  \end{tabular}
 \end{center}
 \vspace{2mm}
  \caption{セットAで訓練,評価したときの再現率・適合率}
  \label{tab:saigen_tekigou3}
\end{table}
}

       本来,訓練セットそのものを評価した場合,再現率,適合率とも100\%に
       なってもよさそうなものである.表\ref{tab:saigen_tekigou3}の値は,
       訓練の限界を示す目安になっており,先の評価結果は,再現率ではこの
       値にかなり近付いている.

 \item 人間の判断のばらつきに関する検討\\
       被験者5人に対して,重要文選択調査した社説のうち4人の結果を正解と
       して,残り1人の結果と比較する.以上の実験を順に5人すべてに対して
       評価を行なった結果を表\ref{tab:only}に示す.なお,正解は4人中2人
       以上が選択した文とする.

       人間同士の比較でも再現率,適合率は表\ref{tab:only}で表される程度
       にばらつきがある.ただし,再現率,適合率間でそれほど大きな差はな
       い.

       自動抄録についての研究の一つの目標を示しているものと考えられる.

{\footnotesize
\begin{table}[hbtp]
 \begin{center}
  \begin{tabular}{|c|c|c|c|c|c|c|} \hline
     被験者  & 1 & 2 & 3 & 4 & 5 & 平均  \\ \hline \hline
     再現率($\%$) & 61.1 & 61.8 & 53.5 & 48.6 & 55.9 & 57.2  \\ \hline
     適合率($\%$) & 56.4 & 67.2 & 67.9 & 58.4 & 68.6 & 63.7  \\ \hline
     要約率($\%$) & 30.8 & 26.3 & 24.2 & 24.9 & 24.8 & 26.2  \\ \hline
  \end{tabular}
 \end{center}
 \vspace{2mm}
  \caption{被験者の抄録についての評価}
  \label{tab:only}
\end{table}
}
\end{itemize}

以上の結果より,本研究で提案する文章構造情報を利用した抄録手法が,利用し
ない抄録手法より有効であると言える.また,文章構造木を部分木に分割し,重
要文抽出の前段階で部分木を削除する手法も提案したが,小規模な部分木削除
(10\%)でもっとも良好な結果を示している.

部分木の削除を考慮する場合,周囲との関連がない重要な文の存在が結果の精度
に大きく影響する.すなわち,ノイズ的にパラメータの評価値が高くなってしま
う文では,部分木の削除を行う場合(手法1),これが削除され,精度は向上する.


\section{おわりに}\label{conclusion}

本論文では,文章中のさまざまな特徴をパラメタとした判定式から文章の構造化
を行ない,さらに文章抄録の観点から選択したパラメタを加えた判定式による自
動抄録手法を提案,実現し,評価,検討した.

従来の抄録手法は文章中の情報を用いて各文について重要度を算出し,重要度の
高い文から順に抽出していくものが多かった.本論文における手法も同様のもの
であるが,判定式に用いるパラメタは抄録作成の観点から選択したものに加えて,
文章構造解析による構造木の修辞関係や部分木の位置などの構造的なものも選択
した.パラメタの重みは,人間による重要文抽出調査の結果をもとに重回帰分析
により求めた.また,文章構造の情報を考慮した場合としない場合での比較を行
なうために,3種類の手法で抄録を作成し,検討を行なった.結果からパラメタ
として文章構造の情報を使用する方が使用しない場合よりも良い結果が得られる
ことが分かった.また,個人差があっても大規模な訓練セット(150編)を用いた
方が,同一の記事を複数人で調査した訓練セット(20編)を用いるよりも結果は良
好であった.

なお,開発途中であるが,照応解析および一文内の圧縮をすることによる抄録か 
ら要約への展開および自動要約の実現を付録に示した.

今後の課題として,他の学習方法についての検討,より大きい訓練セットでの実
験,抄録文章の整形過程の検討,また,システムとしてのユーザインタフェース
の整備があげられる.

\section*{謝辞}

本実験で使用したコーパスは,日本経済新聞CD-ROM'93〜94版から得ている.
同社,および使用に関して尽力された方々に深く感謝します.


\newpage
{\large \bf 付録(文章要約)}

\vspace{5mm}

抄録文章は,原文章から文を抜き書きしたものであり,参照表現,接続詞等も原
文章のままであるので,そのままでは人間が読むときに不自然さを感じる.要約
とは,より少ない文字数で原文章と同じ内容を表現することであるが,1つの文
章として文を越えた何らかの意味的なつながり,つまり論旨が成り立っている必
要がある.

そこで本研究では,照応関係を解析し,その情報を利用することで照応情報の欠
落による抄録文の首尾一貫性の低下を避ける.また,文内で比較的重要度が低い
と考えられる表現を削除することで要約文章の圧縮を行なう.

本方法では,対象とする照応表現は指示詞および指示連体詞とし,以下のような
照応関係の判定を行なう.

\begin{itemize}
 \item 指示対象が直前文である.
 \item 指示対象が前段落(部分木)である.
 \item 指示対象が前出の名詞句である.
 \item 指示対象が文章内に存在しない.
\end{itemize}					  

判定の結果,指示対象が文章内に存在する場合,指示対象を抄録文に採用する.
ただし,指示対象が名詞句の場合はその名詞句を含む文を指示対象と判断する.

抄録処理で文を抜き出すだけでは,要約文章として冗長な部分があったり,文間
の接続が不自然な場合がある(結束性の悪化).そこで,省略することによって結 
束構造を発生することができる主題や文の論旨に影響を及ぼさない表現を削除す 
ることによって,結束性を高めるとともに,一文の量的な圧縮を行なう.

本方法では,以下のような表現を削除の対象とする.

\begin{description}
 \item[削除対象となる表現] \
	    \begin{itemize}
	     \item 接続詞(文頭にある場合のみ)
	     \item 修飾句
 			     \begin{itemize}
 			      \item 固有名詞への修飾句
 			      \item 例示句
 			      \item 話括弧``「」''の前語句
			      \item 時間格
 			     \end{itemize}
	     \item 副詞
 			     \begin{itemize}
 			      \item 量副詞(「ほとんど」,「ほぼ」等)
 			      \item 程度副詞(「きわめて」,「かなり」等)
 			      \item 時制相副詞(「まだ」,「もう」等)
 			     \end{itemize}
	     \item 丸括弧``()''および丸括弧内語句
	     \item 主題(維持されている場合のみ)
	    \end{itemize}
\end{description}

以上の要約手法によって生成された要約文章の例を,原文章,抄録文章(下線部
分)とともに以下に示す.

\vspace{5mm}

{\baselineskip=15pt
{\bf 原文章と抄録文章(下線部分)}
{\normalsize
{\baselineskip=12pt
\begin{itemize}
\item [] 日本経済新聞 93年1月28日の社説
\item []改憲論の前に日本の“自画像”を描け(社説)
\item [] \underline{(1,1): 通常国会の代表質問が二十七日終わったが,今回
の特色は与野党の党首や主要役}\\
\underline{員級の質問者が異なった発想に基づく憲法改正論を展開,
慎重な姿勢をとった宮沢首相}\\ 
\underline{との間で踏み込んだ論議を交わした点にある.}
\item [] \underline{(2,1): もとより憲法には改正条項があり,その長所と
短所を点検し,論議を深めること}\\ \underline{自体は望ましいことである.}
\item [] (2,2): この意味で「神棚に上げず日常的に議論することは,憲法を
自分のものとするためにも有意義だ」(首相)といってよい.
\item [] (3,1): しかし,まず改憲ありき,といった前提で衆参両院に憲法に
関する協議会を設置し,直ちに憲法九条を中心とする改正の具体化に進むとい
う提案(三塚自民党政調会長)はいささか性急にすぎるというほかない.
\item [] (3,2): まず,広く国民各層,各界の間で,国際貢献,人権,環境権,
地方分権など幅広く総点検を進め,じっくりと時間をかけて日本の国づくりの
方向はどうあるべきかを徹底的に議論するのが大事である.
\item [] (3,3): その過程から,法律改正で足りるものと,憲法改正なしには
改善されないものとが自然に区分けされてくるはずである.
\item [] (4,1): 私たちが,拙速を避け,慎重な対処をすべきだと主張する理
由を,別の角度から整理し直してみると次の二点が挙げられよう.
 \item [] (5,1): 第一に,憲法改正の実現のためには,高いハードルを越え
ねばならない.
\item [] (5,2): 憲法九六条では,改正は衆参各議院の総議員の三分の二以上
の賛成で国会が発議し,特別の国民投票または国政選挙の際に行われる投票で
過半数の賛成を必要とする.
\item [] (5,3): この厳しい条件を満たすには,何を実現するための憲法改正
で,その結果,得られるプラス点と甘受すべきマイナス点は何か,という肝心
な点をきちんと国民に説明し,納得を得なければならない.
\item [] (5,4): 政党側に都合のいいことばかり宣伝して,国民にとって苦痛
となりうる点を隠しているようでは政治不信を招くだけである.
\item [] (5,5): この困難な過程を,目をつぶって急ぎ足で通り抜けてはなら
ない.
\item [] (6,1): 第二に,改憲により自衛隊が正規の軍隊と同じ扱いとなった
場合,しばしば海外に派遣されよう.
\item [] (6,2): ガリ国連事務総長は報告書で,紛争防止のための平和維持活
動(PKO)だけでなく,停戦合意が守られない事態に対処する平和執行部隊
の創設を含む平和創造活動を提唱している.
\item [] (6,3): これは武力行使を覚悟した派遣であり,カンボジアに派遣中
のPKOとは質的に異なる.
\item [] \underline{(7,1): 将来,仮に日本が国連安保理の常任理事国になっ
た場合,平和執行部隊派遣に賛}\\ \underline{成しながら自国の部隊派遣を
拒否することは利己的な態度として非難されよう.}
\item [] \underline{(7,2): 日本は国際貢献と犠牲の調和をどこに求めるの
か.}
\item [] \underline{(8.1):その答えを出すには,まず日本はどのようなコストを
負担して世界に寄与するの}\\ \underline{か,という国家としての自画像に
ついて国民的合意を固めねばならない.}
\item [] \underline{(8,2): この意味で首相と自民党政調会長が二十七日,党の憲
法調査会で論議を深めるこ}\\ \underline{とで合意したのは,その第一歩と
して歓迎したい.}
\item [] (8,3): 先入観を捨て,謙虚な姿勢で論議を尽くしてほしい.
\end{itemize}
}
}

\vspace{5mm}

{\bf 要約文章}

\begin{itemize}
{\normalsize
 {\baselineskip=12pt
\item []日本経済新聞 93年1月28日の社説
\item []改憲論の前に日本の“自画像”を描け(社説)

\item [] (1,1): 通常国会の代表質問が二十七日終わったが,今回の特色は与
野党の党首や主要役員級の質問者が異なった発想に基づく憲法改正論を展開,
宮沢首相との間で踏み込んだ論議を交わした点にある.
\item [] (2,1): 憲法には改正条項があり,その長所と短所を点検し,論議を
深めること自体は望ましいことである.
\item [] (7,1): 将来,日本が国連安保理の常任理事国になった場合,平和執
行部隊派遣に賛成しながら自国の部隊派遣を拒否することは利己的な態度とし
て非難されよう.
\item [] (7,2): 日本は国際貢献と犠牲の調和をどこに求めるのか.
\item [] (8,1): その答えを出すには,どのようなコストを負担して世界に寄
与するのか,という国家としての自画像について国民的合意を固めねばならな
い.
\item [] (8,2): この意味で首相と自民党政調会長が二十七日,党の憲法調査
会で論議を深めることで合意したのは,その第一歩として歓迎したい.
}
}
\end{itemize}
}

\vspace{-4mm}
\bibliographystyle{jnlpbbl} 
\bibliography{v06n6_05}
\begin{biography}
\biotitle{略歴}

\bioauthor{比留間 正樹}{
1997年横浜国立大学工学部電子情報工学科卒業.
1999年横浜国立大学大学院工学研究科電子情報工学専攻前期課程修了,
その間,自然言語処理の研究に従事.
同年,日本アイ・ビー・エム入社,現在に至る.
}

\bioauthor{山下 卓規}{
1997年横浜国立大学工学部電子情報工学科卒業.
1999年横浜国立大学大学院工学研究科電子情報工学専攻前期課程修了,
その間,文章要約などの自然言語処理の研究に従事.
同年,株式会社東芝入社,現在に至る.
}

\bioauthor{奈良 雅雄}{
1996年横浜国立大学工学部電子情報工学科卒業.
1998年同大学大学院工学研究科電子情報工学専攻前期課程修了,
自然言語処理,文章要約の研究に従事,
同年,日立ソフトウェアエンジニアリング(株)入社,現在に至る.
}

\bioauthor{田村 直良}{
1985年東京工業大学大学院博士課程情報工学専攻修了,工学博士.
同年東京工業大学大学工学部助手.
1987年横浜国立大学工学部講師,同助教授を経て,
1995年米国オレゴン州立大学客員教授,
1997年横浜国立大学教育人間科学部教授,現在に至る.
構文解析,文章解析,文章要約などの自然言語処理の研究に従事.
情報処理学会,人工知能学会,言語処理学会各会員.
}

\bioreceived{受付}
\biorevised{再受付}
\bioaccepted{採録}

\end{biography}

\end{document}
