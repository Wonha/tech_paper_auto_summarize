\documentstyle[epsf,jnlpbbl]{jnlp_j_b5}
\setcounter{page}{59}
\setcounter{巻数}{6}
\setcounter{号数}{2}
\setcounter{年}{1999}
\setcounter{月}{1}
\受付{1998}{4}{8}
\採録{1998}{7}{27}

\setcounter{secnumdepth}{2}

\title{多段解析法による形態素解析を用いた \\
音声合成用読み韻律情報設定法と \\
その単語辞書構成
}
\author{浅野 久子\affiref{NTT} \and 
松岡 浩司\affiref{NTT} \and
高木 伸一郎\affiref{NTT} \and
小原 永\affiref{NTT}
}

\headauthor{浅野, 松岡,高木,小原}
\headtitle{多段解析法による形態素解析を用いた音声合成用読み韻律情報設定法と
その単語辞書構成}

\affilabel{NTT}{NTT 情報通信研究所}
{NTT Information and Communication Systems Laboratories}

\jabstract{
日本語テキスト音声合成において,自然で聞きやすい合成音声を出力するためには,
読み,アクセント,ポーズ等の読み韻律情報を正しく設定する必要がある.
本論文では,複合語等に対しては部分的に深い解析を行うことを特徴とする
多段解析法に基づく形態素解析を用いて,読み韻律情報を設定する方法,および,
読み韻律情報を設定するために用いる単語辞書情報について述べる.
本方式の主な特徴は,
形態素解析における読み韻律情報付与に対応した長単位認定,
複合語内意味的係り受け情報を用いたアクセント句境界設定,
文節間係り受け情報を用いず,複合語内等の局所構造,およびアクセント句
境界前後単語の品詞情報等から得られるアクセント句結合力を用いて
段階的にポーズを設定する多段階設定法に基づくポーズ設定である.
本方式をニュース文章を対象に2名の評価者により評価した結果,
クローズ評価で95\%,オープン評価で91\%の精度(評価者2名の平均)で,
読み韻律情報を正しく設定でき,その有効性が確認できた.
}

\jkeywords{日本語音声合成,韻律,形態素解析}

\etitle{Reading-and-Prosodic Information \\
Generation Using Morphological Analysis \\
based on Multi-level Analysis Method \\
and its Word Dictionary Components}
\eauthor{
Hisako Asano \affiref{NTT} \and
Koji Matsuoka \affiref{NTT} \and
Shinichiro Takagi \affiref{NTT} \and
Hisashi Ohara \affiref{NTT}}

\eabstract{
In order for Japanese text-to-speech synthesis to provide highly 
natural synthesized speech,it is necessary to correctly generate 
reading-and-prosodic information, 
that is, information about readings, accents, pauses, and so on.
This paper describes a method of generating reading-and-prosodic information 
that uses morphological analysis based on the multi-level analysis method, 
which deeply analyses compound words and heteronyms; also described is 
 the word dictionary information used in the method.
The main characteristics of this generation method are: 
(1) long unit word recognition in the morphological analysis to cope with 
generating reading-and-prosodic information, 
(2) accentual phrase assignment using 
semantic dependent relationships in compound words, 
(3) pause insertion based on multi-level assignment using 
local structures in compound words and connected power of 
accentual phrases instead of dependent relationships of syntactic phrases.
In an evaluation for news-texts, this method generated reading-and-prosodic 
information with 95\% accuracy for closed data and 91\% 
accuracy for open data. 
These results show the effectiveness of this method.
}

\ekeywords{Japanese Text-to-Speech synthesis, Prosody, Morphological analysis}

\renewcommand{\topfraction}{}
\renewcommand{\bottomfraction}{}
\renewcommand{\textfraction}{}
\renewcommand{\dbltopfraction}{}
\renewcommand{\floatpagefraction}{}
\renewcommand{\dblfloatpagefraction}{}


\begin{document}
\maketitle




\section{はじめに}
日本語テキスト音声合成は,漢字かな交じりの日本語テキストに対して,
読み,アクセント(韻律上の基本単位であるアクセント句の設定とその
アクセント型付与),ポーズ等の読み韻律情報
\footnote{本論文では,読みと,アクセントやポーズなどの
韻律情報をまとめて読み韻律情報とよぶ.}
を設定し,これらを元に音声波形を生成して合成音声を出力する.
自然で聞きやすい合成音声を出力するためには,この読み韻律情報を
正しく設定する必要がある.

読みは,形態素解析により認定された単語の読みにより得られるため,
形態素解析の精度が読みの精度に直結する.
ただし,数量表現の読み(例:11本→ジューイ\underline{ッポ}ン:
\mbox{下線部分=読み}が変化)と連濁化(例:子供+部屋→コドモ\underline{ベ}ヤ:
\mbox{下線部分=連濁)については,すべてを単語}として辞書登録するのは
困難であるため,規則により読みを付与する.
数量表現の読みについては\cite{Miyazaki4},連濁化については\cite{Sato}
等により,その手法がほぼ確立されている.

アクセント句のアクセント型設定については,
\cite{Sagisaka}の付属語アクセント結合規則,複合単語(自立語)の
アクセント結合規則,文節間アクセント結合規則により,その手法が
ほぼ確立されている.

アクセント句境界とポーズの設定については,従来から多くの手法が提案されている.
ヒューリスティックスベースの手法としては,係り受けの構造を利用する
\cite{Hakoda1},右枝分かれ境界等の統語情報を用いる\cite{Kawai}等がある.
また,統計的手法によるポーズの設定としては,係り受け情報を利用した手法
\cite{Kaiki}が提案されている.
しかしこれらは,係り受けなどの言語的情報が既知であることを前提として
おり,これらの言語的情報の取得が課題となる.
一方,\cite{Suzuki}ではN文節の品詞情報を用いて局所的な係り受け構造を
推定し,また\cite{Fujio}では,品詞列を入力として確率文脈自由文法を
用いて係り受けを学習し,アクセント句境界や韻律句境界,ポーズの設定を行う.
しかし\cite{Suzuki}は,文節内の処理については言及しておらず,また,
\cite{Fujio}では文節内での設定において,文節内構造の予測誤りによる精度の
低下が問題点として挙げられている.

我々は,\cite{Miyazaki1}の方式をベースとし,多段解析法による形態素解析を
用いて得られた単語情報を利用して規則により読み韻律情報を設定し,
\cite{Hakoda2}の音声合成部を用いて合成音声を出力する日本語テキスト
音声合成システムAUDIOTEXを開発した.
このAUDIOTEXには,現在,数多く開発されている音声合成システムと比較して
以下の2つの特徴がある.
\begin{itemize}
\item 単語辞書の登録単語数が多いため,形態素解析における
未知語認定が少ない.\\
(AUDIOTEX:約37万語,市販の主な音声合成システム:10〜14万語)
\item 単語辞書において,特に名詞と接辞は,他のシステムにはない
意味カテゴリ等の意味情報をもち,これらの意味情報を用いた複合語の
意味的係り受け解析により,複合語の構造を高精度に解析できるため,
複合語の多用されるニュース文などに対しても,正しく読み韻律情報が
設定できる.
\end{itemize}

本論文では,AUDIOTEXにおける読み韻律情報の設定,特に\cite{Miyazaki1}
からの主な改良点として,形態素解析における読み韻律情報付与に対応した
長単位認定,アクセント句境界設定における複数文節アクセント句の設定,
ポーズ設定における多段階設定法の導入について述べ,さらに,
これらの処理で用いる単語辞書の構成について説明する.
この読み韻律情報の設定においては,文節間の係り受け解析は行わず,
多段解析法の形態素解析により得られる複合語内意味的係り受け情報,
品詞等の単語情報のみを用いる.文節間の係り受け解析を行わないのは,
現状,係り受け解析の精度が十分でなく,コストがかかり,
また,文節間係り受けの影響を大きく受けるポーズ設定においては,
アクセント句境界前後の品詞情報等から得られるアクセント句
結合力を導入することにより,実用上十分な精度が得られるためである.
さらに,文節内の構造に対しては,複合語意味的係り受け情報を用いることにより,
その局所構造を元に適切にポーズを設定できる.

以下,\ref{sec:TTS-flow}節ではテキスト音声合成処理の流れ,
\mbox{\ref{sec:morph}節では形態素解析における読み韻律情報設定}のための特徴,
\ref{sec:dic}節では読み韻律情報設定のための単語辞書の情報,
\ref{sec:assign}節では読み韻律情報の設定方法,
\ref{sec:evaluation}節では読み韻律情報設定に対する評価と考察,
\ref{sec:conclusion}節ではまとめを述べる.

\section{テキスト音声合成の流れ}\label{sec:TTS-flow}
テキスト音声合成の一般的な流れを図\ref{fig:system}の左側に示す.
漢字かな交じりの日本語テキストを入力し,テキスト解析部において,
言語的な解析\footnote{形態素解析,構文解析,意味解析等.一般に,現状,
構文解析や意味解析を高精度で行うことは困難であるため,
形態素解析のみ,あるいは形態素解析と局所的な係り受け解析が
行われる場合が多い.}
を行い,その情報を利用して読み韻律情報を生成する.そして
音声合成部では,読み韻律情報を用いてピッチや時間長データを設定して
音声波形を生成し,合成音声を出力する.

\begin{figure*}[tb]
\begin{center}
\epsfile{file=system.eps}
\end{center}
\caption{テキスト音声合成処理の流れ}
\label{fig:system}
\end{figure*}

次にAUDIOTEXにおける読み韻律情報,テキスト解析部,音声合成部の
概略を述べる.

AUDIOTEXでは,読み韻律情報として,
アクセント句単位に読み,アクセント型,音調結合型を指定するアクセント付
カナ文を用いる.
このアクセント付カナ文は,
ピッチパタンを話調成分\footnote{ポーズで区切られる単位(呼気段落)
ごとに設定される.}の上にアクセント成分\footnote{アクセント句
ごとに設定される.}が重畳したものとして表し,
アクセント句の間(アクセント句境界)に音調結合という概念を導入
して,アクセント句成分の相対的な大きさやポーズ挿入の現象を統一的にモデル化
する韻律制御モデル\cite{Hakoda1}に基づいている.
ここで音調結合型は,話調成分,アクセント成分,ポーズの関係により分けられ,
本論文では,表\ref{tab:oncyo}に示す6種類を用いる.

また,読み上げ速度としては,通常,ニュースなどが読み上げられる速度を
想定しており\footnote{ポーズ区間を除いた平均速度6.4モーラ/秒を基準の
\mbox{速度としている.読み上げ速度の変更は可能であるが,読み韻律情}報が
読み上げ速度により変化することはない.},この速度を基準として
読み韻律情報を設定する.

\begin{table}[thb]
\caption{音調結合型}
\label{tab:oncyo}
\begin{center}
\begin{tabular}{l|l|l|l|l}\hline
名称	& 表記	  & ポーズ& 話調成分	& 直後アクセント成分 \\ \hline
強結合	& $\ast$  & なし & 同一	& 抑圧 \\
弱結合	& /	  & なし & 同一  & 抑圧なし  \\
小ポーズ & $\sqcup$(スペース)& 小ポーズ(300msec)	& 再設定& - \\
中ポーズ & ,	& 中ポーズ(500msec) & 再設定 & -  \\
大ポーズ & .	& 大ポーズ(700msec) & 再設定 & -  \\ 
疑問調\footnotemark   & ?	& 大ポーズ(700msec) & 再設定 & - \\ \hline
\end{tabular}
\end{center}
\end{table}

\footnotetext{直前アクセント句末尾のピッチを上げる.}

テキスト解析部は図\ref{fig:system}の右側に示す5部により構成される.
\begin{description}
\item[形態素解析]
文を単語に分割し,品詞や読み等の単語情報を付与する.
この単語情報は,読み韻律情報を設定するために必要な多くの情報を備えている.
この形態素解析で利用する単語辞書には約37万語が登録されている.
詳しくは,\ref{sec:morph}節,\ref{sec:dic}節で述べる.
\item[テキスト書き換え]
入力テキストが,そのまま音声化するのに不適当な表現や形式である場合,
それを音声化に適した表現に書き換える.
書き換え内容は入力テキストの形式やドメインに依存する.
例えば,新聞記事では漢語表現や文末表現の書き換え(例:``今秋''→
``今年の秋'',``〜の予定。''→``〜の予定です。'')が必要であり,
また,電子メールでは,読み上げ対象とならない引用記号やsignatureの削除などが
必要である.
現在,新聞記事用
\cite{Matsuoka},電子メール用の2種類のテキスト書き換えを
備えているが,本論文ではこの詳細については省略する.
\item[読み付与]
数量表現の読み付与,連濁化を規則により行う.その他の読みは,
形態素解析で得られる各単語の読みを用いる.
\item[アクセント付与]
複合語意味的係り受け情報等を用いて,韻律上の基本単位であるアクセント句を
設定する.また,単語固有のアクセント型,単語のアクセント的性質を用いた
アクセント結合規則により,アクセント句のアクセント型を付与する.
詳しくは,\ref{subs:accent}節で述べる.
\item[音調結合型付与]
時間表現,数量表現や同格表現(例えば役職名+人名)など,
独立に扱うことができ,その構造が複合語内意味的係り受け情報より
得られる局所構造内のアクセント句境界,および,句読点の直後など
品詞情報から容易に意味的,構文的な切れ目であることを推定できる
アクセント句境界を対象として,その局所構造に基づき
音調結合型を設定した後,残りのアクセント句境界に
対して,アクセント句結合力を用いて音調結合型を設定する,
段階的な音調結合型設定法(多段階設定法)に基づき,音調結合型を付与する.
詳しくは,\ref{subs:pause}節で述べる.
\end{description}

音声合成部は,波形編集方式の音声合成ソフトウェアFLUET
\cite{Hakoda2}の音声合成部を利用しており,
アクセント付カナ文を音素記号列に変換する韻律パラメータ生成部,
音素片ファイルから隣接する音素環境が一致する音素片データを選択する
音素片選択部,選択した音素片データを結合し,規則で設定されたピッチ,時間長
データに基づいて波形データを加工する音素片接続部からなる.

また,AUDIOTEXはCでコーディングされており(音声合成部はC++ライブラリ),
UNIXおよびWindows上で動作する.

\section{形態素解析における読み韻律情報設定のための特徴}\label{sec:morph}
形態素解析は,
\cite{Miyazaki1}の多段解析法による形態素解析を,より
高精度な読み韻律情報設定を行うために拡張して利用している.
多段解析法は,文字種の違いに着目して仮に設定した範囲内(仮文節)であらゆる
単語の組合わせを検定する局所総当たり法をベースとし,構文や意味の情報が
有効となる複合語解析や同型語判別などには,部分的に深く解析する.
辞書としては,各単語の情報を記述した単語辞書と,文法的接続情報を
記述した接続辞書を用いる.
以下,読み韻律情報設定のための形態素解析という観点から,
その特徴を詳しく説明する.
ここで,連語の認定と用言語幹語尾の一語化は,\cite{Miyazaki1}より
新たに拡張した項目であり,共に長単位認定を行うための手法である.

\begin{figure}[tb]
\begin{center}
\epsfile{file=kakari.eps}
\end{center}
\caption{複合語内意味的係り受け例}
\label{fig:kakari}
\end{figure}

\begin{description}
\item[複合語内意味的係り受け情報]
複合語を構成する単語間の意味的結合関係を係り受けによって
解析する\cite{Miyazaki2}.
係り受け解析では,数詞,固有名詞,接辞,用言性名詞,非用言性名詞など
14種の係り受け規則を用いる.
例として``前平成株式会社社長山田太郎氏51才''に対する複合語内意味的
係り受けを図\ref{fig:kakari}に示す.
この意味的係り受け情報は,アクセント句境界,音調結合型の設定に利用する.
\item[連語の認定]
単語の認定精度の向上は,読み韻律情報の精度向上に直結する.
そこで,認定\mbox{精度が比較的低かった,補助用言(例:``話している''の``いる'')や}
格助詞相当語(例:``彼について''の``について'')の認定精度を向上させるために,
「助詞+補助用言」(例:``て+いる'')や格助詞相当語(例:``に+つい+て'')を
連語として登録し,認定することとした.
しかし,これらの連語内でアクセント句が分割される場合等があるため
(例:彼{\dg に}/{\dg ついて}
\footnote{/はアクセント句境界を表す.以降でも同様に記述する.
また,この例における太字は,1つの連語であることを示す.}
),
アクセント,音調結合型設定時に,連語すべてを1単語として扱うのは問題がある.
そこで,形態素解析では,連語として1語で認定する単語を,連語を構成する
各構成単語に分解できる機能を設けた.
詳しくは,\ref{sec:dic}節で述べる.
\item[用言語幹語尾の一語化]
規則的な活用を行う用言は,単語辞書では,不変化部分(例:似)と変化部分
(例:る,れ,ろ)
に分離し,それぞれを1単語として
登録している.また,サ変名詞(運動)が,サ変動詞(運動する)
の一部を構成する場合があるが,単語辞書上は,サ変動詞型名詞(運動)と
サ変動詞活用形(する)のみを登録し,語幹語尾をまとめた単語(運動する)
としては登録していない.

アクセント句結合力に基づく音調結合型の設定においては,
アクセント句境界前後の単語の品詞情報が重要となる.
しかし,辞書の登録単位である短単位での単語認定では,例えば
``グランドで/運動する''のアクセント句境界の直後単語をサ変名詞(運動)
として扱い,その結果,音調結合型の設定を誤るという問題があった.
そこで,単語認定後,「用言不変化部分+変化部分」,
「用言性名詞+活用語尾」を一語に統合することにした.これにより,
例えば``運動する''を1語のサ変動詞として扱うことになる.
\end{description}

\vspace{3mm}
\section{読み韻律情報設定のための単語辞書情報}\label{sec:dic}

一般に単語辞書は,形態素解析において単語を認定するための情報をもつが,
AUDIOTEXの単語辞書では,読み韻律情報を設定するための様々な情報も保持している.
本節では,読み韻律情報の設定に関連がある単語辞書情報について述べる.

\vspace{3mm}
\subsection{長単位語への対応}\label{subs:cho}
単語辞書に登録する単語の単位は,原則的には語基や接辞などの短単位語であり,
複合語などの長単位語は,短単位語の組み合わせとみなす.
ただし,次に示す語は,以下の理由により例外的に長単位で登録している.
\begin{itemize}
\item 連語:補助用言や格助詞相当語の認定精度の向上のため.
(例)について,ていまし
\item 短単位語から長単位語の意味や読みなどを合成できない,または合成
するのが難しい一部の慣用表現,熟語,複合語,並列語:
短単位語の組み合わせでは,正しく認定できないため.
 (例)十六夜(イザヨイ),一期一会(イチゴイチエ),日仏英(ニチフツエイ)
\item 有名な人名・地名等や一般語で構成される作品名・商品名等の
一部の固有名詞:\\
固有名詞の読みはバリエーションが多く,このうち有名な人名・地名等は
長単位で登録することにより,その読み精度を向上できるため.\\
(例)羽生善治(ハブヨシハル),清水寺(キヨミズデラ)\\
また,一般語のみで構成される固有名詞は短単位語の組み合わせでは
固有名詞として認定できないため.
(例)週間住宅情報
\item 国語辞典に子見出し語や派生語などとして収録されている一般用語:
出現頻度が高く一般性が高いと考えられるため.
(ただし,これらの単語すべてが認定精度向上に役立つとは限らない.)
(例)為替相場,人工呼吸,身分証明書
\end{itemize}

このように,長単位で登録することにより,一般に形態素解析の精度は向上する.
しかし,アクセント,音調結合型の設定においては,長単位語を短単位語と
同様に一語として取り扱うことは,以下のような問題を生じる.



\begin{itemize}
\item 長単位語の内部に設定すべきアクセント句境界に対応できない.\\
並列語:``日仏英''→日/仏/英,
連語:``について''→彼{\dg に}/{\dg ついて}
\item 長単位語内部にアクセント句境界を設定する必要はないが,複合語を構成する
単語数によりアクセント句境界を設定する場合に,複合語を構成する単語数が正しく
得られず設定を誤る.\\
(例)``為替相場''が1語(長単位登録)の場合\\
形態素解析結果:``為替相場+速報+サービス''(3単語扱い,実際は4単語)\\
アクセント句:×``為替相場速報サービス'' と設定.
(○:``為替相場/速報サービス'')\\
cf. ``為替+速報+サービス''(3単語)→○``為替速報サービス''
\end{itemize}

これらの長単位語に対して適切に読み韻律情報を設定するために,
次の3つの単語辞書情報をもつ.

\begin{description}
\item[アクセント句情報]
アクセント句情報は,最大3アクセント句分のモーラ数,読み長(読みの
表記上の長さ),アクセント型を保持する.
最大3アクセント句分の情報をもつのは,固有名詞など,1語で登録されている
複合語内部の特定の位置にアクセント句境界が必ず存在する場合に
対応するためである.
例えば,短単位語の``日本''のアクセント句情報は,\\
\hspace*{1cm}第1アクセント句:モーラ数=3,読み長=3,アクセント型=2\\
\hspace*{1cm}第2,3アクセント句:なし\\
となり,長単位語の``日仏英''(3アクセント句)は,\\
\hspace*{1cm}第1アクセント句:モーラ数=2,読み長=2,アクセント型=1(日)\\
\hspace*{1cm}第2アクセント句:モーラ数=2,読み長=2,アクセント型=1(仏)\\
\hspace*{1cm}第3アクセント句:モーラ数=2,読み長=2,アクセント型=1(英)\\
となる.この情報により,``日/仏/英''と正しくアクセント句境界が
設定できる.
\item[語数]登録単語の内部ではアクセント句境界を生じないが,
長単位で登録されていることを表すための情報であり,登録単語を構成する
単語数を表す.
例えば,
``為替相場(2)+速\mbox{報(1)+サービス(1)''(かっこ内の数字が語数を表す)では,}
形態素解析の認定単語数は3語であるが,語数により4語からなる
複合語であることがわかり,``為替相場/速報サービス''と正しくアクセント句
境界が設定できる.
\item[構成単語情報]
長単位で登録された単語を短単位の構成単語に展開するために,
各構成単語(最大10語)の見出し長,品詞,読み長をもつ.

これは,長単位語の内部に設定すべきアクセント句境界があり,構成単語に付属語
を含む場合,つまり,主に連語に対応するためである.
これらの長単位語は,アクセント付与を行う前に構成単語情報を
元に辞書検索を行い,短単位に展開しておく.上記のアクセント情報で対応する
単語と異なり,辞書検索を行って短単位に展開するのは,付属語の
アクセント結合のための情報(表\ref{tab:dic}の付属語アクセント属性,副次
アクセントフラグ)を長単位語内で保持するのは煩雑なためである.

(例)連語``ていまし(て+い+まし)''\\
\hspace*{1cm}て:見出し長=1, 品詞=接続助詞,読み長=1\\
\hspace*{1cm}い:見出し長=1, 品詞=一段動詞語幹,読み長=1\\
\hspace*{1cm}まし:見出し長=2, 品詞=助動詞連用形(断定),読み長=2
\end{description}

\subsection{読み韻律情報設定で用いる単語辞書情報}
表\ref{tab:dic}に,読み韻律情報を設定する際に利用する単語辞書情報を,
全処理で利用する共通情報と,特定の処理のみで利用する,読み付与用情報,
アクセント句境界設定用情報,アクセント型設定用情報,音調結合型設定用情報の
5種類に分けて示す.
読み付与用には,数量表現の読みの補正,連濁化のための情報をもつ.
アクセント句境界設定用には,複合語の内部構造を表すための各種フラグ,および,
韻律的な特徴を表すフラグ,アクセント型設定用にはアクセント的特徴を示す
各種フラグ,音調結合型設定用には,助詞を構文的に分類した
助詞ポーズ属性をもつ.


\begin{table}[thb]
\caption{読み韻律情報設定用単語辞書情報}
\label{tab:dic}
\begin{center}
\begin{tabular}{|l|l|}\hline
\multicolumn{2}{|l|}{共通情報} \\ \hline
品詞	&  \cite{Miyazaki5}を元にした品詞体系(注1) \\
モーラ数	&  \ref{subs:cho}節のアクセント句情報の一部 \\ \hline
\multicolumn{2}{|l|}{読み付与用情報}  \\ \hline
読み	& 長音化表記 (例:可能→カノー)\\
連濁フラグ & 語頭が連濁する条件(直前語尾条件)\\
数詞音韻フラグ & 助数詞に設定し,数詞の読みを補正
[11\underline{回}→ジューイ\underline{ッ}カイ](注2) \\
助数詞音韻フラグ & 助数詞に設定し,助数詞の読みを補正
[3\underline{本}→サン\underline{ボ}ン](注2)\\ \hline
\multicolumn{2}{|l|}{アクセント句境界設定用情報} \\ \hline
承接語フラグ & \footnotesize{特定品詞(助数詞,地名,姓,名,組織名,その他固有名)に承接する語}\\
連体詞化フラグ & 複合語内で連体詞的(〜の)に使われる語[クリスマス] \\
前置助数詞フラグ & 数詞の前で助数詞的に用いられる語[国道,第] \\
後置助数詞フラグ & 数詞の後で助数詞的に用いられる語[科目,議席] \\
役職・敬称フラグ & 役職,職種,敬称を表す語[先生,様] \\
独立アクセント句フラグ & その語(+付属語)のみでアクセント句を構成
[前(接頭辞)]\\
\footnotesize{前方アクセント句境界フラグ} & 単語前方にアクセント句境界を設定 
[全体] \\
\footnotesize{後方アクセント句境界フラグ} & 単語の後方にアクセント句境界を設定
[生まれ]\\ \hline
\multicolumn{2}{|l|}{アクセント型設定用情報} \\ \hline
アクセント型 & \ref{subs:cho}節のアクセント句情報の一部 \\ 
アクセント結合例外属性 & 自立語アクセント結合での,例外的なアクセント型付与 \\ 
助数詞種別 & 助数詞のアクセント型付与規則種別 \\
副次アクセントフラグ & 副次アクセント属性をもつ語[ます] \\
付属語アクセント属性 & 付属語アクセント結合用属性(直前語品詞(動詞,形容詞,その他)別)\\ \hline
\multicolumn{2}{|l|}{音調結合型設定用情報} \\ \hline
助詞ポーズ属性 & 助詞に設定し,結合力を求めるために利用 \\\hline
\multicolumn{2}{l}{(注1)大分類:名詞,動詞,形容詞,形容動詞,副詞,連体詞,接続詞,感動詞,} \\
\multicolumn{2}{l}{\hspace*{2cm} 接辞,助動詞,助詞,記号}\\
\multicolumn{2}{l}{(注2)表記の下線部はそのフラグが設定される単語を表し,}\\
\multicolumn{2}{l}{\hspace*{1cm} 読みの下線部はそのフラグにより補正が
行われた読みを表す}\\
\multicolumn{2}{l}{(注3)[]内に,当該フラグがonとなる単語例を表す}
\end{tabular}
\end{center}
\end{table}

\normalsize

\section{読み韻律情報の設定法} \label{sec:assign}
本節では,読み韻律情報の設定方法について説明する.特に,新たに多段階設定法
を導入した音調結合型の設定について詳細に述べる.読み付与については,
\cite{Miyazaki1}の手法(数量表現読み付与,連濁化)をそのまま利用するため
省略する.
\subsection{アクセント付与}\label{subs:accent}
複合語内意味的係り受け情報および表\ref{tab:dic}に
示した単語辞書情報を用いて,アクセント句境界およびアクセント型の
設定を行う.
数詞については\cite{Miyazaki4},複合語については
\cite{Miyazaki3}を基本としてアクセント句境界を設定し,
\cite{Sagisaka}に基づきアクセント句のアクセント型を設定する.

ここで
\cite{Miyazaki1}では,文節間のアクセント結合を行っていない(文節境界を
すべてアクセント句境界としている).
これは,文節間のアクセント結合を行わなくても,音調結合型を強結合とする
ことにより実用上十分と判断しているからである.
しかし,次に示す,結び付きの強い文節間においては,強結合としても
不自然に感じるという実験結果が得られたため,アクセント結合を行うこととした.
\begin{itemize}
\item 指示副詞+用言\\
	そう(指示副詞)+思う(動詞)→そう思う
\item 連体詞+名詞($\neq$複合語)\\
この(連体詞)+会議(名詞)→この会議\\
cf. この(連体詞)+国際(名詞)+会議(名詞)→この/国際会議
\end{itemize}

また,複合語内のアクセント句境界設定において,独立性の高い時詞等は
常に独立のアクセント句とされていたが,複合語内意味的係り受け情報,
直後単語の品詞情報(一般名詞,サ変名詞,転生名詞,その他)により,
他の単語と結合するか判断することにした.

(例)\\
\hspace*{1cm}正月(時詞)+番組(一般名詞)+で(格助詞) →正月番組で\\
\hspace*{1cm}今日(時詞)+番組(一般名詞)+で(格助詞)→今日/番組で

さらに,自立語アクセント結合において,付属語アクセント結合における副次
アクセントと同様の現象が存在する.
例えば,

神奈川等:カナ'ガワ/ト'ー, 大阪等:オーサカト'ー
\footnote{'はアクセント位置を表し,/はアクセント句境界を表す.以降も
同様に記述する.}
\\
のうち,先行語(``神奈川'')がアクセントをもつ``神奈川等''は
副次アクセントをもつ.
そこで,自立語でも,付属語と同様に副次アクセントに対応することにした.

\subsection{音調結合型(ポーズ)付与}\label{subs:pause}
音調結合型の設定は,複合語内等から段階的に音調結合型を設定していく
多段階設定法を新たに導入した.
多段階設定法では,はじめに,時間表現,数量表現や同格表現
(例:「役職名+人名」では役職名と人名は同格とみなせる)など,
独立に扱うことができ,その構造が複合語内意味的係り受け情報より
得られる局所構造内のアクセント句境界,および,句読点の直後など
品詞情報から容易に意味的,構文的な切れ目であることを推定できる
アクセント句境界を対象として,意味的,構文的に大きな切れ目となる
アクセント句境界にポーズを,つながりが強いアクセント句境界に
ポーズなしを設定する.
次に,上記において音調結合型が設定されなかったアクセント句
境界に対し,前後の単語の品詞情報等より得られるアクセント句結合力(以降では
結合力と表記)を用いて音調結合型を設定する.ここで,すべてのアクセント句境界
に音調結合型が設定されるが,あるモーラ長以上連続してポーズが
設定されていない連続アクセント句列(ポーズ未設定区間)に対しては,
結合力を用いてポーズ付与のための補正を行う.
結合力の値は1〜10の10段階(値が大きいほど結合力が強い)であり,
アクセント句境界の前後アクセント句の係り受けがありえない場合
(例:用言連用形<アクセント句境界>名詞)では極端に値を小さく,
逆に係り受けが生じる可能性が高い場合(例:用言連体形<アクセント句境界>名詞)
では大きく設定している.

この多段階設定法を導入することにより,従来手法では精度が低かった
複数アクセント句からなる複合語内の設定においても,
その内部構造を反映した適切な音調結合型の設定が可能となり,
また,文節間の係り受け解析を用いなくても,結合力により近似的に
係り受け構造を推定し,結合力の強さとポーズ未設定区間のモーラ数に応じて
段階的にポーズを付与していくので,実用上十分な精度で音調結合型を設定できる.

以下,多段階設定法の各ステップについて説明する.
ポーズ付与の補正以外では,
すでに音調結合型が設定されているアクセント句境界は対象とせず,未設定
のアクセント句境界のみを対象とする.
\subsubsection{記号に基づく音調結合型設定} \label{subsub:kigo}
アクセント句末尾が句読点等の記号の場合,その直後アクセント句境界に対して,
音調結合型を設定する.句点,感嘆符の場合は大ポーズ(700msec),疑問符には,
疑問調(末尾ピッチが上がる+700msecポーズ),読点,開きかっこには
中ポーズ(500msec),その他の中点以外の記号には小ポーズ(300msec)を付与する.
\subsubsection{日時表現・数量表現の音調結合型設定} \label{subsub:num}
日時表現,数量表現は,その局所構造に基づき音調結合型を決定する.

日時表現は,その表現を年要素,月要素,日要素に分類し,
この出現パターンにより,音調結合型を設定する.\\
(例)\\
\{小ポーズ\}
\footnote{\{{\it TYPE}\}は,その位置がアクセント句境界であり,
音調結合型として{\it TYPE}を設定することを表す.以降も同様に記述する.}
年要素\{小ポーズ\}月要素\{強結合\}日要素:\\
\hspace*{1cm}{\dg 会議は\{小ポーズ\}平成10年\{小ポーズ\}6月\{強結合\}1日から}\\
\{小ポーズ\}月要素\{強結合\}日要素:
{\dg 会議は\{小ポーズ\}6月\{強結合\}1日から}

数量表現は,
\cite{Miyazaki4}で,\\
(前置助数詞)+(符号)+数詞+(助数詞)+(接辞)
\footnote{()は省略可を表す.}\\
と定義された表現である.
この数量表現のパターンやその前後アクセント句の品詞に応じて,
数量表現内および,その前後の音調結合型を設定する.\\
(例)\\
\{小ポーズ\}前置助数詞\{小ポーズ\}2つ以上の数詞(+助数詞)アクセント句:\\
\hspace*{1cm}{\dg \{小ポーズ\}第\{小ポーズ\}百/二十/三回}\\
\{小ポーズ\}前置助数詞\{強結合\}1つの数詞(+助数詞)アクセント句:\\
\hspace*{1cm}{\dg \{小ポーズ\}第\{強結合\}三回}

\subsubsection{特定単語の組み合わせによる音調結合型設定}\label{subsub:combi}
連続する2または3アクセント句の品詞等の単語情報を参照して
得られる局所構造により,意味的,構文的な切れ目となり,常にポーズを
付与すべきアクセント句境界に小ポーズを,また,つながりが非常に強く,
常にポーズなしとすべき境界に強結合または弱結合を設定する.\\
(例)\\
役職\{小ポーズ\}人名 [同格表現]:{\dg 社長\{小ポーズ\}山田太郎氏}\\
``など''(副助詞)\{小ポーズ\}用言以外の単語 [例示表現]:\\
\hspace*{1cm}{\dg 証人喚問など\{小ポーズ\}事実審理が/始まります。}

\subsubsection{結合力に基づく音調結合型設定}\label{subsub:ketsugo}
記号に基づく音調結合型設定〜特定単語の組み合わせによる音調結合型設定で
音調結合型が設定されなかったアクセント句境界に対し,
アクセント句境界前後の単語の品詞等により得られる結合力を用いて
音調結合型を設定する.

\begin{table}[t]
\caption{直前単語の品詞分類}
\label{tab:prev}
\begin{center}
\begin{tabular}{|c|l|l|} \hline
分類   & 品詞               & 具体例 \\\hline
P1 & 副詞型名詞,時詞,数詞 & \\ 
P2 & 連体詞型名詞,連体詞   & \\
P3 & 用言,助動詞連体形     & \\
P4 & 用言,助動詞連用形     & \\
P5 & 副詞                   & \\
P6 & P1〜P5, P7〜P11以外    & \\ 
P7 & 助詞:ポーズ属性1 & 副助詞:は (この語のみ)\\
P8 & 助詞:ポーズ属性2 & 接続助詞:から,けれど,ば,副助詞:だけ,しか\\
P9 & 助詞:ポーズ属性3 & 格助詞:が,副助詞:も,接続助詞:たり,つつ\\
P10 & 助詞:ポーズ属性4 & 格助詞:から,で,を 副助詞:ずつ,でも\\
P11 & 助詞:ポーズ属性5 & 格助詞:の,と,や 副助詞:か,なり,やら \\ \hline
\end{tabular}
\end{center}
\end{table}
\begin{table}[t]
\caption{直後単語の品詞分類}
\label{tab:next}
\begin{center}
\begin{tabular}{|l|l|} \hline
分類 & 品詞 \\\hline
N1 & 副詞型名詞,時詞,数詞 \\
N2 & 連体詞型名詞,連体詞 \\
N3 & 本動詞 \\
N4 & 補助動詞 \\
N5 & 形容詞,形容動詞 \\
N6 & 副詞 \\ 
N7 & その他(一般名詞,接辞等) \\\hline
\end{tabular}
\end{center}
\end{table}

結合力を設定するために,品詞により,アクセント句境界の直前単語を
表\ref{tab:prev}に示すP1〜P11の11種類,直後単語を表\ref{tab:next}に
示すN1〜N7の7種類に分類した.
これらは,品詞の構文的性質(例:P3=連体修飾をする用言),および,
独立性\footnote{大分類が名詞と同一でも副詞型名詞,時詞等の方が一般名詞等より
独立性が高いため,名詞を細分類している.(P1・P6,N1・N7)}という
2つの観点により分類を行ったものである.
このため,直前単語の分類においては,助詞を単語辞書情報の
助詞ポーズ属性(構文的性質と独立性により助詞を分類)により5種類に細分類した.
表\ref{tab:prev}のP7〜P11には各ポーズ属性毎に
その属性をもつ助詞(抜粋)をあわせて示している.

結合力は,Pi×Nj(i=1〜11, j=1〜7)の組み合わせマトリックスにより
決定する\footnote{ただし連語から分解された単語の境界には,
無条件に結合力最大値(10)を与える.\\
(例){\dg に}(結合力=10){\dg ついて}}.
この組み合わせマトリックスの各値は,PiとNjの構文的な性質(係り受けが
ありえる,ありえない),および,それぞれの独立性,さらに,係り受けが
ありえる組み合わせにおいては,その韻律的特徴\footnote{アクセント句の
モーラ長や複合語の複数アクセント句化など.}を考慮して経験的に設定している.
ここでは,Pi×N7の組み合わせマトリックスのみを表\ref{tab:ketsugo}に示して
具体的に説明する.
直前単語がP2, P3, P11の場合は,このアクセント句境界の
直前文節が直後文節に係る連体修飾関係となる可能性が高いため,
表\ref{tab:ketsugo2}に示すように,直後アクセント句等の条件に応じて
結合力を変える.表\ref{tab:ketsugo2}の項番1は,直後が複数アクセント句
からなる複合語である場合を表しており,この場合は,連体修飾関係が
成立していても,アクセント句単位の構造としては,右枝分かれ構造となる場合が
ほとんどである\footnote{例えば``昨年/成立した/(A)男女/雇用機会/均等法''
では,文節単位の構造としては``成立した''→``\mbox{男女雇用機会均等}法''と
係り受けが成立するが,アクセント句単位の構造としては,``成立した''→
``均等法''となり,(A)は右枝分かれ境界となる.}ため,結合力を最小とする.

\begin{table}
\caption{Pi×N7の組み合わせマトリックス}
\label{tab:ketsugo}
\begin{center}
\begin{tabular}{|l|l|l|l|l|l|}\hline
P1=2 & P2=表\ref{tab:ketsugo2} & P3=表\ref{tab:ketsugo2} & P4=1  &
P5=2 & P6=9 \\ \hline
P7=1 & P8=1  & P9=2 & P10=2 & P11=表\ref{tab:ketsugo2} & \\ \hline
\end{tabular}
\end{center}
\end{table}

\begin{table}
\caption{P2,P3,P11の結合力}
\label{tab:ketsugo2}
\begin{center}
\begin{tabular}{|c|l|l|c|c|c|}\hline
項  & 直後アクセント句条件  & その他の条件 & \multicolumn{3}{c|}{結合力} \\
\cline{4-6}
番  &                       &              & P2 & P3 & P11 \\ \hline
1 & 名詞,接辞,連体詞のみ & 2つ後のアクセント句先頭=名詞or接辞 & - & 1 & 1 \\
2 & 自立語の総モーラ数$\geq$5 & Pi$\ne$格助詞``の''       & 5 & 4 & 2 \\
3 & 自立語の総モーラ数$\geq$5 & Pi=格助詞``の''           & - & - & 5 \\ \hline
4 & \multicolumn{2}{|l|}{上記以外}                        & 8 & 7 & 8 \\ \hline
\end{tabular}
\end{center}
\end{table}

この結合力は
ポーズ付与の補正でも利用するため,
記号に基づく音調結合型設定〜特定単語の組み合わせによる音調結合型設定で
すでに強結合または弱結合
を設定しているアクセント句境界に対しても求めておく(ただし音調結合型の
設定は行わない).ただし,これらのアクセント句境界に対しては,得られた
結合力に10を加算する.これは,
記号に基づく音調結合型設定〜特定単語の組み合わせによる音調結合型設定
で設定される音調結合型を
結合力に基づく音調結合型設定
で設定される音調結合型より
信頼性が高いと考えるためである.

音調結合型未設定のアクセント句境界に対しては,結合力=1となるアクセント句境界に
小ポーズ,
それ以外のアクセント句境界には,文節境界となるアクセント句境界に弱結合,
それ以外に強結合を付与する.

\subsubsection{ポーズ付与の補正}\label{subsub:unpause}
あるモーラ長以上ポーズが設定されていない場合(このポーズが設定されていない
区間をポーズ未設定区間とよぶ)に,結合力を用いてポーズを
設定する.
ここでは,ポーズ未設定区間の長さと
結合力に基づく音調結合型設定
で設定した結合力の
強さにより,段階的にポーズを付与する.

\begin{description}
\item[20モーラ\hspace{-0.05mm}$\leq$\hspace{-0.05mm}ポーズ未設定区間長\hspace{-0.05mm}$<$\hspace{-0.05mm}30モーラの場合]
\mbox{ポーズ未設定区間先頭から5モーラ目}〜
\mbox{末尾から5モーラ目までに}
\mbox{結合力3以下のアクセント句境界が存在する場合にのみ,そ}の中で最小の
結合力をもつアクセント句境界に小ポーズを付与する.
\mbox{それ以外の場合は}ポーズを付与しない.
\item[ポーズ未設定区間長$\geq$30モーラの場合]
ポーズ未設定区間先頭から5モーラ目〜末尾から5モーラ目までに
結合力が6以下のアクセント句境界が存在する場合にその中で最小の
結合力をもつアクセント句境界に小ポーズを設定する.

条件を満たすアクセント句境界が存在しない場合には,
ポーズ未設定区間先頭から2モーラ目〜末尾から2モーラ目までの
アクセント句境界において,最小の結合力をもつアクセント句境界に小ポーズ
を付与する.
\end{description}

ここで,ポーズ付与の補正を行った後のポーズ未設定区間が上記条件を
満たす場合には,再帰的にポーズ付与の補正を行う.

これは,ポーズ未設定区間長が20モーラ以上30モーラ未満と,ポーズが
挿入されなくてもあまり不自然でない長さの場合には,
結合力が弱い(3以下)アクセント句境界が存在した場合にのみポーズを
付与し,ポーズ未設定区間長が30モーラ以上と,ポーズがなければ不自然となる
長さを越えた場合には,必ずポーズを設定するという2段階のポーズ付与の補正
を行うものである.

\subsubsection{ポーズ付与例}
図\ref{fig:pause}に多段階設定法によるポーズ付与例を示す.
まず,$[$A$]$,$[$B$]$,$[$C$]$に順にポーズが設定された後,結合力が求められ
\footnote{$[$A$]$,$[$B$]$,$[$C$]$のアクセント句境界はすでにポーズ
が設定されているため,結合力を求める必要がない.},
結合力=1である$[$D$]$に小ポーズが付与される.
そして,ポーズ未設定区間``実態調査を共同で行うことで合意しました。''
(27モーラ)において最小結合力2($\leq$3)をもつ$[$E$]$に小ポーズが付与される.
ポーズ未設定区間``放射性廃棄物の海洋投棄に''(23モーラ)は,ポーズ未設定
区間長が20モーラ以上30モーラ未満であり,最小結合力が5($>$3)であるため,
ポーズは付与されない.

\begin{figure}[tb]
\vspace{-4mm}
\begin{center}
\epsfile{file=pause.eps}
\end{center}
\caption{ポーズ付与例}
\label{fig:pause}
\end{figure}

\section{評価}\label{sec:evaluation}
\subsection{読み韻律情報の評価法}\label{subs:kana}
本論文で提案した読み韻律情報設定法の有効性を検証するため,AUDIOTEXで生成した
読み韻律情報(アクセント付カナ文)に対する評価を行った.
具体的には,読み付与,アクセント句境界設定,アクセント型設定,ポーズ設定,
読み韻律設定(総合評価)の5種類の正解率を算出した.

ここで,AUDIOTEXでは6種類の音調結合型を付与しているが,本評価ではポーズの
有無の2段階で評価を行うこととした.
これは,現在の合成音声の品質はまだ十分といえず,ポーズなしのアクセント句
境界に対して強結合と弱結合のどちらを設定するのが正しいかを正確に
聞き分けることができないためである.\footnote{一般には,左枝分かれ
境界に対しては強結合,右枝分かれ境界に対しては弱結合を付与するのが
適切であるが,現在の合成音声の品質では,左枝分かれ境界に対しても
弱結合を付与する方が明瞭性が増し,聞きやすい場合がある.このためAUDIOTEXでは,
ポーズを付与しない文節境界にはすべて弱結合を付与している.}

それぞれの正解率は以下の式により表される.
\begin{eqnarray}
読み正解率
          & = & \frac{C_{al}-C_{er}}{C_{al}} \\
アクセント句境界正解率 
          & = & \frac{(C_{al}-C_{ep})-B_{er}}{C_{al}-C_{ep}} \\
アクセント型正解率
          & = & \frac{A_{sp}-A_{te}}{A_{sp}} \\
ポーズ正解率 
          & = & \frac{A_{sp}-A_{pe}}{A_{sp}}\\
読み韻律正解率
          & = & \frac{A_{cr}}{A_{al}} 
\end{eqnarray}
\begin{small}
\begin{tabbing}
\hspace{1.5cm} \= \hspace{1cm} \= \kill
\> $C_{al}$ \> 全文字数 \\
\> $C_{er}$ \> 読みを誤った文字数 \\
\> $C_{ep}$ \> 読み誤り同一アクセント句文字数 \\
\> $B_{er}$ \> 誤ったアクセント句境界数 \\
\> $A_{sp}$ \> 正しく分割されたアクセント句数 \\
\> $A_{te}$ \> アクセント型を誤ったアクセント句数 \\
\> $A_{pe}$ \> ポーズ有無を誤ったアクセント句数 \\
\> $A_{cr}$ \> 読み,アクセント句境界,アクセント型,ポーズ有無が正しいアクセント句数 \\
\> $A_{al}$ \> 全アクセント句数
\end{tabbing}
\end{small}
ここで,アクセント句境界正解率における「読み誤り同一アクセント句文字数」
($C_{ep}$)
とは,読みを誤った文字,および,読みを誤った文字と同一
アクセント句を構成する文字の総数である.
例えば,``通算1アンダー''は``通算''と``1アンダー''の2アクセント句となるが,
``1''の読みを``イチ''と読み誤った場合(正解は``ワン''),読み誤った文字``1''
と同一アクセント句となる``アンダー''も評価対象から除く.
また,アクセント型,音調結合型正解率の分母となる「正しく分割されたアクセント句
数」とは,前後共に正しいアクセント句境界で区切られたアクセント句数を表す.
さらに,「アクセント型を誤ったアクセント句数」($A_{te}$),
「ポーズ有無を誤ったアクセント句数」($A_{pe}$)は,
正しく分割されたアクセント句を対象とする.
このように正解率を定めたのは,正しく与えられた情報を用いた際の設定精度
評価を行うためである.

ところで,アクセント句境界やポーズはゆれが許容され,単一の正解は
存在しない.例えば,``小型処理装置''は,``小型/処理装置''
(2アクセント句)でも``小型処理装置''(1アクセント句)でもよく,
``初めて/(A)相場水準に/触れ、''の(A)の位置にポーズがあっても
\mbox{なくても許容}される.
また,誤りに対する許容度は個人差も存在する.
そこで,合成音声に慣れ,読み韻律情報の誤りの種類の判定を行うことができ,
誤まった読み韻律情報を正しく修正することができる
2名の評価者(評価者A, B)が,AUDIOTEXにより生成された合成音声を聴き,
不自然に聞こえる個所を誤りとして評価を行った.

評価対象としたテキストは,ニュース文章484文(29829文字)である.
評価は,
(1)評価テキストに対して単語辞書を全く整備しない状態で生成された
アクセント付カナ文(オープン評価),(2)評価テキストに対して
単語辞書の整備を行った後に生成されたアクセント付カナ文(クローズ評価),
そして本手法のベースとなった手法である
\cite{Miyazaki1}により生成されたアクセント付カナ文の3種類を対象に行った.

\subsection{読み韻律情報の評価結果}
各正解率を求めるための各値を表\ref{tab:ev_value},正解率を
表\ref{tab:evaluation}に示す.
表\ref{tab:ev_value},表\ref{tab:evaluation}において,「A$\mid$B正解」は
評価者A,B共に正解と判断したもの,および評価者AまたはBのいずれかが
正解と判断したものを正解としたもの,「A\&B正解」は評価者A,B共に正解と
判断したもののみを正解としたものである
\footnote{ただし,表\ref{tab:ev_value}における$C_{er}$,$B_{er}$などの
誤り数においては,「A$\mid$B正解」は,評価者A,Bが共に誤りとしたもの,
「A\&B正解」は,評価者A,B,一方でも誤りとしたものを表している.}
.

\begin{table}[tb]
\caption{評価式の各値}
\label{tab:ev_value}
\begin{center}
\begin{tabular}{|c|l|cccc|}\hline
\multicolumn{2}{|l|}{} & 評価者A & 評価者B & A$\mid$B正解 & A\&B正解 \\ \hline
共通 & $C_{al}$ & \multicolumn{4}{|c|}{29829} \\ \hline
(1)	 & $C_{er}$ & \multicolumn{4}{|c|}{88} \\ 
単語	 & $C_{ep}$ & \multicolumn{4}{|c|}{281} \\ 
辞書	 & $B_{er}$ & 178 & 146 & 107 & 217 \\
未整備	 & $A_{al}$ & \multicolumn{4}{|c|}{7253} \\ 
(オープン)	 & $A_{sp}$ &  7009 & 7055 & 7122 & 6942 \\
	 & $A_{te}$ & 85 & 141 & 57 & 169 \\
	 & $A_{pe}$ & 80 & 375 & 63 & 392\\
	 & $A_{cr}$ & 6721 & 6425 & 6890 & 6256 \\ \hline
(2) 	 & $C_{er}$ & \multicolumn{4}{|c|}{32} \\ 
単語	 & $C_{ep}$ & \multicolumn{4}{|c|}{75} \\ 
辞書	 & $B_{er}$ & 56 & 69 & 27 & 98 \\
整備後	 & $A_{al}$ & \multicolumn{4}{|c|}{7207} \\ 
(クローズ)	 & $A_{sp}$ &  7133 & 7108 & 7164 & 7077 \\
	 & $A_{te}$ & 18 & 86 & 14 & 90 \\
	 & $A_{pe}$ & 82 & 382 & 64 &  400\\
	 & $A_{cr}$ & 7005 & 6617 & 7072 & 6550 \\ \hline
(3) 	 & $C_{er}$ & \multicolumn{4}{|c|}{97} \\ 
(宮崎,	 & $C_{ep}$ & \multicolumn{4}{|c|}{317} \\ 
大山	 & $B_{er}$ & 403 & 466 & 282 & 587 \\
1986)	 & $A_{al}$ & \multicolumn{4}{|c|}{7360} \\ 
	 & $A_{sp}$ & 6762 & 6580 & 6884 & 6458 \\
	 & $A_{te}$ & 324 & 559 & 281 & 602 \\
	 & $A_{pe}$ & 469 & 1198 & 374 & 1293\\
	 & $A_{cr}$ & 5823 & 4751 & 5986 & 4588 \\ \hline
\end{tabular}
\end{center}
\end{table}

\begin{table}[tb]
\caption{読み韻律情報の正解率}
\label{tab:evaluation}
\begin{center}
\begin{tabular}{|l|l|ccccc|}\hline
評価対象 & 種別	& \multicolumn{5}{|c|}{正解率(\%)}\\ \cline{3-7}
 & 	    	& 評価者A & 評価者B & A$\mid$B正解 & A\&B正解 & A,B平均\\\hline
(1) & 読み	& \multicolumn{5}{|c|}{99.70} \\
単語 & アクセント句境界 & 99.39 & 99.51 & 99.64 & 99.27 & 99.45 \\
辞書 & アクセント型	& 98.79 & 98.00 & 99.20 & 97.57 & 98.40 \\
未整備 &ポーズ		& 98.86 & 94.68 & 99.12 & 94.35 & 96.77 \\
(オープン) & 読み韻律	& 92.67 & 88.58 & 95.00 & 86.25 & 90.63 \\ \hline
(2) & 読み	& \multicolumn{5}{|c|}{99.89} \\
単語 & アクセント句境界 & 99.81 & 99.77 & 99.91 & 99.67 & 99.79 \\
辞書 & アクセント型	& 99.75 & 98.79 & 99.80 & 98.73 & 99.27 \\
整備後 & ポーズ		& 98.85 & 94.63 & 99.11 & 94.35 & 96.74 \\
(クローズ) & 読み韻律	& 97.20 & 91.81 & 98.13 & 90.88 & 94.51 \\ \hline
(3) & 読み	& \multicolumn{5}{|c|}{99.67} \\
(宮崎, & アクセント句境界 & 98.63 & 98.42 & 99.04 & 98.01 & 98.53 \\
大山   & アクセント型	& 95.21 & 91.50 & 95.92 & 90.68 & 93.36 \\
1986)  & ポーズ		& 93.06 & 81.79 & 94.57 & 79.98 & 87.43 \\
       & 読み韻律	& 79.12 & 64.55 & 81.33 & 62.34 & 71.84 \\ \hline
\end{tabular}
\end{center}
\end{table}

ここで,(2)のクローズ評価用に行った単語辞書の整備は,
新語登録(人名等)48語,アクセント句境界設定用情報整備27語,
アクセント型設定用情報整備24語,
複合語意味的係り受け情報を正しく設定するため,および同型語認定精度を
向上させるための情報(意味カテゴリ,承接属性など)整備18語,
不適切な単位や誤った内容で登録されている単語の削除4語であった.

\subsubsection{評価者間の比較}
評価者Aと評価者Bの評価結果を比較すると,全体的な傾向は同じ,すなわち,
(2)クローズ評価,(1)オープン評価,(3)\cite{Miyazaki1}の手法の順に
各正解率が高いといえる.
ただし,全体的に評価者Bの方が評価者Aより正解率が低い.
これは,韻律のゆれの許容範囲は個人差が大きいことを表している.
特に,アクセント型では,評価者Bが判定した誤りは評価者Aの約2〜5倍,
ポーズ有無では,約3〜5倍となっている.
このうち,(2)クローズ評価のアクセント型誤り($A_{te}$)において,
評価者Bのみ誤りとしたアクセント句のうち,44\%(32件)は,
設定されたアクセント型は1型以上であるが,評価者Bは0型(平板型)と
判断したものであった(例:キノ'ー(昨日),クワワ'ル(加わる),
ス'ーカイ(数回)).
また,(2)クローズ評価のポーズ有無誤り($A_{pe}$)において,
評価者Bのみ誤りとしたアクセント句のうち,12\%(37件)は,
連体修飾句が修飾する複合語が複数アクセント句である場合の,
連体修飾句と複合語の間のポーズであった
(例:``大統領の/(A)財政赤字/削減策に''における(A)の位置のポーズ).
また,10\%(31件)は,数量表現内のポーズであった(例:``およそ/(B)七億/
七千万円の''における(B)の位置のポーズ).

\subsubsection{評価対象間の比較}
表\ref{tab:evaluation}の(1)オープン評価と(2)クローズ評価を,
評価者A,Bが共に誤りとしたもの(A$\mid$B正解)を対象として比較すると,
(2)クローズ評価において,読みの誤り($C_{er}$)が\mbox{約$\frac{1}{3}$,
アクセ}ント句境界($B_{er}$),アクセント型($A_{te}$)の誤りが約半数
に減少している.
この減少分は,未知語の新規登録,複合語係り受け用情報の整備による
形態素解析精度の向上,および,表\ref{tab:dic}に示した読み韻律付与用単語情報の
適正さによって改善される割合を示し,本手法ではこの単語辞書情報の正確さが
読み韻律情報の精度に大きな影響を与えるといえる.
ここで,(2)クローズ評価においてポーズ有無誤り($A_{pe}$)が微増しているのは,
読み,アクセント句境界が誤った部分はポーズ評価の対象としていないため,
ポーズにおいては,形態素解析誤りに起因する誤りはほとんどなく,
また,辞書整備により読み,アクセント句境界誤りが減少したために,ポーズ評価の
対象となるアクセント句数が増え,さらに,
ポーズ付与のための単語辞書情報は1つしかない(助詞ポーズ属性)ため,
辞書整備の影響が少ないからである.

(3)の\cite{Miyazaki1}で提案された手法と比較すると,本提案方式((1),(2))は
ポーズ正解率の向上が顕著であり,新たに導入した多段階設定法によるポーズ付与,
および形態素解析における長単位認定が有効であるといえる.

\vspace{2mm}
\subsubsection{(2)クローズ評価における誤り原因}
(2)クローズ評価において,評価者A,Bが共に誤りとしたもの
(A$\mid$B正解)を対象として,各誤りの原因を以下に示す.
\begin{description}
\item[読み誤り]1文節1自立語における同型語の読み分け誤りが75\%(24件)と
誤り原因の大半を占めた.このような同型語の読み分けが多段解析法による
形態素解析の課題であるといえる.
\\
(例)\\
\hspace*{1cm}香港ドルと{\dg 元}(×モト→○ゲン)の交換\\
\hspace*{1cm}{\dg 大勢}(×オオゼー→○タイセー)が判明
\item[アクセント句境界誤り]主要な誤り原因は以下の2点である.
\begin{description}
\item[形態素解析誤り]
品詞認定誤り,単語境界誤り等の形態素解析誤りが原因となる誤りが33\%(9件)を
を占めた.\\
(例)\\
\hspace*{1cm}×国会会/期末 (○国会/会期末):\\
\hspace*{1.5cm}``国会(固有名詞)+会(接尾辞)+期末(名詞)\hspace{-2mm}''
と誤認定されるため\\
\hspace*{1.5cm}(正解:``国会(固有名詞)+会期(名詞)+末(接尾辞)'')
\item[数量表現における誤り]
数量表現において,通常は数詞に承接しない単語等が承接した場合に,
アクセント句境界を誤った.(11\%(3件))\\
(例)\\
\hspace*{1cm}×捜査/一課 (○捜査一課):``捜査''と``一''が関連付けできない
\end{description}
\item[アクセント型誤り]
品詞の認定誤り等(引き上げ:×動詞,○転生名詞)の形態素解析誤りが
原因となる誤りが36\%(5件)であった.
\item[ポーズ有無誤り]
文節間係り受け解析を行っていないことに起因し,係り受けのない``連体形+名詞'',
``が(助詞)+用言''間等,および,並列句での誤りが20\%(13件)を占めた.\\
(例)\\
\hspace*{1cm}×5百万ポンドと\{弱結合\}前の\{小ポーズ\}年の/同じ時期に/比べ\\
\hspace*{1cm}(○5百万ポンドと\{小ポーズ\}前の\{弱結合\}年の/同じ時期に/
比べ)
\end{description}
また,新たに導入した多段階設定法では,係り受けがありえないアクセント句
境界前後の単語の組み合わせについては,極端に結合力を弱く設定したが,

総桐タンスを\{小ポーズ\}納入などと/報道されました。\\
\hspace*{1cm}(結合力=2, ポーズ付与の補正により小ポーズ付与)\\
の``を+納入''のような体言止め等の表現に対応できず,
\mbox{実際には係り受け関係があるにも関わ}らず,誤ってポーズが付与され,
\cite{Miyazaki1}よりディグレードした場合が5\%(3件)生じた.

\subsection{処理性能評価}
\ref{subs:kana}節で用いた評価文(29829文字)に対するテキスト解析部の
処理時間(アクセント付カナ文を生成するまでの時間)を測定した結果を,
表\ref{tab:speed}に示す.使用マシンは,Sun Sparc Station 20である.
表\ref{tab:speed}により,\cite{Miyazaki1}と比較して,AUDIOTEXが約3.9倍も
処理時間がかかることがわかる.この理由として,
形態素解析における用言語幹語尾1語化処理の追加,および,
読み韻律情報設定における規則の追加が考えられる.

\begin{table}[bt]
\caption{テキスト解析処理の処理時間}
\label{tab:speed}
\begin{center}
\begin{tabular}{l|c|c}\hline
	& 処理時間	& 100文字あたりの処理時間 \\ \hline
AUDIOTEX & 280.22秒	& 0.94秒 \\
\cite{Miyazaki1} & 72.57秒	& 0.24秒 \\ \hline
\end{tabular}
\end{center}
\end{table}

しかしAUDIOTEXでは,読み韻律情報設定と平行して合成音声の出力を行うことが
可能であり,100文字の読み上げ時間は約15〜20秒程度であるため,実用上十分な
処理速度であるといえる.

\section{おわりに}\label{sec:conclusion}
多段解析法による形態素解析より得られる単語情報を用いて,読み韻律情報を
規則により設定する方法,および読み韻律情報設定のための単語辞書の構成
を示した.そして,特に音調結合型(ポーズ)の設定において,
多段階設定法を新たに導入し,文節間の係り受け情報を用いなくても,
実用上十分な精度でポーズを付与できることを示し,本手法の有効性を
確認した.

本手法による読み韻律情報の設定精度は,単語辞書の精度に大きく依存
しているため,今後は単語辞書の精度向上を容易に行う手法を考えていく必要が
あると考えられる.

\vspace{-3mm}
\bibliographystyle{jnlpbbl}
\bibliography{v06n2_04}

\begin{biography}
\biotitle{略歴}
\bioauthor{浅野 久子}{
1991年横浜国立大学工学部電子情報工学科卒業.
同年,日本電信電話(株)入社.
音声合成,情報抽出のための自然言語処理,テキスト処理の研究に従事.
現在,NTT情報通信研究所知的通信処理研究部勤務.
情報処理学会会員.}

\bioauthor{松岡 浩司}{
1979年九州大学工学部電子工学科卒業.
同年,日本電信電話公社(現NTT)入社.
現在,NTTマルチメディアシステム総合研究所関西リモートオフィスにて
電子図書館システムの研究開発に従事.
形態素解析,音声合成,情報検索の研究に興味を持つ.
情報処理学会,電子情報通信学会各会員.}

\bioauthor{高木 伸一郎}{
1981年金沢大学大学院電気工学専攻修士課程修了.
同年,日本電信電話公社(現NTT)入社.
日本語形態素解析を用いた校正支援システムなど知的支援サービスの開発に従事.
現在,NTT情報通信研究所知的通信処理研究部主幹研究員.
情報処理学会,電子情報通信学会各会員.}

\bioauthor{小原 永}{
1979年慶応義塾大学大学院電気工学専攻修士課程修了.
同年,日本電信電話公社(現NTT)入社.
機械翻訳,推敲支援,韻律生成技術の研究に従事.
現在,NTT情報通信研究所知的通信処理研究部主幹研究員.
情報処理学会,電子情報通信学会各会員.}

\bioreceived{受付}
\bioaccepted{採録}

\end{biography}

\end{document}
