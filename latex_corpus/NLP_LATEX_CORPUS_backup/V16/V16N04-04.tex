    \documentclass[japanese]{jnlp_1.4}
\usepackage{jnlpbbl_1.2}
\usepackage[dvips]{graphicx}
\usepackage{amsmath}
\usepackage{hangcaption_jnlp}
\usepackage{udline}
\setulminsep{1.2ex}{0.2ex}
\let\underline

\newcommand{\subsubsectionX}[1]{}

\usepackage{ascmac}
\def\mod{} 
\def\modl{} 
\def\moda{}
\def\modk{}


\Volume{16}
\Number{4}
\Month{October}
\Year{2009}

\received{2008}{9}{2}
\revised{2009}{4}{11}
\accepted{2009}{7}{17}

\setcounter{page}{65}


\jtitle{相互教授モデルに基づく学習者向け作文支援システムの実現}
\jauthor{山口 昌也\affiref{Author_1} \and 北村 雅則\affiref{Author_2} \and
棚橋 尚子\affiref{Author_3}}
\jabstract{
本論文では,学習者向けの作文支援手法として,学習者,教師,システム間で互
いに作文に関する知識を教えあう相互教授モデルを提案し,{\moda Webベースの
作文支援システムを実現した.}適用対象として,大学の作文教育を想定する.対
象とする文章は,レポートなどの,一定の書式と文章構造が規定される文章とす
る.従来の作文支援システムの問題点として,(a)文章構成や作文の内容に対する
支援など,意味処理が必要となる支援が困難であること,(b)教師の指導意図をシ
ステムの動作に反映させる仕組みを持たず,実際の授業で運用しにくいことが挙
げられる.相互教授モデルでは,作文の言語的・内容的制約を記述する
「作文規則」を用いて,教師からシステムへの指導意図の伝達を可能にした.さ
らに,学習者による作文へのマークアップ,および,学習者同士の添削を導入し,
文章構成や作文の内容に対する支援を含めた作文支援を行う.システムは学習者
のマークアップ結果を利用しつつ,作文規則を作文に適用し,誤りを指摘する.
{\mod 本論文では,提案手法,従来手法による作文実験を行い,相互教授モデルと作文
規則の有効性を確認した.}
}

\jkeywords{作文支援,相互教授モデル,作文規則}

\etitle{An Implementation of a Writing Aid System for Students Based on
a Mutual Teaching Model}
\eauthor{Masaya Yamaguchi\affiref{Author_1} \and Masanori
kitamura\affiref{Author_2} \and Hisako Tanahashi\affiref{Author_3}} 
\eabstract{
This paper proposed a mutual teaching model for assisting students in
writing compositions, and implemented a writing aid system as a Web
application based on the model where students, teachers and our system
teach each other their knowledge of writing. We designed the system to
use in first language writing courses in the university. The existing
systems have two problems: (a) the limitation of assistance for
structure or contents of composition, (b) few mechanisms that allow
teachers to incorporate their educational objectives into systems. In
our proposed model, a student annotates on his/her own composition and
makes comments on other's compositions. And teachers define
``Composition Rules'' for incorporating their educational objectives
into systems. Using the rules and results of the annotation, our system
provides various assistance for also structure or contents of
composition. {\mod By the proposed model and a coventional model, we made
two composition experiments whose results showed the effectiveness of
the proposed model and ``Composition Rules''.}
}

\ekeywords{Computer-Aided Writing, Mutual Teaching Model, Composition Rules}

\headauthor{山口,北村,棚橋}
\headtitle{相互教授モデルに基づく学習者向け作文支援システムの実現}

\affilabel{Author_1}{国立国語研究所}{The National Institute for Japanese Language}
\affilabel{Author_2}{名古屋学院大学}{Nagoya Gakuin University}
\affilabel{Author_3}{奈良教育大学}{Nara University of Education}


\begin{document}
\maketitle

{\mod
\section{はじめに}

\subsection{本研究の背景}\label{ssec:background}

近年,大学では文章能力向上のため,「文章表現」の授業がしばしば行われてい
る.実際に作文することは文章能力向上のために有効であることから,多くの場
合,学生に作文課題が課される.しかし,作文を評価する際の教師の負担は大き
く,特に,指導する学生数が多いと,個別の学生に対して詳細な指導を行うこと
自体が困難になる\footnote{筆者の一人は,1クラス30名程度のクラスを週10コ
マ担当している.延べ人数にして約300名の学生に対して,毎週添削してフィード
バックすることは極めて困難であるため,半期に数回課題を提出させ,添削する
に留まっている.}.{\modk また,講義だけで,個別の指導がない授業形態では,
学生も教師の指導意図をつかみにくく,ただ漠然と作文することを繰り返すといっ
た受け身の姿勢になりがちである.}

本研究は,上記のような現状に対処するために,大学における作文教育実習で
{\modk 活用できる}学習者向け作文支援システムを提案するものである.


\subsection{既存システムの問題点}\label{ssec:problems}

これまでに多くの作文支援システムが提案されてきた.支援手法という観点から
既存の手法を分類すると,次のようになる.

\begin{enumerate}
\def\theenumi{}
\item 作文中の誤りを指摘する手法
\item 作文する際の補助情報を提供する手法
\item 教師の指導を支援する手法
\item 作文を採点する手法
\end{enumerate}

(a)の手法は,ワードプロセッサなどのスペルチェッカや文法チェッカとして,広
く利用されている.また,より高度な文章推敲や校閲を支援するための手法
\cite{umemura2007,笠原健成:20010515}も考案されている.教育分野への適用で
は,第2言語学習者向けの日本語教育分野での研究が盛んである.例えば,第2言
語学習者の誤りを考慮して,文法誤りなどを指摘する手法
\cite{chodorow2000,imaeda2003,brockett2006}がある.

さらに,(b)の手法としては,文章作成時の辞書引きを支援する手法
\cite{takabayashi2004},翻訳時にコーパスから有用な用例を参照する手法
\cite{sharoff2006}などがある.これらは,学習者用というよりも,ある程度す
でに文章技術を習得している利用者向けの手法である.

(c) のアプローチは,学習者を直接支援するのではなく,作文指導を行う教師を
支援することにより,間接的に学習者の学習を支援する手法である.この種のア
プローチの例としては,教師の添削支援システム\cite{usami2007,sunaoka2006} 
に関する研究がある.これらの研究では,日本語教育の作文教育において,作文
とそれに付随する添削結果をデータベースに蓄積し,教師の誤用分析などを支援
する.

(d)の手法は,小論文などの文章試験を自動的に採点することを目的に開発されて
いる手法である.代表的なシステムとしては,英語の小論文を自動採点する,
ETS の e-rater \cite{burstein1998} がある.また,e-raterを組み込んだオンラ
イン作文評価システムCriterion\footnote{http://criterion.ets.org/}も開発さ
れており,grammar, usage, mechanics, style, organization \& development
という観点から作文を評価し,誤りの指摘などもあわせて行われる.なお,日本
語でも,e-rater の評価基準を踏襲して,石岡らが日本語小論文評価システム
Jess \cite{ishioka-kameda:2006:COLACL}を構築している.また,井上らが Jess
を Windows 用に移植し,大学において日本語のアカデミックライティング講座へ
の導入を検討している\cite{井上達紀:20050824}.

以上の手法のうち,学習者を直接支援対象としうる手法は,(a)(d)である.大学
における作文実習に,これらの手法を適用することを考えた場合,次の二つの問
題があると考える.


\subsubsectionX{問題点1: 意味処理が必要となる支援が困難なこと}
大学の文章表現では,レポート,論文,手紙,電子メール,履歴書などを題材と
して,表記・体裁,文法,文章構成(例:テーマに即した文章の書き方,論理的
な文章の書き方),要約の方法,敬語の使い方など,広範囲な文章技術を習得対
象としている\cite{shoji2007,okimori2007}.

それに対して,現状の作文支援システムは,表記・文法に関しては,手法(a)(d)
で誤りの指摘が行われているが,意味的な解析が必要となる支援については,部
分的に実現されるにとどまっている.例えば,前述のCriterion では,導入部
(introduction material)や結論部(conclusion)などの文章要素を自動的に認識し,
それぞれの部分の一般的な記述方法を表示することができる.しかし,現在の自
然言語処理技術では,学習者の支援に耐えうるほどの精度で意味解析を行うこと
は難しい.そのため,作文課題に必要な記述が含まれているか\footnote{例えば,
得意料理の作り方を記述する課題では,材料や料理手順に関する記述は必須的な
内容であろう.},記述内容の説明が不足していないか,意味的な誤りや矛盾はな
いか,といった深い意味解析を必要とする支援は困難である.


\subsubsectionX{問題点2: 教師の指導意図をシステムの動作に十分反映できないこと}
{\modk 前述のとおり,教師が用意する作文課題には,学術的なものから実社会で
役立つものまで様々なものがある.各課題を課す際には,学習者の作文の質を向
上させるために,それぞれの目的に応じた到達目標やそれに応じた学習支援を設
定する.したがって,}教師が実習で作文システムを利用するには,課題の内容に
応じて,教師がシステムの支援内容をコントロールできなければならない.例え
ば,電子メールの書き方を習得するための課題であれば,電子メールに書かれる
べき構成要素(例:本文,結び,signatureなど)が{\modk 存在するか,また,}適
切な順序で書かれているかを検査し,誤りがあれば,指摘するという支援が考え
られる.このような支援を行うためには,電子メールに書かれるべき構成要素と
その出現順序を,教師が規則として作文支援システム中で定義できなければなら
ない.

現状の作文支援システムの中では,手法(d)の作文採点システムが,作文評価用の
パラメータの設定手段を持っている(自動採点システムにおける作文評価手法は
\cite{石岡恒憲:20040910}に詳しい).例えば,Windows版 Jess の場合は,修辞,
論理構成に関する各種パラメータの採点比率,および,内容評価用の学習用文章
をユーザが指定できるようになっている.このように,既存の規則のパラメータ
を設定することは可能である.{\modk しかし,教師が新たな規則を定義できるま
でには至っておらず,教師の指導意図をシステムの動作に反映することは難しい
のが現状である.}


\subsection{本研究の目的}

そこで,本研究では,上記の二つの問題を解決するための手法を提案し,作文支
援システムとして実現する.

まず,問題点1に対しては,「相互教授モデル」を導入する.このモデルでは,学
習者,教師,システムが互いの作文知識を教授しあうことにより,学習者の作文
技術を向上させる.従来のシステムのように,作文支援システムだけが学習者に
作文技術を教授するのではなく,学習者・システム間,学習者同士で作文技術を
教授しあうことにより,システム単独では実現できない,深い意味処理が必要で,
多様な文章技術に対する支援を可能にする.

また,問題点2に対しては,「作文規則」を用いる.この規則は,学習者の作文
の構造,および,内容を規定するための規則である.教師は,作文課題に基づい
て作文規則を決定する.システムは,作文規則に基づいて,学習者の作文をチェッ
クし,誤りがあれば,それを指摘する.本稿では,作文規則の形式,作文への適
用方法について示す.


本論文の構成は,次のようになっている.まず,\ref{sec:system_structure}章
ではシステムの構成について述べる.\ref{sec:model}章では相互教授モデルの提
案を行い,\ref{sec:composition_rule}章では作文規則の定義と作文への適用方
法を示す.さらに,提案手法の有効性を検証するために,\ref{sec:experiment}
章で提案手法・従来手法による作文実験を行い,\ref{sec:evaluation}章で実験
結果を評価・考察する.そして,最後に\ref{sec:conclusion}章でまとめを述
べる.}


{\mod
\section{システム構成}\label{sec:system_structure}

本システムの構成を図\ref{fig:system}に示す.本システムは,Webサーバ上の
Wikiとして動作し,教師が作文課題用のサイトを構築する.学習者は,Webブラウ
ザを介して作文するとともに,互いの作文を添削する.システムは,教師の設定
した「作文規則」に基づいて,学習者の作文をチェックする.作文,および,添
削結果は,WikiコンテンツとしてWebサーバ上に格納されていく.システムは,教
師に対して,作文を分析する手段を提供する.


\begin{figure}[b]
 \begin{center}
  \includegraphics{16-5ia5f1.eps}
  \caption{システム構成}
  \label{fig:system}
 \end{center}   
\end{figure}

本システムの機能は,Wikiのプラグインとして実現される\footnote{Pukiwiki
(http://pukiwiki.sourceforge.jp/)を利用している.}.プラグインは,次の4
種類に大別される.


\paragraph{編集関連プラグイン:}WYSIWYGエディタとして機能する\footnote{
	TinyMCE (http://tinymce.moxiecode.com/)を拡張している.}.
エディタとしての基本的な機能の他,学習者による作文へのマークアップ,Webサーバへ
の作文の保存(排他処理を含む)ができる.作文結果は,XHTMLとしてWikiコンテ
ンツの中に埋め込まれる.図\ref{fig:edit}に編集プラグインの実行例を示す.

\paragraph{エラー検出プラグイン:}「作文規則」に基づき,作文をチェックす
る.チェックは,作文を保存する際に行われ,結果はWikiページに保存される.
作文規則は,\ref{ssec:model_student_sys}節,\ref{sec:composition_rule}節
で詳しく説明する.



\paragraph{添削関連プラグイン:} 学習者同士の添削を支援する.添削時には,
まず,作文の該当箇所にハイパーリンクを作成する.添削の内容は,リンク先の
Wiki のページに記入する.図\ref{fig:edit}中の「冗」「口」\footnote{それぞ
れ「冗長表現」「口語表現」を指摘する添削である.}などのアイコンは,ハイパー
リンクの例である.

\paragraph{作文分析支援プラグイン:} 作文中の文字数,文数,段落数,添削
数などの統計や作文の検索など,主として教師が学習者の作文を分析するための
役割を果たす.


\begin{figure}[t]
 \begin{center}
  \includegraphics{16-5ia5f2.eps}
  \caption{編集プラグインの実行例}
  \label{fig:edit}
 \end{center}   
\end{figure}


Wikiを利用している理由の一つは,Webコンテンツを作成するのが容易なことであ
る.Wikiは,HTMLのタグではなく,Wikiの簡易的なタグでWebページを作成できる.
作文教育の担当教師は,必ずしも Webシステムの利用やWebコンテンツの作成に精
通しているとは限らない.しかし,もし,作文支援システムを含めたWebコンテン
ツを容易に作成できれば,授業の資料を含めた形の作文用のWebサイトを構築す
るという形で,作文支援システムを授業の中に導入しやすくするなると考えられ
る.

Wikiを利用した,もう一つの理由は,プラグインとして実装することにより,機
能の拡張やその利用が容易になるためである.作文課題のテーマや授業での利用
方法は多様であり,システムの拡張性やその利用の容易性は重要である.Wikiの
プラグインとして実装すれば,必要に応じて,機能を拡張することが可能である
と同時に,拡張したプラグインを通常のプラグインと同様にWikiコンテンツに埋
め込むことができる.
} 



\section{相互教授モデル}\label{sec:model}

\subsection{概要}\label{ssec:model_abst}

本節では,本システムの特徴である「相互教授モデル」について述べる.

このモデルは,学習者,教師,システムが互いの作文に関する「知識」を教授し
あうことにより,学習者の作文を支援する.概要を図\ref{fig:interaction}に
示す.この図のとおり,このモデルには,教師$\Leftrightarrow$システム,学
習者$\Leftrightarrow$システム,学習者$\Leftrightarrow$学習者,という三つ
のタイプのインタラクションがある.

\begin{figure}[b]
 \begin{center}
  \includegraphics{16-5ia5f3.eps}
  \caption{相互教授モデルにおけるインタラクション}
  \label{fig:interaction}
 \end{center}   
\end{figure}

このモデルの中で,学習者は二人以上を想定する.通常は,10名以上の学習者を
想定し,特定の学習者間だけでなく,複数の学習者間でインタラクションが行わ
れるようにする.これは,後述するように,学習者間のインタラクションにおい
て,誤った作文知識が教授されるのを防ぐためである.

教師は作文規則をシステムに「教授」し,システムがその作文規則に基づいて,
個別の学習者とインタラクションしつつ,作文をチェックする.

実際の授業で利用する場合は,次の手順で示すように,作文課題に関連する授業
と本モデルとを組み合わせて運用することを想定している.

\begin{description}
 \item[手順1] 教師が,学習者に対して作文を書く前の事前授業を行う.
 \item[手順2] 教師がシステムに対して「作文規則」
	    (\ref{sec:composition_rule}節を参照のこと)を設定する(教師
	    $\Rightarrow$システム).
 \item[手順3] 学習者がそれぞれ作文する.この際,自分の作文に対して,各種
	    のマークアップを行う(学習者$\Rightarrow$システム).また,作
	    文の過程で,システムが学習者の作文をチェックし,チェック結果
	    に基づいて,学習者が自分の作文を修正する(システム
	    $\Rightarrow$学習者).
 \item[手順4] 他の学習者の添削を行う(学習者$\Leftrightarrow$学習者).
 \item[手順5] 学習者が添削結果に基づき,自分の作文の修正を行う(学習者
	    $\Leftrightarrow$学習者).
 \item[手順6] システムが教師に対して,作文の分析支援を行う(システム
	    $\Rightarrow$教師)
 \item[手順7] 分析結果に基づき,授業で教師が学習者を指導する.
\end{description}

この後の節では,それぞれのインタラクションでどのような作文知識が教授され,
どのように学習者の知識獲得につながるのか,ということを説明する.


\subsection{教師 $\Leftrightarrow$ システム}

まず,教師からシステムへは,教師自身が設定する「作文規則」が教授される
(手順2).作文規則は,作文の表記,文法,文体,文章構造などの言語的な規則
に加えて,作文の形式や内容を規定する規則である.一般的には,作文課題ごと
に学習者に教授したい事柄(以後,「指導項目」)が存在するので,その指導項目
に基づいて,作文規則が決定される.なお,指導項目は,教師が手順1で授業す
る内容と基本的に対応する.

例えば,本論文の\ref{sec:experiment}節で実施した実験では,「章立て」の習
得を目的として,「旅行計画」をテーマとする作文課題を出している.教師は,
この目的とテーマを満たす作文が書かれるように,作文規則を設定する.この実
験の場合,1文の長さ,文体(です・ます調で書く)といった一般的な作文上の規
則に加えて,「章立て」に関する作文規則を設定した.さらに,作文のテーマに
則して,必ず記述しなければならない項目(以後,「必須記述項目」)を設けて,
作文規則とした.具体的には,「旅行計画」というテーマから,「旅行目的」
「スケジュール」などを必須記述項目としている.

一方,システムから教師へは,学習者が作成した作文を分析するための機能が提
供される(手順6).具体的には,(1)作文規則の作文への適用結果の集計,(2)作
文中の文字,文,段落数などの集計,(3)全作文に対する横断的な全文検索,(4) 
学習者同士の添削結果の集計などである.教師は,これらの機能を使って,全学
習者の作文を分析(例えば,誤りの傾向分析)をした上で,授業で学習者を指導す
る(手順7).



\subsection{学習者 $\Leftrightarrow$ システム}
\label{ssec:model_student_sys} 

学習者とシステムとの間のインタラクションは,学習者が作文する過程で行われ
る(手順3).学習者は,自分の作文に対して,マークアップを加える.それに対
して,システムは,作文に対する自動的なマークアップと,学習者のマークアッ
プ結果を利用しつつ,作文規則を作文に適用する.そして,図\ref{fig:check}の
ように,作文規則に適合しない部分を学習者に指摘する.


\begin{figure}[b]
\input{05fig04.txt}
\caption{システムによるチェック例}\label{fig:check}
\end{figure}


学習者が実際にどのようなマークアップを行うかは,教師の指導項目や自動的な
マークアップの精度などに応じて決定する.例えば,\ref{sec:experiment} 節の
実験では,章節タイトル,引用部分などの言語的な要素の他に,「旅行目的」
「スケジュール」といった必須記述項目の記述範囲に対するマークアップを行っ
ている.{\modl これは,この実験の教育目的が「章立て」の習得であり,作文テーマが
「旅行計画」だからである.このように教育目的上,重要な部分については,自
動的なマークアップの精度にかかわらず,学習者によるマークアップを行う.}


学習者のマークアップが,学習者の作文技術習得に与える効果は,二つある.一
つは,学習者自身が自分の作文に対してマークアップすることによって,教師が
学習者に対して習得してほしいと考えている事柄を学習者が自覚的に確認しつつ,
作文を行うことができることである.また,マークアップを行うことは,他者で
あるシステムに対して作文知識を「教える」ことになる.したがって,CAI 関連
研究において,\cite{kotani1989}や\cite{大林史明:20001215}が主張しているよ
うに,学習者の自発的な学習を促し,学習者自身の作文知識が整理・詳細化され
ることが期待される.

学習者がマークアップすることの,もう一つの効果は,現在の自然言語処理技術
では十分な精度で解析することが困難な対象に対しても,マークアップが可能に
なることである.現在のシステムでは実現困難なマークアップ処理を学習者が行
えば,作文規則にそのマークアップ結果を取り込んで,学習者の作文を検査する
ことが可能になる.

例えば,上記でマークアップの例として挙げた「旅行目的」を自動的に検出す
るには,表層的な解析のみならず,意味的な解析が必要になると思われる.しか
し,学習者がマークアップを行えば,「旅行目的」が記述されているか否かを
機械的に判断し,記述されていない場合は,学習者にその誤りを指摘できる.



\subsection{学習者 $\Leftrightarrow$ 学習者}
学習者間のインタラクションは,互いに作文に対して,コメントしたり,質問し
あうことである.コメントや質問の内容についての制限はないが,(a)誤字・脱
字の指摘, (b)語の用法や文法の誤りの指摘,(c) 内容に対する質問,(d) 文章
構成などの改善案などを,教師が学習者に事前に推奨しておくものとする.学習
者が行ったコメント,質問などは,システムが管理し,学習者同士が掲示版シス
テムを用いて対話できるようにする.図\ref{fig:correction}に添削例を示す.

学習者間のインタラクションが学習者の作文技術習得に与える効果は,学習者と
システム間のインタラクションと同様,学習者が他人に教授することによる効果
と,広範な支援内容を実現できる点にある.ただし,その効果の内容は異なる.

\begin{figure}[b]
 \begin{center}
  \includegraphics{16-5ia5f5.eps}
  \caption{添削例}
  \label{fig:correction}
 \end{center}   
\end{figure}

まず,他人に教授することの効果について{\moda 説明する.}学習者が他の学習
者の作文に対してコメントや質問をすることは,コメントするだけの知識を自ら
持たなければならず,学習者自身の知識が整理・詳細化されると考えられる.ま
た,他人の作文を読むということは自分の作文作成の参考にもなる.自分の作文
に対してマークアップする場合と異なるのは,添削した相手や他の学習者から添
削自体に対する反応があることである.そのため,学習者(添削者,被添削者)
の作文知識が矯正・強化される可能性がある.{\modl 具体的には,(a)誤った知識に基づ
いて他人の作文を添削した場合でも,学習者同士の対話の過程で誤った知識が矯
正されうること,(b)複数の添削者が同じ内容の添削を行えば,その添削内容の信
頼性が被添削者や他の添削者にも伝わること\footnote{図\ref{fig:correction}
は,(b)の例である.},などが挙げられる.}

次に,広範な支援内容を実現できるという点について説明する.学習者とシステ
ム間でのインタラクションでも同様の効果が得られるが,学習者間のインタラク
ションはそれを補完する役割を果たす.つまり,学習者・システム間のインタラ
クションでは,作文規則に記述できる範囲でしか支援しかすることができない.
それに対して,学習者間のインタラクションでは,添削する側の学習者の作文知
識の範囲での支援が可能である.

以上のような学習効果がある一方,添削するのが学習者であり,専門家でないこ
とから,添削内容に誤りが含まれる可能性を考慮しなければならない.そこで,
本モデルでは,二つの対策を考えている.一つは,一人の学習者の作文に対して,
複数の学習者が添削し,添削内容について議論できるようにすることである.も
う一つは,添削用の掲示板システム上に,教師が部分的に介入することにより,
添削者・被添削者が互いに解決できないような場合に対処することである.


{\mod
\subsection{他の教授モデルとの比較}

本節では,ユーザ(学習者)による知識の教授という面から,相互教授モデルと
既存の教授モデルとを比較する.

まず,学習者同士の教授という面から見てみると,作文教育では従来から学習者
同士の添削を導入している.例えば,国語教育では,学習者同士が作文を交換し
あって読みあわせる「相互推敲」と呼ばれる手法が用いられている
\cite{tazika2006}.また,第2言語学習者に対する作文教育でも,学習者同士で
推敲しあう,ピア・レスポンスと呼ばれる手法が導入され,成果を挙げている
\cite{harada2006}.

これらの教授モデルと相互教授モデルとの違いは,本教授モデルでは,教師の指
導項目が習得されているかをチェックする手段が,第三者であるシステムに作文
規則として取り込まれている点である.教師の指導意図が学習者に伝わっている
かを確認する場合,従来のモデルでは,各学習者の作文を個別に確認しなければ
ならないが,本教授モデルでは,作文規則を用いて教師の指導意図を徹底させつ
つ,学習者同士の知識教授を実現することができる.このことは,学習者の習得
結果を確認しづらい,多人数を対象とした授業において,特に有効である.


次に,工学的な見地,つまり,ユーザ・システム間の知識教授という側面から本
モデルと既存手法とを比較する.従来から,自然言語処理システムでは,全自動
で十分な精度の解が得られない場合,ユーザとのインタラクションが用いられて
きた.最も一般的に利用されているのは,仮名漢字変換システムである.
\ref{ssec:problems}節の問題点2で取り上げた意味処理に関しても,GDA \cite{橋
田浩一:19980701}を用いて,ユーザが意味的情報をアノテーションする手法が提
案されている(例えば,\cite{綾2005}による要約生成).

本モデルが既存手法と異なる点は,(1)アノテーションが学習者にとって手間にな
るのではなく,作文技術を習得する助けになる(\ref{ssec:model_student_sys}
参照)という,積極的な意味を持つこと,(2)複数の学習者が存在するため,アノ
テーションの誤りを修正できる可能性があることである.二つ目の特徴は,誤っ
たアノテーションを行う可能性がある学習者をユーザとする作文支援システムに
とっては,アノテーション結果を有効利用する上で重要である.} 



\section{作文規則}\label{sec:composition_rule}

\subsection{定義}

作文規則は,学習者の作文が教師の指導項目に適合しているか検査するための規
則である.本システム上で作成される作文は,XML を用いて,内部的に構造化さ
れており,作文規則はその構造に対して適用される.作文規則の例を表
\ref{tbl:example_composition_rule}に示す.なお,これらの規則は,
\ref{sec:experiment}節の実験で,実際に使用した作文規則の一部\footnote{表
\ref{tbl:example_composition_rule}には,「章」の規則しか記載していないが,
実際には,「節」「小節」「小々節」に関する規則がある.「章」と類似する規
則となるので,省略した.}
である.

作文規則は,次の4種類のテンプレートを論理的に結合することにより構成する.
テンプレート中の作文要素 $e_i, e_j$ は,作文自体,文,段落,文字など,作
文を構成する言語的な要素である.なお,作文要素には,学習者がマークアップ
する言語要素も含まれる.


\begin{quote}
\begin{description}
 \item[テンプレート1:] $include(e_i, e_j, N)$
 \item[テンプレート2:] $child(e_i, e_j, N)$
 \item[テンプレート3:] $locate(e_i, e_j, P)$
 \item[テンプレート4:] $correspond(e_i, e_j, R)$
\end{description}
\end{quote}

テンプレート1と2は,作文要素の包含関係を規定する.作文要素 $e_i$ が 
$e_j$ を $N$ だけ含む場合,作文とテンプレートが一致したことになる.$N$ 
は,$e_j$の個数を表し,定数,もしくは,数値範囲で指定する.テンプレート1
と2との違いは,テンプレート1では$e_i$ が $e_j$を単に包含していればよいのに
対して,テンプレート2では,$e_j$ が $e_i$ の子要素となっていなければいけ
ないところである.

この二つのテンプレートの使用例として,表
\ref{tbl:example_composition_rule}の作文規則1, 17を挙げる.作文規則1は,
「作文には,一つの作文タイトルが存在する」ことを規定するもので,作文要素
「作文」の中に,子要素として作文要素「作文タイトル」が一つ含まれることを
意味する.作文を XML で記述した例を図\ref{fig:structured_data}に示す.

一方,作文規則17は,必須記述項目の「旅行の目的」が作文中に書かれているか
を検査するための規則である.したがって,作文規則1とは異なり,作文の子要素
ではなく,作文に単に包含されていればよい.図\ref{fig:structured_data}では,
2 章の章タイトルに「目的」タグとして付与されており,「作文」要素に包含さ
れた構造になっている.

\begin{table}[t]
 \begin{center}
  \caption{作文規則の例(一部)}
  \label{tbl:example_composition_rule}
\input{05table01.txt}
\end{table}

テンプレート3は,作文要素間の位置関係を規定する.作文要素 $e_i$ と $e_j$ 
が位置関係 $P$ のときにテンプレートと一致したことになる.$P$ の値は,
「直前」「直後」「前」「後」「先頭」「末尾」のいずれかを取る.このテンプ
レートの使用例としては,表\ref{tbl:example_composition_rule}の規則3を挙
げる.この規則では,図\ref{fig:structured_data}のように,「作文タイトル」
要素が「作文」要素の先頭にあることを規定する.

\begin{figure}[t]
\input{05fig06.txt}
\vspace{-1\baselineskip}
 \caption{構造化された作文の例}\label{fig:structured_data}
\end{figure}

テンプレート4は,作文要素間の対応関係を規定し,作文要素 $e_i$ と $e_j$ が
対応関係 $C$ であることを示す.対応関係には,「引用・出典」「図表参照」が
ある.このテンプレートの例としては,表\ref{tbl:example_composition_rule}
の作文規則16を示す.この規則は,引用を行った場合,必ず出典を示すことを指
導するためのものである.

以上の四つのテンプレートに対して,論理演算子($and, or, not$)を適用する
ことにより,複数のテンプレートを論理的に結合させることができる.また,演
算子 $desirable$を適用することにより,必須的な作文規則なのか,任意的な作
文規則なのか(従ったほうが好ましい規則なのか)を表現することができる.

作文規則$R$の形式は,次のBNFで規定される.
\begin{align*}
 R  ::= & R' \mid desirable(R') \\
 R' ::= & (R' and R') \mid (R' or R') \mid not(R') \mid \\
    & include(e_i, e_j, N) \mid locate(e_i, e_j, P) \mid
    correspond(e_i, e_j, C)
\end{align*}


\subsection{作文規則に基づく誤り検出}
作文規則に基づく誤り検出は,次の手順で行う.

\begin{enumerate}
 \item 作文要素の付与
 \item 規則の適用
 \item エラーの生成
\end{enumerate}

\subsubsection{作文要素の付与}

作文要素の付与方法には,(a)生テキストに自動付与,(b)学習者が
付与,(c) (b)で付与された情報を利用して自動付与,の3通りがある.どのよう
な作文要素を付与するかは,作文規則(つまり,教師の指導項目)に依存するが,
ここでは,表\ref{tbl:example_composition_rule}に示した作文規則中の作文要
素と関係づけつつ,三つのタイプの作文要素の付与方法を説明する.


まず,(a)の方法で付与される作文要素は,自動的な認定の精度が高いものが対象
となる.認定の精度が低い場合や,精度が高くても,必須記述項目に関連した作
文要素のように教育上の重要性が高い部分については,(b)(c)の付与方法を取る
ことになる.

表\ref{tbl:example_composition_rule}中の作文要素のうち,(a)の方法で付与さ
れた作文要素は,「作文」「段落」「文」「文字」「章番号」「話し言葉形態素」
「だ・である形態素」である.これらは,表層的なパターンマッチングや形態素
解析結果を利用して付与されている.例えば,「文」は,改行情報や句点(。)
などの表層的な情報に基づいて認定している.また,「話し言葉形態素」は,作
文全体を形態素解析し,主として,話し言葉でよく用いられる文末の助詞,助動
詞を「話し言葉形態素」として認定している\footnote{今回は,形態素解析シス
テム MeCab (http://mecab.sourceforge.net/),辞書として Jumandic を使用し,
「か」以外の終助詞,接続助詞「けど」,助動詞「んだ」,動詞語尾「ちゃ」
「りゃ」(例:食べ\underline{りゃ}いい)などを「話し言葉形態素」として検
出した.認定方法は,\cite{juman_manual}を参考にした.}.


次に,(b)のタイプの作文要素は,学習者が作文作成時にマークアップする.表
\ref{tbl:example_composition_rule}中の作文要素のうち,このタイプの作文要
素としては,章節タイトル(「章タイトル」「節タイトル」など),「引用」「出典」
「作文タイトル」「著者」に加えて,必須記述項目に関連する作文要素「目的」
「日程」「予算」「イベント」「導入部」「まとめ」がある.

(c)のタイプの作文要素の例としては,作文要素「章」(図
\ref{fig:structured_data}参照)がある.「章」要素の範囲を自動認定する際に
は,学習者がマークアップした「章タイトル」の付与結果を利用する
\footnote{{\modl 次の章タイトル,もしくは,文章末までを章の範囲として自動認定す
る.}}.また,必須記述項目の「導入部」についても,第1章の「章」要素を「序
章」として認定するのに利用されている(図\ref{fig:structured_data}では,
「章」要素のtype属性値として記述されている).


\subsubsection{作文規則の適用とエラーの生成}

作文規則の適用は,学習者が作文を保存するタイミングで実行され
る.作文が保存されると,まず,(前節で説明した)付与方法(a)(c)の順序で作
文要素を付与する.その後,すべての作文規則を順次,作文に対して,適用する.

もし,作文規則に合致しない場合は,エラーとして学習者に伝達する.エラーメッ
セージは,個々の作文規則ごとに対応づけられているものとする.実際の実行結
果については,\ref{ssec:model_student_sys}節の図\ref{fig:check}を参照され
たい.


\section{実験}\label{sec:experiment}

本節では,提案した相互教授モデルの有効性を検証するために,実現した作文支
援システムを用いて,作文実験を行う(実験1).さらに,従来からの作文教育と
の比較を行うために,システムを利用しない場合の作文実験も行う(実験2).な
お,本論文の実験で検証対象とするのは,提案した相互教授モデルのうち,手順
2から手順5に相当する部分である.システムから教師への知識教授(手順6)につ
いては,本稿では扱わない.

\subsection{実験1}

実験1では,実現したシステムを利用して,作文実験を行った.作文のテーマは
「旅行計画」,教育上の目標は文章の「章立て」に関する技術習得とした.実験
の条件は,次のとおりである.

\begin{itemize}
 \item 被験者: 大学1〜3年生(教育学部)26名
 \item 作文テーマ: 旅行計画(著者自身がこれから行く旅行の計画を他人に説
       明する)
 \item 想定する読者: 年上を含む,不特定多数の読者
 \item 文体: 必要がない限り,口語表現は避ける.また,ですます調で書く.
 \item 作文規則: 表\ref{tbl:example_composition_rule}の作文規則を用いる.
 \item 学習者によるマークアップ対象: 作文タイトル,著者,章節タイトル
       (章,節,小節,小々節),引用,出典,図タイトル,[以降,必須記述
       項目] 旅行の目的,日程,予算,旅行中のイベント,導入部,まとめ
\end{itemize}

実験手順は,次のとおりである.システムは,インターネット上の Web サーバ
上に設置し,各被験者は好みの時間と場所で実験を行った.ただし,各段階の実
施期間を定め,すべての被験者が同じ期間に同じ段階の実験を行うようにした.

\begin{enumerate}
 \item 章立てに関する資料を被験者に配布し,教育上の目標を理解してもらう.
 \item テーマに基づき,個々の被験者が作文を行う.なお,学習者は,作文の
       過程で,自分の作文に対してマークアップを行う.また,システムによ
       る誤りの指摘に基づき,作文を修正する.
 \item 個々の被験者は,それぞれ4名の被験者の作文を添削する.各被験者は他
       の4名の被験者の添削を受ける.なお,添削する相手に添削内容がうまく
       伝わるように,添削を行う前に添削マニュアルを配布した.
 \item 添削に基づき,自分の作文を修正する.なお,被添削者は添削結果に返答
       すること,添削結果を受け入れなくてもよいが,その理由を明記すること
       が求められる.
\end{enumerate}

{\mod
\subsection{実験2}

実験2では,提案手法と従来手法との比較を行うために,作文支援システムを利用
しない作文演習を想定して実験を行う.なお,学習者同士の添削は行わない.

実験条件は,被験者以外,実験1と同一である.被験者は,実験1と重複しない18
名(同一大学同一学部)である.実験手順は,実験1と同様,章立てに関する資料
を被験者に配布し,教育上の目標を理解した上で,作文支援システムを使わない
で作文してもらった\footnote{作文は,ワードプロセッサなどを利用し,電子的
なテキストの形で提出してもらっている.}.また,筆者が実験後に実験1と同様
の基準でマークアップを行った.
} 

\subsection{実験結果}

まず,表\ref{tbl:res_composition}「提案手法」(2行目)に実験1の作文結果を
まとめる.示した数値は,全被験者の平均値である.最左列から,作文に含まれ
る文字数,文数である.「構造」「必須」は,それぞれ,必須記述項目以外に対する学習者のマークアッ
プ数,必須記述項目に対するマークアップ数である.「誤り(章立
て)」は,章立てに関連する作文規則(1〜9)により検出された誤りのうち,被
験者が修正せずに残してしまった誤りの数を表す.「誤り(その他)」は,作文
規則(10〜16)により検出された誤りのうち,被験者が修正せずに残してしまっ
た誤りの数を表す.「誤り(必須)」は,必須記述項目に関連する作文規則(17〜
22)で検出された誤りのうち,被験者が修正せずに残してしまった誤りの数であ
る.


\begin{table}[b]
 \caption{作文,および,マークアップ結果}\label{tbl:res_composition}
\input{05table02.txt}
\end{table}


{\mod 次に,表\ref{tbl:res_composition}「従来手法」に実験2の作文結果をま
とめる.また,提案手法と従来手法とを比較するために,両者の結果に対して,
ウィルコクソンの順位和検定を行った.表\ref{tbl:res_composition}「p値」に,
p~値を示す.} 

表\ref{tbl:res_correction}に実験1における添削結果を示す.示した数値は,全
被験者の合計値である.表\ref{tbl:res_correction}「添削数」(2行目)は他の
学習者から受けた添削数,「修正数」はそのうち作文の著者が修正した数である.
「誤字脱字」から「その他」までが,添削種類別の添削数である.「全体」は,
作文中の個別の箇所でなく,作文全体に対して行われた添削の数である.「合計」
は全添削種類の合計添削数,「添削」は他の被験者の作文に対して行った添削の
総数である.



\begin{table}[t]
  \caption{添削結果}
  \label{tbl:res_correction}
\input{05table03.txt}
\end{table}


\section{評価}\label{sec:evaluation}

\subsection{システム,学習者間のインタラクションの評価}

ここでは,相互教授モデル中のインタラクションのうち,システムと学習者間の
インタラクションを評価する.\ref{sssec:syodate}〜\ref{sssec:jido}節では提
案手法について扱い,\ref{ssse:hikaku}節では提案手法と従来手法との比較を行
う.


\subsubsection{章立てに関する評価(提案手法)}\label{sssec:syodate}

作文の「章立て」を定量的に評価するために,章立てに関する学習
者のマークアップと,作文規則への適用結果を分析してみる.表
\ref{tbl:res_composition}「構造」より,必須記述項目以外の文章構造に関する
マークアップは,平均11.7ヵ所である.このうち,章節タイトル(章,節,小節,
小々節のタイトル)のマークアップ数は,平均9.3ヵ所である.1作文あたりの章
の数は,5.0個であった.また,章節の階層の深さは,平均で1.59だった.表
\ref{tbl:res_composition}「文字数」に示したとおり,作成された作文の文字数
は平均1631文字なので,おおよそ,400字詰め原稿用紙4枚に,5章からなる作文が
作成されたことになる.

作成された作文は表\ref{tbl:example_composition_rule}の作文規則1から9で検
査されているので,表\ref{tbl:res_composition}(「誤り(章立て)」列)に示
されている誤り13ヵ所\footnote{{\mod 章節タグのつけ誤りにより5ヵ所が誤りと
検出されたが,それらは誤りとしてカウントしていない.}}以外は,作文規則(1
から9)に適合した章立て構造を持った作文が作成されたことを意味する.

解決されずに残された誤りの内訳は,(a)章節タイトルに番号がないもの(12ヵ所,
作文規則9により検出),(b)序章が1章以外に存在するもの(1ヵ所,作文規則5に
より検出)だった.(a)は学習者のミス,もしくは,恣意的な「誤り」(エラー表
示の中には誤りの修正が必須でないものも含まれるため,あえてそのままにした
可能性がある)によるものだと思われる.(b)は学習者の章立てに対する理解不足
だと思われる.

以上の結果から,章節タイトル番号がない問題を除けば,作文規則1から9に規定
される範囲内で,章立てがうまく行われたと考えられる.

ただし,現状の作文規則では十分検出できなかった誤りもある.具体的には,
(1)第1章に必要以上の情報を書き込む学習者が存在したこと(2名の学習者),(2) 
章立てを過度に細かくする学習者が存在したことである(例えば,一つの章に1文
しかないものが見受けられた).前者については,序章に含まれる節数や文字数
を制約する作文規則を追加すること,後者については,章の数,もしくは,各章
に含まれる文字数の下限を制約する作文規則を作成することにより,防止できる
ものと考えられる.


\subsubsection{必須記述項目による評価(提案手法)}\label{sssec:hissu}

表\ref{tbl:res_composition}「必須」を見ると,必須記述項目に関
するマークアップ数は,一人当り平均6ヵ所である.マークアップされた部分に関
して,その内容を確認したところ,すべて正しくマークアップされていた.した
がって,作文規則(17〜22)がうまく機能し,必須記述項目が正しく作文された
と考えられる.

必須記述項目に関連する作文規則により検出され,修正されずに残ったエラーは,
表\ref{tbl:res_composition}「誤り(必須)」に示したとおり,被験者一人あた
り平均0.038ヵ所(合計1ヵ所)であった\footnote{{\mod 作文規則17から20により検出
されたエラーは,合計11ヵ所あった.しかし,10ヵ所については,対応する必須
記述項目に関する記述が作文中にあることから,学習者のマークアップのつけ忘
れだと考えられる.}}.エラーとして残った1ヵ所は作文規則22により検出された
誤りで,作文にまとめの記述がなかった.当該の学習者は,他の必須記述項目に
ついては,正しくマークアップを行っていたので,まとめを作文にうまく組み込
めなかったことが考えられる.

上記の誤り例や\ref{sssec:syodate}節の誤り(b)のように教師の指導が必要な場
合も検出される.したがって,本システムを運用する場合は,
\ref{ssec:model_abst}節の手順7で示したように,検出された誤りを教師が分析
し,作文後の授業で,指導することが好ましいと考える.


\subsubsection{自動付与された作文要素による誤り検出の評価(提案手法)}
\label{sssec:jido}

ここでは,自動的に付与された作文要素による誤り検出について評
価するために,作文規則10から15によって検出された誤りを分析する.

表\ref{tbl:res_composition}「誤り(その他)」を見ると,誤りが修正されずに
残っているのは,被験者一人あたり0.27ヵ所(合計7ヵ所)だった
\footnote{{\mod ただし,形態素解析誤りなどの,作文要素自動付与誤りに起因
するものは除いている.}}.

その内訳は,3ヵ所が作文規則14(文体は「ですます」調で書く)に反して「だ・
である」調で書かれていたもの,残り4ヵ所は作文規則15(話し言葉を用いてはい
けない)に違反したものだった.前者・後者の誤りとも,(それぞれ別の)1名の
被験者によるものである.このうち,前者は,箇条書き部分だけ「だ・である」
調で書かれていたので,被験者の意志だと考えられる.後者は,作文規則15に対
する被験者の認識不足か,不注意だと思われる.

以上の結果から,作文規則10から13で規定されている事柄,つまり,文,段落の
長さや図タイトルの位置に関する誤りは,抑制されているものと考えられる.
{\mod 作文規則14,15で規定されている文体関連の事柄については,部分的に誤
りが残るが,従来手法では合計72ヵ所の誤りが検出されたので,作文規則の効果
があったことが確認できる.}


{\mod
\subsubsection{従来手法との比較}
\label{ssse:hikaku}

まず,表\ref{tbl:res_composition}「誤り(章立て)」「誤り(必
須)」「誤り(その他)」の値を合計すると,修正されずに残った誤りは,提案
手法は一人あたりの平均で0.85ヵ所,従来手法は7.1ヵ所となり,提案手法では作
文規則によって,誤りが抑制されていることがわかる.

次に,表\ref{tbl:res_composition}「p値」より,提案手法と従来手法とで有意
な差が出たのは,「誤り(その他)」$(p<.01)$,および,「構造」「誤り(章
立て)」「誤り(必須)」$(p<.05)$である.このことから,提案手法は,従来
手法と比較して,(a)多くの章立てがなされること,(b)章立て,構造(作文規則
10〜16に反する),必須記述項目に関する誤りが少ないこと,が確認された.

従来手法と比較して,多くの章立てがなされた要因としては,二つのことが考え
られる.一つは,学習者自身が自分の作文に対してマークアップを行ったことに
より,章立てが促進されたことである.もう一つは,作文規則4 (作文には章が
三つ以上ある)に基づくチェックを行ったことにより,章の数が結果的に増えた
ことである.

一方,(b)の結果が得られた要因としては,作文規則によるチェックが有効に機能
したことが考えられる.表\ref{tbl:res_composition}「提案手法」の「誤り(章
立て)」「誤り(その他)」「誤り(必須)」が,作文規則を適用したときに
(修正されずに)残る誤り数なので,表\ref{tbl:res_composition}「従来手法」
の結果との差分が,作文規則を適用したことによる効果だと考えられる.この差
分となる効果を生み出した作文規則を明らかにするために,「従来手法」の作文
中の誤りを検出した作文規則を見てみると,特に有効に機能したのは,作文規則
4(「誤り(章立て)」の誤りのうち,32\%),作文規則9 (同32\%),作文規則
14(「誤り(その他)」の誤りのうち,67\%),作文規則15(同22\%)であるこ
とがわかった.} 


\subsection{学習者間のインタラクションの評価}

ここでは,学習者間のインタラクションの評価として,学習者同士の添削結果を
分析する.

\subsubsection{定量的評価}
\label{sssec:u2u_teiryo}

まず,添削の効果を検証する.表\ref{tbl:res_correction}の「合
計」列に示したとおり,合計で182ヵ所の添削がなされ,そのうち,154ヵ所が添
削対象の作文の著者によって修正されている.したがって,今回の実験では,作
者から見て,約85\%の添削が作文の内容を改善するのに役立ったことがわかる.

添削の種類別に添削数を見てみると,「その他」が全体の約27\%を占め,そのあ
と,「文法誤り」「誤字脱字」「口語表現」といった表記に関する添削が続く.
「その他」の内訳については,次節で詳しく見てみることにする.

次に,学習者を「添削者」という観点から見てみる.表
\ref{tbl:res_correction}の「添削」列のとおり,198回の添削のうち
\footnote{「合計」のほうが「添削」よりも少ないのは,前者が添削された場所
の数であり,1ヵ所に複数の学習者が添削を行う場合があるからである.},168回
の添削が添削対象の作文の著者により修正されており,全体的には約85\% の添削
が被添削者に受け入れられている.ただし,学習者別に添削数を見てみると,最
大23ヵ所,最小1ヵ所とばらつきが大きかった.添削数にばらつきがでる原因は,
学習者の能力による要因以外にも,添削対象の作文,添削者との関係が考えられ
る.例えば,添削対象の作文が優秀な作文であれば,添削数は少なくなる.また,
添削対象となる作文の著者が知り合いの上級生であれば,添削するのを遠慮する可
能性がある.特に,後者の要因に関しては,どのような学習者集団を設定すれば,
添削が活発になされるかを今後調査する必要がある.


\subsubsection{定性的評価}

ここでは,種類別に添削の内容を定性的に評価する.表
\ref{tbl:res_correction}に示した種別ごとに,実際の添削例を示す.

\begin{itemize}
 \item 誤字・脱字
       \begin{description}
	\item[修正前:] モネは生涯でたくさんの\underline{水連}の絵を描きましたが
	\item[修正後:] モネは生涯でたくさんの\underline{睡蓮}の絵を描きましたが
	\item[添削内容:] 睡蓮では?
       \end{description}
 \item 説明不足 
       \begin{description}
	\item[修正前:] 満天の星を眺めて満喫します。
	\item[修正後:] 満天の星を眺めて\underline{モンゴルの大自然を}満喫します。
	\item[添削内容:] 満喫するには,何を満喫するのかを書く必要がある
		   かと思います。モンゴルの星空を満喫するのか,モンゴル
		   の大自然を満喫するのか,書いたほうがわかりやすいと思
		   います。
       \end{description}
 \item 冗長表現
       \begin{description}
	\item[修正前:] この計画の発案者は私なので,私が幹事を務めること
		   になりました。
	\item[修正後:] 発案者の私が幹事を務めることになりました。
	\item[添削内容:] 「発案者の私が」とするとすっきりすると思います。
       \end{description}
 \item 文法誤り
       \begin{description}
	\item[修正前:] おおまかな計画\underline{が}立てることができました。
	\item[修正後:] おおまかな計画\underline{を}立てることができました。
	\item[添削内容:] 「を立てる」になると思います。 
       \end{description}
 \item 口語表現
       \begin{description}
	\item[修正前:]僕は\underline{やっぱり}「食べる」というのが一番の醍醐味だと思っています。
	\item[修正後:]僕は\underline{やはり}「食べる」というのが一番の醍醐味だと思っています。
	\item[添削内容:] 「やはり」のほうがよいのではないでしょうか?
       \end{description}
\end{itemize}

なお,「その他」として分類された添削には,多様な種類の添削内容が含まれて
いた.量的に最も多いのが,より適切な語句や表現の提案であり,約半数がこの
種類の添削だった.次に多かったのが,表記の変更を提案するもので,9例あっ
た.それぞれの例を次に示す.


\begin{itemize}
 \item より適切な語句や表現の提案
       \begin{itemize}
	\item 3,まとめ → 3,おわりに (第1章が「はじめに」だったため)
	\item 必ず暖かい服装をご用意ください。 → 必ず暖かい服装でご参加ください。
       \end{itemize}
 \item 表記の変更の提案
       \begin{itemize}
	\item 楽しんできたいとおもいます。 → 楽しんできたいと思います。
	\item 八月十五日から十八日までの四日間 → 8月15日から18日まで
	      の4日間 (横書に漢数字は読みにくいとの指摘があった)
       \end{itemize}
\end{itemize}

以上のように,表記上の誤り(誤字・脱字や文法誤りに示した例を参照)に対す
る支援だけでなく,「説明不足」「冗長表現」「その他」で示した例のように,
周辺文脈の意味を考慮した上で,改善例を示すような支援も可能である.これは,
従来の作文支援システムでは実現が困難だった支援である.学習者の添削のうち
約85\%が作文の改善に寄与することと,他人に作文知識を教授すること効果を考
慮すると,学習者間の添削は,作文支援の方法として,有効であると考える.





{\mod
\section{おわりに}\label{sec:conclusion}

本論文では,学習者向けの作文支援手法として,学習者,教師,システム間で互
いに作文に関する知識を教えあう,相互教授モデルを提案した.本モデルの新規
性は,次の点にある.

\begin{itemize}
 \item 作文規則を用いることにより,教師の指導意図を学習者に徹底させつつ,
       学習者同士の知識教授を実現できること
 \item 学習者のアノテーションが手間になるのではなく,作文技術習得上の助
       けになること.\\
	また,複数の学習者を想定することにより,学習者のア
       ノテーションの誤りを防止していること
\end{itemize}

さらに,相互教授モデルに基づいた作文支援システムを実現した.実現したシス
テムを評価するために,提案手法と従来手法による作文実験を行い,次の結果を
得た.

\begin{itemize}
 \item 問題点1に対しては,相互教授モデルが有効に機能し,次のように意味解
       析が必要となるような支援を行うことができた.
       \begin{itemize}
	\item 学習者・システム間のインタラクションによる,「章立て」や必
	      須記述項目などに対する支援
	\item 学習者同士の添削による,「説明不足」「冗長表現」などに対す
	      る支援
       \end{itemize}
 \item 問題点2に対しては,作文規則により対応した.今回の実験では,教育上
       の目標を「章立て」の習得,作文テーマを「旅行計画」と設定して,作文
       規則を記述した.
 \item 作文規則に照らし合わせて修正されずに残った誤りは,従来手法が一人あ
       たりの平均で7.1ヵ所だったのに対して,提案手法は0.85ヵ所に削減する
       ことができた.また,提案手法で作成した作文は,従来手法の作文よりも,
       有意に誤りが少ないことを確認した.この結果は,学習者・システム間の
       インタラクション,および,作文規則が有効に機能したことを示すもので
       ある.
\end{itemize}

今後は,教師が容易に作文規則を設定できるようなインターフェイスの実現を行
いつつ,さらなる作文課題に対して,提案手法が有効に機能するか検証を進める
予定である.また,今回扱わなかった,システムから教師への知識教授について
も,大量の作文に対する教師用の作文分析支援手法を検討中である.
}



\acknowledgment
本研究を評価するにあたって行われた作文実験に参加してくださった被験者の
方々,実験結果の集計を支援してくださった方々に深く感謝いたします.なお,
本研究は,科学研究費補助金・基盤研究(C)(課題番号20500822)の助成を受けて
行われたものである.




\bibliographystyle{jnlpbbl_1.4}
\begin{thebibliography}{}

\bibitem[\protect\BCAY{綾\JBA 松尾\JBA 岡崎\JBA 橋田\JBA 石塚}{綾 \Jetal
  }{2005}]{綾2005}
綾聡平\JBA 松尾豊\JBA 岡崎直観\JBA 橋田浩一\JBA 石塚満 \BBOP 2005\BBCP.
\newblock 修辞構造のアノテーションに基づく要約生成.\
\newblock \Jem{人工知能学会論文誌}, {\Bbf 20}  (3), \mbox{\BPGS\ 149--158}.

\bibitem[\protect\BCAY{Brockett, Dolan, \BBA\ Gamon}{Brockett
  et~al.}{2006}]{brockett2006}
Brockett, C., Dolan, B., \BBA\ Gamon, M. \BBOP 2006\BBCP.
\newblock \BBOQ Correcting ESL errors using phrasal SMT techniques.\BBCQ\
\newblock In {\Bem Proceedings of the COLING/ACL 2006}, \mbox{\BPGS\ 249--256}.

\bibitem[\protect\BCAY{Burstein, Kukich, Wolff, Lu, Chodorow, Braden-Harder,
  \BBA\ Harris}{Burstein et~al.}{1998}]{burstein1998}
Burstein, J., Kukich, K., Wolff, S., Lu, C., Chodorow, M., Braden-Harder, L.,
  \BBA\ Harris, M. \BBOP 1998\BBCP.
\newblock \BBOQ Automated scoring using a hybrid feature identification
  technique.\BBCQ\
\newblock In {\Bem Proceedings of the COLING/ACL 1998}, \mbox{\BPGS\ 206--210}.

\bibitem[\protect\BCAY{Chodorow \BBA\ Leacock}{Chodorow \BBA\
  Leacock}{2000}]{chodorow2000}
Chodorow, M.\BBACOMMA\ \BBA\ Leacock, C. \BBOP 2000\BBCP.
\newblock \BBOQ An Unsupervised Method for Detecting Grammatical Errors.\BBCQ\
\newblock In {\Bem Proceedings of the NAACL 2000}, \mbox{\BPGS\ 140--147}.

\bibitem[\protect\BCAY{原田}{原田}{2006}]{harada2006}
原田三千代 \BBOP 2006\BBCP.
\newblock
  中級学習者の作文推敲課程に与えるピア・レスポンスの影響—教師添削とその比較—
.\
\newblock \Jem{『日本語教育』}, {\Bbf 131}, \mbox{\BPGS\ 3--12}.

\bibitem[\protect\BCAY{橋田}{橋田}{1998}]{橋田浩一:19980701}
橋田浩一 \BBOP 1998\BBCP.
\newblock GDA意味的修飾に基づく多用途の知的コンテンツ.\
\newblock \Jem{人工知能学会誌}, {\Bbf 13}  (4), \mbox{\BPGS\ 528--535}.

\bibitem[\protect\BCAY{今枝\JBA 河合\JBA 石川\JBA 永田\JBA 桝井}{今枝 \Jetal
  }{2003}]{imaeda2003}
今枝恒治\JBA 河合敦夫\JBA 石川裕司\JBA 永田亮\JBA 桝井文人 \BBOP 2003\BBCP.
\newblock 日本語学習者の作文における格助詞の誤り検出と訂正.\
\newblock \Jem{情報処理学会研究報告. コンピュータと教育研究会報告}, {\Bbf 2003}
   (13), \mbox{\BPGS\ 39--46}.

\bibitem[\protect\BCAY{井上\JBA 佐渡島}{井上\JBA
  佐渡島}{2005}]{井上達紀:20050824}
井上達紀\JBA 佐渡島紗織 \BBOP 2005\BBCP.
\newblock アカデミックライティングへのJess導入の試み.\
\newblock \Jem{日本行動計量学会大会発表論文抄録集}, 33\JVOL, \mbox{\BPGS\
  378--381}.

\bibitem[\protect\BCAY{石岡}{石岡}{2004}]{石岡恒憲:20040910}
石岡恒憲 \BBOP 2004\BBCP.
\newblock 記述式テストにおける自動採点システムの最新動向.\
\newblock \Jem{行動計量学}, {\Bbf 31}  (2), \mbox{\BPGS\ 67--87}.

\bibitem[\protect\BCAY{Ishioka \BBA\ Kameda}{Ishioka \BBA\
  Kameda}{2006}]{ishioka-kameda:2006:COLACL}
Ishioka, T.\BBACOMMA\ \BBA\ Kameda, M. \BBOP 2006\BBCP.
\newblock \BBOQ Automated Japanese Essay Scoring System based on Articles
  Written by Experts.\BBCQ\
\newblock In {\Bem Proceedings of the 21st International Conference on
  Computational Linguistics and 44th Annual Meeting of the Association for
  Computational Linguistics}, \mbox{\BPGS\ 233--240}.

\bibitem[\protect\BCAY{笠原\JBA 小林\JBA 荒井\JBA 絹川}{笠原 \Jetal
  }{2001}]{笠原健成:20010515}
笠原健成\JBA 小林栄一\JBA 荒井真人\JBA 絹川博之 \BBOP 2001\BBCP.
\newblock マニュアルの校閲作業における文書推敲支援ツールの実適用評価.\
\newblock \Jem{情報処理学会論文誌}, {\Bbf 42}  (5), \mbox{\BPGS\ 1242--1253}.

\bibitem[\protect\BCAY{小谷}{小谷}{1989}]{kotani1989}
小谷善行 \BBOP 1989\BBCP.
\newblock IAC: 利用者が教えるというパラダイムによる教育ツール.\
\newblock \Jem{教育におけるコンピュータ利用の新しい方法シンポジウム報告集},
  \mbox{\BPGS\ 49--53}. 情報処理学会.

\bibitem[\protect\BCAY{黒橋\JBA 河原}{黒橋\JBA 河原}{2005}]{juman_manual}
黒橋禎夫\JBA 河原大輔 \BBOP 2005\BBCP.
\newblock 日本語形態素解析システム Juman version 5.1.\
\newblock \newline http://nlp.kuee.kyoto-u.ac.jp/nl-resource/juman.html.

\bibitem[\protect\BCAY{沖森\JBA 半沢}{沖森\JBA 半沢}{1998}]{okimori2007}
沖森卓也\JBA 半沢幹一 \BBOP 1998\BBCP.
\newblock \Jem{日本語表現法}.
\newblock 三省堂.

\bibitem[\protect\BCAY{大林\JBA 下田\JBA 吉川}{大林 \Jetal
  }{2000}]{大林史明:20001215}
大林史明\JBA 下田宏\JBA 吉川榮和 \BBOP 2000\BBCP.
\newblock 仮想生徒へ「教えることで学習する」CAIシステムの構築と評価.\
\newblock \Jem{情報処理学会論文誌}, {\Bbf 41}  (12), \mbox{\BPGS\ 3386--3393}.

\bibitem[\protect\BCAY{Sharoff, Babych, \BBA\ Hartley}{Sharoff
  et~al.}{2006}]{sharoff2006}
Sharoff, S., Babych, B., \BBA\ Hartley, A. \BBOP 2006\BBCP.
\newblock \BBOQ Using comparable corpora to solve problems difficult for human
  translators.\BBCQ\
\newblock In {\Bem Proceedings of the COLING/ACL 2006}, \mbox{\BPGS\ 739--746}.

\bibitem[\protect\BCAY{庄司\JBA 山岸\JBA 小野\JBA 安達原}{庄司 \Jetal
  }{2007}]{shoji2007}
庄司達也\JBA 山岸郁子\JBA 小野美典\JBA 安達原達晴 \BBOP 2007\BBCP.
\newblock \Jem{日本語表現法—21世紀を生きる社会人のたしなみ}.
\newblock 翰林書房.

\bibitem[\protect\BCAY{砂岡\JBA 劉}{砂岡\JBA 劉}{2006}]{sunaoka2006}
砂岡和子\JBA 劉松 \BBOP 2006\BBCP.
\newblock 誤用データ機能を備えるWEB中国語作文添削支援システム設計と開発.\
\newblock \Jem{2006PCカンファレンス論文集}.

\bibitem[\protect\BCAY{Takabayashi}{Takabayashi}{2004}]{takabayashi2004}
Takabayashi, S. \BBOP 2004\BBCP.
\newblock {\Bem Synthetic Assistance for Creation and Communication of
  Information}.
\newblock Ph.D.\ thesis, Nara Institute of Science and Technology.

\bibitem[\protect\BCAY{田近\JBA 井上}{田近\JBA 井上}{2006}]{tazika2006}
田近洵一\JBA 井上尚美 \BBOP 2006\BBCP.
\newblock \Jem{国語教育指導用語辞典(第3版)\inhibitglue}.
\newblock 教育出版.

\bibitem[\protect\BCAY{梅村\JBA 増山}{梅村\JBA 増山}{2007}]{umemura2007}
梅村祥之\JBA 増山繁 \BBOP 2007\BBCP.
\newblock 仕事文推敲支援に向けた連体修飾不足に対する受容性判定法.\
\newblock \Jem{自然言語処理}, {\Bbf 14}  (4), \mbox{\BPGS\ 43--65}.

\bibitem[\protect\BCAY{Usami \BBA\ Yarimizu}{Usami \BBA\
  Yarimizu}{2007}]{usami2007}
Usami, Y.\BBACOMMA\ \BBA\ Yarimizu, K. \BBOP 2007\BBCP.
\newblock \BBOQ Design of XECS (XML-based Essay Correction System): Effects and
  implications.\BBCQ\
\newblock In {\Bem Proceedings of the CASTEL-J in Hawaii 2007}, \mbox{\BPGS\
  182--184}.

\end{thebibliography}

\begin{biography}
\bioauthor{山口 昌也}{
1992年東京農工大学工学部数理情報工学科卒業.1994年同大学院博士前期課程修
了.1998年同大学院博士後期課程修了.博士(工学).同年,同大学工学部助手.2000年国立国
語研究所研究員,現在に至る.自然言語処理の研究,コーパス構築に従事.言語
処理学会,情報処理学会,日本教育工学会,日本語学会,社会言語科学会各会員.}

\bioauthor{北村 雅則}{
1998年関西大学文学部国文学科卒業.2000年名古屋大学大学院文学研究科
国文学専攻国語学専門博士前期課程修了.2005年名古屋大学大学院文学研究科
人文学専攻日本文学日本語学講座博士後期課程満期退学.博士(文学).
2006年国立国語研究所研究開発部門言語資源グループ特別奨励研究員,
2008年名古屋学院大学商学部講師,現在に至る.日本語学(現代語文法,文法史),
作文教育の研究に従事.日本語学会,日本語文法学会,日本語用論学会,
言語処理学会,日本教育工学会各会員.
}

\bioauthor{棚橋 尚子}{
1983年愛知教育大学教育学部小学校課程国語科卒業.1989年兵庫教育大学大学院
学校教育研究科教科・領域教育専攻言語系コース(国語)修士課程修了.名古屋
市立の中学校教諭,広島大学附属小学校教諭を経て1995年群馬大学教育学部専任
講師,1997年同助教授.1999年奈良教育大学教育学部助教授.2006年同教授,現
在に至る.漢字教育を中心とした国語科教育の研究に従事.全国大学国語教育学
会,日本国語教育学会,中国四国教育学会,日本言語政策学会各会員.
}
\end{biography}


\biodate


\end{document}
