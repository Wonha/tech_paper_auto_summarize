    \documentclass[japanese]{jnlp_1.4}
\usepackage{jnlpbbl_1.1}
\usepackage[dvips]{graphicx}
\usepackage{amsmath}
\usepackage{hangcaption_jnlp}


\Volume{16}
\Number{2}
\Month{April}
\Year{2009}
\received{2008}{9}{2}
\revised{2008}{11}{10}
\accepted{2008}{11}{28}

\setcounter{page}{45}


\jtitle{言語処理技術と教材作成の連携—データベース・ソフトウェアを用いた英語学習教材の自動作成—}
\jauthor{神谷 健一\affiref{Author_1} \and 田中 省作\affiref{Author_2} \and 北尾 謙治\affiref{Author_3}}
\jabstract{
本稿ではデータベース・ソフトウェアの1つであるFileMaker Proによる,英語学習教材の自動作成における言語処理技術と教材作成の連携可能性を提案する.著者は,実際の英語の授業でも利用しやすいプリント教材や簡易E-learning教材を出力できるツールを開発し,無料公開している.これらのツールではGUI環境での操作が可能であるため,パソコン利用スキルが限られる一般の英語教員にも利用しやすく,任意の英文素材からPhrase Readingを軸とした精読教材およびClozeテストを利用した学習教材を短時間で作成することができる. 
}
\jkeywords{データベース・ソフトウェア,ファイルメーカー,フレーズ・リーディング,クローズ・テスト,教材作成,E-learning}

\etitle{Language Processing Technology and Educational Material Development---Generating English Educational Material using a Database Software---}
\eauthor{Kenichi Kamiya\affiref{Author_1} \and Shosaku Tanaka\affiref{Author_2} \and Kenji Kitao\affiref{Author_3}} 
\eabstract{
This article provides an example of developing educational material using a database software, linking it with language processing technology.  Teachers can download our software for free and create worksheets for studying phrase reading and e-learning materials based on cloze exercises.  This software makes creating such learning materials very efficient, and provides integrated functions which are almost impossible to do manually.  Since the operations can be done on graphical user interface, or GUI, even computer novices can use the software easily.
}
\ekeywords{Database Software, FileMaker, Phrase Reading, Cloze Test, Educational \\
	Material Development, E-learning}

\headauthor{神谷,田中,北尾}
\headtitle{言語処理技術と教材作成の連携}


\affilabel{Author_1}{大阪工業大学知的財産学部}{Faculty of Intellectual Property, Osaka Institute of Technology}
\affilabel{Author_2}{立命館大学文学部}{College of Letters, Ritsumeikan University}
\affilabel{Author_3}{同志社大学文化情報学部}{Faculty of Culture and Information Science, Doshisha University}



\begin{document}
\maketitle


\section{はじめに}

英語教育の現場でもICT (Information and Communication Technology)の活用により様々な取り組みがなされている.近年ではE-learningのように学習者が教科書ではなく,まずはコンピュータ端末に向かうような形態での学習環境も一部で行われている.しかし大学を含め,CALL教室などが未整備となっている教育機関は少なくない.またE-learningのための教材作成が英語教育に直接関係する教師自身によって行われることは現実的にはほとんどなく,先進的な取り組みを行っている教育機関などにおいても既存のコースウェアが利用される場合が多い.教室で接する学習者のために教員自らがオーサリングソフトなどを利用して積極的に教材を作成するという事例は,英語教員全体の人数からすると極めて少数であると思われる.

近年,パソコンは爆発的に普及してきており,現在ではほぼ全ての英語教員が日常の業務や教材作成でパソコンを利用することが当たり前のこととなった.しかし大多数の英語教員のパソコン利用スキルは基礎的なワープロ操作に限られると言っても過言ではない.結果,ワープロソフトによる教材作成と,E-learningやCALL環境のための教材作成の間にある溝はなかなか埋まりそうにないというのが現状である.

一方,計算機科学の発展に伴い,言語処理技術に関する研究も急速に増加しつつある.そしてこれらの知見を教育や学習に生かすことを目標とする研究も盛んに行われている.しかしここで一つの疑問が浮かぶ.言語処理技術と教育・学習の連携は,いわゆる文系の一般の教員が,極端に言えば翌日の授業からでも応用可能な形で提供されていると言えるのだろうか.

言語処理技術の教育・学習への応用を試みる際,まずはその方法論が優先される.そしてその実装は簡易なプロトタイプにとどまり,実際の使用に耐えうるシステムの構築は別途行わなければならない場合も多い.しかし,たとえどんなに軽微なものであったとしてもCUIベースの処理やプログラミング言語の知識を必要とする手法を一般の英語教員に求めることはほぼ絶望的である.例えばPerl言語を用いたテキスト処理などでさえも,その実行環境をインストールするといった時点で一般の英語教員のコンピュータ利用スキルからすれば十分にハードルが高いことは間違いない.また「UNIX環境」といった文言でさえ,一般の英語教員を遠ざけるには十分な材料となる.これらのアプリケーションがCGIなどを介してWeb上で提供される場合も同様である.通常これらは教育工学などの分野に関心がある一部の英語教員が,データ分析などの研究目的で利用することが多く,授業に生かすという用途からは残念ながらほど遠いという印象がある.

それでは,仮に言語処理技術を教材作成に簡便に応用できるような仕組みが提供されていればどうなるであろうか.例えば教科書に準拠した補助プリントなどを作る場合など,少しでも教員の負担を減らすことができればきっと喜ばれるに違いない.そして草の根的であったとしても,言語処理技術と教育・学習の連携がこれまで以上に有機的に行われていくことが予想される.

本研究では一般の英語教員でも簡単に使えることを念頭に,様々な状況での実際の英語授業や自習環境で利用できるプリント教材およびE-learning教材の作成支援を行う2種類のツールを開発した.これらのツールは無料で公開しており,GUI環境での簡単な操作で,任意の英文から様々な教材を短時間で作成することができる.利用者である一般の英語教員はこれらをダウンロード,解凍し,フォルダ内に含まれている実行ファイルを起動するだけでよい.つまり別途ソフトウェアを購入する必要もなく,プログラミング言語の実行環境をインストールするというような負担もない.また,これらのツールでは言語処理技術によるデータ処理結果をデータベース・ソフトウェアによって教材に加工するが,内部設計はツール利用者である一般の英語教員には見せない形になっている\footnote{ソースファイル相当以上の内容を知ることができるデータベースデザインレポートも公開している.FileMakerでの開発に通じている者であれば内部設計の把握や改変も可能.}.言語処理のアルゴリズムやデータベース・ソフトウェアについての知識は一切必要としない.

以下,2節では,データベース・ソフトウェアの基本的な特徴を確認し,本研究で使用したFileMakerについて概観する.3節では連携事例 I として,言語処理技術を活用したPhrase Reading教材作成支援システムを紹介し,これを応用したプリント教材の自動作成について述べる.4節では連携事例 II として,任意の英文テキストに対して語彙レベルタグや品詞タグを付与するプログラムを紹介し,この処理結果を用いたE-learning教材作成について述べる.


\section{データベース・ソフトウェアについて}

一般にデータベース・ソフトウェアはビジネス用途\footnote{例えば顧客台帳,受注伝票,売上伝票,商品台帳などを互いに関連づけておき,自動処理によって見積書,納品書,請求書,郵送用宛名ラベルなどを作成するといった目的で利用されることが多い.}で用いられることが多く,数十万単位での大量のデータであっても,整合性を保ちながら高速で検索やソートを行うことができる仕組みが用意されている.また保持しているデータと,それを表示するレイアウトを別々に管理するという特徴があるため,同一のデータを異なるレイアウトに当てはめて出力することや,多数のフィールドから必要なもののみを組み合わせて出力することができるという点で優れている.同様の出力をワープロや表計算ソフトを用いて行うことはかなり困難である.

本研究では市販のデータベース・ソフトウェアの1つであるFileMaker\footnote{本稿執筆時(2008年8月)の最新バージョンはFileMaker Pro 9であり,Windows XP/VistaとMac OS Xに対応している}を利用し,教材作成に生かす方法を模索した.FileMakerには高度な自動処理ができるスクリプト言語が搭載されており,この上位パッケージであるFileMaker Pro Advanced\footnote{http://www.filemaker.co.jp/products/fmpa/}を利用して開発を行うとFileMakerを所有しないユーザでも利用できるランタイム・アプリケーションを構築し,自由に頒布することができる.このランタイム・アプリケーションはWindows XP/Vista上でもMac OS X上でも動作するため\footnote{開発者はまずWindows上またはMac OS上のいずれかにインストールしたFileMaker Pro Advancedでデータベース・ファイルを作成する.これをWindows上で処理するとWindows版のランタイム・アプリケーションを,Mac OS上で処理するとMac OS版のランタイム・アプリケーションをそれぞれ作成することができる.ちなみにFileMaker Pro Advancedのライセンス形態は「1ユーザが同時使用しない限り,2台のマシンへのインストールが可能」であるため,開発者がWindowsとMac OSの両方のパソコンを所有する場合,1ライセンスで両方の作成を行うことができる.},大多数の英語教員が日常で触れる機会のあるプラットフォームで利用することができる.
またFileMakerにはWebビューア機能\footnote{詳細は http://www.filemaker.co.jp/products/fmp/wvg/ を参照のこと.現行の1世代前のFileMaker Pro 8.5(2006年9月発売)から搭載された機能.}が搭載されており,インターネット上にある各種情報やオンラインデータベースと連携したデータベース・ソリューションを開発することができる.この機能を使うと,画面レイアウト上にWebブラウザと同等の機能を持つ画面枠を配置することができ,そこで表示されたHTMLソースはFileMakerに搭載されている関数を用いて取得することができる.

本研究で開発した教材作成ツールでは,以下のような手法を用いることで,言語処理技術と教材作成の連携を試みている.

\begin{enumerate}
\item 言語処理技術を用いたWebアプリケーションを,Webビューア機能によって教材作成ツールの画面内に表示する.
\item Webアプリケーションの実行結果をHTMLソースとして取得し,FileMaker側で様々なテキスト関数によって文字列処理を行うことで,教材中に組み込むデータを準備する.
\item このデータを様々なレイアウトに流し込み,教材の形に整形する.
\end{enumerate}



\section{連携事例 I: Phrase Reading 教材の自動作成}

\subsection{Phrase Reading とは}

Phrase Readingとは次の英文のように適当な長さの意味の塊ごとに区切られたものを英語学習者ができるだけ塊の単位で素早く読み進める訓練方法であり,一般的にはスラッシュで区切って表示されることからスラッシュ・リーディングとも呼ばれる.

\begin{quote}
Scientists say / they have made more progress / in developing / malaria-resistant mosquitoes. / The idea / is to release / genetically engineered insects / like these / into mosquito populations / as a way / to control the disease. / 
\end{quote}

このような学習方法を繰り返すことによって,学習者は英語本来の語順で英文をより直接的に理解できるようになることが期待される.しかしこの学習方法を実践する場合,ある程度一貫性を持った方法で区切られた,一定の量の教材をこなす必要がある.また,スラッシュを挿入するタイミングは必ずしも一つではなく,学習者のレベルや作成者の意図によって様々な基準が考え得る.近年では市販の教材でもよく利用されているが,学習者が関心を持つことができる様々なジャンル,難易度の英文でこのような形式の教材がこれまで十分に提供されてきているとは言い難い.


\subsection{既存のシステムと問題点}

田中・木村・北尾(2006)および行野・田中・冨浦・柴田(2007)ではこのような教材不足を解消するために,言語処理技術を用い英文中に自動的にスラッシュを挿入する手法を提案している.ここでは局所的統語構造およびチャンクの大きさが英文のスラッシュ挿入に強く関係していると仮定した上で,一定量のスラッシュ付きの英文から確率モデルを用いてスラッシュを入れる基準を自動学習させた分割モデルを用意している.そしてこれに基づいて任意の英文を自動的に分割する教材作成支援システムがWeb上から実行できるようになっており,さらに異なる分割モデルに基づいてスラッシュを挿入した,様々な難易度の計128英文書,42,529語を教材集として提供している\footnote{http://www.cl.ritsumei.ac.jp/CALL/SR/}.

この教材作成支援システムは数理的に厳密な処理に基づいたものであり,様々な分割モデルに対応する柔軟なスラッシュ付与が可能であったが,自動付与のための手法の開発に主眼が置かれていたため,この手法を実装したプロトタイプでは最終的に英文にスラッシュをつけて出力することしかできていなかった.つまりスラッシュで区切られた英文を次々と読破する形で自習する場合や,多読を中心に据えた授業形態の場合を除けば,教室で即座に使えるような教材の体裁としては不十分という短所があった.

一方,ほぼ同時期に発表された神谷(2006)や岡本・神谷(2006)では,Phrase Reading学習が容易に行える書き込み式のプリント教材を即座にプリンタから出力できるソフトウェア\footnote{当時は「Phrase Reading Worksheet 作成ツール」「階段式 英文読解プリント 教材作成ツール」として別々に公開していたが,現在は両方の機能を1つのツールに統合したものも公開している.}を提案していた.しかし教材化する英文をPhraseごとに分割する作業は,手作業で行うか,あるいは観察によって得られた50語の単語\footnote{岡本・神谷(2006)または神谷・田中(2007)を参照のこと.}の直前で機械的に分割するという単純な手法を導入しているにすぎなかったため,柔軟性がないという短所があった.


\subsection{FileMaker を活用した教材作成システム}

そこで神谷・田中(2007)はPhrase Reading Worksheet作成ツール上に配置したWebビューアからスラッシュを自動付与する教材作成支援システムを呼び出し,その処理結果をPhrase Reading Worksheet作成ツールへ取り込むという方法を提案した.これにより様々な教材パラメタに対応した柔軟なフレーズ分割が行えるようになり,かつ教室でも使いやすいきれいな体裁のプリント教材を作成できるようになった.

このPhrase Reading Worksheet作成ツールは図1のようなGUI環境のものであり,データベース・ソフトウェアで作られていることを利用者には全く意識させない.また言語処理技術によるフレーズ分割と実行結果の取り込みは図2のような画面で行う.そして最終的に学習者にどのような教材として与えるかの諸要因を考慮し,用紙サイズの設定・行間・メモ欄の有無などの様々な条件を組み合わせながらレイアウト上に表示させることで1つの英文素材から1,000通り以上\footnote{図3・図4の形式では用紙サイズ(B5/A4),プリント上に配置する項目とその位置48種,メモ欄の有無,行間4種,フォントサイズを選択できる.またツールに搭載されている別の出力形式(図5の階段式など)でも同様に様々な出力形式の選択が可能.}のレイアウトのプリント教材を作成することができる.このようなレイアウトの柔軟さ\footnote{用紙サイズ,行間,メモ欄の有無の組み合わせによって用意された16種類のレイアウト上にはそれぞれ25種類のフィールドがそれぞれ配置されている.教材作成時に利用しないフィールドは空白のままで出力されるため,様々な見た目で表示されることになる.}はデータベース・ソフトウェアでしか為し得ないものである.

\begin{figure}[t]
 \begin{minipage}[b]{201pt}
  \begin{center}
    \includegraphics{16-2ia2f1.eps}
  \end{center}
  \caption{起動画面}
 \end{minipage}
\hfill
 \begin{minipage}[b]{201pt}
   \begin{center}
     \includegraphics{16-2ia2f2.eps}
   \end{center}
   \caption{確率モデルによるフレーズ分割}
 \end{minipage}
\end{figure}


図3の形式のプリントは,フレーズごとに分割したものが縦方向に配置されており,右側に予習として各フレーズの意味を記入させる\footnote{ツール内部で行っていることを単純化すると次のようになる:「B5行間標準メモ欄あり」というレイアウト上に配置した「左側 中央揃え」のフィールドに「英句」フィールドの内容を表示}.授業時に図4のような右側に予め日本語訳を入れたプリントを配布すると予習チェックなども簡単に行える\footnote{「B5行間極小メモ欄なし」というレイアウト上に配置した「左側 中央揃え」のフィールドに「英句」を,「右側 左揃え」のフィールドに「和句」フィールドの内容を表示.}.これらの形式はツール上では「縦方向」と呼んでいる.

図5の形式は「階段式」と呼んでおり,各フレーズ間の修飾関係や並列関係がわかりやすい階層の形で示すことができるプリントである\footnote{「B5行間狭メモ欄なし」というレイアウト上に配置した「階段1」〜「階段8」のフィールドに,「英句」フィールドの内容を別レイアウト上で指定した「階段設定」に従って表示.右端の「階段式和訳」のフィールド上に「和句」フィールドの内容を表示.}.図6は「クローズテスト型」と呼んでおり,ここでは7語おきに単語を空欄に置き換えている.和訳を見ながら空所を埋める練習や,聞き取らせたい単語に焦点をあてたリスニングの補助教材などの用途で利用できるであろう.これらのプリントには全て各フレーズの位置を表す「文番号」「句番号」が自動で挿入されるため,授業時に指導中の箇所などを指示しやすい.

\begin{figure}[t]
 \begin{minipage}{0.45\textwidth}
  \begin{center}
    \includegraphics{16-2ia2f3.eps}
  \end{center}
  \caption{予習用プリント}
 \end{minipage}
\hfill
 \begin{minipage}{0.45\textwidth}
   \begin{center}
     \includegraphics{16-2ia2f4.eps}
   \end{center}
   \caption{チェック用プリント}
 \end{minipage}
\end{figure}



図3・4の形式の場合,縦方向に2つ折りにして利用することもできるため,右側に和訳があらかじめ書き込まれた状態であっても,再度英語のみを見ながら日本語で意味を考えさせる,日本語を見ながら英語の原文を思い出させる,日本語を見ながら英文を書かせる,日英語の語順の違いを観察させるなど,同一授業内で学習者の習熟度に応じた異なるタスクを同時進行で学習させるということもできる\footnote{具体的な指導実践例および省察は大学英語教育学会授業学研究委員会編著『高等教育における英語授業の研究—授業実践事例を中心に—』松柏社 pp.~64--65に掲載されている.}.

このようなPhrase Reading Worksheetはどちらかと言えば精読が中心となる授業において,これまで消極的に取り入れられがちであった文法訳読式や輪番制に代わる効率的な授業展開を可能にする.

Phrase Reading Worksheet作成ツールの原型は2004年に初公開しており,新たな機能を加えながら随時アップデートを行っている.最新版や関連情報などは http://www.oit.ac.jp/ip/\textasciitilde kamiya/\\prw/prw.htmlに掲載している.このツールは著者勤務校の多くの英語教員に利用されており,インターネットから入手した英字新聞記事などであってもすぐにプリント教材に加工できる点など,好意的な意見が寄せられることが多い.またこのツールから出力したプリント教材は学生の評判も良い.例えば教科書本文を用いて図3の形式で出力したプリント教材の場合,予習→授業→復習のサイクルで効果的に活用できるのみならず,試験前にこのプリントを見直すだけで十分な復習ができる点などは特に好評のようである.

\begin{figure}[t]
 \begin{minipage}{0.45\textwidth}
  \begin{center}
    \includegraphics{16-2ia2f5.eps}
  \end{center}
  \caption{階段式教材}
 \end{minipage}
\hfill
 \begin{minipage}{0.45\textwidth}
   \begin{center}
     \includegraphics{16-2ia2f6.eps}
   \end{center}
   \caption{クローズテスト型教材}
 \end{minipage}
\end{figure}



\section{連携事例 II: Clozeテストの自動作成}

\subsection{Cloze テスト とは}

Clozeテストとは文書のn番目(通常は6〜8番目)の単語を空欄にして,被験者が元の単語を埋めるものであり,英語母語話者に対する読解教材の信頼度や難易度を測定する目的でTaylor(1953)により開発された.以下は原文の8番目の単語を空欄にした場合の例である.

原文:Scientists say they have made more progress in developing malaria-resistant mosquitoes. The idea is to release genetically engineered insects like these into mosquito populations as a way to control the disease.

作成例:Scientists say they have made more progress (     ) developing malaria-resistant mosquitoes. The idea is to (      ) genetically engineered insects like these into mosquito (      ) as a way to control the disease.

その後,選択式問題など他の方式によるテストとの相関が高いことが分かり,読解力を測定するテスト形式として広く使われるようになった.また1970年代以降にも盛んに研究が行われ,第二言語学習者への応用可能性についての実証が行われてきた.国内の研究では佐藤(1988)はClozeテストが従来の言語テストに見られない様々な優れた特性を備えていることを指摘し,日本の英語教育へ応用する際の意義を述べている.


\subsection{既存のシステムと問題点}

Clozeテストを作成する場合,任意のn語ごとに単語を抜き取り,空欄に置き換えるという作業が必要となる.一見単純な作業であるが,手作業で行うには相当の労力が必要となる.北尾(2007)はPerlを利用した自動抜き取りのプログラム\footnote{http://www.cis.doshisha.ac.jp/kkitao/Japanese/library/resource/corpus/perl/ELT/}により,この労力を大幅に軽減する方法を提案した.このプログラムの優れた点は,原文中から抜き取った単語を自動でランダム順に並べ替えたものが出力されるため,これを解答時の選択肢として表示することで,受験者の心理的負担を軽減することもできた.さらに抜き取った順番通りに並べたものも併せて表示されているため,この部分だけを切り取って後で答え合わせに利用することができるという長所があった.

加えて北尾(2007)ではJACET8000\footnote{大学英語教育学会(JACET)基本語改定委員会が2003年3月に改定・刊行した,コーパスでの頻度をもとに選定された1,000語刻みの8つのレベルから成る計8,000語の語彙表.この語彙表を利用して任意のテキストにJACET8000のレベルを表示するCGIプログラム(http://www01.tcp-ip.or.jp/\textasciitilde shin/J8LevelMarker/j8lm.cgi)が清水伸一氏によって公開されている.}の語彙レベルに基づいてタグを付与したテキストを用いて,ある特定のレベル範囲の語のみを抜き取り対象とする,またTreeTagger\footnote{ドイツのStuttgart大学で開発された形態素解析のためのプログラム.(http://www.ims.uni-stuttgart.de/projekte/corplex/TreeTagger/DecisionTreeTagger.html)任意のテキストを解析してそれぞれの単語に品詞タグを付与することができる.後藤一章氏 (http://uluru.lang.osaka-u.ac.jp/\textasciitilde k-goto/treetagger/tt.html) や杉浦正利氏 (http://genbun.gsid.nagoya-u.ac.jp/tagger/) がCGIを介してWeb上でこの処理を実行できるWebアプリケーションを提供している.}によって品詞タグを付与したテキストを用いて,ある特定の品詞のみを抜き取り対象とするというプログラムも公開していた.これにより学習者の習熟度や各教員の指導目標に合致するClozeテスト形式の練習問題を作成する新たな展望が開けたと言える.

しかしこれらのプログラムを利用するには別途Perl実行環境をインストールする必要がある上,抜き取りの間隔や語彙レベル範囲,抜き取る対象とする品詞を指定するための条件を変数として入力する際にはプログラムを一旦書き換える必要があった.またTreeTaggerのタグセットは学校英文法などで扱う品詞よりもはるかに厳密な分類を行うことから,例えば動詞を抜き取り対象とする場合には,動詞に相当するタグであるVB VBD VBG VBN VBP VBZ VH VHD VHG VHN VHP VHZ VV VVD VVG VVN VVP VVZ VBD VBG VBN VBP VBZという非常に数多くの項目を変数に入力しなければならなかったため,現実的には大変な困難を伴う利用形態であったと言える.さらにJACET8000レベルマーカーやTreeTaggerを併用したClozeテストを作成するには,それぞれのタグ付けを行うWebアプリケーションでの処理結果を一旦テキストファイルに保存してからPerlで処理を行うという手間が必要であり,結果的に一般の英語教員にはハードルが高く,利用者が限られてしまうという短所があった.


\subsection{FileMaker を活用した教材作成システム}

そこで神谷・永野・北尾(2007)ではFileMakerを用いてGUI環境でClozeテストを作成できるツールを開発し,無料公開した.またKitao \& Kamiya (2008) ではこのツールの改良が行われ,画面表示を日英両方に切り替えることも可能になった.このツールでは画面上に英文を貼り付けた後,抜き取り間隔であるnの指定をプルダウンメニューで選択するという簡便な操作のみで即座に作成することができる.またツール自体がランタイム環境であるため,実行環境のインストールも不要である.

\begin{figure}[b]
 \begin{minipage}{201pt}
  \begin{center}
    \includegraphics{16-2ia2f7.eps}
  \end{center}
  \caption{単語簡易抜き取り}
 \end{minipage}
\hfill
 \begin{minipage}{201pt}
   \begin{center}
     \includegraphics{16-2ia2f8.eps}
   \end{center}
   \caption{作成結果}
 \end{minipage}
\end{figure}

JACET 8000レベルマーカーやオンライン版TreeTaggerへはFileMakerのWebビューア機能を利用して同一画面の中で処理できるようにしているため,特定の語彙レベルや品詞を考慮に入れたClozeテストの作成もさらに省力化が図れることとなった.

このツールの目的は簡便な方法でClozeテスト形式の教材作成を行うことにあるため,TreeTaggerを併用する際においても膨大な数のタグセットを考慮しながら品詞を選択するのではなく,8種類の品詞から選ぶという操作方法を導入した.チェックボックスから選んだ品詞はTreeTaggerのタグセットに自動的に置き換える仕組みになっており,仮に不必要なタグが含まれる場合でも手作業で取り除くことができる.

このツールは北尾(2007)が提案したClozeテストの自動採点が可能なJavaScriptプログラムに対応したHTMLファイルを出力することもできる\footnote{問題サンプルはhttp://kkitao.e-learning-server.com/javaS/blank/cloze/index.html で公開されている.}ため,簡易E-learning教材作成ツールとしても利用することができる.このHTMLファイルは以下のような形式で書かれており,単語を抜き取って空欄にすべき処理の代わりに,HTMLタグの間に解答となる語を挟み込んで出力するという別の文字列処理を行っている.

\begin{figure}[t]
 \begin{minipage}{0.45\textwidth}
  \begin{center}
    \includegraphics{16-2ia2f9.eps}
  \end{center}
  \caption{JACET8000処理結果}
 \end{minipage}
\hfill
 \begin{minipage}{0.45\textwidth}
   \begin{center}
     \includegraphics{16-2ia2f10.eps}
   \end{center}
   \caption{作成条件設定}
 \end{minipage}
\end{figure}

\begin{figure}[t]
 \begin{minipage}{0.45\textwidth}
  \begin{center}
    \includegraphics{16-2ia2f11.eps}
  \end{center}
  \caption{TreeTagger処理}
 \end{minipage}
\hfill
 \begin{minipage}{0.45\textwidth}
   \begin{center}
     \includegraphics{16-2ia2f12.eps}
   \end{center}
   \caption{作成条件設定}
 \end{minipage}
\end{figure}



Scientists say they have made more progress 1. ($\langle$input okWord=''in''$\rangle$) developing malaria-resistant mosquitoes. The idea is to 2.($\langle$input okWord=''release''$\rangle$) genetically engineered insects like these into mosquito 3. ($\langle$input  okWord=''populations''$\rangle$) as a way to control the disease.

Clozeテストは4.1節で述べたように,元々は英語母語話者に対する読解評材の信頼度や難易度を測定する目的で開発されたものであり,語彙レベルや品詞を考慮したものは本来のClozeテストとは言えない.そのため試用者から寄せられた評価の中には,これらは特に必要な機能ではないとの声もある.しかし学習者に与える教材において難易度を考慮することは重要であり,言語処理技術との連携によって学習内容や学習者の能力に焦点を当てたテストを短時間で作成することに関しては,研究を深めていく余地が大いにあると考えている.

今後,このツールも新たな機能を加えながら随時アップデートを行っていく計画である.最新版や関連情報などは http://www.oit.ac.jp/ip/\textasciitilde kamiya/mwb/mwb.htmlに掲載している.


\section{おわりに}

データベースという概念には様々な意味があり,本来は区別して扱うべき機能や仕組みなどが混同されて用いられることがある.例えば,様々な教材そのものを蓄積し,必要なものを必要な時に自由に取り出せるような仕組みを「教材データベース」と呼ぶことがある.また表計算ソフト上で様々な学習項目などを整理したものを「データベース」と呼ぶこともある.本稿で扱ったものはこのような仕組みとは無関係であり,意地悪な見方をすれば,単にデータベース・ソフトウェアを用いて教材をワープロソフトよりもきれいに清書できる可能性がある,ということを述べているにすぎない.しかしこれまで主にビジネス用途でしか用いられてこなかったデータベース・ソフトウェアを教材作成に利用することは,レイアウトの柔軟さなどの点で非常に有効であると思われる.

本稿では言語処理技術と教材作成の連携について,GUI環境による使いやすいツールを構築していくことにより,今後一層両者のつながりが深まっていく可能性があることを述べた.またオンライン上で提供される言語処理関連のリソースの実行結果をデータベース・ソフトウェアで加工することで,プリント教材やE-learning教材をシームレスかつ効率的に作成・出力できることを提案した.今後も一般の英語教員のニーズに合致し,すぐに使える教材の形式について検討を深めながら,言語処理技術と教材作成の連携の可能性を追求し,教材の自動出力を行うことができる汎用性の高い教材作成ツールの開発を進めていきたい.

本研究で開発したツールによって作成したプリント教材等を普段から授業で利用している筆者らの印象では,予習→授業→復習のサイクルを意識した学習活動の促進や,様々な状況での実際の英語授業や自習環境において,有効に機能していると考えている.また短時間で様々な教材を作成できることは,教員の負担軽減にもつながることであろう.

\acknowledgment
\noindent 本文中で利用したサンプル英文は以下の記事を抜粋したものである.

\noindent
http://www.voanews.com/specialenglish/archive/2007-07/2007-07-01-voa2.cfm

\vspace{1\baselineskip}\noindent
教材作成ツール開発においてはツールをご利用頂く方々も含め,多くの先生方からご指導をいただきました.またオンライン上のリソースをWebビューアを介して利用することを許諾頂いた諸先生方に感謝いたします.

\vspace{1\baselineskip}\noindent 本研究の一部は文部科学省科学研究費補助金・若手研究(B)(課題番号18720153)により実施したものである.


\begin{thebibliography}{}

\item Coffey, G. \& Prosser S. 小山香織ほか訳(2008). FileMaker Pro大全 Ver.7〜9 edition. ラトルズ.

\item 大学英語教育学会基本語改訂委員会(2003). 大学英語教育学会基本語リスト JACET List of 8000 Basic Words. 大学英語教育学会.

\item 大学英語教育学会基本語改訂委員会(2004). 大学英語教育学会基本語リスト活用事例集:教育と研究への応用. 大学英語教育学会.

\item 神谷健一(2006). データベースソフトを用いた読解プリント教材とその作成ツールについて.社団法人 私立大学情報教育協会 平成18年度全国大学IT活用教育方法研究発表会予稿集, pp.~20--21.

\item 神谷健一,田中省作(2007). 言語処理技術を活用したPhrase Reading学習プリント教材作成ツール. 外国語教育メディア学会 第47回全国研究大会発表論文集, pp.~34--37.

\item 神谷健一,永野友雅,北尾謙治(2007). データベース・ソフトウェアを利用したクローズ・テスト学習教材の自動作成. 社団法人 私立大学情報教育協会 平成19年度大学教育・情報戦略大会, pp.~122--123.

\item Kenji Kitao \& Kenichi Kamiya(2008) ``Creating Cloze Exercises Easily and Effectively'' \textit{Proceedings of WorldCALL 2008}, http://www.j-let.org/\textasciitilde wcf/proceedings/g-016.pdf

\item 北尾謙治(2007). 語彙レベルや品詞も考慮したクローズ・テスト方式のeラーニング教材作成の試み. 外国語教育メディア学会 第47回全国研究大会発表論文集, pp.~114--117.

\item 岡本清美,神谷健一(2006). アカデミックリーディング教材—データベースを利用したプリント教材作成ツールを用いて. 外国語教育メディア学会 第46回全国研究大会発表論文集, pp.~243--251.

\item 佐藤史郎(1988). クローズ・テストと英語教育. 南雲堂.

\item Taylor, Wilson L. (1953) ``Cloze procedure: A new tool for measuring readability'' \textit{Journalism Quarterly}, 30, pp.~415--433,

\item 田中省作,木村恵,北尾謙治(2006). 言語処理技術を活用した柔軟性の高いスラッシュ・リーディング用教材作成支援システム, 外国語教育メディア学会 第46回全国研究大会発表要項集, pp.~483--492.

\item 行野顕正,田中省作,冨浦洋一,柴田雅博(2007). 統計的アプローチによる英語スラッシュ・リーディング教材の自動生成. 情報処理学会論文誌, 48(1), pp.~365--374


\end{thebibliography}

\clearpage

\begin{biography}

\bioauthor{神谷 健一}{
2000年大阪大学大学院言語文化研究科言語文化学専攻博士前期課程修了.高等学校教員を経て2004年〜大阪工業大学知的財産学部専任講師,英語科目を担当.外国語教育に生かすコンピュータの活用方法に関する研究に従事.外国語教育メディア学会,大学英語教育学会,e-Learning教育学会,社会言語科学会,英語コーパス学会,全国英語教育学会各会員.}

\bioauthor{田中 省作}{
2000年九州大学大学院システム情報科学研究科博士後期課程修了.同年九州大学情報基盤センター助手,2005年立命館大学文学部助教授(2007年准教授に職名変更).博士(工学)自然言語処理,言語教育への応用に関する研究に従事.情報処理学会,英語コーパス学会各会員.
}

\bioauthor{北尾 謙治}{
同志社大学英文学科卒業後,米国カンザス大学大学院でTESOLを研究,M.A.とPh.D.を取得.現在同志社大学文化情報学部教授.専門は応用言語学と異文化間コミュニケーション.著書はInternet Resources: ELT, Linguistics, and Communication(英潮社),Intercultural Communication: Between Japan and the United States(英潮社),English Teaching: Theory, Research and Practice(英潮社)ほか多数.
}
\end{biography}

\biodate




















\end{document}

