    \documentclass[japanese]{jnlp_1.4}
\usepackage{jnlpbbl_1.2}
\usepackage[dvips]{graphicx}
\usepackage{amsmath}
\usepackage{hangcaption_jnlp}
\usepackage{udline}
\setulminsep{1.2ex}{0.2ex}
\let\underline


\Volume{16}
\Number{4}
\Month{October}
\Year{2009}

\received{2008}{8}{31}
\revised{2009}{4}{15}
\rerevised{2009}{6}{1}
\accepted{2009}{7}{17}

\setcounter{page}{29}

\jtitle{ウェブコーパスと検索システムを利用した推量副詞と\\
	モダリティ形式の遠隔共起抽出と日本語教育への応用}
\jauthor{Srdanovi\'{c} Irena\affiref{tit}\affiref{lju} \and 
Hodo\v{s}\v{c}ek Bor\affiref{tit}\affiref{lju} \and \\
Beke\v{s} Andrej\affiref{lju} \and 
仁科喜久子\affiref{tit}}
\jabstract{
日本語におけるモダリティ形式および推量副詞と文末モダ
リティ形式との共起についての体系的な研究は自然言語処理の分野
において不十分である.さらに,このような情報は日本語教育の分
野においても十分カバーされていない.本稿では,コーパス検索ツ
ールSketch Engine (SkE)を利用した日本語の推量副詞とモダリティ
形式の遠隔共起の抽出を可能にすることとその日本語教育,特に日
本語学習辞典への応用の可能性を示すことを目的とする.そのため
にまず,複数のコーパスを分析した結果として,モダリティ形式と
そのバリエーションの網羅的なリストを作成した.このモダリティ
形式はChaSenでどのように形態素解析されているかを調査し,各モ
ダリティ形式の様々な形態素を新しいモダリティのタグとしてまと
めることによって,ChaSenで形態素解析されているJpWaCという大規
模ウェブコーパスから抽出した2千万語のサンプルへタグの再付与を
行った.最後に,新しくタグ付けされたコーパスをコーパス検索ツ
ールSkEに載せ,「文法関係ファイル」の内容を変更することで,推
量副詞と文末モダリティの共起の抽出を可能にした.抽出された共起
の結果は93\%以上の精度で高く評価された.得られた結果は言語資源
を利用しての日本語教育への応用の一例として,日本語教育における
辞書編集をはじめ様々な教育資源の作成のために,あるいは教室にお
ける直接的に利用可能となることを示した.
}
\jkeywords{推量副詞,モダリティ,遠隔共起,コーパス検索システム,日本語教育}

\etitle{Extraction of Suppositional Adverb and Clause-Final Modality Form Distant Collocations Using a Web Corpus and Corpus Query System and its Application \\
	to Japanese Language Learning}
\eauthor{Srdanovi\'{c} Irena\affiref{tit}\affiref{lju} \and 
Hodo\v{s}\v{c}ek Bor\affiref{tit}\affiref{lju} \and \\
Beke\v{s} Andrej\affiref{lju} \and 
Nishina Kikuko\affiref{tit}} 
\eabstract{
A systematic account of Japanese language modality forms as well
 as distant collocations between modal adverbs and clause-final 
modality forms is lacking in the field of natural language processing. 
The same stands for coverage of this kind of linguistic information 
in Japanese language education. In order to remedy this deficiency, 
in this paper we make extraction of Japanese adverbs and clause-final 
modality forms collocations possible using the corpus query system 
Sketch Engine and examine possibilities for its application in 
Japanese language learning, focusing on learner's dictionaries. 
First, as a result of analyzing various Japanese language corpora,
 we create a long list of modality forms and their variations. 
Then, we examine how ChaSen morphologically analyzes the forms 
and retag a sample of the large-scale Japanese language web corpus,
 JpWaC, by grouping all morphemes that correspond to individual 
modality forms together under a new modality tag. Finally, we load
 the newly tagged corpus into the Sketch Engine (SkE), modify the 
gramrel file and as a result obtain Word Sketch results for 
collocations between suppositional adverbs and modality forms. 
The evaluation of the collocation results shows that the proposed
 method reaches accuracy of above 93\%. The results can be utilized
 in the creation of Japanese learners' dictionaries or other language
 material or directly in language teaching or learning.
}
\ekeywords{suppositional adverbs, modality, distant collocations, 
Sketch Engine, Japanese language learning}

\headauthor{Srdanovi\'{c}, Hodo\v{s}\v{c}ek, Beke\v{s}, 仁科}
\headtitle{ウェブコーパスと検索システムを利用した遠隔共起抽出}

\affilabel{tit}{東京工業大学}{Tokyo Institute of Technology}
\affilabel{lju}{リュブリャーナ大学}{University of Ljubljana, Slovenia}



\begin{document}
\maketitle


\section{はじめに}

近年,コロケーションの研究は自然言語処理及びコーパス言語学において
盛んになっている.このコロケーションの一種である遠隔共起\footnote{
	本稿では,あるテクストにおいて,語と他の言語的要素が
	同時に出現する現象を「共起」と呼ぶ.情報処理およびコーパス言語
	学において,同じような意味で「コロケーション」という用語も利用
	される.文末モダリティ形式との共起は語と文法範疇の共起とも考え
	られ,そのような語と文法範疇の共起は「コリゲーション」という用
	語を使用することがある.それに対して,語と語の共起は「コロケー
	ション」と呼ばれる\cite{IshikawaBook}.本章では,両方の現象を「共起」
	という用語で扱う.「遠隔共起」というのは離れた位置に出現する共
	起である.}
は,頻繁に
現れる言語現象であるにも関わらず,研究としては取り上げられていない.日
本語における遠隔共起の一つとして推量副詞と文末モダリティ形式\footnote{
	たとえば,「\textgt{\underline{たぶん}}最初は発表のスタイルも
	ばらばらで 声もあまり出ない\textgt{\underline{だろう}}」の例には,
	「たぶん」は推量副詞,「だろう」は文末モダリティ形式である.
	推量副詞および文末モダリティは,両方とも,話し手の確信の度合いを
	表している.}
との共起
関係があげられ,日本語教育においても重要な問題の一つである.このような
共起はモダリティを二重に表現していることにより,テクストにおける重要な
語用論的な指標となっている\cite{Bekes}.
    \shortciteA{Srdanovic2008a}では,
    工藤 \citeyear{Kudou}が示した
確率論的性格を有する推量
副詞と文末モダリティの「共起」の振る舞いが,複数のコーパスの分析の結果
においても確認された.さらに,推量副詞と文末モダリティ形式の共起はテク
ストの種類によって著しく異なっていることが示され,日本語コーパス資料の
分類の可能性が明らかにされた.日本語においてモダリティ形式とそのバリエ
ーションが非常に多いにもかかわらず,そのリストが存在しないため,現在の
形態素解析ツールにおいては複数の形態素の連なりからなる様々なモダリティ
形式の認識が不可能である.このことから,本研究ではコーパス検索ツール
Sketch Engine (SkE) 
    \shortcite{Srdanovic2008b,Srdanovic2008d}
において日本語の推量副詞とモダリティ形式の遠隔共起が検索可能になるように機能
の拡張を試みる.実現方法としては,複数のコーパス分析の結果に基づいて,モ
ダリティ形式とそのバリエーションのリストを作成し,ChaSenで認識できるよう
にした上で,語の文法・共起情報を提示するために推量副詞と文末モダリティ形
式との遠隔共起が容易に抽出できるようにする.この抽出結果によって,日本語
の学習者,研究者,教師が推量副詞と文末モダリティ形式の共起表現を簡単に調
べられ,学習辞書や教科書などの作成に効率的に応用できるようになると考えら
れる.


\section{先行研究}

\subsection{言語学と日本語教育の分野における研究}

    南 \citeyear{Minami1,Minami2}は
日本語の文の階層構造を体系的に究明した.文において,
接続形式と述部の要素,および接続形式と述部以外の要素の共起制約から,四
階層の入れ子構造を確認している.以下の例(1)では,A層は述語「いる」1語
,B層は「この町にも五人ぐらいはいる」という核の文があり,C層では副詞
「どうやら」と文末モダリティ表現「〜らしい」が呼応している.

\vspace{0.5\baselineskip}
\begin{description}
\item{(1)} $\{$どうやら $[$この町にも五人ぐらいは(い)$_A$ $]_B$
らしい$\}_C$ \hspace{4zw} \cite{Bekes}
\end{description}
\vspace{0.5\baselineskip}

    南 \citeyear{Minami1}は,
内省ではなく,コーパスにおける共起の頻度を統計的に検証した結
果に基づいて,副詞と特定のモダリティ形式の共起の可能性を明確に示している.

    工藤 \citeyear{Kudou}も,
南と同様にコーパスを利用している.副詞と文末モダリティの
呼応の関係を検討しながら,それぞれの副詞と特定のモダリティ形式との共起の
結びつきの強弱の程度を明らかにし,結びつきの強いものを呼応・共起関係とみ
なした.推量副詞は対象面から言えば事態実現の確実さ,作用面から言えば話し
手の確信の度合いを表している\cite[p.~204,5]{Kudou}.この副詞群は「確信」
「推測」「推定」「不確定」という四つに区分され,「確信,推測,推定,不確
実」の順で,事態実現の確実さおよび話者の確信の度合いが低くなることを連続
的な関係で示すものである.推量副詞と共起するモダリティ形式は四つのモダリ
ティタイプ,即ち「に違いない」「はずだ」のような確信(NEC, necessity),
「だろう」「と思う」のような推測(EXP, expectation),「らしい」「みたい」
のような推定(CON, conjecture),「かもしれない」「かな」のような不確定
(POSS, possibility)に分けられている\cite[p.~203]{Kudou}.例えば,「多分」は
「だろう」「のだろうか」「と思う」「のではないか」などの推測(EXP)
タイプに属するモダリティ形式と強く呼応する.推量副詞と文末モダリティの
呼応の関係は,コーパスにおける共起の統計的な傾向として捉えることができ
る.工藤のデータについては,3.6で詳述する.

    Beke\v{s} \citeyear{Bekes}は,
会話コーパスにおける推量副詞と文末モダリティの呼応を
分析し,会話における副詞とモダリティの呼応関係の振る舞いを明らかにした.

    Srdanovi\'{c} et al. \citeyear{Srdanovic2008a}および Srdanovi\'{c} et al. \citeyear{Srdanovic2008c}は,
推量副詞とモダリティ形式の共起の傾向を
複数のコーパスを対象に分析している.コーパスの種類によって推量副詞の分布は
異なっており,この副詞と共起する文末モダリティ形式およびモダリティタ
イプの頻度の傾向にも違いが見られることを明らかにした.また,コロケー
ション・コリゲーションの研究において語句と語句との共起および語句と文
型との共起関係はしばしば研究されるが,遠隔共起関係についての認識はあ
まりないことから,日本語における推量副詞と文末モダリティ形式の共起を
一つの例として取り上げ,遠隔共起関係の重要性を指摘した.

日本語教育の面から,個別の副詞に関して,研究や教育への示唆がいくつか
あるが\cite{Sunakawa},コーパスに基づいた様々なモダリティ形式の整理及び
複数のコーパスによる推量副詞とそのモダリティ形式との共起関係の振る舞
いについての研究は見られない.また日本語教科書においての扱いも不十分
であることから,本研究では,日本語学習辞典の編纂への提案も考慮に入れる.


\subsection{自然言語処理における研究}

自然言語処理においては,モダリティ形式が機能表現および長単位の研究
において部分的に扱われているのが現状であり,日本語のモダリティ形式
を中心に扱っている研究はほとんど見られない.先行研究を概観するに際
して,推量副詞とモダリティ形式の呼応の研究とモダリティ形式を記載す
る電子化辞書などの自然言語処理のための言語資源に分け,その研究と自
然言語処理の汎用ツールを使用する上での未解決の問題点を述べる.


自然言語処理の汎用ツールを使用する際,意味的にはひとつのまとまりと
なっている「かも知れない」「に違いない」などのモダリティ形式が複数
の形態素または文節に分割されてしまう問題がある.また,「かもしれな
い」「かも知れません」「かも」のように,モダリティ形式の表記が文体
,丁寧さ,形態素の数などの様々な要素によって異なってくることがある.
この場合,モダリティ形式を記載している言語資源を使用すれば,モダリ
ティ形式の自動抽出あるいは認識が可能になる.

    松吉 他 \citeyear{Matsuyoshi}によって
編纂された階層構造を用いた機能表現辞書「つつ
じ」\footnote{
	http://kotoba.nuee.nagoya-u.ac.jp/tsutsuji/から配布されている.} 
はその言語資源の一つで,機能表現の一種とも考えられるモダリティ
形式がその中にある程度収録されている.
    また,富士池 他 \citeyear{Fujiike}は文節
を基にした複合辞と連語を含む「長単位」という言語単位を提唱し,「か
もしれません」「わけではない」などを1長単位として規定している.し
かし,前者は機能表現辞書の作成,後者は「長単位」という形態論の研究
であり,いずれもモダリティ形式に特化したものではないため,本稿で扱
っているモダリティ形式をすべて含んでいるわけではない.例えば,機能表現辞書「
つつじ」には,「といえない」「といえる」「と思う」「と考える」「の
か」「ように思う」「気がする」「気がしない」のモダリティ形式が含ま
れていない.また,「長単位」で認められるのは,助動詞相当として扱われてい
る「かもしれない」「てはいけない」「てはならない」「なければならな
い」「のだ」「のではない」「わけだ」「わけではない」「わけにはいか
ない」のみである\footnote{
	本稿で扱っているモダリティ形式は3.3節を参照されたい.}.
    木田 他 \citeyear{Kida}は
副詞とそれと共起する表現をまとめて呼応表現と呼び,
文中に副詞が現れると,その後に呼応する表現が来ることが予測できるこ
とに焦点をあてた.そのため,人手作業による副詞とモダリティの呼応関
係を含むタグ付きコーパスを作成し,モダリティ階層関係に基づく呼応関
係辞書を編纂した.しかし,人手作業のため,モダリティ形式を自動認識
できる言語資源およびシステムは提供していない.

SkEで推量副詞とモダリティ形式の共起関係を抽出するためには,網羅的
なモダリティ形式のリストおよび表記のゆれなどを認識できるような言語
資源やツールが必要である.そこで,先行研究を参考にしつつ,3.3節に
述べる方法でモダリティ形式およびそのバリエーションのリストを作成し,
見出し語と品詞の認定およびその再付与を行うことで,多様なモダリティ形
式と推量副詞の関係をSkE上で検索可能にすることを目指すことにする.



\section{SkEでの推量副詞とモダリティ遠隔共起の抽出}

\subsection{SkEとその主な機能}

SkEはコーパス検索システムであり,コンコーダンス機能以外に,
語に付随する文法とコロケーション情報を表示する機能Word Sketchや,
シソーラス情報や意味的に類似する語の共通点と差異を示す機能
(ThesaurusとSketch Difference)も備えている.SkEは現在,複数の
    言語に対応しており\shortcite{Kilgarriff2},
その一つに日本語版が
ある\cite{Srdanovic2008b}.日本語版は,JpWaCという4億語の
大規模Webコーパスを形態素解析ツールChaSenで解析したデータと,
日本語の「文法関係ファイル」から成る.このファイルは,様々な文法
関係・共起関係を22項目にまとめ,ChaSenの品詞体系を用いた正規表現
で定義している.SkEの検索の結果において,Word Sketch機能は名詞,
形容詞,形容動詞,動詞などと共起する要素を表示する.図1はその1例
として,「単語」という語との共起の一部の結果を示している.ここに
見られる共起関係は形容詞(modifier\underline{ }Ai),形容動詞
(modifier\underline{ }Ana),
動詞(をverb,が verb)などとの共起である.それぞれの欄に表示され
ている2 列の数字は,1 列目がコーパスの中の共起頻度を示し,2 列目
がその共起の統計的な顕著性(salience)\footnote{
	共起関係はダイス係数の統計値に基づいて抽出されている.}
を示している.表中の 1列
目の数字をクリックすると,例文がコンコーダンスの中で表示される.

\begin{figure}[t]
\begin{center}
\includegraphics{16-5ia2f1.eps}
\end{center}
\caption{Word Sketchでの「単語」の検索例}
\end{figure}

SkEに搭載されている大規模ウェブコーパスJpWaCは4億語からなり,
均衡コーパスを目指して慎重な抽出方法により構築されたものである
    \shortcite{Erjavec,Srdanovic2008b}.
最近の研究によると,ウェブ
コーパスはバランスの取れたデータの傾向を有していることが明らか
になっている\cite{Ueyama,Sharoff}.
    Srdanovi\'{c} et al. \citeyear{Srdanovic2008c} および Srdanovi\'{c} et al. \citeyear{Srdanovic2009}では,
13種の日本語のコーパスにおいても
推量副詞を分析した結果,JpWaCは,均衡コーパスとして設計された
BCCWJ (Balanced Corpus of Contemporary Written Japanese) の中の
書籍コーパス\cite{Yamazaki}ともっとも類似した結果を示している.
石川\citeyear{Ishikawa}はBCCWJと各種のウェブコーパスにおける基本語頻度の
データを比較し,ウェブコーパスは信頼性と有用性を持つことを実証
的に示した.SkEの応用の可能性は,日本語学習辞書,日本語学研究,
日本語教育などにおいて考えられる\cite{Srdanovic2008d}.モダ
リティ形式の多くは形態素の単位が連なった形となっており,
ChaSenにおいては直接認識されていない.そのため,本稿ではSkE日本
語版において形態素が連なったモダリティ形式をひとつの単位として
抽出可能にし,その結果から日本語教育への応用を検討する.

\subsection{推量副詞との遠隔共起の抽出}

SkEの日本語版においてWord Sketch機能で副詞と他の語との隣接共
起関係が見られる.しかしながら,副詞と離れた場所において共起
する語をみるためには,遠隔共起検索機能を追加しなければならない.
その方法として,ChaSenの品詞で識別できる要素ならば,SkEの「文法
関係ファイル」に単純に追加するだけでWord Sketchで表示できる.
ここで副詞と遠隔共起するものとして,1)副詞と(遠隔)共起する動詞
(modifies\underline{ }V), 2)副詞と(遠隔)共起する形容詞・形容動詞
(modifies\underline{ }Adj), 3)副詞と終助詞 (particle\underline{ }fin) 
の三つの共起関係が挙げられる.
図2は,一例として推量副詞の「たぶん」と共起する最も高頻度の動詞,
形容詞,終助詞を示しており,この頻度は遠隔共起も含んでいる.

\begin{figure}[t]
\begin{center}
\includegraphics{16-5ia2f2.eps}
\caption{Word Sketchで「たぶん」との共起の抽出}
(終助詞,形容詞・形容動詞,動詞の場合)
\end{center}
\end{figure}

しかしながら,副詞と文末モダリティ形式との共起については,
SkEの設計を変更するだけでは,抽出できない.その理由は,文末
モダリティ形式は形態素解析ツールChaSenで認識されていないこと,
および日本語のモダリティ形式とそのバリエーションのリストが存在
しないためである.次節では,SkEに推量副詞と文末モダリティとの
遠隔共起の表示を可能にする方法を検討する.


\subsection{モダリティ形式の抽出方法}

本論文では,\cite{Srdanovic2008a}の同様の調査に,さらに
データを追加したものを分析する.まずJpWaCコーパスから形態素数
2千万のサンプルコーパスを作成し,そのコーパスから推量副詞を
含む文を抽出し,モダリティ形式がどのように現れているかを調査
した.さらにこの調査を基に,モダリティ形式のリストの作成,複数
の形態素に分割されたモダリティ形式および推量副詞のタグの再付与
,SkEへの適用をする.

SkE用のコーパスのフォーマットとして,各単語に対し,出現形,
見出し語,品詞の三つのタグを用い,ChaSenの出力結果からそれらのみ
を使用する.また,2.2節で述べたようにモダリティ形式の多くが複数
の形態素から構成されるが,SkEでは1語がChaSenの1形態素に相当する
ので,モダリティ形式を一つの語にまとめ,コーパスにそれを再付与
する必要がある.サンプルコーパスの調査結果から,モダリティ形式
の出現には活用・スタイル・表記などの多数のバリエーションがある
ことが明らかになった.このため,「だろう」「でしょう」「であろ
う」「だろ」などの同意のモダリティ形式は一つの代表的なモダリテ
ィ形式にまとめ,見出し語を31語に統一した.その31項目のモ
ダリティ形式を表1に示す.

さらに,モダリティ形式をより網羅的に認識するために,
モダリティ形式が他のモダリティ形式やモダリティ形式以外の語と
連なって現れる場合も検討した.下記の文例は,実際にコーパス中
に見られたものである.

\begin{description}
\item[ ] 目の前の仕事が新鮮にうつることに\textgt{きっと}驚く\textgt{でしょう.} (例3.3.1)

\item[ ] \textgt{きっと}素晴らしいセミナーをなさる\textgt{ことでしょう!} (例3.3.2)

\item[ ] \textgt{きっと}皆さんもそう\textgt{であろうと思います.} (例3.3.3)
\end{description}

\begin{table}[b]
\caption{JpWaCサンプルコーパスに出現するモダリティ形式の見出し語}
\input{02table01.txt}
\end{table}

モダリティ形式ではない一定の語および形態素がモダリティ形式に接続
する傾向が見られた.例えば,例3.3.1のモダリティ形式の「でしょう」
に形式名詞「こと」が前接する例として3.3.2の「ことでしょう」がある.
このように,あるモダリティ形式が,単独で現れる場合と,他の語・
形態素と結合して,定型的な組み合わせの語句として現れる場合とで,
意味が異なることがある.このことから,モダリティ形式と語・形態素
による組み合わせからなる定型的な表現を新しいモダリティ形式として
扱うことにした.表1に示したモダリティ形式に接続できる12個の語群を
表2に示す.

\begin{table}[b]
\caption{モダリティ形式と接続可能な語・形態素の見出し}
\begin{center}
\input{02table02.txt}
\end{table}

また,例3.3.3のように,モダリティ形式が連続して現れる傾向が明らかになっ
た.あるモダリティ形式が,単独で現れる場合と,他のモダリティ形式と接続し
て現れる場合があり,その語句の出現の様相で意味が異なることがある.本稿で
は,例3.3.3の「であろうと思います」
のようなモダリティ形式を複合的なモダリティ形式,「であろう」のようなモダ
リティ形式を単独モダリティ形式と呼ぶことにする.SkEでは,複合的なモダリ
ティ形式を別々に扱うのではなく,一つの新しいモダリティ形式としてまとめて
扱うことにした.例3.3.3のモダリティ形式は「だろうと思う」になり,連続し
ている形態素をまとめてSkEの1単語とする.

表3と表4からわかるように,同じ意味・機能の表現であるにもかかわらず,
ChaSenによる見出し語が漢字とひらがなで区別されてしまう場合や,品詞タグが
異なっている場合がある.これはモダリティ形式に限らない.「必ずしも」や
「ひょっとしたら」などの複数の形態素から構成される副詞も同じ問題を含んで
いる.そこで,すべてのモダリティ形式と推量副詞が見出し語と品詞のレベルで統
一されるようにコーパス中のモダリティ形式および推量副詞の見出し語と品詞を
再付与した\footnote{
	そのために,SkEで検索する際に,統一した表現で検索する必要がある.例えば,
	すべての副詞は漢字ではなく,ひらがなで記入して検索することである(多分→
	たぶん).}.
ここで,モダリティ形式の品詞タグはすべて「Mod」という新しい
カテゴリに書き換えた.これにより,モダリティ形式と推量副詞の表記が文章の
スタイルおよびChaSenの誤解析によって異なる場合でも,網羅的にそれらをSkE
で検索したり,違いを見たりすることが可能になる.

\begin{table}[t]
\caption{JpWaCコーパスにおいてChaSenによる見出し語が異なる場合の処理}
\input{02table03.txt}
\end{table}
\begin{table}[t]
\caption{JpWaCコーパスにおいてChaSenによる品詞タグが異なる場合の処理}
\input{02table04.txt}
\end{table}

以上の方法によって得られたモダリティ形式の数は,代表的なモダリティ形式は
31個,組み合わせで現れるモダリティ形式は596個,その出現形のバリエーショ
ンは2,641個である.

以上の点を考慮し,抽出する様々な共起関係を定めるSkEの「文法関係ファイル」
に表5の副詞とモダリティ形式の関係を定める規則を追加する.

\begin{table}[t]
\caption{副詞とモダリティ形式の関係を定める規則}
\input{02table05.txt}
\end{table}

この規則は,SkE において実装されている正規表現と品詞及び単語に基づいた,
「corpus query syntax(コーパス検索シンタクス)」 を適用している.「DUAL」
は副詞とモダリティ形式の関係を両方向から検索できる規則である.「Adv」と
「modality」はそれぞれ「副詞」と「モダリティ形式」を意味する.この規則で
は,副詞に続いて最初に現れるモダリティ形式のみとの関係を抽出することでは
なく,文末まで現れる他のモダリティ形式との共起が多くあることから,すべて
のモダリティ形式との可能性をみることとした.また,この規則で得られる推量
副詞とモダリティ形式の共起は構文を考慮していない.そこで係り受け解析器
CaboChaを使用した共起の抽出方法も考えられるが,本手法と比べていくつかの
利用上の欠点がある.一つは,モダリティ形式が係り受けの単位となる文節の境
界を越えることがある.他の一つは,CaboChaが学習したドメインとは異なる
JpWaCのようなウェブ文章における遠隔共起の正確な係り受け解析が難しいと考
えられることである.


\subsection{抽出結果}

JpWaCの2千万語サンプルコーパス中に見られる複数のモ
ダリティ形式とそのバリエーションにモダリティタグを付け,新しい共起関係を
SkEの「文法関係ファイル」に組み込んだことで,推量副詞と文末モダリティの
共起がSkEのWord Sketchの機能の中で表示できるようになった\footnote{
	現時点で通常の使用では出来ないが,ユーザからの要請により,SkEのウェブペー
	ジでJpWaCのサンプルコーパスにおいて副詞とモダリティの共起を検索すること
	ができる.}.

\begin{figure}[b]
\begin{center}
\includegraphics{16-5ia2f3.eps}
\caption{Word Sketchにおける推量副詞とモダリティ形式の共起表現検索結果}
(「たぶん」,「どうやら」,「もしかしたら」,「かならずしも」とモダリティ
 形式との例)
\end{center}
\end{figure}

例として,「たぶん」「どうやら」「もしかしたら」「かならずしも」と
そのモダリティ形式との共起の表示結果を図3に示す.それぞれの推量副詞が多様なモ
ダリティ形式と共起していること,結びつくモダリティ形式のうち代表的なもの
がそれぞれの副詞で異なっていることが分かる.「たぶん」は「のだろう」「だ
ろう」「と思う」「のだと思う」などの推測(EXP)のモダリティタイプとよく
共起する傾向がある.「どうやら」は推定(CON)の「らしい」「ようだ」「のよ
うだ」「みたいだ」など,「もしかしたら」は不確定(POSS)の「かもしれない」
「のかもしれない」「のかな」など,「必ずしも」は確信(NEC)の「とはかぎ
らない」「わけではない」「ものではない」などとよく共起する傾向がある.

表6では,コーパスにおける推量副詞の分布とそれぞれの副詞と顕著性計指数が
高い項目の順に5つのモダリティ形式が並んでいる.

まず,推量副詞の分布から見ると,高頻度の副詞は「たぶん」「かならず」「お
そらく」「きっと」「ぜったい」などである.推量副詞とモダリティ形式の共起
関係を見ると,「のだろう」「だろう」「と思う」などの「推測」を表すモダリ
ティ形式および「はず」「のだ」「に違いない」などの「確信」を表すモダリティ
形式がもっとも高頻度の共起関係として出現することがわかる.ある副詞が「推
測」のモダリティ形式と共起する傾向があれば,その副詞の共起の中では,「確
信」のモダリティ形式もよく見られる.たとえば,「きっと」と「おそらく」と
共起する「に違いない」などである.逆に,副詞が「確信」のモダリティ形式と
共起する場合,その副詞は「推測」のモダリティ形式とも共起する.たとえば,
「かならず」「ぜったい」と,「だろう」「と思う」という「推測」の文末形式
が共起する場合である.

\begin{table}[t]
\caption{JpWaCのサンプルにおける高頻度の推量副詞と文末モダリティ形式の共起関係}
\input{02table06.txt}
\end{table}


JpWaCのサンプルデータと均衡コーパスであるBCCWJの中の書籍のコーパスにおけ
る推量副詞の分布と推量副詞と文末モダリティの共起を比較した結果,類似した
分布及び共起関係の傾向を示していることが明らかになった\cite{Srdanovic2009}.


SkEの中にあるWord Sketch機能だけでなく,コンコーダンス,Sketch
Difference,シソーラスなどの機能においても推量副詞とモダリティ形式につい
て検索できる.たとえば,類似している二つの語の間の共起の傾向における共通
点と差異が閲覧できるSketch Differenceという機能では,「たぶん」と「きっ
と」を比較する際,「きっと」は「のだろう」「に違いない」「ことだろう」
「はず」との共起が多く,確信のモダリティタイプに近い性質が見られる.一方,
「たぶん」のほうは推測を表す「と思う」「のだと思う」「のではないかと」
「と思うのだ」とよく共起することから,「きっと」より推量の度合いが低いと考え
られる.


\subsection{SkEで抽出された推量副詞と文末モダリティ形式の共起の評価}

本節では,Word Sketchによる推量副詞と文末モダリティ形式との共起の判定の
精度を評価する.推量副詞と共起する代表的なモダリティタイプが四つあり
\cite{Kudou,Srdanovic2008a},それぞれのタイプから一つの高頻度の副詞
を評価のために選択する.また,それぞれの副詞と共起するモダリティの検索結
果から,四つのモダリティ形式を選び(Word Sketchに表示するモダリティ形式
のリストの最初から四番目までの四項目),コンコーダンス機能から選んだこの
副詞と共起するモダリティ形式の10例文をランダムに取り出す.従って,合計
160の例文を評価することとする.

選択した例文について日本語教育専門家二名(A,B), 自然言語処理専門家一名
(C)から評価を受ける.選択肢は「(正)副詞とモダリティ形式の共起は正し
い」,「(?)副詞とモダリティ形式の共起は部分的に正しい」,「(誤)副詞
とモダリティ形式の共起は正しくない」であり,コメントを記入できる欄もある.

\begin{table}[t]
\caption{評価の結果}
\input{02table07.txt}
\end{table}

表7に見られるように「(正)副詞とモダリティ形式の共起は正しい」の回答か
ら,精度は高く評価されていることがわかる.正しく評価されていない例文と評
価者のコメントを見ると以下のようなことがわかる.

\begin{itemize}
\item[1)] モダリティ部分とみなす部分に誤解析となる例文がある.
\item[2)] 日本語として正しくない例文がある.
\item[3)] コーパスには重複する例文がある.
\item[4)] 副詞とモダリティ形式が共起していない例文がある.
\item[5)] 表示されているものとは異なる他のモダリティ形式と共起している例
	   文がある.
\end{itemize}

1),2),3)は比較的修正が容易である.1)については,部分的に抽出したモ
ダリティ形式をバリエーションリストに追加すること,タグの付け方の改善で解
決できる.また,2),3)については,正しくない文および重複している文の排
除を行うことで処理できる.4)と5)の原因は,形態素解析の誤りに起因する複
数の文がコーパスに入っている場合と,構文が複雑なため副詞が自動的に指定さ
れたモダリティ形式と共起しない場合が考えられる.今後の課題として,構文中
の呼応関係を示すための構文解析プログラムを改善する必要がある. 


\subsection{工藤データとの比較}

前節では共起項目検索結果の精度が高いと評価されたことを述べたが,さらにこ
の結果を日本語研究における推量副詞と文末モダリティ形式の呼応関係について
よく精査された研究\cite{Kudou}と比較する.工藤によると副詞とモダリティ形
式との呼応は程度の問題だと考えられ,それぞれの副詞は特定のモダリティタイ
プに属するモダリティ形式とよく共起する傾向があるとしている(図4).たと
えば,「きっと」は確信のモダリティタイプに属する「する・のだ」「に違いな
い」などの形式と最も共起する.「おそらく」は推測の「だろう・まい」「と思
われる」「のではないだろうか」と多く共起する.「どうやら」は確定の「らし
い」など,「もしかしたら」は不確定の「かもしれない」などと多く共起する傾
向が見られる.

\begin{figure}[t]
\begin{center}
\includegraphics{16-5ia2f4.eps}
\end{center}
\caption{工藤による推量副詞と文末モダリティのコーパス分析結果}
\end{figure}

本研究で抽出されたデータにも,工藤のデータと同様に推量副詞と共起する四つ
のモダリティタイプが見られる.このデータにも,推測と確信のモダリティタイ
プとの共起は最も多く見られる.工藤のデータに見られる様々なモダリティ形式
はJpWaCにも表れる.しかし副詞の頻度の順序,およびそれぞれの副詞の共起傾
向には差異も見られる.「きっと」との共起は確信のモダリティタイプより,推
測のモダリティタイプの方が多く現れる.工藤のデータにはやや古い用語も現れ,
例えば,「さぞ」の頻度は,JpWaCより頻度がかなり高い.この差異の理由は,
    Srdanovi\'{c} et al. \citeyear{Srdanovic2008c}に
複数のコーパスの分類結果で指摘されているように,コー
パスの種類が異なるためである.


\section{日本語教育への応用}

SkEは既に多数の言語において語学教育,辞書作成などに利用されており
    \shortcite{Smrz,Smith,Kilgarriff1},
日本語版も日本語教育な
どのために役立つシステムになると考えられる\cite{Srdanovic2008d}.
本研究で抽出できるようになった推量副詞とモダリティ形式の共起に関する結果
は,日本語学習のシラバス,教科書,学習辞典などの学習資源を作成するために利用
できる.また,抽出された共起関係は学習者や教師が授業中に参照したり,授業
の準備をしたりするためにも利用できる.さらに,整理したモダリティ形式とモ
ダリティ形式のバリエーションのリストが日本語教育のための言語処理の資源な
ど様々の目的のために利用できる.

検索方法の面からみるとSkEで推量副詞とモダリティ形式の共起に関して,いく
つかのメリットが見られる.(1)ウェブ上で推量副詞と文末モダリティの共起
が瞬時に効率的に抽出できる.(2)共起に関しての高頻度の情報と統計的に重
要な共起情報が得られ,学習項目に優先順位を付けることができる.(3)
Sketch Differenceの機能において,類似している副詞の共起関係の傾向におけ
る差異を簡単に把握できる.(4)コンコーダンス機能で,例文を細かく考察で
き,複雑な遠隔共起も検索でき,様々なコンコーダンスオプションも利用できる.

一つの応用の例として,日本語の学習辞典の編集・改善案を具体的に検討する.
抽出できるようになった推量副詞と文末モダリティ形式の共起関係の結果を3種
の日本語学習辞典と比較する.学習者によく利用されている和英辞典2種(『ジー
ニアス和英辞典』,『ニューセンチュリー和英辞典』)と,中級レベル以上の日
本語学習者にとって問題となる文型を網羅的に集めた辞典(『日本語文型辞典』)
を対象とする.表8は,これらの辞典においてどの推量副詞の項目がカバーされてい
るか,また記載されている各副詞の項目の中で,その例文中にどのモダリティ
形式が現れているかを示している.

\begin{table}[t]
\caption{日本語辞書における推量副詞と文末モダリティ形式との共起}
\input{02table08.txt}
\par\vspace{4pt}
\small
△副詞は辞書にあるが,共起するモダリティ形式は記載されていない\\
▲モダリティ形式が例文にある\\
●モダリティ形式が他の表現として言及されている
\end{table}


推量副詞の分布から見ると,和英辞典2種には対象となるすべての推量副詞が現
われている.「文型辞典」には,現われない副詞もある.「あんがい」「おおか
た」「ことによる」のような最も低頻度の副詞以外に,コーパス中にかなり出現
している「ぜったい」「ぜったいに」が現われていない.本研究で抽出した結果
によると,この二つの副詞はかなり出現することから,中級以上の日本語学習辞典に
も含めることを考慮する必要がある\footnote{
	\shortciteA{Srdanovic2009}の研究によると,
	この副詞はインフォーマルの会話コーパスで
	は非常に高頻度で,均衡コーパスの書籍とJpWaC,新聞コーパス,知恵袋のウェ
	ブコーパス,フォーマルな会話コーパスでもかなり頻度が高い.}.
反対に,「さぞ」「たいがい」はJpWaCのコー
パスにはあまり現れない.特に「さぞ」はやや古い表現だと考えられる\footnote{
	この副詞は教科書系のコーパスでは頻度が非常に高いが,均衡コーパスの書籍で
	は,頻度が非常に低い.}.

推量副詞と共起するモダリティ形式については,和英辞典2種の例文中にモダリ
ティ形式が多く現れている.しかし,それぞれの辞典で例文に現れるモダリティ
形式は異なっており,特定の副詞と共起する代表的なモダリティ形式が足りない
場合もかなりある.たとえば,「たぶん」には,「かもしれない」「ものだ」の
代わりに,「のだろう」「と思う」を入れた方がよい.また,「おそらく」には,
「と思う」「に違いない」も追加したほうが良いと考えられる.「文型辞典」で
は,推量副詞のすべての項目に網羅的にモダリティ形式が現れていることが評価で
きる.しかし,この辞典においても,コーパス中に高い頻度および高い顕著性
で現れるモダリティ形式のデータを導入することについて検討の余地がある.たとえば,「お
そらく」という副詞に関し,「ものと思われる」のかわりに,「のだろう」「こ
とだろう」「と思う」のモダリティ形式を載せたほうがよい.「かならず」の項
目では推量のモダリティ形式が現われていないので,「はずだ」「のだ」「だろ
う」「に違いない」を追加した方がよい\footnote{
	JpWaCにおける最も高頻度の副詞とモダリティ形式の共起関係は均衡コーパスの
	書籍と類似した結果が見られる.}.

以上の指摘は,均衡コーパスに基づく調査結果によるものであり,一般的な日本
語習得を目的とした辞典の作成のためのものである.しかし,コーパスの種類に
よって代表的なモダリティ形式が異なるので\cite{Srdanovic2008a,Srdanovic2008c},学習辞典
にもこの調査結果を学習目的に合わせて反映させることが望ましい.一つの例と
して,理系教科書,自然言語処理の論文集のような理系の特定コーパスの調査に
おいて,均衡コーパスとは異なり,「必ずしも」が他の推量副詞より頻度が非常
に高いことがわかった.このような結果を反映させることで,辞典に特定目的の
学習をカバーすることが可能となる.


\section{まとめと今後の課題}

本稿では,推量副詞と文末モダリティに関連する先行研究を調査した結果から,
代表的なモダリティ形式に関する実証的な情報と網羅的に作成されたモダリティ
形式のリストが欠如していることを明らかにした.次に,コーパス検索システム
SkEの主な機能を紹介し,SkEに推量副詞と文末モダリティ形式などの遠隔共起関
係の機能の追加を,次の1)〜4)までの手順で行った.
\begin{itemize}
\item[1)] モダリティ形式とそのバリエーションのリストの作成,
\item[2)] ChaSenによるモダリティ形式の形態素解析の検定とモダリティ形式の
 タグの作成,
\item[3)] ChaSenで形態素解析されたコーパスへのタグの再付与(モダリティ形
 式のタグの追加),
\item[4)] 「文法関係ファイル」に副詞とモダリティ形式の共起関係を定める規
 則の追加.
\end{itemize}

次に,SkE中のWord Sketchで抽出が可能になった推量副詞と文末モダリティ形式
の共起の評価を行い,抽出された共起項目が高く評価されたことを示した.さら
に,得られた結果を推量副詞についての工藤の日本語研究と比較し,共通点と差
異を明らかにした.最後に,SkEのWord Sketch 機能で推量副詞とモダリティ形
式の遠隔共起をどのように日本語教育に応用できるかを検討した.具体的な例と
して,日本語学習辞典における推量副詞と文末モダリティ形式の共起の扱いを検
討した.

今後の課題として,モダリティ形式とそのバリエーションのリストをxml形式で
作成,ウェブコーパス以外のデータに基づいてさらに改善することを目指す.ま
た,様々なコーパスを利用することで,推量副詞と文末モダリティ形式の共起の
データをジャンル別に比較することで,日本語教育における目的別の学習への活
用を考える.




\bibliographystyle{jnlpbbl_1.4}
\begin{thebibliography}{}

\bibitem[\protect\BCAY{Beke\v{s}}{Beke\v{s}}{2008}]{Bekes}
Beke\v{s}, A. \BBOP 2008\BBCP.
\newblock \BBOQ Japanese suppositional adverbs: probability and structure in
  speaker-hearer interaction.\BBCQ\
\newblock {\Bem Linguistica}, {\Bbf 48}, \mbox{\BPGS\ 277--292}.

\bibitem[\protect\BCAY{Erjavec, Srdanovi\'{c}, \BBA\ Kilgarriff}{Erjavec
  et~al.}{2007}]{Erjavec}
Erjavec, T., Srdanovi\'{c}, I., \BBA\ Kilgarriff, A. \BBOP 2007\BBCP.
\newblock \BBOQ A large public-access Japanese corpus and its query tool.\BBCQ\
\newblock In {\Bem CoJaS 2007}.

\bibitem[\protect\BCAY{富士池 \Jetal }{富士池\Jetal }{2008}]{Fujiike}
富士池優美 \Jetal  \BBOP 2008\BBCP.
\newblock 『現代日本語書き言葉均衡コーパス』における長単位の概要.\
\newblock
  \Jem{科学研究費補助金特定領域研究「日本語コーパス」平成19年度公開ワークショ
ップ(研究成果報告会)予稿集}, \mbox{\BPGS\ 51--58}.

\bibitem[\protect\BCAY{石川}{石川}{2008}]{IshikawaBook}
石川慎一郎 \BBOP 2008\BBCP.
\newblock \Jem{英語コーパスと言語教育}.
\newblock 大修館書店.

\bibitem[\protect\BCAY{石川}{石川}{2009}]{Ishikawa}
石川慎一郎 \BBOP 2009\BBCP.
\newblock 日本語基礎研究における非統制型・統制型・媒介型Web as
  Corpusの可能性—言語コーパスにおける基本語頻度の安定性について.\
\newblock
  \Jem{特定領域研究「日本語コーパス」平成20年度公開ワークショップサテライトセ
ッション予稿集}, \mbox{\BPGS\ 29--38}.

\bibitem[\protect\BCAY{木田\JBA 高梨\JBA 乾\JBA 井佐原}{木田 \Jetal
  }{2002}]{Kida}
木田敦子\JBA 高梨克也\JBA 乾裕子\JBA 井佐原均 \BBOP 2002\BBCP.
\newblock 文構造の漸進的予測を可能にする日本語の諸特徴の分析.\
\newblock \Jem{言語処理学会第8回年次大会発表論文集}, \mbox{\BPGS\ 65--68}.

\bibitem[\protect\BCAY{Kilgarriff \BBA\ Rundell}{Kilgarriff \BBA\
  Rundell}{2002}]{Kilgarriff1}
Kilgarriff, A.\BBACOMMA\ \BBA\ Rundell, M. \BBOP 2002\BBCP.
\newblock \BBOQ Lexical Profiling Software and its Lexicographic
  Applications---a Case Study.\BBCQ\
\newblock In {\Bem EURALEX 2002 Proceedings}, \mbox{\BPGS\ 807--818}.

\bibitem[\protect\BCAY{Kilgarriff, Rychly, Smr\v{z}, \BBA\ Tugwell}{Kilgarriff
  et~al.}{2004}]{Kilgarriff2}
Kilgarriff, A., Rychly, P., Smr\v{z}, P., \BBA\ Tugwell, D. \BBOP 2004\BBCP.
\newblock \BBOQ The Sketch Engine.\BBCQ\
\newblock In {\Bem Proc. Euralex}, \mbox{\BPGS\ 105--116}.

\bibitem[\protect\BCAY{工藤}{工藤}{2000}]{Kudou}
工藤浩 \BBOP 2000\BBCP.
\newblock 副詞と文の陳述のタイプ.\
\newblock 森山卓郎\JBA 仁田義雄\JBA 工藤浩\JEDS, \Jem{日本語の文法3モダリティ},
  \mbox{\BPGS\ 161--234}. 岩波書店.

\bibitem[\protect\BCAY{松吉\JBA 佐藤}{松吉\JBA 佐藤}{2007}]{Matsuyoshi}
松吉俊\JBA 佐藤理史 \BBOP 2007\BBCP.
\newblock 体系的機能表現辞書に基づく日本語機能表現の言い換え.\
\newblock \Jem{言語処理学会第13回年次大会発表論文集}, \mbox{\BPGS\ 899--902}.

\bibitem[\protect\BCAY{南}{南}{1974}]{Minami2}
南不二男 \BBOP 1974\BBCP.
\newblock \Jem{現代日本語の構造}.
\newblock 大修館書店.

\bibitem[\protect\BCAY{南}{南}{1993}]{Minami1}
南不二男 \BBOP 1993\BBCP.
\newblock \Jem{現代日本語文法の輪郭}.
\newblock 大修館書店.

\bibitem[\protect\BCAY{Sharoff}{Sharoff}{2006}]{Sharoff}
Sharoff, S. \BBOP 2006\BBCP.
\newblock \BBOQ Open-source corpora: using the net to fish for linguistic
  data.\BBCQ\
\newblock {\Bem International Journal of Corpus Linguistics}, {\Bbf 11}  (4),
  \mbox{\BPGS\ 435--462}.

\bibitem[\protect\BCAY{Smith, Chen, \BBA\ Kilgarriff}{Smith
  et~al.}{2007}]{Smith}
Smith, S., Chen, A., \BBA\ Kilgarriff, A. \BBOP 2007\BBCP.
\newblock \BBOQ A corpus query tool for SLA: learning Mandarin with the help of
  Sketch Engine.\BBCQ\
\newblock In {\Bem Practical Applications in Language and Computers---PALC
  2007}.

\bibitem[\protect\BCAY{Smr\v{z}}{Smr\v{z}}{2004}]{Smrz}
Smr\v{z}, P. \BBOP 2004\BBCP.
\newblock \BBOQ Integrating Natural Language Processing into E-learning---A
  Case of Czech.\BBCQ\
\newblock In {\Bem Proceedings of the Workshop on eLearning for Computational
  Linguistics and Computational Linguistics for eLearning, COLING 2004},
  \mbox{\BPGS\ 106--111}.

\bibitem[\protect\BCAY{Srdanovi\'{c}, Beke\v{s}, \BBA\ Nishina}{Srdanovi\'{c}
  et~al.}{2008a}]{Srdanovic2008a}
Srdanovi\'{c}, I., Beke\v{s}, A., \BBA\ Nishina, K. \BBOP 2008a\BBCP.
\newblock \BBOQ Distant Collocations between Suppositional Adverbs and
  Clause-Final Modality Forms in Japanese Language Corpora.\BBCQ\
\newblock In Tokunaga, T.\BBACOMMA\ \BBA\ Ortega, A.\BEDS, {\Bem LKR 2008, LNAI
  4938}, \mbox{\BPGS\ 252--266}. Springer-Verlag Berlin Heidelberg.

\bibitem[\protect\BCAY{Srdanovi\'{c}, Erjavec, \BBA\ Kilgarriff}{Srdanovi\'{c}
  et~al.}{2008b}]{Srdanovic2008b}
Srdanovi\'{c}, I., Erjavec, T., \BBA\ Kilgarriff, A. \BBOP 2008b\BBCP.
\newblock \BBOQ A web corpus and word-sketches for Japanese.\BBCQ\
\newblock \Jem{自然言語処理}, {\Bbf 15}  (2), \mbox{\BPGS\ 137--159}.

\bibitem[\protect\BCAY{Srdanovi\'{c} \BBA\ Nishina}{Srdanovi\'{c} \BBA\
  Nishina}{2008}]{Srdanovic2008d}
Srdanovi\'{c}, I.\BBACOMMA\ \BBA\ Nishina, K. \BBOP 2008\BBCP.
\newblock コーパス検索システムSketch Engineの日本語版とその利用方法.\
\newblock \Jem{日本語科学}, {\Bbf 23}, \mbox{\BPGS\ 59--80}.

\bibitem[\protect\BCAY{Srdanovi\'{c}, Beke\v{s}, \BBA\ Nishina}{Srdanovi\'{c}
  et~al.}{2008}]{Srdanovic2008c}
Srdanovi\'{c}, I., Beke\v{s}, A., \BBA\ Nishina, K. \BBOP 2008\BBCP.
\newblock 複数のコーパスに見られる副詞と文末モダリティの遠隔共起関係.\
\newblock
  \Jem{科学研究費補助金特定領域研究,「日本語コーパス」平成19年度公開ワークシ
ョップ(研究成果発表会)予稿集}, \mbox{\BPGS\ 223--230}.

\bibitem[\protect\BCAY{Srdanovi\'{c}, Beke\v{s}, \BBA\ Nishina}{Srdanovi\'{c}
  et~al.}{2009}]{Srdanovic2009}
Srdanovi\'{c}, I., Beke\v{s}, A., \BBA\ Nishina, K. \BBOP 2009\BBCP.
\newblock BCCWJにおける推量副詞とモダリティ形式の共起.\
\newblock
  \Jem{科学研究費補助金特定領域研究「日本語コーパス」平成20年度公開ワークショ
ップ(研究成果報告会)予稿集}, \mbox{\BPGS\ 237--244}.

\bibitem[\protect\BCAY{砂川}{砂川}{2007}]{Sunakawa}
砂川有里子 \BBOP 2007\BBCP.
\newblock 機能語のコロケーション(1)副詞との共起について.\
\newblock \JTR, 科学研究費補助金特定領域研究日本語教育班第2回公開会議.

\bibitem[\protect\BCAY{Ueyama \BBA\ Baroni}{Ueyama \BBA\ Baroni}{2005}]{Ueyama}
Ueyama, M.\BBACOMMA\ \BBA\ Baroni, M. \BBOP 2005\BBCP.
\newblock \BBOQ Automated construction and evaluation of a Japanese web-based
  reference corpus.\BBCQ\
\newblock In {\Bem Proceedings of Corpus Linguistics}.

\bibitem[\protect\BCAY{山崎}{山崎}{2006}]{Yamazaki}
山崎誠 \BBOP 2006\BBCP.
\newblock 代表性を有する現代日本語書き言葉コーパスの設計.\
\newblock \Jem{国立国語研究所(2006)所載}, \mbox{\BPGS\ 63--70}.

\end{thebibliography}

\begin{biography}
\bioauthor[:]{Srdanovi\'{c} Irena}{
1997年ベオグラード大学文学部日本語学科卒業.
2006年リュブリャーナ大学文学部一般言語学・比較言語学研究科修士課程修了.
2009年9月東京工業大学大学院社会理工学研究科人間行動システム専攻博士課程終了.
博士(学術).
言語処理学会,日本語教育学会会員.
}
\bioauthor[:]{Hodo\v{s}\v{c}ek Bor}{
2007年米国メリーランド大学カレッジパーク校人文学部日本語学学科卒業.
同年リュブリャーナ大学文学部一般言語学・比較言語学研究科修士課程入学.
2008年東京工業大学大学院社会理工学研究科人間行動システム専攻修士課程入学,現在に至る.
言語処理学会会員
}
\bioauthor[:]{Beke\v{s} Andrej}{
1971年リュブリャーナ大学理学部数学学科卒業.
1975年大阪大学大学院理学研究科数学専攻修士課程修了.
1981年筑波大学大学院人文社会科学研究科文芸・言語専攻入学.
1986年同課程修了.
1988年リュブリャーナ大学社会学部専任講師.
2002年同大学文学部正教授(日本学),現在に至る.
文学博士.日本語教育学会会員.
}
\bioauthor{仁科喜久子}{
1969年東京女子大学・文理学部卒業.
1977年同大学大学院文学研究科修士課程修了.
1986年埼玉大学専任講師,1988年東京工業大学助教授を経て,1996年同大学留学生センター教授(社会理工学研究科兼任),現在に至る.
博士(学術).言語処理学会,日本語教育学会,教育工学会会員.
}

\end{biography}

\biodate

\end{document}

\jauthor{\affiref{tit}\affiref{lju} \and 
\affiref{tit}\affiref{lju} \and \\
\affiref{lju} \and 
仁科喜久子\affiref{tit}}
