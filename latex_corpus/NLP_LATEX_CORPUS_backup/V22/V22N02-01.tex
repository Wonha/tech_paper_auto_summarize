    \documentclass[japanese]{jnlp_1.4}
\usepackage{jnlpbbl_1.3}
\usepackage[dvipdfm]{graphicx}
\usepackage{amsmath}
\usepackage{hangcaption_jnlp}
\usepackage{udline}
\setulminsep{1.2ex}{0.2ex}
\usepackage{array}

    \usepackage{amssymb}
    \usepackage{bm}
\usepackage[at]{easylist}	
\usepackage{arydshln}		


\Volume{22}
\Number{2}
\Month{June}
\Year{2015}

\received{2014}{9}{21}
\revised{2015}{1}{11}
\accepted{2015}{3}{3}

\setcounter{page}{77}

\jtitle{医療用語資源の語彙拡張と診療情報抽出への応用}
\jauthor{東山 翔平\affiref{Author_1} \affiref{Author_0} \and 関  和広\affiref{Author_2} \and 上原 邦昭\affiref{Author_1}}

\jabstract{
  近年,医療文書の電子化が進み,大規模化する医療データから有用な情報を
  抽出・活用する技術が重要となっている.特に,診療記録中の症状名や診断
  名などの用語を自動抽出する技術は,症例検索などを実現する上で必要不可
  欠である.機械学習に基づく用語抽出では,辞書などの語彙資源の利用が訓
  練データに含まれない用語の認識に有効である.しかし,診療記録では多様
  な構成語彙の組合せからなる複合語が使用されるため,単純なマッチングに
  基づく辞書の利用では検出できない用語が存在し,語彙資源利用の効果は限
  定的となる.そこで,本稿では,語彙資源を有効活用した用語抽出を提案す
  る.資源活用の1点目として,資源中の用語に対して語彙制限を行うことで,
  用語抽出に真に有用な語彙の獲得を行う.2点目として,資源から複合語の構
  成語彙である修飾語を獲得し,元の語彙に加えて獲得した修飾語を活用する
  ことで,テキスト中のより多くの用語を検出する拡張マッチングを行う.検
  出された用語の情報は機械学習の素性として用いる.NTCIR-10 MedNLPテスト
  コレクションを用いた抽出実験の結果,単純な語彙資源の利用時と比較して
  適合率および再現率の向上を実現し,本手法の有効性を確認した.また,肯
  定・否定などのモダリティ属性の分類を含めた抽出では,従来手法に対して,
  本手法が最も高い精度を実現した.
}
\jkeywords{医療言語処理,用語抽出,NTCIR-10 MedNLPタスク}


\etitle{Vocabulary Expansion of Medical Language Resources for Medical Information Extraction}
\eauthor{Shohei Higashiyama\affiref{Author_1} \affiref{Author_0} \and Kazuhiro Seki\affiref{Author_2} \and Kuniaki Uehara\affiref{Author_1}}
\eabstract{
 With the increasing number of medical documents written in an electronic 
format, automatic term extraction technologies from unstructured texts have 
become increasingly important. 
 Particularly, the extraction of medical terms such as complaints and 
diagnoses from medical records is crucial because they serve as the basis 
for more application-oriented tasks, including medical case retrieval.
 For machine-learning-based term extraction, language resources such as 
lexica and corpora are effective for recognizing expressions that rarely 
or do not occur in training data.
 However, the use of lexica by simple word-matching approaches has limited 
effects because there are compound words that comprise various combinations 
of constituent terms in medical records.
 Therefore, this study presents term extraction systems that can exploit 
language resources by the acquisition and utilization of beneficial terms 
and constituents from the resources.
 Our experimental results on the NTCIR-10 MedNLP test collection, which 
comprises medical history summaries, show increased precision and recall, 
indicating the effectiveness of the proposed system.
 Moreover, compared to existing systems developed for the NTCIR-10 MedNLP 
task, the proposed system achieved optimum performance for complaint and 
diagnosis recognition, including the classification of extracted terms 
into modality attributes. 
}
\ekeywords{Medical NLP, Term extraction, NTCIR-10 MedNLP task}

\headauthor{東山,関,上原}
\headtitle{医療用語資源の語彙拡張と診療情報抽出への応用}

\affilabel{Author_1}{神戸大学大学院システム情報学研究科}{Guraduate School of System Informatics, Kobe University}
\affilabel{Author_0}{現在,NEC 情報・ナレッジ研究所}{Presently with NEC Knowledge Discovery Research Laboratories}
\affilabel{Author_2}{甲南大学知能情報学部}{Faculty of Intelligence and Informatics, Konan University}



\begin{document}
\maketitle

\section{はじめに}

近年,電子カルテに代表されるように,医療文書が電子的に保存されることが
増加し,構造化されていないテキスト形式の医療情報が増大している.大規模
な医療データには有用な情報が含まれ,新たな医学的知識の発見や,類似症例
の検索など,医療従事者の意思決定や診療行為を支援するアプリケーションの
実現が期待されている.これらの実現のためには,大量のテキストを自動的に
解析する自然言語処理技術の活用が欠かせない.特に,テキスト中の重要な語
句や表現を自動的に認識する技術は,固有表現抽出や用語抽出と呼ばれ,情報
検索や質問応答,自動要約など,自然言語処理の様々なタスクに応用する上で
必要不可欠な基盤技術である.

用語抽出を実現する方法として,人手で作成した抽出ルールを用いる方法と,
機械学習を用いる方法がある.前者の方法では,新しく出現した用語に対応す
るために随時ルールの修正や追加を行わなければならず,多大な人的コストが
かかる.そのため,近年では,データの性質を自動的に学習することが可能な
機械学習が用いられることが多くなっている.

機械学習に基づく用語抽出では,抽出すべき語句の情報がアノテーションされ
た訓練データを用いてモデルの学習を行い,学習したモデルを未知のデータに
適用することで,新しいデータから用語の抽出を行う.高精度な抽出を可能と
するモデルを学習するには,十分な量の訓練データがあることが望ましい.し
かし,診療記録などの医療文書は,医師や患者の個人情報を含むため,医療機
関の外部の人間が入手することは困難である.幸い,近年は,研究コミュニティ
でのデータ共有などを目的とした評価型ワークショップが開催されてお
り\cite{uzuner20112010,morita2013overview},匿名化などの処理が施された
医療文書データが提供され,小規模なデータは入手可能になっている .とはい
え,依然として,学習に利用できる訓練データの量は限られることが多い.

他方,一般に公開されている医療用語辞書などの語彙資源は豊富にあり,
英語の語彙資源では,生物医学や衛生分野の用語集,シソーラスなどを含む
UMLS (Unified Medical Language System)\footnote{http://www.nlm.nih.gov/research/umls/},
日本語の語彙資源では,広範な生命科学分野の領域の専門用語などからなる
ライフサイエンス辞書\footnote{http://lsd.pharm.kyoto-u.ac.jp/ja/index.html},
病名,臨床検査,看護用語などカテゴリごとの専門用語集を含むMEDIS標準
マスター\footnote{http://www.medis.or.jp/4\_hyojyun/medis-master/index.html}
などが提供されている.
辞書などの語彙資源を利用した素性(辞書素性)は,訓練データに少数回しか
出現しない用語や,まったく出現しない未知の用語を認識する際の手がかり
として有用であるため,訓練データの量が少ない場合でも,こうした語彙資源
を有効活用することで高精度な抽出を実現できる可能性がある.
しかし,既存の医療用語抽出研究に見られる辞書素性は,テキスト中の語句に
対して辞書中の用語と単純にマッチングを行うものに留まっている
\cite{imaichi2013comparison,laquerre2013necla,miura2013incorporating}.
診療記録では多様な構成語彙の組合せからなる複合語が使用されるため,
単純な検索ではマッチしない用語が存在し,辞書利用の効果は限定的であるといえる.

本研究では,類似症例検索などを実現する上で重要となる症状名や診断名(症
状・診断名)を対象とした用語抽出を行う.その際,語彙資源から症状・診断
名の構成要素となる語彙を獲得し,元のコーパスに併せて獲得した語彙を用い
ることで,より多くの用語にマッチした辞書素性を生成する.そして,生成し
た辞書素性を機械学習に組み込むことで,語彙資源を有効活用した抽出手法を
実現する.また,提案手法の有効性を検証するために,病歴要約からな
るNTCIR-10 MedNLPタスク\cite{morita2013overview}のテストコレクションを
用いて評価実験を行う.

本稿の構成は以下の通りである.まず,\ref{chp:related_work}章で医療用語
を対象とした用語抽出の関連研究について述べ,\ref{chp:baseline_system}章
で本研究のベースとなる機械学習アルゴリズムlinear-chain CRFに基づくシステムを説明する.
\ref{chp:util_resources}章では,語彙資源から症状・診断名の構成語彙
を獲得する方法と,獲得した語彙を活用した症状・診断名抽出手法を説明す
る.\ref{chp:experiments}章でMedNLPテストコレクションを用いた評価実験に
ついて述べ,最後に,\ref{chp:conclusions}章で本稿のまとめを述べる.


\section{関連研究} \label{chp:related_work}
\vspace{-0.5\Cvs}

テキスト中の特定の語句や表現を抽出する処理を固有表現抽出や用語抽出とい
う.固有表現は,主に人名・地名・組織名などの固有名詞や時間・年齢などの
数値表現を指し,固有表現抽出では,これらの固有表現を抽出の対象とするこ
とが多い.一方,用語抽出では特定の分野の専門用語などを対象とする.しか
し,対象とする語句を抽出するというタスク自体に違いはないため,本稿では,
特に両者を区別せず,抽出対象として指定される語句を固有表現と呼ぶ.


\subsection{医療用語抽出研究}

日本語の医療文書を対象に医療用語の抽出や探索を行った研究として,
\pagebreak
\cite{inoue2001iryo,kinami2008kango,uesugi2007n-gram}が
ある.井上ら\cite{inoue2001iryo} は,文章の記述形式に定型性のある医療論
文抄録を対象に,パターンマッチングに基づく方法を用いて,病名(「論文が
取り扱っている主病名」)や診療対象症例(「診断,治療の対象とした患者,
症例」)などの事実情報を抽出した.たとえば,病名は「〜症」「〜炎」「〜
腫」などの接尾辞の字種的特徴を手がかりに用い,診療対象症例に対しては,
「対象は〜」「〜を対象とし」のような対象症例と共起しやすい文字列を手が
かりに用いて抽出している.なお,井上らの報告によると,論文抄録中に含ま
れる病名や診療対象症例の出現回数は平均1回強である.アノテーションを行っ
た医療論文抄録を用いた抽出実験では,病名や診療対象症例に関して,90\%前
後の適合率,80\%から100\%近い再現率という高い精度を得ている.しかし,本
稿で対象とする病歴要約では論文抄録と出現の傾向が異なり,同一文書中に様々
な病名が出現することが多いため,井上らのパターンマッチングに基づく手法
では抽出精度に限界があると考えられる.

木浪ら\cite{kinami2008kango} は,専門用語による研究情報検索への応用を目
的として,再現率の向上を優先した看護学用語の抽出手法を提案した.抽出対
象とされた専門用語は,解剖学用語(「血小板」,「破骨細胞」など)や看護
行為(「止血」,「酸素吸入」など)を含み,本稿で対象とする症状・診断名
よりも広い領域の用語である.特定の品詞を持つ語が連続した場合にそれらの
語を連接して抽出するなど,連接ルールに基づくシステムにより専門用語を抽
出した.システムによる抽出を行うフェーズと,抽出されなかった用語や誤っ
て抽出された用語を人手で分析してルールの修正・追加を行うフェーズを繰り
返し,再現率が向上し,再現率が低下しない範囲で適合率が向上するようなルー
ル集合を導出している.実験では,専門家によるアノテーションを行った看護
学文献を用いて評価し,再現率約80\%という抽出結果を得ている.この手法で
は,適用する専門用語の領域が異なる場合,再び導出手順を踏んでルールを導
出し直す必要がある.しかし,抽出ルール修正の過程が人手による判断に依存
しており,再度ルールを導出する際の人的コストが大きい.また,適合率が
約40\%と低く,適合率を改善するにはルール導出の基準自体も修正する必要が
生じる.

上杉\cite{uesugi2007n-gram}は,医療用語抽出の前処理として,医療辞書なし
で医療コーパス中の用語間の分割位置を探索する研究を行った.文字列X,Yの
出現確率に対し,XYが同時に出現する確率が十分に低ければX,Y間を分割でき
るとの考えに基づき,コーパスから求めた文字列の出現確率と相互情報量を使
用して分割位置を決定している.症例報告論文を用いた実験では,約740語に対
して60\%の分割精度\footnote{この文献で63.4\%と報告されている分割位置探
  索の成功事例の中には,分割された単語の内部でさらに不適切な位置で分割
  されたものが含まれている.意味のある区切りで分割された事例のみ考慮す
  ると,分割精度は約60\%であった.}であり,分割に成功した事例の中には,
複合語が1語と認識される場合と複合語の内部でさらに分割される場合がほぼ同
等の割合で存在した.残りの40\%には,助詞が付加される,英字やカタカナ列
の途中で分割されるなどの誤りがあるため,自然言語文に対する分割精度とし
て十分であるとはいえない.なお,医療用語抽出に応用するには,分割された
各単語が医療用語か否かを判定する基準が別途必要となる.


\subsection{医療言語処理ワークショップとNTCIR-10 MedNLP}

近年,医療文書を対象とした共通タスクを設定し,研究コミュニティでのデー
タ共有や,データ処理技術の向上を目的とする参加型ワークショップが開催さ
れている.英語の医療文書を処理の対象としたタスクとしては,2011年およ
び2012年にNISTが主催するTRECにおいてMedical Records trackが設定さ
れ,2006年および2008年から2012年に渡ってi2b2 NLPチャレンジが開催された.
i2b2 NLPチャレンジでは,診療記録からの情報抽出
技術の評価を目的とした共通タスクが実施され,
匿名化のための個人情報\cite{uzuner2007evaluating},患者の喫煙状態\cite{uzuner2008identifying},
医薬品の使用状況\cite{uzuner2010extracting}や医療上のコンセプト\cite{uzuner20112010}の抽出が行われた.
また,日本語の医療文書を使用したタスクとして,
2013年にはNIIが主催するNTCIRにおいてMedNLPタスク\cite{morita2013overview}が設定され,
患者の個人情報や診療情報を対象に情報抽出技術の評価が行われた.

NTCIR-10 MedNLPタスクでは,医師により書かれた架空の患者の病歴要約からな
る日本語のデータセット(MedNLPテストコレクション)が使用された.データ
から患者の年齢,日時などの個人情報を抽出する「匿名化タスク」と,患者の
症状や医師の診断などの診療情報(症状・診断名)を抽出する「症状と診断タ
スク」などが設定された.症状・診断名には,症状の罹患の肯定,否定などを
表すモダリティ属性が定義されており,モダリティ属性の分類もタスクの一部となっ
ている.なお,両方のタスクとも,それぞれ個人情報,診療情報を固有表現と
した固有表現抽出とみなせる.タスク参加者のシステムは,ルールに基づく手
法よりも機械学習に基づく手法が多く,特に,学習アルゴリズムとしてCRF
(Conditional Random Fields) \cite{lafferty2001conditional} の代表的な
モデルであるlinear-chain CRFを用いたシステムが高い性能を発揮した.
また,成績上位のシステムでは,文中の各単語が辞書中の語とマッチしたか否を
表す情報(辞書素性)が共通して用いられており,語彙資源の利用が精度向上
に寄与したことがわかる.一方,匿名化タスクではルールに基づく手法も有効
であり,最高性能を達成したのはルールベースのシステムであった.

Miuraら\cite{miura2013incorporating} は,固有表現抽出タスクを文字単位の
系列ラベリング\footnote{文をトークン(文字や単語)の列とみなし,
各トークンに対して固有表現か否かなどを表すラベルを推定していく方法を指す.}
として定式化してlinear-chain CRFを適用し,
症状と診断タスクで最も高い精度を達成した.固有表現の抽出を行った後,
抽出した固有表現のモダリティ属性を決定するという2段階の方法を使用している.
MEDIS標準マスターおよびICH国際医薬用語集\footnote{https://www.pmrj.jp/jmo/php/indexj.php}
を語彙資源に用いて辞書素性を与えている.

Laquerreら\cite{laquerre2013necla},Imaichiら
\cite{imaichi2013comparison} は,ともに単語単位の系列ラベリングとして
linear-chain CRFを適用し,症状と診断タスクでそれぞれ2番目,3番目の精度を達成している.
Laquerreらは,ライフサイエンス辞書と
UMLS Metathesaurusを利用し,辞書素性を導入している.また,事前知識に基
づくヒューリスティック素性として,「ない」「疑い」などモダリティ属性判別の
手がかりとなる表現を素性としている.Imaichiらは,Wikipediaから収集した
病名,器官名などの用語集に基づく辞書素性を導入している.


\section{Linear-chain CRFに基づく症状・診断名抽出システム}
\label{chp:baseline_system}

本章では,症状・診断名の抽出のためにベースとするシステムを説明する.本
研究では,用語抽出の処理を症状・診断名からなる固有表現を抽出するタスク
とみなし,系列ラベリングとして定式化する.
また,固有表現の抽出は,機械学習アルゴリズムlinear-chain CRFを用いて行う
(以降,本研究で用いたlinear-chain CRFを指す場合,単にCRFと呼ぶ).さら
に,CRFで抽出を行った出力に,人手で作成したルールを用いて抽出誤り訂正の
後処理を行うことで,より誤りの少ない抽出を実現する.


\subsection{MedNLP「症状と診断タスク」の定義}
\label{sec:mednlp_taskdef}

NTCIR-10 MedNLP (Medical Natural Language Processing) タスク
\cite{morita2013overview}は,医療分野における情報抽出技術の評価を目的
として実施されたタスクである.診療記録から症状・診断名を抽出する「症状
と診断タスク」を含む3つのサブタスクが定義された.本研究では,MedNLP
タスクで使用されたデータセット(MedNLPテストコレクション)を使用し,
症状と診断タスクと同様の設定で症状・診断名の抽出を行う.以下,MedNLP
テストコレクションおよび症状と診断タスクについてそれぞれ説明する.


\subsubsection*{MedNLPテストコレクション}

MedNLPテストコレクションは,医師により書かれた架空の患者の病歴要約50文書
からなるデータセットであり,2対1の比率で訓練データとテストデータに分割
されている.同テストコレクションには,患者の個人情報および診療情報が
アノテーションされている.個人情報は患者の名前,年齢,性別と,日時,地名,
医療機関名からなり,診療情報は患者の症状や医師の診断(症状・診断名)を指す.
このうち,症状と診断タスクで対象とされたのは後者の診療情報である.

症状・診断名には,医師の認識の程度などを表すモダリティ属性が定義されて
おり,それぞれ症状の罹患の肯定,否定,推量(可能性の存在)を表す
positive,negation,suspicionに加え,症状が患者の家族の病歴として
記述されていることを表すfamilyの4種類からなる.

以下,{\tt <n></n>,<s></s>,<f></f>}で囲まれた範囲をそれぞれモダリティ属性が
negation,suspicion,familyである症状・診断名であるとして,
各モダリティ属性が付与された症状・診断名の例を示す.
\renewcommand{\labelenumi}{}
\begin{enumerate}
 \item {\tt「<n>糖尿病</n>は\underline{認めず}」}
 \item {\tt「<n>関節症状</n>は\underline{改善した}」}
 \item {\tt「<s>神経疾患</s>の\underline{疑いにて}」}
 \item {\tt「<s>味覚異常</s>の\underline{可能性を考え}」}
 \item {\tt「<n>胆嚢炎</n>を\underline{疑わせる所見を認めず}」}
 \item {\tt「\underline{母親};<f>気管支喘息</f>」}
 \item {\tt「\underline{父}:<f>狭心症</f>、<f>心筋梗塞</f>」}
 \item {\tt「\underline{娘}3人は鼻粘膜の<f>易出血性</f>があり」}
\end{enumerate}
negation,suspicion属性については,(a)〜(d)に示すように,症状・診断名の
係り先の文節が特定の表現を含む場合に,その表現に対応するモダリティ属性と
なることが多い.ただし,(e)のように,直接の係り先がsuspicionを示す表現を
含んでいても,後方に否定表現が現れることによりnegationとなることがある.
family属性の症状・診断名は,(f),(g)のように,続柄名の後に症状・診断名を
列挙する形で記述された場合が該当する他,(h)のように,文全体の主語が続柄
である場合も該当する.
なお,positive属性は,negation,suspicionおよびfamilyであることを示す
表現と共起しない症状・診断名に対して付与される属性とみなせる.

\begin{table}[b]
\caption{訓練データにおけるモダリティ属性の分布}
\label{tab:medtr_tag}
\input{01table01.txt}
\end{table}

表\ref{tab:medtr_tag}に,訓練データにおける各モダリティ属性の出現回数
(``\#''の列)を示す.出現回数はモダリティ属性の種類により偏りがあり,
positiveが7割程度を占めているのに対し,suspicion,familyは非常に少数
となっている.


\subsubsection*{症状と診断タスク}

症状と診断タスクは,テストコレクションから,患者に関連する症状・診断名
\footnote{文献の引用など,症状・診断名への言及が注目している患者の
症状・診断を表すものではない場合には,抽出の対象とされない.}
を抽出し,モダリティ属性を決定するタスクとして定義された.評価は,訓練
データを用いて開発されたシステムに対して,テストデータでの抽出性能を
測ることで行われ,評価尺度として,$F$値 ($\beta=1$) が使用された.
$F$値は,適合率 (Precision) と再現率 (Recall) の調和平均であり,次式で定義される.
\[
 \frac{2\cdot \mbox{Presicion} \cdot \mbox{Recall}}{\mbox{Precision}+\mbox{Recall}}
\]
また,評価方法として,モダリティ属性の分類を
考慮する評価と,考慮しない評価の2通りの方法で評価が行われた.前者の方法では,
抽出された固有表現が正しい属性に分類されないと正解とならないのに対して,
後者の方法は,属性が誤っていても,抽出された固有表現の範囲がアノテーション
された正解情報と一致していれば正解とみなされる.


\subsection{固有表現抽出タスクの定式化}
\label{sec:ner_formulation}

本研究では,各モダリティ属性が付与された症状・診断名を異なるカテゴリの
固有表現とみなして固有表現抽出を行うことで,症状・診断名の抽出範囲および
モダリティ属性の決定を行う.
固有表現抽出は,入力文中の固有表現部分を同定するタスクであり,系列ラベ
リングとして定式化されることが多い.系列ラベリングとは,入力系列に対し
てラベルの系列を出力する問題である.固有表現抽出では,トークンの系列で
ある文を入力として,トークンごとに固有表現の種類等を表すラベルを推定し,
それらラベルの列を出力する.本研究では,形態素をトークンとする系列ラベ
リングとして固有表現抽出の定式化を行う.

定式化の際には,同一の固有表現(チャンク)中の位置を表すためのチャンキ
ング方式として,IOB2フォーマット\cite{sang1999representing}を使用した.
IOB2では,タグ
はB (Begin),I (Inside),O (Outside) の3種類があり,それぞれチャンク
の先頭,チャンクの先頭を除く内部,チャンクの外側を表す.なお,複数のカ
テゴリの固有表現が存在する場合,B,Iはカテゴリ名と併せて使用される.た
とえば,「嘔吐/B-c\_pos~~出現/I-c\_pos~~した/O」のようにラベルを付与
することで,「嘔吐出現」がc\_pos(肯定のモダリティ属性を持つ症状・診断
名)というカテゴリの固有表現であることを表す.なお,入力文の形態素解析
には,linear-chain CRFに基づく日本語形態素解析器であるMeCab
\cite{kudo2004applying} (Ver.~0.996) およびIPADIC (Ver.~2.7.0) を用いた.


\subsection{CRFによる分類}
\label{sec:crf}

本システムでは,機械学習アル
ゴリズムとしてlinear-chain CRFを用いた.
linear-chain CRFは,分類アルゴリズムである最大エントロピー法を出力が
構造を有する問題(構造学習)に拡張したモデルである.
品詞タギング\cite{lafferty2001conditional},基本名詞句同定\cite{sha2003shallow},
形態素解析\cite{kudo2004applying}など,構造学習,特に,系列ラベリングとして
定式化できる自然言語処理の様々な問題に適用され,高い性能が報告されている.
また,固有表現抽出の研究においても広く使用され\cite{mccallum2003early,jiang2011study},
NTCIR-10 MedNLPタスクでも最もよく用いられた
\cite{imaichi2013comparison,laquerre2013necla,miura2013incorporating}.

linear-chain CRFの実装としては,C++ で記述されたオープンソースソフトウェア
であるCRF++ 
\footnote{http://crfpp.googlecode.com/svn/trunk/doc/index.html} (Ver.~0.58) を利用した.CRF++ は,準ニュートン法の一種であ
るL-BFGS (Limited-memory BFGS) を使用して数値最適化を行っており,省メモ
リで高速な学習を実現している.また,素性テンプレートという素性の記述形
式が定義されており,テンプレートを利用することで多様な素性を容易に学習
に組み込むことができるという特長がある.

CRF++ の素性テンプレートでは,出力ラベル系列についてのunigram素性およびbigram
素性が利用可能である.出力ラベルunigram素性は,入力系列についての任意の情報
(系列中の特定のトークンの形態素自体や品詞など)と現在のトークンのラベルの組
からなる素性を指し,出力ラベルbigram素性は,入力系列の任意の情報,現在の
トークンのラベルおよび 1 つ前のトークンのラベルの三つ組からなる素性を指す.
本研究では,すべての素性に対して出力ラベルunigram素性およびbigram素性の両方
を使用することとした.たとえば,品詞素性を$-2$から2までのウィンドウ
で用いると述べた場合,注目するトークンの 2 つ前方から 2 つ後方に位置する
トークンの品詞について,それぞれで出力ラベルunigram素性およびbigram素性を
使用することを意味する.

なお,指定した値未満の出現回数である素性を学習に使用しないことを意味する
素性のカットオフの閾値は1とし,訓練データに出現したすべての素性を使用する
ことにした.また,正則化には$L2$正則化を用いた.


\subsection{抽出に用いる素性}
\label{sec:features}

本システムでは,症状・診断名抽出のための素性として,「形態素」,「形態
素基本形」,「品詞」,「品詞細分類」,「字種」,「辞書マッチング情報」,
「モダリティ表現」の7種類の情報を用いた.
注目するトークンを起点にどこまでの範囲のトークンの情報を素性に用いるか
を表すウィンドウサイズについては,モダリティ素性を除き,
\ref{chp:experiments}章で述べる実験により最適な値を決定する.
モダリティ素性のウィンドウサイズについては後述する.


\subsubsection*{形態素,基本形,品詞,品詞細分類素性}

形態素素性はトークン自体であり,基本形,品詞,品詞細分類素性は,それぞ
れ各形態素の基本形,品詞,品詞細分類である.たとえば,「言っ」の基本形
は「言う」,「速く」の基本形は「速い」となる.品詞は,名詞,動詞,形容
詞,接続詞など10数種類存在し,品詞細分類は,名詞であれば「固有名詞」,
「形容動詞語幹」などがある.
これら四つの素性には,形態素解析器MeCabの出力を利用した\footnote{異なる
  品詞間の同名の細分類を区別するため,「名詞/一般」,「副詞/一般」の
  ように「品詞/品詞細分類」の形で細分類素性を記述した.また,品詞細分
  類がさらに細分化されている場合は,「固有名詞・人名・姓」のように最上
  層から最下層までのすべての細分類を「・」記号で併記し,1 つの品詞細分
  類という扱いとした.}.なお,MeCabではデフォルトの辞書としてIPADICが
採用されており,IPADICで使用されている品詞および品詞細分類は,
IPA品詞体系として定義されている.


\subsubsection*{字種素性}

字種素性は,トークンを構成する文字の字種パターンを表す.本研究では,ひ
らがな,カタカナ,漢字,英大文字,英小文字,ギリシャ文字,数値,記号と
これらの組合せからなる字種パターンを定義した.たとえば,「レントゲ
ン」は「カタカナ」,「考え」は「$\text{ひらがな}+\text{漢字}$」,「MRI」は「英大文字」
が字種パターンとなる.なお,「$\text{ひらがな}+\text{漢字}$」では,1 つのトークン中にひ
らがなと漢字が出現していればこのパターンに相当するものとし,各文字の出
現順は無視した.複数字種の組合せからなるパターンの扱いは,他のパター
ンについても同様であ
\linebreak
る.


\subsubsection*{辞書素性}

辞書素性は,専門用語辞書など外部の語彙資源を利用した素性であり,入力文
中の形態素列が辞書中の語句と一致したか否かという情報を表す.本研究では,
症状・診断名の抽出のために病名等の用語から構成される辞書を使用し,入力
文中のトークン列で,辞書中の語句とマッチした部分にIOB2フォーマットに
基づくタグを付与した.
たとえば,「腎機能障害」という語句が辞書中に含まれる場合,「腎/機能/
障害/の/増悪」と分割された形態素列に対して「B/I/I/O/O」というタグ
が付与される.
なお,辞書マッチは文字数についての最左最長一致で判定し,マッチした範囲
の境界が形態素の内部にある場合,マッチ範囲に完全に包含される形態素に
対してのみタグを付与した.


\subsubsection*{モダリティ素性}

モダリティ素性は,症状・診断名のモダリティ属性を示す表現(モダリティ
表現)を捉えるための素性である.
モダリティ表現の具体例として,先行する
症状・診断名のモダリティがnegationであることを示す「〜なし」や「〜を認
めず」,同様にsuspicionであることを示す「〜疑い」や「〜を考え」などが
ある.また,「母」「息子」など続柄を表す表現は,共起する症状・診断名の
モダリティ属性がfamilyであることを示すモダリティ表現であるといえる.

本素性では,negation,suspicionおよびfamily属性を対象に,モダリティ属性推定
の手がかりとなる表現を正規表現で記述し,文中の各トークンについて,
最左最長一致で正規表現とマッチした範囲の端に位置するトークンまでの距離
(形態素数)と,マッチした表現が表すモダリティ属性を示すタグを付与した.
なお,positive属性は,negation,suspicionおよびfamilyであることを示す
表現と共起しない症状・診断名に対して付与される属性とみなせるため,
これらのモダリティ表現が周辺に存在しないことがpositive属性と推定する
ための手がかりとなる.

例として,negationのモダリティ表現を捉える正規表現の 1 つとして
次の(a)を
使用している\footnote{``$X|Y$''は$X$または$Y$との一致,``[$X_1 X_2 \dots
  X_N$]''は$X_1,X_2,\dots,X_N$のいずれかとの一致,``()''はパターンのグ
  ループ化,``?''は直前のパターンの0または1回の出現を表す.}.
\renewcommand{\labelenumi}{}
\begin{enumerate}
 \item \![はがを]?(([認め$|$みとめ$|$見$|$み$|$得$|$え)(ら
  れ)?)?([無な](い$|$く$|$かっ)$|$せ?ず)
\end{enumerate}
この正規表現を用いると,「運動/麻痺/は/み/られ/ず」という入力に対
して「はみられず」という部分がマッチし,マッチした範囲の先頭トークン
「は」までの距離は,「麻痺」で1,「運動」で2となる.
negationとsuspicionのモダリティ表現は症状・診断名の右側に出現するため,
モダリティ表現が右側に出現しているトークンにのみ,出現を示すタグを
付与した.反対に,familyの表現は「母:脳梗塞」のように左側に出現するた
め,モダリティ表現が左側に出現しているトークンにのみタグを付与した.

なお,negationの正規表現は上述の(a)を含む4件,suspicionの正規表現は
以下の(b)を含む2件,familyの正規表現は以下の(c)~1件を使用した
\footnote{``[父母]''に後続する``(?![指趾])''というパターンは,
医療用語である「母指」や「母趾」とのマッチを防ぐために用いた.}.
モダリティ素性として使用した正規表現の全リストは付録\ref{sec:modal_regex}に記載する.
\begin{enumerate}
\setcounter{enumi}{1}
 \item の?(疑い$|$うたがい)
 \item {[祖伯叔]?[父母](?![指趾])親?$|$お[じば]$|$[兄弟姉妹娘]$|$息子$|$従(兄弟$|$姉妹)}
\end{enumerate}
\renewcommand{\labelenumi}{}

また,過学習を抑制する目的で,モダリティ属性ごとに距離のグループ化を
行った.negationとsuspicionでは,距離1〜2(近くに出現),3〜4(やや近く
に出現),5〜8(やや遠くに出現),9〜12(遠くに出現)の4通りの距離を考慮し,
familyでは$-\infty$〜$-13$,$-12$〜$-9$,$-8$〜$-5$,
$-4{\rm〜}-1$の4通りとした.なお,距離は,注目するトークンの右側にモダリティ
表現が出現している場合に正数,左側に出現している場合に負数で表しており,
距離の絶対値は,注目するトークンから同一文中の任意のモダリティ表現(正規表現
とマッチしたトークン列)内の最も近いトークンまでのトークン数に相当する.
また,``$-\infty$〜''は文の左端以降での出現を意味している.
以上のモダリティ素性の定義に基づくと,前述の例に対しては,
「運動」および「麻痺」に``neg1\_2'',それ以外のトークンに
モダリティ表現の非出現を表すタグ``O''が付与される.

なお,モダリティ素性として付与したタグは,注目しているトークンのタグのみ
学習・推定に用いた(つまり,ウィンドウサイズを1とした).
トークンに付与される素性タグにより,トークンの周辺文脈の情報が表現
されており,実質的に12以上先のトークンまで考慮していることになる.


\subsection{抽出誤り訂正のための後処理ルール}
\label{sec:postprocess}

著者らは,NTCIR-10 MedNLPタスクに参加し,構造化
パーセプトロン\cite{collins2002discriminative}およびlinear-chain CRF
に基づく固有表現抽出システムを開発してきた
\cite{higashiyama2013clinical,higashiyama2013developing}.
開発したシステムについてエラー分析を行ったところ,
観測された誤りの中で,モダリティ属性の分類誤りが大きな割合を占めること
がわかった.この種の誤りは単純なルールで修正できるため,モダリティ属性
の分類誤りに焦点を当て,誤り訂正のためのルールを作成することにした.
観測されたモダリティ属性分類誤りは,さらに次の2通りに大別できたため,
それぞれの誤りに対応する訂正ルールとして,2 つのルールを作成した.
\begin{itemize}
 \item 固有表現の周辺に(非positive属性の)モダリティ表現が存在し,
       かつ推定された属性が同表現が表す属性と異なっている
       (主にpositiveとなっている)誤り
 \item 固有表現の周辺にモダリティ表現が存在せず,
       かつ推定された属性がpositive以外となっている誤り
\end{itemize}
\vspace{8pt}

作成した 2 つのルールを以下に示す.
``[]''の外側(左側)にnegationおよび
suspicion属性に対する条件,``[]''の内側にfamily属性に対する条件を
記述した.正しい推定に対して修正ルールを適用してしまうことを防ぐため,
後述する$d_1$および$d_2$の値は,$d_1=2$,$d_2=8$として,ルールの
適用基準を厳しく設定した.ただし,family属性のモダリティ表現が出現
する文中の症状・診断名が患者の家族に対する言及でないというケースは
極めて少ないとの考えの下,familyに対する距離は$d_1=d_2=\infty$,
つまり,該当位置から到達可能な同一文内の最大トークン数とした.


\subsubsection*{モダリティ属性分類誤り修正のためのルール}

\begin{enumerate}
 \item 固有表現$e$の末尾[先頭]のトークンから右[左]方向に距離(トークン数)
       $d_1$以内に(非positive属性の)モダリティ表現$m$が存在する場合,
       $e$の推定ラベルのモダリティ属性を表現$m$が表す属性$a$に更新する.
       \begin{itemize}
	\item[*] 距離$d_1$以内に複数のモダリティ表現が存在する場合は,
	      family,negation,suspicionの順に優先する.
       \end{itemize}
 \item 固有表現$e$の末尾[先頭]のトークンから右[左]方向に距離$d_2$以内に
       モダリティ表現が存在しない場合,推定ラベルの属性をpositiveに更新する.
\end{enumerate}

なお,上記のルールを導入することは,
固有表現$e$から$e$の最も近くのモダリティ表現までの距離$d$を,
(a)近い ($d \leq d_1$),
(b)中程度の距離である ($d_1 < d \leq d_2$),
(c)遠い,あるいはモダリティ表現が存在しない ($d > d_2$)
の3通りに分け,距離$d$に応じてモダリティ属性決定の挙動を変えていることに対応する.
ルールが適用される(a)および(c)の条件下では,CRFで推定されたモダリティ属性を
ルールで上書きしているため,ルールによりモダリティ属性を決定しているのと等価である.
一方,固有表現から中程度の距離までの範囲内にモダリティ表現が存在する(b)の場合
には,CRFによる推定結果をそのまま採用し,モダリティ属性の決定を学習の結果に委ねた.


\section{医療用語資源の語彙拡張と症状・診断名抽出への利用}
\label{chp:util_resources}

用語抽出への語彙資源の利用は,
訓練データに少数回しか出現しない用語やまったく出現しない用語の認識に有
効であり,医療用語辞書などの語彙資源を用いた医療用語抽出研究が行われて
いる
\cite{imaichi2013comparison,laquerre2013necla,miura2013incorporating}.
しかし,診療記録では症状・診断名として多様な複合語が使用されるため,テ
キスト中の語句に対して辞書中の用語と単純にマッチングを行うだけではマッ
チしない複合語が存在し,辞書利用の効果は限定的となる.そこで,本章では,
語彙資源中の複合語から構成語彙を獲得する方法と,構成語彙を組み合わせて
より多くの複合語にマッチする拡張マッチングの方法を述べ,これらの方法に
基づく語彙資源を活用した症状・診断名の抽出手法を提案する.


\subsection{基本的な考え:医療用語の構成語彙への分解}
\label{sec:basic_idea}

症状・診断名は,複数の医療用語から構成される複合語であることが多い.た
とえば,「水痘肺炎」という用語は「水痘」と「肺炎」から構成され,「甲状
腺出血」は「甲状腺」と「出血」から構成される.診療記録では,多様な構成
語彙を組み合わせた複合語が用いられる一方で,実際に用いられる複合語の中
には辞書中には含まれないものも多い.例として,「水痘感染」と「甲状腺腫
大」は,診療記録コーパスMedNLPテストコレクション中で用いられ,医療用語
辞書MEDIS病名マスター中に存在しなかった症状・診断名である.ただし,上述
の用語の各構成語彙に注目すると,「水痘」,「感染」,「甲状腺」,「腫大」
を構成語彙として含む用語は同辞書中に多数存在した(表\ref{tab:medterm_constituent}).

\begin{table}[t]
 \hangcaption{複合語「水痘感染」,「甲状腺腫大」の構成語彙と,MEDIS病名マスターにおける各構成語彙からなる複合語数} 
 \label{tab:medterm_constituent}
\input{01table02.txt}
\end{table}

したがって,症状・診断名の抽出に,既存の辞書中に含まれる用語をそのまま
用いて対応するには限界があるといえる.また,症状・診断名としてありうる
構成語彙の組合せは膨大な数に上ると考えられ,それらを網羅的に含むような
辞書を構築することも現実的ではない.一方で,症状・診断名の構成語彙とな
る語の多くは,辞書中の用語の部分文字列として辞書に含まれている可能性が
高い.

そこで,本研究では,既存の医療用語資源から症状・診断名の構成語彙となる
語句を獲得し,得られた構成語彙を組み合わせた柔軟なマッチング方法「拡張
マッチング」に基づく症状・診断名の抽出手法を提案する.なお,マッチした
結果は,機械学習の素性(辞書素性)として用いる.語彙資源を活用した拡張
マッチングによって,辞書素性タグが付与される語句が増加し,元の辞書に含
まれない語彙にも対応した抽出が可能になると考えられる.


\subsection{主要語辞書と修飾語辞書に基づく拡張マッチング}
\label{sec:ext_match}

症状・診断名の構成語彙の獲得にあたり,構成語彙には,単独で症状・診断名
として用いられる「主要語」と,主要語と隣接して現れたときにのみ症状・診
断名の一部となる「修飾語」の2種類が存在すると仮定す
る.\ref{sec:basic_idea}節で述べた例では,「水痘」,「感染」,「腫大」
が主要語に相当し,「甲状腺」が修飾語となる.「水痘感染」や「甲状腺腫大」
のように,構成語彙に分解される前の元の用語自体も主要語とみなす.主要語
と修飾語の具体的な獲得方法は次節以降で後述することにし,本節では,主要
語辞書と修飾語辞書の2種類の辞書を用いた症状・診断名のマッチング方法「拡
張マッチング」を説明する.

以下,「消化管悪性腫瘍の$\cdots$」という入力文が与えられた場合を例に,
拡張マッチングの手順を述べる.主要語辞書には「消化管障害」,「腫瘍」,
「悪性腫瘍」が含まれ,修飾語辞書には「消化管」,「悪性」が含まれている
ものとする.


\subsubsection*{拡張マッチングの処理}

\begin{figure}[t]
\begin{center}
\includegraphics{22-2ia1f1.eps}
\end{center}
\caption{入力文「消化管悪性腫瘍の$\cdots$」に対する拡張マッチングの処理} \label{fig:twodic_matcing}
\end{figure}

入力文を文字の列とみなし,主要語辞書中にマッチする文字列がないか検索する処理を
先頭の文字から末尾の文字まで繰り返し行う.
例の入力文に対しては,次のようにマッチング範囲の探索が行われる.
\begin{enumerate}
\item 入力文1文字目の「消」を読み込み,主要語辞書の検索を行う(図 \ref{fig:twodic_matcing}, a1).「消化管悪」まで検索した時点で主要
  語が存在しないことが判明するため (a2),主要語検索を終了して2文字目以
  降について検索を続ける.
 \begin{itemize}
 \item[*] 2文字目「化」および3文字目「管」の検索においてマッチする主要語は
   ない.
 \end{itemize}
\item 4文字目の「悪」を読み込み,主要語辞書の検索を行う (b1).検索した
  結果,「悪性腫瘍」が主要語辞書とマッチする (b2).
\item 続いて,マッチした範囲「悪性腫瘍」の左右両側について,修飾語辞書
  中にマッチする文字列がないか検索する.検索した結果,元の範囲の左側の
  「消化管」が修飾語辞書とマッチする (b3) .
\item マッチした範囲を「消化管悪性腫瘍」に拡張し,引き続き,拡張された
  範囲のさらに左側について,修飾語辞書を検索する.しかし,これ以上マッ
  チする語はないため,「消化管悪性腫瘍」をマッチした範囲として記憶し,
  入力文5文字目以降について検索を続ける.
 \begin{itemize}
 \item[*] 5文字目「性」の検索においてマッチする主要語はない.
 \end{itemize}
\item 6文字目の「腫」を読み込み (c1) ,同様の処理を行った結果,主要語辞
  書で「腫瘍」(c2) ,修飾語辞書で「悪性」および「消化管」がマッチし
  (c3,c4),「消化管悪性腫瘍」をマッチした範囲として記憶する.入力
  文7文字目以降の文字についても同様に検索を続ける.
 \begin{itemize}
 \item[*] 7文字目以降の検索においてマッチする主要語はなかったものとし,入
   力文に対する検索の処理を終了する.
 \end{itemize}
\end{enumerate}
上述の辞書探索処理の結果,例では「消化管悪性腫瘍」がマッチ
した範囲として(二重に)得られる.複数のマッチング範囲が得られ,得られ
た範囲間に重なりがある場合は,範囲内の文字数が最も多いものを残し,残り
を破棄する.例では,得られた 2 つの範囲が同一であるため,残される範囲も
元の範囲と同じものとなる.

以上が拡張マッチングの処理である.
最終的に得られたマッチング範囲には辞書素性タグを付与し,機械学習の素性として利用する.


\subsection{主要語辞書の利用と語彙制限}
\label{sec:maindic_filtering}

本研究では,主要語辞書として,MEDIS病名マスター (Ver.~3.11)
と,MedNLPテストコレクション\cite{morita2013overview}の訓練データを利用
する.MEDIS病名マスター(ICD10対応標準病名マスター)は,一般財団法人医
療情報システム開発センターにより提供されている病名辞書である.病態毎に
選ばれた代表病名を表す「病名表記」に加え,病名表記の読み,ICD10コード
などから構成され,本研究で利用したVer.~3.11では,24,292語が収載されている.
同辞書に含まれる病名表記を抽出し,主要語辞書として用いる.
MedNLPテストコレクションは,NTCIR-10 MedNLPタスクで提供された
模擬患者の病歴要約からなるコーパスである.
同コーパスの訓練データからアノテーションされた症状・診断名部分を抽出し,
主要語辞書MedNE(延べ語数1,922,異なり語数1,068)として用いる.

ただし,主要語辞書をそのまま用いると,一部の語が用語抽出の学習に悪影響
を与える場合がある.たとえば,細菌性皮膚感染症の一種である「よう」(癰)
は,「〜するようになった」などの表現にマッチしてしまう.また,「喫煙歴」
という表現に含まれる「喫煙」は,必ずしも患者の喫煙を表す言及ではないた
め症状・診断名に該当しない.このように,症状・診断名と無関係の表現とマッ
チする用語や,高頻度で非症状・診断名としても使用される用語を辞書が含ん
でいる場合,学習を阻害する要因となる.そこで,MedNLP訓練データを使用し,
高い割合で症状・診断名とマッチする用語を取得する処理を行う.以下,各主
要語辞書に対する語彙制限の方法を述べる.なお,後述の閾値$n_{\rm
  out}$,$r_\mathrm{out}$,$n_\mathrm{in}$および$r_\mathrm{in}$は,\ref{chp:experiments}章で述べる実験により決定する.


\subsubsection*{MEDIS病名マスターの語彙制限}

辞書に含まれる各用語について,MedNLP訓練データ中で固有表現範囲(症状・
診断名としてアノテーションされた範囲)の内部および外部に出現した回数を
それぞれカウントし,次の 2 つの基準を満たす用語を主要語として許容する.
\begin{itemize}
\item 固有表現範囲の外部における出現回数が$n_\mathrm{out}$未満である.
\item 訓練データ中に1回以上出現している場合は,固有表現範囲の内部と外部の両方の出現に対し,
外部に出現した割合が$r_\mathrm{out}$未満である.
\end{itemize}


\subsubsection*{MedNEの語彙制限}

辞書に含まれる各用語について,MedNLP訓練データ中で固有表現範囲の内部お
よび外部に出現した回数をそれぞれカウントし,次の 2 つの基準を満たす用語
を主要語として許容する.
\begin{itemize}
\item 固有表現範囲の内部における出現回数が$n_\mathrm{in}$以上である.
\item 固有表現範囲の内部と外部の両方の出現に対し,内部に出現した割合が$r_\mathrm{in}$以上である.
\end{itemize}


\subsection{医療用語資源からの修飾語の獲得}
\label{sec:gen_mdfydic}

\ref{sec:basic_idea}節で述べたように,主要語に含まれる部分文字列の中に
は有用な修飾語が多く含まれるという考えに基づき,主要語から部分文字列を
切り出すことで修飾語を獲得する.修飾語の獲得には,主要語の取得と同様
にMEDIS病名マスターとMedNEを用いる.
なお,MEDIS病名マスターの関連リソースとして提供されている
「修飾語テーブル」は,本研究では使用しない.

修飾語は,主要語辞書でマッチした範囲を拡張する際にのみ用いるため,意味
をなさない非語が含まれていても,実際にテキスト中で主要語と隣接して現れ
なければ学習・抽出に影響を与えない.ただし,MEDIS病名マスターには動詞や
助詞,記号などを含む用語(「1型糖尿病・関節合併症あり」,「アヘン類使
用による急性精神・行動障害」など)が存在するため,切り出された部分文字
列が助詞などの表現を含んでいると,症状・診断名でない範囲にまで拡張して
しまう可能性がある.

そこで,部分文字列を切り出すことにより機械的に修飾語候補を抽出した後,
意味をなさないと考えられる候補と,過大な拡張を行う可能性のある有害な候
補を除去するという手順により修飾語の獲得を行う.なお,有害な候補の除去
には,
主にひらがなから構成される語40語程度の「ひらがな表現」リストを作成し,
使用した.同リストは,助詞(「が」や「より」),接続詞(「または」や
「かつ」),動詞・助動詞の組合せ(「ならない」)などひらがなのみから
構成される表現の他,一部,「著しい」など漢字を構成要素に持つ表現も含む.
「ひらがな表現」の全リストは付録\ref{sec:mdfy_restrict}に記載する.


\subsubsection*{修飾語獲得の手順}

修飾語取得の対象とする用語集合$T$から修飾語集合$S$を生成する手順を以下
に示す.切り出す部分文字列の最小の長さは2で固定し,
修飾語として許容する部分文字列の最小出現回数を$f_\mathrm{min}$とした.
$f_\mathrm{min}$の値は,主要語の語彙制限と同様に,
\ref{chp:experiments}章で述べる実験により決定する.
\begin{enumerate}
 \item 修飾語候補の抽出
       \begin{enumerate}
	\item 各用語$t \in T$について,$t$の先頭および末尾から,
	      長さ2から$t$の文字列長までの部分文字列を切り出す.
	      切り出した部分文字列は,修飾語候補集合$S_\mathrm{cand}$に追加する.
	      \begin{itemize}
	       \item[*] たとえば,$t=\mbox{``医療用語''}$の場合,
		     ``医療'',``医療用'',``医療用語'',``用語'',``療用語''が切り出される.
	      \end{itemize}
	\item 各修飾語候補$s \in S_\mathrm{cand}$について,$s$を部分文字列として含む$T$の用語全体の集合を求める.
	      これを,$s$の出現元集合 $\mathrm{Parent}(s)$ とする.
	      \begin{itemize}
	       \item[*] たとえば,$s=\text{``用語''}$の場合,
		     $\mathrm{Parent}(s)=\{\text{``用語''},\text{``医療用語''},\text{``専門用語''},$
		     $\text{``用語抽出''}\}$などが得られると考えられる.
	      \end{itemize}
       \end{enumerate}
 \item 修飾語候補の限定(有害な語の除去)
       \begin{enumerate}
	\item $s \in S_\mathrm{cand}$のうち,数字および「%」記号のみからなる語を除去対象候補集合$S_\mathrm{reject}$に追加する.
	\item $s \in S_\mathrm{cand}$のうち,先頭または末尾が「-」(ハイフン)以外の記号か空白である語を$S_\mathrm{reject}$に追加する.
	\item $s \in S_\mathrm{cand}$のうち,先頭または末尾の文字が1字からなる「ひらがな表現」に一致し,
	      一致した文字と隣接する文字がひらがな以外である
	      \footnote{「のう胞」などひらがなからなる医療用語を除去してしまうことを防ぐため,この条件を設定した.}
	      語を$S_\mathrm{reject}$に追加する.
	\item $s \in S_\mathrm{cand}$のうち,先頭または末尾から始まる範囲が2字以上からなる「ひらがな表現」の組合せに一致した
	      語を$S_\mathrm{reject}$に追加する.
       \end{enumerate}
 \item 修飾語候補の限定(有益でない語の除去)
       \begin{enumerate}
	\item $s \in S_\mathrm{cand}$のうち,$T$内での部分文字列としての出現回数が$f_\mathrm{min}$回未満
	      ($|\mathrm{Parent}(s)|<f_\mathrm{min}$)である$s$を$S_\mathrm{reject}$に追加する.
	\item 任意の$s_1,s_2 \in S_\mathrm{cand}$ ($s_1 \neq s_2$) について,出現元集合が等しい場合,
	      $s_1$と$s_2$のうちの長さが小さい方を$S_\mathrm{reject}$に追加する.
	      \begin{itemize}
	       \item[*] たとえば,$\mathrm{Parent}(\mbox{``群''})=\mathrm{Parent}(\mbox{``症候群''})=\{\mbox{症候群},\mbox{かぜ症候群}\}$
		     である場合,``群''が$S_\mathrm{reject}$に追加される.
	      \end{itemize}
       \end{enumerate}
 \item 修飾語候補の決定
       \begin{itemize}
	\item $S_\mathrm{cand}$から$S_\mathrm{reject}$の元を除いた集合$S=S_\mathrm{cand} \setminus S_\mathrm{reject}$が求める修飾語集合である.
       \end{itemize}
\end{enumerate}


\section{評価実験}
\label{chp:experiments}

本研究で開発した用語抽出システムの性能を評価するため,MedNLPテストコレ
クションを用いて評価を行った.次節以降で,基本素性(形態素,品詞,品詞
細分類,字種,基本形)の評価,モダリティ素性の評価,主要語語彙制限およ
び拡張マッチングの有効性の評価を行う.さらに,すべての素性に加えて後処
理ルールを適用した提案システム全体の性能を評価し,MedNLPタスクの参加シ
ステムである従来手法との比較を行う.


\subsection{実験設定}

実験に使用したMedNLPテストコレクションは,病歴要約50文書からなり,全デー
タの3分の2にあたる2,244文が訓練データ,残りの1,121文が評価用のテストデー
タとなっている.訓練データにアノテーションされた情報は,患者の個人情報
と診療情報である.本研究では,これらのうち,患者の症状と医師の診断を指
す診療情報(症状・診断名)を対象とする.各症状・診断名には医師の認識の
程度などを表すモダリティ属性が定義されており,それぞれ症状の罹患の肯定,
否定,可能性の存在を表すpositive,negation,suspicionに加え,症状が患者
の家族の病歴として記述されていることを表すfamilyの4種類からなる.各モダ
リティ属性が付与された症状・診断名を異なるカテゴリの固有表現とみなして
学習を行うことで,モダリティ属性の分類を含めた抽出を行う.

評価は,MedNLPタスクで行われたのと同様に,モダリティ属性を考慮する方法
と,考慮しない方法の2通りで行う.前者は,抽出された固有表現が正しい属性
に分類された場合にのみ正解とみなす評価であり,後者は,属性が誤っていて
も,抽出された固有表現の範囲がアノテーションされた正解情報と一致してい
れば正解とみなす評価である.それぞれの評価方法について,適合率,再現
率,$F$値を評価尺度として用いる.

CRFで学習を行う際に必要となる正則化のためのハイパーパラメータ$c$につい
ては,訓練データでの5分割交差検定により値を決定した.テストデータに対す
る抽出を行うまでの具体的な手順は以下の通りである.
\begin{itemize}
\item[1.] $c$の値を変化させながら,各$c$の値について訓練データで5分割交差検定を行い,
 5回の検定における各精度(適合率,再現率,$F$値)の平均をそれぞれ算出する.
\item[2.] $c$の値の中で,最も$F$値が高い結果となったものを最適値とする.
\item[3.] 得られた$c$の最適値を用いて,訓練データ全体で学習を行い,モデルを学習する.
続いて,学習で得られたモデルを用いてテストデータで抽出を行う.
\end{itemize}
なお,主要語の語彙制限および修飾語獲得の際の閾値については,次のように決定した.
MEDIS病名マスターの語彙制限では,
$n_\mathrm{out}\in\{3,5,8\}$,$r_\mathrm{out}\in\{0.5,0.7,0.9\}$の各組について,
ハイパーパラメータ$c$の値を変えながら,それぞれ訓練データで交差検定を行い,
最も$F$値が高かった$n_\mathrm{out}$,$r_\mathrm{out}$,$c$の値の組合せを
最適値としてテストデータでの評価に用いた.
MedNEの語彙制限については$n_\mathrm{in}\in\{3,5,8\}$,$r_\mathrm{in}\in\{0.5,0.7,0.9\}$とし,
MEDISおよびMedNEからの修飾語獲得については$f_\mathrm{min}\in\{1,2,3,5,8,10\}$として
,同様に最適値を決定した.
その結果,$n_\mathrm{out}=n_\mathrm{in}=3$,$r_\mathrm{out}=0.7$,$r_\mathrm{in}=0.5$,
$f_\mathrm{min}=3$ (MEDIS),$f_\mathrm{min}=3$ (MedNE) が最適値として得られた.

また,素性のウィンドウサイズについては,基本素性(形態素,品詞,品詞細分類,
字種,基本形)を素性セットとしたモデルを用いて次のように決定した.
サイズは各素性について共通の値を使用することにし,候補を3($-1$から1),
5($-2$から2),7($-3$から3),9($-4$から4)の4通りとした.続いて,
サイズおよびハイパーパラメータ$c$の値を変えながら,訓練データで交差検定を
行ったところ,サイズ5において最大の$F$値が得られ,この値を最適値とした.
次節以降,サイズ1で固定しているモダリティ素性を除き,すべての素性に関して
サイズ5として実験を行った結果を報告する.


\subsection{基本素性の有効性の評価}
\label{sec:exp_basic}

形態素,品詞,品詞細分類,字種,基本形からなる素性セットを基本素性とし,
これら5つの素性の有効性の評価を行った.各素性に対するテストデータでの精
度を表\ref{tab:basic_features}に示す.表中の``P'',``R'',``F''はそれぞ
れ適合率,再現率,$F$値を意味し,``2-way''はモダリティ属性を考慮しない
場合の精度,``Total''は考慮した場合の精度を表す.なお,表には,各素性セッ
トに対して30回行った試行の平均精度を記載した.次節以降の実験結果につい
ても同様である.直前の行中の素性セットを用いた場合との差について両
側$t$検定を行い,有意水準1\%で有意であった場合に``$^\star$''を付した.

\begin{table}[b]
 \caption{基本素性の評価}
 \label{tab:basic_features}
\input{01table03.txt}
\end{table}

品詞素性の導入により,2-way,Totalの両評価で,適合率が4ポイント前後向上
し,再現率が1.5〜2ポイント程度向上した.訓練データでは\footnote{テスト
  データについてはアノテーションされた情報が提供されていないため,訓練
  データでの分析を元にして実験結果への考察を行った.したがって,本節以
  降で述べる各素性における固有表現,非固有表現等の割合は,いずれも訓練
  データで算出した数値である.},名詞,接頭詞であるトークンのうちのそれ
ぞれ20\%弱,30\%弱が固有表現(を構成するトークン)であり,残りの品詞に
ついては,いずれも97\%以上が非固有表現であった.したがって,品詞が名詞
または接頭詞であることが固有表現である可能性を示す手がかりとなり,その
他の品詞であることが固有表現でないことを示すほとんど確実な手がかりとなっ
たと考えられる.

品詞細分類素性の導入では,両評価とも多少の適合率の低下が見られたものの,
それを上回る2.5ポイント前後の再現率の向上が得られた.名詞の細分類全体の
約30\%に相当する「数」では,その99\%以上が非固有表現であり,合わせて名
詞の約50\%に相当する「一般」,「サ変接続」,「接尾・一般」は,
各20〜40\%を固有表現が占めた.
したがって,名詞の中のどの細分類であるかという情報が精度向上に寄与したと考えられる.

字種素性の導入による主な変化として,2-way,Totalでの適合率がそれぞ
れ0.3ポイント程度向上した.固有表現が大きな割合を占めた字種は,「漢字」
と「カタカナ」で,それぞれ30\%強,20\%強の割合であった.ただ,接頭詞お
よび名詞/接尾・一般の100\%近い割合を「漢字」が占めるなど,品詞や品詞細
分類素性と重複している情報も多い.名詞/一般,名詞/サ変接続のそれぞ
れ85\%程度が「漢字」と「カタカナ」となっており,字種素性がこれらの品詞
細分類を詳細化する役割を果たしたと考えられる.

基本形素性の導入の結果,主にTotalでの適合率,再現率が向上した.
「ない」「なかった」など表層が異なるモダリティ表現を基本形で同一視することで,
モダリティ属性を考慮した評価の結果が向上したと考えられる.


\subsection{モダリティ素性の有効性の評価}
\label{sec:exp_mdlfeat}

モダリティ素性の有効性を評価するため,基本素性のみを用いたシステム
(Basic;表\ref{tab:basic_features}の最下行と同一のシステム)と,基本素
性に加えてモダリティ素性を用いたシステム ($\text{Basic} + \text{Modality}$) の精度の比
較を行った.テストデータにおける各モダリティ属性および全体の精度を
表\ref{tab:mod_features}に示す.なお,モダリティ属性とは,症状・診断名
に付加されているモダリティの種類(positive,negation等)を指し,モダ
リティ素性は,モダリティ属性を捉えるために用いた素性を指す.
Basicと$\text{Basic} + \text{Modality}$ との差について両側$t$検定を行い,有意水
準1\%で有意であった場合に``$^\star$''を付した.

\begin{table}[b]
 \caption{モダリティ素性の評価}
\label{tab:mod_features}
\input{01table04.txt}
\end{table}

各属性における適合率と再現率は,negationとsuspicionの適合率を除いて,モ
ダリティ素性導入後の方が同等か高いという結果が得られた.属性全体
 (``All (total)'') で見ても各精度は向上しており,本素性が正確なモダリティ
属性の認識に寄与したといえる.なお,本素性の導入前,後ともに,他の属性
に比べてsuspicion属性の精度が低いのは,データのアノテーション誤りに起因
すると考えられる.著者らが訓練データを確認したところ,「疑い」などの表
現が後続する固有表現で,suspicionという属性が付与されていない
(positiveとなっている)事例が十数件存在し,suspicionの事例全体の十数パー
セントという少なくない割合を占めていた.

一方,モダリティ属性を考慮しない2-wayの評価では,適合率が1ポイント弱低
下した.「〜がない」などの表現は固有表現でない語句の後方にも出現するた
め,非固有表現を誤って固有表現と認識してしまうケースが増加し,適合率低
下の原因となったと考えられる.


\subsection{主要語辞書および修飾語辞書利用の有効性の評価}

主要語語彙制限および拡張マッチングの有効性を評価するため,基本素性のみ
のシステム (Basic) と,基本素性に加えて各種辞書に基づく辞書素性を用いた
システムの比較を行った.結果を表\ref{tab:dic_features}にまとめる.最左
列は利用した辞書を示しており,``main($X$)''は辞書$X$を主要語辞書として
使用したことを,``$\lhd$ modify($X$)''は辞書$X$を修飾語辞書として拡張マッ
チングを行ったことを意味する.辞書$X$に対し
て,\ref{sec:maindic_filtering}節の方法による語彙制限を行っていないもの
を``$X$-r'' (raw),行ったものを``$X$-f'' (filtered) で表し,辞書$X$か
ら獲得した修飾語集合を``$X$-s'' (substring) で表した.また,``$X\cup
Y$''で 2 つの辞書$X$,$Y$を統合した辞書を表した.数値右上のシンボルは両
側$t$検定によって有意水準1\%で有意差があった場合に付してお
り,``$\dagger$''が(a)と(b)または(c)との比較,``$\ddagger$''が(b)と(d)と
の比較,``$\star$''が(c)と(e)との比較,``$\S$''が(e)と(f)または(g)との比
較の結果を示している.

\begin{table}[b] 
 \caption{辞書素性の評価}
\label{tab:dic_features}
\input{01table05.txt}
\end{table}

まず,語彙制限を行っていない辞書を単純に利用した場合,MEDIS病名マスター
(表中のMDM;MEDIS Disease Name Master)に基づく辞書素性を導入し
た(b)で,2-wayの評価において適合率・再現率が0.7〜1.5ポイント程度向上し,
その結果,$F$値が約1.1ポイント向上した.Totalにおいても$F$値が向上した
ものの,2-wayよりは効果が小さく,認識された症状・診断名の中にモダリティ
属性の分類に誤ったものが含まれることが示唆される.なお,2-wayとTotalの
いずれの場合も,$F$値の向上は統計的に有意であった.一方,MedNE(表中
のMNE)を追加した場合(c)では,再現率が大きく低下した.訓練データでの交
差検定では96\%を超える$F$値となっており,正解ラベルとほぼ一致する辞書素
性に基づく過学習が起こり,他の素性の情報が学習されなかった可能性が高
い.

上記のような問題を避けるため,次に語彙制限を行った主要語辞書を使用した.
結果を見ると,MEDISのみの(d)では全体的に$F$値がわずかに低下した.これ
は,MEDIS病名マスターから獲得した主要語については語彙制限の必要性が薄い
ことを示している.一方,MEDISに加えてMedNEにも語彙制限を加えた(e)では,
語彙制限を加えない(c)に比べて適合率,再現率,$F$値とも大幅に向上した.
モダリティ属性まで考慮したTotalでは(b)よりもわずかに精度が低いものの,
モダリティ属性を考慮しない2-wayの評価では,これまでの設定で最も良い結果
が得られた.

最後に,語彙制限を加えた主要語辞書(e)に加えてさらに拡張マッチングを適用
した.その結果,MEDISから獲得した修飾語辞書のみを使用し
た(f)で約0.5〜0.6ポイント,MedNEから獲得した修飾語辞書を併せて使用し
た(g)で約0.6〜0.7ポイントの$F$値の向上が見られ,修飾語辞書の導入・語彙
増加にともない,わずかながらも着実な認識精度の向上が得られた.


\subsection{システム全体の性能および従来手法との比較}

\begin{table}[b]
\caption{従来手法との比較}
\label{tab:compare_other}
\input{01table06.txt}
\end{table}

基本素性,辞書素性,モダリティ素性,後処理ルールを含めたシステム全体の
性能の評価を行った.テストデータでの精度を表\ref{tab:compare_other}
(a)に示す.基本素性に加えて語彙制限後の 2 つの辞書と修飾語辞書に基づく辞
書素性を導入したシステム(表\ref{tab:dic_features}の最下行(g)のシステム)
をModifyとし,モダリティ素性の追加,後処理ルールの適用をそれぞ
れModality,P-rulesで表した.なお,モダリティ素性と後処理ルールはモダリ
ティ属性まで含めた分類 (Total) において利用する.表中のシンボル
``$\dagger$''はModifyと $\text{Modify}+\text{P-rules}$,
``$\ddagger$''はModifyと $\text{Modify}+\text{Modality}$ を,``$\star$''は
$\text{Modify}+\text{Modality}$ と $\text{Modify}+\text{Modality}+\text{P-rules}$ を比較した際
,両側$t$検定で有意水準1\%の有意な差があったことを示している.

モダリティ素性の導入後には,適合率,再現率,$F$値がそれぞれ1ポイント前
後向上し,その効果が確認された.
また,後処理ルールについては,Modifyに適用した場合で$F$値で1ポイント
の向上,$\text{Modify}+\text{Modality}$ に適用した場合で0.7ポイント程度の向上が得られた.
すなわち,CRFによって症状・診断
名と認識された語句に対してのみモダリティ属性を修正する処理を行うこと
で,症状・診断名の認識精度を維持したまま,モダリティ属性分類誤りを減少でき
ている.

なお,後処理ルールの効果を確認するため,訓練データを使用して
\footnote{テストデータへのアノテーションの情報は提供されていないため,
\ref{sec:exp_basic}節と同様に,訓練データでの分析を元に考察を行った.
5分割交差検定の5回分のテスト結果に対して,後処理ルールを適用した結果を
合計した値を報告する.},
Modify および $\text{Modify}+\text{Modality}$の推定結果に後処理ルールを適用した場合の
正解数の変化について表\ref{tab:prules_effect}に示した.
\mbox{``対}象件\mbox{数''}はルール適用の対象となった固有表現の件数で,システムが正例と
推定した固有表現の数に相当する.また,``T$\rightarrow$F'',``F$\rightarrow$T''
および``F$\rightarrow$F''は,ルール適用前に正解 (T)/不正解 (F) で
あったものが,ルール適用後に正解/不正解に変化した件数を表す.``合計''は
これら 3 つの値の合計で,ルールが適用された件数に相当する
(``T$\rightarrow$T''となったもの,
つまり,モダリティ属性が変化しなかったものについては,件数に含めていない).
Modifyに適用した場合では,適用対象件数の5\%弱にあたる81件に対してルール
が適用された.適用された事例のうち,実際に固有表現であった事例に対する
正解率は80\%程度であった.$\text{Modify}+\text{Modality}$ に関しては,
モダリティ素性の導入により周辺文脈と矛盾するモダリティ属性の推定結果が減った
と考えられ,ルールが適用された件数自体が減少した.
しかし,ルールの適用により僅かながら正解数が増加している.

\begin{table}[b]
 \caption{後処理ルール適用の効果}
\label{tab:prules_effect}
\input{01table07.txt}
\end{table}

表\ref{tab:compare_other} (a) の下半分には,MedNLP症状と診断タスクの成
績上位3チームであるMiuraら\cite{miura2013incorporating},Laquerreら
\cite{laquerre2013necla}およびImaichiら\cite{imaichi2013comparison}のシ
ステムの精度を掲載した.従来手法の中では,Totalの適合率を除き,いずれ
もMiuraらのシステムの精度が最も高い.これに対し,Totalの評価で比較する
と,本システムは適合率を中心に従来手法よりも高く,モダリティ素性と後処
理ルールを併用したシステム ($\text{Modify} + \text{Modality} + \text{P-rules}$) では,適合率,
再現率,$F$値ともに最も高くなっている.また,提案システムと従来手法の中
で精度が高いMiuraらのシステムについて,モダリティ属性ごとの精度を
表\ref{tab:compare_other} (b)に掲載した.Miuraらのシステムで
はpositiveおよびnegationの再現率が高く,多数派クラスの網羅的な抽出の点
で優れている.しかし,適合率ではどの属性についても本システムの方が高
く,特に,少数派クラスであるsuspicion,familyの認識では,適合率,再現率
ともに大きく上回っている.したがって,モダリティ属性の分類を含めた抽出
では,従来手法と比べて,我々の手法が最も正確な用語抽出を実現できており,
事例数の少ないクラスに対して頑健な推定を実現できているといえる.

一方,表\ref{tab:compare_other} (a)における2-wayの評価では,特に,再現
率がMiuraらのシステムと比べて3ポイントほど低く,本システムは症状・診断
名抽出の網羅性の点で劣っている.修飾語の付加を行う拡張マッチングでは,
主要語が主要語辞書に含まれない場合には対応できないという限界があるため,
再現率を高めるには,コーパスから修飾語だけでなく新たな主要語も獲得する
方法が必要となる.また,診療記録中の症状・診断名と共起しやすい「〜出現」
「〜増悪」などの表現は,未知の主要語を認識する際の手がかりとして有効利
用できる可能性がある.


\section{おわりに}
\label{chp:conclusions}

本稿では,症例検索などの基礎技術として重要である症状名・診断名の抽出に
焦点を当て,語彙資源を有効活用した用語抽出について報告した.語彙資源活
用の1点目として,コーパス中の用語に対して語彙制限を行うことで,用語抽出
に真に有用な語彙の獲得を行った.2点目として,コーパスから複合語の構成語
彙である修飾語を獲得し,語彙制限後のコーパスの語彙に加えて獲得した修飾
語を活用することで,テキスト中のより多くの用語を検出する拡張マッチング
を行った.検出された用語の情報は,機械学習アルゴリズムlinear-chain CRF
をベースとしたシステムの素性として使用した.

NTCIR-10 MedNLPタスクのテストコレクションを用いて抽出実験を行ったところ,
単純な辞書の利用と比較して,$F$値で0.4〜1.1ポイントの有意な精度向上が見
られ,語彙制限および拡張マッチングの有効性を確認した.症状・診断名の認
識では,同タスクで1位のMiuraら\cite{miura2013incorporating}のシステムと
比較し,モダリティ属性の分類を含めた認識では本システムが適合率・再現率
ともに高い精度を実現した.一方,モダリティ属性を考慮しない場合の症状・
診断名認識において再現率が低く,網羅性の点で劣っていた.語彙制限および
修飾語の付加に基づく拡張マッチングでは,元の語彙資源に含まれない主要語
に対応できないという限界があるため,今後の課題として,新たな主要語の獲
得や,未知の主要語を認識する方法が必要である.

また,NTCIR-10 MedNLPに続くシェアドタスクであるNTCIR-11 MedNLP2 \cite{aramaki2014overview}
では,表記が異なりかつ同一の対象を指す症状・診断名を同一視する
「病名・症状正規化タスク」が設定されている.
症例検索などの応用に向けた基礎技術として,
幅広い症状・診断名を抽出することに加え,
抽出した症状・診断名の表記の違いを吸収する技術の開発も重要である.


\acknowledgment
NTCIR-10 MedNLPタスクを主催され,MedNLPテストコレクションをご提供くださいました
京都大学デザイン学ユニットの荒牧英治特定准教授ならびに関係者の皆様に感謝申し上げます.
MEDIS病名マスターをご提供くださいました
一般財団法人医療情報システム開発センターの関係者の皆様に感謝申し上げます.
また,本研究の一部は,JSPS科学研究費補助金25330363,
ならびに私立大学等経常費補助金特別補助「大学間連携等による共同研究」
の助成を受けたものです.


\bibliographystyle{jnlpbbl_1.5}
\begin{thebibliography}{}

\bibitem[\protect\BCAY{Aramaki, Morita, Kano, \BBA\ Ohkuma}{Aramaki
  et~al.}{2014}]{aramaki2014overview}
Aramaki, E., Morita, M., Kano, Y., \BBA\ Ohkuma, T. \BBOP 2014\BBCP.
\newblock \BBOQ Overview of the {NTCIR}-11 {MedNLP}-2 Task.\BBCQ\
\newblock In {\Bem Proceedings of the 11th NTCIR Workshop Meeting on Evaluation
  of Information Access Technologies}, \mbox{\BPGS\ 147--154}.

\bibitem[\protect\BCAY{Collins}{Collins}{2002}]{collins2002discriminative}
Collins, M. \BBOP 2002\BBCP.
\newblock \BBOQ Discriminative Training Methods for Hidden {Markov} Models:
  Theory and Experiments with Perceptron Algorithms.\BBCQ\
\newblock In {\Bem Proceedings of the 2002 Conference on Empirical Methods in
  Natural Language Processing \textup{(}EMNLP\textup{)}}, \mbox{\BPGS\ 1--8}.

\bibitem[\protect\BCAY{Higashiyama, Seki, \BBA\ Uehara}{Higashiyama
  et~al.}{2013a}]{higashiyama2013clinical}
Higashiyama, S., Seki, K., \BBA\ Uehara, K. \BBOP 2013a\BBCP.
\newblock \BBOQ Clinical Entity Recognition using Cost-sensitive Structured
  Perceptron for {NTCIR}-10 {MedNLP}.\BBCQ\
\newblock In {\Bem Proceedings of the 10th NTCIR Conference}, \mbox{\BPGS\
  704--709}.

\bibitem[\protect\BCAY{Higashiyama, Seki, \BBA\ Uehara}{Higashiyama
  et~al.}{2013b}]{higashiyama2013developing}
Higashiyama, S., Seki, K., \BBA\ Uehara, K. \BBOP 2013b\BBCP.
\newblock \BBOQ Developing {ML}-based Systems to Extract Medical Information
  from {Japanese} Medical History Summaries.\BBCQ\
\newblock In {\Bem Proceedings of the 1st Workshop on Natural Language
  Processing for Medical and Healthcare Fields}, \mbox{\BPGS\ 14--21}.

\bibitem[\protect\BCAY{Imaichi, Yanase, \BBA\ Niwa}{Imaichi
  et~al.}{2013}]{imaichi2013comparison}
Imaichi, O., Yanase, T., \BBA\ Niwa, Y. \BBOP 2013\BBCP.
\newblock \BBOQ A Comparison of Rule-based and Machine Learning Methods for
  Medical Information Extraction.\BBCQ\
\newblock In {\Bem Proceedings of the 1st Workshop on Natural Language
  Processing for Medical and Healthcare Fields}, \mbox{\BPGS\ 38--42}.

\bibitem[\protect\BCAY{井上\JBA 永井\JBA 中村\JBA 野村\JBA 大貝}{井上 \Jetal
  }{2001}]{inoue2001iryo}
井上大悟\JBA 永井秀利\JBA 中村貞吾\JBA 野村浩郷\JBA 大貝晴俊 \BBOP 2001\BBCP.
\newblock 医療論文抄録からのファクト情報抽出を目的とした言語分析.\
\newblock \Jem{情報処理学会研究報告. 自然言語処理研究会報告}, {\Bbf 141}  (17),
  \mbox{\BPGS\ 103--110}.

\bibitem[\protect\BCAY{Jiang, Chen, Liu, Rosenbloom, Mani, Denny, \BBA\
  Xu}{Jiang et~al.}{2011}]{jiang2011study}
Jiang, M., Chen, Y., Liu, M., Rosenbloom, S.~T., Mani, S., Denny, J.~C., \BBA\
  Xu, H. \BBOP 2011\BBCP.
\newblock \BBOQ A Study of Machine-learning-based Approaches to Extract
  Clinical Entities and Their Assertions from Discharge Summaries.\BBCQ\
\newblock {\Bem Journal of the American Medical Informatics Association
  \textup{(}JAMIA\textup{)}}, {\Bbf 18}  (5), \mbox{\BPGS\ 601--606}.

\bibitem[\protect\BCAY{木浪\JBA 池田\JBA 村田\JBA 高山\JBA 武田}{木浪 \Jetal
  }{2008}]{kinami2008kango}
木浪孝治\JBA 池田哲夫\JBA 村田嘉利\JBA 高山毅\JBA 武田利明 \BBOP 2008\BBCP.
\newblock 看護学分野の専門用語抽出方法の研究.\
\newblock \Jem{自然言語処理}, {\Bbf 15}  (3), \mbox{\BPGS\ 3--20}.

\bibitem[\protect\BCAY{Kudo, Yamamoto, \BBA\ Matsumoto}{Kudo
  et~al.}{2004}]{kudo2004applying}
Kudo, T., Yamamoto, K., \BBA\ Matsumoto, Y. \BBOP 2004\BBCP.
\newblock \BBOQ Applying Conditional Random Fields to {Japanese} Morphological
  Analysis.\BBCQ\
\newblock In {\Bem Proceedings of the 2004 Conference on Empirical Methods in
  Natural Language Processing \textup{(}EMNLP\textup{)}}, \mbox{\BPGS\
  230--237}.

\bibitem[\protect\BCAY{Lafferty, McCallum, \BBA\ Pereira}{Lafferty
  et~al.}{2001}]{lafferty2001conditional}
Lafferty, J., McCallum, A., \BBA\ Pereira, F. \BBOP 2001\BBCP.
\newblock \BBOQ Conditional Random Fields: Probabilistic Models for Segmenting
  and Labeling Sequence Data.\BBCQ\
\newblock In {\Bem Proceedings of the 18th International Conference on Machine
  Learning \textup{(}ICML\textup{)}}, \mbox{\BPGS\ 282--289}.

\bibitem[\protect\BCAY{Laquerre \BBA\ Malon}{Laquerre \BBA\
  Malon}{2013}]{laquerre2013necla}
Laquerre, P.~F.\BBACOMMA\ \BBA\ Malon, C. \BBOP 2013\BBCP.
\newblock \BBOQ {NECLA} at the Medical Natural Language Processing Pilot Task
  ({MedNLP}).\BBCQ\
\newblock In {\Bem Proceedings of the 10th NTCIR Conference}, \mbox{\BPGS\
  725--727}.

\bibitem[\protect\BCAY{McCallum \BBA\ Wei}{McCallum \BBA\
  Wei}{2003}]{mccallum2003early}
McCallum, A.\BBACOMMA\ \BBA\ Wei, L. \BBOP 2003\BBCP.
\newblock \BBOQ Early Results for Named Entity Recognition with Conditional
  Random Fields, Feature Induction and Web-enhanced Lexicons.\BBCQ\
\newblock In {\Bem Proceedings of the 7th Conference on Natural Language
  Learning \textup{(}CoNLL-2003\textup{)}}, \mbox{\BPGS\ 188--191}.

\bibitem[\protect\BCAY{Miura, Ohkuma, Masuichi, Shinohara, Aramaki, \BBA\
  Ohe}{Miura et~al.}{2013}]{miura2013incorporating}
Miura, Y., Ohkuma, T., Masuichi, H., Shinohara, E., Aramaki, E., \BBA\ Ohe, K.
  \BBOP 2013\BBCP.
\newblock \BBOQ Incorporating Knowledge Resources to Enhance Medical
  Information Extraction.\BBCQ\
\newblock In {\Bem Proceedings of the 1st Workshop on Natural Language
  Processing for Medical and Healthcare Fields}, \mbox{\BPGS\ 1--6}.

\bibitem[\protect\BCAY{Morita, Kano, Ohkuma, Miyabe, \BBA\ Aramaki}{Morita
  et~al.}{2013}]{morita2013overview}
Morita, M., Kano, Y., Ohkuma, T., Miyabe, M., \BBA\ Aramaki, E. \BBOP
  2013\BBCP.
\newblock \BBOQ Overview of the {NTCIR}-10 {MedNLP} Task.\BBCQ\
\newblock In {\Bem Proceedings of the 10th NTCIR Conference}, \mbox{\BPGS\
  696--701}.

\bibitem[\protect\BCAY{Sang \BBA\ Veenstra}{Sang \BBA\
  Veenstra}{1999}]{sang1999representing}
Sang, E.~F.\BBACOMMA\ \BBA\ Veenstra, J. \BBOP 1999\BBCP.
\newblock \BBOQ Representing Text Chunks.\BBCQ\
\newblock In {\Bem Proceedings of the 9th Conference on European Chapter of the
  Association for Computational Linguistics}, \mbox{\BPGS\ 173--179}.

\bibitem[\protect\BCAY{Sha \BBA\ Pereira}{Sha \BBA\
  Pereira}{2003}]{sha2003shallow}
Sha, F.\BBACOMMA\ \BBA\ Pereira, F. \BBOP 2003\BBCP.
\newblock \BBOQ Shallow Parsing with Conditional Random Fields.\BBCQ\
\newblock In {\Bem Proceedings of the 2003 Conference of the North American
  Chapter of the Association for Computational Linguistics on Human Language
  \textup{(}HLT-NAACL\textup{)}}, \mbox{\BPGS\ 213--220}.

\bibitem[\protect\BCAY{上杉}{上杉}{2007}]{uesugi2007n-gram}
上杉正人 \BBOP 2007\BBCP.
\newblock N-gramと相互情報量を用いた医療用語抽出のための分割点の探索.\
\newblock \Jem{医療情報学}, {\Bbf 27}  (5), \mbox{\BPGS\ 431--438}.

\bibitem[\protect\BCAY{Uzuner, South, Shen, \BBA\ DuVall}{Uzuner
  et~al.}{2011}]{uzuner20112010}
Uzuner, {\"O}., South, B.~R., Shen, S., \BBA\ DuVall, S.~L. \BBOP 2011\BBCP.
\newblock \BBOQ 2010 i2b2/{VA} Challenge on Concepts, Assertions, and Relations
  in Clinical Text.\BBCQ\
\newblock {\Bem Journal of the American Medical Informatics Association
  \textup{(}JAMIA\textup{)}}, {\Bbf 18}  (5), \mbox{\BPGS\ 552--556}.

\bibitem[\protect\BCAY{Uzuner, Goldstein, Luo, \BBA\ Kohane}{Uzuner
  et~al.}{2008}]{uzuner2008identifying}
Uzuner, {\"O}., Goldstein, I., Luo, Y., \BBA\ Kohane, I. \BBOP 2008\BBCP.
\newblock \BBOQ Identifying Patient Smoking Status from Medical Discharge
  Records.\BBCQ\
\newblock {\Bem Journal of the American Medical Informatics Association
  \textup{(}JAMIA\textup{)}}, {\Bbf 15}  (1), \mbox{\BPGS\ 14--24}.

\bibitem[\protect\BCAY{Uzuner, Luo, \BBA\ Szolovits}{Uzuner
  et~al.}{2007}]{uzuner2007evaluating}
Uzuner, {\"O}., Luo, Y., \BBA\ Szolovits, P. \BBOP 2007\BBCP.
\newblock \BBOQ Evaluating the State-of-the-art in Automatic
  De-identification.\BBCQ\
\newblock {\Bem Journal of the American Medical Informatics Association
  \textup{(}JAMIA\textup{)}}, {\Bbf 14}  (5), \mbox{\BPGS\ 550--563}.

\bibitem[\protect\BCAY{Uzuner, Solti, \BBA\ Cadag}{Uzuner
  et~al.}{2010}]{uzuner2010extracting}
Uzuner, {\"O}., Solti, I., \BBA\ Cadag, E. \BBOP 2010\BBCP.
\newblock \BBOQ Extracting Medication Information from Clinical Text.\BBCQ\
\newblock {\Bem Journal of the American Medical Informatics Association
  \textup{(}JAMIA\textup{)}}, {\Bbf 17}  (5), \mbox{\BPGS\ 514--518}.

\end{thebibliography}


\appendix

\section{モダリティ素性における正規表現}
\label{sec:modal_regex}

モダリティ素性において,各属性のモダリティ表現を捉えるために用いた
正規表現を記載する.各正規表現の記述形式は,プログラミング言語
Python\footnote{http://www.python.jp/}での正規表現に準拠している.
なお,``$X|Y$''は$X$または$Y$との一致,``[$X_1 X_2 \dots X_N$]''は
$X_1,X_2,\dots,X_N$のいずれかとの一致,``()''はパターンのグループ化,
``?''は直前のパターンの0回または1回の出現を表す.また,``(?!$\cdots$)''
は否定先読みアサーションと呼ばれ,``$X$(?!$\cdots$)''として用いられた
場合に,``$\cdots$''に相当する文字列が後続しない$X$にマッチする.

\subsubsection*{Negationのモダリティ表現}

Negation属性のモダリティ表現のために用いた正規表現を以下に示す.

\begin{itemize}
\item {\small [はがを]?(([認め$|$みとめ$|$見$|$み$|$得$|$え)(られ)?)?([無な](い$|$く$|$かっ)$|$せ?ず)}
\item {\small [はがを]?((消失$|$除外)(?!し?(な[いかくし])$|$せず))}
\item {\small [はがもの](改善$|$陰性$|$寛解)(?!し?(な[いかくし])$|$せず)}
\item {\small (の(所見$|$既往)が?)?[無な](し$|$い$|$く$|$かった)}
\end{itemize}

\subsubsection*{Suspicionのモダリティ表現}

Suspicion属性のモダリティ表現のために用いた正規表現を以下に示す.
\begin{itemize}
\item {\small の?(疑い$|$うたがい)}
\item {\small ((の$|$である$|$であった)可能性)?[はがをもと]?(考え$|$考慮$|$思われ$|$(疑$|$うたが)[わい]$|$含め$|$高い)}
\end{itemize}

\subsubsection*{Familyのモダリティ表現}

Family属性のモダリティ表現のために用いた正規表現を以下に示す.
なお,``[父母]''に後続する``(?![指趾])''というパターンは,
医療用語である「母指」や「母趾」とのマッチを防ぐために用いた.
\begin{itemize}
\item {\small [祖伯叔]?[父母](?![指趾])親?$|$お[じば]$|$[兄弟姉妹娘]$|$息子$|$従(兄弟$|$姉妹)}
\end{itemize}


\section{修飾語の限定に用いたひらがな表現}
\label{sec:mdfy_restrict}

語彙資源から修飾語を獲得する処理において,
修飾語の限定の際に用いた主にひらがなから構成される表現を以下に記載する.
\begin{description}
 \item[助詞・連語] が,の,を,に,へ,と,から,より,で,など,のみ,における
 \item[連体詞] その
 \item[接続詞] または,あるいは,および,かつ
 \item[助動詞] な
 \item[形容詞] ない,なし
 \item[動詞・名詞・助詞・助動詞の組合せからなる語] する,よる,ある,あり,して,した,ならない,なった,のための
 \item[漢字を含む表現] 伴う,伴わない,生じない,疑い,著しい,その他,手当て,比較して
\end{description}



\begin{biography}
\bioauthor{東山 翔平}{
2012年神戸大学工学部情報知能工学科卒業.
2014年神戸大学大学院システム情報学研究科博士前期課程修了.
現在,NEC情報・ナレッジ研究所に在籍.
自然言語処理,情報抽出の研究に従事.
}
\bioauthor{関  和広}{
2002年図書館情報大学情報メディア研究科修士課程修了.
2006年インディアナ大学図書館情報学研究科博士課程修了.Ph.D.
神戸大学助教等を経て現在甲南大学知能情報学部准教授.
情報検索,データマイニングの研究に従事.
情報処理学会,電子情報通信学会,人工知能学会各会員.
}
\bioauthor{上原 邦昭}{
1978年大阪大学基礎工学部情報工学科卒業.
1983年同大学院博士後期課程単位取得退学.工学博士.
同産業科学研究所助手,講師,神戸大学工学部情報知能工学科助教授等を経て,
現在同大学院システム情報学研究科教授.
人工知能,特に機械学習,マルチメディア処理の研究に従事.
電子情報通信学会,計量国語学会,日本ソフトウェア科学会,AAAI各会員.
}
\end{biography}

\biodate


\end{document}
