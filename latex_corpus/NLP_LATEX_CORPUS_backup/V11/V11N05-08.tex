\documentstyle[epsf,jnlpbbl+,graphicx,lingmacros,lflingmacros]{jnlp_j+}

 \usepackage{ifthen}


\setcounter{page}{151}
\setcounter{巻数}{11}
\setcounter{号数}{5}
\setcounter{年}{2004}
\setcounter{月}{10}
\受付{2004}{5}{20}
\再受付{}{}{}
\採録{2004}{7}{5}

\renewcommand{\cite}{}
\renewcommand{\citeA}{}
\renewcommand{\citeauthor}{}
\def\sec#1{}
\def\ssec#1{}
\def\sssec#1{}
\def\app#1{}
\def\eq#1{}
\def\fig#1{}
\def\tab#1{}

\renewcommand{\appendix}{}

\newcommand{\numex}[2]{}
\newcommand{\numexs}[2]{}
\newcommand{\refex}[1]{}
\newcommand{\refexs}[2]{}
\newcommand{\featbox}[1]{}

\newcommand{\uri}[1]{}
\newcommand{\badex}{}
\newcommand{\hatenaex}{}
\newcommand{\ra}{}
\newcommand{\da}{}
\newcommand{\lra}{}
\newcommand{\tra}{}
\newcommand{\rev}{}
\newcommand{\ssr}[1]{}
\newcommand{\cir}[1]{}

\setcounter{secnumdepth}{3}

\title{言い換え技術に関する研究動向}
\author{乾 健太郎\affiref{NAIST} \and 藤田 篤\affiref{NAIST}}
\jkeywords{言い換え,言い換え生成,言い換え認識,言い換え知識獲得,意
味の同一性}

\etitle{A Survey on Paraphrase Generation and Recognition}
\eauthor{Kentaro Inui\affiref{NAIST} \and Atsushi Fujita\affiref{NAIST}}
\ekeywords{paraphrasing, paraphrase generation, paraphrase recognition,
paraphrase acquisition, sameness of meanings}

\affilabel{NAIST}{奈良先端科学技術大学院大学情報科学研究科
}{
Graduate School of Information Science, Nara Institute of Science and
Technology}
\headauthor{\textbf{乾,藤田}}
\headtitle{\textbf{言い換え技術に関する研究動向}}


\jabstract{意味が近似的に等価な言語表現の異形を言い換えと言う.言い換
え技術とは,所与の言語表現からその言い換えを生成する言い換え生成技術,
および所与の言語表現対が言い換え関係にあるか否かを判定する言い換え認識
技術の総称である.これらの技術は,機械翻訳の前編集や読解支援のための文
章簡単化,質問応答や複数文書要約など,様々な応用に貢献する応用横断的な
ミドルウェア技術になると期待されており,近年研究者の関心を集めてきた.
本論文では,こうした言い換え技術について,工学的研究を中心に近年の動向を
紹介する.具体的には,言い換えの定義や言い換え技術の応用可能性について
論じた後,構造変換による言い換え生成,質問応答・複数文書要約のための言
い換え認識に関する研究を概観し,最後に言い換え知識の自動獲得に関する最
新の研究動向を紹介する.}

\eabstract{Paraphrases are alternative ways of conveying the same
content.  The language technology for processing paraphrases, namely,
paraphrase generation and paraphrase recognition, has drawn the
attention of an increasing number of researchers because of its
potential contribution to a wide variety of natural language
applications.  This survey paper overviews recent research trends in
paraphrase generation and recognition, and discusses
future prospects, addressing the issues of the definition of
paraphrases, transformation-based paraphrase generation, paraphrase
recognition in question answering and multi-document summarization,
and finally corpus-based knowledge acquisition.  }

\begin{document}
\maketitle

\section{はじめに}
\label{sec:introduction}

\numexs{hodonai}{
\item \emph{旧友}と飲む酒\emph{ほど}楽しいものは\emph{ない}.
\item \emph{昔の友達}と飲む酒が\emph{一番}楽しい.}
\numexs{kousan}{
\item 内戦状態に\emph{再突入する公算が大きい}.
\item \emph{再び}内戦状態に\emph{なる}\emph{可能性が高い}.}
この例のように,言語には同じ情報を伝える表現がいくつも用意されている.
意味が近似的に等価な言語表現の異形を言い換え(paraphrase)と言う.言い
換えを指す用語には他に,言い替え,換言,書き換え,パラフレーズといった
語も使われるが,統一のため本論文では一貫して「言い換え」という用語を使
う.

これまでの言語処理研究の中心的課題は,曖昧性の問題,すなわち同じ言語表
現が文脈によって異なる意味を持つ問題をどう解決するかにあった.これに対
し,言い換えの問題,すなわち同じ意味内容を伝達する言語表現がいくつも存
在するという問題も同様に重要である.

与えられた言語表現からさまざまな言い換えを自動生成することができれば,
たとえば,所与の文章を読み手の読解能力に合わせて平易な表現に変換したり,
音声合成の前編集として聴き取りやすい表現に変換したりすることができる.
あるいは,機械翻訳の前編集として翻訳しやすい表現に変換するといったこと
も可能になるだろう.また,与えられた2つの言語表現が言い換えであるかど
うかを自動判定することができれば,情報検索や質問応答,複数文書要約といっ
たタスクにおける重要な問題の一つが解決する.

近年,こうした問題に関心を持つ研究者が増え,言い換えというキーワードが
目立つようになってきた.本学会年次大会でも,2001年に言い換えのセッショ
ンが設置されて以来,4件(2001年),9件(2002年),10件(2003年),7件
(2004年)と投稿を集めた.また2001年,2003年には言い換えに関する国際ワー
クショップが開かれ,それぞれ8件,14件の発表,活発な議論が行なわれた
\cite{NLPRSWS:01,IWP:03}.

本論文では,言い換えに関する工学的研究を中心に,近年の動向を紹介する.
以下,まず,\sec{definition}で,言語学的研究および意味論研究における言
い換えに関連の深い話題を取り上げ,言い換えの定義について考察する.次に,
\sec{applications}で言い換え技術の応用可能性について論じた後,
\sec{models}で構造変換による言い換え生成,質問応答・複数文書要約のため
の言い換え認識に関する研究を概観する.最後に\sec{knowledge}で言い換え
知識の自動獲得に関する最新の研究動向を紹介する.

\section{言い換えとは?}
\label{sec:definition}

そもそも言い換えとはどのようなものか? どのような種類があるのか? 直感
的なイメージをつかむには,末尾の\app{taxonomy}に目を通されるのがよいか
もしれない.日本語の言い換え現象を構文的特徴に基づいて整理してある.

では,こうした言い換えはどのように定義されるか? 英語学習辞典COBUILDに
よると,「言い換え」=``paraphrase'' は次のように説明されている.
\begin{quote}
If you \emph{paraphrase} something written or spoken, or the person
who said it, you give its \textbf{meaning} using different words.
\end{quote}
明らかなのは,言い換えを定義するには,``meaning''が何を指すか,つまり
言葉の「意味」とは何かという問題に深く立ち入らなければならないというこ
とである.言語の意味論が多くの議論を要する,形式化が困難な問題であるこ
とは読者の良く知るところであろう.しかも,意味論研究の成果と自然言語処
理技術の現状には依然として深いギャップがある.ここでは言語学研究および
意味論研究における言い換えに関連の深い話題をいくつか紹介し,言い換えと
は何かを考える材料を提供する.

\subsection{言語学から見た言い換え}
\label{ssec:ling_pov}

\subsubsection{変形文法}
\label{sssec:trans}

理論言語学における言い換えの扱いは初期の変形文法まで遡る.変形文法にお
ける変形は,表層構造(統語構造)に対する統語的操作で,たとえば,能動文
から受動文に変形する操作は次のように記述される.

\smallskip
\begin{center}
\begin{tabular}{cccccc}\hline
能動文:  & \featbox{1}~:~NP & \featbox{2}~:~[$+$V, $+$AUX] &    & \featbox{3}~:~[$+$V, $-$AUX] & \featbox{4}~:~NP \\
受動文:  & \featbox{4}    & \featbox{2}                & BE\+ EN & \featbox{3}           & BY \featbox{1} \\
\hline
\end{tabular}
\end{center}
\smallskip
こうした操作は,表す意味を保存したまま統語構造を変えるという意味で構造
的な言い換えと見なすことができる.\cite{harris:81}には,文から文への変
形,名詞句から文へ変形といった具合に対象の粒度によって変形規則を約20種
類に分類したものが掲載されている(ただし,これらの変換規則が構文的言い
換えをどの程度カバーするかは明らかでない).こうした研究の成果は言い換
えの工学的実現にも有益であろう.

変形規則のもう一つの問題は,変形の語彙依存性をうまく扱えていないことで
ある.ある変形の適用が可能となる条件は単語に依存する場合が少なくない.
たとえば,上の受動化の例をとってみても,他動詞であれば常に受動化できる
わけではなく,「``resemble''のような動詞の場合は適用できない」といった
きめの細かい条件指定が必要になる.これは個々の語の特性に立ち入る必要が
あることを意味しているが,変形文法のような伝統的文法理論ではそれらを明
らかにする活発な研究は見られなかった.

\subsubsection{Meaning-Text Theory}
\label{sssec:MTT}

変形の語彙依存性の記述を試みた例として,\citeauthor{melcuk:96}らが発展
させたMeaning-Text Theory(以下,MTT)があげられる
\cite{melcuk:87,iordanskaja:91,wanner:94,melcuk:96,iordanskaja:96}.
MTTでは,意味構造から深層統語構造,表層統語構造を経て音韻構造にいたる
まで7層の表現レベルを用意し,次の2種類の規則で言い換えを説明する.
\begin{itemize}
\item \emph{変形規則:} 各レベル間の対応を変形規則として記述する.同
じ意味構造に対して異なる変形規則を適用すると異なる文が得られる.これら
の文は互いに言い換えの関係にある.
\item \emph{言い換え規則:} 同じレベルの表現どうしの間で起こる変形を
言い換え規則として記述する.言い換え規則を適用すると言い換えが得られる.
\end{itemize}
変形規則と言い換え規則の語彙依存性の記述には語彙関数(lexical function)
を用いる.語彙関数とは,語の共起関係を記述するための道具で,たとえばあ
る動詞$X$に対してその名詞形を返す$S_{0}(X)$と,ある動詞の名詞形$Y$に対
して元の動詞形の主語を同じく主語とするような機能動詞を返す
$Oper_{1}(Y)$という2つの語彙関数を組み合わせると,\refex{MTTrule}のよ
うな深層統語構造の言い換え規則によって,\refex{MTTex}のような言い換え
を記述することができる.

\begin{figure}[t]
\begin{center}
\leavevmode
\includegraphics*[scale=.4]{clip000.eps}
\caption{MTTにおける言い換え規則の例\cite{iordanskaja:96}}
\label{fig:MTTrule}
\end{center}
\end{figure}

\numex{MTTrule}{$X_{verb}$\quad{\lra}\quad $S_{0}(X) +
Oper_{1}(S_{0}(X))$ (\fig{MTTrule}のような構造を仮定)\\
($S_{0}(\emph{decrease}_{verb})=\emph{decrease}_{noun}$,
$Oper_{1}(\emph{decrease}_{noun}) = \emph{show}$)}

\numexs{MTTex}{
\item[s.] Employment \emph{decreased sharply} in October.
\item[t.] Employment \emph{showed a sharp decrease} in October.}

\citeA{melcuk:96}によると,言語の記述に必要な語の共起関係は60種類の言
語独立な語彙関数でカバーできるとしており,それらの関数を組み合わせて表
現する\fig{MTTrule}のような言い換え規則もまた言語に依存しない一般的な
規則で記述する.ただし,実際には個々の言語の語彙について語彙関数の大規
模な辞書が必要となるが,残念ながらそうした辞書はいまだ存在しない.

\subsubsection{言い換えの構成性}

変形文法やMTTにおける言い換えの扱いでもう一つ重要な原理は言い換えの構
成性である.たとえば,次の\refex{synonym+casealt}の言い換えは,
\emph{purchase} {\ra} \emph{buy}という語彙的な言い換えと,\emph{$X$ be
VERB-\textsc{pp} by $Y$} {\ra} \emph{$Y$ VERB $X$} のような一般的な態
交替の規則を組み合わせによって記述できる.
\numexs{synonym+casealt}{
\item[s.] This car \emph{was purchased by} him.
\item[t.] He \emph{bought} this car.}

明らかに,言い換えをよりプリミティブな変形に分解して記述するアプローチ
は,分解せずに記述するアプローチに比べると理論的にも工学的にも理にかなっ
ている.ただし,前者を採用することにしても,それ以上分解できないプリミ
ティブな言い換えにはどのような種類があるのか,それらは網羅的に数えあげ
られるのか,また,そもそも分解の可否を判断する基準を明確に規定できるか,
という問題が残る.たとえば,MTTの研究者らはこれらを究明すべき中心的な
問題の一部と考え,前述のように言い換え規則や語彙関数としてその成果を形
にしてきた.しかし,語彙関数の定義が厳密性を欠くなど,不十分な点も多く,
大規模な評価も試みられていないのが現状である.

\subsubsection{言い換えの言語横断性}

言い換えの言語横断的共通性の解明も注目すべき言語学的知見の一つである.

MTTでは,フランス語や英語など,複数の言語の言い換えを対比させることに
よって,言語に依存しない言い換え規則を規定しようとしてきた.
\citeA{melcuk:96}によると,すべての言語の言い換えは60種類の語彙関数を
組み合わせた言い換え規則で記述できるとされる.こうした成果が本当に正し
いかどうかは十分な経験的評価による証明を待たねばならないが,少なくとも
言い換えのモデル化に関して有益な知見を提供していることは間違いない.

経験的方法と組み合わせた大規模な調査も報告されている.たとえば,
\citeA{kageura:04:b}は,日本語と英語の各々で同じ言い換え現象,とくに複
合語を中心とした専門用語の言い換えを取り上げ,
\refex{compound-parallel}のような日英各々の言い換え規則集合
\cite{jacquemin:97,yoshikane:03}を用いて,両言語における言い換え可能性
の共通性を調査している\footnote{\refex{compound-parallel}の言い換え規
則において,$N_{i}$は名詞,$V_{i}$は動詞,$X_{i}$は任意の内容語,
$V(N_{i})$は名詞$N_{i}$の動詞形を表す.}.
\numexs{compound-parallel}{
\item[j.] 概念学習 {\ra} 概念を学習する
($X_{1}$ $N_{2}$ {\ra} $X_{1}$``を''$N_{2}$する)
\item[e.] word category {\ra} categorize words 
($N_{1}$ $N_{2}$ {\ra} $V(N_{2})$ $N_{1}$)}
彼らは,事例から演繹的に50〜60個の変換規則を作成し,これを用いて,ある
技術用語集の6割強の複合専門用語に対する両言語同時の言い換えを可能にし
ている.

ほかにも,\refex{cleft-parallel}のような分裂文の言い換え
\cite{sunagawa:95,dras:99:a}や\refex{lvc-parallel}のような機能動詞結合
の言い換え\cite{muraki:91,iordanskaja:91,dras:99:a,fujita:04:d}なども,
さまざまな言語に共通の言い換え現象である.
\numexs{cleft-parallel}{
\item[j.] \emph{収録されているのは}約千人の人物\emph{だ}.
\\{\ra} 約千人の人物\emph{が収録されている}.
\item[e.] \emph{It was his best suit that} John wore to the dance
last night.\\{\ra} John wore \emph{his best suit} to the dance last
night. }
\numexs{lvc-parallel}{
\item[j.] \emph{住民の強い要請を受け},廃棄物処理場の建設を中止し
た.\\{\lra} \emph{住民に(から)強く要請され},廃棄物処理場の建
設を中止した.
\item[e.] Employment \emph{showed a sharp decrease} in
October. \\{\lra} Employment \emph{decreased sharply} in October.}
言い換えの定式化・類型化にはこうした言語横断的視点からの分析が欠かせない.

\subsubsection{言い換えの機能論的説明}

人間は言い換える.それはなぜだろうか? この疑問に対する答えはいくつか
考えられる.

\citeA{walker:93}は,言い換えによって意思の疎通をはかっていると指摘し
ている.
また,\citeA{takatsuka:99}は,テクニカルコミュニケーションの立場から,
第二言語学習者にとって有用な語彙的言い換えに着目し,言い換えの機能とし
て次のような項目をあげている.
\begin{itemize}
\item 第二言語の学習者が自らの言語能力を補う.
\item 相手の理解を促進させるために自分の先行発話を言い換える.
\item 意味を確認するために相手の発話を言い換える.
\end{itemize}

上の例のようなコミュニケーションの促進という機能の他にも,言い換えは,
社会的関係を保持する道具として用いられる.
\citeA{kunihiro:00}は,会話の相手のメンツを傷つけることを避ける婉曲表
現,仲間との連体性を保つ集団語として言い換えが用いられると述べている.
自治体,企業のサービス窓口やアナウンサー向けのマニュアルにも社会的に不
適切な表現の使用を避けるような項目が設けられており,代替表現(言い換え)
が提示されていることもある.

\subsection{意味が同じであるとは?}
\label{ssec:equivalence}

\subsubsection{真理値意味論的意味の同一性}

変形文法やMTTを含む多くの現代文法理論では,意味構造(深層構造)が同一
であれば同義であり言い換えであると仮定しており,これが言い換えの定義に
なっている.ここで言う意味とは,語あるいは言語表現の内包的な意味
(intension)を指す.本論文では,内包的意味の同一性に基づく言い換えを,
以下で述べる参照的言い換えや語用論的言い換えと区別するために,\emph{語
彙・構文的言い換え}(lexical and structural paraphrase)と呼ぶ.

内包的意味が同一かどうかの判断は,真理値意味論(モデル論的意味論)を仮
定すると考えやすい場合が少なくない.真理値意味論では,意味を真理値への
写像と見なして定式化する.たとえば,「本」の意味は,個体$x$が本であれ
ば真,本でなければ偽を返す真理関数`本'$(x)$で与えられる.これによると,
二つの表現があるとき,それぞれの真理関数において任意の個体の写像先の真
理値がつねに同じであれば,またそのときに限り,それらの意味は同一である.
「書籍」の関数`書籍'$(x)$を真にする個体の集合,すなわち書籍という概念
の外延(extension)が「本」のそれと同じであれば,「書籍」と「本」は意
味が同じといえる.よく知られるように,こうした見方は,「太郎が本を読む」
のような命題を表す表現にも拡張できる.

ただし,こうした真理値意味論が言葉の意味を表現するのに十分でないことは
明らかであり,真理値意味論の問題に関する議論は枚挙にいとまがない.以下,
言い換えの定義に深く関連する問題について論じる.

\subsubsection{言外の意味}

変形文法では,前述のように能動文と受動文の対は同じ意味構造を持つと仮定
した.誰が何をどうしたかという命題部分の意味は同じと考えて良いだろう.
こうした仮定はたとえばより最近の主辞駆動句構造文法(Head-driven Phrase
Structure Grammar; HPSG)にも受け継がれている.しかし,能動文と受動文
は,話者の視点が違っていたり,どの情報を強調するか,あるいはどの情報が
新情報かといった情報構造の違いがあるので,文脈によっては置換できない.
現在の文法理論が仮定する意味表現は,こうした意味の違いを十分に扱えてい
ない.

こうした問題は,最も基本的なものに見える単語間の言い換えにも見られる.
たとえば,前述のように「書籍」と「本」は真理値意味論的意味,すなわち指
示的意味(denotation)はほぼ同じだが,厳密にはフォーマリティなどの暗示
的意味(connotation,いわゆるニュアンスあるいは言外の意味)の違いがあ
り,いつでも置き換えられるわけではない.言語は同義語を嫌う
\cite{clark:92}.同じ意味を持つ語があったとしても,語用論的な力が作用
して,次第に違う意味,とくに違うニュアンスを帯びるようになる.したがっ
て,実際には完全に意味が同じで常に置換可能な同義語はまれである.

言い換えと認められる表現対の多くになんらかの意味の違いがあるとすれば,
ただちに,言い換え対の意味の差にはどのような種類があり,どのように記述
すればよいかという問題が出てくる.これについては,すでに多くの言語学的
蓄積\cite{halliday:94,miyajima:95:a,miyajima:95:b,kageyama:01}があるも
のの,全貌はまだ遠く見えない.
さらに,こうした言い換え対が所与の文脈で置換可能であるためには,意味の
差がその文脈で無視できるものなくてはならない.したがって,言い換えの理
論化には,言い換え対の意味の差分が所与の文脈に照らして無視できるかどう
かを判別する機構の説明が必要である.

\subsubsection{参照の同一性}

語の意味に関する哲学的考察のなかで古くから論じられてきたように,言語表
現の内包的意味が同じであることは,その表現の参照対象が同一であることと
必ずしも一致しない.たとえば,Fregeの「宵の明星」「明けの明星」の例は
有名である.「宵の明星」と「明けの明星」は同じ参照対象を持つが,明らか
に内包的意味は異なる.両者が言い換え可能(置換可能)となる文脈を考える
のは容易でない.このように,参照対象が同一であることは言い換え可能であ
るための十分な条件にはならない.

ただし,内包的意味が違っていても,参照の同一性に基づいて言い換えること
ができる場合がある.典型的なのは,次の例のような参照表現の言い換えであ
る\cite{sato:99}.
\numexs{semantic_paraphrase1}{
\item[s.] 去年の出来事
\item[t.] 1998年の出来事}
\numexs{semantic_paraphrase2}{
\item[s.] 筆者の考え
\item[t.] 佐藤の考え}
こうした言い換えは特定の大域的文脈,談話の状況でのみ成り立つもので,内
包的意味の同一性に基づく言い換えとは区別するべきである.本論文では,こ
の種の言い換えを\emph{参照的言い換え}(referential paraphrase)と呼ぶ.

\subsubsection{語用論的効果の同一性}

先に意味の同一性を真理値意味論的に捉えた場合を議論したが,真理値意味論
の欠陥を踏まえて状況論的意味論や言語行為理論が登場した経緯から容易に推
測されるように,真理値意味論に基づいて言い換えを論じるのには限界がある.
その一つが,語用論的効果の同一性に基づく言い換えである.言葉の語用論的
効果とは,話者がそれを発することによって達成できると期待するコミュニケー
ションの目的である.次の例のように同じ語用論的効果を持つ発話は言い換え
可能である\cite{sato:99,kawamura:00}.
\numexs{pragmatic_paraphrase1}{
\item[s.] どなたかgccのソースのありかをご存知ないでしょうか.
\item[t.] gccのソースが置いてあるftpサイトを教えてください.}
\numexs{pragmatic_paraphrase2}{
\item[s.] Will you break this bill?
\item[t.] I want to use that vending machine. }
これらの言い換えは,仮に内包的意味が真理値意味論的に与えられたとしても,
同じではない.\citeA{sato:99}はこうした言い換えを語用論的言い換えと呼
んでいる.本論文でもこれに倣い,語用論的効果の同一性に基づく言い換えを
\emph{語用論的言い換え}(pragmatic paraphrase)と呼ぶ.

\bigskip

以上の議論をまとめると,言い換えには少なくとも,語彙・構文的言い換え,
参照的言い換え,語用論的言い換えの3種類がある.参照的言い換えは言葉が
発せられた文脈や談話の状況を参照する必要がある.また,語用論的言い換え
は明らかに,代表的な現代文法理論で仮定している意味論を超えるものである.
このうち工学的実現が最も容易に見えるのは語彙・構文的言い換えである.実
際,言い換えに関する工学的研究のほとんどが対象をこの語彙・構文的言い換
えに限定している.以下,本論文でも語彙・構文的言い換えに話題をしぼる.

\section{言い換え技術の使い方}
\label{sec:applications}

言い換え技術の用途は広い.言い換えの実現方法に話をすすめる前に,さまざ
まな言い換えが自動化できるようになるとどのような使い方ができるかを整理
してみよう.

\subsection{人間のために言い換える}

電子化文書データの爆発的な増加を背景に,そうした文書を利用者や利用形態
に適した形に自動編集する技術の必要性が説かれるようになって久しい.冒頭
の例のように,高齢者や子供,外国人,障害者など,利用者の言語能力にあわ
せて読みやすい平易な文面になおすタスク
\cite{carroll:98:a,canning:99:a,inui:01:a,higashinaka:02,inui:03:a}は
そのような編集の一例である.また,ニュース原稿から字幕を生成したり,
Webの文書を携帯端末に表示したり,ニュースを街頭や新幹線の電光掲示板に
表示したい場合は,1行当たりの字数を考慮してコンパクトな表現に言い換え
る技術が必要になる
\cite{robin:96,kondo:97:a,fukushima:99,mikami:99,ehara:00,kataoka:00,masuda:01:a,SatoDai:04:a,ikeda:04}.

言い換え技術は,人間が文書を書く現場でも有用である.読みやすい文書を書
く,スタイルを統一する,規定の語彙と構文を使って(制限言語文書を)書く,
といった作業を支援する推敲支援でも,読みにくい文や制限言語に合わない文
を自動的に適切な文に言い換える技術が必要とされている
\cite{hayashi:91,takahashi:91:a,takeishi:92,dras:99:a,mitamura:01}.

同様のことは,機械翻訳や要約など,機械が文章を生成する場合にもいえる.
機械が出力した文章をチェックし,適格でない表現があれば自動的に修正する
といった後編集\cite{knight:94:a,mani:99:a,nanba:00:a}が実現するとあり
がたい.また,人間が要約する場合は,原文にはない表現をうまく使って内容
をまとめることができるが,これなども言い換えの一種といえ,そのような言
い換えをいかに自動化するかが自動要約の重要な課題になっている
\cite{kondo:97:a,okumura:99:a,kataoka:00,okumura:02:a}.

\subsection{言語の機械処理のために言い換える}
\label{ssec:for_machine}

言い換えた結果を消費するのは人間ばかりとは限らない.言い換えは入出力が
同一言語であることから,さまざまな言語処理アプリケーションの中に部分タ
スクとして組み込むことができる.

機械翻訳では,前編集段階で機械処理に適した言語表現にあらかじめ書き換え
ておくと訳質が上がる\cite{shirai:95:a,kato:97:a,yoshimi:00:b}.この前
編集を自動化する試みがすでに多数報告されている
\cite{kim:94:a,shirai:95:a,chandrasekar:96:a,nyberg:00,yoshimi:00:a,yoshimi:00:b,imamura:01,YamamotoKazuhide:02:c}.
機械翻訳の他にも,手話への翻訳のための前編集(手話に変換しやすい表現に
言い換える)\cite{adachi:92:a,tokuda:98}や音声合成のための前編集(耳で
聴きとりやすい表現に言い換える)など,前編集としての言い換え技術の潜在
的応用範囲は広い.

言語には同じ内容を指す表現がいくつも用意されている.\cite{sato:01:a}の
例を引こう.
\numexs{お名前}{
\item お名前をお願いしたいのですが.\\{\ra} Could you tell me your
name, please?
\item お名前を頂戴することはできますか.\\{\ra} Could you tell me your
name, please?}
\refexs{お名前}{a}と\refexs{お名前}{b}はだいたい同じ意味の発話で,同じ
訳文を当てることができる.したがって,(a)を翻訳できる人なら,(b)も翻訳
できるだろう.しかし,翻訳システムが(a)と(b)の同義性を理解できず,それ
ぞれ別々に翻訳しようとすると,(a)は翻訳できるが,(b)は翻訳できないといっ
たことになりかねない.その場合でも,仮に前編集段階で(b)を(a)に言い換え
ることができれば,翻訳システムも対応できることになる.前編集が機械翻訳
に効果的と考えられるのは,言語が持つこのような表現の多様性を前編集段階
で吸収できることが期待されるためである.

言語表現の多様性が機械処理を難しくしている例は翻訳にとどまらない.文書
集合に対する情報検索や質問応答では,検索要求や質問に使われる言語表現と
それに該当する記述の言語表現が異なれば,単純なキーワード照合ではうまく
応答できない.これに対し,たとえば,
\numexs{verification}{
\item 《著作名》の著者は《人名》だ.
\item 《人名》が《著作名》を発表する.}
という2つの言い回しが広い意味での言い換えの関係になっていることを認識
できれば,「『坊ちゃん』の著者は誰ですか?」のような質問の答えを,情報
源となる文書中の「夏目漱石が『坊ちゃん』を発表した明治39年は,$\ldots$」
のような記述から探し出すことができる.検索質問拡張(term expansion)は,
その近似的な解決策の一つであるが,より洗練された同義性判定,すなわち言
い換えの認識の仕組みが必要であることは明かであり,すでにさまざまな試み
が報告されている
\cite{jacquemin:97,hikasa:99:a,anick:99:a,hirata:00:a,shiraki:00:a,tomuro:01:a,ravichandran:02,hermjakob:02,sasaki:02,duclaye:03,moldovan:03,takahashi:03:c,yoshikane:03}.
また,情報抽出において多様な言い換え表現から同じ情報を抽出する問題
\cite{sekine:01,shinyama:02,shinyama:03}や,複数文書要約において個々の
文書から抽出したパッセージの中に同じ情報を冗長に伝える記述がないかどう
かを判定する問題
\cite{mckeown:99,barzilay:99,ueda:00,narimatsu:02,barzilay:03:c}なども,
同様に言い換え認識の問題といえる.

\subsection{言い換えを研究の道具として使う}
\label{ssec:as_tools}

言い換え技術は言語処理研究の道具としても使える可能性がある.直接的な使
用例の一つに,機械翻訳システムの評価用正解翻訳例の自動生成がある.機械
翻訳の研究では,評価用の各原文に対して複数の正解翻訳例を用意し,システ
ムの出力を評価するのが一般的になってきた.たとえば,
BLEU~\cite{papineni:02:b}と呼ばれる自動評価手法では,豊富な正解例を用
意することが評価結果の信頼性を確保するのに必要なことが分かっている.し
かしながら,いくつもの正解翻訳例を人手で作るのはコストが高い.こうした
背景から,代表的な正解例からその他の翻訳例を言い換え生成によって自動的
に入手する試みがいくつか報告されている\cite{pang:03,kanayama:03}.

\clearpage

\section{言い換えの実現方法}
\label{sec:models}

\sec{applications}で述べたように,言い換えの工学的な処理は大きく生成と
認識の2種類に分けられる.

言い換え生成は,与えられた言語表現からその言い換えを生成する作業であり,
\sec{applications}で述べたように機械翻訳の前編集や後編集,文章読解の支
援など,さまざまな応用がある.言い換え生成は,言語表現を入力とする生成
という意味で,テキストからのテキスト生成 (text-to-text generation)の
一種と言うことができる.これに対し,自然言語生成の文脈では,同一の意味
構造からさまざまな言い換えを生成する作業を指して「言い換え生成」と呼ぶ
ことがある.両者は相互に深く関係しあう問題であり,独立な問題として考え
るべきではないが,混乱を避けるため,本論文ではもっぱらテキストからのテ
キスト生成の意味で言い換え生成という用語を使う.

言い換え生成は,意味を保存しながらある言語表現を別の意味表現に変換する
という意味で,同一言語内の翻訳と見なすことができる.逆に言い換えの方を
広く捉えて,翻訳は異なる言語間をまたぐ言い換えであると言ってもよい.本
節では,まず\ssec{MT}で言い換え生成の研究を歴史の長い機械翻訳研究と対
比させながら紹介し,\ssec{vsMT}で機械翻訳研究では顕在化しなかった新し
い問題を論じる.以下,「言い換え生成」と「(機械)翻訳」を区別するため,
それぞれを狭い意味で用いる.すなわち,「言い換え生成」は意味を保存した
まま同一言語内の別の表現に変換する作業,「(機械)翻訳」は意味を保存し
たまま他の言語の表現に変換する作業を指す.

一方,言い換え認識は,2つの異なる言語表現が言い換えかどうかを判別する
作業であり,情報検索や質問応答,複数文書要約などの応用がある.この問題
は,一方から言い換えを生成して他方に到達できるかを調べる問題と見なせる
ので,言い換え生成と裏表の関係にあるといえる.\ssec{recognition}ではこ
の点を論じる.

\subsection{同一言語内翻訳としての言い換え生成}
\label{ssec:MT}

\subsubsection{機械翻訳と言い換え生成}

言い換え生成の研究が機械翻訳研究の長い歴史から学べることは多い.たとえ
ば,統語構造変換と意味構造変換の長所短所,用例ベース翻訳や統計的翻訳な
どの経験的手法の有効性,対訳コーパスからの翻訳知識獲得の可能性といった
議論は,言い換えにもほとんどそのまま当てはまる.
実際,言い換え生成の実現方法に関するこれまでの提案や試みの多くは,こう
した既存の機械翻訳技術からのアナロジーに基づいている.むしろ,処理方式
などの方法論的議論に関するかぎり,言い換え生成の研究はこれまでのところ
機械翻訳技術の後追いの域をほとんど出ていないとさえいえる.

\begin{figure}[t]
\begin{center}
\leavevmode
\includegraphics*[scale=.4]{clip001.eps}
\caption{MTTに基づく翻訳と言い換えの統合モデル\cite{lavoie:00}}
\label{fig:lavoie}
\end{center}
\end{figure}

言い換えと翻訳の共通性をうまく切り取って形にした例として,
\citeA{lavoie:00}の翻訳と言い換えの統合モデル(\fig{lavoie})は象徴的
である.このモデルでは,MTT(\sssec{MTT}を参照)に基づいて,まず入力文
から深層の依存構造表現(深層統語構造表現; DSyntS)を生成する.この中間
表現を他言語の中間表現に構造変換(トランスファ)してから生成すると翻訳
になり,同言語内の別の中間表現に変換して生成すると言い換えになる.中間
表現には言語に依存しない共通の表現形式を用いるので,言い換えにせよ翻訳
にせよ同じエンジンを使って構造変換することができる.言い換えと翻訳の違
いは,構造変換に用いる変換知識(変換パターン)が違うだけである.

\subsubsection{構造変換(トランスファ)方式}
\label{sssec:transfer}

既存の言い換えの実現方法の多くは機械翻訳のトランスファ方式に対応する.
ただし,機械翻訳と異なって入力文の全体を変換する必要はないため,対象と
する言い換えの種類ごとに,その現象を捉えるのに都合の良い変換レベルを設
定していて実現可能性を調査するというスタイルの研究が多い.たとえば,言
い換えの対象が語や句のような比較的局所的な場合は,表層レベルでの局所的
な置換によって言い換えを生成できる可能性があるが,埋め込み文を主節から
切り離したり,主辞が交替するような言い換えを実現するためには,少なくと
も依存構造や句構造などの統語レベルのトランスファが必要になる.

\begin{description}
\item[表層レベル] 単語を同義語に言い換える場合や,慣用表現のような,要
素が省略されにくく,語の間に別の語が割り込まない表現を言い換える場合,
単純な文字列置換でも言い換えを生成できる.このレベルの言い換えには完全
に語彙化された表現対が用いられる.実現例としては,単語から単語への置き
換え\cite{edmonds:99,fujita:01,lapata:01:b,pearce:01},数語の単語列か
ら同じく数語の単語列への言い換え
\cite{barzilay:01,barzilay:02,pang:03,shimohata:03:c,quirk:04},慣用表
現の言い換え\cite{fujita:03:c}などがあげられる.
\item[統語レベル] 言い換え生成モデルの中には依存構造を前提としているも
のも多い.統語レベルのトランスファでは,\app{taxonomy}に示すようなさま
ざまな種類の言い換えが実現可能になる.たとえば,\citeA{kurohashi:99:b},
\citeA{kaji:01}のモデルでは,文節レベルの係り受け構造を用いている.そ
して,「AのB」{\lra}連体節\cite{kurohashi:99:b},内容語の言い換え
\cite{kaji:03:b},機能動詞結合の言い換え\cite{kaji:04:a}などを実現して
いる.一方,\citeA{takahashi:01:c}は,言い換えの対象が文節以上の単位で
あっても,その影響が文節よりも小さなレベルに及ぶことに着目し,形態素レ
ベルの依存構造を採用している.また,機能語列や係り受けの順序などを表現
するための記述言語を提供し,依存構造レベルのトランスファ規則の作成を支
援している.
\item[意味レベル] \sssec{MTT}で紹介したMTTの深層統語構造レベルでの言い
換えは,項の順序を区別して語彙関数を定義するなど,依存構造や句構造から
意味的なレベルに一歩踏み込んだ構造変換を仮定している.その他,動詞の格
役割,格要素間の関係を捉えるためにより意味に踏み込んだ例として,語彙概
念構造(Lexical Conceptual Structure; LCS)を用いた複合動詞の言い換え
\cite{takeuchi:02},機能動詞結合の言い換え\cite{fujita:04:d}があげられ
る.
\end{description}

\subsubsection{ピボット方式}
\label{sssec:pivot}

機械翻訳のピボット方式相当のアプローチも見られる
\cite{meteer:88,huang:96,brun:03}.この方式では,対象領域や捕らえるべ
き情報を限定して専用の意味構造を定義し,対象テキストをその構造に当ては
めることで言い換えを生成する.こうした研究では,情報抽出の技術を用いて
入力から意味構造の要素を抽出する技術や,意味構造から表層表現を生成する
技術が各々独立に論じられている.

たとえば,\citeA{brun:03}は,薬品データベースの分析に基づいて7種類の述
語--項構造を定義し,\refex{brun1}の例のようにこの領域専用の意味表現と
して用いている.

\numexs{brun1}{
\item Acetone is a manufactured chemical that is also found naturally
in the environment.
\item \texttt{SUBSTANCE(acetone).
PHYS\_FORM(acetone,chemical).\\
ORIGIN(acetone,natural,the environment,in).\\
ORIGIN(acetone,man-made,NONE,NONE).}}

機械翻訳の場合と同様,ピボット方式では中間意味表現の設計と管理が問題に
なるため,領域を十分に限定して,過度に複雑な意味の問題をうまく避ける必
要がある.

\subsubsection{言い換え知識の表現方法}
\label{sssec:representation}

言い換え知識の表現方法については,言い換え関係にある表現対を表層の単語
列または構文木の対として表現する場合が多い.\sssec{trans}で紹介した変
形文法の操作は統語レベルの言い換え知識とみなせるし,\sssec{MTT}で紹介
したMTTにおける言い換え規則も依存構造の対からなる.

文法を用いて言い換えを表現している例もある.\citeA{dras:99:a}は,木接
合による同期文法(Synchronous Tree Adjoining Grammar; STAG)
\cite{abeille:90,shieber:90}を用いて言い換え知識を表現している.同期文
法の枠組みでは,入力文の解析と同時に,解析に用いられる文法のそれぞれに
対応(同期)する文法が組み合わさり,解析終了と同時に出力が得られる.

\begin{figure}[t]
\begin{center}
\leavevmode
\includegraphics*[scale=.4]{clip002.eps}
\caption{複数単語列アラインメントによって生成される単語ラティス\cite{pang:03}}
\label{fig:lattice}
\end{center}
\end{figure}

その他,言い換えの関係にある複数の文を,複数単語列アラインメント
(Multi-Sequence Alignment; MSA)というアルゴリズムを用いて
\fig{lattice}のような1つの単語ラティスで表現する試みもある
\cite{barzilay:02,barzilay:03:a,pang:03}.これは,英語のように語順の制
約が比較的強い言語ならではのアプローチである.

\subsection{機械翻訳と何が違うのか?}
\label{ssec:vsMT}

\subsubsection{応用横断的技術としての言い換え}

翻訳が異言語間の同義表現であるのに対し,言い換えは同一言語内の同義表現
である.このことのおかげで,言い換え生成・認識技術は,単一言語を対象と
するさまざまな言語処理アプリケーションへの利用が期待できる.別の言い方
をすれば,言い換え技術は形態素・統語解析のような要素技術をさまざまな応
用につながる応用横断的なミドルウェアであると考えてもよい.

\ssec{for_machine}の例\refex{お名前}をもう一度考えよう.機械翻訳の前編
集で「お名前を頂戴することはできますか」を「お名前をお願いしたいのです
が」に言い換えることが有益なのは,翻訳システムが前者を正しく処理できず,
後者を正しく処理できる場合である.しかし,この議論には少し誤魔化しがあ
る.

この議論が成り立つためには,言い換えシステムが「お名前を頂戴することは
できますか」という文を正しく解析し,正しく言い換える能力を持っていなけ
ればならない.しかし,上で述べたように翻訳と言い換え生成は本質的には同
じ問題を扱う技術なので,もし「お名前を頂戴することはできますか」の言い
換えが技術的に可能なのであれば,その技術を翻訳システムに組み込んで,両
方の入力を正しく翻訳するシステムを作ることも原理的には可能なはずである.
逆に,翻訳システムにとって解析が困難な文は言い換えもやはり困難なはずで
あり,翻訳の問題の一部を前編集に移したとしても,問題の難しさは変わらな
い.それでも翻訳の前編集に言い換え技術を使う試みが合理的に見えるのは,
複雑な既存の翻訳システムの中身を触らずに済むといった短期的な利益のため
ばかりでなく,この技術が広く応用横断的でだからである.翻訳という一つの
応用技術から言い換えという応用横断技術を切り出す試みと言ってもよい.

このように,言い換え技術が応用横断的であることは,従来の機械翻訳研究で
は見過ごされてきた重要な特徴である.これまでの言い換えの研究では,特定
の応用/言語/言い換えを想定して知識が構築されてきた.しかし,今後は,
知識やシステムのポータビリティを考慮し,
\begin{itemize}
\item 言い換えのための知識をどのように整理し,分割し,記述しておけば応
用横断的な再利用性が高くなるかを検討し,
\item その成果にもとづいて実際に言い換えの処理や知識を実現し,
\item それらの部品を組み合わせて新しい用途に対応できる仕組みを作る
\end{itemize}
という努力を重ねる必要がある.

\subsubsection{問題解決型タスクとしての言い換え}
\label{sssec:problem_solving}

言い換え生成では,言い換えるべき対象を選択しなければならないという問題
もある.

翻訳では,原文のすべての構成要素を目的言語に変換するという暗黙の前提が
あった\cite{verbmobil:00,EBMT:03}.一方,言い換えは,「なんらかの目的
を満たす表現への変換」\cite{YamamotoKazuhide:01}であるため,「原文のど
の部分を言い換えるべきか」を目的に照らして判断し,その部分だけを選択的
に変換する必要がでてくる.

たとえば,原文を平易な表現に言い換えて文章の読解を支援するといった用途
の場合,原文のままでユーザが理解できる部分は言い換える必要がないし,む
しろ言い換えない方がよい\cite{dras:99:a,inui:01:a,inui:03:a}.
\sec{definition}で述べたように,言い換えは多くの場合原文の意味を厳密に
は保存できないため,不必要な言い換えは原文の情報を過度にねじまげてしま
う恐れがあるからである.\citeauthor{dras:99:a}は,原文を言い換えるたび
に,話し手が伝えたい情報や微妙なニュアンスなど原文のなんらかの情報がか
ならず損われるため,人間の書いたテキストを言い換える際は目的を満たす範
囲で言い換えの程度を最小限に抑えるべきだと指摘している.そして,そのよ
うな言い換えを「外部から与えられた制約を満たすために仕方なくやる言い換
え」という意味で``reluctant paraphrasing''と呼んでいる.

これをもう少し一般化すると,言い換え生成は,「なんらかのテキストの評価
基準が与えられたとき,原文から基準を満たさない言語表現を抽出し,満たす
表現に言い換える」という問題解決型のタスクと見なせる.評価基準は言い換
えの目的によって異なるだろう.読解支援の場合は「人間(外国人,子供,障
害者,特定のユーザなど)にとってのテキストの読みやすさ」が基準になる
\cite{carroll:98:a,canning:99:a,inui:01:a,higashinaka:02,inui:03:a}が,
機械翻訳の前編集では「解析・翻訳の容易性」
\cite{shirai:95:a,kato:97:a,yoshimi:00:b},音声合成(text-to-speech
systhesis)の前編集では「聴覚理解の容易性」ということになる.また,特
定の語彙と構文を基準として与えると制限言語への言い換え
\cite{mitamura:01}というタスクになり,書き言葉から話し言葉への変換
\cite{kaji:04:c}などの応用例がある.

以上から,言い換え生成技術は,次の2つの部分技術に集約できることがわか
る.
\begin{itemize}
\item \emph{言い換え候補生成} 与えられた言語表現に対して,言語的に適格
な種々の言い換えを網羅的に生成する技術
\item \emph{テキスト評価} 与えられた評価基準に基づいてテキストを評価す
る技術
\end{itemize}
言い換え候補生成が通常の意味での機械翻訳にほぼ相当するとすれば,テキス
ト評価は言い換えになって初めて顕在化される問題であるといえる.

\begin{figure}[t]
\begin{center}
\leavevmode
\includegraphics*[scale=.4]{clip003.eps}
\caption{言い換え候補生成とテキスト評価}
\label{fig:gen-and-eval}
\end{center}
\end{figure}

テキスト評価の必要性は,すでに何人もの研究者が指摘するところである.
\citeA{YamamotoKazuhide:01}は,「対象特定」「仮説生成」「仮説選択」,
\citeA{murata:01:c}は,「変換(transformation)」「評価(evaluation)」
と呼んでいる.また,\citeA{mitamura:01}も制限言語への言い換えを
「checking」「rewriting」に分けている.これらはいずれもほぼ同じ分け方
と見てよい.

両者の組み合わせ方にはいろいろ考えられる.最も単純には,次のような3 段
階のカスケード型のモデルが考えられる(\fig{gen-and-eval}).
\begin{itemize}
\item[1.] テキストを評価し,言い換えの対象を選択する
\item[2.] 選択された対象から可能な言い換えを網羅的に生成する
\item[3.] 生成された候補を評価し,最適解を出力する
\end{itemize}
もちろん,そのような単純な`generate and test'方式に計算量的な問題があ
る場合は,評価基準に対する原文の「違反の仕方」に応じて言い換えの種類を
絞り込むといった,いわば「言い換えプランニング」のような機構を検討して
もよい.\citeA{dras:99:a}が試みたように,制約下での最適化問題として定
式化する方向も考えられる.言語生成のセンテンスプランニングを制約充足問
題として定式化した\citeA{beale:98}のアプローチも参考になると思われる.
また,読解支援などの場合,評価基準にユーザの読解能力の個人差(ユーザモ
デル)を反映することができれば,ユーザに適応的な支援も可能になるだろう.
いずれにせよ,テキスト評価と言い換え候補生成を切り離した設計は,テキス
トの評価基準を取り替えることによってさまざまな用途の違いを吸収でき,よ
り汎用的な枠組みを提供することができる点で有利である.

\subsection{言い換えの認識}
\label{ssec:recognition}

言い換えの認識は,質問応答や情報検索の中心的な部分問題の一つである
\cite{shiraki:00:a,kurohashi:01,lin:01,ravichandran:02,sasaki:02,hermjakob:02,duclaye:03,moldovan:03,takahashi:03:c,takahashi:04:a}.
\citeA{takahashi:04:a}は,NTCIR QACトラック\cite{fukumoto:02}で用いら
れた質問文と,人手で作成した質問の合計約400 問とそれに対する解答文書の
関係を分析している.それによると,質問と解答を結び付ける変換操作の約
85\%が含意・前提条件などの推論を含む広義の言い換えと見なせる.
また,複数文書要約では,イベント間の関係を把握したり冗長な要約を避けた
りするために,異なる文書中で同じ内容を指す部分(類似部分)を同定する必
要がある.
\citeA{barzilay:99}は,Topic Detection and Trackingコーパス
\cite{allan:98}中の同じ内容を示す文の対200組を分析し,語彙・構文的言い
換えによって約85\%の文対を結び付けることができると述べている.
この文脈でも,言い換えの認識に関するいくつかの手法が提案されている
\cite{mckeown:99,barzilay:99,ueda:00,narimatsu:02,barzilay:03:c}.

言い換えの認識のアプローチは大きく2種類に分けられる.1つ目は,語彙・構
文的変換の到達可能性を調べるアプローチで,与えられた2つの言語表現のう
ち,一方を語彙・構文的に言い換えて,他方に到達できるか否かを判別する.
2つ目は,意味レベルの照合を明示的に扱うアプローチで,2つの言語表現の各々
をピボット的な意味表現に変換し,それらが一致するか否かを判別する.以下,
それぞれの代表的な研究を紹介する.

\subsubsection{語彙・構文的変換に基づく言い換えの認識}

語彙・構文的変換に基づく方法の例は\citeA{ueda:00}の複数文書要約アルゴ
リズムに見ることができる.
彼らの方法では,複数の入力文書を依存構造の部分木の集合として表現し,そ
の中から入力文書に共通に出現する部分木を取り出すことで複数文書要約を生
成する.ただし,文書によっては同じ情報が別の表現で言語化されている可能
性があるので,同義な(すなわち言い換えの関係にある)部分木どうしも共通
の部分木として扱う必要がある.彼らが扱った言い換えは,\refex{ueda1}の
ような類義語,上位語への言い換え,および\refex{ueda2},\refex{ueda3}の
ような構文的な交替などである.
\numexs{ueda1}{
\item[s1.] \emph{ホウレンソウ}からダイオキシンが検出された.
\item[s2.] \emph{白菜}からダイオキシンが検出された.
\item[t.] \emph{野菜}からダイオキシンが検出された.}
\numexs{ueda2}{
\item[s.] 軽量の携帯\emph{電話が}\emph{フーバー社によって}\emph{発売さ
れる}.
\item[t.] \emph{フーバー社が}軽量の携帯\emph{電話を}\emph{発売する}.}
\numexs{ueda3}{
\item[s.] 全角\emph{スペースが}シンタックス\emph{エラーを}\emph{起こす}.
\item[t.] 全角\emph{スペースで}シンタックス\emph{エラーが}\emph{起き
る}.}
彼らのアプローチは,\citeA{mckeown:99,barzilay:99}の手法と次の点で共通
する.
\begin{itemize}
\item 従来のbag-of-words的な手法に替えて部分構文構造を導入し,類似度を
より正確に見積もる\cite{mckeown:99},
\item WordNet~\cite{WN:90}のsynsetや動詞のクラス\cite{levin:93}を用い
て,部分構造間の同義性を判定する\cite{mckeown:99,barzilay:99},
\item 各種交替などの構文的な言い換えを規則として実装し,表現の多様性を
吸収する\cite{barzilay:99}.
\end{itemize}
\citeauthor{ueda:00}は,語彙・構文的変換によって生成された部分木にペナ
ルティを課すことで,変換によって生じる元文との情報のずれを考慮している.
そして,変換によるペナルティと文書間での共通性から各部分木の重要度を計
算し,上位数個の部分木を要約生成に用いている.
このアルゴリズムは,入力となる$n$文の各々について可能なすべての言い換
えを生成することで,複数の文書間で共通に出現し,かつ言い換え回数が少な
い表現を効率良く選択している.

こうした語彙・構文的変換に基づく方式では,入力表現の言い換え方が組み合
わせ的な数に膨らむ可能性があるので,変換の種類に制限を加える,あるいは
効率的な探索法を導入するといったなんらかの対策が必要になる.
たとえば,\citeA{takahashi:03:c}は,質問応答の文脈で,山登り法探索を実
現する枠組みを提案している.彼らのアルゴリズムでは,質問と解答候補文書
の両方に言い換えを適用し,最も類似度が高い$\langle$質問,解答候補
$\rangle$の組を優先して繰り返し言い換える.また,\citeauthor{ueda:00}
のように,語彙・構文的変換にペナルティを課す場合はそれによって探索空間
が抑えられる.

\subsubsection{意味表現に基づく言い換えの認識}

ある表現の含意や前提条件などの推論は,統語構造上よりも意味表現上で扱う
方が,知識記述や変形操作の実現という点で都合が良い.しかし,意味表現に
基づいて言い換えを認識するには,やはり意味表現の設計が問題になる.ここ
では,質問応答の文脈でのアプローチをいくつか紹介する.

\citeA{ravichandran:02}は,質問応答では多くの場合,質問に対して特有の
表現パターンが解答になるということに着目している.たとえば,
\refexs{ravi1}{a}と\refexs{ravi1}{b}は``Mozart''の誕生年を示す異なる表
現だが,\refexs{ravi2}{a},\refexs{ravi2}{b}のように固有表現を抽象化す
ると,どちらも「ある人物の誕生年」に関する\refex{ravi3}のような意味表
現に対する表現のパターンと見なすことができる.
\numexs{ravi1}{
\item Mozart was born in 1756.
\item Mozart (1756-1791) $\ldots$}
\numexs{ravi2}{
\item $\langle$\emph{name}$\rangle$ was born in
$\langle$\emph{birthdate}$\rangle$
\item $\langle$\emph{name}$\rangle$ (
$\langle$\emph{birthdate}$\rangle$ -}
\numex{ravi3}{\textsc{birthdate}~(\emph{name}, \emph{birthdate})}
彼らは,TREC 2001~\cite{voorhees:01}における質問の分析によって上の誕生
年に関する質問を含む6つの質問タイプを選択し,それぞれを表すような典型
的な表現パターンをWebから自動収集している.質問に対する解答としては,
単純に上のような表現パターン中の,質問対象の固有表現のスロットに対応す
る表現が出力される.
たとえば,``When was Mozart born?''という質問は,\refex{ravi3} の意味
表現に対応付けられ,同時に$\langle$\emph{birthdate}$\rangle$が解答を示
す固有表現スロットだと同定される.質問文中の固有表現``Mozart''を用いて
検索した文書中に,\refexs{ravi1}{a}のような,\refex{ravi2}中の表現パター
ンに対応する表現が見つかれば,その文書と質問は\refex{ravi3}という意味
表現において等価だと認識され,``1756''が解答として取り出される.

\citeA{sasaki:02,moldovan:03}は,質問と解答候補文書を論理形式(Logical
Form)を意味表現とすることで,言い換えの関係にある文間の統語レベルの違
いを捨象し,質問と解答候補文書の対応付けを可能にしている.
\refexs{sasaki1}{\textsc{lf}}は,\citeauthor{sasaki:02}の質問応答シス
テム,SAIQA-Isに\refexs{sasaki1}{q}という質問文を入力したときに得られ
る論理形式である.
\numexs{sasaki1}{
\item[q.] Where is the capical city of Japan?
\item[\textsc{lf}.] \texttt{COUNTRY(Y1:'Japan'), R(Y1,Y2),
'city'(Y2:'capital city'),\\LOCATION(Y3:Z); ORGANIZATION(Y3:Z),
R(Y2,Y3)}}
SAIQA-Isでは,さらに,解答が得られなかったときのみ``\emph{World
Cup}''{\ra}``\emph{W-Cup}''のような同義語レベルの言い換えを適用し,解
答を再度探索する.
一方,\citeauthor{moldovan:03}のシステム,Logic Proverは,
WordNet~\cite{WN:90}を用いて論理形式中の一部の語を同義語・上位語に置き
換えるだけでなく,論理形式レベルで推論に関する書き換え処理を施して,質
問と解答候補文書の照合を試みる.


\section{言い換え知識の獲得}
\label{sec:knowledge}

言い換えの生成や認識を実用規模で実現するには,言い換えに関する知識を既
存の資源から効率的に獲得する手段の開発が必須である.本節では,言い換え
知識の獲得に関するこれまでの試みを翻訳や情報抽出のための知識獲得技術に
照らして紹介する.

\subsection{既存の語彙資源を言い換えに利用する}
\label{ssec:existent}

\subsubsection{シソーラスを使って同概念語に言い換える}

言い換えに利用できる語彙資源と言うと,まず思い浮ぶのはシソーラスである.
たとえば,
WordNet~\cite{WN:90}\footnote{\uri{http://www.cogsci.princeton.edu/\~{}wn/}} 
やEDR日本語単語辞書
\cite{EDR:95}\footnote{\uri{http://www2.crl.go.jp/kk/e416/EDR/J\_index.html}}
には,非常に細かい意味分類に基づく単語間の同義関係が与えられているので,
それを用いれば,入力中の単語を同義語に置換する語彙的言い換えを実現でき
るように思える.しかし,実際には同義語といえども意味や用法になんらかの
差がある場合がほとんどで,無条件で置換できる語のペアは必ずしも多くない
\cite{edmonds:99,fujita:01,lapata:01:b,pearce:01,okamoto:03:b,inkpen:03:b} 
.
たとえば,「随所」と「各地」はEDR日本語単語辞書によると同概念に属する
(同概念語)が厳密には意味が異なる.このため,互いに言い換え可能かどう
かは,\refex{zuisho}のように,その差が周囲の文脈に照らして無視できるか
どうかに依存する.
\numexs{zuisho}{
\item \emph{随所}({\ra} \emph{各地})でがれきの山が生まれ,火災も発生
し,死傷者も多数,確認されている.
\item 片仮名交じりの文語体,しかも難解な言葉が\emph{随所}({\ra}
{\badex}\emph{各地})にあり,法学専攻の学生をすら悩ます現行刑法の法文
が現代用語に書き換えられる.}

\subsubsection{語釈文に言い換える}

国語辞典の語釈文は見出し語の言い換え表現と見なせるので,国語辞典から
$\langle$見出し語,語釈文$\rangle$の対を取り出せば,そのまま大規模な語
彙的言い換え知識として使える.たとえば,「廃材」は「いらなくなった木材
(岩波国語辞典)」という語釈文を持つので,\refex{haizai}のような言い換
えができる.
\numexs{haizai}{
\item[s.] がれきや\emph{廃材}の仮置き場
\item[t.] がれきや\emph{いらなくなった木材}の仮置き場}

しかし,いつも語釈文に置き換えるだけで正しい言い換えが作れるわけではな
い.たとえば,次の例\refexs{ainori}{s}の「相乗り(する)」を語釈文「乗
り物に一緒に乗る」にそのまま置き換えようとすると,
\refexs{ainori}{t1}のような不適格な文になってしまう.正しくは,原文中
の「タクシーに」と語釈文中の「乗り物に」の重複を検出して,
\refexs{ainori}{t2}のように「乗り物に」を削除する必要がある.
\numexs{ainori}{
\item[s.] タクシーに\emph{相乗りする}
\item[t1.] {\badex}タクシーに\emph{\underline{乗り物に}一緒に乗る}
\item[t2.] タクシーに\emph{一緒に乗る}}

語釈文への言い換えでおこる上の問題に注目した
\citeA{kurohashi:01,kaji:01}は,言い換え対象語の周囲の文脈と語釈文の要
素と重なり(上の例では「タクシー」と「乗り物」)を自動的に検出し,重複
をうまく取り除いた適格な言い換えを生成する手法を提案している.彼らのア
プローチは,(a)国語辞典という既存の語彙資源を使うため,カバレッジの広
い多様な語について語彙的言い換えを実現できる,(b)自然言語で書かれた語
釈文を知識源とするので,知識の拡張・保守が容易であるなどの利点があり,
大きな可能性を秘めている.

\subsubsection{語釈文から言い換えを見つける}

さらに,慣用表現など,内容語の特別な用法について,語釈文にヒントが隠さ
れている場合がある.たとえば,岩波国語辞典\cite{RWC:98}の「はこぶ」の
語釈文には,\refex{RWC:98:はこぶ}のように,慣用表現とそれに対応する表
現の対が記述されている.ただし,こうした記述は網羅的なものではないため,
この方法で十分なカバレージを確保することは難しい.また,運用の際には多
義性も考慮する必要がある.
\numex{RWC:98:はこぶ}{\emph{はこ‐ぶ【運ぶ】}〈1〉((五他))何かのた
めに,ものを他の所に進め移す.〈ア〉物を持ったり車に積んだりして,他の
場所まで動かす.「机を別の部屋に—」「恋人の所へせっせと金を—(= みつ
ぐ)」\emph{「筆を—」(文章を書き進める)}\emph{「足を—」(行く.通
う)} 「ようこそお—・び(=おいで)くださいました」}

\subsubsection{対訳辞書から言い換えを見つける}

対訳辞書を利用するという手も考えられる.たとえば,日本語語彙大系
\cite{NTT:97}の構文体系には\refex{NTT:97:surrender}のような記述があり,
そこから\refexs{NTT:97:combi}{r}のような言い換え知識を獲得することがで
きる.
\numexs{NTT:97:surrender}{
\item N1(名詞のクラス:主体)がN2(名詞のクラス:主体)の軍門に下る\\
{\lra} N1 surrender to N2
\item N1(名詞のクラス:主体)がN2(名詞のクラス:主体)に降伏する\\ {\lra}
N1 surrender to N2}
\numexs{NTT:97:combi}{
\item[r.] N1(名詞のクラス:主体)がN2(名詞のクラス:主体)の軍門に下る\\
{\ra} N1がN2に降伏する.
\item[s.] 英国を含む\emph{欧州がヒトラーの軍門に下る}のを黙って見てい
るわけにはいかない.
\item[t.] 英国を含む\emph{欧州がヒトラーに降伏する}のを黙って見ている
わけにはいかない.}

\subsubsection{意味の同一性を考えて言い換えを獲得する}
\label{sssec:distinction}

より厳密に同義表現を獲得するためには,言い換え前後の表現対の共通の意味
や意味の差を捉え,\ssec{equivalence}で述べたさまざまなレベルにおける同
一性の問題に踏み込む必要がある.
ここでは,最もプリミティブなレベル,すなわち類義語間の意味の差と捉えよ
うとする試みをいくつか紹介する.

\begin{figure}[t]
\begin{center}
\leavevmode
\includegraphics*[scale=.4]{clip006.eps}
\caption{語の指示的意味と言外の意味を示すクラスタモデル\cite[p.97]{edmonds:99}}
\label{fig:cluster}
\end{center}
\end{figure}

\citeA{edmonds:99}は,類義語(near-synonym)の意味を記述するオントロジ
を開発し,自然言語生成の語選択に用いている.\fig{cluster}では,類義語
``brunder''と``error''の意味が,それらを含む語のクラスの指示的意味,お
よび一部の言外の意味のリンクによって示されている.この図からは,この2
語が,
(i)非難の激しさの程度(criticism→severity),
(ii)誤りがばかげているか否か(stupidity),
(iii)具体性(concreteness),
(iv)軽蔑的か否か(pejorative),
という4つの点で異なっていることが分かる.
語の意味を記述する意味素の粒度について,\citeauthor{edmonds:99}は,複
数の言語を対象として上のような異なりを表現できているためある程度妥当で
あると評価している.

意味記述に替わる語彙知識のリソースとして国語辞典の語釈文を用いた研究が
いくつかある.
\citeA{tsuchiya:00}は,国語辞典中の各語の語釈文を統語解析器を用いてグ
ラフに変換し,MDS原理に基づいて,辞書全体にわたる部分グラフの抽象化お
よび辞書の圧縮を施している.結果として得られる辞書からは,任意の2語
$w_{1}$と$w_{2}$の共通の意味と個別の付加的意味を容易に取り出すことがで
きる.
\citeA{fujita:01}は,\citeauthor{tsuchiya:00}の手法を各々の$w_{1}$と
$w_{2}$の対に対して適用し,語釈文間の重なりの大きさに基づいて言い換え
の適格性を判定している.\refex{zuisho}の例では,「随所」の語釈が「限定
されないどの場所にも.方々.」,「各所」の語釈が「ある範囲内のところど
ころ.」(いずれも角川類語新辞典\cite{kadokawa:81})と完全に異なる.
\citeauthor{fujita:01}の手法では,語釈文の差分と文脈における制約を独立
にしか捉えていないため,文脈に関わらず常に(例\refexs{zuisho}{a}の言い
換えも)適格でないと判断されるという問題がある.
\citeA{okamoto:03:b}は,ある語が持つ指示的意味とその語が文脈中の他の語
句の選択に与える制約(語彙的制約)を語釈文から取り出している.具体的に
は,語釈文中のすべての内容語を,その意味クラスとコーパス中の共起頻度を
用いて指示的意味,語彙的制約に分類している.彼ら
の手法では,語釈文の比較などを要さずに単語そのものを表現できるため,
\citeA{edmonds:99}のモデルにおける知識獲得につながる可能性がある.
しかし,\citeA{fujita:01,okamoto:03:b}の実験結果を見るかぎり,語釈文の
情報を用いて同概念語間の可換性を判断するためには,かなり深い言語理解を
必要とするように見える場合も多く,越えるべきハードルは高い.

言外の意味(フォーマリティや親密度など)をいかにして獲得するか,という
課題もある.この課題に対する試みとして\citeA{inkpen:03:b}の研究を紹介
する.彼女はまず,\citeA{edmonds:99}のオントロジに基づいて,類義語の意
味を表現するための知識を,指示的意味(denotation),姿勢・態度
(attitude),スタイル(style)の3クラスの知識に分類・形式化している.
それぞれの知識は次のような組で表現される.
\medskip
\begin{description}
\item[指示的意味] $\langle$語,頻度(sometimes, usually, always),強
さ(low, medium, high),指示の間接性(\textsc{suggestion}, 
\textsc{implication}, \textsc{denotation}),周辺的概念(不定形)
$\rangle$
\item[姿勢・態度] $\langle$語,頻度,強さ,姿勢・態度の種類
(\textsc{favorable}, \textsc{neutral}, \textsc{pejorative})$\rangle$
\item[スタイル] $\langle$語,強さ,スタイルの種類(\textsc{formality}, 
\textsc{concreteness}, \textsc{floridity}など)$\rangle$
\end{description}
\medskip
\fig{cluster}で示されている言外の意味のうち,軽蔑的か否か(pejorative)
は姿勢・態度の,具体性(concreteness)はスタイルの下位クラスとして定義
されており,それぞれ種類の項の値となる.
\citeauthor{inkpen:03:b}は次に,これらの知識を類義語の使い分け辞典から
抽出する手法を提案している.この手法では,たとえば,\refexs{inkpen}{a}
の文章から\refexs{inkpen}{b}に示す3つの語彙知識を獲得できる.
\refexs{inkpen}{b}の1つ目は動詞``absorb''のスタイルに関する知識であり,
残りは指示的意味に関する知識である.指示的意味における周辺的概念だけは
不定形であり,\refexs{inkpen}{a}の文章中の句で表現される.
\numexs{inkpen}{
\item \textbf{Absorb} is slightly more informal than the others and
has, perhaps, the widest range of uses.  In its most restricted sense
it suggests the taking in or soaking up specifically of liquids: the
liquid \emph{absorbed} by the sponge.  In more general uses
\emph{absorb} may imply the thoroughness of the action: not merely to
read the chapter, but to \emph{absorb} its meaning.
\item \texttt{$\langle$absorb,low,\textsc{formality}$\rangle$\\
$\langle$absorb,usually,medium,\textsc{suggestion},the taking in of
liquids$\rangle$\\
$\langle$absorb,sometimes,medium,\textsc{implication},the thoroughness
of the action$\rangle$}}


\subsection{パラレルコーパスから言い換え知識を獲得する}
\label{sssec:extraction}

機械翻訳では,対訳コーパスから翻訳知識を自動獲得する試みが多数報告され
ており
\cite{meyers:98,watanabe:00,melamed:01,YamamotoKaoru:01:b,imamura:02},
大規模なパラレルコーパスまたはコンパラブルコーパスがあれば,そこから翻
訳知識を獲得できることがわかっている.一方,言い換えの場合,大量の言い
換え事例の入手は翻訳の場合ほど容易でない.日本語の新聞記事や書籍,ホー
ムページが英語や他の言語に翻訳されることはあっても,わざわざ「外国人日
本語学習者にもわかる日本語」や「朗読して聴きとりやすい日本語」に言い換
えられることはほとんどない.パラレル/コンパラブルコーパスを収集するた
めになんらかの工夫をするか,パラレルでないコーパスからの知識獲得を考え
る必要がでてくる.

\subsubsection{同じ原文に対する複数の翻訳文を集める}

同じ原文に対して複数の翻訳がある場合,それらは言い換えと見なすことがで
きる.機械翻訳では,システムの評価方法として,1つの原文に対して例
\refex{shirai2}のような複数の正解翻訳例を用意するのが一般的になってき
ており,そうした複数の翻訳例を含む対訳コーパスもいくつか整備されつつあ
る\cite{shirai:01:c,shimohata:04:b}.
\numexs{shirai2}{
\item[J0] 競技場は大勢の観客で\emph{膨れ上がった}.
\item[J1] 競技場は大勢の観客で\emph{身動きができなかった}.
\item[E0] The athletic field was swamped with spectators.}
\cite{shirai:01:c}では,既存の対訳コーパスに対して多様な別訳を作る作業
をどうやってうまく制御し効率化するかといった問題も検討されており,今後
も研究者間で共有できる資源が増えるものと思われる.

対訳コーパスからの翻訳知識獲得では,一対一の翻訳対を対象に句や節の対応
を計算するという問題が一般的であった.一方,言い換えの場合は,互いに言
い換え関係にある複数の表現を含む集合を用意することができるので,一般に
は3つ以上の要素間のアラインメントをとるという新しい問題が出てくる.もっ
とも単純なアプローチは,集合内の各要素対ごとにアラインメントをとる方法
である.たとえば,\citeA{imamura:01}は,集合内の各言い換え対について構
文木に基づく階層的アラインメントによって句や節レベルの言い換え対を獲得
する方法を提案している.これに対し,\citeA{pang:03}の方法では,構文木
に基づく複数単語列アラインメントによって3つ以上の言い換え間の対応関係
を1つの単語ラティスで表現する(\sssec{representation}を参照).

一方,既存の翻訳を集めてくるという手もある.\citeA{barzilay:01}は,
『海底二万里\footnote{Jules Verne (1869). \emph{Vingt mille lieues
sous les mers / Two Thousand Leagues Under the Sea}.}』のように同じ原
著から何冊もの訳本がでている作品があることに着目し,そうした訳本から言
い換え事例を大量に獲得しようと試みている.
彼女らによると,複数の翻訳本から得られるパラレルコーパスはノイズが多く,
また従来扱ってきた対訳コーパスに比べるときれいに言い換えの対応がとれる
箇所は必ずしも多くない.
さらに,獲得できた事例は\emph{King's son} {\lra} \emph{son of the
king}や\emph{countless} {\lra} \emph{lots of}のような局所的な語句の言
い換えが多く,\ssec{existent}で述べたような既存の資源から得られそうな
ものも少なくない.一方,\citeA{ohtake:03:b}が同様の実験を旅行対話に関
する対訳コーパスで行ったところ,領域に特化した\refex{ohtake}のような言
い換えが多数獲得できた.
\numexs{ohtake}{
\item[s.] それ以上は安くなりませんか.
\item[t.] それが最終的な値段ですか.}
機械翻訳の場合と同じように,パラレルコーパス→アラインメント→言い換え
知識の獲得,というシナリオが現実的に描けるか,難しいとすればどのような
工夫が必要かなど,興味深い問題が課題として残されている.

\subsubsection{同じ物事に対する複数の説明文を集める}

同じ事件を報道している複数の違った新聞社の記事をコンパラブルコーパスと
見なせば,そこから言い換え表現を発見できる可能性がある.厳密なパラレル
コーパスと違って,記事は日々生産されるので大規模なコーパスを入手できる
という利点がある.ただし,各記事がまったく同じ情報を同じ順序で過不足な
く伝えている保証はないため,文単位,句単位の順番で厳密にアラインメント
をとるという従来の翻訳知識獲得の方法を単純に適用するわけにはいかない.

\begin{figure}[t]
\begin{center}
\leavevmode
\includegraphics*[scale=.4]{clip004.eps}
\caption{コンパラブルな文の対からの言い換え対の抽出\cite{shinyama:03}}
\label{fig:extraction}
\end{center}
\end{figure}

この問題に対し,\citeA{shinyama:02,shinyama:03}の方法では,まず記事対
応をとった後,出現単語の類似度に基づいて文対応を同定する.次に,句単位
のアラインメントをとる代わりに,次の条件をより良く満たす依存構造の部分
木の対だけを言い換え対として獲得する(\fig{extraction}).
\begin{itemize}
\item[(a)] 各部分依存構造木の根は用言である.
\item[(b)] 対となる部分依存構造木が共通の固有表現を含んでいる.
\item[(c)] 各用言が要求する格が部分依存構造木に過不足なく含まれている.
\end{itemize}

一方,\citeA{barzilay:03:a}は,まずコーパスに含まれる各文を単語n-gram 
に基づく類似度でクラスタリングし,各クラスタ内に含まれる類似文から複数
単語列アラインメントによって単語ラティス(\fig{lattice})を生成する.
単語ラティスを見れば,クラスタ内の各類似文のどの箇所が共通でどの箇所が
文ごとに異なるかが分かるので,共通部分を定型表現,それ以外を変数とする
定型パターンが作れる.この方法を各新聞社の記事集合に適用し,
\citeA{shinyama:03}と同様の方法で記事集合間の対応をとれば,定型パター
ン間の言い換え関係を同定できる可能性がある.

句単位のアライメントがとれない場合に問題となるのは,与えられた文の対の
どの部分を言い換え対として抽出すれば良いかの判断が難しいことである.た
とえば,\fig{extraction}の例で言うと,右側の木に対応する言い換えは
``have died''を根とする左側の木の一部であって,``has announced''を根と
する木全体でない.しかし,そうだと判断するに足る手がかりは,この文の対
を見ているだけでは得られない.このとき,有用な手がかりとなるのは,言い
換え関係に立つ表現の「表現らしさ」である.\citeA{shinyama:03}は,上の
条件(c)を追加することによってこの「表現らしさ」を考慮しようとしている.
また,\citeA{barzilay:03:a}が複数単語列アラインメントによって定型パター
ンを事前に収集したのも,そのねらいは同じである.

\subsection{パラレルでないコーパスを使う}

\subsubsection{文脈の類似性を測る}

パラレルでないコーパス(ノンパラレルコーパス)から同義表現を獲得する場
合に基本となるのは,
\begin{quote}
与えられた入力表現と(a)似た文脈で出現する表現,あるいは(b)内部構造が似
ている表現がコーパス中に存在すれば,それは入力の言い換えである可能性が
高い
\end{quote}
という仮定である.とくに,(a)の出現文脈の類似性に基づいて推定される言
語表現の類似度は分布類似度(distributional similarity)と呼ばれ,単語
間の類似度の推定に効果的であることが知られている
\cite{pereira:93,lin:98:a}.言い換えの獲得は同義語の獲得を一般化した問
題と見なせるので,単語間の分布類似度の推定方法をうまく拡張すれば,より
多様な構造の言い換えを獲得できる可能性がある.

\begin{figure}[t]
\begin{center}
\leavevmode
\includegraphics*[scale=.4]{clip005.eps}
\caption{分布類似度に基づく部分依存構造木間の同義性の判定\cite{lin:01}}
\label{fig:dirt}
\end{center}
\end{figure}

代表的なのは\citeA{lin:01}の手法である.DIRTと呼ばれる彼らのアルゴリズ
ムは,\cite{lin:98:a}で提案した単語間分布類似度の推定方法を一般化した
もので,\fig{dirt}のような依存構造の部分木間の類似度を推定する.ここで
対象とする部分木は,両端を名詞の変数スロットとする枝分かれなしのパスで
ある.図の例のように,2つのパスの両端のスロット$X$,$Y$に現れる単語の
分布が互いに十分に似ていれば,それらのパスは言い換えと同定される.

また,\citeA{torisawa:02:a}の手法は,「アメリカの車」のような入力に対
して「アメリカで生産する車」のような言い換えをコーパスから獲得する.二
つの名詞(「アメリカ」と「車」)を出現文脈とし,与えられた文脈と確率的
に良く共起する表現を選択する点は同じで,違うのは,共起の強さを測る方法
と獲得の対象を動詞格構造(「で生産する」)に限定している点である.また,
\cite{torisawa:02:b}では,同様の方法が\cite{kondo:99}と同様の動詞格構
造間の言い換えにも適用できることが報告されている.

コンパラブルコーパスに比べると,ノンパラレルコーパスははるかに容易に入
手できるので,これと分布類似度の組み合わせは良い解決策であるように見え
る.ただし,分布類似度にも問題がある.まず,分布類似度を推定するには参
照する文脈のスコープを固定する必要があるため,予め固定したパターンの言
い換えしか獲得できないという制限がつきまとう.たとえば,\citeA{lin:01} 
の手法では,両端を名詞の変数スロットとする枝分かれなしのパスに対象が限
定されており,3つ以上の変数スロットを持つ部分構文木を同時に扱うことは
できない.分布類似度に基づく方法にはもう一つ,文脈の分布の偏りが大きい
表現の言い換えしか獲得できないという欠点もある.たとえば,「$X$が$Y$を
告訴する」の言い換えを同様の方法で獲得しようとしても,スロット$X$,$Y$ 
に出現する人間や組織の間でなされる行為は「告訴」だけではないので,これ
だけの情報で正しい言い換えを選別するのは困難である.

\subsubsection{内部構造の類似性を測る}

内部構造の類似に基づく方法には次のような例がある.
\citeA{kimura:01:a,tokunaga:03}の手法では,同義表現を探し出す手段とし
て,漢字インデックスによる情報検索を利用する.同じ漢字をより多く共有す
る2つ名詞句は意味が似ている可能性が高い.漢字をインデックスとすること
によって多様な表現の間の類似性が計算できるので,次のように単語の置換だ
けでは抽出できない言い換えも生成できる.
\numexs{kanji-index1}{
\item[s.] 収益の減少
\item[t.] 減収減益}
\numexs{kanji-index2}{
\item[s.] 倍額の増資
\item[t.] 出資額倍増}

また,\citeA{terada:01}は,略語をもとの単語に復元する\refex{terada}の
ような言い換えをとりあげ,言い換えの候補を文脈に応じて候補選択するモデ
ルを略語の多いコーパスと略語の少いコーパスから獲得する手法を提案してい
る.
\numexs{terada}{
\item[s.] TWR
\item[t.] tower / toward}
彼らの手法では,文字ベースの類似性と単語が出現する文脈の類似性の両方を
考慮して,略語と同義な単語をコーパス中から探す.入力と「似ている」表現
をコーパスから探し出すという点で,やはり上述のアプローチと同様の方向性
を持っている.

\citeA{jacquemin:97,yoshikane:03}は,文書検索の文脈で,索引語(ここで
はとくに複合専門用語,multi-word term; MWT)の表現の多様性を吸収する言
い換え規則を単言語コーパスから発見する手法を提案している.
\numexs{Jac}{
\item 構文的変形: technique for performing volumetric
measurements\\{\qquad\qquad\qquad}{\ra} measurement technique
\item 形態的変形: electrophoresed on a neutral polyacrylamide
gel {\ra} gel electrophoresis}
彼らの手法では,まず,専門用語辞書中の複合専門用語(以下,単に複合語)
を種にして,その言い換えをコーパスから抽出する.ここでは,所与の複合語
を構成する内容語がコーパス中で有意に共起しているパターンを,この複合語
の言い換えパターンとして取り出す.
次に,さまざまな複合語について得られた言い換えパターンを人手で類型化し,
大きく6種類(\citeauthor{yoshikane:03}は7種類)の言い換え規則集合を作
成している.
両研究とも,作成した言い換え規則集合によって生成される複合語の言い換え
がどれだけ正しいか,情報検索の精度向上にどれだけ寄与するかの2段階で評
価している.


\subsection{言い換えの適格性を判定するための知識}
\label{ssec:correctness}

言い換えた後の表現が言語的に適格か否かを判定する必要がある.
言い換え生成はテキストの一部に関する操作であるため,適格性の判定も,文
や文章全体の良さではなく,その操作を受けた部分と文脈がうまくあうかどう
かだけを評価すればよさそうに思える.
ただし,言い換えの言語的適格性に関わる要因には,形態素・構文レベルから
意味レベル,談話レベルまで性質が異なるさまざまなものがあり,言い換えの
種類によって共通性は見られるものの,その傾向は異なっている
\cite{fujita:03:c}.これらをまとめて捉えるようなモデルは現状では存在し
ないので,それぞれの適格性に関わる要因を個別に捉えて整理・モデル化し,
うまく融合させる必要がある.

自然言語生成の分野では,近年,出力テキストの候補を複数生成し,最後にラ
ンキングして候補を1つに絞る方式が有力になってきた
\cite{knight:95,langkilde:98,bangalore:00}.
これにならったランキング方式のモデルの一例として,単語の共起の是非を判
定する研究\cite{pearce:01,lapata:01:b,fujita:04:c}を取り上げる.
例\refexs{zuisho}{b}では,「言葉が(法文の)\emph{各地}にある」という
表現について,「法文の」と「各地」が共起しない(修飾関係にならない)こ
とがわかれば,適格ではないと判定できる.しかし,「随所」とは共起するが,
同概念語の「各地」とは共起しないといった粒度の細かい共起制約を必要とす
るということは,「意味クラスに基づく共起データの抽象化」という常套手段
が通用しないことを意味するので,問題は見た目ほど単純ではない.
この共起の是非を判定するためには,厳密には個々の語に関する詳細な知識や
共起に関する知識が必要になるが,個々の単語を区別する統計モデルを洗練す
るだけでも比較的良い成果をあげている.
\citeA{pearce:01,lapata:01:b}は,WordNet~\cite{WN:90}を用いて名詞を修
飾する形容詞を言い換えたときの単語の共起の是非を判定し,単語間の分布類
似度が人間の判断と相関を持っていることを示した.
一方,\citeA{fujita:04:c}は,態の交替や内容語の言い換えなど,さまざま
な言い換えの際に頻繁に不適格になる,動詞とその名詞格要素の共起を対象と
している.\citeauthor{fujita:04:c}は,大規模な生コーパスから得られる共
起用例の分布クラスタリング\cite{pereira:93}によって単語間の潜在的な類
似性を考慮するとともに,人手で収集した不適格な共起用例(負例)を組み合
わせて,共起の適格性を判定するモデルを構築している.

その他,言い換えの適格性判定の研究としては,談話構造や結束性を対象とし
た研究\cite{inui:01:c,nogami:02,siddharthan:03:a}がある.
さらに,言語的適格性を超えたレベルでも評価が必要になる場合がある.言語
生成の分野では,ユーザの知識に合わせて表現を変える\cite{cawsey:92}.語
用論的効果を考慮する\cite{hovy:88}.丁寧さや性別などを考慮する
\cite{kaneko:96,uchimoto:96}など,さまざまな試みが報告されている
\cite{inui:99:b}.


\section{おわりに}
\label{sec:issues}

本論文では,近年研究者間で関心が高まってきた言い換え技術について,最近
の研究動向を紹介した.

言い換え技術は言い換え生成と言い換え認識に大きく分けて考えることができ
る.言い換え生成は,ある言語表現を意味を保存しながら別の言語表現に変換
する作業であり,機械翻訳の前編集や読解支援のための文章簡単化など,さま
ざまな応用に利用できる.変換の方式,曖昧性解消,言語生成,知識の表現方
法と自動獲得など,個別の部分問題に関する限り,言い換え生成に必要な技術
は機械翻訳技術と重なるところが大きい.一方,言い換え認識は,2つの異な
る言語表現が言い換えかどうかを判別する作業であり,情報検索や質問応答,
複数文書要約などの応用がある.この問題は,一方から言い換えを生成して他
方に到達できるかを調べる問題と見なせるので,言い換え生成と裏表の関係に
ある.こうした技術は,これまで,それぞれの応用で別々に必要性が論じられ,
個別に研究されてきた.しかし,それらの間には必要な技術や蓄積すべき言語
知識に共通する部分も多い.今後はこれら個別の試みを統合し,応用横断的な
ミドルウェア技術に発展させていくことが重要である.

最大の技術的関心は知識獲得である.言い換えのパラレルコーパスは,翻訳の
場合よりさらに入手が難しいので,大規模なコンパラブルコーパスを収集する
ためになんらかの新しい工夫をするか,ノンパラレルコーパスから言い換え知
識を獲得する方法を考える必要がある.コンパラブルコーパスの収集について
は,同じ事件を報道した複数の新聞記事を集める方法や,同じ原著に対する複
数の翻訳を集める方法を紹介したが,ノイズの多いコンパラブルコーパス上で
高精度な知識獲得を実現するには克服すべき課題も多い.一方,ノンパラレル
コーパスからの知識獲得も,分布類似度に代表される従来の単語間類似度の推
定手法を拡張・一般化する方向で発展してきたが,実際の応用に耐える実用規
模の知識獲得に至った例はほとんどない.

目指すべき方向性の一つは,コーパスから獲得した言い換え知識と既知の言い
換え知識を組み合わせて知識の洗練をはかるアプローチであろう.これまでに
報告された知識獲得の研究は,同義語の知識以外は既知の言い換え知識を仮
定せず,スクラッチから言い換え知識を獲得しようとする試みがほとんどであ
る.しかし,\ssec{ling_pov}で見た言語学的研究が示すように,我々はかな
り多くの言い換えをすでに知っている.また,
\citeA{asaoka:04,fujita:04:d}の試みに見られるように,規則化がある程度
可能なタイプの言い換えも少なくない(\app{taxonomy}).安易にすべてを知
識獲得技術に頼るのではなく,まずはこうした人手による管理が可能な言い換
え知識を収集・蓄積し,コーパスからの知識獲得に積極的に活用する姿勢が必
要なように思われる.そのためにはまず,言い換えにはどのような種類がある
のか,どのタイプの言い換えの知識を人手で書くのが合理的なのか,自動獲得
に頼るべき言い換え知識はどのようなものかを明らかにしていく必要がある.

言い換え技術をさらに発展させるためには,多様な言い換え現象を包括的に調
査・分類し,それに基づいて言い換えコーパスや言い換え知識などの言語資源
を共有可能な形に設計・蓄積する努力が必要である.しかし,言い換えが言語
表現の同義性に立脚する概念である以上,これに厳密な定義や分類を与えるに
は,意味とは何か,意味が同じとはどういうことかといった深い意味論の問題
に立ち入らなければならない.本論文では,語彙・構文的言い換え,参照的言
い換え,語用論的言い換えを混同すべきでないことに言及したが,これは議論
の出発点の一つに過ぎない.こうした意味に深く根ざす問題にどうやって工学
的にアプローチするか.統計的言語処理技術がある程度成熟を見た今,再度じっ
くり議論してよいテーマである.言い換え技術が現在の言語処理技術をより深
い意味処理に一歩近づけるための良い例題になると期待したい.

\section*{謝辞}

本論文に対する,佐藤理史氏(京都大学),山本和英氏(長岡技術科学大学),
難波英嗣氏(広島市立大学),高橋哲朗氏(奈良先端科学技術大学院大学)の
有益なコメントに感謝する.

\bibliographystyle{jnlpbbl}
\newcommand{\noopsort}[1]{}
\begin{thebibliography}{}

\bibitem[\protect\BCAY{Abeille, Schabes, \BBA\ Joshi}{Abeille
  et~al.}{1990}]{abeille:90}
Abeille, A., Schabes, Y., \BBA\ Joshi, A.~K. \BBOP 1990\BBCP.
\newblock \BBOQ Using lexicalized {TAG}s for machine translation\BBCQ\
\newblock In {\Bem Proceedings of the 13th International Conference on
  Computational Linguistics {\rm (}COLING\/{\rm )}, Vol. 3}, \BPGS\ 1--6.

\bibitem[\protect\BCAY{安達}{安達}{1992}]{adachi:92:a}
安達久博 \BBOP 1992\BBCP.
\newblock \JBOQ 手話通訳のためのニュース文の話しコトバへの変換処理\JBCQ\
\newblock \Jem{電子情報通信学会技術研究報告, NLC92-47}.

\bibitem[\protect\BCAY{Allan, Carbonell, Doddington, Yamron, \BBA\ Yang}{Allan
  et~al.}{1998}]{allan:98}
Allan, J., Carbonell, J., Doddington, G., Yamron, J., \BBA\ Yang, Y. \BBOP
  1998\BBCP.
\newblock \BBOQ Topic detection and tracking pilot study: final report\BBCQ\
\newblock In {\Bem Proceedings of the Broadcast News Understanding and
  Transcription Workshop}, \BPGS\ 194--218.

\bibitem[\protect\BCAY{Anick \BBA\ Tipirneni}{Anick \BBA\
  Tipirneni}{1999}]{anick:99:a}
Anick, P.~G.\,\BBA\ Tipirneni, S. \BBOP 1999\BBCP.
\newblock \BBOQ The paraphrase search assistant: terminological feedback for
  iterative information seeking\BBCQ\
\newblock In {\Bem Proceedings of the 22nd Annual International ACM SIGIR
  Conference on Research and Development in Information Retrieval {\rm
  (}SIGIR\/{\rm )} Workshop on Customised Information Delivery}, \BPGS\
  153--159.

\bibitem[\protect\BCAY{麻岡, 佐藤, 宇津呂}{麻岡\Jetal }{2004}]{asaoka:04}
麻岡正洋,佐藤理史,宇津呂武仁 \BBOP 2004\BBCP.
\newblock \JBOQ 語構成を利用した言い換え表現の自動生成\JBCQ\
\newblock \Jem{言語処理学会第10回年次大会発表論文集}, \BPGS\ 488--491.

\bibitem[\protect\BCAY{Bangalore \BBA\ Ranbow}{Bangalore \BBA\
  Ranbow}{2000}]{bangalore:00}
Bangalore, S.\,\BBA\ Ranbow, O. \BBOP 2000\BBCP.
\newblock \BBOQ Corpus-based lexical choice in natural language
  generation\BBCQ\
\newblock In {\Bem Proceedings of the 38th Annual Meeting of the Association
  for Computational Linguistics {\rm (}ACL\/{\rm )}}, \BPGS\ 464--471.

\bibitem[\protect\BCAY{Barzilay, McKeown, \BBA\ Elhadad}{Barzilay
  et~al.}{1999}]{barzilay:99}
Barzilay, R., McKeown, K.~R., \BBA\ Elhadad, M. \BBOP 1999\BBCP.
\newblock \BBOQ Information fusion in the context of multi-document
  summarization\BBCQ\
\newblock In {\Bem Proceedings of the 37th Annual Meeting of the Association
  for Computational Linguistics {\rm (}ACL\/{\rm )}}, \BPGS\ 550--557.

\bibitem[\protect\BCAY{Barzilay \BBA\ McKeown}{Barzilay \BBA\
  McKeown}{2001}]{barzilay:01}
Barzilay, R.\,\BBA\ McKeown, K.~R. \BBOP 2001\BBCP.
\newblock \BBOQ Extracting paraphrases from a parallel corpus\BBCQ\
\newblock In {\Bem Proceedings of the 39th Annual Meeting of the Association
  for Computational Linguistics {\rm (}ACL\/{\rm )}}, \BPGS\ 50--57.

\bibitem[\protect\BCAY{Barzilay \BBA\ Lee}{Barzilay \BBA\
  Lee}{2002}]{barzilay:02}
Barzilay, R.\,\BBA\ Lee, L. \BBOP 2002\BBCP.
\newblock \BBOQ Bootstrapping lexical choice via multiple-sequence
  alignment\BBCQ\
\newblock In {\Bem Proceedings of the 2002 Conference on Empirical Methods in
  Natural Language Processing {\rm (}EMNLP\/{\rm )}}, \BPGS\ 164--171.

\bibitem[\protect\BCAY{Barzilay \BBA\ Lee}{Barzilay \BBA\
  Lee}{2003}]{barzilay:03:a}
Barzilay, R.\,\BBA\ Lee, L. \BBOP 2003\BBCP.
\newblock \BBOQ Learning to paraphrase: an unsupervised approach using
  multiple-sequence alignment\BBCQ\
\newblock In {\Bem Proceedings of the 2003 Human Language Technology Conference
  of the North American Chapter of the Association for Computational
  Linguistics {\rm (}HLT-NAACL\/{\rm )}}, \BPGS\ 16--23.

\bibitem[\protect\BCAY{Barzilay}{Barzilay}{2003}]{barzilay:03:c}
Barzilay, R. \BBOP 2003\BBCP.
\newblock {\Bem Information fusion for multidocument summarization:
  paraphrasing and generation}.
\newblock Ph.D.\ thesis, Columbia University.

\bibitem[\protect\BCAY{Beale, Nirenburg, Viegas, \BBA\ Wanner}{Beale
  et~al.}{1998}]{beale:98}
Beale, S., Nirenburg, S., Viegas, E., \BBA\ Wanner, L. \BBOP 1998\BBCP.
\newblock \BBOQ De-constraining text generation\BBCQ\
\newblock In {\Bem Proceedings of the 9th International Workshop on Natural
  Language Generation {\rm (}INLG\/{\rm )}}, \BPGS\ 48--57.

\bibitem[\protect\BCAY{Brun \BBA\ Hag\`{e}ge}{Brun \BBA\
  Hag\`{e}ge}{2003}]{brun:03}
Brun, C.\,\BBA\ Hag\`{e}ge, C. \BBOP 2003\BBCP.
\newblock \BBOQ Normalization and paraphrasing using symbolic methods\BBCQ\
\newblock In {\Bem Proceedings of the 2nd International Workshop on
  Paraphrasing: Paraphrase Acquisition and Applications {\rm (}IWP\/{\rm )}},
  \BPGS\ 41--48.

\bibitem[\protect\BCAY{Canning \BBA\ Tait}{Canning \BBA\
  Tait}{1999}]{canning:99:a}
Canning, Y.\,\BBA\ Tait, J. \BBOP 1999\BBCP.
\newblock \BBOQ Syntactic simplification of newspaper text for aphasic
  readers\BBCQ\
\newblock In {\Bem Proceedings of the 22nd Annual International ACM SIGIR
  Conference on Research and Development in Information Retrieval {\rm
  (}SIGIR\/{\rm )} Workshop on Customised Information Delivery}, \BPGS\ 6--11.

\bibitem[\protect\BCAY{Carl \BBA\ Way}{Carl \BBA\ Way}{2003}]{EBMT:03}
Carl, M.\,\BBA\ Way, A.\BEDS\ \BBOP 2003\BBCP.
\newblock {\Bem Recent advances in example-based machine translation}.
\newblock Kluwer Academic Publishers.

\bibitem[\protect\BCAY{Carroll, Minnen, Canning, Devlin, \BBA\ Tait}{Carroll
  et~al.}{1998}]{carroll:98:a}
Carroll, J., Minnen, G., Canning, Y., Devlin, S., \BBA\ Tait, J. \BBOP
  1998\BBCP.
\newblock \BBOQ Practical simplification of English newspaper text to assist
  aphasic readers\BBCQ\
\newblock In {\Bem Proceedings of the 15th National Conference on Artificial
  Intelligence and 10th Conference on Innovative Application s of Artificial
  Intelligence {\rm (}AAAI-IAAI\/{\rm )} Workshop on Integrating Artificial
  Intelligence and Assistive Technology}.

\bibitem[\protect\BCAY{Cawsey}{Cawsey}{1992}]{cawsey:92}
Cawsey, A. \BBOP 1992\BBCP.
\newblock {\Bem Explanation and interaction}.
\newblock The MIT Press.

\bibitem[\protect\BCAY{Chandrasekar, Doran, \BBA\ Srinivas}{Chandrasekar
  et~al.}{1996}]{chandrasekar:96:a}
Chandrasekar, R., Doran, C., \BBA\ Srinivas, B. \BBOP 1996\BBCP.
\newblock \BBOQ Motivations and methods for text simplification\BBCQ\
\newblock In {\Bem Proceedings of the 16th International Conference on
  Computational Linguistics {\rm (}COLING\/{\rm )}}, \BPGS\ 1041--1044.

\bibitem[\protect\BCAY{Clark}{Clark}{1992}]{clark:92}
Clark, E.~V. \BBOP 1992\BBCP.
\newblock \BBOQ Conventionality and contrast: pragmatic principles with lexical
  consequences\BBCQ\
\newblock In {\Bem Kittay and Lehrer {\rm (}Eds.{\rm )}, Frames, fields, and
  contrasts: New essays in semantic and lexical organization}, \BPGS\ 171--188.
  Lawrence Erlbaum Associates.

\bibitem[\protect\BCAY{Dras}{Dras}{1999}]{dras:99:a}
Dras, M. \BBOP 1999\BBCP.
\newblock {\Bem Tree adjoining grammar and the reluctant paraphrasing of text}.
\newblock Ph.D.\ thesis, Department of Computing, Macquarie University.

\bibitem[\protect\BCAY{Duclaye, Yvon, \BBA\ Collin}{Duclaye
  et~al.}{2003}]{duclaye:03}
Duclaye, F., Yvon, F., \BBA\ Collin, O. \BBOP 2003\BBCP.
\newblock \BBOQ Learning paraphrases to improve a question-answering
  system\BBCQ\
\newblock In {\Bem Proceedings of the 10th Conference of the European Chapter
  of the Association for Computational Linguistics {\rm (}EACL\/{\rm )}
  Workshop on Natural Language Processing for Question-Answering}, \BPGS\
  35--41.

\bibitem[\protect\BCAY{Edmonds}{Edmonds}{1999}]{edmonds:99}
Edmonds, P. \BBOP 1999\BBCP.
\newblock {\Bem Semantic representations of near-synonyms for automatic lexical
  choice}.
\newblock Ph.D.\ thesis, CSRI-399, Department of Computer Science, University
  of Toronto.

\bibitem[\protect\BCAY{江原, 福島, 和田, 白井}{江原\Jetal }{2000}]{ehara:00}
江原暉将,福島孝博,和田裕二,白井克彦 \BBOP 2000\BBCP.
\newblock \JBOQ 聴覚障害者向け字幕放送のためのニュース文自動短文分割\JBCQ\
\newblock \Jem{情報処理学会研究報告, NL-138-3}, \BPGS\ 17--22.

\bibitem[\protect\BCAY{藤田 乾}{藤田\JBA 乾}{2001}]{fujita:01}
藤田篤,乾健太郎 \BBOP 2001\BBCP.
\newblock \JBOQ 語釈文を利用した普通名詞の同概念語への言い換え\JBCQ\
\newblock \Jem{言語処理学会第7回年次大会発表論文集}, \BPGS\ 331--334.

\bibitem[\protect\BCAY{藤田 乾}{藤田\JBA 乾}{2003}]{fujita:03:c}
藤田篤,乾健太郎 \BBOP 2003\BBCP.
\newblock \JBOQ 語彙・構文的言い換えにおける変換誤りの分析\JBCQ\
\newblock \Jem{情報処理学会論文誌}, {\Bbf 44}  (11), \BPGS\ 2826--2838.

\bibitem[\protect\BCAY{藤田, 乾, 松本}{藤田\Jetal }{2004}]{fujita:04:c}
藤田篤,乾健太郎,松本裕治 \BBOP 2004\BBCP.
\newblock \JBOQ 自動生成された言い換え文における不適格な動詞格構造の検出\JBCQ\
\newblock \Jem{情報処理学会論文誌}, {\Bbf 45}  (4), \BPGS\ 1176--1187.

\bibitem[\protect\BCAY{Fujita, Furihata, Inui, Matsumoto, \BBA\
  Takeuchi}{Fujita et~al.}{2004}]{fujita:04:d}
Fujita, A., Furihata, K., Inui, K., Matsumoto, Y., \BBA\ Takeuchi, K. \BBOP
  2004\BBCP.
\newblock \BBOQ Paraphrasing of {J}apanese light-verb constructions based on
  lexical conceptual structure\BBCQ\
\newblock In {\Bem Proceedings of the 42th Annual Meeting of the Association
  for Computational Linguistics {\rm (}ACL\/{\rm )} Workshop on Multiword
  Expressions: Integrating Processing}, \BPGS\ 9--16.

\bibitem[\protect\BCAY{Fukumoto, Kato, \BBA\ Masui}{Fukumoto
  et~al.}{2002}]{fukumoto:02}
Fukumoto, J., Kato, T., \BBA\ Masui, F. \BBOP 2002\BBCP.
\newblock \BBOQ Question answering challenge ({QAC}): question answering
  evaluation at {NTCIR} workshop 3\BBCQ\
\newblock In {\Bem Working Notes of the 3rd NTCIR Workshop Meeting: QAC1},
  \BPGS\ 1--10.

\bibitem[\protect\BCAY{福島, 江原, 白井}{福島\Jetal }{1999}]{fukushima:99}
福島孝博,江原暉将,白井克彦 \BBOP 1999\BBCP.
\newblock \JBOQ 短文分割の自動要約への効果\JBCQ\
\newblock \Jem{自然言語処理}, {\Bbf 6}  (6), \BPGS\ 131--147.

\bibitem[\protect\BCAY{Halliday}{Halliday}{1994}]{halliday:94}
Halliday, M. A.~K. \BBOP 1994\BBCP.
\newblock {\Bem An introduction to functional grammar (second edition)}.
\newblock Edward Arnold.

\bibitem[\protect\BCAY{Harris}{Harris}{1981}]{harris:81}
Harris, Z. \BBOP 1981\BBCP.
\newblock \BBOQ Co-occurrence and transformation in linguistic structure\BBCQ\
\newblock In {\Bem Hiz {\rm (}Ed.{\rm )} Papers on Syntax}, \BPGS\ 143--210. D.
  Reidel Publishing Company.

\bibitem[\protect\BCAY{林 菊井}{林\JBA 菊井}{1991}]{hayashi:91}
林良彦,菊井玄一郎 \BBOP 1991\BBCP.
\newblock \JBOQ 日本文推敲支援システムにおける書換え支援機能の実現方式\JBCQ\
\newblock \Jem{情報処理学会論文誌}, {\Bbf 32}  (8), \BPGS\ 962--970.

\bibitem[\protect\BCAY{Hermjakob, Echibahi, \BBA\ Marcu}{Hermjakob
  et~al.}{2002}]{hermjakob:02}
Hermjakob, U., Echibahi, A., \BBA\ Marcu, D. \BBOP 2002\BBCP.
\newblock \BBOQ Natural language based reformulation resource and web
  exploitation for question answering\BBCQ\
\newblock In {\Bem Proceedings of the 11th Text Retrieval Conference {\rm
  (}TREC 2002\/{\rm )}}.

\bibitem[\protect\BCAY{日笠, 藤井, 黒橋}{日笠\Jetal }{1999}]{hikasa:99:a}
日笠亘,藤井綱貴,黒橋禎夫 \BBOP 1999\BBCP.
\newblock \JBOQ
  入力質問と知識表現の柔軟なマッチングによる対話的ヘルプシステムの構築\JBCQ\
\newblock \Jem{情報処理学会研究報告, NL-134-14}, \BPGS\ 101--108.

\bibitem[\protect\BCAY{Higashinaka \BBA\ Nagao}{Higashinaka \BBA\
  Nagao}{2002}]{higashinaka:02}
Higashinaka, R.\,\BBA\ Nagao, K. \BBOP 2002\BBCP.
\newblock \BBOQ Interactive paraphrasing based on linguistic annotation\BBCQ\
\newblock In {\Bem Proceedings of the 19th International Conference on
  Computational Linguistics {\rm (}COLING\/{\rm )}}, \BPGS\ 1218--1222.

\bibitem[\protect\BCAY{平田, 日笠, 藤井, 黒橋}{平田\Jetal }{2000}]{hirata:00:a}
平田大志,日笠亘,藤井綱貴,黒橋禎夫 \BBOP 2000\BBCP.
\newblock \JBOQ 図書館の自動リファレンス・サービス・システムの構築\JBCQ\
\newblock \Jem{言語処理学会第6回年次大会発表論文集}, \BPGS\ 463--466.

\bibitem[\protect\BCAY{Hovy}{Hovy}{1988}]{hovy:88}
Hovy, E.~H. \BBOP 1988\BBCP.
\newblock {\Bem Generating natural language under pragmatic constraints}.
\newblock Lawrence Erlbaum Associates.

\bibitem[\protect\BCAY{Huang \BBA\ Fiedler}{Huang \BBA\
  Fiedler}{1996}]{huang:96}
Huang, X.\,\BBA\ Fiedler, A. \BBOP 1996\BBCP.
\newblock \BBOQ Paraphrasing and aggregating argumentative text using text
  structure\BBCQ\
\newblock In {\Bem Proceedings of the 8th International Workshop on Natural
  Language Generation {\rm (}INLG\/{\rm )}}, \BPGS\ 21--30.

\bibitem[\protect\BCAY{飯田, 徳永, 乾, 衛藤}{飯田\Jetal }{2001}]{iida:01}
飯田龍,徳永泰浩,乾健太郎,衛藤純司 \BBOP 2001\BBCP.
\newblock \JBOQ
  言い換えエンジン\textsc{Kura}を用いた節内構造および機能語相当表現レベルの言
い換え\JBCQ\
\newblock \Jem{第63回情報処理学会全国大会予稿集第二分冊}, \BPGS\ 5--6.

\bibitem[\protect\BCAY{池田, 大橋, 山本}{池田\Jetal }{2004}]{ikeda:04}
池田諭史,大橋一輝,山本和英 \BBOP 2004\BBCP.
\newblock \JBOQ 「新幹線要約」のための文末の整形\JBCQ\
\newblock \Jem{情報処理学会研究報告, NL-163-22}.

\bibitem[\protect\BCAY{池原, 宮崎, 白井, 横尾, 中岩, 小倉, 大山, 林}{池原\Jetal
  }{1997}]{NTT:97}
池原悟,宮崎正弘,白井諭,横尾昭男,中岩浩巳,小倉健太郎,大山芳史,
  林良彦\JEDS\ \BBOP 1997\BBCP.
\newblock \Jem{日本語語彙大系: CD-ROM版}.
\newblock 岩波書店.

\bibitem[\protect\BCAY{今村, 秋葉, 隅田}{今村\Jetal }{2001}]{imamura:01}
今村賢治,秋葉泰弘,隅田英一郎 \BBOP 2001\BBCP.
\newblock \JBOQ 階層的句アライメントを用いた日本語翻訳文の換言\JBCQ\
\newblock \Jem{言語処理学会第7回年次大会ワークショップ論文集}, \BPGS\ 15--20.

\bibitem[\protect\BCAY{今村}{今村}{2002}]{imamura:02}
今村賢治 \BBOP 2002\BBCP.
\newblock \JBOQ 構文解析と融合した階層的句アライメント\JBCQ\
\newblock \Jem{自然言語処理}, {\Bbf 9}  (5), \BPGS\ 23--42.

\bibitem[\protect\BCAY{Inkpen}{Inkpen}{2003}]{inkpen:03:b}
Inkpen, D.~Z. \BBOP 2003\BBCP.
\newblock {\Bem Building a lexical knowledge-base of near-synonym differences}.
\newblock Ph.D.\ thesis, Department of Computer Science, University of Toronto.

\bibitem[\protect\BCAY{乾, 山本, 野上, 藤田, 乾}{乾\Jetal }{1999}]{inui:99:a}
乾健太郎,山本聡美,野上優,藤田篤,乾裕子 \BBOP 1999\BBCP.
\newblock \JBOQ 聾者向け文章読解支援における構文的言い換えの効果について\JBCQ\
\newblock \Jem{電子情報通信学会技術研究報告, WIT1999-2}, \BPGS\ 9--14.

\bibitem[\protect\BCAY{乾}{乾}{1999}]{inui:99:b}
乾健太郎 \BBOP 1999\BBCP.
\newblock \JBOQ 文章生成\JBCQ\
\newblock \Jem{自然言語処理---基礎と応用---}, \BPGS\ 116--158.
  電子情報通信学会.

\bibitem[\protect\BCAY{乾}{乾}{2001}]{inui:01:a}
乾健太郎 \BBOP 2001\BBCP.
\newblock \JBOQ コミュニケーション支援のための言い換え\JBCQ\
\newblock \Jem{言語処理学会第7回年次大会ワークショップ論文集}, \BPGS\ 71--76.

\bibitem[\protect\BCAY{Inui \BBA\ Nogami}{Inui \BBA\ Nogami}{2001}]{inui:01:c}
Inui, K.\,\BBA\ Nogami, M. \BBOP 2001\BBCP.
\newblock \BBOQ A paraphrase-based exploration of cohesiveness criteria\BBCQ\
\newblock In {\Bem Proceedings of the 8th European Workshop on Natulal Language
  Generation {\rm (}EWNLG\/{\rm )}}, \BPGS\ 101--110.

\bibitem[\protect\BCAY{Inui \BBA\ Hermjekob}{Inui \BBA\
  Hermjekob}{2003}]{IWP:03}
Inui, K.\,\BBA\ Hermjekob, U.\BEDS\ \BBOP 2003\BBCP.
\newblock {\Bem The 2nd International Workshop on Paraphrasing: Paraphrase
  Acquisition and Applications {\rm (}IWP\/{\rm )}}.
\newblock ACL-2003 Workshop.

\bibitem[\protect\BCAY{Inui, Fujita, Takahashi, Iida, \BBA\ Iwakura}{Inui
  et~al.}{2003}]{inui:03:a}
Inui, K., Fujita, A., Takahashi, T., Iida, R., \BBA\ Iwakura, T. \BBOP
  2003\BBCP.
\newblock \BBOQ Text simplification for reading assistance: a project
  note\BBCQ\
\newblock In {\Bem Proceedings of the 2nd International Workshop on
  Paraphrasing: Paraphrase Acquisition and Applications {\rm (}IWP\/{\rm )}},
  \BPGS\ 9--16.

\bibitem[\protect\BCAY{Iordanskaja, Kittredge, \BBA\ Polgu\`{e}re}{Iordanskaja
  et~al.}{1991}]{iordanskaja:91}
Iordanskaja, L., Kittredge, R., \BBA\ Polgu\`{e}re, A. \BBOP 1991\BBCP.
\newblock \BBOQ Lexical selection and paraphrase in a meaning-text generation
  model\BBCQ\
\newblock In {\Bem Paris et al. {\rm (}Eds.{\rm )} Natural Language Generation
  in Artificial Intelligence and Computational Linguistics}, \BPGS\ 293--312.
  Kluwer Academic Publishers.

\bibitem[\protect\BCAY{Iordanskaja, Kim, \BBA\ Polgu\`{e}re}{Iordanskaja
  et~al.}{1996}]{iordanskaja:96}
Iordanskaja, L., Kim, M., \BBA\ Polgu\`{e}re, A. \BBOP 1996\BBCP.
\newblock \BBOQ Some procedural problems in the implementation of lexical
  functions for text generation\BBCQ\
\newblock In {\Bem Wanner {\rm (}Ed.{\rm )} Lexical Functions in Lexicography
  and Natural Language Processing}, \BPGS\ 279--297. John Benjamin Publishing
  Company.

\bibitem[\protect\BCAY{Jacquemin, Klavans, \BBA\ Tzoukermann}{Jacquemin
  et~al.}{1997}]{jacquemin:97}
Jacquemin, C., Klavans, J.~L., \BBA\ Tzoukermann, E. \BBOP 1997\BBCP.
\newblock \BBOQ Expansion of multi-word terms for indexing and retrieval using
  morphology and syntax\BBCQ\
\newblock In {\Bem Proceedings of the 35th Annual Meeting of the Association
  for Computational Linguistics and the 8th Conference of the European Chapter
  of the Association for Computational Linguistics {\rm (}ACL-EACL\/{\rm )}},
  \BPGS\ 24--31.

\bibitem[\protect\BCAY{Kageura, Yoshikane, \BBA\ Nozawa}{Kageura
  et~al.}{2004}]{kageura:04:b}
Kageura, K., Yoshikane, F., \BBA\ Nozawa, T. \BBOP 2004\BBCP.
\newblock \BBOQ Parallel bilingual paraphrase rule for noun compounds: concepts
  and rules for exploring web language resources\BBCQ\
\newblock In {\Bem Proceedings of the 4th Workshop on Asian Language Resources
  {\rm (}ALR\/{\rm )}}, \BPGS\ 54--61.

\bibitem[\protect\BCAY{影山}{影山}{2001}]{kageyama:01}
影山太郎\JED\ \BBOP 2001\BBCP.
\newblock \Jem{日英対照 動詞の意味と構文}.
\newblock 大修館書店.

\bibitem[\protect\BCAY{鍜治, 黒橋, 佐藤}{鍜治\Jetal }{2001}]{kaji:01}
鍜治伸裕,黒橋禎夫,佐藤理史 \BBOP 2001\BBCP.
\newblock \JBOQ 国語辞典に基づく平易文へのパラフレーズ\JBCQ\
\newblock \Jem{情報処理学会研究報告, NL-144-23}, \BPGS\ 167--174.

\bibitem[\protect\BCAY{鍜治, 河原, 黒橋, 佐藤}{鍜治\Jetal }{2003}]{kaji:03:b}
鍜治伸裕,河原大輔,黒橋禎夫,佐藤理史 \BBOP 2003\BBCP.
\newblock \JBOQ 格フレームの対応付けに基づく用言の言い換え\JBCQ\
\newblock \Jem{自然言語処理}, {\Bbf 10}  (4), \BPGS\ 65--81.

\bibitem[\protect\BCAY{鍜治 黒橋}{鍜治\JBA 黒橋}{2004}]{kaji:04:a}
鍜治伸裕,黒橋禎夫 \BBOP 2004\BBCP.
\newblock \JBOQ 迂言表現と重複表現の認識と言い換え\JBCQ\
\newblock \Jem{自然言語処理}, {\Bbf 11}  (1), \BPGS\ 81--106.

\bibitem[\protect\BCAY{Kaji, Okamoto, \BBA\ Kurohashi}{Kaji
  et~al.}{2004}]{kaji:04:c}
Kaji, N., Okamoto, M., \BBA\ Kurohashi, S. \BBOP 2004\BBCP.
\newblock \BBOQ Paraphrasing predicates from written language to spoken
  language using the {W}eb\BBCQ\
\newblock In {\Bem Proceedings of the 2004 Human Language Technology Conference
  of the North American Chapter of the Association for Computational
  Linguistics {\rm (}HLT-NAACL\/{\rm )}}.

\bibitem[\protect\BCAY{Kanayama}{Kanayama}{2003}]{kanayama:03}
Kanayama, H. \BBOP 2003\BBCP.
\newblock \BBOQ Paraphrasing rules for automatic evaluation of translation into
  {J}apanese\BBCQ\
\newblock In {\Bem Proceedings of the 2nd International Workshop on
  Paraphrasing: Paraphrase Acquisition and Applications {\rm (}IWP\/{\rm )}},
  \BPGS\ 88--93.

\bibitem[\protect\BCAY{金子, 八木沢, 藤田}{金子\Jetal }{1996}]{kaneko:96}
金子和恵,八木沢津義,藤田稔 \BBOP 1996\BBCP.
\newblock \JBOQ 話し手の性別・年齢を反映する文生成システム\JBCQ\
\newblock \Jem{情報処理学会研究報告, NL-116-19}, \BPGS\ 129--136.

\bibitem[\protect\BCAY{片岡, 増山, 山本}{片岡\Jetal }{2000}]{kataoka:00}
片岡明,増山繁,山本和英 \BBOP 2000\BBCP.
\newblock \JBOQ 動詞型連体修飾表現の{"N1のN2"}への言い換え\JBCQ\
\newblock \Jem{自然言語処理}, {\Bbf 7}  (4), \BPGS\ 79--98.

\bibitem[\protect\BCAY{加藤, 小川, 佐良木}{加藤\Jetal }{1997}]{kato:97:a}
加藤輝政,小川清,佐良木昌 \BBOP 1997\BBCP.
\newblock \JBOQ 英語構文の構文解析と編集、その論理と方法\JBCQ\
\newblock \Jem{情報処理学会研究報告, NL-120-10}, \BPGS\ 65--70.

\bibitem[\protect\BCAY{川村}{川村}{2000}]{kawamura:00}
川村珠巨 \BBOP 2000\BBCP.
\newblock \JBOQ
  言語生成プロセスの一視点---命題を言い換えるというコミュニケーション方略につ
いて---\JBCQ\
\newblock \Jem{大阪府立工業高等専門学校研究紀要, Vol.34}, \BPGS\ 77--82.

\bibitem[\protect\BCAY{金 江原}{金\JBA 江原}{1994}]{kim:94:a}
金淵培,江原暉将 \BBOP 1994\BBCP.
\newblock \JBOQ 日英機械翻訳のための日本語長文自動短文分割と主語の補完\JBCQ\
\newblock \Jem{情報処理学会論文誌}, {\Bbf 35}  (6), \BPGS\ 1018--1028.

\bibitem[\protect\BCAY{木村, 徳永, 田中}{木村\Jetal }{2001}]{kimura:01:a}
木村健司,徳永健伸,田中穂積 \BBOP 2001\BBCP.
\newblock \JBOQ 漢字インデックスを利用したパラフレーズの抽出\JBCQ\
\newblock \Jem{情報処理学会研究報告, NL-146-7}, \BPGS\ 39--45.

\bibitem[\protect\BCAY{木村, 徳永, 田中}{木村\Jetal }{2002}]{kimura:02}
木村健司,徳永健伸,田中穂積 \BBOP 2002\BBCP.
\newblock \JBOQ
  日本語名詞句に対するパラフレーズ事例の自動抽出に関する研究\JBCQ\
\newblock \Jem{言語処理学会第8回年次大会発表論文集}, \BPGS\ 327--330.

\bibitem[\protect\BCAY{金水, 工藤, 沼田}{金水\Jetal }{2000}]{kinsui:00}
金水敏,工藤真由美,沼田善子 \BBOP 2000\BBCP.
\newblock \Jem{時・否定と取り立て}.
\newblock 岩波書店.

\bibitem[\protect\BCAY{Knight \BBA\ Chander}{Knight \BBA\
  Chander}{1994}]{knight:94:a}
Knight, K.\,\BBA\ Chander, I. \BBOP 1994\BBCP.
\newblock \BBOQ Automated postediting of documents\BBCQ\
\newblock In {\Bem Proceedings of the 12th National Conference on Artificial
  Intelligence {\rm (}AAAI\/{\rm )}}, \BPGS\ 779--784.

\bibitem[\protect\BCAY{Knight \BBA\ Hatzivassiloglou}{Knight \BBA\
  Hatzivassiloglou}{1995}]{knight:95}
Knight, K.\,\BBA\ Hatzivassiloglou, V. \BBOP 1995\BBCP.
\newblock \BBOQ Two-level, many-paths generation\BBCQ\
\newblock In {\Bem Proceedings of the 33rd Annual Meeting of the Association
  for Computational Linguistics {\rm (}ACL\/{\rm )}}, \BPGS\ 252--260.

\bibitem[\protect\BCAY{Kondo \BBA\ Okumura}{Kondo \BBA\
  Okumura}{1997}]{kondo:97:a}
Kondo, K.\,\BBA\ Okumura, M. \BBOP 1997\BBCP.
\newblock \BBOQ Summarization with dictionary-based paraphrasing\BBCQ\
\newblock In {\Bem Proceedings of the 4th Natural Language Processing Pacific
  Rim Symposium {\rm (}NLPRS\/{\rm )}}, \BPGS\ 649--652.

\bibitem[\protect\BCAY{近藤, 佐藤, 奥村}{近藤\Jetal }{1999}]{kondo:99}
近藤恵子,佐藤理史,奥村学 \BBOP 1999\BBCP.
\newblock \JBOQ 「サ変名詞+する」から動詞相当句への言い換え\JBCQ\
\newblock \Jem{情報処理学会論文誌}, {\Bbf 40}  (11), \BPGS\ 4064--4074.

\bibitem[\protect\BCAY{近藤, 佐藤, 奥村}{近藤\Jetal }{2001}]{kondo:01}
近藤恵子,佐藤理史,奥村学 \BBOP 2001\BBCP.
\newblock \JBOQ 格変換による単文の言い換え\JBCQ\
\newblock \Jem{情報処理学会論文誌}, {\Bbf 42}  (3), \BPGS\ 465--477.

\bibitem[\protect\BCAY{神田, 藤田, 乾}{神田\Jetal }{2001}]{kouda:01}
神田慎哉,藤田篤,乾健太郎 \BBOP 2001\BBCP.
\newblock \JBOQ 連用節主節化に関する規則の追試と洗練\JBCQ\
\newblock \Jem{第15回人工知能学会全国大会, 1A1-06}.

\bibitem[\protect\BCAY{国広}{国広}{2000}]{kunihiro:00}
国広哲弥 \BBOP 2000\BBCP.
\newblock \JBOQ 人はなぜ言葉を言い換えるか\JBCQ\
\newblock \Jem{言語}, {\Bbf 29}  (10), \BPGS\ 20--25.

\bibitem[\protect\BCAY{Kurohashi \BBA\ Sakai}{Kurohashi \BBA\
  Sakai}{1999}]{kurohashi:99:b}
Kurohashi, S.\,\BBA\ Sakai, Y. \BBOP 1999\BBCP.
\newblock \BBOQ Semantic analysis of {J}apanese noun phrases: a new approach to
  dictionary-based understanding\BBCQ\
\newblock In {\Bem Proceedings of the 37th Annual Meeting of the Association
  for Computational Linguistics {\rm (}ACL\/{\rm )}}, \BPGS\ 481--488.

\bibitem[\protect\BCAY{黒橋 酒井}{黒橋\JBA 酒井}{2001}]{kurohashi:01}
黒橋禎夫,酒井康行 \BBOP 2001\BBCP.
\newblock \JBOQ 日本語の柔軟な照合\JBCQ\
\newblock \Jem{言語処理学会第7回年次大会発表論文集}, \BPGS\ 343--346.

\bibitem[\protect\BCAY{黒川}{黒川}{2003}]{kurokawa:03:master}
黒川和也 \BBOP 2003\BBCP.
\newblock \JBOQ
  日本語機能表現の正用・誤用の判別および日本語学習支援における利用に関する研究
\JBCQ\
\newblock 豊橋技術科学大学大学院工学研究科修士論文.

\bibitem[\protect\BCAY{Langkilde \BBA\ Knight}{Langkilde \BBA\
  Knight}{1998}]{langkilde:98}
Langkilde, I.\,\BBA\ Knight, K. \BBOP 1998\BBCP.
\newblock \BBOQ Generation that exploits corpus-based statistical
  knowledge\BBCQ\
\newblock In {\Bem Proceedings of the 36th Annual Meeting of the Association
  for Computational Linguistics and the 17th International Conference on
  Computational Linguistics {\rm (}COLING-ACL\/{\rm )}}, \BPGS\ 704--710.

\bibitem[\protect\BCAY{Lapata, Keller, \BBA\ McDonald}{Lapata
  et~al.}{2001}]{lapata:01:b}
Lapata, M., Keller, F., \BBA\ McDonald, S. \BBOP 2001\BBCP.
\newblock \BBOQ Evaluating smoothing algorithms against plausibility
  judgements\BBCQ\
\newblock In {\Bem Proceedings of the 39th Annual Meeting of the Association
  for Computational Linguistics {\rm (}ACL\/{\rm )}}, \BPGS\ 346--353.

\bibitem[\protect\BCAY{Lavoie, Kittredge, Korelsky, \BBA\ Rambow}{Lavoie
  et~al.}{2000}]{lavoie:00}
Lavoie, B., Kittredge, R., Korelsky, T., \BBA\ Rambow, O. \BBOP 2000\BBCP.
\newblock \BBOQ A framework for {MT} and multilingual {NLG} systems based on
  uniform lexico-structural processing\BBCQ\
\newblock In {\Bem Proceedings of the 6th Applied Natural Language Processing
  Conference and the 1st Meeting of the North American Chapter of the
  Association for Computational Linguistics {\rm (}ANLP-NAACL\/{\rm )}}, \BPGS\
  60--67.

\bibitem[\protect\BCAY{Levin}{Levin}{1993}]{levin:93}
Levin, B. \BBOP 1993\BBCP.
\newblock {\Bem English verb classes and alternations: a preliminary
  investigation}.
\newblock Chicago Press.

\bibitem[\protect\BCAY{Lin}{Lin}{1998}]{lin:98:a}
Lin, D. \BBOP 1998\BBCP.
\newblock \BBOQ Extracting collocations from text corpora\BBCQ\
\newblock In {\Bem Proceedings of the 1st International Workshop on
  Computational Terminology {\rm (}CompuTerm\/{\rm )}}, \BPGS\ 57--63.

\bibitem[\protect\BCAY{Lin \BBA\ Pantel}{Lin \BBA\ Pantel}{2001}]{lin:01}
Lin, D.\,\BBA\ Pantel, P. \BBOP 2001\BBCP.
\newblock \BBOQ Discovery of inference rules for question answering\BBCQ\
\newblock {\Bem Natural Language Engineering}, {\Bbf 7}  (4), \BPGS\ 343--360.

\bibitem[\protect\BCAY{Mani, Gates, \BBA\ Bloedorn}{Mani
  et~al.}{1999}]{mani:99:a}
Mani, I., Gates, B., \BBA\ Bloedorn, E. \BBOP 1999\BBCP.
\newblock \BBOQ Improving summaries by revising them\BBCQ\
\newblock In {\Bem Proceedings of the 37th Annual Meeting of the Association
  for Computational Linguistics {\rm (}ACL\/{\rm )}}, \BPGS\ 558--565.

\bibitem[\protect\BCAY{Masuda, Yasutomi, \BBA\ Nakagawa}{Masuda
  et~al.}{2001}]{masuda:01:a}
Masuda, H., Yasutomi, D., \BBA\ Nakagawa, H. \BBOP 2001\BBCP.
\newblock \BBOQ How to transform tables in {HTML} for displaying on mobile
  terminals\BBCQ\
\newblock In {\Bem Proceedings of the 6th Natural Language Processing Pacific
  Rim Symposium {\rm (}NLPRS\/{\rm )} Workshop on Automatic Paraphrasing:
  Theories and Applications}, \BPGS\ 29--36.

\bibitem[\protect\BCAY{益岡 田窪}{益岡\JBA 田窪}{1994}]{masuoka:94}
益岡隆志,田窪行則 \BBOP 1994\BBCP.
\newblock \Jem{基礎日本語文法(改訂版)}.
\newblock くろしお出版.

\bibitem[\protect\BCAY{松吉, 佐藤, 宇津呂}{松吉\Jetal }{2004}]{matsuyoshi:04:a}
松吉俊,佐藤理史,宇津呂武仁 \BBOP 2004\BBCP.
\newblock \JBOQ 機能表現「なら」の機械翻訳のための言い換え\JBCQ\
\newblock \Jem{情報処理学会研究報告, NL-159-28}, \BPGS\ 201--208.

\bibitem[\protect\BCAY{McKeown, Klavans, Hatzivassiloglou, Barzilay, \BBA\
  Eskin}{McKeown et~al.}{1999}]{mckeown:99}
McKeown, K.~R., Klavans, J.~L., Hatzivassiloglou, V., Barzilay, R., \BBA\
  Eskin, E. \BBOP 1999\BBCP.
\newblock \BBOQ Towards multidocument summarization by reformulation: progress
  and prospects\BBCQ\
\newblock In {\Bem Proceedings of the 16th National Conference on Artificial
  Intelligence and 11th Conference on Innovative Application s of Artificial
  Intelligence {\rm (}AAAI-IAAI\/{\rm )}}, \BPGS\ 453--460.

\bibitem[\protect\BCAY{Melamed}{Melamed}{2001}]{melamed:01}
Melamed, I.~D. \BBOP 2001\BBCP.
\newblock {\Bem Empirical methods for exploiting parallel texts}.
\newblock The MIT Press.

\bibitem[\protect\BCAY{Mel'\v{c}uk \BBA\ Polgu\`{e}re}{Mel'\v{c}uk \BBA\
  Polgu\`{e}re}{1987}]{melcuk:87}
Mel'\v{c}uk, I.\,\BBA\ Polgu\`{e}re, A. \BBOP 1987\BBCP.
\newblock \BBOQ A formal lexicon in meaning-text theory (or how to do lexica
  with words)\BBCQ\
\newblock {\Bem Computational Linguistics}, {\Bbf 13}  (3-4), \BPGS\ 261--275.

\bibitem[\protect\BCAY{Mel'\v{c}uk}{Mel'\v{c}uk}{1996}]{melcuk:96}
Mel'\v{c}uk, I. \BBOP 1996\BBCP.
\newblock \BBOQ Lexical functions: a tool for the description of lexical
  relations in a lexicon\BBCQ\
\newblock In {\Bem Wanner {\rm (}Ed.{\rm )} Lexical Functions in Lexicography
  and Natural Language Processing}, \BPGS\ 37--102. John Benjamin Publishing
  Company.

\bibitem[\protect\BCAY{Meteer \BBA\ Shaked}{Meteer \BBA\
  Shaked}{1988}]{meteer:88}
Meteer,~M.\,\BBA\ Shaked,~V. \BBOP 1988\BBCP.
\newblock \BBOQ Strategies for effective paraphrasing\BBCQ\
\newblock In {\Bem Proceedings of the 12th International Conference on
  Computational Linguistics {\rm (}COLING\/{\rm )}}, \BPGS\ 431--436.

\bibitem[\protect\BCAY{Meyers, Yangarber, \BBA\ Grishman}{Meyers
  et~al.}{1998}]{meyers:98}
Meyers, A., Yangarber, R., \BBA\ Grishman, R. \BBOP 1998\BBCP.
\newblock \BBOQ Deriving transfer rules from dominance-preserving
  alignments\BBCQ\
\newblock In {\Bem Proceedings of the 36th Annual Meeting of the Association
  for Computational Linguistics and the 17th International Conference on
  Computational Linguistics {\rm (}COLING-ACL\/{\rm )}}, \BPGS\ 843--847.

\bibitem[\protect\BCAY{三上, 増山, 中川}{三上\Jetal }{1999}]{mikami:99}
三上真,増山繁,中川聖一 \BBOP 1999\BBCP.
\newblock \JBOQ ニュース番組における字幕生成のための文内短縮による要約\JBCQ\
\newblock \Jem{自然言語処理}, {\Bbf 6}  (6), \BPGS\ 65--81.

\bibitem[\protect\BCAY{Miller, Beckwith, Fellbaum, Gross, \BBA\ Miller}{Miller
  et~al.}{1990}]{WN:90}
Miller, G.~A., Beckwith, R., Fellbaum, C., Gross, D., \BBA\ Miller, K.~J. \BBOP
  1990\BBCP.
\newblock \BBOQ Introduction to {W}ord{N}et: an on-line lexical database\BBCQ\
\newblock {\Bem International Journal of Lexicography}, {\Bbf 3}  (4),
  \BPGS\ 235--244.

\bibitem[\protect\BCAY{Mitamura \BBA\ Nyberg}{Mitamura \BBA\
  Nyberg}{2001}]{mitamura:01}
Mitamura, T.\,\BBA\ Nyberg, E. \BBOP 2001\BBCP.
\newblock \BBOQ Automatic rewriting for controlled language translation\BBCQ\
\newblock In {\Bem Proceedings of the 6th Natural Language Processing Pacific
  Rim Symposium {\rm (}NLPRS\/{\rm )} Workshop on Automatic Paraphrasing:
  Theories and Applications}, \BPGS\ 1--12.

\bibitem[\protect\BCAY{宮島 仁田}{宮島\JBA 仁田}{1995a}]{miyajima:95:a}
宮島達夫,仁田義雄\JEDS\ \BBOP 1995a\BBCP.
\newblock \Jem{日本語類義表現の文法(上)単文編}.
\newblock くろしお出版.

\bibitem[\protect\BCAY{宮島 仁田}{宮島\JBA 仁田}{1995b}]{miyajima:95:b}
宮島達夫,仁田義雄\JEDS\ \BBOP 1995b\BBCP.
\newblock \Jem{日本語類義表現の文法(下)複文・連文編}.
\newblock くろしお出版.

\bibitem[\protect\BCAY{Moldovan, Clark, Harabagiu, \BBA\ Maiorano}{Moldovan
  et~al.}{2003}]{moldovan:03}
Moldovan, D., Clark, C., Harabagiu, S., \BBA\ Maiorano, S. \BBOP 2003\BBCP.
\newblock \BBOQ {COGEX}: a logic prover for question answering\BBCQ\
\newblock In {\Bem Proceedings of the 2003 Human Language Technology Conference
  of the North American Chapter of the Association for Computational
  Linguistics {\rm (}HLT-NAACL\/{\rm )}}, \BPGS\ 87--93.

\bibitem[\protect\BCAY{森田 松木}{森田\JBA 松木}{1989}]{morita:89}
森田良行,松木正恵 \BBOP 1989\BBCP.
\newblock \Jem{日本語表現文型---用例中心・複合辞の意味と用法}.
\newblock アルク.

\bibitem[\protect\BCAY{森田}{森田}{1994}]{morita:94}
森田良行 \BBOP 1994\BBCP.
\newblock \Jem{動詞の意味論的文法研究}.
\newblock 明治書院.

\bibitem[\protect\BCAY{村木}{村木}{1991}]{muraki:91}
村木新次郎 \BBOP 1991\BBCP.
\newblock \Jem{日本語動詞の諸相}.
\newblock ひつじ書房.

\bibitem[\protect\BCAY{Murata \BBA\ Isahara}{Murata \BBA\
  Isahara}{2001}]{murata:01:c}
Murata, M.\,\BBA\ Isahara, H. \BBOP 2001\BBCP.
\newblock \BBOQ Universal model for paraphrasing --- using transformation based
  on a defined criteria ---\BBCQ\
\newblock In {\Bem Proceedings of the 6th Natural Language Processing Pacific
  Rim Symposium {\rm (}NLPRS\/{\rm )} Workshop on Automatic Paraphrasing:
  Theories and Applications}, \BPGS\ 47--54.

\bibitem[\protect\BCAY{村田 井佐原}{村田\JBA 井佐原}{2002}]{murata:02}
村田真樹,井佐原均 \BBOP 2002\BBCP.
\newblock \JBOQ
  受け身/使役文の能動文への変換における機械学習を用いた格助詞の変換\JBCQ\
\newblock \Jem{情報処理学会研究報告, NL-149-6}, \BPGS\ 39--44.

\bibitem[\protect\BCAY{Nanba \BBA\ Okumura}{Nanba \BBA\
  Okumura}{2000}]{nanba:00:a}
Nanba, H.\,\BBA\ Okumura, M. \BBOP 2000\BBCP.
\newblock \BBOQ Producing more readable extracts by revising them\BBCQ\
\newblock In {\Bem Proceedings of the 18th International Conference on
  Computational Linguistics {\rm (}COLING\/{\rm )}}, \BPGS\ 1071--1075.

\bibitem[\protect\BCAY{成松, 河原, 黒橋, 西田}{成松\Jetal
  }{2002}]{narimatsu:02}
成松深,河原大輔,黒橋禎夫,西田豊明 \BBOP 2002\BBCP.
\newblock \JBOQ 格関係の比較を用いた複数テキスト間の重複・差分の検出\JBCQ\
\newblock \Jem{言語処理学会第8回年次大会発表論文集}, \BPGS\ 535--538.

\bibitem[\protect\BCAY{日本電子化辞書研究所}{日本電子化辞書研究所}{1995}]{EDR:
95}
日本電子化辞書研究所 \BBOP 1995\BBCP.
\newblock \Jem{EDR電子化辞書仕様説明書}.

\bibitem[\protect\BCAY{野上 乾}{野上\JBA 乾}{2001}]{nogami:01}
野上優,乾健太郎 \BBOP 2001\BBCP.
\newblock \JBOQ 結束性を考慮した連体修飾節の言い換え\JBCQ\
\newblock \Jem{言語処理学会第7回年次大会発表論文集}, \BPGS\ 339--342.

\bibitem[\protect\BCAY{野上 乾}{野上\JBA 乾}{2002}]{nogami:02}
野上優,乾健太郎 \BBOP 2002\BBCP.
\newblock \JBOQ 言い換えを用いた結束性評価基準の構築\JBCQ\
\newblock \Jem{言語処理学会第8回年次大会発表論文集}, \BPGS\ 335--338.

\bibitem[\protect\BCAY{Nyberg \BBA\ Mitamura}{Nyberg \BBA\
  Mitamura}{2000}]{nyberg:00}
Nyberg, E.\,\BBA\ Mitamura, T. \BBOP 2000\BBCP.
\newblock \BBOQ The {KANTOO} machine translation environment\BBCQ\
\newblock In {\Bem Proceedings of AMTA 2000 Conference}, \BPGS\ 192--195.

\bibitem[\protect\BCAY{小川 石崎}{小川\JBA 石崎}{2004}]{ogawa:04}
小川修太,石崎俊 \BBOP 2004\BBCP.
\newblock \JBOQ
  概念辞書における深層格の相互作用について---壁塗り構文を例として---\JBCQ\
\newblock \Jem{言語処理学会第10回年次大会発表論文集}, \BPGS\ 572--575.

\bibitem[\protect\BCAY{大橋 山本}{大橋\JBA 山本}{2004}]{ohashi:04}
大橋一輝,山本和英 \BBOP 2004\BBCP.
\newblock \JBOQ 「サ変動詞+名詞」の複合名詞への換言\JBCQ\
\newblock \Jem{言語処理学会第10回年次大会発表論文集}, \BPGS\ 693--696.

\bibitem[\protect\BCAY{大野, 横山, 西原}{大野\Jetal }{2003}]{ohno:03}
大野満,横山晶一,西原典孝 \BBOP 2003\BBCP.
\newblock \JBOQ 日本語敬語表現の変換・解析システム\JBCQ\
\newblock \Jem{言語処理学会第9回年次大会発表論文集}, \BPGS\ 218--222.

\bibitem[\protect\BCAY{Ohtake \BBA\ Yamamoto}{Ohtake \BBA\
  Yamamoto}{2001}]{ohtake:01:b}
Ohtake, K.\,\BBA\ Yamamoto, K. \BBOP 2001\BBCP.
\newblock \BBOQ Paraphrasing honorifics\BBCQ\
\newblock In {\Bem Proceedings of the 6th Natural Language Processing Pacific
  Rim Symposium {\rm (}NLPRS\/{\rm )} Workshop on Automatic Paraphrasing:
  Theories and Applications}, \BPGS\ 13--20.

\bibitem[\protect\BCAY{Ohtake \BBA\ Yamamoto}{Ohtake \BBA\
  Yamamoto}{2003}]{ohtake:03:b}
Ohtake, K.\,\BBA\ Yamamoto, K. \BBOP 2003\BBCP.
\newblock \BBOQ Applicability analysis of corpus-derived paraphrases toward
  example-based paraphrasing\BBCQ\
\newblock In {\Bem Proceedings of the 17th Pacific Asia Conference on Language,
  Information and Computation {\rm (}PACLIC\/{\rm )}}, \BPGS\ 380--391.

\bibitem[\protect\BCAY{大泉, 鍜治, 河原, 岡本, 黒橋, 西田}{大泉\Jetal
  }{2003}]{oizumi:03}
大泉敏貴,鍜治伸裕,河原大輔,岡本雅史,黒橋禎夫,西田豊明 \BBOP 2003\BBCP.
\newblock \JBOQ 書きことばから話しことばへの変換\JBCQ\
\newblock \Jem{言語処理学会第9回年次大会発表論文集}, \BPGS\ 93--96.

\bibitem[\protect\BCAY{Okamoto, Sato, \BBA\ Saito}{Okamoto
  et~al.}{2003}]{okamoto:03:b}
Okamoto, H., Sato, K., \BBA\ Saito, H. \BBOP 2003\BBCP.
\newblock \BBOQ Preferential presentation of {J}apanese near-synonyms using
  definition statements\BBCQ\
\newblock In {\Bem Proceedings of the 2nd International Workshop on
  Paraphrasing: Paraphrase Acquisition and Applications {\rm (}IWP\/{\rm )}},
  \BPGS\ 17--24.

\bibitem[\protect\BCAY{奥}{奥}{1990}]{oku:90}
奥雅博 \BBOP 1990\BBCP.
\newblock \JBOQ 日本文解析における述語相当の慣用的表現の扱い\JBCQ\
\newblock \Jem{情報処理学会論文誌}, {\Bbf 31}  (12), \BPGS\ 1727--1734.

\bibitem[\protect\BCAY{奥村 難波}{奥村\JBA 難波}{1999}]{okumura:99:a}
奥村学,難波英嗣 \BBOP 1999\BBCP.
\JBOQ テキスト自動要約に関する研究動向\JBCQ\
\Jem{自然言語処理}, {\Bbf 6} (6), \BPGS\ 1--26.

\bibitem[\protect\BCAY{奥村 難波}{奥村\JBA 難波}{2002}]{okumura:02:a}
奥村学,難波英嗣 \BBOP 2002\BBCP.
\newblock \JBOQ テキスト自動要約に関する最近の話題\JBCQ\
\newblock \Jem{自然言語処理}, {\Bbf 9}  (4), \BPGS\ 97--116.

\bibitem[\protect\BCAY{大野 浜西}{大野\JBA 浜西}{1981}]{kadokawa:81}
大野晋,浜西正人 \BBOP 1981\BBCP.
\newblock \Jem{角川類語新辞典}.
\newblock 角川書店.

\bibitem[\protect\BCAY{Pang, Knight, \BBA\ Marcu}{Pang et~al.}{2003}]{pang:03}
Pang, B., Knight, K., \BBA\ Marcu, D. \BBOP 2003\BBCP.
\newblock \BBOQ Syntax-based alignment of multiple translations: extracting
  paraphrases and generating new sentences\BBCQ\
\newblock In {\Bem Proceedings of the 2003 Human Language Technology Conference
  of the North American Chapter of the Association for Computational
  Linguistics {\rm (}HLT-NAACL\/{\rm )}}, \BPGS\ 102--109.

\bibitem[\protect\BCAY{Papineni, Roukos, Ward, \BBA\ Zhu}{Papineni
  et~al.}{2002}]{papineni:02:b}
Papineni, K., Roukos, S., Ward, T., \BBA\ Zhu, W.-J. \BBOP 2002\BBCP.
\newblock \BBOQ \textsc{BLEU}: a method for automatic evaluation of machine
  translation\BBCQ\
\newblock In {\Bem Proceedings of the 40th Annual Meeting of the Association
  for Computational Linguistics {\rm (}ACL\/{\rm )}}, \BPGS\ 311--318.

\bibitem[\protect\BCAY{Pearce}{Pearce}{2001}]{pearce:01}
Pearce, D. \BBOP 2001\BBCP.
\newblock \BBOQ Synonymy in collocation extraction\BBCQ\
\newblock In {\Bem Proceedings of the 2nd Meeting of the North American Chapter
  of the Association for Computational Linguistics {\rm (}NAACL\/{\rm )}
  Workshop on WordNet and Other Lexical Resources: Applications, Extensions and
  Customizations}, \BPGS\ 41--46.

\bibitem[\protect\BCAY{Pereira, Tishby, \BBA\ Lee}{Pereira
  et~al.}{1993}]{pereira:93}
Pereira, F., Tishby, N., \BBA\ Lee, L. \BBOP 1993\BBCP.
\newblock \BBOQ Distributional clustering of {E}nglish words\BBCQ\
\newblock In {\Bem Proceedings of the 31st Annual Meeting of the Association
  for Computational Linguistics {\rm (}ACL\/{\rm )}}, \BPGS\ 183--190.

\bibitem[\protect\BCAY{Quirk, Brockett, \BBA\ Dolan}{Quirk
  et~al.}{2004}]{quirk:04}
Quirk, C., Brockett, C., \BBA\ Dolan, W. \BBOP 2004\BBCP.
\newblock \BBOQ Monolingual machine translation for paraphrase generation\BBCQ\
\newblock In {\Bem Proceedings of the 2004 Conference on Empirical Methods in
  Natural Language Processing {\rm (}EMNLP\/{\rm )}}, \BPGS\ 142--149.

\bibitem[\protect\BCAY{Ravichandran \BBA\ Hovy}{Ravichandran \BBA\
  Hovy}{2002}]{ravichandran:02}
Ravichandran, D.\,\BBA\ Hovy, E. \BBOP 2002\BBCP.
\newblock \BBOQ Learning surface text patterns for a question answering
  system\BBCQ\
\newblock In {\Bem Proceedings of the 40th Annual Meeting of the Association
  for Computational Linguistics {\rm (}ACL\/{\rm )}}, \BPGS\ 215--222.

\bibitem[\protect\BCAY{Robin \BBA\ McKeown}{Robin \BBA\
  McKeown}{1996}]{robin:96}
Robin, J.\,\BBA\ McKeown, K. \BBOP 1996\BBCP.
\newblock \BBOQ Empirically designing and evaluating a new revision-based model
  for summary generation\BBCQ\
\newblock {\Bem Artificial Intelligence}, {\Bbf 85}  (1-2), \BPGS\ 135--179.

\bibitem[\protect\BCAY{RWC}{RWC}{1998}]{RWC:98}
RWC \BBOP 1998\BBCP.
\newblock \Jem{RWC テキストデータベース第2版,
  岩波国語辞典タグ付き/形態素解析データ第5版}.

\bibitem[\protect\BCAY{斉藤, 池原, 村上}{斉藤\Jetal }{2002}]{saito:02:b}
斉藤健太郎,池原悟,村上仁一 \BBOP 2002\BBCP.
\newblock \JBOQ 日英文型パターンの意味的対応方式\JBCQ\
\newblock \Jem{言語処理学会第9回年次大会発表論文集}, \BPGS\ 295--298.

\bibitem[\protect\BCAY{酒井 増山}{酒井\JBA 増山}{2003}]{sakai:03}
酒井浩之,増山繁 \BBOP 2003\BBCP.
\newblock \JBOQ コーパスからの名詞と略語の対応関係の自動獲得\JBCQ\
\newblock \Jem{言語処理学会第9回年次大会発表論文集}, \BPGS\ 226--229.

\bibitem[\protect\BCAY{Sasaki, Isozaki, Kokuryou, Hirao, , \BBA\ Maeda}{Sasaki
  et~al.}{2002}]{sasaki:02}
Sasaki, Y., Isozaki, H., Kokuryou, K., Hirao, T., , \BBA\ Maeda, E. \BBOP
  2002\BBCP.
\newblock \BBOQ {NTT}'s {QA} Systems for {NTCIR} {QAC-1}\BBCQ\
\newblock In {\Bem NTCIR Workshop 3 Question Answering Challenge {\rm
  (}QAC\/{\rm )}}.

\bibitem[\protect\BCAY{佐藤, 岩越, 増田, 中川}{佐藤\Jetal
  }{2004}]{SatoDai:04:a}
佐藤大,岩越守孝,増田英孝,中川裕志 \BBOP 2004\BBCP.
\newblock \JBOQ
  Webと携帯端末向けの新聞記事の対応コーパスからの言い換え抽出\JBCQ\
\newblock \Jem{情報処理学会研究報告, NL-159-27}, \BPGS\ 193--200.

\bibitem[\protect\BCAY{佐藤}{佐藤}{1999}]{sato:99}
佐藤理史 \BBOP 1999\BBCP.
\newblock \JBOQ 論文表題を言い換える\JBCQ\
\newblock \Jem{情報処理学会論文誌}, {\Bbf 40}  (7), \BPGS\ 2937--2945.

\bibitem[\protect\BCAY{佐藤}{佐藤}{2001}]{sato:01:a}
佐藤理史 \BBOP 2001\BBCP.
\newblock \JBOQ なぜ言い換え/パラフレーズを研究するのか\JBCQ\
\newblock \Jem{言語処理学会第7回年次大会ワークショップ論文集}, \BPGS\ 1--2.

\bibitem[\protect\BCAY{Sato \BBA\ Nakagawa}{Sato \BBA\
  Nakagawa}{2001}]{NLPRSWS:01}
Sato, S.\,\BBA\ Nakagawa, H.\BEDS\ \BBOP 2001\BBCP.
\newblock {\Bem Workshop on Automatic Paraphrasing: Theories and Applications}.
\newblock NLPRS-2001 Workshop.

\bibitem[\protect\BCAY{関根}{関根}{2001}]{sekine:01}
関根聡 \BBOP 2001\BBCP.
\newblock \JBOQ 複数の新聞を使用した言い替え表現の自動抽出\JBCQ\
\newblock \Jem{言語処理学会第7回年次大会ワークショップ論文集}, \BPGS\ 9--14.

\bibitem[\protect\BCAY{Shieber \BBA\ Schabes}{Shieber \BBA\
  Schabes}{1990}]{shieber:90}
Shieber, S.~M.\,\BBA\ Schabes, Y. \BBOP 1990\BBCP.
\newblock \BBOQ Synchronous tree-adjoining grammars\BBCQ\
\newblock In {\Bem Proceedings of the 13th International Conference on
  Computational Linguistics {\rm (}COLING\/{\rm )}, Vol. 3}, \BPGS\ 253--258.

\bibitem[\protect\BCAY{下畑, 渡辺, 隅田, 松本}{下畑\Jetal
  }{2003}]{shimohata:03:c}
下畑光夫,渡辺太郎,隅田英一郎,松本裕治 \BBOP 2003\BBCP.
\newblock \JBOQ パラレルコーパスからの機械翻訳向け同義表現抽出\JBCQ\
\newblock \Jem{情報処理学会論文誌}, {\Bbf 44}  (11), \BPGS\ 2854--2863.

\bibitem[\protect\BCAY{Shimohata, Sumita, \BBA\ Matsumoto}{Shimohata
  et~al.}{2004}]{shimohata:04:b}
Shimohata, M., Sumita, E., \BBA\ Matsumoto, Y. \BBOP 2004\BBCP.
\newblock \BBOQ Building a paraphrase corpus for speech translation\BBCQ\
\newblock In {\Bem Proceedings of the 4th International Conference on Language
  Resources and Evaluation {\rm (}LREC\/{\rm )}}, \BPGS\ 1407--1410.

\bibitem[\protect\BCAY{Shinyama, Sekine, Sudo, \BBA\ Grishman}{Shinyama
  et~al.}{2002}]{shinyama:02}
Shinyama, Y., Sekine, S., Sudo, K., \BBA\ Grishman, R. \BBOP 2002\BBCP.
\newblock \BBOQ Automatic paraphrase acquisition from news articles\BBCQ\
\newblock In {\Bem Proceedings of the Human Language Technology Conference {\rm
  (}HLT\/{\rm )}}.

\bibitem[\protect\BCAY{Shinyama \BBA\ Sekine}{Shinyama \BBA\
  Sekine}{2003}]{shinyama:03}
Shinyama, Y.\,\BBA\ Sekine, S. \BBOP 2003\BBCP.
\newblock \BBOQ Paraphrase acquisition for information extraction\BBCQ\
\newblock In {\Bem Proceedings of the 2nd International Workshop on
  Paraphrasing: Paraphrase Acquisition and Applications {\rm (}IWP\/{\rm )}},
  \BPGS\ 65--71.

\bibitem[\protect\BCAY{白井, 池原, 河岡, 中村}{白井\Jetal }{1995}]{shirai:95:a}
白井諭,池原悟,河岡司,中村行宏 \BBOP 1995\BBCP.
\newblock \JBOQ 日英機械翻訳における原文自動書き換え型翻訳方式とその効果\JBCQ\
\newblock \Jem{情報処理学会論文誌}, {\Bbf 36}  (1), \BPGS\ 12--21.

\bibitem[\protect\BCAY{Shirai, Yamamoto, \BBA\ Bond}{Shirai
  et~al.}{2001}]{shirai:01:c}
Shirai, S., Yamamoto, K., \BBA\ Bond, F. \BBOP 2001\BBCP.
\newblock \BBOQ Japanese-{E}nglish paraphrase corpus\BBCQ\
\newblock In {\Bem Proceedings of the 6th Natural Language Processing Pacific
  Rim Symposium {\rm (}NLPRS\/{\rm )} Workshop on Language Resources in Asia},
  \BPGS\ 23--30.

\bibitem[\protect\BCAY{白木 黒橋}{白木\JBA 黒橋}{2000}]{shiraki:00:a}
白木伸征,黒橋禎夫 \BBOP 2000\BBCP.
\newblock \JBOQ 自然言語入力と目次との柔軟な照合による図書検索システム\JBCQ\
\newblock \Jem{情報処理学会論文誌}, {\Bbf 41}  (4), \BPGS\ 1162--1170.

\bibitem[\protect\BCAY{首藤, 田辺, 吉村}{首藤\Jetal }{2001}]{shuto:01}
首藤公昭,田辺利文,吉村賢治 \BBOP 2001\BBCP.
\newblock \JBOQ 日本語モダリティ表現とその言い換え\JBCQ\
\newblock \Jem{言語処理学会第7回年次大会ワークショップ論文集}, \BPGS\ 47--50.

\bibitem[\protect\BCAY{Siddharthan}{Siddharthan}{2003}]{siddharthan:03:a}
Siddharthan, A. \BBOP 2003\BBCP.
\newblock \BBOQ Preserving discourse structure when simplifying text\BBCQ\
\newblock In {\Bem Proceedings of the 9th European Workshop on Natulal Language
  Generation {\rm (}EWNLG\/{\rm )}}, \BPGS\ 103--110.

\bibitem[\protect\BCAY{砂川}{砂川}{1995}]{sunagawa:95}
砂川有里子 \BBOP 1995\BBCP.
\newblock \JBOQ 日本語における分裂文の機能と語順の原理\JBCQ\
\newblock \Jem{複文の研究(下),仁田義雄(編)}, \BPGS\ 353--388.
  くろしお出版.

\bibitem[\protect\BCAY{Takahashi, Iwakura, Iida, Fujita, \BBA\ Inui}{Takahashi
  et~al.}{2001}]{takahashi:01:c}
Takahashi, T., Iwakura, T., Iida, R., Fujita, A., \BBA\ Inui, K. \BBOP
  2001\BBCP.
\newblock \BBOQ \textsc{Kura}: a transfer-based lexico-structural paraphrasing
  engine\BBCQ\
\newblock In {\Bem Proceedings of the 6th Natural Language Processing Pacific
  Rim Symposium {\rm (}NLPRS\/{\rm )} Workshop on Automatic Paraphrasing:
  Theories and Applications}, \BPGS\ 37--46.

\bibitem[\protect\BCAY{Takahashi, Nawata, Inui, \BBA\ Matsumoto}{Takahashi
  et~al.}{2003}]{takahashi:03:c}
Takahashi, T., Nawata, K., Inui, K., \BBA\ Matsumoto, Y. \BBOP 2003\BBCP.
\newblock \BBOQ Effects of structural matching and paraphrasing in question
  answering\BBCQ\
\newblock {\Bem IEICE Transactions on Information and Systems}, {\Bbf E86-D}
  (9), \BPGS\ 1677--1685.

\bibitem[\protect\BCAY{高橋, 乾, 関根, 松本}{高橋\Jetal
  }{2004}]{takahashi:04:a}
高橋哲朗,乾健太郎,関根聡,松本裕治 \BBOP 2004\BBCP.
\newblock \JBOQ 質問応答に必要な言い換えの分析\JBCQ\
\newblock \Jem{言語処理学会第10回年次大会発表論文集}, \BPGS\ 309--312.

\bibitem[\protect\BCAY{高橋 牛島}{高橋\JBA 牛島}{1991}]{takahashi:91:a}
高橋善文,牛島和夫 \BBOP 1991\BBCP.
\newblock \JBOQ 計算機マニュアルのわかりやすさの定量的評価方法\JBCQ\
\newblock \Jem{情報処理学会論文誌}, {\Bbf 32}  (4), \BPGS\ 460--469.

\bibitem[\protect\BCAY{高塚}{高塚}{1999}]{takatsuka:99}
高塚成信 \BBOP 1999\BBCP.
\newblock \JBOQ
  コミュニケーション方略としての「言い換え」---その指導内容と方法に関する基礎
的研究---\JBCQ\
\newblock \Jem{岡山大学教育学部研究集録, 第10号}, \BPGS\ 1--12.

\bibitem[\protect\BCAY{武石 林}{武石\JBA 林}{1992}]{takeishi:92}
武石英二,林良彦 \BBOP 1992\BBCP.
\newblock \JBOQ 接続構造解析に基づく日本語複文の分割\JBCQ\
\newblock \Jem{情報処理学会論文誌}, {\Bbf 33}  (5), \BPGS\ 652--663.

\bibitem[\protect\BCAY{竹内, 内山, 吉岡, 影浦, 小山}{竹内\Jetal
  }{2002}]{takeuchi:02}
竹内孔一,内山清子,吉岡真治,影浦峡,小山照夫 \BBOP 2002\BBCP.
\newblock \JBOQ 語彙概念構造を利用した複合名詞内の係り関係の解析\JBCQ\
\newblock \Jem{情報処理学会論文誌}, {\Bbf 43}  (5), \BPGS\ 1446--1456.

\bibitem[\protect\BCAY{Terada \BBA\ Tokunaga}{Terada \BBA\
  Tokunaga}{2001}]{terada:01}
Terada, A.\,\BBA\ Tokunaga, T. \BBOP 2001\BBCP.
\newblock \BBOQ Automatic disabbreviation by using context information\BBCQ\
\newblock In {\Bem Proceedings of the 6th Natural Language Processing Pacific
  Rim Symposium {\rm (}NLPRS\/{\rm )} Workshop on Automatic Paraphrasing:
  Theories and Applications}, \BPGS\ 21--28.

\bibitem[\protect\BCAY{徳田 奥村}{徳田\JBA 奥村}{1998}]{tokuda:98}
徳田昌晃,奥村学 \BBOP 1998\BBCP.
\newblock \JBOQ
  日本語から手話への機械翻訳における手話単語辞書の補完方法について\JBCQ\
\newblock \Jem{情報処理学会論文誌}, {\Bbf 39}  (3), \BPGS\ 542--550.

\bibitem[\protect\BCAY{Tokunaga, Tanaka, \BBA\ Kimura}{Tokunaga
  et~al.}{2003}]{tokunaga:03}
Tokunaga, T., Tanaka, H., \BBA\ Kimura, K. \BBOP 2003\BBCP.
\newblock \BBOQ Paraphrasing Japanese noun phrases using character-based
  indexing\BBCQ\
\newblock In {\Bem Proceedings of the 2nd International Workshop on
  Paraphrasing: Paraphrase Acquisition and Applications {\rm (}IWP\/{\rm )}},
  \BPGS\ 80--87.

\bibitem[\protect\BCAY{徳永}{徳永}{2002}]{tokunaga:02:bachelor}
徳永泰浩 \BBOP 2002\BBCP.
\newblock \JBOQ 取り立て詞に着目した否定表現の言い換えと意味解析\JBCQ\
\newblock 九州工業大学情報工学部知能情報工学科卒業論文.

\bibitem[\protect\BCAY{Tomuro \BBA\ Lytinen}{Tomuro \BBA\
  Lytinen}{2001}]{tomuro:01:a}
Tomuro, N.\,\BBA\ Lytinen, S.~L. \BBOP 2001\BBCP.
\newblock \BBOQ Selecting features for paraphrasing question sentences\BBCQ\
\newblock In {\Bem Proceedings of the 6th Natural Language Processing Pacific
  Rim Symposium {\rm (}NLPRS\/{\rm )} Workshop on Automatic Paraphrasing:
  Theories and Applications}, \BPGS\ 55--62.

\bibitem[\protect\BCAY{鳥澤}{鳥澤}{2002}]{torisawa:02:a}
鳥澤健太郎 \BBOP 2002\BBCP.
\newblock \JBOQ 教師無し学習による名詞句の言い換え\JBCQ\
\newblock \Jem{言語処理学会第8回年次大会発表論文集}, \BPGS\ 323--326.

\bibitem[\protect\BCAY{Torisawa}{Torisawa}{2002}]{torisawa:02:b}
Torisawa, K. \BBOP 2002\BBCP.
\newblock \BBOQ An unsupervised learning method for associative relationships
  between verb phrases\BBCQ\
\newblock In {\Bem Proceedings of the 19th International Conference on
  Computational Linguistics {\rm (}COLING\/{\rm )}}, \BPGS\ 1009--1015.

\bibitem[\protect\BCAY{土屋 黒橋}{土屋\JBA 黒橋}{2000}]{tsuchiya:00}
土屋雅稔,黒橋禎夫 \BBOP 2000\BBCP.
\newblock \JBOQ MDL原理に基づく辞書定義文の圧縮と共通性の発見\JBCQ\
\newblock \Jem{情報処理学会研究報告, NL-140-7}, \BPGS\ 47--54.

\bibitem[\protect\BCAY{土屋, 佐藤, 宇津呂}{土屋\Jetal }{2004}]{tsuchiya:04}
土屋雅稔,佐藤理史,宇津呂武仁 \BBOP 2004\BBCP.
\newblock \JBOQ 機能表現言い換えデータからの言い換え規則の自動生成\JBCQ\
\newblock \Jem{言語処理学会第10回年次大会発表論文集}, \BPGS\ 492--495.

\bibitem[\protect\BCAY{内元, 黒橋, 長尾}{内元\Jetal }{1996}]{uchimoto:96}
内元清貴,黒橋禎夫,長尾眞 \BBOP 1996\BBCP.
\newblock \JBOQ 日本語文生成における語彙選択に必要な要因とその性質\JBCQ\
\newblock \Jem{情報処理学会研究報告, NL-116-21}, \BPGS\ 143--150.

\bibitem[\protect\BCAY{内山 石崎}{内山\JBA 石崎}{2003}]{uchiyama:03}
内山清子,石崎俊 \BBOP 2003\BBCP.
\newblock \JBOQ 複合動詞の多義性解消のための意味解析法\JBCQ\
\newblock \Jem{言語処理学会第9回年次大会発表論文集}, \BPGS\ 163--166.

\bibitem[\protect\BCAY{内山 Baldwin}{内山\JBA Baldwin}{2004}]{uchiyama:04}
内山清子,Baldwin, T. \BBOP 2004\BBCP.
\newblock \JBOQ 機械学習を用いた複合動詞の多義性解消\JBCQ\
\newblock \Jem{言語処理学会第10回年次大会発表論文集}, \BPGS\ 741--744.

\bibitem[\protect\BCAY{上田 小山}{上田\JBA 小山}{2000}]{ueda:00}
上田良寛,小山剛弘 \BBOP 2000\BBCP.
\newblock \JBOQ 共通意味断片の抽出による複数文書要約\JBCQ\
\newblock \Jem{言語処理学会第6回年次大会発表論文集}, \BPGS\ 360--363.

\bibitem[\protect\BCAY{Voorhees}{Voorhees}{2001}]{voorhees:01}
Voorhees, E.~M. \BBOP 2001\BBCP.
\newblock \BBOQ Overview of the {TREC} 2001 question answering track\BBCQ\
\newblock In {\Bem Proceedings of the 10th Text Retrieval Conference {\rm
  (}TREC 2001\/{\rm )}}, \BPGS\ 42--51.

\bibitem[\protect\BCAY{Wahlster}{Wahlster}{2000}]{verbmobil:00}
Wahlster, W.\BED\ \BBOP 2000\BBCP.
\newblock {\Bem Verbmobil: foundations of speech-to-speech translation}.
\newblock Springer.

\bibitem[\protect\BCAY{若尾, 江原, 白井}{若尾\Jetal }{1997}]{wakao:97}
若尾孝博,江原暉将,白井克彦 \BBOP 1997\BBCP.
\newblock \JBOQ テレビニュース番組の字幕に見られる要約の手法\JBCQ\
\newblock \Jem{情報処理学会研究報告, NL-122-13}, \BPGS\ 83--89.

\bibitem[\protect\BCAY{Walker}{Walker}{1993}]{walker:93}
Walker, M.~A. \BBOP 1993\BBCP.
\newblock \BBOQ When given information is accented: repetition, paraphrase and
  inference in dialogue\BBCQ\
\newblock In {\Bem Proceedings of Linguistics Society of America Annual
  Meeting}.

\bibitem[\protect\BCAY{Wanner}{Wanner}{1994}]{wanner:94}
Wanner, L. \BBOP 1994\BBCP.
\newblock {\Bem Current issues in {M}eaning-{T}ext {T}heory}.
\newblock Pinter Publishers.

\bibitem[\protect\BCAY{Watanabe, Kurohashi, \BBA\ Aramaki}{Watanabe
  et~al.}{2000}]{watanabe:00}
Watanabe, H., Kurohashi, S., \BBA\ Aramaki, E. \BBOP 2000\BBCP.
\newblock \BBOQ Finding structural correspondences from bilingual parsed corpus
  for corpus-based translation\BBCQ\
\newblock In {\Bem Proceedings of the 18th International Conference on
  Computational Linguistics {\rm (}COLING\/{\rm )}}, \BPGS\ 933--939.

\bibitem[\protect\BCAY{山口, 乾, 小谷, 西村}{山口\Jetal }{1998}]{yamaguchi:98}
山口昌也,乾伸雄,小谷善行,西村恕彦 \BBOP 1998\BBCP.
\newblock \JBOQ 前編集結果を利用した前編集自動化規則の獲得\JBCQ\
\newblock \Jem{情報処理学会論文誌}, {\Bbf 39}  (1), \BPGS\ 17--28.

\bibitem[\protect\BCAY{山本 松本}{山本\JBA 松本}{2001}]{YamamotoKaoru:01:b}
山本薫,松本裕治 \BBOP 2001\BBCP.
\newblock \JBOQ 統計的係り受け結果を用いた対訳表現抽出\JBCQ\
\newblock \Jem{情報処理学会論文誌}, {\Bbf 42}  (9), \BPGS\ 2239--2247.

\bibitem[\protect\BCAY{山本}{山本}{2001}]{YamamotoKazuhide:01}
山本和英 \BBOP 2001\BBCP.
\newblock \JBOQ 換言処理の現状と課題\JBCQ\
\newblock \Jem{言語処理学会第7回年次大会ワークショップ論文集}, \BPGS\ 93--96.

\bibitem[\protect\BCAY{Yamamoto}{Yamamoto}{2002a}]{YamamotoKazuhide:02:c}
Yamamoto, K. \BBOP 2002a\BBCP.
\newblock \BBOQ Machine translation by interaction between paraphraser and
  transfer\BBCQ\
\newblock In {\Bem Proceedings of the 19th International Conference on
  Computational Linguistics {\rm (}COLING\/{\rm )}}, \BPGS\ 1107--1113.

\bibitem[\protect\BCAY{Yamamoto}{Yamamoto}{2002b}]{YamamotoKazuhide:02:d}
Yamamoto, K. \BBOP 2002b\BBCP.
\newblock \BBOQ Acquisition of lexical paraphrases from texts\BBCQ\
\newblock In {\Bem Proceedings of the 2nd International Workshop on
  Computational Terminology {\rm (}CompuTerm\/{\rm )}}, \BPGS\ 22--28.

\bibitem[\protect\BCAY{Yoshikane, Tsuji, Kageura, \BBA\ Jacquemin}{Yoshikane
  et~al.}{2003}]{yoshikane:03}
Yoshikane, F., Tsuji, K., Kageura, K., \BBA\ Jacquemin, C. \BBOP 2003\BBCP.
\newblock \BBOQ Morpho-syntactic rules for detecting Japanese term variation:
  establishment and evaluation\BBCQ\
\newblock \Jem{自然言語処理}, {\Bbf 10}  (4), \BPGS\ 3--32.

\bibitem[\protect\BCAY{吉見 佐田}{吉見\JBA 佐田}{2000}]{yoshimi:00:a}
吉見毅彦,佐田いち子 \BBOP 2000\BBCP.
\newblock \JBOQ 英字新聞記事見出し翻訳の自動前編集による改良\JBCQ\
\newblock \Jem{自然言語処理}, {\Bbf 7}  (2), \BPGS\ 27--43.

\bibitem[\protect\BCAY{吉見, 佐田, 福持}{吉見\Jetal }{2000}]{yoshimi:00:b}
吉見毅彦,佐田いち子,福持陽士 \BBOP 2000\BBCP.
\newblock \JBOQ 頑健な英日機械翻訳システム実現のための原文自動前編集\JBCQ\
\newblock \Jem{自然言語処理}, {\Bbf 7}  (4), \BPGS\ 99--117.

\end{thebibliography}

\appendix

\section{語彙・構文的言い換えの分類}
\label{app:taxonomy}


これまでの事例研究の中で扱われてきたさまざまな種類の言い換えや,言語学
の分野で示されてきた交替現象,表現の使い分けなどの分析結果を集めた.
さらに,それぞれの言い換えを実現するための課題を考察し,(a)言い換えの
スコープ,(b)内容表現か機能表現か,(c)必要な語彙知識の種類,という観点
から分類した.





\subsection{節間の言い換え}
\label{ssec:category1}

2つ以上の節にまたがる言い換えである.\refex{dc2mc},\refex{cleft}のよ
うな言い換えでは,主題が変更するため,それにともなって対応する名詞述語
表現が必要になる.一方,\refex{adv_clause},\refex{conjunction}のよう
な言い換えでは,節間の修辞的関係を表す接続詞を改めて選択しなければなら
ない.このように,節間の言い換えでは,節間の順序や関係が変化するため,
結束性の評価が必要になる.
\numexs{dc2mc}{
\item[] \hspace{-6mm}\emph{連体節主節化}
~\cite{chandrasekar:96:a,dras:99:a,nogami:01}
\item[s.] 昨年,区制施行70周年という大きな節目を\emph{迎えた本区}は,
新たな10年に向けて順調な区政運営をスタートいたしました.
\item[t.] 昨年,\emph{本区は}区制施行70周年という大きな節目を
\emph{迎えました.そして,}新たな10年に向けて順調な区政運営をスター
トいたしました.}
\numexs{cleft}{
\item[] \hspace{-6mm}\emph{分裂文から非分裂文への言い換え}
~\cite{sunagawa:95,dras:99:a}
\item[s.] 今週当選した\emph{のは},奈良県の男性\emph{でした}.
\item[t.] 今週\emph{は},奈良県の男性\emph{が}当選し\emph{まし
た}.}
\numexs{adv_clause}{
\item[] \hspace{-6mm}\emph{連用節・並列節の分割}
~\cite{takeishi:92,kouda:01,mitamura:01}
\item[s.] 情報化に向けての前向きな意見が多くを占めています\emph{が},
情報格差などの不安もみられます.
\item[t.] 情報化に向けての前向きな意見が多くを占めています.\emph{し
かし},情報格差などの不安もみられます.}
\numexs{conjunction}{
\item[] \hspace{-6mm}\emph{接続表現の言い換え}~\cite{miyajima:95:b}
\item[s.] 用紙は各事務所に置いてあります\emph{から},どしどし意見を
お寄せください.
\item[t.] 用紙は各事務所に置いてあります\emph{ので},どしどし意見を
お寄せください.}



\subsection{節内の言い換え}
\label{ssec:category2}

\refex{comparison}の主題交替や,\refex{voice_alternation}の格交替など,
操作の対象が節内で閉じている言い換えである.変換パターンのバリエーショ
ンはそれほど多くなく,人手で書き尽くせる程度のように見えるが,変換パター
ンによっては適用の可否が語に依存するため,その判断に必要な語彙知識をい
かにして発見・構築するかが課題となる.また,視点のような対人関係的意味
や主題/陳述構造のような文脈レベルの意味が変化するため,これを捉えるモ
デルを形式化する必要もある.
\numexs{negation}{
\item[] \hspace{-6mm}\emph{否定表現の言い換え}
~\cite{hayashi:91,kondo:01,iida:01,tokunaga:02:bachelor}
\item[s.] 返信\emph{しない}と,申込みは取り消され\emph{ます}.
\item[t.] 返信\emph{する}と,申込みは取り消され\emph{ません}.}
\numexs{comparison}{
\item[] \hspace{-6mm}\emph{比較表現の言い換え}
~\cite{kondo:01,saito:02:b}
\item[s.] \emph{隣町は我が町より}山林資源が\emph{乏しい}.
\item[t.] \emph{我が町は隣町より}山林資源が\emph{豊かだ}.}
\numexs{voice_alternation}{
\item[] \hspace{-6mm}\emph{態・使役の交替}
~\cite{yamaguchi:98,kondo:01,murata:02}
\item[s.] 今年は湾岸などの都市基盤の\emph{整備が}\emph{行われまし
た}.
\item[t.] 今年は湾岸などの都市基盤の\emph{整備を}\emph{行いました}.}
\numexs{transitivity_alternation}{
\item[] \hspace{-6mm}\emph{動詞交替(自他)}
~\cite{levin:93,kageyama:01,kondo:01}
\item[s.] 無制限な個人情報の収集に一定の\emph{制限を}\emph{加える}.
\item[t.] 無制限な個人情報の収集に一定の\emph{制限が}\emph{加わる}.}
\numexs{locative_alternation}{
\item[] \hspace{-6mm}\emph{動詞交替(壁塗り/場所格)}
~\cite{levin:93,kageyama:01,ogawa:04}
\item[s.] 課長は\emph{大きな杯に}\emph{日本酒を}満たした.
\item[t.] 課長は\emph{大きな杯を}\emph{日本酒で}満たした.}
\numexs{lightverb}{
\item[] \hspace{-6mm}\emph{機能動詞結合の言い換え}
~\cite{oku:90,muraki:91,iordanskaja:91,morita:94,dras:99:a,kaji:04:a,fujita:04:d}
\item[s.] \emph{住民の熱心な要請を受け},工事を中止した.
\item[t.] \emph{住民に熱心に要請され},工事を中止した.}
\numexs{donatory}{
\item[] \hspace{-6mm}\emph{授受の構文の言い換え}
~\cite{masuoka:94,inui:99:a}
\item[s.] 区民の健康保持の立場から,清掃活動を\emph{頑張ってくれてい
る}.
\item[t.] 区民の健康保持の立場から,清掃活動を\emph{頑張っている}.}
\numexs{enable_verb}{
\item[] \hspace{-6mm}\emph{可能動詞の言い換え}
~\cite{miyajima:95:a,inui:99:a}
\item[s.] 電車は込んでいたけど吊革に\emph{掴まれた}.
\item[t.] 電車は込んでいたけど吊革に\emph{掴まることができた}.}
\numexs{mod_alternation}{
\item[] \hspace{-6mm}\emph{修飾要素の交替}~\cite{miyajima:95:b}
\item[s.] ブロック,レンガなど\emph{大きくて}重いものは,ひも掛けを
してそのまま出してください.
\item[t.] ブロック,レンガなど\emph{大きく}重いものは,ひも掛けをし
てそのまま出してください.}
\numexs{quantity}{
\item[] \hspace{-6mm}\emph{数量詞の遊離}~\cite{muraki:91}
\item[s.] \emph{一件の}開示請求がありました.
\item[t.] 開示請求が\emph{一件}ありました.}












\subsection{内容語の複合表現の言い換え}
\label{ssec:category3}

複数の内容語が複合表現を形成する場合,接続詞や格助詞,共通の動詞などの
関係は明示的には現れない.この隠れている関係を明示的に示すように言い換
えれば(下記s{\ra}t),読解支援という目的に対しては有効だと考えられる.
一方,要約のような応用には,複合表現への言い換え(下記t{\ra}s)の方が
有効である.これらの言い換えの可否は,構成素となる内容語からそれらの関
係(複合表現における結び付き方)が容易に連想できるかどうかにも依存する.
\numexs{compound_word}{
\item[] \hspace{-6mm}\emph{複合名詞の分解・構成}
~\cite{sato:99,kimura:02,takeuchi:02,ohashi:04}
\item[s.] \emph{区政施行70周年}という大きな節目を迎えました.
\item[t.] \emph{区の行政が施行されてから70周年}という大きな節目を迎え
ました.}
\numexs{relative_clause}{
\item[] \hspace{-6mm}\emph{「AのB」{\lra}連体節}
~\cite{kurohashi:99:b,kataoka:00,torisawa:02:a}
\item[s.] 奈良県知事選への\emph{出馬の挨拶}を行った.
\item[t.] 奈良県知事選への\emph{出馬を表明する挨拶}を行った.}
\numexs{compound_verb}{
\item[] \hspace{-6mm}\emph{複合動詞の分解・構成}
~\cite{uchiyama:03,uchiyama:04}
\item[s1.] 人に\emph{頷きかける}.
\item[t1.] 人に\emph{向かって頷く}.
\item[s2.] 夕飯を\emph{食べ過ぎた}.
\item[t2.] 夕飯を\emph{必要以上に食べた}.}

\subsection{機能語/モダリティの言い換え}
\label{ssec:category4}

機能語相当表現(助詞・助動詞)やモダリティのレベルの言い換えは,上で示
した言い換えに比べて語彙的な性格が強く,局所的な情報を参照するだけで言
い換えられるものも多い.このレベルの言い換えを実現するには,同義の機能
語/モダリティ表現をグループ化して辞書を整備するとともに,個々の言い換
えで生じる意味差分をどのように計算するかが課題となる.

\numexs{functional_expression}{
\item[] \hspace{-6mm}\emph{機能語相当表現の言い換え}
~\cite{morita:89,iida:01,kurokawa:03:master,matsuyoshi:04:a,tsuchiya:04}
\item[s.] \emph{市民はもとより}全国に誇れるものにしていきたい.
\item[t.] \emph{市民だけでなく}全国に誇れるものにしていきたい.}
\numexs{move_emphasis}{
\item[] \hspace{-6mm}\emph{取り立て助詞の移動}
~\cite{kinsui:00,tokunaga:02:bachelor}
\item[s.] ご飯は食べずに,辛いおかずを\emph{食べてばかり}いた.
\item[t.] ご飯は食べずに,辛い\emph{おかずばかりを}食べていた.}
\numexs{delete_emphasis}{
\item[] \hspace{-6mm}\emph{助詞による特徴づけの削除}~\cite{morita:94}
\item[s.] 人口は一時10万人を\emph{超えこそしたが},今は7万人まで減少
している.
\item[t.] 人口は一時10万人を\emph{超えたが},今は7万人まで減少してい
る.}
\numexs{modality}{
\item[] \hspace{-6mm}\emph{伝達のモダリティ}
~\cite{miyajima:95:a,inui:99:a,shuto:01}
\item[s.] 秋には紅葉を見に多くの人が\emph{集まるという}.
\item[t.] 秋には紅葉を見に多くの人が\emph{集まるそうだ}.}
\numexs{honorifics}{
\item[] \hspace{-6mm}\emph{敬語表現の言い換え}
~\cite{ohtake:01:b,ohno:03}
\item[s.] お支払いの方は\emph{いかがなさいますか}.
\item[t.] お支払いの方は\emph{どうなさいますか}.}
\numexs{changing_style}{
\item[] \hspace{-6mm}\emph{文体の変換}
~\cite{oizumi:03,kaji:04:c}
\item[s.] 不本意\emph{だが}仕方\emph{ない}.
\item[t.] 不本意\emph{ですが}仕方\emph{ありません}.}

\subsection{内容語句の言い換え}
\label{ssec:category5}

内容語の言い換え表現は個々の単語ごとに記述する必要があるため,パラレル
コーパスのアラインメント,シソーラス中の同概念語,国語辞典の語釈文など
の既存の資源から機械的に収集する手段が検討されている(\sec{knowledge}).



\numexs{oneword_noun}{
\item[] \hspace{-6mm}\emph{名詞の言い換え}
~\cite{fujita:01,pearce:01,YamamotoKazuhide:02:d,okamoto:03:b}
\item[s.] 太平洋を一望する桂浜公園の\emph{丘陵}に完成.
\item[t.] 太平洋を一望する桂浜公園の\emph{高台}に完成.}
\numexs{oneword_verb}{
\item[] \hspace{-6mm}\emph{動詞の言い換え}
~\cite{kondo:97:a,kondo:99,kaji:01,torisawa:02:b,kaji:03:b}
\item[s.] 警官が犯人を\emph{逮捕する}.
\item[t.] 警官が犯人を\emph{捕まえる}.}


\subsection{慣用表現の言い換え}
\label{ssec:category6}

構成語の変形では生成できない特有の言い回しの言い換えは,言い換え表現対
を辞書に蓄える必要がある.
\numexs{idiom}{
\item[] \hspace{-6mm}\emph{慣用句}
~\cite{morita:94,mitamura:01}
\item[s.] 「ひかり都市」として\emph{脚光を浴びる}こととなりました.
\item[t.] 「ひかり都市」として\emph{注目される}こととなりました.}
\numexs{acronym}{
\item[] \hspace{-6mm}\emph{表記のゆれ/略語}
~\cite{wakao:97,terada:01,sakai:03}
\item[s.] 多くの市民が\emph{原発}の建設に反対している.
\item[t.] 多くの市民が\emph{原子力発電所}の建設に反対している.}
\numexs{metonymy}{
\item[] \hspace{-6mm}\emph{換喩}
\item[s.] \emph{シェイクスピアを}読む.
\item[t.] \emph{シェイクスピアが書いた本を}読む}



\newcommand{\email}[1]{}

\begin{biography}
\biotitle{略歴}
\bioauthor{乾 健太郎}{
1967年生.
1995年東京工業大学大学院情報理工学研究科博士課程修了.
同年より同研究科助手.
1998年より九州工業大学情報工学部助教授.
1998年〜2001年科学技術振興事業団さきがけ研究21研究員を兼任.
2001年より奈良先端科学技術大学院大学情報科学研究科助教授.
現在にいたる.
博士(工学).
自然言語処理の研究に従事.
情報処理学会,人工知能学会,ACL各会員.
\email{inui@is.naist.jp}.}
\bioauthor{藤田 篤}{
1977年生.
2000年九州工業大学情報工学部卒業.
2002年同大学大学院情報工学研究科博士前期課程修了.
同年,奈良先端科学技術大学院大学情報科学研究科博士課程入学.
現在にいたる.
自然言語処理,特にテキストの自動言い換えの研究に従事.
情報処理学会,ACL各学生会員.
\email{atsush-f@is.naist.jp}.}

\bioreceived{受付}
\bioaccepted{採録}
\end{biography}

\end{document}
