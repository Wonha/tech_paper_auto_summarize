\documentstyle[epsf,jnlpbbl]{jnlp_j_b5}



\setcounter{page}{45}
\setcounter{巻数}{7}
\setcounter{号数}{2}
\setcounter{年}{2000}
\setcounter{月}{4}
\受付{1999}{9}{28}
\採録{2000}{1}{7}

\setcounter{secnumdepth}{2}
\setlength{\parindent}{\jspaceskip}

\newcommand{\tabtopsp}[1]{}
\setlength{\doublerulesep}{0.4pt}      



\title{新聞の用字の面による変動と時系列変動}
\author{久野 雅樹\affiref{UEC}}

\headauthor{久野}
\headtitle{新聞の用字の面による変動と時系列変動}

\affilabel{UEC}{電気通信大学 電気通信学部 人間コミュニケーション学科}
{Department of Human Communications, 
 Faculty of Electro-Communications, 
 The University of Electro-Communications}

\jabstract{
1991年から1997年までの毎日新聞7年分の電子テキスト(約3.4億文字)
を対象に,使用されている文字種すべて(5,726;空白文字を除く)
について,その出現率(出現頻度)が,面種(e.g., 解説面,スポーツ面,
社会面),月次,年次の3つの要因に関して,どの程度まで系統的な変動を
示すかを検討した.5,726文字種のうち,16の面種間による出現率の差は
69.2\%で,月次による出現率の差は20.3\%で,年次による出現率の差は
43.9\%で認められた.低出現率の文字(0.001‰未満)を除いた2,732
文字種では,さらに変動は顕著で,面種差は98.4\%で,月次差は
33.5\%で,年次差は76.0\%で認められた.このように,紙面の種類と
時系列によって,新聞の文字使用が系統的に変動することが,広範に
確認された.こうした語彙表現に関わる変動現象は,大量のテキストに
基づいて文字や単語の計量を行うような研究ではあまり関心が払われて
こなかったが,変動のもつ規則性は,それ自体,精細な分析の対象と
なりうるものである.}

\jkeywords{文字使用頻度,毎日新聞,語彙の変動,面種変動,時系列分析,季節変動,トレンド}

\etitle{Page-type and time-series variations \\ 
of a newspaper's character occurrence rate}
\eauthor{Masaki Hisano\affiref{UEC}} 

\eabstract{Using 1991--1997 Mainichi Shimbun Newspaper's CD-ROMs containing 
about 340 million characters, nature of character use was explored.  
Significant differences of mean occurrence rates among 16 
page-types (e.g., editorial, sports, local) were observed 
in 69.2 \% of 5,726 character types that covered all the cases 
in the corpus except a space character.  
Similarly, 20.3 \% of those showed significant 
month-level (seasonal) variations of occurrence rates, 
and 43.9 \% showed significant year-level variations (trends).  
Limited to frequent 2,732 character types that severally 
accounted for more than 0.001 ‰ of the corpus, these tendencies became 
more clearly: rates of character types that showed significant 
variations of occurrence rates by page-types, month-levels, 
and year-levels were 98.4, 33.5, and 76.0 \% respectively.  
These results suggest that there could be a vast range of 
systematic variations in lexical use, which have been 
overlooked in simple summing-up of mass corpuses.}

\ekeywords{character occurrence rate (frequency), 
Mainichi Shimbun Newspaper, variation of lexicon, 
page-type variation, time-series analysis, seasonal variation, trend}

\begin{document}
\maketitle


\section{はじめに}

  語彙とは“ある言語に関し(その一定範囲の)あらゆる語を
一まとめにして考えた総体”(水谷,1983, p.1)のことである.
したがって,日本語なら日本語という特定の1言語に限っても,
その内容は一まとめにくくる際の観点をどのように設定するか
によって変化しうる.大きく見れば,語彙は時代の進行にそって
変化するし,同時代の語彙にも地域,職業,社会階層などによって
集団としての差異が存在する.細かく見てゆくならば,
個人によっても語彙は違うであろうし,
特定の書籍,新聞,雑誌等,言語テキストそれぞれに独自の語彙が
存在すると言ってよい.さらに,個人で見ても,その語彙のシステム
(心内語彙=mental lexicon)は,発達・学習によって大きく変化し,
さらに特定の時点における特定の状況に対応した微妙な調整によって,
常に変化しつづけていると考えることができる.
こうした語彙の多様性は,ごく簡単に整理すれば,経時的な変動と,
それと連動しつつ,表現の主体,内容,形式のバラエティに
主に関わる共時的な変動という,縦横の軸からとらえることができる.

  本研究では,新聞という一般的な言語テキストを対象に,
経時的,共時的の両面に関して語彙の系統的な変動を抽出することを試みる.
具体的には1991年から1997年までの毎日新聞7年分の電子化テキストを
用いて,そこで使われている全文字種の使用状況の変動について,
面種と時系列の2つの面から調べる.

  毎日新聞を対象にしたのは,紙面に含まれる記事の内容が広く,
難度も標準的であり,現代日本の一般的な言語表現を観察するのに
適していると考えられること,
面種等のタグ付けが施されたテキストファイルが利用できること,
研究利用条件が整っていて,実際に多くの自然言語処理研究で
利用されているため,知見の蓄積があることなどによる.

  語彙について調べることを目標に掲げる研究で,文字を分析単位と
している理由は,
日本語の場合,文字が意味情報を多く含んでいて単語レベルに近いこと
(特に漢字の場合),
単語と違って単位が明確なために処理が容易であること,
異なり数(タイプ)が多すぎないので悉皆的な調査も可能であることである.
目標と方法の折り合うところとして,
文字という単位にまず焦点を当てたのである
(電子テキストを用いて,日本語の文字頻度の本格的な計量を行った
例としては,横山,笹原,野崎,ロング,1998 がある).

  面種による変動を調べるのは,1種類の新聞の紙面で,どの程度,
語彙(本研究では実際には文字)の内容に揺れ(変位)があるかを吟味することを
ねらいとする.
全体で一まとめにして“毎日新聞の語彙”とくくれる語彙の集合を
紙面の種類によって下位カテゴリに分割しようとする試みであるとも言える.
経済面とスポーツ面とで,使われている語彙に差異があるだろうと
いうこと自体は,容易に想像がつくが,
本研究では,こうした差異がどの程度まで広範に確認されるかを検討する.

  テキストのジャンルによる使用語彙の差を分析したものとして,
国立国語研究所(1962),Ku\v{c}era \& Francis(1967)を挙げることができる.
前者は,1956年に刊行された90の雑誌から抽出した50万語の標本に対して
評論・芸文,庶民,実用・通俗科学,生活・婦人,娯楽・趣味の
5カテゴリを設定し,後者は1961年にアメリカ合衆国で出版された
本,新聞,雑誌等から抽出した100万語のコーパスに
報道記事,宗教,恋愛小説等の15カテゴリを設定している.
ただし,いずれも対象としているテキストの種類が多岐にわたるだけに語彙の差が
検出しやすい条件にあると見ることができるが,
カテゴリ間に見られる差についての検討は十分なものではない.
本研究の場合,新聞1紙の中でどの程度の内容差を検出できるかを,
文字という単位で悉皆的に分析するところに特色がある.

  語彙の時系列的な変動に関しては,世代,時代といった長い時間幅であれば,
様々に研究されているが,7年間という,この種の分析としては短い時間幅で,
どのような変動が観察されるかを詳細に分析するところに本研究の独自性がある.
本研究では,7年全体での変動としてのトレンドに加えて,
循環性のある変動として月次変動(季節変動)も調べる.
時系列的な微細な分析は,経済,自然の分野では多くの実例があるものの,
言語現象への適用は未開拓である.実際,言語テキストの月単位,
年単位でのミクロな分析は,近年の大規模電子コーパスの整備によって
ようやく現実的なものとなったという段階にあるにすぎない.
新聞での用字パタンに時系列な変動が存在すること自体は予想できる.
たとえば,“春”という文字は春に,“夏”という文字は夏に多用されそうである.
しかし,そもそも,“春”なら“春”の字がある時期に多用されるといっても,
実際のパタンがどうであるのか,また,こうした季節変動が他の文字種を含めて
どの程度一般的な現象であるのかというのは調べてみなければわからない.

  時系列変動の中でも,月次変動に関しては,筆者らは既に新聞の
カタカナ綴りを対象とした分析
(久野,野崎,横山,1998; 野崎,久野,横山,1998),
新聞の文字を対象とした分析(久野,横山,野崎,1998)を報告している.
そこでは,月ごとの頻度プロフィールの相関をベースに,
隣接月次の単語・文字の使用パタンが類似したものとなり,
12ヵ月がほぼ四季と対応する形でグルーピングできることを示したが,
本報告では,個々の文字をターゲットとして時系列的変動の検出を試みる.

  この時系列変動の調査は,トレンドに関しては,近年における
日本語の変化の大きさについて考えるための基礎資料となるという点からも
意味が大きい.
また,月次変動,季節変動については,日本の場合,風土的に四季の変化が
明確であり,その変化をめでる文化をもち,様々な生活の営みが1年の
特定時期と結びついているという点から,分析の観点として有効性が高いことが
期待される.

  以下では,面種変動,時系列変動という順序で,分析結果を報告する.
実際の分析は,両方を行き来し,重ね合せながら進めたが,面種変動の方が
結果が単純であり,また,時系列変動の分析では面種要因を考慮に入れる
操作をしているという事情による.


\section{研究1:面種変動の分析}

\subsection{方法}

  分析用のコーパスとして,毎日新聞7年分(1991〜1997)
のテキストを収録したCD-ROM(毎日新聞社,1996--1998)を用いた.
このCD-ROMでは,テキストは,全角文字(S-JISコード)で収録されているが,
本研究では,見出しと本文の,空白文字を除く全角文字すべて
(5,726字種,のべで約3億4千万文字)を分析の対象とした.

  5,726文字種の文字カテゴリ(e.g., ひらがな,JIS漢字第1水準)の
内訳を7年間全体での出現率(‰)のレベルと連関させて表1に示した.
7年間での出現率は,年次ごとに集計した出現率の平均である.
単純に7年分の合計文字数に対する出現率を求めてもほとんど変わらないが,
表2に示したように年次が下るにつれてテキストの規模が大きくなるので,
それを調整したものである.
単語,文字の頻度分布で一般に観察されるように,
高出現率のものの異なり数は少なく,低出現率のもののそれは多い.
文字カテゴリごとに見ると,ひらがな,カタカナ,アラビア数字の出現率は高い.
アルファベット,漢字第1水準がそれに次ぎ,
漢字第2水準では大半が0.001‰以下である.
その他の記号等は,高出現率のものと低出現率のものに分かれている.
なお,実質的な文字ではなく,独特の使われ方(段落冒頭の字下げ,
見出しの区切り,表で桁を揃える等)をする空白文字は
出現率算出時の分母からも除外している
(分析からの空白文字の除外は,以下のすべての集計で同様).
\input{tab1}
\input{tab2}

  集計区分とした面種は,毎日新聞社によって7年すべてに共通して
設定されている16カテゴリ
(1面,2面,3面,解説,社説,国際,経済,
特集,総合,家庭,文化,読書,科学,芸能,スポーツ,社会)である.
面種による文字の延べ数は表2を参照.社会面,総合面のように規模が
大きい面種もあれば,科学,読書,文化のように規模の小さい面種もある.
文字種ごとに各年次の面種別出現頻度を求め,
テキストの規模の異なる面種間で比較可能なように,
集計単位ごとに出現率(‰)を算出した.

  そして,文字種ごとに,面種別年次出現率に対して,
面種を要因として1元配置分散分析を行った.
これは,7つの年次で出現率が他の面に比べて安定して低い面や
高い面種が(1以上)ある場合,それを面種変動として
検出しようというものである.
こうして,面種が16水準で各水準の繰り返し数は7という単純なモデルで,
文字種と同数の5,726回の分散分析を実行した.

  なお,単語の出現頻度(出現率)のデータを用いて,分散分析,
相関分析ほかのパラメトリックな分析をする際には,
しばしば事前処理として対数変換を施すが,本研究では,
もとの出現率に基づく分析結果を一貫して報告する.
実際の分析は,対数変換を施した場合についても行っているが,
変換をしてもしなくても,結果に違いがあまり認められないことが
確認されているので(分散分析で要因の効果が有意となる文字種の割合は,
百分率で数ポイント程度変わるに過ぎない),
表示しやすく理解しやすい,もとの出現率に基づいた結果を示すこととした.

\subsection{結果と考察}

  分散分析の結果を 表3に示した.$F$値の自由度は $(15,96)$ である.
低出現率の文字種を除く大多数の文字種で,
面種による出現率の変動が認められた.
1.000‰以上の高出現率の文字種では,すべて0.1\%水準で有意である.
出現率でほぼ上位半分に相当する0.001‰以上の文字種(2,732種)で見ても,
実に95.5\%が0.1\%水準で有意,
97.2\%が1\%水準までで有意,
98.4\%が5\%水準までで有意となっている.
さらに全5,726文字種で見ても,58.2\%が0.1\%水準で有意となり,
1\%水準までで63.4\%,5\%水準まででは69.2\%が有意となる.
低出現率の文字種で面種による比率の差が有意水準に達しない文字種には,
出現総数が少なすぎるので,統計的検定の対象とするのが適当でない場合が
多いと考えられる.
したがって,実質的には,ほとんどの文字種で面種変動が見られると
してよいと推測される.
面種変動は,ひらがな,カタカナ,記号類等の漢字以外の文字種にも
広く見られるが,これらの場合も,単語との直接的な対応は漢字の場合より
弱いものの,やはり使用されている語彙の系統的な変動を相当に反映していると
考えられる.
たとえば,ある面種ではカタカナで表記される単語が多用されているといった
現象を反映するようなケースである.
\input{tab3}


  面種差の詳細に関しては,個々の文字種ごとの吟味に加えて,面による
比率の高低による文字種のクラスタリングや,出現比率プロフィールの
面種間類似性の分析等を通して,理解を深めてゆくことができるが,
本研究では,面種差の例を示すとともに,面種差をもたらしている独自な
面種に関する予備的調査の結果を示すにとどめる.
\begin{figure}[htb]  
\begin{center}
\epsfile{file=clip000.eps,scale=0.55}
\end{center}
\caption{面種により文字出現率が異なる例:“学”\\
出現率は7年分の年次出現率の平均でエラーバーは標準偏差}
\label{図1}
\end{figure}

  面種差の具体例としては,“学”のケースを示す.
面種別年次出現率を分散分析すると,$F(15,96) = 42.03$ ( $p<.0001$ )となる.
実際の面種別の平均出現率は図1の通りで,見て分かるように,
科学面での出現率が他を大きく引き離している
(Tukeyの方法により多重比較を行うと,他の15面種全てと5\%水準で
差があることが確認できる).
以下,読書面,文化面で出現率が高く,経済面,2面,国際面,芸能面,
1面では低い.
科学面,読書面,文化面といった教養的な内容の面で多く出現するのは
了解しやすく,経済面,国際面,芸能面といった,内容が独自で学問の関与が
薄そうな面で出現率が低いのも理解しやすい.一般的な内容の面が,
その中間にくるようだが,それらの面の間にも差が認められるところが興味深い.

  追加分析として,各面種の独自性の程度について調べるために,上記の分散分析で
$p$値が5\%未満の3,960文字種を対象として,面種ごとに出現率の順位(1〜16)の
標準偏差を求めた.この値が大きい場合,平均的なレベルからずれる文字種が
多いことになり,面種変動の原因となったケースが多いものと推測できる.
標準偏差が大きい面種を挙げると,スポーツ面(5.12),芸能面(5.01),
科学面(4.87),読書面(4.82),国際面(4.73)が上位5つとなる.
これらは掲載する内容の特定性が高い面であり,
他の面に比して出現率の高い文字種,低い文字種が多く存在することで,
文字使用パタンの面種差の大きな原因となっていると考えられる.
一方,標準偏差が小さい方を見ると,総合面(2.65),3面(3.52),
解説面(3.55),社会面(3.64),1面(3.70)の順となる.
内容が一般的な面種では,全体的に平均的なレベルで文字種が出現する
傾向にあると言える.

  以上のように研究1では,文字の使用状況という観点から,
毎日新聞紙面の言語表現の多様性の中に
面種に関連して系統的な成分が広範に含まれていることが確認された.
なお,分散分析によって面種変動を調べる場合,上述のような単純な
一元配置のデザインの他に,年次や月次の要因を含めることによって
分析の精度を高めることができるが,
上記の分析で,面種変動が遍在するということは既に十分に示されたので,
本稿ではこれ以上の分析報告は行わない.


\section{研究2:時系列変動の分析}

\subsection{方法}

  時系列分析では,一般に時系列データを,トレンド,循環変動,
季節変動,不規則変動などに分解することを通して,データの性質を
記述したり,将来の予測を行ったりする.
目的に応じて多様な時系列分析の技法が開発されているが,本研究では,
5,726系列という大量の時系列データに対して実行する初歩的な分析として,
月次変動と年次変動の有無と程度を分散分析と相関係数を通して検討した.

  まず,7年分の電子テキストで,5,726の文字種それぞれに対して,
月を集計単位として,出現率(‰)を求めた(84ヵ月分).

  この時系列上の84の出現率に対して,月次(12水準),年次(7水準)の
2要因の主効果を調べる分散分析を実行した.
これは,月次,年次に関して,相互に他方をブロック要因として,
乱塊法モデルによって分散分析を行ったものと言うことができる.
分析デザインとしては,面種の要因も組み込んだり,
月次・年次を別個に分析したりすることで,他にも様々な選択がありうるが,
時系列の2要因を同時に扱う最も単純なモデルを採用した.
2つの要因はカテゴリ変数により構成される独立した要因として扱っていることになる.
月次の主効果は広義の季節変動を反映するもので,
日常概念としての季節(いわゆる四季といったもの)から外れるものも
含めて12ヵ月以下の周期のサイクル現象を反映する.
年次の主効果が有意になった場合,トレンド,
それも期間が短いので1次のトレンドを主に反映すると考えられるが,
偶発的な変動を反映して有意となることもありうる
(偶発的な変動によって,本来存在するはずのトレンドが検出できずに,
非有意となることももちろんありうる). 

  分散分析に続いて,84の出現率の系列位置を連続的変数として扱い,
相関係数を用いてトレンドの分析を行った.これは,先の分散分析で
示された年次変動に関して,その程度を評価するための材料を提供する
とともに,分散分析では年次をカテゴリ変数としたために,
時系列の連続性の情報を用いていないという点を補うものである.

  具体的には,1991年1月を1,1991年2月を2として,以下,
1997年12月の84まで,7年間の84ヵ月に順次,自然数を与え,
それと月次出現率の相関係数(ピアソンの積率相関係数)を
算出した.トレンドの分析としては,2次以上の項も考慮して回帰分析を
行うことを検討したが,言語の長期変動を見るには7年間という期間は短く,
高次の成分を抽出することの意味は小さいと考えられるので,
1次の直線的な変動傾向のみを分析,考察の対象とすることにした.

  さらに,研究1の結果から,面種によって使用文字に変動があることが
広範に認められているので,面種別に,年次・月次の2要因の分散分析を
実行した.
これは,面種によって時系列変動の様相が違うかどうかを調べようとする
ものである.この目的を実現するためには,面種も分散分析のモデルに
要因として組み込んでしまう方法をとることもでき,その場合,
面種と時系列要因の交互作用の分析が中心となるが,
5,726セットを対象とするには,手続きも結果の表示も複雑である.
そこで,本研究では,それぞれの面種でどの程度,時系列変動が
見られるかを明らかにするために,はじめから面種を区別した上で,
2要因の分散分析を行うこととした.


\subsection{月次変動}

  まず,分散分析の結果に基づいて,月次変動について述べる.
分散分析での月次の主効果を表4に示した.$F$値の自由度は $(11,66)$ である.
\input{tab4}


  面種変動ほど顕著ではないものの,かなり広範に系統的な変動の存在を
確認することができる.出現率0.100 ‰以上の高出現率文字種では,
5\%水準までで有意になるものが41.6\%で,10\%水準の有意傾向まで含めれば
51.0\%と過半数の文字種に月次要因の効果が認められる.
低出現率の文字種では月次要因の効果の認められるものが減少する.
これは,面種変動のところでも述べたように,分散分析の適用が有効である
ほどの出現数がないことに主によると考えられる.
出現率0.001‰以上までとすると,5\%水準で有意が33.5\%,
10\%水準までで41.8\%となる.全文字種では,5\%水準までで有意に
なるのは20.3\%,10\%水準まででは26.6\%となる.

  月次変動のパタンを個々の文字種について見ると,いわゆる季節との
対応が想定できる変動も多く観察されるが,
その対応にはゆるやかに複数の月にまたがって峰ができるものもあれば,
特定の1つの月だけにピークがくるものもある.
また,多峰性のものも少なくない.
以下に,月次変動の具体的なパタンをいくつか例示する.

  季節性の変動の典型として,予想が確認されたという意味で
“春”“夏”“秋”“冬”の場合を図2に示した
(4文字種の月次要因の$F$値はそれぞれ,
$F(11,66)=33.70$, $65.23$, $23.06$, $36.24$;いずれも $p<.0001$ ).
ただこうした典型的なケースでも,パタンを微細に観察すると,
いくつかの特徴を指摘することができる.
たとえば,この4文字種は,四季それぞれを直接に示す文字種であり,
季節とほぼ対応しているものの,そのピークの間隔は3ヵ月ではない.
また,ピークのとがりも文字種によってかなり異なっている.
\begin{figure}[bt]  
\begin{center}
\epsfile{file=clip001.eps,scale=0.55}
\end{center}
\caption{文字出現率が季節に対応する月次変動を示す例}
\label{図2}
\end{figure}

\begin{figure}[bt]  
\begin{center}
\epsfile{file=clip002.eps,scale=0.55}
\end{center}
\caption{文字出現率が単峰性でない月次変動を示す例}
\label{図3}
\end{figure}

  四季に単純に対応しない月次変動が見られるものとして,
図3に単峰性でない月次変動を示す文字種の例を示した.
“貴”“撲”
(それぞれ $F(11,66)=8.41$,$16.55$;いずれも$p<.0001$ )
は奇数の月に出現率が高まるのだが,
これは大相撲の開催月に関連の記事が出ることによるものである.
“甲”( $F(11,66)=11.73$, $p<.0001$ )は3月と8月に山があり,
“誉”( $F(11,66)=23.75$, $p<.0001$ )は4月と11月に山がある.
“甲”のピークは春夏の高校野球甲子園大会に関連した記事に
よるもので,“誉”のそれは春秋の叙勲褒賞によるものである.

  以上のように,月次を単位とした分析を通して,
文字使用の循環的な変動が,“春”“夏”“秋”“冬”といった,
その存在が容易に予測できるものに限らず,かなりの割合の文字種に
おいて,多様な変動パタンで存在することが見出された.


\subsection{年次変動}

  続いて,年次変動について,分散分析と相関分析に基づいて検討する.

  分散分析の結果は,表4を参照.$F$値の自由度は $(6,66)$ である.
年次要因が有意となった文字種は月次要因が有意になったものよりも多い.
1.000‰以上の高出現率の文字種では94.0\%が5\%水準までで
年次の効果が有意である.以下,出現率レベルが下がるにつれて,
年次の効果が有意でない文字種,有意であってもその水準が低い
文字種の割合が高まっている.
特に,\hbox{0.001‰}未満の文字種では,85.3\%が5\%水準に達していない
(10\%水準にも達しないのが78.9\%).
しかし,全文字種でも,5\%水準で43.9\%,10\%水準の有意傾向まで
含めれば,ちょうど半分の50.0\%で年次変動が見られたことになる.

  分散分析で見出された年次変動には,方法のところで述べた通り,
多様なパタンが含まれる.その中の単純な上昇傾向,
下降傾向を見るために求めた,84の月次順序と出現率の相関係数は,
表5を参照.表は相関係数の値で区分してまとめてあるが,
相関係数の有意水準と対応させると,
相関係数の絶対値が .35 で0.1\%水準の有意となる.
以下,.28 で1\%水準の有意,.21 で5\%水準の有意,
.18 で10\%水準の有意傾向となる.
5,726文字種全体での結果を記すと,
0.1\%水準で有意な相関が見られる文字種が 997(17.4\%),
1\%水準で有意な文字種が 483( 8.4\%),
5\%水準で有意な文字種が552( 9.6\%),
10\%水準で有意な傾向が認められる文字種が408( 7.1\%),
10\%水準に至らず非有意な文字種が3,286( 57.4\%)となる.
分散分析の結果と比較して,有意となってもその水準が低くなっている
文字種が多く,非有意な文字種も多くなっているのは,想定通り,
変動の中でも1次のトレンドのみにしぼって見ていることに
主によっていると考えられる.
これは見方を変えれば,単調増加,単調減少以外の変動パタンが
7年間のうちでかなりの程度観察されるということでもある.
このような現象は,大きな事件や連載企画などによって生じうることである.
\input{tab5}

  個々に見ると,正の相関係数で上位は,アラビア数字と一部の記号類
(e.g.,“【” “】” “!”)が占めている.
一方,負の相関係数では,絶対値の大きさで漢数字が上位に並ぶ.

  図4に例として“1”と“一”の出現率の変化を示した.
相関係数は,“1”の場合, $r=.91$,“一”の場合,$r=-.83$ である.
図から明らかなように,1996年の4月に“一”の出現率が大きく低下し,
同じ時点で“1”は上昇している.
これは,この時期,数字表記の原則に方針変換があったことに
対応していて,1996年の4月1日付で1面に掲載された社告に,
“数字を読みやすく”と題して,
“漢数字が原則だった数字の表記をきょうから洋数字に変えます.”とある.
ただし,“1”については,この1996年4月の増加以外でも,
全体的な増加傾向が著しく,さらに“1”と“一”を合わせた出現率も
増加しているので,漢数字からアラビア数字への切り替え以外にも,
紙面変化が生じていることをうかがわせる.
\begin{figure}[htb]  
\begin{center}
\epsfile{file=clip003.eps,scale=0.6}
\end{center}
\caption{文字出現率がトレンドを示す例(数字)}
\label{図4}
\end{figure}

\begin{figure}[htb]  
\begin{center}
\epsfile{file=clip004.eps,scale=0.6}
\end{center}
\caption{文字出現率がトレンドを示す例(数字以外)}
\label{図5}
\end{figure}

  トレンドの別の例として,数字,記号以外で時系列との相関が
正負それぞれで最も強かった文字種を図5に示した.
“思”は $r=.85$,“午”は $r=-.85$ である.
なお,この2例の示すトレンドについては現象の記述にとどめ,
その解釈は本稿では行わない.
一般に,こうしたトレンドが見られる背景としては,記事内容や
紙面構成の変化,文体の変化,表現方針の変化等があることが
推測されるものの,月次変動に比べて,その意味付けは容易でなく,
実際のテキストに即して詳細な吟味が必要である.

  以上の分析を通して,具体的な内容理解は今後の検討にまつところが
多いものの,文字使用の年次変動が多数の文字種において観察された.
年次変動は,月次変動よりも多くの文字種で認められているが,
これは近年における言語表現の変化の大きさを反映していると考えられる.


\subsection{面種別の月次変動と年次変動}

  面種別に月次,年次を要因とする分散分析を実行した結果は表6を参照.
\input{tab6}

  月次変動の主効果を見ると,有意な文字種が全般に少なくなっている.
この結果に関して,面種区分をすることでテキストの規模が小さくなった
ことが当然影響している.
特に,科学面,文化面,読書面といったテキスト規模が他より小さい
面種では影響が大きいと推測される.
ただし,全面種で集計した場合に認められた月次変動が,特定面種に
大きく依存しているために,面種別にしてしまうと,その特定面種の
ほかでは変動が見られなくなってしまうというケースも多いと考えられる.

  月次変動が見られる文字種の多さでは,スポーツ面が際立っている.
全文字種の34.3\%が5\%水準までで有意である.
先に月次変動の例として,相撲関連,野球関連の文字種
(“貴”“撲”“甲”)を挙げたが,他にもスポーツ関連で,
月次変動を示す文字種が多数あることが分かる.
他には,特集面,社会面,経済面で,面種区別をせず全体で分析した
場合と同じくらいの割合の文字種に有意な変動が見られる.
特集面や社会面ではイベント,行事,生活,気象等に関連した記事を
通して,季節の変動が紙面に表れると考えられる.
経済面については,経済現象における季節変動の存在は広く知られて
いることであり,それが紙面にも反映していると見ることができるだろう.
一方,科学面,文化面,読書面,解説面等では月次変動が確認された
文字種が少ない.
これらは,学術的な,あるいはその時々の状況から独立した,
普遍性の高い情報を主に掲載する面種とまとめることができよう.

  年次変動も面種区別をしなかった場合よりも全般に見られる
割合が少なくなっているが,社説面,総合面,経済面,2面では,
面種区分をしなかった場合と同程度の割合の文字種で有意な変動が見られる.
経済面のほかは月次変動の少ない面種であるが,いずれも政治情勢,
社会情勢の変化に関連して時事性の高い面であると言えるだろう.
この4面種に続いて,家庭面,スポーツ面において変動を示す文字種の割合が高い.
ここでも,科学面,読書面,文化面では,有意な変動が観
察される文字種が少ない.
掲載する情報のカテゴリ,スタイルが安定した面種であると言えそうだ.

  こうした面種による時系列変動の分析結果を先の面種変動の分析と
対応させてみると,他の面種に比した場合の独自性の高低と,
面種内での時系列変動の大小の組み合わせで,
大きく4タイプに分けることができる.
それぞれ面種の例とともに挙げれば,第1に,スポーツ面のように,
面種としての独自性が高く,しかも時系列変動も大きい面種がある.
第2に,科学面,読書面のように,面種としての独自性は高いが,
時系列変動は小さい面種がある.
第3に,面種としての独自性は低く一般的な紙面だが,
大き目の時系列変動を示す面種に総合面がある.
\hbox{第4}に,面種としての独自性も低く,時系列変動も小さい面種として
解説面が挙げられる.


\section{全体的考察}

  以上の分析から,面種,時系列の双方の要因に関して,近年の
毎日新聞7年間での文字使用には,系統的な変動が広範に
観察されることが確認された.
5,726文字種を対象として分散分析を行ったところ,有意水準5\%で,
面種(16水準)による出現率の差は69.2\%で,
月次(12水準)による差は20.3\%で,
年次(7水準)による差は43.9\%で認められた.
低出現率の文字種(0.001‰未満)を除いた2,732文字種では,
さらに変動は顕著で,面種差は98.4\%で,月次差は33.5\%で,
年次差は76.0\%で認められた.


  重要なのは,これだけの系統的な変動が,文字種という単位で,
1種類の新聞の7年という限られた期間のテキストを対象として
検出されたということである.
別の言い方をすれば,1紙の語彙に限っても,多様な組織的変動を
含んだものとして,その語彙の体系を把握することができる
ということである.
この結果は,直接に単語を対象として,毎日新聞以外の新聞,
さらには新聞以外の言語テキストをも対象に含めて,
より長期にわたって調べた場合,語彙の系統的な変動は,
さらに多様に検出できるであろうことを,容易に推測させる.
語彙の使用に系統的な変動があること自体は,経験的にも
了解されることだが,それを定量的に評価し,非常に広範に
存在することを確認した点で,本研究の意味は大きい.

  本研究で見出された語彙の系統的変動の存在は,語彙調査,
特に頻度の集計において,暗黙に想定されがちな,スタティックな
語彙観の見直しを促す.
調査の時期,対象とするテキストによって,頻度(出現率)にかなりの
差を生じうることに注意が必要なのは言うまでもないとして,
バランスに配慮した大量のコーパスから単純に平均的な頻度(出現率)
を求めるような方法にも改善の余地があると言える.
現実の語彙が,多くの系統的な変動を含むのであれば,平均は情報の
一面でしかない.
変動は,必ずしも排除すべき誤差ではなく,それ自体が重要な情報と
なりうるのである.
今後,単語や文字の頻度に関して,系統的変動を視野に入れて,
多面的に計量,分析を進めることが望まれる.

  系統的変動に注目することが重要であり,有効であるというのは,
頻度が関わる言語現象の基礎研究に関してそう言えるにとどまらず,
頻度調査の利用という見地からもそのように考えることができる.
たとえば,頻度の情報は辞書や学習用の単語リストの編集に
しばしば用いられるが,その際にも頻度を多面的に見ることは有効だろう.
平均的な頻度集計では上位にならないけれども重要性の高い単語があり,
逆に頻度は高いけれども重要性の低い単語もあるというような,
経験的に知られている現象も,変動現象という観点からは自然なものと
見ることができる.
こうしたケースについて,人手による調整はもちろん不可欠だが,
集計対象による変動を考慮に入れることで,頻度情報の有用性は
改善される可能性がある.

  頻度調査の応用に関して改善が見込まれる例として,
もうひとつ言語素材を用いた心理実験を挙げることができる.
文字,単語などを素材とする心理実験では頻度の統制が大きな課題となり,
その際,平均的な集計値
(e.g., 国立国語研究所,1970; Ku\v{c}era \& Francis, 1967)を
利用するのが一般的であるが,
変動をも考慮に入れた素材の用意や結果の分析も有効であろう.
Gernsbacher(1984)は,頻度表では低頻度であるのに,
主観的にはそう感じられない単語があることを指摘して,
経験的親近性(experiential familiarity)を統制する必要があるとしている.
しかし,親近性は主観的な評定によるものであり,
指標の内容にあいまいな部分がある.
親近性の指標としての意味を見直すことと並行して,
頻度自体の情報の充実を進めてゆくというのが望ましい方向であると考えられる.
一般に心理学の分野では,心内語彙に関する研究は盛んであるが,
外的に存在する語彙の性質についての考察が弱い.
心理実験で調べる心内語彙と,言語テキストの分析で調べる外的語彙は,
本来双方向的であるのだから,双方の対応関係にもっと関心をもつ必要があり,
それを実践するために本稿のような分析を応用することができるだろう
(多面的に集計した単語出現率と心内辞書のパフォーマンスの関連を見た
研究の例として,Hisano, 1999 がある).


\section{おわりに}

  今後の課題を2点挙げる.

  第1に,分析の精緻化,体系化を進めること.
分散分析,相関係数における有意水準を手がかりとして検出した面種変動,
時系列変動が,実際にどのような変動のパタンを示すかについて,
本研究では,代表的なものを例示するにとどまった.
個々のケースについては,各種の時系列分析法の適用や,実際のテキストでの
用例の吟味によって,分析を洗練,深化してゆくことができる.
また,変動のパタンを,語彙全体の構造の中で,より直接的に把握,表現する
ために,変動パタンの分類,クラスタリングを本格的に行うことが望まれる.

  第2に,対象の拡張を進めること.本研究は文字を単位とする分析を行ったが,
これは語彙変動の抽出を念頭に置いてのものであり,
今後,単語レベルでの変動を検討することが重要な課題であるのは言うまでもない.
また,コーパスの拡張は,本研究で観察された変動現象の一般性を
評価する意味で重要である.
まず,今回の分析で扱った期間が短い(特にトレンドを見るには)ことと,
電子テキストがまだ整備途上にあることを考えても,毎日新聞を対象に
継続的な観測を行うことは重要である.
もちろん,毎日新聞以外のコーパス(新聞以外を含む)への拡張も重要であり,
その際には,英語をはじめとする他言語での検討も必要であるだろう.




\acknowledgment

  本研究は日本心理学会第63回大会において発表した研究
(久野,野崎,横山,1999)に,大幅な拡充を加えたものである.
電子化コーパスを用いた日本語語彙の研究を共同で進めている
横山詔一(国立国語研究所),野崎浩成(愛知教育大学)の両先生には,
本研究においても,多くのご示唆,ご助力をいただきました.
記して深い感謝の意を表します.




\bibliographystyle{jnlpbbl}
\bibliography{v07n2_03}

\begin{biography}
\biotitle{略歴}
\bioauthor{久野 雅樹}{
1987年東京大学教育学部教育心理学科卒業.
1995年同大学院博士課程修了.
1996年電気通信大学電気通信学部講師,現在に至る.
専門は認知心理学,言語心理学.
日本心理学会,日本教育心理学会,日本認知科学会,
人工知能学会,言語処理学会,
日本行動計量学会,計量国語学会各会員.}


\bioreceived{受付}
\bioaccepted{採録}

\end{biography}

\end{document}


