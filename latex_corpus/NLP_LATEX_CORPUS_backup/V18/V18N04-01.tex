    \documentclass[japanese]{jnlp_1.4}
\usepackage{jnlpbbl_1.3}
\usepackage[dvips]{graphicx}
\usepackage{amsmath}
\usepackage{hangcaption_jnlp}
\usepackage{udline}
\setulminsep{1.2ex}{0.2ex}
\let\underline


\Volume{18}
\Number{4}
\Month{September}
\Year{2011}

\received{2010}{9}{29}
\revised{2011}{1}{23}
\rerevised{2011}{3}{28}
\accepted{2011}{5}{25}

\setcounter{page}{323}


\jtitle{階層意味論に基づいた心的態度のアノテーション}
\jauthor{小橋 洋平\affiref{Author_1} \and 坂野 達郎\affiref{Author_2}}
\jabstract{
本論文では,日本語コーパス内の命題に書き手の心的態度をアノテーションする基準として階層意味論を検討する.階層意味論とは「命題」と「モダリティ」からなる普遍的な意味構造を規定する概念である.モダリティは心的態度を指す概念として知られているが,既存研究で取り上げられている文法論のモダリティでは対象が文法形式に限定されてしまう.対して階層意味論で定義される「モダリティ」は意味論上の概念であるため形式上の制約が少なく,心的態度を網羅的にアノテーションするという目的により適した概念といえる.ただし,階層意味論で規定される心的態度を母語話者が一貫性を持ってアノテーションできるのか実証的に確認されているとは言い難い.そこで,母語話者に新聞の社説記事に対するアノテーションを実際に行ってもらい,その一貫性を調査した.その結果,4 名の間での Fleiss の $\kappa$ 係数は,真偽判断系,価値判断,拘束判断でそれぞれ 0.49,0.28,0.70 となった.真偽判断系と価値判断は一致度が高いとは言い難いが,真偽判断系に関しては,述語または後続表現の語彙的機能の影響で真偽を読み取ることが困難な命題を取り除くと 0.58 まで改善した.加えて,語彙,文法形式によって明示的に心的態度が表されていない命題でも 0.50,0.28,0.53 の値を示した.このことから,心的態度を表す語句,文法形式が明示されていなくてもある程度の一貫性が得られることが伺える.
}
\jkeywords{心的態度, アノテーション, モダリティ, 階層意味論}

\etitle{Annotation for Writer's Attitude \\
	based on Hierarchical Semantics}
\eauthor{Kobashi Yohei\affiref{Author_1} \and Sakano Tatsuro\affiref{Author_2}} 
\eabstract{
In this paper, we examine Hierarchical Semantics for an annotation of
a writer's attitude toward a proposition in Japanese text. 
Hierarchical Semantics defines a universal semantic structure 
composed of a ``proposition'' and a ``modality''. 
A modality is known as a linguistic concept of writer's attitude 
and there are some linguistic modality theories which have been adopted in previous works. 
But the theories cover only grammatical forms 
because they are grammatical theories. On the other hand,
Hierarchical Semantics defines modality not as syntax but
semantics. The semantic definition is more useful to cover all of
writer's attitudes than the syntactic definition since there is hardly
any excess formal condition. To confirm whether we can annotate
the semantic information consistently, we examined the degree of
consistency among Japanese native speakers' annotations of truth,
value and deontic judgments on Japanese newspaper editorials. The
result shows that Fleiss's kappa coefficients of truth, value and
deontic judgments between them are 0.49, 0.28 and 0.70. The one
of truth judgment can be increased to 0.58 by removing
propositions which are inappropriate to ask the truth-value. In
addition, these coefficients were 0.50, 0.28 and 0.53 even when we
removed propositions which syntactically depend on a modal form or a subjective
expression. It means that Japanese native speakers understand
a writer's attitude consistently
even when it cannot be explained by lexcical or grammatical rules.
}
\ekeywords{attitude, annotation, modality, Hierarchical Semantics}


\headauthor{小橋,坂野}
\headtitle{階層意味論に基づいた心的態度のアノテーション}

\affilabel{Author_1}{筑波大学}{University of Tsukuba}
\affilabel{Author_2}{東京工業大学}{Tokyo Institute of Technology}



\begin{document}
\maketitle


\section{はじめに}

\subsection{背景と目的}

我々が記述や発話によって伝える情報は客観的な事柄のみではない.事柄が真なのか偽なのか,事柄が望ましいか望ましくないか,といった心的態度も併せて伝達する.言語学,日本語学にはこのような心的態度に対応する概念として「モダリティ」または「様相」と呼ばれるものが存在する.

モダリティは,文を構成する主要な要素として規定されている概念である.モダリティ論では「文は,客観的な事柄内容である『命題』と話し手の発話時現在の心的態度(命題に対する捉え方や伝達態度)である『モダリティ』からな」るという規定が多くの学者に受け入れられてきた\cite{Book_01}\footnote{以後,修飾語句なしに「心的態度」と記述するときは書き手の発話時現在の命題に対する捉え方や伝達態度のことを指す.}.そして,活用形と「べき(だ)」「だろう」「か」といった助動詞や終助詞および,それらの相当語句がモダリティに属する文法形式とされている.これらの文法形式はコーパスに心的態度の情報をアノテーションする上で有効な指標になると考えられる.

ただし,前述の文法形式をアノテーションするだけでは心的態度を網羅することはできない.「ことを確信している」「と非常に良さそうだ」等,文法形式以外にも心的態度を表す表現は存在する.そのことは心的態度のアノテーションを目的とする既存研究で指摘されており,それらの研究では「拡張モダリティ」\cite{Article_01}「確実性判断」\cite{Article_02}といった文法形式以外も含む新たな概念が提案されている.

しかし,このように対象を拡張すると,モダリティの持つアノテーションに有利な特徴が失われてしまう.モダリティであれば,文ごとに特定の文法形式の有無を目安にしてアノテーションの判定をすればよい.対して,拡張モダリティや確実性判断にはこのような明確な判断基準がない.よって,作業者によって基準がぶれてアノテーションが安定しない可能性がある.

これに対し本論文では,「階層意味論」で規定される「モダリティ」の概念を用いることで,母語話者の判断による一貫したアノテーションが可能であると考える.階層意味論とは,言語普遍的な意味構造を規定する意味論上の概念であり,この意味構造によって従来の文法論では解釈が困難な複数の言語現象に自然な解釈を与えることができる.この階層意味論で規定される「モダリティ」は文法論上の概念ではない.拡張モダリティや確実性判断と同じく文法形式ではない心的態度も対象とするため,心的態度の網羅という目的に適う概念といえる.

ただし,階層意味論の研究は主に英語の事例を扱っており,日本語の事例研究は限定的である.そのため日本語における普遍性が実証的に確かめられているとは言い難い.そこで,4名の母語話者に新聞の社説記事から「モダリティ」を読み取ってもらう調査を行い,母語話者間での回答の一致度を見る.本論文では「階層意味論の『モダリティ』が普遍性のある概念であれば母語話者間で『モダリティ』に対する認識の仕方に違いは出ない」という前提のもと,母語話者間の一致度を通して普遍性を評価する.

以下,2節で,自然言語処理,言語学,日本語学それぞれにおける「モダリティ」の扱いを概観し,その違いが心的態度のアノテーションに及ぼし得る問題点を論じる.次に3節で,本論文で検討する階層意味論について説明する.そして4節で,日本語の母語話者を対象に,新聞の社説記事から階層意味論に基づき規定された心的態度を読み取ってもらう調査を行い,その判断に対する母語話者間の一致度を示すとともに不一致を引き起こす要因について論じる.最後に,5節でまとめと今後の課題について述べる.


\vspace{-0.5\baselineskip}
\subsection{用語に関する注意事項}

「文」「命題」「モダリティ」といった用語は,特定の言語形式を指す場合と,その形式で表現される意味内容を指す場合とがある.文法論,意味論と自然言語処理との間で横断的な議論を行う場合は,どちらの用法で用いているのか明記しないと混乱を招く恐れがある.そこで,本論文における各用語の便宜的な用法をここで示す.

まず,文については「書き言葉において句点\footnote{文体によっては改行や句点以外の記号で代用されることもある.} で区切られる統語上の言語単位」を指すことにする.話し言葉は本論文では取り上げない.

次に,モダリティは「文法論でモダリティとして扱われている表現の集合\footnote{この集合を厳密に定義する既存研究はないが,日本語のモダリティについては,日本語学でモダリティを体系的に論じた書籍である宮崎他(2002)と日本語記述文法研究会(2003)のいずれかでモダリティとして扱われているかどうかを基準とする.}」を指すことにする.文法論では「モダリティ」が文法形式を指す場合とその機能を指す場合とがあるが,本論文ではもっぱら前者として用いる.

この規定は心的態度とモダリティを明確に区別することを意図している.本論文では,心的態度はアノテーションすべき対象なのに対し,モダリティはあくまでアノテーションの目安となる統語上の特徴の1つということになる.

そして,命題は「補足語+述語」\inhibitglue\footnote{述語とは「動詞,形容詞,形容動詞または『名詞+助動詞「だ」』(+ヴォイス)(+テンス)(+アスペクト)」を指す.},「補足語+述語+形式名詞」および「補足語+述語+形式名詞」に言い換え可能な「(連体修飾語+)名詞」を指す\footnote{形式名詞の規定は\cite{Book_25}に従う.}.例えば「彼が渋谷まで買い物に行った」「A銀行の破たん」といった表現が挙げられる\footnote{名詞の言い換えは文脈に依存するため,ここで「A銀行の破たん」の言い換えを一意に定めることはできないが,文脈さえ定まれば母語話者は困難なく言い換えることができると考える.具体的には4.2の手順2で論ずる.}.

ただし,階層意味論で「命題」や「モダリティ」と呼ばれているものは意味構造を記述するための概念であり,ここで述べた用法とは異なる.そこで「命題’」「モダリティ’」と,「’」をつけて区別する.



\section{自然言語処理とモダリティ論との相違点}

\subsection{自然言語処理でのモダリティと心的態度}

自然言語処理の分野では,拡張モダリティや確実性判断の概念が出てくる前から,大規模コーパスに対しモダリティのアノテーションが行われてきた.英語コーパスに品詞と統語構造の情報を付与したPenn Treebank \cite{Article_03}では,品詞のタグにModal Verbというカテゴリを設け,法助動詞に対しそのタグを付与している\cite{Inproc_03}.また,日本語の京都大学テキストコーパス\footnote{http://www-lab25.kuee.kyoto-u.ac.jp/nl-resource/corpus.html} では,形態素,構文タグとは別にメモ書きとしてモダリティのタグが存在する.ただし,タグの対象となっている表現は「こと」「もの」「ところ」「わけ(だ)」「ほど(だ)」という形式名詞を含む表現が述語に後接しているものに限られる.そのことから,実質,形式名詞が助動詞に近い働きをしていることを示すタグとなっている.

このように既存の大規模コーパスにはモダリティタグが存在するが,これらのモダリティタグは文法形式に限定されており心的態度を網羅するものではない.1節で述べたように,心的態度を網羅することを目的とした研究では拡張モダリティや確実性判断といった新たな概念が提示されている.拡張モダリティでは,「てほしい」に言い換え可能な表現を文法形式かどうかに関わらず全て対象としている\cite{Article_01}.また確実性判断では,命題に対する書き手,話し手の確信度の度合いを読み取るための指標として,モダリティのみならず叙実述語や仮定表現も対象としている\cite{Article_02}.さらに,モダリティ論によるモダリティの細分類に基づいて心的態度のカテゴリを定め,それらを「拡張モダリティ」と呼んでアノテーションする研究も存在する\cite{Inproc_04}.モダリティ論を参考にはしているが,こちらの「拡張モダリティ」でも文法形式に限定せずにアノテーションを行っている.以上,既存の大規模コーパスにモダリティタグがある一方で,心的態度を解析するために新たなアプローチでのアノテーションが行われているのが現状と言えよう.


\subsection{言語学,日本語学のモダリティ論}

現在の言語学,特に文法論におけるモダリティは,文を構成する要素のうち命題とは異質なものを命題と区別するための概念であり,心的態度または ``speaker (or writer)'s attitude'' という規定はモダリティが持つべき前提条件というわけではない.ただし,日本語学では90年代まで心的態度という規定が基本的な位置付けにあったのも事実である.その背景には,日本語学のモダリティ論が一般言語学とは異なる独自の発展をしたことがある.以下,モダリティ論が形成された経緯を日本語学を中心に概観する.

一般言語学で文が命題とモダリティからなるという規定が広まったのは,Fillmoreが示した ``$sentence \rightarrow proposition + modality$'' というモデル\cite{Inbook_01}の影響が大きいと思われる.ただし,Fillmore自身は ``modality'' に高い関心はなく,文からテンス,アスペクト,ムードを分離した上で命題のみに対象を絞って議論するためにこのようなモデルを提示している.そのため,``modality'' の詳細については論じておらず,テンス,アスペクトを含む雑多なものと見なしている.

一方,Fillmoreのモデルを採用しつつモダリティの方に焦点を当てる研究も行われてきた.それらの研究の関心は,主に英語の法助動詞や法助動詞に類する他言語の文法形式にある.それらの研究では認識(epistemic),当為(deontic)といったモダリティの細分類を規定し,その分類に基づいて文法形式の機能を整理している(例えばPalmer 1986, 2001).この傾向は日本語学でも同様で,モダリティの細分類を通して1節で挙げた助動詞,終助詞といった文法形式の特徴を論じている(例えば益岡 1991).

ただし,日本語学のモダリティ論の場合,陳述論の影響により文の成立要件に関する論考も展開しているという特徴がある.陳述論とは,言語表現が文として成立するために必要な機能を山田孝雄が「陳述」と呼んだことに端を発する.そして,時枝誠記,渡辺実へと引き継がれた議論がモダリティ論に影響を与えたとされている\cite{Book_06}.以下,時枝以降の論考の流れを簡単に述べる.

時枝は,文の構造を記述するモデルとして客体的な内容である「詞」を主体的な作用である「辞」が包むという入れ子構造を提案している\cite{Book_19}.詞に属する表現には「山」「運動会」などの名詞や「彼が買い物へ行く」「本が安い」などの命題が,辞に属する表現には「(山)が」「(安い)よ」などの助詞全般や「だろう」などの助動詞といった機能語が挙げられる.その上で,助動詞,終助詞などの「文末辞」が陳述の役割を果たし,述語とその補足語で構成される詞を包むことで文が成立するという考えを示している.

それに対し渡辺は,詞と辞の入れ子構造だけでは文の成立条件を適切に記述できないことを指摘している\cite{Book_23,Book_18}.例えば「命題+助動詞」という構造は,時枝のモデルでは助動詞が辞として陳述の働きをすると解釈される.しかし,実際には文だけではなく従属節も同様の構造を持ち得る.陳述の機能を助動詞に求めている限り,文と従属節との違いを解釈することはできない.そこで,渡辺は辞の概念を見直し,事柄を描き上げる機能を「叙述」と呼んで文を完結させる機能の陳述と区別している.

そして,これらの陳述論は日本語文が統語的な階層構造を持つという主張へと展開する.渡辺は,助動詞や終助詞の相互承接には規則性があり,叙述の機能を持つものが前,陳述の機能を持つものが後ろに来ると述べている.また南不二夫は,渡辺を含めた語順に関する研究を総括し,4段階の階層構造にまとめている\cite{Book_20,Book_10}.南によると,この階層性は主観性という指標で捉える事ができる.具体的には,述語に近い位置に現れる表現は客観的,遠い位置の表現は主観的という傾向があるとしている.助動詞,終助詞で見ると,「れる」「ない」「た」などは客観的表現,「だろう」「か」などは主観的表現に分類される.

以上,Fillmoreのモデルおよび日本語学の陳述論と階層構造について概観した.日本語のモダリティ論で提示されている文構造のモデルは詞と辞の概念や南の階層構造と類似している.実際,モダリティとして扱われる文法形式の多くは,辞や陳述,主観的表現として扱われる文法形式と一致する.一般言語学では必ずしも重視されてこなかった心的態度という規定も,日本語学では陳述論の流れを引き継ぐことで長く受け入れられたものと考えられる.



\subsection{モダリティと心的態度との関連性}

モダリティ論では前述の通り文法形式を対象とする研究が主流である.一方,心的態度は文法形式に限らず様々な言語表現によって表現され得る.このことから,心的態度という概念をモダリティの規定として採用すべきではないという主張がされている\cite{Book_06,Article_04}.この問題を回避するためのアプローチは大きく2つ存在する.ひとつは一般言語学,日本語学の両方で見られるもので,心的態度という規定自体を見直すというものである.心的態度の代わりに採用される規定の代表的なものとしては,非現実(irrealis)\footnote{irrealisは伝統的にはunrealを指す概念であり非現実と訳されるのが一般的なため本論文でも非現実と訳すが,non-assertiveに近い概念と考える研究者もおり\cite{Book_07},検討の余地が残る.} が挙げられる\cite{Book_07,Book_06,Article_04}.

例えば,「昨今の学生は勤勉だ。」と記述した場合,書き手は「昨今の学生が勤勉である」ことを真だと断言しているが,「昨今の学生は勤勉かもしれない。」や「昨今の学生は勤勉でなければいけない。」では,「学生が勤勉である」ことを断言しているとは言い切れない.「昨今の学生は勤勉かもしれないが,そうでないかもしれない。」「昨今の学生は勤勉ではないが,勤勉でなければならない。」といった言明が可能なことがそれを裏付けている.

このように,「かもしれない」「なければならない」といったモダリティの文法形式は非現実と対応した表現であるといえる.この非現実の規定を採用すると,例えば日本語の場合,感嘆詞,間投詞,係助詞,敬語に関わる表現などがモダリティに含まれないことが明確となり,対象を大幅に絞り込むことができるため,心的態度の規定と比べ形式上の雑多さは大幅に解消される.

もうひとつは日本語学で見られるもので,心的態度という意味論上の規定だけではなく統語論上の条件も規定に加えるというものである.例えば,心的態度を表す複合形式をモダリティと見なす基準として文法化しているかどうかを挙げる研究が存在する\cite{Article_05}.また,モダリティの基本的性格として主観性を挙げた代表的な研究者の1人である益岡隆志は,これまでのモダリティ論では主観的な表現が全てモダリティであるかのような誤解を招いているとした上で,モダリティかどうかの判断には文構成における位置付けも考慮する必要があると指摘している\cite{Book_08}.

以上,あくまで形式と意味との相関を明らかにすることが主眼の言語学,日本語学では,形式との高い相関が見出せない場合は規定自体を見直すということが行われる.規定が文法形式を定めるという一方的な関係ではなく,図1のように適切な関係性が得られるよう規定と形式が相互に影響し合う関係だといえよう.

\begin{figure}[t]
 \begin{center}
  \includegraphics{18-4ia911f1.eps}
 \end{center}
 \caption{言語学,日本語学における規定と形式との相互依存関係}
 \label{fig:one}
\end{figure}


\subsection{2つの相違点}

以上,自然言語処理とモダリティ論それぞれにおけるモダリティと心的態度に対する態度を見てきた.心的態度のアノテーションを考える上で考慮すべき相違点を整理すると次の2点を挙げることができる.

\begin{itemize}
\item 自然言語処理では命題を解析単位としているのに対し,モダリティ論,特に日本語学のモダリティ論では文が主な考察単位となる.
\item 心的態度は,自然言語処理ではアノテーションすべき対象であるのに対し,モダリティ論では形式との相関を見る中で放棄され得る規定である.
\end{itemize}

自然言語処理の応用研究のうち,拡張モダリティなどの新たな概念を立ててそのアノテーションを試みている研究は,命題で表される事柄に対する書き手,話し手の真偽判断や価値判断の情報を特定することを目的としている.そのため,アノテーションの単位は命題となる.近年,盛んに行われている医学,生物学系テキストに対する確実性判断のアノテーションでも,当初は文がアノテーションの単位だったが\cite{Inproc_01},その後,心的態度を表す表現とそのスコープとなる命題へのアノテーションが提案されている\cite{Inproc_02}.

対して,モダリティ論では原則,Fillmoreのモデルに象徴されるように文が分析単位となる.特に日本語学のモダリティ論は,陳述論の影響により文のあり方を論じるという文脈で語られることが多い.加えて,日本語の従属節にはモダリティとされる文法形式の出現に制限があるため\cite{Book_10},文以外の単位はそもそも議論の対象になりにくい.

この結果,心的態度を表すにもかかわらずモダリティ論では取り上げられない表現が数多く出てくる.例えば,「主観述語」\cite{Article_06}や ``mental state predicates'' \cite{Book_11}と呼ばれる心的態度を表す用言が挙げられる.以下に例を示す.
\begin{enumerate}
  \item 今回の洪水で予想以上の被害を受けたのは\underline{事実だ}。
  \item 東京での開催が決定したことを\underline{歓迎する}。
  \item 予定通りに到着することが\underline{望ましい}。
\end{enumerate}
その他にも前提,仮定,反実仮想,目的など従属節の種類に対応する心的態度もモダリティとは見なされない.

加えて,2.3で見たように,モダリティ論には心的態度という規定自体が適切ではないという意見もある.実際,モダリティ論で提示されている法則や傾向が,心的態度と形式との普遍的な関係を表すものとは限らない.例えば,力動的モダリティ(dynamic modality)と呼ばれる可能性や能力,性向を表すカテゴリ\cite{Book_07,Book_12}は心的態度の範疇に入るとは考え難い.英語のcanやmayなどは力動的モダリティとして機能することもあるため\cite{Book_12},「英語の法助動詞は心的態度を表す」という一般化は成り立たないことになる.

以上のことから,拡張モダリティ,確実性判断といった概念とモダリティとでは,対象となる表現の範囲が異なるというだけではなく,分析単位および概念自体に相違点があるといえる.



\section{階層意味論に基づく心的態度の規定}

\subsection{階層意味論の概要}

前節で論じた自然言語処理とモダリティ論との相違点は,モダリティ論の成果をアノテーションに応用することや,アノテーションの結果をモダリティ論によって評価する上での障害となる.ただし,モダリティ論には,数は少ないながらも意味論の立場から「モダリティ」(以後,1.2で述べたように「モダリティ’」と記述する)を論じている研究がある.意味論であれば議論の対象が文法形式に限定されるという制約は生じない.その中でも,階層意味論\cite{Book_13}は「話し手(本論文では書き手)の発話時点での心的態度」を典型的なモダリティ’の厳密な定義としている.階層意味論での命題’も同様に,述語と補足語そのものではなく述語と補足語によって表現可能な意味全般(以後,「事柄」と呼ぶ)となる.そこで本論文では,階層意味論で規定されるモダリティ’に着目し,心的態度をアノテーションする基準にモダリティ’の細分類を用いることを検討する.以下,この節では階層意味論について説明する.

中右は,文が命題’とモダリティ’との対立を軸とした一定普遍の意味構造を持つという階層意味論モデルを提唱している(図2).図2は,自然言語において枝分かれの左側の要素を出力するには右側の要素が必須であることを示している.このモデルに従うと,自然言語で談話モダリティ’を表現するときは必ず構文意味を伴い,文内モダリティ’を表現するときは必ず全体命題’を伴う.そして,発話意味は談話モダリティと構文意味から構成され,構文意味は文内モダリティ’と全体命題’から構成されるということも表している.なお,図2のモデルは2節で言及したモダリティ論の階層性\cite{Book_01,Book_04}とは明確に区別する必要がある.モダリティ論の階層性は文法形式の統語的な特徴,言い換えると語順に関する記述である.一方,階層意味論モデルは特定の言語の文法に捉われない普遍的な意味構造を記述するもので,本質的に異なるモデルと言えよう\footnote{\cite{Book_01}では,階層意味論モデルが仁田や益岡の想定する意味構造と同様であると論じている.しかし,仁田や益岡のモデルはあくまでも日本語の文法を記述することが目的である.言語普遍的な特徴も視野に入れているとはいえ,細分類の典型例が必ず文法形式によって与えられる点で階層意味論モデルとは大きく異なる.本稿では,文法論の立場から記述される階層性と階層意味論モデルは明確に区別されるべきだと考える.}.

\begin{figure}[t]
 \begin{center}
  \includegraphics{18-4ia911f2.eps}
 \end{center}
 \caption{階層意味論モデル(中右のモデルから一部抜粋)}
 \label{fig:two}
\end{figure}



\subsection{階層意味論と本論文との違い}

本論文では心的態度のアノテーション方法を原則,中右の階層意味論に基づいて定めるが,次の2点は階層意味論と異なる.
\begin{itemize}
\item 対象を文ではなく事柄(命題’)とする.
\item 瞬間的現在と持続的現在とを区別せず,発話時点であれば典型的と考える.
\end{itemize}
中右は文論の立場から階層意味論を論じており,観察している事例は文に限定されている.それに対し本論文では,コーパス内に現れる全ての事柄に対する心的態度を明らかにするためにアノテーションは事柄単位で行う\footnote{中右は統語上の文しかM(S)を表し得ないと述べているが,本論文での「文」は中右の言う「典型的な文」に対応する.中右の用法に従うと,文論は「典型的な文を扱うもの」,本論文のアプローチは「典型的ではない文まで対象に含めるもの」ということになる.}.

次に,中右はI thinkを瞬間的現在,I am thinkingを持続的現在とみなし,同じ心的態度でも前者の方がモダリティ’として典型的だとしているが,この区別は用言以外で表されている心的態度には適用が難しい.そこで本研究では,心的態度が発話時点のものであれば瞬間的かどうかに関わらず典型性が高いと考える.

なお,中右は(1) John moved,(2) John agreed,(3) John agrees,(4) I agreed,(5) I agree という5つの表現を用いて,(5)が最もモダリティとして典型性が高く,(4)から(1)へと典型性が下がっていくとしている.この段階的な典型性の中でどこまでをアノテーションすべきかは,アノテーション目的や難易度と照らし合わせて検討する必要がある.本稿では(5)に持続的現在を加えたものをアノテーションの対象とするが,今回の結果も含めデータが蓄積された後に,典型性の程度がアノテーションに与える影響を検討することが求められよう.

\subsection{階層意味論に基づいた心的態度の細分類}

図2にあるように,階層意味論ではモダリティ’を文内モダリティ’と談話モダリティ’とに分類している.前者は命題に対する書き手や話し手の認識を表し,後者は文をどのような形で伝えるかという書き手や話し手の発話態度や伝達様式を表す.本論文では,事柄に関する情報を取り出すという目的を踏まえ,事柄に対する心的態度である文内モダリティ’をアノテーションの対象とする.階層意味論では,事柄に対する心的態度は文内モダリティ’の細分類によって包括できると考えられている\footnote{中右(1994, p.~54)では,文内モダリティ’の細分類によって「目下のところ」包括できる見通しがあると述べられており,網羅性に確信があるわけではないことが伺える.とはいえ,文法形式の分類に用いられるモダリティの細分類とは異なり,意味論の立場から事柄に対する心的態度を網羅するよう配慮されて提示された細分類であり,高い網羅性を持つことが期待される.}.そして図2のモデルから,文章,談話中に現れる全ての命題’は,a)何もモダリティ’を伴わない,b)文内モダリティ’のみを伴う,c)文内モダリティ’と談話モダリティ’を伴う,のいずれかである.よって概念上は,事柄に対する心的態度を網羅するためには「命題’+文内モダリティ’」という組み合わせを全てアノテーションできれば十分ということになる.以下,文内モダリティ’の細分類を示す.

\begin{itemize}
	\item 真偽判断のモダリティ’(modality of truth judgment)
	\item 判断保留のモダリティ’(modality of judgment withholding)
	\item 是非判断のモダリティ’(modality of (dis)approval)
	\item 価値判断のモダリティ’(modality of value judgment)
	\item 拘束判断のモダリティ’(modality of deontic judgment)
\end{itemize}

それでは各分類の規定を述べる.「真偽判断のモダリティ’」は,事柄の真理値について肯定的あるいは否定的に断定,推定する心的態度を指す.「判断保留のモダリティ’」は,事柄の真理値について判断を保留し中立的な立場を表明,含意する心的態度である.典型的なものとしては疑問,質問態度があるが,中右は伝聞判断もここに加えている.「是非判断のモダリティ’」は,真理値を評価しているという点は真偽判断と一緒だが,事柄が既定的(pre-established)である点が異なる.既定的とはその情報が既に談話の世界に提示済みであるということである.以上の3つは全て命題で表される事柄の真偽に対する態度であり,以下,本論文では3つをまとめて「真偽判断系」の心的態度と呼ぶ.

次に「価値判断のモダリティ’」は,事柄に対する情緒的な反応や評価を指す.この心的態度は命題が叙実的(factive)であることが前提とされている.最後に「拘束判断のモダリティ’」は,事柄が指し示す未来の行為を拘束することに関する書き手,話し手の立場を表す.中右はdeonticを拘束判断と訳しているが,2節でモダリティの細分類として挙げた当為(deontic)に近い概念と言える.



\subsection{階層意味論のアノテーションへの応用}

以上,階層意味論の概要を見てきた.文法論によるモダリティの細分類と異なり,モダリティ’の細分類は書き手の事柄に対する発話時点での心的態度を分類するものである.そのため,アノテーションの基準として規定を大幅に変えることなく利用できることが期待される.

ただし,大規模コーパスにアノテーションすることを想定すると,次の2点が問題になると思われる.この2点については,アノテーションの簡単化のために便宜的な対処を行う.
\begin{itemize}
\item 既定的かどうかは語用論的な情報である
\item 未来でも叙実的でもない事柄に対する評価は価値判断にも拘束判断にも含まれない
\end{itemize}
前述の通り,「既定的」とは既に談話の中で取り上げられていることを指す.これは明らかに文脈,状況に依存する性質であろう.``I doubt (that)...'' や ``I admit (that)...'' のようにthat以下が既定的であることを表す表現もあるが,既定的であるときは必ずこのような表現を伴わなければならないわけではない.全く同じ表現でも既定的かどうかは文脈,状況に応じて変わり得るだろう.本論文では,語用論的な側面を考慮するとアノテーションの労力が大きくなり過ぎると考え,既定的かどうかは判断せず,真偽判断と是非判断を区別しないこととする.

そして,未来の行為でも叙実的でもない事柄に対する評価はどのカテゴリにも属さないという問題もある.これについては,未来以外の事柄に対する評価は全て価値判断に属するものとして対処する.




\section{アノテーションの不一致を引き起こす要因と対策}

\subsection{アノテーションの一致度による階層意味論の評価}

階層意味論は,語彙,統語的な特徴が十分に明らかになっていない意味論上の概念であるため,アノテーションは母語話者による判断に頼る必要がある.階層意味論では,母語話者間で共通した性質を出発点に議論するという,チョムスキーがSyntactic Structures \cite{Book_22}で提示したアプローチを採用している\cite{Book_13}.図2で示したモデルは,母語話者の普遍的な言語理解を表すものであり,階層意味論による心的態度の細分類も母語話者間で普遍的であることが期待される.普遍性が成立するのであれば,階層意味論のアノテーションは母語話者によって行われることで妥当性が保障される.

ただし,階層意味論を前提とした日本語の事例研究は多くなく,普遍的という仮説が十分に実証されているとは言い難い.そこで,本論文では4名の母語話者に対してアノテーションの一致度を測る調査を行うことでモダリティ’の細分類の普遍性を評価する.一致度が高ければ,階層意味論の細分類が母語話者間で普遍的な概念であるという仮説が裏付けられ,この細分類が心的態度をアノテーションするための基準として適切と言える.対して一致度が低い場合,アノテーションの基準としてだけでなく,普遍性が求められる階層意味論の概念としても不適切ということになる.


\subsection{調査概要}

\begin{table}[b]
\caption{調査概要}
\input{01table01.txt}
\end{table}

では,一致度を評価するために行った調査の概要を示す.まず,被調査者,調査に用いるテキストおよび調査手順を表1に示す.被調査者は,理系,文系によって傾向に違いが生じる可能性を考慮して東京工業大学と一橋大学から5名ずつ募った\footnote{東京工業大学の社会工学科は学際領域だが,大学受験で求められる能力及び学部で学ぶ講義の内容から理系に該当すると見なして差し支えないと判断した.}.

調査に用いるテキストは,文脈による結果への影響を考えると幅広い文体のテキストおよび談話が望ましいが,膨大な量のアノテーションを行うためには多くの予算と時間を要する.大規模なアノテーションの実施は階層意味論の有効性がある程度確認されて信頼できるデータが得られる目途が立ってから行うべきだと考える.今回は有効性を確認する作業の一環として,効率よく全ての細分類に対するアノテーションを得るために新聞の社説記事を用いた.社説記事は,主題に関する事実関係を記述しつつ書き手の意見を述べる構成になっているため,真偽判断系,価値判断,拘束判断の心的態度がバランスよく現れる傾向がある.かつ,文筆を仕事とする人によって書かれているため,語彙,文法上のミスが少ないことが期待できる.



それでは,調査手順の詳細を説明する.
\begin{description}
\item[手順1] \gt 調査者が社説記事から,次のいずれかの条件を満たすものを命題として全て取りだす.
	\begin{itemize}
	\item 必須格を伴っている動詞,形容詞,形容動詞,名詞+助動詞「だ」
	\item 必須格を伴っている動詞,形容詞,形容動詞,名詞+助動詞「だ」に書き換え可能な名詞(書き換え可能かどうかは調査者が判断)
	\end{itemize}
\end{description}

手順1で取り出される命題の形式は,主節と従属節,名詞の3種類である.そのうち従属節は,大きく副詞節,名詞節と連体節に分けることができる.以下にそれぞれの例を示す\footnote{今後このように例文を示す際,下線部は全て着目している命題を表す.}.
\begin{enumerate}
\setcounter{enumi}{3}
  \item \underline{銀行が腐った資産を抱えたまま}では、貸し出し機能が正常化しないためである。(2009/3/30 毎日新聞)
  \item \underline{パチンコ店などの風俗営業法が適用される施設は、禁煙か分煙にする}ことが努力義務となった。(2009/3/30 朝日新聞)
  \item \underline{膨大な準備が必要となる}大学側の「評価疲れ」が指摘される。(2009/3/31 読売新聞)
\end{enumerate}
ここで,主節,従属節,名詞の全てのタイプの命題を含む(6)を用いて手順1の具体例を示す.この文から取り出される命題は
\begin{itemize}
\item 膨大な準備
\item 膨大な準備が必要となる
\item 評価疲れ
\item 膨大な準備が必要となる大学側の「評価疲れ」が指摘される
\end{itemize}
の4つとなる.

\begin{figure}[t]
 \begin{center}
  \includegraphics{18-4ia911f3.eps}
 \end{center}
 \caption{回答画面}
 \label{fig:three}
\end{figure}

\begin{description}
\item[手順2] \gt 取り出した表現を調査者が人手で文の形に書き直す.書き直した文は,手順3において原文とともに図3のような形で被調査者に提示する.書き直す手順は
	\begin{enumerate}
	\item[1.] 命題が名詞の場合は必須格と述語の形に書き直す.
	\item[2.] テンス,アスペクトがない場合はそれらを補ってモダリティの文法形式と接続助詞を取り除いた上で終止形とする.
	\item[3.] 省略されている必須格を補う.ただし,テキスト内で明示されておらず,かつ補わなくても文として理解できる必須格は補わない.この作業は原文に対しても行う.
	\end{enumerate}
\end{description}
ここで命題が名詞の例を以下に示す.名詞を命題(補足語+述語+形式名詞)で書き換えるとき,どのように書き換わるかは文章,談話の上でないと決定できない.例えば,下の(7),(8)の例は,この文だけでは「アップ」「勝利」の主語を特定することはできない.また,「アップ」や「勝利」が「する」なのか「した」なのかも定まらない.その一方で文章,談話の中で現れる場合,文章,談話の書き手,話し手がその言語の母語話者であり誠実に情報を伝えようとしている限り,それを読んだり聞いたりした母語話者が主語やテンスが何かで迷うことはまずない.(7),(8)を文章の中でみると「アップする」のは介護保険制度の改正に携わる人物で\footnote{社説内ではこの人物が誰かははっきりしない.しかし,社説内で法案の可否が話題となる場合,主語が何かは必ずしも重要でない.本稿では,主語が明確にならなくても母語話者は(7)を命題として理解するものと考える.},「勝利する」の主語は民主党推薦の候補であることがわかる.同時に,「アップ」は未来の出来事なので「アップする」,民主党の候補は実際には知事選に勝利していないので「勝利したこと」では不自然となり「勝利すること」が選ばれる.このように,名詞の書き換えは調査者が母語話者として文章を読んだときの理解に基づいて行われる.

\begin{enumerate}
\setcounter{enumi}{6}
  \item \underline{報酬アップ}は介護人材の確保と処遇改善が狙いだ。(2009/3/31 毎日新聞)
  \item しかし、民主党は\underline{千葉県知事選での勝利}を「政権交代への第一歩」にしたいと考えていた。(2009/3/30 産経新聞)
\end{enumerate}
具体的には,下線部の名詞は手順2によって,それぞれ「介護報酬をアップする(こと)」「民主党推薦の候補が千葉県知事選で勝利する(こと)」と書き直される.

\begin{description}
\item[手順3] \gt 被調査者に集まってもらい,各命題に対する心的態度を選択肢形式で回答してもらう.
\end{description}

以下,手順3の詳細を述べる.調査は,被調査者とのスケジュールを調整し,4回に分けて実施した.各作業では,被調査者2人もしくは3人に1つの部屋に集まってもらい,各人に1台ずつノートパソコンを割り当てる.次に,調査の主旨と回答の手順を書いた紙を配布し,それを調査者が15分かけて音読する.その内容を簡単に整理すると
\begin{itemize}
\item 日本語文で表される情報には,事実情報以外にも書き手や話し手の希望,または書き手や話し手が何も判断,態度を示していない情報が存在することを解説する.
\item 各質問を例を用いて説明する.この際,書き手に関することは何もわからないという前提で回答するよう指示する(新聞の社説記事だということも伝えない).
\item 回答するのはあくまで書き手の心的態度であり,被調査者自身の知識や価値観に惑わされないよう注意を促す.
\item 各質問で書き手の心的態度を読み取ることができないときは最後の「わからない」に相当する選択肢を選ぶよう強調する.
\item 普段,新聞や雑誌を読むときと同じ感覚で文を読み,直観的に回答するようお願いする.
\end{itemize}
となる\footnote{配布した紙と回答画面の例をhttp://www.soc.titech.ac.jp/~sakano/modality/に掲載している.}.

音読が終わった後,被調査者はパソコン画面に表示される質問(図3)に答える.
画面には社説記事に載っている元の文とその中の命題を調査者が文の形に書き直したもの,
そして質問が表示される.最初の質問に答えて「次へ」を押すと次の質問が表示され,再び回答するという作業の繰り返しとなる.被調査者は最初1時間回答した後,5分休憩を挟み,続けて40分回答する.回答中と休憩中に被調査者間で質問に関する情報のやり取りをすることは禁止した.以下,表2に各質問を掲載する.



1つの命題に対する回答の流れは次のようになる.まず,質問1では,命題が発話時以前のものなら「事実かどうか」,後なら「将来事実となるかどうか」に対する書き手の事実認識を答える\footnote{発話時以前かどうかは調査者が事前に人手で分類している.}.回答が得られた271個の命題のうち,発話時以前のものは191個,発話時より後のものは80個である.質問1で4と回答された場合は質問2に進み,1と回答された場合は質問3に進む.

質問2は質問1を補足する質問となっている.本来,判断保留なら質問1で3が選ばれるべきだが,伝聞の場合,問1で3が回答されない可能性が高いと考え,別途,質問2を用意した.質問3は既定的かどうかの区別を意図した質問だが,前述の通り本論文では考察の対象としない.

\begin{table}[t]
\caption{各質問の説明}
\input{01table02.txt}
\end{table}

そして最後に,発話時点以前の命題であれば質問4,後であれば質問5に進む.ここでは,発話時以前なら叙実的なものとして価値判断の対象になり,発話時以後なら未来の命題として拘束判断の対象になるという前提を置いている.



\subsection{一致度の評価方法}

本論文では一致度の評価は,一致度によるアノテーションの評価をFleissの $\kappa$ 係数\cite{Article_08}によって行う.以下,Fleissの $\kappa$ 係数の前身であるCohenの $\kappa$ 係数\cite{Article_07}を概説した後,Fleissの $\kappa$ 係数について簡潔に述べる.Cohenの $\kappa$ 係数とは,同じ対象に対し同じ名義尺度で測定した2つのデータ間の一致度を,偶然による一致の可能性を排除して評価する指標である\cite{Article_07}.データが実際に一致した割合を$P(A)$,2つのデータが独立だった場合に偶然一致する割合を$P(E)$としたとき,Cohenの $\kappa$ 係数は次式で表される.
\[
\kappa = \frac{P(A)-P(E)}{1-P(E)}
\]
このP(E)によって偶然による一致の分が修正される.

例として,同一の有権者の集合に対し,内閣を支持するかしないかの2択で,去年と今年の2回に渡り調査した状況を想定する.調査の結果,去年,今年とも全員から回答が得られ,去年の内閣支持率が0.8,今年の内閣支持率が0.7,去年と同じ回答をした人の割合が0.62だったとする.このとき,$P(A)$が0.62と6割以上であるにもかかわらず,P(E)の値も$0.8 \times 0.7 + 0.2 \times 0.3 = 0.62$であるためCohenの $\kappa$ 係数は0となる.Cohenの $\kappa$ 係数は$-1$から1の値を取り,0であれば2つのデータ間には偶然による一致しかないということになる.

そして,Fleissの $\kappa$ 係数は,$\frac{P(A)-P(E)}{1-P(E)}$という計算式は変わらず,データが3つ以上の場合でも計算できるように$P(A)$と$P(E)$の計算方法を拡張したものである.P(E)を計算する際,データ間が独立という仮定に加え,データ間の等質性(本論文では被調査者間の等質性)の仮定も置いているのが特徴である.

これら $\kappa$ 係数を評価する基準としては,$0\le\kappa\le0.2$が低い一致(poor agreement),$0.2<\kappa\le0.4$が軽度の一致(fair),$0.4<\kappa\le0.6$が中程度の一致(moderate),$0.6<\kappa\le0.8$が相当な一致(substantial),$0.8<\kappa\le1$がほとんど完全な一致(almost perfect)という目安が提示されているが\cite{Article_09,Book_24},この目安に客観的な根拠があるわけではない.あくまで参考程度に,0.2,0.4,0.6,0.8を1つの評価ラインとして考える.

また,心的態度のアノテーションを行っている既存研究での $\kappa$ 係数も参考になると思われる.前述の「てほしい」に相当する拡張モダリティ\cite{Article_01}では,工学部の学生3名に対して調査を行い,学生間の一致度をCohenの $\kappa$ 係数で評価したところ,0.48,0.64,0.47\footnote{Cohenの $\kappa$ 係数は2名間の一致度しか測ることができないため,A, B, Cの3名の結果から,AB,AC,BCの3通り計算している.}という結果が得られている.また,モダリティ論に基づいてカテゴリを定めた方の拡張モダリティ\cite{Inproc_04}では,2名の専門家によるマニュアルに従った判断の $\kappa$ 係数(書かれていないがCohenと思われる)を全カテゴリで求めた平均が0.71となっている.



\subsection{調査結果}

最初に各細分類で実際に一致した数と割合{$P(A)$}を表3に示す.真偽判断系と拘束判断は3名以上一致する割合が80\%を超えており高い一致度のように見えるが,4.3で述べたように偶然一致する確率が考慮されていない.そこで,質問1と2を合わせた\footnote{質問1の選択肢で4が選ばれた場合,それを質問2の回答に置き換えている.} 真偽判断系,質問4の価値判断,質問5の拘束判断についてFleissの $\kappa$ 係数\cite{Book_15}を表4に示す.最初に,表4の2者間の一致度において学校の違いの影響を見る.6パターンのうち学校が同じ者同士の $\kappa$ 係数は,真偽判断系では4,5番目,価値判断では1,6番目,拘束判断では3,5番目の大きさとなっており,学校が異なる回答者間の係数と比べ順位が高い傾向にあるとは言えない.よって,以後,学校間の違いは考慮せず,4名における $\kappa$ 係数のみを考察する.

\begin{table}[t]
\begin{minipage}[t]{.5\textwidth}
\caption{回答が一致した命題数(括弧内は\%)}
\input{01table03.txt}
  \end{minipage}
  
  \hfill
  
  \begin{minipage}[t]{.45\textwidth}
\caption{モダリティの一致度(Fleissの $\kappa$ 係数)}
\input{01table04.txt}
  \end{minipage}
\end{table}

4名の $\kappa$ 係数を見ると拘束判断が約0.70で既存研究\cite{Inproc_04}で示された0.71に近い値を示しているが,真偽判断系は約0.49,価値判断は約0.28と,0.71を大幅に下回っている.単純な比較はできないが\footnote{この0.71という値はモダリティ’ではなくモダリティの概念に基づいたアノテーションであり,かつ詳細なマニュアルに基づいたものなので単純比較はできない.現段階では,今後,意味論のモデルに基づいた他のアノテーションが行われた際に比較できるよう,階層意味論に基づいた調査結果の一致度を提示すること自体に意義があると考える.},真偽判断系と価値判断についてはもう1つの既存研究\cite{Article_01}の一致度と比較しても高いとは言えず,普遍性があると見なすことは難しい.以下,この点について考察する.真偽判断系については4.5で詳述するため,ここでは価値判断について論じる.

本論文では,価値判断は本質的に母語話者にとって判断が難しい概念であると考える.そのことを示唆するものに選択肢3が挙げられる.3は,中右が例示している「不思議に思う」「奇異に感ずる」「おもしろいことに」「驚いたことに」といった,肯定的か否定的かがはっきりしない価値判断を想定して設けた選択肢だが,これが一致度を下げる要因となっている.4名の回答者のうち1人以上が3を選んでいる命題を除外して $\kappa$ 係数を求めると0.45と,0.28から大幅に上昇する.このことから,選択肢3で雑多な価値判断をまとめて処理しようとしたことが一致度を下げている一因になっており,少なくとも中右本来の規定のままではアノテーションの基準として適切とは言えないことが予想される.一致度を上げるために,価値判断を「正しさ」「面白さ」といった判断基準ごとに更に細分類しアノテーションする必要があると思われる.



\subsection{真偽判断系で不一致を引き起こす要因と対策}

真偽判断系は,価値判断と異なりその判断基準が真偽に限られるにもかかわらず $\kappa$ 係数が拘束判断と比べて低い.前述の医学,生物学系テキストへのアノテーションでも扱われているように,真偽判断系は自然言語処理の分野で需要の高い情報である.もしアノテーションが難しいのであれば階層意味論の基準としての妥当性が疑われることになる.そこで,この節では不一致の要因とそれを解消するための方法について検討する.

まずは,13の「懸念も消えない」や14の「いけない」のような,述語が書き手もしくは他者の心的態度を表す事例が一致度を下げていると考えられる.書き手や他者の心的態度を表す述語の例を表5に示す.
\begin{enumerate}
\setcounter{enumi}{12}
  \item \underline{西松建設の違法献金事件が自民党議員側に波及する{\gt 懸念も消えない}}。(2009/3/30 読売新聞)
  \item \underline{\mbox{AIG や金融界に対する批判が{\gt いけない}}}というのではない。(2009/3/30 毎日新聞)
\end{enumerate}
全271個の命題を,述語が書き手や他者の心的態度を表す52個と,残りの219個とに分けて $\kappa$ 係数を計算すると,前者が0.26,後者が0.53と大きな差がみられる.

\begin{table}[b]
\caption{心的態度を表す述語}
\input{01table05.txt}
\end{table}

4.2の手順2では品詞のみで述語を決定するため,述語が心的態度を表しているかどうかに関係なく13,14の下線部も命題として取り出される.しかし,論理学の観点からは,これらの真偽を問うことができるのかどうかは非常に難しい問題である.心的態度を適切に扱うための体系が様相論理の分野でいくつか提案されているが\cite{Book_17},まだ標準的な体系が確立されたとは言い難い.論理学者にとっても扱いが難しい問題に対し,母語話者間で一貫した回答が得られることは期待できないだろう.よって,13,14の下線部のように述語が心的態度を表す場合,真偽判断のアノテーションの対象から外すことが適当だと考える.

2つ目に取り上げるのは,命題に後続する表現が他者の心的態度または条件\footnote{ここでは,従属節が主節の命題が真であるための条件を表す場合のみならず,主節が従属節の命題が真であるための条件を表す場合も含む.} を表す事例である.15,16の「考えを示した」「歴史的課題だ」は他者の心的態度,17,18の「れば」「には」は条件の例となる.
\begin{itemize}
\item 他者の心的態度
  \begin{enumerate}
  \setcounter{enumi}{14}
    \item 鳩山氏は\underline{衆院の解散直前に党独自で行う選挙情勢調査を見極めて最終判断する}{\gt 考えを示した}。(2009/3/31 毎日新聞)
    \item \underline{出先機関改革}は中央省庁再編で積み残された{\gt 歴史的課題だ}。(2009/3/31 読売新聞)
  \end{enumerate}
\item 条件
  \begin{enumerate}
  \setcounter{enumi}{16}
    \item \underline{政治への国民の信頼がなけ}{\gt れば}、今の経済状況は乗り切れない。(2009/3/31 毎日新聞)
    \item \underline{年末の改革大綱で具体的な成果を上げる}{\gt には}、政府は早い段階から、周到に調整を進める必要がある。(2009/3/31 読売新聞)
  \end{enumerate}
\end{itemize}

他者の心的態度や条件自体は確実性判断の既存研究でも扱われているが\cite{Article_02,Article_06},今回の調査結果は,これらの表現が真偽判断系の一致度を下げることを示している.271個の命題のうち,これらの表現が後続する38個では $\kappa$ 係数が0.27,残りの233個では0.50となる.他者の心的態度や条件が後続する場合,書き手の心的態度が明示的に表されていない.そのため,本来であれば選択肢4が選ばれるべきところだが,実際には書き手の事実認識を類推してしまう場合があると思われる.

同様に「と言われている」のような伝聞を表す表現が後続する場合も一致度が下がる.先ほどの38個の命題に伝聞も加えた52個では $\kappa$ 係数が0.26,残りの219個では0.54となる.伝聞情報でも書き手の心的態度が明示されていないという点では共通し,同様の結果を引き起こすと思われる.

以上をまとめると,4.2の手順2で特定された命題のうち
\begin{itemize}
\item 述語が書き手または他者の心的態度を表すもの
\item 後続する表現が他者の心的態度や条件,伝聞を表すもの
\end{itemize}
については真偽判断をアノテーションする対象から外すべきと考える.両方を外した残りの命題175個で $\kappa$ 係数を求めると,0.58と,中程度の一致と相当の一致の境界となる0.6に近い値を示す.



\subsection{意味論上の規定を用いることの一致度への影響}

階層意味論のような意味論上の規定をアノテーションの基準に採用する狙いとして,モダリティや主観述語といった客観的な指標が明示されない心的態度も母語話者の判断によって特定することが挙げられる.このとき,客観的な指標がないと高い一致度が得られないのであれば,この狙いを達成することはできない.そこで,4.2の手順2で特定した命題を「(a)『主節』および『後続する表現が語彙的に心的態度を表す命題』」と「(b)それ以外の命題」とに分け,真偽判断系,価値判断,拘束判断の $\kappa$ 係数を求め比較する\footnote{語彙的に心的態度を表す表現とは,表5で示すような表現のことを指す.このとき,各カテゴリの $\kappa$ 係数を求めるときはそのカテゴリに該当する表現のみを(a)に含める.例えば,真偽判断系の $\kappa$ 係数を求めるときに係り先が拘束判断を表していても(a)ではなく(b)とする.}.なお,真偽判断系に関しては4.5で絞り込んだ175個の命題を対象に計算する.

結果を表6に示す.真偽判断系と価値判断では(a)と(b)で一致度にほとんど差が見られない一方で,拘束判断では(a)の0.89の方が(b)の0.53と比べて極めて高い値を示している.このことから拘束判断が他の判断と比べて一致度が高いのは,モダリティや主観述語が示されている場合の一致度が極めて高いためだと考えられる.その一方で,(b)でも真偽判断系,拘束判断の両方で0.5以上と中程度の一致度を示している.モダリティや主観述語で明示的に心的態度が示されていなくても,母語話者は一定の割合で一致した理解を得ていることが伺える.

\begin{table}[b]
\vspace{-0.5\baselineskip}
\caption{形式上の特徴の有無に応じた一致度の違い(Fleissの $\kappa$ 係数,括弧内は該当命題数)}
\input{01table06.txt}
\vspace{-0.5\baselineskip}
\end{table}

ここで,(b)のうち4名全員の回答が一致しているもの(真偽判断系105個中80個,価値判断110個中35個,拘束判断47個中24個)の観察を通して,モダリティや主観述語が存在しなくても一致する要因について検討する.まず真偽判断系だが,最も多い事例として,主節のモダリティの統語的な作用域に従属節の命題が含まれているものを取り上げる.この事例は29個存在する.例として19を示す.
\begin{enumerate}
\setcounter{enumi}{18}
  \item ところが、\underline{閣僚折衝さえ開かれず}、中身の乏しい工程表になってしまった。(2009/3/31 読売新聞)
\end{enumerate}
日本語学では,主節のモダリティの作用域に含まれるかどうかは従属節の種類に依存するとされている\cite{Article_10}.例えば,19のような連用節は独立度が低く作用域に含まれるが,ガ節は独立度が高く主節からの影響を受けない.従属節の独立度には,従属節の内部に補足語や付加語,モダリティの文法形式といった成分を取り得るかによって3種類(主節を加えると4種類)の分類が存在する\cite{Book_10}.19のような事例の心的態度を解析する場合,連用節,ガ節といった従属節の種類,または南(1993)による4分類が素性の1つとして有効だと考えられる.

次に,命題が文の前提(presupposition) \cite{Book_07}を表す事例が19個存在する.前提となる命題は読み手に真として受け取られる傾向があると思われる.既存研究で取り上げているのは20のように命題が従属節のものだが,今回のデータでは,19個のうち16個が21のように命題が名詞の例だった.今後,母語話者が名詞の命題を前提だと認識する条件を明らかにすることが求められる.
\begin{enumerate}
\setcounter{enumi}{19}
  \item 朝日新聞の世論調査で、\underline{\mbox{86%}もの人が小沢氏の説明では「不十分だ」と答えた}のは当然のことだろう。(2009/3/31 朝日新聞)
  \item \underline{森田氏の大量得票}は政党全体への不信感の裏返しといってもいいのだ。(2009/3/31 毎日新聞) \end{enumerate}

以上,真偽判断系の事例は「従属節の独立度」や「前提」のように文法論の概念で解釈可能なものが多く見られる.対して価値判断と真偽判断の(b)は原則,語用論的な側面を考慮する必要があると考えられる.

最初に取り上げるのは,語彙的には心的態度ではないが,語用論的側面まで考慮すると後続する表現が心的態度を表す事例である.この事例は,価値判断では4個,拘束判断では8個存在する.22に価値判断,23と24に拘束判断の例を示す.22の「役立たない」や24の「言行不一致が問われよう」は,客観的な事実を述べつつ,否定的な印象を与える表現である.このとき心的態度としての側面が強調されるかどうかは文脈に大きく左右される.一方,23の「結びつけたい」は,「テストの結果を比較する」ことが「教育施策や学習指導の改善」に繋がることを望んでいることからの推論で,「改善」を望んでいるという帰結が得られると思われる.ただし,「テストの結果を比較する」以外の方法では「改善」を望んでいない可能性もあり,あくまで文脈に左右される解釈と言える.

\begin{enumerate}
\setcounter{enumi}{21}
 \item 公開が原則の国立公文書館という組織はあっても、\underline{そこへ移される文書が極めて少ない}から、あまり{\gt 役立たない}。(2009/3/31 朝日新聞)
 \item テストの結果を比較し、\underline{教育施策や学習指導の改善}に{\gt 結びつけたい}。(2009/3/30 読売新聞)
 \item \underline{与党が出先機関改革をすべて衆院選後に先送りする}ようでは、{\gt 言行不一致が問われよう}。(2009/3/31 読売新聞)
\end{enumerate}

また,価値判断では選択肢4の「読み取ることはできない」での一致が26個,拘束判断では後者では選択肢3の「わからない」での一致が15個と,それぞれ半数以上を占めている.直感的には,価値判断や拘束判断を表す表現がまったくないときに「わからない」と判断すると思われる.ただし,価値判断や拘束判断を表し得る表現の有無は,モダリティ,主観述語の他に22〜24のような語用論的側面も考慮しなければならない.よって,「わからない」と判断するための基準も語彙,文法上の特徴だけにはとどまらないと思われる.

以上,今回の調査によって,(b)でも真偽判断系,拘束判断については,ある程度,一貫性のあるアノテーションが可能なことが示された.とはいえ,アノテーションの過程はできる限りマニュアル化できた方が望ましい.これまでの文法論では扱われていない新たな語彙,統語的な特徴を明らかにし客観性を高めることも求められよう.今後,(b)の事例を数多く含んだ大規模なタグ付きコーパスを作成し,多変量解析によって語彙,統語的な特徴を推定,検証することが望まれる.



\section{おわりに}

本論文では,まず,モダリティ論と自然言語処理とではモダリティと心的態度の扱いに相違点があることを指摘した.前者は心的態度という規定が必ずしも前提ではなく,文論の枠内で形式と意味との相関を捉えることが主旨である一方,後者は命題に対する心的態度を特定することが目的となる.

その上で本論文では,心的態度をアノテーションする基準として階層意味論で提示されているモダリティ’の細分類を用いることを検討した.階層意味論のモダリティ’は書き手の発話時点の心的態度であると定義されており,対象が文法形式に限定されない.よって,心的態度をアノテーションするという目的に適う概念と言える.ただし,階層意味論では日本語の事例を用いた研究が少なく普遍性の実証が十分ではない.そこで本論文では,日本語を母語とする大学の学部生に命題に対する書き手の心的態度を回答してもらい母語話者間の一致度を調査することで普遍性を評価した.その結果,拘束判断は0.70と高い $\kappa$ 係数を示し,真偽判断系は,i)述語が心的態度を表す場合と,ii)後続表現が他者の心的態度や条件節,伝聞である場合,アノテーションの対象から外すことで0.58の値を示した.一方,価値判断は0.28と相対的に低い.中右が規定する価値判断には「面白さ」「奇妙さ」など複数の判断基準が混在しており,概念を判断基準ごとに細分化する必要性があると考えられる.

また,階層意味論に基づくアノテーションでは,語彙,文法形式によって明示的に心的態度が表されない命題も対象となる.このとき明示的な指標がない事例に対して一貫性のあるアノテーションできないのであれば階層意味論を用いる意義が損なわれる.そこで明示的な指標がない事例に限定したときの一致度についても検討を行った.その結果,明示されていない事例でも真偽判断系と拘束判断については0.5以上の $\kappa$ 係数を示し,母語話者は明示的な指標がない事例でも一定の一致度を持ってアノテーションできることが確認された.

なお,今回の調査は10個の社説記事から命題を抜き出したため,回答者が同じテキスト内の命題に触れる頻度が多く,ランダムに並び替えたとはいえ,文脈の情報を排除しきれたとは言い難いものとなってしまった.さらに社説の内容も日本の政治に関わるもので,被調査者が少なからず知識を持っていると予想されるため,被調査者間の知識の共通点や相違点が回答に影響を与えた可能性もある.今後,さらなる調査を行う際には,語用論的な側面のコントロールをより徹底して行うことが求められる.

\acknowledgment

数多くの有益な助言を頂いた査読者様,調査に協力して頂いた東京工業大学と一橋大学の学生10名,およびプレ調査に協力して頂いた坂野研究室の後輩2名に感謝を申し上げます.



\nocite{*}
\bibliographystyle{jnlpbbl_1.5}
\begin{thebibliography}{}

\bibitem[\protect\BCAY{Chomsky}{Chomsky}{1957}]{Book_22}
Chomsky, N. \BBOP 1957\BBCP.
\newblock {\Bem Syntactic Structures}.
\newblock Mouton.

\bibitem[\protect\BCAY{Cohen}{Cohen}{1960}]{Article_07}
Cohen, J. \BBOP 1960\BBCP.
\newblock \BBOQ A coefficient of agreement for nominal scales.\BBCQ\
\newblock {\Bem Educational and Psychological Measurement}, {\Bbf 20}  (1),
  \mbox{\BPGS\ 37--46}.

\bibitem[\protect\BCAY{江口\JBA 松吉\JBA 佐尾\JBA 乾\JBA 松本}{江口 \Jetal
  }{2010}]{Inproc_04}
江口萌\JBA 松吉俊\JBA 佐尾ちとせ\JBA 乾健太郎\JBA 松本裕治 \BBOP 2010\BBCP.
\newblock モダリティ,真偽情報,価値情報を統合した拡張モダリティ解析.\
\newblock \Jem{言語処理学会第16回年次大会論文集}, \mbox{\BPGS\ 852--855}.

\bibitem[\protect\BCAY{Fillmore}{Fillmore}{1968}]{Inbook_01}
Fillmore, C.~J. \BBOP 1968\BBCP.
\newblock {\Bem Universals in Linguistic Theory}, \BCH\ The case for case,
  \mbox{\BPGS\ 1--88}.
\newblock Holt, Rinehart \& Winston of Canada Ltd.

\bibitem[\protect\BCAY{Fleiss}{Fleiss}{1971}]{Article_08}
Fleiss, J.~L. \BBOP 1971\BBCP.
\newblock \BBOQ Measuring nominal scale agreement among many raters.\BBCQ\
\newblock {\Bem Psychological Bulletin}, {\Bbf 76}  (5), \mbox{\BPGS\
  378--382}.

\bibitem[\protect\BCAY{花園}{花園}{1999}]{Article_05}
花園悟 \BBOP 1999\BBCP.
\newblock 条件形複合用言形式の認定.\
\newblock \Jem{國語學}, {\Bbf 197}  (1), \mbox{\BPGS\ 39--53}.

\bibitem[\protect\BCAY{乾\JBA 井佐原}{乾\JBA 井佐原}{2002}]{Article_01}
乾裕子\JBA 井佐原均 \BBOP 2002\BBCP.
\newblock 拡張モダリティの提案−自由回答から回答者の意図を判定するために−.\
\newblock \Jem{電子情報通信学会技術研究報告. NLC,
  言語理解とコミュニケーション}, {\Bbf 102}  (414), \mbox{\BPGS\ 31--36}.

\bibitem[\protect\BCAY{川添\JBA 齊藤\JBA 片岡\JBA 戸次}{川添 \Jetal
  }{2009}]{Article_02}
川添愛\JBA 齊藤学\JBA 片岡喜代子\JBA 戸次大介 \BBOP 2009\BBCP.
\newblock 確実性判断に関わる意味的文脈アノテーションの試み.\
\newblock \Jem{情報処理学会研究報告. 自然言語処理研究会報告}, {\Bbf 2009}  (2),
  \mbox{\BPGS\ 77--84}.

\bibitem[\protect\BCAY{Landis \BBA\ Koch}{Landis \BBA\ Koch}{1977}]{Article_09}
Landis, J.~R.\BBACOMMA\ \BBA\ Koch, G.~G. \BBOP 1977\BBCP.
\newblock \BBOQ The measurement of observer agreement for categorical
  data.\BBCQ\
\newblock {\Bem Biometrics}, {\Bbf 33}, \mbox{\BPGS\ 159--174}.

\bibitem[\protect\BCAY{Light, Qiu, \BBA\ Srinivasan}{Light
  et~al.}{2004}]{Inproc_01}
Light, M., Qiu, X.~Y., \BBA\ Srinivasan, P. \BBOP 2004\BBCP.
\newblock \BBOQ The Language of Bioscience: Facts and Speculations and and
  Statements in Between.\BBCQ\
\newblock In {\Bem HLT-NAACL 2004 Workshop: BioLINK 2004 and Linking Biological
  Literature and Ontologies and Databases}, \mbox{\BPGS\ 17--24}.

\bibitem[\protect\BCAY{牧野}{牧野}{2009}]{Article_06}
牧野武則 \BBOP 2009\BBCP.
\newblock 日本語の主観表現の機能的構造 : 客観文と主観文(日本語処理・文法).\
\newblock \Jem{情報処理学会研究報告. 自然言語処理研究会報告}, {\Bbf 2009}
  (36), \mbox{\BPGS\ 7--14}.

\bibitem[\protect\BCAY{Marcus, Santorini, \BBA\ Marcinkiewicz}{Marcus
  et~al.}{1993}]{Article_03}
Marcus, M.~P., Santorini, B., \BBA\ Marcinkiewicz, M.~A. \BBOP 1993\BBCP.
\newblock \BBOQ Building a large annotated corpus of English: The Penn
  Treebank.\BBCQ\
\newblock {\Bem Computational Linguistics}, {\Bbf 19}  (2), \mbox{\BPGS\
  313--330}.

\bibitem[\protect\BCAY{益岡}{益岡}{1991}]{Book_05}
益岡隆志 \BBOP 1991\BBCP.
\newblock \Jem{モダリティの文法}.
\newblock くろしお出版.

\bibitem[\protect\BCAY{益岡}{益岡}{2007}]{Book_08}
益岡隆志 \BBOP 2007\BBCP.
\newblock \Jem{日本語モダリティ探究}.
\newblock くろしお出版.

\bibitem[\protect\BCAY{益岡\JBA 田窪}{益岡\JBA 田窪}{1992}]{Book_25}
益岡隆志\JBA 田窪行則 \BBOP 1992\BBCP.
\newblock \Jem{基礎日本語文法—改訂版—}.
\newblock くろしお出版.

\bibitem[\protect\BCAY{南}{南}{1974}]{Book_20}
南不二男 \BBOP 1974\BBCP.
\newblock \Jem{現代日本語の構造}.
\newblock 大修館書店.

\bibitem[\protect\BCAY{南}{南}{1993}]{Book_10}
南不二男 \BBOP 1993\BBCP.
\newblock \Jem{現代日本語文法の輪郭}.
\newblock 大修館書店.

\bibitem[\protect\BCAY{宮崎\JBA 安達\JBA 野田\JBA 高梨}{宮崎 \Jetal
  }{2002}]{Book_01}
宮崎和人\JBA 安達太郎\JBA 野田晴美\JBA 高梨信夫 \BBOP 2002\BBCP.
\newblock \Jem{新日本語文法選書4 モダリティ}.
\newblock くろしお出版.

\bibitem[\protect\BCAY{中右}{中右}{1994}]{Book_13}
中右実 \BBOP 1994\BBCP.
\newblock \Jem{認知意味論の原理}.
\newblock 大修館書店.

\bibitem[\protect\BCAY{Narrog}{Narrog}{2005}]{Article_04}
Narrog, H. \BBOP 2005\BBCP.
\newblock \BBOQ On defining modality again.\BBCQ\
\newblock {\Bem Language Sciences}, {\Bbf 27}  (2), \mbox{\BPGS\ 165--192}.

\bibitem[\protect\BCAY{Narrog}{Narrog}{2009}]{Book_11}
Narrog, H. \BBOP 2009\BBCP.
\newblock {\Bem Modality in Japanese: The Layered Structure of the Clause and
  Hierarchies of Functional Categories (Studies in Language Companion Series)}.
\newblock John Benjamins Publishing Company.

\bibitem[\protect\BCAY{日本語記述文法研究会}{日本語記述文法研究会}{2003}]{Book
_04}
日本語記述文法研究会(編集) \BBOP 2003\BBCP.
\newblock \Jem{現代日本語文法〈4〉第8部・モダリティ}.
\newblock くろしお出版.

\bibitem[\protect\BCAY{尾上}{尾上}{2001}]{Book_06}
尾上圭介 \BBOP 2001\BBCP.
\newblock \Jem{文法と意味I}.
\newblock くろしお出版.

\bibitem[\protect\BCAY{Palmer}{Palmer}{1986}]{Book_21}
Palmer, F. \BBOP 1986\BBCP.
\newblock {\Bem Mood and Modality}.
\newblock Cambridge University Press.

\bibitem[\protect\BCAY{Palmer}{Palmer}{2001}]{Book_07}
Palmer, F. \BBOP 2001\BBCP.
\newblock {\Bem Mood and Modality Second edition}.
\newblock Cambridge University Press.

\bibitem[\protect\BCAY{Portner}{Portner}{2009}]{Book_17}
Portner, P. \BBOP 2009\BBCP.
\newblock {\Bem Modality}.
\newblock Oxford University Press.

\bibitem[\protect\BCAY{Rietveld \BBA\ van Hout}{Rietveld \BBA\ van
  Hout}{1993}]{Book_24}
Rietveld, T.\BBACOMMA\ \BBA\ van Hout, R. \BBOP 1993\BBCP.
\newblock {\Bem Statistical Techniques for the Study of Language and Language
  Behaviour}.
\newblock Mouton de Gruyter.

\bibitem[\protect\BCAY{Santorini}{Santorini}{1990}]{Inproc_03}
Santorini, B. \BBOP 1990\BBCP.
\newblock \BBOQ Part-of-Speech Tagging Guidelines for the Penn Treebank Project
  (3rd Revision).\BBCQ\
\newblock In {\Bem University of Pennsylvania Department of Computer and
  Information Science Technical Report No. MS-CIS-90-47}.

\bibitem[\protect\BCAY{澤田}{澤田}{2006}]{Book_12}
澤田治美 \BBOP 2006\BBCP.
\newblock \Jem{モダリティ}.
\newblock 開拓社.

\bibitem[\protect\BCAY{Siegel \BBA\ Jr.}{Siegel \BBA\ Jr.}{1988}]{Book_15}
Siegel, S.\BBACOMMA\ \BBA\ Jr., N. J.~C. \BBOP 1988\BBCP.
\newblock {\Bem Nonparametric Statistics for The Behavioral Sciences, Second
  ed.}
\newblock McGraw-Hill Book Co.

\bibitem[\protect\BCAY{Szarvas, Vincze, Farkas, \BBA\ Csirik}{Szarvas
  et~al.}{2008}]{Inproc_02}
Szarvas, G., Vincze, V., Farkas, R., \BBA\ Csirik, J. \BBOP 2008\BBCP.
\newblock \BBOQ The BioScope Corpus: Annotation for Negation and Uncertainty
  and Their Scope in Biomedical Texts.\BBCQ\
\newblock In {\Bem the Workshop on Current Trends in Biomedical Natural
  Language Processing}, \mbox{\BPGS\ 38--45}.

\bibitem[\protect\BCAY{田窪}{田窪}{1987}]{Article_10}
田窪行則 \BBOP 1987\BBCP.
\newblock 統語構造と文脈情報.\
\newblock \Jem{日本語学}, {\Bbf 6}  (5), \mbox{\BPGS\ 37--48}.

\bibitem[\protect\BCAY{時枝}{時枝}{1941}]{Book_19}
時枝誠記 \BBOP 1941\BBCP.
\newblock \Jem{國語學原論}.
\newblock 岩波書店.

\bibitem[\protect\BCAY{渡辺}{渡辺}{1974}]{Book_23}
渡辺実 \BBOP 1974\BBCP.
\newblock \Jem{国語文法論}.
\newblock 笠間書院.

\bibitem[\protect\BCAY{渡辺}{渡辺}{1996}]{Book_18}
渡辺実 \BBOP 1996\BBCP.
\newblock \Jem{日本語概説}.
\newblock 岩波書店.

\end{thebibliography}


\section*{付録: 調査で得られた回答の集計データ}

調査して得られた4つのデータのクロス集計表を示す.表の項目にあるx-yは回答者xが選択肢yを選んだことを示す.回答者1, 2は東工大生,3,4は一橋大生の回答である.命題ごとに同じ大学の5人のうち2人がランダムに選ばれているため,実際には「各回答者x」の回答には複数の被調査者の回答が混在している.なお,真偽判断系については,質問1で選択肢4を選んだもののうち質問2で1を選んだ場合は4,2を選んだ場合は5としている.また,合計が0の行と列は省略している.

\begin{table}[b]
\caption{真偽判断系のクロス表}
\input{01table07.txt}
\vspace{1\baselineskip}
\end{table}



\begin{table}[b]
\caption{真偽判断系のクロス表(4.5で絞り込んだ命題)}
\input{01table08.txt}
\end{table}

\clearpage

\begin{table}[t]
\caption{価値判断のクロス表}
\input{01table09.txt}
\vspace{1\baselineskip}
\end{table}



\begin{table}[t]
\caption{拘束判断のクロス表}
\input{01table10.txt}
\end{table}




\begin{biography}
\bioauthor{小橋 洋平}{
2005年東京工業大学大学院社会理工学研究科博士前期課程修了.同大学博士後期課程を満期退学後,筑波大学人文社会科学研究科非常勤研究員.現在に至る.修士(工学).心的態度に着目した文章解析および談話分析への応用に関心がある.日本語文法学会,日本計画行政学会各会員.
}

\pagebreak
\bioauthor{坂野 達郎}{
1987年東京工業大学総合理工学研究科博士課程修了.博士(工学).同理工学研究科社会工学科助手,日本社会事業大学助教授を経て,1996年より東京工業大学社会理工学研究科准教授.現在に至る.専門は公共システムデザイン.日本計画行政学会,産業組織心理学会,都市計画学会各会員.
}
\end{biography}




\biodate





\end{document}
