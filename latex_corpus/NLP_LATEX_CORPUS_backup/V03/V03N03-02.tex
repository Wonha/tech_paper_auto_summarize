\documentstyle[jnlpbbl,epsf]jnlp_j_b5}


\setcounter{page}{31}
\setcounter{巻数}{3}
\setcounter{号数}{3}
\setcounter{年}{1996}
\setcounter{月}{7}
\受付{95}{6}{21}
\再受付{95}{9}{18}
\採録{96}{2}{20}

\setcounter{secnumdepth}{2}

\def\atari(#1,#2){}

\title{被喩詞の意味と比喩表現の意味との違いを示す指標}
\author{内山 将夫\affiref{U} \and 板橋 秀一\affiref{I}}

\headtitle{被喩詞の意味と比喩表現の意味との違いを示す指標}
\headauthor{内山 将夫・板橋 秀一}

\affilabel{U}{筑波大学 大学院 工学研究科}{Doctoral Program in Engineering, University of Tsukuba}
\affilabel{I}{筑波大学 電子・情報工学系}{Institute of Information Sciences and Electronics, University of Tsukuba}

\jabstract{本稿では,比喩の理解過程における再解釈の段階について考察し
た.名詞の意味を確率を利用して表現した.比喩表現の意味は,喩詞の意味に
影響された被喩詞の意味である.被喩詞の意味と比喩表現の意味との違いを示
す指標として,明瞭性と新奇性とを 情報量を用いて 定義した.明瞭性が大き
い値となるときは,比喩表現において属性に関する不確定さが減少したときで
ある.新奇性が大きい値となるときは,比喩表現が稀な事象を表わすときであ
る.SD法により比喩表現の意味を測定し,明瞭性と新奇性とを求めた.明瞭性
が属性の顕著性のパターンに対応する数値であることと,明瞭性と新奇性とが
比喩の理解容易性の指標として適当であることとを示した.}

\jkeywords{
  比喩,解釈,SD法,情報量,明瞭性,新奇性
  }

\etitle{The Measurement of the meaning difference \\ 
between tenor and metaphor/simile}

\eauthor{Masao Utiyama \affiref{U} \and Shuichi Itahashi \affiref{I}}

\eabstract{This paper describes the re-interpretation stage of
understanding metaphor/simile. We define the meaning of a noun by
using probability. The meaning of a metaphor/simile is the meaning of
the tenor which is affected by the vehicle. We define {\it
``clarity''\/} and {\it ``novelty''\/}, which are based on the
quantity of information, as measures of the meaning difference between
metaphor/simile and tenor. The {\it clarity\/} of an attribute of a
metaphor/simile measures the decrease in entropy. The {\it novelty\/}
of an attribute of a metaphor/simile measures the increase in
rareness. We measured the meanings of similes by using Semantic
Differential and calculated their {\it clarity\/} and {\it novelty\/}. 
We showed that the {\it clarity\/} of an attribute of a simile agreed
well with the salience pattern of the vehicle, tenor, and simile. We
also showed that the {\it clarity\/} and {\it novelty\/} of a simile
agreed with the comprehensibility of the simile.}

\ekeywords{
  metaphor, simile, interpretation, Semantic Differential, quantity of information, clarity, novelty
  }

\begin{document}
\maketitle
\section{はじめに}
\label{sec:introduction}
比喩は自然言語に遍在する.たとえば,李\cite{Yi82}によると,小説と新聞
の社説とにおいて 比喩表現の出現率に大差はない.また,比喩を表現する者
(話し手)は,比喩により 言いたいことを端的に表現する.したがって,自然
言語処理の対象を科学技術文から評論や小説に拡大するためには,比喩の処理
が必要である.

比喩表現は,喩える言葉(喩詞)と喩えられる言葉(被喩詞)とからなる.話し手
は,それを伝達か強意かに用いる\cite{Nakamura77a}.伝達のために比喩を用
いるときは,伝達したい事柄が相手(聞き手)にとって未知であると話し手が判
断したときである.たとえば,「湖」は知っているが「海」は知らない聞き手
にたいして,「海というのは大きい湖のようなものだ」と言う場合である.強
意のために比喩を用いるときは,伝達したい事柄の 一つの側面を強調したい
ときである.たとえば,「雪のような肌」により「肌」の白さを強調する場合
である.

山梨\cite{Yamanashi88}は,(1)認定 (2)再構成 (3)再解釈の3段階により比
喩が理解されると述べている.認定とは,ある言語表現が文字通りの意味では
ない(比喩的意味である)ことに聞き手が気づくことをいう.再構成とは,喩詞
と被喩詞と文脈とから比喩表現の意味を構成することである.再解釈とは,比
喩表現の意味を 被喩詞に対する新たな視点として認識し,被喩詞に対する考
え方を聞き手が改めることである.

本稿では,強意の比喩に対しての,聞き手の再解釈を考察の対象とする.ただ
し,再解釈を
\begin{quote}
\begin{description}
\item[(3a)] 被喩詞の意味と比喩表現の意味との$\dot{\mbox{ず}}
\dot{\mbox{れ}}$を聞き手が認識する,
\item[(3b)] その$\dot{\mbox{ず}}\dot{\mbox{れ}}$が聞き手の考え方に反
映する
\end{description}
\end{quote}
という2段階に分け,(3a)を対象にする.なお,対象とする比喩が強意の比喩
であるので,聞き手にとって,喩詞の意味と被喩詞の意味とは既知である.

本稿では,「AのようなB」という形の比喩表現を考察の対象とする.また,比
喩表現が使われる文脈については考慮しない.第\ref{sec:formulation}章に
おいて,名詞の意味を確率により表現する.そして,比喩表現を捉える指標と
して明瞭性と新奇性とを定義する.これらは情報量に基づく指標である.明瞭
性は比喩表現における属性の不確定さを示す指標であり,新奇性は比喩表現の
示す事象の希少さに関する指標である.第\ref{sec:sd}章では,これら評価関
数の妥当性を実験により示す.3種類の値,
\begin{quote}
  \begin{description}
  \item[(1)] 喩詞・被喩詞・比喩表現の属性集合(SD法による\cite{Osgood57})
  \item[(2)] 喩詞・被喩詞・比喩表現における,属性の顕著性
  \item[(3)] 比喩表現の理解容易性
  \end{description}
\end{quote}
を測定する.(1)から明瞭性と新奇性とを計算し,それらが属性の顕著性と比
喩表現の理解容易性とを捉える指標として適当であることを示す.第
\ref{sec:summary}章は結びである.

\section{情報量を用いた 比喩の定式化}
\label{sec:formulation}

名\hspace{0.2mm}詞の意\hspace{0.2mm}味\hspace{0.1mm}を\hspace{0.1mm}確\hspace{0.2mm}率を利\hspace{0.2mm}用して表\hspace{0.2mm}現する.比\hspace{0.1mm}喩\hspace{0.1mm}表\hspace{0.1mm}現\hspace{0.1mm}の意\hspace{0.2mm}味は,喩\hspace{0.2mm}詞の意\hspace{0.2mm}味に影\hspace{0.2mm}響された被\hspace{0.2mm}喩\\詞の意味である.被喩詞の意味と比喩表現の意味との$\dot{\mbox
{ず}}\dot{\mbox{れ}}$は,情報量を用いた評価関数である,明瞭性と新奇性
とにより表現する\cite{Utiyama95a,Utiyama95b}.

まず,名詞の意味を定義する.次に,喩詞と被喩詞とから比喩表現を構成する
方法を述べ,最後に,明瞭性と新奇性とを定義する.

\subsection{名詞の意味}
\label{sec:meaning}

名詞の意味を図\ref{fig:meaning}のようなベイジアンネットワーク
\cite{Pearl88}で定義する.ベイジアンネットワークは有向非循環グラフであ
り,それぞれのノードは確率変数に対応する.ある確率変数が特定の値を取る
確率は,アークで連結しているノードとの局所的な関係により決まる.親ノー
ドのないノードをルートノードと言い,これらは互いに確率的に独立である.
ルートノードには事前確率が与えられる.また,子ノードのないノードをリー
フノードという.以下ではノードと確率変数とを区別しない.

図\ref{fig:meaning}におけるリーフノードを属性と呼び,属性の親ノードを
因子と呼ぶ.因子は,名詞においてはルートノードであるが,
\ref{sec:re-composition}節で述べる比喩表現においてはルートノードではな
い.属性の値を属性値と呼び,因子の値を極と呼ぶ.

属性は,「身体の大きさ」とか「力の強さ」とかである.属性値は,名詞の指
示物が明らかなときには,その指示物を観測することで決定する(ある属性値
の確率を 1 にして,他の属性値の確率を 0 にする)ことができると仮定する.
属性は確率的に独立とは限らない.因子は,理論上の仮定である.異なる因子
は互いに確率的に独立である.本稿では,名詞の指示物が与えられていない場
合を考察するため,属性が特定の属性値を取る確率は,その親ノードである因
子の極により決まる.

ある名詞の意味を$N$\hspace{-0.1mm}とし,\hspace{-0.2mm}その属性$F_{i}$が属性値$A_{ij}$からなるとする.
また,$F_i$の親ノード
である因子の集合を${\bf C}^i$とし,${\bf C}^i$の
任意の値割当を${{\bf c}^i} = (c_1^i,c_2^i,...,c_k^i,...,c_n^i)$とする.
$c_k^i$は,因子${\cal C}_k^i (\in {\bf C}^i)$の任意の極である.このと
き,ベイジアンネットワークの条件付き独立性\footnote{ノード\hspace{-0.1mm}$X$\hspace{-0.1mm}の値を\hspace{-0.1mm}$x$\hspace{-0.1mm},
\hspace{-0.1mm}$X$\hspace{-0.1mm}の親ノードの集合\hspace{-0.1mm}$\bf Y$\hspace{-0.1mm}の値割当を\hspace{-0.1mm}$\bf y$,\hspace{-0.1mm}それ以外のノード$Z$の値を\hspace{-0.1mm}
$z$\hspace{-0.1mm}とすると,$p(X\hspace{-0.1mm}=x|{\bf Y\hspace{-0.1mm}=y},Z\hspace{-0.1mm}=z) = p(X=x|{\bf Y = y})$となる.また,
ノード$X$と親ノードの集合を同じくするノード$W$の値$w$について,
$p(X=x,W=w|{\bf Y = y}) = p(X=x|{\bf Y = y})p(W=w|{\bf Y = y})$である.}
から,
\begin{equation}
  \label{eq:NFA}
  p(N.F_i.A_{ij}) = \sum_{N.{\bf c}^i} p(N.F_i.A_{ij}|{N.{\bf c}^i}) p(N.{\bf c}^i) = \sum_{N.{\bf c}^i} p(N.F_i.A_{ij}|N.{\bf c}^i) \prod_k p(N.c_k^i).
\end{equation}
第\ref{sec:sd}章では,SD法の結果から,属性値の確率$p(N.F_i.A_{ij})$を
計算する.本稿では,極の確率
$p\hspace{0.3mm}(\hspace{0.1mm}N.c_k^i\hspace{0.1mm})$\hspace{0.1mm}の設定法と極を条件とする属性値の条件付き確率\hspace{0.2mm}$p\hspace{0.3mm}(\hspace{0.1mm}N.F_i.A_{ij}\hspace{0.2mm}|\hspace{0.2mm}N.{\bf c}^i\hspace{0.2mm})$の設定法,\hspace{0.2mm}及び,\hspace{0.2mm}名詞の意味を表わすベイジアンネットワークの構築法については考察の範囲外である.

なお,
\begin{displaymath}
  p(N.F_i.A_{ij}) \geq 0,\mbox{\hspace{1em}}\sum_{j}p(N.F_i.A_{ij}) = 1.
\end{displaymath}
また,属性$F_i$のエントロピー$S(N.F_i)$は次式である.
\begin{equation}
  \label{eq:entropy}
  S(N.F_i) = - \sum_{j}p(N.F_i.A_{ij}) \lg p(N.F_i.A_{ij}).
\end{equation}
ただし,$\lg$は2を底とする対数.$S(N.F_i)$は,$F_i$の取りうる値に対す
る不確定性を示す.

\begin{figure}[htbp]
  \begin{center}
    \leavevmode
    \epsfile{file=meaning.ps,hscale=0.714,vscale=0.714}
    
  \end{center}
  \caption{名詞の意味}
  \label{fig:meaning}
\end{figure}

\subsection{比喩の再構成}
\label{sec:re-composition}

比喩表現の意味は,喩詞の意味に影響された被喩詞の意味であると我々は考え
る.前節では,名詞の意味をベイジアンネットワークで表現した.したがって,
喩詞の意味と被喩詞の意味とから比喩表現の意味を再構成する問題は,被喩詞
のベイジアンネットワークを喩詞のベイジアンネットワークの影響のもとで変
化させることと同じである.

簡単のため,喩詞と被喩詞とで互いのベイジアンネットワークの構造(ネット
ワークの形と親ノードと子ノードの間の条件付き確率)が同じであるとする.
このとき,比喩表現のベイジアンネットワークは (1) 喩詞の因子から対応す
る被喩詞の因子にアークを張る(図\ref{fig:metaphor}),(2) アークを張る前
の喩詞と被喩詞の因子状態\footnote{ある時点でのノードの状態とは,そのノー
ドに対応する確率変数が取り得る値全てに対して,その値と確率の組とを記述
することにより表される.}から,アークを張った後の被喩詞の因子状態を決
める(これが比喩表現の因子状態となる),(3)アークを張った結果として構成
されたベイジアンネットワークの一部分である被喩詞のベイジアンネットワー
クを,比喩表現のベイジアンネットワークと看做す,という3段階の手順で作
られる.ベイジアンネットワークの構造と因子の状態が決まれば,属性値の確
率は(\ref{eq:NFA})式から求まる.以下では,被喩詞の意味とは,喩詞との因
子間にアークが張られる前の意味を指し,比喩表現の意味とは,その後の被喩
詞の意味であるとする.

\begin{figure}[htbp]
  \begin{center}
    \leavevmode
   \epsfile{file=metaphor.ps,scale=0.714}
  \end{center}
  \caption{比喩表現の構成}
  \label{fig:metaphor}
\end{figure}

喩詞・被喩詞・比喩表現の意味を$V$・$T$・$M$とし,それらの属性$F_i$が属
性値$A_{ij}$からなるとする.また,$F_i$の親ノードである因子の集合を
${\bf C}^i$とし,${\bf C}^i$の任意の値割当を${{\bf c}^i} =
(c_1^i,c_2^i,...,c_k^i,...,c_n^i)$とする.$c_k^i$は,因子${\cal
C}_k^i(\in {\bf C}^i)$の任意の極である.このとき,比喩表現の属性値は,
(\ref{eq:NFA})式と同様に,
\begin{equation}
  \label{eq:MFA}
  p(M.F_i.A_{ij}) = \sum_{M.{\bf c}^i} p(M.F_i.A_{ij}|{M.{\bf c}^i}) p(M.{\bf c}^i) = \sum_{M.{\bf c}^i} p(M.F_i.A_{ij}|M.{\bf c}^i) \prod_k p(M.c_k^i)
\end{equation}
である.このとき,被喩詞と比喩表現とでベイジアンネットワークの構造は変
らないので,$p(M.F_i.A_{ij}|M.{\bf c}^i) = p(T.F_i.A_{ij}|T.{\bf
c}^i)$となる.一方,比喩表現の因子状態は,喩詞への従属関係ができたため,
被喩詞の因子状態とは異なる.因子${\cal C}_k^i$の極を${\cal P}^i_{kl}$
と表わすと,
\begin{equation}
  \label{eq:M}
  p(M.{\cal C}^i_k.{\cal P}^i_{kl}) = \sum_{m} p(M.{\cal C}^i_k.{\cal P}^i_{kl} | V.{\cal C}^i_k.{\cal P}^i_{km}) p(V.{\cal C}^i_k.{\cal P}^i_{km}).
\end{equation}
喩詞の因子と比喩表現の因子との間には(\ref{eq:M})式のような従属関係があ
るが,一旦,比喩表現の因子状態が決まったあとでは,そのような従属関係を
考えずに,(\ref{eq:MFA})式により比喩表現の属性値の確率を決めてよい.こ
のことは,ベイジアンネットワークの条件付き独立性から言える.したがって,
比喩表現の因子状態$p(M.{\cal C}^i_k.{\cal P}^i_{kl})$が決まれば,喩詞
の因子を条件とする比喩表現の因子の条件付き確率$p(M.{\cal C}^i_k.{\cal
P}^i_{kl}| V.{\cal C}^i_k.{\cal P}^i_{km})$を求める必要はない.

喩詞と被喩詞と比喩表現とで構造が同じベイジアンネットワークが作れること
と,喩詞と被喩詞の対応する因子状態から比喩表現の因子状態を決定できるこ
ととは第\ref{sec:sd}章で確かめる.

\subsection{比喩の再解釈}
\label{sec:re-interpretation}

比喩の再解釈において,被喩詞の意味と比喩表現の意味との$\dot{\mbox{ず}}
\dot{\mbox{れ}}$を聞き手が認識するために\\は,それらの意味を比較する必要
がある.以下に述べる明瞭性と新奇性とは,そのための指標である.これらの
指標は,比喩表現の属性と比喩表現自体とについて定義される.比喩表現の明
瞭性と新奇性とは,それぞれ,属性の明瞭性の和と新奇性の和とで近似できる.

\subsubsection{比喩表現の属性の明瞭性と新奇性}

\paragraph{属性の明瞭性}

被喩詞・比喩表現の意味 $T$・$M$ について,属性$F_i$の明瞭性$C(M.F_i)$
は次式である.
\begin{equation}
  \label{eq:clarity}
  C(M.F_i) = S(T.F_i) - S(M.F_i).
\end{equation}

被喩詞の属性\hspace{-0.1mm}$F_i$\hspace{-0.1mm}の確率分布は,\hspace{-0.2mm}被喩詞の因子が喩詞の因子に従属すること
により変動し,比喩表現の確率分布となる.その結果として,比喩表現の不確
定性が大きくなれば,$S(M.F_i)$の値は大きくなり,不確定性が小さくなれば,
$S(M.F_i)$の値は小さくなる.$C(M.F_i)$は,被喩詞の意味と比喩表現の意味
とを比べたときの,属性$F_i$に関する不確定性の減少量であり,その減少は
喩詞の意味に起因する.

ところで,喩詞と被喩詞の属性間の関係の指標として相互情報量を使うことが
考えられる.相互情報量は属性間で対称的な量である.しかし,比喩表現は非
対称的である.つまり,喩詞と被喩詞とを交換すると比喩表現の意味は変化す
る.したがって,相互情報量は比喩表現の意味の指標として適さない.明瞭性
は喩詞と被喩詞とに関して非対称的な量であることから,比喩表現の意味の指
標として,より適している\cite{Utiyama95a}.

\paragraph{属性の新奇性}
被喩詞の意味$T$\hspace{-0.1mm}について,\hspace{-0.2mm}属性$F_i$\hspace{-0.1mm}の属性値$A_{ij}$\hspace{-0.1mm}の確率が
$p(T.F_i.A_{ij})$のとき,比喩表現の意味$M$の属性$F_i$の新奇性
$N(M.F_i)$は次式である.
\begin{equation}
  \label{eq:novelty}
  N(M.F_i) = - \sum_{j} p(M.F_i.A_{ij}) \lg p(T.F_i.A_{ij}) - S(T.F_i).
\end{equation}

$T$において,$F_i$の値として$A_{ij}$が選ばれたときの情報量は$-\lg
p(T.F_i.A_{ij})$である.したがって,$S(T.F_i)$は,獲得が期待される情報
量の平均値である((\ref{eq:entropy})式参照).一方,$p(M.F_i.A_{ij})$は,
被\hspace{0.1mm}喩\hspace{0.1mm}詞\hspace{0.1mm}の因\hspace{0.1mm}子が喩\hspace{0.1mm}詞の因\hspace{0.1mm}子に従\hspace{0.1mm}属することにより変\hspace{0.1mm}動した,被\hspace{0.1mm}喩\hspace{0.1mm}詞の属\hspace{0.1mm}性\hspace{0.1mm}値の
確\hspace{0.1mm}率である.また,それぞれの属性値の情報量は$-\lg p(T.F_i.A_{ij})$であ
るから,$- \sum_{j} p(M.F_i.A_{ij}) \lg p(T.F_i.A_{ij})$は,喩えること
により獲得された被喩詞の属性値の情報量の平均値と考えられる.したがって,
獲得された平均情報量と獲得を期待された平均情報量(エントロピー)との差が
新奇性である.

新奇性が大きい値となるときは,被喩詞の意味において割り当てられた確
率が小さい属性値が,比喩表現の意味においては大きい確率となるときである.
つまり,稀な事柄を表わす比喩表現の属性ほど新奇性が大きい.

\subsubsection{比喩表現の明瞭性と新奇性}
全ての属性が確率的に独立な(因子を仮定する必要がない)ときには,比喩表現
の明瞭性と新奇性とは,それぞれ,属性の明瞭性の和と新奇性の和として表わ
される.

たとえば,被喩詞$T$\hspace{-0.1mm}が属性$F_1$\hspace{-0.1mm}と\hspace{-0.1mm}$F_2$\hspace{-0.1mm}とからなり,\hspace{-0.1mm}$F_1 \hspace{-0.1mm}=\hspace{-0.1mm}
\{A_{11},A_{12},...,A_{1i},...,A_{1n}\}$,$F_2 =
\{A_{21},A_{22},...,A_{2j},...,A_{2m}\}$であるとする.このとき,$T$が
一つの属性$F$からなり,$F = \{A_{1},A_{2},...,A_{k},...,A_{nm}\}$であ
ると考えることができる.\hspace{-0.1mm}ただし,$A_{k}$は$A_{1i}$と\hspace{-0.1mm}$A_{2j}$とが同時に
成立しているような属性値である.また,比喩表現$M$についても対応する属
性がある.

このとき,新奇性を示す(\ref{eq:novelty})式の第1項は,属性同士の確率的
独立性より,
\begin{eqnarray}
  \lefteqn{-\sum_{k}^{nm}p(M.F.A_{k}) \lg p(T.F.A_{k})} \nonumber \\
  &=& -\sum_{i}^{n}p(M.F_1.A_{1i})\lg p(T.F_1.A_{1i}) - \sum_{j}^{m}p(M.F_2.A_{2j}) \lg p(T.F_2.A_{2j}) \nonumber
\end{eqnarray}
のようになる.属性数が三つ以上になっても加法性は成立する.また,エント
ロピーも加法的である.したがって,全ての属性が互いに確率的に独立なとき
には,比喩表現の明瞭性と新奇性とは,それぞれ,属性の明瞭性の和と新奇性
の和とで表現できる.

実際には,属性同士は互いに確率的に独立ではないので,属性の明瞭性の和と
新奇性の和とは,それぞれ,比喩表現の明瞭性と新奇性の近似値となる.

\section{SD法による比喩表現の意味の測定}
\label{sec:sd}
一般に,名詞の意味は,客観的な意味と主観的な意味とに分かれる.客観的な
意味というのは,その名詞のカテゴリカルな意味であり,主観的な意味という
のは,その名詞にたいする聞き手の印象である.たとえば,「狼」という語は,
「動物,哺乳類」というカテゴリカルな意味と,「獰猛,陰険」などという
「狼」にたいする聞き手の印象とに分かれる\cite{Yamanashi88}.

本稿では,客観的な属性とは,任意の喩詞に対して,比喩表現と被喩詞とで対
応する属性値の確率が同じ属性であると定義する.客観的な属性では,明瞭性
と新奇性は 0 となるが,その逆は必ずしも成立しない.なお,主観的な属性
とは,客観的でない属性であると定義する.

本稿では,被喩詞の意味と比喩表現の意味との$\dot{\mbox{ず}}\dot{\mbox
{れ}}$が考察の対象であるので,客観的な属\\性は考慮しない.本稿で測定する
意味は,主観的な属性の集合(名詞の印象)である.SD法(Semantic
Differential)は,そのような属性集合を測定するのに適した手法である
\cite{Osgood57}.

本章では,3種類の値を測定する.
\begin{quote}
  \begin{description}
  \item[(1)] 喩詞・被喩詞・比喩表現の属性集合(SD法による)
  \item[(2)] 喩詞・被喩詞・比喩表現における,属性の顕著性
  \item[(3)] 比喩表現の理解容易性
  \end{description}
\end{quote}
(1)から計算される明瞭性・新奇性の値と(2)および(3)とが対応関係にあるこ
とを示す.

\subsection{実験の方法}
\label{sec:method}
三つの実験を行った.実験1では八つの具体名詞に対してSD法による評定を行
い,実験2・3において,実験1で評定された名詞からなる比喩表現(それぞ
れの実験で六つずつ)についての実験をした.全ての実験で,SD法により属性
集合を測定すると同時に,各名詞・比喩表現において顕著な属性と属性値も求
めた.実験2・3では,各比喩の``理解容易性''についても測定した.これら
の実験に用いた比喩表現は,中村\cite{Nakamura77b}から選択したものを修正
した比喩表現である.

SD法に用いた属性は,芳賀\cite{Haga90}と同じ25の形容詞対である(付録:表
\ref{tab:scale-mean}・\ref{tab:scale-mean-1}).各形容詞が属性値に相当
する.それぞれの刺激語句について,その印象を7段階で記入させた.このと
き,同時に,それぞれの刺激語句において顕著であると被験者が感じた属性の
属性値に対して,丸をつけさせた.丸をつける属性の数は制限しなかった.た
とえば,「向日葵」にたいして,「くらい-あかるい」のうち「あかるい」が
顕著であると被験者が感じたら「あかるい」に丸を付けさせた.

各語句につき1枚の評定表(具体名詞には図\ref{fig:sd-card}(a),比喩表現
には図\ref{fig:sd-card}(b),ただし属性は同じ)を用いた.評定表は小冊子
の形式で被験者に配った.名詞句・属性の順番は無作為である.実験は集団で
実施し,被験者への指示は黒板を用いて口頭で行った.また,実験2・3の被
験者は実験1の被験者の部分集合である.

\begin{figure}[htbp]
  \begin{center}
    \leavevmode
    \epsfile{file=sd-card.ps,hscale=0.9,vscale=0.9}
  \end{center}
  \caption{実験に用いた評定表の例}
  \label{fig:sd-card}
\end{figure}

\subsubsection{実験1} 
図\ref{fig:sd-card}(a)と同様の評定表を用いて,SD法により,八つの具体名
詞の印象を25の属性について評定させた.また,顕著な属性の属性値に丸を付
けさせた.
\begin{quote}
  \begin{description}
  \item[測定日\hspace{1em}] 95/01/12
  \item[被験者\hspace{1em}] 筆者と同じ研究室の日本人の大学(院)生16人(男13・女3)
  \item[刺激語句] 岩,猛獣,熊,向日葵,豚,蝶々,男,娘.
  \end{description}  
\end{quote}

\subsubsection{実験2}

実験2では,各々の評定表の冒頭で,各比喩の``理解容易性''を5段階(容易/
どちらかといえば容易/どちらともいえない/どちらかといえば困難/困難)
で評定させた(図\ref{fig:sd-card}(b)).また,実験1と同じ属性を用いて,
SD法により,六つの比喩表現の評定を求め,かつ,顕著な属性の属性値に丸を
付けさせた.この実験に用いた比喩表現は,実験1で評定した具体名詞からな
る.

\begin{quote}
  \begin{description}
  \item[被験者\hspace{1em}] 筆者と同じ研究室の日本人の大学(院)生12人(男 9 ・女 3)
  \item[測定日\hspace{1em}] 95/01/19  
  \item[刺激語句] \ \\
    \begin{tabular}{lll}
      岩のような男 & 熊のような男 & 猛獣のような男  \\
      豚のような男 & 蝶々のような男 & 向日葵のような男 
    \end{tabular}
  \end{description}
\end{quote}

\subsubsection{実験3}

実験2と同様に,図\ref{fig:sd-card}(b)のような評定表を用い,比喩表現の
``理解容易性''と各属性の評定値と顕著な属性の属性値とを求めた.ただし,
被喩詞が実験2と異なる.

\begin{quote}
  \begin{description}
  \item[被験者\hspace{1em}] 筆者と同じ研究室の日本人の大学(院)生13人(男10・女3)
  \item[測定日\hspace{1em}] 95/01/26
  \item[刺激語句] \ \\
    \begin{tabular}{lll}
      岩のような娘 & 熊のような娘 & 猛獣のような娘  \\
      豚のような娘 & 蝶々のような娘 & 向日葵のような娘 
    \end{tabular}
  \end{description}
\end{quote}

\subsection{実験の結果}
\label{sec:sd-hiyu-results}

SD法により測定した それぞれの刺激語句の属性について 平均評定値を付録:
表\ref{tab:scale-mean}・\ref{tab:scale-mean-1}に示す.

各語句の各属性の顕著性を付録:表\ref{tab:cnum}・\ref{tab:cnum-1}に示す.
顕著性とは,その属性に付けられた丸の数を被験者数で割った値である(空欄
は0).顕著性は0以上1以下であるが,これらの表では,左側の属性値(形容
詞)が顕著であったときには``$-$'',右側の属性値のときには``$+$''の符号
を付け,顕著な属性値を示した.同一の属性に対して,被験者間で,丸を付け
た属性値が食い違うことはなかった.測定した比喩表現がステレオタイプなも
のであったためであろう.

実験2・3で求めた比喩表現の理解容易性について表\ref{tab:easiness}に示す.
表\ref{tab:easiness}は,それぞれの比喩表現に対して,各段階に評定した被
験者の数(空欄は0)を示すクロス表である.このクロス表に荻野の数量化
\cite{Ogino83}を適用し,比喩表現を1次元に配置した.それを「指標」欄に
示す.この値が大きい比喩表現ほど 理解容易性が高いと我々は判断した.

\begin{table}[htbp]
   \begin{center}
      \leavevmode
      \begin{tabular}{|l|ccccc|l|}\hline
                 & 困難 & 困難$-$ & 中立 & 容易$-$ & 容易 & 指標\\ \hline
    岩のような娘 &    2 &     5 &    1 &     4 &    1 & 0.00\\
  猛獣のような娘 &    2 &     4 &    1 &     5 &    1 & 0.05\\
  蝶々のような男 &    1 &     4 &    2 &     4 &    1 & 0.08\\
    熊のような娘 &    2 &     3 &    2 &     4 &    2 & 0.14\\
向日葵のような男 &      &       &    3 &     7 &    2 & 0.41\\
  蝶々のような娘 &      &     2 &    2 &     4 &    5 & 0.47\\
    岩のような男 &      &       &    1 &     7 &    4 & 0.54\\
    豚のような娘 &      &       &    1 &     6 &    6 & 0.63\\
向日葵のような娘 &      &       &    2 &     3 &    8 & 0.72\\
    豚のような男 &      &       &      &     3 &    9 & 0.83\\
    熊のような男 &      &       &      &     2 &   10 & 0.89\\
  猛獣のような男 &      &       &      &       &   12 & 1.00\\ \hline
       \end{tabular}

       \vspace{\baselineskip}

       \begin{minipage}{11cm}
         「困難$-$」・「中立」・「容易$-$」は,それぞれ,「どちらかといえば困難」・「どちらともいえない」・「どちらかといえば容易」に対応する.
       \end{minipage}

   \end{center}
   \caption{比喩の理解容易性}
   \label{tab:easiness}
\end{table}

\subsection{実験結果の解釈}
\label{sec:interpretation}
第\ref{sec:introduction}章で,比喩が,(1)認定,(2)再構成,(3)再解釈,
の3段階から理解されると述べた.本節では,(1)の認定段階は考察外である.
(2)の再構成については,表\ref{tab:scale-mean}・\ref{tab:scale-mean-1}
に示された評定値から,(a)喩詞と被喩詞と比喩表現とで構造が同じベイジア
ンネットワークを構成でき,(b) 喩詞と被喩詞の対応する因子の状態から比喩
表現の対応する因子の状態を設定できることを示す.そして,比喩表現の意味
を喩詞と被喩詞の意味から構成することが基本的に可能であると仮定する.

(3)の再解釈において,``被喩詞の意味と比喩表現の意味との$\dot{\mbox{ず}}
\dot{\mbox{れ}}$を認識した''と判定するた\\めの必要十分条件は,一般に,明
らかではない.しかし,以下の2点は必要条件に含まれるであろう.
\begin{itemize}
\item 比喩表現の属性について,その属性が,被喩詞のときに比べて,強調さ
れたのか,それとも抑制されたのか,が判断できること,
\item ``喩詞において顕著な属性は比喩表現においても顕著である''などとい
う属性の顕著性のパターンを表現できること.
\end{itemize}
さらに,意味の$\dot{\mbox{ず}}\dot{\mbox{れ}}$は,比喩表現の理解容易性
にも影響すると考えられる.

再解釈についての考察では,これら三つの観点と明瞭性・新奇性との関係につ
いて述べる.再構成が可能であると仮定したので,明瞭性・新奇性は 被喩詞
の属性と比喩表現の属性とから 直接 計算する.

\subsubsection{比喩の再構成}

比喩の再構成が基本的に可能なことを示すために,(1) 喩詞と被喩詞と比喩表
現とで構造が同じベイジアンネットワークを作ることができ,(2) 喩詞と被喩
詞の対応する因子の状態から比喩表現の対応する因子の状態を設定できること
を示す.

表\ref{tab:scale-mean}・\ref{tab:scale-mean-1}のデータから属性間相関行
列を作り,それに対して因子分析を行なった.相関行列が特異行列となるのを
避けるために,他の属性との相関係数の2乗和が小さい方の属性から6個を削
除した.削除した属性は,「かっぱつな-かっぱつでない」,「しずかな-さわ
がしい」,「おいしい-まずい」,「まがった-まっすぐな」,「おそい-はや
い」,「ばかな-かしこい」である.

主因子法により因子解を求め,バリマックス回転を施した.設定した因子数は
4である.これらの因子により,全体の分散の約87%が説明できる.表
\ref{tab:Floading}はバリマックス回転後の各属性の因子負荷量である.

\begin{table}[htbp]
   \begin{center}
      \leavevmode
      \begin{tabular}{|l|cccc|}\hline
                          & 第1因子 & 第2因子 & 第3因子 & 第4因子\\ \hline
              わるい-よい &   \ 0.94 &    -0.22 &   \ 0.11 &    -0.10\\
      ただしい-まちがった &    -0.92 &   \ 0.16 &   \ 0.02 &   \ 0.08\\
    ふつうの-ふつうでない &    -0.88 &     0.00 &    -0.10 &   \ 0.09\\
          きらいな-すきな &   \ 0.87 &    -0.38 &   \ 0.14 &    -0.03\\
    しんせつな-ざんこくな &    -0.83 &    -0.09 &    -0.33 &   \ 0.24\\
たいせつな-たいせつでない &    -0.78 &   \ 0.21 &    -0.23 &   \ 0.10\\
      みにくい-うつくしい &   \ 0.66 &    -0.63 &   \ 0.18 &   \ 0.08\\
          くらい-あかるい &   \ 0.64 &    -0.61 &   \ 0.03 &    -0.28\\
            とおい-ちかい &   \ 0.64 &    -0.02 &   \ 0.31 &    -0.20\\ \hline
            はでな-じみな &    -0.26 &   \ 0.89 &    -0.10 &   \ 0.18\\
            おもい-かるい &   \ 0.18 &    -0.83 &   \ 0.49 &   \ 0.04\\
        あたらしい-ふるい &    -0.21 &   \ 0.75 &    -0.16 &   \ 0.38\\
かわりにくい-かわりやすい &    -0.32 &    -0.72 &   \ 0.33 &    -0.38\\ \hline
            つよい-よわい &   \ 0.22 &    -0.11 &   \ 0.88 &    -0.26\\
        おおきい-ちいさい &   \ 0.23 &    -0.52 &   \ 0.82 &   \ 0.00\\
おとこらしい-おんならしい &   \ 0.30 &    -0.30 &   \ 0.76 &    -0.24\\ \hline
        まるい-かどばった &    -0.12 &   \ 0.11 &    -0.36 &   \ 0.89\\
          つめたい-あつい &   \ 0.42 &    -0.23 &    -0.14 &    -0.83\\
        かたい-やわらかい &   \ 0.07 &    -0.29 &   \ 0.49 &    -0.74\\ \hline
                 因子寄与 &   \ 6.47 &   \ 4.12 &   \ 3.14 &   \ 2.71\\ \hline
       \end{tabular}
   \end{center}
   \caption{属性の因子負荷量}
   \label{tab:Floading}
\end{table}

\begin{figure}[htbp]
 \vspace{-1mm}
  \begin{center}
    \leavevmode
    \epsfile{file=Floading.ps,hscale=0.714,vscale=0.714}
      \vspace{-1.5mm}
  \end{center}
  \caption{属性の因子負荷量から生成されるネットワークの一部}
    \vspace{-1mm}
  \label{fig:Floading}
\end{figure}

\paragraph{ベイジアンネットワークの構成}

表\ref{tab:Floading}から図\ref{fig:Floading}のようなネットワークが生成
できる.そのアークには,因子負荷量が重みとして設定されている.このネッ
トワークの各因子に因子得点を与えれば,各語句の属性の評定値(平均 0 ,標
準偏差 1 に正規化されている)を計算できる.ある属性の評定値は,その全て
の親ノードである因子の因子得点にアークの重みを掛けた値の和として与えら
れる.

このネットワークは,喩詞・被喩詞・比喩表現に共通である.また,各因子は
互いに無相関なので,これを確率的にも独立であると仮定する.すると,アー
クの重みから,因子を条件とする属性の条件付き確率を設定できれば,喩詞・
被喩詞・比喩表現で構造が同じベイジアンネットワークが構成できる.更に,
各語句の因子の因子得点を,その因子の極に与える事前確率に変換できれば,
因子の状態を各語句に固有に設定できる.本稿では,これらの変換は可能であ
ると仮定する.

\vspace{-0.4mm}
\paragraph{喩詞と被喩詞の因子状態からの比喩表現の因子状態の推定}
喩詞と被喩詞の因子得点を説明変数,比喩表現の因子得点を目的変数として,
48因子($\mbox{4因子}\times\mbox{12の比喩表現} $)について重回帰分析をし
たところ,重相関係数が 0.89 であった.また,楠見\cite{Kusumi93}による
同様な実験においても,重相関係数は 0.88 以上である.これより,喩詞と被
喩詞の対応する因子の因子得点から比喩表現の対応する因子の因子得点を推定
できると言える.したがって,因子得点と因子状態との相互変換ができれば,
喩詞と被喩詞の因子状態から比喩表現の因子状態を推定できる.

以上から,我々は,喩詞の意味と被喩詞の意味とから比喩表現の意味を構成可
能であると仮定する.

\vspace{-0.4mm}
\subsubsection{比喩の再解釈}

\paragraph{明瞭性・新奇性と顕著性との関係}
比喩の再構成が可能であると仮定したので,明瞭性・新奇性は被喩詞の属性と
比喩表現の属性とから計算した.各属性の評定値は,刺激語句における,それ
ぞれの属性値の出現頻度の比を反映する\cite{Charles57}ので,属性値の確率
は,平均評定値を線形変換して求めた\footnote{$\mbox{左側の属性値の確率}
= (7-\mbox{平均評定値})/6$.}.たとえば,「向日葵のような娘」に対して
は,「くらい-あかるい」の平均評定値が$6.7$であるので,「くらい」の確率
を$0.05$,「あかるい」の確率を$0.95$とした.

図\ref{fig:charnum}に,明瞭性(Clarity)・新奇性(Novelty)と比喩表現の顕
著性との関係を示す.図\ref{fig:charnum}において``Minimum''で示される点
線は,任意の明瞭性に対して,取りうる新奇性の最小値のプロットである
\footnote{最小値はシミュレーションにより求めた.3,0000属性について,被
喩詞と比喩表現とで属性値の確率を独立に設定して,明瞭性と新奇性とを求め,
明瞭性が共通の新奇性のうちで 最小値となるものをプロットした.}.三種類
の記号(菱形,十字,四角)で表示されているプロットは,それぞれ,顕著性が
$[0..1/3)$,$[1/3..2/3)$,$[2/3..1]$の属性である.

図の右側には,被喩詞において顕著でなかった属性で,比喩表現において顕著
になった属性が位置している.左側には,被喩詞において顕著であった属性で,
比喩表現において顕著でなくなった属性が位置している.図には,いくつかの
属性について,``喩詞-被喩詞:その属性において確率の高い方の属性値''と
いう形式で表示してある.図から判断すると,属性を顕著性で分類しているの
は,主に,明瞭性である.

\begin{figure}[htbp]
  \begin{center}
    
    \epsfile{file=fig-charnum.ps,scale=1.0}
  \end{center}
  \caption{明瞭性・新奇性と比喩表現の顕著性}
  \label{fig:charnum}
\end{figure}

次に,明瞭性と顕著性の変化との関係を述べる.$H = \mbox{比喩表現の顕著
性} - \mbox{被喩詞の顕著性}$ とすると,当該の属性が強調されたかどうか
は,$H$により,以下のように示される.
\begin{quote}
  \begin{tabular}{llllll}
    $H < 0$ & 抑圧,& $H = 0$ & 無変化,& $H > 0$ & 強調.
  \end{tabular}
\end{quote}
$H$\hspace{-0.1mm}が正の方向に大きいほど,その属性は強く強調されたことになり,\hspace{-0.1mm}負の方
向に大きいほど,\hspace{-0.1mm}強く抑圧されたことになると言える.なぜなら,顕著性とは,
その属性を顕著であると認めた被験者の割合であるので,その変化量は その
属性を顕著だと認めた被験者の割合の増減を示すからである.この$H$と明瞭
性との相関を全ての比喩表現の属性(12の比喩表現 $\times$ 25属性 $=$ 300
属性)について求めたところ,その相関係数は 0.89 であった.

これらのことは,``属性が強調されたのか抑圧されたのか''を示す指標として
明瞭性が適当なことを示している.

\paragraph{喩詞・被喩詞・比喩表現における,属性の顕著性のパターン}
本小節では,ある属性の顕著性が1/4を超えているとき,その属性を顕著な属
性であるとする.顕著な属性については,その属性の顕著性に付けられたプラ
スとマイナスとに応じて,$\pm 1$により表現する.左側の形容詞が顕著なと
き$-1$であり,右側のときには$+1$である.顕著でない属性は$0$で表わす.
すると,全ての属性について,喩詞・被喩詞・比喩表現における顕著性のパター
ンが$+1$と$-1$と$0$とで表せる.

本実験においては,喩詞と被喩詞と比喩表現とで,顕著な属性において符号が
食い違うことはなかった.そのため,単に,$1$により顕著な属性を表す.喰
い違いが生じなかったのは,評定した比喩表現が典型的なものであったためで
あろう.

表\ref{tab:salience-patterns}には,喩詞・被喩詞・比喩表現の顕著性のパ
ターンと明瞭性との関係を示す.本実験では,パターン``1 1 0''とパターン
``1 1 1''とを除いた6パターンが生じた.

生じなかった2パターンのうち,パターン``1 1 0''は,喩詞と被喩詞とで顕
著な属性が比喩表現において顕著でなくなるパターンである.このようなパター
ンは考えがたい.パターン``1 1 1''に該当するものは,被喩詞で顕著であっ
た属性が喩詞の作用により更に顕著になる場合である.たとえば,「山のよう
な大男」の大きさが そのような属性である.このパターンに該当する属性で
は,比喩表現の顕著性が被喩詞の顕著性よりも大きくなるので,明瞭性は正の
値となる.ただし,典型的な比喩表現においては,被喩詞の顕著でない属性が,
喩詞の作用により,顕著になる.したがって,属性の顕著性のパターンが``1
1 1''になることは少ないと考えられる.

\begin{table}[htbp]
  \begin{center}
    \leavevmode
    \begin{tabular}{|c|ccc|rcc|} \hline
      パターン & 喩詞 & 被喩詞 & 比喩表現 & 頻度 & 明瞭性の平均値 &明瞭性の標準偏差\\ \hline
      A & 1 & 0 & 1 &  11 & $+0.43$ & $0.14$\\
      B & 0 & 0 & 1 &   8 & $+0.22$ & $0.20$\\ \hline
      C & 1 & 0 & 0 &   5 & $+0.11$ & $0.09$\\
      D & 0 & 0 & 0 & 264 & $+0.01$ & $0.08$\\ \hline
      E & 0 & 1 & 1 &   2 & $-0.23$ & $0.08$\\
      F & 0 & 1 & 0 &  10 & $-0.48$ & $0.10$\\ \hline
    \end{tabular}
    
    \vspace{\baselineskip}
    
    明瞭性の平均値と標準偏差は,各パターンに属する属性の明瞭性から計算した.
    
  \end{center}
  \caption{顕著性のパターンと明瞭性}
  \label{tab:salience-patterns}
\end{table}

表\ref{tab:salience-patterns}において,パターンが異なれば それに属する
比喩表現から計算される明瞭性の分布も異なるかどうかを,順位和検定により,
検定した.その結果,全ての隣接パターン間において,有意水準5%で分布の
ズレが確認された.このことは,``喩詞において顕著な属性は比喩表現におい
ても顕著である''などという定性的な表現を 明瞭性が定量的に表現できるこ
とを示している.

\paragraph{比喩表現の理解容易性と明瞭性・新奇性}

比喩表現の明瞭性・新奇性を,比喩表現を構成する属性の明瞭性の和と新奇性
の和とで近似する(\ref{sec:re-interpretation}節).比喩表現の理解容易性
(表\ref{tab:easiness})と比喩表現の明瞭性/新奇性との相関係数は,それぞ
れ,$0.65$/$-0.75$である.また,明瞭性と新奇性とを説明変数,理解容易
性を目的変数として重回帰分析をした結果,明瞭性と新奇性とから計算される
理解容易性の予測値と測定された理解容易性との相関係数は 0.80 であった.

SD法により評定された語句間の関係の指標としては,一般に,Dスコアが用い
られる.
\begin{displaymath}
  \mbox{Dスコア} = \sqrt{\sum_i d_i^2}.
\end{displaymath}
ただし,\hspace{-0.2mm}$d_i$は,\hspace{-0.2mm}属性$_i$における \hspace{-0.2mm}被喩詞と比喩表現の平均評定値の差であ
る.\hspace{-0.2mm}従来の研究では,\hspace{-0.2mm}D\\スコアを比喩理解に関わる要因の一つとしている
\cite[など]{Tourangeau82,Kusumi87}.

Dスコアと理解容易性との相関係数は$-0.61$である.また,明瞭性・新奇性か
ら計算される理解容易性の予測値とDスコアとの相関係数は$-0.79$である.し
たがって,予測値とDスコアが正規分布をしていると仮定すれば,有意水準 1
%で,明瞭性・新奇性からの予測値の方が Dスコアよりも理解容易性との相関
が高い.

\section{おわりに}
\label{sec:summary}
比喩の再解釈のモデルとして,情報量にもとづく評価関数である 明瞭性と新
奇性とを提案した.第\ref{sec:sd}章の実験では,五つの標本(被験者・属性・
喩詞・被喩詞・比喩表現)を用いた.これらの標本は,網羅的でもないし,無
作為抽出されたものでもない.したがって,これらの標本における明瞭性・新
奇性の性質から,母集団における それらの性質を統計的に推測することはで
きない.しかし,本稿の実験に用いられた標本に関しては,
\begin{itemize}
\item 属性の顕著性の変化を示す指標として明瞭性が適当であること,
\item 属性の顕著性のパターンに応じて明瞭性が分布していること,
\item 比喩の理解容易性の指標として,明瞭性と新奇性とが適当なこと
\end{itemize}
を示した.

本稿では,比喩の再解釈を考察の対象にした.比喩の認定と再構成とは今後の
課題である.比喩表現には,その表現が比喩であることが陽には示されていな
いもの(隠喩)も多い.そのような比喩を認定するためには,文脈からの意味の
逸脱などを検出する必要がある.比喩の再構成のためには,喩詞と被喩詞と比
喩表現とに共通なベイジアンネットワークを作成し,喩詞と被喩詞の因子状態
から比喩表現の因子状態を推定する必要がある.いずれの場合にも,名詞の属
性集合が重要である.本稿では名詞の属性集合をSD法により測定した.今後は,
これをコーパスから抽出することを考えている.


\acknowledgment

本稿に対して適切な助言を下さった,本学 山本幹雄 講師に感謝する.

\bibliographystyle{jnlpbbl} 

\bibliography{metaphor}


\section*{付録}
\label{sec:appendix}

SD法により得られた平均評定値を表\ref{tab:scale-mean}と表
\ref{tab:scale-mean-1}とに示す.また,各概念の各属性の顕著性を表
\ref{tab:cnum}と表\ref{tab:cnum-1}とに示す.これらの表では,``のような''
という文字列を省略した.例えば,``猛獣のような男''なら``猛獣男''として
記載されている.

\input{appendix.tex}


\clearpage

\begin{biography}
\biotitle{略歴}
\bioauthor{内山 将夫}{

1992年筑波大学第三学群情報学類卒業.現在,筑波大学大学院工学研究科博士
課程に在学中.知識獲得に興味がある.言語処理学会,情報処理学会の学生会
員.

}
\bioauthor{板橋 秀一}{

東北大学 工学部 通信工学科 卒業(1964).東北大学 大学院 工学研究科 電気
及び通信工学専攻 博士課程 単位取得退学(1970).東北大学 電気通信研究所 
助手(1970). 通産省 工業技術院 電子技術総合研究所 技官(1972).同 主任研
究官(1974).ストックホルム王立工科大学 客員研究員(1977-78). 筑波大学 電
子・情報工学系 助教授(1982).筑波大学 電子・情報工学系 教授(1987). 専
門は音声・自然言語・画像の処理・理解.1982年より,(社)日本電子工業振
興協会の音声入力方式分科会主査,音声入出力方式専門委員会委員長として音
声データベースの検討・構築に従事している.

}

\bioreceived{受付}
\biorevised{再受付}
\bioaccepted{採録}

\end{biography}

\end{document}
