\documentstyle[epsf,nlpbbl]{jnlp_e_b5}

\setcounter{page}{57}
\setcounter{巻数}{4}
\setcounter{号数}{2}
\setcounter{年}{1997}
\setcounter{月}{4}
\受付{May}{13}{1996}
\採録{November}{11}{1996}

\setcounter{secnumdepth}{3}

\title{}
\author{}
\jkeywords{}

\etitle{On Semantic Interpretation\\ of Japanese Compound Nouns}

\eauthor{Masato Shiraishi\affiref{FukuokaUnivEdu}	\and
	Masao Yokota\affiref{FukuokaInsTech}}

\headauthor{Shiraishi,~M. and Yokota,~M.}
\headtitle{On Semantic Interpretation of Japanese Compound Nouns}

\affilabel{FukuokaUnivEdu}
          {Fukuoka University of Education, Faculty of Education}
          {Fukuoka University of Education, Faculty of Education}
\affilabel{FukuokaInsTech}
          {Fukuoka Institute of Technology,
 Language and Information Laboratory}
          {Fukuoka Institute of Technology, 
 Language and Information Laboratory}

\eabstract{ Toward \hspace{0.5mm}the realization of a natural\hspace{0.5mm} language  \hspace{0.5mm}understanding \hspace{0.5mm}system for clinical records, the authors have analyzed a large number
of discharge summaries (a kind of\hspace{-0.3mm} clinical record). \hspace{-1mm}In \hspace{-0.5mm}the \hspace{-0.5mm}records
many Japanese compound nouns appear \hspace{-0.5mm}due \hspace{-0.5mm}to \hspace{-0.5mm}el- lipsis. Therefore, it is
very essential to the understanding system to cope with them. This
 \hspace{-0.3mm}paper \hspace{-0.3mm}describes\hspace{-0.5mm} a\hspace{-0.5mm} system\hspace{-0.3mm} to\hspace{-0.3mm} paraphrase\hspace{-0.3mm} compound\hspace{-0.3mm} nouns\hspace{-0.3mm} by\hspace{-0.3mm} restoring\hspace{-0.3mm}
their\hspace{-0.3mm} ellip- tical constructions in use of their {\it semantic
categories}~\cite{Yokota94} according to the {\it Mental-image
directed semantic theory}~\cite{Yokota88a,Yokota91a}. This system
consists of four major processors : {\it "Word segmentation
processor," "Restoration processor," "Hierarchical relation detector"}
and {\it "Sentence generator"}, and possesses two types of dictionary
: "Word dictionary" and "Hierarchy dictionary". The former of the
dictionaries assigns a semantic category, etc. to each noun, and the
latter contains the hierarchic relations among the concepts of objects
(one of the semantic categories of nouns). The experimental result of
the system has proven to be fairly successful.  }

\ekeywords{Understanding System, Semantic Categories, Compound noun, Paraphrasing System}

\begin{document}

\maketitle

\section{Introduction}
We have been carrying out studies in order to realize a natural
language understanding system~\cite{Yokota91b} to be used in the
field of medicine. The data that could be processed on the system
would be a natural language sentence (i.e., in a combination of
English and/or Japanese) which emerges in many discharge summaries,
that is, a kind of clinical record. Various linguistic phenomena
(e.g., slightly ill-formed sentences and ellipsis of a Japanese
postpositional particle) where it is difficult for the conventional
parsing technique to analyze these sentences were
observed~\cite{Yokota88b}. In particular, the ellipsis of a Japanese
postpositional particle is a very important problem. Because its
frequency of occurrence is extremely high in the summaries, it is
difficult for the conventional lexical analyzers to separate its
object into specific morphemes and a compound noun generated by such
ellipsis implies some problems concerned with ambiguous
meanings~\cite{Miyazaki84,Takeda87}.


Moreover, such a compound noun is often a significant word or phrase
(i.e., key words) which determines the semantic content of the full
sentence~\cite{Ishizaki90,Miyazaki84,Takeda87}. In usual natural
language sentences in addition to those in discharge summaries, the
semantic processing or understanding of the noun is an unavoidable
subject. Generally, in Japanese, various hiragana, katakana, kanji and
alphanumeric characters are used in sentences and portions of words
may not be clear because Japanese is an agglutinative language. In
particular, for words written in kanji characters, the natural
abbreviation of Japanese postpositional particles by concatenation
often results in a new word. Many technical terms used in academic
documents have been affected by this type of
concatenation~\cite{Miyazaki84}. Often in the summaries, a noun
resulted from concatenation according to the styles specific to
medical records is not a proper Japanese noun; this situation also
occurs with compound nouns. Consequently, in order to develop a system
to understand the summaries it is very important to realize the
semantic processing of a compound noun which is idiomatically used or
occurs as a result of abbreviation.

\begin{figure}[t]
\begin{center}
\unitlength=1mm
\hspace{10mm}
\epsfile{file=04fig1new.eps}
\end{center}
\vspace{-2mm}
\caption{Configuration of the paraphrasing system}
\vspace{2mm}
\end{figure}


We developed an experimental system that can estimate and restore
certain words, which were mainly Japanese postpositional particles.
In this paper, the outline of this system and the experimental results
of its application to such nouns are described. This restoring
technique, which is adopted to the restoration of an elliptical word,
makes use of the classification of the individual nouns, into which a
compound noun is separated, five semantic categories~\cite{Yokota94}
by the Mental-image directed semantic theory
(MIDST)~\cite{Yokota88a,Yokota91a} and a hierarchical relationship
among many nouns of the category "object".


In the study related with the processing of a compound noun,
methods and devices to analyze lexical ambiguities or identify the
semantic features of compound nouns have been proposed in the
literatures~\cite{Ishizaki90,Miyazaki84,Takeda87}. However, these
methods do not deal with restoration of the elliptical form of such a
noun. Furthermore, it seems that our proposal is the only one that is
concerned with processing which estimates and restores elliptical
words or phrases from a compound noun.


This paper is structured as follows: First, Section 2 explains
the semantic categories of the nouns adopted in this system.
Succeedingly, Section 3 describes the outline of the paraphrasing
system. Section 4 examines and considers the results of the
paraphrasing experiment that used the outlined system. Lastly, in
Section 5 we provide a brief summary and describe some future
subjects.

\section{Semantic Categories}
\vspace*{-0.1cm}

It is possible to efficiently estimate and restore the elliptical
words around the compound noun using the respective semantic
categories of its constituent nouns. In this case, the five {\it
semantic categories} related to the MIDST are defined~\cite{Yokota94}.
The names of these categories are object:~Obj, event:~Evt,
attribute:~Atr, value:~Val and relation:~Rel.  In this theory, it is
supposed that the meaning of a word consist of the two parts, {\it
"Concept part"} and {\it "Connection part"}. The former is some
logical expressions of mental image evoked by the word, which is free
from any natural language. The latter works for combining concept
parts of its own and another.

Generally, a full meaning structure of one sentence can be produced by
combining concept parts of its constituent words according to their
connection parts.  The MIDST provides general definitions of the
semantic descriptions about the five categories. In short, we only
describe the definitions about the concept parts of an Obj and an Evt
noun as follow.

\subsection{Obj noun}
\vspace*{-0.1cm}
 The concept part of an Obj noun is defined as shown in formula
(\ref{obj}). The Obj noun doesn't have any connection part.
\vspace*{-0.1cm}
\begin{equation}
O(x_0) \stackrel{\triangle}{=} O^{\mbox{\tiny $+$}}(x_0) \wedge
O^{\mbox{\tiny $+\!+$}}(x_0,x_1,...,x_n) \label{obj}
\end{equation}
\begin{tabbing}
 where, \= $\stackrel{\triangle}{=}$ \ \ \ \= : The meta symbol indicating
 that the left side is defined by the right side,\\
	\> $\wedge$ \>: Logical operator AND,\\
        \> $x_i$ \>: Variables or constants which represent an entity
 (i.e. objects/events),\\
        \> $O^{\mbox{\tiny $+$}}$ \>: Concepts about the properties 
of entity $x_0$,\\
        \> $O^{\mbox{\tiny $+\!+$}}$ \>: Concepts about the relations
 among entities $(x_0,x_1,...,x_n)$.
\end{tabbing}

\subsection{Evt noun}
The concept part of an Evt noun is defined as shown in formula (\ref{evt}).
\begin{equation}
 E(y_0) \stackrel{\triangle}{=} E^{\mbox{\tiny $+\!+$}}(y_0,y_1,...,y_n) 
\wedge E^{\mbox{\tiny $+$}}(y_0) \wedge C_1(y_1) \wedge ... 
\wedge C_n(y_n) \label{evt} 
\end{equation}
\begin{tabbing}
 where,  \= $E^{\mbox{\tiny $+$}}$ \ \ \= : Concepts about the 
situations of an event $y_0$ during a time interval $[t_i,t_j]$,\\
        \> $E^{\mbox{\tiny $+\!+$}}$ \> : Concepts about the relations 
among entities $y_0,y_1,...,y_n$ during same interval,\\
        \> $C_i$ \> : Concepts about the entity $y_i$ that appears 
in the event $y_0$,\\
        \>       \> \ \ \ and can be replaced recursively with $O$ 
or $E$ concerning the entity $y_i$.
\end{tabbing}

\section{Paraphrasing System for Elliptical Words}
The system can process the direct concatenation of two nouns, at
present. The configuration of this paraphrasing system is shown in
Fig.1. This system consists of four major processors: {\it "word
segmentation processor," "restoration processor," "hierarchical
relation detector"} and {\it "sentence generator,"} and possesses two
types of dictionaries: "word dictionary" and "hierarchy dictionary."
The former of the dictionaries assigns a semantic category and other
information to each noun. The latter also contains the hierarchical
relations among the Obj concepts (one of the semantic categories of
nouns).

\subsection{Word Dictionary}
The information which is registered in the word dictionary is
expressed by the list structure of the stem of the word, the semantic
category and the combination description, as shown in formula
(\ref{stem}).
\begin{equation}
         [W_d, [L_c, I_c]] \label{stem}
\end{equation}
\begin{tabbing}
 where,  \= $W_d$ \ \= : Stem of word,\\
         \> $L_c$ \> : Semantic category (Obj, Evt, Atr, Val or Rel),\\
         \> $I_c$ \> : Combination description. 
\end{tabbing}

\begin{table}
\begin{center}
\caption{Definitions of  $I_c$.}

\begin{tabular}{c|l} \hline
Semantic category & \multicolumn{1}{|c}{$I_C$}\\ \hline
Obj &	$[ \; < O_i \; |\; \phi > \; ]$ \\[1.5ex] 
Evt &   $[\; \{\; < [ $ def$(O_i),$ dep$(J_i)\; ]\; |\; [ \; $cat$(C_i) ,$
 dep$( < J_i \; | \; \phi> )\; ] > \} ^+ \; ]$ \\[1.5ex]
Atr &   $<  [ \; \{ \; [ \; $cat$(C_i) ,\; $
dep$(J_i)\; ] \; \} ^5 \; ] \; | \; [$ dep$(J) \; ] >$ \\[1.5ex]
Val &  $<  [ \; \{ \; [\; $cat$(C_i) ,\; $
dep$(J_i)\; ] \; \} ^5\; ]\; |\; [ \;$dep$(J)\; ] >$ \\[1.5ex] 
Rel &  $[\; \{ \; [\; $cat$(C_i) ,\; $dep$(J_i) \; ] \; \} ^5\; ]$ \\[1.5ex] 
\hline
\multicolumn{2}{l}{\footnotesize
$\{ \}^n$ is repetition of the list from 1 to n times.}\\
\multicolumn{2}{l}{\footnotesize
$\{ \}^+$ is repetition of the list of one or more times.}\\
\multicolumn{2}{l}{\footnotesize
$< X | Y>$ indicates that $X$ or $Y$ may be used.}\\
\multicolumn{2}{l}{\footnotesize
$\phi$ indicates that the element is empty.}\\
\end{tabular}\\
\end{center}
\end{table}

Table 1 shows the structure of this $I_c$ for every semantic category.
In this table, the variables and identifiers represent the following.
\begin{tabbing}
\ \ \ \ \ \ \= $O_i$ \ \ \ \ \ \ \= : Noun of Obj,\\
          \> $J_i,J$ \> : Postpositional particle,\\
          \> $C_i$ \>: Semantic category,\\
         \> cat($C_i$) \>: Identifier that limits the candidate to be
 combined by this noun \\
         \> \> \ \ to the category of $C_i$,\\
         \> def($O_i$) \>: Identifier with the default of $O_i$,\\
         \> dep($J$) \> : Identifier with the connectable dependent
 of $J$.
\end{tabbing}

Some examples of these description forms are shown as follow.
\begin{tabbing}
        \= Obj noun\  \= : \= [心,[Obj,心臓]],\\
        \> Evt noun \> : \> [投与,[Evt,[[def(医療従事者),dep(が))],
[def(薬物),dep(を)]]]],\\
        \> Atr noun \>: \> [湿度,[Atr,[dep(である)]]],\\
        \> Val noun \>: \> [高級,[Val,[dep(な)]]],\\
        \> Rel noun \>: \> [間,[Rel,[[cat(Obj),dep(における)],[cat(Evt),
dep(している)], \\
        \>          \>  \> [cat(Atr),dep(における)],[cat(Val),
dep(における)],\\
        \>          \>  \> [cat(Rel),dep(と)]]]].
\end{tabbing}

\subsection{Hierarchy Dictionary and Hierarchical Relation Detector}
The authors have adopted selectional restriction for disambiguation of
the semantic relations between two nouns compound. For the
purpose, the hierarchical structure shown in Fig.2 is utilized.

\begin{figure}[t]
\begin{center}
\unitlength=1mm
\epsfile{file=04fig2new.eps}
\end{center}
\vspace{-3mm}
\caption{A part of the hierarchical relations among the objects}
\vspace{1mm}
\end{figure}


Assume a concept $A$ is defined as follows:
\begin{equation}
 \lambda x (A(x) \equiv (\exists(y,...))(R(x,y) \wedge B(y) \wedge ...))
\end{equation}
\begin{tabbing}
 where, \= $\equiv$ \= : Equivalence,\\
       \> $\exists$ \> : Existential quantifier,\\
       \> and the relation $R$ is part of $O^{\mbox{\tiny $+\!+$}}$.
\end{tabbing}
        
The {\it semantic relations} between two concepts $A$ and $B$ in the
hierarchical structure are limited to the three cases below:
\begin{tabbing}
 Case1)  \= $A \stackrel{is-a}{\leftarrow} B$\\
         \> $R(x,y)$ means that $A$ includes $B$ conceptually, that is,
\end{tabbing}
\begin{equation} 
            \forall x(B(x)\supset A(x))
\end{equation}


or $B$ is-a $A$. For example, a dog($B$) is a mammal($A$) or dogs($B$)
form a subset of 

mammal($A$).
\begin{tabbing}
 Case2)  \= $A \stackrel{part-of}{\leftarrow} B$\\
         \> $R(x,y)$ means that $A$ includes $B$ physically, that is,
\end{tabbing}
\begin{equation}
            (\forall x)(\exists y)(B(x)\supset A(y))
\end{equation}
\begin{tabbing}
\ \ \ \ \ \ \= $B$ is part-of $A$. For example, the belly($B$) is 
part of the body($A$) or the body($A$)\\
            \> has the belly($B$) as its part.
\end{tabbing}
\begin{tabbing}
 Case3)  \= $A = B$
\end{tabbing}
\begin{equation}
            (\forall x)(A(x)\supset B(x)) \wedge (B(x)\supset A(x))
\end{equation}
The relations "is-a" and "part-of" can be more analytically defined~\cite{Yokota94}. 

$A$ is {\it equivalent-to} $B$ both conceptually and physically. For
example, an aeroplane ($A$) is equivalent to an airplane ($B$). In the
cases above only, $B$ can satisfy the semantic restriction imposed by
$A$, and $B$ is allowed to fill the slot for $A$. For example in
case1, 自動車(car) can be in place of 物品(goods) in 物品購入(goods
purchase), because a car is a good. For example in case2, 手指(finger)
can be in place of 右腕(right-arm) in 右腕切除(right-arm
dismemberment), because the right-arm has the finger as its part. For
example in case3, 品物(article) is equivalent to 物品, and therefore 
品物 can be in place of 物品 in 物品購入.

 In the paraphrasing system, three ground clauses (is-a($A$,$B$),
part-of($A$,$B$) and equivalent($A$,$B$)) represent its three semantic
relations in our hierarchical dictionary respectively. In Fig.2, a
node represents a concept. A branch of a single (directed) line
indicates a relation "is-a" or "part-of". A branch of a double line
indicates a relation "equivalent-to". There are two hierarchical
relationships in such relations.  A {\it parallel hierarchy}
(Upal($A$,$B$)) exists between two nodes($A$ and $B$) if they have an
identical parent node and same relation. The other relationship is an
{\it upper and lower hierarchy} (Ual($A$,$B$)) where an upper node
links to a lower node by a certain directed branch (i.e.
$\stackrel{is-a}{\leftarrow}$ or $\stackrel{part-of}{\leftarrow}$ ).
Furthermore, when the relation Ual($A$,$B$) holds, the number of
directed branches which extends to the node $B$ from $A$ is called a
{\it path length} from $B$ to $A$.

\subsection{Word Segmentation Processor}
 A\hspace{-0.1mm} compound\hspace{-0.1mm} noun\hspace{-0.1mm} is\hspace{-0.1mm} separated\hspace{-0.1mm} according\hspace{-0.1mm} to\hspace{-0.1mm} the\hspace{-0.1mm} lexical\hspace{-0.1mm} information\hspace{-0.1mm}
(certain\hspace{-0.1mm} stems\hspace{-0.1mm} of\hspace{-0.1mm} words) in the word dictionary. It is also possible
to apply segmentation processing for the case where more than 3 words
are compound. The segmentation technique that we used is based on
the method of minimizing the number of word segmentations~\cite{Yoshimura83}.

\subsection{Restoration Processor}
 The elliptical word from a compound noun is estimated by our
restoration processor according to the combination of the semantic
category of the two nouns ($N_1$・$N_2$) into which it was separated.
It is possible that the elliptical words and their positions can be
decided definitely only with the combination of the semantic
categories of the separated nouns in the restoration processing. We
call restoration information about such an elliptical word a
{\it combination pattern}. This process is called {\it simple
restoration}.

The other case, where the elliptical words cannot be determined only
with their combination so that the {\it combination description}
($I_c$) of the separated nouns is necessary, is called {\it
word-dependent restoration}. Combination patterns and examples of the
simple restoration are shown in Table 2.
\begin{table}
\begin{center}
\caption{Combination patterns and examples of simple restoration of
compound nouns ($N_1$・$N_2$)}
\begin{tabular}{c|c|l|l} \hline
  $N_1$  & $N_2$ & Combination Pattern  & 
\multicolumn{1}{|c}{Restoration Examples}\\ \hline
 Obj & Atr & [$N_1$]が持つ[$N_2$] & 心臓が持つ機能、\ \ \ 大気が持つ温度\\
    &     &                      & function of heart \ \ \ 
temperature of atmosphere\\ 
 Obj & Val & [$N_1$]が[$N_2$]である     & 物品が劣悪である、\hspace{3em}
信号が赤である\\
    &     &                      & The articles are inferior. \ \ 
The signal is red.\\
 Evt & Obj & [$N_1$]する[$N_2$]         & 蛍光する物質、\hspace{4em}
作用する物質\\
     &     &                     & the fluorescent material \ \ acting
material\\
 Evt  & Atr & [$N_1$]する[$N_2$]         & 運動する能力、\hspace{1em}
運動する機能\\
     &     &                      & ability to move \ \ function to move.\\
 Evt  & Val & [$N_1$]が[$N_2$]である     & 睡眠が長時間である、\ \ \ 
歩行が良好である\\
     &     &                      & The dormancy is long. \ 
The walking is good.\\
 Atr  & Obj & [$N_1$]のための[$N_2$]     & 湿度のための計器、\ \ \ \
カラーのためのフィルム\\
     &     &                      & gauge with humidity \ \ color film\\
 Atr  & Atr & [$N_1$]と[$N_2$]     & 容姿と体格、\hspace{5em}時間と場所\\
     &     &                      & diagram and physique \ \ 
time and place\\
 Atr  & Val & [$N_1$]が[$N_2$]である     & 天候が不順である、\hspace{5em}
EFが90%である\\
     &     &                      & The weather is unseasonable. \ \ 
EF is 90\% \\
 Val  & Atr & [$N_1$]である[$N_2$]       & 赤である色、\ \ 
円である形\\
     &     &                      & color of red \ \ \ circular form\\
 Val  & Val & [$N_1$]と[$N_2$]           & 赤と茶、\hspace{4em}青と白\\
     &     &                      & red and brown \ \ \ blue and white\\
 Rel  & Obj & [$N_1$]である[$N_2$]       & 同一である物質\\
     &     &                      & identical material\\
 Rel  & Atr & [$N_1$]である[$N_2$]       & 同一である規格、\hspace{1em}
一定である温度\\
     &     &                      & identical standard \ \ 
constant temperature.\\
 Rel  & Val & $[N_1]\;\varepsilon\; [N_2]$  & 約50m、\ \ \ \ 概略10m\\
     &     &                      & about 50 m \ \ \ roughly 10 m\\
\hline
\multicolumn{2}{l}{\footnotesize $\varepsilon$ means null string.}\\
\end{tabular}
\end{center}
\end{table}

Here, we describe the processor for the word-dependent
restoration of some pattern.
\renewcommand{\thesubsubsection}{}
\subsubsection{Obj・Obj}
 First, the hierarchical relation detector examines whether a parallel
hierarchy Upal($N_1$,$N_2$) between one Obj noun $N_1$ and another
$N_2$ holds. If Upal($N_1$,$N_2$) is true, the processor will combine
$N_1$ and $N_2$ by the use of the particle "と".
\begin{list}{}{}
\item[1)] 右手と左手
\item (= the right hand and the left hand)
\end{list}
If not, it will combine two nouns by the use of the particle 
"にある".
\begin{list}{}{}
\item[2)] 車にある座席
\item (= the seat in the car)
\end{list}

\subsubsection{Obj・Evt}
 In this case, the processor strongly depends on the combination
description of the Evt noun. It uses $I_c$($N_2$): the $I_c$ of $N_2$
(i.e., the Evt noun) in the word dictionary and the hierarchical
relationship of $N_1$ (i.e., the Obj noun) for this restoration.
First, the hierarchical relation detector examines the hierarchical
relationship between $O_i$ within the identifier def($O_i$) in
$I_c$($N_2$) and $N_1$. For the relationship, if $N_1$ exists in the
lower hierarchy of $O_i$, $O_i$ become a candidate to substitute
for $N_1$. For all $O_i$ in $I_c$($N_2$) such that Ual($O_i$,$N_1$)
holds, the path lengths $P_i$ to $N_1$ are computed. This operation is
performed from the left to the right in $I_c$($N_2$). When more than
one candidate exists, the candidate with the smallest path length $P_i$
is chosen and substituted. If there is more than one minimum path
length $P_i$, the detector substitutes $N_1$ for all such candidates
of which length $P_i$ is minimum. So as to not substitute $O_i$ as
itself , this paired with $J_i$. Lastly, it concatenates "する", which
is the character strings of the end form of the sa-hen verb, to the
right of the constructed strings and lastly it output a sentence.
\begin{tabbing}
\ \ \ \ \= 3) \= $<$ Compound noun $>$ \=  自動車購入\\
	\>    \>                      \>(= car purchase)\\
	\>    \> $I_c$(購入): [[def(person),dep(ga)], [def(things),dep(wo)]]\\
	\>    \> $<$ Output sentence $>$ \>  人が自動車を購入する \\
	\>    \>                    \> (= the person buys a car.)
\end{tabbing}
 
If there is no hierarchical relationship with $O_i$ in $I_c$($N_2$),
it is impossible to paraphrase the elliptical structure. In this case is
put out the message that the structure is unrestorable.

\subsubsection{\{ Evt, Atr, Val, Rel \}・Evt}
 In this combination pattern, basically, according to $I_c$($N_2$)
which is the combination description of the governor $N_2$, the
restoration processing is done. This hierarchical relationship between
$N_1$ and $N_2$ is never examined. The processor compares the semantic
category of $N_1$ with $O_i$ which is directly described within some
def($O_i$) of $I_c$($N_2$). If there is the semantic category $O_i$
equal to $N_1$, it combines $N_1$ with $J_i$ of the following
dep($J_i$) in $I_c$($N_2$). When the following dep($J_i$) in
$I_c$($N_2$) doesn't exist (i.e. in the case that the semantic
category of $N_1$ is the either of Atr, Val or Rel), it refers to
$I_c$($N_1$) in order to find out the list [cat(Evt), dep($J_k$)]. It
combines $N_1$ with $J_k$. Lastly, it puts "する" to the right end of the
constructed strings.
\begin{list}{}{}
\item[4)]  経過を観察する
\item (= observe the progress)
\end{list}

\subsubsection{\{ Obj, Evt, Atr, Val \}・Rel}
 When a Rel noun is the governor, the processor restores some
elliptical words depended on the semantic category of $N_1$. It refers
to $I_c$($N_2$) in order to find out the list [cat($C_i$), dep($J_i$)]
in which the $C_i$ is equal to the semantic category of $N_1$. Lastly,
it combines $N_1$ and $N_2$ with $J_i$ in the list.
\begin{list}{}{}
\item[5)] 投与する前
\item (= before giving medicine)
\end{list}

\subsubsection{Val・Obj}
 In this case, the processor searches for the list
[cat(Obj),dep($J_i$)] from the $I_c$($N_1$). It combines $N_1$ and
$N_2$ with $J_i$.
\begin{list}{}{}
\item[6)] 赤色である灯
\item (= red light)
\end{list}

\section{Experimental Results and Discussion}
A prototype of our paraphrasing system was implemented in CESP
language (version 3.2). The program consists of about 900 lines in
total. The analysis data consist of 334 discharge summaries written by
45 doctors of the circulatory internal medicine at Kyushu University
Hospital. We manually extracted 5267 compound nouns from these
data. We randomly selected 100 words of direct concatenation with 2
nouns and input them into our paraphrasing system. Table 3 shows a
part of the results.
\begin{table}
\begin{center}
\caption{Experimental results of restoration processor (a part).}
\begin{tabular}{l|c|l} \hline
 \ \ \ Input     & Combination & \hspace{3em} Output\\ \hline
 右手指先       & Obj・Obj  & 右手にある指先\\
                &           & fingertip of right hand\\
左房右房        & Obj・Obj  & 左房と右房\\
                &           & left tassel and right atrium\\
治療薬選択      & Obj・Evt  & 人が治療薬を選択する\\
                &           & The person chooses a curative medicine.\\
 抗不整脈薬投与 & Obj・Evt  & 医療従事者が抗不整脈薬を投与する\\
                &           & The medical professional prescribes
an antiarrhythmic agent.\\
消化器系精査   & Obj・Evt & 医療従事者が医療器具で消化器系を精査する\\
                &           & The medical professional proves an alimentary
 system\\
                &           & with the medical equipment.\\
 心機能         & Obj・Atr  & 心臓が持つ機能\\
                &           & function of heart\\
 物品劣悪       & Obj・Val  & 物品が劣悪である\\
                &           & The articles are inferior.\\
 左房内         & Obj・Rel  & 左房における内\\
                &           & out of the left tassel\\
 蛍光物質       & Evt・Obj  & 蛍光する物質\\
                &           & the fluorescent material\\
 発作出現       & Evt・Evt  & 発作が出現する\\
                &           & have a fit.\\
 運動機能       & Evt・Atr  & 運動するための機能\\
                &           & function of movement\\
貯留速度 & Evt・Atr & 貯留する速度\\
         &          & speed of collection\\
 歩行良好       & Evt・Val  & 歩行が良好である\\
                &           & The walking is good.\\
呼吸不全 & Evt・Val & 呼吸が不全である\\
         &          & The knack is incomplete.\\
 清掃後         & Evt・Rel  & 清掃した後\\
                &           & after cleaning up\\
 カラーフィルム & Atr・Obj  & カラーのためのフィルム\\
                &           & filmwith colors\\
 血圧測定       & Atr・Evt  & 血圧を測定する\\
                &           & measuring of blood pressure\\
 容姿体格       & Atr・Atr  & 容姿と体格\\
                &           & diagram and physique\\
 天候不順       & Atr・Val  & 天候が不順である\\
                &           & The weather is unseasonable.\\
 体重差         & Atr・Rel  & 体重における差\\
                &           & difference with weight\\
 同一物質       & Rel・Obj  & 同一である物質\\
                &           & identical material\\
 同時記録       & Rel・Evt  & 同時に記録する\\
                &           & recording at the same time\\
 一定温度       & Rel・Val  & 一定である温度\\
                &           & constant temperature\\
 増加減少       & Rel・Rel  & 増加と減少\\
                &           & increase and decrease\\
\hline
\end{tabular}
\end{center}
\end{table}

All answers except one were correct. The word dictionary
used in this experiment contained 246 words. In hierarchy dictionary,
184 items were registered. The following is the incorrect output:
\begin{list}{}{}
\item[7)] [Val・Val] 1丁目と2番地
\item (= 1-chome and 2 banchi (a lot number of an address in Japan))
\end{list}

We believe that the results of this paraphrasing experiment proved
that the system functioned effectively for compound nouns consisting
of 2 nouns. Furthermore, even though Evt nouns sometimes have more
than one corresponding combination description, it was possible to use
an elliptical word in single $I_c$ for every Evt noun appeared in the
discharge summaries.

\section{Conclusions and Future Perspectives}
The paraphrasing of a compound noun was described. It is difficult to
analyze compound nouns in general Japanese texts as well as in
discharge summaries. Many past studies have examined compound nouns
from the viewpoint of morpheme analysis. But the consideration, such
as ours, that a compound noun is a result of ellipsis and the
suggestion to restore them is rarely discussed.  We designed the
paraphrasing system of elliptical words based on the MIDST and tried
the semantic interpretation of compound nouns. To adopt the semantic
categories of nouns based on the MIDST, we classified concatenating
relations and elliptical words systematically.  Therefore, the system
required early-stage knowledge, the word dictionary and the hierarchy
dictionary where the hierarchical relationship among the general
"object" noun concepts was described, which made the interpretation
process very simple and efficient.

Problems which should be addressed in future systems include the
following.
\begin{list}{}{}
\item[(a)] Realization of restoration processing of compound noun of 
more than 2 nouns.
\item[(b)] More detailed analysis and efficient restoration process of the
Rel noun.
\item[(c)] Restoration processing of word that has complex combination 
information.
\item[(d)] Expansion of the word dictionary and hierarchy dictionary.
\end{list}

\vspace{2ex}
\begin{figure}[ht]
\begin{center}
\unitlength=1mm
\epsfile{file=04fig3new.eps}
\end{center}
\vspace{-1mm}
\caption{An example of the recursive paraphrasing for the case of three nouns.}
\end{figure}

We have found that the semantic category of the noun resulted from
concatenation of two nouns becomes that of the last of the two as
shown in Table 4.


\vspace{-1mm}
\begin{table}
\begin{center}
\caption{A total semantic category for the combination between two 
nouns ($N_1$・$N_2$).}
\begin{tabular}{|c|c|c|c|c|c|} \hline
   $N_1$\$N_2$  &  Obj  &  Evt  &  Atr  &  Val  &  Rel\\ \hline
     Obj   &  Obj  &  Evt  &  Atr  &  Evt  &  Rel\\
     Evt   &  Obj  &  Evt  &  Atr  &  Evt  &  Rel\\
     Atr   &  Obj  &  Evt  &  Atr  &  Val  &  Rel\\
     Val   &  Obj  &  Evt  &  Val  &  Val  &  Rel\\
     Rel   &  Obj  &  Evt  &  Atr  &  Val  &  Rel\\
\hline
\end{tabular}
\end{center}
\end{table}


This fact implies such a generalization that the semantic category of
N($\geq2$) compound nouns becomes that of the last noun, which in turn will
allow recursion of the paraphrasing for the case of two nouns as shown
in Fig.3.


\bibliographystyle{nlpbbl}
\bibliography{paper}

\begin{biography}

\biotitle{}

\bioauthor{Masato Shiraishi}
{ Masato Shiraishi received the B.E. degree in electric engineering
from Miyazaki University, Miyazaki, Japan, in 1982, and the M.E.
degree in information systems from Kyushu University, Fukuoka, Japan,
in 1984. From 1984 to 1989 he worked for the Kyushu Matsushita
Electric Co., Ltd. He has been with the Faculty of
Education, Fukuoka University of Education, Munakata, Japan, where he
is currently a Research Associate.  He is a member of the Information
Processing Society of Japan, and the Japan Society for Artificial
Intelligence. He is interested in natural language processing,
semantic understanding and corpus linguistics. }

\bioauthor{Masao Yokota}
{
Masao Yokota (Regular Member) was born in Miyazaki, Japan, \hspace{-0.1mm}on\hspace{-0.1mm} September \hspace{-0.1mm}12, \hspace{-1mm}1949.\hspace{-1mm} He \hspace{-0.1mm}received \hspace{-0.1mm}the \hspace{-0.1mm}B.E. \hspace{-0.1mm}degree\hspace{-0.1mm} in\hspace{-0.1mm} electronic
engineering from Kyushu Institute of Technology, Japan, in 1972, and
the M.E. \hspace{-0.1mm}and \hspace{-0.1mm}the \hspace{-0.1mm}D.E. \hspace{-0.1mm}degree \hspace{-0.1mm}in \hspace{-0.1mm}communication \hspace{-0.1mm}engineering\hspace{-0.1mm} from\hspace{-0.1mm} Kyushu \hspace{-0.1mm}
Uni- versity, Japan, in 1974 and in 1982 respectively.  He is now a
Professor of Fukuoka Institute of Technology and interested in
integrated media understanding.}


\bioreceived{Received}
\bioaccepted{Accepted}

\end{biography}

\end{document}
