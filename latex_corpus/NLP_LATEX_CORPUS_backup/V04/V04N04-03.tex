\documentstyle[graphicx,jnlpbbl]{jnlp_j_b5_2e}

\setcounter{page}{41}
\setcounter{巻数}{4}
\setcounter{号数}{4}
\setcounter{年}{1997}
\setcounter{月}{10}
\受付{1996}{11}{13}
\再受付{1997}{3}{11}
\採録{1997}{7}{18}


\title{表層表現による日本語動詞句のアスペクトの推定}
\author{大石 亨\affiref{NAIST} \and 松本 裕治\affiref{NAIST}}

\headauthor{大石 亨・松本 裕治}
\headtitle{日本語動詞句のアスペクトの推定}

\affilabel{NAIST}{奈良先端技術大学院大学 情報科学研究科} {Graduate School
  of Information Science, Nara Institute of Science and Technology}

\jabstract{日本語のアスペクトの研究は,継続相,完成相というような分類と
  それぞれの意味を記述していく段階から,副詞的成分などの関わりを含め,ア
  スペクト的な意味の決まり方のプロセスを整理する方向へと発展してきている.
  本稿では,アスペクト形式や副詞句の意味を時間軸におけるズーミングや焦点
  化といった認知的プロセスを表示するものとしてとらえ,動詞句の意味に対す
  る動的な操作であると考える.その上で,動詞句の意味をコーパスに現れた表
  層表現から推定し,素性構造として表現する.実験の結果得られた動詞句の分
  類を評価するために,最も基本的なアスペクトの形態である「シテイル」形の
  意味を自動的に決定する処理を行なった.200文における正解率は71\%で
  あった.これらの情報は,動詞句のアスペクト的な意味のあり方の類型を与え
  るだけではなく,文間の関係をとらえる際の手がかりを提供するものであると
  考える.}

\jkeywords{動詞句,アスペクト,ズーミング,焦点化,表層表現}

\etitle{An Estimate of Aspect of Japanese Verb Phrases \\
from Surface Expressions}
\eauthor{Akira Oishi\affiref{NAIST} \and Yuji Matsumoto\affiref{NAIST}} 

\eabstract{ The study on aspect in Japanese has evolved from the
  description of the meaning for each type such as ``progressive'' or
  ``perfective'' into the process of the determination of the aspectual
  meaning including adverbial phrases etc.  In this paper, we consider
  the meaning of the aspectual markers or adverbs as the indicator of
  the cognitive process like ``zooming'' or ``focusing'' on the
  time-line and define them as the dynamic operations on the meaning of
  the verb phrases.  We estimate the aspectual meaning of verbs from
  surface expressions in a corpus and represent them as a bundle of
  features.  To evaluate the result of the experiment, we examined the
  meaning of {\em si-teiru} which is one of the most fundamental
  aspectual markers, and obtained the correct recognition score of
  71\% for the 200 sentences.  }

\ekeywords{Verb phrase, Aspect, Zooming, Focusing, Surface expressions}

\begin{document}
\maketitle

\section{はじめに}

アスペクトとは,動きの時間的な局面を問題にして,どの局面をどのように(動
きとして,あるいは状態として)とらえるか,ということを表すカテゴリーであ
る.『国語学大辞典』\cite{Kokugo93}で「アスペクト」の項をひくと,
\begin{quotation}
  動詞のあらわす動作が一定時点においてどの過程の部分にあるかをあらわす,
  動詞の形態論的なカテゴリー.たとえば,「よみはじめる」はよむ動作がはじ
  まることを,「よんでいる」は進行の途中にあることを,「よんでしまう」は
  動作がおわりまでおこなわれることを,「よんである」は,動作終了後に一定
  の結果がのこっていることをあらわす.アスペクトは,時間にかかわるカテゴ
  リーであるが,テンスとちがって,はなしの時点との関係は問題にしない.
  (後略)
\end{quotation}
とされている.当然のことながら,動詞句の実現するアスペクト的な意味は,動
詞の性格と密接に関係する.金田一は,動詞を継続動詞と瞬間動詞にわけ,継続
動詞が「している」になると進行態(進行中の意味)となり,瞬間動詞は既然態
(結果の状態の意味)になるとした\cite{Kindaiti76}.このほか,結果動詞と
非結果動詞,さらに,変化動詞,出現動詞,消滅動詞,設置動詞などと,さまざ
まな分類がなされてきた.英語においても,Vendlerによるactivities,
achievements, accomplishments, statesというような分類\cite{Vendler57},
あるいはComrieによるactions, states, processes, eventsのような分類がある
\cite{Comrie76}.

しかし,近年の研究は,動詞句の分類とそれぞれの意味を記述していく段階から,
副詞的成分などの関わりを含め,アスペクト的な意味の決まり方のプロセスを整
理する方向へと発展してきている.たとえば,森山は,「結婚している」という
句が,通常,結果の状態をあらわすのに対し,「多くの友達が次々と結婚してい
る」といった場合には,繰り返しとしての進行中と解釈されるなどの例を挙げ,
最終的なアスペクトの意味が,格成分,副詞などを含めた包括的なレベル(森山
氏はこれをアスペクトプロポジションAPと呼んでいる)において決められること
を指摘している\cite{森山83,森山88}.

本稿では,アスペクト形式
\footnote{
  派生的にとらえられる文法的な形態素を形式と呼ぶ.本稿では,「シ始メル」
  などの複合動詞も含め,動詞に後続する要素をアスペクト形式とよぶ.
}
や副詞句の意味を時間軸におけるズーミングや焦点化といった認知的プロセスを
表示するものとしてとらえ,動詞句の意味に対する動的な操作であると考える.
次節では,これらの概念について,一般的な説明を与える.第3節では,動詞句
の意味を素性によって表現し,それに対してアスペクト形式や副詞句が具体的に
どのような操作をするかを明らかにする.第4節では,動詞句の意味をコーパス
に現れた表層表現から推定し,6種類のクラスに分類する実験の方法と結果およ
び評価を述べる.実験結果の評価は,最も基本的なアスペクトの形態である「シ
テイル」形の意味を自動的に決定する処理によって行なった.動詞句の分類自体
は,客観的に評価することが難しいからである.

\section{ズーミングと焦点化}

ここでは,次節で述べるアスペクト形式や副詞の行なう操作的な作用の基礎とな
る概念である{\bf ズーミング}と{\bf 焦点化}について説明する
\footnote{
  本稿は,これらの操作に対して,形式的な表現を与えるものではない.形式化
  によって,一般的な適用可能性と柔軟性が失われるおそれがあるからである.
  時間表現の形式的表現については,\cite{原88},\cite{郡司94},\cite{金子
    95}など参照.}.

ズーミングとは,テレビカメラが行なうように,対象をアップで撮るか,ロング
で撮るかを自由に変えることをいう.前者をズームイン,後者をズームアウトと
呼ぶ.人間は,同一の客観的事象に対して,異なる表現をすることができる.た
とえば,人が死んだのを見て,他殺であっても加害者を明示せず,自動詞を用い
て「死んだ」といえるし,自殺や病死であっても,「社会(病気)が人を殺した」
と他動的表現をすることができる.ある事態を叙述するためには,その事態に関
与する人および物の中から,表現する範囲を限定し,枠組を設定しなければなら
ない.この枠組のことを「認知のスコープ」と呼び,それを変更すること,すな
わち,新たな枠組を設定することをズーミングと呼ぶ.

他方,焦点化とは,レンズの焦点を絞るように,ある対象に注目して事態を叙述
することをいう.「地震が家を倒す」のと,「家が地震で倒れる」のとで,実際
の地震に区別はない.ただ,人間が地震を主にして考えるか,家を主にするかの
違いである.どの概念に焦点をあてるかによって,動詞句の構造は大きく変わる
ことになる.

ズーミングや焦点化という用語は,カメラのメタファーを用いているために,視
覚的操作であるかのような印象を与えるかもしれない.実際,人間は視覚器の水
晶体や虹彩を使って,これらの処理を行なっている.しかし,ここでいうズーミ
ングや焦点化は,感覚器の行なう処理ではない.われわれの眼が言語表現を決定
するわけではない.知覚系を通過して脳に入ってくる情報を処理する形式そのも
のとして,これらの概念を想定しているのである.したがって,これらの処理を
視覚系に限定する必要はない.本稿では,これを時間軸上に適用する.

人間は,存在を三次元空間と時間とを用いて認識している.そして,時間軸は空
間の軸とは独立なものとして取り扱い,それを当然のことだと思っている.それ
で,ズーミングや焦点化といった空間のメタファーをアスペクトのような時間に
関わる概念に対して適用すると,奇異な感じをもたれるかもしれない.しかし,
時間と空間とは,いつも相関関係をもっている\cite{本川92}.空間と時間,構
造と機能などの二分法をもたらすものは何か.それは,視覚と聴覚という二つの
認識系であると考える.視覚の特質は,時間という要素が抜け落ちていることで
ある.写真は瞬間の像である.一方,音は,画像とは違って,時間軸の上を単線
で進む.ところで,言語には視覚によるものと聴覚によるものがある.われわれ
は両方を言語と呼ぶ.視覚と聴覚の一次中枢からの情報に,順次,高次の処理を
重ねてゆく.すると,いつのまにか視覚と聴覚の情報が体系的に一致し,ある程
度交換可能になる.ここに言語が成立する.刺激の種類も時間に関わる性質も全
く異なる二つの感覚を「言語」として統一する
\footnote{
  この考え方は,\cite{養老89,養老96}による.
}.
その統一の具体的な過程のなかに,ここでズーミングと焦点化と呼ぶ機能が含ま
れていると考えるのである.ごく単純化していえば,焦点化とは,入力に対する
「重みづけ」である.人間が知覚に基づいて世界像を形成する際に,われわれは
その入力に対して適当に重みづけをすることができる.最も重みづけされた入力
の部分,それが焦点化された部分である.入力に対する重みづけに関して,ある
閾値を設定する.この閾値を越えた部分だけが意識にのぼり,認知のスコープを
形成する.さらに,重みづけの順位により,ある種の構造が創出される.すなわ
ち,ズーミングとは,入力に対する重みづけの変更に他ならない.

さて,上述のように,視覚には時間の要素が抜けている.一方,聴覚・運動系で
は時間は流れる.両者の連合を可能にするためには,単位的な時間が必要である.
これをわれわれは瞬間という.ユークリッド幾何学では直線を点の集合とみなす.
視覚の生理学は,視覚系におけるニューロンの直線状の受容野は,同心円状の受
容野をもつ複数の下位のニューロンからの入力であることを示した\cite{養老
  89}.早い話が,そこでは,直線は点の集合なのである.同様の処理が時間と
いう仮想的な次元で行なわれていると考える.瞬間の集合が時間軸を構成する.
具体的には,映画のフィルムのコマ送りを思い浮かべればよい.それぞれのコマ
の集合に重みをつけることによって,ズーミングや焦点化という概念を時間軸上
に適用することができる.

時間を基礎的に特徴づけるものは変化である.変化がなければ時間はない.それ
が絵や写真である.しかし,変化のみであれば,ふたたび時間はないであろう.
そこでは,時間は変化と同義になってしまう.そこに持続が必要となる.変化と
持続の繰り返しのパターンが,動詞のアスペクトの類型を与える.本稿では,そ
の類型を素性によって表現する.また,変化は時間軸をいくつかの部分に分割す
る.この時間的な断層によって,焦点化の単位が発生する.ある種の格成分は,
その内在的な特性によって,この断層をもたらすことができる.「パンを食べ」
てしまえば,それ以上「食べる」ことはできない.「学校に」着いてしまえば,
同じ「学校に行く」ことはできない.ある種の副詞は,この分割された時間軸の
一部分を修飾する.修飾するとは,より詳しく述べることである.詳しく述べる
ためには,ピントが合っていなければならない.これを焦点化というのである.
さらに,アスペクト形式は,動詞語幹によって叙述されるコト的意味(プロポジ
ション)に新たな枠組を押しつける.多くの形式が与える枠組は,時間軸上で分
割された部分あるいは分割点の内の一つである.もとの動詞のアスペクトから見
れば,枠が縮小されることになる.これをズームインというのである.副詞によ
る焦点化も形式によるズームインも,もとの動詞のアスペクトと整合していなけ
ればならない.この制約を用いて動詞のアスペクトの類型を推定することが,本
稿の目的の一つである.

\section{アスペクト決定の過程}
\subsection{素性による動詞句分類}

本節では,動詞句の最終的なアスペクトの意味の決まり方を前節に述べた操作に
基づいて記述するのであるが,その前に,操作が適用されるべき動詞のアスペク
ト的な意味を考えておく必要がある.ここでは,森山1983で抽出された素性に基
づいて,動詞を六つのカテゴリーに分類する.森山は,動詞のアスペクチュアル
な素性として,持続性,過程性,終結性,進展性の四つを挙げている.ただし,
対象とされているのはアスペクトの対立のある動詞(「スル」形と「シテイル」
形が意味的に異なる動詞)であるので,これらを状態性の動詞(「ある」,「い
る」など)から区別するための素性として,動作性を追加する.なお,このうち,
動作性(dynamic),一点性(atomic)
\footnote{
  atomicとは,動きが点的であることを表し,持続性と対立する概念である.し
  たがって,持続性のあるものは,$-$atomicである.
}
,終結性(telic)については,\cite{Bennett90}や\cite{Dorr93}などでも用い
られている.

{\bf 動作性(dynamicity)}とは,述語が動きを表す(+d)か,状態を表すか($-$d)を
区別するための素性である.一般に,日本語の動詞は無標では動きを表し,「テ
イル」をつけて状態化するのが普通であるが\cite{Talmy85},状態的な動詞は
「スル」形で状態を表す.動きを表す動詞の「スル」形は一般に未来を表すのに
対し,状態的な動詞の「スル」形は現在を表す.この違いは,意味的なものであ
り,単に「テイル」がつくつかないという違いではない.

{\bf 持続性(durationality --- non-atomicity)}とは,動詞の表す動きや,そ
れによって生ずる事態に,何らかの持続的な期間が存在することを表す素性であ
る.持続性があれば($-$a),「シ続ケル」「シナガラ」などの形式が承接し,単
純な期間成分(「〜間」,「暫く」など)が共起しうる.持続性には,動きその
ものが展開する持続(「勉強し続ける」)と,動きの結果を維持する持続(「下
を向き続ける」)の二種類がある.前者を{\bf 過程持続},後者を{\bf 結果持
  続}とよぶ.

{\bf 過程性(process)}とは,上述した過程持続に関する素性であり,動きが展
開する持続があるかどうかを区別するものである.過程性があれば(+p),「シテ
イル」形が進行中の意味で承接しうるし,「シ始メル」,「シ出ス」,「シカカ
ル」,「シカケル」などの始動を表す形式が承接する.ただし,「シカカル」,
「シカケル」は,動きの前段階をあらわすこともあるので,純粋に始動を表すと
はいえない.

{\bf 終結性(telicity)}とは,動きに終わりの点があるということを表す素性で
ある.もちろん,過程性を前提とする.終結性があれば(+t),「シ終ワル」,
「シ終エル」など,終結点を取り出す形式が承接するほか,「〜間カカッテ」と
いう稼働期間を表す成分と共起しうる.

{\bf 進展性(graduality)}とは,動きの中に何らかの変化が内在していて,時と
ともにその程度が深化進展するという素性である.進展性があれば(+g),「シテ
イク」,「シテクル」,「シツツアル」などの形式が承接する(これらは,進展
性以外の用法もある)ほか,「次第に」,「徐々に」などの副詞が共起しうる.

以上の五つの素性の組合せにより,動詞を分類することができる.各素性には依
存性がある(過程性がなければ終結性はないなど)ので,6種類のカテゴリーが
考えられる(表\,\ref{tab:category}).ここでは,わかりやすさのために各素性
を ``+'', ``$-$''によって二分したが,各素性には段階性があり,明確な境界が
存在するわけではない.形式の承接についての容認性にも,言いやすいものと言
いにくいものなどの段階性があろう.したがって,それによって定義されるカテ
ゴリーもプロトタイプ性を含んだものとなる.すなわち,そのメンバーには典型
的なものと周辺的なものが存在し,カテゴリーの間に明確な境界は存在しない.

\begin{table}[h]
\caption{素性による動詞分類}
\label{tab:category}
\centering
\begin{tabular}{|l|l|l|}  \hline
カテゴリー       & 素性表現               & 動詞の例 \\ \hline\hline
1.状態的         & [$-$d]                 & ある,いる,そびえる \\ \hline
2.一点的         & [+d,+a]                & ひらめく,見かける,飽きる \\ \hline
3.変化+結果持続 & [+d,$-$a,$-$p]         & 座る,立つ,ぶらさがる \\ \hline
4.過程+結果持続 & [+d,$-$a,+p,+t]        & 殺す,着る,開ける \\ \hline
5.非進展的過程   & [+d,$-$a,+p,$-$t,$-$g] & 歩く,言う,歌う \\ \hline
6.進展的過程     & [+d,$-$a,+p,$-$t,+g]   & 腐る,高まる,近付く \\ \hline
\end{tabular}
\end{table}
\vspace*{-2mm}
表\,\ref{tab:category}において,{\bf 1.状態的動詞}とは,動作性のない動詞
($-$d)である.動作性がないとは,動きのあり方に質的な断層を前提しないという
ことであり,その動詞の表す状態が時間的に連続的なものとして取り上げられる
(後述).状態的な動詞には,存在を表すものと,性質を表すものがある.

{\bf 2.一点的動詞}とは,一時点的な動きを表す動詞である.これには,「ひら
めく」,「命中する」など無変化で一時点的な動きの他に,「死ぬ」のように,
永続的な変化を表すものが含まれる.永続的な変化は,結果が非可逆的であるの
で,結果の持続を取り上げることができない.そこで,「*死に続けた」,「*
しばらく死んだ」などが言えないとともに,「シテイル」形が,変化の結果のあ
りさまを述べる意味と,かつてその動きがあったということを表す経歴の意味と
中和的になる(後述).

{\bf 3.変化+結果持続動詞}とは,変化によってある状態が成立し,その結果が
持続されるという意味の動詞である.ただし,過程性がないので,動きの展開や
変化の過程が取り上げられることはなく,「シテイル」形は進行中の意味になら
ない.主に姿勢や態勢を表す動詞が多い.

{\bf 4.過程+結果持続動詞}とは,過程によって主体あるいは客体に変化が生じ,
その結果が持続されるという動詞である.「窓を開け続ける」,「窓を開けてい
る」のように,「シテイル」,「シ続ケル」の意味は,進行中の意味と結果の持
続の意味の二つがありうる.

{\bf 5.非進展的過程動詞}とは,動きの展開する過程のみを有するものであり,
変化を表さない.動詞だけを取り上げれば終結性はないが,対象にくるものの性
質や明示的に終点を表す格成分によって終結点が設定されうる.また,動きの全
体量を規定することによっても終結性を持ちうる(後述).

{\bf 6.進展的過程動詞}とは,進展性を持つ動詞であって,変化をもたらす過程
の部分が漸次変化を表すものである.進展的な場合は終結点が設定されないのが
普通であるが,副詞等によって,終結点が設定される場合がある(後述).

Vendlerの分類では,1がstates,2,3がachievements,4,6がaccomplishments,5
がactivitiesとなる.それぞれのカテゴリーは,変化があればその変化が主体の
変化か客体の変化か,結果の持続があればその持続が主体的(意志的)になされ
るか否かによって,さらに細分することができるが,副次的なものであると考え
る.

\subsection{アスペクトに関わる領域}

動詞句全体のアスペクトは,{\bf 動詞 $\rightarrow$ 格成分 $\rightarrow$ 
  副詞 $\rightarrow$ 形式} の順序で未分化な状態から分化したものへと変わっ
てゆく.その際,動詞に固有に備わっている素性によって制約を受ける.ここで
は,上に述べた動詞のアスペクトに対して,これらの要素がどのように働きかけ
るかという点について述べる.

\subsubsection{格成分}

動詞句の意味を考えるうえで,最も重要なものは格成分である.Tennyは, 
``mesuring out''および ``terminus''という概念を導入して,動詞のとる項
(argument)とアスペクトの関係を詳細に記述している\cite{Tenny94}.

``mesuring out''とは,動きの全体量を規定することであり,これによって終結
性のない動詞に終結性を付加することがある.この役割は,internal direct
argument(概略,他動詞目的語および非対格動詞(unaccusative)の主語に相当す
る)のみが担うことができる.例として,Tennyは,incremental theme verbs
(`eat an apple', `build a house'など),change-of-state verbs (`ripen the
fruit'など),path objects of route verbs (`climed the ladder', `play a
sonata'など)の三種類を挙げている.

``terminus''とは,動きの終点を設定するものであり,これも動詞に終結性を付
加する.この役割を担うのは,internal indirect argument(英語では前置詞 
``to'',日本語では「に」格または「まで」格)である.このように,格成分と
アスペクトは密接に関連している.ただし,すべての格成分がアスペクトに関わ
るわけではなく,あくまで,特定の動詞クラスにおいて,特定の格が特定の役割
を担うことがありうるということである.Tennyは,同じ動詞が異なる構文に入
ることによって,そのアスペクチュアルな意味が転移するということも述べてい
るが,日本語では,以下で述べる副詞やアスペクト形式によって表現されるもの
が多い.日本語において格成分が関与する例としては,移動の動詞が,終点を表
す「に」格または「まで」格をとって,終結性を獲得すること(「学校まで走る」),
非能格動詞(unergative)が同族目的語をとって,全体量が設定されること(「短
い一生を生きる」)などがある.これらは,焦点をあてるべき動きの全体枠を設
定する.

\subsubsection{副詞}
\label{fukushi}

一般に,副詞は動きのある部分に焦点をあて,その部分をより詳細に述べる働き
をする.その焦点をあてる部分によって,以下のように分類される\cite{森山
  88}.

{\bf 過程修飾副詞(Process modifiers)}は,過程性のある動詞を修飾する.
「がさがさ」,「ばたばた」,「すいすい」,「せっせと」,「ぶつぶつ」,
「がらがら」のような畳語オノマトペや,「ゆっくり」,「手早く」,「足早に」
などの速度を表す副詞がある.これらは,動きの展開過程に焦点をあてる働きを
する.

{\bf 進展的副詞(Gradual change indicators)}とは,「段々」,「すこしずつ」,
「徐々に」,「ぐんぐん」,「しだいに」などのように,変化の進展を表す副詞
である.過程のある主体変化の動詞句は,「熱が下がっている」のように,それ
だけなら「シテイル」形では結果の状態の読みが優先されるのに対し,「熱が次
第に下がっている」のように,この種の副詞の修飾を受けると,過程の部分に焦
点があてられ,進行中の意味に読まれる.

{\bf 持続副詞(Continuous adverbs)}とは,「ずっと」「いつまでも」のように,
過程,結果持続のどちらも修飾しうるものである.どちらに焦点があてられるか
は,動詞の意味による(後述).いずれにしろ,持続性がなければならず,「一
時間」などの期間を表す成分と同じ共起属性を持つ.また,場所を示す「で」な
どの成分も,過程と結果の両方を修飾することができるが,持続性は関与しない.

{\bf 一時点化副詞(Atomic adverbs)}とは,動きを一点的なものとしてとらえる
副詞である.持続的な動きでも,特にある一点だけを取り出して修飾するもので
ある.「さっと」,「ぽんと」,「がたっと」,「ぽたりと」,「瞬間」,「一
瞬」,「あっというまに」などがある.これらが共起すれば,動きが一時点的な
ものとして把握されることになる.ただし,動詞が一時点的であるとは限らない.
あくまで,とらえ方の問題である.

{\bf 量規定副詞(Quantity regulators)}とは,「五キロ歩く」のように,動き
の全体量を規定する副詞である.量を規定する副詞なら何でもよく,時間,距離,
内容の量などがある.

{\bf 結果修飾副詞(End state modifiers)}は,特に変化の結末を表す副詞であ
り,「まっぷたつに」,「こなごなに」,「ぺちゃんこに」,「ばらばらに」な
どがある.この副詞は,変化の最終的な様子を修飾するものである.

以上述べたのは,一回的な動きのレベルであるが,繰り返しは,さらにこれらの
動きの全体的なあり方を規定する.繰り返しによって,動きが固有に持っている
素性とは無関係に繰り返しとしての過程が問題にされる.

{\bf 繰り返しの規定(Repetition adverbs)}には,「三回」,「何度も」,「い
くたびか」のように,繰り返しの全体量を規定するものと,「いつも」,「しょっ
ちゅう」,「つねづね」のように,習慣的な繰り返しを規定するものがある.い
ずれも,主体と動きの関与(事態)が複数である.

最後に,{\bf 過去の副詞(Time in the past adverbs)}についても述べておく.
「私はかつて留学している」のように,「シテイル」という現在・未来形と,
「かつて」,「昔」,「以前」のような過去の副詞が共起することがある.これ
は,動きを表すというよりも,経歴を述べる場合であるが,このような過去の副
詞は,それ事態でテンス相関的な時制構造を決めてしまう.「シテイル」の意味
については,次節で述べる.

\subsubsection{アスペクト形式}

アスペクト形式には,「始メル」,「続ケル」,「終ワル」,「カケル」のよう
な統語的複合動詞,「テイル」,「テクル」,「テアル」のように「テ」形に接
続する補助動詞,「ママダ」,「バカリダ」,「トコロダ」などの形式名詞など
がある.ここでは,次節の実験で用いたものについてのみ,簡単に述べる.

「シヨウトスル」,「シカケル」は,出来事の発生だけを問題にする形式である.
それで,アスペクトの対立がありさえすれば(+d),これらの形式が共起すること
ができる.「シテシマウ」も,モダリティー的な意味があるので,原則的には,
出来事が発生しさえすれば共起する.ただ,「料理を全部食べてしまう」のよう
に,終結性がある場合には,その終結点を取り出す意味になる.

「シ続ケル」は,持続性がある($-$a)動詞に承接する.先に述べたように,動きの
展開する過程と結果の持続の両方を取り上げることができる.「シ始メル」は,
過程性のある(+p)動詞に承接し,その過程の始まりを取り上げる.これに対し,
「シ終ワル」,「シ終エル」は,過程の終結を取り上げる.したがって,終結性
を持つ(+t)動詞に承接する.ただし,これらが動詞が固有に持っている素性と関
わるのは,一回的な動きのレベルであり,繰り返しによって複数の事態が過程化
される場合には,動詞の素性とは無関係に,状態的動詞以外のあらゆる動詞に承
接することが可能である(「多くの人々が,戦争で死に続けている」等).

「シツツアル」,「シテイク」,「シテクル」は,変化の進展的過程を取り上げ
る.進展しつつある変化を状態として取り上げるのが「シツツアル」であり,変
化の元の様子に視点を置いたのが「シテイク」,変化の行く先に視点を置いたの
が「シテクル」である.ただし,「シテイク」,「シテクル」には単に方向的な
移動を表す用法(「持っていく」等)の他,以前から,または,以後への継続を
表す用法がある(「昔から,村の人々はここで祭りを行なってきた」等).

\section{動詞の分類実験}

ここでは,コーパスに現れた表層表現から,動詞のアスペクチュアルな素性を獲
得する,すなわち,表\,\ref{tab:category}の六つのカテゴリーに動詞を分類す
る実験について述べる.動詞の分類が得られれば,それに前節で述べた他の要素
による操作を順に適用することによって,動詞句全体のアスペクト属性を推定す
ることが可能となる.

\begin{figure}
\begin{center}
	\leavevmode
      	
      	\includegraphics[width=70mm,clip]{category.eps}
	
\end{center}
\caption{動詞の分類と素性間の関係}
\label{fig:category}
\end{figure}

図\,\ref{fig:category}に示すように,それぞれのカテゴリーは,アスペクト形
式と動詞の共起制限に基づいて決定されるものである.しかも,これらの形式は
動詞に直接後続するものであるから,構文解析をするまでもなく,形態素解析だ
けで容易にデータを収集することができる.しかし,図\,\ref{fig:category}に
おいて,形式の表示していない枝は,相対する枝に表示してある形式に対する負
例を表すものである.我々は,負例を用いることができない.コーパス中に見つ
からなかったといって,絶対に言えないとは限らないからである.したがって,
負例を用いずに,正例のみで知識を獲得する何らかの手法を確立する必要がある.
また,先に述べたように,持続性を表す($-$a)とされる「シ続ケル」などの形式は,
事態全体の繰り返しによって,あらゆる動詞に承接する可能性がある.さらに,
「テイク」,「テクル」,「ナガラ」などは,アスペクト以外の意味でも用いら
れる場合もある.

\subsection{アルゴリズム}

これまでの考察に基づいて,動詞を分類するためのアルゴリズムを以下に示す.
なお,実験には「EDR日本語コーパス」(約21万文)および「EDR日本語共起辞書」
\cite{EDR95aj}を利用した.

\begin{description}
\item[STEP:1 データの抽出] EDR日本語共起辞書から,係り側単語の品詞が副詞
  で,受け側単語の品詞が動詞であるデータを抽出し,共起頻度とともに,配列
  PAIR (表\,\ref{tab:pair})に登録する.
\item[STEP:2 副詞の分類] 配列PAIRに含まれる副詞を, \ref{fukushi}節で述
  べた基準によって分類し,分類ラベル(英語の頭文字)を与え,配列ADVERBに
  登録する(表\,\ref{tab:adverb0},\ref{tab:adverb1}参照).
\item[STEP:3 対象動詞の決定] 配列PAIRから,配列ADVERBに登録されている副
  詞を含むものを抽出し,動詞をキーとして,副詞との共起頻度を集計する.共
  起頻度が5以上の動詞を実験対象とし,リストVERBに登録する.
\item[STEP:4 形式・副詞の獲得] EDR日本語コーパス(解析済)の全文に対して,
  動詞を発見し,それがリストVERBに存在すれば,
  \begin{description}
  \item[STEP:4-1 形式の獲得] 動詞に直接後続している形式が,あらかじめ用
    意してあるリスト(表\,\ref{tab:form1})に存在すれば,配列FORMを更新
    する(当該形式のカラムを1とする.表\,\ref{tab:form2}参照).ただし,
    クラスRの副詞(繰り返しの規定)に修飾されている場合は登録しない.
    「テクル」,「テイク」は,クラスGの副詞(進展的副詞)に修飾されてい
    る場合のみ登録する.
  \item[STEP:4-2 副詞の獲得] 動詞を修飾している副詞が,配列ADVERBに存在
    すれば,その分類ラベルを参照し,配列MODIFIEDを更新する(当該副詞クラ
    スの頻度に1加算する.表\,\ref{tab:modified}参照).ただし,副詞クラ
    スがCの場合(持続副詞),後続形式が「テイル」を含んでいる場合(C1)と,
    それ以外の場合(C2)を分けて登録する.また,「ナイ」,「ズ」などの否定
    を表す形式
    \footnote{
      「*しばらく爆発する」 $\rightarrow$ 「しばらく爆発しない」のよう
      に,否定することによって,一時点的なものも,その動きがないという持
      続を持ちうる.
      }や,「レル」,「サレル」,「セル」,「サセル」など,ボイスに変更
    をもたらす形式
    \footnote{
      「太郎がロープを切っている」 $\rightarrow$ 「ロープが切られている」
      のように,能動,受動の対立が,動作継続か結果継続というアスペクト的
      な対立と結び付いている.
      }が承接している場合は登録しない.
  \end{description}
\item[STEP:5 動詞カテゴリーの決定] リストVERBに存在するすべての動詞に対
  して,
  \begin{description}
  \item[STEP:5-1 形式による絞り込み] 配列FORMに基づき,表
    \,\ref{tab:form1}にしたがって,可能な動詞カテゴリーを絞り込む.
  \item[STEP:5-2 副詞による決定] STEP:5-1で動詞のカテゴリーが一意に決定
    できない場合(カテゴリー6以外のとき),配列MODIFIEDを用いて,以下の
    ように動詞カテゴリーを決定する.
\[\cases{
      {\rm 進展的副詞(G)に修飾されている場合} \cr
      \hfill  \Rightarrow カテゴリー6 \cr
      {\rm 過程修飾(P)があり,結果修飾(E)がない場合} \cr
      \hfill \Rightarrow カテゴリー5 \cr 
      {\rm 過程修飾(P)があり,結果修飾(E)がある場合} \cr
      \hfill \Rightarrow カテゴリー4 \cr
      {\rm 過程修飾(P)がなく,結果修飾(E)がある場合} \cr
      \hfill \Rightarrow カテゴリー3 \cr
      {\rm 一時点化副詞(A)のみに修飾されている場合} \cr
      \hfill \Rightarrow  カテゴリー2 \cr
      {\rm 持続副詞の原形(スル形)修飾(C2)があり,} \cr
      {\rm 過程修飾(P),進展的(G),結果修飾(E)の副} \cr
      {\rm 詞に修飾されていない場合} \cr
      \hfill  \Rightarrow カテゴリー1\cr
      {\rm 上記以外のとき,} 
      \hfill \Rightarrow あいまいなまま出力 \cr
     }
\]
  \end{description}
\end{description}

\begin{table}[htb]
\begin{minipage}[t]{80mm}
\caption{配列PAIR(一部)}
\label{tab:pair}
\begin{tabular}{|l|l|r|}  \hline
副詞 & 動詞 & 共起頻度 \\ \hline \hline
ああ & いう &   1 \\ \hline     
ああ & する &   1 \\ \hline             
ああ & なる &   1 \\ \hline             
相   & 会う &   1 \\ \hline             
あいかわらず & いる &   1 \\ \hline     
あいかわらず & 落ち着く & 1 \\ \hline   
あいかわらず & 加える & 1 \\ \hline     
\end{tabular}
\end{minipage}
\hspace{1mm}
\begin{minipage}[t]{80mm}
\caption{配列ADVERB(一部)}
\label{tab:adverb0}
\begin{tabular}{|l|l|}  \hline
副詞 & ラベル \\ \hline \hline
あいかわらず &  C \\ \hline
あえぎあえぎ &  P \\ \hline
あかあかと   &  P \\ \hline
あくせく     &  P \\ \hline
あたふた     &  P \\ \hline
あたふたと   &  P \\ \hline
あっという間 &  A \\ \hline
\end{tabular}
\end{minipage}
\end{table}

\begin{table}[htb]
\caption{副詞の分類結果}
\label{tab:adverb1}
\centering
\begin{tabular}{|l|r|l|}  \hline
ラベル & 種類数 & 例 \\ \hline \hline
過程修飾副詞   {\sf P}    &   470  & ゆっくり がさがさ ばたばた  ... \\ \hline
進展的副詞     {\sf G}    &    52  & 次第に ますます 徐々に  ... \\ \hline
持続副詞       {\sf C}    &    78  & そのまま ずっと いつまでも ... \\ \hline
一時点化副詞   {\sf A}    &   294  & さっと ぽんと がたっと ... \\ \hline
量規定副詞     {\sf Q}    &    12  & 180度 一杯 一歩 一時間 ... \\ \hline
結果修飾副詞   {\sf E}    &    86  & まっぷたつに こなごなに ... \\ \hline 
繰り返しの規定 {\sf R}    &   122  & 何度も いつも しょっちゅう ... \\ \hline 
過去の副詞     {\sf T}    &    11  & かつて むかし 以前 ... \\ \hline
\end{tabular}
\end{table}
\vspace*{20mm}
\begin{table}[htb]
\caption{実験に用いたアスペクト形式}
\label{tab:form1}
\centering
\begin{tabular}{|l|l|}  \hline
形式                     & 共起可能な動詞カテゴリー \\ \hline\hline
ヨウトスル,カケル       & 2, 3, 4, 5, 6 \\ \hline
続ケル                   & 3, 4, 5, 6 \\ \hline
始メル                   & 4, 5, 6 \\ \hline
終ワル,終エル           & 4, 5, 6 \\ \hline
ツツアル,テクル,テイク & 6 \\ \hline
\end{tabular}
\end{table}

\vspace*{50mm}
\begin{table}[hbt]
\caption{配列FORM(一部)}
\label{tab:form2}
\centering
\begin{tabular}{|l|r|r|r|r|r|}  \hline
動詞      & \multicolumn{5}{|c|}{形式}            \\ \hline
          & カケル & 続ケル & 始メル & 終ワル & テクル \\ \hline \hline
悪化する & 0 & 0 & 1 & 0 & 1 \\ \hline
握る     & 1 & 1 & 0 & 0 & 0 \\ \hline
安定する & 0 & 0 & 1 & 0 & 1 \\ \hline
意識する & 0 & 0 & 1 & 0 & 1 \\ \hline
異なる   & 0 & 0 & 0 & 0 & 0 \\ \hline
移動する & 0 & 0 & 1 & 0 & 0 \\ \hline
維持する & 0 & 1 & 0 & 0 & 0 \\ \hline
違う     & 0 & 0 & 0 & 0 & 0 \\ \hline
育つ     & 0 & 0 & 0 & 0 & 1 \\ \hline
育てる   & 0 & 0 & 0 & 0 & 1 \\ \hline
一致する & 0 & 0 & 1 & 0 & 0 \\ \hline
\end{tabular}
\end{table}

\clearpage
\begin{table}[htb]
\caption{配列MODIFIED(一部)}
\label{tab:modified}
\centering
\begin{tabular}{|l|r|r|r|r|r|r|r|}  \hline
動詞       &  \multicolumn{7}{|c|}{副詞クラス}            \\ \hline
          & {\sf P} & {\sf G} & {\sf C1} & {\sf C2} & {\sf A} & {\sf Q} & {\sf E} \\ \hline \hline
悪化する & 0 & 5 & 0 & 0 & 1 & 0 & 0 \\ \hline
握る     & 0 & 1 & 0 & 1 & 0 & 0 & 1 \\ \hline
安定する & 0 & 1 & 1 & 1 & 0 & 0 & 1 \\ \hline
意識する & 0 & 1 & 0 & 1 & 0 & 0 & 0 \\ \hline
異なる   & 0 & 1 & 0 & 0 & 0 & 0 & 1 \\ \hline
移動する & 1 & 1 & 0 & 1 & 1 & 0 & 0 \\ \hline
維持する & 0 & 0 & 0 & 4 & 0 & 0 & 0 \\ \hline
違う     & 0 & 1 & 0 & 0 & 1 & 0 & 0 \\ \hline
育つ     & 5 & 3 & 0 & 0 & 0 & 1 & 1 \\ \hline
育てる   & 3 & 1 & 0 & 1 & 0 & 0 & 0 \\ \hline
一致する & 0 & 0 & 0 & 0 & 3 & 0 & 2 \\ \hline
\end{tabular}
\end{table}

\begin{table}[htb]
\caption{副詞クラスと共起可能な動詞カテゴリー}
\label{tab:adverb2}
\centering
\begin{tabular}{|l|l|}  \hline
副詞クラス      & 動詞カテゴリー \\ \hline\hline
過程修飾副詞   {\sf P}    & 4, 5, 6 \\ \hline
進展的副詞     {\sf G}    & 6 \\ \hline
持続副詞       {\sf C}    & 1, 3, 4, 5, 6 \\ \hline
一時点化副詞   {\sf A}    & 2, 3, 4, 5, 6 \\ \hline
量規定副詞     {\sf Q}    & 1, 3, 4, 5, 6 \\ \hline
結果修飾副詞   {\sf E}    & 3, 4, 6 \\ \hline 
\end{tabular}
\end{table}

STEP:1〜3は,実験対象とする動詞を決定するための処理である.STEP:2は,前
節で述べた副詞の分類を与える処理である.この処理は手作業で行なったが,副
詞は,動詞よりも少数であり(EDR共起辞書に存在する副詞は2,563件,動詞は
12,766件である),イコン性(形式と意味の同型性)が高いので
\footnote{
  「がらがら」,「ころころ」,「ゆらゆら」など,反復型の語は,反復継続す
  る状態を描写するのに用いられ,「どっと」「ガラッと」「デンと」など,一
  音節で促音や撥音を含む語の後に「と」が付いたものは,一回限りの,動きの
  激しい状態を描写するすることが多い.また,母音の組み合わせ方には,変化
  の状態が反映される\cite{大坪82}. },分類は比較的容易である.また,こ
の結果は,動詞のカテゴリー決定のみではなく,以下で述べる「シテイル」形の
意味の推定においても重要な役割を果たす.これらの副詞と5回以上共起してい
る動詞は,431個あり,これを実験対象とした.

STEP:4は,動詞のカテゴリー決定に使用するアスペクト形式および副詞を,動詞
ごとに登録する処理であり,この処理で得られたデータを用いて,STEP:5で動詞
のカテゴリーを決定した.先に述べたように,形式のみでは動詞のカテゴリーを
一意に決定することはできない.そこで,共起する副詞の情報を用いて,カテゴ
リーを推定した.表\,\ref{tab:adverb2}に示したように,副詞は,クラスごと
に,共起できる動詞のカテゴリーが制限されるからである.

\subsection{評価}

実験対象とした431個の動詞のうち,375個は一意にカテゴリーが得られた.残り
の56個のうち,37個はSTEP:5-1で絞り込まれたカテゴリーと,STEP:5-2で決定さ
れたカテゴリーが矛盾するものである.これは,STEP:5-1で形式を獲得する際に,
繰り返しの規定を発見できなかったものと思われるので,STEP:5-2で決定したカ
テゴリーを優先した.結果を表\,\ref{tab:result}に示す.

\begin{table}[h]
\caption{実験で得られた動詞分類}
\label{tab:result}
\centering
\begin{tabular}{|l|r|l|}  \hline
カテゴリー       & 件数               & 動詞の例 \\ \hline\hline
1.状態的             &  30 & 見つめる 維持する 住む 存在する \\
                     &     & 眺める 黙る 繰り返す 使える ... \\ \hline
2.一点的             &  19 & 投げる はね上がる 気づく 見かける \\
                     &     & 合意する 切れる 踏み切る ... \\ \hline
3.変化+結果持続     &  29 & ぬれる つまる つながる 合う 座る \\
                     &     &  暮れる たたむ 当てはまる ... \\ \hline
4.過程+結果持続     &  30 & たてる 立てる 並ぶ のばす まとめる \\
                     &     & 包む 交わる 散る 取り囲む ... \\ \hline
5.非進展的過程       &  94 & 飲む 運ぶ 楽しむ 観察する 震える \\
                     &     & 響く 飛び回る 過ごす 食べる ...\\ \hline
6.進展的過程         & 210 & 悪化する 強まる 強める 高まる 高める \\
                     &     &  深刻化する 活発化する 成長する ...\\ \hline
一意に決まらな       &  19 &  加わる つとめる 伴う 訪ねる 来日する \\
かったもの           &     &  つきまとう 果たす 誇る 上回る ... \\ \hline
\end{tabular}
\end{table}

約80\%以上の動詞については,正しいカテゴリーに分類されていると思われるが,
この判断は主観的なものである.そこで,これを客観的に評価するために,「シ
テイル」形の意味を,先の実験によって得られた動詞のカテゴリーおよび副詞の
分類を用いて,自動的に推定するための実験を行なった.

\begin{figure}
\begin{center}
	\leavevmode
	
      	\includegraphics[width=100mm,clip]{timeline.eps}
	
\end{center}
\caption{動詞カテゴリーの時間軸による表現}
\label{fig:timeline}
\end{figure}

「シテイル」形は,動きを状態的に取り上げるものである.状態的に取り上げる
とは,あり方に質的な断層を前提しないということであり,この断層の欠如によっ
て,時間的に連続するものとしてとらえることが可能となる.すなわち,「スル」
対「シテイル」のアスペクト的な対立とは,閉的区間と開的区間の違いである
\cite{森山84}.図\,\ref{fig:timeline}に,それぞれのカテゴリーの時間軸に
よる表現を示す.図\,\ref{fig:timeline}において,太線で示したところが動詞
の表示する動きや状態であり,破線は,時間的に続くこと,つまり,開的区間を
示している.また,丸印は,動きの始まりや終結点を表しており,この点によっ
て,時間的な断層,つまり,閉的区間を表したものである.「テイル」は,動き
を状態的に取り上げるものであるから,この点を含むことはできない.したがっ
て,「シテイル」形の意味は,時間軸の下に示した区間のどれかでなければなら
ない.(1)は,状態述語の状態を取り上げるものであり,この場合,特に現前の
状態を述べる意味になる.金田一以来,「シテイル」形の意味は,「動作,作用
が進行中であること」,「動作,作用が終わって,その結果が残存していること」,
「単なる状態」の三つの類型が考えられているが,三番目の,「単なる状態」が,
これにあたる.また,(4),(7)が,「結果の状態」に,(6),(9),(11)が「進行中」
の意味にそれぞれ相当する.図\,\ref{fig:timeline}には記載していないが,5.
非進展的過程,6.  進展的過程の動詞でも,明示的に終結点が設定されれば,
「結果の状態」が取り上げられる.

工藤は,さらに,「進行中」,「結果の状態」に,基本的な意味と,派生的な意
味があることを指摘して,次のような位置づけをしている\cite{工藤82}.
\begin{itemize}
\item[(i)]   「進行中」の基本的な意味 「動きの継続」
\item[(ii)]  「進行中」の派生的な意味 「反復」
\item[(iii)] 「結果の状態」の基本的な意味 「変化の結果の継続」
\item[(iv)]  「結果の状態」の派生的な意味 「現在有効な過去の運動の実現」
\end{itemize}

(ii)は,先に述べた繰り返しとしての過程を問題にするものであり,いわば,
「点の集合としての線」として,複数の事態をとらえるものである.(iv)は,い
わゆる「経歴」を表すものであり,図\,\ref{fig:timeline}の
(2),(3),(5),(8),(10),(12)に相当する.これらの派生的なものは,派生的であ
るがゆえに,構文的,あるいは,文脈的に条件づけられている.すなわち,副詞
等で明示的に派生的な意味であることが表現されることが多い.その一方,動詞
の語彙的な意味からは解放されている.すなわち,ほぼあらゆる動詞がこれらの
意味をあらわすことができ,動詞のカテゴリーと直接関係しない.

以上の考察に基づいて,(i)から(iv)に(v)「単なる状態」を加えた五つの意味を
表層表現によって区別する実験を行なった.以下に手順を示す.
\begin{itemize}
\item[1.] 繰り返しを規定する副詞(R)が含まれているとき \\ 
  $\Rightarrow$ (ii)「反復」
\item[2.] 過去の副詞(T)が共起しているか,動詞のカテゴリーが2(一点的)
  のとき \\
  $\Rightarrow$ (iv)「経歴」
\item[3.] 動詞のカテゴリーが1(状態的)のとき \\ 
  $\Rightarrow$ (v)「単なる状態」
\item[4.] 動詞のカテゴリーが3(変化+結果持続)のとき \\ 
  $\Rightarrow$ (iii)「結果の継続」
\item[5.] 上記以外のとき
  \begin{itemize}
  \item[5-1.] 過程修飾副詞(P)または進展的副詞(G)が共起しているとき \\ 
  $\Rightarrow$ (i)「進行中」
  \item[5-2.] 終結点が設定されているとき(結果修飾副詞(E),量規定副詞(Q),
  「に」格,「まで」格の共起) \\
  $\Rightarrow$ (iii)「結果の継続」
  \item[5-3.] 過程が取り上げられない条件があるとき(一時点化副詞(A),「すで
  に」,「もう」などの共起) \\
  $\Rightarrow$ (iii)「結果の継続」
  \item[5-4.] 上記のいずれにもあてはまらないとき
    \begin{itemize}
    \item[5-4-1.] 動詞カテゴリーが5(非進展的過程)または6(進展的過程)
      のとき  \\ $\Rightarrow$ (i)「進行中」
    \item[5-4-2.] 動詞カテゴリーが4(過程+結果持続)のとき \\ $\Rightarrow$
      (i) 「進行中」or (iii)「結果の継続」であいまい
    \end{itemize}
  \end{itemize}
\end{itemize}

実験は,EDR日本語コーパスから,文末に「テイル」を含んでいる200文をラ
ンダムに抽出して行なった.結果を表\,\ref{tab:evaluation}に示す.

\begin{table}[htb]
\caption{評価実験の分析結果}
\label{tab:evaluation}
\centering
\begin{tabular}{|l|r|r|r|r|r|}  \hline
「シテイル」形   & 人間による & プログラム & 判断が一致  & 再現率(\%) & 正解率(\%) \\ 
 の意味          & 判断(a)    & の出力(b)  & したもの(c) & c/a×100   & c/b×100 \\ \hline\hline
(i) 動きの継続   &  95 & 137 &  88 & 93 & 64 \\ \hline
(ii) 反復        &   4 &   2 &   2 & 50 & 100 \\ \hline
(iii) 結果の継続 &  29 &  15 &  14 & 48 & 93 \\ \hline
(iv) 経歴        &  39 &  15 &  14 & 36 & 93 \\ \hline
(v) 単なる状態   &  19 &  19 &  15 & 79 & 79 \\ \hline
あいまいなもの   &  14 &  12 &   9 & 64 & 75 \\ \hline
合  計           & 200 & 200 & 142 & 71 & 71 \\ \hline
\end{tabular}
\end{table}
\vspace*{-2mm}
表\,\ref{tab:evaluation}によると,全体の正解率は71\%であるが,(i)「動き
の継続」は再現率が高く,正解率が低い.一方,(iii)「結果の継続」と(iv)
「経歴」は,逆に正解率は高いが再現率が低いことがわかる.これは,本来,
(iii)「結果の継続」または(iv)「経歴」とすべきものを,(i)「進行中」として
いるものが多いことを示している.この原因は,テストセンテンスの中に,「主
張する」,「説明する」,「表明する」のように,引用の「と」格をとる動詞を
含んでいるものが多かったことによる.これらは,カテゴリーが5(非進展的過
程)として分類されているので,上記の手順5-4-1により,(i)「進行中」の意味
と決定されたが,この場合,発話内容を表す「と」格が動きの全体量を規定して
おり,これが終結点を設定していると考えられるので,(iii)「結果の継続」ま
たは(iv)「経歴」の読みが優先されるものである.また,これ以外の原因として
は,「かかる」,「あたる」のような多義的な動詞が挙げられる.これらの動詞
は,格成分によって,そのアスペクト的な意味が変わってくる.たとえば,
\begin{itemize}
\item [] 今,彼は,コップの水をビーカーに移している.(進行中)
\item [] 今,彼は,住民表を生駒市に移している.(結果の状態)
\end{itemize}
のように,格成分が変わることによって,同じ動詞でもその解釈は異なってくる
のである.どのような名詞句が,どのような動詞と組合わさって,アスペクトの
解釈にどのような影響を与えるかは,複雑かつ微妙な問題であり,現在のところ,
我々は,根本的な解決には至っていない.今後の課題である.

\section{おわりに}

本稿では,コーパスに現れた表層表現に基づいて,動詞を六つのカテゴリーに分
類し,この結果をもとに,「シテイル」形の意味を推定する手法について述べた.

動詞句のアスペクトは,出来事の時間的様態を表すだけではなく,出来事間の時
間関係を考えるための基礎となるものである.さらに,時間関係抜きの因果関係
はありえず,また,時間関係をとらえるとすれば,そこに暗示的に因果関係が含
み込まれてくる\cite{工藤95}.

また,動詞の語彙的アスペクトは,表層格とは直接的で有意味な関係を持たない
が,動詞の意味のタイプと直接的関係を持つ.この意味のタイプは,動詞とその
補語構造と関係するので,この点において,いわゆる深層格と有意味な関係を持
つ\cite{金子95}.したがって,表層格と語彙的アスペクトを組み合わせること
によって,動詞の意味のタイプを細かく分類することができる\cite{Oishi96ae}.

このように,ここで得られた動詞のアスペクト情報は,文章理解や機械翻訳など
の分野で用いるべき意味的な情報の一部として利用可能であると考える.



\bibliographystyle{jnlpbbl}
\newcommand{\kokuken}{}
\begin{thebibliography}{}

\bibitem[\protect\BCAY{Bennett, Herlick, Hoyt, Liro, \BBA\ Santisteban}{Bennett
  et~al.}{1990}]{Bennett90}
Bennett, S.~W., Herlick, T., Hoyt, K., Liro, J., \BBA\ Santisteban, A. \BBOP
  1990\BBCP.
\newblock \BBOQ A Computational Model of Aspect and Verb Semantics\BBCQ\
\newblock {\Bem Mashine Translation}, {\Bbf 4}  (4), 247--280.

\bibitem[\protect\BCAY{Comrie}{Comrie}{1976}]{Comrie76}
Comrie, B. \BBOP 1976\BBCP.
\newblock {\Bem Aspect}.
\newblock Cambridge Textbooks in Linguistics. Cambridge University Press.

\bibitem[\protect\BCAY{Dorr}{Dorr}{1993}]{Dorr93}
Dorr, B.~J. \BBOP 1993\BBCP.
\newblock {\Bem Machine Translation --- A View from the Lexicon}.
\newblock The MIT Press.

\bibitem[\protect\BCAY{Gunji}{Gunji}{1994}]{郡司94}
Gunji, T. \BBOP 1994\BBCP.
\newblock \BBOQ A Proto-lexical Analysis of Temporal Properties of Japanese
  Verbs\BBCQ\
\newblock \Jem{日本語句構造文法に基づく効率的な構文解析の研究 平成5
  年度科学研究費補助金研究成果報告書(03452169)}, pp.65--75.

\bibitem[\protect\BCAY{Oishi \BBA\ Matsumoto}{Oishi \BBA\
  Matsumoto}{1996}]{Oishi96ae}
Oishi, A.\BBACOMMA\  \BBA\ Matsumoto, Y. \BBOP 1996\BBCP.
\newblock \BBOQ Detecting the Organization of Semantic Subclasses of Japanese
  Verbs\BBCQ\
\newblock \BTR\ NAIST-IS-TR96019, Nara Institute of Science and Technology.

\bibitem[\protect\BCAY{Talmy}{Talmy}{1985}]{Talmy85}
Talmy, L. \BBOP 1985\BBCP.
\newblock {\Bem Lexicalization patterns: semantic structure in lexical forms},
  \lowercase{\BVOL}~3 of {\Bem Language Typology and Syntactic Description},
  \BCH~2, \BPGS\ 57--308.
\newblock Cambridge University Press.

\bibitem[\protect\BCAY{Tenny}{Tenny}{1994}]{Tenny94}
Tenny, C.~L. \BBOP 1994\BBCP.
\newblock {\Bem Aspectual Roles and the Syntax-Semantics Interface},
  \lowercase{\BVOL}~52 of {\Bem Studies in Linguistics and Philosophy(SLAP)}.
\newblock Kluwer Academic.

\bibitem[\protect\BCAY{Vendler}{Vendler}{1957}]{Vendler57}
Vendler, Z. \BBOP 1957\BBCP.
\newblock \BBOQ Verbs and times\BBCQ\
\newblock {\Bem Philosophical Review}, {\Bbf 66}, pp.143--160.

\bibitem[\protect\BCAY{(株)日本電子化辞書研究所}{(株)日本電子化辞書研究所}{199
5}]{EDR95aj}
(株)日本電子化辞書研究所 \BBOP 1995\BBCP.
\newblock \Jem{EDR電子化辞書仕様説明書(第2版)}.

\bibitem[\protect\BCAY{金子}{金子}{1995}]{金子95}
金子亨 \BBOP 1995\BBCP.
\newblock \Jem{言語の時間表現}.
\newblock ひつじ研究叢書(言語編) 第7巻. ひつじ書房.

\bibitem[\protect\BCAY{国語学会編}{国語学会編}{1993}]{Kokugo93}
国語学会編 \BBOP 1993\BBCP.
\newblock \Jem{国語学大辞典(第8版)}.
\newblock 東京堂出版.

\bibitem[\protect\BCAY{金田一}{金田一}{1976}]{Kindaiti76}
金田一春彦 \BBOP 1976\BBCP.
\newblock \Jem{日本語動詞のアスペクト}.
\newblock むぎ書房.

\bibitem[\protect\BCAY{工藤}{工藤}{1982}]{工藤82}
工藤真由美 \BBOP 1982\BBCP.
\newblock \JBOQ シテイル形式の意味記述\JBCQ\
\newblock \Jem{武蔵大学 人文学会雑誌}, {\Bbf 13}  (4).

\bibitem[\protect\BCAY{工藤}{工藤}{1995}]{工藤95}
工藤真由美 \BBOP 1995\BBCP.
\newblock \Jem{アスペクト・テンス体系とテクスト 現代日本語の時間の表現}.
\newblock 日本語研究叢書. ひつじ書房.

\bibitem[\protect\BCAY{森山}{森山}{1983}]{森山83}
森山卓郎 \BBOP 1983\BBCP.
\newblock \JBOQ 動詞のアスペクチュアルな素性について\JBCQ\
\newblock \Jem{待兼山論叢}, 17\JVOL, \BPGS\ 1--22. 大阪大学国文学研究室.

\bibitem[\protect\BCAY{森山}{森山}{1984}]{森山84}
森山卓郎 \BBOP 1984\BBCP.
\newblock \JBOQ テンス、アスペクトの意味組織についての試論\JBCQ\
\newblock \Jem{語文}, 44\JVOL, \BPGS\ 1--14. 大阪大学国文学研究室.

\bibitem[\protect\BCAY{森山}{森山}{1988}]{森山88}
森山卓郎 \BBOP 1988\BBCP.
\newblock \Jem{日本語動詞述語文の研究}.
\newblock 明治書院.

\bibitem[\protect\BCAY{本川}{本川}{1992}]{本川92}
本川達雄 \BBOP 1992\BBCP.
\newblock \Jem{ゾウの時間 ネズミの時間}.
\newblock 中公新書. 中央公論社.

\bibitem[\protect\BCAY{大坪}{大坪}{1982}]{大坪82}
大坪併治 \BBOP 1982\BBCP.
\newblock \JBOQ 象徴語彙の歴史\JBCQ\
\newblock 森岡健二, 宮地裕, 寺村秀夫, 川端善明\JEDS, \Jem{語彙史}, \Jem{講座
  日本語学}, 4\JVOL, \BPGS\ 228--250. 明治書院.

\bibitem[\protect\BCAY{養老}{養老}{1989}]{養老89}
養老孟司 \BBOP 1989\BBCP.
\newblock \Jem{唯脳論}.
\newblock 青土社.

\bibitem[\protect\BCAY{養老}{養老}{1996}]{養老96}
養老孟司 \BBOP 1996\BBCP.
\newblock \Jem{考えるヒト}.
\newblock ちくまプリマーブックス. 筑摩書房.

\bibitem[\protect\BCAY{原, 北上, 中島}{原\Jetal }{1988}]{原88}
原裕貴, 北上始, 中島淳 \BBOP 1988\BBCP.
\newblock \JBOQ 時間概念の表現とデフォルト推論\JBCQ\
\newblock \Jem{人工知能学会誌}, {\Bbf 3}  (2), pp.216--223.

\end{thebibliography}

\newpage

\begin{biography}
\biotitle{略歴}
\bioauthor{大石 亨}{
昭和37年生.
昭和59年大阪大学文学部文学科卒業.
同年奈良県庁入庁.
平成7年奈良先端科学技術大学院大学情報科学研究科博士前期課程修了.
現在,同大学院博士後期課程在学中.
自然言語処理の研究に従事.
情報処理学会, ACL各会員.}
\bioauthor{松本 裕治}{
昭和30年生.昭和52年京都大学工学部情報工学科卒.昭和54年同大学大
学院工学研究科修士課程情報工学専攻修了.同年電子技術総合研究所入
所.昭和59〜60年英国インペリアルカレッジ客員研究員.昭和60〜62年
(財)新世代コンピュータ技術開発機構に出向.京都大学助教授を経て,
平成5年より奈良先端科学技術大学院大学教授,現在に至る.専門は自然
言語処理.人工知能学会,日本ソフトウェア科学会,情報処理学会,
AAAI, ACL, ACM各会員.
}


\end{biography}

\end{document}

