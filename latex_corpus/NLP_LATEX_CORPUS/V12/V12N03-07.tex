\documentclass{nlp}
\usepackage{tabularx}

\begin{document}

\newcounter{example}
\setcounter{example}{1}

\setcounter{page}{3}
\setcounter{Volume}{2}
\setcounter{Number}{3}
\setcounter{Month}{7}
\received{1995}{5}{6}
\revised{1995}{7}{8}
\accepted{1995}{9}{10}

\title{名詞格フレーム辞書の自動構築とそれを用いた名詞句の関係解析}
\author{笹野 遼平\affiref{KUEE} \and 河原 大輔\affiref{KUEE} \and 
黒橋 禎夫\affiref{KUEE}}

\headauthor{笹野,河原,黒橋}

\affilabel{KUEE}{東京大学大学院情報理工学系研究科,
Graduate School of Information Science and Technology, University of Tokyo}

\begin{abstract}
 本稿では、コーパスから名詞句「AのB」を収集し、国語辞典を用いて意味解
 析を行なうことにより、名詞格フレーム辞書を自動構築する手法を提案する。
 また、自動構築した名詞格フレーム辞書の有用性を調べるため、名詞格フレー
 ム辞書に基づく名詞間の関係解析システムを構築する。自動構築した名詞格フ
 レーム辞書の評価を人手で作成したものとの比較により行ったところ高い精度
 で構築されていることが確認でき、また、関係解析実験の結果から名詞格フレー
 ムの有用性を確認できた。
\end{abstract}

\keywords{名詞格フレーム,照応解析、間接照応}

\etitle{Automatic Construction of Nominal Case Frames and \\ 
its Application to Indirect Anaphora Resolution}
\eauthor{Ryohei Sasano\affiref{KUEE}\and  Daisuke Kawahara\affiref{KUEE}
\and Sadao Kurohashi\affiref{KUEE}}

\begin{eabstract}
This paper proposes a method to automatically construct Japanese
nominal case frames. The point of our method is the integrated use of a
dictionary and example phrases from large corpora. To examine the
practical usefulness of the constructed nominal case frames, we built a
 system of indirect anaphora resolution based on the constructed case 
frames. The case frames were evaluated by hand, and were confirmed to be
 good quality. Experimental results of indirect anaphora 
resolution also indicated the effectiveness of our approach.
\end{eabstract}

\ekeywords{Nominal Case Frame, Anaphora Resolution, Indirect Anaphora}

\maketitle

 \section{はじめに}

 文章が表す意味内容は複雑な関係を持っている。しかし、文章は一次元の文字
 列で表現されるためこれらの関係の多くは明示されない。このような文章中の
 要素間の隠れた関係の解析を行うことは、文章理解の第一歩であり、自動要約、
 機械翻訳、質問応答などの言語処理アプリケーションを高度化するために必要
 となる。このような隠れた関係の一つに、間接照応表現とその先行詞の関係が
 ある。ここで間接照応とは、文章中に現れていないがすでに言及されたことに
 関係する事物を間接的に指示する用法である。
 \vspace{1ex}\\
 \hspace{1em}(\theexample)
 \stepcounter{example}
 チケットを買った。値段は2000円だった。\vspace{1ex}\\
 \hspace{1em}(\theexample)
 \stepcounter{example}
 池のほとりに小屋があった。屋根は白かった。\vspace{1ex}\\
 例えば、(1)における「値段」とは「チケットの値段」であり、「チケット」と
 「値段」は強い関連性を持っている。「何かの値段」であることは明示的に表
 現されてはいないが、「値段」とは「何かの値段」であり、この場合はその
 「何か」にあたるものが「チケット」である。同様に(2)の文における「屋根」
 とは「小屋の屋根」であり、「小屋」と「屋根」は強い関連性を持っている。

 (1)における「チケット」と「値段」や、(2)における「小屋」と「屋根」の間
 の関係は、主に後者の名詞の性質に起因していると考えられる。「値段」とは
 「ある商品の値段」であり、「値段」には「何という商品の値段であるか」と
 いう情報が必要となる。同様に「屋根」には、「何という建物の屋根であるか」
 という情報が必要となる。このように多くの名詞にはその名詞にとって必須的
 な要素が存在しており、名詞における名詞とその必須要素との関係は、用言に
 おける用言とその必須格との関係に相当すると考えられる。そこで本稿では、
 その名詞にとって必須的な名詞の働きを名詞の必須格と呼ぶことにする。
 
 名詞の必須格にあたる語が何であるかを解析することは、文章理解を行うため
 に重要である。このような解析を行うためには、その名詞がどのような必須格
 をとるのか、各必須格にはどのような用例があるのかという知識が必要となる。
 しかし、それらの情報を記述した名詞格フレーム辞書は今のところ存在せず、
 また、数万語あるいは数十万語という規模が必要となることから人手で作成す
 ることはほとんど不可能である。
 
 動詞・形容詞など用言の格フレーム辞書については、自動構築する手法がすで
 に提案されている\cite{Briscoe1997,Kawahara2002e}。これらの手法ではコー
 パス中に表層的に明示された格情報とその用例を用いて格フレームを自動構築
 している。名詞でこれらの手法を応用する場合、ノ格として対象の名詞に係る
 表現の用例を用いることが考えられる。しかし、用言の場合はガ格やヲ格といっ
 た格の種類によって対象の語との関係がある程度制限されるのに対し、名詞の
 場合、ノ格により2つの名詞が関係を持つことが明示されていても、それらの持
 つ関係は様々である。このため、名詞格フレーム辞書を構築する場合には名詞
 句の関係の解析を行うことが必要となる。
 
 英文における間接照応に関連した研究としては、人手で作成したルールを使用
 したものや、WordNetのような人手で作成した辞書的知識を用いたものがある
 \cite{Hahn1996,Vieira2000,Strube1999}。Poesioらは''B of A''という名詞句
 から辞書的知識を獲得する手法を提案しているが、対象としているのは(2)の例
 における「小屋」と「屋根」のような全体-部分関係のみであり、(1)における
 「チケット」と「値段」の関係は解析できない\cite{Poesio2002}。

 また、日本語における間接照応解析の研究としては、村田らが名詞句「AのB」
 を用いた照応解析手法を提案している\cite{Murata1999}。しかし、基本的に全
 ての「AのB」の用例を名詞格フレーム辞書として使用しており、名詞句「A
 のB」の解析を高精度に行うことにより間接照応解析の精度が上がる可能性に
 言及している。
 
 そこで本稿では、コーパスから名詞句「AのB」を収集し、名詞句の意味解析
 を国語辞典の定義文を利用した手法\cite{Kurohashi1999}を用いて行い、その
 結果から名詞格フレーム辞書を自動構築する手法を提案する。また、自動構築
 した名詞格フレーム辞書の有用性を調べるため、自動構築した名詞格フレーム
 に基づいた名詞句の関係解析システムを構築し、評価を行う。

\section{名詞句「AのB」の意味解析}
 \label{意味解析}
 
 名詞句「AのB」における表層的に同じ接続助詞「の」で結ばれる二つの名詞
 は、所有、目的、道具などの様々な意味的関係を持ち得る。黒橋らは国語辞典
 に出現する必須的な要素に注目し、さらに一部の語についてはシソーラスを用
 いることで、「AのB」の意味解析が高精度に行えることを示した。以下では
 その手法を簡単に示す。
 
  \subsection{国語辞典を用いた名詞句意味解析}
  
  国語辞典の定義文には、その語に必須的な要素が含まれていることが多い。例
  えば、『例解小学国語辞典』\cite{Reikai}における「値段」の定義文は次の
  ようになっており、「品物」という必須要素が含まれている。
  \vspace{1.5ex}\\
  \hspace{1em}【値段】品物を売り買いするときの金額。
  \vspace{1.5ex}\\ 
  このような定義文に含まれる必須要素との対応付けを行うことにより、例えば
  「チケットの値段」という名詞句は「チケットを売り買いするときの金額」と
  いう意味であると解析できる。
  \vspace{1.5ex}\\
  \hspace{1em}【コピー】◯\hspace{-0.77em}1\hspace{0.6em}書類など、もと
  のものと同じものを写しとること\\
  \hspace{5.8em}◯\hspace{-0.77em}2\hspace{0.6em}広告の文案。
  \vspace{1.5ex}\\
  また、「コピー」のように多義性のある名詞の場合、「資料のコピー」という
  名詞句では「資料」と定義文中の「書類」を、「CMのコピー」という名詞句
  では「CM」と定義文中の「広告」を対応付けることにより多義性を解消する
  ことが可能となる。
  
  解析手順を以下に示す。
  \begin{enumerate}
   \item Bの定義文から、Aと対応付ける候補語としてBの上位語以外の名詞
	 Wをとりだす。基本的に定義文の末尾の語を上位語として扱う。
   \item シソーラスを用いてWとAとの類似度を計算し、その値をWとAとの
	 対応付けのスコアとする。
   \item 以上の処理により最大のスコアをとるWにAを関連付け、そのスコア
	 を国語辞典を用いた意味解析のスコアとする。
  \end{enumerate}

  シソーラスとしては日本語語彙大系\cite{GoiTaikei}を用いる。日本語語彙大
  系では約30万語を約3,000種類の意味属性に分類している。以下では、意味属
  性を《...》で表記する。
  
  2語の類似度はシソーラスの木構造での2語の近さから算出する、XとYの類似
  度を計算する場合、木構造におけるXの深さd$_{X}$、Yの深さd$_{Y}$、Xと
  Yの共通の親の深さd$_{S}$を用いて、(d$_{S}$$\times$2)
  /(d$_{X}$+d$_{Y}$)を類似度として用いる。例えばXとYが同じ位置にある場
  合、類似度は最大値1となる。ただし、「ことがら」、「物ごと」、「何か」
  などの表現は例外的に扱い、Aの意味属性が《人》、《時間》である場合は類
  似度0、それ以外の場合ば類似度を0.8としている。
   
  \subsection{意味属性パターンを用いた名詞句意味解析}
  
  国語辞典中に適当な定義文のない語や、国語辞典に載っていない語が存在する。
  例えば、『例解小学国語辞典』における「姉」の定義文は次のようになってお
  り、この定義文だけを用いて「彼の姉」における「彼」と「姉」の関係を解析
  するのは困難である。
  \vspace{1.5ex}\\
  \hspace{1em}【姉】年上の、女のきょうだい。 
  \vspace{1.5ex}\\  
  このような語に対しても高精度な解析を行えるようにするため、国語辞典を用
  いた解析に加えて、意味属性パターンに関するルールを設定し、ルールに基づ
  いた意味解析を行う。意味解析の結果として与える関係として、\texttt{<必
  須格(親族)>、<必須格(対人)>、<必須格(位置)>、<必須格(属性)>、<所属>、<
  所有>、<主体>、<場所>、<時間>、<修飾>}を設定する。このようなルールの例
  を表\ref{RuleEx}に示す。このルールに基づいた解析によって、例えば「私の
  姉」には\texttt{<必須格(親族)>、「公式の会談」には<修飾>}の関係を与
  えることができる。

  \begin{table}
   \small
   \caption{意味属性パターンに関するルールの例}
   \label{RuleEx}
   \begin{center}
    \begin{tabular}{ll}\hline 
     \multicolumn{1}{c}{ルール} & 「AのB」の例\\\hline
     A:《人》, B:《人間<親族関係>》 → \texttt{<必須格(親族)>}
     & 私の姉, 監督の子供\\
     A:《人》, B:《人間<対人関係>》 → \texttt{<必須格(対人)>}
     & 父の上司, 母の友人\\
     A:《*》,  B:《度量衡》 → \texttt{<必須格(属性)>}
     & 子供の年齢, 川の深さ\\
     A:《*》, B:《場》 → \texttt{<必須格(位置)>}
     & 家の中, 町のはずれ\\
     A:《性質》or《状態》or《形状》,B:《*》 → \texttt{<修飾>}
     & 公式の会談, 三角の屋根\\
     A:《資材》, B:《具体》 → \texttt{<修飾>}
     & 木の机, レンガの家\\
     A:《人》, B:《人》 → \texttt{<修飾>}
     & 男性の客, 女性の店員\\
     A:《人》, B:《動作》 → \texttt{<主体>}
     & 医師の薦め, 彼の誘い\\
     A:《時間》, B:《*》 → \texttt{<時間>}
     & 夏の大会, 昼の番組\\
     A:《組織》, B:《主体》 → \texttt{<所属>}
     & 高校の教師, 病院の医師\\
     A:《場所》or《家屋(部分<場>)》, B:《具体》 → \texttt{<場所>}
     & 庭のベンチ, 日本の子供\\
     A:《主体》, B:《*》 → \texttt{<所有>}
     & 自分の能力, 母の趣味\\\hline
    \end{tabular}
   \end{center}
  \end{table}

  \subsection{意味解析の統合}

  多くの場合、国語辞典を用いた解析も意味属性パターンを用いた解析も解析結
  果を出力する。このため、国語辞典を用いた解析と意味属性パターンを用いた
  解析、両方の解析で解析結果が得られる場合がある。このような場合、
  \texttt{<必須格(親族)>}という関係が得られたものや、国語辞典を用いた解
  析で高いスコアが得られたものなど、より信頼性が高いと考えられる解析結果
  を使用する(基本的には、国語辞典を用いた解析のスコアが0.8を越えている場
  合は国語辞典を用いた解析の結果を、それ以外の場合は意味属性パターンを用
  いた解析の結果を使用している)。また、いずれの解析においても適当な解析
  結果が得られない場合は、名詞Aと名詞Bの間に関係がないものと判断し、解
  析結果なしとする。
  
 \section{名詞格フレーム辞書の構築}
  
  本章では名詞格フレーム辞書を自動構築する方法について説明する。名詞格フ
  レーム辞書とは、ある名詞がどのような必須格をとるのか、また各必須格の用
  例にはどのようなものがあるのかを記した計算機用の辞書のことであり、語義
  ごとに必要となる。
  
  このような辞書を作るために使用できる情報源としては、国語辞典の定義文と
  コーパス中に出現する名詞句「AのB」が考えられる。しかし、国語辞典の定
  義文中にはその語の必須要素が含まれていることが多いものの、国語辞典を使
  用するだけでは定義文中のどの名詞が必須要素であるか判断することは困難で
  あり、また、全ての必須要素が定義文中に含まれているとも限らない。一方、
  名詞の必須格の多くは対象の名詞にノ格でかかることから名詞句「AのB」の
  用例を使用することが考えられるが、名詞句「AのB」を使用するだけでは、
  どのような要素が必須要素であるか判断することが困難であり、多義性に対応
  することもできない。

  そこで、本稿ではこれらの2つの情報源を統合的に使用することにより名詞格
  フレーム辞書の自動構築を行うことを考える。

  \subsection{名詞句「AのB」の収集・解析}
  \label{収集}

  コーパスから名詞Bにノ格で係る名詞Aの用例を収集する。名詞Bの格フレー
  ムの用例となる名詞Aを集めるのが目的である。したがって、「AのB」の形
  をしていても、「AのBのC」や「AのBC」のようにAがBに係っているの
  が明らかでないものは収集せず、「AのBが」や「AのBを」のようにBがガ
  格やヲ格であるものなど、AがBに係っている可能性が高いもののみ収集する
  \footnote{「AのBを」という形であっても、「監督の優勝を決めた瞬間の表
  情」のようにAがBに係らない場合もある。しかし、このような用例は稀であ
  る。}。

  次に、収集された「AのB」に含まれる各Bについて集まったAの用例を
  \ref{意味解析}章で述べた方法で解析し(国語辞典として『例解小学国語辞
  典』を用いる)、解析結果ごとにまとめる。解析結果のまとめ方は次のとおり
  である。
  
  \begin{itemize}
   \item 定義文中の語に関係付けられた用例は定義文中の語ごとにまとめる。
	 ただし、定義文中の並列語に集まった用例はひとつにまとめる。
   \item 意味属性パターンを用いた解析で関係が得られたものはその関係でま
	 とめる。
   \item 意味解析結果の得られなかったものは\texttt{<その他>}としてまとめ
	 る。
  \end{itemize}
  
  以下ではこのようにまとめられたそれぞれを格スロットと呼ぶことにし、国語
  辞典の定義文に関連付けられた格スロットを``\texttt{〔...〕}''で、意味属
  性パターンに基づいて作成された格スロットを``\texttt{<...>}''、意味解析
  結果が得られなかったものを\texttt{<その他>}で表記する。Bが「ひさし」
  である場合、Aの格スロットおよび用例は表\ref{ResultEx}のようになる(表
  中の格スロットに続く数字はその格スロットに分類される用例の出現回数の合
  計、用例に続く数字はその用例の出現回数を表している)。

  \begin{table}[t]
   \small
   \caption{「Aのひさし」の用例の収集・解析結果}
   \label{ResultEx}
   \begin{center}
    \begin{tabular}{ll}\hline 
     \multicolumn{2}{l}{\textbf{国語辞典を用いた解析の結果}}\\
     \multicolumn{2}{l}{\begin{tabular}{lp{6.4cm}}
			 1. 家の出入り口や
			 窓などの上にさし出た小さな屋根。
			\end{tabular}} \\[4pt]
     \hspace{2em}〔家〕11 & 玄関2,ビル1,家屋1,建物1,…\\
     \hspace{2em}〔窓〕8 & 屋根2,窓2,入り口1,天井1,…\\
     \multicolumn{2}{l}{\begin{tabular}{lp{6.4cm}}
			 2. ぼうしの前についているつば。
			\end{tabular}} \\[4pt]
     \hspace{2em}〔ぼうし〕13 & 帽子8,兜1,フード1,よけ1,…\\\hline
     \multicolumn{2}{l}{\textbf{意味属性パターンを用いた解析の結果}}\\
     \hspace{2em}\texttt{<場所>}18 & 駐車場3,店舗3,コーナー2,店2,…\\   
     \hspace{2em}\texttt{<修飾>}10 & コンクリート1,下1,金属製1,銀色1,…\\\hline
     \multicolumn{2}{l}{\textbf{意味解析結果が得られなかったもの}}\\   
     \hspace{2em}\texttt{<その他>14} & 部分3,式1,信号機1,電話1,…\\\hline
    \end{tabular}
   \end{center}
   \scriptsize{\vspace{-1.5ex}\hspace{20em}
   コーパス中の「ひさし」の出現回数:292回}
  \end{table}   
          
 \subsection{格スロットのクラスタリング}

   同じ必須要素を表す格スロットが重複して作成される場合がある。例えば表
   \ref{ResultEx}の「ひさし」の場合、〔家〕と〔窓〕、\texttt{<場所>}はほ
   ぼ同じ内容を表している。このような格スロットをまとめるため、類似度が
   0.5以上である格スロットを統合する。格スロットの類似度は各用例間の類似
   度の上位1/4の値の平均値として計算する。ただし、異なる定義文に関係付け
   られた格スロット同士は統合の対象としない。また、統合する格スロットの
   中に定義文に関連付けられた格スロットが含まれている場合は、以降では統
   合された格スロットを定義文に関連付けられた格スロットとして扱う。
       
   「ひさし」の場合、まず〔家〕と〔窓〕の類似度が0.80と計算されるので、
   これらの格スロットは統合され〔家・窓〕となる。次に、〔家・窓〕と
   \texttt{<場所>}の類似度も0.67と計算されるので、これらの格スロットも統
   合される。〔ぼうし〕と〔家・窓〕の類似度も0.58と計算されるが、これら
   は異なる定義文に関連付けられていることから統合されない。

  \subsection{格スロットの選択}
  
  用例の解析結果をまとめた格スロットには必須的でない要素も含まれている。
  このため、どの格スロットが必須的か判断し、格フレーム辞書に含める格スロッ
  トを絞りこむことが必要となる。

  必須的な格スロットであればその用例も多いと考えられる。また、定義文中に
  含まれる語に関連付けれられた格スロットなどは必須要素である可能性が高く、
  意味属性パターンを用いた解析で\texttt{<修飾>}と解析された格スロットは
  必須要素にまずならないなど、格スロットの種類によって必須要素である可能
  性が異なると考えられる。そこで、格スロットの種類ごとに必須格とみなす閾
  値を設定し、閾値を満足した格スロットのみを残す。閾値は50個の名詞の用例
  の収集結果を参考に設定した。設定した閾値を表\ref{thre}に示す。
  
  「ひさし」の場合、「ひさし」の出現回数は292回であるので、閾値を満足す
  る格スロットは用例の出現回数が19回の〔家・窓〕と用例の出現回数が13回の
  〔ぼうし〕の2つだけであり、これら2つの格スロットのみが残される。

  \begin{table}[h]
   \small{
   \caption{必須要素と判断する閾値}
       \label{thre}
   \begin{center}
    \begin{tabular}{cc}\hline
     格スロットの種類 & 用例の出現頻度\\\hline
     定義文に関連付け & 0.5%\\
     \texttt{<必須格(親族・対人・属性・位置)>} & 2.5%\\
     \texttt{<所属>} & 2.5%\\
     \texttt{<所有>・<主体>・<場所>} & 5%\\
     \texttt{<その他>} & 10%\\
     \texttt{<修飾>・<時間>} & (使用しない)\\\hline
    \end{tabular}
   \end{center} 
  \scriptsize{\vspace{-1.5ex}\hspace{10.7em}
   出現頻度:対象の名詞の出現回数に対する格スロットの用例の出現回数の割合}
   }
  \end{table}    
 
   \subsection{格フレームの構築}
   \label{格フレームの構築}
   
   格フレームとは、ある語のとり得る格の制約を記述したものであり語義ごと
   に必要となる。このため、同一の語義に対する格スロットはひとつの格フレー
   ムの異なる格スロットとして扱い、異なる語義に対応する格スロットは別々
   の格フレームとして扱う。ここで問題となるのは、どの格スロットがどの語
   義に対応していると考えるかである。定義文に関連付けられた格スロットは
   語義との対応関係が明らかなので、意味属性パターンを用いた解析で得られ
   た格スロットと\texttt{<その他>}としてまとめられた格スロットをどう扱う
   かが問題となる。

   まず、意味属性パターンを用いた解析で\texttt{<必須格>}としてまとめられ
   た格スロットについて考える。このような格スロットは国語辞典適切な定義
   文がないために構築されたものと考えられることから、定義文に関連付けら
   れた格スロットとは異なる語義に対応していると考える。また、\texttt{<そ
   の他>}としてまとめられた格スロットについても、\texttt{<その他>}として
   まとめられた格スロットが残っている場合はその出現頻度は10\%以上と高頻
   度であることから(表\ref{thre}参照)、\texttt{<必須格>}としてまとめられ
   た格スロットと同様に、定義文に関連付けられた格スロットとは異なる語義
   に対応していると考える
   \footnote{もし、定義文に関連付けられた格スロットと同じ語義に対応して
   いるとすると、これだけの高頻度で出現する極めて必須的な要素が定義文に
   記述されていないことになるが、このような可能性は低いと考えられる。}。

   一方、意味属性パターンを用いた解析結果によって
   \texttt{<所有>、<場所>、<所属>、<主体>}としてまとめられた格スロットに
   対しては、これらの格スロットが残るのはこれらの要素が国語辞典の定義文
   中にあまり記述されない要素であるためと捉え、他の必須要素とも同時に必
   要とされる要素、すなわち、他の格スロットと同じ語義に関連付けられる格
   スロットであると考える。

   このような考えに基づく格フレームの構築アルゴリズムを以下に示す。
  \begin{itemize}
   \item 定義文に関連付けられた格スロットがある場合は定義文ごとに1つの格
	 フレームを構築する。
   \item 意味属性パターンを用いた解析によって\texttt{<必須格>としてまと
	 められた格スロットおよび<その他>}としてまとめられた格スロットが
	 統合されず残っている場合は、それぞれに対して新たな格フレームを
	 構築する。
  \item  意味属性パターンを用いた解析結果によって\texttt{<所有>や<場所>、
	 <所属>、<主体>}としてまとめられた格スロットが残っている場合は、
	 既に存在している格フレームに付け加える。ただし他の格スロットが
	 存在しない場合は新しい格フレームを作成する。
  \end{itemize}
  
  \begin{table}
   \small{
   \caption{自動構築した名詞格フレームの例}
   \label{caseframe}
   \begin{center}
    \begin{tabular}{lcl}
     \hline
     & 格スロット & 用例\\\hline\hline
     ひさし(1) & 
     \multicolumn{2}{l}{\begin{tabular}{l}
			 \footnotesize{(家の出入り口や
			 窓などの上にさし出た小さな屋根。)}
			\end{tabular}}\\
     & 〔家・窓〕 & 駐車場3,店舗3,玄関2,屋根2,窓2,…\\\hline
     ひさし(2) & 
     \multicolumn{2}{l}{\begin{tabular}{l}
			 \footnotesize{(ぼうしの前についているつば。)}
			\end{tabular}}\\
     & 〔ぼうし〕 & 帽子8,兜1,フード1,よけ1,…\\\hline\hline
     表情(1) & 
     \multicolumn{2}{l}{\begin{tabular}{l}
			 \footnotesize{(自分の気持ち
			 を顔や身ぶりにあらわすこと。)}
			\end{tabular}}\\
     & 〔自分〕 & 人々137,人80,市民61,相手49,首相45,…\\
     & 〔気持ち〕 & 安ど599,余裕397,困惑247,苦渋245,…
     \\\hline\hline
     引き出し(1) & 
     \multicolumn{2}{l}{\begin{tabular}{l}
			 \footnotesize{(机・たんすなどにある、
			 引いて出し入れができる箱。)}
			\end{tabular}}\\
     & 〔机・たんす〕& 机216,たんす36,タンス33,鏡台6,デスク4…\\\hline
     引き出し(2) & \texttt{<その他>} & 預金250,資金31,配当金21,貯金13,
     資産13,…\\\hline\hline
     コーチ(1) & 
     \multicolumn{2}{l}{\begin{tabular}{l}
			 \footnotesize{(スポーツなどで
			 そのやり方などを教えること。)}
			\end{tabular}}\\
     & 〔スポーツ〕 & 野球12,サッカー9,水泳6,スケート6,体操4,…\\
     & \texttt{<所属>} & チーム97,部36,クラブ10,母校10,…\\
     \hline\hline
     ショップ(1) & 
     \multicolumn{2}{l}{\begin{tabular}{l}
			 \footnotesize{(商品をならべて売る所。商店。みせ
			 や。)}
			\end{tabular}} \\
     & 〔商品〕 & 商品4,品3,雑貨3,工芸品2,衣料2,…\\
     \hline\hline
    チケット(1) & 
     \multicolumn{2}{l}{\begin{tabular}{l}
			 \footnotesize{(乗り物や劇場などで、料金をはらっ
			 たしるしにくれる小さい紙のふだ。券。)}
			\end{tabular}}\\
     &〔乗り物・劇場・その他〕\footnotemark & 戦54,コンサート45,公演44,
     試合33,…\\
     \hline\hline
     値段(1) & 
     \multicolumn{2}{l}{\begin{tabular}{l}
			 \footnotesize{(品物を売り買いするときの金額。)}
			\end{tabular}}\\
     & 〔品物〕 &  土地112,モノ103,商品82,米55,野菜51,…\\\hline\hline
     株式(1) & 
     \multicolumn{2}{l}{\begin{tabular}{l}
			 \footnotesize{(株式会社の資本構成単位。)}
			\end{tabular}}\\
     & 〔会社〕 & 企業1078,会社628,銀行169,子会社142,…
     \\\hline\hline
     需要(1) & 
     \multicolumn{2}{l}{\begin{tabular}{l}
			 \footnotesize{(商品を買いたいと思って求めるこ
			 と。)}
			\end{tabular}}\\
     & 〔商品〕 & 製品679,機器329,装置244,機械234,住宅198,…\\
     & \texttt{<所有>} & 企業413,アジア203,客186,顧客185,市場156,…
     \\\hline\hline
     盟友(1) & \texttt{<必須格(対人)>} & 首相9,大統領8,大佐2,父2,将軍1,…
     \\\hline\hline 
     レバー(1) & 
     \multicolumn{2}{l}{\begin{tabular}{l}
			 \footnotesize{(動物のかんぞう。)}
			\end{tabular}}\\
     & 〔動物〕 & 牛4,鶏4,豚3\\\hline
     レバー(2) & 
     \multicolumn{2}{l}{\begin{tabular}{l}
			 \footnotesize{(機械を運転するために、手でにぎる棒。)}
			\end{tabular}}\\
     & 〔機械〕 & トランク4,コック3,エンジン2,機2,本体2,…\\\hline\hline
     終着駅(1) & \texttt{<場所>} & 線10,鉄道9,新幹線4,列車4,問題2,…
     \\\hline\hline 
     遅れ(1) & \texttt{<その他>} & 回復1720,対応774,整備630,改革459,調整
     408,…\\\hline\hline
     著者(1) & 
     \multicolumn{2}{l}{\begin{tabular}{l}
			 \footnotesize{(本を書き著した人。著作者。)}
			\end{tabular}}\\
     & 〔本〕 & 本74,シリーズ5,書5,教科書4,論文4,…\\\hline\hline
     季節(1) & 
     \multicolumn{2}{l}{\begin{tabular}{l}
			 \footnotesize{(あることが、よくおこなわれる時期。)}
			\end{tabular}}\\
     & 〔こと〕 & 政治158,桜87,人事61,選挙42,異動36,花粉症35,…\\\hline\hline
     相談(1) & 
     \multicolumn{2}{l}{\begin{tabular}{l}
			 \footnotesize{(物ごとを決めるために話し合うこと。)}
			\end{tabular}}\\
     & 〔物ごと〕 & ローン130,トラブル100,悩み87,問題81,融資:79,…
     \\\hline\hline
     孫(1) & \texttt{<必須格(親族)>} & 私20,首相18,自分18,天皇12,大統領
     9,会長7,…\\\hline
    \end{tabular}
   \end{center}
   }
  \end{table} 

  この方法で自動構築された名詞格フレームの例を表\ref{caseframe}に示す。
  「ひさし」の場合、〔家・窓〕と〔ぼうし〕の2つの格スロットは異なる定義
  文に関連付けられた格スロットがあるので各定義文ごとに異なる格フレーム
  が構築される。一方「表情」の場合は、同一の定義文に関連付けられた複数
  の格スロットであるのでこれらの格スロットは同一の格フレームの別の格ス
  ロットとして扱われる。「引き出し」の場合、定義文に関連付けられた格ス
  ロットの他に、\texttt{<その他>}としてまとめられた格スロットが存在して
  いるので、これらは異なる語義に対する格スロットであると判断され2つの格
  フレームが構築される。「コーチ」の場合は定義文に関連付けられた格スロッ
  トの他に存在するのが\texttt{<所属>}という格スロットであるので多義性は
  ないと判断され1つの格フレームにまとめられる。
  
 \section{名詞句の関係解析}
 
 自動構築した名詞格フレーム辞書の有用性を調べるために、構築した名詞格フ
 レーム辞書に基づく名詞句の関係解析システムを構築した。名詞句の関係解析
 とは解析対象の名詞が間接的に照応している先行詞を特定する処理である。本
 稿ではあくまで名詞格フレームの有用性を調べるのが目的であるので、構文解
 析の結果得られる係り受けの情報以外の情報は使用していない。

 解析は入力文の前から順に用言の省略解析と並行して行う。まず、JUMAN、KNP
 を用いて入力文を形態素・構文解析する。続いて、入力文中に含まれる名詞に
 ついて格フレームが存在しているかを調べ、格フレームが存在している名詞に
 対しては格スロットの用例と先行詞候補との類似度をそれぞれ計算する。閾値
 α(現在は0.95)以上となる組合わせがあった場合は、先行詞候補を格スロット
 に対応付ける。格スロットが複数ある場合は、最も類似している格スロットに
 対応付け、残った格スロットに対して、さらに先行詞候補を調べていく。ここ
 で、先行詞候補とは対象の文および2文前までに含まれる名詞のことである。先
 行詞候補は解析対象の名詞に直接係っているものから調べられていき、閾値を
 越える組合わせが見つからない場合は、同一文中、一文前といった順で徐々に
 解析対象の名詞との構造的距離の遠いものも調べていく。
  
  \begin{figure}[t]
   \begin{center}
    \includegraphics[width=60mm]{relation.eps}    
    \vspace{1ex}
    \small{
    \begin{tabular}
     {c|c|l||c}
     \hline
     & 格スロット & 用例 & 解析結果\\\hline\hline
     ショップ & 〔商品〕 & 商品4,品3,雑貨3,工芸品2,… & 金券 \\\hline
     チケット & 〔乗り物・劇場・その他〕\footnotemark[3]
     & 戦54,コンサート45,公演44,… & なし\\\hline
     値段 & 〔品物〕 &  土地112,商品82,チケット23,… 
     & チケット\\\hline 
    \end{tabular}
    }
   \end{center}
   \footnotesize{
   \hspace{11em}【ショップ】商品をならべて売る所。\\
   \hspace{11em}【チケット】乗り物や劇場などで、
   料金をはらった印にくれる紙の札。\\
   \hspace{11em}【値段】品物を売り買いするときの金額。\\
   }
   \vspace{-2ex}
   \caption{関係解析}
   \label{analysis}
  \end{figure}
  
  \footnotetext{格スロット〔乗り物・劇場・その他〕とは、〔乗り物・劇場〕
  と\texttt{<その他>}が統合された格スロット。}

  例として、次のような文の解析を考える。
  \vspace{1ex}\\
   \hspace{1em}(\theexample)
   \stepcounter{example}
   金券ショップではチケットが何倍もの値段で売られていた。\vspace{1ex}\\
   形態素・構文解析に続いて「金券」の解析が行われ、「金券」に対する格フ
   レームがないことから「金券」の必須要素はないと判断される。「ショップ」
   については、直接係っている「金券」が閾値αを満足するので格スロット
   〔商品〕に「金券」が対応付けられる。
   
   次に、「チケット」の解析が行われる。直接係っている名詞がないことから、
   先行詞の侯補となる「値段」、「ショップ」、「金券」に対して格スロット
   に含まれる用例との類似度が計算される。しかし、閾値αを満足するものが
   ないため必須要素はないと判断される
   \footnote{必須要素なしと判断されるのは、あくまで入力文中に必須要素が
   含まれていないためである。本来、チケットとは「\texttt{<試合・公演>}の
   チケット」であり、この文の前にコンサートなどの単語が出現している場合
   は、その単語が必須要素であると判断される。}。
   
   最後に、「値段」の解析が行われる。「値段」には「何倍」が直接係ってい
   るが閾値αを満足しないことから他の要素から先行詞が探される。この場合、
   まず「チケット」と格スロット〔品物〕の用例との類似度が計算され。用例
   中に「チケット」があることから閾値αを満足するので格スロット〔品物〕
   にチケットが対応付けられる(図\ref{analysis}参照)。

 \section{実験と考察}
 
  \subsection{自動構築された辞書の評価}
  
  毎日新聞12年分および日経新聞13年分の約2,500万文を用いて名詞格フレーム
  辞書の自動構築を行った。約1,000万個の「AのB」から、約17,000語の名詞
  について格フレームが構築された。1語あたりの格フレーム数の平均は1.06個、
  1つの格フレームに含まれる格スロット数の平均は1.09個であった。

  まず、自動構築された格フレーム辞書の評価を行うため、コーパス中に1万回
  以上出現した普通名詞から無作為に抽出した100個の名詞について正しいと考
  えられる格フレームを人手で与えた。65個の名詞に対して68個の格フレームを
  与え、35個の名詞については必須要素がないと判断した。

  続いて、人手で与えた格フレームと自動構築された格フレーム辞書を比較する
  ことにより自動構築された辞書の評価を行った。その結果を表
  \ref{evaluation}に示す。必須格スロットが完全に一致しているものを正しい
  格フレームとして評価している。70個の格フレームが自動構築され、そのうち
  58個の格フレームが正しいと判断できた。
 
 \begin{table}[t]
  \small{
  \caption{格フレームの精度}
  \label{evaluation}
  \begin{center}
   \begin{tabular}{c|c|c} \hline
    適合率 & 再現率 & F\\\hline
    58/70 (0.829) & 58/68 (0.853) & 0.841 \\\hline
   \end{tabular}
  \end{center}
  }
 \end{table}
 
 ここで、必要な格スロットが構築されないものの中には、格スロットの用例を
 収集する際に「AのB」の形をした用例しか集めていないことが原因と考えら
 れるものが存在した。例えば「メーカー」の場合、\texttt{<製品>}にあたるも
 のが必須的要素であると考えられる。
 \vspace{1.5ex}\\
 \hspace{1em}(\theexample)
 \stepcounter{example}
 パソコン\underline{メーカー}各社はそろってサーバの開発に力を入れている。
 \vspace{1.5ex}\\
 ところが、この例のように「\texttt{<製品>メーカー}」の形の用例は多いもの
 の、「\texttt{<製品>}のメーカー」の用例は少ないため、\texttt{<製品>}に
 対応する格スロットは構築されなかった。実際にメーカーに対して収集された
 \texttt{<製品>のメーカー}」、および「\texttt{<製品>}メーカー」の用例の
 出現回数は表\ref{maker}のようになった。この表からも判るように、この問題
 は名詞格フレーム辞書を構築する際に集める用例として複合名詞「AB」など
 「AのB」以外の用例も用いることにより改善できると考えられる。
 
\begin{table}[t]
   \small{
   \caption{「\texttt{<製品>のメーカー}」、「\texttt{<製品>}メーカー」
 の出現回数の比較}
 \label{maker}
 \begin{center}
  \begin{tabularx}{40em}{c|X|c}\hline
   & 主な用例 & 出現回数\\\hline
    \texttt{<製品>のメーカー} & 製品18,機器16,装置13,商品10,器9,車8,機械
   7,剤7,自動車7,容器7,用品7,ウエア5,ビール5,ワープロ4,…
   & 534\\\hline 
   \texttt{<製品>メーカー} & 自動車2652,電機944,機器861,部品860,鉄鋼
   587,食品575,半導体488, 機械378,化学359,品356,パソコン336,… 
   & 18510\\\hline
   \end{tabularx}
 \end{center}
 }
  \end{table}

  誤った格フレームが構築されているものの中には、定義文中の適切でない語に
  関連付けられ表\ref{thre}の定義文に関連付けられた格スロット用の緩い閾値
  が適用されてしまった結果、必須要素とみなされたものが存在した。例えば表
  \ref{shimaclub}に示すように、自動構築した格フレーム辞書では「島」の必
  須要素として格スロット〔まわり〕が構築されているが、これは以下に示すよ
  うな「島」の定義文中の重要でない要素、「まわり」に用例を関連付けてしまっ
  たために構築されたものである。
  \vspace{1.5ex}\\
  \hspace{1em}【島】まわりを水で囲まれた、小さい陸地。
  \vspace{1.5ex}\\
  この問題の解決策としては、定義文中の名詞であっても出現位置によっては用
  例との関連付けを行わないことが考えられるが、定義文中のどの位置と重要性
  の関係を一般化するのは容易ではない。

  他にも、定義文中の適切でない語に関連付けられてしまうものとして「クラブ」
  があった。
  \vspace{1.5ex}\\
  \hspace{1em}【クラブ】◯\hspace{-0.77em}1\hspace{0.6em}同じ目的を
  持った人たちの集まり。\\
  \hspace{5.8em}◯\hspace{-0.77em}2\hspace{0.6em}ゴルフの球を打つ道具。
  \vspace{1.5ex}\\
 「サッカーのクラブ」という名詞句中の「クラブ」は、本来は「同じ目的を持っ
  た人たちの集まり」という意味であり、この定義文に関連付けられなければな
  らないのに、「ゴルフ」という名詞が含まれている「ゴルフの球を打つ道具」
  という定義文に関連付けられてしまっている(表\ref{shimaclub})。このこと
  は格フレームを用いて「クラブ」の必須要素を解析する際にはあまり問題とな
  らないが、格フレームを用いて多義性の解消を行う際には問題となる。このよ
  うな定義文中の適切でない語への関連付け防止するためには、 定義文中での
  働きが例示である場合以外は定義文中に出現する語そのもの、またはその下位
  語としか関連付けないようにすることなどが考えられる。しかし単純に適用し
  ただけでは、正しい関連付けが行われなくなるものが出るなどの問題が生じて
  しまう。

  \begin{table}
    \small{
   \caption{自動構築された「島」、「クラブ」の格フレーム}
   \label{shimaclub}
   \begin{center}
    \begin{tabular}{cccl}
     \hline
     & 格スロット & 用例 & 割合\\\hline\hline
     島 & 
     \multicolumn{2}{l}{\begin{tabular}{l}
			 \footnotesize{(まわりを水で囲まれた、小さい陸地。)}
			\end{tabular}}\\
     & 〔まわり〕54 & 周辺16,南端13,端8,先4,近く2,周囲2,付近2,… & 0.50\% 
     \\\hline
     クラブ &
     \multicolumn{2}{l}{\begin{tabular}{l}
			 \footnotesize{(ゴルフの球を打つ道具。)}
			 \end{tabular}}\\
     & 〔ゴルフ〕48 & ゴルフ9,趣味4,運動2,歌舞伎2,野球2,サッカー2,… & 0.59\%\\\hline
    \end{tabular}
   \end{center}
   }
  \end{table}
	  
  \subsection{関係解析実験}
  
  新聞10記事を用いて名詞の関係解析の実験を行い、解析システムの出力結果と
  人手で付けられた正解付きコーパス\cite{Kawahara2002ec}を比較することに
  より解析システムの評価を行った。

  実験に用いたコーパスは無作為に選ばれた10記事で、217個の名詞が含まれて
  いた。そのうち正解付きコーパスにおいて必須要素があると判断されている名
  詞は106個で、2つの先行詞を持つ名詞が2つあったため先行詞の総数は108個で
  あった。このうち解析対象の名詞に直接係っている先行詞が59個、解析対象の
  名詞と直接係り受け関係にない先行詞が49個であった。また、108個の先行詞
  のうち、自動構築した格フレームに対応した格スロットが存在したのは91個あ
  り、したがって、自動構築した格フレーム辞書に基づく関係解析システムの再
  現率の上限は84.3\%(91/108)である。

 10記事に対する関係解析の結果を表\ref{result}に示す。表中の「係り受けあ
 り」とは先行詞が解析対象の名詞に直接係っているもの、「係り受けなし」と
 は先行詞と解析対象の名詞が係り受け関係にないもののことである。係り受け
 がある場合の解析の方が容易であるにもかかわらず、係り受けありの場合の再
 現率があまり良くないのは、解析に必要な格フレームが構築されていないのが
 主な原因である。直接係り受け関係にある先行詞の多くは解析対象の名詞と複
 合名詞を形成しており、また、複合名詞の形を多くとる名詞は「メーカー」の
 ように用例が収集されにくいため、このような名詞の場合、対応する格フレー
 ムは構築されていないことが多い。直接係り受けが無い場合の主な誤り原因を
 以下に示す。
  
  \begin{table}[t]
   \small{
   \caption{関係解析の精度}
   \label{result}
   \begin{center}
    \begin{tabular}{c|c|c|c} \hline
     係り受け & 適合率 & 再現率 & F\\\hline
     係り受けあり & 40/46 (0.870) & 40/59 (0.678) & 0.762\\ 
     係り受けなし & 31/61 (0.508) & 31/49 (0.633) & 0.564\\\hline
     合計 & 71/107 (0.664) & 71/108 (0.657) & 0.660\\\hline 
    \end{tabular}
   \end{center}
   }
  \end{table}

    \paragraph{必須要素の文脈依存性}\ \\
    解析システムの誤った出力には、文脈によって必須となる情報が異なること
    に起因するものがある。例えば「株式」という名詞に対しては、どの「会社」
    のものかという情報が必須となる場合があるため、「株式」という名詞には
    〔会社〕を格スロットとする格フレームが構築される
    (表\ref{caseframe}参照)。
    \vspace{1.5ex}\\
    \hspace{1em}(\theexample)
    \stepcounter{example}
    \underline{株式}相場の押し上げ要因となる。\vspace{1.5ex}\\
    しかし「株式」の使われ方にはこの例のように、「ある会社の株式」ではな
    く一般的な「株式」を指し、格フレームに記されている要素が必要とならな
    いものがある。構築した関係解析システムでは、ある名詞が一般的な意味で
    使われているかなどを考慮しないため、このような場合でも〔会社〕にあた
    るものを探してしまう。この問題は、解析対象の名詞が一般的なものを意味
    しているのかどうかの判断を行うことによって改善できると考えられるが、
    実際に一般性の判断を行うのは容易でない。
    
    \paragraph{同じ意味属性をもつ名詞の存在}\ \\
    必須要素の取り違えには、同じ意味属性を持つ名詞の存在に起因するものが
    ある。
    \begin{table}[h]
     \begin{center}
      \begin{tabularx}{40em}{p{1.2em}X}
       & \hspace{-2em}(\theexample)
       \stepcounter{example}
       ブッシュ\textgt{米}政権は外交・安保政策で「単独主義」を意識的に控
       え、国際社会との協調路線を打ち出した。たが、\textgt{ロシア}との弾
       道弾迎撃ミサイル制限条約からの一方的脱退宣言や、地下核実験の将来
       の再開を示唆する姿勢にはブッシュ\underline{大統領}らが...
      \end{tabularx}
     \end{center}
    \end{table}\\
    このような文章があると、2文目に出現する「大統領」という名詞には
    \texttt{<国>}という要素が必須であることが自動構築された格フレーム辞
    書からわかるので、システムは\texttt{<国>}にあたるものを探す。ところ
    が「米の大統領」と解析すべきであるのに、もっとも近くにある
    \texttt{<国>}は「ロシア」であるため、「ロシアの大統領」であると解析
    してしまう。この問題を解決する方法としては、1文目に出現する「ブッシュ
    米政権」という情報を用いることや、社会・文化に対する知識をシステムに
    持たせることなどが考えられる。

    現在は、閾値を満足する先行詞候補のうち、構文的に最も近くに存在するも
    のを先行詞とみなしているが、一定の範囲の中で最も評価の高い先行詞候補
    を先行詞とする方法も考えられ、どのような方法が良いかは今後調査する予
    定である。
          
 \section{おわりに}
   
 本稿では、国語辞典を用いた名詞句「AのB」の意味解析を行うことにより、
 コーパス中の名詞句「AのB」の用例から名詞格フレーム辞書を自動構築する
 手法を提案した。
 
 名詞格フレーム辞書を自動構築した結果、約17,000個の名詞について格フレー
 ムが構築された。人手で正しいと思われる格フレーム辞書を作成し比較を
 行ったところ適合率は82.9\%、再現率は85.3\%であり、自動構築した名詞格
 フレーム辞書を用いた関係解析の結果からもある程度実用的な名詞格フレーム辞
 書を構築できていることが確認できた。

 今後、名詞格フレーム辞書に含める必須要素を集める際に複合名詞ABなどの
 用例も集めることによりカバレージのさらに大きい名詞格フレーム辞書を作成
 していき、また、自動構築した格フレーム辞書を用いた名詞句の関係解析シス
 テムに関しても、より高精度な解析を行えるシステムを作成していく予定であ
 る。

\bibliographystyle{jnlpbbl}
\begin{thebibliography}{}

\bibitem[\protect\BCAY{Briscoe \BBA\ Carroll}{Briscoe \BBA\
  Carroll}{1997}]{Briscoe1997}
Briscoe, T.\BBACOMMA\  \BBA\ Carroll, J. \BBOP 1997\BBCP.
\newblock \BBOQ Automatic Extraction of Subcategorization from Corpora\BBCQ\
\newblock In {\Bem Proceedings of the 5th Conference on Applied Natural
  Language Processing}, \BPGS\ 356--363.

\bibitem[\protect\BCAY{Hahn\JBA Strube \BBA\ Markert}{Hahn
  et~al.}{1996}]{Hahn1996}
Hahn, U.\JBA Strube, M.\JBA  \BBA\ Markert, K. \BBOP 1996\BBCP.
\newblock \BBOQ Bridging Textual Ellipses\BBCQ\
\newblock In {\Bem Proceedings of the 16th International Conference on
  Computational Linguistics}, \BPGS\ 496--501.

\bibitem[\protect\BCAY{Kawahara \BBA\ Kurohashi}{Kawahara \BBA\
  Kurohashi}{2002}]{Kawahara2002e}
Kawahara, D.\BBACOMMA\  \BBA\ Kurohashi, S. \BBOP 2002\BBCP.
\newblock \BBOQ Fertilization of Case Frame Dictionary for Robust {J}apanese
  Case Analysis\BBCQ\
\newblock In {\Bem Proceedings of the 19th International Conference on
  Computational Linguistics}, \BPGS\ 425--431.

\bibitem[\protect\BCAY{Kawahara\JBA Kurohashi \BBA\ Hasida}{Kawahara
  et~al.}{2002}]{Kawahara2002ec}
Kawahara, D.\JBA Kurohashi, S.\JBA  \BBA\ Hasida, K. \BBOP 2002\BBCP.
\newblock \BBOQ Construction of a {J}apanese Relevance-tagged Corpus\BBCQ\
\newblock In {\Bem Proceedings of the 3rd International Conference on Language
  Resources and Evaluation}, \BPGS\ 2008--2013.

\bibitem[\protect\BCAY{Kurohashi \BBA\ Sakai}{Kurohashi \BBA\
  Sakai}{1999}]{Kurohashi1999}
Kurohashi, S.\BBACOMMA\  \BBA\ Sakai, Y. \BBOP 1999\BBCP.
\newblock \BBOQ Semantic Analysis of {J}apanese Noun Phrases: A New Approach to
  Dictionary-Based Understanding\BBCQ\
\newblock In {\Bem Proceedings of the 37th Annual Meeting of the Association
  for Computational Linguistics}, \BPGS\ 481--488.

\bibitem[\protect\BCAY{Murata\JBA Isahara \BBA\ Nagao}{Murata
  et~al.}{1999}]{Murata1999}
Murata, M.\JBA Isahara, H.\JBA  \BBA\ Nagao, M. \BBOP 1999\BBCP.
\newblock \BBOQ Pronoun Resolution in {J}apanese Sentences Using Surface
  Expressions and Examples\BBCQ\
\newblock In {\Bem Proceedings of the ACL'99 Workshop on Coreference and Its
  Applications}, \BPGS\ 39--46.

\bibitem[\protect\BCAY{Poesio\JBA Ishikawa\JBA im~Walde \BBA\ Vieira}{Poesio
  et~al.}{2002}]{Poesio2002}
Poesio, M.\JBA Ishikawa, T.\JBA im~Walde, S.~S.\JBA  \BBA\ Vieira, R. \BBOP
  2002\BBCP.
\newblock \BBOQ Acquiring Lexical Knowledge for Anaphora Resolution\BBCQ\
\newblock In {\Bem Proceedings of the 3rd International Conference on Language
  Resources and Evaluation}, \BPGS\ 1220--1224.

\bibitem[\protect\BCAY{Strube \BBA\ Hahn}{Strube \BBA\ Hahn}{1999}]{Strube1999}
Strube, M.\BBACOMMA\  \BBA\ Hahn, U. \BBOP 1999\BBCP.
\newblock \BBOQ Functional Centering {--} Grounding Referential Coherence in
  Information Structure\BBCQ\
\newblock {\Bem Computational Linguistics}, {\Bbf 25}  (3), 309--344.

\bibitem[\protect\BCAY{Vieira \BBA\ Poesio}{Vieira \BBA\
  Poesio}{2000}]{Vieira2000}
Vieira, R.\BBACOMMA\  \BBA\ Poesio, M. \BBOP 2000\BBCP.
\newblock \BBOQ An Empirically Based System for Processing Definite
  Descriptions\BBCQ\
\newblock {\Bem Computational Linguistics}, {\Bbf 26}  (4), 539--592.

\bibitem[\protect\BCAY{池原\JBA 宮崎\JBA 白井\JBA 横尾\JBA 中岩\JBA 小倉\JBA
  大山\JBA 林}{池原\Jetal }{1997}]{GoiTaikei}
池原\JBA 宮崎\JBA 白井\JBA 横尾\JBA 中岩\JBA 小倉\JBA 大山\JBA  林\JEDS\ \BBOP
  1997\BBCP.
\newblock \Jem{日本語語彙大系}.
\newblock 岩波書店.

\bibitem[\protect\BCAY{田近洵一}{田近洵一}{1997}]{Reikai}
田近洵一\JED\ \BBOP 1997\BBCP.
\newblock \Jem{例解小学国語辞典}.
\newblock 三省堂.

\end{thebibliography}


\section*{\large{略歴}}
\begin{description}
 \item[笹野 遼平:] 2004年東京大学工学部電子情報工学科卒業。現在、東京大
	    学大学院情報理工学系研究科修士課程在学中。省略解析、照応解析
	    の研究に従事。
 \item[河原 大輔:] 1997年京都大学工学部電気工学第二学科卒業。1999年同大
	    学院修士課程修了。2002年同大学院博士課程単位取得認定退学。現
	    在、東京大学大学院情報理工学系研究科学術研究支援員。構文解析、
	    省略解析の研究に従事。
 \item[黒橋 禎夫:] 1989年京都大学工学部電気工学第二学科卒業。1994年同大
	    学院博士課程修了。京都大学工学部助手、京都大学情報学研究科講
	    師を経て、2001年東京大学大学院情報理工学系研究科助教授、現在
	    に至る。自然言語処理、知識情報処理の研究に従事。
\end{description}
 
\end{document}
