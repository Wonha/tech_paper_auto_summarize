    \documentclass[japanese]{jnlp_JS2.0}
\usepackage{jnlpbbl_1.3}
\usepackage[dvips]{graphicx}
\usepackage{amsmath}
\usepackage{hangcaption_jnlp}
\usepackage{udline}
\setulminsep{1.2ex}{0.2ex}
\let\underline

\usepackage{lingmacros}
\usepackage{pifont}
\newcommand{\word}[1]{}

\def\SP{}
\newcommand{\kuuhaku}{}
    \newcommand{\key}[1]{}
\newcommand{\rei}[1]{}
\newcommand{\lf}{}
\newcommand{\eos}{}
\newcommand{\tyuu}[1]{}
\newcommand{\srcbox}[2]{}




\Volume{18}
\Number{2}
\Month{June}
\Year{2011}

\received{2011}{1}{5}
\accepted{2011}{2}{23}

\setcounter{page}{175}

\jtitle{構文・照応・評価情報つきブログコーパスの構築}
\jauthor{橋本  力\affiref{NICT}\affiref{KU} \and 黒橋 禎夫\affiref{KU}\affiref{NICT} \and 河原 大輔\affiref{KU} \and 新里 圭司\affiref{KU} \and 永田 昌明\affiref{NTT}}
\jabstract{
近年,ブログを対象とした情報アクセス・情報分析技術が盛んに研究されてい
る.
我々は,この種の研究の基礎データの提供を目的とし,249 記事,4,186 文
からなる,解析済みブログコーパスを構築した.
主な特長は次の 4 点である.
i) 文境界のアノテーション.
ii) 京大コーパス互換の,形態素,係り受け,格・省略・照応,固有表現のアノテーション.
iii) 評価表現のアノテーション.
iv) アノテーションを可視化した HTML ファイルの提供.
記事は,大学生 81 名に「京都観光」「携帯電話」「スポーツ」「グルメ」のい
ずれかのテーマで執筆してもらうことで収集した.
解析済みブログコーパスを構築する際,不明瞭な文境界,括弧表現,誤字,方言,顔文字等,多様な形態素への対応が課題になる.
本稿では,本コーパスの全容とともに,いかに上記の課題に対応しつつコーパ
スを構築したかについて述べる.
}
\jkeywords{ブログ,解析済みコーパス}

\etitle{Construction of a Blog Corpus with Syntactic, Anaphoric, and Sentiment Annotations}
\eauthor{Chikara Hashimoto\affiref{NICT}\affiref{KU} \and Sadao Kurohashi\affiref{KU}\affiref{NICT} \and Daisuke Kawahara\affiref{KU} \and Keiji Shinzato\affiref{KU} \and Masaaki Nagata\affiref{NTT}} 
\eabstract{
There has been a growing interest in the technologies of information
 access and analysis targetting blog articles recently. 
In order to provide the research community with the basic data, 
we constructed a blog corpus that consists of 249 articles (4,186
 sentences) and has the following features: 
i) Annotated with sentence boundaries. 
ii) Annotated with grammatical information about
 morphology, dependency, case, anaphora, and named entities, in a way
 consistent with Kyoto University Text Corpus. 
iii) Annotated with sentiment information. 
iv) Provided with HTML files that visualize all the annotations
 above. 
We asked 81 university students to write blog articles about either
the sightseeing of Kyoto, cellphones, sports, or gourmet. 
In constructing the annotated blog corpus, we faced problems concerning 
sentence boundaries, parentheses, 
errata, dialect, a variety of smiley, and other morphological
 variations. 
In this paper, we describe the specification of the corpus and how we
 dealt with the above problems.
}
\ekeywords{Blog, Annotated Corpus}

\headauthor{橋本,黒橋,河原,新里,永田}
\headtitle{構文・照応・評価情報つきブログコーパスの構築}

\affilabel{NICT}{情報通信研究機構}{National Institute of Information and Communications Technology}
\affilabel{KU}{京都大学}{Kyoto University(第 4 著者の現在の所属は 楽天技術研究所)}
\affilabel{NTT}{NTTコミュニケーション科学基礎研究所}{NTT Communication Science Laboratories}



\begin{document}
\maketitle


\section{はじめに}

近年の自然言語処理技術は,新聞記事等のフォーマルな文章だけでなく,
ブログ等のインフォーマルな文章をもその射程に入れつつある
\cite{ICWSM:2008,ICWSM:2009}.
この背景の一つには,世論や消費者のニーズ等をブログを含めたWeb文書から取
り出そうとする,自然言語処理技術を援用した情報アクセス・情報分析研究の盛
り上がりがある\cite{Sriphaew:Takamura:Okumura:2009,Akamine:Kawahara:Kato:Nakagawa:Inui:Kurohashi:Kidawara:2009,Murakami:Masuda:Matsuyoshi:Nichols:Inui:Matsumoto:2009}.近年の自然言語処理技術は,機械学習等のコーパス
ベースの手法の発展により高い精度が得られるようになったが,
これらの手法の成功の鍵は,処理対象の分野/ジャンルの解析済みコーパスの充実
にある
\cite{McClosky:Charniak:Johnson:2006}.
ブログに自然言語処理技術を高精度に適用するには,同様に,
解析済みのブログコーパスの整備/充実が必須である.

我々は,ブログを対象とした自然言語処理技術の高精度化に寄与することを目的
とし,249記事,
4,186文からなる解析済みブログコーパス(以下,KNBコーパス\footnote{
\textbf{K}yoto University and \textbf{N}TT \textbf{B}log コーパス
})を構築し,配布を開始した.

本研究でアノテーションしている言語情報は,多くの自然言語処理タスク
で基盤的な役割を果たしている形態素情報,係り受け情報,格・省略・照応情報,
固有表現情報と,文境界である.
これらのアノテーションの仕様は,コーパスユーザの利便性を重視し,世の中に
広く浸透している京都大学テキストコーパス
\cite{Kawahara:Kurohashi:Hashida:2002j}\footnote{http://nlp.kuee.kyoto-u.ac.jp/nl-resource/corpus.html}(以下,京大コーパス)と極力互換性のあるものにした.
これらのアノテーションに加えて,ブログを対象とした情報アクセス・情報分析
研究にとっての要となる評価表現情報もKNBコーパスのアノテーション対象に含
めた.

ブログ記事は,京都の大学生81名に「京都観光」「携帯電話」「スポーツ」「グルメ」
のいずれかのテーマで執筆してもらうことで収集した.
執筆者らは記事執筆に際し,記事の著作権譲渡に同意しているため,
アノテーションだけでなく本文も併せてKNBコーパスとして無料配布している.

KNBコーパス構築の過程で,我々は次の問題に直面した.

\setlength{\widelabel}{18pt}
\eenumsentence{
 \item 不明瞭な文境界
 \item 構文構造の解析を困難にする文中の括弧表現
 \item 誤字,方言,顔文字等の多様な形態素
}

これらは,校閲等の過程を経た上で世に公開される新聞記事等のフォーマルな文
章とは異なる,ブログ記事,あるいはCGM (Consumer Generated Media) テキス
トの特徴と言える.

KNBコーパス構築の際には,このようなブログ記事特有の現象を可能な限りその
ままの形で残すよう心がけた.
一方で,新聞記事を対象にして作られた京大コーパスとの互換性も重視した.
本稿では,KNBコーパスの全容とともに,京大コーパスとの互換性の保持と,
ブログの言語現象の正確な記述のために我々が採用した方針について詳述する.

なお本稿では,京大コーパスの仕様からの拡張部分に焦点を当てる.
本稿に記述されていない詳細については,京大コーパスに付属のマニュアル
\cite{KUCorpus:syn:2000,KUCorpus:rel:2005}
を参照されたい.

以下,\ref{sec:related-work}節で関連研究について述べた後,
\ref{sec:spec}節でKNBコーパスの全体像を具体例とともに詳述する.
\ref{sec:construction}節で記事収集から構築,配布までの過程を説明し,
\ref{sec:conclusion}節で結論を述べる.



\section{関連研究 \label{sec:related-work}}

近年,コーパスベースの手法の発展とともに,言語やタスクを問わず,自然言語
処理技術の精度は向上してきた.
本稿では,コーパスベースの技術とは,人手で正解が付与されたコーパスから
機械学習に基づき言語処理システムを実現するアプローチと捉える.
例えば
英語の品詞タグ付けでは
Laffertyら\cite{Lafferty:McCallum:Pereira:2001}が,
日本語の形態素解析では
工藤ら\cite{Kudo:Yamamoto:Matsumoto:2004}が
Conditional Random Fieldsを用いてそれぞれのタスクで高精度化を果たした.
また,
言語に依存しない係り受け解析として,
SVMによる機械学習を用いたNivreらの手法
\cite{Nivre:Hall:Nilsson:senEryigit:Marinov:2006}や
Miraによる機械学習を用いたMcDonaldらの手法
\cite{McDonald:Lerman:Pereira:2006}が注目されている.

これら解析技術の高精度化は,人手でアノテーションされた解析済みコーパスに
支えられてきた.
日本語の代表的な解析済みコーパスとして,
京大コーパスと
NAISTテキストコーパス\cite{NAISTCorpus:2007}がある.
前者は,
新聞記事を対象に,4万文に対して形態素・構文情報を,
5,000文に対して格関係,照応・省略関係,共参照の情報を付与したものである.
加えて,
IREX \cite{IREX:2000}
において,京大コーパス中の1万文に対する固有表現アノテーションが配布され
ている.
後者は,
京大コーパスの4万文に対して,
述語と表層格の関係,
事態性名詞と表層格の関係,
名詞間の共参照関係の情報を付与したものである.

これらのコーパスは新聞記事から作られているため,それらによって訓練
された解析器は,新聞記事あるいはそれに文体が近い文章は高精度で解析できる
が,ブログやWWW上の掲示板等の,文体が新聞記事とは大きくかけ離れている文
章の解析は不得手となることが知られている
\cite{McClosky:Charniak:Johnson:2006}.

一方,自然言語処理技術の適用領域は,WWWの爆発的な拡大とともに,
ブログやWWW上の掲示板等の文章へと広がりを見せている.
ブログを対象とした自然言語処理技術の高精度化の鍵は,解析済みブログコーパ
スを充実させることができるかどうかにかかっている.
そして,解析済みブログコーパスを充実させるためには,構築ノウハウの蓄積と
実際に構築した解析済みブログコーパスの流通が欠かせない.
我々はこの役割を担うべく,
KNBコーパスとして,ブログ記事に各種言語情報(文境界,形態素,係り受け,
格・省略・照応,固有表現,評価表現)をアノテーションした.
アノテーションの仕様は,
ブログ特有の現象に対応するため一部KNBコーパス独自の仕様を策定したが,
京大コーパスのものと極力互換性のあるものとした.
KNBコーパスの独自仕様の一部には,
話し言葉を対象とした代表的な解析済みコーパスである
日本語話し言葉コーパス
\cite{日本語話し言葉コーパスの構築法:2006}\footnote{
	http://www.kokken.go.jp/katsudo/seika/corpus/}(以下,CSJ)の仕様と類似したものが含まれている.
\ref{sec:spec}節でKNBコーパスの仕様を述べる際,
話し言葉と類似する現象のアノテーションに関しては,適宜CSJの仕様と
比較する.

ブログを対象とした既存のコーパスとして,
Spinn3r Blog Dataset
\cite{ICWSM:2009}\footnote{http://www.icwsm.org/2009/data/} がある.
これは44,000,000記事からなる大規模なものだが,係り受けや照応等の言語情報
は付与されていない.

KNBコーパスには,京大コーパスにはない評価表現のアノテーションがなされて
いる.
既存の評価表現コーパスとして
NTCIR-6 意見分析パイロットタスクのテストコレクション\footnote{
http://research.nii.ac.jp/ntcir/permission/ntcir-6/perm-ja-OPINION.html} や,
小林ら\cite{Kobayashi:Inui:Matsumoto:2007},
宮崎ら\cite{宮崎:森:2008},
川田ら\cite{評価情報タグ付与基準:2009}
のものがあるが,
本コーパスの評価表現アノテーションは,
評価表現を「当為」「要望」「採否」等の意味的なタイプに分類する
川田らの仕様に基づく.



\section{コーパスの全体像 \label{sec:spec}}

一般に,コーパスを設計する際は少なくとも次の点を考慮する必要がある.

\begin{enumerate}
 \item コーパスの使用目的は何か.
 \item 文章に対してどのような言語情報をアノテーションするか.
 \item 何をコーパスの素材(元となる文章)にするか.
 \item どのような仕様でアノテーションするか.
\end{enumerate}

KNBコーパスは,ブログを対象とした自然言語処理技術のためのデータの提供を
目的としている.
具体的には,
多くの自然言語処理タスクで基盤的な役割を果たしている形態素解析,
係り受け解析,格・省略・照応解析,固有表現抽出と,
ブログを対象とする場合に重要になる文境界の検出,
情報アクセス・情報分析研究にとっての要となる評判分析を対象としている.

従って,KNBコーパスにアノテーションする言語情報は,形態素情報,係り受け情報,
格・省略・照応情報,固有表現情報,文境界位置,評価表現情報となる.

次に,何をKNBコーパスの素材とするかであるが,選択肢としては,
WWW上に既に存在するブログ記事を用いるか,本研究のために新たにブログ記事
を執筆するかの2つが考えられる.
前者の場合,記事を大量に収集できるが,記事の著作権処理が困難になることが
予想される.
後者は逆に,記事の著作権譲渡にあらかじめ同意してもらうことで,著作権に
関する問題はクリアできるが,収集できる記事の量が前者に比べて大幅に少なく
なることが予想される.
結局,我々は,
多数の大学生にアルバイトとして記事を執筆してもらうことで,ある程度の記事
数を確保できる見通しが得られたため,
後者のアプローチを採用した.
記事のテーマは執筆者である大学生に自由に決めてもらうことも可能だが,
我々は,
アノテーション対象の評価表現が文中に含まれやすく,かつ,
大学生にとって比較的身近であると考えられる
「京都観光」\inhibitglue\footnote{執筆者である大学生は皆,京都の大学に在籍している.}「携帯電話」「スポーツ」「グルメ」の4つのテーマをあらかじめ
設定した.
執筆してもらったブログ記事は全て文章のみであり,画像や動画等は含まれていない.

\begin{table}[b]
\caption{テーマごとの記事数,文数,形態素数}
\label{tab:breakdown}
\input{06table01.txt}
\vspace{-1\baselineskip}
\end{table}

アノテーション仕様については,選択肢として,KNBコーパス用に新規
に設計するか,既存のタグ付きコーパスの仕様を流用,拡張するかの2つ
が考えられる.
前者にはコーパスをブログ記事に特化したものにすることができるという利点が
あるが,コーパスユーザにとっては,全く新規の仕様は
広く流通している既存のコーパスの仕様に比べて扱いにくいものと
なることが予想される.
後者の場合,ユーザにとって扱いやすいコーパスとなるが,
既存の仕様ではカバーできないブログの言語現象が存在する恐れがある.
我々は当初,既存の仕様を用いる場合,評価表現に関しては川田らのものを,
それ以外のアノテーションには京大コーパスのものを第一候補として考えていた.
川田らの仕様は,他の評価表現アノテーションの仕様と異なり,評価表現を単に
マークするだけでなく,「当為」「要望」「採否」等の意味的な細分類を付与す
る.
この細分類情報は,従来以上に詳細な情報分析技術を実現するための重要な要素
であると我々は考えている.
一方,京大コーパスの仕様は,京大コーパスだけでなく,近年盛んに研究利用さ
れている日本語WWWコーパスのアノテーションにも使用されており
\cite{新里:橋本:河原:黒橋:2007},解析済みコーパスの仕様の中でもっとも
広く流通しているものということができる.
そこで我々は,川田らの評価表現アノテーションの仕様と京大コーパスの仕様を
ブログ記事に予備的に適用し,記述しきれない言語現象が存在するかを
調査した.
その結果,後述するように,誤変換,脱字,衍字,口語的表現などの形態素に関
する仕様の追加と,係り受けに関しての若干の仕様拡張,文境界と括弧表現に
ついての仕様の策定を行えば,ほぼ対応可能であるという見通しが得られた.
最終的に,川田らの仕様と京大コーパスの仕様をベースに,以降で述べる
いくつかの拡張を加えることでKNBコーパスのアノテーション仕様とすることに
決定した\footnote{WWW上に存在するブログ記事の中には,文章の他,本文と密接に関連した
画像や動画等が掲載されていることがある.
そのような記事の文章の構造や意味的な情報を十分に記述するためには,既存の
解析済みテキストコーパスの仕様では明らかに不十分であり,画像や動画等の
他のメディアの情報を本文と統一的に記述する枠組みが必要である.
上述の通り,KNBコーパスは文章のみを対象としたコーパスであるため
既存のテキストコーパスの仕様を用いることができたが,
今後,
WWW上に存在する多様なブログ記事をありのままの姿で解析済みコーパスに
収録していくには,他メディアの掲載を許し,
その情報を十分に記述できるような仕様を開発する必要がある.}.

完成したKNBコーパスは249記事,4,186文から成る.
テーマごとの記事数,文数,形態素数は表\ref{tab:breakdown}の通りである.
執筆に加わった大学生は計81名である.
京都観光について執筆したのは計72名,携帯電話については計67名,
グルメについては計34名,スポーツは計18名である.



以下では,KNBコーパスへアノテーションされている各種言語情報の仕様
について例とともに説明する.



\subsection{文境界}

新聞記事とは違い,ブログ記事では,例文(\ex{1})のような境界が明確な文だけで
なく,例文(\ex{2})にあるような境界が不明確な文もある
(以下,
\eos{}はアノテーションされた文境界を,\lf{}は元の記事における改行を表す
).

\eenumsentence{
\item 私はプリペイド携帯をずっと使っている.\eos{}
\item ヒマな大学生の人はチケット買って西京極でボクと握手!\lf \ \eos{}
}

\eenumsentence{
\item なぜか清水寺に着きました笑\ \eos{}\footnote{
元の記事では「着きました笑」の後に空白が挿入されている.
}
\item 京都のほうだったような・・・ \ \eos{}
まぁ観光スポット多いですもんね?京都は.\ \eos{}
}


KNBコーパスでは,原則的には,母語話者の直観に基づき一文として最も適切だ
と思われる個所で文を区切った.
それ以外に,(\ex{1})に挙げる個別的な方針を導入した.

\eenumsentence{
 \item {日付だけからなる行も一文とする}
     \begin{itemize}
      \item 2006年10月09日\lf \ \eos{}
     \end{itemize}
 \item {URLだけからなる行も一文とする}
     \begin{itemize}
      \item http://www.shigureden.com/\lf \ \eos{}
     \end{itemize}
 \item {箇条書きの一行一行をそれぞれ一文とする}
     \begin{itemize}
      \item[] ・藤井大丸\lf \ \eos{}
      \item[] ・紀伊國屋書店\lf \ \eos{}
     \end{itemize}
 \item {途中にURLが含まれていても,全体を一文とする}
     \begin{itemize}
      \item http://url/とでも入力すると,\lf \ 
	    http//url/ \lf \ と出ます. \eos{}
     \end{itemize}
 \item {途中に文末によく現れる記号があっても,明らかな一文である場合は区切らない}
     \begin{itemize}
      \item 散歩??かな.\lf \ \eos{}
     \end{itemize}
 \item {文頭,文末の記号は一文に含める}\footnote{文頭と文末の空白文字はブログ記事をHTMLからテキストに変換する際に削除した.}
     \begin{itemize}
      \item そんな日本語ないか.\underline{笑} \lf \  \eos{}
      \item 脱力.\underline{ORZ} \lf \  \eos{}
      \item \underline{P.S.}数年前,電車の...\eos{}
     \end{itemize}
}

CSJでは,対象が話し言葉であり,文の終わりが不明確であるという特徴がより
顕著である.
そこで「節単位」という「文」に概ね相当する単位を規定し,
文境界アノテーションを施している\cite{丸山:高梨:内元:2006}.


\subsection{括弧表現}

京大コーパスでは括弧表現は削除されていたが,ブログ記事の括弧表現
は,新聞記事と比べると,
本文と密接不可分な内容のものが多く無視できない.
例文(\ex{1})はKNBコーパスの括弧表現の例である.

\eenumsentence{
\item ここもFood\key{(}パンメインでカレーとか\key{)}の量は少なかったなー
\item 貴重な\key{(}まぁどのへんが貴重なのかはわからないけど\key{)}時間
を無駄にしてしまう.
\item どでかい神楽松明\key{(}激しく燃えている!\key{)}を担いで,狭い鞍
馬街道をどこからともなく練り歩き出す.
}

一方で,ブログ記事の括弧表現は多種多様で,文内に埋め込まれたままだと,係
り受け等のアノテーションが困難になる.
そこで,括弧表現を文中から取り出して一つの独立した文とした.
例文(\ex{2})は例文(\ex{1})から括弧表現を抽出したものである.

\enumsentence{
{\small{\# S-ID:KN012\_Gourmet\_6-1-23}}\\
ここもFood\key{(}パンメインでカレーとか\key{)}の量は少なかったなー
}

\eenumsentence{
\item {\small{\# S-ID:KN012\_Gourmet\_6-1-23-01}}\\
ここもFoodの量は少なかったなー
\item {\small{\# S-ID:KN012\_Gourmet\_6-1-23-02 \ 
括弧タイプ:例示 \
括弧位置:7 \ 括弧始:( \ 括弧終:)}}\\
パンメインでカレーとか
}

「\#」の行には文IDを表すS-IDをはじめ,直下の文の各種情報が記述されている.
抽出され一文になった括弧表現は,元の文の直後に置かれ,新たに文IDが与え
られる.
具体的には,元の文のID末尾に「-01」を付与し,抽出されて別の文となった文の
IDには「-02」「-03」「-04」等を付与する.
例えば例文(\ex{-1})はIDが「KN012\_Gourmet\_6-1-23」だが,括弧抽出後は,
「KN012\_Gourmet\_6-1-23-01」(\ex{0}a)と「KN012\_Gourmet\_6-1-23-02」
(\ex{0}b)の二文になる.


なお,括弧表現の元の文における位置情報が記録されており,復元も可能である.
例文(\ex{0}b)における「括弧位置:7」がその情報にあたり,「7」は括弧の開始
位置を字数により示している.

さらに,抽出された括弧文には,
読み,
日付,
金額,
場所,
同義,
例示,
その他
のいずれかの括弧タイプが与えられる.
以下にKNBコーパスにおける括弧タイプとその例を挙げる.

\enumsentence{
\begin{description}
 \item[読み:] 印象に残ったのは,「御髪(みかみ)神社」.
 \item[日付:] ソフトバンクが前夜に予想外割を発表したこともあって,
この日(10月24日)の情報通信株は面白いことになりそう
という意識があった.
 \item[金額:] また付いてくる麦飯とキャベツと赤だしの味噌汁がおかわり自
	    由で,私は豚カツ120g(1050円)と一緒に,ごはんを4杯,
	    みそしるを2杯,キャベツをもともとよそられていた量のの1.3
	    倍は食べてしまう.
 \item[場所:] まずは自転車で三条口(西大路三条)へと向かう.
 \item[同義:] 15〜20,ブル(ど真ん中)を狙って例えば15を三回あてれ
	    ば自分の陣地になりそっから点数が入る,というものである.
 \item[例示:] Webが使えない,通話料がやけに高いので電話をわざわざ公
	    衆電話からかけたりする,携帯会社が謳うオトクげなサービス(誕
	    生日割りなど)をほとんど受けられない...
 \item[その他:] 私は携帯電話が嫌いで(高校入学時に買わされたが),電源
	    を入れているときの方が少ない.
\end{description}
}


括弧表現はコーパス全体で137回出現した.
その内訳を表\ref{tab:paren-breakdown}に挙げる.
「その他」に分類される括弧文のほとんどは,上の例にあるような,
本文に対する補足説明と呼べるものだった.

\begin{table}[b]
\caption{括弧表現の出現数}
\label{tab:paren-breakdown}
\input{06table02.txt}
\end{table}

なお,表\ref{tab:breakdown}では,抽出された括弧表現も一文としてカウント
している.



\subsection{形態素}


形態素アノテーションでは,形態素と呼ばれる,当該言語において意味をもつ最
小の単位に文を分割し,各形態素に読み,原形,品詞,活用型,活用形を付与す
る.
形態素は,文の構造を解析する上での最も基本的な単位である.

KNBコーパスでは京大コーパスと同様に,形態素単位,タグ単位,文節単
位の3段階で階層的に文を分割している.
タグ単位とは,基本的に自立語1語を核として,その前後に存在する接頭辞,接
尾辞,助詞,助動詞などの付属語をまとめたものである.
タグ単位は,格・省略・照応関係と係り受け関係のアノテーションで使用される.
文節単位は,1語あるいは複数の自立語を核として,その前後に存在する接頭辞,
接尾辞,助詞,助動詞などの付属語をまとめたものであり,
係り受け関係のアノテーションで使用される\footnote{
つまり,\ref{sec:dep}節で述べるように,係り受け関係はタグ単位と文節単位
の両方でアノテーションされる.
}.
また,文節境界はタグ単位境界でもある.
例として表\ref{tab:morph}に,「半年ほど前に携帯電話の機種変更をしました.」
という文を対象にした形態素,タグ単位,文節単位のアノテーションを示す.

\begin{table}[b]
\caption{形態素,タグ単位,文節単位アノテーションの例}
\label{tab:morph}
\input{06table03.txt}
\end{table}


形態素単位と文節単位の他にタグ単位を用意するのは,格・省略・照応アノテー
ションの際,形態素より大きく文節より小さい単位でアノテーションする必要が
あるためである.
例えば「500メートル地点」という表現に対しては,「500メートル」と
「地点」の間に修飾格関係が成立する旨をアノテーションするが,「500メー
トル地点」は全体で1文節であり,また,「500メートル」は「500」と
「メートル」の2つの形態素に分かれる.
このため,「500メートル」を1つの単位としてまとめるタグ単位の導入が必
要になる.

KNBコーパスの形態素アノテーションは,品詞・活用体系,フォーマットともに
京大コーパスの仕様に準拠しているが,次のようなブログの特徴に対応するため
仕様を拡張した.

\eenumsentence{
\item 誤変換,脱字,衍字\footnote{
衍字とは,語句の中に間違って入り込んでいる,あるいは,
語句の前後に間違って隣接している不必要な文字のことである.
}
\item 口語的表現(方言,外国語,擬音・擬態語,言い淀み)
\item 創造的表現(記号,Webで頻出のスラング)
}

以下,それぞれについて詳述する.


\subsubsection{誤変換,脱字,衍字}

下に例を挙げる.
矢印$\to$の右側が誤変換,脱字,衍字の例である.

\eenumsentence{
\item 誤変換:「通信機能が\underline{内蔵}されたもの」$\to$「通信機能が\underline{内臓}されたもの」
\item 脱字:
\begin{enumerate}
 \item 「早めに\underline{行かないと}」$\to$「早めに\underline{かないと}」
 \item 「属する\underline{ように}なり」$\to$「属する\underline{よう}なり」
\end{enumerate}
\item 衍字:
\begin{enumerate}
 \item 「何が安いのか,考えて\underline{買って}いきます.」 
       $\to$
       「何が安いのか,考えて\underline{買いって}いきます.」 
 \item 「高級な\underline{料亭}や,焼肉屋,$\cdots$ダイニングも多い.」 
       $\to$
       「高級な\underline{り料亭}や,焼肉屋,$\cdots$ダイニングも多い.」 
\end{enumerate}
}

誤変換あるいは脱字を含む文は,元の誤った表現を残しつつ,正式な書き方・表現に基
づいてアノテーションすることとした.
例えば「早めにかないと」なら「早めに行かないと」としてアノテーションした.
加えて,コーパス中に用意されているメモ欄に,
「ER:(正しい書き方)」のように,誤りフラグERと正しい書き
方を記載することとした.
このメモはタグ単位に与えられている.
表\ref{tab:misspelling}左に脱字のアノテーション例を挙げる\footnote{
表\ref{tab:misspelling}ではタグ単位境界と文節境界が一致しているため,
タグ単位境界は明示していない.}.

\begin{table}[t]
\caption{脱字,衍字のメモ例}
\label{tab:misspelling}
\input{06table04.txt}
\end{table}

衍字を含む文は,衍字が語句の内部にある場合
と,
語句の前あるいは後ろに接している場合
の2通りに分けて対応した.
前者の場合,上述した誤変換あるいは脱字の場合と同様にアノテーションする.
つまり,元の表現を残しつつ,正式な書き方・表現に基づいてアノテーション
し,
メモ欄には誤りフラグERと正しい書き方を記載する.
後者の場合,衍字とそれが接している語句を別々の形態素としてアノテーション
し,
メモ欄には誤りフラグERと正しい書き方を記載する.
一方,タグ単位としては1つにまとめる.
表\ref{tab:misspelling}右に衍字のアノテーション例を挙げる.

KNBコーパス全体では,誤変換,脱字,あるいは衍字が102回出現した.
その内訳は,「京都観光」では43回,「携帯電話」では30回,「グルメ」では20
回,「スポーツ」では9回である.


\subsubsection{口語的表現}

方言や外国語,擬音・擬態語,意図的な言い淀み等がこれに該当する.
後述するように,結局これらの表現は元の形のままでアノテーションしている.
CSJにおいても,融合,省略,フィラー,断片化といった口語表現特有の現象を,
元の表現そのままでアノテーションしており,元のテキストを可能な限りそのま
まの形で正確に記述するという我々の方針と一致している.


\textbf{方言}では活用をどう記述するかが問題となる.
我々は,京大コーパスとの互換性と文法記述の正確性を最大限確保するため,
既存の活用に該当しない方言に対して,
形態素解析器JUMAN\footnote{http://nlp.kuee.kyoto-u.ac.jp/nl-resource/juman.html}
の活用記述法に準拠しつつ,新たな活用を定義した.
例として,関西方言の「や」に関連するもの
と,それに対応する標準語の(つまりJUMANに既存の)活用情報
を例文(\ex{1})に挙げる\footnote{下線の形態素の活用情報を記載している.}.

\eenumsentence{
 \item 「\underline{面倒や}ん」$\cdots$ ナ形容詞ヤ列基本形
 \item[] (「\underline{面倒だ}」$\cdots$ ナ形容詞基本形)
 \item 「悲しくなったりする\underline{んやろ〜}?」$\cdots$ ナ形容詞ヤ列
 基本推量形異表記
 \item[] (「悲しくなったりする\underline{んだろう}?」$\cdots$ 
 ナ形容詞ダ列基本推量形)
 \item 「小さいもん\underline{や}のに」$\cdots$ 判定詞ヤ列基本連体形
 \item[] (「小さいもの\underline{な}のに」$\cdots$ 判定詞ダ列基本連体形)
 \item 「もの大切にしい\underline{や}」$\cdots$ (終助詞につき無活用)
}

また,全ての方言に対して,メモ欄に「DI」(dialect)と記載した.
方言は全体で114回出現した.
その内訳は,「京都観光」では40回,「携帯電話」では45回,「グルメ」では3
回,「スポーツ」では26回である.



本研究における\textbf{外国語}とは,固有表現やURL(の一部),日本語として
日常的に用いられている名詞(「HDD」「PDF」等)を除く,
外国語の表現全てを指す.
例として,
「今日は3限をさぼらせて友達を連れて祇園界隈へGO!!」
という文における「GO」がある.
外国語のアノテーションでは,どのような日本語の品詞を割り当てるべきかが問
題になる.
KNBコーパスでは,外国語の形態素に対し,原則として,サ変名詞,普通名詞,
形容詞のいずれかを割り当てることとした.

\eenumsentence{
 \item サ変名詞:「祇園界隈へ\underline{GO}!!」
 \item 普通名詞:「予想\underline{GUY}」
 \item 形容詞:「京都\underline{LOVEな}ので」
 (ナ形容詞 ダ列基本連体形)
}
読みは英単語アルファベットそのままとした.

外国語のフレーズは,1単語1形態素とし,フレーズ全体で1タグ単位と
した.
例えば「with\kuuhaku{}my\kuuhaku{}friend」なら,
3つの形態素と2つの空白から成る1フレーズなので,5形態素1タグ単位とな
る.

アルファベットで書かれたものに限定してKNBコーパスにおける外国語を集計し
た結果,外国語に該当する表現が16個存在した.
テーマ別に見ると,「京都観光」には8表現,「携帯電話」には3表現,
「グルメ」には5表現で,「スポーツ」には存在しなかった.


\textbf{擬音語・擬態語}として,
KNBコーパスでは,辞書には登録されていない,『ピリリリリリ』『ポフュー』
等が出現した.

\eenumsentence{
 \item 「携帯『ピリリリリリ』」
 \item 「『ポフュー』って粉が出て」
}

JUMANでは,「ごくごく(飲む)」や「どしどし(応募する)」のような既知の
擬音語・擬態語を副詞としているため,KNBコーパスでもこれに倣い,これらを
全て副詞とした.

JUMANに未登録の擬音語・擬態語をKNBコーパスから抽出したところ,16語が得ら
れた.
テーマ別に見ると,「京都観光」から3語,「携帯電話」から8語,
「グルメ」から2語,「スポーツ」から3語が得られた.


\textbf{言い淀み}の例を以下に挙げる.

\eenumsentence{
\item 牛乳を入れて$\cdots$
\underline{ぎゅうにゅ}$\cdots$
\underline{にゅ}$\cdots$
牛乳ねぇぇぇー!
\item 現\underline{voda}$\cdots$もといソフトバンクモバイル
}

これらは未定義語として,メモ欄に「言い淀み」と記載する.
KNBコーパス全体で言い淀みは4回出現した.
その内訳は,「携帯電話」では1回,「グルメ」では2回,「スポーツ」では1回である.

\subsubsection{創造的表現}

顔文字等の記号や,「サーバ」を意味する「鯖」等のWeb上で多用されるスラン
グはブログ特有の表現といえるが,これらを創造的表現と呼ぶことにする.

KNBコーパスにおけるスラングは,「サーバ」を意味する「鯖」と,「マスコミ」
を意味する「マスゴミ」,「終わった」を意味する「オワタ」があった.

\eenumsentence{
\item MNP\underline{鯖}が落ちているから.
\item \underline{マスゴミ}で報道されてしまったし,
}

スラングも,他の表現と同様に京大コーパスとの互換性を最大限保つよう配慮す
る.
例えば上記の「鯖」は,普通名詞として扱う.
なおスラングには,メモ欄に「スラング:(正式あるいは一般的な表記)」を付
記している.
KNBコーパス全体におけるスラングの出現回数は3回で,
いずれも「携帯電話」に関する記事において出現した.


人がうなだれている姿を現す「orz」や顔文字,「!!」「...」のような
同じ記号の連続は1形態素の記号とした.

\eenumsentence{
\item 自分の部屋で紅茶をいれ,一人で味わいました\underline{orz}
\item 京都大好きな人が増えていけばいいなと密かに思ってます\underline{\mbox{(*^-^*)}}
\item 住んでみて改めて思います\underline{!!}
\item 厳かさに畏れを抱く\underline{...}
}

「orz」(変種も含む)と顔文字の出現頻度を調べたところ,
KNBコーパス全体では26回だった.
テーマ別で見ると,「京都観光」では11回,「携帯電話」では8回,「グルメ」では5
回,「スポーツ」では2回だった.

一方,「ー」「〜」などの長音記号は,それを含む形態素,あるいは直前の形態
素の一部とし,独立した記号とはしない.
また,長音記号付きの形態素は,対応する標準的な表現の異表記としてアノテー
ションした.

\eenumsentence{
\item 気分悪くなってきたのでやめ\underline{まーす}笑
\item[] 動詞性接尾辞ます型 基本形異表記
\item[] (cf. 「(やめ)ます」は「動詞性接尾辞ます型 基本形」)
\item 誰か一緒に\underline{いこ〜}!!
\item[] 子音動詞カ行促音便形 意志形異表記
\item[] (cf. 「(一緒に)行こう」は「子音動詞カ行促音便形 意志形」)
}

長音記号に関するもの以外で,標準的表現の異表記としてアノテーションされて
いる表現として「分厚っ」や「すげぇ」,「大好きだあ」などがある.

これら標準的表現の異表記を集計したところ,KNBコーパス全体では38表現存在
した.
その内訳は,「京都観光」が15表現,「携帯電話」が8表現,「グルメ」が11表
現,「スポーツ」が4表現である.


\subsection{係り受け \label{sec:dep}}

係り受けアノテーションでは,タグ単位間と文節間の2種類の係り受け関係を
京大コーパスに準拠する形式でアノテーションした.
係り受け関係は,自然言語処理において,文の構文的意味的構造を表す最も一般
的な手段である.

タグ単位間と文節間の係り受け関係の例を
図\ref{fig:dep-tag}と
図\ref{fig:dep-bunsetsu}に挙げる.
係り受け関係として,京大コーパスと同様,通常の係り受け関係の他に,並列関
係と同格関係がある.

\begin{figure}[b]
\begin{center}
\includegraphics{18-2ia6f1.eps}
\end{center}
\caption{タグ単位間の係り受け}
\label{fig:dep-tag}
\end{figure}
\begin{figure}[b]
\begin{center}
\includegraphics{18-2ia6f2.eps}
\end{center}
\caption{文節間の係り受け}
\label{fig:dep-bunsetsu}
\end{figure}
\begin{figure}[b]
\begin{center}
\includegraphics{18-2ia6f3.eps}
\end{center}
\hangcaption{CSJにおける倒置の係り受けの例(「これは」が倒置され,左の文節
「耐えられないんです」に 係っている) }
\label{fig:dep-csj-inversion}
\end{figure}


CSJでは話し言葉特有の現象に対応すべく独自の仕様を設けている.
例えば,文節の倒置に対応するために右から左への係り受けを許し(図
\ref{fig:dep-csj-inversion}の実線.
「これは」が倒置され,左の文節
「耐えられないんです」に 係っている)\inhibitglue\footnote{図\ref{fig:dep-csj-inversion}と図\ref{fig:dep-csj-nejire}における
破線は,KNBコーパスの仕様に基づく係り受けアノテーションを示している.
},
また,言い差し,ねじれに対応するために係り先のない文節を認めている(図
\ref{fig:dep-csj-nejire}の実線.
「目標は」の係り先として適切なものがない)\inhibitglue\footnote{
図\ref{fig:dep-csj-inversion}と図\ref{fig:dep-csj-nejire}の例は
CSJの係り受けアノテーションマニュアル
\cite{CSJ:係り受けマニュアル}
から引用した.
}.

KNBコーパスでは,京大コーパスとの互換性を重視し,上記のような特殊な仕様
は避けることとした.
つまり図\ref{fig:dep-csj-inversion}と図\ref{fig:dep-csj-nejire}に相当す
る現象に対しても,図中の破線にあるように,左から右へ,最も適切と思われる文
節に係るようにアノテーションするという方針にした.
例えば,
例文(\ex{1})(図\ref{fig:dep-knb-nejire})のように,
記事の執筆者の誤りによって係り先の文節が括弧に入れられ,括弧抽出処理に
よって別の文になっている場合があった\footnote{
つまり,本来なら「前に」の直前に「)」が来るべきところが,ここでは後に来
ている.
}.


\enumsentence{
壊れる(画面が見えなくなる前に)他の携帯を手に入れようと
}

\begin{figure}[t]
\begin{center}
\includegraphics{18-2ia6f4.eps}
\end{center}
\caption{CSJにおけるねじれの係り受けの例(「目標は」の係り先として適切な
 ものがない)}
\label{fig:dep-csj-nejire}
\end{figure}
\begin{figure}[t]
\begin{center}
\includegraphics{18-2ia6f5.eps}
\end{center}
\hangcaption{KNBコーパスにおけるねじれの係り受けの例(「壊れる」の係り先とし
 て適切なものが括弧抽出により無くなっている)}
\label{fig:dep-knb-nejire}
\end{figure}

この例文は括弧抽出処理を経て次のように分けられる.

\eenumsentence{
 \item 壊れる他の携帯を手に入れようと
 \item 画面が見えなくなる前に
}

つまり,閉じ括弧位置の間違いのため,「壊れる」の本来の係り先「前に」がな
くなっている.
これに対して,上記方針に従い,抽出された本来の係り先文節(例文(\ex{0})
の「前に」)の係り先文節(「入れようと」)に係るものとした.



\subsection{格・省略・照応 \label{sec:case-ellipsis-anaphora}}

京大コーパスに準拠する形式で,格・省略・照応のアノテーションを付与した.
これらの情報は,係り受け関係よりさらに踏み込んだ文の意味構造や,文よりさ
らに大きい単位である談話の構造について知る手段となる.

格アノテーションでは,用言の場合,その用言に係る要素との間の意味的関
係を当該用言に付与する.
ここでいう意味的関係とは,ガ格,ヲ格,ニ格,ト格,デ格,カラ格などの格助
詞に対応する関係や,「〜を通じて」「〜として」などの複合辞に対応する関係,
さらには,格助詞や複合辞に対応するものが無い「外の関係」と呼ばれる関係が
ある\footnote{我々は,日本語の格助詞は用言とその用言に係る要素の意味的関係を
(比較的荒い粒度で)表していると考える.
しかし,査読者が指摘した通り,日本語の格助詞はいわゆる深層格を表すもの
ではないという点で意味的関係の表現としてさらなる詳細化の余地が残されて
いる.
用言とその用言に係る要素の意味的関係の表現方法の洗練は今後の課題とする.}.
「は」や「も」などの副助詞でマークされる要素に対しては,
例文(\ex{2})のように,
これらのいずれかの関係のうち意味的に合致するものが選ばれる.
格アノテーションは体言間の意味的関係も対象とする.
その場合,体言の必須格を表すノ格や体言の補足的情報を表す修飾格などが
付与される.
例文(\ex{1})と(\ex{2})に用言格アノテーションの例を,
例文(\ex{3})
と(\ex{4})
に体言格アノテーションの例を挙げる.
下線が引いてある用言あるいは体言への格アノテーションが括弧内に書かれている.

\enumsentence{
一ユーザとしては,かなり使いでが\underline{ある.}
(ガ格:使いで,トシテ格:ユーザ)
}

\enumsentence{
Foodは量が\underline{少ない}.
(ガ格:量,外の関係:Food)
}


\enumsentence{
フツウの携帯ユーザーの\underline{仲間}入り
(ノ格:ユーザー)
}

\enumsentence{
年間最大\underline{1万8千円程度}
(修飾:年間,修飾:最大)
}

省略アノテーションは,それと係り受け関係にあるはずの要
素が省略されている場合,その要素の指示対象と意味的関係を付与する.
意味的関係は格アノテーションのものと同様である.
以下に省略アノテーションの例を挙げる.

\enumsentence{
4,5回しか\underline{行ったことないけど}.
(ガ格:一人称)
}

\enumsentence{
だって携帯の会社変えたらアドレスとか全部変わるもんね.
\underline{面倒やん.}
(ガ格:変わる)
}


\enumsentence{\label{ex:sigureden}
時雨殿に行った.$\cdots$
残念ながら\underline{閉館}していたので…(ガ格:時雨殿) 
}

\enumsentence{\label{ex:yokosen}
私の使っている携帯電話に最近原因不明の黒い横線が入るようになってしまった.…
もちろんプリペイド携帯電話.…
現実の\underline{使用}の上ではそんなに困ることもない…
(ガ格:私,ヲ格:電話)
}

表\ref{tab:cases-breakdown}に,格・省略アノテーションの結果明らかになっ
た,KNBコーパスにおける最頻出の意味的関係上位 5 つの出現回数を挙げる.
いずれのテーマにおいても,ガ格,ヲ格,ニ格,ノ格,修飾格が上位5つを占め
た.

照応アノテーションでは,ある表現が既出のものと同じ対象を指示している場合,
その表現(照応表現と呼ぶ)に対して
対応する既出表現の位置情報を付与する.
例文(\ex{1}),(\ex{2}),(\ex{3})に照応アノテーションの例を挙げる.

\enumsentence{
父と野球.
\underline{父}は野球が好きだった.(父 $=$ 1文前)
}

\enumsentence{
実は,この2大勢力時代という表現は正確ではない.
本当はもう1勢力あるのだ.
\underline{それ}が『ネギ』である.
(勢力 $=$ 1文前)
}

\enumsentence{
困ったことに,友人たちの間では僕は携帯電話を持ち歩かないことで有名だ.
が,それは事実とは反している.
\underline{携帯}は持ち歩いている.
(電話 $=$ 2文前)
}

\begin{table}[t]
\caption{意味的関係の出現数(出現頻度上位 5)}
\label{tab:cases-breakdown}
\input{06table05.txt}
\end{table}

例文(\ex{0})に関して,下線の「携帯」は2文前の「携帯電話」と同じ対象を指示し
ているが,照応アノテーションの単位はタグ単位なので,「電話」のみと
$=$で結んでいる.

照応アノテーションは,明示的に書かれている表現に対してだけでなく,
省略された表現に対してもなされる.
例えば,例文(\ref{ex:sigureden})と(\ref{ex:yokosen})に対しては,
それぞれ例文(\ex{1})と(\ex{2})のようにアノテーションする.

\enumsentence{
時雨殿に行った.…
残念ながら\underline{閉館}していたので…
(時雨殿 $=$ 1文前) 
}

\enumsentence{
私の使っている携帯電話に最近原因不明の黒い横線が入るようになってしまった.…
もちろんプリペイド携帯電話.…
現実の\underline{使用}の上ではそんなに困ることもない…
(私 $=$ 2文前,電話 $=$ 2文前)
}

結局,照応アノテーションでは,形態素や係り受けのアノテーションと違い,京
大コーパスからの仕様拡張の必要は無かった.

KNBコーパス全体における照応表現の出現回数は9,881回だった.
テーマ別に見ると,「京都観光」では3,620回,「携帯電話」では3,146回,
「グルメ」では1,939回,「スポーツ」では1,176回だった.



\subsection{固有表現}

固有表現アノテーションでは,文中の人名や地名,日付,時間表現等をマークす
る.
固有表現は解析システムの辞書に登録されていない場合が多く,何も手段を講じ
なければ未知語として扱われ,解析誤りを引き起こしうる.
解析システムをより頑健にするには,本コーパスの固有表現アノテーションのよ
うな,固有表現自動抽出を学習するためのデータが必要である.

KNBコーパスの固有表現は,京大コーパスと同様に,IREXの仕様に準
拠してアノテーションされている.
固有表現は次の8つのいずれかに分類される.

\eenumsentence{
\item ORGANIZATION …組織名
\item LOCATION …地名
\item PERSON …人名
\item ARTIFACT …固有物名
\item PERCENT …割合表現
\item MONEY …金額表現
\item DATE …日付表現
\item TIME …時間表現
}

上記分類以外に,それら分類のいずれにも属さない,あるいは判定が困難な場合
に用いるOPTIONALという分類基準がある.

例文(\ex{1})に例を挙げる.

\eenumsentence{
\item \underline{近鉄}ファンだった.
(近鉄:ORGANIZATION)
\item \underline{京都}を回ってみようと思います☆
(京都:LOCATION)
\item あの\underline{孫}社長が携帯業界に参入してきた.
(孫:PERSON)
\item 初めて\underline{バンホーテンココア}を飲んだときも衝撃でしたが,
(バンホーテンココア:ARTIFACT)
\item 乗車率\underline{90%}程度だろうか.
(90%:PERCENT)
\item 蕎麦の一杯や定食の一人前が\underline{800円}から\underline{10
00円}もする.
(800円:MONEY,1000円:MONEY)
\item \underline{来シーズン}もプロ野球戦線が盛り上がるといいですね〜.
(来シーズン:DATE)
\item \underline{9時}過ぎに出発する.(9時:TIME)
}


\ref{sec:case-ellipsis-anaphora}節の格・省略・照応アノテーションと同様,
固有表現アノテーションにおいても,京大コーパス (IREX) の仕様の拡張は必要
なかった.

\begin{table}[b]
\caption{固有表現の出現数}
\label{tab:NE-breakdown}
\input{06table06.txt}
\end{table}

アノテーションの結果,KNBコーパスには固有表現が2,073個含まれていることが
分かった.
その内訳を表\ref{tab:NE-breakdown}に挙げる.



\subsection{評価表現 \label{sec:evalexp}}

KNBコーパスにおける評価とは,
ある対象に対して述べられた肯定的,
もしくは否定的判断や態度,叙述を指す.
典型的には,「お酒は美味しかったですよ.」のような,ある対象に対する肯定
的/否定的な判断や態度がそれにあたる.
また,「昼食が2万3千円〜だった.」のような事実的な言明であっても,その
事実がある対象への肯定的/否定的評価に結びつくなら,それも評価に含める.


評価表現アノテーションでは,
    川田ら (川田 他 2009) に基づき,
何らかの評価を含む文に対して次の情報を付与する.

\eenumsentence{
\item 評価保持者:その文における評価を発信している人や団体.
\item 評価表現:文中での評価を表している部分.
\item 評価タイプ:評価の種類と評価の極性.
極性は,「$+$」が肯定的評価で「$-$」が否定的評価を表す.
 \begin{description}
  \item[当為:] 評価保持者による提言や助言,対策を表す言明である.
	     典型的には,「〜すべきだ」「〜しましょう」といったものがこ
	     れにあたる.
  \item[要望:] 評価保持者の要望や要請を表す言明である.
	     典型的には,「〜してほしい」「〜を求める」などがこれにあた
	     る.
  \item[感情$+/-$:] 評価保持者の欲求や喜怒哀楽,好き嫌いといった感情を
	     表す言明である.
	     典型的には,「〜が好き」「〜が悲しい」などがこれにあたる.
  \item[批評$+/-$:] 評価保持者による賛成や反対,称賛,批判などの感情の
	     言明がこれにあたる.典型的には,「〜が素晴らしい」「〜が納
	     得できない」などがこれにあたる.
  \item[メリット$+/-$:] 評価保持者による,評価対象に対する利点や欠点,
	     特徴や課題について述べた言明である.典型的には,「〜効果が
	     ない」「〜がうるさい」などがこれにあたる.
  \item[採否$+/-$:] 評価保持者が評価対象を積極的に利用したり,新たな製
	     品や制度などを採用する姿勢を述べた言明である.典型的には,
	     「〜を利用する」「〜を導入する」「〜を採用する」などがこれ
	     にあたる.
  \item[出来事$+/-$:] 評価対象によって引き起こされた良い/悪い状況や個
	     別的経験について述べられた言明である.
	     典型的には,「〜が壊れた」「〜を受賞した」などがこれにあた
	     る.
 \end{description}
\item 評価対象:評価の対象.
}

例文(\ex{1})と(\ex{2})に,「おかきやせんべいの店なのだが,これがオ
イシイ.」と「貧乏人臭い,なんか怪しげな人っぽいといった類のものです.」
という文への評価表現アノテーションを挙げる.

\eenumsentence{
\item[] \hspace*{-5mm}おかきやせんべいの店なのだが,これがオイシイ.
\item 評価保持者:[著者]
\item 評価表現:オイシイ
\item 評価タイプ:批評$+$
\item 評価対象:おかきやせんべい
}

\eenumsentence{
\item[] \hspace*{-5mm}貧乏人臭い,なんか怪しげな人っぽいといった類のもの
です.
\item 評価保持者:[不定]
\item 評価表現:貧乏人臭い,なんか怪しげな人っぽいといった類のものです.
\item 評価タイプ:批評$-$
\item 評価対象:[プリペイドユーザー]
}

[$\cdots$]でマークされているものは,文中に存在しない評価保持者,評価対
象である\footnote{
(\ex{0})で評価対象が[プリペイドユーザー]となっているが,これは直前の文が
「加えて,プリペイドユーザーというものが持つ社会に対するマイナスイメージ
も考えてみました.」だったためである.
}.
(\ex{0})で評価保持者が[不定]となっているのは,その評価が執筆者や特
定の個人,団体によるものではなく,世間一般の評価であることを示している.

一文に複数の評価が含まれていることもある.
次の例では,「しょぼかった」「画面も小さかった」「音も3和音とかやった」
の3つの評価表現が一文に含まれている.

\eenumsentence{
\item[] 最初はカメラもしょぼかったし,画面も小さかったし,音も3和音
	    とかやった.
\item 評価保持者: 
\begin{enumerate}
 \item [著者]
 \item [著者]
 \item [著者]
\end{enumerate}
\item 評価表現:
\begin{enumerate}
 \item  しょぼかった
 \item 画面も小さかった
 \item 音も3和音とかやった
\end{enumerate}
\item 評価タイプ: 
\begin{enumerate}
 \item 批評$-$
 \item メリット$-$
 \item メリット$-$
\end{enumerate}
\item 評価対象: 
\begin{enumerate}
 \item カメラ
 \item 画面\vspace{-0.5pt}
 \item 音
\end{enumerate}
}

(\ex{0})の (1) (2) (3) は対応しており,
例えば,評価「しょぼかった」の評価保持者は「[著者]」,評価タイプ
は「批評$-$」,評価対象は「カメラ」となる.

評価表現アノテーションは一文内で完結しない場合が頻繁にある.
例えば例文(\ex{1})のように,評価の対象と評価表現が(近接する)異なる文に
書かれている場合がそうである.
例文(\ex{1})の2文目は,1文目に書かれている対象「ココア・オレ」に対する評
価を述べている\footnote{
(\ex{1})のaからdは2文目に付与される情報である.
}.

\eenumsentence{
\item[] 私はココア・オレを買います.\\これは何度飲んでも衝撃です.
 \item 評価保持者:[著者]
 \item 評価表現:衝撃です
 \item 評価タイプ:感情$+$
 \item 評価対象:[ココア・オレ]
}

異なる文に分断された評価対象と評価表現は多くの場合,格・省略・照応関係で
結ばれている.
従って評価表現アノテーションは,格・省略・照応関係のアノテーションと合わ
せてなされるべきである.
\ref{sec:case-ellipsis-anaphora}節で述べた通り,KNBコーパスでは格・省略・
照応関係もアノテーションされており,例文(\ex{0})の2文は照応表現「これ」
を介して次のように関連づけられる.

\eenumsentence{
\item[] 私はココア・オレを買います.
\item[] \underline{これ}は何度飲んでも衝撃です.
(これ $=$ ココア・オレ)
}

\begin{table}[b]
\caption{評価表現の出現数}
\label{tab:eval-breakdown}
\input{06table07.txt}
\end{table}

KNBコーパス全体を評価表現アノテーションした結果,
2,045文(48.85\%)の文に何らかの評価表現が含まれていた.
その内訳は表\ref{tab:eval-breakdown}の通りである\footnote{評価表現を含む文の総数が2,045なのに対し,表\ref{tab:eval-breakdown}の合
計が2,510となっているが,これは複数の評価表現を含む文が存在するためであ
る.}.
表において,「当」「要」「感$+$」等はそれぞれ「当為」「要望」「感情$+$」
等を表す.




\subsection{アノテーション可視化HTMLファイル}

以上で述べた通り,KNBコーパスのアノテーションは多岐にわたり,
そのままでは(人間にとって)可読性が低い.
そこで,アノテーションを可視化したHTMLファイルを別途用意した.

図\ref{fig:visualize}は,「悩んだ末,カシオのG’zoneという,衝撃
や水濡れに強いのがウリの機種に決めました.」という文へのアノテーションを
可視化した例である.
図\ref{fig:visualize}の上段が,係り受けと格・省略・照応,固有表現,評価
表現\footnote{アノテーション可視化HTMLファイル中では「評判表現」となって
いる.} のアノテーションを可視化したもので,下段が形態素のアノテーションを可
視化したものである.

\begin{figure}[t]
\begin{center}
\includegraphics{18-2ia6f6.eps}
\end{center}
\caption{アノテーション可視化HTMLファイル}
\label{fig:visualize}
\end{figure}

係り受け関係はタグ単位に基づくものを可視化している.
また,「濡れに」に係っている「衝撃や」と「水」は並列関係にあるとアノテー
ションされているので「P」(Parallel) とラベル付けされている.

格・省略アノテーションに関して,例えば「決めました.」はガ格とニ格を取り,それ
ぞれ,省略されている一人称と,直前のタグ単位「機種」がそれらの要素となっ
ている.
加えて,「(悩んだ)末」という時間格を取る.
照応アノテーションの例として,「機種」が一文前の「機種」と照応関係にある
とアノテーションされている.

固有表現として「カシオ」と「G’zone」の2つがこの例文中にある.
前者は企業名なので「ORGANIZATION」,後者は製品名であり「ARTIFACT」となる.

この例文中の評価情報として「衝撃や水濡れに強いのがウリの機種に決めました」
がある.
この評価保持者は[著者]である.
評価保持者が[著者]あるいは[不定]の場合,可視化HTML中では明示しない.
評価タイプは「採否$+$」であり,
評価対象は「衝撃や水濡れに強いのがウリの機種」となる.
評価表現は「衝撃や水濡れに強いのがウリの機種に決めました」で,この箇所は
係り受けの可視化において当該文字列が黄色でマークされている.

図\ref{fig:visualize}下段の形態素アノテーションの可視化では,
形態素ごとに,表出形,読み,原形,品詞,活用型と活用形を示している.
また,文節境界とタグ単位境界もこの箇所で明記している.



\section{構築から配布まで \label{sec:construction}}

KNBコーパスの構築から配布までの手順は次の通りである.

\begin{enumerate}
 \item 記事の収集
 \item 文境界/括弧抽出アノテーション
 \item 形態素/係り受け/格・省略・照応/固有表現
       アノテーション
 \item 評価表現アノテーション
 \item 誹謗・中傷・宣伝的内容の削除
 \item アノテーション可視化HTMLの生成
\end{enumerate}

記事の収集は,独自にブログサーバを設置し,大学生にアルバイトとして記事を
書いてもらうことで収集した.
その際,記事執筆にあたった全ての大学生から記事の著作権譲渡の承諾を得た.
手順 (2)と (3) は,自動でアノテーションした後,人手修正を施した.
(3) の自動アノテーションでは,京大コーパスと同様,JUMAN/KNP\footnote{http://nlp.kuee.kyoto-u.ac.jp/nl-resource/knp.html} を用いた.
一方,(4) は全て人手で行った.
また(2) (3) (4) に関して,京大コーパスと同様,アノテーション基準の
見直しと再アノテーションというサイクルを何度か繰り返して行った.
その後,元のブログ記事にある,特定の個人,団体に対する誹謗や中傷,
宣伝的な内容を人手でチェックし,削除,あるいは伏せ字化した.
最後に,全てのアノテーションを可視化したHTMLファイルを自動生成した.

完成したKNBコーパスのパッケージ (4.2~MB) は,
京都大学情報学研究科—NTTコミュニケーション科学基礎研究所共同研究ユニッ
トのホームページ\footnote{http://nlp.kuee.kyoto-u.ac.jp/kuntt/}
からダウンロードできる.
京大コーパスとは異なり,コーパスのパッケージには,アノテーションだけでなく
記事の本文も含まれている.



\section{おわりに}
\label{sec:conclusion}

本稿では,我々が構築した,他に類を見ない,解析済みブログコーパス
について報告した.
特に,ブログ特有の現象とそれらの正確な記述,京大コーパスとの互換性の重視
について述べた.

今後は,コーパスの大規模化と,
KNBコーパスを用いた各種解析システム(形態素解析器や構文解析器等)のブロ
グドメインへの適応を試みる予定である.

自然言語処理技術のブログへの適用が今後ますます活発化すること
は明らかである.
その成否は,本コーパスのような解析済みブログコーパスを充実させることが出
来るかどうかにかかっている.
一方,解析済みブログコーパスの充実にとって欠かせないのは,
本稿で述べたような構築ノウハウの蓄積と構築されたコーパスの流通である.
本研究の貢献はこの点にある.



\acknowledgment

本研究は,京都大学情報学研究科—NTTコミュニケーション科学基礎研究所共同
研究ユニット 「グローバルコミュニケーションを支える言語処理技術」の活動
の一環として行われました.
共同研究ユニットのメンバーの方々に感謝申し上げます.
また,評価情報のアノテーションについて御協力いただいた情報通信研究機構知識処理
グループのメンバーの方々に感謝申し上げます.



\bibliographystyle{jnlpbbl_1.5}
\begin{thebibliography}{}

\bibitem[\protect\BCAY{Adar, Hurst, Finin, Glance, Nicolov, \BBA\ Tseng}{Adar
  et~al.}{2008}]{ICWSM:2008}
Adar, E., Hurst, M., Finin, T., Glance, N., Nicolov, N., \BBA\ Tseng, B. \BBOP
  2008\BBCP.
\newblock {\Bem Proceedings of the 2nd International Conference of Weblogs and
  Social Media}.
\newblock The AAAI Press.

\bibitem[\protect\BCAY{Adar, Hurst, Finin, Glance, Nicolov, \BBA\ Tseng}{Adar
  et~al.}{2009}]{ICWSM:2009}
Adar, E., Hurst, M., Finin, T., Glance, N., Nicolov, N., \BBA\ Tseng, B. \BBOP
  2009\BBCP.
\newblock {\Bem Proceedings of the 3rd International Conference of Weblogs and
  Social Media}.
\newblock The AAAI Press.

\bibitem[\protect\BCAY{Akamine, Kawahara, Kato, Nakagawa, Inui, Kurohashi,
  \BBA\ Kidawara}{Akamine
  et~al.}{2009}]{Akamine:Kawahara:Kato:Nakagawa:Inui:Kurohashi:Kidawara:2009}
Akamine, S., Kawahara, D., Kato, Y., Nakagawa, T., Inui, K., Kurohashi, S.,
  \BBA\ Kidawara, Y. \BBOP 2009\BBCP.
\newblock \BBOQ WISDOM: A Web Information Credibility Analysis
  Systematic.\BBCQ\
\newblock In {\Bem Proceedings of the ACL-IJCNLP 2009 Software Demonstrations},
  \mbox{\BPGS\ 1--4}.

\bibitem[\protect\BCAY{飯田\JBA 小町\JBA 乾\JBA 松本}{飯田 \Jetal
  }{2007}]{NAISTCorpus:2007}
飯田龍\JBA 小町守\JBA 乾健太郎\JBA 松本裕治 \BBOP 2007\BBCP.
\newblock
  {NAIST}テキストコーパス:述語項構造と共参照関係のアノテーション(解析・対話
)\inhibitglue.\
\newblock \Jem{情報処理学会研究報告. 自然言語処理研究会報告}, \mbox{\BPGS\
  71--78}. 社団法人情報処理学会.

\bibitem[\protect\BCAY{河原\JBA 黒橋\JBA 橋田}{河原 \Jetal
  }{2002}]{Kawahara:Kurohashi:Hashida:2002j}
河原大輔\JBA 黒橋禎夫\JBA 橋田浩一 \BBOP 2002\BBCP.
\newblock 「関係」タグ付きコーパスの作成.\
\newblock \Jem{言語処理学会 第8回年次大会}, \mbox{\BPGS\ 495--498}.

\bibitem[\protect\BCAY{河原\JBA 笹野\JBA 黒橋\JBA 橋田}{河原 \Jetal
  }{2005}]{KUCorpus:rel:2005}
河原大輔\JBA 笹野遼平\JBA 黒橋禎夫\JBA 橋田浩一 \BBOP 2005\BBCP.
\newblock \Jem{格・省略・共参照タグ付けの基準}.
\newblock
  http://nlp.kuee.kyoto-u.ac.jp/nl-resource/corpus/KyotoCorpus4.0/doc/rel\_gui
deline.pdf.

\bibitem[\protect\BCAY{川田\JBA 中川\JBA 赤峯\JBA 森井\JBA 乾\JBA 黒橋}{川田
  \Jetal }{2009}]{評価情報タグ付与基準:2009}
川田拓也\JBA 中川哲治\JBA 赤峯亨\JBA 森井律子\JBA 乾健太郎\JBA 黒橋禎夫 \BBOP
  2009\BBCP.
\newblock \Jem{評価情報タグ付与基準}.
\newblock http://www2.nict.go.jp/x/x163/project1/eval\_spec\_20090901.pdf.

\bibitem[\protect\BCAY{Kobayashi, Inui, \BBA\ Matsumoto}{Kobayashi
  et~al.}{2007}]{Kobayashi:Inui:Matsumoto:2007}
Kobayashi, N., Inui, K., \BBA\ Matsumoto, Y. \BBOP 2007\BBCP.
\newblock \BBOQ Opinion mining from web documents.\BBCQ\
\newblock {\Bem Transactions of JSAI}, {\Bbf 22}  (2), \mbox{\BPGS\ 227--238}.

\bibitem[\protect\BCAY{国立国語研究所}{国立国語研究所}{2006}]{日本語話し言葉コ
ーパスの構築法:2006}
国立国語研究所.
\newblock 日本語話し言葉コーパスの構築法(国立国語研究所報告書
  124)\inhibitglue.\
\newblock
  \Turl{http://www.kokken.go.jp/katsudo/seika/corpus/csj\_report/CSJ\_rep.pdf}.

\bibitem[\protect\BCAY{Kudo, Yamamoto, \BBA\ Matsumoto}{Kudo
  et~al.}{2004}]{Kudo:Yamamoto:Matsumoto:2004}
Kudo, T., Yamamoto, K., \BBA\ Matsumoto, Y. \BBOP 2004\BBCP.
\newblock \BBOQ Applying Conditional Random Fields to Japanese Morphological
  Analysis.\BBCQ\
\newblock In {\Bem 2004 Conference on Empirical Methods in Natural Language
  Processing (EMNLP2004)}, \mbox{\BPGS\ 230--237}.

\bibitem[\protect\BCAY{黒橋\JBA 居蔵\JBA 坂口}{黒橋 \Jetal
  }{2000}]{KUCorpus:syn:2000}
黒橋禎夫\JBA 居蔵由衣子\JBA 坂口昌子 \BBOP 2000\BBCP.
\newblock \Jem{形態素・構文タグ付きコーパス作成の作業基準 version 1.8}.
\newblock
  http://nlp.kuee.kyoto-u.ac.jp/nl-resource/corpus/KyotoCorpus4.0/doc/syn\_gui
deline.pdf.

\bibitem[\protect\BCAY{Lafferty, McCallum, \BBA\ Pereira}{Lafferty
  et~al.}{2001}]{Lafferty:McCallum:Pereira:2001}
Lafferty, J., McCallum, A., \BBA\ Pereira, F. \BBOP 2001\BBCP.
\newblock \BBOQ Conditional Random Fields: Probabilistic Models for Segmenting
  and Labeling Sequence Data.\BBCQ\
\newblock In {\Bem Proceedings of the Eighteenth International Conference on
  Machine Learning (ICML2001)}, \mbox{\BPGS\ 282--289}.

\bibitem[\protect\BCAY{丸山\JBA 高梨\JBA 内元}{丸山 \Jetal
  }{2006}]{丸山:高梨:内元:2006}
丸山岳彦\JBA 高梨克也\JBA 内元清貴 \BBOP 2006\BBCP.
\newblock 第5章 節単位情報.\
\newblock \Jem{日本語話し言葉コーパスの構築法(国立国語研究所報告書
  124)\inhibitglue}, \mbox{\BPGS\ 255--322}. 国立国語研究所.

\bibitem[\protect\BCAY{McClosky, Charniak, \BBA\ Johnson}{McClosky
  et~al.}{2006}]{McClosky:Charniak:Johnson:2006}
McClosky, D., Charniak, E., \BBA\ Johnson, M. \BBOP 2006\BBCP.
\newblock \BBOQ Reranking and Self-Training for Parser Adaptation.\BBCQ\
\newblock In {\Bem Proceedings of the 21st International Conference on
  Computational Linguistics and 44th Annual Meeting of the Association for
  Computational Linguistics (COLING-ACL' 06)}, \mbox{\BPGS\ 337--344}.

\bibitem[\protect\BCAY{McDonald, Lerman, \BBA\ Pereira}{McDonald
  et~al.}{2006}]{McDonald:Lerman:Pereira:2006}
McDonald, R., Lerman, K., \BBA\ Pereira, F. \BBOP 2006\BBCP.
\newblock \BBOQ Multilingual Dependency Analysis with a Two-Stage
  Discriminative Parser.\BBCQ\
\newblock In {\Bem Proceedings of CoNLL-X}, \mbox{\BPGS\ 216--220}.

\bibitem[\protect\BCAY{宮崎\JBA 森}{宮崎\JBA 森}{2008}]{宮崎:森:2008}
宮崎林太郎\JBA 森辰則 \BBOP 2008\BBCP.
\newblock 製品レビュー文に基づく評判情報コーパスの作成とその特徴の分析.\
\newblock \Jem{情報処理学会研究報告. 自然言語処理研究会報告2008 (90)},
  \mbox{\BPGS\ 99--106}.

\bibitem[\protect\BCAY{Murakami, Masuda, Matsuyoshi, Nichols, Inui, \BBA\
  Matsumoto}{Murakami
  et~al.}{2009}]{Murakami:Masuda:Matsuyoshi:Nichols:Inui:Matsumoto:2009}
Murakami, K., Masuda, S., Matsuyoshi, S., Nichols, E., Inui, K., \BBA\
  Matsumoto, Y. \BBOP 2009\BBCP.
\newblock \BBOQ Annotating Semantic Relations Combining Facts and
  Opinions.\BBCQ\
\newblock In {\Bem Proceedings of the Third ACL Workshop on Linguistic
  Annotation Workshop}, \mbox{\BPGS\ 150--153}.

\bibitem[\protect\BCAY{新里\JBA 橋本\JBA 河原\JBA 黒橋}{新里 \Jetal
  }{2007}]{新里:橋本:河原:黒橋:2007}
新里圭司\JBA 橋本力\JBA 河原大輔\JBA 黒橋禎夫 \BBOP 2007\BBCP.
\newblock 自然言語処理基盤としてのウェブ文書標準フォーマットの提案.\
\newblock \Jem{言語処理学会第13回年次大会発表論文集}, \mbox{\BPGS\ 602--605}.

\bibitem[\protect\BCAY{Nivre, Hall, Nilsson, sen Eryi~git, \BBA\ Marinov}{Nivre
  et~al.}{2006}]{Nivre:Hall:Nilsson:senEryigit:Marinov:2006}
Nivre, J., Hall, J., Nilsson, J., sen Eryi~git, G., \BBA\ Marinov, S. \BBOP
  2006\BBCP.
\newblock \BBOQ Labeled Pseudo-Projective Dependency Parsing with Support
  Vector Machines.\BBCQ\
\newblock In {\Bem Proceedings of CoNLL-X}, \mbox{\BPGS\ 221--225}.

\bibitem[\protect\BCAY{Sekine \BBA\ Isahara}{Sekine \BBA\
  Isahara}{2000}]{IREX:2000}
Sekine, S.\BBACOMMA\ \BBA\ Isahara, H. \BBOP 2000\BBCP.
\newblock \BBOQ IREX: IR and IE Evaluation-Based Project in Japanese.\BBCQ\
\newblock In {\Bem Proceedings of the 2nd International Conference on Language
  Resources and Evaluation (LREC2000)}, \mbox{\BPGS\ 1475--1480}.

\bibitem[\protect\BCAY{Sriphaew, Takamura, \BBA\ Okumura}{Sriphaew
  et~al.}{2009}]{Sriphaew:Takamura:Okumura:2009}
Sriphaew, K., Takamura, H., \BBA\ Okumura, M. \BBOP 2009\BBCP.
\newblock \BBOQ Cool Blog Classification from Positive and Unlabeled
  Examples.\BBCQ\
\newblock In {\Bem Advances in Knowledge Discovery and Data Mining, 13th
  Pacific-Asia Conference (PAKDD)}, \mbox{\BPGS\ 62--73}.

\bibitem[\protect\BCAY{内元\JBA 丸山\JBA 高梨\JBA 井佐原}{内元 \Jetal
  }{2004}]{CSJ:係り受けマニュアル}
内元清貴\JBA 丸山岳彦\JBA 高梨克也\JBA 井佐原均 \BBOP 2004\BBCP.
\newblock \Jem{『日本語話し言葉コーパス』における係り受け構造付与}.
\newblock
  http://www.kokken.go.jp/katsudo/seika/corpus/public/manuals/dependency.pdf.

\end{thebibliography}



\begin{biography}
\bioauthor{橋本  力}{
1999年福島大学教育学部卒業.2001年北陸先端科学技術大学院大学博士前期課程
修了.2005年神戸松蔭女子学院大学大学院博士後期課程修了.京都大学情報学研
究科産学官連携研究員を経て,2007年山形大学大学院理工学研究科助教,
2009年より独立行政法人情報通信研究機構専攻研究員.
現在に至る.自然言語処理の研究に従事.
博士(言語科学).
情報処理学会,言語処理学会,各会員.
}
\bioauthor{黒橋 禎夫}{
1989年京都大学工学部電気工学第二学科卒業.1994年同大学院博士課程終了.京
都大学工学部助手,京都大学大学院情報学研究科講師,東京大学大学院情報理
工学系研究科助教授を経て,2006年京都大学大学院情報学研究科教授,現在に至
る.自然言語処理,知識情報処理の研究に従事.
言語処理学会,情報処理学会,人工知能学会,電子情報通信学会,
ACL,ACM,各会員.
}
\bioauthor{河原 大輔}{
1997年京都大学工学部電気工学第二学科卒業.1999年同大学院修士課程修了.
2002年同大学院博士課程単位取得認定退学.東京大学大学院情報理工学系研究
科学術研究支援員を経て,2006年より独立行政法人情報通信研究機構研究員.
現在,同主任研究員.自然言語処理,知識処理の研究に従事.博士(情報学).
情報処理学会,言語処理学会,人工知能学会,ACL,各会員.
}
\bioauthor{新里 圭司}{
2006年北陸先端科学技術大学院大学情報科学研究科博士後期課程修了.
博士(情報科学).同年10 月より京都大学大学院情報学研究科特任助教.2009年4月
より特定研究員.2011年4月より楽天技術研究所研究員.自然言語処理の研究に従事.
言語処理学会会員.
}
\bioauthor{永田 昌明}{
1987年京都大学大学院工学研究科修士課程修了.
同年,日本電信電話株式会社入社.
現在,コミュニケーション科学基礎研究所 主幹研究員.
工学博士.統計的自然言語処理の研究に従事.
電子情報通信学会,情報処理学会,人工知能学会,言語処理学会,ACL 各会員.
}

\end{biography}


\biodate




\end{document}
