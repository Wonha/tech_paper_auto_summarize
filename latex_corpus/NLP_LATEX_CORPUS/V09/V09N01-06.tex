



\documentstyle[epsf,jnlpbbl]{jnlp_j}

\setcounter{page}{117}
\setcounter{巻数}{9}
\setcounter{号数}{1}
\setcounter{年}{2002}
\setcounter{月}{1}
\受付{2001}{6}{4}
\再受付{2001}{8}{3}
\採録{2001}{9}{28}

\setcounter{secnumdepth}{2}

\title{日英機械翻訳のための日本語抽象名詞の文法的・意味的用法の分類}
\author{池原 悟\affiref{Tottori} \and 村上 仁一\affiref{Tottori}
 \and 車井 登\affiref{Fujitsu}}

\headauthor{池原 悟, 村上 仁一, 車井 登}
\headtitle{日英機械翻訳のための日本語抽象名詞の文法的・意味的用法の分類}

\affilabel{Tottori}{鳥取大学工学部知能情報工学科}{Department of
Information and Knowledge Engineering, Faculty of Engineering, Tottori
University}
\affilabel{Fujitsu}{富士通株式会社}{Fujitsu Corporation}
\jabstract{
日英機械翻訳において,翻訳が難しいと見られる抽象名詞,「の」,「こと」,
「もの」,「ところ」,「とき」,「わけ」の6種類を対象に,文法的用法と意味
的用法を分類し,英語表現との対応関係を検討した.具体的には,名詞「の」は,
意味的に他の抽象名詞に置き換えられる場合(交替現象)の多いことに着目して,
置き換え先となる抽象名詞の種類と置き換え可能となる条件について検討した.次
に,置き換え後の5種類の抽象名詞の用法を「語彙的意味の用法」,「文法的意味
の用法」に分け,このうち,「文法的意味の用法」を,さらに,「補助動詞的用法」
と「非補助動詞的用法」に分類した.さらに,これらの分類を詳細化し,英語表現
との対応関係を「日英対応表」にまとめた.交替現象の解析精度と「日英対応表」
の精度を調べるため,新聞記事から抽出した抽象名詞の用例に適用した結果では,
「の」の交替現象の解析精度は,97%,「日英対応表」の平均カバー率は89%,平均
正解率は73%であった.
}

\jkeywords{日英機械翻訳,抽象名詞,形式名詞,翻訳規則,交替現象,文法属
性,意味属性,意味解析,訳語選択,語彙的意味,文法的意味}

\etitle{Classification of Syntactic and Semantic Usage of Japanese \\
Abstract Nouns and Their Translations}
\eauthor{Satoru Ikehara \affiref{Tottori} \and Jin'ichi Murakami \affiref{Tottori}
\and Noboru Kurumai \affiref{Fujitsu}}

\eabstract{
Translation of highly abstract nouns has been one of the most
difficult problems in Japanese to English machine translation. In order
to develop the translation rules, meanings and usage of typical Japanese
abstract nouns of "no", "koto", "mono", "tokoro", "toki", "wake" were
studied and syntactic and semantic usage were classified.  First, taking
notice of the semantic equivalency of "no" and other abstract nouns,
exchange rules for noun "no" was studied. Next, semantic usage and
syntactic usage of these abstract nouns were analyzed and translations
for them were summalized. \\
The results were applied to 741 expressions extracted from newspaper
and translation quality was evaluated. The results showed that the
quality for exchange rules was 96.9% and the classification accuracy of
5 abstract nouns was 73% in average. The results are very useful to
improve the translation quality of abstract nouns.
}

\ekeywords{Japanese to English Machine Translation, Translation Rule,
Abstract Noun, Syntactic Attribute, Semantic Attribute, Word Selection,
Word Sense Disambiguation, Syntactic Usage, Semantic Usage}

\begin{document}
\maketitle


\section{まえがき}


自然言語処理の最大の問題点は,言語表現の構造と意味の多様性にある.機械翻
訳の品質に関する分析結果(麻野間ほか 1999)によれば,従来の機械翻訳において,
期待されるほどの翻訳品質が得られない最大の原因は,第1に,動詞や名詞に対す
る訳語選択が適切でないこと,第2に,文の構造が正しく解析できないことである
と言われている.

ところで,日本語表現で,訳語選択と文の構造解析を共に難しくしている問題の
一つとして,「もの」,「こと」,「の」などの抽象名詞の意味と用法の問題があ
る.抽象名詞は,高度に抽象化された実体概念を表す言葉で,話者が,対象を具体
的な名詞で表現できないような場合や明確にしたくないような場合にも使用される
傾向を持ち,その意味と用法は多彩である.そのため,従来の機械翻訳において,
これらの抽象名詞が適切に訳される例は,むしろ少ない.学校文法では,これらの
語の一部を形式名詞と呼んでいるが,これは,それらの語が,実体概念を表すとい
う名詞本来の機能を越えて,対象に対して話者の抱いた微妙なニュアンスを伝える
ような機能を持ち,文法上,他の名詞とは異なる用法を有することを意味している.

訳語選択の観点から見ると,従来,動詞の訳し分けでは,結合価文法が有効であ
ることが知られており,大規模な結合価パターン辞書(池原ほか 1997)が開発さ
れたことによって,その翻訳精度は大幅に向上した.これに対して,名詞の訳し分
けの研究としては,結合価文法で定義された名詞の意味属性を用いることの有効性
を検証した研究(桐澤ほか 1997)や形容詞に修飾された名詞についての訳し分けな
どがあるが,動詞の場合に比べて得られる効果は小さい.名詞は動詞に比べてその
種類も多く意味が多彩である(笠原ほか1997).なかでも,抽象名詞は本来の名詞
としての機能のほか,文法的にも多彩な機能を持つため,個別に検討する必要があ
ると考えられる.

従来の抽象名詞の研究としては,形式名詞「もの」の語彙的意味と文法的意味の
連続性を明らかにする目的で,これを他の抽象名詞「こと」と「ところ」を対比し
た研究(佐々ほか 1997)がある.また,抽象名詞「こと」が,「名詞+の+こと」
の形式で使用された場合を対象に,「こと」が意味的に省略可能であるか否かを述
語の種類によって判定する研究(笹栗,金城 1998)等もある.しかし,これらの研
究では,文中での意味的役割については検討されておらず,従って,また,英語表
現との対応関係も明らかでない.

そこで,本検討では,抽象度の高い6種類の名詞「の」,「こと」,「もの」,
「ところ」,「とき」,「わけ」を対象に,文法的,意味的用法を分類し,英語表
現との対応関係を調べる.このうち,名詞「の」は,多くの場合,その意味を変え
ることなくより抽象度の低い名詞「こと」,「もの」,「ところ」,「とき」,
「わけ」,「ひと」に置き換えられることが知られている.これに着目して,本稿
では,以下の2段階に分けて検討を行う.

まず,単語「の」を対象に,それが,抽象名詞であるか否かを判定するための条
件を示し,抽象名詞である場合について,他のどの抽象名詞に交替可能であるかを
判定する方法を検討する.次に,5種類の抽象名詞「こと」,「もの」,「ところ」,
「とき」,「わけ」を対象に,文中での役割に着目して,用法を「語彙的意味の用
法」と「文法的意味の用法」に分け,「文法的意味の用法」をさらに,「補助動詞
的用法」と「非補助動詞的用法」に分類する.その後,表現形式と意味の違いに着
目して,文法的,意味的用法と英語表現形式との対応表を作成する.また,得られ
た対応表を新聞記事の標本データに適用し,その適用範囲と適用精度を評価する.



\section{抽象名詞とその用法について}

\subsection{抽象名詞の概念について}

従来の文法では,物理的な対象概念を表す言葉を「具象名詞」と称するのに対し
て,観念的な実体概念を表す言葉を「抽象名詞」と呼んでいる。これに対して,本
検討では,具体から抽象へ向かう実体概念の把握において,高度に抽象化された実
体概念を表現する言葉を「抽象名詞」と呼び
\footnote{学校文法では、文法的機能を説明する立場から、名詞「の」、「もの」、
「こと」を「形式名詞」と読んでいるが、「ところ」、「とき」、「わけ」などは
「形式名詞」と呼ばれていない。これに対して、本研究では、名詞の表す対象概念
の抽象度に着目して、これらの名詞も合わせて「抽象名詞」と呼ぶ},
対象が物理的であるか観念的であるかの区別はしない.

認識論的な品詞分類の観点(時枝1950)から見れば,名詞は,個別的な実体をあ
る種類に属するものとして普遍的に把握し,その種類の持つ特殊性に応じて表現す
るための言葉だとされる.一般に,言語表現では,万人に共通する対象のあり方が
そのまま表現されているわけではなく,対象のあり方が話者の認識を通して表現さ
れている.そのため,同一の対象を表現する場合でも,人と場合によって認識の仕
方は異なり,それに応じた言葉が使用される.

例えば,名詞「魚」は,「虫」や「鳥」と区別して対象を表すときに使用される
が,これらを区別する必要のない時は,より抽象度の高い言葉として「動物」が使
用される.「生物」と「無生物」の区別を必要としないときは,さらに抽象度の高
い「もの」が使用される.また,行動を固定してとらえ,客体化した概念を表現す
る名詞として「働き」,「眠り」などが使用されるが,これらを区別しないときは,
より抽象的な「こと」が使用される.このように,言語表現では,話者の対象に対
する個別性と普遍性の認識に適合する抽象度の言葉が使用される.

ここで,名詞「もの」と名詞「こと」は,それぞれ物理的実体と観念的実体を代
表する最も抽象度の高い名詞であるが,さらに抽象度の高い言葉として名詞「の」
がある.「の」は,両者の区別を必要としない場合に使われることから,日本語に
おいて最も抽象度の高い言葉と言える.また,逆に,「もの」,「こと」より,若
干,抽象度の低い名詞としては,「ところ」,「とき」,「わけ」,「あいだ」,
「ばかり」,「ほど」等がある
\footnote{「日本語語彙体系」(池原ほか1997)では、このような名詞の表す概念
の抽象度の関係が、名詞意味属性の包含関係(is-a関係)として、12段階の木構造に
整理されている}.

ところで,名詞「の」は,機械翻訳において最も翻訳が困難な名詞の一つである
が,文脈によって,他の抽象名詞「もの」,「こと」,「ところ」,「とき」,
「わけ」,「ひと」に置き換えられる(交替現象)から,名詞「の」を翻訳するに
は,置き換えられた抽象名詞の翻訳方法に従えばよいと考えられる.そこで,本検
討では,「の」の交替現象の解析方法と交替後の抽象名詞の翻訳方法について検討
する.但し,置き換え先となる抽象名詞のうち,「ひと」の翻訳方法は比較的単純
であるため検討対象外とする.

\subsection{抽象名詞の用法}

{\bf (1)抽象名詞「の」の対象範囲について}

抽象名詞の中でも,「の」の用法は多彩であり,従来,その意味と機能の面から
国語学者によってさまざまな解釈がされている(大野, 柴田1976).本検討では,
言語過程説の立場(時枝1941; 三浦1967, 1975)から提案された抽象名詞「の」の
定義(宮崎ほか 1995)に従う.従来の学校文法との違いは,表1の通りである.

\begin{table}[htbp]
\caption{抽象名詞「の」範囲}
\begin{center}
\begin{tabular}{|c|c|} \hline
学校文法での解釈 & 本検討での扱い\\ \hline
形式名詞「の」 & \\
準体助詞 「の」& 抽象名詞\\
終助詞「の」 & \\ \hline
接続助詞「ので」 & 抽象名詞「の」+格助詞「で」\\
 & 抽象名詞「の」+肯定判断の助動詞「だ」の連体形「で」\\ \hline
接続助詞「のに」 & 抽象名詞「の」+格助詞「に」\\ \hline
終助詞「のだ」 & 抽象名詞「の」+肯定判断の助動詞「だ」\\ \hline
\end{tabular}
\end{center}
\end{table}

{\bf (2)抽象名詞の用法分類}

英語でも日本語と同様,抽象度に応じて対象概念を表現する名詞があり,本検討
で対象とする日本語抽象名詞には,おおよそ以下の英語が対応する.

 「の」:one,thing, 「もの」:thing,  「こと」:thing,matter,
affair,

 「ところ」:place,  「わけ」:reason,  「とき」:time

しかし,これらの単語の場合,その表す概念は必ずしも一致しないだけでなく,
本来の意味から転じてさまざまな用法が発達しているため,その訳し方は単純でな
い.例えば,下記の例文1では,「もの」に対する名詞訳語は存在せず,過去の習
慣の意味で,連語"used to"によって訳されている.また,例文2は,「とき」に
係っている節をまとめあげて後置する節へ接続する機能を持ち,英語表現では接続
詞"when"に訳されている.

\vspace{6pt}

例文1:子供の時はいつも学校へ歩いて行った\underline{もの}です.

    As a child, I \underline{used} always to walk to school. 

例文2:桜のきれいな\underline{とき}に日本に来たい.

    I would like to come to Japan \underline{when} the cherry blossoms are
  pretty.

\vspace{6pt}

そこで,本検討では,抽象名詞が実体概念を表すという名詞本来の意味で使用さ
れている場合(「語彙的意味の用法」と呼ぶ)と,本来の意味を失って,文法的な
機能を持つ言葉として使用されている場合(「文法的意味の用法」と呼ぶ)を区別
する.

ここで,「語彙的意味の用法」とは,例文3のような場合である.この文では,
抽象名詞「ところ」は本来の意味で"place"に訳されている.また,「文法的意味
の用法」とは,例文4のような場合である.抽象名詞は語彙的な意味が薄れ,接続
助詞や助動詞のように,文法的な意味を有する機能語として使用される.

\vspace{6pt}

例文3:分からない\underline{ところ}は兄が教えてくれました.

    My elder brother explained to me \underline{the places} I didn't
understand.

例文4:私はその話はいくらか聞いた\underline{こと}がある.

    I \underline{have} heard something about that subject.

\vspace{6pt}

この2つの用法の違いは「文法化(grammaticalization)」として説明される
(Hopper and Traugott 1993).ここで,文法化とは,一般に語彙的意味を持つ語が
機能語へ変化していくプロセスのことである.

次に,「文法的意味の用法」を「補助動詞的用法」と「非補助動詞的用法」に分
類する.「補助動詞的用法」は,例文4のような場合で,機能動詞を伴った文末表
現として訳されることが多いのに対して,「非補助動詞的用法」は,例文5のよう
な場合で,主に接続表現として訳される.

\vspace{6pt}

例文5:電話をする\underline{とき}は電話番号をよく確かめてから,かけなさ
い.

    \underline{When} you are telephoning, make sure of the number before
phoning.

\vspace{6pt}

以上から,本検討では,抽象名詞の用法を図1の通り分類する.

\begin{figure}
\begin{center}
\atari(120,40)
\caption{抽象名詞の分類}
\end{center}
\end{figure}



\section{抽象名詞「の」の交替現象の解析}

\subsection{抽象名詞「の」の交替現象について}

本検討で対象とする6つの抽象名詞の中で,最も抽象度の高い名詞は,「の」で
ある.この名詞は,多くの場合,意味を変えることなく,他の抽象名詞に置き換え
ることができる
\footnote{もちろん、抽象名詞である以上、「こと」、「もの」、「ところ」、
「とき」、「わけ」も、他のより具体性のある名詞に置き換え可能な場合は多いが、
ここでは、それ以上の置き換えは考えない}.
置き換え可能(「交替可能」)な場合と置き換え不能(「交替不能」)な場合の例
を以下に示す.

\vspace{6pt}

{\bf [交替可能な例]}

例文6:年内に景気が回復する\underline{の(こと)}は,極めてむずかしい.

例文7:日本料理はなんでも食べられるが,一番好きな\underline{の(もの)}は
スキヤキです.

例文8:昨日,彼女がカバンを持ってバスに乗り込む\underline{の(ところ)}を
見た.

例文9:この前お会いした\underline{の(とき)}は,いつでしたか.

例文10:彼の成績が悪い\underline{の(わけ)}は,しょっちゅう学校を休んでば
かりいるからだ.

\vspace{6pt}

{\bf [交替不能な例]}

例文11:見つかるといけない\underline{の}で,彼は隠れた.

例文12:彼は勉強した\underline{の}に,試験に落ちた.

例文13:きみは切符を買う必要はなかった\underline{の}だ.

そこで,抽象名詞「の」の機械翻訳では,図2に示すように,他の抽象名詞に交
替可能か否かを判定し,交替可能な場合は,交替後の抽象名詞として翻訳するもの
とする.

\vspace{6pt}

\begin{figure}
\begin{center}
\atari(120,70)
\caption{抽象名詞「の」の翻訳手順}
\end{center}
\end{figure}


\subsection{交替現象の解析規則}

抽象名詞には,指示代名詞と類似した用法があり,一度文中に出現した対象を改
めて取り上げるために使用される場合がある.また,逆に,初めに抽象名詞で表現
した対象が,後でより具体的な名詞で表現されることもある.このような場合は,
抽象名詞に対して,置き換えることのできる名詞が存在する.そこで,新聞記事
\footnote{大手新聞社数社の記事から、解説記事、政治、経済、社会面等の記事を
1,000文づつ(合計10,000文)取り出して、対訳を付与したもの}
や小説(新潮社1995),日英機械翻訳機能試験文集
\footnote{英訳しにくい日本語表現(6,200文)に対訳を付け、機械翻訳の機能試験
用にまとめたもの(池原1994)}
,日本語問題集(名柄監修1987)から得られた抽象名詞「の」の用例を対象に,後
置する単語の種類,文型,係り受け関係などに着目して,置き換えを可能とする条
件と置き換え先の抽象名詞を決定する方法について検討した.その結果を表2に示
す.

\begin{table}[htbp]
\caption{「の」の交替現象の解析規則}
\begin{center}
\begin{tabular}{|c|c|c|c|c|c|} \hline
{\verb+#+} & 区分 & \multicolumn{3}{|c|}{「の」の交替現象の判定条件} & 判定結果\\ \hline
 & & \multicolumn{2}{|c|}{後置単語による判定} & 助動詞「だ」の場合 & 格助詞「の」
 +助動詞「だ」\\ \cline{5-6}
 & & \multicolumn{2}{|c|}{} & 接続助詞「に」の場合 & 接続助詞「のに」\\ \cline{5-6}
1 & 交替不能 & \multicolumn{2}{|c|}{「の」に後置する単語が} & 接続助詞「で」の場
 合 & 接続助詞「ので」\\ \cline{5-6}
 & の判定 & \multicolumn{2}{|c|}{助動詞,接続助詞,} & 終助詞「か」の場合 & 終
 助詞「のか」\\ \cline{5-6}
 & & \multicolumn{2}{|c|}{終助詞の場合} & 終助詞「よ」の場合 & 終助詞「のよ」
 \\ \cline{5-6}
 & & \multicolumn{2}{|c|}{} & 終助詞「ね」の場合 & 終助詞「のね」\\ \hline
2 & & 前置単語による判定 & \multicolumn{2}{|c|}{形容詞が単独で係っている
 場合} & 「もの」に置換\\ \cline{1-1} \cline{3-6}
3 & & 文型による判定 & \multicolumn{2}{|c|}{「〜するのは,〜からだ」の文
 型の場合} & 「わけ」に置換\\ \cline{1-1} \cline{3-6}
4 & 交替可能 & & 係り先が & 「見る」「発見する」の場合 & 「ところ」に置換
 \\ \cline{1-1} \cline{5-6}
5 & の判定 & & 動詞の場合 & その他の用言の場合 & 「もの」に置換\\ \cline{1-1} \cline{4-6}
 & & 係り先による判定 & & <時詞>の場合 & 「とき」に置換\\ \cline{5-6}
6 & & & 係り先が & <{\verb+#+}4人>の場合 & 「ひと」に置換\\ \cline{5-6}
 & & & 名詞の場合 & <{\verb+#+}1000抽象>の場合 & 「こと」に置換\\ \cline{5-6}
 & & & & その他の名詞の場合 & 「もの」に置換\\ \hline
\end{tabular}
\end{center}
\end{table}

表で,<時詞>の形式の表記は,名詞の文法的属性を表し,<{\verb+#+}4人>,<{\verb+#+}1000抽象
>の表記は,「日本語語彙大系」で定義された「意味属性体系」上の一般名詞意味
属性を表す
\footnote{本検討では名詞の文法的属性として、「言語過程説に基づく日本語品詞の体系化とその効用」(宮崎ほか1995)の品詞体系を使用
し、意味属性体系として「日本語語彙体系」(池原ほか1997)に掲載されている一般
名詞意味属性体系を用いる。一般名詞意味属性体系は、約40万語の一般名詞の意味的用
法を2,710のカテゴリに分類し、意味属性としてラベル付けされている}
.各規則は番号順に適用される
\footnote{文法的属性と意味的属性は形態素解析の結果を使用する。また、係り受
け関係などの構文情報は構文解析結果を使用する}
.以下,本節では,各規則の概要を説明する.

{\bf (1)交替不能の判定}

交替現象の解析では,まず,抽象名詞「の」が,他の抽象名詞に交替可能か否か
を判定する.表1の{\verb+#+}1は,そのための規則である.学校文法で,接続助詞とされ
る「ので」,「のに」,および,終助詞とされる「のだ」に相当する表現は,高度
に文法化した用法であり,他の抽象名詞に置き換えることができない(例文
11-13参照).これらの用法に該当するか否かについては,「の」に後置する単語
の品詞を調べることによって判定することとし,判定後は,表に示されたような解
釈とする.

なお,接続助詞的用法である「ので」,「のに」については,順接複文を構成す
る「ので」と逆接複文を構成する「のに」の研究 (西沢ほか 1995)など,多くの先
行研究があるため,以下では,対象外とする.

規則{\verb+#+}1で交替不能と判定されなかった場合は,以下に述べるように,{\verb+#+}2以降
の規則を順に適用し,置き換え先の抽象名詞を決定する.

\vspace{6pt}

{\bf (2)後置単語による判定}

 前置単語を調べる.前置単語が形容詞(形容動詞を含む)で,節を構成せずに名
詞「の」に単独で係っている場合は,「もの」に交替させる(例文7参照).そのた
めの規則が,表2の{\verb+#+}2である.

\vspace{6pt}

{\bf (3)文型による判定}

 抽象名詞「の」の使用された文型を調べる.文型が,「〜する(な)のは,〜か
らだ」の形式を持つときは,{\verb+#+}3の規則によって,「わけ」に交替させる(例文
10参照).

\vspace{6pt}

{\bf (4)係り先による判定(動詞に係る場合)}

係り先を調べる.係り先が,知覚動詞の場合は,「ところ」に交替させる.「と
ころ」には空間的な位置や場所を表す場合,抽象的な事柄についての位置や場面を
表す場合,時間的な断面を表す場合などがある.例文8は,場面を表している.

係り先となる知覚動詞としては,「見る」,「発見する」が代表的であるので,
規則{\verb+#+}4では,これを規則に使用した.これに対して,抽象名詞「の」の係り先が,
その他の動詞である場合は,規則{\verb+#+}5によって「もの」に交替させる.

\vspace{6pt}

{\bf (5)係り先による判定(名詞に係る場合)}

最後に,係り先が名詞であるかどうかを調べる.すでに述べたように概念化の過
程で,抽象化が進むと,殆どの名詞は,「こと」,「もの」,「ところ」,「とき」,
「わけ」,「ひと」の何れかに縮退されるが,どの名詞に縮退されるかは,名詞の
文法的・意味的範疇から判断できる.ここでは,名詞の文法的属性と意味的属性の
体系を使用して判断することとし,交替先の抽象名詞を規則{\verb+#+}6のよう
に定める.

従って,例えば,例文14では,抽象名詞「の」は,「灯台の明かり」に係ってお
り,「灯台の明かり」は,物理的実体概念として意味属性<{\verb+#+}1具体>の配下にあ
るため,「もの」に交替させる.

\vspace{6pt}

 例文14:遠くの方でぴかぴか光っている\underline{の(もの)}は,夜の海を照らす灯台の明か
りです.


\section{抽象名詞の用法と英語表現の対応関係}

すでに述べたように,抽象名詞「の」では,あらかじめ交替現象の解析によって
交替先の抽象名詞を決定するものとした.そこで,本章では,交替先となる「こと」,
「もの」,「ところ」,「とき」,「わけ」の5種類の抽象名詞の用法と英語表現
との関係を検討する.

具体的には,3章で使用した文献と同一の文献
\footnote{新聞記事10,000文、新潮文庫、翻訳機能試験文集、日本語問題集の4種
}
からこれらの抽象名詞の用例を
収集し,その用法を第2章で述べたように,「語彙的意味の用法」,「補助動詞的
用法」,「非補助動詞的用法」に分類した後,用法と意味を,その前後に現れる語
に着目して細分類し,対応する英語表現を表3(「日英対応表」)にまとめた。以下,
各抽象名詞の用法と英語表現との対応表を示す.

\subsection{「こと」の用法と英語表現}

「こと」は係っている語によって,「語彙的意味の用法」と「文法的意味の用法」
に分類する.具体的には,係っている語が節を成していないか,一語だけであると
き,「語彙的意味の用法」とし,それ以外を「文法的意味の用法」とする.

\begin{table}[htbp]
\caption{日英対応表(「こと」の用法と英語表現)}
\begin{center}
\begin{tabular}{|c|c|c|c|c|c|} \hline
区 分 & \multicolumn{3}{|c|}{表現形式} & 意 味 & 英語表現\\ \hline
 & \multicolumn{3}{|c|}{名詞+の+こと} & 内容 & about+名詞\\ \cline{5-6}
語彙的意味の用法 & \multicolumn{3}{|c|}{} & − & 名詞のみの訳語\\ \cline{2-6}
 & \multicolumn{2}{|c|}{連体詞+コト} & あの,あんな & 指示 & that\\
 \cline{4-4} \cline{6-6}
 & \multicolumn{2}{|c|}{} & この,こんな & & this\\ \cline{4-4}
 \cline{6-6}
 & \multicolumn{2}{|c|}{} & どんな & & what\\ \cline{2-6}
 & \multicolumn{3}{|c|}{形容詞+こと} & − & 形容詞の訳語\\ \cline{2-6}
 & \multicolumn{3}{|c|}{形容動詞+こと} & − & 形容動詞+matter\\ \cline{2-6}
 & \multicolumn{3}{|c|}{動詞+こと} & 動作 & to+動詞,動詞ing\\ \hline
 & \multicolumn{3}{|c|}{ことになる} & 予定 & will\\ \cline{5-6}
補助動詞的用法 & \multicolumn{3}{|c|}{} & 事象の成立 & decide\\ \cline{2-6}
 & \multicolumn{3}{|c|}{ことができる} & 可能 & can(be able to)\\ \cline{2-6}
 & ことにする & \multicolumn{2}{|c|}{したことにする} & 偽りの事象 &
 assume\\ \cline{3-6}
 & & \multicolumn{2}{|c|}{することにする} & 決定 & decide to\\ \cline{2-6}
 & \multicolumn{3}{|c|}{ことが多い} & 短期間の反復 & frenquently\\ \cline{5-6}
 & \multicolumn{3}{|c|}{} & 傾向 & tend to\\ \cline{5-6}
 & \multicolumn{3}{|c|}{} & 通例 & usually\\ \cline{5-6}
 & \multicolumn{3}{|c|}{} & 頻度 & often\\ \cline{2-6}
 & ことがある & \multicolumn{2}{|c|}{したことがある} & 過去の経験 & have
 been\\ \cline{5-6}
 & & \multicolumn{2}{|c|}{} & 過去の習慣 & used to\\ \cline{3-6}
 & & \multicolumn{2}{|c|}{することがある} & 可能性 & there are times\\ \cline{5-6}
 & & \multicolumn{2}{|c|}{} & 頻度 & sometimes\\ \cline{5-6}
 & & \multicolumn{2}{|c|}{} & 頻度 & frequently\\ \cline{3-6}
 & & \multicolumn{2}{|c|}{したいことがある} & 要求の存在 & there is one
 thing\\ \cline{5-6}
 & & \multicolumn{2}{|c|}{} & 要求の存在 & have a favor\\
 & & \multicolumn{2}{|c|}{} & (依頼の動詞を伴う) & \\ \cline{5-6}
 & & \multicolumn{2}{|c|}{} & 要求の存在 & have something\\
 & & \multicolumn{2}{|c|}{} & (内容が不明瞭) & \\ \hline
 & \multicolumn{2}{|c|}{{\it X}+こと} & 主格無し &  & {\it V} to {\it X}\\ \cline{4-4} \cline{6-6}
非補助動詞的用法 & \multicolumn{2}{|c|}{+格助詞+動詞} & 主格有り & &
 {\it V}
 that {\it X}\\ \cline{2-4} \cline{6-6}
 & \multicolumn{2}{|c|}{{\it X1} は{\it X2} だ} & 主格有り & & It is 〜that\\
 \cline{4-4} \cline{6-6}
 & \multicolumn{2}{|c|}{} & 主格無し & & It is 〜to\\ \cline{2-4} \cline{6-6}
 & \multicolumn{3}{|c|}{{\it X}+こと+から} & & because 〜\\ \cline{2-4} \cline{6-6}
 & \multicolumn{3}{|c|}{{\it X}+こと+で} & & with 〜\\ \hline
\end{tabular}
\end{center}
\end{table}

次に,「こと」が機能動詞を伴って現れる場合を「補助動詞的用法」とし,それ
以外を「非補助動詞的用法」とする.それぞれの詳細と英語表現との対応関係を表
3に示す.

ここで,表中の最上欄の「名詞+の+こと」に該当する例文を示す.

\vspace{6pt}

例文15:私は彼の\underline{こと}を知らない.

    I don't know \underline{about} him.

例文16:彼女は彼の\underline{こと}が嫌いだ.

    She does not like him.

\vspace{6pt}

これらの例文には,いずれも「彼のこと」という表現が含まれている.このうち,
例文15は,「彼」についての属性(年齢や職業などの「彼」にまつわる事柄)を表し,
英語では,''about him''が用いられている.これに対して,例文16では,「彼」
そのものの意味で,英語では,単に,''him''が使用されている.

また,例文17は,第3章の交替現象解析の規則が適用される例で,「の」が用言
に係っていることから「こと」に交替した後,表3が適用され,"that"が使用され
る.

\vspace{6pt}

例文17:その場合,環境問題は多様であり,国,企業,専門家,非政府組織
(NGO)など多様な層で構成する\underline{の}が望ましい.<訳語:「こと」に交替後 that>

\subsection{「もの」,「ところ」,「とき」,「わけ」の用法と英語表現}

「もの」では,文中の「もの」が用言の格要素の名詞として用いられ,かつ表4
で示された「補助動詞的用法」と「非補助動詞的用法」の表現形式以外の場合を
「語彙的意味の用法」とする.この分類方法は,「ところ」,「とき」,「わけ」
も同様である.以上4種類の抽象名詞の用法と英語表現の対応を表4〜表7に示す.

ここで,「もの」(表4),「ところ」(表5)の規則では,語彙的意味の用法
の場合,表層的な表現形式から英語表現を決定することは困難であり,「もの」も
しくは「ところ」に相当する名詞を文脈によって特定し,その意味を調べる必要が
ある点に注意が必要である。ところで,語彙的意味の用法では,抽象名詞が,埋め
込み修飾節に対して「内の関係」で使用されているから,結合価パターン(池原ほ
か1997)を使用した埋め込み文解析が適用できて,先行詞である名詞の文中での意
味属性(意味的用法)が決定できる。そこで,得られた名詞の意味属性から対応す
る表の意味を決定する。

以上の日英対応表の適用例を以下に示す.

\vspace{6pt}

例文18:名詞はこんなに信用させる\underline{もの}なのに簡単に作ることができる.<訳語:
thing>

例文19:あと数十センチでぶつかる\underline{ところ}だったが,赴任して二週間,危ない目
にあったのは,これが初めてではない.<訳語:be about to>

\vspace{6pt}

例文18は「もの」が語彙的意味で用いられ,訳語は"thing"が使用される.例文
19は「ところ」が補助動詞的用法で用いられ,動作の直前を表す''be about
to''が訳語候補となる.

\begin{table}[htbp]
\caption{日英対応表(「もの」の用法と英語表現)}
\begin{center}
\begin{tabular}{|c|c|c|c|} \hline
区 分 & 表現形式 & 意 味 & 英語表現\\ \hline
 & & 有形の物 & thing\\ \cline{3-4}
語彙的意味の用法 & & 品物 & article\\ \cline{3-4}
 & & 物質 & goods\\ \cline{3-4}
 & & 材料 & material\\ \cline{3-4}
 & & 資源 & resource\\ \cline{3-4}
 & & 所有物 & possession\\ \cline{3-4}
 & & 物理 & thing, matter\\ \cline{3-4}
 & & 道理 & reason\\ \cline{3-4}
 & & 者 & person\\ \hline
 & ものだ & 当然の帰結 & \\ \cline{2-3}
補助動詞的用法 & ものとする & 決断 & なし\\ \cline{2-3}
 & ものである & 断定 & \\ \cline{2-4}
 & 〜もの. & − & it be -ed\\ \cline{2-4}
 & ようなものだ(主語が人) & 比喩 & should\\ \cline{2-2} \cline{4-4}
 & ようなものだ(主語が人以外) &  & like\\ \hline
 & もので〜 & 利用 & by -ing\\ \cline{3-4}
非補助動詞的用法 &  & 前述 & −\\ \cline{2-4}
 & 〜ものと〜 & − & that\\ \cline{2-4}
 & ものの & 逆接 & thought, but\\ \hline
\end{tabular}
\end{center}
\end{table}

\begin{table}[htbp]
\caption{日英対応表(「ところ」の用法と英語表現)}
\begin{center}
\begin{tabular}{|c|c|c|c|} \hline
区 分 & 表現形式 & 意 味 & 英語表現\\ \hline
 & & 場所 & place\\ \cline{3-4}
語彙的意味の用法 &  & 住所 & place\\ \cline{3-4}
 & & 個所 & passage\\ \cline{3-4}
 & & 点 & point\\ \cline{3-4}
 & & 特徴 & feature\\ \cline{3-4}
 & & 時 & time\\ \cline{3-4}
 & & 事 & thing\\ \hline
 & するところだ & 動作の直前 & be about to\\ \cline{2-4}
補助動詞的用法 & したところだ & 動作の直後 & have just -ed\\ \cline{2-4}
 & しているところだ & 動作の進行 & just\\ \cline{2-4}
 & したいところだ & 動作の希望 & なし\\ \cline{2-4}
 & しようとするところだ & 動作の寸前 & just be about to\\ \hline
 & 〜ところ,〜 & & なし\\ \cline{2-2} \cline{4-4}
非補助動詞的用法 & するところに〜 & & just as\\ \cline{2-2} \cline{4-4}
 & しているところを〜 & & in act of -ing\\ \cline{2-2} \cline{4-4}
 & しているところに〜 & & when\\ \hline
\end{tabular}
\end{center}
\end{table}

\begin{table}[htbp]
\caption{日英対応表(「とき」の用法と英語表現)}
\begin{center}
\begin{tabular}{|c|c|c|c|c|} \hline
区 分 & \multicolumn{2}{|c|}{表現形式} & 意 味 & 英語表現\\ \hline
 & & あのとき & & then\\ \cline{3-3} \cline{5-5}
語彙的意味の用法 & 連体詞 & このとき & & (at)this point in time\\ \cline{3-3} \cline{5-5}
 & & そのとき & & (at)that moment\\ \hline
補助動詞的用法 & \multicolumn{2}{|c|}{ときがある} &  時々 &  sometimes\\ \hline
 & \multicolumn{2}{|c|}{動作動詞+するとき} & & when\\ \cline{2-3} \cline{5-5}
非補助動詞的用法 & \multicolumn{2}{|c|}{形容詞+とき} & & when\\ \cline{2-3} \cline{5-5}
 & \multicolumn{2}{|c|}{状態動詞+ているとき} & & while\\ \hline
\end{tabular}
\end{center}
\end{table}

\begin{table}[htbp]
\caption{日英対応表(「わけ」の用法と英語表現)}
\begin{center}
\begin{tabular}{|c|c|c|c|} \hline
区 分 & 表現形式 & 意 味 & 英語表現\\ \hline
語彙的意味の用法 & & 理由 & reason\\ \hline
 & わけだ & 当然 & so, (it is)no wonder\\ \cline{2-4}
補助動詞的用法の用法 & わけがない & 主観的不可能 & can not possibly\\ \cline{2-4}
 & わけではない & 否定の強調 & not\\ \cline{2-4}
 & するわけにはいかない & 感情の否定 & can not, not possible to\\ \cline{2-4}
 & しないわけにはいかない & 義務 & can not stand by without -ing\\
 & & & can not avoid -ing\\
 & & &(否定の動詞が存在する場合はcan not)\\ \cline{2-4}
 & というわけだ & 帰結 & なし\\ \cline{2-4}
 & わけはない & 当然の否定 & certainly\\ \hline
 & わけで & 背景 & and\\ \cline{3-4}
非補助動詞的用法 & & 論拠 & because\\ \cline{2-4}
 & わけなので & 論拠 & because\\ \hline
\end{tabular}
\end{center}
\end{table}

\section{評価と考察}

第3章で示した抽象名詞「の」の交替現象の解析規則と第4章で示した抽象名詞
の個別の日英対応表を新聞記事の用例に適用したときの精度を手作業で評価した.
各対応表を適用するに際しては,評価対象とする文を,あらかじめ,形態素解析プ
ログラム maja(高橋ほか 1993)によって解析して,判定に必要な単語の文法属性と
意味属性を求めると共に,構文解析プログラムALT-JAWSによって得られた構文情
報を使用した.なお,テストに使用する標本は,検討で使用したものとは異なり,
オープンテストである.

\subsection{交替現象の解析規則の評価と考察}
\subsubsection{評価結果}

1995年度の毎日新聞記事から抽象名詞「の」を含む194文を取り出し,第3章の
規則を適用して,交替現象の解析精度を評価した.評価結果を表8に示す.表8は,
交替先である抽象名詞を正しく判断できた割合を示している.

\begin{table}[htbp]
\caption{「の」交替現象解析規則の評価結果}
\begin{center}
\begin{tabular}{|c|c|c|c|c|c|c|c|c|c|} \hline
抽象名詞種別 & 交替 & \multicolumn{7}{|c|}{交 替 可 能} &  合計\\ \cline{3-9}
 & 不能 & こと & もの & ところ &とき & わけ & ひと & 小計 & \\ \hline
正解数 & 120 & 45 & 6 & 3 & 3 & 6 & 5 & 68 & 188\\ \hline
標本数 & 121 & 46 & 8 & 3 & 4 & 7 & 5 & 73 & 194\\ \hline
正解率 & 99% & 98% & 75% & 100% & 75% & 86% &  100% & 93% & 96.9%\\ \hline
\end{tabular}
\end{center}
\end{table}

 表から,以下のことが分かる.

\vspace{6pt}

(1)交替不能の欄のデータから,交替可否の判定では,全標本194件中,192件が正
しく判定され,高い判定精度(99%)が得られたことが分かる.

(2)交替不能を含む全体の解析精度は,96.9%,また,交替可能と判定されたもの
のうち,交替先の名詞が正しく決定されたのは,93%で,いずれも良い精度と言え
る.

\subsubsection{考察}

不正解の表現には,意味辞書を変更すれば正解となる表現や動詞の情報が役立ち
そうな表現などがあったが,本手法では解析が困難な表現も存在する.そこで,交
替現象の解析規則の改良の可能性について考察する.

(1)短絡的表現への適用性

名詞は,短絡的用法や比喩的な用法などにより,本来の意味の枠を越えて使用さ
れることがある.「の」の係り先がこのような名詞の場合,作成した規則は適用で
きない.

\vspace{6pt}
例文20:その時でさえ飲める\underline{の}は,僅に喉をうるおすに足る少量で
ある.

\vspace{6pt}

例えば,例文20は,「の」の係り先が「少量」という名詞であり,意味属性体系
上では ({\verb+#+}1000抽象)の配下に属するために,本論文の規則では「こと」に交替さ
れるが,正解は「もの」である.これは,「少量の液体」,もしくは,「少量の酒」
などというべきところを短絡的に「少量」と表現していることが原因である.この
ような現象に対応するためには,短絡的表現と比喩に関する解析規則を併用するこ
とが必要と考えられる.

\vspace{6pt}

(2)動詞情報の必要性

抽象名詞「の」の交替先の解析で,係り先が動詞となる場合は,その動詞の意味
をより細かく分類する必要があると考えられる.

\vspace{6pt}

例文21:ショーウィンドウのガラスに映った\underline{の}をチラリと眺めると

\vspace{6pt}

例えば,例文21は「の」の係り先が用言であるために,本規則では,交替先は
「こと」と判断されるが,正解は「もの」である.ここで,「眺める」に着目する
と「ことを眺める」とは言いがたく,「ものを眺める」であることが分かる.

\vspace{6pt}
(3)文脈情報の必要性

与えられた1文だけでは,交替先を決定できない場合として,下記のような例が
ある.

\vspace{6pt}

例文22:テーブル(が/は)新しい\underline{の}に置き換えられた.

\vspace{6pt}

この例文22は,2通りの解釈が存在する.一つは,「テーブルは,まだ新しいの
に,置き換えられた」のであり,何に置き換えられたのかは不明である.この場合,
「のに」は,接続助詞的な用法となる.もう一つの解釈は,「テーブルは,別の新
しいテーブルに置き換えられた」と言う場合であり,この場合,「の」は,名詞
「テーブル」の代わりに使用された抽象名詞であるから,「に」は,格助詞の解釈
となる.この文では,主題提示の文節で助詞「が」が使用されている場合は,おお
よそ,後者の意味だろうと思われるが,助詞「は」が使用された場合は,前者の解
釈の可能性も増してくる.

いずれにしてもこのように,決め手のないような文では,文脈情報が必要である.

\vspace{1em}

\subsection{日英対応表の評価と考察}

\subsubsection{評価結果}

1995年度の毎日新聞記事より,「こと」,「もの」,「ところ」,「わけ」,
「とき」を含む合計741文を対象に日英対応表を適用し,翻訳結果を求めた.評価
では,原文と該当場所に対して得られた翻訳結果を翻訳家に提示し,以下の3段階
の基準で評価してもらった.

 ○:適切である.  △:意味は通じるが,より良い翻訳方法がある. ×:不
正解

評価結果を表9に示す.

\begin{table}[htbp]
\caption{抽象名詞に対する日英対応表の評価結果}
\begin{center}
\begin{tabular}{|c|c|c|c|c|c|c|c|c|c|} \hline
\multicolumn{3}{|c|}{区 分} & 項 目 & こと & もの & ところ & とき & わけ
 & 小計\\ \hline
\multicolumn{3}{|c|}{} & 標本数 & 187 & 186 & 182 & 90 & 96 & 741\\ \cline{4-10}
\multicolumn{3}{|c|}{カバー率} & 対応表適用数 & 161 & 147 & 158 & 78 &
 85 & 629\\ \cline{4-10}
\multicolumn{3}{|c|}{} & カバー率 & 86.1% & 79.0% & 86.8% & 86.7% &
 90.6% & 85%\\ \hline
 & \multicolumn{2}{|c|}{} & 標本数 & 31 & 53 & 108 & 3 & 0 & 213\\ \cline{4-10}
 & \multicolumn{2}{|c|}{語彙的意味を} & ○+△ & 20 & 31 & 57 & 0 & 0 &
 115\\ \cline{4-10}
 & \multicolumn{2}{|c|}{持つもの} & △ & 4 & 16 & 26 & 0 & 0 & 46\\ \cline{4-10}
 & \multicolumn{2}{|c|}{} & 正解率 & 65% & 58% & 53% & 0% & --- & 54%
 \\ \cline{2-10}
 & & & 標本数 & 34 & 59 & 5 & 1 & 82 & 181\\ \cline{4-10}
正 & & 補助動詞 & ○+△ & 25 & 56 & 5 & 1 & 53 & 140\\ \cline{4-10}
 & 文法的 & 的用法 & △ & 1 & 0 & 2 & 0 & 3 & 6\\ \cline{4-10}
解 & 意味を & & 正解率 & 74% & 95% & 100% & 100% & 65% & 77%\\ \cline{3-10}
 & 持つ & & 標本数 & 96 & 35 & 45 & 74 & 3 & 329\\ \cline{4-10}
率 & もの & 非補助動 & ○+△ & 68 & 35 & 35 & 72 & 2 & 262\\ \cline{4-10}
 & & 詞的用法 & △ & 9 & 0 & 0 & 21 & 0 & 72\\ \cline{4-10}
 & & & 正解率 & 71% & 100% & 78% & 97% & 67% & 80%\\ \cline{2-10}
 & \multicolumn{2}{|c|}{} & ○+△ & 113 & 122 & 97 & 73 & 55 & 460\\ \cline{4-10}
 & \multicolumn{2}{|c|}{合 計} & △ & 14 & 16 & 28 & 21 & 3 & 82\\ \cline{4-10}
 & \multicolumn{2}{|c|}{} & 正解率 & 70% & 83% & 61% & 55% & 63% &
 73.1%\\ \hline
\end{tabular}
\end{center}
\end{table}

この表から以下のことが分かる.

\vspace{6pt}

(1)対応表全体のカバー率は,85%で,比較的広い範囲の表現に適用できる.

(2)5種類の抽象名詞の中で,最大の正解率は「もの」(83%),最低の正解率は,
「とき」(55%)である.また.平均の正解率は,73%である.

(3)用法の違いから見ると,「文法的用法」の場合に比べて,「意味的用法」の場
合の精度が悪い.

\vspace{6pt}

交替現象の解析精度に比較すると,各抽象名詞の翻訳精度は,あまり良いとは言
えないが,従来,この種の名詞の翻訳は,困難な問題の一つであったのに対して,
解決のための糸口が得られた
\footnote{抽象名詞の意味解析は、指示代名詞の照応解析と似た側面があることに
ついては、既に述べたが、この精度は、従来の指示代名詞の照応解析(村田、長尾
1996,1997a,1997b,吉見1997)の解析精度(カバー率77%、適合率79%)と、ほぼ同じ
値である}
.

\subsubsection{失敗例と日英対応表改良の可能性}

失敗した例を見ると,現状で使用している情報の範囲で,改良が可能と思われる
ものもあるが,多くは,本検討で使用しなかった情報を必要としている.以下では,
その例を挙げて,今後の問題点について考察する.

(1)追加登録の可能性

対応表が使用できなかった例の中には,未登録であった表現を登録するか,既に
登録されている表現に対する英語表現を変更することで改善する表現がある.

\vspace{6pt}

例文23:こうした工作が功を奏したのか,今の\underline{ところ},ムーディーズは三行の長
期債を格下げしていない.

例文24:「ビジネスの現場が望んでいるのは,これらの地道な措置」という\underline{わけ}
だ.

\vspace{6pt}

例文23は,「今のところ」が"now"と訳され,語彙的意味である「ところ」の直
接の訳語は現れない.このような決まり文句は,登録することで改善できる.

また,例文24は「というわけだ」という表現を含む文である.本対応表では「と
いうわけだ」は訳語として現れないとしているが,この場合の正解は''the
point is''となっており,誤りと判定されたが,微妙である.意味の強調と考えて,
''the point is''と訳出する方がよいとすれば,「というわけだ」に対する英語表
現を変更することで改善できる.

\vspace{6pt}

(2)類推の必要性

表記レベルの解析では訳語を特定するのが困難な場合で,単語間の意味的関係を
ネットワークとして体系化した新たな概念体系や世界の一般知識を必要とする場合
として,以下のような例がある.

\vspace{6pt}

 例25:特に,第三のグループがかなりの\underline{もの}になって鼎立関係になるのか,キャ
スチングボートを握るようになるのか.

\vspace{6pt}

この文の場合,本稿の対応表では,「もの」が''thing''と訳されるが,正解は,
''force''である.この一文から「もの」の訳語として''force''を導くのは,大変
困難と思われる.「鼎立関係」や「キャスチングボート(を握る)」などの語から連
想する仕掛けが可能かどうかについて,今後の検討する必要がある.

\vspace{6pt}

(3)文脈,常識の必要性

評価データの中には一文だけでは訳語が特定できない表現がある.

\vspace{6pt}

例文26 :青年海外協力隊などを想定した\underline{ところ}が多く,国内の災害を想定した
のは少なかったためだ.

\vspace{6pt}

例文26では,「ところ」が語彙的意味で用いられ,場所の意味で使用されている
という判断から"place"が訳語候補として選択された.しかし,実際には この一文
だけで「ところ」の訳語を決定することは難しい.この場合,「想定した」に対す
る主格の補完機能があれば,格関係の解析から決定できると考えられる.すなわち,
その場合は,底の名詞である抽象名詞が埋め込み文において内の関係であるとき,
抽象名詞の指し示す概念を前後の文から補完し,補完された名詞の種類によって訳
語を決定すればよい.

\vspace{1em}

\section{おわりに}

抽象名詞「の」,「こと」,「もの」,「ところ」,「とき」,「わけ」を対象
に,英語表現を決定するための日英対応表について検討した.具体的には,まず,
「の」について,他の抽象名詞への交替現象に着目して,意味的に置き換え可能で
あるか否かを判定する規則と置き替え可能な場合について置き換え先の名詞を決定
するための規則を作成した.次に,「の」を除く各抽象名詞の用法を「語彙的意味
の用法」と「文法的意味の用法」に分類し,このうち,「文法的意味の用法」を,
さらに,「補助動詞的用法」と「非補助動詞的用法」に分類して,日英対応表を作
成した.

また,作成した日英対応表を新聞記事の例文に適用し,その精度を評価した.そ
の結果によれば,交替現象の解析では,正解率97%の精度が得られたのに対して,
個別の抽象名詞に対する対応表の精度は,平均カバー率89%,平均正解率73%であっ
た.この精度は,あまり高い精度とは言えないが,従来,難問の一つとされてきた
抽象名詞の翻訳において,翻訳方式を決定する上で,有力な手がかりが得られたと
考えられる.

今後は,誤り分析の結果に基づき,動詞の意味情報の詳細化を行うと共に,文脈,
一般常識を使用した推論,補完技術などを組み合わた翻訳方式について検討していき
たい.






\bibliographystyle{jnlpbbl}
\bibliography{ikebib}





\begin{biography}

\biotitle{略歴}
\bioauthor{池原 悟}{
1967年 大阪大学基礎工学部電気工学科卒業.1969年 同大学院修士課程終了.同
年日本電信電話公社に入社.数式処理、トラフィック理論,自然言語処理の研
究に従事.1996年 スタンフォード大学客員教授.現在、鳥取大学工学部教授.
工学博士.1982年 情報処理学会論文賞,1993年 同研究賞,1995年 日本科学技術
情報センタ賞(学術賞),同年人工知能学会論文賞受賞.電子情報通信学会,
人工知能学会,言語処理学会,各会員
}

\bioauthor{村上 仁一}{
1984年 筑波大学第3学群基礎工学類卒. 
1986年 筑波大学修士課程理工学研究科理工学専攻修了.
1996年 NTTに入社.NTT情報通信処理研究所に勤務.
1991年 国際通信基礎研究所(ATR)自動翻訳電話研究所に出向.
1994年 NTT情報通信研究所
1997年 鳥取大学工学部知能情報工学科
現在に至る.
主に音声認識のための言語処理の研究に従事
電子通信情報処理学会,日本音響学会,言語処理学会,各会員.
}

\bioauthor{車井 登}{
1999年3月 鳥取大学工学部知能情報学科卒
1999年4月 鳥取大学大学院工学研究科知能情報工学専攻入学
2000年3月 鳥取大学工学部工学研究科知能情報工学専攻修了
2000年4月 富士通株式会社に入社 ソフトウェア事業本部に所属
現在にいたる.
}
\bioreceived{受付}
\biorevised{再受付}
\bioaccepted{採録}

\end{biography}

\end{document}
