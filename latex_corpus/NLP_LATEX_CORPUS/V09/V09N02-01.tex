



\documentstyle[epsf,jnlpbbl]{jnlp_j}

\setcounter{page}{3}
\setcounter{巻数}{9}
\setcounter{号数}{2}
\setcounter{年}{2002}
\setcounter{月}{4}
\受付{2001}{3}{28}
\再受付{2001}{6}{20}
\再々受付{2001}{9}{3}
\採録{2002}{1}{10}

\setcounter{secnumdepth}{2}

\title{意味グラフのマッチングによる \\
事故問い合わせ文からの判例検索システムJCare}
\author{原田 実\affiref{KUEE} \and 鈴木 亮\affiref{KUEE KUEE} \and 南 旭瑞\affiref{KUEE KUEE}}

\headauthor{原田 実, 鈴木 亮, 南 旭瑞}
\headtitle{意味グラフのマッチングによる
事故問い合わせ文からの判例検索システムJCare}

\affilabel{KUEE}{青山学院大学理工学部情報テクノロジー学科}{Department of Integrated Information Technology, Faculty of Science and Engineering, Aoyama Gakuin University}
\affilabel{KUEE KUEE}{青山学院大学理工学研究科経営工学専攻}{Graduate School of Industrial and System Engineering, Department of Science and Engineering, Aoyama Gakuin University}

\jabstract{
意味解析を用いた情報検索の一手法を提案し,「判例」を検索対象とし日本語文章で記述した「問い合わせ文」を検索質問とした検索システム{\bf JCare}を開発する.本研究では,文章が表す内容を,語が格納されたノードと語間の関係(深層格)を表すアークからなる意味グラフとして捉え,判例文と問い合わせ文の意味グラフ間における位相同型部分の大きさをもとに,文章間の内容類似度を算出する.このとき検索の高速化・精度向上の目的で{\it View}という考え方を導入する.視点({\it View})により意味グラフを分割した{\it View}グラフの類似度を求めることで,内容的に関連性の低い文章間の計算時間,またそこから生まれるノイズを排除する.
}

\jkeywords{情報検索, 意味解析, 深層格, 意味グラフ, グラフマッチング, EDR電子化辞書}

\etitle{Judicial CAse REtriever JCare \\
from Japanese Query Sentences \\
based on Semantic Graph Matching}
\eauthor{Minoru Harada\affiref{KUEE} \and Ryo Suzuki\affiref{KUEE KUEE} \and Akimizu Minami\affiref{KUEE KUEE}} 

\eabstract{
We propose the technique to realize the information retrieval based on the semantic analysis and actually develop the judicial case retrieval system called JCare. It accepts a query written in Japanese sentences and retrieves a judicial case containing the sentences describing the similar situation specified by the query. It first transforms both a judicial case and the query into semantic graphs that have nodes representing the meaning of word and arcs representing the relations (deep case) between the words. Next, it calculates the similarity between the case and the query by searching the maximum common parts that are topologically equivalent. The graph matching is speeded up by separating each semantic graph into sub-graphs based on the ``View'' point about a judicial case.
}

\ekeywords{Information retrieval, Semantic analysis, Deep case, Semantic graph, Graph matching, EDR}

\begin{document}
\maketitle



\section{はじめに}
\subsection{研究背景}
今日ある検索システムは,索引語を用いたキーワード検索が主流となっている.検索漏れを防ぐために,キーワードに指定した語の同意語や関連語も自動的に検索対象にするといった工夫が凝らされているものも幾つか存在する.しかし,一般にキーワードによる絞込みは難しく,検索結果からまさに必要とする情報に絞り込むには,その内容についての説明文などを検索要求と比べる必要があった.例えば,判例検索システムで今担当している事件に似ている状況で起こった過去の事件の判例を調査するとき,当該事件を記述する適切な5つ位のキーワードを指定してand検索をしても,該当して表示される判例数は100件程度になり,この中から当該事件の当事者の関係や諸事実の時間的・因果的関係などが最も類似している事件の判例を人手で探すには大変な労力が必要となる.検索システムが有能な秘書のように,必要な情報の説明を文章で与えるだけで検索対象の要約などの解説文の内容を考慮して最適な情報を掲示してくれると,ユーザの検索労力は大幅に軽減される.この検索を支援する研究のポイントは,2つの文章に記述されている内容の類似性を如何に機械的に計算するかである.本研究の詳細に入る前に,文章の類似性を評価することを要素として含むこれまでの研究についてまず述べることにする.


篠原\cite{sinohara}らは,一文ごとの要約を行う目的で,コーパスから類似した文を検索しこれとの対比において省略可能な格要素を認定する手法を提案している.ここでの文章間の類似性の計算方法は,2文間に共通する述語列を求め,これに係っている格要素について,それらが名詞である場合,その意味属性を元に対応関係を,同一関係,同義関係,類似関係に分け,類似度の算出式を設定し,総合的な文間の類似度を求めている.ただし,ここでは,格が表層格であり,文間の関係や述語間(用言間)の格(時間的順序,論理関係,条件関係など)についての類似性は考慮されていない.
黒橋ら\cite{kurohashi}は,係り受け構造解析における並列構造の範囲の同定において,キー文節前後の文節列同士の類似性を,自立語の一致,自立語の品詞の一致,自立語の意味的類似度,付属語の一致を元に計算し,類似度最大の文節列の組を求める方法を提案している.
宇津呂ら\cite{uturo}は,用例間の類似度を用いて構造化された用例空間中を効率よく探索することにより,全用例探索を行わずに類似用例を高速に検索する手法について提案している.ここでは類似度テンプレートを用いた用例高速化に重きを置いている.この研究においては文章間の類似度を対応する語同士の表層格の対応および格要素の名詞の意味カテゴリの類似度をもとに計算している.
兵藤ら\cite{hyoudo}は,表層的情報のみを用いて安定的かつ高精度に構文解析を行う骨格構造解析を用いて辞典の8万用例文について構文付きコーパスを作成し,これを対象として類似用例文検索システムを構築している.ここでの類似用例文検索では,入力された検索対象文を構文解析し,自立語,意味分類コード,機能語を対象とした索引表を作成し,それを用いて検索の絞込みを行い,次に索引表にコード化されている構造コード中の文節番号,係り受け文節番号,文節カテゴリコードを参照して用例文との構造一致があるかを検査している.
田中ら\cite{tanaka}は,用例提示型の日英翻訳支援システムにおける検索手法として入力キーワードの語順とその出現位置の感覚を考慮した手法を提案している.検索手法としては,入力文字列を形態素解析して自立語を抽出し,これをキーワードとしAND検索を行っている.この際,AND検索だけでは不必要な文を拾いやすいので語順と変異を考慮した検索を行っている.これにより構文解析した結果と同じような効果を得ることができるとしている.
村田ら\cite{murata}は,自然言語でかかれた知識データと質問文を,類似度に基づいて照合することにより,全自動で解を取り出すシステムを開発している.ここでの文間の類似度計算には,自立語同士の類似度については基本的にIDFの値を用い,同義語の場合はEDRの概念辞書などを用い,質問側の文節が疑問詞などを含む文節の場合は意味制約や選考に従った類似性を用いている.
日本語文章を検索インタフェースに用いている研究には,京都大学総合情報メディアセンターで公開されているUnixの利用方法に関する藤井ら\cite{kyoudai}のアドバイスシステムがある.このシステムは質問文の構文木と解説文の条件部の構文木を比較し,一致点に対して重みを付けて合計することによって類似度を求め,最も類似する解説文の結果部を表示するというものである.


一方,法律文を対象とした自然言語処理の研究としては,平松ら\cite{hiramatu}の要件効果構造に基づいた統語構造の解析や高尾ら\cite{takao}の並列構造の解析の研究がある.前者では,法律文の論理構造を的確に捉えるために,条文中の要件・効果などを表す表層要素を特定し,これを用いた制限言語モデルを単一化文法として記述し,これに基づく法律文の構文解析を行い,解析木と素性構造を出力している.後者では,前者の研究を受けて,係り受け解析時の並列構造の同定において,経験則に基づく制約を用いて間違った構文構造を除去し,次に並列要素の長さ,表層的・深層的類似性などに基づく評価を行い,並列構造の範囲を推定している.なお,ここでの並列構造の類似性判定においては黒橋らの方法を用いている.


このように,これまでの研究における文の類似性は,述語を中心として,それに構文的に係っている語についてその表層格と意味素を基に計算しているものである.これらでは,2つの文章中の対応する語間の論理的や時系列的やその他の意味的な関係による結合の類似性については比較の対象外になっており,本研究の目的とする文章に記述されている事実の内容的な類似性を評価するには十分でない.


\subsection{研究目的}
本研究では,意味解析を用いた情報検索の一手法を提案する.具体的には「判例」を検索対象とし,自然言語で記述された「問い合わせ文」を検索質問とした判例検索システムJCare(Judicial CAse REtriever based on semantic graph matching)を開発する.判例検索は社会的にも有用性が高いので,これを検索対象とした.
本システムでは,自然語意味解析により「問い合わせ文」と「判例」の双方を意味グラフに展開し,意味的に同型な部分グラフを求めることで類似度を算出する.これにより両者の内容にまで踏み込んだ検索を実現する.
検索対象は「判例」の中でも「交通事故関連の判例」に絞り込む.「交通事故」の判例には,被告,原告,被害者などの``当事者''が存在し,それぞれの``当事者''が相互に「関係」を持つという特徴がある.この特徴により,照合時における比較基準が設定しやすくなる.

\section{意味グラフ}
\subsection{検索対象 -- 「判例」}
「最高裁判所判例集」は最高裁判所判例委員会により選択されまとめられたもので,事件の判決を集めた「事例集」の中から特殊な事件,特殊な判決(裁判上の先例となる判決)とされるものを集めたものである.現在約50万件に達する「事例集」に対して「判例集」には約3000〜4000件が含まれている.この「判例集」に含まれるひとつひとつが「判例」である.

「判例」は大きく分けて,判例ID,事件の種類,判決日,当事者の氏名などといった(論文で言えば「書誌情報」に相当するような)付帯情報と,その判例の内容自体を記述した内容情報から構成されている.内容情報は平均約8000字で記述されている.この内容情報はさらに,事件の事実を表すもの,その事実から裁判官が下した判決とその理由に分けることができる.

判例の中に記述された事実文は大きく2つに分けることができる.ひとつは「当事者間に争いのない事実」を記述したもの,もうひとつは「当事者それぞれが主張する事実」を記述したものである.前者は当事者のもつ属性や事故のおきた場所など,確実に事実として認められるものを示す.これに対して後者は,確実に事実といえないものである.最終的に裁判官が,後者のものに対して確かな「事実」として認めるか否かの判断を下すのである.この時,裁判官は前者の「争いのない事実文」を事故当時の状況を判断するための要素として利用する.従って,判決を下す上で事故の状況を示す「争いのない事実文」が大きな役割を担っていると考えられる.そこで本研究では,検索対象を「当事者間に争いのない事実」文とする.図~\ref{fig:judicialcase}に,ある交通事故裁判の判例から取り出した事実文を示す.

\begin{figure}
\begin{center}
 \atari(130,21)
 \caption{交通事故裁判の判例から取り出した事実文の例}
 \label{fig:judicialcase}
\end{center}
\end{figure}

実際の判例検索システムでは,「判例」の中からこの「争いのない事実文」を検索対象として取り出す必要がある.この抽出プロセスは,本研究では手作業による前処理として実現する.その自動化は今後の課題とする.

\subsection{検索質問 -- 「問い合わせ文」}
本システムを利用するにあたって利用者は,図~\ref{fig:query}に示すような,``事件の状況を示す事実を記述した文章''を「問い合わせ文」として与えるものとする.ここでは,事故当時の現場の状況,当事者の行動,当事者の属性などの,事故の発生に関する事実関係を明確に記述する.JCareはこの問い合わせ文に記述されている事故に意味的に類似した事故に関する判例を検索する.

\begin{figure}
\begin{center}
 \atari(130,21)
 \caption{問い合わせ文の例}
 \label{fig:query}
\end{center}
\end{figure}

\subsection{意味解析(自然言語処理)}
実際に検索を行うには,問い合わせ文と判例をその意味的な内容を表す内部表現に変換する必要がある.このためには言語表現からその意味を抽出するための一連の自然言語処理を行う必要がある.本研究ではこのために,形態素解析に奈良先端科学技術大学院大学の松本研究室が開発した茶筌\cite{chasen}を,係り受け解析に同研究室が開発した茶掛を,意味解析に青山学院大学の原田研究室が開発したSAGE\cite{sage99,harada,sage2000}を,さらに文脈解析に同研究室が開発したInSeRA\cite{insera}を用いることとする.

\begin{figure}
\begin{center}
 \atari(120,113)
 \caption{図~\ref{fig:query}の問い合わせ文から抽出されたframe述語とinterRel述語}
 \label{fig:frame}
\end{center}
\end{figure}

\begin{figure}
\begin{center}
 \atari(120,76)
 \caption{図~\ref{fig:query}の問い合わせ文を表わす意味グラフ(見やすくするため一部のフレームを併合してある)}
 \label{fig:graph}
\end{center}
\end{figure}

\subsection{意味グラフ}
先に述べた4つの自然言語処理を経て判例や問い合わせ文の文章は,図~\ref{fig:frame}に示すように,文内の語の語意と係り受け関係にある他の語との間の文内深層格を表すframe述語リストと,文間の文間深層格を表すinterRel述語リストに変換される.さらに,手作業で用意した照応解析データやラベル解析データ(詳細は3章で述べる)をこれらと併合して,問い合わせ文は最終的に図~\ref{fig:graph}に示すような意味グラフという形に変換される.ここでは,frame述語が表すひとつひとつの語が頂点(ノード)で表現され,また,frame述語間の文内深層格および文間深層格,照応解析・ラベル解析による深層格が弧(アーク)で表わされている.

意味グラフ中の各弧が表す深層格には,EDR電子化辞書\cite{edr}にて定義された全部で28個の文内深層格に加え,「日本語文書の意味解析システムSAGEの高速化と精度評価」\cite{sage2000}にて追加定義された7個の文内深層格及び「日本語文間の意味的関係解析システムInSeRAの開発研究」\cite{insera}における21個の文間深層格が含まれる.

\begin{figure}
\begin{center}
 \atari(110,108)
 \caption{JCareの処理概要}
 \label{fig:jcare}
\end{center}
\end{figure}

\section{JCareの概要}
JCareシステムの処理の概要を図~\ref{fig:jcare}に示す.まず前処理として「判例」から検索対象となる「争いのない事実文」を手作業で抽出し,自然言語処理による構造解析,意味解析,照応解析などを行う.「問い合わせ文」はそのまま同様な自然言語処理を行う.この結果得られた意味グラフをViewグラフに分割し,問い合わせ文と判例をViewグラフレベルで照合し,類似度の高い順に判例を提示する.

\subsection{前処理}
先に述べたように,照応解析とラベル解析は現状では手作業で行う.たとえば照応解析では,指示詞が文脈中に登場した人物を指す場合,図~\ref{fig:equalto}のようなequalTo述語を出力する.equalTo述語は第1引数に照応元のフレームIDを持ち,第2引数に照応先のフレームID群のリストを持つ.例えば,図中の最初の事例は,図~\ref{fig:frame}の34番のフレームで表される指示詞『この』が,23番のフレームで表される『交差点』に照応していることを表している.

\begin{figure}
\begin{center}
 \atari(100,30)
 \caption{照応解析結果を表わすequalTo述語}
 \label{fig:equalto}
\end{center}
\end{figure}

\begin{figure}
\begin{center}
 \atari(120,85)
 \caption{ラベル解析結果を表わすlabel述語}
 \label{fig:label}
\end{center}
\end{figure}

ラベル解析では,図~\ref{fig:label}に示すような判例文における箇条書きの記述パターンを解析し,label述語を出力する.label述語は第1引数に文番号,第2引数に接頭語,第3引数に箇条書き番号,第4引数に接尾語,第5引数に箇条書き項目名をもつ.この第2から第5引数の記述パターンの解析から得た箇条書き構造により,どの内容文がどの箇条書き項目を説明したものなのかという情報を抽出する.これは深層格という形で意味グラフに取り込まれ,意味グラフを部分グラフへ分割するために有用な情報となる.

\subsection{当事者の同定}
言語処理の出力データ群から意味グラフを作成する前に,当事者の同定を行う.当事者とは事故に関係した人物であり,本研究では被告(defendant),原告(plaintiff),被害者(victim),遺族(bereaved),その他の当事者(others)の5つのクラスを用意する.当事者の記述表現が固有名詞であったり,被害者と原告が同一人物であるときに生じうる照合時の当事者不整合を排除するために当事者の同定を行う.具体的には,「被害者」,「被害者太郎」など,同一人物について多様な表現を含む文章があったとき,これらを同一視するために図~\ref{fig:instanceof}に示すinstanceOf述語を出力する.これは第2引数のリスト中の各フレームが第1引数の当事者クラス(図中で<の後にクラス名を示した)に属すことを表す.例えば,図中の2番目の事例は,図~\ref{fig:frame}のフレーム29の『被害者』とフレーム33の『被害者太郎』が共に原告クラスに属していることを表している.

\begin{figure}
\begin{center}
 \atari(90,34)
 \caption{当事者同定結果を表わすinstanceOf述語}
 \label{fig:instanceof}
\end{center}
\end{figure}

\subsection{意味グラフ生成部}

言語処理で出力する意味解析データ,文脈解析データに加え,前処理で用意する照応解析データ,ラベル解析データ,以上4つの解析データをまとめて,図~\ref{fig:graph}に示した意味グラフを生成する.具体的には図~\ref{fig:graphdata}に示すgraph述語で表す.graph述語には頂点の情報を保持するgraphNode述語と,弧の情報を保持するgraphArc述語がある.それぞれの述語句名のあとには,検索質問(Query)か検索対象(Object)かを判別する接尾語が続く.graphNode述語では第1引数にフレームID,第2引数に概念IDを保持する.また,graphArc述語では引数を,係り元フレームID,深層格,係り先フレームIDの順で保持する.例えば,図~\ref{fig:graphdata}に示すgraph述語では,頂点1が図~\ref{fig:frame}内の『南北』を,頂点2が『走る』を表しており,その間に『goal』格の弧が存在していることが宣言されており,これらは図~\ref{fig:graph}に示した意味グラフに視覚的に表現されている.

\begin{figure}
\begin{center}
 \atari(120,49)
 \caption{意味グラフの述語表現}
 \label{fig:graphdata}
\end{center}
\end{figure}

\section{Viewへの分割}
本研究では,意味グラフ間の類似度算出のために個々の語意だけでなく,語と語を結ぶ深層格も考慮する.一般に判例の事実文の意味グラフのサイズは頂点数が平均約1000個,弧数は平均約1200個であり,頂点と弧の両者を考慮して意味グラフ間の同型部分グラフ探索を行うと,計算量が膨大になる.そのため探索空間を狭める方法として,意味グラフを視点(View)により部分グラフ(Viewグラフと呼ぶ)群に分割することにした.

\begin{table}
\begin{center}
 \caption{Primary View と Secondary View}
 \atari(70,62)
 \label{fig:view}
\end{center}
\end{table}

\subsection{Viewとは}
Viewは,事実文中に記述されている語をそれらが表す意味(EDRでは概念idと読んでいる)を考慮して,「事故現場の状況」,「当事者の行動」,「当事者の静的特徴」などの特定の``視点''でグループ分けする単位である.
本研究では,このViewとして,``誰(who)'',``いつ(when)'',``どこ(where)'',``なに(what)'',``どのように(how)''という事件の基本構造を表すPrimary Viewと,「述語句」の意味により静的なものと動的なものへさらに細分割するSecondary Viewの2種類を用意した.これらを判例用語で表わすと,表\ref{fig:view}に示す計11種のViewとなる.

\begin{description}
\item[1)Primary View:]検索対象は,判例に含まれる「事件の状況を記述した事実文」であり,基本構造上の``when'',``where'',``what''では,これらの視点から選ばれたこの「事件」に直接に関連するものを対象とする.本研究では,対象が交通事故であることから,``when''は『発生日時』View,``where''は『発生場所』View,``what''は『関係車両』Viewを表現する.また``how''にあたる『事故態様』Viewは,事件の状況をまとめて記述した(要約した)部分グラフに当たり,たとえば「事故は被害者と被告が車を競わせたことでおきた.」などがこれにあたる.
``who''では,『被告』View,『被害者』View,『原告』Viewの3つを用意する.先に述べた当事者の1つである「遺族」は,判例において存在しないことが多いと判断し,Viewからは外すことにした.また,事実文中に``被告''が複数人存在したとしても,被告らが関係する事実文は,同じ『被告』Viewへ割り振り,被告同士の区別は,意味グラフのレベルで行うこととした.以上で述べた6つのViewどれにも当てはまらないものは,『その他』Viewへ割り振る.
\end{description}

\begin{description}
\item[2)Secondary View:]当事者に関する3つのPrimary View(「被告」,「被害者」,「原告」)グラフを,それぞれそこに含まれる「述語句」の意味によって,さらに2つのSecondary Viewグラフへ分割する.一般的に「述語句」は,当事者の属性を示すものと,当事者の行動を示すものに大きく2種類に分けることが出来る.Secondary Viewは,この2種類の視点に異なった類似度算出方法を選択することで,視点の性格に応じた比較を行うために用意した.具体的には当事者に関する3つのViewを,当事者の静的特徴についての記述を格納する『静的特徴』Viewと,当事者の行動についての記述を格納する『単独行動』Viewに分割する.たとえば,「被告は20歳の男性である」は前者に割り振られ,「被告は自動車を運転していた」は後者に割り振られる.また,行動に関する部分グラフの中に当事者が2人以上含まれていた場合は,『単独行動』Viewへ割り振らず,Primary Viewの『事故態様』Viewへ割り振る.これは,当事者別にViewを分けたことで生じうる比較精度の低下を防ぐためである.たとえば,「問い合わせ文」での事故の状況を示す記述が,被告中心の表現となっており,反対に「判例」の「争いのない事実文」での記述が原告中心の表現である場合,それぞれが異なるSecondary Viewへと割り振られてしまう.そのため,この事故の記述が似ているものであろうとなかろうと類似度はゼロとなる.このような照合の漏れを防ぐために,当事者が相互に関連しあいながら生じた行動を記述した部分グラフは,『事故態様』Viewへ割り振ることとする.
\end{description}

\subsection{主体}
意味グラフ中の用言を表す頂点に接続している頂点群には,この用言が表す述語の動作主体や動作対象などを表す語を含む頂点が存在し,これを主体と呼ぶ.具体的には,EDR電子化辞書の深層格で表される主体には,「有意志動作を引き起こす主体」,「属性をもつ対象」および「動作・変化の影響を受ける対象」の3つがある.1つ目は思考的・知的動作を含む意志を持って行われる動作の主体を表し,agent格にあたるものである.2つ目は,感情を表す動作の主体を表し,a-object格にあたるものである.3つ目は,自然現象・生理現象・物理現象の主体を表し,object格にあたるものである.

\subsection{代表語}
実際に意味グラフの各頂点をViewグラフに分配するには各頂点がどのViewに属するかを判定する必要がある.この判定は,各々のViewを代表する語(代表語)を用意し,意味グラフの割り振り判定対象となる頂点が表す語がどのViewの代表語のEDR概念体系辞書中の下位概念であるかに基づいて行う.
意味グラフ中の主体を表す各頂点を,Primary Viewに分類するための代表語を表にしたのが,表\ref{fig:primaryview}である.これに対し述語句を表す各頂点をSecondary Viewに分類するための代表語は,図~\ref{fig:secondaryview}に示すようにEDR概念体系辞書の最上位概念群を利用し,『静的特徴』Viewでは図中Aで表した計5つの概念を代表語とし,『単独行動』Viewでは図中Bで表した計4つの概念を代表語として用意する.

\begin{table}
\begin{center}
 \leavevmode
 \caption{Primary View の代表語}
 \atari(120,24)
 \label{fig:primaryview}
\end{center}
\end{table}

\begin{figure}
\begin{center}
 \leavevmode
 \atari(120,52)
 \caption{Secondary View の代表語}
 \label{fig:secondaryview}
\end{center}
\end{figure}

\subsection{Viewへの分割手法}
意味グラフ生成部で生成した意味グラフをViewに分配する際の単位は,「述語句」を1つ,「主体」を少なくとも1つ含む部分グラフとする.この単位で,意味グラフから述語句頂点に連結する頂点群を取り出し,Primary Viewでは主体となる頂点,Secondary Viewでは述語句となる頂点のもつ語が先に述べたどの代表語の下位概念になっているかを判定し,この部分グラフ全体を適切なViewへ分配する.この際Primary Viewでは,当事者View(「被告」, 「被害者」, 「原告」)の代表語の下位概念にあたる主体が2種類以上存在する場合は,事故態様Viewへ割り振る.それ以外は割り振り先Primary Viewの優先順位を,当事者View,発生場所View,発生日時View,関係車両View,その他Viewの順番として割り振る.またSecondary Viewでは,Primary Viewにて当事者Viewに割り振られた部分グラフの述語句の概念が,『静的特徴』Viewと『動的特徴』Viewのどちらの代表語の下位概念であるかを判定し,割り振りを実現する.意味グラフの切断箇所は「述語句」頂点間の弧とする.この際複数のグラフに共有されている頂点は,個々のViewグラフにおいて重複して存在させることとする.たとえば「原告は信号を確認せずに,横断歩道を直進した.」という文章は図~\ref{fig:graphrule}のように,2つのViewグラフとして扱われる.

\begin{figure}
\begin{center}
 \atari(100,50)
 \caption{意味グラフのViewへの割振り単位}
 \label{fig:graphrule}
\end{center}
\end{figure}

\section{Viewごとのマッチング手法}
問い合わせ文章と判例文章の各々11個のViewグラフ同士をViewごとに両者間の位相同型部分の大きさをもとに,内容類似度を算出する.この手順について以下に説明する.

\subsection{語意類似度算出部}
まず2つのViewグラフから任意の頂点を1つずつ取り出し,それらの語意類似度を算出する.2ノード間の語意類似度の算出は,EDR概念体系辞書を用いて2つの語が表す概念の概念体系木における深さと,2つの共通上位語の深さを求め,図~\ref{fig:wordsimilarity}左に示す式で算出する.たとえば``交差点''と``現場''というノード間の語意類似度は,図左下に示すように算出される.
次に,語意類似度の値が一定値(現在は経験的に40としている)以上ならば仮のペアを1つづつ作る.すべてのノード同士の判定終了後,ノードが重複した仮ペア群の中で類似度の最も高いものを正式ペアとする.


\begin{figure}
\begin{center}
 \atari(130,86)
 \caption{EDR概念体系辞書を用いた語意類似度算出}
 \label{fig:wordsimilarity}
\end{center}
\end{figure}

正式ペアの語意類似度の合計を問い合わせ文のViewグラフノード数で除算し,11個のView各々における総語意類似度を求める.

\subsection{格類似度算出部}
語意類似度算出において,類似しているとした正式ペアの頂点対から,任意の2つの頂点対を取り出し,その間にある最短路(関係パスと呼ぶ)上の深層格リストの類似性を評価する.この評価値を格類似度と呼ぶ.

\begin{figure}
\begin{center}
 \atari(70,91)
 \caption{格類似度算出}
 \label{fig:casesimilarity}
\end{center}
\end{figure}

\begin{table}
\begin{center}
 \caption{深層格類似度の割り当てパターン(一部)}
 \atari(100,50)
 \label{fig:deepcasesimilarity}
\end{center}
\end{table}


具体的には,関係パス上の深層格群の類似性評価は,図~\ref{fig:casesimilarity}に示すようにグラフ1とグラフ2の頂点対A,B間及びA',B'間に有向路があれば,その最短路(関係パス)を求めそれらが類似する深層格を共有するごとに得点を加算することで実現する.加える得点は深層格ごとに表\ref{fig:deepcasesimilarity}の類似度割り当てパターンAに示すように定めた.この表で例えば,第1行目は,「グラフ1とグラフ2の両方の関係パス上に深層格agentが存在した時に,得点10を加える」ということである.この深層格類似度値は,深層格同士の類似性,事実文中における深層格の重要性から,同じ深層格であり重要度の高いものを10,同じ深層格であるが多少重要なものを9,同じ深層格であるが重要度の低いもの及び,重要度の高いもので互いに似ている深層格を8,多少重要なもので互いに似ている深層格を7としている.
頂点対ごとの関係パスに対するこの類似度得点値の合計値を頂点対の数だけ合計し,その結果を問い合わせ文のViewグラフにおける格の数で除算する.この値を各々のViewにおける総格類似度とする.

ここで,関係パスにおける格類似度得点値をViewごとに設定することで,全体の類似度を算出する上でのViewの独自性を表現する.具体的には表\ref{fig:deepcasesimilarity}の最右列に示すように,もう1つの類似度割り当てパターンBを用意した.これは主として時間軸における前後関係に関する格の類似性により多くの重みを与えたものである.パターンBは当事者の行動を記述した『単独行動』Viewグラフおよび『事故態様』Viewグラフ用に,パターンAはそれ以外のViewグラフ用として利用している.

\subsection{意味グラフ類似度算出部}
先で求めたViewグラフごとの総語意類似度と総格類似度の和をsigmoid関数を用いて算出された値が0〜100の範囲になるよう正規化する.この値をViewごとのViewグラフ類似度とする.そして11種類のViewごとのこの値の合計値を,判例各々の問い合わせ文に対する意味グラフ類似度とする.
蓄積されている全ての判例の問い合わせ文に対する意味グラフ類似度を算出する.この類似度で判例をソートして,高いものからユーザへ提示することで,判例検索システムを実現する.


\begin{figure}
\begin{center}
 \atari(100,89)
 \caption{発生場所Viewに関する類似度比較の例}
 \label{fig:viewResult}
\end{center}
\end{figure}

\begin{figure}
\begin{center}
 \atari(110,29)
 \caption{すべてのViewに関する類似度比較結果と意味グラフ類似度}
 \label{fig:result}
\end{center}
\end{figure}

\subsection{事例}
本研究では実際に,事故の状況を文章で記述した図~\ref{fig:query}の問い合わせ文により図~\ref{fig:judicialcase}に示した判例を含む計3つの判例を対象として類似度評価を行った.
問い合わせ文と3つの判例から,本システムでの解析における途中結果を分析して,発生場所Viewに関する類似度を計算する上で抽出された語と語間の関係を取り出し,人が分かりやすいように,これらを含む原文を取り出して示したのが,図~\ref{fig:viewResult}である.さらに,この図において発生場所Viewについての総語意類似度,総格類似度,Viewグラフ類似度の計算結果を列挙した.
問い合わせ文の事故現場は信号機の設置された丁字路交差点であり,判例2および判例3も同様である.これに対し,判例1の事故現場は高速道路上り線上であり,問い合わせ文と異なる.この結果,まず判例1との比較では語意が近い語があることにより総語意類似度はある程度の値となったが,語間の関連性の類似性がなく総格類似度は0であった.一方判例2では,事故現場の交差点が共に2つの道路が『交差している』ということ,さらにその『交差する』で連体修飾されている『交差点』であること,『信号機が設置されている』ことなどの記述があることで,総語意類似度と総格類似度共に高い点になっている.一方判例3は,語の関連の仕方は判例2ほど似ていないので総格類似度は判例2ほど高くないが,『環八通り』という語が共に含まれているので総語意類似度は非常に高くなっている.これらと同様に他のViewについての比較計算が行われた結果,それぞれのViewごとの類似度は図~\ref{fig:result}に示すようになり,これらの和である意味グラフ類似度はそれぞれ138, 202, 264と算出された.

\section{まとめ}
本研究では,日本語文章による問い合わせ文による情報検索として,一群の文章群から,出現する語の意味およびそれらの意味的関係において問い合わせ文と類似した記述をできるだけ多く含む文章を検索する手法を提案した.具体的には,検索対象を交通事故の判例文章に特定し,これら文章群に意味解析を中心とした自然言語処理を施して意味グラフ群へ展開し,それらの間に存在する意味的な位相同型部分の大きさを基に類似性を判定する判例検索システムJCareを開発した.ここでは,意味グラフを交通事故を特徴づける11個の視点によりViewグラフと呼ぶ部分グラフに分割し,Viewグラフごとの照合を行うことで類似度を算出する手法を構築し,検索の高速化・精度向上を図った.

実際に交通事故の3つの判例に対し問い合わせ文との照合を行ったところ,おおよそこの類似度は内容的な類似性の程度をよく表わしているとの評価を得た.

今後は類似度算出における関係パスの深層格の評価値を経験的に求めていく必要がある.また,前処理部などにおける手作業を自動化する必要がある.

\acknowledgment

本研究を進めるにあたり,日本語形態素解析システム『茶筌』,係り受け解析システム『茶掛』を提供してくださった奈良先端科学技術大学院大学の松本裕治教授に深く感謝致します.また,判例検索システム構築において多くのアドバイスおよび判例データを提供してくださった第一法規株式会社に深く感謝致します.

\bibliographystyle{jnlpbbl}
\bibliography{ipjpaper2000}

\begin{biography}
\biotitle{略歴}
\bioauthor{原田 実(正会員)}
{1951年生.1975年東京大学理学部物理学科卒業.1980年東京大学理学系大学院博士課程修了.理学博士.(財)電力中央研究所研究員を経て,1989年より青山学院大学 理工学部 経営工学科 助教授,2000年より同情報テクノロジー学科 教授,現在に至る.1986年電力中央研究所経済研究所所長賞.1992年人工知能学会全国大会優秀論文賞.1996--1998年EAGL推進事業機構「ソフトウェア開発の統合的自動化」プロジェクトリーダー.主たる研究は,自動プログラミング,ソフトウェア分析・設計の自動化,自然語意味理解,ルールベースの自動更新,アクティブメッセージなど.編著書「自動プログラミングハンドブック」など.情報処理学会,電子情報通信学会,人工知能学会,ソフトウエア科学会,IEEE,ACM,AAAI各会員.}
\bioauthor{鈴木 亮(非会員)}
{1999年青山学院大学理工学部経営工学科卒業.2001年同大学大学院修士課程修了.4月よりソニー株式会社.}
\bioauthor{南 旭瑞(非会員)}
{2000年青山学院大学理工学部経営工学科卒業.現在同大学大学院.} 

\bioreceived{受付}
\biorevised{再受付}
\biorerevised{再々受付}
\bioaccepted{採録}
\end{biography}

\end{document}
