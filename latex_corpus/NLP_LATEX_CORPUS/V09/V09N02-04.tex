


\documentstyle[epsf,nlpbbl]{jnlp_e}

\setcounter{page}{77}
\setcounter{巻数}{9}
\setcounter{号数}{2}
\setcounter{年}{2002}
\setcounter{月}{4}
\受付{August}{10}{2001}
\採録{January}{10}{2002}

\setcounter{secnumdepth}{2}

\title{}
\author{}
\jkeywords{}

\etitle{A Method for Similarity-based\\  Lexical Disambiguation}

\eauthor{Qujiang Peng\affiref{UECUniv}   \and
        Haodong Wu\affiref{DokkyoUniv}   \and
        Teiji Furugori\affiref{UECUniv}}

\headauthor{Peng,~Q.~et~al.}
\headtitle{A Method for Similarity-based Lexical Disambiguation}

\affilabel{UECUniv}
          {The University of Electro-Communications, Department of Computer Science}
          {The University of Electro-Communications, Department of Computer Science}
\affilabel{DokkyoUniv}
          {Dokkyo University, Faculty of Foreign Languages}
          {Dokkyo University, Faculty of Foreign Languages}

\eabstract{
This paper describes a method for word sense disambiguation using a 
similarity metric.  In this method, we first obtain context-similarity vectors 
for the senses of a polysemous word using a corpus and also define the context 
representation for the same polysemous word appearing in text.  We then 
calculate distributional matrix between each context-similarity vector and the 
context representation for the word to be disambiguated.  Finally, comparing 
the values of distributional matrices, we select the sense with the highest 
value as the meaning of the polysemous word.  An experiment with 682
instances for 10 polysemous words shows that we are able to disambiguate at 
a rate of almost 92\%.
}

\ekeywords{Ambiguity, Word Sense, Disambiguation, Similarity, Context, Distributional Matrix, Density}


\begin{document}

\maketitle

\section{Introduction}

The meaning of a word is not unique.  This makes natural languages nothing but a convenient
 and flexible tool to use for communication.  However, it is a big problem in natural language 
processing.  To understand a passage like,

\begin{quote}
{\it Water the plant in the garden.}
\end{quote}
we need to know what {\it plant} means: a living thing or factory?  Unless this 
problem is resolved, we will have no better systems, not to mention a complete 
one, for machine translation, natural language understanding, information 
retrieval, etc.


In what follows,we present a method and its experimental result for resolving
 lexical ambiguities based on a word similarity measure.  

\section{Word Sense Disambiguation}

The history tells us that early work in lexical or word sense disambiguation (WSD) took 
a pure AI approach (e.g., Hirst 1987; Small and Rieger 1982).  
Many researchers manually set up a knowledge base for 
each word and described its senses in various linguistic usages.  Then, some started to 
use machine-readable dictionaries (MRDs) to select the proper sense of a polysemous 
word (e.g., Guthrie, Guthrie, and Aidinejad 1991; Lesk 1986; Walker 1987). 

The MRDs are widely available, but they have deficiencies for WSD.  For instance, the 
definitions of a word in them are often too short and too uneven to cover necessary 
collocations.  We encounter another difficulty in retrieving necessary information from
 the on-line dictionaries since their contents are usually expressed for the use of 
human beings rather than for the computers.

The research on WSD relies on corpora in the most of work being done nowadays.  
The corpus-based approach has an advantage in its general applicability.  It is 
also possible to get around costly and perhaps error-prone parsing or semantic analysis 
otherwise needed.  Along this line, Yarowsky (Yarowsky 1992) presented a statistical
 method of resolving 
lexical ambiguities with the use of Roget's Thesaurus and a large corpus.
  Dagan, Marcus, and Markovitch (Dagan, Marcus, and Markovitch 1995) proposed a
 method based on the adjacent word of 
a polysemous word to disambiguate the word senses.  Leacock, Chodorow, 
and Miller (Leacock, Chodorow, and Miller 1998) determined the sense of a 
word in text using a corpus and 
WordNet (WordNet 2001) relations.  Sch\"{u}tze (Sch\"{u}tze 1998) proposed a method 
for WSD based on clustering
 on contextual similarities.  Karov and Edelman (Karov and Edelman 1998) devised 
a disambiguation method 
using a corpus and MRD, and attempted to determine the word senses by the different
 contextual usages of the polysemous word being analyzed.


\section{Measurement of Similarity}

Word similarity between two words $w_1$ and $w_2$  depends on the degree of their co-occurrences
 in text.  The contextually similar words are the words co-occurring frequently in a distance 
in text.  They need not be synonyms, or belong to the same syntactic or semantic category.  
For instance, the words {\it doctor} and {\it health} are not synonyms in 
thesauruses, but they are 
likely to be contextually similar as they tend to co-occur in a text.


\subsection{Mutual Information}

The contextually similar word may be measured by using mutual information.  The mutual
 information, $I$, estimates the strength of association between two words $w_1$  and $w_2$ 
(Church and Hanks 1990):
\begin{equation}
I(w_1,w_2)=\log_2\biggl(\frac{N\ast f(w_1,w_2)}{f(w_1)f(w_2)}\biggl)
\end{equation}
Here, $N$ in (1) is the size of the corpus used in the estimation, $f(w_1,w_2)$ is the frequency 
of co-occurrences of $w_1$ and $w_2$, and $f(w_1)$ or $f(w_2)$ is the frequency of each word.
  If there is 
a strong association between $w_1$ and $w_2$, then $I(w_1,w_2)\gg0$.  If there is a weak association 
between $w_1$ and $w_2$, then $I(w_1,w_2)\approx0$.  If $I(w_1,w_2)\ll0$, then $w_1$ and $w_2$ 
are said to be in complementary distribution.

If $w_1$ and $w_2$ are contextually similar, then $I(w,w_1)$ and $I(w,w_2)$ have similar mutual
 information 
for some word $w$ in lexicon.  We will use the third word $w$ in the next section to 
measure the similarity of two words $w_1$ and $w_2$.

\subsection{Similarity Metric of Two Words}

Mathematically, the error can be used to express the dissimilarity of two
 words $w_1$ and $w_2$.  For example, $\bigl| I(w,w_1)-I(w,w_2) \bigl|$ shows the
 absolute error in that it indicates 
how much the two numbers $I(w,w_1)$ and $I(w,w_2)$ differ.  The absolute error does not express the 
precision of the difference, however.  So, we define the proportional error:
\[\frac{\bigl| I(w,w_1)-I(w,w_2) \bigl|}{I(w,w_1)+I(w,w_2)}\]
We know that the bigger the similar values of $I(w,w_1)$ and $I(w,w_2)$, the stronger the similarity
 of $w_1$  and $w_2$.  In the proportional error, however,
 $\bigl( I(w,w_1), I(w,w_2) \bigl)=\bigl( 2.7, 3.2 \bigl)$ and 
$\bigl( I(w,w_1), I(w,w_2) \bigl)=\bigl( 0.27, 0.32 \bigl)$  , for instance, have the 
same effect on measuring the similarity, though in reality the numbers in the second
 pair are far smaller and thus far less important (sometimes harmful) in measuring 
the similarity.

\subsubsection{3.2.1   Strong Bridge Set}

It is desirable to measure the word similarity collecting the set
 of word $w$ so that its member has bigger mutual information 
with $w_1$  and $w_2$.  We define such a word set as:
\[S=\Bigl\{w \bigl| I(w,w_1)>t_0, I(w,w_2)>t_0, t_1<\frac{I(w,w_1)}{I(w,w_2)}<t_2, w\in Lexicon \Bigl\}\]
and call it the strong bridge set.  Here, $t_0$, $t_1$, and $t_2$ are 
threshold values,  $t_0$ makes eliminate small mutual information
 values, and $t_1$ and $t_2$ make it possible to get the word $w$ for 
which $w_1$  and $w_2$ have similar mutual information.  We set $t_2=t_1^{-1}$ .  The 
similarity of $w_1$  and $w_2$ gets stronger as $\|S\|$  becomes bigger.

\subsubsection{3.2.2   Definition of Similarity}

We will define a similarity metric of two words $w_1$  and $w_2$ using
 the proportional error and the strong bridge set $S$.
\begin{equation}
sim(w_1,w_2)=\log_{10}\sum_{w \in S}e^{-\frac{| I(w,w_1)-I(w,w_2) |}{I(w,w_1)+I(w,w_2)}}
\end{equation}
The function $f(x)=e^{-x}$ in (2) can convert dissimilarity to similarity.  
The function $f(x)=\log_{10}(x)$ is to make the function value small.


Evidently, equation (2) is symmetric, i.e., $sim(w_1,w_2)=sim(w_2,w_1)$.  This agrees
 with an intuition that the similarity of two
 words $w_1$  and $w_2$ is equal to that of $w_2$ and $w_1$.


In equation (2), the word $w_1$ is most similar to itself.  
It is proven by showing $sim(w_1,w_2) \leq sim(w_1,w_1)$.  Put it simply,
\[S(w_1,w_2)=\Bigl\{w \bigl| I(w,w_1)>t_0, I(w,w_2)>t_0, t_1<\frac{I(w,w_1)}{I(w,w_2)}<t_2,
 w\in Lexicon \Bigl\}\] 
\[\subset\{w| I(w,w_1)>t_0, w\in Lexicon\}=S(w_1,w_1)\]
and then we get:
\[sim(w_1,w_2)=\log_{10}\sum_{w \in S(w_1,w_2)}e^{-\frac{| I(w,w_1)-I(w,w_2) |}{I(w,w_1)+I(w,w_2)}} \leq 
\log_{10}\sum_{w \in S(w_1,w_2)}e^{-0}\]
\[\leq \log_{10}\sum_{w \in S(w_1,w_1)}e^{-0}=sim(w_1,w_1)\]

\section{A Method of Word Sense Disambiguation}

We now devise a method of WSD based on equation (2).  It takes the steps in Figure 1.

\begin{center}   
\begin{picture}(300,265)
     \put(10,245){\underline{Content Word Sequence in Text ($w=$ word with ambiguities)}} 
     \put(65,235){$w_1, w_2, \cdots, w_i, w, w_{i+1}, \cdots, w_{n-1}, w_n$}
     \put(85,230){\vector(0,-1){15}}
     \put(205,230){\vector(0,-1){25}}
     \put(82,207){$w$}
     \put(85,203){\vector(0,-1){15}}
     \put(58,175){equation (2)}     
     \put(85,177){\oval(60,17)}
     \put(152,197){\underline{Context Representation}} 
     \put(135,185){$V_{CR}=(w_1, \cdots, w_i, w_{i+1}, \cdots, w_n)$}
     \put(85,165){\vector(0,-1){15}}
     \put(28,141){\underline{Context-similarity Vectors}} 
     \put(37,127){$V_{s_1}=(w_1^1, w_2^1, \cdots, w_k^1)$}
     \put(37,114){$V_{s_2}=(w_1^2, w_2^2, \cdots, w_k^2)$}
     \put(56,101){\vdots}
     \put(37,94){$V_{s_r}=(w_1^{r}, w_2^{r}, \cdots, w_k^{r})$}
     \put(3,65){Comparison:$sense(V_{s_1},V_{CR}), sense(V_{s_2},V_{CR}),
                            \cdots, sense(V_{s_r},V_{CR})$}
     \put(148,67){\oval(293,17)}
     \put(85,90){\vector(1,-2){7}}
     \put(205,181){\vector(-1,-3){35}}
     \put(148,55){\vector(0,-1){10}}
     \put(142,38){$s_m$}
     \put(106,28){(Word Sense of $w$)}
     \put(67,1){{\bf Figure 1}~~~~~~~ The Process of WSD}
   \end{picture}
\end{center}
Word sense disambiguation starts with obtaining context-similarity 
vector, $V_{s_1}, V_{s_2}, \cdots, V_{s_r}$, for each sense of a polysemous word, $w$, 
using the word similarity metric in equation (2).  At the same time, we get the context
 representation, $V_{CR}$, for the same word, $w$, appearing in text.  We then calculate 
the similarity, $sense(V_{s_m},V_{CR})$, between each vector and the context representation. 
 Finally, we compare the values of 
similarities, $sense(V_{s_1},V_{CR}), sense(V_{s_2},V_{CR}), \cdots, 
sense(V_{s_r},V_{CR})$, and select the sense, $s_m$, with the 
highest value as the meaning of the polysemous word in question.

\subsection{Context Representation and Context-similarity Vector}

The context representation (CR) we use here is the sequence of content 
words (nouns, verbs, adjectives) that appear in a distance $l$ with the 
polysemous word in question.  It is defined as the vector:
\begin{equation}
V_{CR}=(w_1, w_2, \cdots, w_i, w_{i+1}, \cdots, w_n)
\end{equation}

Let $s_m$ be the $m$th sense of a polysemous word $w$.  Using equation (2), we 
can get a vector of each sense of a polysemous word. 
\begin{equation}
V_{s_m}=(w_1^m, w_2^m, \cdots, w_k^m)
\end{equation}
Here $w_1^m, w_2^m, \cdots, w_k^m$ are the set of the $k$ most contextually similar words of
 the $m$th sense of a polysemous word $w$.  $s_m$ and $w_i^m (i=1, \cdots, k)$  have big and 
similar mutual information with some other word.  We set $k$ to be 6 in this paper.

When the $m$th sense of a polysemous word $w$ in the corpus faces the data 
sparseness problem, we seek its synonym set in the WordNet, choose from the 
set a word that does not have the problem, and use it to calculate $V_{s_m}$.  When 
no such word is found, we use its coordinate word (WordNet 2001) that does not have 
the problem to calculate $V_{s_m}$.

\subsection{Distributional Matrix and the Meaning of Polysemous Word}

Using (3) and (4), we will get a distributional matrix of mutual information.
\begin{equation}
M(V_{s_m}, V_{CR})=(I(w_i^m, w_j))_{k \times n}~~~~~~~  (i=1, \cdots, k; j=1, \cdots, n)
\end{equation}

The matrix $M(V_{s_m}, V_{CR})$ expresses a distribution of the tight degree of association
 between the $m$th sense of a polysemous word $w$ and the context $V_{CR}$.  And we 
can get the density of the matrix $M(V_{s_m}, V_{CR})$.
\begin{equation}
\rho (V_{s_m}, V_{CR})=\frac{\sum_{i=1}^k \sum_{j=1}^n I(w_i^m, w_j)}{k \times n}
\end{equation}

If $I(w_i^m, w_j)$ is only a pure number, then $\rho (V_{s_m}, V_{CR})$ is 
only a mean value of some numbers.  However, $I(w_i^m, w_j)$ has linguistic significance,
 i.e., it also expresses the tight degree of association between two words.  
Consequently, $\rho (V_{s_m}, V_{CR})$ is not only a mean 
value but also expresses the tight degree of association between the $m$th sense 
of a polysemous word $w$ and the context $V_{CR}$.

In the context of natural language, there are noisy words or irrelevant
 information (Dagan and Itai 1994) in $V_{CR}$ to express the topic involved.  So we set the 
threshold $t$ to modify (6) so that we select the most informative context
 words (Gale, Church and Yarowsky 1992) for word sense disambiguation.  We 
throw away the mutual information
 of some words that have little contribution for disambiguation.

Consequently, we make the distance between densities enlarged and the difference 
of the senses of a polysemous word more evident.  The process will raise our 
confidence in identifying the word sense, especially, in the 
case $\rho (V_{s_1}, V_{CR}) \approx \rho (V_{s_2}, V_{CR})$.

In the matrix $M(V_{s_m}, V_{CR})$, we throw away the values of mutual information in $d$ most 
small values if the value is less than $t$.  For the rest, we calculate the density 
of mutual information.  We mark the density with $sense(V_{s_m}, V_{CR})$.  
The parameters $d$ and $t$ are adjustable.

Yarowsky (Yarowsky 1993, 1994a, 1994b) suggests that semantic or topic-based
 ambiguities require a 
large window of 20-50 words.  In our experiment, we use windows of $l$=10, 35, and 50 
words before and after the polysemous word to be disambiguated.  When $l=10$, we do 
not use $t$ and $d$.  For $l=35$ and $l=50$, we set $t$ to be
\[t=\frac{\sum_{i=1}^k \sum_{j=1}^n I(w_i^m, w_j)}{4k \times n}
\]
and $d$ to be $9k$ and $12k$, respectively.

Suppose the word appearing in a text is $w$, and its lexical meanings 
are $s_1, s_2, \cdots, s_r$.  We calculate the
 density $sense(V_{s_1},V_{CR}), sense(V_{s_2},V_{CR}), \cdots, sense(V_{s_r},V_{CR})$.
  The meaning of the word in consideration is $s_m$ that satisfies the maximal value.

\section{Experiment and Result}

To test our method, we use EDR English Corpus (EDR 1998) as training data and 10 polysemous
 words for which 682 instances were selected randomly from (Text1 1998; Text2 1998) and other 
materials.  The EDR corpus contains 160,000 sentences with annotated morphological, 
syntactic and semantic information.  And we set $t_0=0.5$, $t_1=0.75$, and $t_2=1.33$.

We predetermined the meaning of each instance of the polysemous word by two human subjects.  
The meaning of $w$ is determined by the following procedure:

\begin{enumerate}
\item Obtain the vectors $V_{s_1}, V_{s_2}, \cdots, V_{s_r}$ for the lexical meanings of $w$.
\item Get $V_{CR}$ in consideration using a window of $l$ words each before and after $w$ in the text. 
\item Calculate the density $sense(V_{s_1},V_{CR}), sense(V_{s_2},V_{CR}), \cdots, sense(V_{s_r},V_{CR})$.
\item Select $s_m$ that got the maximal density value to be the meaning of $w$.
\end{enumerate}

\subsection{Examples}

Consider the polysemous word in consideration to be {\it cabinet} in the following text and see how 
its meaning is determined:

\vspace{0.3cm}
\begin{quote}
\small $\cdots$ Chicago was the industrial inferno of the nineteenth century A.D.  A curious anecdote 
has come down to us of John Burns, a great English labor leader and one time member of the British 
{\it cabinet}.  In Chicago, while on a visit to the United States, he was asked by a newspaper reporter
 for his opinion of that city.  `Chicago,' he answered, `is a pocket edition of hell.'  Some time 
later $\cdots$ ({\it a text on Internet})
\end{quote}
\vspace{0.2cm}
{\it Cabinet} is given two nominal meanings: administrative organ ($s_1$) and a shelf ($s_2$).  Using 
equation (2), we get for them the context-similarity vectors:
\begin{quote}
$V_{s_1}$=({\it minister, administration, council, approve, prime, regular}) \\
$V_{s_2}$=({\it display, exhibit, store, maker, tea, shelf})
\end{quote}
The context representation $V_{CR}$ ($l=35$) from the text is:
\begin{quote}
$V_{CR}$=({\it $\cdots$, time, member, British, Chicago, visit, $\cdots$})
\end{quote}
The calculation of the distributional matrix produces the density 0.411462 
for $sense(V_{s_1},V_{CR})$ and 0.180325 for $sense(V_{s_2},V_{CR})$.  So, we decide the meaning
 of the {\it cabinet} to be $s_1$ (administrative organ).

Take another example for the word {\it sentence} ($l=35$) in:

\vspace{0.3cm}
\begin{quote}
\small $\cdots$ and there he died like a brave man.  He refused to have his eyes bandaged, saying that he was 
not at all afraid of death; and he admitted the justice of his {\it sentence}, and was much regretted by 
the people.  Although Mary had shrunk at the most important time from disproving her guilt, she was 
very careful never to do anything that would admit $\cdots$ ({\it Also a text on Internet})
\end{quote}
\vspace{0.2cm}
Again, {\it sentence} has two meanings: punishment ($s_1$) and group of words ($s_2$).  
The context-similarity vectors and the context representation are:
\begin{quote}
$V_{s_1}$=({\it prison, judge, murder, trial, crime, ruling}) \\
$V_{s_2}$=({\it word, code, language, meaning, symbol, accent}) \\
$V_{CR}$=({\it $\cdots$, death, admit, justice, regret, people, $\cdots$})
\end{quote}
Now, the density for $sense(V_{s_1},V_{CR})$ becomes 0.497328 and $sense(V_{s_2},V_{CR})$ becomes
 0.179224.  Thus, we get the meaning of the {\it sentence} as $s_1$ (punishment).
\subsection{Experimental Results}

Table 1 shows the 10 polysemous words, their senses, the number of their instances, and the 
instances identified correctly in our experiment using $l=35$ (Experiment 2).
\vspace{0.2cm}
\begin{center} \footnotesize 
  \begin{tabular}{llrr} 
   \multicolumn{4}{c}{\small {\bf Table 1}~~~~ Polysemous Words Tested} \\ \hline
   Words & Senses & Instances & Resolved \\ \hline
   band  &  group of musicians	&  19	&  18	\\
	      & strip or stripe	  & 4	 &  4	\\			
   cabinet & administrative organ &	 24 & 22 \\	
	      & shelf	&  17	&  16	\\				
   court &	judicial &	163 &	158	\\
	      & area for ball game &  9	&   8	\\				
   crane &	machine &	 16 &	 16	\\
	      & bird	& 21 &	 18	\\				
   palm &	tree &	 11 &	 11	\\
	     & hand	 & 52	&  42	\\				
   plant &	living thing &	 86 &	 81 \\	
      	& factory	&  18	 & 15	\\				
   sentence &	group of words &	 25 &	 22	\\
	       & punishment &	 67	&  59	\\				
   slug &	bullet	&   6	&   4	\\
	     & animal	&  16	&  16	\\				
    tank &	combat vehicle &	 12 &	 12	\\
      	& water-filled place &	 10 &	  8 \\				
    trial &	action of judging	 & 89 &	 82	\\
	     & test	&  17	&  15	\\ \hline
  \end{tabular}
\end{center}
\vspace{0.2cm}


Table 2 shows the disambiguation result (success rate).  Experiment 1, Experiment 2, and 
Experiment 3 are for $l$=10, 35, and 50, respectively.
\begin{center} \scriptsize 
  \begin{tabular}{c|cccccccccc|c} 
   \multicolumn{12}{c}{\small {\bf Table 2}~~~~~~~~~~~~~ Disambiguation Result(\%)} \\ \hline
      & band	& cabinet & court & crane & palm & plant & sentence & slug & tank & trial & Average \\ \hline
       Experiment 1 & 87.0 & 82.9 & 91.9 & 91.9 & 87.3 & 92.3 & 87.0 & 77.3 & 95.5 & 89.6 & 89.4 \\	
       Experiment 2 & 95.7 & 92.7 & 96.5 & 91.9 & 84.1 & 92.3 & 88.0 & 90.9 & 90.9 & 91.5 & 91.9 \\	
       Experiment 3 & 95.7 & 97.6 & 95.9 & 97.3 & 84.1 & 91.3 & 80.4 & 72.7 & 90.9 & 96.2 & 91.3 \\ \hline 
  \end{tabular}
\end{center}
\vspace{0.2cm}
As is seen in Table 2, the overall average success rates are 89.4\%, 91.9\%, and 91.3\%, respectively, 
for Experiment 1, Experiment 2, and Experiment 3.  The best overall result is got 
when $l=35$ (Experiment 2).  The best single result, 97.6\%, is got for {\it cabinet} when $l=50$ and
 the worst is for {\it slug} when $l=50$.

\section{Evaluation}

The upper-bound for the success rate in our experiment is supposed to be 100\%, which is that 
of human beings.  The lower-bound or the base-line success rate is 80.5\%, which is the sum 
of bigger instances in each word divided by the total instances, i.e., 549/682.  Obviously, 
our result is much better than that of the base-line.

Methodologically, our approach is similar to Dagan, Marcus, and 
Markovitch (Dagan, Marcus and Markovitch 1995) and Sch\"{u}tze (Sch\"{u}tze 1998) in 
that all use some kinds of word similarity measures.  Dagan, Marcus, and Markovitch 
(Dagan, Marcus and Markovitch 1995) used 
Augmented TWS method to disambiguate the word senses.  It relied on the adjacent word only of a 
polysemous word to disambiguate the word senses and was doomed to fail, 
at least for noun disambiguation, 
since not enough disambiguating information would be provided by the adjacent word alone.  
In this paper, we considered the global context using a window of $l$ words each before and after the 
ambiguous word in the text, and took more words into account to make use of enough disambiguating 
information.  Sch\"{u}tze (Sch\"{u}tze 1998) used an idea of Word Space that was
 a high-dimensional, real-valued 
space.  Then he got vectors of huge dimensions, in which many irrelevant information to express 
the topic involved was put.  Here in our paper, we used the $k$ most similar words and got better 
vectors, capturing effective information for disambiguating word sense.

Experimentally, our result is better than the ones in other studies 
(e.g., Chen and Chang 1998; Hiro, Wu and Furugori 1996; Li, Szpakowicz and Matwin 1995).  But this 
does not necessarily mean the superiority of our method: no experiments are done under the same 
settings and they differ in conditions and data used.

Previously, Peng, Takakura, and Furugori (Peng, Takakura and Furugori 2001) experimented
 a Dagan-like word similarity 
measure (Dagan, Marcus and Markovitch 1995) for the same data we used in the present paper.  The similarity measure used was:
\begin{equation}
sim(w_1,w_2)=\frac{\sum_{w \in Lexicon}min(I(w,w_1),I(w,w_2))}{\sum_{w \in Lexicon}max(I(w,w_1),I(w,w_2))}
\end{equation}
The success rate got from this experiment was 91.5\%.  Percentage-wise, this result is 
comparable with ours, but our present method is superior in the following two points.

First and computationally, the equation (2) makes the calculation time less than that of 
using equation (7) since $\|S\| \ll \|Lexicon\|$.

Second and methodologically, the equation (2) is free from a deficiency seen in 
equation (7).  Let us see this in examples for $\|Lexicon\|=8$.  There would be cases in
 using equation (7).
\begin{center} 
  \begin{tabular}{ccl} 
    {\it case 1} & (a) & $sim(w_1,w_2)=\frac{0.27+0.27+0.27+0.27+0.27+0+0+0}{0.32+
       0.32+0.32+0.32+0.32+0+0+0}$ \\
                    & (b) & $sim(w_3,w_4)=\frac{2.7+2.7+2.7+2.7+2.7+0+0+0}{3.2+3.2+3.2
       +3.2+3.2+0+0+0}$ \\
    {\it case 2} & (a) & $sim(w_1,w_2)=\frac{2.7+2.7+0.2+0.2+0.2+0+0+0}{3.2+
       3.2+0.3+0.3+0.3+0+0+0}$ \\
                    & (b) & $sim(w_3,w_4)=\frac{2.7+2.7+2.7+2.7+2.7+0.2+0.2+0}{3.2+3.2+3.2
       +3.2+3.2+0.3+0.3+0}$ \\
   \end{tabular}
\end{center}
\vspace{0.3cm}


In {\it case 1}, 2.7 in $(b)$ is far bigger than 0.27 in $(a)$ and the same 
is true for 3.2 and 0.32.  Nevertheless, we get $sim(w_1,w_2)=sim(w_3,w_4)$.  
In {\it case 2}, the numbers of the first two terms $(2.7,3.2)$ in $(a)$  are 
very big, while those of the rest are very small, and the numbers of the first five terms 
in $(b)$ are very big, while those of the rest are very small.  Nevertheless, 
we get $sim(w_1,w_2) \approx sim(w_3,w_4)$.

As is clear already, these problems are eliminated in equation (2).  But then a question remains 
that why our result is not much better than that of using equation (7).  An answer to this question 
is that $sim(w_1,w_2)$ and $sim(w_1,w_3)$ share the mutual information $I(w,w_1)$ in
 equation (7), so unfavorable effect of {\it cases 1} and {\it 2} is there not in full force.
  Another is that the noise effect of {\it cases 1} and {\it 2} is negligible as we used the
 small value of $k$ (=6) in both experiments.  We are 
sure that the difference in performances between equations (2) and (7) would become clearer 
as the value of $k$ becomes bigger.

\section{Conclusion}

We proposed a method to calculate the similarity of two words.  Using the similarity metric
 in equation (2) we designed an approach for WSD that uses the context-similarity vector for
 resolving ambiguities of polysemous words in text.

Our method is intuitional and its performance is in an acceptable level. However, to 
improve the performance, it may be worthwhile to consider such points as:
\begin{itemize}
\item Dynamic adjustment of parameters $d$ and $t$ for extracting the topic effectively from the context.
\item Invertion of the distributional matrix to distributional image in the two dimensional coordinates, 
      and the determination of the sense according to distributional character of the image.
\end{itemize}

In our method and many others, context is the means to identify the meaning of a polysemous word.  But 
some word senses are nonspecific to topics and a word with such a sense can appear freely in many 
domains of discourse.  It is desirable to give a thought to deal with the meaning of polysemous words 
used in the nontopical sense (Leacock, Chodorow and Miller 1998). However, within a restriction, 
we contend that our method is 
theoretically sound and better in generality and simplicity.






\bibliographystyle{nlpbbl}
\bibliography{epaper}



\begin{biography}

\biotitle{}

\bioauthor{Qujiang Peng}
{
Qujiang Peng received a B.S. degree in Mathematics and Computer Science 
from Chongqing Normal University, China, in 1994 and a M.S. degree in 
Applied Mathematics from Chongqing University, China, in 1997. He is a Ph.D 
candidate in Computer Science at the University of Electro-Communicatins, 
Tokyo, Japan.
}

\bioauthor{Haodong Wu}
{
Haodong Wu received a B.S. and a M.S. in Computer Science from Chongqing 
University, China, in 1983 and 1986. He got a Ph.D. in Computer Science from 
the University of Electro-Communications, Tokyo, Japan, in 1997. He is an 
associate professor at Dokkyo University. His current research interests 
include Computational Linguistics, Information Retrieval, and Artificial 
Intelligence. He is a member of the Information Processing Society of Japan, 
the Association for Natural Language Processing of Japan, and the Association 
for Computational Linguistics.
}

\bioauthor{Teiji Furugori}
{
Teiji Furugori is a professor in the Department of Computer Science at the 
University of Electro-Communications, Tokyo, Japan. He has a Ph.D in Computer 
Science from SUNY at Buffalo. His research interests include Computational 
Linguistics, Natural Language Processing, and Cognitive Science. He is a member 
of ACM, the Information Processing Society of Japan, the Institute of Electronics, 
Information and Communication Engineers of Japan, and the Association for 
Natural Language Processing of Japan.
}


\bioreceived{Received}
\bioaccepted{Accepted}

\end{biography}

\end{document}


