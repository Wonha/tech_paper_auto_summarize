\documentstyle[jnlpbbl]{jnlp_j}

\setcounter{page}{21}
\setcounter{巻数}{9}
\setcounter{号数}{1}
\setcounter{年}{2002}
\setcounter{月}{1}
\受付{2001}{5}{10}
\採録{2001}{9}{28}

\setcounter{secnumdepth}{2}

\makeatletter
\def\le{}
\def\ge{}
\def\gl@align#1#2{}

\let\@ARRAY \def\@array{}
\def\<{}
\def\>{}
\def\|{}
\def\Underline{}
\def\endUnderline{}
\def\LATEX{}
\def\LATEx{}
\def\LATex{}
\def\iLATEX#1{}
\def\LATEXe{}
\def\LATExe{}
\def\Quote{}
\let\endQuote
\def\TT{}
\def\CS#1{}
\makeatother


\title{手指動作記述文間の類似性に基づく\\
手話単語の検索方法}
\author{安達 久博\affiref{UIS}}


\headauthor{安達 久博}
\headtitle{手指動作記述文間の類似性に基づく手話単語の検索方法}

\affilabel{UIS}{宇都宮大学 工学部 情報工学科}
{Department of Information Science, Utsunomiya University}

\jabstract{
手話は聴覚障害者と健聴者との重要なコミュニケーション
手段の 1つであり,手話を学習する健聴者の数も年々増加
する傾向にある.この様な背景から,近年,手話の学習支
援システムや手話通訳システムなどの研究が各所で盛んに
行われている.
特に,これらの自然言語処理システムの知識辞書となる手
話電子化辞書の構築は重要な課題であり,手話側から対応
する日本語ラベルを効率良く検索する手段の実現は,日本
語と手話との対訳辞書の検索機能として,必要不可欠な要
素技術といえる.
従来の検索方法の多くは,手話単語の手指動作特徴を検索
項目とし,検索条件を詳細に設定する必要があった.その
ため,初心者には満足する検索結果を得ることが難しいと
いう問題点が指摘されている.この主な原因の 1つは,検
索条件の複雑さや検索項目間の類似性から選択ミスが生じ
やすく,結果として,利用者の要求に適合する検索結果が
出力されないという問題点にある.
本論文では,市販の手話辞典に記載されている手指動作記
述文に着目した検索方法を提案する.
本手法の特徴は,検索キーとして入力された手指動作記述
文と類似の手指動作記述文を検索辞書から検索し,対応す
る手話単語の日本語ラベルを利用者に提示する点にある.
すなわち,手話単語の検索問題を文献検索問題と捉えたア
プローチといえる.
実験の結果,本手法の妥当性を示す結果が得られた.一方,
実験により明らかになった問題点の 1つとして,手指動作
記述文で表現された手指動作の一部に曖昧さがあることが
分かった.この問題を含め,本手法の問題点と今後の課題
について,例を示しながら詳細に議論する.
}

\jkeywords{手話,情報検索,類似度,手指動作記述文,電子化辞書}


\etitle{A Retrieval Method of Signs Based on\\
Similarity between Manual Motion\\
        Descriptions}
\eauthor{Hisahiro Adachi \affiref{UIS}}

\eabstract{
Sign language is an important path for us to communicate
with hearing impaired people. Therefore, learners has been
increasing in recent years.
There are several researches into learning aid systems
and electronic dictionaries for sign language.
Especially, when users want to look up the Japanese word
labels corresponding with manual motion properties,
most of previous retrieval methods are necessary to set
various and many retrieval conditions in detail.
There is a serious problem that it is hard for beginners
to look up the most appropriate sign by the setting
wrong conditions.
To overcome this problem, this paper proposes a method
which uses a different approach from
the previous methods.
The point of the method is based on the similarity between
manual motion descriptions(MMDs) appeared in ordinary sign
dictionaries.
By computing the similarity between an inputted MMD as
a query and the MMDs in the database, 
retrieval results are outputted in similarity order.
The retrieval results formed by the similarity
can be considered as a set of signs that are similar to
each other.
As an interesting point, a subject of sign retrieval can
be considered as the document retrieval.
The results of evaluation experiments show the applicability
and usability of the proposed method.
We also discuss a problem that there are ambiguous MMDs
by demonstrating examples. 
}

\ekeywords{Sign language, Information retrieval, Similarity, Manual motion description, Electronic dictionary}

\begin{document}
\thispagestyle{plain}
\maketitle


\section {はじめに}\label{No1}

近年,聴覚障害者の重要なコミュニケーション手段の 1つである
手話と,健聴者のコミュニケーション手段である日本語とのコミ
ュニケーション・ギャップの解消を目的とする手話通訳システム
や手話の学習支援システムなどの研究が各所で盛んに行われてい
る\cite{Adachi1992a}.

これら手話を対象とした自然言語処理システムを実現するための重要な
要素技術の1つである手話の認識や生成処理技術は,動画像処理の研究分野であるが,
対象が限定されているため,動画像構成の単位を明確に規定できる可能性があり,
手話の知的画像通信や手話画像辞書への特徴素の記述法が提案されている
\cite[など]{Kurokawa1988,Kawai1990,Sagawa1992,Terauchi1992,Nagashima1993,JunXU1993}.

また,日本語文の手話単語列文への言語変換処理に関する基礎検討としては,
\cite[など]{Kamata1992,Adachi1992b,Adachi1992c,Kamata1994,Terauchi1996}が
報告されている.
さらに,手話表現の認識結果を基に日本語文を生成する研究としては,
\cite[など]{Sagawa1992,Abe1993}がある.
なお,これらの処理精度に影響を与える電子化辞書の構成方法に関しては,
\cite[など]{Adachi1993,Nagashima1993,Tokuda1998}が提案されている.

さて, 2言語間の対訳電子化辞書システムを構築する場合の
重要な要素技術の 1つとして,原言語側と目標言語側との
双方向から単語を検索できる機能の実現が挙げられる.
ここで,手話単語を対象とした場合の課題の 1つは,視覚言語としての特徴
から,手指動作表現を検索キーとし,対応する日本語の単語見出し(以後,
本論文では,日本語ラベルと略記する.)を調べる手段をどのように実現す
るかという点である.
すなわち,視覚情報としての手指動作特徴をどのように記号化して,検索要
求に反映させ,検索辞書をどのように構成するかという問題といえる.
この問題に対する従来のアプローチは,手の形,動き,位置などの手指動作
特徴の属性を詳細な検索項目として用意し,これらの項目間の組合せとして検
索条件を設定し,同様に,これらの検索項目を基に手話単語を分類したデータ
ベースを検索辞書としていた\cite[など]{KatoYuji1993,Naitou1994}.
この検索アプローチは,手話言語学における手話単語の表記法(単語の構造記
述におけるコード法)に基づいている\cite{Kanda1984,Kanda1985}.

しかし,これらの表記法と分類観点は,本来,個々の手話単語の表現を
厳密に再現(記述)することを目的としているため\cite{Stokoe1976},項目
数が多く,また,項目間の類似性もあり,初心者には難解なコード体系といえ
る.
そのため,このアプローチによる検索システムの問題点が,\cite{Naitou1996}
により指摘されている.それによると,手の形,動き,位置などの検索条件を
指定する場合,

\begin{enumerate}
\item 検索項目間に類似性が高いものがあり,利用者が区別しにくく,
\item 検索条件や検索項目が多くなると,利用者は選択操作が煩わしく
なり,
\end{enumerate}

\noindent 結果として選択ミスを生じ,満足のゆく結果が得られないとされる.
これは,利用者が認知した手話表現の手指動作特徴を再生し,検索条件に設定
する場合に,不必要な検索条件までも指定してしまう点に原因があるといえる.

一方,認知された外界の情報を,ある表現形式(表象)から別の表現形式に
変換することを,一般に,コーディング(符号化)と呼ぶ.
また,視覚的な特性を持つ「視覚的コード」と言語的な特性を持つ「言語的
コード」を重要視する「二重コード説」によると,写真などの視覚情
報を記憶する場合に,視覚的コーディングに加え,「赤い色をした車」のよう
に言語的コーディングも同時に行われているとされる\cite{Ohsima1986}.
さらに,単語は文字(あるいは音素)の組み合わせで構成されるが,例えば,
(1)「キ」を提示した後で,それは「キ」あるいは「シ」だったのかを質問した
場合と,(2)「テンキ」を提示した後で,それは「テンキ(天気)」あるいは
「テンシ(天使)」だったのかを質問した場合とでは,
(2)の方が成績が良いとされ,文字の弁別が単語という文脈内で規定された方が
より正確に記憶するとされる「単語(文脈)優位性効果\cite{Reicher1969}」が
知られている.
これら認知科学の成果を手話単語の検索問題に当てはめて考えてみると,
人間が手指動作表現を認知する場合,「両手を左右に動かす」というように,
言語文として言語的コーディングを行い記憶しているとすれば,
記憶された言語的コード,すなわち,手指動作特徴を記憶した際の文脈環境
を保持する言語文そのものを検索キーとするアプローチが考えられる.

本論文では,検索条件を細かく指定する従来の方法とは異なり,
手話単語の手指動作特徴を日本語文で記述した手指動作記述文
を検索条件とみなし,辞書にある類似の手指動作記述文を類似検索し,
検索結果に対応付けられている日本語ラベルを提示する方法を提案する.
本手法の特徴は,手話単語の手指動作特徴間の類似性を
手指動作記述文間の類似性と捉え,入力された手指動作記述文と
辞書に格納された手指動作記述文との類似度を計算する点にある.
すなわち,「{\bf 手話単語の検索問題を文献検索問題と捉えたアプローチ}」といえる.
また,この手指動作記述文は,一般に,市販の手話辞典に記載されており,
手話の学習者の多くが,慣れ親しんでいる文形式と捉えることができる.

なお,本提案手法に関連する研究として,翻訳支援を目的
とした対訳用例の類似検索に関する研究が
幾つか報告されている\cite[など]{NakamuraNaoto1989,SumitaEiichiro1991,SatoSatoshi1993,TanakaHideki1999}.
これらにおいては,文間の類似度の計算に用いる照合要素として,文字を対象
とする方式と単語を対象とする方式に大別することができる.
また,これらの要素間の照合戦略としては,出現順序を考慮しながら共有要素を計
算する方式(以後,順序保存と略記する.)と,出現順序を考慮しない計算方式
(以後,順序無視と略記する.)に大別することができる.
ここでは,照合要素が文字列と単語列という違いはあるが,
順序無視と順序保存の照合戦略を用いた代表的な 2つの手法について概説する.

\cite{SatoSatoshi1993}は文字の連続性に着目した文間の類似性を基準に,
「最適照合検索」として,順序保存を採用した検索システム(CTM1)\cite{SatoSatoshi1992}と順序無視を採用した(CTM2)の検索効率を比較し,ほぼ同等であるが順序無視
の方が若干優位としている.
一方,\cite{TanakaHideki1999}は,放送ニュース文の
単語列を対象に,AND検索に順序保存の制約条件を加え,
長文に対する効果的な用例検索法を提案し,順序保存の方が優位としている.
なお,両者とも類似性を計る指標として,語順(あるいは文字の出現順序)を考慮
するアプローチの重要性を指摘している.
このことは,文構造の類似性を表層情報として得られる
文形式(単語の配列順序)の類似性を文間の類似度に加味することの意義を示唆している.

以下,2章で,手指動作記述文の特徴について述べ,
3章では,手指動作記述文間の類似度と手話単語の検索方法について述べ,
4章で,提案手法の妥当性を検証するために行った実験結果を示す.
5章では,実験により明らかとなった問題点について議論し,
今後の課題について述べる.最後に,6章で,まとめを行う.

\section {手指動作記述文の特徴}\label{tokuchou}

一般に,市販されている手話辞典の多くは,
手話単語の日本語ラベルに対して,その手指動作表現の手続きをイラスト
以外に,日本語文で表現した手指動作記述文
を記述している\cite[など]{MaruyamaKoji1984,Watashi1986}.
そこで,以下に示す手話単語対の「午前」と「午後」を例に考えてみると,
この単語対は,手指動作特徴の中で「手の動き(右に倒すのか左に倒すのか)」に
関する手指動作特徴だけが異なる手話単語の最小対を構成している.
明らかに,与えられた手指動作記述文間の比較からも,この関係を導出すること
ができる.

\begin{enumerate}
\item {\gt 午前} 右手の人差指と中指を立て額の中央に当て\underline{右に倒す}
\item {\gt 午後} 右手の人差指と中指を立て額の中央に当て\underline{左に倒す}
\end{enumerate}

このように,手指動作記述文は,手話単語の手指動作表現の特徴構造を,
構造を持つ1次元の記号系列(言語文)に写像した特徴系列と捉えることができる.
この特徴系列は,日本語文であり,文法構造を内在しており,
手話単語の手指動作表現を生成するための手続きを記述したプログラム
(手続き文)と捉えることもできる.

また,手指動作記述文は一般の自然言語文に比べ,
文中で用いられる語彙には,ある種の制約があると捉えることができる.
例えば,名詞は手指動作表現を特徴付ける手や顔,胸などの身体の上半身の部位を表す
ものに限定されている.
また,動詞は手指の形や動きを表現するものに限定されている.
さらに,副詞は手指動作の強弱や反復などの程度を表すものに限定されている.

同様に,動作表現を生成する手続き文としての特徴から,
文中での単語の配列(文形式)に,ある種の構文パターンがあると考えられる.
例えば,「午前,午後」の例が示すように,「手の形,手の位置,手の動き」の順に
手指動作特徴を表す単語列が配列されている.

これらの特徴から,手指動作記述文は言語の使用環境が,一般の自然言語文に比べて,
語彙的にも構文的にも強い制約を受けた有限の文集合と捉えることができる.


\section {手指動作記述文を用いた手話単語の類似検索}\label{Sign}


\cite{AdachiHisahiro2000}は,与えられた単語集合から
類似の手指動作特徴を含む手話単語対を抽出する方法として,
手話単語間の手指動作特徴の類似性を手指動作記述文間の
類似性と捉えたアプローチによる方法を報告している.
そこでは,手指動作記述文を,文を構成する文字配列を特徴観点
とする$n$次元の特徴ビットベクトルで表現し,以下に示す
文間の類似度の計算式(\ref{sim})を導出している.

\begin{equation}
\label{sim}
S(A,B) = \frac{LCS(A,B)^2}{M・N}
\end{equation}


\noindent ここで,$M,N$はそれぞれ,手指動作記述文$A,B$の長さ(文字数)を
表し,$LCS(A,B)$は最長共有部分列の長さ\cite{Thomas1990}を表している.


\subsection {手話単語の類似検索への適用}

\cite{AdachiHisahiro2000}では,
ある1つの辞書に記載の手指動作記述文の文集合をその処理対象としている.
一方,本論文で対象とする手指動作記述文の文集合は,検索データベース,
すなわち,検索辞書としての手指動作記述文(以後,{\bf 辞書記述文}と略記する.)
と,検索要求としての手指動作記述文(以後,{\bf 検索記述文}と略記する.)
というように,異なる 2つの文集合を対象とする必要がある.
すなわち,一般の情報検索システムで問題となる利用者の質問と辞書との表現上の差\cite{Nagao1983}の問題と同様に,
検索記述文と辞書記述文とのギャップについて考慮する必要がある.

例えば,辞書記述文は手話表現を生成する手続き文(プログラム文)としての特徴から,
手の形,手の位置,手の動きなどの手指動作特徴を詳細に記述していると捉えることができる\footnote{必ずしも,全ての手指動作特徴を洩れなく記述している訳ではなく,イラストの情報との相補関係にあり,省略されている特徴素もある}.
一方,検索記述文は手話の学習者の入力を想定しており,辞書記述文に比べ特徴素の
省略などにより,より簡潔な文となる可能性が高い.
しかし,手話の学習者が日常,手話単語を学習する際に参考としている手話辞典等で,
慣れ親しんでいる辞書記述文と類似の文形式で入力するであろうと考えることができる.

そこで,本研究では,このようなギャップがある手指動作記述文間の類似性の計算に
適応させるため,
検索記述文と辞書記述文の両者を
疑似文節列に分割し,各疑似文節から平仮名以外の文字列のみを抽出した
非平仮名キーワード列を
特徴観点とした類似度の計算を行うこととする.


\subsection {疑似文節列と検索キーワード列の抽出}

漢字かな混じりの日本語文を形態素解析処理を用いないで,文節列に区切る
ヒューリスティック・ルール(経験則)として,
以下に示す字種の違いに着目したものが考えられる.

\subsubsection {文節区切りの経験則}

\begin{description}
\item 平仮名文字が非平仮名文字へ字種が変位する境界部分を文節の区切りとする
\end{description}

\medskip
例えば,「両手を交互に上下させながら左右に引き離す」をこの経験則で分割すると,
『両手を/交互に/上下させながら/左右に/引き/離す』となる.
この分割結果を本論文では,疑似文節列と呼ぶ.
ここで,「疑似」という用語を冠した理由は,以下に述べるように,一般に,文節
とは認定できない分割結果を含む場合があるためである.

\subsubsection {経験則の問題点と利点}

先に示した経験則による文節分割の問題点は,
上記の例文にある複合動詞「引き離す」のような{\bf 混ぜ書き}で表現される
文要素を分割してしまう点と,「上下させながら」のように
平仮名で表記される文要素は分割できない点である.
しかし,上記の例文を辞書記述文とし、
検索記述文に同じ意味を表す別の表現として「〜を左右に引く」や
「〜を左右に離す」があった場合,
混ぜ書きされた文要素を分割してしまう経験則の不具合は,
逆に,このような同義関係にある文要素同士を類似度に反映できる利点があると考える.
一方,この文字列の照合処理において,完全一致では動詞の活用部分の差や
用いられる助詞の差(例えば,「左右に」と「左右へ」の関係)など,
自然言語表現の持つ「語形の多様性」により同一視が難しいのは明らかである.

\subsubsection {疑似文節列から非平仮名キーワード列へ}

そこで,疑似文節列から語幹に相当する非平仮名列のみを
抽出し,照合処理の対象キーワードとし,語形の多様性を吸収することとする.
これにより,与えられた手指動作記述文からのキーワード列の抽出は,
非平仮名文字列\footnote{漢字に限定しないのは,指文字等は一般にカタカナで表記される傾向がある点と,
平仮名書きの文要素を,カタカナで表記あるいは置換できれば,比較的容易に類似度に反映させることが可能と考えたためである.なお,カタカナへの置換は今後の課題とする.}
だけを対象とし,平仮名文字列は無視することとする.
このため,漢字やカタカナで表記されていない文要素は
類似度の計算では考慮されず,
仮に,その文要素がその手話表現を特徴付ける場合であっても,
類似度には反映されないという問題は残るが,
本論文では,他の文要素間の類似性により文間の類似性を近似する
こととし,この問題は今後の課題とする.

以上の議論から,検索記述文と辞書記述文との類似性を近似する類似度は,
与えられた手指動作記述文から非平仮名の連続文字列を
照合要素とし,以下に示す式(\ref{sim2})を用いて計算する.
ここで,式(\ref{sim})と区別するため,$M_k,N_k$はそれぞれ,手指動作記述文$A,B$の非平仮名キーワード列の総数であり,$LCS_k(A,B)$は両者の最長共有キーワード部分列の長さを表す.
なお,照合の際に次節で述べる「語選択の多様性」に対処する処理を適用する.

\begin{equation}
\label{sim2}
S(A,B) = \frac{{LCS_k(A,B)}^2}{M_k・N_k}
\end{equation}

\subsection {キーワード照合における不要語と同義語の扱い}

一般に,文献検索システムでは,出現頻度が高く
文献の特徴付けに寄与しないキーワードを不要語(stop word)として,
検索対象キーワードから除外する方法が採られる.
手話単語は,一般に,両手の形が同一である「両手同形」,形が異なる「両手異形」
と片手で表現する「片手」の3種類に大別される.
ここで,両手同形の手話単語を例に考えると,
市販の手話辞典に記載の手指動作記述文には,以下に示す例文(1)と(2)のように,
「手の形」を規定する部分が「両手」に対して,前置される場合と後置される
場合が混在している.

\begin{enumerate}
\item \underline{五指を折り曲げた}両手を交互に上下する
\item 両手\underline{の五指を折り曲げて}交互に上下する
\end{enumerate}

キーワードの出現順序を考慮する照合戦略では,利用者の検索要求を表す
検索記述文が例文(2)と同一であった場合,例文(1)の共通キーワード数は例文(2)
より少なくなる.この照合洩れを抑止するために,「両手」を不要語として,
検索記述文と辞書記述文の両方から除外すれば,
例文(1)と(2)の共通キーワードの数は同一となる.
同様に,片手手話に分類される単語は,手話単語の基本形であり,一般に,
「右手」を用いて表現する.
そのため,利用者の検索記述文中で省略される可能性がある.
このように,検索システム全体
の構成として,最初に,両手同形なのか片手の手話なのかを利用者に指定させる
ことで,片手手話の検索の際には「右手」を,両手同形手話の場合には「両手」を
その照合対象から除外する戦略は,類似検索の照合処理に有効に機能すると考える.

次に,検索記述文と辞書記述文のキーワードの一致を前提に照合を行なう
本手法では,本来,同一であるべき照合要素が言語の多様性により別の表現となり,
一致しない場合がある.
例えば,「前方に出す」と「前に出す」の関係における「前方」と
「前」との不一致である.
また,手の動作位置などに関して,利用者の認識と辞書側
の記述表現の差,すなわち,全体/部分の関係の捉え方の違いに起因する
不一致が考えられる.
例えば,辞書記述文では「胸の前」と記述されている部分に対して,
利用者の検索記述文では「体の前」と記述する場合である.
同様に,「口の前」と「顔の前」などが対応する,
そこで,本手法では表\ref{dougi}に示すように,キーワード間の照合で
同一視する同義置換表を用意し,これら「語選択の多様性」の一部を
吸収することにする.

このように,前節で述べた平仮名文字列を照合要素から除外する戦略が,
語形に関する多様性に対処する枠組であり,
本節で述べた処理は語選択の多様性に対する枠組と捉えることができる.


\begin{table}[htbp]
\caption{照合時に同一視するキーワード間の同義置換表の一部}
\label{dougi}
\begin{center}
\footnotesize\tabcolsep=3pt
\begin{tabular}{l|l}\hline
同一視するキーワード群   & 文中で用いられる例   \\ \hline\hline
前,前方      & (前,前方)に出す \\ \hline
伸,立        & 親指を(伸ばす,立てる) \\ \hline
開,離,広    & 左右に(開く,離す,広げる) \\ \hline
体,胸,腹    & 両手を(体,胸,腹)の前に\\ \hline
\end{tabular}
\end{center}
\end{table}


\section {実験と評価}

本手法の有効性を確認するため,本論文では,両手同形の手話単語集合を
対象として,実験を行う.
\cite{Naitou1996}の調査分析によれば,
両手同形の手話単語は手話単語全体(2,524語)
の41.7\%を占め,その中の90\%は両手の移動を伴う手話単語であり,
移動を伴う手話単語は手話単語全体の84\%と報告されている.
また,\cite{Kamata1991}は,
手指動作特徴の中で「手の動き」の重み付けは,他の特徴素よりも大きいと指摘
している.これは,\cite{Naitou1996}の分析でも両手同形の手話単語は
他のタイプの手話単語に比べ,「手の形」が限定されており,手の動きが単語の弁別特徴素として
働く割合が高い手話単語であるとの調査結果と一致する.

これらの分析から,手話単語の大部分が動きを伴う表現であり,
特に,両手同形の手話単語では,その割合が高く,
本論文で提案する手法の妥当性を検証するのに適していると考える.

\subsection {実験データとその特徴}

実験データは以下の手順で準備したものを用いた.
まず,検索記述文として「わたしたちの手話(1)」\cite{Watashi1986}に
記載の手指動作記述文を,
辞書記述文として「イラスト手話辞典」\cite{MaruyamaKoji1984}
に記載の手指動作記述文を用い,
文字列「両手」を含む両手同形の手話単語を抽出した.
なお,複合手話表現と明示されている手話単語は,実験対象から除外した.

次に,実験結果の分析・評価を明確にするため,抽出された
検索記述文の日本語ラベルが,辞書記述文の文集合(613文)の日本語ラベル
と対応関係にある検索記述文を選択し,最終的な検索記述文の文集合(87文)とした.
ここで,両者の原辞書間で日本語ラベルが一致しないものがあるため
\footnote{例えば,(離れる,別れる),(月日,いつ),(つまり,まとめる),(失う,なくす)などの対応関係を同定した.},
原辞書の索引等を比較し,検索記述文の手話単語が辞書記述文に確実に含まれる
検索記述文を選定し,人出により計算機に入力したものを実験データとして準備した.

なお,表\ref{sample}には,検索記述文と辞書記述文の文字数の分布状況を示す.
ここで,辞書記述文より検索記述文の方が
短い文で構成されていることから,
大まかな動作特徴を検索キーとする傾向にある利用者側の要求を
反映した実験データと捉えることができる.

\begin{table}[htbp]
\caption{実験に用いた手指動作記述文の文字数による比較}
\label{sample}
\begin{center}
\footnotesize\tabcolsep=3pt
\begin{tabular}{c|r|r|r|r|r|r} \hline
           &総文数 & 平均字数 & 最大字数 & 最小字数 & 20字未満 & 20字以上 \\
\hline\hline
検索記述文 &   87文&    18.85 &  32  &    7 &   48文     &  39文   \\ \hline
辞書記述文 &  613文&    31.71 &  78  &   10 &   48文     & 565文   \\ \hline
\end{tabular}
\end{center}
\end{table}


\subsection {実験方法}

実験は,まず,
1次情報である検索記述文から非平仮名文字のみを
検索キーワード列として抽出した2次情報を作成する.
同様に,辞書記述文から
抽出した非平仮名キーワード列の2次情報を作成する.
そして,2次情報同士の文字列照合により類似度を求め,
類似度の値が高いものほど
上位に位置するように整列したものを検索結果とする.
なお,共通キーワード数の計算には,順序保存と
順序無視の照合方式をそれぞれ用いた類似度を計算し,
検索結果を比較する.
ここで,順序保存の類似度は式(\ref{sim2})で計算し,
順序無視の場合には,式(\ref{sim2})の$LCS_k(A,B)$を,重複を
許す形で照合を行なった共通キーワード数$C_k(A,B)$として類似度を計算する.
また,2次情報との比較のため,
与えられた1次情報の文字を照合要素とする類似度を
式(\ref{sim})を用いて同様に計算し,
検索結果を比較する.

\subsection {評価方法}

一般に,情報検索システムあるいは手法を評価する場合,利用者側の立場からみた
検索効率を評価する尺度として,従来,呼出率と
適合率が用いられてきた.
ここで,呼出率とは,被検索対象である文書集合の中で,検索要求文に適合すると判断
できる文書総数と,実際に検索された適合文書数との比で計算される.
一方,適合率は検索された文書総数と,その中の適合文書数との比で計算される.
なお,最近では,これらの指標が利用者の多様な検索要求に対して,必ずしも適切な
評価尺度とはならないとの問題点が指摘されている\cite{Tokunaga1999}.
本論文で対象とする手話単語を検索する場合,検索要求に適合する辞書の手指動作
記述文は原則として1つである.
すなわち,複数の適合手指動作記述文(に対する手話単語)を
前提としていないため,
適合率と呼出率による検索効率の評価は適していないと考え\footnote{現実には,予め,同一の手話表現に対する手指動作記述文を1つにマージしてない場合,複数の適合文が存在することになるが,
すべての検索要求に対して,複数の適合文がある訳ではなく,その数は限られている.},
以下で述べる評価尺度を用いることにする.

\subsubsection {平均検索成功率}

手話単語の検索の場合,一般的に,
利用者の第一義的な検索要求は,「当該手話表現の日本語ラベルは何かを知りたい」
であり,順序付けられた検索結果の上位に,対応する日本語ラベルが位置することが
検索システムには求められる.

さらに,現実的には,利用者は当該手話表現に適合する
検索結果を上位から逐次調べる必要があるため,その手間数が少ない方が良い.
すなわち,利用者の立場から見ると,
その当該手話単語が見つかれば要求は満たされ,原則,それ以上は調べる必要性はない.

しかし,手話単語を学習(手話表現を習得し理解を確実に)するためには,類似の動作特徴を持つ他の
手話単語との関係を同時に学習することは重要である.一般に,音声言語の単語習得
においても,他の単語との関係から当該単語の特徴や役割について理解することは,
学習効果を高めるといわれている.
この類似の動作特徴を持つ手話単語も同時に検索することが,本論文で
提案する検索方法の重要な目的の一つである.

そこで,本論文では,適合性の評価尺度として,検索結果の上位に位置する
ある規定範囲内で検索されたか否かの二値的な判断により,質問集合全体の
総数と各質問に対する検索数の比で求まる「平均検索成功率」を評価指標とする.

\subsubsection {平均到達度数}

一方,検索結果の順序付けは類似度に応じて整列されるが,類似度の値が同じため
同順位となる場合があり,適合部分集合の要素は複数となる.
そのため,1位で検索された場合でも,同順位のものが10個あると,
正解とみなされる手話単語に到達するまで,利用者は最悪で10回の手間を必要とする.
この利用者が適合する結果に到達するまでの手間を評価尺度とする方法として,
\cite{Cooper1968}の平均探索長(expected search length)がある.

そこで,本論文では,評価尺度として,平均探索長の考えに基づく,
「平均到達度数」を定義し,評価指標として用いることにする.
平均到達度数は,平均探索長が同順位以外
も含め,複数の適合文書の検索を想定して計算する必要があるが,
本実験では,適合文書に対応する辞書記述文は,検索記述文の手話単語に
対応する唯一の手指動作記述文を対象として評価したいため,
適合辞書記述文を含む同順位の組み合わせだけを考慮すれば良いため,
以下のように計算は比較的簡単化される.

任意の検索要求に対して,
同順位 $i$ 番目で検索される部分集合の要素数を
 $P(i)$ とする.
また,当該手話単語が含まれる順位を $n$ とすると,期待値としての
平均到達度数 $EL$ は次式で求められる.

\begin{equation}
EL = \sum_{i=1}^{n-1} P(i) + \sum_{i=1}^{P(n)} \frac{i}{P(n)}
\end{equation}

ここで,すべての検索要求に対して,必ず$m$個の検索結果を出力するものとすると,
到達度数の最小値は$1$であり,第一番目に当該手話単語が検索され,かつ,
同順位のものがない場合となる.また,最大値は出力された$m$個をすべて調べる
場合で$m$となる
\footnote{検索に失敗した場合も,すべての出力結果を調べるため手間数は同等になる.
また,同順位の要素数を含め $m$個を超えた場合,超えた分は
検索されなかったものとみなす.}.

\subsubsection {有用性の評価}

有用性の観点からの評価と適合性の観点からの評価とは,一般に,
直交する概念と捉えられる.
すなわち,適合する文書ではなくても,その文書が,利用者にとって有用な
情報を含んでいる場合があり,その観点で有用性が評価される.
最近では,二値的な判断でなく多値的な判断を採り入れ,
より詳細に適合性や有用性を評価
している\cite{SatoSatoshi1993,TanakaHideki1999}.

本論文では,検索結果を有用性の観点から評価するため,各検索記述文に対する
上位10位までの検索結果の手話単語を求め,
得られた手話単語に対して,以下に示す評価値をつける.

\begin{itemize}
\item [A] 検索記述文の手話単語と一致する.
\item [B] 検索記述文の手話単語と手の形,位置,動きの中で1つだけ異なる最小対である.
\item [C] 検索記述文の手話単語と類似の動作特徴を含む.
\item [F] 類似の動作特徴を何も含まない.
\end{itemize}

また,各検索記述文に対する検索結果全体の評価は,
上記の評価値の組み合わせとして,
以下に示す総合評価値を与える.
すなわち,Aを含むか否か(適合性)も加味された評価となる.

\begin{description}
\item [{\bf AB}] AとBを含む
\item [{\bf AC}] AとCを含む
\item [{\bf A\ \ }] Aを含む
\item [{\bf BC}] BとCを含む
\item [{\bf B\ \ }] Bを含む
\item [{\bf C\ \ }] Cを含む
\item [{\bf F\ \ }] A,B,C いずれも含まない.
\end{description}



\subsection {結果と評価}

\subsubsection {平均検索成功率による評価}

表~\ref{kekka_hyouka}は,同順位で10位(上位から10番目)までの検索結果
に対する平均検索成功率を示す.
ここで,順序保存と順序無視による照合について,
文字を照合要素とした場合,
非平仮名キーワード列を照合要素とした場合
との組み合わせによる検索結果を示している.
欄中の最初の数値が検索要求総数(87件)の中で10位以内で検索に成功した数
を示し,括弧の中は平均検索成功率を示している.

\begin{table}[htb]
\label{kekka_hyouka}
\caption{上位から10番目までの平均検索成功率}
\begin{center}
\footnotesize\tabcolsep=3pt
\begin{tabular}{c|p{7zw}|p{7zw}}\hline
 照合方式  & 文字 & キーワード \\ \hline\hline
 順序保存 & 52\ (59.8\%) & 61\ (70.1\%) \\ \hline
 順序無視 & 44\ (50.6\%) & 55\ (63.2\%) \\ \hline
\end{tabular}
\end{center}
\end{table}

明らかに,順序保存による照合が,順序無視に比べて平均検索成功率で
優位にあることが分かる.
同様に, 照合要素をキーワードとした照合が,文字に比べて優位に
あることが分かる.
以上の結果から,本提案方式の順序保存・キーワード照合の組み合わせが
優位にあり,上位10位以内で
$70.1\%$の平均検索成功率を示している.


\subsubsection {平均到達度数による評価}

次に,平均到達度数をキーワード列を用いた
順序保存による照合と順序無視による照合との計算結果を
表~\ref{kekka_one}に示す.
その結果,順序保存による照合方式の
平均到達度数は 5.01となり,順序無視による照合に比べて 0.24 優位に
あることが分かる.
一方,検索に成功した場合の平均到達度数では,逆に 0.24 の差がある.
このように,上位10位以内に適合手話単語が含まれる場合,
両者とも平均すると検索結果の上位3位以内に適合手話単語が位置している
ことを示している.

\begin{table}[htbp]
\caption{非平仮名キーワード列による検索結果}
\label{kekka_one}
\begin{center}
\footnotesize\tabcolsep=3pt
\begin{tabular}{c|c|c}\hline
照合方式   & 成功時の平均到達度数 & 全検索要求に対する平均到達度数 \\
\hline\hline
順序保存 &  2.99     &     5.01   \\
\hline
順序無視 &  2.75     &     5.25   \\ \hline
\end{tabular}
\end{center}
\end{table}







\subsubsection {有用性の観点からの評価}

図\ref{sample_hyouka}に評価結果の例を示す.
この例の場合,入力された検索記述文の手話単語は【競技】であり,
第 1 位で検索されたものが対応し,評価 Aが与えられる.
また,【競う】,【試験】は「手の動き」が異なる最小対であり
\footnote{明らかに,図\ref{sample_hyouka}に示した「競う」と「試験」は同一の手話表現である.}
,
【売買】は「手の形」が異なる最小対であり,いずれも評価 Bとなる.
一方,【話】は「手の形」と「手の位置」が異なるため,最小対とは認められず,
評価 Cとなる.
その結果,検索結果全体の総合評価値は,AとBを含むため,
最良値として,{\bf AB}のグレードが与えられる.

\begin{figure}[tb]
\setbox0\vbox{\footnotesize\tabcolsep=3pt
\hbox{\|【競技 (137)】=(競技) : 親指を立てた両手を交互に前後させる |}
\hbox{\| key token( 4) = 親指 立 交互 前後                         |}
\hbox{\|--------------------------------------------------------------|}
\hbox{\| A 1 0.800 lcs( 4) m( 5) 【競技】親指 立 交互 二度 前後  |}
\hbox{\| B 2 0.450 lcs( 3) m( 5) 【競う】親指 立 交互 二三度 上下|}
\hbox{\| B 3 0.375 lcs( 3) m( 6) 【試験】親指 左右 立 二度 交互 上下|}
\hbox{\| C 4 0.321 lcs( 3) m( 7) 【話】人差指 立 口 前 二度 交互 前後|}
\hbox{\| B 5 0.281 lcs( 3) m( 8) 【売買】|}
\hbox{\|                          親指 人差指 輪 作 左右 交互 前後 動|}
}
\centerline{\fbox{\hbox to \textwidth{\hss\box0\hss}}}
\caption{検索結果の有用性の評価例}
\label{sample_hyouka}
\end{figure}


評価結果を表~\ref{Hyouka}に示す.
総合評価が{\bf B}以上は71個で,全体の81.6\%である.
また,総合評価が{\bf C}以上は78個で,全体の89.7\%を示している.
このことから,検索に成功した場合には,類似の動作特徴を含む
他の手話単語との比較が可能であり,適合手話表現と類似手話表現との弁別特徴の差や,
意味の類似性や相違性を確認できるなど,
手話単語の学習効果に貢献する有用な情報が検索できたといえる.

\begin{table}[htb]
\caption{検索結果の有用性の評価}
\label{Hyouka}
\begin{center}
\footnotesize\tabcolsep=3pt
\begin{tabular*}{\columnwidth}{@{\hspace{\tabcolsep}
\extracolsep{\fill}}c|p{2zw}|p{2zw}|p{2zw}|p{2zw}|p{2zw}|p{2zw}|p{2zw}|r}\hline
評価値  & \multicolumn{1}{c|}{\bf AB} & \multicolumn{1}{c|}{\bf AC} & \multicolumn{1}{c|}{\bf A}  & \multicolumn{1}{c|}{\bf BC} & \multicolumn{1}{c|}{\bf B} & \multicolumn{1}{c|}{\bf C} & \multicolumn{1}{c|}{\bf F} & \multicolumn{1}{c}{合計} \\ \hline\hline
 計     & \multicolumn{1}{r|}{52} & \multicolumn{1}{r|}{9} & \multicolumn{1}{r|}{0}  & \multicolumn{1}{r|}{8} & \multicolumn{1}{r|}{2} & \multicolumn{1}{r|}{7} & \multicolumn{1}{r|}{9} & \multicolumn{1}{r}{87}  \\ \hline
A区分総計   & \multicolumn{3}{c|}{61\ (71.1\%)}&\multicolumn{4}{c|}{26\ (29.9\%)}&87\\
\hline
B区分総計   & \multicolumn{5}{c|}{71\ (81.6\%)}&\multicolumn{2}{c|}{16}& 87\\ \hline
C区分総計   & \multicolumn{6}{c|}{78\ (89.7\%)}   &9& 87\\
\hline
\end{tabular*}
\end{center}
\end{table}


一方,該当手話単語の検索に失敗した場合には,
その中の38.5\%は適合手話単語の最小対を,また,65.4\%は類似の動作特徴を
含む手話単語の検索に成功していることから,
利用者が「この表現は,ちょっと違うんだけど,よく似てる」と判断した場合,
その辞書記述文を新たな検索記述文として再利用できる可能性がある.
すなわち,検索のリカバリー処理を同じ枠組で実現できる.
この場合,次章で議論するように,辞書側の手指動作記述文の正規化が
リカバリー処理による検索成功率を向上させるための課題である.

これらの評価結果から,
検索結果の上位10位以内に,検索要求に適合する手話単語を含む割合は約70\%であり,
平均すると上位から5個程度までを調べれば,当該手話単語を見つけられる.
また,最小対などの類似の動作特徴を含む手話単
語を検索結果に含んでおり,
手話単語の学習効果の向上に貢献し,上位での検索に失敗した場合のリカバリー処理にも利用できる
有用な情報を含む手話単語(辞書記述文)を検索しているなど,
本論文で提案した手話単語の検索方法の妥当性を示す結果が得られたと考える.


\section {検討}

本章では,実験により明らかになった問題点と今後の課題に
ついて議論する.特に,上位での検索に失敗した検索結果を
分析し,検索例を示しながら議論を行う.
ここで取り上げる問題の幾つかは,本論文で提案した手法に
限らず,他のアプローチによる手話単語の検索方法にも共通
の課題を含んでいると考える.

\subsection {手指動作記述文の解釈における曖昧さ}

実験により明らかになった問題点の 1つは,
同一の手指動作記述文の表す手指動作表現の解釈に,曖昧さが
ある点である.すなわち,文が複数の解釈を持つ場合があると
いう自然言語表現に特有の多義性の問題と捉えることができる\footnote{
人工言語と自然言語の違いは,この多義性の有無といわれ,自然言語処理はこの多義性の問題との取り組みが重要な課題といえる.}.
その結果,検索記述文に非常に類似した辞書記述文が検索された場合でも,
利用者の意図した検索要求の手話表現とは,まったく異なる手話表現
が上位で検索されてしまう問題である.

例えば,手話単語の日本語ラベル【太陽】の検索記述文
に対する検索結果の例を図\ref{fig:ambiguity}に示す.
なお,図中の点線の下部は検索結果を示し,
1番目の数字は順位を示し,2番目の数値は類似度を示す.
また,
``{\tt lcs}''の括弧付きの数字は順序保存による共通キーワード数,
``{\tt m}''の括弧付きの数字は辞書記述文のキーワード数をそれぞれ示している.

\begin{figure}[tb]
\setbox0\vbox{\footnotesize\tabcolsep=3pt
\hbox{\|【太陽 (137)】=(太陽) : 両手の親指と人差指で輪を作り上げる |}
\hbox{\| key token( 5) = 親指 人差指 輪 作 上                         |}
\hbox{\|--------------------------------------------------------------|}
\hbox{\| 1 0.625 lcs( 5) m( 8) 【上がる(値段などが)】   |}
\hbox{\|                        親指 人差指 輪 作 並  同時 上 上 |}
\hbox{\| 2 0.556 lcs( 5) m( 9) 【木曜日】                 |}
\hbox{\|                        親指 人差指 大 輪 作 上 上 左右 開 |}
}
\centerline{\fbox{\hbox to \textwidth{\hss\box0\hss}}}
\caption{手指動作記述文の解釈に曖昧さがある例}
\label{fig:ambiguity}
\end{figure}

ここで,検索記述文「両手の親指と人差指で輪を作り上げる」が意図する
手話表現は,『両手を使って太陽を模倣した{\bf 大きな輪を1つだけ}作り,
それを上にあげる』という手指動作を意味している.
一方,第1位で検索された手話単語の日本語ラベル【上がる(値段などが)】の
手話表現は,
『右手と左手のそれぞれで親指と人差指を使って{\bf 小さな輪}を作り
\footnote{「お金」の意味として,しばしば用いられる手話表現の1つである.},
その{\bf 2つの輪を}並べて同時に上に上げる』手指動作を意味している.

このように,両手を用いた表現に対して,検索側と辞書側の記述文の解釈に差が
生じる場合がある.
しかし,類似の文構造を持つ「両手の親指と人差指を立て(伸ばし)て〜」などでは,
手話表現に 上記のような 2つの解釈は生じない.
また,同様な多義性を持つものとして,「親指と人差指(の指先)を付け合わせる」の
例を考えてみる.
片手でこの表現を自然に行うと,いわゆる「小さな輪」になるであろう.
一方.両手で表現する場合には,大きな輪以外に四角や三角も表現できる可能性がある.
そのため,可能な解釈を絞り込む情報が付加された,
「両手の親指と人差指(の指先)を付け合わせて輪を作り〜」のような,冗長な表現と
も取れる手指動作記述文が,実験で用いた辞書記述文に存在する.

さらに,第2位で検索された【木曜日】の例では,「輪」に関する解釈は
ほぼ一致しているが,【太陽】の輪は,動作主である人間の身体に対して平行な
位置関係であるのに対し,【木曜日】のそれは,垂直な位置関係で表現する違い
がある.この位置関係の差に関する情報は,検索側と辞書側の記述文には陽に表
現されていない.
実験に用いた手指動作記述文は,市販の手話辞典に記載のもの
であり,イラストとの併用を前提として記述されているため,イラストで理解で
きる情報の一部が手指動作記述文から省略されている場合がある.



\subsection {語彙の多様性}

次に,構造的な解釈の曖昧さと同様に,語彙的な表現の差が検索結果に与える影響
について議論する.
図\ref{fig:ambiguity2}に,辞書側の【太陽】の記述文から抽出された
キーワードを示す.
ここで,「輪」が「丸」,「作(る)」が「表現」というように,
使われている表現が異なるため,文字列照合の不一致が類似度に反映されない問題である.
そこで,表\ref{dougi}に示したキーワード間の同義置換表に,
この2つの同義関係(輪,丸)と(作,表現)を追加した結果,
図\ref{fig:ambiguity3}に示すように,同順位で第2位で検索されることを確認した.
このように,今後,キーワード間の同義置換表を整備することで検索精度を
向上できる可能性がある.
しかし,同義置換表による「同一視」は,安易な追加・変更が文字列の照合処理に
副作用を生じる可能性もあり,同義置換表の拡充・利用法の検討は今後の課題とする.


\begin{figure}[tb]
\setbox0\vbox{\footnotesize\tabcolsep=3pt
\hbox{\|【太陽 (137)】=(太陽) : 両手の親指と人差指で輪を作り上げる |}
\hbox{\| key token( 5) = 親指 人差指 輪 作 上                         |}
\hbox{\|--------------------------------------------------------------|}
\hbox{\| ( 中略 ) |}
\hbox{\| 19 0.200 lcs( 3) m(9) 【太陽】 |}
\hbox{\|                        親指 人差指 丸 形 表現 上 上 輪 頭上 |}
}
\centerline{\fbox{\hbox to \textwidth{\hss\box0\hss}}}
\caption{辞書記述文と検索記述文の違い}
\label{fig:ambiguity2}
\end{figure}

\begin{figure}[tb]
\setbox0\vbox{\footnotesize\tabcolsep=3pt
\hbox{\|【太陽 (137)】=(太陽) : 両手の親指と人差指で輪を作り上げる |}
\hbox{\| key token( 5) = 親指 人差指 輪 作 上                         |}
\hbox{\|--------------------------------------------------------------|}
\hbox{\| 1 0.625 lcs( 5) m( 8) 【上がる(値段などが)】 |}
\hbox{\|                        親指 人差指 輪 作 並 同時 上 上 |}
\hbox{\| 2 0.556 lcs( 5) m( 9) 【太陽】                   |}
\hbox{\|                        親指 人差指 丸 形 表現 上 上 輪 頭上 |} 
\hbox{\| 2 0.556 lcs( 5) m( 9) 【木曜日】                   |}
\hbox{\|                        親指 人差指 大 輪 作 上 上 左右 開  |}
}
\centerline{\fbox{\hbox to \textwidth{\hss\box0\hss}}}
\caption{同義置換表の変更による検索結果の違い}
\label{fig:ambiguity3}
\end{figure}

\subsection {視点の違いによる記述表現のゆれ}\label{douitsushi}

\begin{enumerate}
\item 指先を上に向けて付け合わせた両手を左右に引き離す
\item 掌を前方に向けて付け合わせた両手を左右に引き離す
\end{enumerate}

上記の(1)と(2)に示した手指動作記述文は,同一の手話表現を表し,
(1)が検索記述文であり,(2)が辞書記述文の適合文である.
結果として,検索結果の上位では(2)は検索されなかった.
上位で検索されたものは,すべて共通のキーワード列「指先,付,合,左右」をこの
配列順序で照合されたものであった.このキーワード列から類推できるのは,
「指先を付け合わせた手の形」が自然であり,前節で議論した「輪」の表現に
相当する.
このため,「手の形」が異なる類似の手指動作特徴を含む手話表現(手話単語)は上位で検索されたが,利用者の第一義的な要求に応える検索結果を
上位で提供できないという問題である.

一般に,2つのオブジェクトがある位置関係を持つ場合,その位置関係を
文で記述するには2つの視点が考えられる.例えば,「Aの左にBがある」と
「Bの右にAがある」の関係である.手話単語の手指動作表現を記述する場合,
(1)と(2)で記述される手話表現の手の形は,基本形の1つであり,
「掌の方向」と「指先の方向」は連動して変化する場合が多い.例えば,
「掌を下に向ける」ことは「指先を前方に向ける」ことになる.

この「視点」すなわち,手指動作特徴のどこに焦点をおき,手指動作記述文を
記述するかについて,実験に使用した2つの手話辞典では,辞書記述文と
検索記述文のいずれでも統一はされていない.
このように,手指動作記述文の正規化の問題は,
前節の議論と同様に,本提案手法の検索精度を向上させる上で重要な
課題の1つといえる.

なお,今回の実験では,市販の手話辞典に記載の手指動作記述文を辞書記述文と
検索記述文の双方に用いているため,顕在化していないが,実際に利用者の
検索記述文を入力とする場合に考慮すべき「視点の問題」として,左右方向
の逆転現象がある.手話辞典の多くは,手話表現の動作主体から見た左右方向
を記述している.一方,利用者は手話表現の動作主体と対面する形で手話表現
を見ているため,利用者から見た左右の方向を記述してしまう傾向がある.
しかし,この利用者が間違った記述をしたか否かの判断は,
検索システム側では対処不能と考える.


\subsection {動作表現の認識に関する曖昧さ}

手話単語の検索問題では,この語彙に関する問題以上に,
利用者側の手話表現の捉え方と辞書側の手話表現の捉え方
が問題となる.前述した事例以外に,(a)「両手を二度ほど下におろす」表現と
(b)「両手を上下に動かす」表現を「同一視」するか否かである.
厳密には,(a)はある始点位置を動作の上限位置と定め,その範囲で下方に二度,
両手を下げることを意味し,(b)はある始点位置を基準に上方/下方両方の範囲で
両手を上げ下げすることを意味すると捉えることができる.辞書側で(a)を
「手を下にだけ動かす」手話単語群に分類し,(b)を「手を上下双方に動かす」
手話単語群に分類していた場合,利用者が(a)を「上下運動」と捉えたなら,
(b)に分類された手話単語群しか検索されないことになる.
このように,この問題は,動作表現を言葉で表現した手指動作記述文を
用いる本論文で提案するアプローチだけでなく,
検索項目や検索条件を詳細に設定する(すなわち,これらの基準で分類された
検索辞書を用いる)従来のアプローチにも共通の問題と捉えることができる.
次節では,これらの議論を基に,
手指動作記述文の正規化の方法と利用について検討する.

\subsection {手指動作記述文の正規化の方法と利用}

2章で述べたように,手指動作記述文で用いられる語彙は,
通常の日本語文と比べて制約が強く,文形式にパターンが
存在すると捉えることができる.
一般に,このような制限された文を受理する有限オートマ
トンは比較的容易に構成できることが知られている.
すなわち,正規文法で手指動作記述文の構文規則を記述できる可能性がある.
この文法を構成することは,検索記述文と
辞書記述文の両方を正規化する言語処理的かつ汎用性
のあるアプローチといえる.

一方,\cite{AdachiHisahiro1998}は限られた動作に限定しているが,
人物を背景画像とするパレット上での
マウスの位置とその移動軌跡に基づき,正規化された手指動作記述文を生成し,
検索記述文とする検索アプローチを提案している.
これは,前節で議論した視点や曖昧さの問題に
対する 1つの解と考えることができる.
すなわち,検索記述文を辞書記述文に
近づける正規化の試みの1つと捉えることができる.
また,\cite{AdachiHisahiro1999}は,手指動作記述文が手話表現
を生成する手続き文(プログラム)と捉えた手話アニメーションの生成
方法について報告している.
これら一連の研究では,手指動作記述文を電子化辞書の情報の中心に据え,
言語と画像の接点として,
手話単語の検索と生成を同じ枠組で構成しようとする提案である.
今後の展開を期待する試みの1つである.


実験では,「両手」を文の特徴付けに寄与しない不要語とみなし,
手の形を記述する部分が「両手」に対して前置される場合と後置される
場合の照合洩れを抑止する戦略を採った.
一方,手指動作記述文の正規化,特に,単語の配列(文形式)に関して,
このキーワード「両手」は
有効利用できる可能性がある.すなわち,「手の形」の手指動作特徴は「両手」
の前に置くように正規化する方法である.
これにより,キーワード照合の際に,
「両手」の前部と後部を分けて照合すれば,前部は「手の形」に関する
類似性であることが明確になり,「手の形」を無視(手の位置と動きに注目)した
検索や「手の形」に注目した検索など,検索手段の幅が広がる可能性があると考える.
本手法では,キーワードに対して重み付けは考慮していない.
手指動作特徴の各パラメータに関連するキーワードに重み付けをし,
類似度の計算に反映させる枠組の検討は,今後の課題である.

ここで,これまでの議論をまとめると,
本論文で提案した検索方法は,
手話単語の特徴構造を言語表現に写像した手指動作記述文を用いた検索方法であり,
手話単語の検索問題を文献検索問題と捉えた点に特徴がある.
この手指動作記述文は一般の自然言語文に比べて構文的にも語彙的にも
制限のある文集合と捉えることができる.
一方,このような制約のある文集合でも,自然言語表現の持つ特徴である
「多義性」や「多様性」に起因する「解釈の曖昧さ」という自然言語処理全体に
共通の問題点を解消する必要がある.
この「曖昧さ」を除去するには,手指動作記述文から得られる情報と得られない
情報を区別し,得られる情報でも現在は無視している部分(辞書記述文と検索記述文で
共通でないキーワードの数やキーワードの持つ意味)を類似度の計算に組み入
れる検討と,
与えられた文からは得られない情報(利用者が手話表現を認知する場合の視点や
省略された手指動作特徴など)
については,\cite{AdachiHisahiro1998}の手法
のように,言語処理の枠組内ですべて解決するのではなく,言語表現以外の手段により,
得られた情報を基に正規化された言語表現を再構成し,検索記述文とするアプローチな
ども考慮しながら検討する必要がある.

このように,本論文で提案した手法を実用レベルの検索精度に高めるには,
ここで議論した手指動作記述文の正規化の問題を解決する必要があるなど,
残された課題は少なくない.
また,検索に失敗した場合の効果的なリカバリー処理についても,今後,
検討する必要がある.


\section {おわりに}

従来の手話単語の検索方法が,利用者に検索項目や検索条件を詳細に設定させ,
検索単語の候補を絞り込むアプローチを採用していたため,
初心者には適切な設定が困難であり,かつ煩わしく,その結果として,
選択ミスを生じやすく,検索精度を低下させる問題点があった.
本論文では,従来の検索項目や検索条件を設定する代わりに,
手指動作記述文を用いた手話単語の検索方法を提案した.
本手法の特徴は,与えられた入力文に類似している辞書中の
手指動作記述文を検索し,対応する手話単語の日本語ラベルを提示する
点にある.すなわち,手話単語の検索問題を文献検索問題と捉えたアプ
ローチといえる.
両手同形の手話単語を対象に検索実験を行った結果,
本手法の妥当性を示す結果が得られた.
また,実験により明らかとなった,手話表現を日本語文で記述した場合に
生じる手指動作記述文の曖昧さに関する問題について検討を行った.
今後の課題として,入力された手指動作記述文の曖昧さの検出,
検索に失敗した場合のリカバリー処理の検討,および文の正規化による
検索精度の向上が挙げられる.


\acknowledgment

本研究を進めるにあたり,有益なご示唆,ご討論を頂いた宇都宮大学鎌田一雄教授,
熊谷毅助教授に心より感謝する.
また,データ整理,実験等に協力頂いた研究室の学生諸氏に感謝する.
なお,本研究の一部は文部省科研費,厚生省科研費,
実吉奨学会,電気通信普及財団,放送文化基金,
トヨタ自動車,栢森情報科学振興財団,大川情報通信基金の援助によった.




\bibliographystyle{jnlpbbl}
\bibliography{C}

\begin{biography}
\biotitle{略歴}
\bioauthor{安達 久博}{
1981年宇都宮大学工学部情報工学科卒業.
1983年同大学院工学研究科修士課程修了.
同年,東京芝浦電気株式会社(現.(株)東芝)入社.
同社総合研究所情報システム研究所に所属.
この間,(株)日本電子化辞書研究所(EDR)に出向.
1992年より宇都宮大学工学部助手.博士(工学).
現在,聴覚障害者の情報獲得を支援する手話通訳システムに関する研究に従事.
言語処理学会,情報処理学会,電子情報通信学会,人工知能学会,
日本認知科学会,計量国語学会,各会員.
}

\bioreceived{受付}
\bioaccepted{採録}

\end{biography}

\end{document}
