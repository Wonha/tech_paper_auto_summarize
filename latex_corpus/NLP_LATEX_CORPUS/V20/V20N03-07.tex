    \documentclass[japanese]{jnlp_1.4}
\usepackage{jnlpbbl_1.3}
\usepackage[dvips]{graphicx}
\usepackage{amsmath}
\usepackage{hangcaption_jnlp}
\usepackage{udline}
\setulminsep{1.2ex}{0.2ex}
\let\underline
\usepackage{multirow}
\usepackage{array}

\Volume{20}
\Number{3}
\Month{June}
\Year{2013}

\received{2012}{12}{14}
\revised{2013}{2}{18}
\accepted{2013}{3}{28}

\setcounter{page}{485}

\jtitle{マイクロブログにおける流言の影響の分析}
\jauthor{宮部 真衣\affiref{TOKYO} \and 田中 弥生\affiref{KANAGAWA} \and 西畑  祥\affiref{KONAN} \and 灘本 明代\affiref{KONAN} \and 荒牧 英治\affiref{KYOTO}\affiref{JST}}
\jabstract{
マイクロブログの普及により,ユーザは様々な情報を瞬時に取得することができるようになった.
一方,マイクロブログでは流言も拡散されやすい.
流言は適切な情報共有を阻害し,場合によっては深刻な問題を引き起こす恐れがある.
これまで,マイクロブログ上の流言拡散に関する分析は多かったが,ある流言がどのような影響を引き起こすかについての考察はない.
本論文では,東日本大震災直後のTwitterを材料とし,どのような流言が深刻な影響を与えるかを,
有害性と有用性という観点からの主観評価および修辞ユニット分析により分析した.
その結果,震災時の流言テキストの多くは行動を促す内容や,状況の報告,予測であること,
また,情報受信者の行動に影響を与えうる表現を含む情報は,震災時に高い有用性と有害性を持つ可能性があることを明らかにした.
}
\jkeywords{マイクロブログ,流言,災害}

\etitle{Effects of Rumors on Microblogs}
\eauthor{Mai Miyabe\affiref{TOKYO} \and Yayoi Tanaka\affiref{KANAGAWA} \and Sho Nishihata\affiref{KONAN} \and \\
	Akiyo Nadamoto\affiref{KONAN} \and Eiji Aramaki\affiref{KYOTO}\affiref{JST}} 
\eabstract{
Microblogs have enabled us to exchange information in real time, which has led to the spread of not only beneficial but also potentially harmful information, such as rumors.
Rumors may block the process of adequate information sharing, which may, in turn, cause serious problems.
Several studies have already analyzed the impact of rumors on microblogging media; however, the way in which these rumors can cause potential problems largely remains unclear.
This paper analyzes people’s perceptions of rumors on Twitter during disasters using subjective evaluation and rhetorical unit analysis.
The results showed that many subjects perceived rumors as containing information that instigates people to take action, reports on a current situation, or predicts future events.
Moreover, information that instigates people to take action has been perceived as beneficial in some contexts, while it is also seen as harmful in other cases.
}
\ekeywords{Microblogging system, Rumor, Disaster}

\headauthor{宮部,田中,西畑,灘本,荒牧}
\headtitle{マイクロブログにおける流言の影響の分析}

\affilabel{TOKYO}{東京大学知の構造化センター}{Center for Knowledge Structuring, The University of Tokyo}
\affilabel{KANAGAWA}{神奈川大学外国語学部}{Faculty of Foreign Languages, Kanagawa University}
\affilabel{KONAN}{甲南大学知能情報学部}{Faculty of Intelligence and Informatics, Konan University}
\affilabel{KYOTO}{京都大学デザイン学ユニット}{Design School, Kyoto University}
\affilabel{JST}{科学技術振興機構さきがけ}{JST PRESTO}



\begin{document}
\maketitle


\section{はじめに}\label{sec:intro}

近年,Twitter\footnote{http://twitter.com/}などのマイクロブログが急速に普及している.
主に自身の状況や雑記などを短い文章で投稿するマイクロブログは,ユーザの情報発信への敷居が低く,
現在,マイクロブログを用いた情報発信が活発に行われている.
2011年3月11日に発生した東日本大震災においては,緊急速報や救援物資要請など,
リアルタイムに様々な情報を伝える重要な情報インフラの1つとして活用された\cite{Book_Hakusho,Article_Nishitani,Book_Tachiiri}.
マイクロブログは,重要な情報インフラとなっている一方で,情報漏洩や流言の拡散などの問題も抱えている.
実際に,東日本大震災においても,様々な流言が拡散された\cite{Book_Ogiue}.
{\bf 流言}については,これまでに多くの研究が多方面からなされている.流
言と関連した概念として{\bf 噂},{\bf 風評},{\bf デマ}といった概念があ
る.これらの定義の違いについては諸説あり,文献毎にゆれているのが実情で
ある.
本研究では,{\bf 十分な根拠がなく,その真偽が人々に疑われている情報を流言と定義し,その発生過程(悪意をもった捏造か自然発生か)は問わない}ものとする.
よって,最終的に正しい情報であっても,発言した当時に,十分な根拠がない場合は,流言とみなす.

本論文では,マイクロブログの問題の1つである,流言に着目する.
流言は適切な情報共有を阻害する.
特に災害時には,流言が救命のための機会を損失させたり,誤った行動を取らせたりするなど,深刻な問題を引き起こす場合もある.
そのため,マイクロブログ上での流言の拡散への対策を検討していく必要があると考えられる.

マイクロブログの代表的なツールとして,Twitterがある.
Twitterは,投稿する文章(以下,ツイート)が140字以内に制限されていることにより,
一般的なブログと比較して情報発信の敷居が低く\cite{Article_Tarumi},
またリツイート (RT) という情報拡散機能により,流言が拡散されやすくなっている.
実際に,東日本大震災においては,Twitterでは様々な流言が拡散されていたが,同じソーシャルメディアであっても,
参加者全員が同じ情報と意識を持ちやすい構造を採用しているmixi\footnote{http://mixi.jp/}やFacebook\footnote{http://www.facebook.com/}では深刻なデマの蔓延が確認されていないという指摘もある\cite{Book_Kobayashi}.


マイクロブログ上での流言の拡散への対策を検討するためには,まずマイクロブログ上の流言の特徴を明らかにする必要がある.
そこで本論文では,マイクロブログとして,東日本大震災時にも多くの流言が
拡散されていたTwitterを材料に,そこから481件の流言テキストを抽出した.

さらに,どのような流言が深刻な影響を与えるか,有害性と有用性という観点から被験者による評価を行い,
何がその要因となっているか,修辞ユニット分析の観点から考察を行った.
その結果,震災時の流言テキストの多くは行動を促す内容や,状況の報告,予測であること,
また,情報受信者の行動に影響を与えうる表現を含む情報は,震災時に高い有用性と有害性という全く別の側面を持つ可能性があることが明らかとなった.

以下,2章において関連研究について述べる.
3章では分析の概要について述べる.
4章で分析結果を示し,マイクロブログ上での流言について考察する.
5章で将来の展望を述べ,最後に6章で本論文の結論についてまとめる.


\section{関連研究}\label{sec:reference}

本論文では,災害時のマイクロブログ上での流言について分析を行う.
そこで本章では,まず,流言に関するこれまでの定義について述べた後,
災害や流言について扱ったソーシャルメディアに関する研究について述べる.


\subsection{流言の定義と流言の伝達}

本節では,実社会における流言の先行研究について述べる.

流言の分類としては,ナップによる第2次世界大戦時の流言の分類があ
る\cite{ナップ1944}.ナップは,流言を「恐怖流言(不安や恐れの投影)」
「願望流言(願望の投影)」「分裂流言(憎しみや反感の投影)」の3つに分類
している.また,これらの流言がどのように流通するかは,例えば不景気,
災害など,社会状況に依存すると述べている.

また,社会状況だけでなく,流言の伝達に影響する要素として,流言の内
容,特に,{\bf 曖昧さ},{\bf 重要さ},{\bf 不安}という3つの要因が知られてい
る\cite{Book_Kawakami}.オルポートとポストマンは,流言の流布量につい
て,$R \sim i \times a$のように定式化し,「流言の流布量 (R) は,重要さ
 (i) と曖昧さ (a) の積に比例する」と述べている\cite{Book_dema}.

このように,流言に関しては古くから研究が行われてきたが,主に口伝えでの
流言の伝達を対象としてきた.
本論文では,口伝えより,より迅速に,また,広範囲に広まりうるネットワーク上での
流言を扱った点が新しい.


\subsection{災害,流言とソーシャルメディア}

本節では,災害を扱ったTwitterをはじめとするソーシャルメディアの
先行研究について概観する.

  災害時のソーシャルメディアの利用方法について分析した研究として
  は,まずLonguevilleらやQuら,Backら,Cohnら,Viewegらの研究があ
  る\cite{Inproc_Longueville,Inproc_Qu2009,Inproc_Qu,Article_Back,Article_Cohn,Inproc_Vieweg}.
  Longuevilleらは,2009年にフランスで発生した森林火災に関し
  て,Twitterに発信されたツイートの分析を行ってい
  る\cite{Inproc_Longueville}.この研究においては,ツイートの発信者の分
  類や,ツイートで引用されたURLの参照内容に関する分析などを行ってい
  る.Quらは四川大地震および青海地震において中国のオンラインフォーラ
  ム (BBS) がどのように利用されたのかを分析してい
  る\cite{Inproc_Qu2009,Inproc_Qu}.
  また,BackらやCohnらは,9.11時のブログの書き込み内容を分析し,人々の感情の変
  化を分析している\cite{Article_Back,Article_Cohn}.
Viewegら\cite{Inproc_Vieweg}は,
  2009年のオクラホマの火事 (Oklahoma Grassfires) やレッドリバーでの洪水
   (Red River Floods) におけるTwitterの利用方法を調査している.これらの
  研究では発信された内容を分類し,情報の発信の方法(情報発信か返信か)
  や,その位置関係について議論しているが,情報が流言かどうかといった観
  点からの分析は行われていない.


  流言については,災害時に限らず,多くのソーシャルメディア上の研究がある.
Qazvinianらは,マイクロブログ
   (Twitter) における特定の流言に関する情報を網羅的に取得することを目的
  とし,流言に関連するツイートを識別する手法を提案してい
  る\cite{Inproc_Qazvinian}.Mendozaらは,2010年のチリ地震におけ
  るTwitterユーザの行動について分析を行っている\cite{Inproc_Mendoza}.
  この研究では,正しい情報と流言に関するツイートを,「支持」「否定」
  「疑問」「不明」に分類し,支持ツイート,否定ツイートの数について,正
  しい情報と流言との違いを分析している.分析結果として,正しい情報を否
  定するツイートは少ないが (0.3\%),流言を否定するツイートは約50\%に上
  ることを示している.



\section{分析の概要}

マイクロブログ上での流言拡散への対策を検討するためには,マイクロブログ上の流言の特徴を明らかにする必要がある.
まず,なぜ人間は流言を拡散させるのであろうか.
一般に,人々がある情報を他者に伝える場合,その情報が正しいと思って伝えていることが多く,
本人がでたらめだと思う話を,悪意をもって他者に伝えることは少ない\cite{Book_Kawakami}.
また,流言とは,曖昧な状況に巻き込まれた人々が,自分たちの知識や情報を寄せ集めることにより,
その状況について意味のある解釈を行おうとするコミュニケーションであるという考察もある\cite{Article_Sato}.
つまり,災害時の流言は,何らかの役に立ち得る(有用性のある)情報を含み,
それを共有するために善意で拡散されている可能性がある.


次に,流言が拡散された場合,どのような問題が起きるかという点を考えると,
\ref{sec:intro}章で述べたように,情報受信者を誤った行動に導き,様々な損失を与えるということが考えられる.
つまり,特に対策を講じるべき流言とは,情報受信者にとって有害性のある情報である.

また,上述した何らかの役に立ち得る(有用性のある)情報は,人々の行動などに影響を与える可能性もある.
すなわち,流言の内容が有用と判断される場合には,情報受信者の何らかの行動を引き起こし得ると考えられ,
有用性の高さは有害性と関連する可能性がある.



そこで本研究では,上述した「有害性」および「有用性」という観点に着目し,次の2つの分析を行う.
\begin{enumerate}
\item {\bf 流言の有害性/有用性}:どのような流言が有害または有用とみなされるのかの主観評価を行う.
\item {\bf 流言の修辞ユニット分析}:どのような特徴が先の有害性/有用性
  に影響を与えているのか,後述する修辞ユニット分析という手法を用いて解析する.
\end{enumerate}


\subsection{材料:対象データセット}
\label{sec:dataset}

本研究では,分析対象のデータとして,東日本大震災ビッグデータワークショッ
プにおいてTwitter Japan株式会社により提供された,3月11日から1週間分のツ
イートデータを用いた.

\ref{sec:intro}章で述べたように,本論文では, 十分な根拠がなく,その真偽が人々に疑われている情報を流言と定義する.
そこで,ある情報の真偽について言及しているツイートが投稿されている場合,真偽を疑問視された内容は流言と見なし,
それらを分析対象の流言として用いることとする.

\begin{table}[b]
\caption{流言抽出のパターン}
\label{正規表現}
\input{24table01.txt}
\end{table}

流言は以下の手順で抽出した.
\begin{enumerate}
\item データ全体から,情報の真偽について言及しているツイートをキーワー
  ド「デマ」をもとに抽出する\footnote{今回は,真偽を言及する際に「デマ」「流言」など
特定のキーワードが用いられる可能性があると考え,表\ref{正規表現}のPTN2 に相当する文字列を含むツイー
    トを抽出した.}.
\item 手順 (1) で抽出したツイートから,``「〜」というデマ''というパターン
を用いて流言内容(「〜」部)を抽出する.
用いたパターンは表\ref{正規表現}に示す.
\item 抽出された流言内容を人手で確認し,内容を理解可能なもののみを抽出する.
\end{enumerate}

上記の手順により,486件\footnote{なお,抽出した486件の流言テキストのうち,5件は分析対象外としたため,実際の分析対象となった流言テキストは481件である.詳細は\ref{sec:result}章で述べる.}の流言テキストを抽出した.
抽出されたテキストの一部を表~\ref{table:example}に示す.
なお,本手順では,同じ流言の異なる表現のバリエーションも抽出されうる.
例を表~\ref{table:difEx}に示す.
本研究では,同じ流言を意図していても,伝え方によって印象が異なる可能性があると考え,
1つの流言に対する分析対象を1つのテキストとするのではなく,
複数のテキストを扱うこととする.

\begin{table}[t]
\caption{抽出された流言の例}
\label{table:example}
\input{24table02.txt}
\end{table}
\begin{table}[t]
\caption{同じ流言の表現のバリエーションの例}
\label{table:difEx}
\input{24table03.txt}
\end{table}


\subsection{分析1:流言内容の影響度に関する主観評価}

前述したように,災害時の流言拡散において,実際的に問題となるのは,
その流言が実際に流言(虚偽の情報)であった場合,どれくらい有害であるか,
また,逆に,それが流言でなかった場合,どれくらい有用であるのかという2つの問題である.

そこで,以下の2項目について主観評価を実施した.
\begin{description}
\item[有害性:] この情報が間違っている場合,この情報は人にとって有害である.
\item[有用性:] この情報が正しい場合,この情報は人にとって有用である.
\end{description}
なお,本評価では,評価者自身にとって有害・有用でない情報であっても,
ある人にとって有害・有用であると考えられる場合は,有害・有用と判断してもらうこととした.

各項目の評価は,5段階評価(1:強く同意しない,2:同意しない,3:どちらともいえない,4:同意する,5:強く同意する)を用いることとし,
共著者を含む7名の評価者により評価を行った.また,評価者が上記のいずれの
評価値もつけることができないと判断した場合,評価不能 ($-1$) とすることと
した.


\subsection{分析2:流言内容の分類}

分析1では,流言の有害性と有用性という 2 つの尺度から,流言について主観評価を行った.
次に問題となるのは,流言のどのような要素が有害性や有用性に影響を与えているかである.
そこで,2つ目の分析として,流言内容をいくつかの特徴から分析した.
この際に,先行研究で観られた分類(行動を促進するかどうか,ネガティブな内容であるかどうか)に加え,
知識伝達の分析に用いられる修辞ユニット分析を用いた.


\subsubsection{従来の分類}
流言内容を分類した先行研究\cite{Article_Umejima}では,
「ネガティブである」「不安を煽る」「行動を促進する」といった観点により流言の分類を行っている.

そこで,先行研究における分類に基づき,以下の5項目について主観評価を実施した.
\begin{description}
\item[ネガティブさ:] この情報はネガティブな内容である.
\item[行動促進:] この情報は行動を促している.
\item[不安扇動:] この情報は不安を煽る.
\item[尤もらしさ:] この情報は尤もらしい.
\item[伝聞情報:] この情報には伝聞情報が含まれる.
\end{description}

各項目の評価は,5段階評価(1:強く同意しない,2:同意しない,3:どちらともいえない,4:同意する,5:強く同意する)を用いることとし,
共著者を含む7名の評価者により評価を行った.


\subsubsection{修辞ユニット分析}
\label{sec:rua}


修辞ユニット分析(Rhetorical Unit Analysis 以下,RUA)\cite{Cloran99}は,
談話分析手法の1つであり,
分析の過程で,伝達される内容の{\bf 修辞機能}の特定を行い,文脈化の程度を知ることができる.
ここでいう文脈とは,一般的な話であるほど脱文脈化されてお
り,個人的な話であるほど文脈化されているとみなす尺度である.

例えば,「ホウ素は特殊な結晶構造をとるため放射線を吸収します。」という
のは一般性を持つため脱文脈化されているとみなす.逆に,「ホウ素サプリを
採りましょう」というのは聞き手に行動を促しており,文脈化されているとみ
なす.

先行研究では,
母子会話や生徒—教師の解析\cite{Cloran94,Cloran99,Cloran2010} ,作文指
導\cite{Article_Sano},Q\&Aサイトの解
析
\cite{Inproc_TanakaandSano,Inproc_TanakaandSano2,Inproc_TanakaandSano3,Inproc_Tanaka}
などに用いられてきた.
日本語への適用については文献\cite{Web_Sano,Article_Sano2011}が詳しい.
本稿では,その概要のみを述べるものとする.

\begin{table}[b]
\caption{修辞機能の特定と脱文脈化指数}
\label{table:rua}
\input{24table04.txt}
\vspace{0.5zw}\small
「n/a」は該当なし/太字の部分が修辞機能の種類/[ ]内は脱文脈化指数\par
\end{table}

RUAは通常次の手続きを踏む.
\begin{enumerate}
\item 発話のメッセージ(基本的には節)の{\bf 発話機能}を認定し,{\bf 中核要素}と{\bf 現象定位}を確認する.
{\bf 発話機能}は,「与える」と「要求する」の「交換における役割」と,
「品物/行為」と「情報」という「交換されるもの」の二項の組み合わせで構成され,
「品物/行為」の交換を「提言」,「情報」の交換を「命題」とする\cite{Book_Halliday}.
{\bf 中核要素}は,基本的には発話内容の主語で判断し,
「状況外」など4つのカテゴリからなる.
{\bf 現象定位}は,発話機能が「命題」と認定されたメッセージについて,
その発話内容の出来事が起こった,あるいは起こる時を,基本的にはテンスや時間を表す副詞句などから判断し,
「過去」など6つのカテゴリからなる.
\item この,発話機能と中核要素と現象定位の組み合わせから,14のレベルに細分
化された修辞機能が特定され,文脈化の程度(脱文脈化指数と呼ばれる)が測られる(表\ref{table:rua}).
脱文脈化指数の数値が大きいものほど脱文脈化の程度が高く一般的・汎用的で,
小さいものほど脱文脈化の程度が低く個人的・特定的である.
\end{enumerate}

各修辞機能と脱文脈化指数へと分類されるテキストの例を以下に示す.[ ]内は脱文脈化指数を示す.
\begin{description}
\item[[01]行動] みんなで節電しましょう\\
(中核要素対象:みんなで,現象定位対象:節電しましょう)
\item[[02]実況] 血が流れている。\\
(中核要素対象:血が,現象定位対象:流れている)
\item[[03]状況内回想] ラックが倒壊した。\\
(中核要素対象:ラックが,現象定位対象:倒壊した)
\item[[04]計画] お水買っといた方がいいんじゃない?\\
(中核要素対象:$\phi$=あなたは(あるいはわたしは),現象定位対象:お水買っといた方がいいんじゃない?)
\item[[05]状況内予想] もうすぐ肉不足で焼肉食べられなくなる\\
(中核要素対象:$\phi$=あなたは(あるいはわたしは),現象定位対象:もうすぐ肉不足で焼肉食べられなくなる)
\item[[06]状況内推測] 放射能が来ても自転車のチューブがあれば助かるらしいぞ\\
(中核要素対象:$\phi$=あなたは,現象定位対象:自転車のチューブがあれば助かるらしいぞ)
\item[[07]自己記述] 国際線で1回飛ぶと宇宙線を1ミリシーベルト近く被曝します\\
(中核要素対象:$\phi$=あなたは,現象定位対象:宇宙線を1ミリシーベルト近く被曝します)
\item[[08]観測] このそば屋の店主はいつも愛想がない\\
(中核要素対象:このそば屋の店主,現象定位対象:いつも愛想がない)
\item[[09]報告] 301号にけが人がいます\\
(中核要素対象:けが人が,現象定位対象:います)
\item[[10]状況外回想] 阪神大震災の際ははじめの地震から三時間後に一番強い地震がきた\\
(中核要素対象:一番強い地震が,現象定位対象:きた)
\item[[11]予測] 関東の方は深夜に地震が起きる可能性があるそうです。\\
(中核要素対象:地震が,現象定位対象:起きる可能性があるそうです)
\item[[12]推量] 首都圏で買いだめすると被災地に物資が届かなくなる\\
(中核要素対象:物資が,現象定位対象:届かなくなる)
\item[[13]説明] 日本ユニセフ協会は募金をピンハネする\\
(中核要素対象:日本ユニセフ協会は,現象定位対象:募金をピンハネする)
\item[[14]一般化] ホウ素は特殊な結晶構造をとるため放射線を吸収します。\\
(中核要素対象:ホウ素は,現象定位対象:特殊な結晶構造をとるため放射線を吸収します。)
\end{description}

なお,修辞ユニット分析については,修辞ユニット分析に精通した1名の作業者が分類作業を行った.



\section{分析結果と考察}
\label{sec:result}

本論文では,分析1における評価結果において,評価者7名の内4名以上が判定不能と判断したもの,
および分析2における修辞機能と脱文脈化指数の認定ができなかったものは分析対象から除外することとした.
確認の結果,分析1において評価者4名以上が判定不能と判断したものは存在しなかったため,
分析2における修辞機能と脱文脈化指数の認定ができなかった5件のみを除外した,481件の流言テキストを分析対象とした.
また,分析1の評価結果については,7名の評価者による全ての評価結果(481件×7名分,3,367件)および
7名による評価結果の中央値を用いて考察する\footnote{なお,評価者が「判定不能」と評価した場合,中央値の算出や頻度の計算においては,その評価値は除外する.}.


\subsection{流言内容の影響度に関する主観評価結果}
\label{sec:subjective}

\begin{table}[b]
\caption{有害性,有用性に関する分類結果}
\label{table:evalRes}
\input{24table05.txt}
\par\vspace{0.5zw}\small
・各評価値は,1:強く同意しない,2:同意しない,3:どちらともいえない,4:同意する,5:強く同意する,を意味する.\\
・「判定不能」と評価した評価者がいた場合,その値は除外して中央値を取っている.
表中の1.5,2.5,3.5,4.5欄は,除外後の評価結果が偶数個の場合の中央値(中央に近い2つの値の算術平均)である.\par
\end{table}

本節では,分析1(流言内容の影響度に関する主観評価)の結果について述べる.
流言テキストの有害性,有用性に関する主観評価結果を表~\ref{table:evalRes}に示す.
表~\ref{table:evalRes}より,481件に対する7名の全評価値\footnote{表中の該当数の合計が全評価結果数(3,367件)に満たないのは,「判定不能」と評価されたものは除外しているためである.}についてみると,
有害性,有用性のどちらについても,「同意する」(評価値4または5)が多い傾向が見られる.
また,481件の各流言テキストに対する評価値の代表値として,中央値をとった場合の分類結果を見ると,
全評価値と同様に,有害性,有用性のどちらも「同意する」(評価値4または5)に分類された流言が多く,
震災時に発信された流言テキストは,有害性や有用性が高い傾向が見られる.

\begin{table}[b]
\caption{有害性評価と有用性評価の分布(中央値を用いた場合)}
\label{table:harmful_median}
\input{24table06.txt}
\end{table}
\begin{table}[b]
\caption{有害性評価と有用性評価の分布(7名の全評価結果)}
\label{table:harmful_all}
\input{24table07.txt}
\end{table}

また,481件に対する7名の有害性,有用性の評価結果ペア(3,367ペア)をもとに順位相関係数を調査した結果,
順位相関係数は0.601であり,正の相関がみられた.
また,流言テキスト1件毎に中央値をとった場合の,
481ペアの有害性,有用性評価結果の順位相関係数は0.628となり,同様に正の相関がみられた.
有害性評価と有用性評価の分布を表~\ref{table:harmful_median},\ref{table:harmful_all}に示す.
表~\ref{table:harmful_median},\ref{table:harmful_all}より,
一部,有害性と有用性の分類結果に相関がみられないものも見られる.
例えば,「ほくでんが東京電力に電力提供する準備を始めた」という流言テキストは,有害性の評価結果(中央値)は2であったが,有用性の評価結果(中央値)は4であった.
また,「韓国で日本の大地震を記念したTシャツが売られている」という流言テキストは,有害性の評価結果(中央値)は4であったが,有用性の評価結果(中央値)は2であった.
これらの一部例外となる流言テキストはあるものの,大部分の流言テキストについては,有害性と有用性の分類結果は類似している.
つまり,{\bf 有用性と有害性は表裏一体の関係にあることが多く,情報が正しい場合に有用性の高い内容は,その情報が間違っていた場合に有害となりうると言える.}


\subsection{流言内容の分類結果}
\label{sec:ruaRes}

\begin{table}[b]
\caption{主観評価による分類結果}
\label{table:evalRes2}
\input{24table08.txt}
\par\vspace{0.5zw}
\small
・各評価値は,1:強く同意しない,2:同意しない,3:どちらともいえない,4:同意する,5:強く同意する,を意味する.\\
・「判定不能」と評価した評価者がいた場合,その値は除外して中央値を取っている.
表中の1.5,2.5,3.5,4.5欄は,除外後の評価結果が偶数個の場合の中央値(中央に近い2つの値の算術平均)である.\par
\end{table}
\begin{table}[b]
\caption{主観評価結果の相関係数(中央値を用いた場合)}
\label{table:correl}
\input{24table09.txt}
\end{table}
本節では,分析2(流言内容の分類)の結果について述べる.
まず,先行研究に基づく流言内容の主観評価結果を表~\ref{table:evalRes2}に,
各項目および有害性,有用性の評価結果の順位相関係数を表~\ref{table:correl},\ref{table:correl2}にそれぞれ示す.

表~\ref{table:evalRes2}より,震災時に流れた流言内容は,
ネガティブで,不安を煽るものであることがわかる.
これは,先行研究における結論\cite{Book_dema}と一致する.

なお,表~\ref{table:correl},\ref{table:correl2}に示した各項目と有害性,有用性の評価結果の相関を見ると,
中央値を用いた場合は,行動促進と有害性,有用性との間や,
不安扇動と有用性との間に相関が見られる.
全評価値を用いた場合の上記の関連は,中央値を用いた場合よりも相関は弱くなるものの,
同様の傾向が見られる.

一方,尤もらしさや伝聞情報に関しては,上述した指標と比較して相関が弱く,
これらの影響で有害性や有用性が決定されているわけでないと言える.



次に,修辞ユニット分析による分類結果を表~\ref{table:ruaRes}に示す.

\begin{table}[b]
\caption{主観評価結果の相関係数(全評価値を用いた場合)}
\label{table:correl2}
\input{24table10.txt}
\end{table}
\begin{table}[b]
\caption{修辞機能と脱文脈化指数による分類結果}
\label{table:ruaRes}
\input{24table11.txt}
\vspace{0.5zw}\small
* 修辞ユニット分析は節ごとに分類を行うため,1つのツイートに複数の修辞機能が認定され,脱文脈化指数が付与される場合がある.
そこで,表~\ref{table:ruaRes}には1つのツイートに付与された脱文脈化指数のうち,最大値および最小値を代表値とした場合の該当数を提示している.\par
\end{table}

まず,脱文脈化指数の観点から考察する.
\ref{sec:rua}項で述べたように,脱文脈化指数は,
数値が大きいものほど一般的・汎用的で,小さいものほど個人的・特定的であるとされる.
表~\ref{table:ruaRes}を見ると,各脱文脈化指数に分類される流言テキストの数にはばらつきがみられ,
発信された流言について,各脱文脈化指数の大きさとの関連は見られなかった.
つまり,内容が一般的か,個人的かに関わらず,流言は発信されると考えられる.

次に,修辞機能の観点から考察する.
表~\ref{table:ruaRes}より,
[01] 行動,[09] 報告,[10] 状況外回想,[11] 予測に分類されたものが合計397件(代表値が最大値の場合)および408件(代表値が最小値の場合)で,
代表値を最大値,最小値とした場合のいずれについても,分析対象の80\%以上となる.
つまり,{\bf 震災時の流言のカテゴリは 4 つ(行動を促す内容,状況の報告,
  状況外回想,予測)が大部分を占めていることがわかる.}



\subsection{修辞機能と脱文脈化指数による分類結果から見た有害性,有用性}
\label{sec:ResTotal}

本節では,修辞機能および脱文脈化指数による分類結果をもとに,
有害性,有用性との関連について考察する.
修辞ユニット分析は節ごとに分類を行うため,
1つのツイートに複数の修辞機能が特定され脱文脈化指数が付与される場合がある.
表~\ref{table:ruaRes}に示したように,代表値を最大値,最小値とした場合の分布は類似している.
それぞれの結果をもとに有害性,有用性との関連を確認した結果,いずれも同様の傾向を示したが,
最小値を用いた場合により顕著な傾向が見られたため,以降の分析では脱文脈化指数の最小値を代表値とした場合の分類結果をもとに議論する.


\subsubsection*{有害性との関連}

図~\ref{fig:bargraph_yugai}に,有害性の各評価値に分類された流言に関する,修辞機能と脱文脈化指数の割合を示す.
なお,図~\ref{fig:bargraph_yugai}では,
有害性の評価結果(中央値)に基づき,有害性の低いもの(評価値1,1.5,2),
中程度のもの(評価値2.5,3,3.5),高いもの(評価値4,4.5,5)に分類されたものをまとめた際の修辞機能と脱文脈化指数の割合を提示している.
各評価値における修辞機能と脱文脈化指数の割合については,付録における図~\ref{fig:bargraph1}として提示している.
また,分類結果の例として,[01] 行動と[09] 報告の例を表~\ref{table:ruaRes1}に示す.

図~\ref{fig:bargraph_yugai}より,有害性が高いと評価された流言(評価値4〜5)は,
修辞機能と脱文脈化指数の分類結果としては,[01] 行動に約30\%が,[11] 予測に約25\%の流言が分類されている.
ここでいう行動には注意喚起や救援要請など,情報受信者の行動を促進するものが含まれ(表~\ref{table:ruaRes1}),
この結果は,\ref{sec:ruaRes}節で述べた,行動促進が有害性と相関していることを裏付けている.

逆に,有害性が低いと評価された流言(評価値1〜2)の70\%程度は,修辞機能と脱文脈化指数が[09] 報告や[10]状況外回想に分類されている.

このように,本結果から,行動促進のみが有害性と相関するだけでなく,
有害性を低くする要素として,回想や報告があることが伺える.

\begin{figure}[p]
\begin{center}
\includegraphics{20-3ia24f1.eps}
\end{center}
\caption{有害性と修辞機能および脱文脈化指数}
\label{fig:bargraph_yugai}
\end{figure}
\begin{table}[p]
\caption{有害性評価結果における特徴的な分類結果の例}
\label{table:ruaRes1}
\input{24table12.txt}
\end{table}

\subsubsection*{有用性との関連}

図~\ref{fig:bargraph_yuyou}に,有用性の各評価値に分類された流言に関する,
修辞機能と脱文脈化指数の割合を示す.
図~\ref{fig:bargraph_yuyou}についても,図~\ref{fig:bargraph_yugai}と同様に,
有用性の評価結果(中央値)に基づき,有用性の低いもの(評価値1,1.5,2),
中程度のもの(評価値2.5,3,3.5),高いもの(評価値4,4.5,5)に分類されたものをまとめた際の修辞機能と脱文脈化指数の割合を提示している.
各評価値における修辞機能と脱文脈化指数の割合については,付録における図~\ref{fig:bargraph2}として提示する.

\begin{figure}[b]
\begin{center}
\includegraphics{20-3ia24f2.eps}
\end{center}
\caption{有用性と修辞機能および脱文脈化指数}
\label{fig:bargraph_yuyou}
\end{figure}

先の有害性と同じく,有用性が高いと評価された流言の30\%前後が[01] 行動に,25\%程度が[11] 予測に分類され,
有用性が低いと評価された流言の74\%程度が[09] 報告や[10] 状況外回想に分類された.
表~\ref{table:ruaRes2}に,分類結果の例として[10] 状況外回想と[11] 予測の例を示す.

このように,有害性と有用性は基本的には同様の傾向を示すことがわかった.


\subsubsection*{有害性,有用性と修辞機能との関連のまとめ}

以上の有害性,有用性との関連の結果から,行動を促すテキストおよび将来発生し得る事象の予測を含むテキストは,震災時高い有用性と有害性を持つと判断される.
また,回想や報告を含むテキストは,震災時の有用性と有害性が低い傾向がある.
つまり,{\bf 情報受信者の未来の行動に影響を与えうる表現を含む情報は,震災時に高い有用性と有害性を持ち,
過去に発生したことの報告については,有用性・有害性が低いと考えられる.}

\begin{table}[t]
\caption{有用性評価結果における特徴的な分類結果の例}
\label{table:ruaRes2}
\input{24table13.txt}
\vspace{0.5zw}\small
* 「〜と聞いた」のような形式のテキストについては,「〜」の部分が分析対象となる.\par
\end{table}


\subsection{表現の違いによる影響}
\label{sec:diff}

\ref{sec:dataset}節で述べたように,本論文における抽出手順では,同じ流言の異なる表現のバリエーションも抽出されうる.
本論文では,同じ流言を意図していても,伝え方によって印象が異なる可能性があると考え,
1つの流言に対する分析対象を1つのテキストに限定せず,複数のテキストを扱った.
しかし,同じ流言を意図する表現が大量に含まれる場合,それらが結果に影響する可能性がある.
そこで,本節では,1つの流言に対するテキストを限定した場合の結果について述べる.

まず,表~\ref{table:difEx}に示したような,同じ内容を取り扱っているが異なる表現を持つものを1つの流言と見なした場合の,
データセット中の流言数を確認した.
流言テキストに含まれるキーワードをもとに分類し,さらに人手で内容を確認しながら流言内容毎の表現バリエーション数を調査した.
確認の結果,481件の流言テキストに含まれる独立した流言内容は256件であった.
表~\ref{table:variation}に,表現バリエーション数を示す.
2つ以上の表現バリエーションを持つ流言内容は256件中44件であり,
1つの流言内容に対する最大の表現バリエーション数は,81バリエーションであった.

次に,複数の表現バリエーションを持つ流言内容から,代表となるテキストをランダムに抽出した.
なお,同じ流言を意図する複数の表現が,すべて同じ修辞機能に認定されるとは限らない.
修辞機能により違いがある可能性もあるため,
今回はある流言に対して単純に1つのテキストを抽出するのではなく,
修辞機能に違いのあるテキストが含まれる場合は,修辞機能ごとに1つずつ抽出することとした.
上記の条件で抽出されたテキストは300件である.

300件のテキストを用いて,\ref{sec:subjective}節〜\ref{sec:ResTotal}節と同様の分析を行った結果,
\ref{sec:subjective}節〜\ref{sec:ResTotal}節で示した結果と同様の傾向が見られた\footnote{300件のテキストによる結果については,付録として提示する.}.
したがって,今回の分析においては,流言における複数の表現は,分析結果に大きな影響は与えていないと考えられる.

\begin{table}[t]
\caption{表現バリエーション数}
\label{table:variation}
\input{24table14.txt}
\end{table}


\subsection{分析結果の限定性}

本章では,震災時の流言テキストを対象として流言内容の主観評価,分類を行い,
流言テキストの持ち得る性質についてまとめた.

本論文で得られた結果は,流言テキストのみを対象として調査した結果得られたものであり,
流言ではないものについても,今回明らかにした流言テキストと同様の性質を持つ可能性もある.
つまり,本論文で得られた結論が,流言のみにあてはまるものであるかどうかという点までは,本論文では検証できていない.
今後,流言以外のテキストを対象とした調査を行い,流言との違いの有無を確認し,
今回得られた結論が流言のみに限定されるものか,テキスト全般に適用されるものかを明らかにする必要がある.


\section{将来への展望}

本研究により,流言において,有害性と有用性に影響を与える要素として,行動の促進,予測がある
ことが分かった.
一部の行動を促す表現は「〜して下さい」「に注意!」など,典型な表現を含んでいるため,
本研究の知見により,大量の流言の中から,有害または有用であるものをある程度
ピックアップすることも可能だと思われる.

我々は自動的に流言を収集するサービスをすでに動かしているが\cite{Article_MiyabeRakuten,Article_MiyabeDICOMO},今後,本知見
による有害性,有用性推定システムを組み込む予定である.


\section{おわりに}

本研究では,マイクロブログ上での流言の特徴を明らかにするために,Twitterを例とした分析を行った.
分析対象として,東日本大震災時のTwitterデータから抽出した481件の流言テキストを用いた.
流言テキストに対する主観評価および修辞ユニット分析を行い,
震災時に発生したマイクロブログ上の流言テキストには,以下の傾向があることを明らかにした.

\begin{enumerate}
\item 情報が正しい場合に有用性の高い内容は,その情報が間違っていた場合に有害性がある.
\item 震災時に拡散する流言テキストは,行動を促す内容や,状況の報告,回想,予測が大部分を占める.
\item 情報受信者の行動に影響を与えうる表現,または,予想を含む情報は,高い有用性と有害性を持つと考えられる.
\end{enumerate}

ただし,上記の結論は,流言テキストのみを対象として調査した結果得られたものであり,
これらの性質が流言のみにあてはまるものであるかどうかは不明である.

今後は,流言以外のテキストを対象とした調査を行い,上述した結論が流言のみに限定されるものなのかどうかの検証が必要である.
また,得られた知見に基づき,流言拡散を防ぐための仕組みを検討していく必要がある.




\acknowledgment

本研究の一部は,JST戦略的創造研究推進事業による.



\bibliographystyle{jnlpbbl_1.5}
\begin{thebibliography}{}

\bibitem[\protect\BCAY{Back, Kufner, \BBA\ Egloff}{Back
  et~al.}{2010}]{Article_Back}
Back, M.~D., Kufner, A. C.~P., \BBA\ Egloff, B. \BBOP 2010\BBCP.
\newblock \BBOQ The Emotional Timeline of September 11, 2001.\BBCQ\
\newblock {\Bem Psychological Science}, {\Bbf 21}  (10), \mbox{\BPGS\
  1417--1419}.

\bibitem[\protect\BCAY{Cloran}{Cloran}{1994}]{Cloran94}
Cloran, C. \BBOP 1994\BBCP.
\newblock {\Bem Rhetorical units and decontextualisation: an enquiry into some
  relations of context, meaning and grammar}.
\newblock Ph.D.\ thesis, Nottingham University.

\bibitem[\protect\BCAY{Cloran}{Cloran}{1999}]{Cloran99}
Cloran, C. \BBOP 1999\BBCP.
\newblock \BBOQ Instruction at home and school.\BBCQ\
\newblock In Christie, F.\BED, {\Bem Pedagogy and the shaping of consciousness:
  Linguistic and social processes}, \mbox{\BPGS\ 31--65}. Cassell, London.

\bibitem[\protect\BCAY{Cloran}{Cloran}{2010}]{Cloran2010}
Cloran, C. \BBOP 2010\BBCP.
\newblock \BBOQ Rhetorical unit analysis and Bakhtin's chronotype.\BBCQ\
\newblock {\Bem Functions of Language}, {\Bbf 17}  (1), \mbox{\BPGS\ 29--70}.

\bibitem[\protect\BCAY{Cohn, Mehl, \BBA\ Pennebaker}{Cohn
  et~al.}{2004}]{Article_Cohn}
Cohn, M.~A., Mehl, M.~R., \BBA\ Pennebaker, J.~W. \BBOP 2004\BBCP.
\newblock \BBOQ Linguistic Markers of Psychological Change Surrounding
  September 11, 2001.\BBCQ\
\newblock {\Bem Psychological Science}, {\Bbf 15}  (10), \mbox{\BPGS\
  687--693}.

\bibitem[\protect\BCAY{De~Longueville, Smith, \BBA\ Luraschi}{De~Longueville
  et~al.}{2009}]{Inproc_Longueville}
De~Longueville, B., Smith, R.~S., \BBA\ Luraschi, G. \BBOP 2009\BBCP.
\newblock \BBOQ ``OMG, from here, I can see the flames!'': a use case of mining
  location based social networks to acquire spatio-temporal data on forest
  fires.\BBCQ\
\newblock In {\Bem Proceedings of the 2009 International Workshop on Location
  Based Social Networks}, LBSN '09, \mbox{\BPGS\ 73--80}. ACM.

\bibitem[\protect\BCAY{G.W.オルポート\JBA L.ポストマン}{G.W.オルポート\JBA
  L.ポストマン}{2008}]{Book_dema}
G.W.オルポート\JBA L.ポストマン \BBOP 2008\BBCP.
\newblock \Jem{デマの心理学}.
\newblock 岩波書店.

\bibitem[\protect\BCAY{Halliday \BBA\ Matthiessen}{Halliday \BBA\
  Matthiessen}{2004}]{Book_Halliday}
Halliday, M. A.~K.\BBACOMMA\ \BBA\ Matthiessen, C. M. I.~M. \BBOP 2004\BBCP.
\newblock {\Bem An introduction to functional grammar\/} (3rd ed \BEd).
\newblock Arnold, Hodder Education.

\bibitem[\protect\BCAY{インプレス~R\&D}{インプレス~R\&D}{2011}]{Book_Hakusho}
インプレス~R\&Dインターネットメディア総合研究所 \BBOP 2011\BBCP.
\newblock \Jem{インターネット白書2011}.
\newblock インプレスジャパン.

\bibitem[\protect\BCAY{川上}{川上}{1997}]{Book_Kawakami}
川上善郎 \BBOP 1997\BBCP.
\newblock \Jem{うわさが走る 情報伝搬の社会心理}.
\newblock サイエンス社.

\bibitem[\protect\BCAY{Knapp}{Knapp}{1944}]{ナップ1944}
Knapp, R.~H. \BBOP 1944\BBCP.
\newblock \BBOQ A Psychology of Rumor.\BBCQ\
\newblock {\Bem Public Opinion Quarterly}, {\Bbf 8}  (1), \mbox{\BPGS\ 22--37}.

\bibitem[\protect\BCAY{小林}{小林}{2011}]{Book_Kobayashi}
小林啓倫 \BBOP 2011\BBCP.
\newblock
  \Jem{災害とソーシャルメディア〜混乱、そして再生へと導く人々の「つながり」〜}.
\newblock 毎日コミュニケーションズ.

\bibitem[\protect\BCAY{Mendoza, Poblete, \BBA\ Castillo}{Mendoza
  et~al.}{2010}]{Inproc_Mendoza}
Mendoza, M., Poblete, B., \BBA\ Castillo, C. \BBOP 2010\BBCP.
\newblock \BBOQ Twitter under crisis: can we trust what we RT?\BBCQ\
\newblock In {\Bem Proceedings of the First Workshop on Social Media
  Analytics}, SOMA '10, \mbox{\BPGS\ 71--79}. ACM.

\bibitem[\protect\BCAY{宮部\JBA 梅島\JBA 灘本\JBA 荒牧}{宮部 \Jetal
  }{2011}]{Article_MiyabeRakuten}
宮部真衣\JBA 梅島彩奈\JBA 灘本明代\JBA 荒牧英治 \BBOP 2011\BBCP.
\newblock 流言訂正情報に基づいた流言情報クラウドの提案.\
\newblock \Jem{第4回楽天研究開発シンポジウム}, \mbox{\BPGS\ 1--4}.

\bibitem[\protect\BCAY{宮部\JBA 梅島\JBA 灘本\JBA 荒牧}{宮部 \Jetal
  }{2012}]{Article_MiyabeDICOMO}
宮部真衣\JBA 梅島彩奈\JBA 灘本明代\JBA 荒牧英治 \BBOP 2012\BBCP.
\newblock 人間による訂正情報に着目した流言拡散防止サービスの構築.\
\newblock \Jem{マルチメディア,分散,協調とモバイル(DICOMO2012)シンポジウム},
  \mbox{\BPGS\ 1442--1449}.

\bibitem[\protect\BCAY{西谷}{西谷}{2010}]{Article_Nishitani}
西谷智広 \BBOP 2010\BBCP.
\newblock I 見聞録:Twitter研究会.\
\newblock \Jem{情報処理学会誌}, {\Bbf 51}  (6), \mbox{\BPGS\ 719--724}.

\bibitem[\protect\BCAY{荻上}{荻上}{2011}]{Book_Ogiue}
荻上チキ \BBOP 2011\BBCP.
\newblock \Jem{検証 東日本大震災の流言・デマ}.
\newblock 光文社新書.

\bibitem[\protect\BCAY{Qazvinian, Rosengren, Radev, \BBA\ Mei}{Qazvinian
  et~al.}{2011}]{Inproc_Qazvinian}
Qazvinian, V., Rosengren, E., Radev, D.~R., \BBA\ Mei, Q. \BBOP 2011\BBCP.
\newblock \BBOQ Rumor has it: Identifying Misinformation in Microblogs.\BBCQ\
\newblock In {\Bem EMNLP}, \mbox{\BPGS\ 1589--1599}. ACL.

\bibitem[\protect\BCAY{Qu, Huang, Zhang, \BBA\ Zhang}{Qu
  et~al.}{2011}]{Inproc_Qu}
Qu, Y., Huang, C., Zhang, P., \BBA\ Zhang, J. \BBOP 2011\BBCP.
\newblock \BBOQ Microblogging after a major disaster in China: a case study of
  the 2010 Yushu earthquake.\BBCQ\
\newblock In {\Bem Proceedings of the ACM 2011 conference on Computer supported
  cooperative work}, CSCW '11, \mbox{\BPGS\ 25--34}. ACM.

\bibitem[\protect\BCAY{Qu, Wu, \BBA\ Wang}{Qu et~al.}{2009}]{Inproc_Qu2009}
Qu, Y., Wu, P.~F., \BBA\ Wang, X. \BBOP 2009\BBCP.
\newblock \BBOQ Online Community Response to Major Disaster: A Study of Tianya
  Forum in the 2008 Sichuan Earthquake.\BBCQ\
\newblock In {\Bem HICSS}, \mbox{\BPGS\ 1--11}. IEEE Computer Society.

\bibitem[\protect\BCAY{佐野}{佐野}{}]{Web_Sano}
佐野大樹.
\newblock
  日本語における修辞ユニット分析の方法と手順ver.0.1.1—選択体系機能言語理論(
システミック理論)における談話分析—(修辞機能編).\
\newblock \Turl{http://researchmap.jp/systemists/資料公開/}.

\bibitem[\protect\BCAY{佐野}{佐野}{2010}]{Article_Sano}
佐野大樹 \BBOP 2010\BBCP.
\newblock
  特集選択体系機能言語理論を基底とする特定目的のための作文指導方法について—修
辞ユニットの概念から見たテクストの専門性.\
\newblock \Jem{専門日本語教育研究}, {\Bbf 12}, \mbox{\BPGS\ 19--26}.

\bibitem[\protect\BCAY{佐野\JBA 小磯}{佐野\JBA 小磯}{2011}]{Article_Sano2011}
佐野大樹\JBA 小磯花絵 \BBOP 2011\BBCP.
\newblock
  現代日本語書き言葉における修辞ユニット分析の適用性の検証—「書き言葉らしさ・
話し言葉らしさ」と脱文脈化言語・文脈化言語の関係—.\
\newblock \Jem{機能言語学研究}, {\Bbf 6}, \mbox{\BPGS\ 59--81}.

\bibitem[\protect\BCAY{佐藤}{佐藤}{2007}]{Article_Sato}
佐藤健二 \BBOP 2007\BBCP.
\newblock 関東大震災後における社会の変容.\
\newblock
  \Jem{立命館大学・神奈川大学21世紀COEプログラムジョイントワークショップ報告書
『歴史災害と都市—京都・東京を中心に—』}, \mbox{\BPGS\ 81--89}.

\bibitem[\protect\BCAY{立入}{立入}{2011}]{Book_Tachiiri}
立入勝義 \BBOP 2011\BBCP.
\newblock \Jem{検証 東日本大震災 そのときソーシャルメディアは何を伝えたか?}
\newblock ディスカヴァー・トゥエンティワン.

\bibitem[\protect\BCAY{田中\JBA 佐野}{田中\JBA
  佐野}{2011}]{Inproc_TanakaandSano3}
田中弥生\JBA 佐野大樹 \BBOP 2011\BBCP.
\newblock
  Yahoo!知恵袋における質問と回答の分類—修辞ユニット分析を用いた脱文脈化—文脈
化の程度による検討—.\
\newblock \Jem{社会言語科学会第27回大会発表論文集}, \mbox{\BPGS\ 208--211}.

\bibitem[\protect\BCAY{田中}{田中}{2011}]{Inproc_Tanaka}
田中弥生 \BBOP 2011\BBCP.
\newblock 修辞ユニット分析を用いた Q \& A
  サイトの質問と回答における修辞機能の展開の検討.\
\newblock \Jem{社会言語科学会第28回大会発表論文集}, \mbox{\BPGS\ 226--229}.

\bibitem[\protect\BCAY{田中\JBA 佐野}{田中\JBA
  佐野}{2011a}]{Inproc_TanakaandSano}
田中弥生\JBA 佐野大樹 \BBOP 2011a\BBCP.
\newblock
  Yahoo!知恵袋における質問の修辞ユニット分析—脱文脈化—文脈化の程度による分類
—.\
\newblock \Jem{信学技報}, {\Bbf 110}  (400), \mbox{\BPGS\ 13--18}.

\bibitem[\protect\BCAY{田中\JBA 佐野}{田中\JBA
  佐野}{2011b}]{Inproc_TanakaandSano2}
田中弥生\JBA 佐野大樹 \BBOP 2011b\BBCP.
\newblock 修辞ユニット分析からみたQ\&Aサイトの言語的特徴.\
\newblock \Jem{言語処理学会第17回年次大会(NLP2011)論文集}, \mbox{\BPGS\
  248--251}.

\bibitem[\protect\BCAY{垂水}{垂水}{2010}]{Article_Tarumi}
垂水浩幸 \BBOP 2010\BBCP.
\newblock 実世界インタフェースの新たな展開:4.ソーシャルメディアと実世界.\
\newblock \Jem{情報処理学会誌}, {\Bbf 51}  (7), \mbox{\BPGS\ 782--788}.

\bibitem[\protect\BCAY{梅島\JBA 宮部\JBA 荒牧\JBA 灘本}{梅島 \Jetal
  }{2011}]{Article_Umejima}
梅島彩奈\JBA 宮部真衣\JBA 荒牧英治\JBA 灘本明代 \BBOP 2011\BBCP.
\newblock 災害時Twitterにおけるデマとデマ訂正RTの傾向.\
\newblock \Jem{情報処理学会研究報告.データベース・システム研究会報告}, {\Bbf
  2011}  (4), \mbox{\BPGS\ 1--6}.

\bibitem[\protect\BCAY{Vieweg, Hughes, Starbird, \BBA\ Palen}{Vieweg
  et~al.}{2010}]{Inproc_Vieweg}
Vieweg, S., Hughes, A.~L., Starbird, K., \BBA\ Palen, L. \BBOP 2010\BBCP.
\newblock \BBOQ Microblogging during two natural hazards events: what twitter
  may contribute to situational awareness.\BBCQ\
\newblock In {\Bem Proceedings of the SIGCHI Conference on Human Factors in
  Computing Systems}, CHI '10, \mbox{\BPGS\ 1079--1088}. ACM.

\end{thebibliography}



\appendix
\section{各評価値における修辞機能と脱文脈化指数の割合}
\label{sec:append}

有害性および有用性の評価値毎に分類された流言に関する修辞機能と脱文脈化指数の割合を,
図~\ref{fig:bargraph1}および図~\ref{fig:bargraph2}にそれぞれ示す.

\begin{figure}[h]
\begin{center}
\includegraphics{20-3ia24f3.eps}
\end{center}
\caption{有害性(各評価値)と修辞機能および脱文脈化指数}
\label{fig:bargraph1}
\end{figure}
\clearpage

\begin{figure}[t]
\begin{center}
\includegraphics{20-3ia24f4.eps}
\end{center}
\caption{有用性(各評価値)と修辞機能および脱文脈化指数}
\label{fig:bargraph2}
\end{figure}

\clearpage

\section{1つの流言に対するテキストを限定した場合の分析結果}

\ref{sec:diff}節で述べた,1つの流言に対するテキストを限定し,300件のテキストを用いた場合の分析結果として,以下のデータを提示する.

\begin{enumerate}
\item 各主観評価結果の相関係数(表\ref{table:correl_300},\ref{table:correl2_300})
\item 修辞機能と脱文脈化指数による分類結果(表\ref{table:ruaRes_300})
\item 有害性および有用性の評価値毎に分類された流言に関する修辞機能と脱文脈化指数の割合(図\ref{fig:bargraph_yugai_300},\ref{fig:bargraph_yuyou_300})
\end{enumerate}


\begin{table}[h]
\caption{300件のテキストにおける主観評価結果の相関係数(中央値を用いた場合)}
\label{table:correl_300}
\input{24table15.txt}
\end{table}
\begin{table}[h]
\caption{300件のテキストにおける主観評価結果の相関係数(全評価値を用いた場合)}
\label{table:correl2_300}
\input{24table16.txt}
\end{table}
\clearpage

\begin{table}[h]
\caption{300件のテキストにおける修辞機能と脱文脈化指数による分類結果}
\label{table:ruaRes_300}
\input{24table17.txt}
\vspace{0.5zw}\small
* 修辞ユニット分析は節ごとに分類を行うため,1つのツイートに複数の修辞機能が認定され,脱文脈化指数が付与される場合がある.
そこで,表~\ref{table:ruaRes_300}には1つのツイートに付与された脱文脈化指数のうち,
最大値および最小値を代表値とした場合の該当数を提示している.\par
\end{table}

\clearpage
\begin{figure}[p]
\begin{center}
\includegraphics{20-3ia24f5.eps}
\end{center}
\caption{300件のテキストにおける有害性と修辞機能および脱文脈化指数}
\label{fig:bargraph_yugai_300}
\end{figure}
\begin{figure}[p]
\begin{center}
\includegraphics{20-3ia24f6.eps}
\end{center}
\caption{300件のテキストにおける有用性と修辞機能および脱文脈化指数}
\label{fig:bargraph_yuyou_300}
\end{figure}

\clearpage

\begin{biography}
\bioauthor{宮部 真衣}{
2006年和歌山大学システム工学部デザイン情報学科中退.
2008年和歌山大学大学院システム工学研究科システム工学専攻博士前期課程修了.
2011年和歌山大学大学院システム工学研究科システム工学専攻博士後期課程修了.
博士(工学).現在,東京大学知の構造化センター特任研究員.
コミュニケーション支援に関する研究に従事.
}
\bioauthor{田中 弥生}{
1997年青山学院大学大学文学部第二部英米文学科卒業.
1999年青山学院大学大学院国際政治経済学研究科国際コミュニケーション専攻修士課程修了.
修士(国際コミュニケーション学).
現在,神奈川大学外国語学部,青山学院女子短期大学非常勤講師.英語およびコミュニケーション論を担当.
}
\bioauthor{西畑  祥}{
2013年甲南大学知能情報学部知能情報学科卒業.
在学中は,マイクロブログ上の流言情報の特徴分析に関する研究に従事.
}
\bioauthor{灘本 明代}{
東京理科大学理工学部電気工学科卒業.
2002年神戸大学大学院自然科学研究科情報メディア科学専攻後期博士課程修了.
博士(工学).
現在,甲南大学知能情報学部教授.
Webコンピューティング,データ工学の研究に従事.
ACM,IEEE,情報処理学会,電子情報通信学会会員.
}
\bioauthor{荒牧 英治}{
2000年京都大学総合人間学部卒業.
2002年京都大学大学院情報学研究科修士課程修了.
2005年東京大学大学院情報理工系研究科博士課程修了(情報理工学博士).
以降,東京大学医学部附属病院企画情報運営部特任助教,東京大学知の構造化センター特任講師を経て,
現在,京都大学デザイン学ユニット特定准教授,科学技術振興機構さきがけ研究員(兼任) .
自然言語処理,医療情報学の研究に従事.
}
\end{biography}



\biodate






\end{document}
