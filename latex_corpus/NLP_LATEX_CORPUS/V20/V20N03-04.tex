    \documentclass[japanese]{jnlp_1.4}
\usepackage{jnlpbbl_1.3}
\usepackage[dvips]{graphicx}
\usepackage{amsmath}
\usepackage{hangcaption_jnlp}
\usepackage{udline}
\setulminsep{1.2ex}{0.2ex}
\let\underline

\usepackage{url}
\usepackage{ascmac}
    \newcommand{\twitter}[3]{}
\newcommand{\mytwitter}[2]{}



\Volume{20}
\Number{3}
\Month{June}
\Year{2013}

\received{2012}{12}{14}
\revised{2013}{2}{28}
\accepted{2013}{3}{29}

\setcounter{page}{405}

\jtitle{「東日本大震災関連の救助要請情報抽出サイト」による\\
	救助活動支援}
\jauthor{相田  慎\affiref{TUT} \and 新堂 安孝\affiref{FREE} \and 内山 将夫\affiref{NICT}}
\jabstract{
  東日本大震災初期,Twitter に寄せられた膨大なツィートには,
  緊急性の高い救助要請候補が多数含まれていたものの,
  他の震災関連ツィートや「善意のリツィート」によって,
  通報されるべき情報が埋もれてしまった.
  この様な状況を解消するために,
  筆者らは2011年3月16日,Twitter 上の救助要請情報をテキストフィルタリングで抽出し,
  類似文を一つにまとめ一覧表示するWebサイトを開発・公開した.
  本論文では,本サイト技術のみならず,通報支援活動プロジェクト {\tt \#99japan} との
  具体的な連携・活用事例についても詳述する.
  なお {\tt \#99japan} は,救助状況の進捗・完了報告を重視する Twitter を用いた活動であると共に,
  発災2時間後に2ちゃんねる臨時地震板ボランティアらによって
  立ち上げられたスレッドに由来する.
}
\jkeywords{災害,Twitter,2ちゃんねる,救助要請,情報支援,テキストフィルタリング}

\etitle{Rescue Activity for the Great East Japan Earthquake Based on a Website that Extracts Rescue Requests \\
	from the Net}
\eauthor{Shin Aida\affiref{TUT} \and Yasutaka Shindo\affiref{FREE} \and Masao Utiyama\affiref{NICT}} 
\eabstract{
In the early stages of the Great East Japan Earthquake, 
a vast number of tweets were related to high-urgency rescue requests; 
however, most of these tweets were buried under many other tweets, 
including some well-intentioned retweets of the rescue requests. 
To better handle such a situation, 
the authors have developed and published a website that
automatically lists similar statements to extract rescue requests 
from Twitter on March 16, 2011. 
This paper describes not only the technology of the system but also the start of a rescue project {\tt \#99japan}. 
The project takes particular note of the progress and completion reports of the rescue situations, 
using this site as sources of rescue information.
Note that {\tt \#99japan} originated from a thread of the Japanese textboard
\textit{2channel}, which was launched by some volunteers within two hours of the disaster's occurrence.
}
\ekeywords{disaster, Twitter, 2channel, rescue requests, information support, text filtering}

\headauthor{相田,新堂,内山}
\headtitle{「東日本大震災関連の救助要請情報抽出サイト」による救助活動支援}

\affilabel{TUT}{豊橋技術科学大学大学院工学研究科情報・知能工学系}{Department of Computer Science and Engineering, Toyohashi University of Technology}
\affilabel{FREE}{フリー}{Free}
\affilabel{NICT}{独立行政法人情報通信研究機構ユニバーサルコミュニケーション研究所}{Universal Communication Research Institute, National Institute of Information and Communications Technology}



\begin{document}
\maketitle


\section{はじめに}  
2011年3月11日に発生した東日本大震災の
被災範囲の広大さは記憶に新しい.
この震災では,既存マスメディア(放送・新聞・雑誌等)だけでなく,
Twitter などのソーシャルメディアによる情報発信が盛んに行われた \cite{Shimbun,Endo2}. 
しかしながら,大手既存メディアは被災報道を重視していた.
実際,被災者にとって有用な報道として,
災害時でも乾電池で駆動可能なラジオ,並びに,
無料で避難所等へ配布された地元地方紙が役に立ったことが,
\cite{Fukuda}の被災者アンケートで調査報告されている. 


この様な震災初期の状況の理由として,
阿部正樹(IBC岩手放送社長)は,
震災発生当時の被災地において,テレビは
「テレビ報道は系列間競争の中でどうしても全国へ向かって放送せざるを得ない.(中略)
被災者に面と向き合う放送がなかなか出来ない,被災者のためだけの放送に徹し切れない.
」
というジレンマがあったとする一方,
「しかしラジオは違う.地域情報に徹することが出来る.
(中略)
テレビではどこそこの誰が無事だという情報はニュースになりづらい.
しかし,ラジオでは大切な情報なのだ.
いつしかラジオが安全情報,安否情報へと流れていったのは自然なことだったと思う.」
と述懐している \cite{IBC}. 

震災初期から,ソーシャルメディアの一つである Twitter には,
救助要請ハッシュタグ {\tt \#j\_j\_helpme}
\cite{Twitter_tags} が付与された 大量の救助の声が寄せられていた.
(被災地マスメディアの一つであるラジオ福島は,当時生きていた3~G回線を用いて,
Twitter による情報収集・発信を行っている \cite{rfc}.)
ただし,これら救助要請の多くには「【拡散希望】」という文字列が含まれていたため,
それを見た「善意の第三者」は, 
Twitter のリツィート機能(全文引用機能)を用いる傾向が高かった\cite{Ogiue,Tachiiri}.
結果として,リツィートによって救助要請の類似情報が Twitter へ膨大に流れたものの,
「実際に救助要請情報が警察など関係機関へ適切に通報されたかどうか」という
最も重要な情報のトレースは,著しく困難なものになった. 


この様な状況を解消するために,
我々は2011年3月15日,Twitter 上の救助要請情報をテキストフィルタリングで抽出し,
類似文を一つにまとめて一覧表示するWebサイトを開発し,翌16日に公開した\cite{Aida0,extraction,Aida1}.
本論文では,本サイトの技術のみならず,救助要請の情報支援活動プロジェクト {\tt \#99japan} と
本サイトとの具体的な連携・活用事例について述べる.

ここで {\tt \#99japan} とは,救助状況の進捗・完了報告を重視する Twitter を用いたプロジェクトであると共に,
発災2時間後,2ちゃんねる臨時地震板ボランティアらによって
立ち上げられたスレッド「【私にも】三陸沖地震災害の情報支援【できる】」\cite{2ch}を由来する.
このスレッドは,「震災初期におけるネット上のアウトリーチ活動記録」として,特筆に値する.


\section{Twitter}

{\bf Twitter}とは,
Web上で140字以内の短文による
{\bf ツィート(Tweet)}
を投稿することで情報発信出来る,
{\bf ソーシャルメディア}の一種である.
Twitterのブラウザ表示例を図\ref{fig:Twitter}に示す.
以降,本論文に関連する Twitter の用語・概念について述べるが,
より詳細については\cite{Twitter_help}を参照されたい.

\begin{figure}[b]
  \begin{center}
    \includegraphics{20-3ia5f1.eps}
  \end{center}
  \caption{Twitterのブラウザ表示の一例(一関コミュニティFM・{\tt @FMasmo}のツィート)}
  \label{fig:Twitter}
\end{figure}

{\bf ユーザ名(User Name)}とは,Twitter へのログインIDであり,
{アルファベット・数字・アンダースコアの組合せ文字列}として
(既存ユーザ名と重複がなければ)自由に決められる.

従来の電子掲示板やチャットとは異なり,
ユーザ同士が{\bf フォロー(Follow)}し合うことで,ソーシャルネットワークを成す.
このネットワークにおいて,フォロー元のユーザを{\bf フォロワー}と呼ぶ.
ユーザのページには,自分のツィートと,フォローしている他ユーザのツィートなどが時系列順に表示される.
これを,ユーザの{\bf タイムライン(Time Line)}と呼ぶ.
{\bf パブリックタイムライン(Public Time Line)}とは,
「(非公開ユーザを除く)全ユーザのタイムライン」を指す.
ここで,{\bf 非公開ユーザ}とは,フォローを許可制にしているユーザを指す.
非公開ユーザでなければ誰からもフォローされ得るが,
任意のフォロワーを
{\bf ブロック}すると,フォローを拒否出来る.


{\bf 返信(Reply)}とは,ユーザのツィートの内容に対して,
メールの返信のようにそのユーザへ向けてツィートすることである.
返信する場合,「返信ボタン」(図\ref{fig:Twitter}の左方向矢印)をクリックすると,
文頭に返信先ユーザ名が``{\tt @ユーザ名}''のように表示される.
返信に限らず,任意のツィート中に``{\tt @ユーザ名}''を含めると,
ツィートはそのユーザのタイムラインにも反映される.
それを{\bf 言及(Mention)}と呼ぶ.言及は通例,
ユーザのツィートの出典が解るよう
``{\tt RT @ユーザ名: ツィートの一部}'' 付きで引用し,
それについて言及するツィートを前置する場合が多い.

{\bf リツィート(Retweet, RT)}とは,「全文引用すること」を意味する.
リツィートによって,ユーザのタイムライン上のツィートを全文引用し,
それをフォロワーへ一斉に拡散出来る.
リツィートを受信したフォロワーは,さらにそれをリツィートすることも出来るため,
一般に興味・関心を惹く深いツィートは伝播されやすい.
特に,フォロワー数の多いユーザ(有名人など)のリツィートは伝播されやすい傾向があり,
ソーシャルメディアの特徴と言える.

リツィートをする場合は,
「リツィートボタン」(図\ref{fig:Twitter}の折れ線矢印二組の記号)をクリックする.
この機能を{\bf 公式リツィート}と呼ぶ.
一方,{\bf 非公式リツィート}とは,
リツィートボタンを押さずに,コピー\&ペーストなどによる全文引用を指す.
非公式リツィートは公式リツィートとは異なり,
引用元ツィート日時ではなく,引用日時が記録されてしまう.
そのため,最初のツィート日時の把握が困難になる,
引用元がツィートを削除した場合も引用先情報は削除されない,など様々な問題点が指摘されている\footnote{ただし,かつてリツィートはユーザが情報拡散のために始めた試みであり,「{\tt RT @発言ユーザ名: ツィート} 」形式で全文引用していた.この機能を,Twitter社がRTボタンを押すだけでリツィート出来る機能を実装し,それを「公式リツィート」と呼んでいる.}.

{\bf ハッシュタグ}とは,ツィートの任意の位置に挿入出来る,自分の興味・関心に関係する
``{\tt \#}''で始まるアルファベット・数字・アンダースコアの組合せで表現される文字列
である.
(なお,2011年7月13日より「{\tt \#日本語}」などマルチバイト日本語文字列のハッシュタグが利用可能になった\cite{nihongo}.)
図\ref{fig:Twitter}中の ``{\tt \#cfmasmo}'' はハッシュタグの使用例である.

ツィート内の``{\tt @user}''や``{\tt \#hashtag}''から自動的にリンクが張られ,
そのリンクをクリックすることで,
「{\tt @user} のタイムライン」や「{\tt \#hashtag}のタイムライン」をそれぞれ得られる.
ハッシュタグに限らず,Twitterでは任意の検索語でパブリックタイムラインを,
図\ref{fig:Twitter}上部の検索フォーム等から検索出来る.

Twitter は,上述のツィート・リツィート・返信・言及・ハッシュタグなどによって,
フォロワーだけでなく,共通の関心事を持つユーザを意識した投稿が出来る {\bf マイクロブログ}でもある.

以降,本論文ではツィートを以下の形式で表す.
\begin{quote}
 \twitter{@user2 user2 user3 への言及ツィート \#tag${}_1$ ... \#tag${}_n$ \\ RT @user3: user3 のツィートの部分引用 ...}{YYYY/MM/DD hh:mm:ss}{user1}
\end{quote}



\section{救助要請情報}
\label{request}

震災初期の救助活動において,検討すべき救助要請情報を整理した結果,以下を識別した:
\begin{enumerate}
\item 
  救助要請の1次情報.
\item 
  救助要請の2次情報.(あるいは重複情報.)
\item 
  救助要請とは無関係な情報.
\item 
  救助完了報告.
\end{enumerate}
我々は,まず(1)〜(3)に着目した.
本論文では,(1) 及び (2) を{\bf 救助要請情報},
(3) を{\bf 非救助要請情報}と呼ぶ.
ここで (2) は,Twitter 上での情報拡散を希望する文字列(【拡散希望】等)が付与されてリツィートされた情報も含む.
このような情報は,救助活動においては情報が拡散するだけで通報されたのかどうか判明しない恐れがあるが,
遠隔地から被災地の現状をある程度把握できる利点も有り得る.


\subsection{Twitter 上の救助要請情報}

Twitter 上での救助要請の「1次情報」はオリジナルツィート,「2次情報」はリツィート・返信・言及にそれぞれ該当する.
その文例を以下に挙げる:
\begin{itemize}
\item {\bf 1次情報:}
  \begin{quote}
  \twitter{【拡散希望】宮城県の○○病院で100人以上孤立している模様. \#j\_j\_helpme\\}{2011/3/12 01:23:45}{foo}        
  \end{quote}
\item {\bf 2次情報(リツィート):}
  \begin{quote}
    \twitter{RT @foo:【拡散希望】宮城県の○○病院で100人以上孤立している模様. \#j\_j\_helpme}{2011/3/12 01:23:45}{bar}      
  \end{quote}
\item {\bf 2次情報(返信):}
  \begin{quote}
    \twitter{@foo 本件について,どなたも通報されていないようでしたので,警察へ通報致しました.\#j\_j\_helpme}{2011/3/12 12:34:50}{baz}      
  \end{quote}
\item {\bf 2次情報(言及):}
  \begin{quote}
    \twitter{どなたか本件の情報発信元をご存じの方はいらっしゃらないですか? 
      RT @foo: 【拡散希望】宮城県の○○病院で100人以上孤立している模様.\#j\_j\_helpme\\
    }{2011/3/12 01:30:00}{bar}
  \end{quote}
\end{itemize}
一般に,2次情報の場合は引用・返信先を明示する``{\tt @ユーザ名}'' が含まれ,
なおかつ,そのユーザのタイムラインにも2次情報のツィート本文が表示される.
また,1次情報でも,任意の ``{\tt @ユーザ名}'' を含むツィートが出来る.
本論文では,リツィートの場合,文頭に ``{\tt RT}''(ReTweet の略)と``{\tt @引用元ユーザ名}'' を付記する.
なお,言及の場合は,``{\tt RT}'' の他に ``{\tt QT}''(Quote Tweet の略)が用いられることもある.

Twitter では,
下記のように先頭から時系列を遡って言及文は前置,
引用文は後置され,一次情報が最後置される場合が顕著である.
\begin{quote}
  \twitter{user2への言及文 RT @user2: user3への言及文 RT @user3 \\$\ldots$ RT @user$n$: 一次情報}{YYYY/MM/DD hh:mm:ss}{user1}    
\end{quote}

救助要請の情報源は,被災地から直接寄せられたものだけでなく,
知人宛のメール伝聞や,電子掲示板情報など,様々ありえるが,
我々は{\bf 1次情報を「重要度の高い救助要請ツィートの情報源」}と定めた.
2次情報については,リツィートの場合,全文引用であるため1次情報と同一視可能であり,
また言及・返信の場合,上述の考察より最後置引用文が1次情報となる可能性が高いため,
重要度は1次情報よりも相対的に下がる.


\subsection{救助要請情報の傾向}
\label{key}

救助要請情報に含まれやすい語による検索結果には,
それ以外の情報も数多く含まれうる.
そこで,救助要請情報を抽出するために,それらの語の特徴を正規表現としてまとめた.

\begin{enumerate}
\item 救助要請情報は,以下を含むものとする:
  \begin{itemize}
  \item 
    「地名 + 県・市・区・町・村」など住所に含まれる語.{(5語)}(救助要請の発信先特定のため.)
  \item 
    「孤・救・助・命・探・願・捜・求・送・食・水・届・死・衰」や,
    「消息・深刻・要請・避難」など,
    ライフライン未復旧や安否確認に起因する語.{(21語)}
  \end{itemize}
\item 非救助要請情報は,以下を含むものとする:
  \begin{itemize}
  \item 過去のデマに含まれていた固有名詞{など.}

    {例:デマ,花山村,競輪場,ピースボート,ヨーグルト,納豆,など.(14語)}
  \item 報道機関の Twitter 公式アカウント.(報道機関{が発信した}情報は警察など関係機関へ既報済みの可能性が高いため.)

    {例:{\tt radio\_rfc\_japan}, {\tt fct\_staff}, {\tt [Aa]sahi}, {\tt FKSminpo}, {\tt [Nn][Hh][Kk]}, {\tt nhk\_seikatsu}, {\tt i\_jijicom\_eqa}, 
      {\tt kahoku\_shimpo}, {\tt akt\_akita\_tv}, {\tt NTV, telebee\_tnc}, {\tt NISHINIPPON}, {\tt zakdesk}, {\tt 781fm}, など.(20語)}
  \item 政治家や著名人など特定人名,国名,政党名,団体名.(思想・信条を含む情報は救助要請の可能性が低いため.)

    {例:民主党,自民党,社民党,共産党,公明党,みんなの党,など.(9語)}

  \item {特定の国名・国際機関.(国名・機関名は報道等の二次情報の可能性が高いため.}

    {例:アメリカ,フランス,ドイツ,ケニア,国連,ユニセフ.(6語)}
  \item 原発事故関連の語.(救助要請に
    科学用語を含む可能性は低いため.)
    
    {例:セシウム,ヨウ素,ウラン,プルトニウム,ストロンチウム,マイクロ,ベクレル,シーベルト,放射線,放射能.(10語)}

  \item {救助要請情報に用いられないであろう語.}

    {例:笑,批判,テロ,など.(8語)}

  \item 
    {ハッシュタグが濫用されたツィートに含まれていた語.}
    (ハッシュタグ{本来の意味}と無関係な内容のため.)

    {例:予測市場,リスクマネジメント情報,{\tt \#oogiri}, など.(6語)}
  \end{itemize}
\end{enumerate}


\subsection{救助要請情報の表示方針}

2011年3月15日は震災初期であったため,
救助要請情報の抽出処理サイトを速やかに構築・公開する必要があった.
我々は,必要な救助要請情報が埋もれないように敢えて単一ページ内に大量表示することにした.(当初は300件,2012年1月30日当時1000件.)
このようにした理由を以下に挙げる:
\begin{enumerate}
\item 抽出した情報はノイズが含まれうる.
  正規表現によるフィルタリング規則を厳しくすると,
  本当の救助要請情報が表示されない恐れがあるため,
  「本当にそれが真の救助要請かどうか」は,閲覧者の判断に委ねることとした.
\item 表示された救助要請情報を閲覧して通報活動を行うボランティアは,
  被災地以外の者でPC用電源環境が確保されていると仮定した.
  また,2011年時点の一般世帯のPC環境では,大量の情報を表示しても,
  ブラウザ動作が不安定になることは無い.
\item 同一ページ表示であれば,どのブラウザもデフォルトで備えている検索機能で,
  任意の語で文を検索できる.
\item ページ分割(pagination)されていると,利用者にリンク遷移作業などを強いるため,
  ページ分割を行わない.
\end{enumerate}
この方針に基づき,次節で述べる情報抽出アルゴリズムの試作版を2011年3月15日に実装,
翌16日にインターネット上に公開\cite{extraction}し,その後も改良に努めた.

また,デマ等のノイズは事前に想定出来ないため,
それらに含まれうる語を分析し,
前節の「非救助要請情報が含む語」の定義に適宜追記した.


\subsection{救助要請情報の抽出手法}
\label{algorithm}

\begin{figure}[t]
  \begin{center}
    \sloppy
    \begin{screen}
      \tt
      \#j\_j\_helpme \#j\_i\_helpme \#hinan \#jishin \#jisin \#tunami 
      \#311sppt \#311care \#311sien \#itaisousaku 99japan
      \#anpi \#aitai \#Funbaro \#hope4japan \#prayforjapan
      \#ganbappe \#save\_busshi \#save\_volunteer \#save\_gienkin \#save\_kids 
      \#saigai \#shinsai \#tasukeai \#fukkou \#fukko
      \#save\_miyagi \#save\_fukushima \#save\_iwate \#save\_aomori 
      \#save\_ibaraki \#save\_chiba \#save\_nagano \#save\_sendai
      \#save\_ishinomaki \#save\_iwaki \#ishinomaki \#shiogama 
      \#rikuzentakata 
      緊急地震 余震 火事 怪我 負傷者 
      自宅避難 避難所 孤立 餓死 \\緊急+救助 食料+不足 
      物資+不足 食糧+不足 救援 支援 安否 消息 栄村 陸前高田 
      釜石 大船渡 \\ 気仙沼 南三陸 歌津 志津川 石巻 松島 亘理 山元 相馬 いわき 飯舘
    \end{screen}
  \end{center}
  \caption{救助活動が最も盛んだった時期の検索語一覧(2012年4月22日確認時)}
  \label{keyword}
\end{figure}

救助要請情報の抽出アルゴリズムの手法概要は以下のとおりである:
\begin{enumerate}
\item 
  図\ref{keyword}に示した検索語それぞれについて,それを含むツィート情報を
  「1500ったー」\cite{1500}より HTML 取得する.
\item 取得したすべての情報に対して,以下を行う:
  \begin{enumerate}
  \item HTMLからツィートのみ抽出し,
    直前までに抽出していたものにマージ後,
    そのログを保存する.
  \item 
    抽出した救助要請情報候補から,非救助要請情報と思われるものを,\ref{key}節の規則に従ってフィルタリングする.
  \item フィルタリング済みの救助要請情報候補から,
    下記手続きで{\bf 類似ツィート判定キー}を生成する:
    \begin{enumerate}
    \item 文頭から``{\tt @}''までの最長文字列を削除する(ユーザ名付き1次情報抽出).
    \item ASCII文字(ユーザ名含む)・全角記号・仮名文字などを除去したマルチバイト文字列の前15字分を類似ツィート判定キーとする.
    \end{enumerate}
  \item 
    類似ツィート判定キーが一致するツィートは同値と見なし,ツィート回数をカウントする.
    同値類の中で,最も古い日時のものを,その同値類の{\bf 代表ツィート}とし,
    同値のツィートの最新日時を更新時刻として記録する.
  \item 更新時刻の新しい順でソートする.
  \end{enumerate}
\item 
  得られたツィート情報をそれぞれ下記5項目としてまとめ,上位1000件をHTMLへ変換し,サイトを更新する.
  \begin{itemize}
  \item 救助要請情報の通し番号
  \item 救助要請代表ツィート本文
  \item 最新ツィート日時
  \item 最古ツィート日時(本サイトで最初に観測した代表ツィートの日時)
  \item ツィート回数(代表ツィートの同値類の数)
  \item 推定情報源URL(``http://twitter.com/ユーザ名/statuses/発言ID'' の形式)
  \end{itemize}                                             
\end{enumerate}

ここで,ツィートには任意の位置に``{\tt @ユーザ名}''を含めることが出来るため,
以下のようにユーザ名を後置している場合,
1次情報が ``{\tt@}'' の前置であるため類似ツィート判定キーは空語になるが,
Twitter の慣習上,このような文例はほとんど出現しないことより無視する.
\begin{quote}
  \twitter{1次情報 $\ldots$ @user2}{YYYY/MM/DD hh:mm:ss}{user1}
\end{quote}
  


\section{救助活動}

Webベースでの救助活動には,ボランティア同士で救助要請の通報状況を共有する必要がある.
本サイト開設後,
このような活動を個人で行なっていた
株式会社エコヒルズ代表取締役・田宮嘉一氏との出会いがきっかけとなって,
\ref{99japan}節で述べる {\tt \#99japan} へ我々も参加した.


\subsection{震災初期のネットの状況}

震災初期は,ネット上において多くのボランティア活動が立ち上げられた.
また,Google や Yahoo! などのポータルサイトにおいて,情報提供が行われた.
特に,2010年1月のハイチ地震から使われた安否確認システム ``Google Person Finder'' はよく知られている\cite{GPF}.
さらに,様々なネット上の情報源から Google Person Finder への安否情報を強化するプロジェクト``ANPI NLP''\shortcite{anpi_nlp}が
自然言語処理研究者らボランティアによって立ち上げられ,
Twitter からは 33242 に及ぶツイートにタグ付けされた\cite{anpi_nlp_proc}.

救助活動を含めたシステムとしては,
``sinsai.info''\shortcite{sinsai.info}が知られている.
このサイトは,2007年のケニア大統領選挙以降,多くの人災・天災で用いられた
クラウドソーシングツール Ushahidi を用い,震災当日に構築されたシステムとして
有名である\cite{sinsai.info_TechWave}.
その他,放射線量マップや,Twitter の震災関連タグが付与されたツィートのタイムライン表示サイトなど,
様々なシステムが公開された.

しかしながら,無数に生まれたシステムと,
震災初期の全国各地のボランティアとの連携は
必ずしも迅速に出来たとは言えない.
寧ろ,ボランティア側が既存システムを利用して,
試行錯誤的に規則を決めて活動していた事例も多い.
その一例として,2ちゃんねる臨時地震板において震災当日に立てられた
スレッド「【私にも】三陸沖地震災害の情報支援【できる】」\cite{2ch}を中心として,
「東北大震災 まとめ Wiki」\cite{atwiki} や
「共同編集:被害リアルマップ東北地方太平洋沖地震(救助用マップ)」\cite{map1} が挙げられる.
これらは,以下に示す経緯で開設された:
\begin{enumerate}
\item 発災直後,2ちゃんねる臨時地震板スレッド「震度7」\cite{2ch_shindo7_1}が立ち上がる.
\item その後「震度7その2」〜「震度7その5」\cite{2ch_shindo7_2,2ch_shindo7_3,2ch_shindo7_4,2ch_shindo7_5} 
  の順序でスレッドが立ち上がる.
\item 「震度7その5」\cite{2ch_shindo7_5} 47番目の投稿ユーザ {\tt ID: nx64KwTT} が,発災2時間後にスレッド
「【私にも】三陸沖地震災害の情報支援【できる】」\cite{2ch}を立て,
  11番目の投稿で
  「東北大震災 まとめ Wiki」\cite{atwiki} 開設を表明する.
\item 翌12日,ユーザ {\tt ID:EJYqO+nC0} が立てたスレッド「被害状況まとめGoogleマップ希望スレ」 \cite{2ch_map}
  で「救助用マップ」\cite{map1} が開設されたことが,同日 \cite{2ch} 436番目の投稿で表明される.
\end{enumerate}
後にこれらのサイトの情報は,「東日本大震災(東北地方太平洋沖地震)@ウィキ」\shortcite{matome_wiki} に一元化された.



\subsection{「東日本震災支援  {\tt \#99japan}」}
\label{99japan}

2011年3月15日,田宮嘉一氏(Twitter ID: {\tt @ktamiya})は,
ブログのコメント欄を用いて通報活動履歴を管理していた\cite{ktamiya}.
同氏は3月18日,自身のTwitterのフォロワー数(10万超)が多い利点を活かして情報支援活動のメンバーを募集,
{\bf 「東北関東大震災救助支援プロジェクト {\tt \#99japan}」}が組織された(現「東日本震災支援 {\tt \#99japan}」).
募集時の活動概要は以下のとおりである\cite{99japan1}:
\begin{itemize}
\item {\bf 目的:}Twitter等での救助要請の声を適正機関に伝達する支援を行い,被災者を救う.
\item {\bf 活動内容:}主に被災者の情報の整理,内容確認,アドバイス,
  救助要請の代行.期間は物資が行き渡り,復興段階に入る頃までを予定.
\end{itemize}
このプロジェクトでは,
Google マップのデフォルト機能を用いて構築された
前述の「救助用マップ」\cite{map1} と「物資要請・提供マップ」\cite{map2}の使用,並びに編集規則が採用された.
なお,これらマップ開設者らも {\tt \#99japan} に合流していることより,
{\tt \#99japan}は「東日本大震災最初期の情報支援プロジェクト」の一つと言える.

``{\tt \#99japan}'' はプロジェクト名であると共に,
ハッシュタグでもある.このタグ付きでツィートすることで,
プロジェクトメンバーはいつでも救助要請情報や通報状況情報を共有出来る.
もちろん,メンバー以外の人でもツィートは閲覧可能である.
この{\bf 「ハッシュタグによる緩やかなユーザ間の情報共有」が,
  「Twitter ベースのボランティア活動」
  
  最大の特徴である.
}

    ``{\tt \#99japan}'' における情報共有・マップ編集・情報元確認作業の流れを以下に示す:(\texttt{@ma\_chiman} 2011a)\nocite{99japan2}
\begin{itemize}
\item {情報共有・マップ編集作業}
  \begin{enumerate}
  \item {各活動者は,本サイトなどで得た救助要請情報や関連情報に対して,
    ``{\tt \#99japan}'' を付加したツィートを発言することで,
    有益な情報を集約させ,各活動者が短時間で情報の把握ができるようにする.}
  \item {共有された情報に従い,
    各活動者はマップから通報対象地点を選び,警察など関係機関に電話・メール・ツィートする.
    通報対象地点は,(1)未通報地点,(2)通報済みだが経過不明,(3)解決済み,(4)その他,に大別される.}
  \item {通報地点のポップアップに通報内容を記入する.
    (記入例:【(日付) (要請先)に(電話)で連絡済(自分のID)】)}
  \end{enumerate}
\item {情報元確認作業}
  \begin{enumerate}
\item 
  ツィート時点から一定日数経過した現地情報の鮮度を保つために,
  情報元ツイート者を辿り,経過を訪ねる.
\item 近辺にお住まいの方とコンタクトを取り,状況を尋ね,確認中であることをマップに記入する.
\item 返信の有無や内容に応じて,次の情報をマップに反映する:
  \begin{itemize}
  \item 【解決:(現在の日付)(自分のID)】
  \item 【新しい情報得られず:(現在の日付)(自分のID)】
  \item 【未連絡:(現在の日付)(自分のID)】
  \item  その他情報.
  \end{itemize}
\end{enumerate}
\end{itemize}
我々が開発したサイトの目標は,``{\tt \#99japan}'' において,
救助要請情報を通報ボランティアが発見する機会を増やすことである.



\subsection{本サイトと {\tt \#99japan} の連携活動}

我々は,{\tt \#99japan}メンバーの要望に沿うよう本サイトを改善した結果,
本論文著者の相田(Twitter ID: {\tt @aidashin})に対して,
公式サイト開設者・{\tt @ma\_chiman}氏や「救助用マップ」管理者・{\tt @juntaro33}氏より,
以下のような評価を得た:
\begin{itemize}
\item 3月20日の活動開始直後(発足後,最初の日曜日):
  \begin{itemize}
  \item \twitter{【提案】@aidashinさんのツイートで http://is.gd/THMebZ というものを見かけました.
      対応済み・アバウトな内容なものも多いですが,対応してなさそうなものも見かけたのですがこちらを照会する作業も必要だと思われますか?
          \\\mbox{\rm (\texttt{@ma\_chiman} 2011b)\nocite{ma_chiman_0}}}
    {2011/03/20 20:39:37}{ma\_chiman}
  \item \twitter{@ma\_chiman @ktamiya @aidashin これもツイッターからの情報であれば全部拾う必要があると思います
      \mbox{\rm \cite{juntaro33_00}}}
    { 2011/03/20 20:47:22}{\mbox{juntaro33}}
  \end{itemize}
\item 1次情報のツィート日時の表示機能追加時:
  \begin{itemize}
  \item \twitter{@aidashin すごいです!お疲れ様です! \mbox{\rm \cite{juntaro33_0}}\\}{2011/3/25 18:19:14}{\mbox{juntaro33}}
  \item \twitter{@aidashin http://bit.ly/hfDcW0,確認致しました.相変わらず最新ツイートを拾うのに便利ですね.
   観測機能がついたのはこのツールにとっては大きな進展になりそうですね^^最中に開発,さすがです. {\tt \#99japan} 
       \mbox{\rm (\texttt{@ma\_chiman} 2011c)\nocite{ma_chiman_1}}\\}
   {2011/03/25 17:09:46}{ma\_chiman}
  \end{itemize}
\item 
  救助要請と安否確認の色分け機能追加時:
  \begin{itemize}
  \item \twitter{@aidashin これは見やすいです! \#99japan
        \mbox{\rm (\texttt{@ma\_chiman} 2011d) \nocite{ma_chiman_2}}\\}
  {2011/3/31 15:28:46}
  {ma\_chiman}
  \end{itemize}
\item 本サイトへの反応に驚いていたことに対して:
  \begin{itemize}
  \item \twitter{@aidashin 今,一番有用に使わせてもらってます!ありがとうございます!\mbox{\rm \cite{juntaro33_1}}}
  {2011/4/2 16:16:24}
  {\mbox{juntaro33}}
  \end{itemize}
\item 
  マップを作成された方々の見解を伺った際:
  \begin{itemize}
  \item \twitter{@aidashin いえいえ.今,マップの情報のほとんどがあいださんのシステムからの情報ですから.
      \mbox{\rm \cite{juntaro33_2}}}
    {2011/4/6 23:52:14}{\mbox{juntaro33}}
  \end{itemize}
\end{itemize}

本サイトの豊橋技術科学大学以外からのアクセス数の遷移を図\ref{log}に示す.
公開日や3月20日の{\tt \#99japan}活動開始直後は,特にアクセス数が多い.
4月初め頃に緊急状態を脱し,アクセス数は一時的に減少したが,
4月7日の最大震度6強(宮城),4月11・12日の最大震度6弱(福島)の大余震時
アクセス数が再び増加しており,相関が見られる.

\begin{figure}[b]
  \begin{center}
    \includegraphics{20-3ia5f2.eps}
  \end{center}
  \caption{「救助要請情報抽出サイト」アクセス数の遷移}
  \label{log}
\end{figure}
\begin{figure}[b]
  \begin{center}
    \includegraphics{20-3ia5f3.eps}
  \end{center}
  \caption{共同編集:物資 要請・提供マップ東北関東大震災(2012月1日30日当時)}
  \label{map}
\end{figure}


「救助用マップ」
や
「物資要請・提供マップ」
(図\ref{map})を用いた活動によって,
4月上旬までの3週間程で,救助・物資支援それぞれ200地点を超える通報・支援活動が行われた
\footnote{「救助用マップ」
(「共同編集:被害リアルマップ東北地方太平洋沖地震」)は公開が凍結されているが,
2011年3月29日 9:17までのバックアップは閲覧可能である \cite{map1_backup}.}.
なお,本活動は \cite{Utada} において評価されている.


\section{考察}

{\tt \#99japan} を振り返ると,
まず「大地震発生時に集う場所」として,「2ちゃんねる臨時地震板が存在していたこと」が重要であった.
今回,震災当日に2ちゃんねる臨時地震板ユーザらがボランティアとなって,
WikiやGoogleマップの他に,Twitter や mixi などソーシャルメディアなど
既存の情報技術を活用した情報共有がなされた.
次に,Twitterユーザらが,彼らと{\tt \#99japan}に自然合流し,
Twitterによってより多くの情報が共有できた.
とりわけ,ハッシュタグとしての``{\tt \#99japan}''は,
通報報告のみならず,進捗報告や救助完了報告についても共有出来る仕組みとして極めて意義がある.
このような進捗・完了報告の重要性は,
「東日本大震災ビッグデータワークショップ」の Twitter ブレインストーミングにおいても指摘されている
\shortcite{bigdata}.

通報活動には,「救助要請情報の鮮度」が重要であったが,
我々の「救助要請情報抽出サイト」は,{\tt \#99japan}への情報源提供元として活用され,
効率的な通報活動支援に貢献した.

大震災後の社会について考察した \cite{Endo} において,
遠藤は序章3節「われわれはいま,何を考えるべきか」の中で,以下を留意点として述べている:
\begin{enumerate}
\item 未曾有の大災害におけるミクロな「現実」の精査
\item マクロな社会システムの分析と再設計
\item 非常時における社会的コミュニケーション回路の再構築
\item 地域コミュニティにおける社会資本と情報蓄積
\item ボランティア活動の組織化とソーシャルメディア
\item 国際社会との対話—世界問題としての大震災
\end{enumerate}
{\tt \#99japan} は,(5)をいち早く実施した事例と言える.
また,実社会における(3)の一部として機能した.
\cite{Utada} は本活動を評価しながらも,(1)と(2)に関連する課題点として,「公共の組織と連携していない点」を指摘している:
\begin{quote}
  それぞれのマップ
  \footnote{
    マップは\cite{map1,map2}を指す.
  }
  の冒頭には,
  「通報が無ければ、このマップに書き込んでも救助されません!
  通報が原則です!!
  通報をしないと国は助けられません!!」
  と注意喚起している。
  しかし、国の救援組織がこのマップを採用すれば、こうした心配はなくなる。
  通報したかどうかではなくて、 
  「対応中」 とか 「救援済」 といったより具体的で確実
  な情報が反映できる。 

  通信ネットワークが十分に機能しなかった今回の教訓で、
  いずれは避難所などにも、
  衛星を使ったネット回線など災害対応の通信ネットワーク環境が整備されていくだろう。
  そうなってくれば、こうした地図を使った情報共有の仕組みはますます役に立つ。
  公の組織も利用することを考えるべきではないか。
  検索できない名簿を作っている時代ではもはやなくなっているはずなのだから。 
\end{quote}
なお,(4)は震災復興対策,(6)は震災に関する正確な情報発信をそれぞれ含意するが,
これらのための「研究者らと地域住民一体となったアウトリーチ活動」は広く行われている.
そして,このような活動の収集・記録・分析活動が,
(1)と(2)に対する復興方法論の提案になると考えられる.



\section{おわりに}

我々は,震災初期の2011年3月16日,
Twitter上の全情報から救助要請情報を一覧表示するサイトを早期開発し,
Web上に公開した.
特に,Twitterベースの東日本震災支援プロジェクト {\tt \#99japan} の活動に参加・連携することで,
本サイトで抽出した数多くの救助要請情報に基づいて適宜通報活動が行われたことが判った.
{また,{\tt \#99japan} のメンバーの要望に従い,救助要請情報の抽出精度向上と機能改善に努めることが出来た.}
本活動に限らず,
震災初期支援活動に「ソーシャルメディア」が活用されていることも
報告されている\cite{NHK,Shimbun,Tachiiri}.

今後の課題として,現在も稼働し取得し続けている本サイトログ
からニーズを分析し,
適応的な震災復興支援システム構築が挙げられる.
また,今回は {\tt \#99japan} は偶然幾つかの重要な要素が繋がったが,
\cite{Endo,Utada}で指摘されている様に,{\tt \#99japan} のような活動を一過性のものとせずに,
次の災害時,より迅速的かつ効率的な救助・復興活動になるように,
「災害ボランティアに関する社会システムの枠組み」を今のうちに洗練化しておくべきであろう.



\section*{この度の東日本大震災で被災された皆様へ}

この度の東日本大震災によって亡くなられた方々へ謹んで哀悼の意を表しますと共に,
被災された皆様,ご家族,並びに関係者の皆様へお見舞い申し上げます.
そして,被災地の一日も早い復興を心よりお祈り申し上げます.


\acknowledgment

まず,本サイト開発を勧めて頂いた {\tt @nkanada} 様に,
御礼申し上げます.
次に,本研究成果は,
{\tt \#99japan} を立ち上げた田宮嘉一様,
マップ開設者の {\tt @juntaro33}様,
{\tt \#99japan} のサイト管理者の {\tt @ma\_chiman}様,
並びに多くの {\tt \#99japan} ボランティアの皆様
のお力添え賜りましたことへ深謝致します.
また,本サイト開発・公開に際して,有益な助言賜りました
``ANPI NLP''\shortcite{anpi_nlp} ボランティア研究者の方々に,
感謝致します.
最後に,震災初期に出会った ``311HELP.com''\cite{311help} 開発者の
株式会社42・田原大生様が,
2011年7月31日, 
{\tt \#99japan} について公の場で初めて発表の機会を与えて下さいましたことへ,
改めて感謝致します.
(本論文の内容の一部は,言語処理学会第18回年次大会で発表した
ものである\cite{Aida1}.)


\bibliographystyle{jnlpbbl_1.5}
\begin{thebibliography}{}

\bibitem[\protect\BCAY{{\tt \#99japan}有志ら}{{\tt
  \#99japan}有志ら}{2011a}]{map1}
{\tt \#99japan}有志ら \BBOP 2011a\BBCP.
\newblock 共同編集:被害リアルマップ東北地方太平洋沖地震.\
\linebreak
\newblock \url{https://maps.google.co.jp/maps/ms?ie=UTF8&hl=ja&brcurrent=3,0x5f8892ddfbe0dc71:0xce6fb9385107a4ad,0&msa=0&msid=209051486000599298555.00049e33f3610df1bed86&z=9}.

\bibitem[\protect\BCAY{{\tt \#99japan}有志ら}{{\tt
  \#99japan}有志ら}{2011b}]{map2}
{\tt \#99japan}有志ら \BBOP 2011b\BBCP.
\newblock 共同編集:物資 要請・提供マップ東北関東大震災.\
\newblock \url{https://maps.google.co.jp/maps/ms?ie=UTF8&hl=ja&brcurrent=3,0x34674e0fd77f192f:0xf54275d47c665244,0&oe=UTF8&num=200&msa=0&msid=212756209350684899471.00049ea27cf60c4292136&ll=37.827141,140.306396&spn=2.290855,3.488159&z=8}.

\bibitem[\protect\BCAY{{\tt \#99japan}有志ら}{{\tt
  \#99japan}有志ら}{2011c}]{map1_backup}
{\tt \#99japan}有志ら \BBOP 2011c\BBCP.
\newblock 共同編集:被害リアルマップ東北地方太平洋沖地震(3/29
  9:17までのバックアップ分).\
\newblock
  \url{https://maps.google.co.jp/maps/ms?ie=UTF8&hl=ja&brcurrent=3,0x34674e0fd
77f192f:0xf54275d47c665244,0&msa=0&ll=38.255436,140.998535&spn=10.259815,16.54
541&z=6&msid=212756209350684899471.00049f93fb04a48b1dce9}.

\bibitem[\protect\BCAY{{\tt @juntaro33}}{{\tt
  @juntaro33}}{2011a}]{juntaro33_00}
{\tt @juntaro33} \BBOP 2011a\BBCP\
\newblock \url{https://twitter.com/#!/juntaro33/status/49436866939863040}.

\bibitem[\protect\BCAY{{\tt @juntaro33}}{{\tt @juntaro33}}{2011b}]{juntaro33_0}
{\tt @juntaro33} \BBOP 2011b\BBCP\
\newblock \url{https://twitter.com/#!/juntaro33/status/51211528241819648}.

\bibitem[\protect\BCAY{{\tt @juntaro33}}{{\tt @juntaro33}}{2011c}]{juntaro33_1}
{\tt @juntaro33} \BBOP 2011c\BBCP\
\newblock \url{https://twitter.com/#!/juntaro33/status/54079719091605504}.

\bibitem[\protect\BCAY{{\tt @juntaro33}}{{\tt @juntaro33}}{2011d}]{juntaro33_2}
{\tt @juntaro33} \BBOP 2011d\BBCP\
\newblock \url{https://twitter.com/#!/juntaro33/status/55643984638394368}.

\bibitem[\protect\BCAY{{\tt @ma\_chiman}}{{\tt @ma\_chiman}}{2011a}]{99japan2}
{\tt @ma\_chiman} \BBOP 2011a\BBCP.
\newblock 東日本震災支援 \#99japan(救急ジャパン)公式サイト.\
\newblock \url{https://sites.google.com/site/sharp99japan/}.

\bibitem[\protect\BCAY{{\tt @ma\_chiman}}{{\tt
  @ma\_chiman}}{2011b}]{ma_chiman_0}
{\tt @ma\_chiman} \BBOP 2011b\BBCP\
\newblock \url{https://twitter.com/#!/ma_chiman/status/49434917083414528}.

\bibitem[\protect\BCAY{{\tt @ma\_chiman}}{{\tt
  @ma\_chiman}}{2011c}]{ma_chiman_1}
{\tt @ma\_chiman} \BBOP 2011c\BBCP\
\newblock \url{https://twitter.com/#!/ma_chiman/status/51194045803921408}.

\bibitem[\protect\BCAY{{\tt @ma\_chiman}}{{\tt
  @ma\_chiman}}{2011d}]{ma_chiman_2}
{\tt @ma\_chiman} \BBOP 2011d\BBCP\
\newblock \url{https://twitter.com/#!/ma_chiman/status/53342954399600640}.

\bibitem[\protect\BCAY{{\tt ID:nx64KwTT}}{{\tt ID:nx64KwTT}}{2011}]{atwiki}
{\tt ID:nx64KwTT} \BBOP 2011\BBCP.
\newblock 東北大震災 まとめ Wiki.\
\newblock \url{http://www45.atwiki.jp/acuser001}.

\bibitem[\protect\BCAY{相田}{相田}{2011}]{Aida0}
相田慎 \BBOP 2011\BBCP.
\newblock Twitter
  からどのようにして救助要請情報を抽出したのか?—「東日本震災支援{\tt
  \#99japan}」活動を通して—.\
\newblock \Jem{LAシンポジウム会誌, 第57号}, \mbox{\BPGS\ 9--24}.

\bibitem[\protect\BCAY{相田\JBA 新堂\JBA 内山}{相田 \Jetal }{2011}]{extraction}
相田慎\JBA 新堂安孝\JBA 内山将夫 \BBOP 2011\BBCP.
\newblock 救援要請ツィート抽出サイト.\
\newblock \url{http://www.selab.cs.tut.ac.jp/~aida/}.

\bibitem[\protect\BCAY{相田\JBA 新堂\JBA 内山}{相田 \Jetal }{2012}]{Aida1}
相田慎\JBA 新堂安孝\JBA 内山将夫 \BBOP 2012\BBCP.
\newblock 「東日本大震災関連の救助要請情報抽出サイト」構築と救助活動について.\
\newblock \Jem{言語処理学会第18回年次大会講演論文集}, {\Bbf 13}  (1),
  \mbox{\BPGS\ 1236--1239}.

\bibitem[\protect\BCAY{荒蝦夷}{荒蝦夷}{2012}]{IBC}
荒蝦夷\JED\ \BBOP 2012\BBCP.
\newblock \Jem{その時,ラジオだけが聴こえていた—IBC岩手放送 3・11震災の記録}.
\newblock 竹書房.

\bibitem[\protect\BCAY{遠藤}{遠藤}{2012}]{Endo2}
遠藤薫 \BBOP 2012\BBCP.
\newblock
  \Jem{メディアは大震災・原発事故をどう語ったか—報道・ネット・ドキュメンタリ
ーを検証する}.
\newblock 東京電機大学出版局.

\bibitem[\protect\BCAY{遠藤\JBA 高原\JBA 西田\JBA 新\JBA 関谷}{遠藤 \Jetal
  }{2011}]{Endo}
遠藤薫\JBA 高原基彰\JBA 西田亮介\JBA 新雅史\JBA 関谷直也 \BBOP 2011\BBCP.
\newblock \Jem{大震災後の社会学}.
\newblock 講談社.

\bibitem[\protect\BCAY{林\JBA 山路}{林\JBA 山路}{2012}]{GPF}
林信行\JBA 山路達也 \BBOP 2012\BBCP.
\newblock
  東日本大震災と情報、インターネット、Google「パーソンファインダー、東日本大震
災での進化(1)」.\
\newblock \url{http://www.google.org/crisisresponse/kiroku311/chapter_06.html}.

\bibitem[\protect\BCAY{本田}{本田}{2012}]{sinsai.info_TechWave}
本田正浩 \BBOP 2012\BBCP.
\newblock オープンソースマインドが支えた sinsai.info
  の1年—関治之氏インタビュー.\
\newblock \url{http://techwave.jp/archives/51731021.html}.

\bibitem[\protect\BCAY{片瀬\JBA ラジオ福島}{片瀬\JBA ラジオ福島}{2012}]{rfc}
片瀬京子\JBA ラジオ福島 \BBOP 2012\BBCP.
\newblock \Jem{ラジオ福島の300日}.
\newblock 毎日新聞社.

\bibitem[\protect\BCAY{久世}{久世}{2010}]{1500}
久世宏明 \BBOP 2010\BBCP.
\newblock 1500ったー.\
\newblock \url{http://xtter.openlaszlo-ason.com/XTTER/1500ttr/}.

\bibitem[\protect\BCAY{村上\JBA 他\JBA 他}{村上 \Jetal }{2011}]{anpi_nlp}
    村上浩司 他 \BBOP 2011\BBCP.
\newblock \BBOQ ANPI NLP.\BBCQ\
\newblock \url{http://trans-aid.jp/ANPI_NLP/}.

\bibitem[\protect\BCAY{村上\JBA 萩原}{村上\JBA 萩原}{2012}]{anpi_nlp_proc}
村上浩司\JBA 萩原正人 \BBOP 2012\BBCP.
\newblock 安否情報ツイートコーパスの詳細分析とアノテーションに関する一考察.
言語処理学会第18回年次大会発表論文集, 
\newblock {\Bbf 13}  (1), \mbox{\BPGS\ 1232--1235}.

\bibitem[\protect\BCAY{{\tt nextutozin}\JBA 他\JBA 他}{{\tt nextutozin} \Jetal
  }{2011}]{matome_wiki}
    {\tt nextutozin} 他 \BBOP 2011\BBCP.
\newblock 東日本大震災(東北地方太平洋沖地震)@ウィキ.\
\newblock \url{http://www46.atwiki.jp/earthquakematome/}.

\bibitem[\protect\BCAY{NHK総合テレビ}{NHK総合テレビ}{2011}]{NHK}
NHK総合テレビ \BBOP 2011\BBCP.
\newblock
  クローズアップ現代「いま,私たちにできること〜“ソーシャルメディア”支援〜」
.\
\newblock \url{http://cgi4.nhk.or.jp/gendai/kiroku/detail.cgi?content_id=3022}.

\bibitem[\protect\BCAY{2ちゃんねる}{2ちゃんねる}{2011a}]{2ch_shindo7_1}
2ちゃんねる臨時地震板 \BBOP 2011a\BBCP.
\newblock 震度7.\
\newblock \url{http://www.logsoku.com/r/eq/1299822821/}.

\bibitem[\protect\BCAY{2ちゃんねる}{2ちゃんねる}{2011b}]{2ch_shindo7_2}
2ちゃんねる臨時地震板 \BBOP 2011b\BBCP.
\newblock 震度7その2.\
\newblock \url{http://www.logsoku.com/r/eq/1299823601/}.

\bibitem[\protect\BCAY{2ちゃんねる}{2ちゃんねる}{2011c}]{2ch_shindo7_3}
2ちゃんねる臨時地震板 \BBOP 2011c\BBCP.
\newblock 震度7その3.\
\newblock \url{http://www.logsoku.com/r/eq/1299825048/}.

\bibitem[\protect\BCAY{2ちゃんねる}{2ちゃんねる}{2011d}]{2ch_shindo7_4}
2ちゃんねる臨時地震板 \BBOP 2011d\BBCP.
\newblock 震度7その4.\
\newblock \url{http://www.logsoku.com/r/eq/1299826741/}.

\bibitem[\protect\BCAY{2ちゃんねる}{2ちゃんねる}{2011e}]{2ch_shindo7_5}
2ちゃんねる臨時地震板 \BBOP 2011e\BBCP.
\newblock 震度7その5.\
\newblock \url{http://www.logsoku.com/r/eq/1299830742/}.

\bibitem[\protect\BCAY{2ちゃんねる}{2ちゃんねる}{2011f}]{2ch}
2ちゃんねる臨時地震板 \BBOP 2011f\BBCP.
\newblock 【私にも】三陸沖地震災害の情報支援【できる】.\
\newblock \url{http://logsoku.com/thread/hayabusa.2ch.net/eq/1299829654/}.

\bibitem[\protect\BCAY{2ちゃんねる}{2ちゃんねる}{2011g}]{2ch_map}
2ちゃんねる臨時地震板 \BBOP 2011g\BBCP.
\newblock 被害状況まとめGoogleマップ希望スレ.\
\newblock \url{http://logsoku.com/thread/hayabusa.2ch.net/eq/1299892327/}.

\bibitem[\protect\BCAY{荻上}{荻上}{2011}]{Ogiue}
荻上チキ \BBOP 2011\BBCP.
\newblock \Jem{検証 東日本大震災の流言・デマ}.
\newblock 光文社.

\bibitem[\protect\BCAY{関\JBA 他\JBA 他}{関 \Jetal }{2011}]{sinsai.info}
    関治之 他 \BBOP 2011\BBCP.
\newblock \BBOQ sinsai.info.\BBCQ\
\newblock \url{http://www.sinsai.info/}.

\bibitem[\protect\BCAY{福田}{福田}{2012}]{Fukuda}
福田充 \BBOP 2012\BBCP.
\newblock \Jem{大震災とメディア}.
\newblock 北樹出版.

\bibitem[\protect\BCAY{新聞通信調査会}{新聞通信調査会}{2013}]{Shimbun}
新聞通信調査会 \BBOP 2013\BBCP.
\newblock \Jem{大震災・原発とメディアの役割—報道・論調の検証と展望—}.
\newblock 公益財団法人 新聞通信調査会.

\bibitem[\protect\BCAY{田原}{田原}{2011}]{311help}
田原大生 \BBOP 2011\BBCP.
\newblock 必要物資・支援要求マップ 311HELP.com.\
\newblock \url{http://311help.com/}.

\bibitem[\protect\BCAY{田宮}{田宮}{2011a}]{ktamiya}
田宮嘉一 \BBOP 2011a\BBCP.
\newblock 東北関東大震災,救助依頼連絡先 (アメーバブログ).\
\newblock \url{http://ameblo.jp/ktamiya/entry-10829792004.html}.

\bibitem[\protect\BCAY{田宮}{田宮}{2011b}]{99japan1}
田宮嘉一 \BBOP 2011b\BBCP.
\newblock 東北関東大震災の救助支援プロジェクト「{\tt
  \#99japan}」参加者を募集します.\
\newblock \url{http://twipla.jp/events/6133}.

\bibitem[\protect\BCAY{立入}{立入}{2011}]{Tachiiri}
立入勝義 \BBOP 2011\BBCP.
\newblock \Jem{検証 東日本大震災 そのときソーシャルメディアは何を伝えたか?}
\newblock ディスカヴァー・トゥエンティワン.

\bibitem[\protect\BCAY{Twitter社}{Twitter社}{2011a}]{Twitter_tags}
Twitter社 \BBOP 2011a\BBCP.
\newblock 東北地方太平洋沖地震に関して.\
\newblock \url{http://blog.jp.twitter.com/2011/03/blog-post_12.html}.

\bibitem[\protect\BCAY{Twitter社}{Twitter社}{2011b}]{nihongo}
Twitter社 \BBOP 2011b\BBCP.
\newblock {\tt \#}日本語ハッシュタグ.\
\newblock \url{http://blog.jp.twitter.com/2011/07/blog-post.html}.

\bibitem[\protect\BCAY{Twitter社}{Twitter社}{2013}]{Twitter_help}
Twitter社 \BBOP 2013\BBCP.
\newblock Twitter ヘルプセンター.\
\newblock \url{https://support.twitter.com/}.

\bibitem[\protect\BCAY{歌田}{歌田}{2011}]{Utada}
歌田明弘 \BBOP 2011\BBCP.
\newblock 仮想報道 Vol.~673「地図を使った震災情報の共有」.\
\newblock \Jem{週刊アスキー4月12日号}, \mbox{\BPGS\ 94--95}.

\bibitem[\protect\BCAY{山崎\JBA 他\JBA 他}{山崎 \Jetal }{2012}]{bigdata}
    山崎富美 他 \BBOP 2012\BBCP.
\newblock {\tt \#shinsaidata}
  「東日本大震災ビッグデータワークショップ---Project 311 ---」ブレスト.\
\newblock \url{http://togetter.com/li/372103}.

\end{thebibliography}


\begin{biography}

\bioauthor{相田  慎}{
  1975年生.
  2002年名古屋大学大学院人間情報学研究科 物質・生命情報学博士後期課程修了.
  博士(学術).
  2002年豊橋技術科学大学工学部 知識情報工学系助手.
  2007年同助教.
  2007年同大大学院工学研究科 情報・知能工学系助教.
  計算量理論・アルゴリズム理論の研究に従事.
}
\bioauthor{新堂 安孝}{
  1974年生.
  2000年大阪市立大学 大学院 理学研究科 博士前期課程修了.
  2004年大阪市立大学 博士(理学)取得.
  PC周辺機器メーカー,機械メーカー,通信キャリア,ソフトウェア・メーカーに
  勤務.
  音声言語処理・自然言語処理の研究などに従事.
}
\bioauthor{内山 将夫}{
  1969年生.
  1997年筑波大学大学院工学研究科電子情報工学専攻修了.
  博士(工学).
  1997年信州大学工学部電気電子工学科助手.
  1999年郵政省通信総合研究所非常勤職員.
  2001年独立行政法人通信総合研究所任期付き研究員.
  2004年独立行政法人情報通信研究機構主任研究員.
  自然言語処理の研究に従事.
}

\end{biography}


\biodate

\end{document}
