\documentstyle[epsbox,jnlpbbl]{jnlp_j_b5}


\setcounter{page}{91}
\setcounter{巻数}{7}
\setcounter{号数}{2}
\setcounter{年}{2000}
\setcounter{月}{4}
\受付{1999}{10}{25}
\再受付{1999}{12}{6}
\採録{2000}{1}{7}

\setcounter{secnumdepth}{2}
\setlength{\parindent}{\jspaceskip}

\title{統計的手法による換喩の解釈}
\author{内山 将夫\affiref{U} \and 村田 真樹\affiref{U} \and 馬 青\affiref{U} \and 内元 清貴\affiref{U} \and 井佐原 均\affiref{U}}


\headauthor{内山,村田,馬,内元,井佐原}
\headtitle{統計的手法による換喩の解釈}

\affilabel{U}{郵政省通信総合研究所}
{Communications Research Laboratory, Ministry of Posts and Telecommunications}

\jabstract{本稿では,換喩を統計的に解釈する方法を述べた.換喩と
  は,喩える言葉(喩詞)と喩えられる言葉(被喩詞)との連想に基づい
  た比喩である.たとえば,「漱石を読む」という換喩は,「漱石の
  小説を読む」というように解釈できる.この場合,喩詞である「漱
  石」と被喩詞である「(漱石の)小説」との間には,「作者-作品」と
  いう連想関係が成立する.本稿では,以下の方針で換喩を解釈する
  ことを試みた.
  \begin{enumerate}
  \item 「名詞$A$,格助詞$R$,述語$V$」というタイプの換喩が与え
    られたとき,与えられた喩詞$A$から連想される名詞群を求めるた
    めにコーパスを利用する.
  \item 連想された名詞群のなかから,与えられた視点($R$,$V$)に適
    合するような名詞を被喩詞として統計的に選択する.
  \end{enumerate}
  その結果,コーパスが連想名詞の供給源として有効なことが例証さ
  れ,かつ,提案手法を用いることにより,喩詞から連想された名詞
  群の中から,換喩の視点に適合する名詞を被喩詞として選択できる
  ことが分かった.また,提案手法による換喩解析の正解率は,厳し
  い評価を適用した場合には0.47であり,緩い評価を適用した場合に
  は0.65であった.これらは提案手法が換喩の解析に有効であること
  を示している.}

\jkeywords{比喩,換喩,コーパス,連想,視点,統計}

\etitle{Statistical Approach to \\ the Interpretation of Metonymy}
\eauthor{Masao UTIYAMA\affiref{U} \and Masaki Murata\affiref{U} \and Qing Ma\affiref{U} \and Kiyotaka Uchimoto\affiref{U} \and Hitoshi Isahara\affiref{U}}

\eabstract{
This paper describes a statistical approach to the
interpretation of metonymy.  In metonymy, the name of one
thing ({\it the source}) is substituted for that of another
related to it ({\it the target}).  For example, in {\it
Souseki wo yomu} (read a Souseki), the source `a Souseki' is
substituted for the target `a novel written by Souseki.'  In
this case, they are related by an {\it Artist for Artform}
relation.  The method in this paper follows the procedure
below in interpreting a metonymy.
\begin{enumerate}
 \item Given a metonymy `noun $A$, case-marker $R$, predicate $V$,' nouns related to the source $A$ are collected in a corpus.
 \item From the collected nouns, a candidate for the target that satisfies the constraints imposed by $R$ and $V$ is selected by applying a statistical criterion.
\end{enumerate}
The method was tested experimentally. It was shown that a
corpus is valuable for extracting nouns that are related to
a given source and it was also shown that the proposed
statistical criterion can select a good target from the
extracted nouns. The precision of the experiment, when based
on a rigorous judgment, was 0.47 and when based on a less
rigorous judgment it was 0.65. The effectiveness of the
proposed method has thus been demonstrated.}

\ekeywords{figure of speech, metonymy, corpus, association, view, statistical approach}

\begin{document}
\thispagestyle{plain}
\maketitle

\section{はじめに}
\label{sec:intro}

本稿では,比喩の一種である換喩を統計的に解釈する方法を述べる.

比喩は大別すると,直喩・隠喩的なものと換喩的なものとに分けられ
る\cite{Ye90}.まず,直喩・隠喩的な比喩とは,喩えるもの(喩詞)と喩えら
れるもの(被喩詞)との類似性に基づいた比喩である.たとえば,「あ
の男は狼のようだ」という直喩,あるいは,「あの男は狼だ」という
隠喩は,喩詞である「狼」と被喩詞である「あの男」との間の何らか
の類似性(獰猛さなど)に基づいている.ここで,直喩と隠喩との違い
は,直喩が比喩であることを言語的に明示するのに対して,隠喩はそ
のようなことを明示しない点にある.一方,換喩的な比喩とは,喩詞
と被喩詞との連想関係に基づいた比喩である.たとえば,「漱石を読
む」という換喩は,「漱石の小説を読む」というように解釈できる.
この場合,喩詞である「漱石」と被喩詞である「(漱石の)小説」との
間には,「作者-作品」という連想関係が成立する\cite{yamanashi88}.

比喩の処理は,検出と解釈の2段階に分けて考えることができる.まず,
比喩の検出とは,与えられた言語表現が比喩であるかどうかを判定す
る処理である.次に,比喩の解釈とは,与えられた言語表現が比喩で
あるとして,その比喩の非字義的な表現から字義的な表現を求める処
理である.たとえば,比喩の検出の段階では,「漱石を読む」が比喩
であり,「小説を読む」が比喩でないことを区別する.また,比喩の
解釈の段階では,既に比喩であることが分かっている「漱石を読む」
という非字義的な表現から,「漱石の小説を読む」という字義的な表
現を導出する.

本稿では,直喩・隠喩的なものと換喩的なものとに大別できる比喩の
うちで,換喩を対象とする.また,換喩の検出と解釈のうちでは,換
喩の解釈を対象とする.なお,本稿の対象をこのようにした理由は,
まず,第1に,直喩や隠喩や換喩などは,上述のように,一応区別でき
るものであるので,それらを別々のものとして,そのうちの一つを研
究対象とすることは可能であるからである.次に,換喩の解釈を対象
とする理由は,換喩の解釈は換喩のみを考慮すれば実現可能なのに対
して,換喩の検出は直喩や隠喩なども考慮しなければ実現不可能なた
めである.すなわち,換喩を検出するには,まず,比喩を検出し,そ
の後でその比喩が換喩かどうかを検出しなければならないので,換喩
検出を直喩や隠喩と別々に研究することは困難であるのに対して,換
喩の解釈の場合には,既に換喩が与えられたものとすれば,他の比喩
のことは考慮せずに独立に研究できるためである.

本稿では,換喩のなかでも,「$名詞A$,$格助詞R$,$述語V$」という
タイプの換喩を対象とする.そして,以下の方針に基づいて,換喩を
解釈する.
\begin{enumerate}
 \item 「$A$,$R$,$V$」というタイプの換喩が与えられたとき,与えられた喩詞$A$から連想される名詞群を求めるためにコーパスを利用する(\ref{sec:corpus}章).
 \item 連想された名詞群のなかから,与えられた視点($R$,$V$)に適合するような名詞を被喩詞として統計的に選択する(\ref{sec:measure}章).
\end{enumerate}

たとえば,「一升瓶を飲む」という換喩が与えられたとすると,喩詞
である「一升瓶」から連想される名詞として「酒,栓,...」をコー
パスから求め,その中から「を飲む」という視点に適合する「酒」を
被喩詞として選択する.一方,「一升瓶を開ける」という換喩に対し
ては,「一升瓶」から連想される名詞群は同じであるが,被喩詞とし
ては「栓」を選択する.

上述の(1)と(2)は本稿の手法を特徴付けるものである.そして,これ
らは\cite{yamamoto98}の方法を発展させたものと考えることができる.

まず,(1)については,これまでの換喩の研究としては,連想される
(名詞とは限らない)単語群を求めるために,意味ネットワークや規則
などを利用したものがある
\cite{iverson92:_metal,bouaud96:_proces_meton,fass88:_meton_metap}
が,そのような知識は人手で構築するのが困難であるという欠点があ
る.それに対して,コーパスを利用すれば,意味ネットワークのよう
な知識を人手で構築する必要はない.そのため,コーパスを利用すれ
ば,相当多くの換喩を解析できる可能性がある.すなわち,コーパス
に基づく手法の方が,意味ネットワークなどに基づく手法よりも,広
い範囲の換喩を解析できる可能性が高い.

なお,\cite{yamamoto98}は,名詞$A$から連想される名詞の候補とし
て,「名詞$A$の名詞$B$」における$B$と,「名詞$A$ 名詞$B$」にお
ける$B$を用いていたが,本稿では,(i)「名詞$A$の名詞$B$」におけ
る$B$と,(ii)名詞$A$と同一文中に出現した名詞$B$とを連想される名
詞の候補に用いる\footnote{(i)における名詞の候補は(ii)における候
  補に包含されるが,\ref{sec:measure}章で述べる統計的尺度の計算
  において別扱いを受ける.}.(ii)を用いることにより,
\cite{yamamoto98}の方法ではカバーできない名詞を連想の候補として
利用できることが期待できる.

次に,(2)については,換喩の解釈を絞り込むための情報源として換喩
の視点($R$,$V$)を利用していると考えられる.このような絞り込みは,
従来の研究では,意味ネットワークや規則により実現されてきたが,
本稿では,コーパスにおける統計情報を利用して実現する.

なお,\cite{yamamoto98}は,換喩の解釈を絞り込むために,与えられ
た述語の格フレーム($R$,$V$)に適合する名詞のうちで喩詞$A$との共
起頻度が最大のものを被喩詞として選ぶという方法を用いている.し
かし,全ての述語について格フレームが利用できるとは限らないので,
本稿では格フレームを利用せず,統計的手法に基づいて被喩詞を選択
する手法を提案する.なお,格フレームが利用できる場合には,その
格フレームに適合する名詞のみを候補として,本稿で提案する手法を
適応すれば良いので,本稿で提案する手法と共に格フレームを利用す
ることは容易である\footnote{\cite{yamamoto98}では,本稿と同様に,
  換喩の解釈のみを対象にしているが,入力される換喩としては,
  「名詞$A_1$,格助詞$R_1$,名詞$A_2$,格助詞$R_2$,$\ldots$,名詞
  $A_n$,格助詞$R_n$,述語$V$」を想定している.そして,その入力に
  含まれる名詞のなかで述語$V$の格の選択制限に合致しないものを喩
  詞と特定し,その喩詞の被喩詞を求めている.たとえば,「私が漱
  石を読む」という換喩の場合には,「漱石」が「読む」の選択制限
  を満たさないことを特定し,「漱石」の被喩詞として「小説」を求
  めている.一方,本稿では,喩詞が特定済みの入力を想定している.
  つまり,入力としては,「漱石を読む」のようなものを想定してい
  る.この点では,\cite{yamamoto98}の方法の方が優れている.この
  ような喩詞の特定は今後の課題である.ただし,本稿の方法に加え
  て,格フレームを利用できれば,\cite{yamamoto98}と同様の方法を
  使うことにより,喩詞を特定できる.}.

以下,\ref{sec:sort}章では,換喩の種類と本稿の対象とする換喩に
ついて述べ,\ref{sec:corpus}章では,喩詞に関連する名詞群をコー
パスから求めるときに使う共起関係について述べ,\ref{sec:measure}
章では,被喩詞らしさの統計的尺度について述べる.そして,
\ref{sec:experiments}章において,提案尺度の有効性を実験により調
べ,\ref{sec:discussion}章で,その結果を考察する.
\ref{sec:conclusion}章は結論である.なお,付録の表\ref{tab:1}か
ら表\ref{tab:5}には,提案尺度に基づいて換喩を解釈した結果がある.

\section{換喩の種類と対象とする換喩}
\label{sec:sort}

換喩の種類を区別するものとして,構文的区別,意味的区別,文脈的
区別を考える.そして,それぞれの場合において,本稿で対象として
いる換喩の特徴を述べる.

\subsection{構文的区別}
\label{sec:syntax}

構文的な区別とは,換喩の表現形式による区別のことである.

まず,1項だけで,たとえば,名詞だけで,換喩となる場合が考えられ
る.この例\footnote{本稿における換喩の例は\cite{yamanashi88}ま
  たは\cite{yamamoto98}からの引用もしくは著者らの作例である.}
としては,「コニャック」という地名が「コニャック産のブランデー」
という産物を表すというものがある.ただし,この例のように,名詞1
語だけで換喩という場合は,語源を考えれば換喩ということであり,
現在の用法としては,換喩というよりは,語義の多義性であると言え
るであろう.

次に,2項関係により換喩の解釈が決まるものがある.これは,たとえ
ば,「名詞$A$,格助詞$R$,述語$V$」という形\footnote{格助詞は項
  数に含めない.}で,かつ,$A$が換喩かどうかが$R$と$V$のみに依
存するものである.そのような例としては「漱石を読む」がある.こ
の例は,「漱石の小説を読む」と解釈できるが,そのように解釈でき
る理由は,「漱石」は人であり,「人を読む」ことはできないので,
その他の解釈として,蓋然性が高い,「漱石の小説を読む」が選択さ
れると考えられる.

3項関係としては,「名詞$A_1$,格助詞$R_1$,名詞$A_2$,格助詞
$R_2$,述語$V$」のようなものが考えられる.この形の換喩の場合に
は,$A_1$が換喩かどうかは,$R_1$と$V$だけでなく,$A_2$と$R_2$に
も依存する.このような例としては,「(観光ブームで)祇園に笑顔が
戻った」のようなものがある.この例では「$A_1=祇園$,$R_1=に$,
$A_2=笑顔$,$R_2=が$,$V=戻った$」であり,その解釈は,「祇園の
人々に笑顔が戻った」というものである.この場合には,$A_1$の「祇
園」が喩詞であり,その被喩詞は「(祇園の)人々」である.ここで,
この喩詞と被喩詞との関係は,$A_2$の「笑顔」がなければ成立しない
ものである.これは,たとえば,「祇園に太郎が戻った」という文で
は,「祇園という場所に太郎が戻った」という字義通りの解釈となる
ことから分かる.

項数の他の構文的区別としては,換喩において,喩詞がどのような表
現形式をしているかがある.上記では,喩詞の表現としては名詞しか
考えていなかったが,その他に,「唇を噛むこと」が「くやしさ」を
喩えていたり,「バンドの穴が一つ縮むこと」が「やせたこと」を喩
えている場合がある.このように,換喩において,喩詞の表現形式は
名詞とは限らない.

上記のように,構文的な区別としては様々なものが考えられるが,本
稿で対象とするものは,「$A$,$R$,$V$」という単純なものである.

\subsection{意味的区別}

意味的な区別とは,喩詞と被喩詞との意味的な関係による換喩の区別
のことである.

このような関係としては,表\ref{tab:metonymies}に示すように,
「作者-作品」「主体-手段」「容器-中味」「主体-付属物」のような
ものがある\cite{yamanashi88}.

\begin{table}[htbp]
  \begin{center}
    
    \caption{換喩における意味的な関係の例}
    
    \begin{tabular}{|c|c|c|}
      \hline
      意味的関係 & 換喩の例 & 解釈 \\ \hline
      作者-作品 & 漱石を読む & 漱石の小説を読む\\
      主体-手段 & クラリネットを笑う & クラリネットの奏者を笑う \\
      容器-中味 & 鍋を食べる & 鍋の料理を食べる\\
      主体-付属物 & 詰め襟が歩く & 詰め襟の学生が歩く\\
      主体-近接物 & ハムサンドが勘定を払う & ハムサンドの客が勘定を払う\\\hline
    \end{tabular}
    \label{tab:metonymies}
  \end{center}
\end{table}

本稿での換喩の解釈においては,このような意味的な区別はせずに,
単に,被喩詞としての名詞を同定することを目標とする.たとえば,
「一升瓶を飲む」については,「一升瓶の酒を飲む」のように,「一
升瓶」の被喩詞として「酒」を取り出すことを目標とするが,「一升
瓶」と「酒」の間にある「容器と中味」という意味的な関係について
推定することは解釈には含めない.その理由は,本稿では,換喩の被
喩詞となる名詞をコーパス中から求めるのだが,そのコーパスには,
「一升瓶の酒」というような事例はあっても,その事例における「一
升瓶」と「酒」との間の意味関係については何の情報もないからであ
る.そのため,そのような意味関係を推定するのは今後の課題とする.
なお,そのような意味関係の推定のためには,
\cite{kurohashi99:_seman_analy_japan_noun_phras}の方法が利用で
きるであろう.

\subsection{文脈的区別}
\label{sec:context}

ここで述べる文脈的区別とは,ある換喩が発話されたとき,その換喩
の解釈を一意に決めるために必要な文脈の程度による区別のことであ
る.

まず,解釈を一意に決めるために,特に文脈を必要としない換喩があ
る.これは,その換喩の解釈に一般的な知識のみを必要とするような
ものである.そのような例としては「漱石を読む」がある.この換喩
を解釈するために必要な知識は,「漱石の小説」というものが存在す
るという知識のみであり,かつ,そのような知識は常識的な知識であ
るので,この換喩を一意に解釈するためには,特に文脈は必要ない.

一方,一意に解釈するためには,それが発話された状況までも知らな
いとならない換喩がある.たとえば,「黒がいい」という換喩の場合,
洋服売場においては,「黒のスラックスがいい」という意味かもしれ
ないし,囲碁の対局においては,「黒石を持つ人の情勢がいい」とい
う意味かもしれない.

このように換喩の解釈に必要な文脈の程度は様々であるが,本稿が対
象とする換喩は,一意に解釈するために文脈が不要な換喩である.

\section{喩詞と関連する名詞群を求めるときに使う共起関係}
\label{sec:corpus}

コーパスにおける単語間の共起頻度を利用することにより,互いに意
味的に関連がある単語対を抽出できる
\cite[など]{church89:_word_assoc_norms_mutual_infor_lexic,brown92:_class_based_model_natur_languag}.このこと
に基づき,喩詞から連想される名詞群を求めるためにコーパスを利用
する.

本稿では,特に,「の」による共起関係と「同一文内」における共起
関係とに基づいて喩詞と関連する名詞群を求める.ただし,「の」に
よる共起関係とは,係り受け関係にある名詞$A$と$B$とが「$A$の$B$」
の形で出現した場合を言う.さらに,$A$と「の」による共起関係にあ
る名詞としては,$A$に「の」を介して後接する名詞のみを利用し,前
接する名詞は利用しない.たとえば,「$A$の$B$」と「$X$の$A$」と
があった場合には,$A$と「の」による共起関係にある名詞は$B$のみ
である.一方,「同一文内」における共起関係では,「の」による共
起関係のような非対称性はない.たとえば,上例と同様に,$A$と$B$
とが同一文内にあり,$X$と$A$とが同一文内にある場合には,$A$の共
起名詞としては,$A$との相対的な出現順序に関わらず$B$と$X$とが抽
出される\footnote{「の」による共起関係が非対称であるのに対して,
  「同一文内」における共起関係が対称である理由は以下の通りであ
  る.まず,「の」による共起関係が非対称である理由は,喩詞と被
  喩詞との間にある連想の非対称性と,「の」による共起関係におけ
  る出現頻度の非対称性とがよく一致すると考えたからである.たと
  えば,「漱石を読む」という換喩では,喩詞「漱石」と被喩詞「小
  説」との間には,「漱石」から「小説」を連想するのであって,
  「小説」から「漱石」を連想するのではないという非対称性がある.
  そして,「の」による共起関係についても,出現頻度に関して,こ
  のような非対称性がある.たとえば,\ref{sec:experiments}章の実
  験に用いたコーパスでは,「漱石の小説」の頻度は14であるが,
  「小説の漱石」の頻度は0である.このことから,喩詞「漱石」から
  連想される名詞群を求めるときには,「漱石の$B$」における$B$を
  利用すれば良いと考えられる.我々は,このようなことが一般にも
  成立すると予想し,喩詞から連想される名詞群を精度良く求めるた
  めに,「の」による共起関係を非対称なものとした.一方,「同一
  文内」における共起関係については,そのような連想の非対称性は
  顕著ではないと考えた.更に,以下で述べるように,「同一文内」
  における共起関係は,共起名詞のカバー率を上げることを目的とし
  て利用しているため,なるべく多くの共起名詞を拾うためには対称
  の方が都合が良いため,対称的な共起関係を利用した.}.

ここで,被喩詞としての妥当性という観点から,それぞれの共起関係
により求められる名詞群を比べると,一般的に言って,「の」による
共起関係による名詞群の方が優れている.なぜなら,「の」による共
起関係にある名詞は,「同一文内」における共起関係にある名詞と違
い,二つの名詞間に係り受け関係という構文的な関係が成立している.
つまり,構文的な関係という観点からは,「の」による共起関係にあ
る名詞の方が,「同一文内」における共起関係にある名詞よりも明らか
に強い.そのため,意味的な関係についても「の」による共起関係に
ある名詞の方が強いと考えることができるからである.たとえば,表
\ref{tab:metonymies}では,様々な意味的な関係が「の」による共起
関係として表現されている.これは,「の」による共起関係が,意味
的な関係が強い名詞と起こることを例示している.なお,「の」によ
る共起関係の有用性は,\cite{murata97}でも確認されている.

一方,「同一文内」における共起関係にある名詞群は,「の」による
共起関係にある名詞群よりも数が多い.実際,「の」による共起関係
にある名詞群は,「同一文内」における共起関係にある名詞群の部分
集合である.そのため,「の」による共起関係でカバーできないよう
な連想についても「同一文内」における共起名詞によりカバーできる
可能性がある.

このように,「の」による共起関係と「同一文内」における共起関係
とは,互いに,意味的な関連の強さとカバー率において性質が異なる
関係であると言える.そのため,これらを区別して喩詞と共起関係に
ある名詞群を求める.

たとえば,付録の表\ref{tab:1}から表\ref{tab:5}には,
\ref{sec:experiments}章の実験で使われた換喩について,その喩詞と
共起関係にある名詞群を載せてある.ここで,それらの表において,
「AのB」とあるのは,喩詞と「の」による共起関係にある名詞群であ
り,「A近B」とあるのは,喩詞と「同一文内」における共起関係にあ
る名詞群である.なお,それらの名詞群は,喩詞との共起頻度が多い
ものから最大10個が並べられている.ただし,共起名詞の数が10個以
下のものについては,全ての共起名詞が共起頻度の降順に記載されて
いる.なお,各名詞の右下の添字は,その名詞と喩詞との,その共起
関係における共起頻度である.

これらの表のうちで,表\ref{tab:2}にある,10番の換喩「顔を剃る」
を見ると,喩詞「顔」と,「の」による共起関係にある名詞群として
は,「表情,部分,前,輪郭,しわ,大きさ,筋肉,色,アップ,形」
があるが,「同一文内」による共起関係にある名詞群としては,「日
本,人,前,女性,首相,目,自分,男,手,東京」がある.これら
を比べたとき,「の」による共起関係にある名詞群の方が,「顔」と
の意味的な関係が強い(明瞭である)と言える.このことは,その他の
換喩の例についても言えると考える.一方,12番の換喩「アデランス
が歩く」を見ると,「同一文内」における共起名詞として「かつら,
アートネイチャー,東京,新宿,髪,男性,女性,アルシンド,メー
カー,大手」があるが,「の」による共起関係にある名詞は存在しな
い\footnote {この理由の一つは,\ref{sec:material}節でも述べるが,
  「の」による共起関係が成立するかどうかの判定を厳しくしている
  ことにもある.}.このことは,「同一文内」における共起関係が,
連想される名詞のカバー率を上げるために有用であることを示してい
る.これらのことからも分かるように,二つの共起関係は性質が異な
るものである.そのため,共起名詞を得るときには,二つの関係を区
別しなければならないと言える.

\section{被喩詞らしさの統計的尺度}
\label{sec:measure}

本章では,与えられた名詞群から,換喩の被喩詞として適当な名詞を
選択するための統計的尺度について述べる.

与えられた換喩が「名詞$A$,格助詞$R$,述語$V$」のとき,$A$と関係
$Q$にある被喩詞を$B$とすると,「$A$,$R$,$V$」は,
「$A$,$Q$,$B$,$R$,$V$」の省略形であると考えることができる
\cite{yamamoto98}.たとえば,「漱石を読む」という換喩は,「漱石
の小説を読む」という字義的な表現の省略形であると考えることがで
きる.ただし,「$A=漱石$」「$Q=の$」「$B=小説$」「$R=を$」「$V=
読む$」である.これらの関係を図\ref{fig:dep}に示す.
\begin{figure}[htbp]
  \begin{center}
    \epsfile{file=depAQBRV.eps}
    \caption{「$A$,$Q$,$B$,$R$,$V$」における依存構造}
    \label{fig:dep}
  \end{center}
\end{figure}

このことを統計的に表現するために,換喩「$A$,$R$,$V$」に対して,
関係$Q$を用いたときの被喩詞$B$の尤もらしさを表す尺度を
\begin{equation}
  \label{eq:L}
  L_Q(B|A,R,V) \doteq \Pr(B|A,Q,R,V)
\end{equation}
により定義する.そして,$B$の被喩詞としての尤もらしさを表す尺度
を
\begin{equation}
  \label{eq:M}
  M(B|A,R,V) \doteq \max_Q L_Q(B|A,R,V)
\end{equation}
と定義する.尺度$M$が大きい名詞ほど,被喩詞として尤もらしい.こ
こで,$\Pr(\cdots)$は,そのような事象が生じる確率である.また,
$Q$は,本稿の場合には,「の」による共起関係か,「同一文内」での
共起関係であり,$B$は,$A$と関係$Q$にあるような名詞の一つである.

(\ref{eq:L})式を計算するために,以下のような変形をする.

\begin{eqnarray}
  \label{eq:L2}
  \lefteqn{L_Q(B|A,R,V)}\nonumber\\
  & = & \Pr(B|A,Q,R,V) \nonumber\\
  & = & \frac{\Pr(A,Q,B,R,V)}{\Pr(A,Q,R,V)} \nonumber\\
  & = & \frac{\Pr(A,Q,B) \Pr(R,V|A,Q,B)}{\Pr(A,Q) \Pr(R,V|A,Q)} \nonumber\\
  & \simeq & \frac{\Pr(A,Q,B) \Pr(R,V|B)}{\Pr(A,Q)\Pr(R,V)} \nonumber\\
  & = & \frac{\Pr(A,Q,B)}{\Pr(A,Q)}\frac{\Pr(B,R,V)}{\Pr(B)}\frac{1}{\Pr(R,V)}.
\end{eqnarray}
上式において,4行目から5行目への近似は,形態素間の依存構造を根
拠としている.すなわち,「$A$,$Q$,$B$,$R$,$V$」には,図
\ref{fig:dep}のような依存構造を考えることができることから,
$A,Q$と$R,V$との関係は密接でないと考えられる.そのため,$A,Q$と
$R,V$とは確率的に独立と近似し,4行目から5行目において,
$\Pr(R,V|A,Q,B) \simeq \Pr(R,V|B)$および$\Pr(R,V|A,Q) \simeq
\Pr(R,V)$とした.

(\ref{eq:L2})式の要素式を以下のように定義する\footnote{本節で後
  述する頻度の計数法に従うと,(\ref{eq:aqb})式と(\ref{eq:aq2})
  式については,$\Pr(A,Q) = \sum_B \Pr(A,Q,B)$である.しかし,
  (\ref{eq:b})式と(\ref{eq:brv})式については,$\Pr(B) \neq
  \sum_{R,V}\Pr(B,R,V)$である.このように,(\ref{eq:b})式と
  (\ref{eq:brv})式には確率の観点からは不整合がある.この不整合
  性を失くすためには,$\Pr(B)$を$\sum_{R,V} \Pr(B,R,V)$と定義す
  れば良いのだが,それを計算するのはコストが高いため,
  (\ref{eq:b})式のように定義した.}.
\begin{equation}
  \label{eq:aqb}
  \Pr(A,Q,B) \doteq \frac{f(A,Q,B)}{N_0}.
\end{equation}
\begin{equation}
  \label{eq:aq2}
  \Pr(A,Q) \doteq \frac{f(A,Q)}{N_1}.
\end{equation}
\begin{equation}
  \label{eq:b}
  \Pr(B) \doteq \frac{f(B)}{N_2}.
\end{equation}
\begin{equation}
  \label{eq:brv}
  \Pr(B,R,V) \doteq \left\{
    \begin{array}{ll}
      \frac{f(B,R,V)}{N_3} & \mbox{if } f(B,R,V) > 0,\\
      \frac{\sum_{C \in Classes(B)} \Pr(B|C) f(C,R,V)}{N_3} & \mbox{otherwise}.
    \end{array}
    \right.
\end{equation}
\begin{equation}
  \label{eq:bc}
  \Pr(B|C) \doteq \frac{f(B)/|Classes(B)|}{f(C)}.
\end{equation}
ただし,$f(\cdots)$は当該事象の頻度であり,$N_i(0\le i \le 3)$
は確率の総和が1となるように定める正規化定数であり,$Classes(B)$は
$B$が所属する(たとえばシソーラスにおける)意味的なクラスの集合で
ある.なお,$N_i$は定数であり,さらに,(\ref{eq:L2})式の
$\Pr(R,V)$についても,換喩「$A$,$R$,$V$」が与えられているという
条件下では定数である.よって,これらの値は,$L_Q(B|A,R,V)$の大
小の比較には影響を与えない.そのため,\ref{sec:experiments}章の
実験では,簡単のため,$N_i=1$,$\Pr(R,V)=1$として(\ref{eq:L2})式
を計算している.

(\ref{eq:aqb})式と(\ref{eq:brv})式とを比べると,(\ref{eq:aqb})
式では,$f(A,Q,B)$のみを利用しているため,$A$と関係$Q$にないよ
うな名詞$B$は,$f(A,Q,B)=0$であるので,無視される.一方,
(\ref{eq:brv})式では,$f(B,R,V)=0$であるような名詞$B$についても,
$B$の所属するクラスを利用することにより,0より大きい確率を付与
する.このようにした理由は,まず,(\ref{eq:aqb})式については,
$A$と$B$とは,喩詞と被喩詞という,ある程度強い連想関係にあるは
ずなので,共起頻度が0であるような名詞は無視しても良いと考えたか
らである.それに比べて,(\ref{eq:brv})式では,$B$と$V$とは必ず
しも強い連想関係にあるとは限らないので,$f(B,R,V)=0$であっても,
$B$が被喩詞として不適当とは限らない可能性がある.そのため,
$f(B,R,V)=0$のようなものについても,$B$の意味的なクラスを利用す
ることにより,確率を付与することにした.

(\ref{eq:bc})式は,意味的クラス$C$が名詞$B$として出現する確率で
ある.これは,もし,名詞$B$に多義性がなく,一つのクラス$C$にし
か所属しないとすれば,名詞$B$の頻度$f(B)$を,クラス$C$の頻度
$f(C)$で割れば求めることができるが,名詞$B$が多義で複数のクラス
に所属する場合には,名詞$B$はそれらのクラスに等確率で所属するも
のとして,$f(B)$を$|Classes(B)|$で割った頻度を$f(C)$で割ったもの
を$\Pr(B|C)$とする.


なお,各事象の頻度は以下のようにして求めた.まず,
(\ref{eq:aqb})式における$f(A,Q,B)$の計数法は,詳しくは
\ref{sec:material}節で述べるが,基本的には,コーパス中における,
そのような事象を直接計数する.たとえば,Qが「の」による共起関係
の場合は,たとえば,「漱石,の,小説」のような共起を計数するし,
「同一文内」による共起関係の場合は,たとえば,「漱石」と「小説」
が同一文内で共起した回数を数える.一方,(\ref{eq:aq2})式におけ
る$f(A,Q)$の場合は,$f(A,Q) \doteq \sum_B f(A,Q,B)$と定義する.
つまり,$f(A,Q)$は,$A$と関係$Q$により共起した名詞$B$について,
$f(A,Q,B)$を足し合わせた値である.更に,(\ref{eq:b})式の$f(B)$
の場合には,単に,コーパス中における$B$の出現頻度を数える.また,
(\ref{eq:brv})式の$f(B,R,V)$については,詳しくは
\ref{sec:material}節で述べるが,基本的には,コーパス中における
「$B$, $R$ ,$V$」という形態素列の個数を数える.以上をまとめると,
$f(A,Q,B)$, $f(B)$, $f(B,R,V)$は,コーパス中における事象を計数
し,$f(A,Q)$については,$f(A,Q)=\sum_B f(A,Q,B)$と定義する,と
いうことである.

また,各クラスの頻度$f(C)$は,そのクラスに属する名詞$B$の頻度に
より,次のようにして求めた.
\begin{displaymath}
  f(C) = \sum_{B \in C} \frac{f(B)}{|Classes(B)|}.
\end{displaymath}
ここで,$f(B)$を$|Classes(B)|$で割っているのは,(\ref{eq:bc})式の
説明で述べたように,$B$が複数の意味的クラスに属する場合に,
$f(B)$を各クラスに等分割するためである.また,$f(C,R,V)$も同様
にして求めた.すなわち,
\begin{displaymath}
  f(C,R,V) = \sum_{B \in C} \frac{f(B,R,V)}{|Classes(B)|}.
\end{displaymath}

以上,(\ref{eq:aqb})式から(\ref{eq:bc})式までを利用することによ
り,(\ref{eq:L2})式を計算し,(\ref{eq:M})式により,名詞の被喩詞
らしさを求める.

これらの式の適用例として,たとえば,「森鴎外を読む」という換喩
における被喩詞の候補として,「小説」と「戯曲」があるとする.こ
こで,「$A=森鴎外$」「$R=を$」「$V=読む$」である.また,$Q$は,
「の」による共起関係か,「同一文内」での共起関係であり,$B$は
「小説」か「戯曲」である.このとき,それぞれの頻度を,
$f(A,Q,B)$などで分類して表示すると以下の通りである.
\paragraph{$f(A,Q,B)$}
\begin{eqnarray}
  f(森鴎外,の,小説) & = & 4, \nonumber\\
  f(森鴎外,の,戯曲) & = & 2, \nonumber\\
  f(森鴎外,同一文内,小説) & = & 10, \nonumber\\
  f(森鴎外,同一文内,戯曲) & = & 2, \nonumber
\end{eqnarray}
\paragraph{$f(A,Q)$}
\begin{eqnarray}
  f(森鴎外,の) & = & 18, \nonumber\\
  f(森鴎外,同一文内) & = & 1311,\nonumber
\end{eqnarray}
\paragraph{$f(B,R,V)$}
\begin{eqnarray}
  f(小説,を,読む) & = & 88, \nonumber\\
  f(戯曲,を,読む) & = & 4, \nonumber
\end{eqnarray}
\paragraph{$f(B)$}
\begin{eqnarray}
  f(小説) & = & 7545,\nonumber\\
  f(戯曲) & = & 781. \nonumber
\end{eqnarray}
ただし,これらの頻度は,\ref{sec:experiments}章の実験で用いら
れたコーパスにおいて,\ref{sec:material}節の方法により頻度を集
計した結果である.

次に,$L$の大小比較に関係のない$N_0,N_1,N_2,N_3,\Pr(R,V)$を1と
して,$L$を計算すると,(\ref{eq:L2})式は,$f(B,R,V)>0$のときに
は,
\begin{equation}
  L_Q(B|A,R,V) = \frac{f(A,Q,B)}{f(A,Q)}\frac{f(B,R,V)}{f(B)}.
\end{equation}
のように簡単化できるので,
\begin{eqnarray}
  \label{eq:Ls}
  L_{の}(小説|森鴎外,を,読む) & = & \frac{f(森鴎外,の,小説)}{f(森鴎外,の)}\frac{f(小説,を,読む)}{f(小説)} \nonumber\\
  & = & \frac{4}{18}\frac{88}{7545}= 2.59 \times 10^{-3},\nonumber\\
  L_{同一文内}(小説|森鴎外,を,読む) & = & \frac{f(森鴎外,同一文内,小説)}{f(森鴎外,同一文内)}\frac{f(小説,を,読む)}{f(小説)} \nonumber\\
  & = & \frac{10}{1311}\frac{88}{7545}= 8.90 \times 10^{-5},\nonumber\\
  L_{の}(戯曲|森鴎外,を,読む) & = & \frac{f(森鴎外,の,戯曲)}{f(森鴎外,の)}\frac{f(戯曲,を,読む)}{f(戯曲)} \nonumber\\
  & = & \frac{2}{18}\frac{4}{781}= 5.69 \times 10^{-4},\nonumber\\
  L_{同一文内}(戯曲|森鴎外,を,読む) & = & \frac{f(森鴎外,同一文内,戯曲)}{f(森鴎外,同一文内)}\frac{f(戯曲,を,読む)}{f(戯曲)} \nonumber\\
  & = & \frac{2}{1311}\frac{4}{781}= 7.81 \times 10^{-6}.\nonumber
\end{eqnarray}
これより,
\begin{eqnarray}
  \label{eq:Ms}
  \lefteqn{M(小説|森鴎外,を,読む)}\nonumber\\
  & = & \max\{L_{の}(小説|森鴎外,を,読む),L_{同一文内}(小説|森鴎外,を,読む)\}\nonumber\\
  & = & 2.59 \times 10^{-3},\nonumber\\
  \lefteqn{M(戯曲|森鴎外,を,読む)}\nonumber\\
  & = &\max\{L_{の}(戯曲|森鴎外,を,読む),L_{同一文内}(戯曲|森鴎外,を,読む)\} \nonumber\\
  & = & 5.69 \times 10^{-4}\nonumber
\end{eqnarray}
となり,「森鴎外を読む」の被喩詞としては,「小説」の方が「戯曲」
よりも適当となる.この結果は妥当であると考える.なぜなら,森鴎
外は,確かに戯曲も書いているが,それよりも,「舞姫」「青年」
「高瀬舟」などの小説で有名であるからである.

\section{実験}
\label{sec:experiments}

実験材料,実験方法,実験結果について順に述べる.

\subsection{実験材料}
\label{sec:material}

\paragraph{換喩}

実験に用いた換喩は,\cite{yamanashi88}において例文として採用さ
れているものを,「名詞$A$,格助詞$R$,述語$V$」の形に適合するよう
に変形した33例\footnote{実験に用いたのは,\ref{sec:syntax}節で
  述べたように,2項関係により換喩の解釈が決まる例文のうちで,喩
  詞が名詞であるようなものである.}に,著者らによる作例1例
\footnote{著者らの作例は,換喩解釈における視点の役割を例示する
  ために加えた.}を加えた34例である\footnote{\cite{yamamoto98}
  は35例で実験をしているが,それらにおける異なり述語数は18であ
  るので,実質上は似たタイプの換喩における実験であると言える.
  それに対して本稿の実験では,述語の異なり数が31であるので,取
  扱う換喩のタイプ数としては,本稿の実験の方が多いと言える.ま
  た,\cite{yamamoto98}は\cite{yamanashi88}から換喩の例文を選ん
  でいるのだが,\cite{yamanashi88}にある全ての換喩を選んでいる
  わけではない.それに対して,本稿では,\cite{yamanashi88}にお
  いて,完全な文として成立するような例40例の中から,
  \ref{sec:syntax}節の規準に合致する32例(32/40=0.80.ここで32例
  のうち1例は二つの換喩を含んでいたので,それを2例に分け,最終
  的に33例の換喩を得た)を選んだものであるので,本稿の実験は,換
  喩のタイプを良く網羅するものであると考える.}.なお,これらの
例では,格助詞としては,「が」と「を」しか出現していないが,そ
の他の格助詞についても提案手法は適用できるし,もっと一般に,任
意の2項関係についても提案手法は適用できる.

\paragraph{コーパス}
 
単語間の共起頻度を計数するためのコーパスとしては,「CD-毎日新聞」
の91年度版から97年度版の7年間分を用いた.このコーパスを自動処理
により一文単位に分けたあとで,茶筌version
2.0b6\cite{matsumoto97}により形態素解析し,共起頻度を計数した.

ここで,(\ref{eq:aqb})式における$\Pr(A,Q,B)$を計算するときに必
要な,名詞間の共起頻度を求めるとき,「同一文内」における共起関
係については,特別な処理をすることなく,単に頻度を計数した.し
かし,「の」による共起関係については,名詞間に係り受け関係が高
確率で成立するような共起関係を同定することを目的として,「非名
詞,名詞$A$,の,名詞$B$,非名詞」という単語列からのみ,$A$と
$B$の共起頻度を計数した.なお,ここでの名詞には,茶筌にとっての
未知語を含む.

また,(\ref{eq:brv})式における$\Pr(B,R,V)$を計算するときに必要
な,「名詞,格助詞,述語」の共起頻度を求めるときには,格助詞が
「を」の場合には,単に,「名詞,を,述語」という単語列の生起頻
度をもって共起頻度としたが,格助詞が「が」の場合には,「が」だ
けでなく,「は」「も」「の」を間に狭んだ名詞と述語の連続,すな
わち,「名詞,は,述語」や「名詞,も,述語」や「名詞,の,述語」
についても,「が」による共起関係の頻度とした.たとえば,「僕が
行く」「僕は行く」「僕も行く」という例が,それぞれ1例ずつあった
場合には,「僕,が,行く」の頻度を3にした.このようにした理由は,
「を」については,「を」だけで共起頻度が充分に利用できるが,
「が」については,「が」だけでは充分な共起頻度が得られないため
である.なお,構文解析の結果に基づいた係り受け情報を利用すれば,
「が」についても充分な共起頻度が得られる可能性はあるが,今回は
構文解析をしなかったため,上記のような措置により,共起頻度情報
を得た.

\paragraph{名詞の意味的クラス}

(\ref{eq:brv})式を計算するためには,各名詞の意味的なクラスが必
要である.そのクラスとしては,分類語彙表増補版\cite{bgh96}にお
ける上位3桁の分類番号を用いた.これによるクラスの種類は,全体で
は,90種であり,名詞に限れば,43種である.なお,本稿では,名詞
のクラスのみを用いた.

分類語彙表増補版には約85,000語が記載されているが,これに記載さ
れていない名詞については,もし,その名詞の細分類が,茶筌におけ
る,「人名」「組織」「地域」「数」のいずれかである場合には,以
下の分類番号を割当てた.
\begin{quote}

\vspace{1em}

\begin{tabular}{|l|l|l|}
  \hline
  細分類 & 分類番号 & 分類語彙表にない名詞例\\ \hline
  人名 & 1.20, 1.21, 1.22, 1.23, 1.24 & 佐藤,田中,鈴木\\
  組織 & 1.26, 1.27, 1.28 & 自民党,筑波大,日本相撲協会\\
  地域 & 1.25 & 大阪,神戸,兵庫\\
  数   & 1.19 & 一一〇,一万八千,七十一\\ \hline
\end{tabular}

\vspace{1em}

\end{quote}
ただし,これらの細分類に該当する名詞については,それに関連する
全ての分類番号を割当てた.たとえば,「佐藤」という人名の場合に
は,「1.20, 1.21, 1.22, 1.23, 1.24」の全てを分類番号として割当
てた.つまり,「佐藤」は複数の意味的クラスに所属する多義的な名
詞ということである.

また,品詞細分類が上記四つ以外の名詞については,それが分類語彙
表で未登録名詞の場合には,その名詞自体を一つのクラスとした.
\vspace{-3mm}
\subsection{実験方法と実験結果}
\label{sec:results}

各換喩について,喩詞と,「の」または「同一文内」での共起関係に
ある名詞群を得て,それぞれの名詞について,(\ref{eq:M})式の尺度
$M$を計算した.

その結果を付録の表\ref{tab:1}から表\ref{tab:5}に示す.これらの
中で,表\ref{tab:1}と表\ref{tab:2}は正解例であり,表\ref{tab:3},
表\ref{tab:4},表\ref{tab:5}は不正解例である.また,解析結果の
一部を表\ref{tab:ex}に示す.

\begin{table}[htbp]
  \footnotesize
  \begin{center}
    
    \caption{解析結果の一部}
    
    \begin{tabular}{|l|l|}
            \hline
1.& 一升瓶 を 飲む\\
被喩詞 & $○酒^{近}_{10,1}(2.6\times{}10^{-3})$ $ビール^{近}_{6,6}(5.0\times{}10^{-4})$ $日本酒^{近}_{7,4}(3.9\times{}10^{-4})$ $ぶどう酒^{近}_{1,79}$\\
 & $ワイン^{近}_{2,26}$ $牛乳^{近}_{2,26}$ $ジュース^{近}_{2,26}$ $ウイスキー^{近}_{2,26}$ $地酒^{近}_{1,79}$ $薬^{近}_{1,79}$\\
AのB & $栓_{1}$ $ラベル_{1}$\\
A近B & $酒_{10}$ $ビール瓶_{10}$ $瓶_{10}$ $日本酒_{7}$ $空_{7}$ $ビール_{6}$ $手_{6}$ $足_{4}$ $量_{4}$ $程度_{4}$\\
\hline
21.& 押し入れ を かきまわす $\rightarrow$ 押し入れの中味をかきまわす\\
被喩詞 & $△奥^{の,*}_{17,1}(2.5\times{}10^{-7})$ $天袋^{の,*}_{5,2}(2.3\times{}10^{-7})$ $天井^{の,*}_{2,4}(9.2\times{}10^{-8})$ $戸^{の,*}_{2,4}$\\
 & $上段^{の,*}_{2,4}$ $中段^{の,*}_{2,4}$ $金庫^{の,*}_{1,11}$ $相手^{近}_{1,393}$ $和室^{近,*}_{17,5}$ $前^{の,*}_{3,3}$\\
AのB & $奥_{17}$ $天袋_{5}$ $前_{3}$ $天井_{2}$ $戸_{2}$ $中段_{2}$ $上段_{2}$ $布団_{2}$ $整理_{2}$ $中身_{2}$\\
A近B & $自宅_{25}$ $奥_{23}$ $調べ_{22}$ $遺体_{19}$ $和室_{17}$ $容疑_{17}$ $部屋_{16}$ $疑い_{16}$ $布団_{14}$ $県_{13}$\\
\hline
27.& ベルイマン を 見る\\
被喩詞 & $×息子^{の}_{1,1}(8.1\times{}10^{-4})$ $心情^{の}_{1,1}(4.9\times{}10^{-4})$ $○映画^{近}_{10,2}(4.7\times{}10^{-4})$ $父^{の}_{1,1}$\\
 & $脚本^{の}_{1,1}$ $作品^{近}_{6,4}$ $夢^{近}_{1,26}$ $秘蔵っ子^{の,*}_{1,1}$ $シーン^{近}_{1,26}$ $風景^{近}_{1,26}$\\
AのB & $息子_{1}$ $心情_{1}$ $父_{1}$ $脚本_{1}$ $秘蔵っ子_{1}$\\
A近B & $監督_{19}$ $映画_{10}$ $イングマール_{8}$ $作品_{6}$ $スウェーデン_{5}$ $脚本_{4}$ $息子_{3}$ $野_{3}$ $アメリカ_{3}$ $時代_{3}$\\
\hline
    \end{tabular}
    \label{tab:ex}
  \end{center}
\end{table}

表\ref{tab:ex}に示すように,各解析結果の第1行には,解析対象の換
喩が番号と共に記述されている.また,解析対象の換喩の横には
「$\rightarrow$」で示された換喩の解釈がある.ただし,この解釈は,
第2行で示される「被喩詞」のなかに,\cite{yamanashi88}で想定され
ている正解がない場合にのみ記述されている.次に,「被喩詞」の行
には,喩詞とのコーパス中での共起名詞が,(\ref{eq:M})式の尺度$M$
について,その降順に上位から最大10個並んでいる.ここで,第1番目
の名詞については,○/△/×が付与されているが,これらの意味は,
○は当該の名詞が\cite{yamanashi88}で想定されている正解に合致す
る例であり,△は\cite{yamanashi88}では想定されていないが意味的
には成立可能な例\footnote{△を付けられた例の許容度は様々である.
  特に,表\ref{tab:4}の26番,表\ref{tab:5}の33,34番の例は×に近
  いとも言える.}であり,×は完全な間違い例である.さらに,第1
番目の名詞が△か×の場合には,それ以降に最初に現れた名詞で被喩
詞として適当なものに○か△を付けている.なお,○/△/×の判断は
著者らによる.また,各名詞は,
\begin{displaymath}
  名詞^{共起関係,*}_{頻度,順位}(尺度Mの値)
\end{displaymath}
のように表現されている.ただし,$*$や尺度$M$の値は特別の場合(後
述)にしか表示しない.ここで,まず,「共起関係」とは,
(\ref{eq:M})式の尺度$M$を計算した際に選ばれた共起関係であり,そ
れが「の」の場合には「の」による共起関係であることを示し,「近」
の場合には「同一文内」における共起関係であることを示す.また,
「頻度」とは,その共起関係における喩詞との共起頻度であり,「順
位」とは,その共起関係における,その共起頻度の名詞の順位である.
なお,同頻度の名詞は同順位である.たとえば,「$名詞_1$,$名詞_
2$,$名詞_ 3$,$名詞_ 4$,$名詞_5$,$名詞_6$」の頻度が,それぞれ,
「7,5,5,3,3,1」であるとすると,それぞれの順位は,「1,2,2,4,4,6」
である.また,名詞の右肩に$*$が付いているものは,その名詞と換喩
の述語との共起頻度が0であったため,(\ref{eq:brv})式において意味
的クラスが利用されたことを示す.たとえば,21番の「押し入れをか
きまわす」では「奥をかきまわす」などの頻度が0であったことを示す.
さらに,上位3位までの名詞と○または△が付いた名詞については,
(\ref{eq:M})式の尺度$M$の値を括弧内に示す.最後に,「AのB」の行
にある名詞群は,喩詞と「の」による共起関係にある名詞群であり,
「A近B」の行にある名詞群は,喩詞と「同一文内」における共起関係
にある名詞群である.なお,これらの名詞群は,共起頻度の降順に上
位から最大10個並んでいる.また,各名詞の右下の添字は,その名詞
のその共起関係における共起頻度である.ただし,共起頻度が同じ名
詞群については,それらを尺度$M$の値により降順にソートして表示し
ている.

\subsubsection{実験結果の解釈}

\paragraph{全体的な精度}

(\ref{eq:M})式の尺度$M$の全体的な精度を調べるために,表
\ref{tab:1}から表\ref{tab:5}までの被喩詞の第1候補について,○△
×を数えた.その結果を表\ref{tab:total}に示す.この表によると,
○のみを正解とする厳しい評価では,正解率は$16/34 \simeq 0.47$で
あり,○と△を正解とする緩い評価では,正解率は$(16+6)/34 \simeq
0.65$である.これらの数値は,扱う対象が換喩という従来あまり解析
の対象とされていない現象であることを考えると,高い値であると言
えると考える.すなわち,提案手法は,換喩の解析に対して有効な手
法であると考える.ただし,これらの値は少数例についてのものであ
るので,定量的に確定的なことを言うためには,更に大規模な実験が
必要である.このことは本節で以下で述べることにも同様に言える.
なお,実験結果の定性的な解釈については,\ref{sec:discussion}章
で考察する.

\begin{table}[htbp]
  \begin{center}
    
    \caption{提案尺度の全体的な精度}
    
    \begin{tabular}{|c|c|c|c|}
      \hline
      ○ & △ & × & 計\\
      \hline
      16 & 6 & 12 & 34\\
      \hline
    \end{tabular}
    \label{tab:total}
  \end{center}
\end{table}

\begin{table}[htbp]
  \begin{center}
    
    \caption{共起関係毎の精度}
    
    \begin{tabular}{|c|c|c|c|c|}
      \hline
      &○ & △ & × & 計\\
      \hline
      「の」による共起関係 & 14 & 8 & 12 & 34\\
      「同一文内」における共起関係 & 12 & 8 & 14 & 34\\
      \hline
    \end{tabular}
    \label{tab:cmp}
  \end{center}
\end{table}

\paragraph{二つの共起関係を使った効果}

二つの共起関係を使った効果を見るために,「の」による共起関係に
ある名詞のみを被喩詞の候補とした場合と「同一文内」での共起関係
にある名詞のみを被喩詞の候補とした場合とについて,それぞれの共
起名詞について尺度$M$を計算し,被喩詞の第1候補を得た.そのとき
の○△×の数を表\ref{tab:cmp}に示す.

まず,表\ref{tab:total}と表\ref{tab:cmp}とを比べると,二つの共
起関係を利用した表\ref{tab:total}の結果の方が,それぞれ一つだけ
の共起関係を利用した表\ref{tab:cmp}の結果のどちらよりも優れてい
ることが分かる.これから,二つの共起関係を同時に用いることが有
効であると言える\footnote{\cite{yamamoto98}と同様に「名詞$A$の
  名詞$B$」における$B$と「名詞$A$ 名詞$B$」における$B$とを名詞
  $A$の被喩詞の候補に使った場合には,全体の精度は,○=12,△=10,
  ×=12,であり,「名詞$A$の名詞$B$」における$B$のみを被喩詞の候
  補とした場合の精度は,表\ref{tab:cmp}における「の」による共起
  関係の場合と同じであり,「名詞$A$ 名詞$B$」における$B$のみを
  被喩詞の候補とした場合の精度は,○=3,△=13,×=18,であった.こ
  のことから,「名詞$A$ 名詞$B$」における$B$を被喩詞の候補とす
  ることは,有効ではないと言える.}.

次に,表\ref{tab:cmp}における二つの結果を比べると,「の」による
共起関係の精度が若干良いことが分かる.一方,「同一文内」におけ
る共起関係のカバー率が高いことは,たとえば,表\ref{tab:1}の1番
の換喩「一升瓶を飲む」における「酒」や表\ref{tab:2}の12番の換喩
「アデランスが歩く」における「男性」など,「の」による共起関係
に出現していないような名詞でも,「同一文内」における共起関係を
利用することにより被喩詞として選択できることからわかる.これら
のことは,\ref{sec:corpus}章で述べたことを例証している.
\vspace{-3mm}
\paragraph{意味的クラスの有効性}

(\ref{eq:brv})式では,被喩詞と述語との共起頻度のスパース性を考
慮して,意味的クラスを導入した.そのことは,ある程度は有効であっ
たと考える.その理由は以下の通りである.

まず,表\ref{tab:1}から表\ref{tab:5}までを見ると,尺度$M$の計算
において意味的クラスが使用された名詞については,$*$が右肩に付さ
れている.これから,被喩詞の第1候補で,意味的クラスが使用された
ものは,5例あり(13,18,19,21,30番の換喩),そのうち,○が1例,△
が2例,×が2例である.一方,もし,意味的クラスを使用しない場合
\footnote{$\Pr(B,R,V)$の計算に(\ref{eq:brv})式の
  $\frac{f(B,R,V)}{N_3}$のみを利用し,$f(B,R,V)=0$のときには,
  $\Pr(B,R,V)=0$とした場合.}には,この5例について,△が3例,×
が2例となる.

次に,各換喩ごとに,1位から10位までの候補について,○か△の数の
大小を,意味的クラスを使用した場合と使用しない場合とで比べる.
たとえば,7番の「平安神宮が満開」の場合には,1位から10位までを
順に示すと,意味的クラスを使用した場合には「○桜,△ハナショウ
ブ,△花,池,柳,枝,幕,○さくら,△植物,花びら」が候補であ
り,意味的クラスを使用しない場合には「○桜,△ハナショウブ,△
花,幕,今年,春,今,人,日,一つ」が候補である.したがって,
意味的クラスを使用した場合の方が○か△の数が多い.

このように,各換喩ごとに○か△の数を比較すると,意味的クラスを
使用した場合の方が○か△の数が多い換喩は12例であり,意味的クラ
スを使用しない場合の方が○か△の数が多い換喩は5例である.この結
果から,片側検定により符合検定をすると,有意水準2.5\%で,意味的
クラスを使用した場合の方が,○か△の数が多い換喩が多いと言える.
これより,意味的クラスの使用は有効であると言える.

\section{考察}
\label{sec:discussion}

\subsection{連想名詞の供給源としてのコーパスの有用性}

コーパスが連想名詞の供給源として有効なことを第\ref{sec:intro}章
で述べた.そのことは,付録の表\ref{tab:1}から表\ref{tab:5}まで
に例証されていると考える.なぜなら,本実験においては,様々な名
詞を喩詞として利用したが,それらの大半において妥当な名詞が連想
されていることが表\ref{tab:1}から表\ref{tab:5}を見れば分かるか
らである\footnote{このことを数値的に示すことは難しいが,
  \ref{sec:results}章の実験における,○と△の数を一応は妥当な連
  想の数であるとすると,少くとも,$(16+6)/34 \simeq 0.65$は妥当
  な連想であると言える}.

\subsection{視点を考慮した連想}

換喩を解析するためには,喩詞から連想された名詞の中から,換喩の
(格助詞と述語という)視点に適合する名詞を(被喩詞として)選択する
必要があると第\ref{sec:intro}章で述べた.その規準を,
(\ref{eq:M})式の尺度$M$は満たすと言える.そのことを端的に示す例
が,1番と2番の換喩の対,および,3番と4番の換喩の対である.まず,
1番の「一升瓶を飲む」という換喩と2番の「一升瓶を開ける」という
換喩を見ると,同じ喩詞であっても,異なる被喩詞として「酒」と
「栓」が選ばれている.次に,3番の換喩「鍋が煮える」と4番の換喩
「鍋を食べる」を見ると,それぞれ,「出汁」と「料理」が選択され
ている.また,その他の例についても,被喩詞の候補として選択され
ているのは,格助詞を介して述語に継がるような例がほとんどなので,
尺度$M$は,換喩の視点に適合する名詞を被喩詞として選択していると
言える.

\subsection{不正解例の分析}
\label{sec:analysis}

不正解となった原因について分析し,それらの誤りを改善する可能性
について述べる.

\subsubsection{頻度が少ないことによる不正解例}

表\ref{tab:3}には,頻度が少ないことによる不正解例を載せてある.

まず,喩詞の頻度が少ない場合として,17番の「藁草履が来る」があ
る.これは,喩詞である「藁草履」がコーパス中で1回も出現しなかっ
たため解析に失敗した例である.このように喩詞自体の頻度が少ない
ような例は本手法では解析できない.このような単語を含む換喩を解
析するためには,より大規模なコーパスが必要であろう.ただし,
「藁草履」の場合には,「藁草履」の頻度は0であるが,「わら草履」
の頻度は0ではない.そのため,このような表記のゆれを吸収すること
ができれば,比較的頻度が少ない被喩詞を含む換喩についても解析で
きる可能性がある.あるいは,「藁草履」でなく「草履」を喩詞とみ
なして換喩の解析をすることも考えられるが,これらを試みるのは今
後の課題である.

次に,喩詞と被喩詞との共起頻度が少ない例として,18,19,20番の換
喩がある.たとえば,19番の「傘が行く」は「傘をさした人が行く」
などと解釈できるが,「傘」と「人」などとの共起頻度は,「傘」と
「先」とか「下」とか「柄」とかとの共起頻度と比べれば少ない
\footnote{より正確に言えば,(\ref{eq:L2})式における
  $\frac{\Pr(A,Q,B)}{\Pr(A,Q)}$が小さい.}.そのため,被喩詞と
して優先されない.このような例を提案手法で解析することはできな
い.

ただし,18,19,20番の換喩に限っていえば,被喩詞が全て「人」であ
ると特徴付けることができる.このことは,「人」の場合には,たと
え,喩詞との関連性が低くても被喩詞となりうることを示していると
解釈できる.もし,この解釈が正しければ,「人」に類するものが被
喩詞の候補としてある場合には,それを優先するようにすれば,これ
らの換喩を解釈できる可能性がある.ただし,このことを検証するの
は今後の課題である.

最後に,被喩詞と述語との共起頻度が少ない例として,21,22,23,24番
がある.たとえば,23番の「川が氾濫」は「川の水が氾濫」と解釈で
きるが,「水」と「氾濫」の共起頻度は,「水路」や「河川」や「ダ
ム」と「氾濫」との共起頻度よりも小さい\footnote{より正確に言え
  ば,(\ref{eq:L2})式における$\frac{\Pr(B,R,V)}{\Pr(B)}$が小さ
  い.}.そのため被喩詞として優先されない.このような例も提案手
法で解析することはできない.

ただし,21,22,23,24番の換喩に限れば,述語が場所を対象格として取
りうると特徴付けることができる.そして,これらについては,換喩
として与えられた入力が,現在ではそのまま字義的な表現として通用
すると言える.すなわち,これらの換喩は語源的には換喩であっても,
現在の用法としては既に換喩ではなく,慣用的に場所を対象格に取り
得るといえる.そのため,これらは換喩として解析する必要はないと
考える.ただし,与えられた入力を換喩として解釈すべきかどうか決
めるためには,その入力が換喩かどうかを検出する必要がある.

\subsubsection{解析に文脈が必要な例}

表\ref{tab:4}にある25,26,27番の換喩は,述語の対象格が漠然として
いるため,被喩詞を決めることができない例である.このような例を
解析するためには,\ref{sec:context}節で述べたように,換喩が発話
された文脈を考慮する必要がある.

\subsubsection{その他の原因による不正解例}

表\ref{tab:5}には,その他の原因による不正解例を載せる.

まず,28,29,30番は,被喩詞の一般化が足りない例である.たとえば,
29番の「大阪がげんなり」は「大阪の人々がげんなり」と解釈できる
が,被喩詞の候補としては「委員,弁護士,業者,...」となって
いる.これらの候補はいずれも「人」または「人々」の下位語である
ので,これらの候補を適切に一般化できれば,この換喩を解釈できる
可能性がある.それと同様なことが28,30番にも言える.

次に,31,32番は,「名詞,が,述語」という共起関係を求めるために,
\ref{sec:material}節で述べたように,「名詞,は,述語」などの共
起関係も利用したために生じた間違いである.つまり,31番の場合に
は「ドラマは興奮」という共起関係を利用し,32番の場合には「今日
は行く」という共起関係を利用したための間違いである.このような
ことを回避するためには,名詞と述語の,「が」や「は」を介した共
起頻度をとるときには,その名詞が主格となっている場合についての
み共起頻度を取らなければならない.そうするためには,構文解析を
利用して共起頻度を求める必要があるであろう.なお,現在の方法に
より共起頻度を求めた場合には,「が」のみを利用した場合の解析結
果が,その他のものも利用した場合と比べて良くないことは予備実験
で確かめてある.

最後に,33,34番は,換喩を解釈した結果が依然として換喩的な意味を
持っている場合である.たとえば,33番の「理論が主張」の解析結果
は「党が(理論を)主張」であるが,「党」の背後には,更に「人」が
暗黙の内にいると考えられる.このような例を解釈するためには,解
析結果が換喩的意味を持つかどうかを調べ,もしそれが換喩的意味を
持つならば,それをもう一度解釈する必要がある

\section{おわりに} 
\label{sec:conclusion}

本稿では,比喩の一種である換喩を統計的に解釈する方法について述
べた.

そして,換喩のなかでも,「$名詞A$,$格助詞R$,$述語V$」というタ
イプの換喩を対象とし,以下の方針に基づいて,換喩を解析すること
を試みた.

\begin{enumerate}
\item 「$A$,$R$,$V$」というタイプの換喩が与えられたとき,与えられた喩詞$A$から連想される名詞群を求めるためにコーパスを利用する.
\item 連想された名詞群のなかから,与えられた視点($R$,$V$)に適合するような名詞を被喩詞として統計的に選択する.
\end{enumerate}

その結果,コーパスが連想名詞の供給源として有効なことが例証され,
かつ,提案手法を用いることにより,喩詞から連想された名詞群の中
から,換喩の(格助詞と述語という)視点に適合する名詞を(被喩詞とし
て)選択できることが分かった.また,提案手法による換喩解析の精度
は,扱う対象が換喩という従来あまり解析の対象とされていない現象
であることを考えると,高い値であると我々は判断した.

また,実験結果の不正解例を分析した結果,
\begin{itemize}
\item 喩詞と被喩詞や被喩詞と述語の共起頻度が少ない例
\item 解釈に文脈が必要な例
\item 被喩詞の適切な一般化が必要な例
\end{itemize}
などがあることが分かった.今後は,そのような問題も解決できるよ
うに提案手法を拡張していきたい.

\newpage
\acknowledgment

本稿に対して有益なコメントを下さった筑波大学山本幹雄助教授に感
謝する.

\appendix
\input{tables2}

\clearpage
\bibliographystyle{jnlpbbl}
\bibliography{v07n2_05}

\begin{biography}
\biotitle{略歴}
\bioauthor{内山 将夫}{
筑波大学第三学群情報学類卒業(1992).
筑波大学大学院工学研究科博士課程修了(1997).博士(工学).
信州大学工学部電気電子工学科助手(1997).
郵政省通信総合研究所非常勤職員(1999).
}
\bioauthor{村田 真樹}{
1993年京都大学工学部卒業.
1995年同大学院修士課程修了.
1997年同大学院博士課程修了,博士(工学).
同年,京都大学にて日本学術振興会リサーチ・アソシエイト.
1998年郵政省通信総合研究所入所.研究官.
自然言語処理,機械翻訳,情報検索の研究に従事.
言語処理学会,情報処理学会,ACL,各会員.}
\bioauthor{馬 青}{
1983年北京航空航天大学自動制御学部卒業.
1987年筑波大学大学院理工学研究科修士課程修了.
1990年同大学院工学研究科博士課程修了.工学博士.
1990 $\sim$ 93年株式会社小野測器勤務.
1993年郵政省通信総合研究所入所,主任研究官. 
人工神経回路網モデル,知識表現,自然言語処理の研究に従事. 
日本神経回路学会,言語処理学会,電子情報通信学会,各会員.}
\bioauthor{内元 清貴}{
1994年京都大学工学部卒業.
1996年同大学院修士課程修了.
同年郵政省通信総合研究所入所,郵政技官.
自然言語処理の研究に従事.
言語処理学会,情報処理学会,ACL,各会員.}
\bioauthor{井佐原 均}{
1978年京都大学工学部電気工学第二学科卒業.
1980年同大学院修士課程修了.博士(工学).
同年通商産業省電子技術総合研究所入所.
1995年郵政省通信総合研究所
関西支所知的機能研究室室長.自然言語処理,機械翻訳の研究に従事.
言語処理学会,情報処理学会,人工知能学会,日本認知科学会,ACL,各会員.}
\bioreceived{受付}
\biorevised{再受付}
\bioaccepted{採録}
\end{biography}

\end{document}
