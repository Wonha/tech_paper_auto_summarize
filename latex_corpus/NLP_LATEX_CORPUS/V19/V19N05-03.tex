    \documentclass[japanese]{jnlp_1.4}
\usepackage{jnlpbbl_1.3}
\usepackage[dvips]{graphicx}
\usepackage{amsmath}
\usepackage{hangcaption_jnlp}
\usepackage{udline}
\setulminsep{1.2ex}{0.2ex}
\let\underline

\newcommand{\DownCell}[2]{}
\newcommand{\DownCellB}[2]{}



\Volume{19}
\Number{5}
\Month{December}
\Year{2012}

\received{2012}{5}{7}
\revised{2012}{7}{15}
\accepted{2012}{8}{22}

\setcounter{page}{401}

\jtitle{マイクロブログにおける感情・コミュニケーション・動作タイプの推定に基づく顔文字の推薦}
\jauthor{江村 優花\affiref{TsukubaB}\affiref{Fukoku} \and 関  洋平\affiref{TsukubaM}}
\jabstract{
現在,電子メール,チャット,マイクロブログなどのメディアで,顔文字は日常的に使用されている.顔文字は,言語コミュニケーションで表現できない,ユーザの感情やコミュニケーションの意図を表すのに便利であるが,反面,その種類は膨大であり,場面に合った顔文字を選ぶことは難しい.本研究では,ユーザの顔文字選択支援を目的として,ユーザが入力したテキストに現れる感情,コミュニケーション,動作のタイプ推定を行い,顔文字を推薦する方法を提案する.感情,コミュニケーション,動作のタイプは,{\it Twitter} から収集したコーパスを用いてカテゴリを定義し,推定システムは,$k$-NN に基づき実現した.また,システムが推薦する顔文字がユーザの意図にどの程度適合しているか,5名の被験者により評価した結果,91件のつぶやきに対して66.6\%の顔文字が適切に推定されており,感情カテゴリのみを用いて推薦された結果と比べて,提案手法の顔文字推薦の精度が有意に向上していることがわかった.
}
\jkeywords{顔文字推薦,感情カテゴリ,コミュニケーションタイプ,動作タイプ,マイクロブログ}

\etitle{Facemark Recommendation based on Emotion, Communication, and Motion Type Estimation \\in Microblogs}
\eauthor{Yuka Emura\affiref{TsukubaB}\affiref{Fukoku} \and Yohei Seki\affiref{TsukubaM}} 
\eabstract{
Many users use facemarks everyday in recent computer mediated communication environments such as e-mail, chatting, and Microblogs. Although facemarks are useful to express the emotion or communication intentions beyond natural language communication, many users feel difficult to choose the right one from lots of candidates according to the situation. We propose a method to recommend facemarks based on the estimation of emotions, communication, or motion types in texts written by users. Emotion, communication, or motion types are defined with {\it Twitter} corpus, and estimation system is implemented with $k$-NN. Five assessors evaluated the relevance of recommended facemarks for their intention, and found that 66.6\% of facemarks for 91 tweets were recommended properly, which improved significantly over the recommendation only from emotion categories.
}
\ekeywords{Facemark Recommendation, Emotion Category, Communication Type, Motion Type, and Microblog}

\headauthor{江村,関}
\headtitle{感情・コミュニケーション・動作タイプの推定に基づく顔文字の推薦}

\affilabel{TsukubaB}{筑波大学情報学群知識情報・図書館学類}{College of Knowledge and Library Sciences, School of Informatics, University of Tsukuba}
\affilabel{Fukoku}{現在,フコク情報システム株式会社}{Presently with Fukoku Information Systems}
\affilabel{TsukubaM}{筑波大学図書館情報メディア系}{Faculty of Library, Information and Media Science, University of Tsukuba}



\begin{document}
\maketitle


\section{はじめに}

現在,電子メール,チャット,{\it Twitter}\footnote{http://twitter.com}に代表されるマイクロブログサービスなど,文字ベースのコミュニケーションが日常的に利用されている.これらのコミュニケーションにみられる特徴の一つとして,顔文字があげられる\cite{ptas2012}.旧来の計算機を介した電子メールなど,ある程度時間のかかることを前提としたコミュニケーションでは,直接会った際に現れる非言語的な情報,具体的には,表情や身振りから読み取ることのできる感情やニュアンスなどの手がかりが少なくなることから,フレーミングなどのリスクを避けようとすると,個人的な感情を含まない目的のはっきりした対話に用いることが適切とされる\cite{derks2007}.一方,利用者のネットワークへのアクセス時間の増加に伴い,マイクロブログや携帯メールなど,リアルタイム性の高いコミュニケーションメディアが発達するとともに,親しい友人同士の非目的志向対話への需要は増している.このようなコミュニケーションにおいては,顔文字が,対面コミュニケーションにおける非言語情報の一部を補完するとされている\cite{derks2007}.

顔文字とは``(\verb|^|−\verb|^|)''のように,記号や文字を組みあわせて表情を表現したもので,テキスト中で表現された感情を強調・補足できる,という利点がある.一方,マイクロブログや携帯メールなど,リアルタイム性の高いコミュニケーションメディアの発達と時期を同じくして,その種類は増加の一途をたどっている.その中から,ユーザが文章で伝えたい感情に適切な顔文字を,ただひとつだけ選択するのは困難である.また,顔文字入力の主な方法である顔文字辞書による選択では,指定された分類カテゴリ以外の意味での使用を目的とした顔文字を入力することは難しく,予測変換機能では,単語単位を対象としてしか顔文字を提示できない.そのほかの手段として,他のテキストからのコピーアンドペーストやユーザ自身による直接入力があるが,これらは操作数が多く,効率的ではない.

そこで,本研究では,ユーザによる適切な顔文字選択の支援を目的とし,{\bf ユーザの入力文章から,感情カテゴリやコミュニケーションや動作を反映したカテゴリを推定}し,顔文字を推薦するシステムの構築を目指す.

本論文の構成は以下のとおりである.\ref{sec:related}節では,関連研究を紹介する.\ref{sec:category}節では,顔文字推薦のために本研究で定義したカテゴリについて説明する.\ref{sec:implementation}節では,顔文字推薦システムの実現について紹介し,\ref{sec:evaluation}節では,評価実験について説明する.最後に,\ref{sec:conclusion}節で結論をまとめる.


\section{関連研究}
\label{sec:related}

顔文字推薦を目指した先行研究として,\citeA{Suzuki2006}では,ユーザの入力した文章から Plutchik の体系に基づき感情を推定し,顔文字を推薦することを試みている.また,感情と顔文字の関係を利用した別の応用事例として,感情値を入力とした顔文字パーツの生成\cite{nakamura2003b},文章と顔文字を利用した感情推定\cite{shino},顔文字の抽出と感情解析\cite{ptas2010b}等の研究が行われている.本研究では,感情に限らず,コミュニケーションや動作を反映したカテゴリに基づく顔文字推薦の効果について検証を進める.

また,\citeA{mura}は,顔文字に関する調査や考察を行っている.彼らの調査の結果,顔文字は,単純な極性にはわけられず,複雑な感情にわかれることや,顔文字自体の極性と文の極性とが異なる場合があることが明らかとなった.\citeA{kawa}の研究では,顔文字の表す感情について調査しており,1つの顔文字は複数の感情を表す場合があることを示している.

一方,\citeA{kono}は,年代や性別による顔文字の使用理由について調査を行った結果,文字だけではそっけない画面をにぎやかにするため,自分の伝えたい感情を強調する,文章の深刻さを和らげる,などの理由を明らかにした.\citeA{kato}は,顔文字が感情を表すことを感情表現機能,顔文字を使うことで言葉での表現が減ることをメール本文代替機能と呼んでいる.彼らは,顔文字を使用することでメール本文の文字数が減ること,親しい間柄では感情表現機能,メール本文代替機能がより使用されることを明らかにしている.また,\citeA{yamaguchi2000}では,電子掲示板,ネットニュース,メーリングリストなどを対象として,顔文字から受ける情緒・感情について内容分析を行った結果,礼,謝罪,要請などのカテゴリが存在することを明らかにした.

加藤らの研究から,文の感情を顔文字が代替する機能,つまり{\bf 文の感情を強調する}機能を持っていることが明らかとなった一方,河野らの研究からは,{\bf 画面をにぎやかにする},{\bf 文章の感情を和らげる},といった機能を持つことがわかった.このことから,顔文字を推薦するためには,単純に文章の感情のみを手がかりとするだけでなく,文章に現れる特定の表現なども考慮したルールを設定しなければならない.本研究では,これらの傾向をより詳細な形で明らかにするため,顔文字を含む文章を分析し顔文字の使われ方を調査する.そしてその結果から,親密なやりとりが多く出現するマイクロブログにおいて,顔文字推薦を行うためには,感情以外に必要な,{\bf コミュニケーションや動作を反映したカテゴリ}を,\citeA{yamaguchi2000}よりも詳細に区別する必要があることを明らかにし,顔文字推薦システムを構築する.

また,顔文字ではないが,{\it Twitter} の文章に絵文字を推薦する研究として\citeA{hashi}がある.橋本の研究では,まず入力文章を形態素解析し,それを単語3グラムに分割する.そして絵文字入りコーパスを用い,単語3グラムと類似するコーパス中の文で用いられている絵文字を文章に挿入している.

しかし,絵文字は,顔文字と比べると,キャラクター,動物,ハートマーク,アイスキャンディーのような記号など,装飾要素が強く,文の任意の位置に挿入される傾向が強い.一方,顔文字は,感情やコミュニケーションを反映した表現の直後の位置以外には入りにくい.したがって,感情やコミュニケーションを反映した手がかり語辞書を用いて,カテゴリを推定した方が,顔文字推薦の目的には有効と考える.また,感情推定については,ポジネガ推定を前処理とする2段階推定にすることで,精度が向上することが\citeA{toku}から明らかになっている.

本研究では,マイクロブログ ({\it Twitter}) の文章を収集し,コーパスを作成する.また,入力文に含まれる感情語や特定の表現などを手がかりにコーパスから類似文を見つけ,感情やコミュニケーションのカテゴリ推定を行う.カテゴリ推定については,\citeA{toku}にならい,2段階での推定を行う.その結果を用いて,作成した顔文字データベースから顔文字を推薦する.

マイクロブログを対象とした感情推定については,感情語辞書から音韻論に基づき長音化したものを検出すること\cite{brody2011}や,顔文字そのものやハッシュタグを使って訓練データを拡張すること\cite{purver2012}により,感情推定の精度向上を目指す研究がある.これらのアプローチは興味深いが,前者は言語に依存しており,後者は少数の感情を除いて対応関係が曖昧とされていることから,本研究では,これらのデータの拡張は行わずに,定義したカテゴリの顔文字推薦についての効果を明らかにすることに主眼を置く.



\section{顔文字推薦に用いるカテゴリの定義}
\label{sec:category}

本研究で提案する顔文字推薦システムは,大きく分けてカテゴリ推定と顔文字推薦の2種類の処理から構成される.本節では,それぞれの処理を行う際に必要となる情報が何かを分析し,その結果,必要となる基本情報を定義する.


\subsection{カテゴリの調査方法}

顔文字推薦のためのカテゴリ推定を行うにあたって,必要な感情カテゴリ,また,コミュニケーションや動作を反映したカテゴリを明らかにするため,調査を行った.
調査データには,きざしラボ\cite{kizashi}が提供する顔文字92個を含む {\it Twitter} のつぶやき1,722件を収集し使用した.これらのデータに,極性(ポジティブ,ネガティブ,なしの3種類)と感情カテゴリ(\citeA{naka},「喜」「怒」「哀」「恥」「怖」「好」「厭」「昂」「安」「驚」の10種類)をそれぞれ分類して付与した.また,10種類のカテゴリに分類することができず,よく現れるような表現の場合は,新たにカテゴリを定義した.

調査の結果,顔文字推薦に必要な情報として,定義した情報の例を以下に示す.

\begin{enumerate}
\item 感情の種類について
\begin{enumerate}
\item 「不快」「不安」など,頻出する項目を中心に,感情カテゴリを定義した.また,{\it Twitter} のつぶやきには,期待を表す表現や,疲れを表す表現などを含む文章が多く見られた.しかし,\citeA{naka}の感情分類にはそれらを表す感情カテゴリはないため,新たに「期待」,「疲れ」という感情カテゴリを定義した.
\end{enumerate}
\item 感情以外の特徴について
\begin{enumerate}
\item ``m(\_ \_)m ''は,「ごめんなさい」や「申し訳ない」などの謝罪表現を含む文に付与されることが多かった.しかし,謝罪を表すような感情カテゴリは\citeA{naka}の分類には含まれておらず,また,{\bf 謝罪は感情とは言い難い側面を持つ}\footnote{\citeA{ortony1990}のように,感情を広くとらえると,謝罪自体は何らかの感情により誘発されると考えることもできるが,本研究では,顔文字の使用者が,顔文字を使用することで伝えたい意図は,原因となる感情ではなく,コミュニケーションにおける行為そのものと考える.}.そこで,感情分類とは別のカテゴリとして,新たに「謝罪」カテゴリを定義した.

このような,誰かとのやり取りや,人に伝えることを前提としたカテゴリが多く存在したため,{\bf コミュニケーションタイプ}として,新たに定義した.
\item ``(-\_-)''や``(*\_*)''は,睡眠を表す文章で使用されていた.{\bf 睡眠は動作であり,感情ではない}.感情カテゴリに当てはめることはできないので,これも新たに「睡眠」カテゴリを定義した.この動作を表すカテゴリを,{\bf 動作タイプ}として新たに定義した.
\end{enumerate}
\end{enumerate}



\subsection{定義したカテゴリ体系}

前節の結果に基づき,顔文字推薦に必要な感情,感情以外のカテゴリを決定した.感情カテゴリの定義を表\ref{table:em}に,感情以外の特徴を,コミュニケーションタイプ,動作タイプとして定義した結果を表\ref{table:com}に示す.

感情カテゴリは全部で27種類ある.中村の感情分類を基礎カテゴリとして用いて,それら27種類の感情を詳細カテゴリとして10種類の基礎カテゴリに分類した\footnote{「残念」,「悔しい」は,感情表現辞典\cite{naka}では「厭」に分類されているが,文章に含まれる顔文字は「悲しい」に含まれる顔文字と同じことが多く,「哀」に分類した.}.また,各詳細カテゴリごとに,極性(ポジティブ/ネガティブ/なし)の設定を行った.


\begin{table}[p]
\caption{感情カテゴリの定義}
\label{table:em}
\input{03table01.txt}
\end{table}

コミュニケーション・動作タイプは全部で10種類ある.感情には分類できないが,表現としてよく現れるものを分類した.コミュニケーションタイプは,人とのやり取りや他者に物事を伝えることを中心とした表現のタイプであり,人にどのように伝えるかによって「やりとり型(他者との会話に含まれる表現)」「つぶやき型(他者に見られることを前提としているが,会話にはなっていない文章に含まれる表現)」「不特定多数型(やりとり,つぶやきのどちらにも使われる表現)」の3つに分類しており,それらの分類の下に詳細カテゴリを設定した.また,感情カテゴリ,コミュニケーションタイプに分類できず,文章の中心が動作表現となるものを動作タイプとして定義した.現在,動作タイプとして定義できる程度に頻出した動作表現は「睡眠」だけであるため,動作タイプは1種類のみである.

\begin{table}[t]
\caption{コミュニケーション・動作タイプの定義}
\label{table:com}
\input{03table02.txt}
\end{table}


\subsection{カテゴリ定義の妥当性についての検証}
\label{subsec:kappa}

作成したコーパスの文章に付与したカテゴリがどの程度安定しているかを調べるため,第1著者と協力者1名(ともに大学生,女性と男性)が,コーパス中からランダムに選んだ190件について,(1)極性と(2)感情カテゴリ,コミュニケーションタイプ,動作タイプのいずれかを付与した場合の一致度($\kappa$ 係数)を調べた.その結果,極性の一致については,$\kappa$ 値が,0.850(almost perfect/ほとんど一致\cite{landis1977}),感情カテゴリ,コミュニケーションタイプまたは動作タイプの一致については,$\kappa$ 値が,0.747(substantial/かなり一致\cite{landis1977})となり,付与方針が安定していることを示すことができた.

本カテゴリのような分類体系の関連研究としては,\citeA{kikui2012}の提案がある.基本的な考え方は,両方の体系で共通しており,菊井の陳述型は,ひとりごと型に対応し,発話行為タイプは,やりとり型に対応すると考えられる.ただし,その下位のカテゴリの分け方に関しては,本研究では,顔文字を推薦する,という前提のもとに,顔文字のつけやすいカテゴリ体系になっている.


\section{顔文字推薦システムの実現}
\label{sec:implementation}

\ref{sec:category}節で定義したカテゴリ体系に基づき,顔文字推薦システムを実現した.構築した顔文字推薦システムは,カテゴリ推定処理と顔文字推薦処理の2つから成り立っている.以下に,顔文字推薦システム全体の流れを示す.

\begin{enumerate}
\item ユーザが文章を入力する.
\item 入力文章を用いて,カテゴリ推定を行う({\bf カテゴリ推定処理}).
\item カテゴリ推定の結果を用いて,顔文字データベースから適切な顔文字を取り出す({\bf 顔文字推薦処理}).
\item 取り出した5件の顔文字を推薦候補として,画面に表示する.
\end{enumerate}

\noindent
本節では,上記の処理に必要なデータの作成について,\ref{subsec:data}節で具体的な内容を紹介する.(2)(3)の2つの処理については,\ref{subsec:estimate}節,\ref{subsec:recommend}節で説明する.


\subsection{必要なデータの作成}
\label{subsec:data}

本研究では,カテゴリ推定処理と顔文字推薦処理に必要なデータとして,顔文字辞書,タグ付きコーパス,手がかり語辞書,顔文字データベースを作成した.以下,詳細を説明する.


\subsubsection*{顔文字辞書}

顔文字推薦に使用する顔文字は,予備実験で用いた92個では各感情を表す顔文字を推薦するのに十分な数ではなかったため,アンケートを実施し追加を行い,163種類とした.


\subsubsection*{タグ付きコーパス}

\ref{sec:category}節で定義したカテゴリ体系を用いて,感情,コミュニケーションタイプ,動作タイプのタグ付きコーパス(以下,タグ付きコーパス)を作成した.コーパスの文章は,{\it Twitter} の 2011年5月1日〜31日,7月1日〜16日のパブリックタイムライン上のツイートから,前述の顔文字163個を含む1,369件(ツイート数)を収集し,それらの文章に,それぞれ(1)極性,(2)感情カテゴリ,コミュニケーションタイプ,動作タイプのいずれかを,\ref{subsec:kappa}節の判定者2名で協議して付与した.さらに,日常的に顔文字を使用する別の20代の協力者2名と協議しつつ,コーパスの修正と拡張を行い,最終的に,3,975件(各詳細カテゴリごと100件強)とした.


\subsubsection*{手がかり語辞書}

\ref{sec:category}節のカテゴリ体系を用いて,手がかり語辞書を作成した.手がかり語辞書は感情表現辞典\cite{naka},単語感情極性対応表\cite{taka}を参考にし,全1,440語の項目を設定した.それぞれの語には,極性(ポジティブ,ネガティブ,なしのいずれか)と感情カテゴリ,コミュニケーションタイプまたは動作タイプのいずれかを付与した.なお,手がかり語は,おはよう(おはよ,おはよー,おはよ〜,おはしゃす
),さようなら(さよーなら,グッバイ,さらば)のような,オンライン上の挨拶表現の変形も含む.


\subsubsection*{顔文字データベース}

顔文字推薦処理を実装するために,収集した文章に含まれる顔文字の使われ方を分析し,顔文字データベースの作成を行った.作成したタグ付きコーパスには顔文字が含まれているため,それを分析することで各顔文字の感情ごとの出現頻度を調べた.

なお,単純なコーパス中での出現頻度以外に,同じカテゴリに分類される顔文字でも,文章中の表現の違いで使用のされ方が違う場合には,補足ルールを作成し,補足ルールにおいて設定した下位カテゴリ別の出現頻度も調べた.補足ルールは2つあり,以下のとおりである.

\begin{enumerate}
\item 「あいさつ」カテゴリ

「あいさつ」カテゴリでは,どの時間帯のあいさつか(例:おはよう→``(⌒▽⌒)'',おやすみ→``(( \_ \_ ))..zzzZZ''),あるいは時間帯に無関係なあいさつ(例:よろしく→\\``(‾\verb|^|‾)ゞ'',さようなら→``(´\_ゝ`)'')かによって,使用頻度の高い顔文字が変わってくる.そこで,「あいさつ」カテゴリでは,あいさつ表現別に9つの下位カテゴリを作成した.また,オンライン上の挨拶表現の変形についても,考慮した上で,各顔文字がどの分類に当てはまるのかを分析し,それぞれの分類ごとに出現頻度を計算した.

\item 「感謝」カテゴリ

「感謝」カテゴリでは,丁寧な口調の表現を含む場合,泣き顔や申し訳なさを表す顔文字(例:``ヽ(;▽;)ノ'')がより使用されていた.そこで,「ございます」「ありがたいです」などの言葉を含む文章について,顔文字ごとの頻度を計算した.
\end{enumerate}

また,入力文章がこれらの補足ルールに一致する語を含んでいるかを判断する必要があるため,補足ルールの下位カテゴリごとに,それらを表す表現を集め,補足ルール辞書を作成した.

顔文字データベースは,顔文字とカテゴリを組み合わせて主キーとし,コーパス中でのカテゴリ別の顔文字の出現頻度,補足ルールで指定した下位カテゴリ別の出現頻度を加えて,1レコードとして格納した.データベースの例を,表\ref{table:database}に示す.

\newcommand{\TateMoji}[1]{}
\begin{table}[t]
\caption{顔文字データベースの例}
\label{table:database}
\input{03table03.txt}
\end{table}

顔文字データベースの傾向を確認するため,{\it CAO}システム\cite{ptas2010b}との比較を行った.このシステムでは,分類体系は,\citeA{naka}の感情10種類に基づいており,本研究で定義したカテゴリを,表\ref{table:em}の体系に基づき感情10種類により対応付けることで,顔文字に対する感情カテゴリの傾向を比較した.具体的には,\citeA[5.4節]{ptas2010b}の論文で有効と判定されている {\it Unique Frequency} と,本データベースの相対出現頻度との相関係数を調査した.なお,顔文字は双方で分析可能なもの,かつある程度の出現回数がみられる43個を対象として分析を行った.その結果,相関係数が0.4以上のものが,53.5\%,正の相関を持つものが72.1\%となり,ある程度の相関がみられた.相関係数の低い顔文字は,泣き笑いや,汗付きの笑いを表す顔文字などが含まれており,本研究のデータベースでは,``厭'',``哀''などの感情を表す傾向があるのに対して,{\it CAO}システムでは,他の感情の値の方が大きい傾向がみられる.これは,{\it CAO}システムではWeb全般やブログを対象としているのに対して,本研究ではマイクロブログを対象としていることから,使用傾向の違いが現れていると考えている.

なお,表\ref{table:com}のコミュニケーション・動作タイプについては,上記のシステムとの比較が行えないことから,この比較には用いていない.


\subsection{カテゴリ推定処理}
\label{subsec:estimate}

カテゴリ推定を行うために,分類器を構築した.分類器には$k$-NNを用い,文章を入力すると,各学習データとの類似度を計算し,類似している上位$k$件の中で最も多い感情を推定結果として返す.学習データには,\ref{subsec:data}節で説明した,タグ付きコーパスを用いた.

分類器によるカテゴリ推定の手順を以下に示す.
\begin{enumerate}
\item 極性(``A. ポジティブ'',``B. ネガティブ'',``C. なし''のいずれか)を推定する.
\begin{enumerate}
\item まず,手がかり語辞書に含まれる語の,入力文における出現頻度を計算する.
\item (a)の結果から得られた各手がかり語の頻度を用いて,極性ごとに素性ベクトルを構築する.
\item 学習データ中の各文との平方距離を計算する.
\item 平方距離の近い上位$k$件のうち,学習データにおいて最も多く付与されている極性を推定結果とする.
\end{enumerate}
\item 推定した極性に対応するカテゴリとして,A. 感情(ポジティブ):嬉しい,めでたい 等;B. 感情(ネガティブ):怒り,悲しい 等;C. 感情(なし):興奮,安らぎ 等 か,コミュニケーション・動作タイプ:感謝,謝罪 等 のいずれかを推定する.
\begin{enumerate}
\item 素性ベクトルの構築や各学習データとの距離の計算は,1の(a)--(c)と同様に推定する.
\item 平方距離の近い上位$k$件のうち,学習データにおいて最も多く付与されている感情カテゴリ,コミュニケーションタイプ,動作タイプのいずれかを推定結果とする.
\end{enumerate}
\end{enumerate}

また,入力文に複数の手がかり語が含まれる場合,その位置関係によって推定を誤る可能性がある.

たとえば,「おかえり〜今日はありがとね」という文には,「おかえり」と「ありがとう」という2つのコミュニケーションタイプの手がかり語が含まれている.この文の場合では,文全体のコミュニケーションタイプは,「感謝」になるのだが,「あいさつ」が推定結果として出力されてしまう可能性がある.各カテゴリの出現頻度を計算して推定すると,どちらも同じ頻度なため,どちらが推定されてもおかしくない.これは,文章内に現れる手がかり語の位置関係を考慮していないことによる.この対策として,入力文内に現れる順序によって,手がかり語に重みを付与する,といった処理を追加した.具体的には,以下のような計算を行う.なお,$word_{n}$は入力文内に $n$ 番目に現れる手がかり語,$N$ は入力文内に現れる手がかり語の総数である.
\[
weight(word_{n}) = \left[\frac{n}{N}\right]
\]

次に,構築した分類器の推定精度ならびに,重み付け処理による推定精度の向上を評価するため,10分割交差検定により実験を行った.具体的には,作成したタグ付きコーパス3,975件を,398件を5個,397件を5個の,10個に分割し,10分割の交差検定を行い,推定した極性,カテゴリと,人手でつけた極性,カテゴリがどの程度一致しているかを計算した.推定精度の評価尺度としては,正確さ (accuracy) を採用した.

実験の結果,重み付け処理を行っていない場合では,極性の推定精度が69.7\%,極性とカテゴリの双方が正解していたものが43.0\%であったのに対して,重みづけ処理を行った場合では,極性の推定精度が70.2\%,極性とカテゴリの双方が正解していたものが43.5\%であり,統計的有意差はないが,わずかに精度が向上した.重みづけ処理の失敗分析を30件を対象に行ったところ,手がかり語を手動で付与した場合には,28件について,最後の手がかり語のカテゴリが文章のカテゴリと一致することがわかり,手がかり語辞書との照合の精度を向上する必要性が明らかになった.

カテゴリの推定精度が十分でない別の理由としては,``報告''カテゴリの分類が難しいことがあげられる.このカテゴリは,{\it Twitter} 特有のひとりごとである ``〜なう'' のような,個別の行動の報告が含まれており,機械学習に適した共通の手がかりが得られにくい.この辺りのカテゴリの推定精度の向上には,ユーザの履歴,文脈といった情報が必要になると考える.なお,\mbox{``報}告''カテゴリを含まずに推定した場合には,推定精度は1.1\%向上した.


\subsection{顔文字推薦処理}
\label{subsec:recommend}

次に,顔文字推薦処理の実装を行った.顔文字推薦処理では,カテゴリ推定の結果に適合した顔文字を,\ref{subsec:data}節で説明した顔文字データベースを用いて判定し,出現頻度の多い順に,最大5件を推薦する.また,もしカテゴリ推定の結果が,\ref{subsec:data}節で説明した補足ルールにマッチする場合には,補足ルール用辞書を用いてさらに細かい分類を行い,適合した顔文字を推薦する.補足ルールに当てはまらないカテゴリであれば,カテゴリ推定の結果だけを用いて,顔文字データベースから顔文字を取り出す.


\section{顔文字推薦システムの評価}
\label{sec:evaluation}

顔文字推薦システムを実際に被験者に使わせ,推薦する顔文字がどの程度ユーザの入力した文章に対して適切であるか調べるため,被験者実験を行った.
被験者は,日常的に顔文字を使用する21〜22歳の男性2名,女性3名である.

\subsection{実験内容}

被験者実験は,2つの項目から構成される.

\subsubsection*{実験1: マイクロブログを対象とした顔文字推薦のカテゴリに基づく評価}

\begin{itemize}
\item 実験1では,{\it Twitter} のつぶやきを用意し,そのつぶやきに対して推薦される顔文字が適切かを回答させた.なお,つぶやきは全部で91個あり,\ref{subsec:kappa}節で記述した,判定者間一致率の計算に用いた190件から,人手で判定した各感情カテゴリ,コミュニケーションタイプ,動作タイプにつき1--3個のつぶやきをランダムに抜き出し使用した.なお,本データは,\ref{sec:implementation}節で説明したカテゴリ推定に用いた学習データとは収集方法が異なり,データに重複はないことも確認済みである.
\item 1つの文章につき,次の2つの推薦結果に対して回答させた.
\begin{enumerate}
\item({\bf 提案手法})感情カテゴリ,コミュニケーションタイプ,動作タイプのいずれかを推定したカテゴリに対して,推薦した顔文字.
\item({\bf ベースライン})分類器の学習データを感情カテゴリに限定して,推薦した顔文字.
\end{enumerate}
\item 顔文字は,各文章について上位5件を推薦し,評価した.評価は,``○(=適してい\mbox{る){\kern-0.5zw}'',}\mbox{``△}(=どちらともいえない){\kern-0.5zw}'',``×(=不適切){\kern-0.5zw}''の3段階で行い,適していると判定された数を分子,カテゴリごとの正解数を分母として推薦精度を計算した.
\end{itemize}


\subsubsection*{実験2: 自由入力を対象とした顔文字推薦の極性に基づく評価}

\begin{itemize}
\item 実験2では,被験者に実際に顔文字推薦システムを使用させた.被験者自身が,実際に {\it Twitter} や携帯メールで使用しそうな文章を考え,システムに入力し,その結果推薦された顔文字が適切かを回答させた.なお,10文はポジティブな文章,10文はネガティブな文章,10文は極性がない文章として,全部で30文入力させた.なお,極性がない文章については,コミュニケーション・動作タイプおよび感情の極性なしのタイプの文章を,被験者に例示した上で,文章を考えさせた.
\item こちらも,実験1と同様に提案システムとベースラインの2つを利用して評価した.
\item 顔文字の推薦件数も実験1と同様に5件であり,評価方法は,入力したポジティブ,ネガティブ,極性なしの文章数を分母として推薦精度を計算する点を除いて,実験1と共通である.
\item また,実験後に,適切でなかった顔文字とその理由について,自由に感想を記入させた.
\end{itemize}


\subsection{実験結果}

\begin{table}[b]
\caption{実験1の結果:マイクロブログを対象とした顔文字推薦のカテゴリに基づく評価}
\label{table:kekka1}
\input{03table04.txt}
\end{table}

実験1,2の結果を,表\ref{table:kekka1},表\ref{table:kekka2}に示す.表4中の$^{*}$は,実験1において,$t$-検定を用いた結果,提案手法が,感情カテゴリのみに基づいて推薦したベースラインと比べて,有意水準5\%,両側検定で,推薦精度に有意な改善があることを示す.



\subsection{実験の考察}

実験1,2の結果に基づく考察を以下に示す.

\begin{table}[t]
\caption{実験2の結果:自由入力を対象とした顔文字推薦の極性に基づく評価}
\label{table:kekka2}
\input{03table05.txt}
\end{table}

\subsubsection*{実験1の考察:マイクロブログを対象とした顔文字推薦のカテゴリに基づく評価}

\begin{itemize}
\item 表\ref{table:kekka1}において,提案手法は,ベースラインと比べて,コミュニケーションタイプ,動作タイプのみならず,感情カテゴリにおいても精度の向上が見られる.これは,コミュニケーション・動作タイプを導入することが顔文字推薦において有用であるだけでなく,感情に基づく顔文字推薦についても,より適切な顔文字を推薦できることを示唆している.
\item 推薦精度が大きく向上したカテゴリは,コミュニケーション・動作タイプでは,``ねぎらい'',``あいさつ'',``心配'',``感謝'',``謝罪'',``睡眠''などであり,感情カテゴリでは,\mbox{``不}安'',``安らぎ'',``期待''などであった.
\item このうち,``謝罪'',``睡眠''のように,特定の顔文字が文章にぴったり適合するカテゴリの場合,コミュニケーション・動作タイプを推定し,顔文字を推薦することが有効である.
\item また,``あいさつ''については,「こんにちは」などは曖昧な表現であり,ポジティブな顔文字であれば,カテゴリ推定が間違っていても,ユーザの要求に適合する可能性はある.一方,「よろしく〜」,「おやすみ」などの挨拶表現については,\ref{subsec:data}節で説明した補足ルールに基づき時間帯を考慮して表現を区別し,顔文字を推薦することが有効である.
\item 例として,「昼からすみません」という文に対する顔文字を推薦する場合を考える.この文章は``謝罪''を表しているが,感情カテゴリのみを推定するベースラインシステムでは,別の感情(この場合,``興奮'')の推定が行われる.その結果,下記のような顔文字が推薦される.\\
\verb|\(^o^)/| (*´Д`) (*´д`*) (*´∀`*) (´;ω;`)\\
この中の 2〜4 番目の顔文字では,``謝罪''とはまったく逆の印象を与えてしまう.\\
しかし,コミュニケーションタイプを推定できる提案手法では,``謝罪''を推定でき,\\
\verb|m(_ _)m orz (´;ω;`) (´・ω・`) (^_^;)|\\
のように,5件とも文章の意図に適合した顔文字を推薦している.
\item 適切でないと回答が多くあったのは,システムの推定したカテゴリが,人手で付与したものと異なる場合である.特に正反対の極性のカテゴリが推定されていると,推薦された顔文字候補はまったく役に立たないこととなる.よって,システムの精度を上げるためには,極性推定の精度の向上が必要と考えられる.
\end{itemize}


\subsubsection*{実験2の考察:自由入力を対象とした顔文字推薦の極性に基づく評価}

\begin{itemize}
\item 実験2についての実験後の感想には,「ネガティブな文章に対する顔文字推薦の精度がよい」とあった.これは,ポジティブな文章よりもネガティブな文章の方が,直接的な手がかり語が文中に現れやすいため,カテゴリを推薦しやすいことが原因と考えられる.
\item 極性なしの文章に対する推薦精度が,一部の被験者で低い理由として,``報告''カテゴリの推定がうまくいかない点の影響がある.なお,手がかり語が存在しない場合に``報告''カテゴリとして顔文字を推薦した場合について調査した結果,極性なしの文章に対する推薦精度の合計は,60.4\%に向上した.
\item その他,実験後の感想として,「同じカテゴリの文章に対して,表示される顔文字の種類・順序が一様である」とあった.現在,手がかり語辞書に収録している,同じカテゴリの各手がかり語には,一意に極性が付与されている.そのため,含まれている語が異なる場合でも,推定結果に違いがなく,同じ顔文字が推薦される.この点については,同じカテゴリの手がかり語でも,極性を考慮して手がかり語辞書を作成すること,また,極性を考慮した顔文字データベースを作成することにより,表示される顔文字の種類・順序を豊富にするだけでなく,顔文字推薦自体の精度向上にもつながると期待できる.
\end{itemize}


\section{おわりに}
\label{sec:conclusion}

本研究では,ユーザの顔文字選択の支援を目的とし,ユーザの入力文から感情や,感情以外のコミュニケーションや動作を反映したカテゴリを推定し,顔文字を推薦する手法を提案した.提案手法は,2つの処理にわかれており,カテゴリ推定処理と顔文字推薦処理から構成される.

カテゴリ推定処理については,{\it k}-NNを用いて分類器を構築した.すなわち,文章を入力すると,手がかり語辞書内の語が含まれていないかを調べ,各カテゴリの手がかり語の出現頻度から構築した素性ベクトルを用いて,各学習データとの類似度を計算し,類似度の高い上位{\it k}件から推定結果を決定する.

顔文字推薦処理では,カテゴリ推定の結果を用いて,顔文字データベースから適切な顔文字を取り出し,推薦する.また,コミュニケーションタイプについて,各カテゴリより細かい単位で補足ルールを作成し,入力文の推定結果が補足ルールに当てはまる場合は,補足ルールも用いて顔文字推薦を行う.

構築した顔文字推薦システムの推薦の正確さを評価するために,被験者実験を行った.その結果,66.6\%の顔文字が文章に対して適切に推薦されており,感情カテゴリのみの推定に基づく推薦に比べて,有意に向上していることを明らかにした.

今後は,活用形も考慮した手がかり語辞書の拡張や,極性を考慮した顔文字データベースの構築による顔文字推薦の精度向上に取り組む予定である.また,感情,コミュニケーション・動作タイプなどのカテゴリがどのような理由により発生するか(例:嬉し涙,嫌なニュースに驚いた)を手がかりとすることで,さらに文章に適した顔文字推薦ができる可能性があると考えている.このような,テキスト含意関係などを考慮した取り組みについても検討を進めていきたい.

\acknowledgment

{\it CAO}システムによる顔文字の解析結果を提供してくださった,北海学園大学工学研究所の Michal PTASZYNSKI さんに感謝いたします.また,アンケートや被験者実験にご協力いただいた皆様に感謝いたします.

本研究の一部は,科学研究費補助金基盤研究C(課題番号 24500291)ならびに筑波大学図書館情報メディア系プロジェクト研究の助成を受けて遂行された.



\bibliographystyle{jnlpbbl_1.5}
\begin{thebibliography}{}

\bibitem[\protect\BCAY{{Brody} \BBA\ {Diakopoulos}}{{Brody} \BBA\
  {Diakopoulos}}{2011}]{brody2011}
{Brody}, S.\BBACOMMA\ \BBA\ {Diakopoulos}, N. \BBOP 2011\BBCP.
\newblock \BBOQ Cooooooooooooooollllllllllllll!!!!!!!!!!!!!! Using Word
  Lengthening to Detect Sentiment in Microblogs.\BBCQ\
\newblock In {\Bem Proceedings of the 2011 Conference on Empirical Methods in
  Natural Language Processing}, \mbox{\BPGS\ 562--570}, Edinburgh, Scotland,
  UK.

\bibitem[\protect\BCAY{{Derks}, {Bos}, \BBA\ von {Grumbkow}}{{Derks}
  et~al.}{2007}]{derks2007}
{Derks}, D., {Bos}, A. E.~R., \BBA\ von {Grumbkow}, J. \BBOP 2007\BBCP.
\newblock \BBOQ Emoticons and Social Interaction on the Internet: The
  Importance of Social Context.\BBCQ\
\newblock {\Bem Computer in Human Behavior}, {\Bbf 23}, \mbox{\BPGS\ 842--849}.

\bibitem[\protect\BCAY{橋本}{橋本}{2011}]{hashi}
橋本泰一 \BBOP 2011\BBCP.
\newblock Twitter への絵文字自動挿入システム.\
\newblock \Jem{言語処理学会第17回年次大会発表論文集}, \mbox{\BPGS\ 1151--1154}.

\bibitem[\protect\BCAY{株式会社きざしカンパニー}{株式会社きざしカンパニー}{201
1}]{kizashi}
株式会社きざしカンパニー \BBOP 2011\BBCP.
\newblock kizashi.jp:きざしラボ.\ \\http://kizasi.jp/labo/lets/wish.html.

\bibitem[\protect\BCAY{加藤\JBA 加藤\JBA 島峯\JBA 柳沢}{加藤 \Jetal
  }{2008}]{kato}
加藤尚吾\JBA 加藤由樹\JBA 島峯ゆり\JBA 柳沢昌義 \BBOP 2008\BBCP.
\newblock
  携帯メールコミュニケーションにおける顔文字の機能に関する分析—相手との親しさ
の程度による影響の検討—.\
\newblock \Jem{日本教育情報学会学会誌}, {\Bbf 24}  (2), \mbox{\BPGS\ 47--55}.

\bibitem[\protect\BCAY{川上}{川上}{2008}]{kawa}
川上正浩 \BBOP 2008\BBCP.
\newblock 顔文字が表す感情と強調に関するデータベース.\
\newblock \Jem{大阪樟蔭女子大学人間科学研究紀要}, {\Bbf 7}, \mbox{\BPGS\
  67--82}.

\bibitem[\protect\BCAY{河野}{河野}{2003}]{kono}
河野道子 \BBOP 2003\BBCP.
\newblock
  フェイスマークを用いた感情表現におけるコミュニケーション・ギャップに関する研
究.\ http://www.sonoda-u.ac.jp/dic/kenkyu/2003/14.pdf.

\bibitem[\protect\BCAY{菊井}{菊井}{2012}]{kikui2012}
菊井玄一郎 \BBOP 2012\BBCP.
\newblock なにをつぶやいているのか?:マイクロブログの機能的分類の試み.\
\newblock \Jem{言語処理学会第18 回年次大会発表論文集}, \mbox{\BPGS\ 759--762}.

\bibitem[\protect\BCAY{{Landis} \BBA\ {Koch}}{{Landis} \BBA\
  {Koch}}{1977}]{landis1977}
{Landis}, J.~R.\BBACOMMA\ \BBA\ {Koch}, G.~G. \BBOP 1977\BBCP.
\newblock \BBOQ {The Measurement of Observer Agreement for Categorical
  Data}.\BBCQ\
\newblock {\Bem Biometrics}, {\Bbf 33}, \mbox{\BPGS\ 159--174}.

\bibitem[\protect\BCAY{村上\JBA 山田\JBA 萩原}{村上 \Jetal }{2011}]{mura}
村上浩司\JBA 山田薫\JBA 萩原正人 \BBOP 2011\BBCP.
\newblock 顔文字情報と文の評価表現の関連性についての一考察.\
\newblock \Jem{言語処理学会第17回年次大会発表論文集}, \mbox{\BPGS\ 1155--1158}.

\bibitem[\protect\BCAY{中村}{中村}{1993}]{naka}
中村明 \BBOP 1993\BBCP.
\newblock \Jem{感情表現辞典} (第1 \JEd).
\newblock 東京堂出版.

\bibitem[\protect\BCAY{中村\JBA 池田\JBA 乾\JBA 小谷}{中村 \Jetal
  }{2003}]{nakamura2003b}
中村純平\JBA 池田剛\JBA 乾伸雄\JBA 小谷善行 \BBOP 2003\BBCP.
\newblock 対話システムにおける顔文字の学習.\
\newblock \Jem{情報処理学会, 第154回自然言語処理研究会}, \mbox{\BPGS\
  169--176}.

\bibitem[\protect\BCAY{{Ortony}, {Clore}, \BBA\ {Collins}}{{Ortony}
  et~al.}{1990}]{ortony1990}
{Ortony}, A., {Clore}, G., \BBA\ {Collins}, A. \BBOP 1990\BBCP.
\newblock {\Bem The Cognitive Structure of Emotions}.
\newblock Cambridge University Press.

\bibitem[\protect\BCAY{{Ptaszynski}}{{Ptaszynski}}{2012}]{ptas2012}
{Ptaszynski}, M. \BBOP 2012\BBCP.
\newblock
  顔文字処理—取るに足らない表現をコンピュータに理解させるに足るには—.\
\newblock \Jem{情報処理}, {\Bbf 53}  (3), \mbox{\BPGS\ 204--210}.

\bibitem[\protect\BCAY{{Ptaszynski}, {Maciejewski}, {Dybala}, {Rzepka}, \BBA\
  {Araki}}{{Ptaszynski} et~al.}{2010}]{ptas2010b}
{Ptaszynski}, M., {Maciejewski}, J., {Dybala}, P., {Rzepka}, R., \BBA\ {Araki},
  K. \BBOP 2010\BBCP.
\newblock \BBOQ {CAO: A Fully Automatic Emotion Analysis System Based on Theory
  of Kinesics}.\BBCQ\
\newblock {\Bem IEEE Transactions On Affective Computing}, {\Bbf 1}  (1),
  \mbox{\BPGS\ 46--59}.

\bibitem[\protect\BCAY{{Purver} \BBA\ {Battersby}}{{Purver} \BBA\
  {Battersby}}{2012}]{purver2012}
{Purver}, M.\BBACOMMA\ \BBA\ {Battersby}, S. \BBOP 2012\BBCP.
\newblock \BBOQ {Experimenting with Distant Supervision for Emotion
  Classification}.\BBCQ\
\newblock In {\Bem Proceedings of the 13th Eurpoean Chapter of the Association
  for Computational Linguistics}, \mbox{\BPGS\ 482--491}, Avignon, France.

\bibitem[\protect\BCAY{篠山\JBA 松尾}{篠山\JBA 松尾}{2010}]{shino}
篠山学\JBA 松尾朋子 \BBOP 2010\BBCP.
\newblock 顔文字を考慮した対話テキストの感情推定に関する研究.\
\newblock \Jem{香川高等専門学校研究紀要}, {\Bbf 1}, \mbox{\BPGS\ 51--53}.

\bibitem[\protect\BCAY{{Suzuki} \BBA\ {Tsuda}}{{Suzuki} \BBA\
  {Tsuda}}{2006}]{Suzuki2006}
{Suzuki}, N.\BBACOMMA\ \BBA\ {Tsuda}, K. \BBOP 2006\BBCP.
\newblock \BBOQ {Express Emoticons Choice Method for Smooth Communication of
  e-Business}.\BBCQ\
\newblock In {\Bem {Knowledge-Based Intelligent Information and Engineering
  Systems}}, \lowercase{\BVOL}\ 4252 of {\Bem Lecture Notes in Computer
  Science}, \mbox{\BPGS\ 296--302}.

\bibitem[\protect\BCAY{高村\JBA 乾\JBA 奥村}{高村 \Jetal }{2006}]{taka}
高村大也\JBA 乾孝司\JBA 奥村学 \BBOP 2006\BBCP.
\newblock スピンモデルによる単語の感情極性抽出.\
\newblock \Jem{情報処理学会論文誌}, {\Bbf 47}  (2), \mbox{\BPGS\ 627--637}.

\bibitem[\protect\BCAY{徳久\JBA 乾\JBA 松本}{徳久 \Jetal }{2009}]{toku}
徳久良子\JBA 乾健太郎\JBA 松本裕治 \BBOP 2009\BBCP.
\newblock Webから獲得した感情生起要因コーパスに基づく感情推定.\
\newblock \Jem{情報処理学会論文誌}, {\Bbf 50}  (4), \mbox{\BPGS\ 1365--1374}.

\bibitem[\protect\BCAY{{山口}\JBA {城}}{{山口}\JBA {城}}{2000}]{yamaguchi2000}
{山口}英彦\JBA {城}仁士 \BBOP 2000\BBCP.
\newblock 電子コミュニティにおけるエモティコンの役割.\
\newblock \Jem{神戸大学 発達科学部 研究紀要}, {\Bbf 8}  (1), \mbox{\BPGS\
  131--145}.

\end{thebibliography}


\begin{biography}
\bioauthor{江村 優花}{
2012年 筑波大学情報学群知識情報・図書館学類卒業.現在,フコク情報システム株式会社所属.
}

\bioauthor{関  洋平}{
1996年 慶應義塾大学大学院理工学研究科計算機科学専攻修士課程修了.2005年 総合研究大学院大学情報学専攻博士後期課程修了.博士(情報学).同年豊橋技術科学大学工学部情報工学系助手.2008年 コロンビア大学コンピュータサイエンス学科客員研究員.2010年 筑波大学図書館情報メディア系助教,現在に至る.自然言語処理,意見分析,情報アクセスの研究に従事.ACM, ACL, 情報処理学会,電子情報通信学会,言語処理学会,日本データベース学会各会員.
}

\end{biography}


\biodate



\end{document}
