\documentstyle[epsbox,jnlpbbl]{jnlp_j_b5}

\setcounter{page}{55}
\setcounter{巻数}{10}
\setcounter{号数}{5}
\setcounter{年}{2003}
\setcounter{月}{10}
\受付{2003}{1}{15}
\再受付{2003}{3}{16}
\採録{2003}{4}{20}

\newcounter{sentcounter}
\newenvironment{SENT2}{}{}

\newenvironment{JSENT}{}{}


\title{人間による翻訳文と機械翻訳文の語彙的差異の計量分析}
\author{吉見 毅彦\affiref{Ryukoku}}

\headauthor{吉見}
\headtitle{人間による翻訳文と機械翻訳文の語彙的差異の計量分析}

\affilabel{Ryukoku}{龍谷大学理工学部情報メディア学科}
{Department of Media Informatics, Faculty of Science and Technology, 
Ryukoku University}

\jabstract{本稿では,ニュース記事から無作為抽出した英文を英日機械翻訳シ
ステムで翻訳した結果と,これらの英文を人間が翻訳した結果を照らし合わせ,
両者の間にどのような違いがあるのかを計量的に分析した.
その結果,次のような量的な傾向があることが明らかになった.
(1) 人間による翻訳に比べ,システムによる翻訳では,英文一文が複数の訳文に
分割されにくい傾向が見られる.
(2) システムによる翻訳と人間による翻訳の間で訳文の長さの分布に統計的有意
差が認められる.
(3) 用言の連用形と連体形の分布に有意差が認められ,システムによる翻訳の
ほうが人間による翻訳よりも複雑な構造をした文が多いことが示唆される.
(4) 体言と用言の分布には有意差は認められない.\\
さらに,動詞と名詞に関して比較検討を行ない,システムによる翻訳を人間によ
る翻訳に近づけるために解決すべき課題をいくつか指摘した.}

\jkeywords{機械翻訳,人間による翻訳,語彙的差異,比較分析,計量分析} 

\etitle{Quantitative Analysis of \\
Morpholexical Difference between \\
Human-Translated and Machine-Translated \\
Sentences}

\eauthor{Takehiko Yoshimi\affiref{Ryukoku}} 

\eabstract{This paper carries out a quantitative analysis of 
morpholexical difference between machine-translated Japanese sentences 
and human-translated ones, both of which are obtained from English 
sentences selected randomly from news articles.
The analysis gives the following results.
(1) A tendency to translate one English sentence into multiple Japanese 
sentences is less observed in machine translation than in human 
translation.
(2) Significant difference exists in the distribution of the sentence 
length between machine- and human-translated sentences.
(3) Significant difference in the distribution of the adverbial form and 
the attributive form of verbs and adjectives intimates that 
machine-translated sentences have more complex syntactic structure than 
human-translated ones do.
(4) No significant difference exists in the distribution between verbs, 
adjectives and nouns. \\
A further investigation on verbs and nouns reveals what kind 
of technical challenges must be solved to improve the quality of machine 
translation up to the extent of human translation.}

\ekeywords{Machine Translation, Human Translation, 
Morpholexical Difference, Comparative Analysis, Quantitative Analysis}

\begin{document}

\maketitle
\thispagestyle{empty}

\section{はじめに}
\label{sec:intro}

1980年代に市販され始めた機械翻訳システムはその後改良が重ねられ,
システムの翻訳品質は確実に向上してきている.
しかし,現状のシステムには解決すべき課題が数多く残されており,
高品質の翻訳が可能なシステムは未だ実現されていない. 

翻訳品質を高めるためにシステムを評価改良していく方法としては,
(1) システムの新バージョンによる訳文と旧バージョンによる訳文との比較や,
異なるシステム間での比較によって行なう方法\cite{Niessen00,Darwin01}と,
(2) システムによる訳文と人間による訳文を比較することによって行なう方法
\cite{Sugaya01,Papineni02}
がある.
前者の方法では,システムによる翻訳(以降,MT訳と呼ぶ)と人間による翻訳(人
間訳)を比較することによって初めて明らかになる課題が見逃されてしまう恐れ
がある.
これに対して,後者の方法では,MT訳と人間訳の間にどのような違いがあるのか
を発見し,その違いを埋めていくために取り組むべき課題を明らかにすることが
できる.

このように,MT訳と人間訳の比較によるシステムの評価改良は有用な方法である.
しかしながら,MT訳と人間訳の違いを明らかにするために両者の比較分析を計量
的に行なった研究は,従来あまり見られない.

ところで,人間によって書かれた文章間の比較分析は,文体論研究の分野におい
て以前から行なわれてきている\cite{Yamaguchi79}.
文体論研究の目的は,比較対象の文章の個別的あるいは類型的特徴を明らかにす
ることにある.
文体論研究は,文章に対する直観的な印象を重視する立場と,文章が持つ客観的
なデータ(文長や品詞比率など)を主に扱う計量的立場\cite{Hatano65}に分ける
ことができる.
また,別の観点からは,言語表現の特徴を作家の性格や世界観に結び付けて扱う
心理学的文体論と,言語表現の特徴を記述するに留める語学的文体論
\cite{Kabashima63}に分けられる. 
計量的・語学的文体論に分類される研究のうち,同一情報源に基づく内容を伝
える文章を比較対象とした研究として,文献\cite{Horikawa79,Hasumi91}など
がある.
堀川は,四コマ漫画の内容を説明する文章を童話作家,小説家,学者に書いても
らい,それらの違いを分析している.
蓮見は,古典の源氏物語を複数の翻訳者が現代語に翻訳した文章において,文数,
文長,品詞比率などを比較分析している.

本研究では,英日機械翻訳システムの翻訳品質の向上を目指し,その第一歩とし
て,英文ニュース記事に対する人間訳とMT訳を比較し,それらの違いを計量的に
分析する.
人間訳とMT訳の違いは多岐にわたるため様々な観点から分析を行なう必要がある
が,本稿では,英文一文に対する訳文の数,訳文の長さ,文節レベルの現象につ
いて量的な傾向を明らかにする.

なお,特にMT訳には誤訳の問題があるが,本研究は,訳文の意味内容ではなく
訳文の表現形式について分析するものである.
すなわち,翻訳の評価尺度として忠実度と理解容易性\cite{Nagao85}を考えた場
合,後者について,MT訳の分かりにくさ,不自然さの原因がどこにあるのかを人
間訳とMT訳を比較することによって明らかにしていくことが本研究の目的である.

以下,\ref{sec:method}\,節で人間訳とMT訳の比較分析方法について述べ,
\ref{sec:result}\,節で分析結果を示し,考察を加える.
最後に\ref{sec:conc}\,節で今回の比較分析で明らかになった点をまとめる.


\section{調査方法}
\label{sec:method}

\subsection{調査対象とした標本}
\label{sec:method:corpus}

コーパスとしてBilingual Net News
\footnote{http://www.bnn-japan.com/}
の英文ニュース記事を使用し,2001年5月26日から2002年1月15日までの記事を構
成する英文5592文を母集団とした.
これらの英文にはあらかじめ人間訳が与えられている.

この母集団から乱数によって500文を単純無作為抽出した.
この500文を市販されている代表的な機械翻訳システムの一つ
で翻訳し,入力文全体を覆う構文構造が得られなかった文と,
一文の認識に失敗した文
\footnote{例えば,`Bush backs the more comprehensive ban sponsored by 
Rep. David Weldon (R) of Florida.'という文は,`Rep.'のところで一つの文が
完結し,`David'のところから新たな文が始まると認識されてしまう.},
合わせて24文を除いた476文を標本とした.

これらの英文476文の文長の平均値は,11.5であり,標準偏差は6.2であった.
文長の計測は,冠詞や前置詞などの機能語と,ピリオドや引用符などの記号を除
き,内容語の数で行なった.

また,訳文の数を地の文の句点の数で数えた場合,英文476文に対する訳文数は,
人間訳では559文であり,MT訳では519文であった.


\subsection{形態素解析と文節解析}
\label{sec:method:parse}

人間訳とMT訳の各文に対して,形態素解析と文節解析を行なった.
これらの処理には,形態素解析システム茶筅
\footnote{http://chasen.aist-nara.ac.jp/chasen/}と
係り受け解析システム南瓜
\footnote{http://cl.aist-nara.ac.jp/\symbol{126}taku-ku/software/cabocha/}
をデフォルトの状態で利用した. 


\subsection{解析結果の変更}
\label{sec:method:modify}

茶筅と南瓜による解析結果に対して,必要な変更を人手で加えた.
品詞タグ付け誤りの修正以外の主な変更点は次の通りである.
なお,変更前の文節区切りを記号`$|$'で,変更後の文節区切りを記号`/'で表わ
す.  
また,茶筅での品詞タグを二重引用符で括って示す.
\begin{enumerate}
\item
``未知語''は名詞に変更する.
\item
南瓜による解析では``名詞-非自立''が自立語として扱われている場合が
ある(「難色を$|$示した$|$ため」など)が,これを付属語とみなす(「難色
を/示したため」).
\item 
「する」,「できる」が``名詞-サ変接続''と結合しているとき,全体をサ変動
詞とする.
\item 
「だ」,「である」,「です」が``名詞-形容動詞語幹''と結合しているとき,
全体を形容動詞とする.
\item
「だ」,「である」,「です」が``名詞-形容動詞語幹''以外の名詞と結合して
述語を構成しているとき,これらを判定詞とする.
\item 
「に$|$よる」,「に$|$基づく」などは,原則として,全体を助詞とみなす.
\item 
``記号''は,原則として,``接頭詞''または付属語と同様に扱う.
\end{enumerate}


\subsection{調査分析項目}
\label{sec:method:items-checked}

まず,概略的な調査として,英文一文に対する訳文の数,訳文の長さ,訳文に含
まれる連体修飾節の数について人間訳とMT訳を比較する.
これらの調査は,人間訳とMT訳の複雑さに違いがあるかを明らかにするためのも
のである.

次に,体言と用言の分布に人間訳とMT訳で違いがあるかを調査する.
この調査の目的は,名詞を中心として展開していく英語の構造が人間訳とMT訳で
それぞれどのように訳されているかを明らかにすることにある.
なお,本稿では,動詞,形容詞,形容動詞の他に判定詞も用言とみなす.

さらに,動詞,名詞,代名詞について若干詳細な比較分析を行なう.
これらの品詞に着目したのは,動詞と名詞(代名詞)は,出現比率が高く,これら
を適切に処理することが特に重要な課題になっているからである.


\section{結果と考察}
\label{sec:result}


\subsection{英文一文に対する訳文数}
\label{sec:result:num-of-sent}

自然な翻訳では,英文一文が訳文一文に対応しているとは限らず,複数の文に
分けて訳されることも少なくない.
そこで,まず,人間訳とMT訳それぞれにおいて訳文数が$n$文である英文の数を
$n$毎に集計した.
その結果を表\ref{tab:ej-sent-num}\,に示す.
表\ref{tab:ej-sent-num}\,において,上段が頻度,下段が比率である.
\begin{table}[htbp]
\caption{訳文数が$n$文である英文の数}
\label{tab:ej-sent-num}
\begin{center}
\begin{tabular}{|c||r|r|r|r|}\hline
$n$    & \multicolumn{1}{c|}{1} & \multicolumn{1}{c|}{2} & 
\multicolumn{1}{c|}{3} & \multicolumn{1}{c|}{4} \\\hline\hline
 & 403 & 64 & 8 & 1 \\
\raisebox{1.5ex}[0pt]{人間訳} & 84.7\,\% & 13.4\,\% & 1.7\,\% & 0.2\,\% \\\hline
 & 433 & 43 & 0 & 0 \\
\raisebox{1.5ex}[0pt]{MT訳} & 91.0\,\% & 9.0\,\% & 0.0\,\% & 0.0\,\%\\\hline
\end{tabular}
\end{center}
\end{table}

表\ref{tab:ej-sent-num}\,を見ると,人間訳では英文一文が三文以上に分けて
訳されていることもあるのに対して,MT訳では高々二文にしか分割されていない
\footnote{MT訳での文の分割は,機械翻訳システムが正常に動作した結果である.
\ref{sec:method:corpus}\,節で述べたように,入力文全体を覆う構文構造が得
られないために分割された文は標本に含まれていない.}.
また,MT訳では,英文一文が二文に訳されることは比較的少なく,一文に訳され
ることが多い.
人間訳とMT訳における訳文数の分布の差は,有意水準5\,\%で統計的に有意である.
これらのことから,MT訳には,人間訳に比べて,複数の文に分割されにくい傾向
があると言える.

表\ref{tab:ej-sent-num}\,から,英文一文に対する訳文数の分布に人間訳とMT
訳とで差があることが分かったが,さらに詳細な分析を行なうために,
人間訳での訳文数$n\,(n=1,2,3,4)$とMT訳での訳文数$m\,(m=1,2)$との対応関係
を調査した.
その結果を表\ref{tab:ej-sent-num-hum-mt}\,に示す.
\begin{table}[htbp]
\caption{人間訳とMT訳の訳文数の対応}
\label{tab:ej-sent-num-hum-mt}
\begin{center}
\begin{tabular}{|c||r|r|r|r|}\hline
MT訳($m$)$\backslash$人間訳($n$) & \multicolumn{1}{c|}{1} & \multicolumn{1}{c|}{2} & 
\multicolumn{1}{c|}{3} & \multicolumn{1}{c|}{4} \\\hline\hline
 & 371 & 55 & 7 & 0 \\
\raisebox{1.5ex}[0pt]{1}
 & 77.9\,\% & 11.6\,\% & 1.5\,\% & 0.0\,\% \\\hline
 &  32 &  9 & 1 & 1 \\
\raisebox{1.5ex}[0pt]{2}
 & 6.7\,\% & 1.9\,\% & 0.2\,\% & 0.2\,\% \\\hline
\end{tabular}
\end{center}
\end{table}

表\ref{tab:ej-sent-num-hum-mt}\,によれば,全体的な傾向として,
人間訳とMT訳とで訳文数が等しい場合が380件(79.8\,\%)あり,
人間訳とMT訳とで訳文数が異なる場合が96件(20.2\,\%)ある.

表\ref{tab:ej-sent-num-hum-mt}\,では,
英文一文が人間訳では一文に訳されているのに対してMT訳では二文に分けて訳さ
れている場合が32件ある.
この32件について,英文のどのような表現のところで分割されているかを調べた.
その結果を表\ref{tab:ej-sent-div-mt}\,に示す.
\begin{table}[htbp]
\caption{MT訳のみでの訳文分割箇所}
\label{tab:ej-sent-div-mt}
\begin{center}
\begin{tabular}{|l|r@{}r|}\hline
\multicolumn{1}{|c|}{分割箇所} & \multicolumn{2}{c|}{頻度} \\\hline\hline
従属接続詞(名詞節) & 14 & (43.8\,\%) \\
等位接続詞          &  9 & (28.1\,\%) \\
従属接続詞(副詞節) &  6 & (18.8\,\%) \\
現在分詞            &  1 & (3.1\,\%) \\
関係副詞            &  1 & (3.1\,\%) \\
コロン              &  1 & (3.1\,\%) \\\hline
\end{tabular}
\end{center}
\end{table}

表\ref{tab:ej-sent-div-mt}\,を見ると,分割が生じる表現のほとんどは,
名詞節を導く従属接続詞(`that'など),等位接続詞(`and'など),副詞節を導く
従属接続詞(`because'など)で占められている.
なお,名詞節を導く従属接続詞の集計には,従属接続詞が省略されている場合も
含めている.

逆に,英文一文が人間訳では二文に分けて訳されているのに対してMT訳では一文
に訳されている55件について,英文のどのような表現のところで分割されている
かを調べた結果を表\ref{tab:ej-sent-div-hum}\,に示す.
\begin{table}[htbp]
\caption{人間訳のみでの訳文分割箇所}
\label{tab:ej-sent-div-hum}
\begin{center}
\begin{tabular}{|l|r@{}r|}\hline
\multicolumn{1}{|c|}{分割箇所} & \multicolumn{2}{c|}{頻度} \\\hline\hline
関係代名詞          & 14 & (25.5\,\%) \\
前置詞              & 10 & (18.2\,\%) \\
等位接続詞          &  7 & (12.7\,\%) \\
従属接続詞(副詞節) &  7 & (12.7\,\%) \\
現在分詞            &  6 & (10.9\,\%) \\
従属接続詞(名詞節) &  4 & (7.3\,\%) \\
to不定詞            &  2 & (3.6\,\%) \\
関係副詞            &  2 & (3.6\,\%) \\
その他              &  3 & (5.5\,\%) \\\hline
\end{tabular}
\end{center}
\end{table}

表\ref{tab:ej-sent-div-hum}\,を見ると,MT訳に見られない分割箇所として
関係代名詞や前置詞が目立つ.
次の文(H\ref{SENT:ej-sent-div-rel})のように,文を関係代名詞のところ
で二分割し,それらの間を同一名詞の繰り返しによってつなぐ手法は,よく知ら
れている英日翻訳技法の一つである\cite{Anzai82,Kondo92,Kamei94}. 
なお,本稿では,英文,人間訳,MT訳をそれぞれ(E\ref{SENT:ej-sent-div-rel}),
(H\ref{SENT:ej-sent-div-rel}),(M\ref{SENT:ej-sent-div-rel})のように参照
する.
英文と人間訳はすべてBilingual Net Newsからの引用である.
記事の掲載年月日を英文の後ろに示す.
\begin{SENT2}
\sentE House and Senate negotiators quickly began hammering out a final 
compromise, {\it which} Republicans hoped to present for Bush's 
signature as soon as today. [2001年5月26日]
\sentH 下院と上院の交渉者たちはすぐに最終的な{\bf 妥協案}を成立させること
にとりかかった.
共和党は今日にも{\bf 妥協案}をまとめて提出しブッシュの署名を得たいと望ん
でいる. 
\sentM 下院,そして,上院交渉者は,迅速に最終の妥協(共和党員がブッシュの
サインのために今日と同じくらいすぐに提示することを望んだ)を打ち出し始め
た. 
\label{SENT:ej-sent-div-rel}
\end{SENT2}

表\ref{tab:ej-sent-div-hum}\,は,このような手法が調査対象のシステムに取
り入れられていないことを示している.
関係節を伴う名詞句を機械翻訳システムにおいて適切に処理するための自動前編
集手法が文献\cite{KatoTerumasa97,Saraki01}などで報告されていたり,
今回の調査に用いたシステムとは異なる別の市販システムでは関係代名詞が先行
詞に置き換えられることがあったりするが,
人間が行なう関係節の翻訳は,柔軟性に富み,かつ,様々な工夫がなされている
ため,関係節に関する翻訳規則をより高度化していく必要がある.

人間訳では,前置詞のところで文を単に分割するだけでなく,自然な翻訳になる
ように工夫が施されている.
その一つは,前置詞句が一つの完全な文になるように翻訳されている点である
\cite{Nakamura82}.
例えば次の文(H\ref{SENT:ej-sent-div-prep})では,「のものである」が補
われている.
また別の工夫として,二つの文を滑らかにつなぐために,「これは」という照応
表現が補われている.
\begin{SENT2}
\sentE In an epic tennis match, Pete Sampras edged Andre Agassi to reach 
the US Open semi-finals {\it after} a dramatic confrontation in which 
neither legend lost a service game. [2001年9月8日]
\sentH 叙事詩のようにすごいテニスの戦いで,ピート・サンプラスはアンドレ・
アガシに競り勝ち,全米オープン準決勝に進んだ.
{\bf これは,}どちらの伝説的選手もサービスゲームを落とさないという,劇的
な対決の末{\bf のものである}.
\sentM 叙事詩のテニスの試合において,ピート・サンプラスは,劇的な直面(伝
説のいずれもサービスゲームに負けなかった)の後で全米オープン準決勝に達す
るために,アンドレ・アガシを研いだ.
\label{SENT:ej-sent-div-prep}
\end{SENT2}


\subsection{訳文の長さ}
\label{sec:result:sent-len}

文の長さは,
文章の類型設定に関する心理学的研究\cite{Hatano65}や,
文章の難易度の測定\cite{Morioka88}, 
手書き文章とワープロ書き文章の比較分析\cite{Jim93}など
様々な研究において,比較尺度として用いられている.

本節では,文長を,訳文の複雑さを近似的に測るための尺度の一つとして用いる.
日本語の文の長さを測る単位としては,文字,単語,文節などが考えられるが,
ここでは文節数(自立語数)を計測単位とする.

人間訳とMT訳それぞれにおける文長の度数分布表を
表\ref{tab:sent-length-hum}\,と表\ref{tab:sent-length-mt}\,に示す.
表\ref{tab:sent-length-hum}\,と表\ref{tab:sent-length-mt}\,の累積比率を
比べると,人間訳では文長14までの文が全体の85.8\,\%を占めるのに対して,MT訳
では73.8\,\%しかないことなどから,人間訳よりMT訳のほうが長い文が多いと言え
る.
Wilcoxonの順位和検定の結果,人間訳とMT訳の文長の分布の間には有意水準5\,\%で
有意差が認められた.
\begin{table}[htbp]
\caption{人間訳における文長の分布}
\label{tab:sent-length-hum}
\begin{center}
\begin{tabular}{|r||r|r|r|r|}\hline
\multicolumn{1}{|c||}{文長} & \multicolumn{1}{c|}{度数} & 
\multicolumn{1}{c|}{比率} & \multicolumn{1}{c|}{累積度数} & 
\multicolumn{1}{c|}{累積比率} \\\hline\hline
   1〜2 & 47 &  8.4\,\% &  47 &  8.4\,\% \\
   3〜4 & 65 & 11.6\,\% & 112 & 20.0\,\% \\
   5〜6 & 62 & 11.1\,\% & 174 & 31.1\,\% \\
   7〜8 & 89 & 15.9\,\% & 263 & 47.0\,\% \\
  9〜10 & 82 & 14.7\,\% & 345 & 61.7\,\% \\
 11〜12 & 81 & 14.5\,\% & 426 & 76.2\,\% \\
 13〜14 & 57 & 10.2\,\% & 483 & 86.4\,\% \\
 15〜16 & 31 &  5.6\,\% & 514 & 92.0\,\% \\
 17〜18 & 22 &  3.9\,\% & 536 & 95.9\,\% \\
 19〜20 & 12 &  2.2\,\% & 548 & 98.1\,\% \\
 21〜22 &  3 &  0.5\,\% & 551 & 98.6\,\% \\
 23〜24 &  3 &  0.5\,\% & 554 & 99.1\,\% \\
 25〜26 &  4 &  0.7\,\% & 558 & 99.8\,\% \\
 27〜28 &  1 &  0.2\,\% & 559 & 100.0\,\% \\\hline
\hline
\end{tabular}
\end{center}
\end{table}
\begin{table}[htbp]
\caption{MT訳における文長の分布}
\label{tab:sent-length-mt}
\begin{center}
\begin{tabular}{|r||r|r|r|r|}\hline
\multicolumn{1}{|c||}{文長} & \multicolumn{1}{c|}{度数} & 
\multicolumn{1}{c|}{比率} & \multicolumn{1}{c|}{累積度数} & 
\multicolumn{1}{c|}{累積比率} \\\hline\hline
   1〜2 & 47 &  9.0\,\% &  47 &  9.0\,\% \\
   3〜4 & 56 & 10.8\,\% & 103 & 19.8\,\% \\
   5〜6 & 60 & 11.5\,\% & 163 & 31.3\,\% \\
   7〜8 & 57 & 11.0\,\% & 220 & 42.3\,\% \\
  9〜10 & 46 &  8.9\,\% & 266 & 51.2\,\% \\
 11〜12 & 68 & 13.1\,\% & 334 & 64.3\,\% \\
 13〜14 & 51 &  9.8\,\% & 385 & 74.1\,\% \\
 15〜16 & 50 &  9.6\,\% & 435 & 83.7\,\% \\
 17〜18 & 32 &  6.2\,\% & 467 & 89.9\,\% \\
 19〜20 & 27 &  5.2\,\% & 494 & 95.1\,\% \\
 21〜22 & 11 &  2.1\,\% & 505 & 97.2\,\% \\
 23〜24 &  4 &  0.8\,\% & 509 & 98.0\,\% \\
 25〜26 &  6 &  1.2\,\% & 515 & 99.2\,\% \\
 27〜28 &  4 &  0.8\,\% & 519 & 100.0\,\% \\\hline
\end{tabular}
\end{center}
\end{table}

文長の平均値は人間訳で9.25,MT訳で10.42,標準偏差は人間訳で4.94,MT訳で
5.95と,平均値,標準偏差ともMT訳のほうが若干大きい.

MT訳での文長の最頻値は階級では11〜12であるが,観測値では2である.
MT訳での最頻値が2となるのは,
次の文(M\ref{SENT:ej-sent-div-say})のように,伝達文の主節だけを独立した文
として訳している場合が多いためである.
\begin{SENT2}
\sentE Hastert said both sides are trying to strike a compromise. [2001年7月28日]
\sentH ハスタートは,両党とも妥協を試みていると語っている.
\sentM Hastertは言った.両側は,妥協を打とうとしていると.
\label{SENT:ej-sent-div-say}
\end{SENT2}


\subsection{連体修飾節の数}
\label{sec:result:rentai}

文の長さは文の複雑さを測る尺度の一つとなりえるが,それだけでは十分な近似
ではない.
例えば次の二つの文は文長(文節数)は等しいが,文構造としては連体修飾節を含
む文(J\ref{SENT:renyo-rentai}')のほうが複雑である
\footnote{連体修飾節は,複雑な文を簡単な文に書き換える際の書き換え対象候
補の一つとなっている\cite{Nogami00}が,これは,連体修飾節を含む
文(J\ref{SENT:renyo-rentai}')のような文が複雑になる傾向があるからであろ
う.}.
\begin{JSENT}
\sentJ シングテルは政府所有の会社で,経営を世界規模に広げて業界大手にな
ることを目指している.
\sentJJ シングテルは,経営を世界規模に広げて業界大手になることを目指して
いる政府所有の会社である.
\label{SENT:renyo-rentai}
\end{JSENT}

文(J\ref{SENT:renyo-rentai})のような文のほうが読み手の負担が軽いことに関
して文献\cite{Yanabu79}では次のように述べられている.
\begin{quote}
読者は,動詞が現れたところで,だいじな意味を語ることばが分り,思考の流れ
はひと区切りつくのである.ひと区切りついた部分は一応前ヘ預けておいて,そ
の先へ読み進んで行ける.
文は,全体としては長いが,読者の頭脳には,この文の長さは決して過重な負担
とはならないのである.
\end{quote}

意味的に一区切りつけることができるのは,用言(動詞,形容詞,形容動詞,
判定詞)の連用形が現れるところである.
文(J\ref{SENT:renyo-rentai})の場合,判定詞「だ」の連用形「で」のところで
意味的なまとまりを認識することができる.
これに対して,文(J\ref{SENT:renyo-rentai}')における「目指している」のよ
うに用言の連体形が現れるところでは,意味的な区切りをつけることができない.
そこで,人間訳とMT訳とで用言の連用形と連体形の分布に差があるかどうかを調
べた.

茶筅の活用形の分類では,終止形と連体形が区別されず,これらの代わりに``基
本形''という一つのタグが与えられている.
このため,``基本形''を含む文節において次の三つの条件のうちいずれかが成り
立つ場合に,``基本形''を連体形とみなし,それ以外の場合には終止形とみなす
ことにした.
\begin{enumerate}
\item\label{enum:katsuyo:matsubi}
``基本形''が文節末尾の形態素である.
\item\label{enum:katsuyo:keishiki}
``基本形''に``名詞-非自立''が後続する. 
\item\label{enum:katsuyo:toten}
``基本形''に``記号-読点''が後続し,その``記号-読点''が文節末尾の形態素で
ある.  
\end{enumerate}

表\ref{tab:yogen-katsuyo}\,に集計結果を示す.
連用形と連体形の分布に差があるかどうかを$\chi^2$検定したところ,有意水準
5\,\%で差があると認められた.
\begin{table}[htbp]
\caption{用言の連用形と連体形の分布}
\label{tab:yogen-katsuyo}
\begin{center}
\begin{tabular}{|c||r|r|}\hline
活用形 & \multicolumn{1}{|c}{連用形} & \multicolumn{1}{|c|}{連体形} \\\hline\hline  
 & 190 & 576 \\
\raisebox{1.5ex}[0pt]{人間訳} & 24.8\,\% & 75.2\,\% \\\hline
 & 146 & 626 \\
\raisebox{1.5ex}[0pt]{MT訳} & 18.9\,\% & 81.1\,\% \\\hline
\end{tabular}
\end{center}
\end{table}

連用形は人間訳でより多く,連体形はMT訳でより多く出現する.
このことから,MT訳のほうが人間訳よりも複雑な文構造をした文が多い可能性が
示唆される.
連体形の数が,長く複雑な連体修飾節の数に直接結びつくわけではないが,ある
程度の傾向を知ることはできる.
より正確な傾向を把握するためには,文中において係り先が未だ決まっていない
文節数を数える\cite{Murata99}などの構文的なレベルでの検証が必要である.


\subsection{体言と用言の分布}
\label{sec:result:pos-ratio}

英語は名詞を中心として展開していく構造であるのに対して,
日本語は用言を中心として展開していく構造である\cite{Yanabu79}.
従って,人間による自然な翻訳では,英語の名詞中心の構造は,日本語の用言中
心の構造に変換されていると考えられる.
他方,英語の名詞的表現を日本語の動詞的表現に変換することは現状の機械翻訳
システムでは容易ではなく,一般的な方法は実現されていない
\footnote{この問題に取り組んだ研究として,文献\cite{Yoshimi01b}などがあ
る.}.
このため,体言の出現比率はMT訳のほうが高くなると予想される.

この点を確認するために,体言(代名詞を含む名詞)と用言(動詞,形容詞,
形容動詞,判定詞)の分布を求めた.
その結果を表\ref{tab:pos-total}\,に示す.
表\ref{tab:pos-total}\,を見ると,体言の比率はMT訳のほうが若干高いが,統
計的には人間訳とMT訳で体言と用言の分布に有意差は認められない(有意水準5\,\%).
\begin{table}[htbp]
\caption{体言と用言の分布}
\label{tab:pos-total}
\begin{center}
\begin{tabular}{|c||r|r|r|r|}\hline
 & \multicolumn{1}{|c}{体言} & \multicolumn{1}{|c|}{用言} \\\hline\hline 
 &   3441 & 1238 \\
\raisebox{1.5ex}[0pt]{人間訳} & 73.5\,\% & 26.5\,\% \\\hline
 &   3513 & 1191 \\
\raisebox{1.5ex}[0pt]{MT訳} & 74.7\,\% & 25.3\,\% \\\hline
\end{tabular}
\end{center}
\end{table}

以下,\ref{sec:result:verb}\,節と\ref{sec:result:noun}\,節で動詞と名
詞についてそれぞれ分析する.


\subsection{動詞に関する分析}
\label{sec:result:verb}

\begin{table}[htbp]
\caption{出現頻度5以上の動詞の一覧}
\label{tab:hifreq-verb}
\begin{center}
\begin{tabular}{|r|p{0.35\textwidth}||r|p{0.35\textwidth}|}\hline
\multicolumn{2}{|c||}{人間訳} & \multicolumn{2}{c|}{MT訳} \\\hline\hline
\multicolumn{1}{|c}{頻度} & \multicolumn{1}{|c||}{訳語(基本形)} & 
\multicolumn{1}{c}{頻度} & \multicolumn{1}{|c|}{訳語(基本形)} \\\hline
74 & なる* & 51 & 言う(*) \\
55 & する* & 34 & する* \\
38 & ある* & 18 & ある* \\
32 & 述べる* & 15 & 発表する* \\
24 & 発表する* & 13 & 持つ* \\
20 & 行う* &  12 & 示す*,なる* \\
13 & 受ける & & \\
11 & いる*,語る & 11 & 拒絶する \\
10 & 求める*,見る*,開く & 10 & 獲得する, 望む \\
9 & 考える* & 9 & 先導する,与える*,使う,考える* \\
8 & 主張する,持つ*,出す & 8 & 予測する,見る*,述べる*,行う*,分かる,
思う* \\ 
7 & 非難する,起きる,示す* & 7 & 報告する,カットする,増加する*,
もたらす,求める* \\
6 & かける,みる,いう(*),得る,向ける & 6 & 独立する,告発する,
確認する,提案する,置く,できる,戻る,含む*,殺す,いる* \\ 
5 & 開始する,死亡する,殺害する,警告する,調査する,拒否する,
減少する,予定する,実現する,増加する*,行なう(*),含む*,支払う,思う*,
なす,出る,認める,報じる,与える* & 5 & 保持する,宣告する,
所有する,サポートする,支持する,提供する,終える,会う,告げる \\\hline  
\end{tabular}
\end{center}
\end{table}

出現頻度が5以上の動詞の基本形の一覧を表\ref{tab:hifreq-verb}\,に示す.
記号`*'が付いている語は,人間訳でもMT訳でも頻度5以上で出現するものである.
また,例えば「言う」と「いう」のように記号`(*)'が付いている語は,それら
を異綴りとみなせば,人間訳でもMT訳でも頻度5以上で出現する語である.

表\ref{tab:hifreq-verb}\,から次のような特徴が読み取れる.
\begin{enumerate}
\item\label{enumerate:suru-naru}
「なる」は人間訳にもMT訳にも現れるが,出現頻度が人間訳では74と高いのに対
して,MT訳では12と比較的低い.
\item\label{enumerate:say}
「言う」の出現頻度がMT訳では51と高いのに対して,人間訳では6と低い.
\item\label{enumerate:func-verb}
 人間訳では,「行(な)う」,「受ける」,「出す」,「かける」などの機
能動詞\cite{Muraki91}が多い.
\end{enumerate}


\subsubsection{「する」的表現と「なる」的表現}

よく知られているように,英語では行為者が対象に能動的に働きかけるという捉
え方がされる傾向が強いのに対して,日本語では物事が自然にある状態になると
いう表現が好まれる\cite{Ikegami81,Anzai83}.
上記の特徴(\ref{enumerate:suru-naru})は,このような英語と日本語の言語的
慣習の違いに現状の機械翻訳システムが対処できていないことを示唆している.
例えば,次の文(H\ref{SENT:suru-naru})では,「法律によって政府は$\cdots$
提出できるようになる」という表現がなされている.
これに対して,文(M\ref{SENT:suru-naru})では,英語表現と同様に,行為者
「規則」が対象「政府」に働きかけるという表現になっている.
\begin{SENT2}
\sentE Rules for such a court also give the government a freer hand to 
introduce evidence. [2001年11月16日]
\sentH さらに,このような裁判所で適応される法律によって,政府は,証拠を
より自由に提出できるようになる.
\sentM 同じくそのような法廷のための規則は,証拠を提出するために,政府に
更に自由な援助を行う. 
\label{SENT:suru-naru}
\end{SENT2}
なお,表\ref{tab:hifreq-verb}\,の集計では「する」が自動詞か他動詞かの区
別や能動態か受動態かの区別をしていないが,
「する」的表現と「なる」的表現の量的な違いを厳密に把握するためにはこれら
の区別を考慮する必要がある.


\subsubsection{`say'の訳し分け}

MT訳で「言う」と訳されているのは`say'である.
「言う」の出現頻度が人間訳で低いのは,`say'が「述べる」や「語る」,「発
表する」などに訳し分けられているためである.
調査対象がニュース記事であるため記者会見での発言が多いが,このような発言
では,`say'を「言う」と訳すより,「述べる」などと訳したほうが適切である
ことによるものであろう.


\subsubsection{機能動詞表現}

機能動詞表現は,例えば「注目を集める」のように,表現全体の実質的意味のほ
とんどを担う名詞と,文法的な機能を果たすだけで動詞本来の意味を持たない動
詞とから構成される表現である\cite{Muraki91}.

機能動詞は同一表現の繰り返しを避けたい場合や受動態にすると不自然になる
場合などに用いられ,これによって表現の豊富さや自然さがもたらされる.
例えば,次の文(H\ref{SENT:func-verb-treat})で用いられれている機能動詞表
現「治療を受ける」と,文(M\ref{SENT:func-verb-treat})で用いられれている
受動態「治療される」を比べると,後者には不自然さが感じられる.
\begin{SENT2}
\sentE Thousands of postal and Capitol workers {\it were being treated} 
with antibiotics as a precaution. [2001年10月19日]
\sentH 数千の郵便や議事堂の労働者は,予防として抗生物質での{\bf 治療を受
けている}. 
\sentM 何千もの郵便の,そして,国会議事堂労働者は,事前対策として抗生物
質で{\bf 治療されつつあった}.
\label{SENT:func-verb-treat}
\end{SENT2}

この点を確認するために,ウェブ検索エンジンGoogleとAltaVistaを用いて,
ウェブ文書における「治療され」の出現頻度と「治療を受け」の出現頻度を比較
したところ,前者の出現頻度がGoogleで1210,AltaVistaで3859であるのに対して, 
後者の出現頻度はGoogleで21900,AltaVistaで34374であった
\footnote{これは2002年8月23日の検索結果である.}.
「治療を受ける」において,助詞「を」は「も」,「は」,「さえ」など他の助
詞との交替が可能であり,さらに,名詞と機能動詞の間に他の語句の挿入が可能
であるが,このような交替や挿入が生じた表現は検索の対象としていないので,
これらを考慮した場合の出現頻度はさらに高くなる.
このことから,「治療される」は「治療を受ける」に比べて用いられにくい表現
であると言えそうである.


\newpage
\subsection{名詞に関する分析}
\label{sec:result:noun}

\begin{table}[htbp]
\caption{出現頻度5以上の名詞の一覧}
\label{tab:hifreq-noun}
\begin{center}
\begin{tabular}{|r|p{0.35\textwidth}||r|p{0.35\textwidth}|}\hline
\multicolumn{2}{|c||}{人間訳} & \multicolumn{2}{c|}{MT訳} \\\hline\hline
\multicolumn{1}{|c}{頻度} & \multicolumn{1}{|c||}{訳語} & 
\multicolumn{1}{c}{頻度} & \multicolumn{1}{|c|}{訳語} \\\hline
 & & 69 & 彼* \\
 & & 65 & それら \\
21 & アメリカ(*) & 53 & 米国* \\
16 & 彼* & 40 & それ* \\
15 & ブッシュ大統領* & 17 & 他*,状態,政府*,会社 \\
14 & これ & 16 & イスラエル* \\
13 & 政府* & 14 & 攻撃,人々* \\
11 & 予定*,大統領 & 11 & アフガニスタン*,最初*,日本*,パレスチナ \\
10 & それ* & 10 & 2人*,取扱い,多く \\
 9 & アフガニスタン*,可能性,ロシア*,女性*,州,今回& 9 & ホーム,ブッシュ,予定* \\
 8 & 米国*,9月11日,日本*,議会*,人々*,以下,他* & 
	8 & ブッシュ大統領*,近く,ロシア*,ジョブ,闘士,最近*,パキスタン*,英国(*),全て,会議 \\
 7 & 選挙*,ワシントン*,従業員,パレスチナ人*,調査,ニューヨーク*,2人*,
イギリス(*),先週*,イスラエル*,人* &  
	7 & リーダ,何千,レイオフ,総裁,都市*,ワシントン*,先週*,ニューヨーク*,5.,6.,あなた \\ 
 6 & 同社,一環,最近*,家,国,上院,今,裁判所,最初*,その他,本社,都市* & 
	6 & テロリスト*,一部,必要,武器,首相,軍人,女性*,1つ,選挙*,パレスチナ人*,準備,2つ,子供*,タリバン,問題*,道,10. \\ 
 5 & アラファト議長,北アイルランド,彼ら,人物,東芝,買収,1人,ホワイトハウス,
テロリスト*,報道,映画,第6位,取引,国連,現在,問題*,パキスタン*,その後,爆弾,家庭,子供*,罪 &
	5 & 人*,ヤセル・アラファト,停戦,イスラム教,結果,連邦,歴史,イラン,ケース,国家,人間,メーカー,4つ,時間,前,戦い,暴力,議会*,日曜日,新聞,今年,昨年,有罪,アナリスト,軍隊 \\\hline
\end{tabular}
\end{center}
\end{table}

出現頻度が5以上の名詞の一覧を表\ref{tab:hifreq-noun}\,に示す.
表\ref{tab:hifreq-noun}\,から次のような特徴があることが読み取れる.
\begin{enumerate}
\item\label{enum:noun:pronoun}
人間訳に比べてMT訳での出現頻度が高い(頻度差が10以上の)名詞は,
「彼」(頻度差:53),「それら」(65),「それ」(30),「米国」(24),
「状態」(15),「会社」(14),「攻撃」(11),「パレスチナ」(11),
「取扱い」(10)であるが,特に代名詞の出現頻度差が大きい.
\item\label{enum:noun:anaph}
逆に,MT訳に比べて人間訳での出現頻度が高い名詞は,「これ」(頻度差:12)で
ある.
\end{enumerate}

まず,代名詞「彼」,「それら」,「それ」の出現頻度差について検討する.
MT訳で「彼」,「それら」,「それ」と訳されている人称代名詞が人間訳ではど
のように訳されているかを調べた.
その結果を表\ref{tab:pron-hum}\,に示す.
なお,「それら」と訳されているのが人称代名詞`they',`their',`them'では
なく,定冠詞や指示代名詞`those'である場合があるが,この場合は集計に含め
ていない. 
このため,表\ref{tab:pron-hum}\,での「それら」の頻度は,
表\ref{tab:hifreq-noun}\,での「それら」の頻度とは一致しない.
同様に,「それ」と訳されているのが人称代名詞`it',`its'ではなく,
指示代名詞`that'である場合は集計に含めていないため,
「それ」についても表\ref{tab:pron-hum}\,での頻度と
表\ref{tab:hifreq-noun}\,での頻度は一致しない.
\begin{table}[htbp]
\caption{人間訳における人称代名詞の翻訳}
\label{tab:pron-hum}
\begin{center}
\begin{tabular}{|l||l|r@{}r|}\hline
\multicolumn{1}{|c||}{MT訳} & \multicolumn{1}{c|}{人間訳} & 
\multicolumn{2}{c|}{頻度} \\\hline\hline
& ゼロ代名詞 & 29 & (42.0\,\%) \\
& 先行詞   & 19 & (27.5\,\%) \\
& 「彼」     & 16 & (23.2\,\%) \\
\raisebox{1.5ex}[0pt]{「彼」} 
& 「自分」 &  1 &  (1.5\,\%) \\
& 「その」 &  1 &  (1.5\,\%) \\
& その他   &  3 &  (4.3\,\%) \\\hline
           & ゼロ代名詞 & 28 & (56.0\,\%) \\
           & 先行詞   & 13 & (26.0\,\%) \\
「それら」 & 「自分」   &  3 & (6.0\,\%) \\
           & 「その」   &  3 & (6.0\,\%) \\
           & 「彼ら」   &  3 & (6.0\,\%) \\\hline
& ゼロ代名詞 & 26 &  (74.3\,\%) \\
& 先行詞   &  6 &  (17.1\,\%) \\
\raisebox{1.5ex}[0pt]{「それ」} 
& 「それ」   &  2 &  (5.7\,\%) \\
& その他     &  1 &  (2.9\,\%) \\\hline
\end{tabular}
\end{center}
\end{table}

表\ref{tab:pron-hum}\,を見ると,
人間訳では,人称代名詞がゼロ代名詞化されるか先行詞に置き換えて訳される割
合が高く,`he',`his',`him'の場合で69.5\,\%,
`it',`its'の場合で82.0\,\%,
`they',`their',`them'の場合では91.4\,\%を占めていることが分かる.

英語では人称代名詞による照応が一般的に用いられるのに対して,日本語では,
人称代名詞による照応よりも,同一名詞の繰り返し,ゼロ代名詞,
再帰代名詞などによる照応が自然である\cite{Kanzaki94}.
このことを反映した英日翻訳技法として,英語の人称代名詞を先行詞,ゼ
ロ代名詞,再帰代名詞に置き換えて翻訳する手法が知られている
\cite{Umegaki75,Nakamura82}.
今後,人称代名詞の翻訳に関するこのような手法を機械翻訳システムに実装して
いく必要があると言える.

次に,表\ref{tab:hifreq-noun}\,における「これ」の出現頻度差について検討
する. 
人間訳でどのような英語表現が「これ」と訳されているのかを調べたところ,
14件のうち7件は,「これ」に直接対応する英語表現は存在せず,訳文で補われ
たものであった.
「これ」の補充は,次の文(H\ref{SENT:kore})のように,一文に訳すと複雑にな
る文を二文に分割し,それらの間を滑らかにつなぐために行なわれている
\footnote{代名詞「これ」に類似した表現として連体詞「この」などがあるが,
「この」のMT訳における出現頻度が5であるのに対して,人間訳では55であった.
詳細な分析を行なっていないので断定できないが,「この」の頻度差も,MT訳と 
人間訳における分かりやすさの違いにつながっているのではないかと予想され
る.}.
\begin{SENT2}
\sentE Leftist rebels in Colombia freed 62 government police and 
soldiers as part of a mass prisoner release hailed as a major boost for 
peace talks to end Colombia's 37-year-old civil conflict. [2001年7月3日]
\sentH コロンビアの左翼ゲリラが,大量の捕虜釈放の一環として,警官と兵士
62人を解放した.
{\bf これは,}37年間にわたる内戦を終結させるための和平会議への重要な起爆
剤として歓迎されている.
\sentM コロンビアにおける左派の反逆者は,和平交渉がコロンビアの37年を経
た民間の対立を終えるために,メジャーな増大と認められた大規模な囚人リリー
スの一部として62の政府警察,及び,軍人を解放した.
\label{SENT:kore}
\end{SENT2}
なお,\ref{sec:result:num-of-sent}\,節で挙げた
文(H\ref{SENT:ej-sent-div-prep})も同様の例である.


\subsection{代名詞に関する分析}
\label{sec:result:pron}

英語の代名詞を適切に翻訳することは,
(1) 代名詞を直訳すると誤った訳文や不自然な訳文となることが多い,
(2) 代名詞の出現頻度は比較的高い
\footnote{我々が英字新聞を対象にして行なった調査では,4240文中1845文
(43.5\,\%)に人称代名詞が含まれていた.},
などの理由から,
英日機械翻訳において重要な課題となっている\cite{Yoshimi01a}.
本節では,代名詞について\ref{sec:result:noun}\,節とは別の観点からさらに
分析する.


\subsubsection{代名詞とそれ以外の名詞の分布}
\label{sec:result:pron:pron-others}

名詞全体のうちで代名詞がどの程度を占めるかを明らかにするために,
人間訳とMT訳それぞれにおいて,名詞全体に占める代名詞の割合を算出した.
その結果を表\ref{tab:pron-others}\,に示す.
\begin{table}[htbp]
\caption{代名詞と他の名詞の分布}
\label{tab:pron-others}
\begin{center}
\begin{tabular}{|c||r|r|}\hline
品詞 & \multicolumn{1}{c|}{代名詞} & 
\multicolumn{1}{c|}{代名詞以外} \\\hline\hline
&  66 & 3375 \\
\raisebox{1.5ex}[0pt]{人間訳} & 1.9\,\% & 98.1\,\% \\\hline
& 213 & 3300 \\
\raisebox{1.5ex}[0pt]{MT訳} & 6.1\,\% & 93.9\,\% \\\hline
\end{tabular}
\end{center}
\end{table}

表\ref{tab:pron-others}\,の分布には,統計的有意差が認められ(有意水準5\,\%),
名詞全体に占める代名詞の割合は人間訳よりMT訳のほうが高い. 
表\ref{tab:pron-others}\,の結果は,表\ref{tab:pron-hum}\,とも合致し,
人間訳では英語の代名詞がゼロ代名詞化されるか先行詞に置き換えられる割合が 
高いことを示している.


\subsubsection{人称代名詞の訳語}
\label{sec:result:pron:trans}

英語の人称代名詞のうち人間を指すものを直訳すると,多くの場合,日本語の代
名詞「私」,「我々」,「あなた」,「彼」,「彼女」,「彼ら」になる.
人間訳とMT訳におけるこれらの出現頻度を表\ref{tab:pron-distri}\,に示す.
表\ref{tab:pron-distri}\,の分布には統計的に有意な差が認められる(有意
水準5\,\%).
\begin{table}[htbp]
\caption{人称代名詞の訳語の分布}
\label{tab:pron-distri}
\begin{center}
\begin{tabular}{|c||r|r|r|r|r|r|r|r|r|}\hline
訳語 & \multicolumn{1}{|c}{私} & \multicolumn{1}{|c}{我々} & \multicolumn{1}{|c}{あなた} & \multicolumn{1}{|c}{彼} & \multicolumn{1}{|c}{彼女} & \multicolumn{1}{|c|}{彼ら} \\\hline\hline
 & 0 & 1 & 2 & 16 & 2 & 5 \\
\raisebox{1.5ex}[0pt]{人間訳} & 0.0\,\% & 3.9\,\% & 7.7\,\% & 61.5\,\% & 7.7\,\% & 19.2\,\% \\\hline
 & 2 & 4 & 7 & 69 & 2 & 2 \\ 
\raisebox{1.5ex}[0pt]{MT訳} & 2.3\,\% & 4.7\,\% & 8.1\,\% & 80.3\,\% & 2.3\,\% & 2.3\,\% \\\hline
\end{tabular}
\end{center}
\end{table}

人間訳での出現頻度よりMT訳での出現頻度が高いもののうち,「私」,「あなた」
について検討する.
「彼」については,\ref{sec:result:noun}\,節で既に分析した.

人称代名詞`I'がMT訳では「私」と訳されているが人間訳ではゼロ代名詞化され
ている2件は,いずれも次の文(H\ref{SENT:pron-trans-i})のように,`I'が
被伝達節の主語であり,主節の主語を指している場合であった.
\begin{SENT2}
\sentE ``{\it I}'ve already won this one,'' Coach Duke said before the 
ceremony. [2001年7月17日]
\sentH 「もうこのゲームは勝ち取ったよ.」と,デュークコーチは式の前に言
いました.
\sentM 「{\bf 私は},既にこれを獲得した」と, Coach 公爵は,セレモニーの
前に,言った.
\label{SENT:pron-trans-i}
\end{SENT2}
このようなゼロ代名詞化は,主節の主語と従属節の主語が同一の場合,日本語で
は一方の主語を省略するという原則に沿うものである.

「あなた」については,人間訳で現れていない5件のうち2件は,ゼロ代名詞化が
文脈上可能であるものであった.
さらに,別の2件は,`you'が歌詞の一部でありその歌詞が原語のまま訳文に現れ
ているものであり,残りの1件は,`you'が総称的に用いられているためゼロ代名
詞化されているものであった.
文脈によるゼロ代名詞化が行なわれていたのは,次のような
テキスト(E\ref{SENT:pron-trans-you})における第二文の`you'である.
なお,テキスト(E\ref{SENT:pron-trans-you})の第一文は,無作為抽出した標本
には含まれていない.
\begin{SENT2}
\sentE Is your last name Dunlop? If so, {\it you}'re eligible for a big 
payoff. [2001年12月19日]
\sentH もしかして,あなたの苗字はダンロップさんではないですか. 
もしそうなら,大金をもらう資格があります. 
\sentM あなたの姓は,ダンロップであるか?もしそうであるならば,{\bf あな
たは},大きな利益に適格である.
\label{SENT:pron-trans-you}
\end{SENT2}

現在市販されている機械翻訳システムでは一文を越える範囲での処理はほとんど
行なわれていないが,このようなゼロ代名詞化を実現するためには,文間照応の
解析などを実装していく必要がある.


\subsubsection{一文に含まれる人称代名詞の数}
\label{sec:result:pron:num-of-pron}

人間訳とMT訳とで,一文中に含まれる代名詞の数に違いがあるかどうかを明らか
にするために,代名詞を$n$個含む文の分布を調べた.
その結果を表\ref{tab:num-of-pron}\,に示す.
表\ref{tab:num-of-pron}\,の分布には,統計的な有意差は認められない(有意
水準5\,\%).
\begin{table}[htbp]
\caption{人称代名詞の訳語を$n$個含む文の分布}
\label{tab:num-of-pron}
\begin{center}
\begin{tabular}{|c||r|r|r|r|}\hline
$n$ & \multicolumn{1}{|c}{1} & \multicolumn{1}{|c}{2} & 
\multicolumn{1}{|c}{3} & \multicolumn{1}{|c|}{4} \\\hline\hline
 & 25 &  1 & 0 & 0 \\
\raisebox{1.5ex}[0pt]{人間訳} & 96.2\,\% & 3.8\,\% & 0.0\,\% & 0.0\,\% \\\hline
 & 64 & 9 & 0 & 1 \\
\raisebox{1.5ex}[0pt]{MT訳} & 86.5\,\% & 12.1\,\% & 0.0\,\% & 1.4\,\% \\\hline
\end{tabular}
\end{center}
\end{table}

表\ref{tab:num-of-pron}\,を見ると,
人間訳では一文に最大で二つしか人称代名詞が含まれないのに対して,
MT訳では四つも含む文がある.
MT訳で人称代名詞を四つ含むのは次の文(E\ref{SENT:4prons})に対する
文(M\ref{SENT:4prons})である. 
\begin{SENT2}
\sentE In a repeat of previous run-ins between the pair, May then turned 
off Milosevic's microphone, told {\it him} {\it he} would have {\it his} 
chance to make {\it his} case during the trial and closed the hearing. 
[2002年1月11日] 
\sentH 前回交わされた言い合いの繰り返しになり,メイ氏はミロシェビッチ氏
のマイクロホンのスイッチを切り,ミロシェビッチ氏に対し公判中に自己弁護で
きる機会があると告げ審問手続きを終了させた.
\sentM ペアの間の前の口げんかの反復において,5月は,それからミロセビッチ
のマイクロホンをオフにし,{\bf 彼に}トライアルの間に{\bf 彼の}ケースを作
る{\bf 彼の}チャンスがあるであろう,と{\bf 彼に}告げ,そして,ヒアリング
を閉じた.
\label{SENT:4prons}
\end{SENT2}

文(M\ref{SENT:4prons})には,訳語選択において不適切な点がいくつかあるので,
人称代名詞以外の部分を人間訳と同じにして,比較してみる.
次の文(J\ref{SENT:4prons-mod})は人称代名詞以外の部分を人間訳に置き換え
たものである.
なお,`make his case'の訳語を慣用句として「自己弁護する」と辞書に登録し
ておけば,`his'は訳出されないものと仮定した.
\begin{JSENT}
\sentJ 前回交わされた言い合いの繰り返しになり,メイ氏はミロシェビッチ氏
のマイクロホンのスイッチを切り,{\bf 彼に対し}公判中に自己弁護できる
{\bf 彼の}機会が{\bf 彼に}あると告げ審問手続きを終了させた.
\label{SENT:4prons-mod}
\end{JSENT}

文(H\ref{SENT:4prons})では,`him'は「ミロシェビッチ氏」と訳され,
`he'と`his'はゼロ代名詞化されている.
これに対して,文(J\ref{SENT:4prons-mod})では,`him'を「ミロシェビッチ
氏」ではなく「彼」としているのは許容できるが,`he'と`his'を「彼」と訳し
ているのは不自然に感じられる.


\section{おわりに}
\label{sec:conc}

本稿では,人間訳とMT訳にどのような違いがあるのかを明らかにするための第一
歩として,英文一文に対する訳文の数,訳文の長さ,訳文に含まれる連体修飾節
の数の違いを計量分析し,さらに,動詞,名詞,代名詞について人間訳とMT訳を
比較した.
その結果,次のような点が明らかになった.
\begin{enumerate}
\item
人間訳に比べMT訳には,英文一文が複数の訳文に分割されにくい傾向がある.
MT訳では分割されないが人間訳では分割される箇所として,関係代名詞や前置詞
などが目立つ.
関係代名詞での分割では,関係代名詞が先行詞に置き換えられることが多い.
前置詞での分割では,前置詞句を日本語に翻訳したとき,完全な文になるように
工夫が施されている.
分割した文の間を滑らかにつなぐために照応表現が補われることがある.
\item
訳文の長さ(文節数)の分布に統計的有意差が認められる.
\item
用言の連用形と連体形の分布に有意差が認められ,MT訳のほうが人間訳よ
りも複雑な構造をした文が多いことが示唆される.
\item
体言と用言の分布に有意差は認められない.
\item
動詞に関するMT訳の主な課題としては,英語の「する」的表現を日本語で好まれ
る「なる」的表現に翻訳すること,ある内容を言語形式化する際の表現手段の選
択肢の一つとして機能動詞表現を考慮に入れることなどが挙げられる.
\item
名詞と代名詞に関しては,
複雑な関係代名詞節を連体修飾節に翻訳しないようにする処理,
人称代名詞をゼロ代名詞化するかあるいは先行詞に置き換える処理など
を実現することが課題である.
\end{enumerate}

人間訳とMT訳の計量的比較分析は,機械翻訳システムが抱える問題点について改
めて考え直したり,これまで見逃されていた問題点を見つけたりする上で有用で
ある.
今後,構文レベルでの分析を行なっていく予定である.

\acknowledgment
本稿の改善に非常に有益なコメントを頂いた査読者の方,内山将夫氏(通
信総合研究所),小谷克則氏(関西外国語大学)に感謝いたします. 

\bibliographystyle{jnlpbbl}
\bibliography{complexity}

\begin{biography}
\biotitle{略歴}
\bioauthor{吉見毅彦}
{1987年電気通信大学大学院計算機科学専攻修士課程修了.
1999年神戸大学大学院自然科学研究科博士課程修了.
(財)計量計画研究所(非常勤),シャープ(株)を経て,
2003年より龍谷大学理工学部情報メディア学科勤務.
} 

\bioreceived{受付}
\biorevised{再受付}
\bioaccepted{採録}
\end{biography}

\end{document}
