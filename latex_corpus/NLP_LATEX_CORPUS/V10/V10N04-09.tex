\documentstyle[epsf,jnlpbbl]{jnlp_j_b5}


\setcounter{page}{177}
\setcounter{巻数}{10}
\setcounter{号数}{4}
\setcounter{年}{2003} 
\setcounter{月}{7}
\受付{2002}{12}{5}
\採録{2003}{4}{10}

\newcommand{\maru}[1]{}
\newcommand{\underlines}[1]{}
\newcommand{\kanji}[1]{}

\newcommand{\hako}[1]{}

\title{日中機械翻訳におけるテンス・アスペクトの処理}
\author{謝 軍\affiref{GIFUU} \and 卜 朝暉\affiref{GIFUU} \and 池田 尚志\affiref{GIFUU}}

\headauthor{謝,卜,池田}
\headtitle{日中機械翻訳におけるテンス・アスペクトの処理}

\affilabel{GIFUU}{岐阜大学大学院工学研究科応用情報学科}
{Department of Information Science, Faculty of Engineering, Gifu University}

\jabstract{
本稿では,日本語のテンス・アスペクト表現を中国語に機械翻訳する手法を提案した.具体的には,日本語のテンス・アスペクト表現で主要な役割を果す「タ/ル/テイル/テイタ」を,両言語の文法特徴・共起情報,中国語述語の時間的性格を主要な手がかりとして,中国語のアスペクト助字(了/着/在/\kanji{001})または無標識の$\phi$に翻訳するアルゴリズムを提案した.まず先行研究から両言語におけるテンス・アスペクト表現の意味用法およびその意味用法間の対応関係をまとめた.そして,対応の曖昧さを解決するために,機械翻訳の立場から,「タ/ル/テイル/テイタ」と中国語アスペクト助字の対応関係を定めるアルゴリズムを提案した.最後に,作成した翻訳アルゴリズムを評価し,約8割正解という良好な結果を得た.
}

\jkeywords{テンス,アスペクト,構文特徴,共起情報,意味用法,述語の時間的性格,日中機械翻訳}

\etitle{Handling of Tense and Aspect\\
for Japanese-Chinese Machine Translation}

\eauthor{Jun Shieh\affiref{GIFUU} \and Zhaohui Bu\affiref{GIFUU} \and Takashi Ikeda\affiref{GIFUU}}

\eabstract{
In this paper we propose a method for mechanical translation of tense and aspect expressions from Japanese into Chinese. We deal with the expressions of ‘タ/ta’, ‘ル/ru’, ‘テイル/teiru’ and ‘テイタ/teita’ that play an important role in Japanese tense and aspect expressions. Based on syntactic characteristics and co-occurring information of both Japanese and Chinese, and temporal feature of Chinese predicates, the method shows how to translate these Japanese expressions into Chinese aspectual particles such as ‘了/le’, ‘着/zhe’, ‘在/zai’, ‘\kanji{001}/guo’, and unmarked null. We make past researches into shape, which discuss usages of tense and aspect of both languages. Our method determines matching relationship to lay down the matching ambiguities between the expressions. We evaluate our algorithm by hand and get more than 80\,\% of accuracy. The evaluation shows the method is effective and acceptable within machine translation.
}

\ekeywords{tense, aspect, syntactic feature, collocation, semantic usage, temporal feature of predicate, Japanese-Chinese Machine Translation}

\begin{document}
\maketitle
\thispagestyle{empty}

\section{はじめに}
日本語のテンス・アスペクトは,助動詞「タ/テイル/テアル/シツツアル/シテイク/…」などを付属させることによって表現される.中国語では「了/着/\kanji{001}(過)/在」などの助字がテンス・アスペクトの標識として用いられるが,テンス・アスペクトを明示的に表示しない場合も多い.言語学の側からの両言語のテンス・アスペクトに関する比較対照の先行研究においては,次のような文献がある.

\begin{enumerater}
\renewcommand{\labelenumi}{}
\renewcommand{\theenumi}{}
\item \cite{Ryu1987}は両言語の動詞を完成と未完成に分類しながら,「タ」と「了」の意味用法を対比した.
\item \cite{Cho1985}は,「了」と「た」の対応関係を描き,その微妙に似通ったり,食い違ったりする原因,理由を探している.
\item \cite{Shu1989}は,「タ」と「了」のテンス・アスペクトの性格について論じている.
\item \cite{Oh1996}は,「シテイル」形の意味用法を基本にして,日本語動詞の種別に対する中国語の対応方法を考察している.
\item \cite{Ryu2000}は,中国語の動詞分類によって,意味用法上で日本語のテンス・アスペクトと中国語のアスペクト助字との対照関係を述べている.\\
\end{enumerater}

これらの言語学側の先行研究では,日中両言語間のテンス・アスペクト表現の対応の多様性(すなわち曖昧性)を示すと同時に,動詞の時間的な性格や文法特徴の角度から曖昧性を解消する方法も論じている.しかしながらこれらの先行研究では,例えば「回想を表す場合」や「動作が完了或いは実現したことを表す場合」などといった表現での判断基準を用いており,そのまま計算機に導入することは難しい.すなわち,これらの判断基準は人間には了解できても,機械にとっては「どのような場合が回想を表す場合であるのか」「どのような場合が完了あるいは実現したことを表す場合であるのか」は分からない.

本論文では,機械翻訳の立場から,日本語のテンス・アスペクト助辞である「タ/ル/テイル/テイタ」に対して,中国語側で中国語のテンス・アスペクト用助字である「了/着/\kanji{001}(過)/在」を付属させるか否かについてのアルゴリズムを考案した.その際,\maru{4}では日本語述語の時間的性格を分析して中国語への対応を論じているが,我々は日中機械翻訳においては対応する中国語の述語はすでに得られていると考えてよいから,中国語の述語の時間的性格も同時に判断の材料としてアルゴリズムに組み込んだ.そのほか両言語における述語のいくつかの文法特徴や共起情報も用いた.

以下,第2章で両言語におけるテンス・アスペクト表現の意味用法およびその間の対応関係についてまとめ,第3章で,「タ/ル/テイル/テイタ」と中国語アスペクト助字の対応関係を定めるアルゴリズムについて述べた.さらに第4章で,作成した翻訳アルゴリズムの評価を手作業で行った結果を説明し,誤った箇所について分析も行った.評価の結果は約8割の正解率であった.

\section{日中両言語におけるテンス・アスペクト助辞の意味用法とその対照}
まず両言語におけるテンス・アスペクト助辞の意味用法を整理する.

\subsection{日本語側のテンス・アスペクト助辞の意味用法}
 本論文で取り上げる「タ」「ル」「テイル/テイタ」の主な意味用法は以下のように整理できる\cite{Shu1989,Kanemizu2000,Masuoka1992,Oh1996,Teramura1991}.\\

「タ」には以下の意味用法がある.
\begin{enumerate}
\item 動作や作用,変化が発話時点あるいは注目時点より前に完成したという過去の意味を表す.
\item アスペクト(完了または実現)を表す.
\item ムードの働きをする.\\
\end{enumerate}

基本形「ル」には以下の意味用法がある.
\begin{enumerate}
\item 現在あるいは現在までの状態,発話時での知覚・思考あるいは話し手の行為を表す.
\item 実現が確実な場合に,未来の状態,出来事,動作を表す.
\item 習慣や反複される出来事・動作が現在に及んでいる場合,基本形で表現される.
\item 時間を超越した事態を表す.\\
\end{enumerate}

「テイル/テイタ」には以下の意味用法がある.
\begin{enumerate}
\item 動作の進行中を表す.
\item 動作・作用の結果の残存を表す.
\item 習慣または繰り返し行う動作を表す.
\item 現在に意義を持つ過去の事象を表す.
\end{enumerate}


\subsection{中国語のテンス・アスペクト助字の意味用法}

中国語におけるテンス・アスペクト的なものは「了」,「着」,「\kanji{001}」などの助字,時間副詞(「已\kanji{002}」,「就」,「在」など),趨向補助語(「去」,「来」,「起来」など)および結果補助語(「完」,「到」,「\kanji{003}」,「在 ... 上」など)で表される.時間副詞「已\kanji{002}」,「就」は日本語の「もう/すでに」,「すぐに」などの時間表現に対応するものであるので,我々は既に翻訳処理は為されているものと考える.また,趨向補助語および結果補助語は,機械翻訳の立場から見れば,動詞・形容詞に対応する訳語の一部として辞書に記載され既に訳出されていると考えられる.従って,以下では助字「了」,「着」,「\kanji{002}」,および副詞「在」について考察することとした.

これらの語の意味用法については以下のように整理できる\cite{Ryu1996,Kanemizu2000,Shu1989,Ro1980,Cho1985}.\\

「了」には以下の意味用法がある.
\begin{enumerate}
\item 過去を表す.
\item 完了または実現を表す.
\item 変化が生じた事を表す.
\item 語気の役を担う.\\
\end{enumerate}

文中の位置と役割によって,「了」は「了1」,「了2」の二つに区別される.動詞の直後に用いられる「了」を「了1」と書くことにする.「了1」は主に動作の完了を表す.その動詞が目的語/補助語を伴えば,「了1」は目的語/補助語の前に置かれることになる.文末に置かれる「了」を「了2」と書くことにする.「了2」は主に事柄に変化が起こったことを表すが,変化が起ころうとしていること(語気)を表す働きもある.文中の述語が目的語/補助語を伴えば,「了2」は目的語/補助語の後に置かれることになる.

「了1」は主に完了,「了2」は主に変化が生じたことを表すが,「\kanji{004}杏\kanji{005}叶全部落光\underlines{了} ./銀杏の葉が全部落ちてしまった.」のように「了」は変化が生じたこと(木の葉の状況が変化したこと)を表すか,動作の完了(動作「落ちる」が完了したこと)を表すかの区別は,明確であるとは言えない.\\

「\kanji{001}(過)」には以下の意味用法がある.
\begin{enumerate}
\item 動作が済んだことを表す.
\item 経験を述べる.\\
\end{enumerate}

(1)の用法に対しては「\kanji{001}」を「了」に入れ替えることができるが,(2)の用法に対しては「了」に入れ替えることはできない.例えば,「他去\kanji{001}中国./彼が中国に行ったことがある.」は経験の意味を表すが,「他去了中国./彼が中国に行った.」または「他去中国了./彼が中国に行った.」では経験という意味を表現できない.\\

「着」には以下の意味用法がある.
\begin{enumerate}
\item 動作の進行中を表す.
\item 動作・状態の持続を表す.
\item 語気を表す.\\
\end{enumerate}

「在」の意味用法:

副詞である「在」は文中の主語と述語の間に介入し,動作の進行中を表す.「着」も同じ動作の進行中を表すが,「着」に接する動詞は状態性が強い,「在」に接する動詞は動作性が強いという相違点がある.

\subsection{両言語のテンス・アスペクトの意味用法の対照}

日本語のタ/ル/テイル/テイタ,中国語の了/着/在には,前述したようにさまざまの意味用法があり,単純にタ→了,ル→$\phi$,テイル→着,テイタ→在と対応させるようなわけにはいかない($\phi$はアスペクト助字を使わないことを指す).

「タ」は「了」より多くの場面で使われ,「了」をつけることができない場合や結果補助語,趨向補助語と対応する場合などがある\cite{Ryu1987,Cho1985}.「テイル/テイタ」は「在」,「着」,「了」などに対応する\cite{Oh1996}.「ル」は現在・未来の出来事,習慣あるいは時間を超越した事態を表すが,中国語ではそのような場合ふつう$\phi$に対応する.しかし変化が生じたことを表すときは「了」と対応する.例えば,「この町へ引越ししてきてから,かれこれ5年になる.」の中国語訳では「搬到\kanji{006}个城市一\kanji{017}眼五年\underlines{了} .」と「了」を使う.

テンス・アスペクトの意味はその標識のみによって決定されるのではなく,動詞の時間的な性格や修飾語などによっても影響される.両言語の動詞等の時間的な性格や修飾語,文法特徴には違いがあるから,意味用法間の対応関係には曖昧性が生じてくる.第3章では,機械翻訳の立場から,両言語の文法特徴および中国語述語の時間的性格に基づいてその曖昧性を解決する方法について考察する.


\section{「タ/ル/テイル/テイタ」と中国語アスペクト助字の対応関係を定めるアルゴリズム}

ここで述べるアルゴリズムは,「タ/ル/テイル/テイタ」によるテンス・アスペクトの表現以外の処理は終わっているものとして,「タ/ル/テイル/テイタ」をどのように中国語に翻訳すれば,整合の取れた正しい翻訳になるかという立場からのものである.従って,原文である日本語文に関する情報だけでなく,「タ/ル/テイル/テイタ」以外に関する中国語翻訳文についての情報も使えるものとしている.

このアルゴリズムは,主語,目的語等の文法特徴や共起する語彙,中国語動詞の時間的性格属性などを主な手がかりとしている.

この章ではまず中国語動詞の時間的性格分類について述べる.


\subsection{中国語述語の時間的性格分類}

\cite{Ryu2000}は,中国人の日本語学習の観点から日中両言語を対照し,中国語の動詞が未然・継続と已然・非継続において,対立を持っているか否かという視点から,時間的性格分類を行っている.本文ではその分類を参考にして多くの例文に対して翻訳アルゴリズムを検討し,結果として表1に示すように分類した.

劉の分類との主な違いは,\maru{1}静態動詞に分類されていた心理・生理的状態動詞を動作性が強いか,状態性が強いかによって心理活動動詞と心理状態動詞の二つに分けたこと,\maru{2}形容詞は事柄の性質・状態を表すという点で静態動詞と機能が同じであるので,静態動詞に分類したことである.結果動詞については,「着」を付加できるか否かを判定条件とした(付加できなければ結果動詞).

\begin{table}[htbp]
 \label{HYO1}
 \caption{中国語述語の時間的性格分類}
 \begin{center}
 \def\arraystretch{}
 \begin{tabular}{|c|l|l|} \hline
  大 分 類 & \multicolumn{1}{|c|}{細 分 類} & \multicolumn{1}{|c|}{例} \\ \hline\hline
                                  & \maru{1}動作行為動詞         & \kanji{008}\kanji{009},\kanji{010},拍 \\
  \cline{2-2}\cline{3-3}          & \maru{2}動作状態動詞         & 挂,穿,吊 \\
  \cline{2-2}\cline{3-3} \raisebox{0.5\normalbaselineskip}[0pt][0pt]{動態動詞} & \maru{3}心理活動動詞         & 回\kanji{011},体会,\kanji{012}料 \\
  \cline{2-2}\cline{3-3}          & \maru{4}移動動詞(趨向動詞)   & 来,上,去 \\ \hline
                                  & \maru{1}属性動詞             & 当做,是,缺乏 \\
  \cline{2-2}\cline{3-3}          & \maru{2}存在動詞             & 有,在,\kanji{013}有 \\
  \cline{2-2}\cline{3-3} \raisebox{0.5\normalbaselineskip}[0pt][0pt]{静態動詞} & \maru{3}心理状態動詞         & 知道,佩服,后悔 \\
  \cline{2-2}\cline{3-3}          & \maru{4}形容詞               & 静,来不及,少 \\ \hline
                                  & \maru{1}「V+V」構造          & 采用,取得,下降 \\
  \cline{2-2}\cline{3-3}          & \maru{2}「V+Adj」構造        & 提高,放\kanji{014},打乱 \\
  \cline{2-2}\cline{3-3} \raisebox{0.5\normalbaselineskip}[0pt][0pt]{結果動詞} & \maru{3}「V+趨向補助語」構造 & 冲出,走上,跨\kanji{090} \\
  \cline{2-2}\cline{3-3}          & \maru{4}瞬間変化動詞         & \kanji{015}\kanji{016},死,看\kanji{003} \\ \hline
 \end{tabular}
 \end{center}
\end{table}

結果動詞の\maru{1}「V+V」構造はその動詞の内部構造が二つの動詞からなるもの;\maru{2}「V+Adj」構造はその動詞の内部構造が動詞と形容詞からなるもの;\maru{3}「V+趨向補助語」構造はその動詞の内部構造が動詞と趨向動詞(補助語)からなるものという意味である.

付録に,第4章で評価する際に対訳される中国語の述語についての時間的性格分類を示す.


\subsection{「タ」の翻訳アルゴリズム}

「タ」は「了/\kanji{001}」などのアスペクト助字と対応するばかりでなく,結果補助語,趨向補助語と対応する場合がある\cite{Cho1985}.しかし,前述のように,機械翻訳の立場から見れば,結果補助語や趨向補助語は述語部分に対応する訳語の一部として辞書に記載され,既に訳出されていると考えられる.例えば,次の例の訳文中の趨向補助語「出来」は述語「整理/とりまとめる」と一緒に訳出される.\\

例文:残された外来語の表記の見直しについては62年1月以来,審議を続け,委員会試案として\underlines{とりまとめた} .

訳文:剩下的外来\kanji{019}\kanji{020}写法的修\kanji{021}工作,自一九八七年一月以来,\kanji{002}多次\kanji{022}\kanji{023},才作\kanji{024}委\kanji{025}会的\kanji{026}行方案\underlines{整理\underline{出来}} .\\

従って,以下では「タ」に「了」を対応させて訳出すべきか否かを中心に,「タ」と「了/\kanji{001}」などのアスペクト助字との対応について考察する.
以下では,いくつかの構文的特徴から「了」の使用,不使用を決められる場合と,動詞の時間的性格から決められる場合とについて考察し,次いで全体をひとつのアルゴリズムにまとめる.

\paragraph{(i) 連体節の場合} 

日本語では連体節中の「タ」は,連体修飾される名詞に対する視点が修飾する事柄の前のものにあるのか後のものにあるのかという,相対的な時間関係を表す.
\begin{enumerater}
\renewcommand{\labelenumi}{}
\renewcommand{\theenumi}{}
\item アラスカへ行った次郎	視点は``アラスカへ行った後''の次郎にある;
\item アラスカへ行く次郎	視点は``アラスカへ行く前''の次郎にある.
\end{enumerater}

中国語では連体節と連体修飾される名詞を「的」でつないで表現するが,その際日本語のように時間的関係を表現することは必須ではない.すなわち,\maru{1}の場合「了」を使う表現「去\underlines{了}阿拉斯加的次郎」もあるが,「了」を使わない表現「去阿拉斯加的次郎」も許される.その場合に時間的な関係は文脈や常識から判断される.つまり動詞が連体節を作るとき,単に状態や属性を示すだけのものに変わり,時間性から解放される方向を辿る\cite{Cho1985}.従って我々のアルゴリズムでは連体節の「タ」は$\phi$と対応させることとした.

ただし,形式名詞が連体修飾されている場合には,若干事情が異なる.「〜たことがある」,「〜たことがない」は経験を表す表現であるので,$\phi$でもまた「了」でもなく,「\kanji{001}」を対応させなければならない.「〜たことがある」は全体が連体節になる場合,例えば,「アラスカに行っ\underlines{たことがある}次郎はカナダにも行きたい.」は「去\underlines{\kanji{001}}阿拉斯加\underlines{的}次郎\kanji{027}想去加拿大.」のように,「\kanji{001}」を使う連体節と連体修飾名詞を「的」でつないだ文に訳される.
「連体節+形式名詞」が完了を表す「もう/すでに」を伴うとき,「了1」を使う.形式名詞によって受け止められる連体節でこれら以外の場合は,述語がそのまま主語あるいは目的語になって「了」を使わない.従って,「タ」の対応は$\phi$とする.

\begin{figure}[hbtp]
\begin{center}
  \epsfile{file=figure/1.eps,width=135mm,height=28.5mm}
\caption{連体節中の「タ」の翻訳}
\label{fig:1}
\end{center}
\end{figure}

\paragraph{(ii) 否定表現の場合} 

中国語では「没有/没」で動作・作用の起こったことあるいは完了したことを否定する.変化の意味を含まない限り,「没有/没」は過去のことを表すと了解されるから,「了」を使う必要はない.また,「不」で性質・状態を否定する.性質,状態およびその否定は時間性と関係ないから,「没有/没」と同じように変化の意味を含まない限り,「了」を使わない.「没有/没」と「不」を伴うとき,事態の変化の意味を含むなら,「了2」を使う(ivの\maru{2}参照).しかし,変化の意味を含むか否かの判別については,日本語側の「形容動詞+でなくなった(静かでなくなった)」,「形容詞+なくなった(明るくなくなった)」,「名詞+でなくなった(子供でなくなった)」,「動詞+なくなった(食べなくなった)」の場合を変化の意味を含むとし,その他の場合については判別できていない.

\begin{figure}[hbtp]
\begin{center}
  \epsfile{file=figure/2.eps,width=127.2mm,height=26mm}
\caption{否定表現と「タ」の翻訳}
\label{fig:2}
\end{center}
\end{figure}

\paragraph{(iii) 文法的特徴から「了」を使わないと判断できるその他の場合} 

以上の他,日本語では「タ」を使うが対応する中国語では「了」を使わない場合として,次のような場合が観察される\cite{Ryu1996}.
\begin{enumerater}
\renewcommand{\labelenumi}{}
\renewcommand{\theenumi}{}
\item 介詞フレーズ補助語を伴う場合.介詞フレーズ(介詞「于/自/向」+名詞)は動詞の後ろに用いられて補助語になる.これを介詞フレーズ補助語と呼ぶ.介詞フレーズ補助語は単に状況を紹介するためだけの表現である.中国語で,このような介詞フレーズ補助語を伴っている文では介詞「向」以外の場合は「了」を用いてはいけない.例えば,「他一九〇七年生\underlines{于}北海道旭川町.」/「彼は明治40年,北海道・旭川町に生まれ\underlines{た} .」において「了」を使ってはいけない.「向」の場合には,直後に完了を表す「了1」を使っても使わなくても同じ意味(状況を紹介する)を表す.例えば,「他把目光\kanji{017}\,\underlines{向}我.」と「他把目光\kanji{017}\,\underlines{向了}我.」/「彼は視線を私に転じた.」において,「了1」を使っても使わなくても同じ「視線を転じた」ことを表す.以上のことから,本アルゴリズムでは介詞フレーズ補助語を伴う場合,「了」を使わないとした.
\item 「是…的.」構文の場合.中国語「是…的.」構文は二つの機能がある.一つは述語の動作が過去においてすでに完了したことを表すが,述べようとする重点は動作自体にあるのではなく,動作の時間や場所,やり方,条件,目的,対象,仕手,動作に関係する何らかの側面にある.この場合すでに完了したことであっても「了」を使わない.例えば,「我\kanji{028}\,\underlines{是}坐\kanji{029}\kanji{030}去\underlines{的} .」/「私達は電車に乗って行ったのです.」では動作「行く」に関するやり方の側面である「電車に乗って」を重点として述べている.もう一つの機能は主に話者の見方や見解,態度を表す機能である.このような話者の判断を表現する場合,専ら完了を表す「已\kanji{002}」(「すでに」の意味)を伴わない限り,「了」を使わない.例えば,「\kanji{006}个\kanji{031}\kanji{032},我\kanji{028}\,\underlines{是}很注意\underlines{的} .」/「この問題に関して,我々は関心をいだいた.」には「了」を使わない.「已\kanji{002}」がある場合には「我\kanji{028}已\kanji{002}是注意到\kanji{006}个\kanji{031}\kanji{032}了的.」のように「了」を使う(この場合日本語では「我々はこの問題にすでに関心をいだいていた.」のように「ていた」が使われる).これらのことから,本アルゴリズムでは,「是…的.」構文においては,副詞「已\kanji{002}」を伴っているとき以外,「了」を使わないとした.
\item 述語が結果補助語あるいは趨向補助語を伴っている場合.この場合は,補助語によってアスペクトが表現されており,「了」と共存してもしなくてもよい.例えば,「\kanji{037}\kanji{038}的地\kanji{033}情况\kanji{034}修\kanji{035}\kanji{018}\,\underlines{来}了困\kanji{036}.」/「複雑な地質の状況が橋工事に困難をもたらした.」には,「了」を使っても使わなくてもよい(「来」$\in$趨向補助語).本アルゴリズムでは,明確に変化の意味を含んでいると判断される場合以外は,「了」を使わないとした.変化の意味を含むか否かについてはivの\maru{2}と同様に判断する.
\end{enumerater}

\paragraph{(iv) 文法的特徴から「了」を使うと判断できるその他の場合} 

逆に以下の場合には,完了或いは変化の意味を表すと判断できるので,「了」を使わなければならない.
\begin{enumerater}
\renewcommand{\labelenumi}{}
\renewcommand{\theenumi}{}
\item 日本語の文で,もっぱら完了を表す「もう」・「すでに」を伴っている場合.
\item 日本語の文で,変化あるいは完了を表す「になった」/「なった」を伴っている場合.この場合,述語が目的語を伴う時,完了を表す「了1」,伴わなければ,変化を表す「了2」とする.\\例文:彼は勤勉によって実業家\underlines{になった} .\\訳文:他通\kanji{001}努力成\underlines{了}一个\kanji{039}\kanji{040}家.
\item 日本語の従属文が「したら」であり,主文が「タ」である場合.このような文は,過去に実際には起こらなかったことを起こりえたこととして主張している.主文の「タ」形に対して主文の中国語の訳では変化が起ころうとしている語気を表す,「了2」を使う.\\例文:もし君が昨年まじめに勉強しなかっ\underlines{\underline{たら}} ,今この大学の学生でいられなかっ\underlines{た}ろう.\\訳文:如果\kanji{041}去年没\kanji{042}真地学\kanji{043},\kanji{016}在就不是\kanji{006}个大学的学生\underlines{了} .
\item 日本語の文末が「〜てしまった」である場合.状況が好ましいものでない「〜てしまった」は行為・状態の実現を明示する「了2」と対応する.\\例文:誕生日におばのくれた指輪を昨日なくし\underlines{てしまった} .\\訳文:生日那天姑母\kanji{034}我的戒指昨天\kanji{044}\,\underlines{了} .
\item 中国語の文が数量補助語を伴っている場合.この場合,すでに到達した数量,または完了までの持続時間を表す.例えば,「\kanji{014}\,\underlines{了}\,\kanji{045}天会.」/「2日間会議を開い\underlines{た} .」には会議が終わるまでの持続時間(2日間)を表す.述語の直後,数量補助語の前に「了1」あるいは「\kanji{001}」を使う可能性があるが,「了1」で「\kanji{001}」を代用しても到達した数量,または完了までの持続時間を同様に表現できるから,本アルゴリズムでは「了1」を使うとした.
\end{enumerater}

\paragraph{(v) 述語の時間的性格からの「了」の使用/不使用の判断} 

\begin{enumerater}
\renewcommand{\labelenumi}{}
\renewcommand{\theenumi}{}
\item 一般に静態動詞は状態や性質を表し,時間性と関係ないため「了」を使わない.
\item 引用を表す動態動詞の場合.引用表現の重点が動作の完了に置かれるならば「了」を取らないこともないが,たいていの場合には表現の重点は引用された内容の紹介,描写に置かれているのであって,引用の媒介である動作が完了したか否かに関係なく,「了」は使わない\cite{Cho1985}.本アルゴリズムでは,この場合も「了」を使わないとした.なお,引用を表す動詞(「\kanji{046}…」/「…」と話した,「以\kanji{024}…」/「…」と思った,「倡\kanji{023}…」/「…」と提唱した など)については,辞書に引用を表す動態動詞であるという属性が付与されているものとする.
\item 残った動態動詞と結果動詞の場合には,「タ」を過去あるいは完了を表す「了1」に対応させる.\\
\end{enumerater}

表2は,以上の考察をアルゴリズムの形にまとめたものである.表中の処理1〜13は文法特徴による個別的・特殊的な判別条件,処理14〜16は(v)の中国語述語の時間的性格による一般的な判別条件である.この処理順は,前述の(i)〜(v)の考察をより個別的・特殊的な条件を先に調べるという原則で並べたものである.ただし,以下の ``[...]''中の処理は順序に関係がない.
\begin{center}
1→2→3→[4,5,6]→7→[8,9]→[10,11]→[12,13]→[14,15]→16\\
$\longleftarrow$ 特殊性が強い       一般性が強い $\longrightarrow$
\end{center}
$\phi$は「了」を使わないという意味である.また,(日)は日本語に関する条件であること,(中)は中国語に関する条件であることを表している.

\begin{table}[htbp]
 \label{HYO2}
 \caption{「タ」形の翻訳を決める手順}
 \begin{center}
 \def\arraystretch{}
 \begin{tabular}{|c|l|c|} \hline
  処理順 & \multicolumn{1}{|c|}{判 別 条 件}            & 処理   \\ \hline\hline
  1      & (日)「〜たことがある/ない」                     & [\kanji{001}]    \\ \hline
  2      & (日)連体節(「連体節+形式名詞」を除く)           & $\phi$ \\ \hline
  3      & (日)「もう」・「すでに」を伴う時                & [了1]  \\ \hline
  4      & (日)「連体節+形式名詞」の場合                   & $\phi$ \\ \hline
  5      & (中)「是…的.」構文                            & $\phi$ \\ \hline
  6      & (中)介詞フレーズ補助語を伴う                    & $\phi$ \\ \hline
  7      & (日)「になった」/「なった」:(中)目的語を伴う時 & [了1]  \\ \hline
  8      & (日)「になった」/「なった」                     & [了2]  \\ \hline
  9      & (日)従属文が「したら」である場合,主文のタ形    & [了2]  \\ \hline
  10     & (中)否定                                        & $\phi$ \\ \hline
  11     & (中)趨向・結果補助語がある場合                  & $\phi$ \\ \hline
  12     & (中)数量補助語を伴う場合                        & [了1]  \\ \hline
  13     & (日)「〜しまった」の場合                        & [了1]  \\ \hline
  14     & (中)静態動詞(述語性格で判別する.以下同様)      & $\phi$ \\ \hline
  15     & (中)引用を表す動態動詞                          & $\phi$ \\ \hline
  16     & (中)動態動詞・結果動詞                          & [了1]  \\ \hline
 \end{tabular}
 \end{center}
\end{table}


\subsection{「テイル/テイタ」の翻訳アルゴリズム}

日本語で「テイル/テイタ」が使われている場合に,中国語への翻訳としては2.1節で述べた「テイル/テイタ」の意味用法に概ね対応して,「了」,「着」,「在」などのアスペクト助字を使う,あるいはこれらのどれも使わない($\phi$).これらの使い分けは,主として中国語の述語を中心とする事情によるものであり,日本語の側での意味解析・意味分類によるよりも,主として中国語側の状況に依存して使い分けるほうが合理的であり,有利である.以下では主として中国語側の構文的特徴と動詞の時間的性格から,これらのアスペクト助字を使用する場合について考察し,次いで全体をひとつのアルゴリズムにまとめた.

\paragraph{A.  $\phi$とする場合} 

\begin{enumerater}
\renewcommand{\labelenumi}{}
\renewcommand{\theenumi}{}
\item 日本語で習慣を表す,あるいは動作が繰り返し行われる「毎〜」や「たいてい」を伴う場合.この場合は,進行中と見なされ,中国語では「在」を使うか,あるいはアスペクト助字のどれも使わない.例えば,「父は最近毎朝走っている.」に対しては「父\kanji{047}最近\kanji{048}天早晨\underlines{在}\,\kanji{049}\kanji{241}.」と「父\kanji{047}最近\kanji{048}天早晨\kanji{049}\kanji{241}.」はどちらでも繰り返して「走る」ことを表すが,本アルゴリズムで「在」を使わない,つまり$\phi$とする.
\item 述語が中国語の静態動詞に訳される場合.静態動詞は状態や性質を表し,時間性と関係ないため,アスペクト助字のどれも使わない.例えば,次の例の訳文中の「\kanji{050}次\kanji{052}比」は形容詞(静態動詞)であり,アスペクト助字のどれも使わない.\\例文:(裏寺町通は)わずか500メートルほどの通りだが,16の寺が軒を\underlines{連ねている} .\\訳文:(里寺町街)\kanji{053}不\kanji{001}五百米,却有十六座寺院\underlines{\kanji{050}次\kanji{052}比} .
\item 述語が中国語に訳すとき名詞化される場合.この場合には時間性が失われるため,$\phi$にする.例えば,次の例文で日本語の連体修飾「述語+形式名詞(の)」は,訳文で「的」によって名詞化されて主文の主語になり,アスペクト助字のどれも使わない.\\例文:一方,\underlines{折り詰め宅配弁当と銘打って展開をしている}のは,24時間営業の「KAKIEMON」.\\訳文:与此同\kanji{055},\underlines{以``送木制盒\kanji{054}''\kanji{024}名\kanji{014}展(\kanji{040}\kanji{056})的}有昼夜二十四小\kanji{055}\kanji{057}\kanji{040}的``柿\kanji{058}\kanji{059}''.
\item 「タ」形の翻訳アルゴリズムと同じ理由で介詞フレーズ補助語を伴う((iii)の\maru{1})場合, および「是…的」構文である((iii)の\maru{2})場合,$\phi$にする.例えば,次の例の訳文では「是…的」構文が使われているので,アスペクト助字のどれも使わない.\\例文:非信者の教会挙式について,結婚式相談会社社長でクリスチャンでもある比留間宗生氏によれば,教会としては布教活動の一環と\underlines{とらえている}という.\\訳文:\kanji{060}于非教徒到教堂\kanji{061}行婚礼一事,据婚礼咨\kanji{062}公司\kanji{063}\kanji{002}理比留\kanji{064}宗生(基督教徒)\kanji{046},教会\underlines{\underline{是}}把它\underlines{当作}一件\kanji{066}教工作来\kanji{067}待\underlines{\underline{的}} .
\item 結果補助語あるいは趨向補助語を伴う動作行為動詞である場合.この場合,「タ」形の翻訳アルゴリズムと同じ理由((iii)の\maru{3})で,補助語がすでに時間性を表しているため,$\phi$にする.\\例文:ゆうべ私たちは11時まで話し\underlines{ていた} .\\訳文:昨天\kanji{272}上我\kanji{028}\kanji{068}\,\underlines{到}十一点\kanji{069}.(到$\in$結果補助語)
\end{enumerater}

\paragraph{B.  「着」とする場合} 

中国語での動作状態動詞は動詞の主体が動作の終わった瞬間の姿をそのまま維持していくことを表す.「テイル/テイタ」に接する述語が動作状態動詞に訳される場合は,以前の動作・作用の結果が現在に残っていることを表す,つまりその動作・作用の持続を表すため,「着」を使う.例えば,「彼は帽子をかぶっ\underlines{ている} .」あるいは「彼は帽子をかぶっ\underlines{ていた} .」に対して「他戴\underlines{着}帽子.」の「戴」は動作状態動詞であり,直後に「着」が使われる.

\paragraph{C.  「了」とする場合} 

\begin{enumerater}
\renewcommand{\labelenumi}{}
\renewcommand{\theenumi}{}
\item 述語が数量補助語を伴う場合.「タ」形の翻訳アルゴリズムと同じ理由((ivの\maru{5}))で,「了1」を使う.例えば,次の例文の訳文に,述語「快」が数量補助語「十分\kanji{069}」を伴って,その間に「了1」を使う.\\例文:ぼくの時計は10分進ん\underlines{でいます} .\\訳文:我的表快\underlines{了}十分\kanji{069}.
\item 述語が結果動詞である場合.中国語での結果動詞は完了したあるいは変化した結果を表す.この場合,目的語を伴う時,完了の意味を表す「了1」を使う;目的語を伴わない結果動詞では,変化の意味を表す場合「了2」を使う.\\例文:平安神宮,吉田神社周辺などでは高層建築や底地買いに反対する市民運動が\underlines{起きている}\,.\\訳文:在平安神\kanji{070}和吉田神社等\kanji{071}也\kanji{072}起\underlines{了}反\kanji{067}在其周\kanji{073}修建高\kanji{074}建筑或\kanji{075}置地\kanji{076}的市民\kanji{077}\kanji{078}.(「\kanji{072}起」$\in$ 結果動詞)
\end{enumerater}

\paragraph{D.  「在」とする場合} 

A〜Cの記述に含まれていないケースは動作状態動詞を除く動態動詞である(動作行為動詞と心理活動動詞).この二種の動態動詞に訳す場合,動作の進行中を表すと判断できるので,「在」を使うことにする.\\例文:同じ情報を定期的,広域に提供する媒体のひとつとして成長を\underlines{続けている} .\\訳文:它作\kanji{024}一\kanji{079}定期向广泛地区提供同\kanji{080}信息的媒体,\underlines{在}持\kanji{081}\kanji{072}旺\kanji{015}\kanji{082}.(「持\kanji{081}」$\in$ 動作行為動詞)\\

以上のことから,中国語述語の時間性格分類と「テイル/タイタ」とアスペクト助辞との一般的な対応関係は図3のように整理される.

\begin{figure}[hbtp]
\begin{center}
  \epsfile{file=figure/3.eps,width=119mm,height=38.1mm}
\caption{中国語述語の時間性格分類と「テイル/テイタ」の翻訳}
\label{fig:3}
\end{center}
\end{figure}


なお,「シテイル」と「シテイタ」は中国語に翻訳するとほとんどの場合区別はなくなる.これは,日本語では「シテイル」と「シテイタ」で非過去・過去の対立が表現されるが,中国語では時間副詞,結果・趨向補助語などで表現される場合が多い.実際,「シテイル」形の過去形「シテイタ」形に対して,新聞記事150文を調査した結果,「ていた」を「ている」に入れ替えても,以下の二文以外は中国語訳語が同じでよいことが確認できた.
\begin{enumerate}
\item 松下幸之助さんは,大阪電灯で屋内配線工事の手車をひいて働いていた.
\item フランスではショコラ,ドイツではショコラーデ,アメリカではチョコレート,日本では昔,長康霊糖,猪口令糖などの当て字を使っていた.
\end{enumerate}

この二つの文は「シテイタ」形が過去の事象を表し,中国語では経験を表す「\kanji{001}」に対応させるのが自然である.「シテイル」形に変えると,進行中の意味を表し,中国語では(1)に対しては$\phi$を,(2)に対しては「在」を対応させるのが自然である.(2)に関しては,中国語での違いはニュアンスの違いといってもよいが,(1)の違いは区別すべきであろう.

我々のアルゴリズムでは,(1)の場合のみこの区別を考慮し(表3の12),(2)の場合およびその他の場合は中国語ではこのような区別を表現しないとして,同一視することとした.つまり,「学\kanji{043}」,「工作」,「\kanji{083}\kanji{078}」などの持続性を持つ動作行為動詞であるとき,「〜ている」であれば$\phi$にする;「〜ていた」であれば「\kanji{001}」にする.

表3は,以上の考察をアルゴリズムの形にまとめたものである.

\begin{table}[htbp]
 \label{HYO3}
 \caption{「テイル/テイタ」形の翻訳を決める手順}
 \begin{center}
 \def\arraystretch{}
 \begin{tabular}{|c|l|c|} \hline
  処理順 & \multicolumn{1}{|c|}{判 別 条 件}             & 処理   \\ \hline\hline
  1      & (日)「毎〜」,「たいてい」(習慣を表す)           & $\phi$ \\ \hline
  2      & (中)数量補助語を伴う                             & [了1]  \\ \hline
  3      & (中)「是…的.」文                               & $\phi$ \\ \hline
  4      & (中)介詞フレーズ補助語を伴う                     & $\phi$ \\ \hline
  5      & (中)静態動詞                                     & $\phi$ \\ \hline
  6      & (中)動作状態動詞                                 & [着]   \\ \hline
  7      & (中)結果動詞:目的語ある場合                     & [了1]  \\ \hline
  8      & (中)結果動詞:目的語ない場合                     & [了2]  \\ \hline
  9      & (中)動態動詞:述語が名詞句(主語,目的語)になる時 & $\phi$ \\ \hline
  10     & (中)動作行為動詞:結果・趨向補助語を伴う         & $\phi$ \\ \hline
  11     & (日)「〜ている」:(中)持続性を持つ動作行為動詞   & $\phi$ \\ \hline
  12     & (日)「〜ていた」:(中)持続性を持つ動作行為動詞   & [\kanji{001}]    \\ \hline
  13     & (中)動態動詞:他                                 & [在]   \\ \hline
 \end{tabular}
 \end{center}
\end{table}

表3で,処理1〜4は文法特徴による特殊的な判別条件,処理5~13は中国語述語の時間的性格による一般的な判別条件である.ただし,``[...]''中の処理は順序に関係がない.
\begin{center}
1 → 2 → [3,4] → [5,6,7,8] → 9 → [10,11,12] → 13\\
$\longleftarrow$ 特殊性が強い   一般性が強い $\longrightarrow$
\end{center}

\subsection{「ル」形を「了」に訳す場合の判断}

日本語の基本形「ル」は現在・未来または習慣を表すので,一般には$\phi$に対応し,過去・完了を表す「了」とは対応しない.しかし,日本語の「ル」形が事態の変化を表現する場合は中国語の「了」を対応させなければならない.以下にそれらの場合を考察する.

日本語で「(形容詞)なる」,「(形容動詞)になる」,「(名詞)になる」などいわゆる「なる」系の表現は変化の意味を表すので,中国語では「了」を使う(ただし,「気になる」,「こうなる」などの慣用節を除く).このほか,中国語側が以下のようである場合にも,変化または完了の意味を表すことになるから,「了」を使う.
\begin{enumerater}
\renewcommand{\labelenumi}{}
\renewcommand{\theenumi}{}
\item 時間副詞「已\kanji{002}」(すでに,もはや,もう)を伴う時,動作や変化が完了し,またはある程度に達していることを表す(呂1980).\\例文:いちいち訳語を作っていては,\underlines{もう間に合わない}のかも知れません.\\訳文:也\kanji{084}是因\kanji{024}\kanji{067}\kanji{048}一个新\kanji{085}都去\kanji{051}造\kanji{086}\kanji{085}\,\underlines{已\kanji{002}来不及了} .
\item 副詞「就」+ 形容詞である時,状態の変化を表す.\\例文:大手の広告代理店が1番や2番を引き当てると,中小の広告代理店への影響が\underlines{大きい}\,.\\訳文:大广告代理店一旦抓到1号,2号,\kanji{067}中小广告代理店的影\kanji{087}\,\underlines{就大了} .
\item 程度副詞「太」+ 形容詞である時,程度が高いあるいは過ぎるという話者の気持ち(語気)を表す.\\例文:とにかく大学の数が\underlines{多すぎる} .\\訳文:\kanji{063}之,大学\underlines{太多了} .
\item 副詞「一下子」を使う時,話者が短い時間に変化または完了したことを表す.\\例文:昭和40年代に始まったマンガブームは昭和50年代にはテレビや映画も巻き込んで\underlines{ぱっと広がる} .\\訳文:从昭和40年代\kanji{014}始的漫画\kanji{088}潮,到昭和50年代把\kanji{029}\kanji{003},\kanji{029}影都卷了\kanji{090}来,\underlines{一下子\kanji{091}展\kanji{014}了} .
\item 動詞「成」+ 名詞目的語である時,事柄の変化または完了を表す.\\例文:輸入ものを養殖池に入れ,日本の水を吸わせれば堂々の\underlines{日本産だ} .\\訳文:将\kanji{090}口\kanji{093}\kanji{094}放\kanji{090}\kanji{095}\kanji{094}池内,\kanji{096}它\kanji{028}吸\kanji{092}日本的水,\kanji{006}\kanji{080}便\underlines{成了}堂堂的\underlines{日本\kanji{097}} .
\item 動詞「\kanji{232}」+ 形容詞である時,変化の意味を表す.\\例文:今年も,春闘では,労働時間の短縮が大きな目標に掲げられているが,時短が\underlines{進む} .\\訳文:今年的``春斗''仍将\kanji{098}短\kanji{083}\kanji{078}\kanji{055}\kanji{064}作\kanji{024}一大目\kanji{099}提出,但\kanji{083}\kanji{078}\kanji{055}\kanji{064}\,\underlines{\kanji{232}短了} .\\
\end{enumerater}

以上の条件はすべて論理ORの関係であるから,判断の順番は関係しない.ただし,これらの条件は二つ以上が共存する場合では,完了を表す「了1」を対応させるか,変化を表す「了2」を対応させるかに関しては,判断条件の順番が影響する.本論文では,三つ以上が共存することは無いと考え,二つが共存する場合についてこれらの場合を二つずつ比較し,三角表で優先順位を考察した(図4).「了1/了2」の使用判断に影響するのが強いほうを表中に書く.「×」は不可能な組み合わせ,Nは名詞目的語,Aは形容詞である.優先順位は表4に処理順序で示す.\\

\begin{figure}[hbtp]
\begin{center}
  \epsfile{file=figure/4.eps,width=140mm,height=40mm}
\caption{「ル」形で「了」を使う判断条件の優先順位の考察}
\label{fig:4}
\end{center}
\end{figure}


以上の考察をまとめ,「ル」形を「了」に訳す判定アルゴリズムを作成した(表4).この中で,副詞「一下子」がある場合,動詞「成」+ 名詞目的語の場合,副詞「已\kanji{002}」がある場合は,計算機処理のため,「了1」にしたが,「了2」にしてもかまわない.これらの条件を満たさないなら,他の「ル」形は$\phi$とする.


\begin{table}[htbp]
 \label{HYO4}
 \caption{「ル」→「了」の翻訳を決める手順}
 \begin{center}
 \def\arraystretch{}
 \begin{tabular}{|c|l|c|c|} \hline
  処理順序 & \multicolumn{1}{|c|}{判 別 条 件} & 処理    & 「了」意味用法 \\ \hline\hline
                                    & (中)副詞「一下子」がある場合     & 「了1」 & 完了・変化 \\
  \cline{2-2}\cline{3-3}\cline{4-4} \raisebox{0.5\normalbaselineskip}[0pt][0pt]{1} & (中)副詞「太」+ 形容詞の場合     & 「了2」 & 語気 \\ \hline
                                    & (中)動詞「成」+ 名詞目的語の場合 & 「了1」 & 完了・変化 \\
  \cline{2-2}\cline{3-3}\cline{4-4} 2 & (中)動詞「\kanji{232}」+ 形容詞の場合 & 「了2」 & 変化 \\
  \cline{2-2}\cline{3-3}\cline{4-4}  & (中)副詞「就」+ 形容詞の場合    & 「了2」 & 変化 \\ \hline
  3        & (中)副詞「已\kanji{002}」がある場合            & 「了1」 & 完了・変化 \\ \hline
  4        & (日)「なる」系                       & 「了2」 & 変化 \\ \hline
 \end{tabular}
 \end{center}
\end{table}

\section{評価}

\subsection{評価資料}
「日本報刊選読 (日中文対照) 」(蘇1995){1989年〜1993年の「読売新聞」,「朝日新聞」,「毎日新聞」などに載った46新聞記事の日中対訳集}中の1412文を対象として,各文の述語に対して第3章のアルゴリズムを手作業で評価した.

\subsection{評価結果}
評価した結果を表5に示す.

\begin{table}[htbp]
 \label{HYO5}
 \caption{評価結果}
 \begin{center}
 \def\arraystretch{}
 \begin{tabular}{|c|c|c|c|c|c|} \hline
  翻訳アルゴリズム & 評価文数 & A: 一致文数   & B: 容認文数 & A+B: 正解文数 & 誤った文数 \\ \hline\hline
「タ」形           & 432      & 335(77.5\,\%) & 49          & 384(88.9\,\%) & 48         \\ \hline
「テイル」形       & 248      & 186(75.0\,\%) & 33          & 219(88.3\,\%) & 29         \\ \hline
「ル」形           & 732      & 596(81.4\,\%) & 35          & 631(86.2\,\%) & 101        \\ \hline
 \end{tabular}
 \end{center}
\end{table}

ここで一致文数は,アルゴリズムの結果と「日本報刊選読」の中国語訳文に使われている「了」,「着」,「在」,$\phi$とが一致した文の数である.Bは,一致はしていないが,機械翻訳としては意味的に大きな問題はなく正解と考えてもよいと筆者が判断した文の数である.正解文数はこの二つを加えたものである.


\subsection{誤り分析}

\paragraph{(1) 「タ」形の翻訳アルゴリズムの誤り分析} 

「タ」形の翻訳では48文の誤りがあった.以下にそられの誤り分析を行う.

\paragraph{A. 事態変化を表すときの問題 (3/48)} \\
文1:仮にこれらの問題が新実験線で解決され,実用化の\underlines{メドがついた}としても,総額数兆円の規模にのぼる``中央リニア新幹線''の財源を,どこがどう負担するのか.\\
訳文:即使\kanji{006}些\kanji{031}\kanji{032}通\kanji{001}新\kanji{039}\kanji{100}\kanji{101}而得到解决,从而\kanji{039}用化\kanji{031}\kanji{032}\,\underlines{有了\kanji{102}\kanji{103}} ,\kanji{063}金\kanji{104}\kanji{082}数万\kanji{105}日元的``中央里尼\kanji{106}新干\kanji{101}''建\kanji{107}\kanji{002}\kanji{108}又由\kanji{109}里,如何来承担\kanji{007}?\\
誤り訳:即使\kanji{006}些\kanji{031}\kanji{032}通\kanji{001}新\kanji{039}\kanji{100}\kanji{101}而得到解决,从而\kanji{039}用化\kanji{031}\kanji{032}\,\underlines{有\kanji{102}\kanji{103}} ,\kanji{063}金\kanji{104}…\\
文2:吉田英司店長は,「カラオケは,女性にも娯楽の一つとして\underlines{定着しました} .\\
訳文:店\kanji{053}吉田英司\kanji{046}:``\kanji{265}拉OK也\underlines{成\kanji{024}}\,\kanji{110}女\kanji{028}的一\kanji{079}\kanji{111}\kanji{112}\,\underlines{了} .\\
誤り訳:店\kanji{053}吉田英司\kanji{046}:``\kanji{265}拉OK也\underlines{成\kanji{024}}\,\kanji{110}女\kanji{028}的一\kanji{079}\kanji{111}\kanji{112}.\\
\\
\hako{分析}:中国語静態動詞(存在動詞)「有(…\kanji{102}\kanji{103})」と静態動詞(属性動詞)「成\kanji{024}」に対して,アルゴリズムでは「了」を使わないという判断になる.しかし,これらの文は事態変化の意味を表しており,「了」が必要である.

\paragraph{B. 日本語連体節の問題 (9/48)} \\
文3:シャワーからたっぷり湯の出るOLの「朝シャン」風景に対し,東京都が自粛を\underlines{求めた}結果だった.\\
訳文:(\kanji{006}部广告片)是女\kanji{065}\kanji{025}在\kanji{113}\kanji{113}地\kanji{114}着\kanji{088}水的淋浴器下``朝香''的\kanji{115}\kanji{102},\kanji{067}此\kanji{116}京都政府\underlines{提出了}自我\kanji{117}束的要求.\\
誤り訳:…\kanji{067}此\kanji{116}京都政府\underlines{提出}自我\kanji{117}束的要求.\\
\\
\hako{分析}:アルゴリズムでは日本語の連体節は中国語でも連体節に翻訳されるものと考え,従って「了」を使わないとしている.しかし,これらの誤ったケースでは連体節に翻訳されない.(「選読」では連体節に翻訳されていない.また,連体節に翻訳しようとすると不自然になる.)

\paragraph{C. 日本語「〜なった」の問題 (10/48)} \\
文4:「\underlines{苦しくなった}」理由としては65.5\,\% が「家計にゆとりがなくなった」を挙げている.\\
訳文:65.5\,\%的人提出`` \underlines{不好\kanji{001}} ''的原因是``家庭收支\underlines{不\kanji{119}裕} ''.\\
誤り訳:65.5\,\%的人提出`` \underlines{不好\kanji{001}了} ''的原因是``家庭收支\underlines{不\kanji{119}裕了} ''.\\
\\
\hako{分析}:アルゴリズムでは日本語の「〜なった」の表現は変化を表すものと考えて「了1/2」を使うとしたが,この例のように中国語訳語では変化を表す意味を持たず,状態を表す場合があり,その場合には「了」は使われない.

\paragraph{D. 中国語「開始+V」の問題 (6/48)} \\
文5:給食指導のあり方を見直すため,今年から「中堅栄養職員研修会」を\underlines{スタートさせた} .\\
訳文:\kanji{024}了改革供餐指\kanji{248}方\kanji{120},今年\underlines{\kanji{014}始}\,\kanji{061}\kanji{121}``中\kanji{122}\kanji{057}\kanji{095}\kanji{065}\kanji{025}研修会'',\kanji{026}\kanji{123}\kanji{124}少吃剩\kanji{016}象.\\
誤り訳:…今年\underlines{\kanji{014}始了}\,\kanji{061}\kanji{121}``中\kanji{122}\kanji{057}\kanji{095}\kanji{065}\kanji{025}研修会'',\kanji{026}\kanji{123}\kanji{124}少吃剩\kanji{016}象.\\
\\
\hako{分析}:中国語の「\kanji{014}始」は動態動詞(動作行為動詞)であるが,後ろに動詞がくると「了」を使わない.「開始動詞」という新しい分類を追加した方がよい.

\paragraph{E. 他の問題 (20/48)} \\
文6:日本人は明治の近代化にあたって外国の思想などを紹介するのに,懸命になって漢字による日本語の訳語を\underlines{生みだしました} .\\
訳文:日本人在明治\kanji{055}代\kanji{153}\kanji{016}代化\kanji{055},\kanji{024}了介\kanji{266}外国思想,都力求\underlines{\kanji{051}造}采用\kanji{125}字的日\kanji{126}\kanji{127}.\\
誤り訳:…\kanji{024}了介\kanji{266}外国思想,都力求\underlines{\kanji{051}造了}采用\kanji{125}字的日\kanji{126}\kanji{127}.\\
\\
\hako{分析}:中国語「\kanji{051}造」は動態動詞であるが,副詞「力求」で修飾されると,過去に完了したことでも「了」を使わなくなる.\\
\\
文7:清水寺は境内隣接地のマンション建設を食い止めるため,10億円でその用地を\underlines{買い取った}\,.\\
訳文:清水寺\kanji{024}了制止在\kanji{267}接的土地上修建公寓,花\kanji{108}10\kanji{105}日元将那\kanji{128}地皮\underlines{\kanji{010}了下来} .\\
誤り訳:清水寺\kanji{010}了制止在\kanji{267}接的土地上修建公寓,花\kanji{108}10\kanji{105}日元将那\kanji{128}地皮\underlines{\kanji{010}下来} .\\
\\
\hako{分析}:アルゴリズムでは趨向補助語を伴う時は既にアスペクトが趨向補助語で表現されているので,「了」を使わないとしたが,この文では従属節が目的を表現しているので主節の動詞に「了1」を使わないと将来のテンスを表現することになる.

\paragraph{(2) 「テイル/テイタ」形の翻訳アルゴリズムの誤り分析}
\paragraph{A. 日本語「〜なっている/た」の問題 (2/29)} \\
文1:日本人がさらに\underlines{長生きになっている}ことが判明した.\\
訳文:\kanji{129}\kanji{130}表明,日本人更加\underlines{\kanji{053}寿了} .\\
誤り訳:\kanji{129}\kanji{130}表明,日本人更加\underlines{\kanji{053}寿} .\\
\\
\hako{分析}:中国語では形容詞(静態動詞)には一般に「了」を使わないが,変化の意味が付加されると「了」を使う場合がある.しかしアルゴリズムでは形容詞(静態動詞)に対しては,既定値として$\phi$と定めているのみで,「了」を使う場合の条件判断ができていない.

\paragraph{B. 他の問題 (23/29)} \\
文2:実は,銀行には国際化と自由化の荒波が\underlines{押し寄せている} .\\
訳文:原来,国\kanji{131}化和自由化的\kanji{132}\kanji{132}浪潮\underlines{正}向\kanji{004}行\underlines{冲来} .\\
誤り訳:原来,国\kanji{131}化和自由化的\kanji{132}\kanji{132}浪潮向\kanji{004}行\underlines{冲来了} .\\
\\
\hako{分析}:アルゴリズムでは中国語訳語が結果動詞であれば,結果の残存の意味を表すと考えて,「了」を使うことにしているが,結果動詞であっても進行中の意味を表す「正(在)」を使う場合がある.その条件判断ができていない.

\paragraph{(3) 「ル」形の翻訳アルゴリズムの誤り分析}
\paragraph{A. 副詞「就」がある場合の問題 (7/101)} \\
文1:「本を読んでいる」が32\,\%などの順で,(車内)混雑すればするほどポスターを見る傾向が\underlines{強い} .\\
訳文:``看\kanji{020}''的占32\,\%,\kanji{030}里越\kanji{134},\kanji{135}于看广告的人\underlines{就}越多.\\
誤り訳:``看\kanji{020}''的占32\,\%,\kanji{030}里越\kanji{134},\kanji{135}于看广告的人\underlines{就}越多\underlines{了} .\\
\\
\hako{分析}:「就」+ 形容詞の場合は変化の意味を表し,アルゴリズムでは「了2」を使うとした.しかし,文1で用いられている「越 ... 越 ...」は動作または変化がますます深まった状態・性質を表すので「了」は使われない.

\paragraph{B.「なる」系の問題 (28/101)} \\
文2:二つがいっしょになって大きな銀行となり,多くの預金を集め,多くの企業などに貸し出しすることができるので,利益は\underlines{多くなる} .\\
訳文:\kanji{045}家并\kanji{024}一家大\kanji{004}行之后,由于可\kanji{270}加存款\kanji{104}和企\kanji{040}\kanji{097}款\kanji{104},所以利益\underlines{很大} .\\
誤り訳:\kanji{045}家并\kanji{024}一家大\kanji{004}行之后,由于可\kanji{270}加存款\kanji{104}和企\kanji{040}\kanji{097}款\kanji{104},所以利益\underlines{很大了} .\\
\\
\hako{分析}:アルゴリズムでは日本語の「なる」系は変化の意味を表すので「了」を使うとした.しかし,この訳文の全体は利益の性質を表し,「了」を使わない.

\paragraph{C. 他の問題 (66/101)} \\
文3:(京都は)4年後に建都1200年を\underlines{迎える} .\\
訳文:再\kanji{001}四年,京都就要迎接建都一千二百年\underlines{了} .\\
誤り訳:再\kanji{001}四年,京都就要迎接建都一千二百年.\\
\\
\hako{分析}:表4以外の「ル」形はアルゴリズムで「了」を使わないとした.しかし,この文は未来に完了することを表し,「了」を使うものになる.


\section{終わりに}
人間が「タ/ル/テイル/テイタ」を中国語アスペクト助字に翻訳する際には,文脈と述語自身の時間性格などの要素および諸要素間の優先関係などで総合的な意味を理解して判断する.本研究では,機械翻訳の立場から,表層の文法情報と述語の性格分類のデータをもとに計算機で中国語の訳語を定める手順について考察した.

まず,両言語におけるテンス・アスペクトの性格・意味用法の研究成果を調査し,比較・整理を行って,意味用法間の対応関係について考察した.次に,両言語の文法特徴・共起情報,中国語述語の時間的性格を主な手がかりとして,「タ/ル/テイル/テイタ」と中国語アスペクト助字の対応関係を定めるアルゴリズムを作成した.最後に日中対訳文集を材料として,作成した翻訳アルゴリズムを手作業で評価した.評価実験の結果は,正解率は約八割であり,本稿で提案したテンス・アスペクトの翻訳処理手法の有効性が確認できた.

さらに翻訳の精度が上がるようにアルゴリズムを整備することと,これらのアルゴリズムを我々の翻訳システム\cite{Imai2002,Imai2003}に組み込んでいくことが今後の課題である.


\bibliographystyle{jnlpbbl}
\bibliography{paper}

\appendix
\section*{評価資料に現れた中国語述語の時間的性格分類リスト}

\paragraph{・動態動詞} \\
\maru{1}動作行為動詞\\
安排,委托,慰\kanji{031},影\kanji{087},往返,卸,加,加班,画,戒(烟),学,学\kanji{043},冠以,刊行,看,寄(信),寄宿,起(作用),吃,吸,供,叫,教,研究,糊口,公演,公布,工作,行\kanji{078},合并,采\kanji{136},参与,指\kanji{137},指\kanji{138},支援,施行,写,出差,出售,升,商量,唱,招待,照射,照搬,笑,上升,推广,睡\kanji{139},制作,征集,整理,接送,撰写,租借,操心,争\kanji{269},装,走,送,打(球),打\kanji{140},置\kanji{171},采取,着手,注意,注目,跳舞,追\kanji{271},提供,展望,努力,搭(\kanji{030}),等,答\kanji{037},宣\kanji{141},逃\kanji{049},播映,播放,波及,表示,表明,表\kanji{142},表\kanji{082},扶持,保持,募集,包\kanji{073},放映,翻\kanji{126},\kanji{144}\kanji{145},蔓延,面\kanji{146},利用,警告,命令,遭受,留学,倡\kanji{023},做,听,哭,喊,喘气,找,拿,撼\kanji{078},敲,靠,\kanji{061}(例),\kanji{061}行,\kanji{010},吻,\kanji{141}播,\kanji{147},\kanji{124},\kanji{051}造,\kanji{148}\kanji{046},\kanji{121},\kanji{121}理,\kanji{149},拍,\kanji{015}火,\kanji{015}展,\kanji{015}\kanji{225},\kanji{150}叫,\kanji{071}理,\kanji{039}行,\kanji{022}\kanji{023},\kanji{014},\kanji{014}始,\kanji{014}展,\kanji{014}列,\kanji{014}辟,\kanji{014}\kanji{015},\kanji{151}\kanji{129},\kanji{152}\kanji{038},\kanji{153},\kanji{154}示,\kanji{155}迎,\kanji{156}集,\kanji{157}算,\kanji{158},\kanji{159}透,\kanji{160},\kanji{161}正,\kanji{002}\kanji{162},\kanji{002}\kanji{057},\kanji{163},\kanji{164}定,\kanji{165}算,\kanji{042}可,\kanji{008}(\kanji{166}),\kanji{008}\kanji{009},\kanji{167}\kanji{031},\kanji{168},\kanji{084}\kanji{169},\kanji{107},\kanji{026},\kanji{046},\kanji{046}明,\kanji{046}\kanji{170},\kanji{145},\kanji{129}\kanji{130},\kanji{138}\kanji{171},\kanji{172}予,\kanji{090}行,\kanji{173}售,\kanji{031},\kanji{012}\kanji{117},\kanji{174}演,禁止,忠告,要求,邀\kanji{175},\kanji{175}求,\kanji{206}呼,持\kanji{081},称呼,假装,\kanji{176}\kanji{176}欲\kanji{026}\\
\\
\maru{2}動作状態動詞\\
空,迎接,写,住,住宿,穿,停,保留,刮,挂,收,矗立,\kanji{118},吊,站,\kanji{095},\kanji{073},\kanji{014}\kanji{107},\kanji{178}\kanji{179},\kanji{180},\kanji{181}\kanji{182},\kanji{183},\kanji{184},\kanji{185}\kanji{186},\kanji{187}\\
\\
\maru{3}心理活動動詞\\
回\kanji{011},考\kanji{188},肯定,自\kanji{042},想,打算,忘,体会,着想,同意,理解,了解,估\kanji{165},决定,\kanji{165}\kanji{177},\kanji{042}\kanji{189},\kanji{012}料\\
\\
\maru{4}移動動詞(趨向動詞)\\
出去,出来,来,到,回,去

\paragraph{・静態動詞} \\
\maru{1}属性動詞\\
(以…)\kanji{024}…,蔚然成\kanji{192},下下停停,居,共有,叫做,限于,高\kanji{082},合\kanji{165},作(作\kanji{024}) ,是,占,属,属于,成\kanji{024},定\kanji{024},当作,不同,不如,富有,包括,缺乏,\kanji{024},\kanji{082},\kanji{001},\kanji{193}先,\kanji{050}次\kanji{052}比,\kanji{194}名,笑\kanji{195}\kanji{195},停滞不前,用于,已婚,徘徊不前,\kanji{196}有尽有,不相上下,\kanji{112}不可支,默不作声,\kanji{135}于,\kanji{135}向,陷于,需要,蜂\kanji{013}而至,火上加油,可\kanji{003}一斑,冲突,\kanji{162}任,取而代之,\kanji{191}保\\
\\
\maru{2}存在動詞\\
活,在,在于,在\kanji{200},不在,没有,有,有\kanji{060},\kanji{171}有,\kanji{018}有,\kanji{013}有\\
\\
\maru{3}心理状態動詞\\
以期,以求,以\kanji{024},感到,感\kanji{072}趣,感\kanji{201},看待,喜\kanji{155},希望,后悔,熟悉,信任,相信,尊敬,担心,知道,佩服,\kanji{139}得,\kanji{042}\kanji{024}\\
\\
\maru{4}形容詞\\
一致,快,快\kanji{112},活生生,活\kanji{176},吃\kanji{206},共同,激烈,固定,好,好\kanji{001},幸福,高,高\kanji{072},自由,寂静,弱,受\kanji{155}迎,准\kanji{202},小,少,深孚\kanji{203}望,水泄不通,成功,静,清楚,生机\kanji{204}然,先\kanji{090},争奇斗\kanji{205},相\kanji{196},多,大,担\kanji{206}受怕,超群出\kanji{203},低,泥\kanji{207},登峰造\kanji{208},突出,如火如荼,悲\kanji{209},漂亮,不安,不足,\kanji{210}用,普及,慢,猛烈,有限,来得及,沮\kanji{211},\kanji{273},\kanji{212}\kanji{039},\kanji{014}心,\kanji{072}隆,来不及,\kanji{213}具一格,\kanji{119}裕,\kanji{119}敞,\kanji{214}底,\kanji{154}眼,\kanji{154}著,\kanji{272},\kanji{088}烈,\kanji{191}定,\kanji{215}\kanji{178},\kanji{117}定俗成,\kanji{053}寿,\kanji{036},大\kanji{216}全\kanji{217}

\paragraph{・結果動詞} \\
\maru{1}「V+V」構造\\
下降,下跌,采用,取得,到\kanji{082},招收,制定,接受,接\kanji{218},逃脱,听取,捐献,\kanji{270}加,\kanji{060}\kanji{219},\kanji{051}\kanji{121},\kanji{014}\kanji{121},\kanji{220}用,\kanji{221}制,\kanji{222}放,\kanji{270}\kanji{053}(zhang3)\\
\\
\maru{2}「V+Adj」構造\\
延\kanji{053},加速,看破,固定,降低,湿透,推\kanji{014},睡着,打乱,提高,登高,腐\kanji{223},放\kanji{014},泡大,用尽,冲淡,站\kanji{191},\kanji{270}多,\kanji{270}大,\kanji{224}斜,\kanji{124}少,\kanji{091}大,\kanji{153}\kanji{226},\kanji{098}短,\kanji{046}服,弄\kanji{226},打碎\\
\\
\maru{3}「V+趨向補助語」構造\\
引起,看出,叫\kanji{014},遇到,迎来,跨\kanji{090},使出,失去,写出,升起,伸出,推出,染上,前往,前来,超\kanji{001},跳入,提出,度\kanji{001},渡\kanji{001},逃出,撞上,播出,落下,留下,列入,冲来,剩下,售出,掀起,陷入,\kanji{271}上,追上,\kanji{072}起,跟上,\kanji{018}来,\kanji{227}出,\kanji{228}去,\kanji{229}出,\kanji{163}入,\kanji{096}出,逃出来,\kanji{129}往,\kanji{017}入,代之而起,\kanji{090}入,\kanji{230}出,\kanji{231}入\\
\\
\maru{4}瞬間変化動詞\\
化成,改成,看到,看\kanji{003},受到,制成,造就,提到,得到,听\kanji{003},找到,抓住,抛\kanji{014},\kanji{141}到,\kanji{232}成,\kanji{118}掉,\kanji{234}住,\kanji{017}成,\kanji{017}\kanji{235},\kanji{082}到,去世,出\kanji{016},消\kanji{236},想到,停\kanji{121},突破,突\kanji{232},破\kanji{076},曝光,弄湿,冲走,\kanji{044},\kanji{237}死,\kanji{015}生,\kanji{236},\kanji{238}掉,\kanji{238}死,\kanji{229}束,\kanji{229}婚,\kanji{021}婚,咽气,死,死亡,自\kanji{239},成,形成,一\kanji{240}而光,一\kanji{241}登天,下\kanji{242},解决,回国,回\kanji{243},改任,改\kanji{245},外出,完成,幸免,合理化,坐\kanji{001}站,作(分析),自\kanji{239},辞\kanji{065},失常,出生,出版,升\kanji{246},倒\kanji{197},淘汰,到\kanji{082},得(病,\kanji{198}分),得救,\kanji{233}\kanji{209},\kanji{259},\kanji{027}\kanji{243},\kanji{264}成,\kanji{229}冰,\kanji{217}利,\kanji{199}\kanji{270},\kanji{015}行,\kanji{015}\kanji{016},\kanji{015}\kanji{247},\kanji{232},\kanji{248}致,\kanji{014}\kanji{178},\kanji{249}\kanji{040},淡化,中止,\kanji{164}范化,\kanji{250}生,\kanji{251}\kanji{030},翻番,迷路,猛\kanji{270},乱套,通\kanji{001},拒收,限制,借,集中,住院,出借,商定,承\kanji{042},招\kanji{252},尽力,成,晴,生,精疲力竭,相\kanji{189},停(\kanji{029}),定,表\kanji{016},分送,包括,包租,冷落,收走,缺席,\kanji{043}\kanji{253},\kanji{254}及,\kanji{255}水,\kanji{256}\kanji{108}(苦心),\kanji{003},\kanji{021}做,\kanji{107}立,\kanji{257}苦,\kanji{201}罪,\kanji{090},\kanji{090}京,\kanji{090}口,\kanji{258}\kanji{217},\kanji{053}(zhang3),\kanji{031}及,\kanji{012},定降,参加,新\kanji{107},霹,告\kanji{257}

\newpage
\begin{biography}
\biotitle{略歴}

\bioauthor{謝 軍}{
1993年中国瀋陽工業大学計算機学部卒.
2000年岐阜大学大学院工学研究科電子情報工学専攻修士課程修了.
工学修士.
現在同大学院工学研究科電子情報システム工学専攻博士課程在学中.
日中機械翻訳,中国語処理の研究に従事.
情報処理学会学生会員.}

\bioauthor{ト 朝暉}{
1991年中国広西大学外国語学部日本語科卒.
2001年岐阜大学教育学研究科国語科教育専修修了.
教育学修士.
現在同大学工学研究科電子情報システム工学専攻後期課程に在学中.
日中機械翻訳に興味を持つ.
言語処理学会,情報処理学会各学生会員.}

\bioauthor{池田 尚志(正会員)}{
1968年東大・教養・基礎科学科卒.
同年工業技術院電子技術総合研究入所.
制御部情報制御研究室,知能情報部自然言語研究室に所属.
1991年岐阜大学工学部電子情報工学科教授.
現在,同応用情報学科教授.
工博.
人工知能,自然言語処理の研究に従事.
情報処理学会,電子情報通信学会,人工知能学会,言語処理学会,ACL,各会員.}


\bioreceived{受付}
\bioaccepted{採録}

\end{biography}

\end{document}
