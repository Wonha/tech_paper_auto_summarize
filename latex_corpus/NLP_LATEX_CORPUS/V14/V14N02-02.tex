    \documentclass[japanese]{jnlp_1.3a}
\usepackage{jnlpbbl_1.1}
\usepackage[dvips]{graphicx}
\usepackage{url}
\usepackage{multirow}
\usepackage{hhline}
\usepackage{amsmath}

\setcounter{secnumdepth}{3}

\Volume{14}
\Number{2}
\Month{Apr.}
\Year{2007}
\received{2006}{8}{20}
\revised{2006}{10}{24}
\accepted{2006}{11}{24}

\setcounter{page}{33}

\jtitle{ウェブから収集した専門分野コーパスと要素合成法を用いた\\専門用語訳語推定}
\jauthor{外池 昌嗣\affiref{Kyoto} \and 宇津呂武仁\affiref{Tsukuba} \and 佐藤 理史\affiref{Nagoya}}
\jabstract{
本論文では,ウェブを利用した専門用語の訳語推定法について述べる.
これまでに行われてきた訳語推定の方法の1つに,パラレルコーパス・コンパラ
ブルコーパスを用いた訳
語推定法があるが,既存のコーパスが利用できる分野は極めて限られている.
そこで,本論文では,訳を知りたい用語を構成する単語・形態素の訳語を既存の
対訳辞書から求め,これらを結合することにより訳語候補を生成し,単言語コー
パスを用いて訳語候補を検証するという手法を採用する.
しかしながら,単言語コーパスであっても,研究利用可能なコーパスが整備されて
いる分野は限られている.
このため,本論文では,ウェブをコーパスとして用いる.
ウェブを訳語候補の検証に利用する場合,サーチエンジンを通してウェブ全体を
利用する方法と,訳語推定の前にあらかじめ,ウェブから専門分野コーパスを収
集しておく方法が考えられる.
本論文では,評価実験を通して,この2つのアプローチを比較し,その得失を論
じる.
また,訳語候補のスコア関数として多様な関数を定式化し,訳語推定の性能と
の間の相関を評価する.
実験の結果,ウェブから収集した専門分野コーパスを用いた場合,ウェブ全体を
用いるよりカバレージは低くなるが,その分野の文書のみを利用して訳語候補の検証を
行うため,誤った訳語候補の生成を抑える効果が確認され,高い精度を達成できることがわかった.
}
\jkeywords{要素合成法,訳語推定,ウェブ,専門用語}

\etitle{Compositional Translation Estimation of Technical Terms Using a Domain/Topic-Specific Corpus Collected \\from the Web}
\eauthor{Masatsugu Tonoike\affiref{Kyoto} \and Takehito
	Utsuro\affiref{Tsukuba} \and Satoshi Sato\affiref{Nagoya}} 
\eabstract{
This paper studies how to compile a bilingual
lexicon for technical terms using the Web.
In the task of estimating bilingual term correspondences of technical
terms, it is usually rather difficult to find an existing corpus for the
domain of such technical terms.
In this paper, we adopt an approach of collecting a corpus for the
domain of such technical terms from the Web.
As a method of translation estimation for technical terms, we employ a
compositional translation estimation technique, where
translation candidates of a term are compositionally generated
by concatenating the translation of the constituents of the term.
Then, the generated translation candidates are validated using the
domain/topic-specific corpus collected from the Web.
This paper further quantitatively compares the proposed approach with
another approach of validating translation candidates directly through a
search engine.
We show that the domain/topic-specific
corpus collected from the Web contributes to achieving higher precision
in translation candidate validation.
}
\ekeywords{Compsitional translation estimation, Web, Echnical term}

\headauthor{外池,宇津呂,佐藤}
\headtitle{ウェブから収集した専門分野コーパスと要素合成法を用いた専門用語訳語推定}

\affilabel{Kyoto}{京都大学大学院情報学研究科}{
	Graduate School of Informatics, Kyoto University}
\affilabel{Tsukuba}{筑波大学大学院システム情報工学研究科}{
	Graduate School of Systems and Information Engineering, University of Tsukuba}
\affilabel{Nagoya}{名古屋大学大学院工学研究科}{
	Graduate School of Engineering, Nagoya University}


\begin{document}
\maketitle



\section{はじめに}
\label{sec:intro}


本論文では,ウェブを利用した専門用語の訳語推定法について述べる.
専門用語の訳語情報は,技術翻訳や同時通訳,機械翻訳の辞書の強化などの場面
において,実に様々な分野で求めれらている.
しかしながら,汎用の対訳辞書には専門用語がカバーされていないことが多く,
対訳集などの専門用語の訳語情報が整備されている分野も限られている.
その上,専門用語の訳語情報が整備されていたとしても,最新の用語を追加して
いく作業が必要になる.
このため,あらゆる分野で,専門用語の訳語情報を人手で整備しようとすると,
大変なコストとなる.
そこで,本論文では,対象言語を英語,日本語双方向とし,自動的に専門用語の
訳語推定を行う方法を提案する.

これまでに行われてきた訳語推定の方法の1つに,パラレルコーパスを用いた訳
語推定法がある~\cite{Matsumoto00a}.
しかしながら,パラレルコーパスが利用できる分野は極めて限られている.
これに対して,対訳関係のない同一分野の2つの言語の文書を組にしたコンパラ
ブルコーパスを利用する方法\cite{Fung98as,Rapp99as}が研究されている.
これらの手法では,コーパスにそれぞれ存在する2言語の用語の組に対して,各
用語の周囲の文脈の類似性を言語を横断して測定することにより,訳語対応の推
定が行われる.
パラレルコーパスに比べれればコンパラブルコーパスは収集が容易であるが,訳
語候補が膨大となるため,精度の面で問題がある.
また,この方法では,訳語推定対象の用語を構成する単語・形態素の情報を利用
していない.
これに対して,\cite{Fujii00,Baldwin04multi}では,訳を知りたい用語を構成
する単語・形態素の訳語を既存の対訳辞書から求め,これらを結合することによ
り訳語候補を生成し,単言語コーパスを用いて訳語候補を検証するという手法を
提案している.
(以下,本論文では, 用語の構成要素の訳語を既存の対訳辞書から求め,これ
らを結合することにより訳語候補を生成する方法を「要素合成法」と呼ぶ.)

要素合成法による訳語推定法の有効性を調査するために,既存の専門用語対訳辞
書の10分野から,日本語と英語の専門用語で構成される訳語対を617個抽出した
\footnote{\ref{sec:evaluation_set}節で述べる未知訳語対集合$Y_{ST}$ に対
応する.}.
そして,それぞれの訳語対の日本側の用語と英語側の用語の構成要素が対応して
いるかを調べたところ,88.5\%の訳語対で日英の構成要素が対応しているという結
果が得られた.
このことから,専門用語に対して要素合成法による訳語推定法を適用することは
有効である可能性が高いことがわかった.
(以下,本論文では,訳語対において各言語の用語の構成要素が対応しているこ
とを「構成的」と呼ぶものとする.)

しかしながら,単言語コーパスであっても,研究利用可能なコーパスが整備され
ている分野は限られている.
このため,本論文では,大規模かつあらゆる分野の文書を含むウェブをコーパス
として用いるものとする.
ウェブを訳語候補の検証に利用する場合,\cite{Cao02as}の様に,サーチエンジ
ンを通してウェブ全体を利用して訳語候補の検証を行うという方法がまず考えら
れる.
その対極にある方法として,訳語推定の前にあらかじめ,ウェブから専門分野コー
パスを収集しておくことも考えられる.
サーチエンジンを通してウェブ全体を利用するアプローチは,カバレージに優れ
るが,様々な分野の文書が含まれるため誤った訳語候補を生成してしまう恐れも
ある.
また,それぞれの訳語候補に対してサーチエンジンで検索を行わなければいけな
いため,サーチエンジン検索の待ち時間が無視できない.
これに対して,ウェブから専門分野コーパスを収集するアプローチは,ウェブ全
体を用いるよりカバレージは低くなるが,その分野の文書のみを利用して訳語候
補の検証を行うため,誤った訳語候補を削除する効果が期待できる.
また,ひとたび専門分野コーパスを収集すれば,訳語推定対象の用語が大量にあ
る場合でも,サーチエンジンを介してウェブにアクセスすることなく訳語推定を
行うことができる.
しかしながら,これまで,この2つのアプローチの比較は行われてこなかったた
め,本論文では,評価実験を通して,この2つのアプローチを比較し,その得失
を論じる.

さらに,上記の2つのアプローチの比較も含めて,本論文では,訳語候補のスコ
ア関数として,多様な関数を以下のように定式化する.
要素合成法では,構成要素に対して,対訳辞書中の訳語を結合することにより訳
語候補が生成されるので,構成要素の訳語にもとづいて訳語候補の適切さを評価
する.
これを対訳辞書スコアと呼ぶ.
また,それとは別に,生成された訳語候補がコーパスに生起する頻度に基づいて,
訳語候補の適切さを評価する.
これをコーパススコアと呼ぶ.
本論文では,この2つスコアの積で訳語候補のスコアを定義する.
本論文では,対訳辞書スコアに頻度と構成要素長を考慮したスコアを用い,また,
コーパススコアには頻度に基づくスコアを用いたスコア関数を提案し,
確率に基づくスコア関数\cite{Fujii00}と比較する.
さらに,対訳辞書スコア,コーパススコアとしてどのような尺度を用いるか,
に加え,訳語候補の枝刈りにスコアを使うかどうか,コーパスとしてウェブ全体を用いるか
専門分野コーパスを用いるか,といったスコア関数の設定を変化させて合計12種
類のスコア関数を定義し,訳語推定の性能との間の相関を評価する.

実験の結果,コーパスとしてウェブ全体を用いた場合,ウェブには様々な分野の
文書が含まれるため誤った訳語候補を生成してしまうことが多い反面,カバレー
ジに優れることがわかった.
逆に,ウェブから収集した専門分野コーパスを用いた場合,ウェブ全体を用いる
よりカバレージは低くなるが,その分野の文書のみを利用して訳語候補の検証を
行うため,誤った訳語候補の生成を抑える効果が確認された.
また,ウェブから収集した専門分野コーパスを用いる方法の性能を向上さ
せるためには,専門分野コーパスに含まれる正解訳語の割合を改善することが課
題であることがわかった.

以下,本論文では,第\ref{sec:web_yakugosuitei}章でウェブを用いた専門用語
訳語推定の枠組みを導入し,専門分野コーパスの収集方法について述べる.
第\ref{sec:compo-method}章では要素合成法による訳語推定の定式化を行い,訳
語候補のスコア関数を導入する.
第\ref{sec:experiments}章では実験と評価について述べる.
第\ref{sec:related_work}章では関連研究について述べ,本論文との相違点を論
じる.



\section{ウェブを用いた専門用語訳語推定}
\label{sec:web_yakugosuitei}

\subsection{概要}


\begin{figure}[b]
 \centering
     \includegraphics[width=300pt]{jnlp-compo-fig-overview1.eps}
 \caption{ウェブを用いた専門用語訳語推定}
 \label{fig:overview1}
\end{figure}

ウェブを用いた専門用語の訳語推定の全体像を図\ref{fig:overview1}に示す.
本論文では,言語$S$の専門用語が複数個が与えられたとき,それらの用語に対
して,言語$T$における訳語を推定するという問題を考える.
このような状況としては,例えば,ある専門分野において,まとまった数の専門
文書が与えられ,それらの文書から用語を抽出し,専門用語の対訳辞書を作成す
る場合が考えられる.
あるいは,ある専門分野の文書と既存の汎用対訳辞書があり,この文書を翻訳家
が翻訳したい場合などが考えられる.
ここで,一般に,与えられた複数の専門用語は,既存の汎用対訳辞書に含まれる訳語の
個数にしたがって,訳語が1個である用語の集合$X_S^U$,訳語が2個以上である
用語の集合$X_S^M$,そして,訳語が得られない用語の集合$Y_S$という3つの部
分集合に分けられる.
本論文では,既存の辞書に訳語が1個だけ含まれる用語の集合$X_S^U$の訳語は
正しいと仮定し,集合$X_S^U$の用語の訳語の集合$X_T^U$を用いてウェブから専
門分野コーパスを収集し,訳語推定に利用するものとする.

本論文では,既存の対訳辞書で訳語が得られない用語の集合$Y_S$を訳語推定の
対象とする.
一方,集合$X_S^M$の用語に対しては,既存の対訳辞書にある訳語の中から最も適切な
ものを選択する必要がある.
例えば,論理回路分野に属する日本語の専門用語「レジスタ」の訳語としては,
サッカー用語の``regista''ではなく,用語``register''が選択されなければな
らない.
この訳語選択の課題については,\cite{Tonoike05cs}において,すでに一定の成
果が得られており,ウェブから収集した専門分野コーパスに生起する頻度の最も
大きい訳語を選択することにより,英日方向で69\%,日英方向で75\%の正解率が
得られたと報告されている.
そこで,本論文では,集合$X_S^M$の用語の訳語選択の課題は取り扱わない.


\subsection{専門分野コーパスの収集}
\label{sec:corpus}

本論文では,言語$T$の専門分野コーパスをウェブから収集して訳語推定に利用
する.
この専門分野コーパスを集める際には,既存対訳辞書に訳語が1つだけ存在する
専門用語の訳語の集合$X_T^U$を利用する.
具体的には,集合$X_T^U$に含まれる用語$x_T^U$を含むサーチエンジンのクエリー
を用いてウェブから上位100ページを収集する
\footnote
{本論文では,サーチエンジンを用いる場合,日本語のクエリーの場合はgoo
(\url{http://www.goo.ne.jp})を用い,英語のクエリーの場合はYahoo!
(\url{http://www.yahoo.com/})を用いる.}.
それらのページに,用語$x_T^U$がアンカーテキストとなっているアンカーが存
在する場合は,そのアンカー先ページも入手する.
これを,集合$X_T^U$に含まれる用語すべてに対して行い,収集されたウェブペー
ジを集めて,専門分野コーパスとする.
日本語のコーパスを収集する際に用いたクエリーは,“$x_T^U$とは”,
“$x_T^U$という”,“$x_T^U$は”,“$x_T^U$の”,及び,``$x_T^U$''である.
一方,英語のコーパスを収集する際に用いたクエリはー,``$x_T^U$ AND
what's'',``$x_T^U$ AND glossary'',及び,``$x_T^U$''である.
ここでは,専門用語$x_T^U$について記述されている文書,
例えば,オンライン用語集などを上位にランクするために,経験的にこれらの
クエリーを用いている.

ここで,計算コストや記憶容量の問題を考慮した上で,できるだけ訳語推定の性
能を向上させるためには,訳語推定対象の分野の用語を十分に含むできるだけ小
さいコーパスを収集することが望ましい.
これを実現する方法として,サーチエンジンのクエリーに複数の用語を含めたり,
取得すべきページ数を変更することなどが考えられる.
しかしながら,本論文では,ウェブから収集した専門分野コーパスを利用す
る方式と,サーチエンジンを通してウェブ全体を利用する方式の比較に焦点を当
てるため,訳語推定対象の分野の用語を十分に含むできるだけ小さいコーパスを収
集する方式を確立することは,論文の対象外とする.
この問題に関連する知見としては,\cite{Takagi05aj}の研究がある.
\cite{Takagi05aj}では,評価用の正解訳語の集合を設定し,上記の方法によっ
てウェブから収集したコーパスに,評価用の正解訳語を含む割合の評価を行った.
その結果,サーチエンジンのヒット数に上限を設けてコーパス収集に使用する用
語の数を絞り込むことにより,コーパス収集に使用する用語の数を少なくしても,
評価用の正解訳語を含む割合が下がらないことが報告されている.
また,\cite{Takagi05aj}では,訳語推定対象の用語(および評価用の正解訳語)
とは異なる分野のコーパスを利用する場合についても評価を行っている.
これによると,評価用の正解訳語を含む割合は,訳語推定対象の用語と近い分野
のコーパスを用いた場合は低下しないが,全く異なった分野のコーパスを用いた
場合は低下することを実験的に確かめている.
このことは,訳語推定対象の用語の分野となるべく近い分野のコーパスを用いて
訳語推定をすべきであることを示している.


\section{要素合成法による専門用語訳語推定の定式化}
\label{sec:compo-method}

\subsection{概要}

\begin{figure}[b]
 \centering
     \includegraphics[width=300pt]{jnlp-compo-fig-compositional.eps}
 \caption{日本語の専門用語「応用行動分析」の要素合成法による訳語推定}
 \label{fig:compositional}
\end{figure}

要素合成法による訳語推定の例として,日本語の専門用語「応用行動分析」の訳
語推定の様子を図~\ref{fig:compositional}に示す.
まず,既存の対訳辞書を参照し,日本語の見出しを検索することにより,日本語
の専門用語「応用行動分析」を構成要素に分割する\footnote{ここで,既存の
対訳辞書として,「英辞郎」Ver.79 (\url{http://www.eijiro.jp/})と英辞郎の訳語対から作
成した部分対応対訳辞書(詳細は\ref{sec:bcl}節で述べる)を用いる.}.
この例の場合,構成要素分割の結果は,図~\ref{fig:compositional}に``a''で
示した“応用”,“行動”,“分析”という分割及び,``b''で示した“応用”,
“行動分析”という分割の2種類になる.
次に,それぞれの構成要素を英語に翻訳する.
それぞれの構成要素の訳語には,ある信頼度のスコアが与えられる.
そして,``a'',``b''それぞれの分割に対して,
それらの構成要素の訳語を結合することによって訳語候補を生成する.
この例では,構成要素に与えられているスコアの乗算で訳語候補のスコアを計算
する.
``applied behavior analysis''のように分割``a'',``b''で同じ訳語候補
が生成される場合には,それぞれの分割でスコアを計算し両者のスコアを加算す
るものとした.
分割``a''では,``applied''と``behavior''と``analysis''を結合して
``applied behavior analysis''が生成され,スコアは,$1.6 \times 1 \times
1 = 1.6$となる.
また,分割``b''では,``applied''と``behavior analysis''を結合して
``applied behavior analysis''が生成され,スコアは,$1.6 \times 10=16$となる.
そして,最終的に``applied behavior analysis''のスコアは,$1.6+16=17.6$と
計算される.



\subsection{部分対応対訳辞書の作成}
\label{sec:bcl}

専門用語の訳語推定をするためには,既存の対訳辞書の訳語情報だけでは不十分
である.
複合語中の単語はどのように訳されるのが自然かという情報が重要となる.
そこで,複合語中の単語の訳し方を,既存の対訳辞書の複合語エントリから収集
することを試みる.
一般に対訳辞書のエントリは見出し語と1つ以上の訳語から構成される.
このエントリを展開し,見出し語と訳語を一対一の語の組にしたものを本論文で
は訳語対と呼ぶ.
本節では,\cite{Fujii00}の語基辞書の作成方法を参考にして,既存の対訳
辞書(英辞郎)の複合語の訳語対から,英語及び日本語の用語の構成要素の訳語
対応を推定し,このような訳語対応を集めて新たな対訳辞書を作成する方法につ
いて述べる.
本論文では,この,既存の訳語対の構成要素を利用して作成された対訳辞書を部
分対応対訳辞書と呼ぶ.
既存の対訳辞書を部分対応対訳辞書で補う方法を,
図~\ref{fig:bubunicchi-rei}の例を用いて説明する.
既存の対訳辞書に“applied: 応用”という訳語対自体は含まれないが,1番目
の英単語が``applied''かつ1番目の日本語単語が「応用」であるような複合語
の訳語対が数多く含まれていると仮定する\footnote{日本語のエントリは,形
態素解析器JUMAN (\url{http://nlp.kuee.kyoto-u.ac.jp/nl-resource/juman.html})で
形態素列に分解されているものとする.}.
このようなとき,それらの訳語対を対応付け,構成要素の訳語対応“applied: 応用”を推定する.

\begin{figure}[t]
 \small
 \centering
  \begin{tabular}{ccccc}
   applied & mathematics & : & 応用 & 数学\\
   applied & science & : & 応用 & 科学\\
   applied & robot & : & 応用 & ロボット\\
   & & $\vdots$ & &  頻度\\
   & & $\Downarrow$ & &  ↓\\
   \hline
   \multicolumn{1}{|c}{applied} & & : & 応用 & \multicolumn{1}{c|}{: 40}\\
   \hline
  \end{tabular}
\vspace{8pt}
 \caption{構成要素の訳語対の推定の例(前方一致)}
 \label{fig:bubunicchi-rei}
\end{figure}

より詳細には,既存の対訳辞書から,まず,日本語及び英語の用語がそれぞれ2
つの構成要素からなる訳語対を抽出し,これを別の対訳辞書$P_2$とする.
次に,$P_2$中の訳語対から英語及び日本語の第一構成要素だけを抜き出して訳
語対とし,これを集めて構成した対訳辞書を前方一致部分対応対訳辞書$B_P$と
呼ぶ.
同様に,$P_2$中の訳語対から英語及び日本語の第二構成要素だけを抜き出して訳
語対とし,これを集めて構成した対訳辞書を後方一致部分対応対訳辞書$B_S$と
呼ぶ.
本論文では,部分対応対訳辞書$B_P$と$B_S$に,以下の2つの制約を課す.
\begin{itemize}
 \item 前方一致部分対応対訳辞書$B_P$は,用語の先頭および中間位置の構成要
       素の訳語を得る場合にのみ参照することとし,用語の最後尾の構成要素
       の訳語を得るために参照することはできない.
 \item 後方一致部分対応対訳辞書$B_S$は,用語の中間位置および最後尾の構成
       要素の訳語を得る場合にのみ参照することとし,用語の先頭の構成要素
       の訳語を得るために参照することはできない.
\end{itemize}
これらの制約は,不適切な訳語候補が生成されるのを防ぐために課した.
なお,\cite{Fujii00}においては,$B_P$と$B_S$を統合した部分対応対訳辞書
(以下,本論文では,この辞書を部分対応対訳辞書$B$と呼ぶ)を
作成しており,上記のような制約を課していない.
$B_P$と$B_S$を統合すると訳語対の数が増える利点はあるが,例えば,“システ
ム応用”という用語の訳語推定を行うときに,第二構成要素である“応用”の訳
語を得るために,$B_P$に含まれる訳語対〈“応用”,``applied''〉が参照されるなど,過剰に参照される恐れがある.

そこで,実際に,部分対応対訳辞書に$B_P$及び$B_S$を利用する場合と,$B$を
利用する場合の比較を行った.
まず,英辞郎と部分対応対訳辞書を用いて,与えられた用語に対して正解訳語が
生成できるかどうかの評価を行った.
詳細は\ref{sec:evaluation_set}節で述べるが,部分対応対訳辞書として$B$を
用いた場合,$B_P$及び$B_S$を用いた場合に比べて,正解訳語が生成できる用語
の割合は2\%程度しか上回らなかった.
また,訳語推定の性能においては
\footnote{
詳細は\ref{sec:evaluation}節で述べるが,訳語候補のスコア付けの方法として
は,ウェブから収集した専門分野コーパスを利用する方法の中では総合的に最も
性能のよかった`DF-CO'というスコア関数を用いた.
},
英日方向では再現率は$B$を用いる場合の方が1\%程度高く,逆に精度
は$B_P, B_S$を用いる場合の方が1\%弱高かった.
一方,日英方向では,$B_P, B_S$を用いる場合の方が,再現率は1\%弱高く,精
度は数\%高かった.
この結果を総合的に判断して,本論文では,部分対応対訳辞書として$B_P, B_S$
を用いることとした.
表\ref{tab:entry_number}に,英辞郎,対訳辞書$P_2$,および,部分対応対訳
辞書$B_P, B_S, B$における,見出し語の数
\footnote{英辞郎は英日辞書であるので,本来は日本語の見出し語は存在しない.
本論文では英辞郎を編集することによって日英版を作成したので,表
\ref{tab:entry_number}にはその見出し語数を掲載している.}
と訳語対の個数を示す.




\begin{table}[t]
\small
 \centering
 \caption{対訳辞書の見出し語数と訳語対数}
 \label{tab:entry_number}
 \begin{tabular}{|c|r|r|r|}
  \hline
  \multirow{2}{*}{対訳辞書} & \multicolumn{2}{c|}{見出し語数} &
  訳語対の個数\\
  \hhline{|~|--|~|}
  & \multicolumn{1}{c|}{英語} & \multicolumn{1}{c|}{日本語} & \\
  \hline
  英辞郎 & 1,292,117 & 1,228,750 & 1,671,230 \\ 
  $P_2$  &   217,861 &   186,823 &   235,979 \\
  $B_P$  &    37,090 &    34,048 &    95,568 \\
  $B_S$  &    20,315 &    19,345 &    62,419 \\
  $B$    &    48,000 &    42,796 &   147,848 \\
  \hline
 \end{tabular}

\begin{tabular}{ccl}
 英辞郎 & : & 既存の汎用対訳辞書 (Ver.79)\\
 $P_2$  & : & 両言語とも2構成要素からなる英辞郎の訳語対の集合\\
 $B_P$  & : & 前方一致部分対応対訳辞書$B_P$\\
 $B_S$  & : & 後方一致部分対応対訳辞書$B_S$\\
 $B$    & : & 部分対応対訳辞書$B$
\end{tabular}
\end{table}

\subsection{訳語候補の生成}
\label{sec:generation}


本節及び次節において,\ref{sec:bcl}節で準備した対訳辞書及
び,\ref{sec:corpus}節で述べた専門分野コーパスまたはウェブ全体を利用して,
与えられた専門用語の訳語推定を要素合成法によって行う方法の詳細を述べる.
本節では,要素合成法により訳語候補を生成する過程を定式化する.
そして,次節で,本論文で実際に評価した訳語候補のスコア関数の詳細について
述べる.

まず,$y_S$を訳語推定すべき専門用語とする.
ここで,$S$が英語であれば$w_i$を単語,$S$が日本語であれば$w_i$を形態素と
して,$y_S$は以下のように$w_i$の列で表される.
\begin{equation}
 \label{eq:y_S}
 y_S=w_1,w_2,\ldots,w_m
\end{equation}
例えば,$y_S$が“応用行動分析”であれば,$w_1=$“応用”,$w_2=$“行動”,
$w_3=$“分析”となる.
要素合成法では,対訳辞書中の訳語対の見出し語と照合する用語は,一個以上の
単語もしくは形態素から構成されると考える.
そして,$y_S$を一個以上の単語もしくは形態素から構成される単位に分割し,
各単位の訳語を結合することにより訳語候補を生成する.
以下,まず$y_S$を上記の単位$s_j$の列に分割する.
\begin{equation}
 y_S=w_1,w_2,\ldots,w_m \equiv s_1,s_2,\ldots,s_n
\end{equation}
ただし,各$s_j$は一個以上の$w_i$の列を表す.
例えば,$y_S$を“応用 行動 分析”とすると,以下の3通りの分割が考えられる
\footnote{\ref{sec:evaluation_set}節で導入する評価用用語集合では,訳語推定
すべき用語$y_S$全体が英辞郎に含まれる用語は除外している.これを除くと,
この例の場合,3通りの分割が考えられる.}.
\begin{equation}
 \begin{array}{l}
  \label{eq:bunkatsu_ouyou}
  s_1=\mbox{“応用”,}\ s_2=\mbox{“行動 分析”}\\
  s_1=\mbox{“応用”,}\ s_2=\mbox{“行動”,}\ s_3=\mbox{“分析”}\\
  s_1=\mbox{“応用 行動”,}\ s_2=\mbox{“分析”}
 \end{array}
\end{equation}
また,$y_S$を``applied behavior analysis''とすると,以下の3通りの分割が
考えられる
\begin{equation}
 \begin{array}{l}
  \label{eq:bunkatsu_applied}
  s_1=\mbox{``applied'',}\ s_2=\mbox{``behavior analysis''}\\
  s_1=\mbox{``applied'',}\ s_2=\mbox{``behavior''},\ s_3=\mbox{``analysis''}\\
  s_1=\mbox{``applied behavior'',}\ s_2=\mbox{``analysis''}
 \end{array}
\end{equation}
次に,対訳辞書から得られた$s_i$の訳語を$t_i$とすると,$y_S$の訳語候
補$y_T$は,以下のように$y_S$と同じ語順で構成される
\footnote{
\label{fn:of_hyphen}
英語の専門用語の中には``angle of radiation''のように前置詞を含むものがある.
この用語の日本語訳語は「放射 角」であるが,英語用語と日本語用語の間で双
方向に適切な訳語候補として生成でき
るようにするためには,``of''の前後の語順の入れ替えや``of''を挿入または削除
する操作を考慮する必要がある.
本論文では,上記の場合も含めて,以下の様な英語・日本語の用語の組において,
双方向に訳語候補の生成ができるような規則を実装した.
\begin{eqnarray*}
 \mbox{英語の用語\ \ \ \ \ } & & \mbox{日本語の用語}\\
 \left.
  \begin{array}{@{\,}ll}
   \mbox{angle of radiation} \\
   \mbox{radiation angle}
  \end{array}
 \right\} &  \Longleftrightarrow & 
 \left\{
  \begin{array}{@{\,}ll}
   \mbox{放射 角}\\
   \mbox{放射 の 角}
  \end{array}
 \right.
\end{eqnarray*}
なお,本論文では前置詞``of''のみに関してこの規則を実装した.
また,〈光ファイバーケーブル,optical-fiber cable〉のよう
に,英語または日本語の用語どちらかにのみ,ハイフン記号を含む場合がある.
このような場合に双方向に訳語候補生成を行うためには,ハイフンの挿入及び削
除を考慮する必要がある.
前置詞を含む場合と同様に,ハイフンが挿入または削除される可能性を考慮した
訳語候補生成を行う規則を実装した.
}.
\begin{equation}
 y_T=t_1,t_2,\ldots,t_n
\end{equation}
そして,訳語候補$y_T$にスコアを与えることを考える.
先行研究と同様に,本論文においても,対訳辞書を用いて$y_S$と$y_T$の対応の
適切さを推定し,スコアを与える(これを対訳辞書スコアと呼ぶ.).
ただし,$y_T$全体の対訳辞書スコアは,訳語対$\langle s_i,t_i\rangle$のス
コア$q(\langle s_i,t_i\rangle)$の積で構成される.
また,それとは別に,目的言語コーパス中で訳語候補$y_T$がどの程度出現する
かによって$y_T$の適切さを評価し,スコアを与える(これをコーパススコア
$Q_{corpus}(y_T)$と呼ぶ.).
訳語候補$y_T$のスコアは,この2つのスコアの積により構成されるとする.
\begin{equation}
 \label{eq:score_one}
 \prod_{i=1}^{n} q(\langle s_i,t_i\rangle) \cdot Q_{corpus}(y_T)
\end{equation}
(ただし,訳語対のスコア$q$とコーパススコア$Q_{corpus}$の詳細は
\ref{sec:score_details}節で述べる.)

実際には,例(\ref{eq:bunkatsu_ouyou}), (\ref{eq:bunkatsu_applied})で示し
たように,$y_S$ には複数の分割の仕方が考えられるので,本論文ではそれぞれの分割
の仕方に対して式(\ref{eq:score_one})によりスコアを計算し,それらの和を訳語候補
$y_T$のスコアとする.
\begin{equation}
 \label{eq:score}
 Q(y_S, y_T)=\sum_{y_S=s_1,s_2,\ldots,s_n}\prod_{i=1}^{n} q(\langle s_i,t_i\rangle) \cdot Q_{corpus}(y_T)
\end{equation}

例えば,$y_S=$“応用 行動 分析”,$y_T$=``applied behavior analysis''の
場合を考える.
訳語対$\langle y_S,y_T\rangle$が既存の対訳辞書に含まれず,かつ,訳語対
〈“応用”,``applied''〉,〈“行動”,``behavior''〉,
〈“分析”,``analysis''〉,〈“行動分析”,``behavior analysis''〉が
既存の対訳辞書に含まれるとき,$y_T$を生成することがで
きる$y_S$の分割を,$y_T$の生成に用いる訳語対と共に以下に示す.
\begin{itemize}
  

 \item $s_1=\mbox{“応用”,}\ s_2=\mbox{“行動”,}\ s_3=\mbox{“分析”}$\\
       〈“応用”,``applied''〉,〈“行動”,``behavior''〉,\\
	〈“分析”,``analysis''〉
 \item $s_1=\mbox{“応用”,}\ s_2=\mbox{“行動 分析”}$\\
       〈“応用”,``applied''〉,〈“行動 分析”,``behavior analysis''〉
\end{itemize}
$\langle y_S,y_T\rangle$のスコアは,上記の2通りの分割に対して,それぞれ
対訳辞書スコアとコーパススコアの積を求めたものの和となる.

次に,訳語候補生成の方法を説明する.
単語または形態素数の多い用語の訳語推定を行う場合,単語または形態素の訳語のすべての組み合
せを生成すると,計算機のメモリ消費量が指数関数的に増えてしまう.
そこで,本論文では,この問題を避けるために,動的計画法のアルゴリズムを採
用し,訳語候補の生成と枝刈りを行う.

式(\ref{eq:y_S})で,訳すべき用語$y_S$を以下のように単語または形態素の列
で定義した.
\[
 y_S=w_1,w_2,\ldots,w_m
\]
ここで,単語または形態素$w_i$の区切りの位置に,位置を表すラベル$0, \ldots,
m$を付与する.
\begin{equation}
 y_S = \ _0 \ w_1 \ _1 \ w_2 \ _2 \ \cdots \ _{m-1} \ w_m \ _{m}
\end{equation}
この位置を表すラベルを利用して,位置$i$と位置$j$の間の単語または形態素の列
を表す記号$w_{j,k}$を導入する.
\begin{equation}
 w_{j,k} \equiv w_{j+1}, w_{j+2}, \ldots, w_{k}
\end{equation}
ただし,$w_{0,0} \equiv \varepsilon$とする.
ここで,$\varepsilon$は空文字を表すものとし,
$y$を1つ以上の単語または形態素の列とすると$\varepsilon y=y$とする.

まず,動的計画法による$y_S$の訳語推定の概略を述べる.
先頭から$k$番目までの単語または形態素の列$w_{0,k}$に対して生成された訳語
候補の集合を$Tran(w_{0,k})$とすると,$y_S=w_{0,m}=w_{1}, \ldots, w_{m}$
の訳語を得るには$Tran(y_S=w_{0,m})$からスコア1位の訳語候補を取り出せば
よい.
ここで,各$Tran(w_{0,k})$ $(k=1,\ldots,m)$は以下の式に従って,再帰的に計
算される.
\begin{equation}
 \label{eq:Tran_calc}
  \begin{array}{ll}
   Tran(w_{0,k}) = top( \hspace{-0.2cm} & merge(\ \bigcup\limits_{i=0}^{k-1}
    concat(Tran(w_{0,i}),\ tran(w_{i,k}))), \\
   & r\ )
  \end{array}
\end{equation}
この式では,$w_{0,k}=w_{1}, \ldots, w_{k}$をある位置$i$で$w_{0,i}=w_{1},
\ldots, w_{i}$と$w_{i,k}=w_{i+1}, \ldots, w_{k}$の2つに分割する.
分割の場所$i$は先頭$i=0$から順に$i=k-1$まで移動させていく.
それぞれの分割の仕方において,$w_{0,i}$に対しては再帰的に訳語候補の集合
$Tran(w_{0,i})$を求め,$w_{i,k}$に対しては$w_{i,k}$を見出し語として対訳
辞書から訳語の集合$tran(w_{i,k})$を得る.
そして,両者を$concat$により結合することにより,新しい訳語候補を生成する.
このとき,同一の訳語候補が複数の異なる分割の仕方から生成される場合がある.
その場合は,$merge$により,それらの訳語候補のスコアがまとめられる.
最後に,$top$によりスコア上位$r$個の訳語候補のみ出力することで,訳語候補の枝
刈りを行い,この出力を$Tran(w_{0,k})$とする.

実際に式(\ref{eq:Tran_calc})を用いて,図\ref{fig:compositional}の例をも
とに$y_S$=“応用”,“行動”,“分析”の訳語候補の集合
$Tran(y_S=w_{0,3})$が生成される様子を説明する.
ただし,ここでは枝刈り後出力される訳語候補数$r$を3とする.
式(\ref{eq:Tran_calc})の$i=0,\ldots,k-1$のループに注目すると,
以下のように訳語候補が生成されていくことがわかる.
\begin{description}
 \item[($i=0$)] $w_{0,3}$は$w_{0,0}=\varepsilon$と$w_{0,3}=$“応用”,
	    “行動”,“分析”に分割される.
	    
	    $w_{0,3}=$“応用”,“行動”,“分析”は対訳辞書に訳語がない
	    ので,$concat$の出力は空集合となる.
 \item[($i=1$)] $w_{0,3}$は$w_{0,1}=$“応用”と$w_{1,3}=$“行動”,“分
	    析”に分割される.
	    
	    $Tran(w_{0,1})$により,$w_{0,1}$の訳語候補の集合を再帰的に求め,
	    $tran(w_{1,3})$
	    により,対訳辞書から得られる$w_{1,3}$の訳語の集合を求める.
	    
	    そして,それらの訳語候補を$concat$により結合し訳語候補を生成
	    する.
	    
	    $Tran(w_{0,1})=\{$``application'', ``practical'',
	    ``applied''$\}$,$tran(w_{1,3})=\{$``behavior analysis''$\}$のとき
	    生成される訳語候補を以下に示す.
       \begin{itemize}
	\item ``application behavior analysis''
	\item ``practical behavior analysis''
	\item ``applied behavior analysis''
       \end{itemize} 
 \item[($i=2$)] $w_{0,3}$は$w_{0,2}=$“応用”,“行動”と$w_{2,3}=$“分
	    析”に分割される.
	    
	    $Tran(w_{0,2})$により,$w_{0,2}$の訳語候補の集合を再帰的に求め,
	    $tran(w_{2,3})$
	    により,対訳辞書から得られる$w_{2,3}$の訳語の集合を求める.
	    
	    そして,それらの訳語候補を$concat$により結合し訳語候補を生成
	    する.
	    
	    $Tran(w_{0,2})=\{$``applied action'', ``applied activity'',
	    ``applied behavior''$\}$,$tran(w_{2,3})=\{$``analysis'',
	    ``diagnosis'', ``assay''$\}$のとき生成される訳語候補を以下に
	    示す.
       \begin{itemize}
	\item ``applied action analysis''
	\item ``applied action diagnosis''
	\item ``applied action assay''
	\item ``applied activity analysis''
	\item ``applied activity diagnosis''
	\item ``applied activity assay''
	\item ``applied behavior anaylysis''
	\item ``applied behavior diagnosis''
	\item ``applied behavior assay''
       \end{itemize} 
\end{description}
以上の操作が終了したら,$y_S$に対して複数個の訳語候補が生成された状態と
なる.
生成された訳語候補に同じものが存在した場合,関数$merge$によりこれらがま
とめられ,最後に関数$top$によりスコア上位$r$個の訳語候補が出力される.


\subsection{訳語候補のスコア付け}
\label{sec:score_details}

\subsubsection{対訳辞書スコア}
\label{sec:lexicon_score}
訳語推定対象の用語$y_S$と訳語候補$y_T$の対応の適切さを対訳辞書を用いて測
定するための「対訳辞書スコア」を\ref{sec:generation}節で導入した.
この対訳辞書スコアは,訳語候補$y_T$を生成するときに使用した訳語対
$\langle s_i,t_i\rangle$のそれぞれの適切さを関数$q$により測定し,それら
の積で計算されるものであった.
本節では,対訳辞書に基づいて訳語対$\langle s_i,t_i\rangle$のスコアを計算
するための関数$q$を2種類定義する.

\subsubsection*{頻度-長さ(DF)}

``natural language processing''という用語
が,既存の対訳辞書に含まれないため,訳語推定の対象となる場合を考え
 る.
〈``natural'', “自然な”〉,〈``language'', “言語”〉,
〈``processing'', “処理”〉の3つの訳語対から生
成される“自然な言語処理”という訳語候補よりも,〈``natural language'', “自然言語”〉のような単語数または形態
素数の多い訳語対と〈``processing'', “処理”〉を利用して得られる訳語候補“自然言語処理”の方が信頼度が
高いと思われる.
また,表~\ref{tab:entry_number}の部分対応対訳辞書に含まれる訳語対は,英
辞郎に含まれる複合語の訳語対から,英語及び日本語の構成要素の訳語対応を推
定することにより作成された訳語対であるため,対訳辞書$P_2$に出現する頻度
の少ない訳語対よりも,出現する頻度の多い訳語対の方が信頼度が高いと思われ
る.
以上のような,訳語対の長さと頻度に基づく経験的な選好に基づいて,訳語対を
順位付けする方法について述べる.

まず,スコア付けの対象となる対訳辞書の訳語対は,以下のように分類できる.
\begin{itemize}
 \item 英辞郎の訳語対(利用できる情報:単語数または形態素数)
       \begin{itemize}
	\item 単語数または形態素数が2以上の訳語対 ((a)とする)
	\item 1単語または1形態素の訳語対 ((b)とする)
       \end{itemize}
 \item 部分対応対訳辞書の訳語対(利用できる情報:対訳辞書$P_2$に出現する
       頻度) ((c)とする)
\end{itemize}
ここではスコア付けの方針を決める問題を,上記の(a),(b),(c)で示した3種
類の訳語対の間に優先順位を付けることに帰着させて考える.
(a),(b),(c)の優先順位として,本論文では,
まず,(a)の訳語対に与えるスコアを極めて高く設定し,(b)または(c)の訳語対
のスコアを必ず上回るようにする.
次に,(b)と(c)の訳語対の間のスコアの大小関係については,
(c)の訳語対が対訳辞書$P_2$に出現する頻度に閾値を設け,
(c)の訳語対の頻度が頻度閾値と同じであれば,(b)の訳語対のスコアと同じにし,
(c)の訳語対の頻度が頻度閾値より大きければ,(b)の訳語対のスコアより大きく
し,そして,(c)の訳語対の頻度が頻度閾値より小さければ,(b)の訳語対のスコ
アより小さくする.
本論文では,この頻度閾値を10に設定した.
この頻度閾値を変化させることにより,英辞郎に含まれる1単語または1形態素
の訳語対のスコアと,部分対応対訳辞書に含まれる訳語対のスコアの大小関係が変化する
ため,訳語推定の性能にもある程度影響を与える.
しかしながら,本論文の目的は,コーパスとしてウェブ全体を用いる方法と,ウェ
ブから収集した専門分野コーパスを利用する方法の比較にあるので,最適なパラ
メータの値の追求は行わなかった.

この優先順位を実現するため,英辞郎の訳語対のスコアには単語数または形態素
数を指数とする関数を用い,部分対応対訳辞書の訳語対のスコアには頻度の対数
を用いることで,単語数または形態素数が2以上の英辞郎の訳語対のスコアが,
部分対応対訳辞書の訳語対のスコアよりも大きくなるようにした.
訳語対$\langle s,t\rangle$のスコア$q(\langle s,t\rangle)$の定義として,
以下の式を採用する.
\begin{equation}
 \label{eq:DF}
 q(\langle s,t\rangle) =
 \left\{ \begin{array}{ll}
  10^{(compo(s)-1)} & \mbox{($\langle s,t\rangle$ in 英辞郎)}\\
	  \log_{10}f_p(\langle s,t\rangle) & \mbox{($\langle s,t\rangle$
	   in } B_P \mbox{)}\\
	  \log_{10}f_s(\langle s,t\rangle) & \mbox{($\langle s,t\rangle$
	   in } B_S \mbox{)}
	 \end{array} \right.
\end{equation}
ここで,$compo(s)$は$s$の単語または形態素の数を表すものとし,
$f_p(\langle s,t\rangle)$は,$P_2$中に第一要素として$\langle s,t\rangle$
が出現する回数を表すものとし,$f_p(\langle s,t\rangle)$は,$P_2$中に第二
要素として$\langle s,t\rangle$が出現する回数を表すものとする.
式(\ref{eq:DF})では,対数関数の底の値が部分対応対訳辞書の訳語対の頻度閾
値に対応する.
すなわち,部分対応対訳辞書の訳語対で対訳辞書$P_2$に10回出現する訳語対と,
英辞郎に含まれる1単語または1形態素の訳語対のスコアが等しくなる.
なお,このスコアでは,部分対応対訳辞書に一度しか現れない訳語対のスコアは
ゼロとなる.
この場合,訳語として利用しないものとする.


式(\ref{eq:DF})に示した訳語対のスコア関数の積で定義される対訳辞書スコア
を,以下ではDFと呼ぶものとする.

\subsubsection*{確率(DP)}
\cite{Fujii00}は,対訳辞書に基づく$y_S$と$y_T$の対応の適切さを,確率
$P(y_S|y_T)$を計算することにより評価した.
このスコアは,条件付き確率$P(s_i|t_i)$の積で定義される.
\cite{Fujii00}は対訳辞書として部分対応対訳辞書$B$のみを用いているため,
同じ設定とするには,本論文でも部分対応対訳辞書$B$のみを用いなければなら
ない.
しかしながら,部分対応対訳辞書$B$のみを用いた実験を行った結果,英辞郎と
部分対応対訳辞書$B$を併用する場合に比べ,訳語推定の性能(精度・再現率)
が10\%前後も低いことがわかった.
このため,本論文では,部分対応対訳辞書$B$に加え英辞郎も用いて,条件付き
確率$P(s_i|t_i)$に基づく対訳辞書スコアを評価することとした.
本論文では,英辞郎と部分対応対訳辞書$B$を併用できるようにするために,以
下の式に示す拡張を行った.
\begin{eqnarray}
 & q(\langle s,t\rangle) = P(s|t) = \frac{f_{prob}(\langle
  s,t\rangle)}{\sum_{s_j}f_{prob}(\langle s_j,t\rangle)} & \label{eq:DP1}\\
 & f_{prob}(\langle s_j,t\rangle) = 
  \begin{cases}
		   10 & \text{($\langle s,t\rangle$ in 英辞郎)} \\
		   f_B(\langle s_j,t\rangle) & \text{($\langle s,t\rangle$ in B)} 
				     
  \end{cases} & \label{eq:DP2}
\end{eqnarray}
上式では,英辞郎の訳語対の頻度は10とみなすものとした\footnote{辞書スコア
`DF'では,頻度10の部分対応対対訳辞書の訳語対のスコアと,構成要素長が1の
用語の英辞郎の訳語のスコアと同じにしている.これに合わせるため,英辞郎の
訳語対の頻度を10とみなすものとした.}.

式(\ref{eq:DP1}), (\ref{eq:DP2})に示した訳語対のスコア関数から計算される
対訳辞書スコアを以下ではDPと呼ぶものとする.


\subsubsection{コーパスに基づくスコア}
\label{sec:corpus_score}

訳語候補$y_T$の適切さを目的言語コーパスを用いて測定するための「コーパス
スコア」を\ref{sec:generation}節で導入した.
本論文では,コーパススコアとして以下に示す3種類を評価した.
\begin{itemize}
 \item 頻度(CF): 目的言語コーパスにおける訳語候補$y_T$の生起頻度
\begin{equation}
 Q_{corpus}(y_T)= freq(y_T)
\end{equation}
 \item 確率(CP): 以下のバイグラムモデルによって推定される,訳語候補$y_T$
       の生起確率.
       \cite{Fujii00}で用いられたコーパススコアの評価を目的
       とする.
       本来は$t_i$を単語または形態素とすべきであるが,実装の都合上,
       $t_i$を対訳辞書から得られた訳語とする.
       したがって,$t_i$は1つ以上の単語または形態素から
       構成される\footnote{
	前節で述べたように,\cite{Fujii00}のスコア関数の評価に際しては,
       対訳辞書スコアにおいて,部分対応対訳辞書$B$と英辞郎を併用している.
       ここで,英辞郎の訳語には複数の単語または形態素で構成されるものが
       あるが,このような場合,厳密には,訳語を単語また形態素に分割して,
       単語また形態素のバイグラムに基づいて式(\ref{eq:CP})の計算をしなけ
       ればならない.しかしながら,実装上の手間を避けるため,ここでは,
       対訳辞書から得られた訳語をそのまま用い,$t_i$は1つ以上の単語また
       は形態素から構成されるとした.}.
\pagebreak
\begin{equation}
 \label{eq:CP}
 Q_{corpus}(y_T)= P(t_1) \cdot \prod_{i=1}^{n-1} P(t_{i+1}|t_i)
\end{equation}
 \item 生起(CO): 目的言語コーパスに訳語候補$y_T$が生起するかどうか
\begin{equation}
 Q_{corpus}(y_T)= 
  \begin{cases}
	   1 & \text{$y_T$ がコーパス中に生起する} \\
	   0 & \text{$y_T$ がコーパス中に生起しない} 
  \end{cases}
\end{equation}
\end{itemize}


\subsubsection{スコア関数}

\begin{table}[b]
 \small
 \centering
 \caption{訳語候補のスコア関数と構成要素}
 \label{tab:param_method}
\begin{tabular}{|c||c|c||c|c|c||c|c|}
 \hline
 & \multicolumn{2}{|c||}{対訳辞書スコア} &
 \multicolumn{3}{|c||}{コーパススコア} & \multicolumn{2}{|c|}{コーパス}\\
 \cline{2-8}
 スコア関数 & 頻度-長さ & 確率 & 頻度 & 確率 & 生起 & 専門分野 & ウェブ \\
 & (DF) & (DP) & (CF) & (CP) & (CO) & コーパス & 全体 \\
 \hline
 \hline
 DF-CF     &  p/f &  & p/f & & & o &\\  
 \hline
 DF-CF${\rm _f}$    &  p/f &  & f & & & o &\\  
 \hline
 DF-CP     & p/f & & & p/f & & o & \\  
 \hline
 DF-CO     & p/f & & & & p/f & o & \\  
 \hline
 DF-CO${\rm _f}$     & p/f & & & & f & o & \\  
 \hline
 DP-CF     & & p/f & p/f & & & o & \\  
 \hline
 DP-CP     & & p/f & & p/f & & o & \\  
 \hline
 CF        &  &  &  p/f & & & o &\\  
 \hline
 CP        &  &  &  & p/f & & o &\\  
 \hline \hline
 DF-CF${\rm _f}$-w  &  p/f &  & f & & & & o\\  
 \hline
 DF-CO${\rm _f}$-w  &  p/f &  &  & & f & & o\\  
 \hline \hline
 DF        & p/f &  &  & &  & & \\  
 \hline
\end{tabular}
\vspace{4pt}

p(prune):枝刈りに利用,f(final):最終スコアに利用
\end{table}

表\ref{tab:param_method}に示すように,本論文では,辞書に基づくスコアとコー
パスに基づくスコアに対して,12種類の組み合せのスコア関数を作成し評価を行っ
た.
この表において,`p(prune)'は,動的計画法のアルゴリズムを用いた訳語候補生
成の過程において,式(\ref{eq:Tran_calc})の$top$を実行することで,生成された
訳語候補の部分列の順位付けと枝刈りにそのスコアが用いられることを示す.
`f(final)'は,生成された訳語候補の最終結果の順位付けにそのスコアが用いら
れることを示す.
また,列`コーパス'において,`専門分野コーパス'は,あらかじめウェブから専
門分野コーパスを収集し,その後,このコーパスを用いて生成された訳語候補の
検証を行うことを示す.
`ウェブ全体'は,サーチエンジンを通してウェブ全体を利用して訳語候補の検証
を行うことを示す.

スコア関数の命名方法は,`対訳辞書スコア名-コーパススコア名'の原則に基づ
く.
例えば,スコア関数`DF-CO'は,対訳辞書スコアに`DF'を用い,コーパススコア
に`CO'を用いたスコア関数である.
ここで,式(\ref{eq:Tran_calc})の$top$による訳語候補の枝刈りについて考え
ると,不要な候補を早い段階で削減するため,基本的には対訳辞書スコアとコー
パススコアの両方を用いるべきである.
しかしながら,コーパスとしてウェブ全体を用いる場合は,サーチエンジンの検
索に要する時間を考慮すると,訳語候補の生成過程でコーパススコアを利用する
ことは効率的ではない.
そこで,訳語候補の枝刈りにはコーパススコアを用いず,訳語候補の最終的なス
コア計算のみにコーパススコアを用いる.
コーパススコアを枝刈りに用いない場合は,`DF-CO${\rm _f}$'の様に,
コーパススコア名の後ろに`${\rm _f}$'を付加する.
そして,コーパスとしてウェブ全体を用いる場合は,`DF-CO${\rm _f}$-w'の様に,
`-w'を付加する.

本論文で評価したスコア関数は,コーパススコアの計算において用いるコーパスの
違いにより,ウェブから収集した専門分野コーパスを用いるタイプ,サーチエン
ジンを通してウェブ全体を用いるタイプ,コーパスを一切用いないタイプの,3
つのタイプに分けることができる.
対訳辞書スコアには,訳語対が部分対応対訳辞書に出現する頻度と訳語対の構成
要素長に基づく`DF'と,条件付き確率$P(s|t)$に基づく`DP'の2つがある.
コーパススコアには,訳語候補がコーパスに生起する頻度に基づく`CF',訳語
候補がコーパスに生起する確率に基づく`CP',訳語候補がコーパスに生起する
か否かに基づく`CO'の3つがある.
ここで,\ref{sec:evaluation}節で示す実験結果においては,対訳辞書`DF'を用
いたスコア関数と`DP'を用いたスコア関数の間で性能に大きな差はないが,`DF'
を用いた方が若干精度が高かった.
そこで本論文では,精度を重視する立場に立ち,
対訳辞書スコアとして主に`DF'を用いて評価を行う
\footnote{
本論文の焦点は,ウェブから収集した専門分野コーパスを用いる方法と,サーチ
エンジンを通してウェブ全体を用いる方法の比較にある.
従って,定義し得るスコア関数を網羅的に評価することは行っていない.
}.

以下,本論文で実際に評価した辞書スコアとコーパススコアの組み合わせに
ついて説明する.

コーパスとしてウェブから収集した専門分野コーパスを用いる場合には,
対訳辞書スコア`DF'を用いたスコア関数とコーパススコアとの組み合わせでは,
`CF',`CP',`CO'の3種類を網羅したが,それらの性能に大差はなかった.
そこで,対訳辞書スコア`DP'では,大きく性質の異なるコーパススコアである
`CF',`CP'との組み合わせを評価した.
ここで,スコア関数`DP-CP'は,\cite{Fujii00}で提案されたモデルに,部分対
応対訳辞書に加え英辞郎自体も用いることができるように拡張を加えたスコア関
数である.

一方,コーパスとしてウェブ全体を用いる場合は,
辞書スコアとしては`DF'を用いた.
また,コーパススコア`CP'は,`CF'や`CO'と比べ,サーチエンジンの検索回数が
多くなるので,評価の対象から除外した.
さらに,上述したように,サーチエンジンの検索時間の都合で,コーパススコア
による枝刈りは行わない.
以上をまとめると,コーパスとしてウェブ全体を用いるスコア関数として
は,`DF-CF${\rm _f}$-w'と`DF-CO${\rm _f}$-w'の2種類を評価する.
そして,この2つのスコア関数との直接的な比較のため,ウェブから収集した専
門分野コーパスを用いるスコア関数として,コーパススコアによる枝刈りを行わ
ないDF-CF${\rm _f}$とDF-CO${\rm _f}$についても評価を行う.

最後に,対訳辞書スコアまたはコーパススコアどちらかのみを用いるスコア関数
の評価のために,次のスコア関数を評価する.
辞書スコアのみで訳語候補のスコア付けをした場合の評価のため,スコア関数
`DF'を評価する
\footnote{
\ref{sec:evaluation}節の評価実験では,スコア関数`DF'は極めて低いF値であっ
た.
本論文ではスコア関数`DP'は評価していないが,同様の傾向であると思われる.
}.
コーパススコアのみで訳語候補のスコア付けをした場合の評価のため,スコア関
数`CF'及び`CP'を評価する
\footnote{
スコア関数`CO'は,辞書スコアは利用せずコーパススコア`CO'のみを用いるスコ
ア関数であるが,訳語候補のスコアが0か1となってしまい,順位付けできないの
で取り扱わない.
}.




\section{実験と評価}
\label{sec:experiments}

\subsection{評価用用語集合}
\label{sec:evaluation_set}

\begin{table}[b]
 \small
 \centering
 \caption{評価用用語の数}
 \label{tab:mondai_number}
 \begin{tabular}{|c|c|r|r|r|r|r|}
  \hline
  \multirow{2}{*}{辞書} & \multirow{2}{*}{分野} &
  \multirow{2}{*}{$|Y_{S}|$} & \multicolumn{2}{|c|}{$S$ = 英語} &
  \multicolumn{2}{|c|}{$S$ = 日本語}\\
  \hhline{|~|~|~|--|--|}
  & & & $|X_S^U|$ & MBytes & $|X_S^U|$ & MBytes\\
  \hline
  \hline
  マグローヒル & 電磁気学       & 30 & 36 & 28 & 32 & 99\\
  科学技術用語 & 電気工学       & 41 & 34 & 21 & 25 & 71\\
  大辞典       & 光学           & 31 & 42 & 37 & 22 & 48\\
  \hline				      	   
  岩波         & プログラム言語 & 28 & 37 & 34 & 38 & 135\\
  情報科学辞典 & プログラミング & 28 & 29 & 33 & 29 & 110\\
  \hline
  英和コンピュータ & \multirow{2}{*}{(コンピュータ)} & 
  \multirow{2}{*}{99} & \multirow{2}{*}{91} & \multirow{2}{*}{67} &
  \multirow{2}{*}{69} & \multirow{2}{*}{232}\\
  用語大辞典     & & & & & &\\
  \hline
                 & 解剖学             &  91 & 91 & 73 & 33 &  66\\
  25万語         & 疾患               &  86 & 91 & 83 & 53 & 100\\
  医学用語大辞典 & 化学物質及び薬物   &  84 & 94 & 54 & 74 & 131\\
                 & 物理化学及び統計学 &  99 & 88 & 56 & 58 & 135\\
  \hline
  \hline
  \multicolumn{2}{|c|}{合計} & 617 & 633 & 482 & 433 & 1127\\
  \hline
 \end{tabular}
\end{table}


実験では,図\ref{fig:overview1}で示したように,既存の対訳辞書に含まれて
いる用語と含まれていない用語が混在した形で複数の専門用語が与えられるものとし,
既存の対訳辞書に載っていない用語の訳語推定の評価を行う.
本論文では,言語$\langle S, T\rangle$の組を〈英語,日本語〉
または,〈日本語,英語〉とする.
評価セットを作成するため,まず,表\ref{tab:mondai_number}に示す既存の4
種類の日英専門用語対訳辞書「マグローヒル科学技術用語大辞典」
\cite{dic-McGraw-Hill},「岩波情報科学辞典」\cite{dic-iwanami-info},
「英和コンピュータ用語大辞典」\cite{dic-computer},「25万語医学用語大辞
典」\cite{dic-25igaku}の10分野から,以下の条件を満たす2種類の訳語対集合
を無作為に選定した.
1種類目は,英辞郎にも訳語対として存在し,かつ英語用語・日本語用語共にヒッ
ト数が100以上の訳語対であり,この種類の訳語対の集合を既知訳語対集合
$X_{ST}$とする.
2種類目は,以下の条件を満たす訳語対$\langle t_e, t_j\rangle$の集合であり,
これを未知訳語対集合$Y_{ST}$とする.
ただし,$t_e$は英語の用語,$t_j$は日本語の用語を表すものとする.
\begin{itemize}
 \item $t_e$,$t_j$共に英辞郎に見出し語として存在しない
 \item $t_e$は2語以上,$t_j$は2形態素以上からなる
 \item $t_e$及び$t_j$のヒット数は10以上
\end{itemize}


次に,$S=英語$,$S=日本語$,それぞれの場合において,既知訳語対集合
$X_{ST}$から,英辞郎に含まれる訳語が一個の用語の集合$X_S^U$を作成した.
そして,\ref{sec:corpus}節で述べた方法で,$X_S^U$の用語に対して,それら
用語の英辞郎に含まれる訳語の集合$X_T^U$を利用して,各分野毎にウェブから
専門分野コーパスを収集した.
同様に,$S=英語$,$S=日本語$,それぞれの場合において,未知訳語対集合
$Y_{ST}$から,評価用用語集合$Y_S$を作成した.
そして,$Y_S$のそれぞれの用語に対して,未知訳語対集合$Y_{ST}$にもともと含まれる
訳語に加えて,必要であれば人手で一個以上の正解訳語を付与した.
この結果,正解訳語の個数の平均は,$S=英語$のとき1.31個,$S=日本語$のとき
1.62個となった.
10分野のそれぞれに対して,表~\ref{tab:mondai_number}に$X_S^U$及び
$Y_{S}$に含まれる用語の個数,及び,ウェブから収集したコーパスのサイズを
分野毎に示す.

続いて,$Y_S$及び,それに属する用語の正解訳語の性質について述べる.
$Y_S$に属する用語の構成要素数の平均は,$S$が英語のとき2.28語(2語の用語
は437個で全体の70.8\%,3語以上の用語は180個で全体の29.2\%),$S$が日
本語のとき2.47形態素(2形態素の用語
は396個で全体の64.2\%,3形態素以上の用語は221個で全体の35.8\%)であった.
また,$Y_S$の用語とその正解訳語が構成的に対応しているかどうかを調べると,
$S$が英語のとき90.6\%,$S$が日本語のとき92.5\%であった
\footnote{
\ref{sec:intro}章では,未知訳語対集合$Y_{ST}$に含まれる訳語対が構成的かどうかを
調査した結果を述べている.
これに対して,本節では,未知訳語対集合$Y_{ST}$に含まれる訳語対に対して,
新たに人手で正解訳語を追加したものに対して構成的かどうか評価を行っている
ので,\ref{sec:intro}章の結果よりも割合が増加し,また,$S$が英語のときと
日本語のときで割合も異なる.
}.

\begin{figure}[t]
 \centering
     \includegraphics[width=300pt]{jnlp-compo-fig-benzu.eps}
      $ t \in Y_S, S(t) \mbox{は$t$の正解訳語の集合}$ 
 \begin{align*}
  T_g & = \{t | \exists s \in S(t),s \mbox{は生成可能}\}\\
  T_c & = \{t | \exists s \in S(t),s \mbox{はコーパスに存在}\}\\
  T_{gc} & = \{t | \exists s \in S(t),s \mbox{は生成可能かつ} s \mbox{はコー
   パスに存在}\}
 \end{align*}
 \caption{正解訳語の生成可能性/コーパス中の出現による集合$Y_S$の部分
 集合の分類}
 \label{fig:benzu}
\end{figure}

次に,集合$Y_S$の用語のどの程度が訳語推定可能かを調べるために,$Y_S$の部
分集合として,図\ref{fig:benzu}に示す$T_g$,$T_c$,$T_{gc}$を定義する.
$T_g$は,対訳辞書として英辞郎と部分対応対訳辞書$B_P$,$B_S$を利用して,
\ref{sec:generation}節で述べた方法で,出力される訳語候補数$r$を無限大に
したときに,正解訳語を生成可能な用語の集合である.
$T_c$は,その用語が属する分野の専門分野コーパスに正解訳語が
含まれている用語の集合である.
$T_{gc}$は,図\ref{fig:benzu}に示したように,生成可能かつ専門分野コーパスに
含まれる正解訳語が存在するという条件を満たす用語の集合である.
言い換えると,英辞郎,および部分対応対訳辞書$B_P$,$B_S$を用いた訳語候補
生成手法において,専門分野コーパスを利用した場合には,$T_{gc}$に属する用
語に対してのみ,正解訳語を生成できる可能性がある
\footnote{
コーパススコア`CP'を用いた場合は,訳語推定対象の用語が$T_g$に属し,かつ,
コーパススコアの確率の各項$P(t_1)$,$P(t_{i+1}|t_i)$がゼロでなければ,正解
訳語そのものがコーパスに存在しなくても,正解訳語を生成できる可能性が{\linebreak}ある.
}.
$T_g$と$T_c$の積集合の中には,生成可能かつ専門分野コーパスに含まれる正解
訳語が存在しない用語も含まれているが,このような用語に対しては正解訳語を
生成することができない.
$T_{gc}$は,$T_g$と$T_c$の積集合の部分集合となっていることに注意されたい.

\begin{table}[t]
 \small
 \centering \caption{英辞郎と部分対応対訳辞書$B_P, B_S$における正解訳語
 の生成可能性と正解訳語がコーパスに存在す   るかどうかの調査}
 \label{tab:seiseika}
 \begin{tabular}{|c|r|r|r|r|r|r|}
  \hline
  & \multicolumn{3}{|c|}{英語→日本語} & \multicolumn{3}{|c|}{日本語→英
  語} \\
      \cline{2-7}
  & & コーパス & 生成可能かつ & & コーパス & 生成可能かつ\\
      \raisebox{0.5\normalbaselineskip}[0cm][0cm]{分野} 
& 生成可能 & に存在 & コーパスに存在 & 生成可能 & に存在 & コー
  パスに存在 \\
  & \multicolumn{1}{|c|}{($T_g$)} & \multicolumn{1}{|c|}{($T_c$)} & \multicolumn{1}{|c|}{($T_{gc}$)} & \multicolumn{1}{|c|}{($T_g$)} & \multicolumn{1}{|c|}{($T_c$)} & \multicolumn{1}{|c|}{($T_{gc}$)} \\
  \hline \hline
  電磁気学           & 60\% & 93\% & 57\% & 87\% &  90\% & 73\%  \\
  電気工学           & 71\% & 78\% & 59\% & 71\% &  68\% & 51\%  \\
  光学               & 61\% & 65\% & 35\% & 71\% &  68\% & 52\%  \\
  プログラム言語     & 86\% & 93\% & 79\% & 82\% & 100\% & 82\%  \\
  プログラミング     & 75\% & 96\% & 71\% & 82\% &  86\% & 75\%  \\
  コンピュータ       & 74\% & 52\% & 34\% & 80\% &  61\% & 46\%  \\
  解剖学             & 78\% & 92\% & 74\% & 80\% &  55\% & 45\%  \\
  疾患               & 69\% & 83\% & 55\% & 76\% &  70\% & 56\%  \\
  化学物質及び薬物   & 54\% & 63\% & 39\% & 56\% &  54\% & 36\%  \\
  物理化学及び統計学 & 85\% & 71\% & 58\% & 80\% &  61\% & 53\%  \\
  \hline
  全体               & 71.8\% & 74.9\% & 53.8\% & 75.7\% & 65.3\% & 51.9\%  \\
  \hline
 \end{tabular}
\end{table}


以上の定義をふまえて,$T_g$,$T_c$,$T_{gc}$に属する用語の割合を分野毎に調べ
た結果を表\ref{tab:seiseika}に示す.
これより,英辞郎と部分対応対訳辞書$B_P$,$B_S$ を利用して正解訳語が生成
可能な用語の割合は,英日方向で71.8\%,日英方向で75.7\%であることがわかる.
一方,英辞郎のみを利用して生成可能な用語の割合を評価すると,英日方向
で50.4\%,日英方向で56.6\%であった.
このことから,部分対応対訳辞書が有効であることがわかる.
また,英辞郎,および部分対応対訳辞書$B_P$,$B_S$を用いた訳語候補生成手法
において,専門分野コーパスを利用した場合には,正解訳語を生成できる用語の
割合の上限は,$T_{gc}$欄より,英日方向で53.8\%,日英方向で51.9\%であるこ
とがわかる.
参考として,対訳辞書として,英辞郎と部分対応対訳辞書$B$を利用して,正解
訳語の生成可能性を調べた結果を表\ref{tab:fujii_seiseika}に示す.
この結果を見ると,対訳辞書として,英辞郎と部分対応対訳辞書$B$を用いる方
が,正解訳語を生成可能な用語数が若干多い.
しかしながら,\ref{sec:bcl}節で述べたように,本論文では両者の性能を総合
的に比較して,部分対応対訳辞書として$B_P, B_S$ を用いている.

\begin{table}[t]
 \small
 \centering
 \caption{英辞郎と部分対応対訳辞書$B$における正解訳語の生成可能性}
 \label{tab:fujii_seiseika}
 \begin{tabular}{|c|r|r|}
  \hline
  分野 & 英語→日本語 & 日本語→英語 \\
  \hline \hline
  電磁気学           & 60\% & 87\% \\
  電気工学           & 80\% & 78\% \\
  光学               & 61\% & 71\% \\
  プログラム言語     & 86\% & 82\% \\
  プログラミング     & 79\% & 82\% \\
  コンピュータ       & 75\% & 81\% \\
  解剖学             & 79\% & 81\% \\
  疾患               & 71\% & 78\% \\
  化学物質及び薬物   & 55\% & 57\% \\
  物理化学及び統計学 & 86\% & 81\% \\
  \hline
  全体               & 73.6\% & 77.0\% \\
  \hline
 \end{tabular}
\end{table}

ここで,\ref{sec:generation}節の脚注\ref{fn:of_hyphen}で述べた,前置詞
``of''及びハイフンの挿入・削除に関する規則の効果について述べる.
未知訳語対集合$Y_{ST}$の訳語対617個のうち,前置詞``of''を含むものは24個
存在した.
また,英語の用語のみにハイフンを含む訳語対は33個,日本語の用語のみにハイ
フンを含む訳語対は2個存在した.
英日方向において,この2つの規則を加えることで,$T_g$に含まれる用語の数
が27個しか増加しなかった.
逆に日英方向の場合,この2つの規則を加えることで,$T_g$に含まれる用語の数
が7個しか増加しなかった.
$T_g$に含まれる用語数の増加が少ないのは,正解訳語を人手で付与することに
より,ofやハイフンを含まない正解訳語が追加されたためである.

\subsection{スコア関数の評価}
\label{sec:evaluation}



表~\ref{tab:param_method}に示したスコア関数を用いて,集合$Y_S$に対して訳
語推定の評価実験を行った.
実験の条件として,動的計画法による訳語生成過程で,保持する訳語候補の数
$r$は10とした.
専門分野コーパスを用いる場合は,対象の用語が属する分野の専門分野コー
パスを用いる.
ウェブ全体を用いる場合は,サーチエンジンとして,英日方向の場合はgooを,
日英方向の場合はYahoo!を用いた.
また,日英方向では,日本語の用語の分かち書きは人手で行った.


\begin{table}[b]
 \small
 \centering
 \caption{集合$Y_S$全体に対するスコア関数の評価(再現率)}
 \label{tab:evaluation}
 \begin{tabular}{|c|r|r|r|r|}
  \hline
  & \multicolumn{2}{|c|}{英語→日本語} & \multicolumn{2}{|c|}{日本語→英
  語} \\
      \cline{2-5}
      \raisebox{0.5\normalbaselineskip}[0pt][0pt]{スコア関数} 
	& top 1 & top 10 & top 1 & top 10 \\
  \hline \hline
  DF-CF     & 42.5\% & 50.1\% & 41.8\% & 48.1\% \\
  DF-CF${\rm _f}$    & 38.2\% & 41.5\% & 39.5\% & 44.1\% \\
  DF-CP     & 43.6\% & 52.4\% & 44.6\% & 51.9\% \\
  DF-CO     & 44.7\% & 47.6\% & 43.9\% & 48.6\% \\
  DF-CO${\rm _f}$ & 39.9\% & 41.7\% & 39.7\% & 44.1\%\\
  DP-CF     & 46.0\% & 54.3\% & 44.7\% & 51.5\% \\
  DP-CP     & {\bf 46.7\%} & 56.2\% & {\bf 48.9\%} & 56.1\% \\
  CF        & 26.3\% & 48.0\% & 31.3\% & 43.8\% \\
  CP        & 25.9\% & 48.6\% & 32.6\% & 46.5\% \\
  \hline
  DF-CF${\rm _f}$-w  & {\bf 52.0\%} & 59.0\% & {\bf 51.1\%} & 65.8\% \\
  DF-CO${\rm _f}$-w  & 44.1\% & 59.0\% & 50.1\% & 65.0\% \\
  \hline
  DF        & 35.7\% & 59.0\% & 45.1\% & 63.2\% \\
  \hline
 \end{tabular}
\end{table}

$Y_S$全体に対する実験の結果を表\ref{tab:evaluation}に示す.
列`top 1'には,スコア1位の訳語候補が正解である割合を,列`top 10'には,正解訳語が
スコア10位以内に含まれる割合を示す.
ここで,$Y_S$全体に対して,スコア1位の訳語候補が正解である用語の割合を再
現率と定義する.
次に,訳語候補が1つ以上生成される用語に限定した評価の結果を表
\ref{tab:precision}に示す.
表の「出力あり」の欄には,訳語候補が1つ以上生成される用語数を示した.
訳語候補が1つ以上生成される用語に対して,スコア1位の訳語候補が正解であ
る割合を精度と定義する.
また,「F値」の欄は,表\ref{tab:evaluation}の値を再現率として計算した.

\begin{table}[t]
 \small
 \centering
 \caption{訳語候補が1つ以上生成される用語に対する評価}
 \label{tab:precision}
 (a) 英語→日本語
\vspace{4pt}

 \begin{tabular}{|c|r|r|r|r|r|r|r|}
  \hline
  & & \multicolumn{3}{|c|}{1位正解} & \multicolumn{3}{|c|}{10位以内正解}\\
      \cline{3-8}
    \raisebox{0.5\normalbaselineskip}[0pt][0pt]{スコア関数} & \raisebox{0.5\normalbaselineskip}[0pt][0pt]{出力あり} 
	& 個数 & 精度 & F値 & 個数 & 精度 & F値\\
  \hline \hline
  DF-CF     & 396 & 262 & 66.2\% & 51.7\% & 309 & 78.0\% & 61.0\%\\
  DF-CF${\rm _f}$    & 303 & 236 & 77.9\% & 51.3\% & 256 & 84.5\% & 55.7\%\\
  DF-CP     & 428 & 269 & 62.9\% & 51.5\% & 323 & 75.5\% & 61.8\%\\
  DF-CO     & 379 & 276 & 72.8\% & {\bf 55.4\%} & 294 & 77.6\% & 59.0\%\\
  DF-CO${\rm _f}$ & 303 & 246 & {\bf 81.2\%} & 53.5\% & 257 & 84.8\% & 55.9\%\\
  DP-CF     & 455 & 284 & 62.4\% & 53.0\% & 335 & 73.6\% & 62.5\%\\
  DP-CP     & 495 & 288 & 58.2\% & 51.8\% & 347 & 70.1\% & 62.4\%\\
  CF        & 456 & 162 & 35.5\% & 30.2\% & 296 & 64.9\% & 55.2\%\\
  CP        & 497 & 160 & 32.2\% & 28.7\% & 300 & 60.4\% & 53.9\%\\
  \hline            
  DF-CF${\rm _f}$-w  & 481 & 321 & {\bf 66.7\%} & {\bf 58.5\%} & 364 & 75.7\% & 66.3\%\\
  DF-CO${\rm _f}$-w  & 481 & 272 & 56.5\% & 49.5\% & 364 & 75.7\% & 66.3\%\\
  \hline            
  DF        & 559 & 220 & 39.4\% & 37.4\% & 364 & 65.1\% & 61.9\%\\
  \hline
 \end{tabular}

     \vspace{\baselineskip}
 (b) 日本語→英語
\vspace{4pt}

 \begin{tabular}{|c|r|r|r|r|r|r|r|}
  \hline
  & & \multicolumn{3}{|c|}{1位正解} & \multicolumn{3}{|c|}{10位以内正解}\\
      \cline{3-8}
  \raisebox{0.5\normalbaselineskip}[0pt][0pt]{スコア関数} & \raisebox{0.5\normalbaselineskip}[0pt][0pt]{出力あり} 
	& 個数 & 精度 & F値 & 個数 & 精度 & F値\\
  \hline \hline
  DF-CF     & 372 & 258 & 69.4\% & 52.2\% & 297 & 79.8\% & 60.1\% \\
  DF-CF${\rm _f}$    & 317 & 244 & 77.0\% & 52.2\% & 272 & 85.8\% & 58.2\% \\
  DF-CP     & 418 & 275 & 65.8\% & 53.1\% & 320 & 76.6\% & 61.8\% \\
  DF-CO     & 369 & 271 & 73.4\% & {\bf 55.0\%} & 300 & 81.3\% & 60.9\% \\
  DF-CO${\rm _f}$ & 317 & 245 & {\bf 77.3\%} & 52.5\% & 272 & 85.8\% & 58.2\%\\
  DP-CF     & 428 & 276 & 64.5\% & 52.8\% & 318 & 74.3\% & 60.9\% \\
  DP-CP     & 489 & 302 & 61.8\% & 54.6\% & 346 & 70.8\% & 62.6\% \\
  CF        & 428 & 193 & 45.1\% & 36.9\% & 270 & 63.1\% & 51.7\% \\
  CP        & 488 & 201 & 41.2\% & 36.4\% & 287 & 58.8\% & 51.9\% \\
  \hline            
  DF-CF${\rm _f}$-w  & 522 & 315 & {\bf 60.3\%} & {\bf 55.3\%} & 406 & 77.8\% & 71.3\% \\
  DF-CO${\rm _f}$-w  & 522 & 309 & 59.2\% & 54.3\% & 401 & 76.8\% & 70.4\% \\
  \hline            
  DF        & 565 & 278 & 49.2\% & 47.0\% & 390 & 69.0\% & 66.0\% \\
  \hline
 \end{tabular}
\end{table}

\subsubsection*{ウェブから収集した専門分野コーパスとウェブ全体の比較}
まず,ウェブから収集した専門分野コーパスを用いる方法とウェブ全体を用いる
方法の比較を行う.
英日方向,日英方向で平均を取ってみると,
ウェブ全体を用いるスコア関数の方が再現率が高いことがわかる.
これは,表\ref{tab:seiseika}からわかるように,集合$Y_S$全体に対して,収集した専門分野コーパスに正解訳語が含ま
れる割合$T_c$が,英日方向で74.9\%,日英方向で65.3\%と,あまり高くないこ
とが原因と考えられる.
精度に関しては,専門分野コーパスを用いるスコア関数であれば`DF-CO${\rm
_f}$'の平均79.3\%が最も高く,ウェブ全体を用いるスコア関数であれば
`DF-CF${\rm _f}$-w'の平均63.5\%が最も高い.
このことから,専門分野コーパスを用いるスコア関数の方が精度が高いことがわ
かる.
これは,ウェブ全体には一般語や訳語推定対象の分野
以外の用語が多数含まれており,不正解の訳語にも大きいスコアが与えられてし
まうためと考えられる.
F値に関しては,専門分野コーパスを用いるスコア関数であれば`DP-CO'の平均
55.2\%が最も高く,ウェブ全体を用いるスコア関数であれば`DF-CF${\rm
_f}$-w'の平均56.9\%が最も高い.
このことから,F値には大きな差はないことがわかる.
以上より,コーパスとしてウェブ全体を用いる手法は再現率を重視した手法と言
える一方,専門分野コーパスを用いる手法は精度を重視した手法と言える.
また,これらの考察から,ウェブから収集した専門分野コーパスを用いる方法に
おいて,精度を下げることなく,再現率を上げるためには,対象分野の用語を十
分に含み,かつ,できるだけ小さなコーパスを収集する必要があることがわかる.
また,両者を相補的に統合する方法としては,まず,ウェブから収集した専門分
野コーパスを用いて高い精度で訳語推定を行い,訳語候補が1つも得られなかっ
た用語に対しては,ウェブ全体を用いて訳語推定を行うことが考えられる.

\subsubsection*{ウェブから収集した専門分野コーパスを用いるスコア関数の評価}
次に,ウェブから収集した専門分野コーパスを用いるスコア関数を比較し評価する.

まず,最も精度が高かったスコア関数は,英日方向,日英方向とも,スコア
関数`DF-CO${\rm _f}$'であった.
2番目に精度が高かったスコア関数は,英日方向,日英方向とも,スコア関数
`DF-CF${\rm _f}$'であった.
`DF-CO${\rm _f}$'や`DF-CF${\rm _f}$'のように,訳語候補の生成途中でコーパ
ススコアを利用しないスコア関数が高い精度となったのは,辞書スコアのみを利
用して生成される訳語候補のほとんどはコーパスに存在せず,最後にコーパスで
検証することにより,これらがすべて消えてしまい,正解訳語が高いスコアとなっ
て残った場合のみ,これが出力されるという現象による.
このため,この2つのスコア関数の「出力あり」の個数は他のスコア関数に比べ
て低い値となっており,再現率は,他のスコア関数に比べ若干低い値となってい
る.
F値で評価した場合は,他のスコア関数と大きな差はない.
F値を下げずに,精度を上げたい場合は,このスコア関数が有効である.

次に,辞書スコア`DF'を用いるスコア関数と辞書スコア`DP'を用いるスコア関数
の比較を行う.
スコア関数`DF-CF'と`DP-CF'を比較し,同様に,スコア関数`DF-CP'と`DP-CP'の
比較を行う.
再現率を比べると,英日方向,日英方向共に,辞書スコア`DP'を用いるスコア
関数の方が再現率が若干高いことがわかる.
これは,表\ref{tab:seiseika},表\ref{tab:fujii_seiseika}で示したように,
対訳辞書スコア`DF'よりも対訳辞書スコア`DP'の方が,正解訳語を生成可能な用
語数が多いことによると考えられる.
一方,精度に関しては,対訳辞書スコア`DF'を用いるスコア関数の方が,若干高
いことがわかる.
F値を見た場合,対訳辞書スコア`DF'を用いるスコア関数と対訳辞書スコア`DP'
を用いるスコア関数にはほとんど差がない.

また,`CF',`CP',`CO'の3つのコーパススコアを用いたスコア関数の再現率を
比較すると,`DF-CF',`DF-CP',`DF-CO'の間で大きな差はない.
ただし,`DF-CF'と`DF-CO'を比較すると,`DF-CO'の再現率及び精度が若干高い.
これは,生成された不正解の訳語が一般的な語であった場合,コーパススコア
`CF'では高いスコアが与えられてしまうことが原因と考えられる.
これに対して,`DF-CO'においては,訳語候補のスコアの値が対訳辞書スコア`DF'の値の
みによって決定されるので,コーパス中に高頻度に出現する一般語に対して過剰
に高いスコアを与えるということはない.
一方,コーパススコア`CP'に注目すると,コーパススコア`CP'を用いたスコア関
数`CF-CP'及び`DP-CP'の精度が他のスコア関数より低いことがわかる.
コーパススコア`CP'を用いると訳語候補全体がコーパスに存在しなくても,
スコアが付与されることとなり,不適切な訳語候補が数多く出力されていると考
えられる.

上記のコーパススコア`CP'の評価と関連して,\cite{Fujii00}で用いられた確率
に基づくスコア関数の評価として,部分対応対訳辞書に加えて英辞郎自体も用い
ることができるように拡張を加えた`DP-CP'に注目する.
このスコアは,他のスコア関数と比べ,性能に大きな差はないが,再現率が若干
高く,精度が若干低い値となっている.
したがって,F値でみると,他のスコア関数とほとんど差がないことがわかる.

対訳辞書スコアのみを用いるスコア関数`DF'は,英日方向では再現率が他のスコ
ア関数より低いが,日英方向では他のスコア関数に比べて遜色のない結果となっ
た.
しかしながら,精度及びF値は他のスコア関数に比べ極めて低い.
このことから,訳語候補の順位付けには,コーパスに基づくスコアを用いること
が必要であることがわかる.


コーパススコアのみを用いるスコア関数`CF'及び`CP'に着目すると,英日方向,
日英方向共に,再現率,精度,F値が最も低い.
このことより,訳語候補の順位付けには,辞書に生起する頻度に基づく何らかの
スコアを利用することが必要であることがわかる.




\subsubsection*{前置詞とハイフンの規則の評価}
\ref{sec:generation}節の脚注\ref{fn:of_hyphen}で述べた前置詞``of''とハイ
フンの挿入・削除に関する規則の効果について述べる.
スコア関数として`DF-CO'を用いて,これらの規則の評価を行ったところ,
英日方向において,この2つの規則を加えることで,正解数が18個増加した.
逆に日英方向の場合,この2つの規則を加えることで,正解数が7個増加した.
このことから,この2つの規則は正解数の向上に有効であることがわかる.

\subsubsection*{分野別の再現率と精度に関する考察}



ここでは,分野別に訳語推定の性能を評価する.
まず,$Y_S$全体に対する再現率の定義と同様に,各分野に属する用語のうち,
スコア1位の訳語候補が正解である用語の割合を分野別再現率と定義する.
同様に,各分野に属する用語のうち,訳語候補が1つ以上生成される用語に対し
て,スコア1位の訳語候補が正解である割合を分野別精度と定義する.

\begin{table}[b]
 \small
 \centering
 \caption{スコア関数`DF-CO'の分野別再現率}
 \label{tab:category_seikairitsu}
 \begin{tabular}{|c|r|r|r|r|}
  \hline
  & \multicolumn{2}{|c|}{英語→日本語} & \multicolumn{2}{|c|}{日本語→英
  語} \\
      \cline{2-5}
      \raisebox{0.5\normalbaselineskip}[0pt][0pt]{分野} 
	& top 1 & top 10 & top 1 & top 10 \\
  \hline \hline
  電磁気学           & 40\% & 47\% & 60\% & 63\%\\
  電気工学           & 46\% & 49\% & 44\% & 46\%\\
  光学               & 32\% & 32\% & 42\% & 48\%\\
  プログラム言語     & 68\% & 71\% & 75\% & 82\%\\
  プログラミング     & 61\% & 64\% & 57\% & 71\%\\
  コンピュータ       & 32\% & 32\% & 38\% & 43\%\\
  解剖学             & 52\% & 57\% & 36\% & 41\%\\
  疾患               & 48\% & 49\% & 50\% & 52\%\\
  化学物質及び薬物   & 35\% & 37\% & 33\% & 35\%\\
  物理化学及び統計学 & 51\% & 56\% & 43\% & 51\%\\
  \hline
  全体               & 44.7\% & 47.6\% & 43.9\% & 48.6\%\\
  \hline
 \end{tabular}
\end{table}

ウェブから収集した専門分野コーパスを用いるスコア関数の中では最もF値の値
が高かったスコア関数`DF-CO'を対象とし,分野別再現率を表
\ref{tab:category_seikairitsu}に示す.
これと,表\ref{tab:seiseika}の$T_{gc}$に属する用語の割合を比較すると,$T_{gc}$ 
に属する用語の割合が小さい分野ほど,再現率が低くなっていることがわかる.
正解訳語が生成可能な用語を増やすための対訳辞書の強化と,コーパスに正解訳
語が含まれる割合の改善が課題となる.

\begin{table}[t]
 \small
 \centering
 \caption{訳語候補が1つ以上生成される用語に対する分野別精度とF値: スコア関数`DF-CO'の結果}
 \label{tab:category_seido}
 (a) 英語→日本語
\vspace{4pt}

 \begin{tabular}{|c|r|r|r|r|r|r|r|}
  \hline
  & & \multicolumn{3}{|c|}{1位正解} & \multicolumn{3}{|c|}{10位以内正解} \\
      \cline{3-8}
    \raisebox{0.5\normalbaselineskip}[0pt][0pt]{分野} & \raisebox{0.5\normalbaselineskip}[0pt][0pt]{出力あり} 
	& 個数 & 精度 & F値 & 個数 & 精度 & F値\\
  \hline \hline
  電磁気学           & 16 & 12 & 75\% & 52\% & 14 & 88\% & 61\% \\
  電気工学           & 26 & 19 & 73\% & 57\% & 20 & 77\% & 60\% \\
  光学               & 14 & 10 & 71\% & 44\% & 10 & 71\% & 44\% \\
  プログラム言語     & 25 & 19 & 76\% & 72\% & 20 & 80\% & 75\% \\
  プログラミング     & 21 & 17 & 81\% & 69\% & 18 & 86\% & 73\% \\
  コンピュータ       & 53 & 32 & 60\% & 42\% & 32 & 60\% & 42\% \\
  解剖学             & 64 & 47 & 73\% & 61\% & 52 & 81\% & 67\% \\
  疾患               & 54 & 41 & 76\% & 59\% & 42 & 78\% & 60\% \\
  化学物質及び薬物   & 36 & 29 & 81\% & 48\% & 31 & 86\% & 52\% \\
  物理化学及び統計学 & 70 & 50 & 71\% & 59\% & 55 & 79\% & 65\% \\
  \hline	         	         
  全体               & 379 & 276 & 72.8\% & 55.4\% & 294 & 77.6\% & 59.0\% \\
  \hline
 \end{tabular}

     \vspace{\baselineskip}
 (b) 日本語→英語
\vspace{4pt}

 \begin{tabular}{|c|r|r|r|r|r|r|r|}
  \hline
  & & \multicolumn{3}{|c|}{1位正解} & \multicolumn{3}{|c|}{10位以内正解} \\
      \cline{3-8}
      \raisebox{0.5\normalbaselineskip}[0pt][0pt]{分野} & \raisebox{0.5\normalbaselineskip}[0pt][0pt]{出力あり} 
	& 個数 & 精度 & F値 & 個数 & 精度 & F値\\
  \hline \hline
  電磁気学           & 21 & 18 & 86\% & 71\% & 19 & 90\% & 75\% \\
  電気工学           & 26 & 18 & 69\% & 54\% & 19 & 73\% & 57\% \\
  光学               & 16 & 13 & 81\% & 55\% & 15 & 94\% & 64\% \\
  プログラム言語     & 25 & 21 & 84\% & 79\% & 23 & 92\% & 87\% \\
  プログラミング     & 22 & 16 & 73\% & 64\% & 20 & 91\% & 80\% \\
  コンピュータ       & 62 & 38 & 61\% & 47\% & 43 & 69\% & 53\% \\
  解剖学             & 51 & 33 & 65\% & 46\% & 37 & 73\% & 52\% \\
  疾患               & 51 & 43 & 84\% & 63\% & 45 & 88\% & 66\% \\
  化学物質及び薬物   & 34 & 28 & 82\% & 47\% & 29 & 85\% & 49\% \\
  物理化学及び統計学 & 61 & 43 & 70\% & 54\% & 50 & 82\% & 63\% \\
  \hline	         	         
  全体               & 369 & 271 & 73.4\% & 55.0\% & 300 & 81.3\% &60.9\%\\
  \hline
 \end{tabular}
\end{table}

表\ref{tab:category_seido}に,訳語候補が1つ以上生成される用語に対する分
野別精度を示す.
スコア関数としては`DF-CO'を用いた.
表\ref{tab:category_seikairitsu}において$Y_S$全体に対する分野別再現率が
他の分野と比べて低かったのは「化学物質及び薬物」の分野であったが,表
\ref{tab:category_seido}においては,「化学物質及び薬物」の分野の精度
は英日方向,日英方向とも80\%以上という高い精度となっている.
訳語候補が1つ以上生成される用語に対する評価では,$T_{gc}$に属する割合との
相関はないことがわかる.

\subsubsection*{翻訳ソフトによる翻訳性能との比較}
$Y_S$の用語を市販の翻訳ソフトで翻訳し,その翻訳性能と表\ref{tab:evaluation}の再現率を比較する.
翻訳ソフトとしては,富士通の「ATLAS 翻訳パーソナル 2003」,東芝の「The 
翻訳オフィスV6.0」,IBMの「インターネット翻訳の王様バイリンガル Version
5」の3種類を用いた.
この実験は,構成要素の訳語選択がどの程度できるのかを調べるのが目的である
ので,翻訳ソフトのオプションの専門用語辞書は使用しなかった.
このうち最も性能が良かったのは,富士通の翻訳ソフト「ATLAS 翻訳パーソナル 
2003」で翻訳した場合で,翻訳結果が正解であった用語の割合は英日方向26.7\%,
日英方向で38.1\%であった.
スコア関数`CF'及び`CP'を除くすべてのスコア関数の再現率は,翻訳ソフトによ
る翻訳結果が正解であった用語の割合を上回っている.


\subsection{正解訳語が生成できない原因の分析}
\label{sec:seiseifuka_analysis}


\begin{table}[p]
 \small
 \centering
 \caption{生成不可の原因分析:集合$Y_S$のうち,正解訳語が生成不可能な用語を対象}
 \label{tab:seiseifuka}
 (a) 英語→日本語
\vspace{4pt}

 \begin{tabular}{|l|r|r|}
  \hline
  生成不可の主原因 & 個数 & 割合 \\
  \hline
  非構成的 & 58 & 33\%\\
  辞書にエントリがない & 73 & 42\%\\
  表記の揺れ & 6 & 3\%\\
  前置詞による順序交換 & 6 & 3\%\\
  前置詞なし順序交換 & 18 & 10\%\\
  訳語に「の」が必要 & 2 & 1\%\\
  「性」を挿入する必要 & 1 & 1\%\\
  アルファベットのままにすべき & 2 &1\%\\
  正解訳語に「・」を含む & 1 & 1\%\\
  複数形で辞書引き失敗 & 6 & 3\%\\
  ハイフンの挿入が必要 & 1 & 1\%\\
  \hline
  合計 & 174& 100\%\\
  \hline
 \end{tabular}

     \vspace{\baselineskip}
 (b) 日本語→英語
\vspace{4pt}

 \begin{tabular}{|l|r|r|}
  \hline
  生成不可の主原因 & 個数 & 割合 \\
  \hline
  非構成的 & 46 & 31\%\\
  辞書にエントリがない & 66 & 44\%\\
  表記の揺れ & 15 & 10\%\\
  前置詞による順序交換 & 4 & 3\%\\
  前置詞なし順序交換 & 13 & 9\%\\
  「性」を外す必要 & 2 & 1\%\\
  アルファベットのままにすべき & 1 & 1\%\\
  用語に「・」を含む & 1 & 1\%\\
  正解訳語が複数形 & 1 & 1\%\\
  訳語中に冠詞が必要 & 1 & 1\%\\
  \hline
  合計 & 150 & 100\%\\
  \hline
 \end{tabular}
    \par\vspace{\baselineskip}
 \small
 \centering
 \caption{スコア関数DF-COの訳語推定結果の分析:集合$Y_{S}$を対象}
 \label{tab:DF-CO}
 \begin{tabular}{|l||r|r|r|r|r|r|}
  \hline
  & \multicolumn{6}{|c|}{スコア1位の訳語候補}\\
  \hhline{|~|---|---|}
  & \multicolumn{3}{|c|}{英語→日本語} & \multicolumn{3}{|c|}{日本語→英
  語}\\
  \hhline{|~|---|---|}
  & 正解 & 不正解 & なし & 正解 & 不正解 & なし\\
  \hline\hline
  $T_{gc}$            & 276 & 30 & 26 & 271 & 36& 13\\
  \hline
  $T_g-T_{gc}$        & --  & 40  & 71 & -- & 40  & 107\\
  \hline
  $\overline{T_g}$ & --  & 33  & 141 & -- & 22  & 128\\
  \hline\hline
  $Y_S$ & 276 & 103  & 238 & 271 & 98 & 248\\
  \hline
 \end{tabular}
\end{table}



本節では,対訳辞書として,英辞郎と部分対応対訳辞書$B_P$,$B_S$を用いたと
きに,正解訳語が生成できない場合に関して,その主原因を調査した結果につい
て述べる.
分析対象は,表\ref{tab:seiseika}に示した生成可能な用語の集合($T_g$)に属
さない用語であり,英日方向で174語,日英方向で150語である.
分析結果を表\ref{tab:seiseifuka}に示す.
まず,用語とその正解訳語が構成的に対応しておらず,本手法では扱えないもの
が,英日方向で33\%,日英方向で31\%存在した.
これは英日方向で,集合$Y_S$全体の9.4\%,日英方向で7.5\%に相当する.
次に,辞書にエントリがないことが原因であるものが,英日方向で42\%,日英方
向で44\%存在した.
「ディジタル」と「デジタル」のような表記の揺れが原因であるものが,英日方
向で3\%,日英方向で10\%存在した.
これらに対しては,部分対応対訳辞書の強化が課題となる.
英語用語中に前置詞を含まず英語と日本語で語順が異なるものが,英日方向で
10\%,日英方向で9\%存在した.
これらは,医学分野に多いため,特定の語が構成要素に現れた場合は,語順の入
れ替えを行うという対処法が考えられる.

\subsection{スコア関数`DF-CO'とスコア関数`DF-CF${\rm _f}$-w'の併用と誤り分析}
\label{sec:error_analysis}


\ref{sec:evaluation}節の評価では,ウェブから収集した専門分野コーパスを用
いる方法は精度に優れ,ウェブ全体を用いる方法は再現率に優れることを示した.
そこで,本節では,両者を相補的に統合するために,ウェブから収集した専門分
野コーパスを用いるスコア関数で訳語推定を行い,その結果,訳語候補が1つも
生成されない場合は,サーチエンジンを通してウェブ全体を利用するスコア関数
を用いるというアプローチの評価と誤り分析を行う.
ここでは,各々の方法におけるスコア関数としては,それぞれ,最もF値が大き
かった`DF-CO',および,`DF-CF${\rm _f}$-w'を用いる.

まず,スコア関数`DF-CO'で訳語推定を行った結果を,図\ref{fig:benzu}で示した
生成可能性に関する分類を利用して整理した.
その結果を表\ref{tab:DF-CO}に示す.
\ref{sec:evaluation_set}節で説明したように,$T_{gc}$は生成可能かつ専門分
野コーパスに含まれる正解訳語が存在する用語の集合である.
$T_g$は,正解訳語を生成可能な用語の集合なので,$T_g-T_{gc}$は,いずれか
の正解訳語は生成可能であるが,生成可能かつ専門分野コーパスに含まれる正解
訳語は存在しない用語の集合となる.
そして,$\overline{T_g}$は正解訳語が生成不可能な用語の集合である.
これらの,$T_{gc}$,$T_g-T_{gc}$,及び$\overline{T_g}$に含まれる用語を,
スコアが1位の訳語候補が正解か,不正解か,もしくは,訳語候補が出力されな
いかによって,それぞれ再分類した.

\begin{table}[b]
 \small
 \centering
 \caption{スコア関数DF-COにおいて,$T_{gc}$中の用語に対して,正解訳語のスコアが1位とならない原因の\hspace*{32pt}分析}
 \label{tab:sukoa-make}
 \begin{tabular}{|l|r|r|r|r|}
  \hline
  & \multicolumn{2}{|c|}{英語→日本語} & \multicolumn{2}{|c|}{日本語→英語}\\
      \cline{2-5}
      \raisebox{0.5\normalbaselineskip}[0pt][0pt]{正解訳語のスコアが1位とならない主原因} 
	& 個数 & 割合 & 個数 & 割合\\
  \hline
  正解訳語の辞書スコアがゼロ & 13 & 23\% & 10 & 20\% \\
  正解訳語が生成過程で枝刈りされる & 25 & 45\% & 10 & 20\% \\
  その他 & 18 & 32\% & 29 & 59\%\\
  \hline
  合計 & 56 & 100\% & 49 & 100\% \\
  \hline
 \end{tabular}
\end{table}

さらに,$T_{gc}$の用語のうち,スコア1位の訳語候補が「不正解」のものと
「なし」のものに関して,正解訳語が1位とならなかった原因の分析を行った.
英日方向では56個,日英方向では49個がその対象である.
その結果を表\ref{tab:sukoa-make}に示す.
まず,正解訳語の辞書スコアがゼロとなることが原因であるものがあった.
これは,辞書スコア`DF'が,部分対応対訳辞書に一度しか現れない訳語対のスコ
アをゼロとするように設計されているためである.
次に,正解訳語が訳語候補の生成過程で枝刈りされてしまっていることがあった.
また,「その他」の理由としては,コーパス中に出現する頻度を考慮したスコア
関数を用いていないことなどが挙げられる.

\begin{table}[b]
 \small
 \centering
 \caption{スコア関数DF-COとスコア関数DF-CF${\rm _f}$-wの差分の分析:スコア関数DF-COで訳語候補が生\hspace*{32pt}成されない用語を対象}
 \label{tab:DF-CFf-w}
 \begin{tabular}{|l|r|rr|r|rr|r|}
  \hline
      \multicolumn{2}{|c}{} &\multicolumn{3}{|c|}{英語→日本語} 
	& \multicolumn{3}{|c|}{日本語→
  英語}\\
  \hhline{|~~|---|---|}
      \multicolumn{2}{|c}{} & \multicolumn{3}{|c|}{正解訳語生成可能性} 
	& \multicolumn{3}{|c|}{正
  解訳語生成可能性}\\
  \hhline{|~~|---|---|}
      \multicolumn{2}{|c}{} & \multicolumn{2}{|c|}{可} 
	& 不可 & \multicolumn{2}{|c|}{可}
  & 不可 \\
  \hline\hline
   &  & 1位 & 63 &  & 1位 & 83 & \\
  \hhline{|~~|--~|--~|}
  訳語 & あり & 2〜10位 & 4 & \multirow{2}{*}{39} & 2〜10位 & 13 & \multirow{2}{*}{43}\\
  \hhline{|~~|--~|--~|}
  候補 &  & 10位以下 & 0 &  & 10位以下 & 1 & \\
  \hhline{|~~|--~|--~|}
  出力 &  & 出力されない & 8 &  & 出力されない & 16 & \\
  \hhline{|~|-|--|-|--|-|}
   & なし & & 22 & 102 &  & 7 & 85\\
  \hline \hline
  \multicolumn{2}{|c|}{合計} & \multicolumn{3}{|c|}{238} & \multicolumn{3}{|c|}{248}\\
  \hline
 \end{tabular}
\end{table}

次に,スコア関数`DF-CO'で,訳語候補が生成されなかった用語を対象として,
スコア関数`DF-CF${\rm _f}$-w'を利用して訳語推定を行い,その性能を評価した.
対象は,英日方向では238個,日英方向では248個の用語である.
評価結果を表\ref{tab:DF-CFf-w}に示す.
ここではまず,1つ以上の訳語候補が出力されたか否かにより,「あり」と「な
し」に分類している.
さらに,それぞれの分類に対し,正解訳語が生成可能か否かによって,さらに分
類している.
そして,正解訳語が生成可能かつ,訳語候補が出力される用語に対しては,正解
訳語のスコアの順位でさらに分類を行っている.
ここで,正解訳語がスコア1位となったものは,英日方向で63個,日英方向で83
個である.
これと,スコア関数`DF-CO'の正解を合わせると,正解数は英日方向で339個,日
英方向で354個となる.
集合$Y_S$全体に対する再現率を求めると,英日方向で54.9\%,日英方向
で57.4\%となり,他のどのスコア関数よりも高い.
また,最終的に訳語候補が1つ以上出力されるものを対象にした評価を行うと,
精度は,英日方向で68.8\%,日英方向で67.4\%となり,スコア関数`DF-CO'に比
べて精度を大きく下げることなく,正解数を増やすことに成功した.
さらに,F値は,英日方向で61.1\%,日英方向で62.0\%となり,他のどのスコア
関数よりも高い.
このことから,スコア関数`DF-CO'とスコア関数`DF-CF${\rm _f}$-w'を組み合わ
せるアプローチが有効であることがわかる.


ここで,正解訳語が2位以下に出力される用語に対しては,\cite{Kida06}で提
案されている用語の分野判定の技術により訳語候補の分野判定を行い,分野外の
訳語候補を削除することによって,正解訳語を1位にできる可能性があると考え
られる.
また,正解訳語が生成不可で訳語候補が1つ以上出力されている用語に対しては,訳
語候補の分野判定を行うことによって,候補数をゼロにできる可能性がある.

\section{関連研究}
\label{sec:related_work}

関連研究として,\cite{Fujii00}は,言語横断情報検索の目的のため
に要素合成法による訳語推定法を提案した.
本論文では,ここで提案されているスコア関数に対して,部分対応対訳辞書だけ
でなく,英辞郎自体も利用できるように拡張し(スコア関数`DP-CP'),他のスコ
ア関数との比較を行った.
その結果,スコア関数`DP-CP'は他のスコア関数と比べ,$Y_S$全体に対する評価
では最も再現率が高かったが,不正解訳語も多く生成されるため,訳語候補が1
つ以上生成される用語に対する評価では,精度は高くないことがわかった.
そして,F値に関しては,他のスコア関数とほとんど差がなかった.
\cite{Fujii00}の手法と,本論文で提案した手法の重要な違いの一つは,
\cite{Fujii00}においては,訳語推定対象の用語が属する分野の文書のみを含む
コーパスではなく,様々な専門分野にわたる65種類の日本の学会から出版された
技術論文を集めたものをコーパスとして利用していることである.
また,\cite{Fujii00}において,彼らは言語横断情報検索の性能のみを評価し,
訳語推定の性能評価はしていない.

\cite{Adama04}も,言語横断情報検索の性能を評価対象として,クエリー翻訳の
方法を提案している.
この研究では,コーパススコアはNTCIR-1\cite{Kando99-NTCIR1},または,NTCIR-2\cite{Kando01-NTCIR2-JEIR}の言語横断情報検索タスクの検索課題文
書(論文等の技術文書)から求める.
コーパススコアには,\cite{Fujii00}で提案されたスコアと合わせて$\chi^2$検
定を用いたスコアを併用している.
しかしながら,2つのコーパススコアの併用は,言語横断情報検索の精度向上に
は貢献しなかったと報告されている.
また,カタカナ語に関しては翻字技術を適用している.

\cite{Baldwin04multi}も要素合成法による訳語推定手法を提案している.
コーパスに基づく8つの素性と辞書に基づく6つの素性とテンプレートに基づく2 
つの素性を立て,SVMを利用して訳語候補のスコア関数を学習している.
この論文でも,辞書に基づく素性でのみ,もしくは,コーパスに基づく素性での
みスコア関数を構成するよりも,両者を利用した方が精度が良いことが報告され
ている.
\cite{Baldwin04multi}の手法と,本論文で提案した手法の重要な違いは,
\cite{Baldwin04multi}においては,コーパスとして,
英語側ではReuters Corpusを,日本語側では毎日新聞を利用しているの
に対し,本論文では,専門分野コーパスを利用した場合と,サーチエンジンを通
してウェブ全体を利用した場合の比較を行っている.
また,\cite{Baldwin04multi}においては,訳語推定対象の用語を,英語2単語
または,日本語2形態素のものに限定している.

\cite{Cao02as}もまた,複合語に対する要素合成法による訳語推定法を提案した.
\cite{Cao02as}の手法では,用語の訳語候補は,用語の構成要素の訳語を結合す
ることによって構成的に生成され,サーチエンジンを通してウェブ全体を用いて
検証される.
本論文では,サーチエンジン通してウェブ全体を用いて訳語候補の検証をするス
コア関数を導入することによって,\cite{Cao02as}で提案されたアプローチの評
価を行い,ウェブから収集した専門分野コーパスを用いる方法と比較した.
その結果,訳語候補が一つ以上出力される場合においては,サーチエンジンを通
してウェブ全体を用いるよりも,ウェブから収集した専門分野コーパスを用いる
方が精度が良いことがわかった.
その一方で,再現率に優れるのは,サーチエンジンを通してウェブ全体を用いる
方法であった.
そこで,この2つの方法の長所を生かすために,まず,ウェブから収
集した専門分野コーパスを用いる方法で訳語推定を行い,訳語候補が一つも得ら
れなかった場合,サーチエンジン通してウェブ全体を用いて訳語推定を行う方法
を評価した.
その結果,本論文で評価したどのスコア関数よりも高いF値を達成できることが
わかった.
なお,\cite{Cao02as}においては,英語の用語に対して中国語の訳語を推定して
いるが,訳語推定対象の用語は英語2単語から構成されるものに限定されている.

\cite{Maeda00}は,言語横断情報検索のためのクエリー翻訳の方法を提案してい
る.
まず,要素合成法によりクエリーの訳語候補を生成する.
次に,ウェブ上の頻度が一定数以上の訳語候補に対して,それぞれの訳語候補の
スコアは,訳語候補の構成要素間の相互情報量を拡張した尺度で計算される.
最後に,スコアの閾値を越える訳語候補を(サーチエンジンの)OR演算子で結合
したものを,クエリーの翻訳結果とする.
評価は言語横断情報検索の性能に関して行っているので,訳語推定結果の比較は
できないが,\cite{Maeda00}らの手法は,本論文で言えば,コーパススコアのみ
のスコア関数を利用した手法に対応する.

\cite{Kimura04}も,言語横断情報検索のために,クエリー翻訳における訳語
曖昧性解消の方法を提案している.
準備として,あらかじめYahoo!の日英のウェブディレクトリのそれぞれのカテゴ
リにおいて,特徴語の抽出と重み付与をし,日英のカテゴリの対応付けをしてし
ておく.
検索をするときは,まず,クエリーに含まれる単語とカテゴリの特徴語を利用
して適合するカテゴリを決める.
次に,クエリーに含まれる各単語に対して,訳語を対訳辞書で調べる.
そして,適合カテゴリの特徴語となっている訳語のうち,特徴語の重みが最も大
きいものをその単語の訳語と決定する.

\section{おわりに}
\label{sec:end}


本論文では,ウェブを利用した専門用語の訳語推定法について述べた.
これまでに行われてきた訳語推定の方法の1つに,パラレルコーパス・コンパラ
ブルコーパスを用いた訳
語推定法があるが,既存のコーパスが利用できる分野は極めて限られている.
そこで,本論文では,訳を知りたい用語を構成する単語・形態素の訳語を既存の
対訳辞書から求め,これらを結合することにより訳語候補を生成し,単言語コー
パスを用いて訳語候補を検証するという手法を採用した.
しかしながら,単言語コーパスであっても,研究利用可能なコーパスが整備されて
いる分野は限られている.
このため,本論文では,ウェブをコーパスとして用いた.
ウェブを訳語候補の検証に利用する場合,サーチエンジンを通してウェブ全体を
利用する方法と,訳語推定の前にあらかじめ,ウェブから専門分野コーパスを収
集しておく方法が考えられる.
本論文では,評価実験を通して,この2つのアプローチを比較し,その得失を論
じた.
また,訳語候補のスコア関数として多様な関数を定式化し,訳語推定の性能と
の間の相関を評価した.
実験の結果,ウェブから収集した専門分野コーパスを用いた場合,ウェブ全体を
用いるよりカバレージは低くなるが,その分野の文書のみを利用して訳語候補の検証を
行うため,誤った訳語候補の生成を抑える効果が確認され,高い精度を達成できることがわかった.
また,ウェブ全体を用いる方法とウェブから収集した専門分野コーパスを用いる
方法を相補的に結合することにより,再現率とF値を改善できることを示した.

今後の課題として,訳語推定対象の分野の用語を十分に含むできるだけ小さいコー
パスを収集することが挙げられる.
また,本論文で提案した,ウェブを用いた要素合成法による訳語推定法を,他の
訳語推定技術と相補的に用いることが挙げられる.
相補的な技術としては,用語とその訳語が併記されたテキストの利用
\cite{Nagata01asl,huang-zhang-vogel:2005:HLTEMNLP}や,固有名詞の翻字の技
術\cite{Knight98,Oh05}などが挙げられる.
また,用語の分野判定の技術\cite{Kida06}を利用することにより,不適切な訳
語候補を削除することが挙げられる.
応用的な課題としては,本論文で提案した専門用語の訳語推定手法を,例えば,
ウェブからの関連語収集手法\cite{Sasaki06}や,論文からの用語抽出
\cite{Banba06aj}の結果に対して適用することが考えられる.


\bibliographystyle{jnlpbbl_1.2}
\newcommand{\gengoshori}{}\newcommand{\kokuken}{}
\begin{thebibliography}{}

\bibitem[\protect\BCAY{阿玉\JBA 橋本\JBA 徳永\JBA 田中}{阿玉\Jetal
  }{2004}]{Adama04}
阿玉泰宗\JBA 橋本泰一\JBA 徳永健伸\JBA 田中穂積 \BBOP 2004\BBCP.
\newblock \JBOQ 日英言語横断情報検索のための翻訳知識の獲得\JBCQ\
\newblock \Jem{情報処理学会論文誌:データベース}, {\Bbf 45}  (SIG10(TOD23)),
  \mbox{\BPGS\ 37--48}.

\bibitem[\protect\BCAY{Baldwin \BBA\ Tanaka}{Baldwin \BBA\
  Tanaka}{2004}]{Baldwin04multi}
Baldwin, T.\BBACOMMA\ \BBA\ Tanaka, T. \BBOP 2004\BBCP.
\newblock \BBOQ Translation by Machine of Compound Nominals: Getting it
  Right\BBCQ\
\newblock In {\Bem Proc. ACL 2004 Workshop on Multiword Expressions:
  Integrating Processing}, \mbox{\BPGS\ 24--31}.

\bibitem[\protect\BCAY{馬場\JBA 外池\JBA 宇津呂\JBA 佐藤}{馬場\Jetal
  }{2006}]{Banba06aj}
馬場康夫\JBA 外池昌嗣\JBA 宇津呂武仁\JBA 佐藤理史 \BBOP 2006\BBCP.
\newblock \JBOQ 対訳辞書とウェブを利用した専門文書中の用語の訳語推定\JBCQ\
\newblock \Jem{言語処理学会第12回年次大会論文集}, \mbox{\BPGS\ 416--419}.

\bibitem[\protect\BCAY{Cao \BBA\ Li}{Cao \BBA\ Li}{2002}]{Cao02as}
Cao, Y.\BBACOMMA\ \BBA\ Li, H. \BBOP 2002\BBCP.
\newblock \BBOQ Base Noun Phrase Translation Using {Web} Data and the {EM}
  Algorithm\BBCQ\
\newblock In {\Bem Proc. 19th {COLING}}, \mbox{\BPGS\ 127--133}.

\bibitem[\protect\BCAY{コンピュータ用語辞典編集委員会}{コンピュータ用語辞典編
集委員会}{2001}]{dic-computer}
コンピュータ用語辞典編集委員会\JED\ \BBOP 2001\BBCP.
\newblock \Jem{英和コンピュータ用語大辞典}.
\newblock 日外アソシエーツ.

\bibitem[\protect\BCAY{藤井\JBA 石川}{藤井\JBA 石川}{2000}]{Fujii00}
藤井敦\JBA 石川徹也 \BBOP 2000\BBCP.
\newblock \JBOQ 技術文書を対象とした言語横断情報検索のための複合語翻訳\JBCQ\
\newblock \Jem{情報処理学会論文誌}, {\Bbf 41}  (4), \mbox{\BPGS\ 1038--1045}.

\bibitem[\protect\BCAY{Fung \BBA\ Yee}{Fung \BBA\ Yee}{1998}]{Fung98as}
Fung, P.\BBACOMMA\ \BBA\ Yee, L.~Y. \BBOP 1998\BBCP.
\newblock \BBOQ An {IR} Approach for Translating New Words from Nonparallel,
  Comparable Texts\BBCQ\
\newblock In {\Bem Proc. 17th {COLING} and 36th {ACL}}, \mbox{\BPGS\ 414--420}.

\bibitem[\protect\BCAY{Huang, Zhang, \BBA\ Vogel}{Huang
  et~al.}{2005}]{huang-zhang-vogel:2005:HLTEMNLP}
Huang, F., Zhang, Y., \BBA\ Vogel, S. \BBOP 2005\BBCP.
\newblock \BBOQ Mining Key Phrase Translations from Web Corpora\BBCQ\
\newblock In {\Bem Proc. HLT/EMNLP}, \mbox{\BPGS\ 483--490}.

\bibitem[\protect\BCAY{医学用電子化AI辞書研究会}{医学用電子化AI辞書研究会}{199
6}]{dic-25igaku}
医学用電子化AI辞書研究会\JED\ \BBOP 1996\BBCP.
\newblock \Jem{25万語医学用語大辞典}.
\newblock 日外アソシエーツ.

\bibitem[\protect\BCAY{Kando, Kuriyama, \BBA\ Yoshioka}{Kando
  et~al.}{2001}]{Kando01-NTCIR2-JEIR}
Kando, N., Kuriyama, K., \BBA\ Yoshioka, M. \BBOP 2001\BBCP.
\newblock \BBOQ Overview of Japanese and English Information Retrieval Tasks
  (JEIR) at the Second NTCIR Workshop\BBCQ\
\newblock In {\Bem Proc. 2nd NTCIR Workshop Meeting}, \mbox{\BPGS\ 73--96}.

\bibitem[\protect\BCAY{Kando, Kuriyama, \BBA\ Nozue}{Kando
  et~al.}{1999}]{Kando99-NTCIR1}
Kando, N., Kuriyama, K., \BBA\ Nozue, T. \BBOP 1999\BBCP.
\newblock \BBOQ NACSIS test collection workshop (NTCIR-1)\BBCQ\
\newblock In {\Bem Proc. 22nd SIGIR}, \mbox{\BPGS\ 299--300}.

\bibitem[\protect\BCAY{木田\JBA 外池\JBA 宇津呂\JBA 佐藤}{木田\Jetal
  }{2006}]{Kida06}
木田充洋\JBA 外池昌嗣\JBA 宇津呂武仁\JBA 佐藤理史 \BBOP 2006\BBCP.
\newblock \JBOQ ウェブを利用した専門用語の分野判定\JBCQ\
\newblock \Jem{電子情報通信学会論文誌}, {\Bbf J89-D}  (未定).

\bibitem[\protect\BCAY{木村\JBA 前田\JBA 宮崎\JBA 吉川\JBA 植村}{木村\Jetal
  }{2004}]{Kimura04}
木村文則\JBA 前田亮\JBA 宮崎純\JBA 吉川正俊\JBA 植村俊亮 \BBOP 2004\BBCP.
\newblock \JBOQ Webディレクトリを言語資源として利用した言語横断情報検索\JBCQ\
\newblock \Jem{情報処理学会論文誌:データベース}, {\Bbf 45}  (SIG7(TOD22)),
  \mbox{\BPGS\ 208--217}.

\bibitem[\protect\BCAY{Knight \BBA\ Graehl}{Knight \BBA\
  Graehl}{1998}]{Knight98}
Knight, K.\BBACOMMA\ \BBA\ Graehl, J. \BBOP 1998\BBCP.
\newblock \BBOQ Machine Transliteration\BBCQ\
\newblock {\Bem Computational Linguistics}, {\Bbf 24}  (4), \mbox{\BPGS\
  599--612}.

\bibitem[\protect\BCAY{前田\JBA 吉川\JBA 植村}{前田\Jetal }{2000}]{Maeda00}
前田亮\JBA 吉川正俊\JBA 植村俊亮 \BBOP 2000\BBCP.
\newblock \JBOQ 言語横断情報検索におけるWeb文書群による訳語曖昧性解消\JBCQ\
\newblock \Jem{情報処理学会論文誌:データベース}, {\Bbf 41}  (SIG6(TOD7)),
  \mbox{\BPGS\ 12--21}.

\bibitem[\protect\BCAY{Matsumoto \BBA\ Utsuro}{Matsumoto \BBA\
  Utsuro}{2000}]{Matsumoto00a}
Matsumoto, Y.\BBACOMMA\ \BBA\ Utsuro, T. \BBOP 2000\BBCP.
\newblock \BBOQ Lexical Knowledge Acquisition\BBCQ\
\newblock In Dale, R., Moisl, H., \BBA\ Somers, H.\BEDS, {\Bem {\em Handbook of
  Natural Language Processing}}, \BCH~24, \mbox{\BPGS\ 563--610}. Marcel Dekker
  Inc.

\bibitem[\protect\BCAY{マグローヒル科学技術用語大辞典編集委員会}{マグローヒル
科学技術用語大辞典編集委員会}{1998}]{dic-McGraw-Hill}
マグローヒル科学技術用語大辞典編集委員会\JED\ \BBOP 1998\BBCP.
\newblock \Jem{マグローヒル科学技術用語大辞典}.
\newblock 日刊工業新聞社.

\bibitem[\protect\BCAY{長尾\JBA 石田\JBA 稲垣\JBA 田中\JBA 辻井\JBA 所\JBA
  中田\JBA 米澤}{長尾\Jetal }{1990}]{dic-iwanami-info}
長尾真\JBA 石田晴久\JBA 稲垣康善\JBA 田中英彦\JBA 辻井潤一\JBA 所真理雄\JBA
  中田育男\JBA 米澤明憲\JEDS\ \BBOP 1990\BBCP.
\newblock \Jem{岩波情報科学辞典}.
\newblock 岩波書店.

\bibitem[\protect\BCAY{Nagata, Saito, \BBA\ Suzuki}{Nagata
  et~al.}{2001}]{Nagata01asl}
Nagata, M., Saito, T., \BBA\ Suzuki, K. \BBOP 2001\BBCP.
\newblock \BBOQ Using the {Web} as a Bilingual Dictionary\BBCQ\
\newblock In {\Bem Proc. Workshop on Data-driven Methods in Machine
  Translation}, \mbox{\BPGS\ 95--102}.

\bibitem[\protect\BCAY{Oh \BBA\ Choi}{Oh \BBA\ Choi}{2005}]{Oh05}
Oh, J.\BBACOMMA\ \BBA\ Choi, K. \BBOP 2005\BBCP.
\newblock \BBOQ Automatic Extraction of English-Korean Translations for
  Constituents of Technical Terms\BBCQ\
\newblock In {\Bem Proc. 2nd IJCNLP}, \mbox{\BPGS\ 450--461}.

\bibitem[\protect\BCAY{Rapp}{Rapp}{1999}]{Rapp99as}
Rapp, R. \BBOP 1999\BBCP.
\newblock \BBOQ Automatic Identification of Word Translations from Unrelated
  {English} and {German} Corpora\BBCQ\
\newblock In {\Bem Proc. 37th {ACL}}, \mbox{\BPGS\ 519--526}.

\bibitem[\protect\BCAY{佐々木\JBA 宇津呂\JBA 佐藤}{佐々木\Jetal
  }{2006}]{Sasaki06}
佐々木靖弘\JBA 宇津呂武仁\JBA 佐藤理史 \BBOP 2006\BBCP.
\newblock \JBOQ 関連用語収集問題とその解法\JBCQ\
\newblock \Jem{自然言語処理}, {\Bbf 13}  (3), \mbox{\BPGS\ 151--175}.

\bibitem[\protect\BCAY{高木\JBA 木田\JBA 外池\JBA 佐々木\JBA 日野\JBA
  宇津呂\JBA 佐藤}{高木\Jetal }{2005}]{Takagi05aj}
高木俊宏\JBA 木田充洋\JBA 外池昌嗣\JBA 佐々木靖弘\JBA 日野浩平\JBA
  宇津呂武仁\JBA 佐藤理史 \BBOP 2005\BBCP.
\newblock \JBOQ
  ウェブを利用した専門用語対訳集自動生成のための訳語候補収集\JBCQ\
\newblock \Jem{言語処理学会第11回年次大会論文集}, \mbox{\BPGS\ 13--16}.

\bibitem[\protect\BCAY{Tonoike, Kida, Takagi, Sasaki, Utsuro, \BBA\
  Sato}{Tonoike et~al.}{2005}]{Tonoike05cs}
Tonoike, M., Kida, M., Takagi, T., Sasaki, Y., Utsuro, T., \BBA\ Sato, S. \BBOP
  2005\BBCP.
\newblock \BBOQ Effect of Domain-Specific Corpus in Compositional Translation
  Estimation for Technical Terms\BBCQ\
\newblock In {\Bem Proc. 2nd IJCNLP, Companion Volume}, \mbox{\BPGS\ 116--121}.

\end{thebibliography}


\begin{biography}
\bioauthor{外池 昌嗣}
{2001年京都大学 工学部 情報学科卒業.
2003年同大学大学院情報学研究科修士課程 知能情報学専攻修了.
2007年同大学大学院情報学研究科博士後期課程修了予定.
自然言語処理の研究に従事.
}
\bioauthor{宇津呂武仁}
{
1989年京都大学工学部 電気工学第二学科 卒業.
1994年同大学大学院工学研究科 博士課程電気工学第二専攻 修了.
京都大学博士(工学).
奈良先端科学技術大学院大学情報科学研究科助手,
豊橋技術科学大学 工学部 情報工学系 講師,
京都大学 情報学研究科 知能情報学専攻 講師を経て,
2006年より
筑波大学 大学院システム情報工学研究科 知能機能システム専攻 助教授.
自然言語処理の研究に従事.
}
\bioauthor{佐藤 理史}{
1983年京都大学工学部電気工学第二学科卒業.1988年同大学院博士課程研究指
導認定退学.京都大学工学部助手,北陸先端科学技術大学院大学情報科学研究
科助教授,京都大学情報学研究科助教授を経て,2005年より名古屋大学大学院
工学研究科教授.工学博士.自然言語処理,情報の自動編集等の研究に従事.}

\end{biography}







\biodate

\end{document}
