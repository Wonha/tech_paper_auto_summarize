    \documentclass[japanese]{jnlp_1.3e}
\usepackage{jnlpbbl_1.1}
\usepackage[dvips]{graphicx}
\usepackage{fancybox}
\usepackage{amsmath}
\usepackage{multirow}


\Volume{14}
\Number{5}
\Month{October}
\Year{2007}

\received{2007}{4}{18}
\revised{2007}{5}{7}
\accepted{2007}{6}{27}

\setcounter{page}{199}

\jtitle{ゲーム理論に基づく参照結束性のモデル化と日本語・英語の\\
	大規模コーパスを用いた統計的検証}
\jauthor{白松  俊\affiref{KUI} \and 駒谷 和範\affiref{KUI} \and 
	橋田 浩一\affiref{AIST} \and 尾形 哲也\affiref{KUI} \and 奥乃  博\affiref{KUI}}
\jabstract{
参照結束性(referential coherence) は,主題の連続性や代名詞化によってもたらされる,文章の滑らかさを表す.
では,なぜ参照結束性が高い表現/解釈が選択されるのだろうか.
参照結束性の標準的理論であるセンタリング理論は,従来,この行動選択のメカニズムをモデル化していなかった.
本研究の目的は以下の2つである.
(1)この行動選択原理をゲーム理論でモデル化した仮説 \shortcite{hasida1996,siramatu2005nlp}を,
複数言語のコーパスで検証すること.
(2)ゲーム理論の期待効用という値が選択基準になり得るか確かめ,
様々な言語の談話処理で利用可能な表現/解釈の選択機構をモデル化すること.\\
そのために本稿では,
意味ゲームに基づくセンタリングモデル(MGCM; Meaning-Game-based Centering Model) を多言語化した.
具体的には,言語依存パラメタの定義を統計的にし,コーパスからのパラメタ獲得を可能にした.
本改良により多言語への適用が可能になったので,
日本語と英語のコーパスを用いてモデルを検証した.
その結果,参照結束性の高い繋がりを持つ発話では,
期待効用が高い表現/解釈が選択されているという統計的な証拠 
を得た.
これにより,MGCMの言語をまたぐ一般性,
すなわち,「参照結束性はゲーム理論の期待効用原理によって引き起こされている」という仮説の言語一般性が示された.
}
\jkeywords{談話解析, 参照結束性, センタリング理論, ゲーム理論, 期待効用, 顕現性, \\参照確率, 代名詞化, 知覚効用}

\etitle{A Game-Theoretic Model of Referential Coherence \\
	and Its Statistical Verification Based on Large \\
	Japanese and English Corpora}
\eauthor{Shun Shiramatsu\affiref{KUI} \and Kazunori Komatani\affiref{KUI} \and  K\^oiti Hasida\affiref{AIST} \and \\
	Tetsuya Ogata\affiref{KUI} \and Hiroshi G. Okuno\affiref{KUI}} 
\eabstract{
Referential coherence represents smoothness of discourse 
resulting from topic continuity and pronominalization. 
By what principle do we select coherent expressions and interpretations? 
Centering theory, the standard theory of referential coherence, 
has not modeled the mechanism for selection of coherent expressions and interpretations. 
Our goals are as follows: 
(1) We aim to verify the hypothesis that models 
the principle of selecting expressions and interpretations 
on the basis of game theory (Hasida et al. 1995; Shiramatsu et al. 2005), 
using corpora of multiple languages. 
(2) We aim to investigate 
whether we can use expected utility as selection criterion, 
and to develop the mechanism of selecting expressions and interpretations 
for discourse processing systems in various languages. \\
For these purposes, we improved the meaning-game-based centering model (MGCM). 
Our improvement, the statistical design of the language-dependent parameters, 
enabled to acquire the parameters from a corpus of the target language. 
It also enabled verification of MGCM using corpora of various languages. 
We verified MGCM using Japanese and English corpora. 
We found out statistical evidences which supported the hypothesis that 
referential coherence was caused by selection with higher expected utility. 
This result indicates language universality of MGCM and the hypothesis.
}
\ekeywords{Discourse analysis, Referential coherence, Centering theory, Game theory, Expected utility, Salience, Reference probability, Pronominalization, Perceptual utility}

\headauthor{白松,駒谷,橋田,尾形,奥乃}
\headtitle{ゲーム理論に基づく参照結束性のモデル化と日本語・英語の大規模コーパスを用いた統計的検証}

\affilabel{KUI}{京都大学大学院情報学研究科}{Graduate School of Informatics, Kyoto University}
\affilabel{AIST}{産業技術総合研究所情報技術研究部門}{ITRI, National Institute of Advanced Industrial Science and Technology (AIST)}


\makeatletter
\renewcommand{\paragraph}{}
\makeatother




\begin{document}
\maketitle



\section{はじめに}
\label{sec:intro}

\subsection{背景と動機}

参照結束性 (referential coherence) とは,
主題の連続性や代名詞化によってもたらされる,
談話の局所的な繋がりの滑らかさである.
本研究の目的は,
参照結束性を引き起こすメカニズムの定量的モデル化である.
この問題を扱う動機を以下に示す.
\begin{itemize}
\item[1.] {\bf 認知言語学的動機: } 談話参与者(発話者, 受話者; 筆者, 読者)が
高い参照結束性で繋がる表現/解釈を選択するのは,
どのような行動選択メカニズムによるものだろうか?~
参照結束性の標準的理論であるセンタリング理論\shortcite{grosz1983,grosz1995}は,
この行動選択メカニズムをモデル化していないという課題を残している.
上記の問いに対する仮説として,Hasida et al. \citeyear{hasida1995} は
ゲーム理論\shortcite{osbone1994,neumann1944} に基づく定式化を提案した.
この仮説を{\bf 意味ゲーム (Meaning Game)}と呼ぶ.
意味ゲーム仮説は日本語コーパスで検証された\shortcite{siramatu2005nlp}が,
日本語以外の言語では未検証である.
近年,語用論や談話現象などの言語現象を
ゲーム理論で説明しようとする研究が増えている \shortcite{parikh2001,rooij2004,benz2006} ことからも,
意味ゲーム仮説が
言語をまたぐ一般性を有するか否かを実データ上で検証することは重要な課題である.
\item[2.] {\bf 工学的動機: } 対話システムや自動要約処理では,
参照結束性が高い順序で発話や文を並べ,
理解しやすい談話構造を出力することが重要である.
そのためには,発話$U_i$までの先行文脈$[U_1,\cdots,U_i]$と
後続発話$U_{i+1}$との間の参照結束性の
モデル化が不可欠である.
工学的に,$U_{i+1}$の候補群から1つの候補を選択する基準として用いるためには,
参照結束性の高い候補を選択するためのメカニズムを定量的にモデル化し,
そのモデルによって参照結束性の高さを定量的な値として推定できることが望ましい.
つまり,本研究が目指す処理の出力は,
{\bf 先行文脈$[U_1,\cdots,U_i]$と
後続発話$U_{i+1}$との間の参照結束性を表す定量的な値}である.
これを,様々な言語の談話処理システムから利用可能にすることを目指す.
\end{itemize}



本研究が扱う参照結束性という談話現象は,
談話参与者の認知的な負荷削減と密接に関連する.
もし,談話参与者の負荷を削減しようとする発話行動が,
様々な言語で参照結束性を引き起こす基本原理となっているのならば,
その原理を定式化することで,
言語をまたぐ一般性を備えた参照結束性のモデルを構築できるはずである.
われわれは,意味ゲーム仮説に基づいてセンタリング理論を一般化するという
アプローチを踏襲することで,そのような言語一般性を備えたモデルを構築できると考える.
これにより,ゲーム理論に基づく定量的・体系的な
参照結束性の分析が,様々な言語で可能になると期待される.



\subsection{目的と課題}
本研究の目的は,(1)「参照結束性はゲーム理論の期待効用原理で説明できる」という仮説\shortcite{hasida1996,siramatu2005nlp} を,性質の異なる様々な言語の実データを用いて検証し,(2)それによって言語一般性を備えた参照結束性の定量的モデルを構築することである\footnote{
	本研究の目的は照応解析の精度向上ではない.また,機械学習を用いた照応解析研究\shortcite{ng2004,strube2003,iida2004}は,参照結束性を引き起こす行動選択メカニズムの解明を目指してはいないので,本研究とは目的が異なる.}.

\begin{figure}[b]
\begin{center}
      \includegraphics{14-5ia8f1.eps}
  \caption{2つの課題}
  \label{fig:issues}
\vspace{-\normalbaselineskip}
 \end{center}
\end{figure}


本研究の目的のために重要な2つの課題を,図\ref{fig:issues}および以下に示す.
\begin{itemize}
\item[1.] {\bf 言語独立な行動選択原理のモデル化: } 
談話参与者は,コミュニケーションを阻害しない程度に知覚的負荷の軽減が見込まれる表現と解釈を選択する.
この行動選択原理から,様々な言語の上での参照結束性のメカニズムを説明できるとわれわれは予想する.
この原理を意味ゲームのフレームワークに基づいて定式化することで,
参照結束性の選好を談話参与者の知覚的な因子
(談話中で参照される実体に向ける注意や,
参照表現を用いる際の知覚的なコスト)
からボトムアップにモデル化する.
そのモデルが参照結束性を説明できるか否かを,
様々な言語の実データで確かめる.
\item[2.] {\bf 言語依存な特性を統計的に獲得可能なモデル化: }
原理的には言語に独立な行動選択機構があるにしても,
表層表現の知覚においては各言語の特性に依存する言語的因子があると考えられる.
具体的には,談話参与者が実体に向ける注意の度合(顕現性; salience)や,
参照表現を用いる際の知覚的なコストは,
各言語に特有な言語表現に影響されるはずである.
言語毎の手作業の設計を避け,より精緻に当該言語に適応させるためには,
当該言語のコーパスからその特性を自動獲得可能なモデルが望ましい.
そのためには,顕現性や知覚コストを統計的に定式化する必要がある.
\end{itemize}




\subsection{アプローチ}
本稿では,上で述べた2つの課題に対して以下のアプローチをとる.

\begin{itemize}
\item[1.] {\bf 参照結束性を引き起こす行動選択原理の言語をまたぐ一般性を検証: } 
Hasida \citeyear{hasida1995}, 白松他 \citeyear{siramatu2005nlp} の仮説(ゲーム理論に基づく参照結束性の定式化)が,言語をまたぐ一般性を有するか否かを検証する.
具体的には,性質が異なる2つの言語である日本語と英語のコーパスを用いて検証実験を行う.
\item[2.] {\bf 言語依存な知覚的要因を表すパラメタを統計的に設計: } 
参照結束性に影響する知覚的な要因(実体の顕現性, 参照表現を使う際の知覚コスト)
は,言語毎に特有な表現に影響されると考えられるので,
これらをコーパスに基づく統計的なパラメタとして設計する.
具体的には,言語特有の表現に依存するパラメタ分布を,
日本語と英語のコーパスから獲得する実験を行う.
これにより,統計的定義の妥当性を検証する.
\end{itemize}

以下,\ref{sec:issues}章で従来研究の概要と問題点を説明し,
\ref{sec:ut}章では白松他\citeyear{siramatu2005nlp} のモデルを多言語に適用する際の問題点を解決する.
\ref{sec:verification}章では,性質が異なる大きく異なる2つの言語である日本語と英語のコーパスを用い,
モデルの言語一般性を検証する.
\ref{sec:discussion}章では,参照結束性の尺度としての期待効用の性質,および,
代名詞化の傾向に関する日本語と英語の違いを考察する.
最後に,\ref{sec:conc}章で結論を述べる.


\section{従来研究の概要と問題点}
\label{sec:issues}

以下では,まずセンタリング理論による非定量的なモデル化の概要と問題点を述べる.
次に,その定量的な一般化である白松他\citeyear{siramatu2005nlp}のモデルの概要と,
言語一般性に関する問題点,および,
多言語への適用のために解決する必要のある問題点を述べる.


\subsection{センタリング理論}

\paragraph{センタリング理論の概要}

センタリング理論\shortcite{grosz1983,grosz1995}は,参照結束性に関する標準的理論である.
センタリング理論では,局所的に共同注意の焦点になっている実体を
{\bf 中心 (center)} と呼ぶ.
実体に対する共同注意の度合,すなわち実体の目立ち具合を{\bf 顕現性 (salience)}と呼ぶ.
顕現性の順に文法役割を順序付けするヒューリスティクスは,{\bf Cf-ranking} と呼ばれる \shortcite{walker1994}.
Cf-rankingには言語によって違いがあり,日本語と英語では以下のように定義される.\\
\hspace{2mm} 日本語: 主題(助詞ハ)$\succ$主語(助詞ガ)$\succ$間接目的語(助詞ニ)$\succ$目的語(助詞ヲ)$\succ$その他 \\
\hspace{2mm} 英語: ~~~主語 $\succ$目的語 $\succ$間接目的語 $\succ$補語 $\succ$その他\\
ただし${\it gram}_1 \succ {\it gram}_2$ は,${\it gram}_1$ の顕現性が${\it gram}_2$ より高いことを表す.
談話[U$_1$, U$_2$, $\cdots$, U$_n$] における各発話単位$U_i$の中心は,
Cf-rankingの顕現性順序に基づき
以下の3つの制約によって定義される.
\begin{itemize}
    \item {\bf 制約1:} 
発話単位$U_i$ は1つの{\bf 後向き中心 (backward-looking center)} Cb($U_i$) を持つ.
    \item {\bf 制約2:} 
{\bf 前向き中心 (forward-looking centers)} Cf($U_i$) の全要素は$U_i$ で参照される.
    \item {\bf 制約3:} 
Cb($U_i$) は,$U_i$ で参照される実体のうちで
Cf($U_{i-1}$) における顕現性順序が最も高い.
\end{itemize}
なお,Cf($U_i$) の要素のうち最も顕現性順序が高い実体Cp($U_i$) は{\bf 優先中心 (preferred center)} と呼ばれ,
後続発話のCb$(U_{i+1})$の予測であると解釈される.
これらの制約に基づき,参照結束性は以下の2つのルールで形式化される.
\begin{itemize}
    \item {\bf ルール1:}  
Cf($U_i$) の要素のうち1つでも代名詞化されるなら,
Cb($U_i$) は代名詞化される.
    \item {\bf ルール2:}  
隣接する発話単位対では,以下のように中心の連続性が高い遷移ほど好まれる.\\
~~~~~~~{\bf Continue} (Cb($U_i$)$=$Cb($U_{i+1}$), Cb($U_{i+1}$)$=$Cp($U_{i+1}$))\\
~~~~$\succ$ {\bf Retain} ~~~~(Cb($U_i$)$=$Cb($U_{i+1}$), Cb($U_{i+1}$)$\neq$Cp($U_{i+1}$))\\ 
~~~~$\succ$ {\bf Smooth-Shift} (Cb($U_i$)$\neq$Cb($U_{i+1}$), Cb($U_{i+1}$)$=$Cp($U_{i+1}$))\\ 
~~~~$\succ$ {\bf Rough-Shift} ~~(Cb($U_i$)$\neq$Cb($U_{i+1}$), Cb($U_{i+1}$)$\neq$Cp($U_{i+1}$))
\end{itemize}
ただし${\it trans}_1 \succ {\it trans}_2$は,遷移タイプ${\it trans}_1$が${\it trans}_2$よりも好まれることを表す.
ルール1 は「後向き中心Cb は代名詞化されやすい」という傾向を意味し,
ルール2 は「中心は連続しやすい」という傾向を意味する.
図\ref{fig:coherence} に,ルール1, 2 に従う例,すなわち参照結束性の高い例と,ルール1, 2に従わない例,
すなわち参照結束性の低い例を示す.

\begin{figure}[t]
\begin{center}
      \includegraphics{14-5ia8f2.eps}
  \caption{参照結束性が高い繋がり,低い繋がりの例}
  \label{fig:coherence}
 \end{center}
\end{figure}


\paragraph{センタリング理論の問題点}
センタリング理論を構成する制約やルールについては,
研究者ごとに様々なバリエーションが提案されている \shortcite{poesio2004}.
それらの試みは理論の改良を意図したものだが,
「なぜルール1,2を満たす表現が選択されるのか」という行動選択原理の
形式化が不十分という点においては改善されていない.
このことから,原理を欠いたまま,
際限なくヒューリスティクスのバリエーションが
提案されていくという状況が危惧される.
これに対してわれわれは,参照結束性を引き起こす行動選択機構の,
よりシンプルな定量的モデルが必要であると考える.
言い換えると,センタリング理論は,
談話参与者の行動選択原理から参照結束性のメカニズムを説明するモデルによって
一般化されるべきである.
この一般化により,定量的な観点からの,より体系的な分析が可能になると期待される.

また,Cf-rankingは言語依存な因子の一例であるが,
これを各言語特有の文法役割の上で設定するには人手が必要である上,
その定義には統計的根拠を欠く.
たとえば,Walkerら\citeyear{walker1994}の日本語Cf-rankingにおける
目的語と間接目的語の順序は,根拠が明らかでない.





\subsection{意味ゲームに基づくセンタリングモデル}
\label{sec:mgcm}


\paragraph{意味ゲームの概要} 
{\bf 意味ゲーム (Meaning Game)} \shortcite{hasida1995,hasida1996} とは,自然言語による
意図的コミュニケーションを複数のプレイヤーによるゲームと見なし,
ゲーム理論\shortcite{neumann1944,osbone1994} によって数理的にモデル化する枠組である.
発話者を$S$, 受話者を$R$, 
意味内容の集合を$C$, 表現(メッセージ)の集合を$M$, 
発話者が意図した内容$c_S$, 発話した表現$m$, 受話者が解釈した内容$c_R$ の
三つ組$\langle c_S, m, c_R \rangle$の集合を$O=C \times M \times C$ とおき,
$S$と$R$が$o=\langle c_S, m, c_R \rangle$を選択する確率をPr$(o)$,
談話参与者$X$にとっての$o$の効用($o$が選ばれることによって$X$が得る利得)
をUt$(o, X)$とすると,
$X$にとっての{\bf 期待効用}(効用の期待値)は
$$\sum_{o \in O} {\rm Pr}(o) {\rm Ut} (X, o)$$
と表され,$X$は期待効用
が大きくなるような$o$を選択しやすい.
この行動選択原理は{\bf 期待効用原理}と呼ばれ,ゲーム理論や意思決定理論で一般的な仮説である.
また以下の仮定を置くと,期待効用を最大化する$o$ がPareto最適解\footnote{
	どのプレーヤについても
単独で戦略を変えることによって自分の利得が高くならないような
プレーヤ達の戦略の組合せを(Nash)均衡と言い,
全プレーヤにとってより望ましい均衡がないような均衡を
Pareto最適であると言う.}
となる.
\begin{itemize}
\item {\bf 確率Prに関する仮定: } ${\rm Pr}(o)$は,$S$と$R$の共有信念から定まる.また,${\rm Pr}(o)$の分布自体が$S$と$R$の共有信念に含まれる.
\item {\bf 効用Utに関する仮定: } ${\rm Ut} (S, o)$と${\rm Ut} (R, o)$の分布に相関がある.すなわち,${\rm Ut} (S, o_1)>{\rm Ut} (S, o_2)ならば{\rm Ut} (R, o_1)>{\rm Ut} (R, o_2)$が成り立つ.
\item {\bf コミュニケーション成立に関する仮定: } 常に$c_S = c_R$である.すなわち,意味内容の伝達が必ず成功する.
\end{itemize}
\noindent
Hasida \citeyear{hasida1996}は,この枠組に基づいて
$S$による照応詞選択,および,$R$による照応解消を定式化することにより,
センタリング理論のルール1が導けることを示した.

\paragraph{意味ゲームに基づくセンタリングモデルの概要} 
白松他 \citeyear{siramatu2005nlp}は,
ルール1の導出に関するHasida の議論を
以下のように定式化することでセンタリング理論を一般化し,
その仮説を日本語の大規模コーパスを用いて検証した.
ただし,以下では発話単位$U_i$までの先行文脈$[U_1,\cdots,U_i]$を${\rm pre}(U_i)$ と表記する.
\begin{itemize}
\item {\bf 参照確率Prに関する仮定: } 発話者$S$と受話者$R$が先行文脈${\rm pre}(U_i)$を共有し,
これを手がかりにして後続発話$U_{i+1}$で参照される実体$e_1, e_2, \cdots$を決定するとき,
後続発話$U_{i+1}$で実体$e$が参照される確率${\rm Pr}(e|{\rm pre}(U_i))$も,
$S$と$R$の共有信念に含まれる.この確率を{\bf 参照確率 (reference probability)} と呼び,実体$e$の顕現性を表す量と見なせる.センタリング理論のCf-rankingに対応する.
\item {\bf 知覚効用Utに関する仮定: } 後続発話$U_{i+1}$に含まれる参照表現$w$を$S$が言う(書く)知覚的なコストと,$R$が聞く(読む)知覚的なコストを考えると,$S, R$双方にとって代名詞の方が非代名詞よりも低コストなので,
$S$,$R$双方にとっての$w$の効用として共通の値Ut($w$)を用い,$w$が代名詞のとき${\rm Ut}(w)=2$, $w$が非代名詞のとき${\rm Ut}(w)=1$と仮定する.この効用を{\bf 知覚効用 (perceptual utility)} と呼び,参照表現$w$の簡略性を表す量と見なせる.センタリング理論のルール1における代名詞/非代名詞の区別に対応する.
\item {\bf コミュニケーション成立に関する仮定: } 後続発話$U_{i+1}$に含まれる参照表現$w_1, w_2, \cdots$と実体$e_1, e_2, \cdots$との参照関係$\langle w_1, e_1 \rangle, \langle w_2, e_2 \rangle, \cdots$について,$S$の意図と$R$の解釈とが必ず一致する.
\end{itemize}
\noindent
これら3つの仮定の下で,発話$U_{i+1}$の{\bf 期待効用}${\rm EU}(U_{i+1})$は以下のように表される\footnote{
	「$w_1$が$e_1$を参照する」という事象と「$w_2$が$e_2$を参照する」という事象は互いに排反ではないので,${\rm EU}(U_{i+1})$の定義は厳密な意味での期待効用とは異なる.しかし,「大きな期待効用を持つ表現/解釈が好まれる」という期待効用原理においては,
複数の候補を期待効用の大小に基づいて選択できることが重要である.
${\rm EU}(U_{i+1})$がこの性質を満たすことはHasida et al.\citeyear{hasida1995}に例示されているので,本稿では省略する.}.
$${\rm EU}(U_{i+1})=\hspace{-1mm}\sum_{w\mbox{ refers to }e\mbox{ in }U_{i+1}}\hspace{-4mm}{\rm Pr}(e|{\rm pre}(U_i)){\rm Ut}(w)$$


この期待効用${\rm EU}(U_{i+1})$を最大化するような参照関係$\langle w, e \rangle$から成る$U_{i+1}$がPareto最適解となるので,
$U_{i+1}$の表現/解釈の選択メカニズムは単純な期待効用原理に帰着できる.
その期待効用原理から,センタリング理論ルール1, 2 の一般化である以下の選好1a, 1b, 2が導かれる.
\begin{itemize}
\item {\bf 選好1a:} 1つの発話単位$U_{i+1}$が複数の参照表現を含む場合,
その中から${\rm Ut}(w_1)>{\rm Ut}(w_2)$であるような参照表現のペア$w_1, w_2$を選び,
それぞれ指示対象との組が$\langle w_1, e_1 \rangle, \langle w_2, e_2 \rangle$であるとき,
実体$e_1, e_2$の参照確率の大小関係は
${\rm Pr}(e_1|{\rm pre}(U_i))>{\rm Pr}(e_2|{\rm pre}(U_i))$になりやすい.
すなわち,知覚効用が高い参照表現$w_1$の方が,参照確率が高い実体$e_1$を参照しやすい(図\ref{fig:crossed_or_uncrossed}).
\item {\bf 選好1b:} 照応詞$w$の効用{\it Ut$(w)$}とその指示対象$e$の参照確率Pr$(e|{\rm pre}(U_i))$の間には正の相関がある.
\item {\bf 選好2: } より大きな期待効用EU($U_{i+1}$)を持つ$U_{i+1}$が選ばれやすい.
\end{itemize}

\begin{figure}[b]
\begin{center}
      \includegraphics{14-5ia8f3.eps}
  \caption{期待効用原理から導かれる選好1a, 1b}
  \label{fig:crossed_or_uncrossed}
 \end{center}
\vspace{-\normalbaselineskip}
\end{figure}



選好1aは,期待効用原理から導かれ,センタリング理論のルール1の一般化になっている.
図\ref{fig:crossed_or_uncrossed}に示すように,実体$e$をPrの降順に,参照表現$w$をUtの降順に並べたときに,
交差しない参照関係(A)に伴う期待効用${\rm EU_A}(U_{i+1})$の方が,
交差する(B)に伴う${\rm EU_B}(U_{i+1})$よりも常に大きくなる.
よって期待効用原理より,(A)の方が(B)よりも選ばれやすい.
このとき,実体$e$がCbならば参照確率が高くなり,
参照表現$w$が代名詞ならば知覚効用が高くなることから,
(A)はセンタリング理論のルール1を満たしており,よって選好1aはルール1の一般化であることがわかる.
ただし,(A)と(B)の期待効用の差${\rm EU_A}(U_{i+1})-{\rm EU_B}(U_{i+1})$が小さいと
選好1aは弱く働き,差が大きいと選好1aは強く働くと予想される.

選好1bは,選好1aから導かれ,選好1aの更なる一般化になっている.
図\ref{fig:crossed_or_uncrossed}の(A)が選ばれやすいならば,
高い参照確率Prを有する(目立っている)実体$e$は,
高い知覚効用Utを有する(簡略化された)参照表現 $w$によって
参照されやすいはずであり,${\rm Pr}(e|{\rm pre}(U_i))$と${\rm Ut}(w)$には正の相関があると予想される.


選好2は,期待効用原理そのものであると同時に,センタリング理論のルール2の一般化になっている.
センタリング理論のルール2における条件式Cb($U_i$)$=$Cb($U_{i+1}$)が成り立つとき,
Cbの参照確率が高くなると同時に,
選好1bの予測からCbを参照する照応詞の効用も高くなると考えられ,
したがって現在の発話の期待効用が増すからである.
また条件式Cb(U$_{i+1}$)$=$Cp(U$_{i+1})$が成り立つときも,
やはりCbの参照確率と効用が高くなり,
期待効用が高くなると考えられる.
さらに,RetainとSmooth-Shiftは共に一方の条件式のみが成り立つ遷移タイプであるが,
Cbは既に観測されたセンターであるのに対し,
Cpは次のセンターの予測に過ぎないので,
Cb($U_i$)$=$Cb($U_{i+1}$)が成り立つRetainの方が,
Cb(U$_{i+1}$)$=$Cp(U$_{i+1})$が成り立つSmooth-Shiftよりも
期待効用が大きくなると予想される.





この白松他\citeyear{siramatu2005nlp}のモデルを,本稿では{\bf MGCM (Meaning-Game-based Centering\linebreak Model)}と呼ぶ.
MGCM における期待効用を構成する2つのパラメタ,すなわち参照確率と知覚効用について,
以下に定義と算出方法を述べる.


    \noindent\hangafter=1\hangindent=1zw\textbf{参照確率 (Reference Probability) の定義}\\
 発話単位$U_i$までの先行文脈${\rm pre}(U_i)$における実体$e$の素性ベクトルが与えられた下で,実体$e$が後続する発話単位$U_{i+1}$で参照される条件付き確率Pr($e|{\rm pre}(U_i)$)で参照確率を定義する.この確率は,$U_i$における実体$e$の顕現性を表す.

    \noindent\hangafter=1\hangindent=1zw\textbf{参照確率の算出方法}\\
 実体$e$の顕現性は,先行文脈${\rm pre}(U_i)$において$e$がどのように参照されているかに影響される.
白松他 \citeyear{siramatu2005nlp} が用いた表\ref{tab:features}の3つの素性{\it dist, gram, chain}は,
顕現性に影響すると考えられる素性である.
{\it dist} は,$e$が最後に参照された発話単位から,発話単位$U_{i+1}$への距離を表す.
{\it gram} は,$e$の最後に参照された発話単位における文法役割を表す.
文法役割の種類ごとの実数値としては,
まずコーパス中の当該文法役割の名詞句を全て抽出し,
そのうち次の発話で参照されている割合Pr({\it gram})を用いる.
{\it chain} は,$U_i$までに$e$が参照された回数を表す.
{\it dist やchain }の設計で$\log$ を用いているのは,
ウェーバー・フェヒナーの法則\shortcite{fechner1860}として知られる
人間の知覚の対数関数的性質を反映させるためである.
本稿でも,これら3つの素性から成る素性ベクトルを用いて
参照確率を算出する.

\begin{table}[t]
\begin{center}
\caption{Pr$(e|{\rm pre}(U_i))$の計算で用いる素性}
\label{tab:features}
\input{08t01.txt}
\end{center}
\end{table}

図\ref{fig:pr}は,参照確率算出の基本的アイディアを表している.
コーパス中のサンプルが3次元の素性空間上で充分に密に分布しているような
理想的な場合に限り,図\ref{fig:pr}のような
単純な計算方法で参照確率を計算することができる.
以下では,まず最初に図\ref{fig:pr}の基本的アイディアを説明し,
次に現実的な算出手法であるロジスティック回帰分析を用いる計算手法を説明する.

\begin{figure}[b]
\begin{center}
      \includegraphics{14-5ia8f4.eps}
  \caption{Pr($e|{\rm pre}(U_i)$) 計算の基本的アイディア}
  \label{fig:pr}
 \end{center}
\end{figure}


\begin{itemize}
\item {\bf 基本的アイディア: }
仮に,図\ref{fig:pr}の左側の発話$U_i$における
``Tom''の参照確率\linebreak
Pr(``Tom''$|$pre($U_i$))を求めたいとする.
このときの,対象事例(``Tom'', pre($U_i$)) の素性は以下のようになる.
\begin{itemize}
\item[・] {\it dist=}$\log$(2+1)(``Tom'' が最後に参照された$U_{i-2}$から$U_{i+1}$までの間の発話数が2)
\item[・] {\it gram=}Pr(subject)(``Tom'' が最後に参照された$U_{i-2}$での``Tom''の文法役割がsubject)
\item[・] {\it chain}=$\log$(2+1)(``Tom'' が$U_i$までに参照された回数が2)
\end{itemize}
この素性ベクトル[{\it dist=}$\log$(2+1), {\it gram=}Pr(subject), {\it chain}=$\log$(2+1)] と同じ素性ベクトルを持つサンプル($e$, pre($U_j$)) の集合を,コーパスから抽出する.
そのうち,$e$が$U_{j+1}$で参照されているサンプルの割合が,
対象事例における``Tom''が$U_{i+1}$で参照される確率の近似値となる.

\item {\bf ロジスティック回帰による参照確率の算出: }
実際には対象事例と全く同じ素性ベクトルを持つサンプルが
コーパス中に充分にあるとは限らない.
よって,何らかの内挿・外挿法を用いてデータスパースネスに対応する必要がある.
そのため本稿では白松他\citeyear{siramatu2005nlp} と同様に,ロジスティック回帰
によって参照確率を算出する.
具体的には,サンプル($e$, pre($U_j$)) の素性{\it dist, gram, chain} を説明変数として用い,$e$が$U_{j+1}$で参照されるか否か(参照されるとき1, 参照されないとき0)を被説明変数として用いる.
ロジスティック回帰では,被説明変数が1になる確率${\rm Pr}(e|{\rm pre}(U_j))$のロジット$\log({\rm Pr}/(1-{\rm Pr}))$を,以下のような説明変数の線形結合で表す.
$$\log \frac{{\rm Pr}(e|{\rm pre}(U_j))}{1-{\rm Pr}(e|{\rm pre}(U_j))}=b_0 + b_1 {\it dist} + b_2 {\it gram} + b_3 {\it chain}$$

ロジスティック回帰モデルの事前学習では,
コーパス中のサンプル($e$, pre($U_j$))で観測された説明変数・被説明変数の値を学習データとして与え,
最尤法によって回帰重み$b_0, b_1, b_2, b_3$ を求める.
また,事前学習したロジスティック回帰モデルを用いて新たな事例
($e$, pre($U_i$)) の参照確率${\rm Pr}(e|{\rm pre}(U_i))$ を計算するには,
素性{\it dist, gram, chain} の観測値を以下の回帰式に代入すればよい.
$${\rm Pr}(e|{\rm pre}(U_i))=(1+{\rm exp}(-(b_0+b_1{\it dist}+b_2{\it gram}+b_3{\it chain})))^{-1}$$
\end{itemize}
    \par\noindent\hangafter=1\hangindent=1zw\textbf{知覚効用 (Perceptual Utility) の定義}\\
 参照表現$w$が発話者から受話者へ伝達される際の知覚的な負荷低減の度合 Ut($w$)を知覚効用として定義する.
ただし,照応解消などの意味的解釈に伴う認知的負荷は含まず,表層的な記号伝達処理
(発話, 聞き取り; 筆記, 読み取り)に伴う
知覚的な負荷のみを対象とする.
    \par\noindent\hangafter=1\hangindent=1zw\textbf{知覚効用の算出方法}\\
 白松他\citeyear{siramatu2005nlp}は,
$w$が代名詞のとき${\rm Ut}(w)=2$, $w$が非代名詞のとき${\rm Ut}(w)=1$という先験的な仮定を置いているので,
このままでは知覚効用をコーパスから測定することはできない.


\noindent
認知的負荷を排除して知覚的負荷のみを用いて効用を設計する意図は,図\ref{fig:issues} に示したように,
表層の知覚(実体の顕現性と名詞句使用の知覚的コスト)からボトムアップに
決定される行動選択原理(期待効用EU($U_{i+1}$)に基づく行動選択)により,
発話者の照応詞選択および受話者の照応解消の過程をモデル化するためである\footnote{
	なお,センタリング理論やMGCM が表現する参照結束性の選好は,
照応処理過程において述語項構造や参照表現自身の選択制限によって上書きされる.
よって,参照結束性の選好だけを用いて照応解析しようとするアプローチは現実的でなく,
それは本研究の目的(参照結束性の定量的モデル化)の範疇外である.}.



表\ref{tab:corresp}は,参照確率,知覚効用,期待効用がそれぞれ
センタリング理論のどのルールに対応しているのかを示している.
参照確率は顕現性の尺度なので,センタリング理論におけるCf-ranking に対応する.
知覚効用は参照表現伝達時の負荷低減の尺度なので,センタリング理論のルール1における代名詞化に対応する.
期待効用は参照結束性の尺度なので,センタリング理論のルール2における遷移のランキングに対応する.

\begin{table}[t]
\begin{center}
\caption{センタリング理論とMGCM の対応}
\label{tab:corresp}
\input{08t02.txt}
\end{center}
\end{table}


MGCMにおける期待効用は,
$U_{i+1}$が発話者$S$から受話者$R$へ伝達される際に,
「その両者にとって,参照関係の処理 
(発話者による出力と受話者による予測)にかかる負荷削減がどれだけ見込めるか」
という期待値を表す.
例えば,顕現性の高い(参照確率の高い)実体が,
後続発話単位において知覚コストの低い(知覚効用の高い)表現によって
参照される場合,その後続発話単位の期待効用は高くなる.
つまり,そのような後続発話は認知的に低負荷であると期待されるので,
選択されやすい\footnote{
ただし,意味ゲームではコミュニケーションの成立(意味内容伝達の成功)を仮定しているので,
「伝えるべき意味内容があるのに何も発話しない」という行動は選択されない.}.

言い換えると,期待効用が大きくなるような$U_{i+1}$を選択することによって,
${\rm pre}(U_i)$と$U_{i+1}$の間の高い参照結束性が実現されていると予想される.
また対偶をとって,${\rm pre}(U_i)$ と$U_{i+1}$の間の参照結束性が低い場合には,
期待効用が小さな$U_{i+1}$が選択されてしまっていると予想される.
この期待効用原理は理論的には各言語に独立なので,
様々な言語において参照結束性が高い談話構造が好まれる理由を説明できる
と予想される.
また工学的には,期待効用${\rm EU}(U_{i+1})$ は
${\rm pre}(U_i)$と$U_{i+1}$の間の参照結束性を表す尺度として,
様々な言語の上での談話処理において,$U_{i+1}$の選択基準として使えると期待される.

\paragraph{意味ゲームに基づくセンタリングモデルの問題点}
従来のMGCM \shortcite{siramatu2005nlp} には,以下の課題が残されている.
\begin{itemize}
\item[1.] {\bf 行動選択原理の言語一般性が未検証: }
行動選択のメカニズムとしての期待効用原理は
言語一般に関して成り立つと予想されるが,
従来,MGCMは日本語のコーパスでしか検証されていなかった.
「期待効用原理が,参照結束性の背後にある言語非依存な原理である」という仮説を
実際のデータで裏付けるために,
複数の言語のコーパスで
検証する必要がある.
\item[2.] {\bf 先験的な知覚効用の設計: } 
従来のMGCMでは,参照確率$w$の知覚効用Ut($w$)は
単純かつ先験的な仮定に基づいて定義されていた
(Ut(代名詞)=2, Ut(非代名詞)=1).
このような先験的な設計では,言語毎の参照表現の違いに対応できない.
また,
代名詞/非代名詞という参照表現の分類は粗すぎると考えられる.
各言語に適応させるためにも,より細かい粒度に対応可能な統計的定義が必要である.
\end{itemize}
\noindent
上記2における代名詞/非代名詞という分類の粗さの問題を,
以下の例で説明する.

\newenvironment{fminipage}{}{}

    \vspace{1\baselineskip}\begin{fminipage}{39zw}
\noindent
自民党は十一日,次期衆院選に向けた選挙対策本部(本部長・\underline{河野洋平総裁})の会合を開いた.\\
\underline{河野氏}は「新制度の下で戦い,政権奪還を目指すのが基本的な考え」と強調.\\
しかし,$\left\{ 
\begin{array}{lc}
{\rm (a)}&(\underline{φ}ハ)\\
{\rm (b)}&\underline{彼}は\\
{\rm (c)}&\underline{河野氏}は\\
{\rm (d)}&\underline{河野洋平総裁}は\\
\end{array}
\right \}
$「いつ行われても不思議ではない衆院選」とも述べ,...\\
    \end{fminipage}\vspace{\baselineskip}

下線部は全て「河野洋平総裁」を参照している.
(a)〜(d)は参照表現を変化させた例であり,原文では(a)のゼロ代名詞が使用されている.
従来のMGCMの知覚効用の定義や,センタリング理論のルール1においては,
(a),(b)の区別と,(c),(d) の区別が無かった.
しかし本来ならば,(a)が最も知覚効用が高い(知覚コストが低い) 参照表現であり,
(b), (c), (d) と知覚効用が低くなると考えられる.
モデルをより現実に近づけるには,
より詳細な知覚効用の定義が必要である.












\section{改良MGCM: 知覚効用の統計的な定義}
\label{sec:ut}

以下では,知覚効用を様々な言語のコーパスから統計的に獲得できるように定義する.
この新たな定義を用いたMGCMを,本稿では{\bf 改良MGCM}と呼ぶ.

\subsection{知覚効用とは}
参照表現の知覚効用とは,前述したように,その表現を発話者・受話者間で伝達する際の
処理(発話, 聞き取り; 筆記, 読み取り)にかかる知覚的負荷の低さを表す.
つまり,慣れなどの要因によって少ない知覚コストで処理できる参照表現ほど,
高い知覚効用を持つ.
ただし,ここでの知覚コストとは表層的なシンボルの伝達にかかる負荷のみを指し,
照応解消などの意味理解に伴う負荷は含まない.
何故なら,言語特有の表現に依存する知覚的処理を2つのパラメタ(参照確率と知覚効用)
に割り当て,そこからボトムアップに構成される期待効用原理から,
言語独立な表現/解釈の選択メカニズムをモデル化することを意図しているからである
(\ref{sec:intro}章の図\ref{fig:issues}参照).
この知覚効用を,白松ら \citeyear{siramatu2005nlp} は「代名詞の効用は非代名詞の効用よりも高い」という単純な仮定のみによって定義していた.


\subsection{知覚コストの統計的定義}
われわれは,参照表現の知覚コストは
その表現の頻度に基づいて計算できると考える.
何故なら,頻繁に使用される参照表現ほど,
発話者・受話者はその表層的伝達処理に慣れてゆくため,
知覚コストが低くなるためである.
また,発話者・受話者が一旦その参照表現に慣れてしまうと,
その表現の使用が知覚コストの削減に繋がるので,
更に頻繁に使用されるようになる,とも考えられる.
よって参照表現$w$ の知覚コストは,
コーパスの各発話単位で$w$ が照応詞として使用される確率$p(w)$ に
基づいて定義できる.
具体的には,「感覚量は物理量の対数に比例する」という
ウェーバー・フェヒナーの法則\shortcite{fechner1860}に従い,
以下のように$p(w)$の対数を用いて$w$の知覚コストI($w$)を定義する.
\begin{align*}
{\rm I}(w)&= - \log p(w)\\
&= - \log \frac{\# 照応詞としての w }{\# \rm 全発話単位}~[{\rm nat}].
\end{align*}

つまり,知覚コストI($w$) を,当該発話単位で$w$ が使用されるという事象の自己情報量として定義する.
ただし$[{\rm nat}]$は自然対数を用いたときの情報量の単位であり,$1~[{\rm nat}]$ = $\log_2 e~[{\rm bit}]$ である.



\subsection{知覚効用の統計的定義}
\label{subsec:utdef}
$w$ の知覚効用${\rm Ut}(w)$を,知覚コスト${\rm I}(w)$の削減の度合として定義する.知覚効用${\rm Ut}(w)$は,非負の値をとる尺度として設計する必要がある.
その理由を以下に示す.

MGCMでは,期待効用EU($U_{i+1}$)を参照結束性の尺度として設計する.
参照結束性が強くなるのは,
参照確率(顕現性)の高い実体$e$が,
知覚効用の高い(知覚コストの低い代名詞などの)参照表現$w$によって参照されるときである.
期待効用EU($U_{i+1}$)は,参照確率${\rm Pr}(e|{\rm pre}(U_i))$と知覚効用${\rm Ut}(w)$の積和なので,
EU($U_{i+1}$)を参照結束性の尺度として設計するためには,
\begin{align*}
&{\rm Pr}(e_1|{\rm pre}(U_i))>{\rm Pr}(e_2|{\rm pre}(U_i)) , {\rm Ut}(w_1)>{\rm Ut}(w_2)\\
&\Rightarrow {\rm Pr}(e_1|{\rm pre}(U_i)){\rm Ut}(w_1)>{\rm Pr}(e_2|{\rm pre}(U_i)){\rm Ut}(w_2)
\end{align*}
という条件が常に満たされるように${\rm Ut}(w)$を設計する必要がある.

知覚効用を非負の尺度として設計した場合,
${\rm Pr}(e_1|{\rm pre}(U_i))>{\rm Pr}(e_2|{\rm pre}(U_i))\ge 0 , {\rm Ut}(w_1)>{\rm Ut}(w_2)\ge 0$のとき,
常に${\rm Pr}(e_1|{\rm pre}(U_i)){\rm Ut}(w_1)>{\rm Pr}(e_2|{\rm pre}(U_i)){\rm Ut}(w_2)$が成り立つ.

一方,知覚効用を負の値として設計した場合,
${\rm Pr}(e_1|{\rm pre}(U_i))>{\rm Pr}(e_2|{\rm pre}(U_i))\ge 0 , 0>{\rm Ut}(w_1)>{\rm Ut}(w_2)$のとき,
${\rm Pr}(e_1|{\rm pre}(U_i)){\rm Ut}(w_1)>{\rm Pr}(e_2|{\rm pre}(U_i)){\rm Ut}(w_2)$が成り立つとは限らない.

以上の考察より,知覚効用${\rm Ut}(w)$を非負の値をとる尺度として設計する.
具体的には,
知覚コスト${\rm I}(w)$を逆転させた非負の尺度として以下のように定義する.
\[
 {\rm Ut}(w) = {\it Ut}_0 - {\rm I}(w).
\]
基準値${\it Ut}_0$ は,${\rm Ut}(w)\ge 0$ を保証するための${\rm max}~{\rm I}(w)$以上の定数である.
上記の議論より,この条件 ${\it Ut}_0 \ge {\rm max}~{\rm I}(w)$ さえ満たされていれば
EU($U_{i+1}$)は参照結束性を表す尺度になると予想される.
ただし,この条件は選好2 とセンタリング理論のルール2 との整合性を保つためには必要だが,
選好1a, 1b には影響しない.
何故なら,選好2 とルール2 が整合するのは
条件${\it Ut}_0 \ge {\rm max}~{\rm I}(w)$
が成立するときのみであると(上記の議論から)予想されるのに対し,
選好1a, 1b の成立/不成立を決める参照表現ペア$w_1, w_2$の知覚効用の差 
${\rm Ut}(w_1)-{\rm Ut}(w_2) = {\rm I}(w_2) - {\rm I}(w_1)$ は,
${\it Ut}_0$ の値に依存しないからである.
よって,${\it Ut}_0$の値は,
選好2 とルール2 の整合性を基準として決定すればよい.

具体的に本稿では,
センタリング理論のルール2における遷移タイプ順序 (Continue $\succ$ Retain $\succ$ Smooth-Shift $\succ$ Rough-Shift)
と期待効用EU($U_{i+1}$)とのスピアマン順位相関係数が最大になるように,
コーパスから基準値${\it Ut}_0$を決定する.
このように決定した${\it Ut}_0 $の値は,
${\it Ut}_0 \ge {\rm max}~{\rm I}(w)$を満たしていると予想される.
また,仮に${\it Ut}_0 \ge {\rm max}~{\rm I}(w)$を満たす範囲で${\it Ut}_0$の値を動かしたとしても,
選好2とルール2の整合性は保たれると予想される.


以上の統計的な定式化によって,言語特有の表現に依存した知覚効用の分布を,
対象言語のコーパスから獲得可能にした.



\section{日本語・英語の大規模コーパスによる検証}
\label{sec:verification}

\paragraph{検証実験の位置づけ}
改良MGCMにおける各言語の特性の統計的獲得と,期待効用原理に基づく行動選択機構の言語一般性を,
大きく性格の異なる2つの言語,日本語と英語のコーパスで検証する.
日本語コーパスとして毎日新聞の記事を,英語コーパスとしてWall Street Journal (WSJ) の記事を用いる.
表\ref{tab:verification}に,本節で行う検証実験の位置づけを示す.
特に新規性のある実験は,英語コーパス上での検証実験と,統計的に定義された知覚効用${\rm Ut}(w)$の獲得に関する検証実験である.

\begin{table}[b]
\begin{center}
\caption{本稿の改良MGCM検証実験の位置づけ}
\label{tab:verification}
\input{08t03.txt}
\end{center}
\end{table}



\paragraph{前提となるコーパスの仕様}
毎日新聞コーパスは1,356記事,63,562述語節,16,728照応詞から成る.
WSJコーパス は 2,412記事,135,278述語節,95,677照応詞から成る.
どちらのコーパスにも,形態素,品詞,係り受け構造,照応の情報を表す
GDA (Global Document Annotation) \shortcite{hasida1998gda} のタグが付与されている.
形態素,品詞,係り受け構造に関しては,
自動的な解析結果を人手で修正したタグが付与されている.
照応に関しては,完全に人手によるタグが付与されている.
以下は毎日新聞の記事に対するアノテーション例である.
ただし見やすくするため,形態素・品詞・細かい係り受けを表すタグは省いてある.
\vspace{1\baselineskip}

{\small\tt
\noindent
~<su syn="f">\\
~~~<adp opr="topic.fit.agt">自民党は</adp>\\
~~~<adp>十一日</adp>,\\
~~~<adp opr="obj">\\
~~~~~<np>\\
~~~~~~~<adp>\\
~~~~~~~~~<n>次期衆院選に向けた選挙対策本部</n>\\
~~~~~~~~~<np>(本部長・<np \underline{id="KonoYohei"}>河野洋平総裁</np>)</np>の
	\hfill …\textcircled{\small 1}     \\
~~~~~~~</adp>\\
~~~~~~~<n>会合</n>\\
~~~~~</np>を</adp>\\
~~~<v>開い</v>た.</su>\\
\\
~<su syn="f">\\
~~~<adp opr="topic.fit.agt"><np \underline{eq="KonoYohei"}>河野氏</np>は</adp>
	\hfill …\textcircled{\small 2}     \\
~~~<adp opr="cnt">「新制度の下で戦い,政権奪還を目指すのが基本的な考え」と</adp>\\
~~~<v>強調</v>.</su>\\
\\
~<su syn="f">\\
~~~<adp>しかし</adp>,\\
~~~<vp>\\
~~~~~<adp opr="cnt">「いつ行われても不思議ではない衆院選」とも</adp>\\
~~~~~<v \underline{agt="KonoYohei"}>述べ</v>,
	\hfill …\textcircled{\small 3}     \\
~~~</vp>\\
~~~<adp opr="obj">現行制度での準備も怠らない構えを</adp>\\
~~~<v \underline{agt="KonoYohei"}>示し</v>た.</su>
	\hfill …\textcircled{\small 4}     
}
\vspace{1\baselineskip}


上記の例の照応タグについて説明する.
まず,先行詞のエレメント\textcircled{\small 1}には,id 属性
({\tt id={\linebreak}"KonoYohei"}) が付与されている.
省略以外の照応詞を表すために,照応詞のエレメント\textcircled{\small 2}に
eq 属性 ({\tt eq="KonoYohei"}) が付与されている.
ゼロ代名詞などの省略を表すために,省略を受けるエレメント\textcircled{\small 3},\textcircled{\small 4}に
省略の格を表す関係属性({\tt agt="KonoYohei"}, この場合は主格)が付与されている.


Wall Street Journal の記事に対しても,同様の照応タグが付与されている.
\vspace{\baselineskip}

{\small\tt
\noindent
~<su syn="b">\\
~~~<q opr="cnt">``We have no useful information on whether users are at risk,'' </q>\\
~~~<v>\\
~~~~~<v>said</v>\\
~~~~~<np opr="agt">\\
~~~~~~~<persnamep \underline{id="DrTalcott"}>James A. Talcott </persnamep>\\
~~~~~~~<adp>of Boston's Dana-Farber Cancer Institute</adp>\\
~~~~~</np></v>.</su>\\
\\ 
~<su syn="b">\\
~~~<persnamep \underline{eq="DrTalcott"} opr="agt">Dr. Talcott </persnamep>\\
~~~<v>\\
~~~~~<v>led </v>\\
~~~~~<np opr="obj">\\
~~~~~~~<n>a team of researchers </n>\\
~~~~~~~<adp>from the National Cancer Institute</adp>\\
~~~~~</np></v>.</su>
}
\vspace{\baselineskip}


\subsection{検証実験の準備}
検証実験の準備として,サンプルの抽出と,参照確率のロジスティック回帰モデル(2.2節 ロジスティック回帰式の回帰重み$b_0, b_1, b_2, b_3$)の事前学習を行う.
ロジスティック回帰モデルの事前学習には,統計ソフトウェアR \shortcite{Rsite,funao2005}を用いる.
このとき,上記の人手で付与された照応タグを手がかりとして用いる.
ただしKameyama\citeyear{kameyama1998}に従い,時制節・述語節を発話単位と見なす.\\


\paragraph{サンプルの抽出}
まず,発話単位列$[U_1,\cdots,U_n]$の各$U_i$に対し,
先行文脈${\rm pre}(U_i)=$\linebreak$[U_1,\cdots,U_i]$で参照されている実体を全て抽出する.
このとき,実体$e$と先行文脈${\rm pre}(U_i)$の組,すなわち
($e, {\rm pre}(U_i)$)を1サンプルと見なす.

サンプル($e, {\rm pre}(U_i)$)において
実体$e$が発話$U_{i+1}$で参照されている場合,
そのサンプルをここでは「正例」と呼ぶ.
$e$が$U_{i+1}$で参照されていない場合,
そのサンプルをここでは「負例」と呼ぶ.
表\ref{tab:samples}に,毎日新聞とWSJから抽出されたサンプル数を示す.

参照確率のロジスティック回帰分析のためには,
正例と負例を用いる.
後述する選好1a, 1b, 2 の検証には,正例のみを用いる.

\begin{table}[b]
\caption{抽出されたサンプル数}
\label{tab:samples}
\input{08t04.txt}
\end{table}



\paragraph{参照パターンの抽出}
参照確率のロジスティック回帰のため,各サンプル($e$, ${\rm pre}(U_i)$)の参照パターンを抽出する.
本研究では,表\ref{tab:features} に示した素性(特徴量)によって参照パターンを表す.





\subsection{参照確率の言語依存特性の獲得実験}
\label{sec:verify_salience}

\paragraph{回帰モデルの事前学習}
参照確率Prの計算のためには,2つの準備が必要である.
まず,\ref{sec:mgcm}節で示した
参照確率のロジスティック回帰式
$${\rm Pr}(e|{\rm pre}(U_i))=(1+{\rm exp}(-(b_0+b_1{\it dist}+b_2{\it gram}+b_3{\it chain})))^{-1}$$
における{\it gram}に割り当てる値として,
文法役割ごとの参照確率の平均値Pr({\it gram})を求める.
Pr({\it gram})は,ロジスティック回帰は使わずに
コーパス中のサンプルを数えるだけで計算できる.
具体的には,まずコーパス中の当該文法役割の名詞句を全て抽出し,
そのうち次の発話で参照されている割合をPr({\it gram})とする.
回帰モデルを用いる必要がないのは,各言語でよく使用される文法役割は
十数種類に限定できるので,データスパースネスが生じないためである.
表\ref{tab:sal_jpn}, \ref{tab:sal_eng} にその結果を示す.

\begin{table}[b]
\caption{文法役割ごとの参照確率の平均値 Pr({\it gram})(毎日新聞)}
\label{tab:sal_jpn}
\input{08t05.txt}
\end{table}

\begin{table}[t]
\caption{文法役割ごとの参照確率の平均値 Pr({\it gram}) (WSJ)}
\label{tab:sal_eng}
\input{08t06.txt}
\end{table}


次に,ロジスティック回帰分析における重み$b_i$ を
コーパスから獲得する.
処理時間が膨大になるのを避けるため,表\ref{tab:samples}の全サンプルを使うのではなく,
12,000サンプルをサブサンプリングして回帰重みを事前学習した.
その結果を,表\ref{tab:weights_jpn}, \ref{tab:weights_eng} に示す.

\begin{table}[t]
\begin{minipage}[t]{.5\textwidth}
\caption{ロジスティック回帰分析の回帰重み(毎日新聞)}
\label{tab:weights_jpn}
\input{08t07.txt}
\end{minipage}
\hfill
\begin{minipage}[t]{.47\textwidth}
\caption{ロジスティック回帰分析の回帰重み (WSJ)}
\label{tab:weights_eng}
\input{08t08.txt}
\end{minipage}
\end{table}


以下,表\ref{tab:sal_jpn}, \ref{tab:sal_eng}, \ref{tab:weights_jpn}, \ref{tab:weights_eng}が示す事前学習結果の妥当性を検証する.


\paragraph{言語依存性の確認}
日本語と英語には,それぞれに特有の文法役割がある.
このことは表\ref{tab:sal_jpn}, \ref{tab:sal_eng}にも明らかであり,
たとえば日本語における主題(係助詞の「ハ」)のような文法役割は,英語には無い.
よって必然的に,日本語と英語では文法役割毎の参照確率の分布も異なるはずである.
実際に表\ref{tab:sal_jpn}と表\ref{tab:sal_eng}では,
参照確率の分布(文法役割間の比)における各言語の特性が観察された.
このことは,参照確率の測定における言語依存特性の獲得の必要性を示している.

\paragraph{英語コーパスから獲得した参照確率の検証}
表\ref{tab:sal_eng} の文法役割順序は,
従来のセンタリング理論のCf-ranking との整合性を示している.
すなわちWSJコーパスにおける主語,目的語,補語の間の順序が,
センタリング理論におけるCf-rankingと一致した.
この結果は,参照確率による顕現性定義の,英語における妥当性を示している.
よって,表\ref{tab:sal_eng}の値を回帰分析に用いることで,
文法役割における顕現性の英語特性を獲得できると考えられる.

次に,図\ref{fig:dist_chain_refpr_wsj}に,WSJコーパスから得られた
ロジスティック回帰モデルを用いた参照確率の推定値を示す.
文法役割{\it gram}を主語に固定し,距離{\it dist} と参照回数{\it chain} を
変動させた場合の参照確率の推定値をプロットした.
また,表\ref{tab:weights_eng} は,WSJコーパスから
得られたロジスティック回帰モデルの回帰重みを表している.

表\ref{tab:weights_eng}の,素性{\it dist}(最近参照された箇所からの距離)の回帰重み$b_1$ は負の値であった.
これは,「最近参照された実体ほど注意が向かいやすい」という知見と整合し,
図\ref{fig:dist_chain_refpr_wsj}にもその傾向が現れている.
素性{\it gram}(最近参照された箇所の文法役割)の回帰重み$b_2$ は正の値であった.
これは,上述したCf-rankingとの整合性からしても妥当である.
素性{\it chain}(共参照連鎖の長さ)の回帰重み$b_3$ は正の値であった.
これは,「多く参照された実体ほど注意が向かいやすい」という知見と整合し,
図\ref{fig:dist_chain_refpr_wsj}にもその傾向が現れている.
これらの結果は,
WSJコーパスから統計的に獲得された英語の参照確率分布の妥当性を示している.

\begin{figure}[t]
\begin{center}
      \includegraphics{14-5ia8f5.eps}
  \caption{{\it dist}, {\it chain}による参照確率の推定結果(毎日新聞)}
  \label{fig:dist_chain_refpr_mainiti}
 \end{center}
\end{figure}
\begin{figure}[t]
\begin{center}
      \includegraphics{14-5ia8f6.eps}
  \caption{{\it dist}, {\it chain}による参照確率の推定結果 (WSJ)}
  \label{fig:dist_chain_refpr_wsj}
 \end{center}
\end{figure}




\paragraph{日本語コーパスから獲得した参照確率の検証}
白松ら\citeyear{siramatu2005nlp} の結果と同様に,
表\ref{tab:sal_jpn} の文法役割順序はセンタリング理論におけるCf-ranking と一致した.
また,表\ref{tab:weights_jpn}に示す回帰重みと,
図\ref{fig:dist_chain_refpr_mainiti}に示す参照確率の推定結果も,
顕現性に関する従来の知見と整合していた.
これらの結果は,毎日新聞コーパスから統計的に獲得された日本語の参照確率分布の妥当性を示している.







\subsection{知覚効用の言語依存特性の獲得実験}
\label{sec:verify_cost}


\paragraph{基準値${\it Ut}_0$のコーパスに基づく決定} 
\ref{subsec:utdef}節で述べた議論から,
知覚効用${\rm Ut}(w)={\it Ut}_0 - {\rm I}(w) $
は非負の値として設計する必要がある.
そのためには,
条件${\it Ut}_0 \ge {\rm max}~{\rm I}(w)$ を満たすように
基準値${\it Ut}_0$ を設定する必要があるが,
この条件さえ満たされていれば,
期待効用EU($U_{i+1}$)は参照結束性を表す尺度となると予想される.


本稿では,センタリング理論との整合性が最大になるように基準値${\it Ut}_0$を定める.
\ref{subsec:utdef}節で述べたように,
${\it Ut}_0$の値に依存するのは
改良MGCMの選好2 とセンタリング理論のルール2 との整合性であり,
選好1a, 1b は ${\it Ut}_0$の値に影響されない.
よって,選好2とルール2の整合性,すなわち
期待効用EU($U_{i+1}$)とルール2の遷移タイプ順序
(Continue $\succ$ Retain $\succ$ Smooth-Shift $\succ$ Rough-Shift)
とのスピアマン順位相関係数が最大になるように,
基準値${\it Ut}_0$を定める.

\begin{figure}[b]
\begin{center}
      \includegraphics{14-5ia8f7.eps}
  \caption{知覚効用Utの基準値${\it Ut}_0$のコーパスに基づく決定}
  \label{fig:imax}
 \end{center}
\end{figure}



図\ref{fig:imax}は,基準値${\it Ut}_0$を変動させて
EU($U_{i+1}$)と遷移タイプ順序のスピアマン相関係数を計測した結果である.
毎日新聞コーパスでは${\it Ut}_0=15.1$, WSJコーパスでは${\it Ut}_0=12.6$のとき,
それぞれ最大のスピアマン順位相関係数を観測した.
よって,ここでは基準値${\it Ut}_0$をこれらの値に定める.

これらの値は\ref{subsec:utdef} 節の予想どおり,
条件 ${\it Ut}_0 \ge {\rm max}~{\rm I}(w)$ 
(毎日新聞では${\rm max}~{\rm I}(w)=11.06$  , WSJでは${\rm max}~{\rm I}(w)=11.82$)を満たしている.
また,図\ref{fig:imax}の${\it Ut}_0 \ge {\rm max}~{\rm I}(w)$ 
の領域においては,スピアマン相関係数はほぼ平坦になっている.
この観測結果は,「条件${\it Ut}_0 \ge {\rm max}~{\rm I}(w)$さえ満たされていれば
期待効用EU($U_{i+1}$)が参照結束性を表す尺度になるはずである」という
\ref{subsec:utdef} 節の予想と合致する.


\begin{table}[b]
\caption{参照表現ごとの知覚効用(毎日新聞)}
\label{tab:cost_jpn}
\input{08t09.txt}
\end{table}
\begin{table}[t]
\caption{参照表現ごとの知覚効用 (WSJ)}
\label{tab:cost_eng}
\input{08t10.txt}
\end{table}


\paragraph{知覚効用の事前測定}
毎日新聞コーパスとWSJコーパスに出現する全ての参照表現$w$について,
その知覚効用を計測した.
ただし複数の形態素から成る参照表現については,
その主辞である形態素の出現確率を用いて
知覚効用を計算した.
表\ref{tab:cost_jpn},\ref{tab:cost_eng}は,それぞれ
毎日新聞とWSJにおいて知覚効用が上位ランクであった参照表現を示している.
ただし,日本語のゼロ代名詞,英語の空範疇 (empty category) はともに省略された文法要素を指す用語である.
空範疇は生成文法の用語であり,
NP痕跡, pro(定形節の音形を持たない主語代名詞), PRO(不定形節や動名詞の音形を持たない主語代名詞), wh痕跡の
4つのタイプがあるとされる\shortcite{jeita2005}.

以下,表\ref{tab:cost_jpn},\ref{tab:cost_eng}が示す知覚効用の測定結果の妥当性を検証する.



\paragraph{言語依存性の確認}
日本語と英語には,それぞれに特有の参照表現がある.
このことは,表\ref{tab:cost_jpn}と表\ref{tab:cost_eng}にも明らかである.
よって必然的に,参照表現毎の知覚効用の分布も異なるはずである.
実際に表\ref{tab:cost_jpn}と表\ref{tab:cost_eng} では,
知覚効用の分布(参照表現間の比)における各言語の特性が観察された.
このことは,知覚効用の測定における言語依存特性の獲得の必要性を示唆している.



\paragraph{日本語コーパスから獲得した知覚効用の検証}
表\ref{tab:cost_jpn}が示すように,毎日新聞コーパスではゼロ代名詞(省略)の知覚コストが最も低く,
したがって知覚効用が高かった.
また,「私」「その」などの指示詞・代名詞がランク上位を占め,
大部分の非代名詞は指示詞・代名詞よりも低い知覚効用であった.
表\ref{tab:cost_jpn}の下部3行が示すとおり,ゼロ代名詞,指示詞・代名詞,非代名詞というカテゴリ毎の知覚効用の平均は,
\[
 \text{ゼロ代名詞} > \text{指示詞・代名詞} > \text{非代名詞}
\]
となった.この順序は,直感的な負荷低減の順序と一致しており,妥当である.
つまり,毎日新聞コーパスで計測された日本語の知覚効用の妥当性を示している.


\paragraph{英語コーパスから獲得した知覚効用の検証}
表\ref{tab:cost_eng}が示すように,WSJコーパスでは空範疇(省略)の知覚コストが最も低く,したがって知覚効用が高かった.
また,``it'', ``he'' などの代名詞がランク上位を占め,
大部分の非代名詞は代名詞よりも低い知覚効用であった.
表\ref{tab:cost_eng}の下部3行が示すとおり,空範疇,代名詞,非代名詞というカテゴリ毎の知覚効用の平均は,
\[
 \text{空範疇} > \text{代名詞} > \text{非代名詞}
\]
となった.この順序は,直感的な負荷低減の順序と一致しており,妥当である.
つまり,WSJコーパスで計測された英語の知覚効用の妥当性を示している.

\vspace{\baselineskip}
以上により,毎日新聞,WSJの両コーパスから,言語特有の表現に依存した知覚効用(知覚コスト低減)の分布を,
統計的に獲得できることが示された.



\paragraph{知覚効用の統計的定義の効果}
従来の単純かつ先験的な知覚効用の定義
(Ut(代名詞)=2, Ut(非代名詞)=1)は,
参照表現の分類が粗過ぎることが問題であった.
例えば\ref{sec:mgcm}節で示した例文では,
(a)ゼロ代名詞(φ)と(b)代名詞「彼」,(c)「河野氏」と(d)「河野洋平総裁」の
区別ができなかった.
表\ref{tab:ut_example}は,改良MGCMにおける知覚効用の定義に基づき
毎日新聞コーパスから統計的に獲得した知覚効用Ut$(w)$の値である.
これにより,(a)と(b),(c)と(d)の知覚効用の違いが扱えることが示された.

\begin{table}[t]
\caption{従来の知覚効用の定義との比較}
\label{tab:ut_example}
\input{08t11.txt}
\end{table}



\subsection{改良MGCMの検証}
\label{sec:verify_coherence}
MGCM は,センタリング理論のルール1, 2 の一般化である選好1a, 1b, 2 を含んでいる. 
以下では,期待効用原理から導出された選好1a, 1b, 2 を日本語と英語のコーパスを用いて統計的に検証することにより,
期待効用原理が言語をまたいで成り立つ原理であるかを検証する.
各選好の検証では,
正規分布が仮定できないパラメタ同士の相関や,
センタリング理論における選好順序との相関をとるのに適している,スピアマン順位相関係数を用いる.
何故なら,スピアマン順位相関係数は分布を仮定しないノンパラメトリックな推定により求まるからである.
また参考のために,一般的に用いられるピアソン積率相関係数も観測する.




\paragraph{選好1aの検証} 
選好1aはセンタリング理論のルール1の一般化であり,期待効用原理から導かれる.
図\ref{fig:crossed_or_uncrossed}において,
常に${\rm EU_A}(U_{i+1})-{\rm EU_B}(U_{i+1})>0$が成り立つので,
(A)の方が(B)より選ばれやすい,というのが選好1aである.
ただし,(A)の選ばれやすさは
${\rm EU_A}(U_{i+1})-{\rm EU_B}(U_{i+1})$の大きさに影響されるはずである.
つまり,期待効用原理から以下の予想が導かれる.

    \vspace{\baselineskip}{\setlength{\leftskip}{1zw}\noindent
 選好1aの(A)と(B)の期待効用の差
${\rm EU_A}(U_{i+1})-{\rm EU_B}(U_{i+1})$が大きい参照表現ペアの場合,
(A)が好まれるという選好が強く働き,選好1a合致率が100\%に近づくと予想される.
一方,(A)と(B)の期待効用の差がほどんど無い参照表現ペアの場合は, 
(A), (B)どちらの候補を選んでも参照結束性に差が無いので,
(A)が好まれるという選好が弱くしか働かず,選好1a合致率が50\%に近づくと予想される.
    \par}\vspace{\baselineskip}

\noindent
以下では,この予想を裏付ける観測結果が得られるかどうかにより,
選好1aを検証する.

まず,検証の準備として,
図\ref{fig:crossed_or_uncrossed}における($w_1$, $w_2$) のような同一発話単位内の参照表現ペアをコーパスから抽出する.
次に,
コーパスから抽出した参照表現ペアから
(A)と(B)の期待効用の差${\rm EU_A}(U_{i+1})-{\rm EU_B}(U_{i+1})$を計算し,
その差の大きさと,実際の選好1a合致率((A)が選ばれている率)の関係を観測する.

\begin{figure}[b]
\begin{center}
      \includegraphics{14-5ia8f8.eps}
  \caption{選好1a: ${\rm EU_A}(U_{i+1})-{\rm EU_B}(U_{i+1})$と選好適合率(毎日新聞)}
  \label{fig:pref1a_eu_mainiti}
 \end{center}
\end{figure}


図\ref{fig:pref1a_eu_mainiti}, \ref{fig:pref1a_eu_wsj} に,
実際の観測結果を示す.
(A)と(B)の期待効用の差が3以上の参照表現ペアのみに限定すると,
選好1a合致率は毎日新聞で0.825, WSJで0.822となり,(A)が選ばれる選好が強い.
それに対し,(A)と(B)の期待効用の差が0.5未満の参照表現ペアのみに限定すると,
選好1aの合致率は
毎日新聞で0.564, WSJで0.529となり,(A)が選ばれる選好は非常に弱くなる.
この結果は,期待効用原理から導かれた上記の予想と合致する.
また,表\ref{tab:pref1a_mainiti}, \ref{tab:pref1a_wsj}は,
(A)と(B)の期待効用の差と,選好1a合致率との相関係数を計算したものである.
スピアマン順位相関係数の観測結果では,
毎日新聞では0.833, WSJでは0.981という強い相関を示している.
これは,「参照結束性は期待効用原理によって引き起こされる」
という仮説の妥当性を強く支持する結果である.



\begin{figure}[t]
\begin{center}
      \includegraphics{14-5ia8f9.eps}
  \caption{選好1a: ${\rm EU_A}(U_{i+1})-{\rm EU_B}(U_{i+1})$と選好適合率 (WSJ)}
  \label{fig:pref1a_eu_wsj}
 \end{center}
\end{figure}
\begin{table}[t]
\caption{${\rm EU_A}(U_{i+1})-{\rm EU_B}(U_{i+1})$と選好1a合致率との相関(毎日新聞)}
\label{tab:pref1a_mainiti}
\input{08t12.txt}
\end{table}
\begin{table}[t]
\caption{${\rm EU_A}(U_{i+1})-{\rm EU_B}(U_{i+1})$と選好1a合致率との相関 (WSJ)}
\label{tab:pref1a_wsj}
\input{08t13.txt}
\end{table}


\paragraph{選好1bの検証}

選好1bは,期待効用原理から導かれた選好1aの更なる一般化である.
もし選好1aの予想どおり,図\ref{fig:crossed_or_uncrossed}の(A)の方が(B)より
選ばれやすいのならば,
{\rm Pr$(e|{\rm pre}(U_i))$}と{\rm Ut$(w)$}に正の相関があるはずである.
これが選好1bであり,定性的には
顕現性の高い実体$e$ほど,知覚的に簡単な参照表現$w$で参照されやすいという傾向を表す.

ここで,{\rm Pr$(e|{\rm pre}(U_i))$}と{\rm Ut$(w)$}のスピアマン順位相関係数を
\begin{itemize}
\item 選好1aの(A)が選ばれた参照表現ペアに限定して計測した相関係数$\rho_A$
\item (B)が選ばれた参照表現ペアに限定して計測した相関係数$\rho_B$
\item 全てのサンプルで計測した相関係数$\rho$
\end{itemize}
の3つに分けて計測すると,図\ref{fig:crossed_or_uncrossed}より,明らかに$\rho_A$は正,$\rho_B$は負になるはずである.

このとき,もしコーパス中のデータで選好1bが成り立つのならば,
$$|\rho_A| > |\rho_B|,~\rho > 0$$
が観測されると予想される.一方,もし選好1bが成り立たないのならば,
$$|\rho_A| \simeq |\rho_B|,~\rho \simeq 0$$
が観測されると予想される.
以下では,選好1bが成り立つ場合の予想に合致する観測結果が得られるか否かにより,
選好1bを検証する.

\begin{table}[b]
\caption{選好1b: ${\rm Pr}(e|{\rm pre}(U_i))$と${\rm Ut}(w)$の相関(毎日新聞)}
\label{tab:pref1b_mainiti}
\input{08t14.txt}
\end{table}
\begin{table}[b]
\caption{選好1b: ${\rm Pr}(e|{\rm pre}(U_i))$と${\rm Ut}(w)$の相関 (WSJ)}
\label{tab:pref1b_wsj}
\input{08t15.txt}
\end{table}

表\ref{tab:pref1b_mainiti}, \ref{tab:pref1b_wsj}は,
実際に$\rho_A, \rho_B, \rho$を観測した結果である.
毎日新聞での観測結果は,$\rho_A = 0.540$, $\rho_B = -0.086$, $\rho = 0.377$ であった.
WSJでの観測結果は,$\rho_A = 0.454, \rho_B = -0.120, \rho = 0.237$ であった.
この結果は,コーパス中のデータで選好1bが成り立たない場合の予想とは合致せず,
選好1bが成り立つ場合の予想と合致していた.
また,表の95\%信頼区間が示すとおり,この値は統計的に有意である.
よって,この結果は選好1bの妥当性を裏付けるものである.

次に,従来のMGCMにおける単純な知覚効用(Ut(代名詞)=2, Ut(非代名詞)=1)との比較のため,
従来の単純なUtとPrとのスピアマン順位相関係数を計測した結果,毎日新聞で0.358, WSJで0.236であった.
一方,本稿で統計的に設計されたUtの場合,Prとのスピアマン順位相関係数が毎日新聞で0.377, WSJで0.237であった(表\ref{tab:pref1b_mainiti}, \ref{tab:pref1b_wsj}).
よって,本稿で示した知覚効用の統計的設計を用いることにより,
「顕現性の高い実体$e$ほど,知覚的に簡単な参照表現$w$で参照されやすい」という傾向を,
人手による先験的な設計とほぼ同程度以上に表現できることが確認できた.

なお,順位相関係数$\rho_A, \rho_B, \rho$の値の意味については,\ref{sec:scale_validity}節で考察する.
また,毎日新聞コーパスの方がWSJコーパスよりも相関係数が大きかった理由については,\ref{sec:pron_ratio}節で考察する.



\paragraph{選好2の検証}
センタリング理論のルール2は,4つの遷移タイプの選好順序で表される.
ここでは,ルール2の選好順序に対する選好2(EU$(U_{i+1})$が高い$U_{i+1}$が好まれる)の整合性を検証した.
その検定の手順を以下に示す.
\begin{itemize}
\item[1.] センタリング理論のCb($U_i$), Cb($U_{i+1}$), Cp($U_{i+1}$)を決定する.Cb, Cpの定義については2.1節を参照のこと.ただし,以下のようにCf-ranking の代わりに参照確率を用いる.\\
Cb($U_i$): $\bigcup_{k=1}^i {\rm Cf}(U_k)$の要素のうち,Pr($e|{\rm pre}(U_{i-1})$)が最も大きい実体\\
Cb($U_{i+1}$): $\bigcup_{k=1}^{i+1} {\rm Cf}(U_k)$の要素のうち,Pr($e|{\rm pre}(U_i)$)が最も大きい実体\\
Cp($U_{i+1}$): $\bigcup_{k=1}^{i+1} {\rm Cf}(U_k)$の要素のうち,Pr($e|{\rm pre}(U_{i+1})$)が最も大きい実体
\item[2.] 1 に基づいて,センタリング理論のルール2と同じ方法で遷移タイプ (Continue, Retain, Smooth-Shift, Rough-Shift) を決定する.遷移タイプの定義については2.1節を参照のこと.
\item[3.] 2 で決定した4 つの遷移タイプごとに期待効用EU($U_{i+1}$) の平均を計算する.
\item[4.] Wilcoxon の順序和検定により,
「ルール2の遷移タイプ間のEU($U_{i+1}$)平均の順序が,
ルール2の順序 (Continue $\succ$ Retain $\succ$ Smooth-Shift $\succ$ Rough-Shift) と一致する」
という整合性の統計的有意性を検証する.
これにより,MGCMの選好2とセンタリング理論のルール2の整合性を検証する.
\item[5.] 4つの遷移タイプに,ルール2の順序を表す値(Continue: 4, Retain: 3, Smooth-Shift: 2, Rough-Shift: 1) を
割り当てた上で,EU($U_{i+1}$) との相関係数を計算する.
この相関係数が高いほど,MGCMの選好2とセンタリング理論のルール2とが整合すると言える.
\end{itemize}





表\ref{tab:tran_jpn}, \ref{tab:tran_eng} と
図\ref{fig:tran_eu_mainiti}, \ref{fig:tran_eu_wsj} に,
手順3で計算した4つの遷移タイプごとの期待効用EU($U_{i+1}$)の平均を示す.
また図\ref{fig:tran_eu_mainiti}, \ref{fig:tran_eu_wsj}には,
EU($U_{i+1}$)の平均だけでなく分布も示してある.




このEU($U_{i+1}$)平均値の順序は,毎日新聞,WSJの両コーパスにおいて,
ルール2の遷移タイプ順序と一致した\footnote{
	なお,サンプル数がContinueとSmooth-Shiftに偏っているのは,
各サンプルにおいて4種類の遷移がすべて選択可能とは
限らないためである\shortcite{kibble2001,siramatu2005nlp}.
他のセンタリング理論の研究においても,ルール2の遷移タイプ順序とサンプル数の多さの順序は一致しない\shortcite{iida1996,takei2000}}.
また,手順4 のWilcoxonの順序和検定の結果,
有意水準$2.2\times 10^{-16}$未満で遷移タイプ順序とEU($U_{i+1}$)の平均順序は整合していた.
更に,手順5 での遷移タイプ順序と期待効用EU($U_{i+1}$)との相関係数の測定結果(95\%信頼区間)を表\ref{tab:pref2_mainiti}, \ref{tab:pref2_wsj}に示す.
これにより,毎日新聞で0.639, WSJで0.482 のスピアマン順位相関係数が観測された.
表の95\%信頼区間が示すように,両コーパスで観測された正の相関は統計的に有意である.

\begin{table}[t]
\caption{選好2: 遷移タイプごとの期待効用EU($U_{i+1}$)の平均(毎日新聞)}
\label{tab:tran_jpn}
\input{08t16.txt}
\end{table}
\begin{table}[t]
\caption{選好2: 遷移タイプごとの期待効用EU($U_{i+1}$)の平均 (WSJ)}
\label{tab:tran_eng}
\input{08t17.txt}
\end{table}
\begin{figure}[t]
\begin{center}
      \includegraphics{14-5ia8f10.eps}
  \caption{選好2: 遷移タイプと${\rm EU}(U_{i+1})$の分布(毎日新聞)}
  \label{fig:tran_eu_mainiti}
 \end{center}
\end{figure}
\begin{figure}[t]
\begin{center}
      \includegraphics{14-5ia8f11.eps}
  \caption{選好2: 遷移タイプと${\rm EU}(U_{i+1})$の分布 (WSJ)}
  \label{fig:tran_eu_wsj}
 \end{center}
\end{figure}


また,従来のMGCMにおける先験的な知覚効用の設計(Ut(代名詞)=2, Ut(非代名詞)=1)と
本稿における統計的な設計を比較するために,
従来の知覚効用を用いてスピアマン順位相関係数を測定し,
毎日新聞で0.640, WSJで0.479という結果を得た.
この値は,本稿の知覚効用を用いた場合の値とほぼ等しい.
よって,
統計的設計によって獲得した知覚効用を用いた場合に,
参照結束性の尺度としての期待効用が,
従来の人手による設計とほぼ同程度の表現能力を持つことが確認できた.
つまり,参照結束性の尺度としてのEU($U_{i+1}$)の妥当性が,
日本語・英語の両コーパスにおいて示された.


\begin{table}[t]
\caption{選好2: 遷移タイプ順序と期待効用${\rm EU}(U_{i+1})$との相関係数(毎日新聞)}
\label{tab:pref2_mainiti}
\input{08t18.txt}
\end{table}
\begin{table}[t]
\caption{選好2: 遷移タイプ順序と期待効用${\rm EU}(U_{i+1})$との相関係数 (WSJ)}
\label{tab:pref2_wsj}
\input{08t19.txt}
\end{table}


以上より,両コーパスにおいて,改良MGCMの選好2と,センタリング理論のルール2との整合性を確認できた.
これは,期待効用EU($U_{i+1}$)に基づく表現/解釈の選択原理が,
両言語で参照結束性を引き起こす行動選択原理となっていることを示唆する.











\section{考察}
\label{sec:discussion}

\subsection{参照結束性の尺度としての期待効用}
\label{sec:scale_validity}

本節では,期待効用が
参照結束性を表す尺度としての性質を満たすか否か,および,
期待効用原理が後続発話$U_{i+1}$の表現/解釈の選択基準になっているか
否かを考察する.


\paragraph{選好1aの検証結果からの考察} 
選好1aは,期待効用原理から導かれる参照結束性の選好である.
\ref{sec:verify_coherence}節の図\ref{fig:pref1a_eu_mainiti}, \ref{fig:pref1a_eu_wsj}では,「図\ref{fig:crossed_or_uncrossed}の(A)と(B)の期待効用の差が大きいほど,(A)を選ぶ選考が強くなる」という,期待効用原理から導かれた予想を裏付ける検証結果が得られた.
具体的には,${\rm EU_A}(U_{i+1})-{\rm EU_B}(U_{i+1})$と選好1a合致率の
スピアマン順位相関係数が,毎日新聞で0.833, WSJで0.981と,非常に強い相関を示した.
${\rm EU_A}(U_{i+1})-{\rm EU_B}(U_{i+1})\ge 3$の場合,
選好1a合致率が毎日新聞で0.825, WSJで0.822となり,
${\rm EU_A}(U_{i+1})-{\rm EU_B}(U_{i+1})< 0.5$の場合,
選好1a合致率が毎日新聞で0.564, WSJで0.529となっていた.
この結果は,期待効用の差が$U_{i+1}$の選択基準としての性質を持ち,
参照結束性の差を表すという仮説と整合する.



ここで,参照表現ペア全体で選好1aの合致率を計算すると,毎日新聞で0.622, WSJで0.605であった.
つまり,(B)が選ばれている事例も全体の40\%ほど存在した.
これは(A)と(B)の期待効用の差がわずかな参照表現ペア\footnote{
	参照表現ペア全体のうち,${\rm EU_A}(U_{i+1})-{\rm EU_B}(U_{i+1})< 0.5$のサンプルの割合は,毎日新聞で65.0\%,WSJで45.8\%である.}
もあわせて計算しているためであり,
期待効用の差が0.5以上に限ると合致率が毎日新聞で0.729, WSJで0.670 となり, (B)が選ばれる事例が約30\%に減少した.
また2以上に限ると合致率が毎日新聞で0.808, WSJで0.772 となり, (B)が選ばれる事例が約20\%に減少した.

(A)と(B)の期待効用の差がわずかな参照表現ペアでは,(A),(B)どちらを選んでも,
見込める知覚コスト削減量に差が無いため,
選好1aが弱くなると考えられる.
白松他\citeyear{siramatu2005nlp}は,述語項構造の選択制限によって
選好1aが上書きされる事例を示したが,
\ref{sec:verify_coherence}節の検証結果を踏まえると,(A)と(B)の期待効用の差がわずかな場合ほど
高頻度で選択制限による上書きが起こっていると考えられる.
この考察は,期待効用の差がわずかな候補間の選択においては,
期待効用原理よりも選択制限の方が行動選択原理として強いことを示唆するが,
期待効用の差が参照結束性の差を表すという仮説とは矛盾しない.

では,(A)と(B)の期待効用の差が大きいにもかかわらず(B)が選択されているのは,どのような事例であろうか.
われわれは,段落の変わり目や話題の転換点においては,(B)のような期待効用が低い
(参照結束性が低い)遷移が起こりやすいと予想する.
何故なら話題の転換点においては,顕現性が高い実体ではなく,新たな実体がトピックとして参照される可能性が高いからである.
そのように文脈が切り替わる箇所では,
文脈を利用した知覚コストの削減ができないため,参照結束性が低くなると予想される.


以下に示す毎日新聞コーパスの投書記事における$U_{i+1}$は,
(A)と(B)の期待効用の差が大きいにもかかわらず,
(B)が選択されていた事例である.

    \vspace{\baselineskip}\begin{fminipage}{39zw}
\noindent
$U_{i-5}:$ 私のクラスに,足に障害のある生徒がいます.\\
$U_{i-4}:$(φガ)学校を一日も休むことなく,\\
$U_{i-3}:$(φガ)勉学に励んでいます.\\
$U_{i-2}:$ 彼を送り迎えするのは,お母さんですが,\\
$U_{i-1}:$(φガ)最近体の調子を崩し,\\
$U_{i\phantom{-0}}:$ 彼はタクシーで帰宅することが多いのです.\\
$U_{i+1}:$ 先日,(φガ)彼の家の最寄り駅まで送迎しました.
    \end{fminipage}\vspace{\baselineskip}

\noindent
この例では,$U_{i+1}$に含まれる参照表現ペア $(w_1={\bf φ}, w_2={\bf 彼})$ と,その指示対象候補$(e_1={\bf 生徒}, e_2={\bf 私})$に対し,
${\rm Pr}({\bf 生徒}|{\rm pre}(U_i)) > {\rm Pr}({\bf 私}|{\rm pre}(U_i))$, ${\rm Ut}({\bf φ})>{\rm Ut}({\bf 彼})$が観測され,(A)・(B)の期待効用の差は4.31と大きかった.
それにもかかわらず,(A)ではなく(B),すなわち$\langle w_1={\bf φ},e_2={\bf 私} \rangle$, $\langle w_2={\bf 彼},e_1={\bf 生徒} \rangle$ の参照関係が選択されていた.

実はこの記事では$U_i$と$U_{i+1}$の間で段落が変わっているので,
ここが話題の転換点であると見なせる.
「日本語では一人称がゼロ代名詞φで参照されることが多い」という性質の影響も無視できないが,
そうだとしても,話題の転換点で
期待効用の低い(参照結束性の低い)表現が選択された事例である.
つまり,「(A)と(B)の期待効用の差が大きいにもかかわらず(B)が選ばれているのは
話題の転換点であろう」という予想に合致する事例である.

この考察は,期待効用が参照結束性の尺度であるという仮説とは矛盾しない.
しかし,話題の転換点や段落の変わり目においては,
期待効用原理(および,参照結束性の選好)は表現/解釈の選択基準になっていないということを示唆している.
このことから,工学的に,
期待効用が小さい隣接発話の間を話題の転換点と見なせる可能性がある.




\paragraph{選好1bの検証結果からの考察} 
選好1bは,期待効用原理から導かれる選好1aの更なる一般化であり,
定性的には「目立っている実体は低コストな参照表現で参照できる」
という,文脈を利用した知覚コスト低減の傾向を表す.
\ref{sec:verify_coherence}節の表\ref{tab:pref1b_mainiti}, \ref{tab:pref1b_wsj}が示す
選好1bの検証結果によると,
(A)が選ばれた参照表現ペアに限定した場合の${\rm Pr}(e|{\rm pre}(U_i))$と${\rm Ut}(w)$のスピアマン相関係数$\rho_A$は,
毎日新聞で0.540, WSJで0.454と,中程度の正の相関を示した.
(B)に限定した場合の$\rho_B$は,毎日新聞で$-$0.086, WSJで$-$0.120と,ほぼ無相関と言えるほど弱い負の相関を示した.
全ての照応詞における$\rho$は,毎日新聞で0.377, WSJで0.237と,弱い正の相関を示した.
この,$|\rho_A| > |\rho_B|,~\rho > 0$ という結果は,
「目立っている実体は低コストな参照表現で参照できる」という,
文脈を利用した知覚コスト低減傾向を表現できていると考えられる.
また,
期待効用原理(選好1a)に従って(A)を選択した場合,文脈を利用したコスト低減傾向が強く($\rho_A>0$),
期待効用原理に従わず(B)を選択した場合は,文脈を利用したコスト低減傾向が無かった ($\rho_B \le 0$)
とも解釈できる.
この解釈は,期待効用が参照結束性を表す尺度であるという仮説と矛盾しない.
\\

ここで,全体の正の相関$\rho$が弱かった原因と,
期待効用原理に従わない場合のわずかな負の相関$\rho_B$の原因について考察する.
これも,
表現/解釈の候補間で期待効用の差がわずかな場合に,
選好1bが述語項構造の選択制限などによって上書きされた影響と考えられる.
これにより,$\rho_B$のわずかな負の相関が生まれ,
全体での$\rho$の正の相関が弱まったと考えられる.
しかし,期待効用原理に従った場合の正の相関$\rho_A$(毎日新聞で0.540, WSJで0.454)に比べると,
期待効用原理に従わない場合の負の相関$\rho_B$(毎日新聞で$-$0.086, WSJで$-$0.120)の絶対値は非常に小さなものであった.
この結果は,選好1bの妥当性を示唆していると考える.


\paragraph{選好2の検証結果からの考察}
選好2は,期待効用原理そのものである.
\ref{sec:verify_coherence}節では,
センタリング理論のルール2との整合性を検証するため,
ルール2の遷移タイプ順序 
(Continue $\succ$ Retain $\succ$ Smooth-Shift $\succ$ Rough-Shift) 
と期待効用${\rm EU}(U_{i+1})$とのスピアマン順位相関計数を求めた.
その結果,毎日新聞で0.639, WSJで0.482の正の相関を観測した.
この観測結果は,期待効用が参照結束性の尺度であるという仮説を
裏付けるものであると考える.

ただし\ref{sec:verify_coherence}節の図\ref{fig:tran_eu_mainiti}, \ref{fig:tran_eu_wsj}
によると,
確かに遷移タイプ毎の${\rm EU}(U_{i+1})$平均の順序は
ルール2における遷移タイプ順序と合致するが,
${\rm EU}(U_{i+1})$が0に近い領域にも,
Continueが分布することが見てとれる.
これは,ContinueのサンプルとRough-Shiftのサンプルの間で
${\rm EU}(U_{i+1})$の大小関係が逆転するケースも0ではない,
ということを意味する.
このことと,選好1aの検証結果を併せて考察すると,
以下の予想が導かれる.

    \vspace{1\baselineskip}{\setlength{\leftskip}{1zw}\noindent
ある先行文脈${\rm pre}(U_{i+1})$を有する
ひとつの発話$U_{i+1}$における表現/解釈の候補同士を比べる場合には,
期待効用${\rm EU}(U_{i+1})$は${\rm pre}(U_i)$と$U_{i+1}$の間の
参照結束性の尺度として有効である.
しかし,先行文脈が異なる発話同士を比べる場合は,
文脈に応じて期待効用${\rm EU}(U_{i+1})$に何らかの正規化を加えた方が,
より参照結束性の尺度に適した値になる可能性がある.
    \par}\vspace{1\baselineskip}
文脈に応じてどのような正規化を期待効用の値に加えるべきかは,
今後の課題とする.
ただし本稿で目指すのは,
定まった先行文脈を有する
ひとつの後続発話$U_{i+1}$における表現/解釈の候補同士からの選択である.
その目的のためには正規化は必要なく,
本稿の期待効用は参照結束性の尺度としての性質を充分に備えていると考える.






\subsection{文脈を利用した知覚コスト低減傾向に関する日本語と英語の比較}
\label{sec:pron_ratio}
選好1b の検証の結果,
参照確率が高い実体ほど,知覚効用が高い表現(代名詞など)で
参照されやすいという正の相関が確認された.
言い換えると,「顕現性が高く目立っている実体ほどコストが低い表現で参照できる」という相関であり,
文脈を利用した知覚コストの低減傾向を表していると考えられる.
具体的には,毎日新聞では0.377,
WSJでは0.237 のスピアマン順位相関係数(表\ref{tab:pref1b_mainiti}, \ref{tab:pref1b_wsj})が観測され,
毎日新聞の方がWSJよりも文脈を利用したコスト低減傾向が高かった.


その理由は,図\ref{fig:pron_ratio}に示されている.
図\ref{fig:pron_ratio}は,参照確率の変化により,代名詞化される実体の割合がどう変化するかを
プロットしたグラフである.
これによると,顕現性が低い範囲(${\rm Pr}<0.75$)では毎日新聞とWSJの違いは顕著ではない.
この範囲では,参照確率が上がるにつれて代名詞化率も上がっていたことが
両コーパスで確認できる.
一方,顕現性が高い範囲(${\rm Pr}\ge 0.75$) においては,
両コーパスの代名詞化傾向に大きな違いがあることがわかる.
毎日新聞では参照確率が上がるにつれて代名詞化率が上がっていたが,
WSJ では代名詞化率は横ばいとなっていた.



\begin{figure}[t]
 \begin{center}
      \includegraphics{14-5ia8f12.eps}
  \caption{参照確率による代名詞化率の変化}
  \label{fig:pron_ratio}
 \end{center}
\end{figure}
\begin{figure}[t]
\begin{center}
      \includegraphics{14-5ia8f13.eps}
  \caption{参照確率と知覚効用の散布図(毎日新聞)}
  \label{fig:salutil_scatter_mainiti}
\end{center}
\end{figure}
\begin{figure}[t]
\begin{center}
      \includegraphics{14-5ia8f14.eps}
  \caption{参照確率と知覚効用の散布図 (WSJ)}
  \label{fig:salutil_scatter_wsj}
 \end{center}
\end{figure}


次に,この選好1bが示す現象(実体の顕現性と,参照表現の簡単さとの相関)の性質を更に詳細に調べるため,
参照確率{\rm Pr$(e|{\rm pre}(U_i))$}と知覚効用{\rm Ut$(w)$}の散布図を描く.
各コーパスに含まれる約3,000照応詞を無作為抽出して描いた散布図が,
図\ref{fig:salutil_scatter_mainiti}, \ref{fig:salutil_scatter_wsj} である.
毎日新聞,WSJの双方に共通して観察できるのは,
省略(ゼロ代名詞と空範疇)がPrの値域全域に渡って分布しているのに対し,
省略以外の名詞句はPr $<$ 0.2 に偏って分布しているという現象である.
ただし,省略以外のPr $<$ 0.2 への偏りは,WSJよりも毎日新聞の方がより顕著である.
一方,WSJ(図\ref{fig:salutil_scatter_wsj})のみに見られる現象としては,Pr $>$ 0.9 付近に
非代名詞(Utが3から5程度)が集まっているのが見てとれる.
これは,図\ref{fig:pron_ratio}のPr $>$ 0.75における代名詞化率の横ばい傾向と整合する.


そこで,顕現性が高い範囲 (Pr $>$ 0.75) において代名詞化されていない実体の割合を調べると,
毎日新聞では17.6\%,WSJ では55.3\%(11,367例)であった.
さらに詳しく調べると,WSJ の11,367例のうちの41.7\%(4,735例)は固有名詞であった.
具体的には以下のような事例である.

    \vspace{\baselineskip}\begin{fminipage}{39zw}
\noindent
``We have no useful information on whether users are at risk,'' said \underline{James A. Talcott} of Boston's Dana-Farber Cancer Institute. \underline{Dr. Talcott} led a team of researchers from ...
    \end{fminipage}\vspace{\baselineskip}


この例のように,英語の新聞においては代名詞``he'' などで参照可能な顕現性が高い実体であっても,
``Dr. Talcott'' のような略称で参照されることも多い.
また,この顕現性が高い範囲において,WSJでは定冠詞句による参照も多く観察された.
これらの事例が,図\ref{fig:salutil_scatter_wsj}のPr$>$0.9 に分布する非代名詞サンプルの正体であり,
図\ref{fig:pron_ratio}の${\rm Pr}\ge 0.75$ の範囲における日本語と英語の代名詞化傾向の違いとなって現れたと考えられる.
この結果は,「日本語の新聞では,英語の新聞よりも文脈を利用した知覚的な労力削減が行われやすい」ということを示しており,
日本語の談話構造における文脈依存性の高さを示唆している.

なお,本節で議論した言語間の定量的比較を可能にしたのは,
MGCMにおける定量的な定式化である.
つまり,このような定量的な議論ができること自体が,
定量的なモデル化の効果の一つである.



\subsection{パラメタ設計における今後の課題}
\ref{sec:scale_validity}節, \ref{sec:pron_ratio}節の議論を踏まえて,
MGCMのパラメタ設計において残されている課題を述べる.
\begin{itemize}
\item {\bf 知覚コストの定義において残された課題: } 
本稿の知覚コストの定義では,英語の``Dr. Talcott'' のような略称の
知覚コストは高く評価されてしまうが,
本来ならば略称の使用も負荷削減を引き起こしているはずである.
よって,モデルの更なる改良のためには,
固有名詞の略称化や
名詞句の指示性を考慮したI($w$)の測定方法が必要であると考えられる.
これにより,英語だけでなく日本語の上でのモデル表現能力も向上すると予想される.
\item {\bf 参照確率の計測方法において残された課題: }
本稿では,参照確率の推定に用いる素性として,
白松他\citeyear{siramatu2005nlp}のものを踏襲して用いた.
モデルの更なる改良のためには,
他の素性設計(例えば,対数$\log$を用いない等)も検討し,比較する必要がある.
更に,本稿の測定方法では,\ref{sec:scale_validity}節で議論したような話題の転換点(低い参照結束性で繋がる隣接発話)を考慮していない.
話題の転換点の後では,
談話参与者が注目する実体が大きく変化する可能性がある.
参照確率は実体の顕現性を表すので,
話題の転換点の前と後で参照確率の値も変化すると考えられる.
よって今後は,先行文脈における話題の転換点を考慮した素性設計も検討する予定である.
\end{itemize}



\section{まとめ}
\label{sec:conc}

本稿では,意味ゲーム仮説に基づく参照結束性のモデルMGCMを多言語化するため,
改良MGCMを設計した.
また,日本語と英語の大規模新聞記事コーパスを用いて改良MGCMを検証し,
言語をまたぐ一般性を備えていることを示した.
これにより,
日本語と英語という性質の大きく異なる言語において,
参照結束性のメカニズムがゲーム理論で説明できるという証拠を示した.
同時に,期待効用が,参照結束性の尺度としての性質(すなわち,
先行文脈が与えられた上で後続発話の表現/解釈を選択する基準としての性質)を
備えていることを,両言語のコーパス上で示した.
\\



まず従来のMGCMには,
(1)日本語のコーパスでしか検証されていない,(2)参照表現の知覚効用${\rm Ut}(w)$を統計的に計測できない,
という2つの課題が残されていた.
様々な言語にモデルを適用するためにも,言語依存特性をコーパスから獲得可能なパラメタ設計が課題であった.

本稿では,参照表現の知覚効用を統計的に再設計することにより,
多言語に適用可能な改良MGCMを構築した.
この統計的パラメタ設計に基づき,日本語と英語のコーパスから
パラメタ(参照確率と知覚効用)の値の分布を獲得できることを示した
(\ref{sec:verification}章の表\ref{tab:sal_jpn}, \ref{tab:sal_eng}, \ref{tab:cost_jpn}, \ref{tab:cost_eng}).

この結果から,以下の2つの知見が得られた.
\begin{itemize}
\item 2つのコーパスの間には,確かに各言語に特有な言語表現と,
その上でのパラメタ分布の違いが見られた.
よって,コーパスからの統計的獲得が必要である.
\item 一方,2つのコーパスから獲得されたパラメタ分布は,言語間の違いを吸収し,従来の知見との整合性を示した.よって,各言語に特有な表現に適応したパラメタ分布が獲得できたと考えられる.
\end{itemize}

この改良MGCMの設計により多言語への適用が可能になったので,
日本語,英語という性質が大きく異なる2つの言語の大規模コーパスを用い,MGCMの統計的な検証実験を行った.
具体的には,期待効用原理から導かれた選好1a, 1b, 2 を,
両言語のコーパスの上で検証した.

選好1aの検証(\ref{sec:verification}章の図\ref{fig:pref1a_eu_mainiti}, 図\ref{fig:pref1a_eu_wsj})では,2つの候補の期待効用の差と,選好1a合致率とのスピアマン順位相関係数が,
日本語コーパス上で0.833, 英語コーパス上で0.981 となり,非常に強い正の相関を示した.
これは,両言語において「参照結束性は期待効用原理によって引き起こされる」という仮説を強く支持する結果である.
同時にこの結果から,候補間の期待効用の差が,参照結束性の差を表現しているという知見を得た.

選好1bの検証,つまり「目立っている実体ほど,知覚的に低コストな参照表現で参照されやすい」という傾向の検証 (\ref{sec:verification}章の表\ref{tab:pref1b_mainiti}, \ref{tab:pref1b_wsj})では,
選好1aに従う事例に限定した場合の参照確率と知覚効用とのスピアマン順位相関係数が,
日本語コーパス上で0.540, 英語コーパス上で0.454となり,中程度の正の相関を示した.
このことから,期待効用原理に従って表現/解釈を選択した場合,
確かに文脈を利用して知覚コストの低減を図る傾向がある,という知見を得た.
また,選好1bに従わない事例に限定した場合は,日本語コーパス上で$-$0.086, 英語コーパス上で$-$0.120 となり,
ほぼ無相関と言えるほど弱い負の相関を示した.
このことから,期待効用原理に従わずに表現/解釈を選択した場合,文脈を利用した知覚コストの低減がなされない
という知見を得た.
さらに全ての事例では,日本語コーパス上で0.377, 英語コーパス上で0.237となり,弱い正の相関を示した.
このことから,
文脈を利用して知覚コストを低減する傾向が弱いながらも存在し,
その傾向は英語コーパスより日本語コーパスの方が強いという知見を得た.
これらの結果は,両言語において「参照結束性は期待効用原理によって引き起こされる」という仮説と矛盾しない.

選好2の検証(\ref{sec:verification}章の図\ref{fig:tran_eu_mainiti}, \ref{fig:tran_eu_wsj})では,選好2とセンタリング理論のルール2との整合性を検証した.
期待効用と,ルール2における遷移タイプ順序とのスピアマン順位相関係数が,
日本語コーパス上で0.639, 英語コーパス上で0.482となり,
中程度からやや強い正の相関を示した.
これは,両言語において「参照結束性は期待効用原理によって引き起こされる」という仮説を支持する結果である.

以上の検証結果から,日本語と英語の両言語において,参照結束性は期待効用原理によって引き起こされている
という経験的な証拠を得た.
同時に両言語において,期待効用原理に従えば参照結束性の高い表現/解釈を選択できるという知見を得た.





さらに本稿では,英語と日本語のコスト低減化傾向を比較するために
両コーパスにおける参照確率と知覚効用の相関関係を比較した 
(\ref{sec:discussion}章の図\ref{fig:pron_ratio}, \ref{fig:salutil_scatter_mainiti}, \ref{fig:salutil_scatter_wsj}).
その結果,参照確率が高い実体を参照するためには
日本語の方が英語よりも低いコストの表現が選ばれやすく,
期待効用原理がより強く働いていることを発見した.
このような定量的分析が可能になったこと自体が,
MGCMにおける定量的なモデル化の効果の一つである.


今後,更に多くの言語でも改良MGCM を検証できれば,
ゲーム理論から導かれた行動選択原理が
参照結束性の認知機構として言語に独立に成り立っていることを確認できるであろう.
このゲーム理論に基づく一般化により,
従来のセンタリング理論に含まれていなかった原理的観点からの,
体系的かつ定量的な分析が可能になると期待される.






\acknowledgment

本研究を進めるにあたって有意義なコメントや励ましの言葉を戴いた奥乃研究室の学生諸君と,
丁寧な御指摘を戴いた匿名のレフェリーに深謝致します.
また,日本語新聞記事GDAコーパスの研究利用を許諾して下さった三菱電機株式会社と,
調整の労をお取り下さったGSK(言語資源協会)に感謝致します.
本研究は,科研費(特別研究員奨励費, 19・91)の助成を受けたものです.





\bibliographystyle{jnlpbbl_1.3}
\begin{thebibliography}{}

\bibitem[\protect\BCAY{Benz, Jager, \BBA\ van Rooij}{Benz
  et~al.}{2006}]{benz2006}
Benz, A., Jager, G., \BBA\ van Rooij, R. \BBOP 2006\BBCP.
\newblock {\Bem {Game Theory and Pragmatics}}.
\newblock Palgrave Macmillan, Basingstoke.

\bibitem[\protect\BCAY{Fechner}{Fechner}{1860}]{fechner1860}
Fechner, G. \BBOP 1860\BBCP.
\newblock {\Bem {Elemente der Psychophysik{\rm , translated to} Elements of
  Psychopysics{\rm , translated by H.E. Adler, 1966}}}.
\newblock Holt, Rinehart and Winston, New York.

\bibitem[\protect\BCAY{Grosz, Joshi, \BBA\ Weinstein}{Grosz
  et~al.}{1983}]{grosz1983}
Grosz, B., Joshi, A., \BBA\ Weinstein, S. \BBOP 1983\BBCP.
\newblock \BBOQ {Providing a Unified Account of Definite Noun Phrase in
  Discourse}\BBCQ\
\newblock In {\Bem Proceedings of the 21st ACL}, \mbox{\BPGS\ 44--50}.

\bibitem[\protect\BCAY{Grosz, Joshi, \BBA\ Weinstein}{Grosz
  et~al.}{1995}]{grosz1995}
Grosz, B., Joshi, A., \BBA\ Weinstein, S. \BBOP 1995\BBCP.
\newblock \BBOQ {Centering: A Framework for Modeling the Local Coherence of
  Discourse}\BBCQ\
\newblock {\Bem Computational Linguistics}, {\Bbf 21}  (2), \mbox{\BPGS\
  203--225}.

\bibitem[\protect\BCAY{Hasida}{Hasida}{1996}]{hasida1996}
Hasida, K. \BBOP 1996\BBCP.
\newblock \BBOQ {Issues in Communication Game}\BBCQ\
\newblock In {\Bem Proceedings of COLING'96}, \mbox{\BPGS\ 531--536}.

\bibitem[\protect\BCAY{Hasida, Nagao, \BBA\ Miyata}{Hasida
  et~al.}{1995}]{hasida1995}
Hasida, K., Nagao, K., \BBA\ Miyata, T. \BBOP 1995\BBCP.
\newblock \BBOQ {A Game-Theoretic Account of Collaboration in
  Communication}\BBCQ\
\newblock In {\Bem Proceedings of the First International Conference on
  Multi-Agent Systems}, \mbox{\BPGS\ 140--147}.

\bibitem[\protect\BCAY{Iida}{Iida}{1997}]{iida1996}
Iida, M. \BBOP 1997\BBCP.
\newblock \BBOQ Discourse Coherence and Shifting Centers in Japanese
  Texts\BBCQ\
\newblock In Walker, M., Joshi, A., \BBA\ Prince, E.\BEDS, {\Bem Centering
  Theory in Discourse}, \mbox{\BPGS\ 161--180}. Oxford University Press.

\bibitem[\protect\BCAY{Kameyama}{Kameyama}{1998}]{kameyama1998}
Kameyama, M. \BBOP 1998\BBCP.
\newblock \BBOQ {Intrasentential Centering: A Case Study}\BBCQ\
\newblock In Walker, M., Joshi, A., \BBA\ Prince, E.\BEDS, {\Bem Centering
  Theory in Discourse}, \mbox{\BPGS\ 89--112}. Oxford University Press.

\bibitem[\protect\BCAY{Kibble}{Kibble}{2001}]{kibble2001}
Kibble, R. \BBOP 2001\BBCP.
\newblock \BBOQ {A Reformulation of Rule 2 of Centering Theory}\BBCQ\
\newblock {\Bem Computational Linguistics}, {\Bbf 27}  (4).

\bibitem[\protect\BCAY{Ng}{Ng}{2004}]{ng2004}
Ng, V. \BBOP 2004\BBCP.
\newblock \BBOQ {Learning noun phrase anaphoricity to improve coreference
  resolution: Issues in representation and optimization}\BBCQ\
\newblock In {\Bem Proceedings of the 42nd ACL}, \mbox{\BPGS\ 152--159}.

\bibitem[\protect\BCAY{Osborne \BBA\ Rubinstein}{Osborne \BBA\
  Rubinstein}{1994}]{osbone1994}
Osborne, M.\BBACOMMA\ \BBA\ Rubinstein, A. \BBOP 1994\BBCP.
\newblock {\Bem {A Course in Game Theory}}.
\newblock The MIT Press, Cambridge, Massachusetts.

\bibitem[\protect\BCAY{Parikh}{Parikh}{2001}]{parikh2001}
Parikh, P. \BBOP 2001\BBCP.
\newblock {\Bem {The Use of Language}}.
\newblock CSLI Publications, Stanford, California.

\bibitem[\protect\BCAY{Poesio, Stevenson, Eugenio, \BBA\ Vieira}{Poesio
  et~al.}{2004}]{poesio2004}
Poesio, M., Stevenson, R., Eugenio, B.~D., \BBA\ Vieira, R. \BBOP 2004\BBCP.
\newblock \BBOQ {Centering: A parametric theory and its instantiations}\BBCQ\
\newblock {\Bem Computational Linguistics}, {\Bbf 30}  (3), \mbox{\BPGS\
  309--363}.

\bibitem[\protect\BCAY{R-Foundation}{R-Foundation}{2003}]{Rsite}
R-Foundation \BBOP 2003\BBCP.
\newblock \BBOQ {The R Project for Statistical Computing}\BBCQ\
\newblock http:{\slash}{\slash}www.{R}-project.org{\slash}.

\bibitem[\protect\BCAY{Strube \BBA\ Muller}{Strube \BBA\
  Muller}{2003}]{strube2003}
Strube, M.\BBACOMMA\ \BBA\ Muller, C. \BBOP 2003\BBCP.
\newblock \BBOQ {A machine learning approach to pronoun resolution in spoken
  dialogue}\BBCQ\
\newblock In {\Bem Proceedings of the 41st ACL}, \mbox{\BPGS\ 168--175}.

\bibitem[\protect\BCAY{van Rooij}{van Rooij}{2004}]{rooij2004}
van Rooij, R. \BBOP 2004\BBCP.
\newblock \BBOQ Signaling games select Horn strategies\BBCQ\
\newblock {\Bem Linguistics and Philosophy}, {\Bbf 27}, \mbox{\BPGS\ 493--527}.

\bibitem[\protect\BCAY{von Neumann \BBA\ Morgenstern}{von Neumann \BBA\
  Morgenstern}{1944}]{neumann1944}
von Neumann, J.\BBACOMMA\ \BBA\ Morgenstern, O. \BBOP 1944\BBCP.
\newblock {\Bem {Theory of Games and Economic Behavior}}.
\newblock Princeton University Press, Princeton.

\bibitem[\protect\BCAY{Walker, , Iida, \BBA\ Cote}{Walker
  et~al.}{1994}]{walker1994}
Walker, M., , Iida, M., \BBA\ Cote, S. \BBOP 1994\BBCP.
\newblock \BBOQ {Japanese Discourse and the Process of Centering}\BBCQ\
\newblock {\Bem Computational Linguistics}, {\Bbf 20}  (2), \mbox{\BPGS\
  193--232}.

\bibitem[\protect\BCAY{Yamura-Takei, Takada, \BBA\ Aizawa}{Yamura-Takei
  et~al.}{2000}]{takei2000}
Yamura-Takei, M., Takada, M., \BBA\ Aizawa, T. \BBOP 2000\BBCP.
\newblock \BBOQ {The Role of Global Topic in Japanese Zero Anaphora Resolution
  (in Japanese)}\BBCQ\
\newblock {\Bem Technical Report of IPSJ}, {\Bbf 135}  (10), \mbox{\BPGS\
  71--78}.

\bibitem[\protect\BCAY{橋田}{橋田}{1998}]{hasida1998gda}
橋田浩一 \BBOP 1998\BBCP.
\newblock \JBOQ GDA: 意味的修飾に基づく多用途の知的コンテンツ\JBCQ\
\newblock \Jem{人工知能学会誌}, {\Bbf 13}  (4), \mbox{\BPGS\ 528--535}.

\bibitem[\protect\BCAY{白松\JBA 宮田\JBA 奥乃\JBA 橋田}{白松\Jetal
  }{2005}]{siramatu2005nlp}
白松俊\JBA 宮田高志\JBA 奥乃博\JBA 橋田浩一 \BBOP 2005\BBCP.
\newblock \JBOQ
  ゲーム理論による中心化理論の解体と実言語データに基づく検証\JBCQ\
\newblock \Jem{自然言語処理}, {\Bbf 12}  (3), \mbox{\BPGS\ 91--110}.

\bibitem[\protect\BCAY{舟尾}{舟尾}{2005}]{funao2005}
舟尾暢男 \BBOP 2005\BBCP.
\newblock {\Bem {The R Tips}}.
\newblock 九天社.

\bibitem[\protect\BCAY{電子情報技術産業協会}{電子情報技術産業協会}{2005}]{jeit
a2005}
電子情報技術産業協会 \BBOP 2005\BBCP.
\newblock \JBOQ 言語情報処理 用語集 (in 言語情報処理 ポータル)\JBCQ\
\newblock
  http:{\slash}{\slash}nlp.kuee.kyoto-u.ac.jp{\slash}{NLP\_{}Portal}{\slash}gl
ossary{\slash}index.html.

\bibitem[\protect\BCAY{飯田\JBA 乾\JBA 松本}{飯田\Jetal }{2004}]{iida2004}
飯田龍\JBA 乾健太郎\JBA 松本裕治 \BBOP 2004\BBCP.
\newblock \JBOQ
  文脈的手がかりを考慮した機械学習による日本語ゼロ代名詞の先行詞同定\JBCQ\
\newblock \Jem{情報処理学会論文誌}, {\Bbf 45}  (3), \mbox{\BPGS\ 906--918}.

\end{thebibliography}



\begin{biography}
\bioauthor{白松  俊(学生会員){\unskip}}{
2000年東京理科大学理工学部情報科学科卒業.
2003年同大学院修士課程修了,JST CREST研究補助員を経て,
現在京都大学大学院情報学研究科博士後期課程在学中.
談話文脈研究に従事.
2007年よりJSPS特別研究員 (DC2).
}

\bioauthor{駒谷 和範(正会員){\unskip}}{
1998年京都大学工学部情報工学科卒業.2002年同大学院情報学研究科博士後期
課程修了.博士(情報学).現在,同大学院情報学研究科助教.音声対話シス
テムの研究に従事.情報処理学会平成16年度山下記念研究賞等を受賞.
}

\bioauthor{橋田 浩一(正会員){\unskip}}{
1981年東京大学理学部情報科学科卒業.1986年同大学大学院理学系研究科博士
課程修了.理学博士.同年,電子技術総合研究所入所.現在,産業技術総合研
究所情報技術研究部門長.日本認知科学会優秀論文賞等を受賞.
}

\bioauthor{尾形 哲也(非会員){\unskip}}{
1993年早稲田大学理工学部機械工学科卒業.博士(工学).
2003年京都大学大学院情報学研究科講師,現在同准教授.
ロボット模倣,神経回路モデル等の研究に従事.
情報処理学会,日本ロボット学会,IEEEなどの会員.
}

\bioauthor{奥乃  博(非会員){\unskip}}{
1972年東京大学教養学部 基礎科学科卒業. 博士(工学).
2001年より京都大学大学院情報学研究科教授.
音環境理解,ロボット聴覚の研究に従事.情報処理学会, 人工知能学会,
ロボット学会,ACM,IEEE等会員.
}

\end{biography}


\biodate



\end{document}
