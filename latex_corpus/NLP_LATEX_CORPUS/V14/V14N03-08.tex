    \documentclass[japanese]{jnlp_1.3b}
\usepackage{jnlpbbl_1.1}
\usepackage[dvips]{graphicx}

\setcounter{secnumdepth}{3}


\Volume{14}
\Number{3}
\Month{Apr.}
\Year{2007}
\received{2006}{4}{20}
\revised{2006}{8}{18}
\rerevised{2006}{11}{13}
\accepted{2006}{11}{27}

\setcounter{page}{131}

\jtitle{感情が表現された音声の$F_0$パターン制御指令と\\言語的情報との関係}
\jauthor{河津 宏美\affiref{TEU} \and 大野 澄雄\affiref{TEU}}
\jabstract{
表現豊かな合成音声への応用を目的として,「喜び」と「悲しみ」の2種類の感情について複数の程度の感情情報を含む音声に対し,基本周波数パターン生成過程モデルに基づく韻律的特徴の分析を行った.分析により得られた各モデルパラメータの生起タイミングおよびその大きさの変化に関して,発話の言語的情報に基づき検討を行った.その結果,基底周波数に関しては,発話内容に依存する傾向の差は特にみられず,「喜び」「悲しみ」とも感情の程度が強くなるに従って,基底周波数は高くなる傾向にあった.文中でのフレーズ指令の生起に関しては,文節境界の枝分かれ種別と,直前のフレーズ指令以降のモーラ数に影響を受けることを確認した.フレーズ指令の大きさ関しては,その生起位置が文頭の場合と文中の場合とで,感情の程度に対する大きさの変化に違いがみられた.また,文中で生起したフレーズ指令は,その境界の枝分かれ種別により,異なる変化の傾向がみられた.アクセント指令の生起タイミングは,感情の有無やその程度の影響はほとんど見られず,アクセント型にのみ依存することが確認された.アクセント指令の大きさに関しては,文頭からの韻律語数に大きな影響を受け,またアクセント型による違いがみられた.
}
\jkeywords{感情表現,基本周波数パターン,$F_0$パターン制御指令,言語的要因}

\etitle{Relationship between Commands of $F_0$ Contour Control and Linguistic Information for Emotional Utterances}
\eauthor{Hiromi Kawatsu\affiref{TEU} \and Sumio Ohno\affiref{TEU}} 
\eabstract{
In order to make synthetic speech rich in its expression, fundamental frequency contours were analyzed for the utterances of several emotional degrees with joy and sadness based on a model for the process of generation.  Changes in controlling parameters of the model with regard to degrees of emotion were examined in terms of linguistic factors of the utterances.  As a result, the baseline frequency increases as emotional degree increases, especially for sadness utterances.  About the phrase commands, the rate of occurrence increases as emotional degree increases at the right branch boundary in the grammatical structure for both joy and sadness, while the rate of occurrence at the left branch boundary for sadness is almost constant for emotional degrees.  The change of the amplitude of phrase commands is depended on the kind of position of grammatical structure.  About the accent commands, timings of their onsets and offsets are almost constant for emotional degrees.  They are depended on the accent types of prosodic words.  The magnitude of the accent commands changes as emotional degree increases depending on the positions of prosodic words from the beginning of the utterance.
}
\ekeywords{Emotional expression, Fundamental frequency contours, Prosodic control commands, Linguistic factor}

\headauthor{河津,大野}
\headtitle{感情が表現された音声の$F_0$パターン制御指令と言語的情報との関係}

\affilabel{TEU}{東京工科大学 バイオ・情報メディア研究科}{
	Graduate School of Bionics, Computer and Media Sciences, Tokyo University of Technology}



\begin{document}
\maketitle



\section{まえがき}
音声言語処理の研究・開発は,コンピュータの高性能化を背景にし,ここ数年の間に飛躍的な発展を遂げ,特に大量のデータに基づく,確率・統計的なモデル化のアプローチは,音響処理面および言語処理面の双方において大きな成功を収めた.これらの技術的進展により,音声合成・音声認識技術は一気に実用レベルに達し,人間とコンピュータとのインタフェースとして広範囲に応用されるに至った.一方,応用範囲が広範になるにつれ,その精度・品質に対して,より高いレベルのものが要求されるようになっている.例えば,音声合成においては,テキストを単に読み上げるだけのものから,パラ言語情報や感情などを表現する柔軟な合成音声が望まれる.このような合成音声には,音声機能障害者を対象とした対話支援システム\cite{Ii2}や,癒し系ロボットへの応用など様々なものが提案されている.

言語情報だけでは伝わらないこのような表現豊かな音声での感情や意図の表現には,韻律的特徴が大きく寄与することは明らかである\cite{Rai}, \cite{Fuji}, \cite{Nikku}.そのため,従来から,感情音声の韻律的特徴に関する研究が行われており,基本周波数パターン(以下,$F_0$パターンと呼ぶ)の統計的な傾向を音声心理学的な観点からとらえた研究\cite{Naga}, \cite{As}, \cite{Sige}や,音声言語コーパスに基づく工学的な観点からの研究\cite{Koba}, \cite{Sagi}がある.日本語音声における感情表現に関する研究は,感情の種類に着目して,特徴付けや識別を試みている例が多い\cite{ITI}, \cite{Ii}, \cite{Kita}.一方,表現豊かな音声合成のために,特定の感情について,その程度を数段階に分けた研究も行われるようになってきている.\cite{Hasi}, \cite{kw2}, \cite{Nsima}.また,韻律的特徴が,文の統語構造と関連を持つことも明らかとなっており\cite{Hir}, \cite{Ume}, 韻律的特徴であるイントネーションやアクセントの生起タイミングにおいては,言語的情報を的確に利用することで,よりよい決定が行われると考えられる.

筆者らは,表現豊かな音声合成の実現を目的として,特に感情に着目し,複数の程度の感情情報を含む典型的な発話に対する韻律制御指令の生成について検討を行っている.本論文は,感情の種類として「喜び」「悲しみ」の2つの感情を取り上げ,それぞれの感情を3段階の程度で表現した音声に対し,発話の言語的情報と$F_0$パターン制御指令のパラメータとの関係について検討することによって,感情を表現する音声合成への応用を目指すものである.すなわち,本論文の主な目的は,感情の種類の判別・差異に着目することではなく,同一感情の程度に対する影響について明らかにするものである.そのための足がかりとして,R. Plutchikが提案した心理学上の感情の立体モデル\cite{PLU}の基本8感情のうち「喜び」「悲しみ」のみを対象として取り上げるにとどめた.

「喜び」「悲しみ」の2感情をその程度まで考慮し分析した先行研究\cite{MD}では,4〜6モーラの単語を発声した際の感情音声を対象として,韻律的特徴が分析されている.この先行研究では,韻律的特徴のうち,時間構造に関するパラメータと$F_0$パターンに関するパラメータを取り扱っている.しかし,孤立に発声された特定の単語発話に対する詳細な検討であり,任意の文章を対象とした音声合成に直接応用することは困難であると考えられる.一方,模擬対話を行って数段階の程度で感情音声を収集した先行研究\cite{Kawana}では,非常に限られた種類の文を対象として文発話を収録している.収録音声の$F_0$パターンとモーラ持続時間短縮率について分析を行い,感情の程度と韻律的特徴との間に一定の傾向を見い出しているが,それは話者や感情の種類によって大きく異なるものと結論づけるにとどまり,一般化には及んでいない.

本論文では,任意の文章への感情音声合成への応用を目指して,感情ごとに異なる10文を用意して分析対象とした.ここで,言語的要因の1つである係り受け関係を網羅するため,対象を4文節からなる文に限定した.また,韻律的特徴には,$F_0$パターン・発話速度・発話強度・声質など様々あるが,日本語音声の場合,高さに関する特徴である$F_0$パターンが韻律情報を支配する直接的要因であると考えられているため,本論文では特に$F_0$パターンに着目することとする.$F_0$パターンについては,その生成過程モデル\cite{Fuji3}に基づいた分析を行い,韻律的特徴の定量化を行う.これは$F_0$パターン生成過程モデルが,音声を生成する人間の生理的・物理的な特性を捉えたものであり,また,言語的内容とも整合した制御指令が得られることが確認されているためである.このモデルの$F_0$パターン制御指令の変化傾向をとらえることで,テキスト音声合成時の$F_0$パターン生成に直接に結びつけることが可能であると期待できる.

以下,\ref{mt}. では,発話内容の言語的情報と音声資料の収録方法について述べる.\ref{ln}. において,$F_0$パターンの分析手法について述べ,\ref{ex}. で言語的情報に基づき$F_0$パターン制御指令のパラメータとの関係について検討した結果について述べる.\ref{sa}. で本論文をまとめる.

\section{音声資料}\label{mt}

\subsection{発話内容の言語的情報}\label{yoso}

発話テキストとして,「喜び」「悲しみ」のそれぞれの感情に対して,4文節からなるテキスト10文を用意した.
特定の感情を前提としない中立な文を発話テキストとして用いて複数感情を同一文で表現させる手法もあるが,ここでは感情ごとに適切な文をそれぞれ用意した.これは,話者に無理のない状況設定を理解させ,より自然な音声資料を収集するためである.表~\ref{hatuwa}に,用意した発話テキストの一部を示す.

韻律制御指令の生起タイミングは,感情の種類,その程度のほか,発話内容の言語的情報である係り受け関係,モーラ数,アクセント型等の影響を受けると考えられる.表~\ref{factors}に,発話テキストから得られる言語的情報に基づく要因をまとめた.これらの要因はテキスト音声合成の際に,一般的に考慮されているものである.ここで,文の係り受け構造に関しては,文節境界の枝分かれ種別に着目した.図~\ref{bdc}に文の係り受け構造と境界の枝分かれ種別を示す.以上,ここでは文節数と係り受け構造に配慮したほかは,特にテキストの制約を与えないこととした.\ref{ex}. 以降の結果の検討において,これらの要因との影響について調べていく.

\begin{table}[b]
\begin{center}
\caption{発話テキストの例}
\label{hatuwa}
\begin{tabular}{p{0.06\textwidth}p{0.42\textwidth}p{0.42\textwidth}}
\hline
&「喜び」&「悲しみ」\\
\hline
1.& 海が透き通って珊瑚まで見えるよ.&いつまで待っても帰ってこない.\\
2.& こんなステーキを食べられて幸せだ.&彼に伝える勇気がありません.\\
3.& 5年掛かってついに司法試験に合格したよ.&これで二度と息子に会えない.\\
4.& あの大きいぬいぐるみが欲しい.&センターの数学のテストができなかった.\\
5.& 今日は僕のウチで遊ぼうよ.&二度ときれいな花が見られない.\\
\hline
\end{tabular}
\end{center}
\end{table}

\begin{table}[b]
\begin{center}
\caption{言語的情報に基づく要因}
\label{factors}
\begin{tabular}{ll}
\hline
要因 & とり得る値 \\
\hline
韻律語の文中での位置&1,2,3,4.\\
境界の枝分かれ種別&右枝分かれ,左枝分かれ.\\
韻律語のモーラ数&2,3,…,8.\\
アクセント型&平板型,頭高型,起伏型.\\
修飾関係&連体修飾,連用修飾.\\
\hline
\end{tabular}
\end{center}
\end{table}

\begin{figure}[t]
\begin{center}
    \includegraphics[width=1.0\textwidth]{bdc.eps}
\end{center}
\caption{係り受け構造と境界の枝分かれ種別(L:左枝分かれ境界 / R:右枝分かれ境界)}
\label{bdc}
\end{figure}

\begin{table}[t]
\caption{発話の状況設定の例}
\label{jyoukyou}
\begin{tabular}{p{0.09\textwidth}p{0.42\textwidth}p{0.42\textwidth}}
\hline
&「喜び」&「悲しみ」\\
\hline
テキスト&「海が透き通って珊瑚まで見えるよ.」&「いつまで待っても帰ってこない.」\\
状況設定& 待ちに待った夏休み.海がとても綺麗と評判の島に行きました.本当に透き通るように綺麗な海で珊瑚が砂浜からでも見られたことに感激して一言.&
一緒に生活するのが当たり前だと思っていた夫(or妻)を不慮の事故で亡くしました.いつもの元気のよい“ただいま”の声が聞こえてきません.悲しんで一言.\\
\hline
\end{tabular}
\end{table}

\subsection{録音条件}

中立な(感情を込めない)発話と,弱・中・強の3段階の程度で「喜び」と「悲しみ」の2種類の感情を表現した発話を収録した.発話の際,指定した感情を表現しやすくするために発話の状況を設定した.表~\ref{jyoukyou}に,状況設定の例を示す.収録ではディスプレイに表示される発話テキストおよび状況設定に従い,演劇経験のある成人話者8名(男6名,女2名)が簡易防音室内において「中立」→「中」→「弱」→「強」の順に発話した音声をそれぞれ3回録音した.

\subsection{感情の程度の主観評価}\label{subeva}
話者の意図によって数段階の程度で収録した音声に対し,聴取実験を行い,聞き手側の観点から感情の程度の主観評価を行った.この聴取実験は,話者の意図した感情表現の程度と,聞き手が受容した感情の程度との一致度を把握することが目的である.被験者は大学4年生の男女6名である.まず,各感情の発話テキスト10文の中からランダムに5文を選択し,480発話(5文×4段階×3セット×8人)を実験用の音声資料とした.実験では,話者ごとにデータセットを用意し,``これから聞こえてくる音声の感情の程度を判定して下さい''という指示に続いて,中立を含むそれぞれの感情に関するすべての発話をランダムな順序で呈示し,聴取した音声に指定した感情がどの程度表れていると感じたかを「まったく感情が表れていない,他の感情に聞こえる」という場合の``0"と,「僅かに感情が表れている」という場合の``1"から「とても強く感情が表れている」という場合の``5"までの6段階で評価し,その数字を回答させた.

\begin{figure}[b]
\begin{center}
    \includegraphics[width=0.8\textwidth]{14-3ia8f2.eps}
\end{center}
\caption{感情の程度の主観評価値と有意差検定の結果(上段:話者 MTI,下段:話者 MTS)}
\label{mos}
\end{figure}

    \subsubsection{有意差検定}
図~\ref{mos}に,話者の意図した感情の程度別の主観評価値の分布結果と,それぞれの程度間で行った片側検定による有意差検定の結果の一部を示す.
ここでは,話者の意図した感情の程度が最も的確に聞き手側に伝わったと考えられる話者MTIに対する結果と,逆に,話者の意図した感情の程度が聞き手側にうまく伝わらなかったと考えられる話者MTSに対する結果を示す.図中の点線は聴取実験において得られた評価値の平均値を表している.

話者MTIについては,「喜び」に関して感情の程度強の平均値が小さく,また分散も大きくなっているが,それぞれの程度の違いについては有意に区別された.また,「悲しみ」に関しては,それぞれ有意な差があり,話者の意図した程度が聞き手に伝達された.一方,話者MTSについては,「喜び」に関して,すべての程度において評価値の分布が重なった.「悲しみ」に関しても,一部有意な差がないと判断され,明確な有意差を見い出せない組み合わせもあり,必ずしもうまく話者の意図した感情の程度が聞き手に伝わらなかった.


\section{韻律的特徴の分析手法}\label{ln}


\begin{figure}[b]
\begin{center}
    \includegraphics[width=0.8\textwidth]{f0model.eps}
\end{center}
\caption{$F_0$パターン生成過程モデル}
\label{model}
\end{figure}

韻律的特徴の分析には,藤崎らによって提案された$F_0$パターン生成過程モデル(図~\ref{model})\cite{Fuji3}を用いた.このモデルは,対数$F_0$パターンが句頭から句末に向かって上昇とその後の緩やかな下降を示すフレーズ成分と,語のアクセントに対応して局所的な起伏を示すアクセント成分,および発話単位中で,ほぼ一定値をとるベースライン成分(基底周波数)の総和として表現できるとするものであり,各成分がそれぞれの指令に対する一定の応答から生成されるとしている.このモデルの入力パラメータは,音声を生成する人間の生理的・物理的な特性を捉えたもので,言語的内容とも整合した制御パラメータが得られることが確認されている.河井らは,韻律上の単位を$F_0$パターン生成過程モデルにおけるフレーズ指令・アクセント指令に基づいて定義しており\cite{kawa},本論文でも「韻律語」「韻律句」を同じ定義で取り扱う.すなわち,一つのアクセント指令に対応し,かつ,一定のアクセント型を示す音素連鎖を「韻律語」,また,一つのフレーズ指令に対応する韻律語の連鎖を「韻律句」と定義する.

\begin{figure}[b]
\begin{center}
    \includegraphics[width=0.85\textwidth]{sample.eps}
\end{center}
\caption{$F_0$パターン生成過程モデルに基づく分析例}
\label{sample}
\end{figure}

モデルパラメータである基底周波数$F_b$,フレーズ指令の大きさ$A_{pi}$とその生起位置$T_{0i}$,および,アクセント指令の大きさ$A_{aj}$とその生起位置$T_{1j}$,  $T_{2j}$について,録音した音声の$F_0$パターンの分析を行った.具体的にはまず,収録した音声資料を10kHz・16bitでデジタル化し,LPC予測残差に対する変形自己相関関数法を用いて10ms間隔で$F_0$の値を抽出した.その$F_0$パターンに対し,視察によりモデルのパラメータを定め,次にそれを初期値として,AbS法に基づき最良近似を与えるパラメータを求めた.図~\ref{sample}に,「喜び」に関する発話「海が透き通って珊瑚まで見えるよ.」と,「悲しみ」に関する発話「いつまで待っても帰ってこない.」について,$F_0$パターン生成過程モデルに基づいて分析した結果を示す.上から,音声波形,$F_0$の実測値(+印)・モデルによる最良近似(実線)・フレーズ成分(破線)・基底周波数$F_b$(点線),フレーズ指令$A_p$,およびアクセント指令$A_a$を示している.


\section{$F_0$パターン制御指令と言語的要因との関係についての検討}\label{ex}
感情の程度の違いが,$F_0$パターン制御指令の生起タイミングおよび大きさに,どのように影響するかについて,文の言語的要因との関連で求めた.
収録音声の感情の程度を聞き手側の観点から主観評価した結果,話し手の意図での感情の程度が最も的確に聞き手に伝達されていた男性話者1名(MTI)の音声資料に対して,検討を行った結果を以下に示す.

\subsection{基底周波数$F_b$}
図\ref{Fbkekka}は収集した発話について,各発話の感情の程度に対する基底周波数の分布を示したものである.
発話セットによる固有の傾向は認められず,また,発話内容に依存する傾向の差は特にみられなかった.
以降の検討では,発話セットに関する傾向の違いはないものとする.
「喜び」「悲しみ」とも感情の程度が強くなるに従って,基底周波数は高くなる傾向にある.「喜び」では,感情の程度が弱いときはほぼ一定あるいは微増の変化傾向がみられ,感情の程度が強いとき,増加傾向がみられた.一方,「悲しみ」では,感情の程度が強くなるにつれて,ほぼ一様な増加傾向があった.

\begin{figure}[t]
\begin{center}
    \includegraphics[width=0.9\textwidth]{fbokisa.eps}
\end{center}
\caption{感情の程度に対する基底周波数の分布}
\label{Fbkekka}
\end{figure}

\subsection{フレーズ指令$A_p$}\label{Ap}
\subsubsection{文節境界におけるフレーズ指令の生起率}
文中でのフレーズ指令の生起に関しては,文節境界の枝分かれ種別と,直前のフレーズ指令以降のモーラ数に大きく影響を受けることを予備的に確認した.図~\ref{kekka-1}に文節境界の枝分かれ種別ごとに,感情の程度に対するフレーズ指令の生起率と直前のフレーズ指令以降のモーラ数との関係を示した.
「喜び」では,いずれの枝分かれ境界に対しても感情の程度が強くなるにつれて,より短い韻律句の後でも生起率が増加する傾向があった.一方「悲しみ」では,右枝分かれ境界の場合,感情の程度が強くなるに従って生起率が大きく増加し,左枝分かれ境界の場合,感情の程度の影響を受けず生起率はほぼ一定であった.限られたデータ数から得た結果ではあるが,枝分かれ境界種別,直前のフレーズ指令からのモーラ数をパラメータとして,感情の程度の影響が明確に表われることが確認できた.

\begin{figure}[t]
\begin{center}
    \includegraphics[width=1.0\textwidth]{Aptiming.eps}
\end{center}
\caption{直前のフレーズ指令からのモーラ数に対するフレーズ指令生起率}
\label{kekka-1}
\end{figure}

\begin{figure}[t]
\begin{center}
    \includegraphics[width=1.0\textwidth]{Apokisa.eps}
\end{center}
\caption{感情の程度に対するフレーズ指令の大きさ}
\label{kekka-3}
\end{figure}

\subsubsection{フレーズ指令の大きさ}
フレーズ指令の大きさ関しては,その生起位置が文頭の場合と文中の場合とで,感情の程度に対する大きさの変化に違いがみられた.また,文中で生起したフレーズ指令は,その境界の枝分かれ種別により,異なる変化の傾向がみられた.図~\ref{kekka-3}に,文頭・右枝分かれ境界・左枝分かれ境界ごとに,感情の程度に対するフレーズ指令の大きさを示した.感情を込めない中立の発話の場合,フレーズ指令の大きさは,文頭>右枝分かれ境界>左枝分かれ境界,の関係がみられた.中立発話の場合,文の統語構造である切れ目の深さが深いほど,大きなフレーズ指令が生起するものと考えられる.また「喜び」「悲しみ」ともに,感情の程度が強くなると,文頭のフレーズ指令の大きさは小さくなり,左枝分かれ境界のフレーズ指令は大きくなった.右枝分かれ境界のフレーズ指令の大きさは,感情の種類により異なる傾向がみられ,「喜び」では,感情の程度が強くなるに従って,大きくなり,感情の程度強において文頭でのフレーズ指令の大きさとほぼ一致した.一方「悲しみ」では,感情の程度が強くなるに従って,小さくなり,発話内のすべてのフレーズ指令の大きさが近づいた.

\subsection{アクセント指令$A_a$}\label{Aa}
\subsubsection{アクセント指令の生起タイミング}


\begin{table}[b]
\begin{center}
\caption{立ち上がり・立ち下がりの基準点}
\label{basetime}
\begin{tabular}{cll}
\hline
型&立ち上がりの基準点&立ち下がりの基準点\\
\hline
平板型&第2モーラの母音開始時点&最終モーラの終了時点\\
頭高型&第1モーラの母音開始時点&第2モーラの母音開始時点\\
起伏型&第2モーラの母音開始時点&アクセント核を持つ次の母音開始時点\\
\hline
\end{tabular}
\end{center}
\end{table}


表~\ref{basetime}に示すように,各韻律語ごとのアクセント型の情報に基づきアクセント指令の立ち上がり・立ち下がりの基準点を求めた\cite{kawa}.
この基準点に対し,$F_0$パターン生成過程モデルを用いた分析によって得られたアクセント指令の立ち上がり・立ち下がりのタイミングの相対時間が感情の程度によってどのように変化するかについて検討を行った.

表\ref{onoffset}に基準点に対するアクセント指令の立ち上がり・立ち下がりのタイミングの相対時間の平均値を示した.その結果,感情の有無やその程度の影響はほとんど見られず,アクセント指令の立ち上がり・立ち下がりのタイミングはアクセント型にのみ依存した.したがって,音声合成時に感情の程度を制御するにあたって,アクセント指令のタイミングについて考慮する必要はないことを確認した.


\begin{table}[t]
\begin{center}
\caption{基準点に対するアクセント指令の生起タイミング}
\label{onoffset}
\begin{tabular}{lcc}
\hline
&立ち上がり[s]&立ち下がり[s]\\
\hline
平板型&$-0.077$&$-0.062$\\
頭高型&$-0.075$&$\phantom{-}0.027$\\
起伏型&$-0.104$&$-0.034$\\
\hline
\end{tabular}
\end{center}
\end{table}

\subsubsection{アクセント指令の大きさ}

\begin{figure}[b]
\begin{center}
    \includegraphics[width=0.9\textwidth]{Aaokisa.eps}
\end{center}
\caption{文頭からの韻律語数ごとの感情の程度に対するアクセント指令の大きさの増加率}
\label{Aakekka}
\end{figure}


アクセント指令の大きさに関しては,予備的な検討の結果,文頭からの位置(韻律語数)が大きく影響することが分かっている\cite{kw2}.図~\ref{Aakekka}に,感情の程度に対するアクセント指令の大きさの増加率を,文頭からの生起位置に着目して示した.ここでは,中立を0,感情の程度が弱の場合を1,中の場合を2,強の場合を3として,これらの感情の程度の値に対するアクセント指令の大きさの回帰直線を求め,この回帰直線の傾きを増加率と定義し,感情の程度に対する増加率と捉える.つまり,この増加率が大きいということは,感情の程度が強くなるに従って,アクセント指令の大きさがより大きくなるということを表す.「喜び」では,文頭からの位置が離れている韻律語のアクセントに増大傾向がみられ,一方「悲しみ」では,文頭のアクセントに最も顕著に減少の変化傾向がみられた.また,図~\ref{Aakekka1}には,平板型,頭高型,起伏型のアクセント型ごとに,感情の程度に対するアクセント指令の大きさの増加率を示した.
「喜び」「悲しみ」いずれの感情においても,平板型・起伏型の韻律語に対するアクセント指令の大きさは,感情の程度の影響を受けずほぼ一定,あるいは,若干の減少傾向がみられた.一方,頭高型の韻律語に対するアクセント指令の大きさは,感情の程度が強くなるにつれて減少する変化傾向がみられた.


\begin{figure}[t]
\begin{center}
    \includegraphics[width=0.9\textwidth]{Aaokisa1.eps}
\end{center}
\caption{アクセント型ごとの感情の程度に対するアクセント指令の大きさの増加率}
\label{Aakekka1}
\end{figure}

\section{あとがき}\label{sa}
本論文では,表現豊かな感情音声の合成を目的として,「喜び」「悲しみ」の2つの感情を取り上げ,それらの感情の程度と韻律的特徴との関係について,特に$F_0$パターンの特徴に着目して分析を行った.$F_0$パターンの特徴としては,$F_0$パターン生成過程モデルの各パラメータを採用し,得られる結果がテキスト音声合成時に容易に適用できることを目指した.分析の際,テキスト音声合
成で考慮される言語的要因との関係について検討を行った.

その結果,$F_0$パターン生成過程モデルのパラメータのうち基底周波数,フレーズ指令の大きさ,およびアクセント指令の大きさに関して,感情の程度に対する各パラメータの変化傾向が得られた.また,フレーズ指令の生起に対する感情の程度の影響については,文節境界の枝分かれ種別と直前のフレーズ指令からモーラ数といった言語的要因ごとに異なることを確認した.アクセント指令の生起タイミングに関して,感情の有無や程度の違いが,アクセント型の変形を伴うような影響を与えないことを確認した.

本論文で得られた結果は,4文節からなる文発話を対象として変化傾向を求めたものであるが,基底周波数,フレーズ指令の生起位置および大きさについては,他の文節数からなる一般の文への適応が可能であると考えられる.アクセント指令の大きさの変化傾向に関しては,文頭からの文節数が大きく寄与することが明らかとなっており,前述の一般文へ適応可能なパラメータに対する妥当性の検証も含めて,さらなる検討が必要である.

また,本論文では,音声合成の工学的な応用を第一の目的として捉え,聞き手に対して感情の種類・程度が的確に表現されている典型的な1名の発話を分析対象とした.しかし,川波らの先行研究\cite{Kawana}によれば,感情表現に関する韻律的特徴には個人差が大きいことが指摘されており,今後,本論文と同様のアプローチにより複数話者の分析を進め,話者に独立な傾向と話者に依存する傾向とに分離して把握することを試みる予定である.

さらに,本論文では韻律的特徴として$F_0$パターンのみを分析の対象としたが,筆者らの予備的検討によると,感情の種類によっては,発話速度がその伝達に大きく寄与するという結果を得ており,今後,$F_0$パターンのほか,発話速度,パワーを含めた総合的な韻律制御規則を導出するべく検討を進める計画である.


{\renewcommand{\baselinestretch}{}\selectfont
\bibliographystyle{jnlpbbl_1.2}
\begin{thebibliography}{}

\bibitem[\protect\BCAY{Banse \BBA\ Scherer}{Banse \BBA\ Scherer}{1996}]{Rai}
Banse, R.\BBACOMMA\ \BBA\ Scherer, K.~R. \BBOP 1996\BBCP.
\newblock \BBOQ Acoustic Profiles in Vocal Emotion Expression\BBCQ\
\newblock {\Bem Journal of Personality and Social Psychology}, {\Bbf 70}  (3),
  \mbox{\BPGS\ 614--636}.

\bibitem[\protect\BCAY{藤崎}{藤崎}{1994}]{Fuji}
藤崎博也 \BBOP 1994\BBCP.
\newblock \JBOQ
  音声の韻律的特徴における言語的・パラ言語的・非言語的情報の表出\JBCQ\
\newblock \Jem{信学技報}, \mbox{\BPGS\ HC94--09}.

\bibitem[\protect\BCAY{Fujisaki \BBA\ Nagashima}{Fujisaki \BBA\
  Nagashima}{1969}]{Fuji3}
Fujisaki, H.\BBACOMMA\ \BBA\ Nagashima, S. \BBOP 1969\BBCP.
\newblock \BBOQ A model for the synthesis of pitch contours of connected
  speech\BBCQ\
\newblock {\Bem Annual Report of the Engineering Research Institute}, {\Bbf
  28}, \mbox{\BPGS\ 53--60}.

\bibitem[\protect\BCAY{Hashizawa, Takeda, Hamzah, \BBA\ Ohyama}{Hashizawa
  et~al.}{2004}]{Hasi}
Hashizawa, Y., Takeda, S., Hamzah, M.~D., \BBA\ Ohyama, G. \BBOP 2004\BBCP.
\newblock \BBOQ On the Differences in Prosodic Features of Emotional
  Expressions in Japanese Speech according to the Degree of the Emotion\BBCQ\
\newblock {\Bem Speech Prosody 2004}, \mbox{\BPGS\ 655--658}.

\bibitem[\protect\BCAY{廣瀬\JBA 尾関\JBA 高木}{廣瀬\Jetal }{2001}]{Hir}
廣瀬幸由\JBA 尾関和彦\JBA 高木一幸 \BBOP 2001\BBCP.
\newblock \JBOQ
  日本語読み上げ文の係り受け解析における韻律的特徴量の有効性\JBCQ\
\newblock \Jem{自然言語処理}, {\Bbf 8}  (4), \mbox{\BPGS\ 71--89}.

\bibitem[\protect\BCAY{市川\JBA 中山\JBA 中田}{市川\Jetal }{1967}]{ITI}
市川熹\JBA 中山剛\JBA 中田和男 \BBOP 1967\BBCP.
\newblock \JBOQ 合成音声の自然性に関する実験的考察\JBCQ\
\newblock \Jem{音響講論(秋)}, {\Bbf 1}, \mbox{\BPGS\ 95--96}.

\bibitem[\protect\BCAY{飯田\JBA 伊賀\JBA 樋口\JBA ニック\JBA 安村}{飯田\Jetal
  }{2000}]{Ii2}
飯田朱美\JBA 伊賀聡一郎\JBA 樋口文人\JBA ニックキャンベル\JBA 安村通晃 \BBOP
  2000\BBCP.
\newblock \JBOQ 対話支援のための感情音声合成システムの試作と評価\JBCQ\
\newblock \Jem{ヒューマンインタフェース学会論文誌}, {\Bbf 2}  (2), \mbox{\BPGS\
  169--176}.

\bibitem[\protect\BCAY{Iida, Campbell, Iga, Higuchi, \BBA\ Yasumura}{Iida
  et~al.}{1998}]{Ii}
Iida, A., Campbell, N., Iga, S., Higuchi, F., \BBA\ Yasumura, M. \BBOP
  1998\BBCP.
\newblock \BBOQ Acoustic nature and perceptual testing of corpora of emotional
  speech\BBCQ\
\newblock {\Bem 5th ICSLP'98}, \mbox{\BPGS\ 1559--1562}.

\bibitem[\protect\BCAY{河井\JBA 広瀬\JBA 藤崎}{河井\Jetal }{1994}]{kawa}
河井恒\JBA 広瀬啓吉\JBA 藤崎博也 \BBOP 1994\BBCP.
\newblock \JBOQ 日本文章音声の合成のための韻律規則\JBCQ\
\newblock \Jem{音響誌}, {\Bbf 50}  (6), \mbox{\BPGS\ 433--442}.

\bibitem[\protect\BCAY{川波\JBA 広瀬}{川波\JBA 広瀬}{1997}]{Kawana}
川波弘道\JBA 広瀬啓吉 \BBOP 1997\BBCP.
\newblock \JBOQ 態度・感情音声における韻律的特徴の考察\JBCQ\
\newblock \Jem{信学技報}, \mbox{\BPGS\ SP97--67}.

\bibitem[\protect\BCAY{河津\JBA 長島\JBA 大野}{河津\Jetal }{2005}]{kw2}
河津宏美\JBA 長島大介\JBA 大野澄雄 \BBOP 2005\BBCP.
\newblock \JBOQ
  生成過程モデルパラメータに基づく感情制御規則を適用した合成音声の評価\JBCQ\
\newblock \Jem{音響講論(秋)}, {\Bbf 1}, \mbox{\BPGS\ 229--230}.

\bibitem[\protect\BCAY{北原\JBA 東倉}{北原\JBA 東倉}{1989}]{Kita}
北原義典\JBA 東倉洋一 \BBOP 1989\BBCP.
\newblock \JBOQ 音声の韻律情報と感情表現\JBCQ\
\newblock \Jem{信学技報}, \mbox{\BPGS\ SP88--158}.

\bibitem[\protect\BCAY{小林\JBA 徳田}{小林\JBA 徳田}{2004}]{Koba}
小林隆夫\JBA 徳田恵一 \BBOP 2004\BBCP.
\newblock \JBOQ コーパスベース音声合成技術の動向[4]—HMM音声合成方式—\JBCQ\
\newblock \Jem{信学誌}, {\Bbf J87}  (4), \mbox{\BPGS\ 322--327}.

\bibitem[\protect\BCAY{M.~Dzulkhiflee\JBA 武田\JBA 大山}{M.~Dzulkhiflee\Jetal
  }{2003}]{MD}
DzulkhifleeHamzah, M., 武田昌一\JBA 大山玄 \BBOP 2003\BBCP.
\newblock \JBOQ
  声優が発声した日本語「喜び」,「悲しみ」表現音声の感情の程度に応じた韻律的特
徴の比較\JBCQ\
\newblock \Jem{音響講論(秋)}, {\Bbf 1}, \mbox{\BPGS\ 367--368}.

\bibitem[\protect\BCAY{Nagasaki \BBA\ Komatsu}{Nagasaki \BBA\
  Komatsu}{2004}]{Naga}
Nagasaki, Y.\BBACOMMA\ \BBA\ Komatsu, T. \BBOP 2004\BBCP.
\newblock \BBOQ Can People Perceive Different Emotions from a Non-emotional
      Voice by Modifying its F0 and Duration?''\
\newblock {\Bem Speech Prosody 2004}, \mbox{\BPGS\ 667--670}.

\bibitem[\protect\BCAY{長島\JBA 大野}{長島\JBA 大野}{2004}]{Nsima}
長島大介\JBA 大野澄雄 \BBOP 2004\BBCP.
\newblock \JBOQ 感情表現の韻律的特徴の分析—喜びと悲しみについて\JBCQ\
\newblock \Jem{音響講論(秋)}, {\Bbf 1}, \mbox{\BPGS\ 273--274}.

\bibitem[\protect\BCAY{ニック}{ニック}{1997}]{Nikku}
ニックキャンベル \BBOP 1997\BBCP.
\newblock \Jem{プラグマティック・イントネーション:韻律情報の機能的役割}.
\newblock くろしお出版.

\bibitem[\protect\BCAY{Paeschke}{Paeschke}{2004}]{As}
Paeschke, A. \BBOP 2004\BBCP.
\newblock \BBOQ Global Trend of Fundamental Frequency in Emotional Speech\BBCQ\
\newblock {\Bem Speech Prosody 2004}, \mbox{\BPGS\ 671--674}.

\bibitem[\protect\BCAY{Plutchik}{Plutchik}{1980}]{PLU}
Plutchik, R. \BBOP 1980\BBCP.
\newblock \BBOQ Emotion---A Psychoevolutionary Synthesis\BBCQ\
\newblock {\Bem Harper and Row}.

\bibitem[\protect\BCAY{匂坂\JBA ニック}{匂坂\JBA ニック}{2000}]{Sagi}
匂坂芳典\JBA ニックキャンベル \BBOP 2000\BBCP.
\newblock \JBOQ 音声合成のための規則とデータの表現, 獲得, 評価\JBCQ\
\newblock \Jem{信学論(D-2)}, {\Bbf J83-D-2}  (11), \mbox{\BPGS\ 2068--2076}.

\bibitem[\protect\BCAY{重永}{重永}{2000}]{Sige}
重永實 \BBOP 2000\BBCP.
\newblock \JBOQ 感情判別分析からみた感情音声の特性\JBCQ\
\newblock \Jem{信学論(A)}, {\Bbf J83-A}  (6), \mbox{\BPGS\ 726--735}.

\bibitem[\protect\BCAY{梅村\JBA 原田\JBA 清水\JBA 杉本}{梅村\Jetal
  }{2000}]{Ume}
梅村祥之\JBA 原田義久\JBA 清水司\JBA 杉本軍司 \BBOP 2000\BBCP.
\newblock \JBOQ
  音声合成におけるポーズ制御のための決定リストを用いた局所係り受け解析\JBCQ\
\newblock \Jem{自然言語処理}, {\Bbf 7}  (5), \mbox{\BPGS\ 51--70}.

\end{thebibliography}
}


\begin{biography}

\bioauthor{河津 宏美}{
2002年東京工科大学工学部情報通信工学科卒業.同年,同大工学部助手.2005年東京工科大学大学院バイオ・情報メディア研究科コンピュータサイエンス専攻.現在に至る.韻律的特徴の分析・合成に関する研究に従事.電子情報通信学会,日本音響学会,日本シミュレーション学会各会員.}

\bioauthor{大野 澄雄}{
1988年東京大学工学部電気工学科卒.1993年同大大学院工学系研究科電子工学専攻博士課程了.同年,東京理科大学基礎工学部助手.1999年東京工科大学工学部講師.現在,同大コンピュータサイエンス学部助教授.音声言語処理,特に音声の韻律の分析・合成・認識処理の研究に従事.博士(工学).電子情報通信学会,日本音響学会各会員.
}
\end{biography}







\biodate

\end{document}
