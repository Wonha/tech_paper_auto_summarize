    \documentclass[english]{jnlp_1.3e}
\usepackage{jnlpbbl_1.1}
\usepackage[dvips]{graphicx}
\usepackage{hangcaption_jnlp}


\Volume{14}
\Number{4}
\Month{July}
\Year{2007}

\received{2006}{4}{17}
\revised{2006}{8}{9}
\rerevised{2007}{1}{15}
\accepted{2007}{3}{20}

\setcounter{page}{83}

\etitle{Emotion Recognition using Mel-Frequency \\Cepstral Coefficients}
\eauthor{
	Nobuo Sato\affiref{ARL}  \and 
	Yasunari Obuchi\affiref{CRL}} 
\eabstract{
In this paper, we propose a new approach to emotion recognition. Prosodic features are currently used in most emotion recognition algorithms. However, emotion recognition algorithms using prosodic features are not sufficiently accurate. Therefore, we focused on the phonetic features of speech for emotion recognition. In particular, we describe the effectiveness of Mel-frequency Cepstral Coefficients (MFCCs) as the feature for emotion recognition. We focus on the precise classification of MFCC feature vectors, rather than their dynamic nature over an utterance. To realize such an approach, the proposed algorithm employs multi-template emotion classification of the analysis frames. Experimental evaluations show that the proposed algorithm produces 66.4\% recognition accuracy in speaker-independent emotion recognition experiments for four specific emotions. This recognition accuracy is higher than the accuracy obtained by the conventional prosody-based and MFCC-based emotion recognition algorithms, which confirms the potential of the proposed algorithm.
}
\ekeywords{Emotion Recognition, Speech Processing, Human Interface}


\headauthor{Sato, Obuchi}
\headtitle{Emotion Recognition using Mel-Frequency Cepstral Coefficients}

\affilabel{ARL}{}{Advanced Research Laboratory, Hitachi, Ltd.}
\affilabel{CRL}{}{Central Research Laboratory, Hitachi, Ltd.}


\begin{document}

\maketitle


    \let\thline

\section{Introduction}

Emotion recognition of speech has gained increasing attention in recent years \cite{Cowie2001}. Emotion recognition is a procedure that converts a human's voice into an emotional symbol, such as anger, sadness, or happiness. Computers can flexibly react to the user, using this symbolized information. Applications of emotion recognition include a speech dialog system, a call center system, and a security system. If the system's reactions change depending on the user's feelings, the man-machine interface would become more friendly and easier to use. Since it is difficult even for human to discriminate emotions from speech, there are some applications in which emotion recognition helps, if a certain level of accuracy can be achieved.

There have been many studies on emotion recognition. Most of them use prosodic information as their feature parameters \cite{Tato2002,Yacoub2003,Oudeyer2003,Yu2001,Kwon2003,Schuller2003}. However, the accuracy of emotion recognition using prosodic information is low. In particular, the accuracy falls below 50\% in most speaker-independent emotion recognition systems for four (or more) emotions.

Feature extraction and classification are two important modules of emotion recognition. Many classification techniques have been proposed, such as neural network (NN) \cite{Tato2002,Yacoub2003,Oudeyer2003}, decision tree \cite{Oudeyer2003}, k-nearest neighbor (K-NN) \cite{Oudeyer2003}, support vector machine (SVM) \cite{Yacoub2003,Oudeyer2003,Yu2001}, discriminant analysis \cite{Kwon2003}, and hidden markov model (HMM) \cite{Kwon2003,Schuller2003}. However, the difference between the recognition rates of these techniques is rather small, and therefore we conclude that it is more important to find the best feature parameters for emotion recognition.

Prosodic features are currently used in most emotion recognition systems. It is commonly thought that the prosodic features of speech have useful information for discriminating emotions. Prosodic features are made of fundamental frequency and energy, which means that there are only two independent components in each frame, even though a number of utterance-level variables (minimum, maximum, average, and regression coefficients, etc) can be derived. We think that this is a part of the reason why emotion recognition algorithms that use prosodic features are not sufficiently accurate. However, it is said that phonetic features have less information for discriminating emotions. Actually, there are more independent components in the phonetic features of speech than in the prosodic features of speech. For example, 12--16 dimensional Mel-Frequency Cepstral Coefficients (MFCCs) have been used as the effective phonetic features for speech recognition. If even a small amount of useful information is kept in the phonetic features, the accuracy of emotion recognition can be improved by increasing the number of independent phonetic features.

We describe the effectiveness of MFCC as the feature for emotion recognitions. Various algorithms of emotion recognition using MFCC have been proposed. The algorithms proposed by Kwon \cite{Kwon2003} and by Schuller \cite{Schuller2005} are both based on the utterance-level features such as the maximum, average, variance, etc. Even though these features are made from MFCCs, the detailed structure of the frame-level features is omitted. Contrastingly, Nwe et al. used the frame-level MFCCs without summarizing them over the utterance \cite{Nwe2003}. Their algorithm yields an average accuracy of 78\% in speaker-dependent emotion recognition for six specific emotions. They use discrete HMMs to deal with the dynamics of the input feature, but the precise information of the MFCCs was lost in the clustering process of each frame.

Therefore, we propose an emotion recognition algorithm that focuses more on precise classification of the MFCCs. To realize such a precise classification, we give the emotion label to each frame using multi-template MFCC clustering. The proposed algorithm is simple enough to realize immediate response even in a low-end computer, as well as the higher accuracy than the conventional method.

Section 2 describes the emotion recognition algorithm that we propose. In Section 3, the performance evaluation and a comparison between the proposed and conventional prosody-based and MFCC-based algorithms are described. The concluding remarks are presented in Section 4.



\section{Emotion Recognition Algorithm}

The proposed algorithm is composed of three modules (Feature extraction, Frame-level classification, and Utterance-level voting) and one database. The processing flow is shown in Figure \ref{fig01}. First, the voice data is divided into analysis frames. Next, each analysis frame is converted to a feature vector. Then, an appropriate emotion label is attached to each feature vector. Finally, the emotion labels of the entire utterance are collected and the emotion of the utterance is determined.

\begin{figure}[b]
\begin{center}
    \includegraphics{14-4ia5f1.eps}
\caption{Emotion recognition flowchart.}
\label{fig01}
\end{center}
\end{figure}
\begin{figure}[b]
\begin{center}
    \includegraphics{14-4ia5f2.eps}
\caption{Feature extraction flowchart.}
\label{fig02}
\end{center}
\end{figure}



\subsection{Feature Extraction}

We use Mel-Frequency Cepstral Coefficient (MFCC) \cite{Davis1980} as the main feature for emotion recognition. Figure \ref{fig02} shows a feature extraction flowchart using MFCC. First, the voice data is divided into frames. Each frame is windowed using a Hamming window. Second, the analysis frame is converted to the frequency domain using a short time Fourier transform. Third, a certain number of sub-band energies are calculated using a mel filterbank, which is a nonlinear-scale filterbank that imitates a human's aural system. Fourth, the logarithm of the sub-band energies is calculated. Finally, the MFCC is computed by an inverse Fourier transform. In the proposed algorithm, we use 28 mel filters and 16 MFCCs.


\subsection{Frame-level Classification}

\begin{figure}[b]
\begin{center}
    \includegraphics{14-4ia5f3.eps}
\caption{Illustration of distance between inputs and codewords in feature space.}
\label{fig03}
\end{center}
\end{figure}
\begin{figure}[b]
\begin{center}
    \includegraphics{14-4ia5f4.eps}
\caption{Codebook training flowchart.}
\label{fig04}
\end{center}
\end{figure}

In this section we will attempt to describe the method for classifying analysis frames. Each emotion is expressed by a codebook, and each codeword is represented as a vector in the feature space. When we have an input feature vector, we calculate the distance between the input and all the codewords. Finally, the emotion label of the nearest codeword becomes the classification result of the analysis frame. To avoid the effect of a different scaling of dimensions, we use a Mahalanobis-generalized distance instead of a Euclidean distance. Figure \ref{fig03} illustrates how we calculated the distances between the input and all the codewords in the featured space.

Next, we will describe the codebook training procedure. This database stores the codebooks for the target emotions. The training procedure is illustrated in Figure \ref{fig04}. First, the training data is classified by emotion. Next, the training data is divided into analysis frames, and then converted to feature vectors. Then, all the feature vectors are collected. Finally, the codebook is generated by clustering these feature vectors together using the LBG algorithm \cite{Linde1980}.


\subsection{Utterance-level Voting}

In this process, emotion of one utterance is decided by voting of frame-level emotion labels. An emotion with the largest number of votes becomes the emotion label of the utterance. Figure 5 illustrates the voting for the emotions of hot anger and neutral within one utterance. In this example, ``hot anger'' has 39 votes and ``neutral'' has 11 votes (Figure \ref{fig05} (1)).

\begin{figure}[b]
\begin{center}
    \includegraphics{14-4ia5f5.eps}
\caption{Example of utterance-level voting.}
\label{fig05}
\end{center}
\end{figure}

A modification was added to reduce the count of incorrect votes. We observed that the labeling result is made of some long queues of a specific label and additional short spots of other labels as shown in Figure \ref{fig05} (1). We assumed that the reliability of the labeling is high in the long queues and low in the short spots. Therefore, we used only the labels of the frames in the long queues to decide the label of the utterance. In the proposed algorithm, a threshold $L$ was introduced, and the label $e_i$ (of the frame $i$) was included in the vote only if $\{e_k|k=i-L+1,\ldots,i-1,i\}$  are all the same. This processing is called ``utterance-level smoothing''. Figure \ref{fig05} (2) shows an example of $L=4$.


\section{Evaluation Experiments}

Two sets of experiments were conducted to evaluate the performance of the proposed algorithm. The first experiment was for the recognition of the two emotions (hot anger, and neutral), and the second one was for the recognition of four emotions (hot anger, neutral, sadness, and happiness). The conditions for the experimental analysis are shown in Table \ref{tab01}.

\begin{table}[b]
\input{table1.txt}
\end{table} 
\begin{figure}[b]
\begin{center}
    \includegraphics{14-4ia5f6.eps}
\caption{Accuracy of two- and four-emotion experiments, according to number of codewords.}
\label{fig06}
\end{center}
\end{figure}


In our experiments, we used the emotional speech database from the Linguistic Data Consortium \cite{LDC2002}. Each utterance corresponds to one emotion. One utterance is composed of 3--4 words, either numbers (``two thousand one''; ``four hundred ten'') or dates (``September ninth''; ``December tenth''). The utterances were spoken by seven actors (three males and four females), mostly in their mid-20's.

The experiments were conducted in a speaker-independent manner to avoid the effect of individuality. The utterances of the six actors were used for training, and the utterances of one actor were used for the evaluation. The sessions were repeated seven times by switching their roles. The average of the seven sessions was calculated to obtain the final recognition accuracy.


\subsection{Performance Evaluation}

\begin{table}[b]
\input{table2.txt}
\end{table} 
\begin{table}[b]
\input{table3.txt}
\end{table}


The results of the two- and four-emotion experiments are shown in Figure \ref{fig06}. The vertical axis represents the accuracy, and the horizontal axis represents the codebook size for each emotion. In the two-emotion experiments, a high accuracy was obtained even with very small codebooks. In the four-emotion experiments, the accuracy was low with the small codebooks, but it improved with the increase in the number of codewords up to 64, where the performance became saturated.

Table \ref{tab02} shows the confusion matrices for the experiments with 64 codewords. The column is the emotion included in the utterance. The row is the recognition result by the proposed algorithm. Two-emotion utterances are accurately recognized as shown in Table \ref{tab02} (A). The accuracy of the two emotions is 98.8\%. There is little difference in the accuracy between hot anger and neutral. However, the accuracies for the four-emotions are not even, as shown Table \ref{tab02} (B). In particular, the accuracies of sadness and happiness are quite low. The accuracy of the four emotions is 66.4\%.


\begin{table}[b]
\vspace{-\baselineskip}
\input{table4.txt}
\end{table} 


Table \ref{tab03} and Table \ref{tab04} shows confusion matrices of accuracy of each speaker. We confirmed the difference of the recognition rate was small in the two-emotion experiment (Table \ref{tab03} (A)). However, in the four-emotion experiment, the difference of the recognition rate was rather large (Table \ref{tab04} (B)).

\begin{table}[p]
\input{table5.txt}
\vspace{2\baselineskip}
\input{table6.txt}
\end{table} 


\subsection{Comparison between Proposed and Conventional Prosody-based Algorithm}

The proposed algorithm and the conventional algorithm were compared to evaluate the performance of the proposed algorithm. The proposed algorithm was implemented as described in the previous section. That is, the LBG algorithm was used for the training of the emotion clusters, and k-nearest neighbor algorithm was used for the emotion recognition.

We referred to Tato et al. \cite{Tato2002} to implement the most well-known emotion recognition algorithm. We extracted 43 prosodic features from voiced/unvoiced speech, fundamental frequency, jitter, and energy (Table \ref{tab05}). The normalized prosodic features were used for this examination.


The evaluation results are shown in Table \ref{tab06}. It was confirmed that the accuracy of the proposed algorithm was higher than the prosody-based algorithm. In particular, the accuracy of the four discrete emotions was improved by 16.5 points. From these results, it was confirmed that the feature of this proposed algorithm was more effective than the prosodic feature.


\subsection{Comparison between Proposed and Conventional MFCC-based Algorithms}

We compare the recognition rates of our proposed algorithm and the conventional MFCC-based algorithm. We referred to Nwe et al. \cite{Nwe2003} to implement the MFCC-based emotion recognition algorithm. In Nwe, et al. \cite{Nwe2003}, they use MFCC and log frequency power coefficients (LFPCs) to represent the speech signals and a four-state discrete ergodic HMM as the classifier.

We use Hidden Markov Model Toolkit (HTK) \cite{Young2002} to implement the Nwe's algorithm. However, HTK does not support discrete ergodic HMM. Instead, we evaluate various topologies of left-to-right HMM. If the variation of the accuracy among various topologies is small and the accuracy of the proposed algorithm is a higher than that of the best topology of the left-to-right HMM, the advantage of the proposed algorithm would be confirmed.

To evaluate the conventional algorithm, we had made two types of HMM as shown in Figure 7. The first one is the HMM with skips between states (Figure \ref{fig07} (a)). In this type of HMM, any forward transition is permitted. The second one is the HMM without skips (Figure \ref{fig07} (b)). Moreover, the number of states of HMM was varied (1, 2, 4, and 8). 

\begin{figure}[t]
\begin{center}
    \includegraphics{14-4ia5f7.eps}
\hangcaption{Four-state left-to-right HMMs:(a) HMM with the skip between states, (b) HMM without the skip between states.}
\label{fig07}
\end{center}
\end{figure}
\begin{figure}[t]
\begin{center}
    \includegraphics{14-4ia5f8.eps}
\caption{Comparison between proposed and conventional MFCC-based algorithms.}
\label{fig08}
\end{center}
\end{figure}


The experimental results are shown in Figure \ref{fig08}. The vertical axis represents the recognition accuracy, and the horizontal axis represents number of HMM states. HMM (Skip) corresponds to the results obtained with HMM with skips. HMM (NoSkip) corresponds to the result obtained with HMM without skips. The recognition accuracy of the proposed algorithm (Proposed) is also shown by the horizontal line for reference.

We confirmed that the accuracy of the proposed algorithm was higher than the conventional MFCC-based algorithm. In particular, the accuracy of the four-emotions was improved by 4 points. The accuracy of the conventional MFCC-based algorithm has not changed much through the experiments with various topologies.

\subsection{Effective Features of MFCC in Emotion Recognition}

We investigated the effect of each feature of the MFCC feature vector. Recognition that used only one feature was evaluated. In this experiment, the codebook was fixed at 64 words. The recognition rate of each feature element is shown in Figure \ref{fig09}. The vertical axis represents the accuracy, and the horizontal axis represents the features. The accuracy of MFCCs is higher than that of  MFCCs. In particular, the accuracy of a low-dimensional MFCC is high. The 1st MFCC exceeds 80\% accuracy.

\begin{figure}[t]
\begin{center}
    \includegraphics{14-4ia5f9.eps}
\caption{Accuracy of two-emotion experiments, according to features.}
\label{fig09}
\end{center}
\end{figure}


\section{Conclusion}

We reported a new approach to emotion recognition. We proposed an emotion recognition algorithm using MFCC. Evaluation experiments showed that the proposed algorithm produces 66.4\% recognition accuracy in speaker-independent emotion recognition experiments for four emotions (hot anger, neutral, sadness, and happiness). This recognition accuracy was higher than the accuracy of the conventional prosody-based and MFCC-based emotion recognition algorithms, which confirmed the potential of the proposed algorithm. The accuracy of 66.4\% is not high enough for general use, but the improvement would make some existing applications more effective. However, we are far from satisfied with the accuracy of the proposed algorithm. Further study is needed to explore additional features for classifying more emotions, and developing an improved emotion recognition algorithm using these features.






\acknowledgment

We would like to thank Mr. Moriwaki, N., Mr. Horry, Y., Dr. Yano, K. 
\pagebreak
and Dr. Osakabe, N., Advanced Research Laboratory, Hitachi, Ltd., and Mr. Masui, 
S., Hitachi (China) Research \& Development Corporation for their support during this research.

\bibliographystyle{jnlpbbl_1.3}
\begin{thebibliography}{}

\bibitem[\protect\BCAY{Cowie, Douglas-Cowie, Tsapatsoulis, Votsis, Kollias,
  Fellenz, \BBA\ Taylor}{Cowie et~al.}{2001}]{Cowie2001}
Cowie, R., Douglas-Cowie, E., Tsapatsoulis, N., Votsis, G., Kollias, S.,
  Fellenz, W., \BBA\ Taylor, J. \BBOP 2001\BBCP.
\newblock \BBOQ Emotion recognition in human-computer interaction\BBCQ\
\newblock {\Bem IEEE Signal Processing Magazine}, {\Bbf 18}  (1), \mbox{\BPGS\
  80--82}.

\bibitem[\protect\BCAY{Davis \BBA\ Mermelstein}{Davis \BBA\
  Mermelstein}{1980}]{Davis1980}
Davis, S.~P.\BBACOMMA\ \BBA\ Mermelstein, P. \BBOP 1980\BBCP.
\newblock \BBOQ Comparison of parametric representations for monosyllabic word
  recognition in continuously spoken sentences\BBCQ\
\newblock {\Bem IEEE Transaction on Acoustics, Speech, and Signal Processing},
  {\Bbf ASSP-28}  (4), \mbox{\BPGS\ 357--366}.

\bibitem[\protect\BCAY{Kwon, Chan, Hao, \BBA\ Lee}{Kwon
  et~al.}{2003}]{Kwon2003}
Kwon, O.~W., Chan, K., Hao, J., \BBA\ Lee, T.~W. \BBOP 2003\BBCP.
\newblock \BBOQ Emotion Recognition by Speech Signals\BBCQ\
\newblock In {\Bem Proceedings of 8th European Conference on Speech
  Communication and Technology (EUROSPEECH-2003)}, \mbox{\BPGS\ 125--128}\
  Geneva, Switzerland.

\bibitem[\protect\BCAY{Liberman, Davis, Grossman, Martey, \BBA\ Bell}{Liberman
  et~al.}{2002}]{LDC2002}
Liberman, M., Davis, K., Grossman, K., Martey, N., \BBA\ Bell, J. \BBOP
  2002\BBCP.
\newblock \BBOQ Emotional Prosody Speech and Transcripts\BBCQ.
\newblock http://www.ldc.upenn.edu/Catalog/CatalogEntry.jsp ?catalogId=
  LDC2002S28.

\bibitem[\protect\BCAY{Linde, Buzo, \BBA\ Gray}{Linde et~al.}{1980}]{Linde1980}
Linde, Y., Buzo, A., \BBA\ Gray, R.~M. \BBOP 1980\BBCP.
\newblock \BBOQ An algorithm for vector quantizer design\BBCQ\
\newblock {\Bem IEEE Transaction on Communications}, {\Bbf COM-28}  (1),
  \mbox{\BPGS\ 84--95}.

\bibitem[\protect\BCAY{Nwe, Foo, \BBA\ De~Silva}{Nwe et~al.}{2003}]{Nwe2003}
Nwe, T.~L., Foo, S.~W., \BBA\ De~Silva, L.~C. \BBOP 2003\BBCP.
\newblock \BBOQ Speech emotion recognition using Hidden Markov Models\BBCQ\
\newblock {\Bem Speech Communication}, {\Bbf 41}  (4), \mbox{\BPGS\ 603--623}.

\bibitem[\protect\BCAY{Oudeyer}{Oudeyer}{2003}]{Oudeyer2003}
Oudeyer, P.~Y. \BBOP 2003\BBCP.
\newblock \BBOQ The production and recognition of emotions in speech: features
  and algorithms\BBCQ\
\newblock {\Bem International Journal of Human Computer Interaction}, {\Bbf 59}
   (1--2), \mbox{\BPGS\ 157--183}.

\bibitem[\protect\BCAY{Schuller, Muller, Lang, \BBA\ Rigoll}{Schuller
  et~al.}{2005}]{Schuller2005}
Schuller, B., Muller, R., Lang, M., \BBA\ Rigoll, G. \BBOP 2005\BBCP.
\newblock \BBOQ Speaker independent emotion recognition by early fusion of
  acoustic and linguistic features within ensembles\BBCQ\
\newblock In {\Bem Proceedings of Interspeech 2005}, \mbox{\BPGS\ 805--809}\
  Lisbon, Portugal.

\bibitem[\protect\BCAY{Schuller, Rigoll, \BBA\ Lang}{Schuller
  et~al.}{2003}]{Schuller2003}
Schuller, B., Rigoll, G., \BBA\ Lang, M. \BBOP 2003\BBCP.
\newblock \BBOQ Hidden Markov Model-based Speech Emotion Recognition\BBCQ\
\newblock In {\Bem Processing of The 2003 IEEE International Conference on
  Acoustics, Speech, and Signal Processing (ICASSP 2003)},
  \lowercase{\BVOL}~II, \mbox{\BPGS\ 1--4}\ Hong Kong, China.

\bibitem[\protect\BCAY{Tato, Santos, Kompe, \BBA\ Pardo}{Tato
  et~al.}{2002}]{Tato2002}
Tato, R., Santos, R., Kompe, R., \BBA\ Pardo, J. \BBOP 2002\BBCP.
\newblock \BBOQ Emotional space improves emotion recognition\BBCQ\
\newblock In {\Bem Proceedings of 7th International Conference on Spoken
  Language Processing (ICSLP 2002)}, \lowercase{\BVOL}~3, \mbox{\BPGS\
  2029--2032}\ Denver, USA.

\bibitem[\protect\BCAY{Yacoub, Simske, Lin, \BBA\ Burns}{Yacoub
  et~al.}{2003}]{Yacoub2003}
Yacoub, S., Simske, S., Lin, X., \BBA\ Burns, J. \BBOP 2003\BBCP.
\newblock \BBOQ Recognition of emotions in interactive voice response
  systems\BBCQ\
\newblock In {\Bem Proceedings of 8th European Conference on Speech
  Communication and Technology (EUROSPEECH-2003)}, \mbox{\BPGS\ 729--732}\
  Geneva, Switzerland.

\bibitem[\protect\BCAY{Young}{Young}{2002}]{Young2002}
Young, S. \BBOP 2002\BBCP.
\newblock \BBOQ The HTK book (for HTK version 3.2)\BBCQ.
\newblock http://htk.eng.cam.ac.uk/ docs/docs.shtml.

\bibitem[\protect\BCAY{Yu, Chang, Xu, \BBA\ Shum}{Yu et~al.}{2001}]{Yu2001}
Yu, F., Chang, E., Xu, Y.~Q., \BBA\ Shum, H.~Y. \BBOP 2001\BBCP.
\newblock \BBOQ Emotion Detection from Speech to Enrich Multimedia
  Content\BBCQ\
\newblock In {\Bem Proceedings of the Second IEEE Pacific Rim Conference on
  Multimedia}, \mbox{\BPGS\ 550--557}\ Beijing, China.

\end{thebibliography}


\vspace{-0.5\baselineskip}
\begin{biography}

\bioauthor[:]{Nobuo Sato}
{Nobuo Sato received the B.S., M.S., and Ph.D degrees in computer science and engineering
from University of Aizu, Fukushima, Japan, in 1997, 1999, and 2005, respectively.
From 1999 to 2002, he worked as a part-time lecturer in Junior College of Aizu, Fukushima, Japan.
From 2002 to 2003, he worked for Central Research Laboratory, Hitachi, Ltd., Tokyo, Japan.
Since 2003, he has been working at Advanced Research Laboratory, Hitachi, Ltd., Saitama, Japan.
He is a member of the Acoustical Society of Japan,
the Information Processing Society of Japan, 
and the Institute of Electronics, Information and Communication Engineers.
His current research interests include speech processing,
signal processing, and sensornet.
}

\bioauthor[:]{Yasunari Obuchi}
{Yasunari Obuchi received the B.S. degree and the M.S. degree in physics in 1988 and
1990 respectively, and the Ph.D. in information science and technology
in 2006, all from the University of Tokyo. Since 1992, he has been
working in Central Research Laboratory and Advanced Research
Laboratory, Hitachi Ltd. He worked in Carnegie Mellon University as a
Visiting Researcher from 2002 to 2003. He has also been a Visiting
Researcher of Waseda University from 2005 to the present. Currently he
is a Senior Researcher of Central Research Laboratory, Hitachi Ltd.
His research interests include robust speech recognition, spoken
dialog systems, speech recognition in small devices, speech-to-speech
translation, language identification, and emotion recognition. Dr.
Obuchi was a co-recipient of the Technology Development Award of the
Acoustical Society of Japan in 2000. He is a member of IEEE, ISCA,
IEICE, IPSJ, and ASJ.
}
\end{biography}

\biodate


\end{document}
