    \documentclass[english]{jnlp_1.3c}
\usepackage{jnlpbbl_1.1}
\usepackage[dvips]{graphicx}
\usepackage{url}
\usepackage{hangcaption_jnlp}
\setlength{\captionwidth}{\textwidth}

\Volume{14}
\Number{2}
\Month{Apr.}
\Year{2007}

\received{2006}{9}{20}
\revised{2006}{12}{1}
\accepted{2007}{1}{4}

\setcounter{page}{95}

\etitle{Indonesian-Japanese Transitive Translation \\using English for CLIR}
\eauthor{
  Ayu Purwarianti\affiref{tutics} \and
  Masatoshi Tsuchiya\affiref{tutimc} \and
  Seiichi Nakagawa\affiref{tutics}} 
\eabstract{
We propose a query transitive translation system of a CLIR
(Cross Language Information Retrieval) for a source language with a poor
data resource. Our research aim is to do the transitive translation with
a minimum data resource of the source language (Indonesian) and exploit
the data resource of the target language (Japanese). We did two kinds of
translation, a pure transitive translation and a combination of direct
and transitive translations. In the transitive translation, English is
used as the pivot language. The translation consists of two main
steps. The first is a keyword translation process which attempts to make
a translation based on available resources. The keyword translation
process involves many target language resources such as the Japanese
proper name dictionary and English-Japanese (pivot-target language)
bilingual dictionary. The second step is a process to select some of the
best available translations. We combined the mutual information score
(computed from target language corpus) and TF×IDF score in order to
select the best translation. The result on NTCIR 3 (NII-NACSIS Test
Collection for IR Systems) Web Retrieval Task showed that the
translation method achieved a higher IR score than the machine
translation (using Kataku (Indonesian-English) and Babelfish/Excite
(English-Japanese) engines). The transitive translation achieved about
38\% of the monolingual retrieval, and the combination of direct and
transitive translation achieved about 49\% of the monolingual retrieval
which is comparable to the English-Japanese IR task.
}
\ekeywords{Transitive Translation, Bilingual Dictionary, Limited Resource Language, Cross Language Information Retrieval, Indonesian-Japanese}

\headauthor{Purwarianti, Tsuchiya, Nakagawa}
\headtitle{Indonesian-Japanese Transitive Translation for CLIR}

\affilabel{tutics}{}{Department of Information and Computer Sciences, Toyohashi University}
\affilabel{tutimc}{}{Information and Media Center, Toyohashi University of Technology}


\begin{document}

\maketitle





\section{Introduction}
\label{Intoroduction}

Nowadays, there are many web resources on the Internet written in
languages other than English. CLIR (Cross Language Information
Retrieval) has served as a bridge for Internet sources and users with
different languages. Indonesia, with a population of 241 million people,
also has an interest in utilizing the CLIR. But, unfortunately,
Indonesian is a language with minimum data resources. Here, we found
that there is a need to build a CLIR for language with minimum data
resources such as the Indonesian language. In order to do this kind of
translation, transitive translation with bilingual dictionaries has been
an alternative method. Even though there are other data resources
available such as machine translation or parallel corpus, an electronic
bilingual dictionary is the most widely available. Some studies have
been done in the field of transitive translation with bilingual
dictionaries in the CLIR system such as
\cite{ballesteros2000,gollins2001}. \cite{ballesteros2000} translated
Spanish queries into French with English as the interlingua. Ballesteros
used Collins Spanish-English and English-French
dictionaries. \cite{gollins2001} translated the Germany queries into
English using two pivot languages (Spanish and Dutch), while Gollins
used the Euro Wordnet as a data resource. To our knowledge, no CLIR is
available with transitive translation for a source language with a
minimum data resource such as the Indonesian one.

Translation with a bilingual dictionary usually provides many
translation alternatives, only a few of which are appropriate. A
transitive translation gives more translation alternatives than a direct
translation. In order to select the most appropriate translation, a
monolingual corpus can be used. \cite{ballesteros1998} used an English
corpus to select some English translation based on Spanish-English
translation and analyzed the co-occurrence frequencies to disambiguate
phrase translations. The occurrence score is called the \textit{em} score. Each
set is ranked by an \textit{em} score, and the highest ranking set is taken as
the final translation.  \cite{gao2001} used a Chinese corpus to select
the best English-Chinese translation set. They modified the EMMI
weighting measure to calculate the term coherence score. \cite{qu2002}
selected the best Spanish-English and Chinese-English translation using
the English corpus. The coherence score calculation was based on 1) web
page count; 2) retrieval score; and 3) mutual information score.
\cite{mirna2000} translated Indonesian into English and used an English
monolingual corpus to select the best translation, employing a term
similarity score based on the Dice similarity
coefficient. \cite{federico2002} combined the N-best translation based
on an HMM model of a query translation pair and relevant document
probability of the input word to rank Italian documents retrieved by
English query.  \cite{kishida2004} used all terms to retrieve a document
in order to obtain the best term combination, and chose the most
frequent term in each term translation set that appears in the
top-ranked document.

Here, we translated Indonesian queries into a Japanese keyword list in
order to retrieve Japanese documents. Because of resource limitations
between Indonesian and Japanese, we conducted a transitive translation
with English as the pivot language. Even though there are alternatives
to use machine translation or parallel corpus in the transitive
translation, we prefer to employ a bilingual dictionary as the most
available resource. To filter the translation results, we combined the
TF×IDF engine score and the mutual information score (taken from a
monolingual target language corpus) to select the most appropriate
translation.

Another problem in the translation with bilingual dictionaries is
out-of-vocabulary (OOV) words. This problem becomes critical if the OOV
words are proper nouns which usually are important keywords in the
query. Therefore, if the proper noun keywords are not translated, the IR
system will return almost no relevant document. We found that some OOV
words that are not available in the Indonesian dictionary are borrowed
words. In our Indonesian-Japanese CLIR, the borrowed words come from the
English and Japanese languages. Therefore, in order to translate these
OOV, we used the English-Japanese dictionary and Japanese proper name
dictionary.

The experiments were done on NTCIR 3 Web Retrieval
Task data \cite{NTCIR3WEB}.  We compared the bilingual dictionary-based transitive
translation with the machine transitive translation and direct
translation using the available Indonesian-Japanese bilingual
dictionary.  We also compared our proposed keyword selection method with
one based only on the mutual information score.

The rest of the paper is organized as an overview of the Indonesian
query, Indonesian-Japanese keyword translation system, the keyword
selection method using mutual information and IR engine score, the
comparison methods, experiments and conclusions.

\section{Overview of indonesian Query}
\label{Overview of indonesian Query}

Indonesian is the official language of Indonesia. It is understood by
people in Indonesia, Malaysia, and Brunei. The Indonesian language
family is Malayo-Polynesian (Austronesian), which extends across the
islands of Southeast Asia and the Pacific.  Indonesian is not related to
either English or Japanese.

Unlike other languages used in Indonesia such as Javanese, Sundanese and
Balinese that use their own scripts, Indonesian uses the familiar Roman
script. It uses only 26 letters as in the English alphabet. A
transliteration module is not needed to translate an Indonesian
sentence.

Indonesian is quite a simple language. For example, it does not have
declensions or conjugations. The basic word order is SVO
(Subject-Verb-Object). Verbs are not inflected for person or
number. There are no tenses. Tense is denoted by the time adverb or some
other tense indicator. The time adverb can be placed at the front or end
of the sentence.

A rather complex characteristic of the Indonesian language is that it is
an agglutinative language. Words in Indonesian, usually verbs, can be
attached by many prefixes or suffixes. Affixes used in Indonesian
language include \cite{kosasih2003} me(n)-, ber-, di-, ter-, pe(n)-, per-,
se-, ke-, -el-, -em-, -er-, -kan, -i, -n ya, -an, me(n)-kan, di-kan,
memper-i, diper-i, ke-an, pe(n)-an, per-an, ber-an, ber-kan,
se-nya. Words with different affixes might have a different translation
but also might have the same translation. Examples of a different word
translation include ``membaca'' and ``pembaca'', which are translated
``read'' and ``reader'', respectively. Examples of the same word
translation are the word ``baca'' and ``bacakan'', which are both
translated into ``read'' in English. Other examples are the word
``membaca'' and ``dibaca'', which are translated into ``read'' and
``being read'', respectively. Using a stop word elimination, the
translation result of ``membaca'' and ``dibaca'' will give the same
English translation, ``read''.

An Indonesian dictionary usually contains words with affixes (that have
different translations) and the base words. For example, the ``se-nya''
affix declares a ``most possible'' pattern, such as ``sebanyak-banyaknya''
(as much as possible), ``sesedikit-sedikitnya'' (as little as
possible), and ``sehitam-sehitamnya'' (as black as possible). This affix
can be attached to many adjectives with the same meaning
pattern. Therefore, words with the ``se-nya'' affix usually are not
included in an Indonesian dictionary.

\begin{table}[b]
  \caption{Indonesian Query Examples}
  \label{table:Indonesian Query Examples}
  \begin{center}
    \begin{tabular}{|p{0.9\textwidth}|}\hline
     Query 1 \\
     Saya ingin mengetahui siapa yang telah menjadi peraih {\it Academy Awards}
     beberapa generasi secara berturut-turut \\
     (I want to know who have been the recipients of successive
     generations of Academy Awards) \\ \hline
     Query 2 \\
     Temukan buku-buku yang mengulas tentang novel yang ditulis oleh
     {\it Miyabe Miyuki} \\
     (Find book reviews of novels written by Miyabe Miyuki) \\ \hline
     Query 3\\
     Saya ingin mengetahui metode untuk belajar bagaimana menari salsa
     \\
     (I want to know the method of studying how to dance the salsa) \\
     \hline
     Query 4\\
     Saya ingin belajar tentang akibat perusakan lapisan ozon dan
     pelebaran lubang ozon terhadap tubuh manusia \\
     (I want to learn about the effects of destruction of the ozone layer
     and expansion of the ozone \\ hole have on the human body) \\ \hline
    \end{tabular}
  \end{center}
\vspace{-1\baselineskip}
\end{table}

Indonesian sentences usually consist of native (Indonesian) words and
borrowed words. The first three query examples in Table
\ref{table:Indonesian Query Examples} contain borrowed words. ``Academy
Awards'' of the first query and ``salsa'' of the third query are
borrowed from the English language. ``Miyabe Miyuki'' in the second
query is transliterated from the Japanese. To obtain a good translation,
the query translation in our system must be able to translate those
words, the Indonesian (native) words and the borrowed words.



\section{Schema of Indonesian-Japanese Query Translation System}
\label{Schema of Indonesian - Japanese Query Translation System}

Indonesian-Japanese query translation is a component of the
Indonesian-Japanese CLIR. The query translation system aims to translate
an Indonesian query sentence into a Japanese keyword list. The Japanese
keyword list is then used in the Japanese IR system to retrieve the
relevant document. The schema of the Indonesian-Japanese query
translation system can be seen in Figure~\ref{fig:query_translation}.

The query translation system consists of 2 subsystems; the keyword
translation and translation candidate filtering. The keyword translation
system seeks to obtain Japanese translation candidates for an Indonesian
query sentence. The translation candidate filtering aims to select the
most appropriate among all Japanese translation alternatives. The
Japanese translation resulting from the translation filtering is used as
the input for the Japanese IR system. The keyword translation and
translation filtering process is described in the next section.

\begin{figure}[b]
  \begin{center}
    \includegraphics[width=0.28\textwidth]{figure1.eps}
  \end{center}
  \caption{Indonesian-Japanese Query Translation Schema}
  \label{fig:query_translation}
\end{figure}



\subsection{Indonesian-Japanese Key Word Translation Process}

The keyword translation system is a process by which Indonesian keywords
are translated into Japanese keywords. We chose to do a transitive
translation with bilingual dictionaries to do the keyword
translation. Other approaches such as direct translation or machine
translation are employed for the comparison method. The schema of our
keyword transitive translation using bilingual dictionaries is shown in
Figure~\ref{fig:keyword_translation}.  

The keyword translation process consists of native (Indonesian) word 
translation and borrowed word translation. The native words are 
translated using Indonesian-English and English-Japanese dictionaries. 
Because the Indonesian tag parser is not available, we do the 
translation of a single word and consecutive
pair of words that exist as a single term in the Indonesian-English
dictionary. As mentioned in the previous section dealing with the affix
combination in the Indonesian language, not all words with the affix
combination are recorded in an Indonesian dictionary. Therefore, if a
search does not reveal the exact word, it will search for other words
that are the basic word of the query word or have the same basic
word. For example, the Indonesian word, ``munculnya'' (come out), has a
basic word ``muncul'' with the postfix ``-nya''. Here, the term
``munculnya'' is not available in the dictionary. Therefore, the search
will take ``muncul'' as the matching word with ``munculnya'' and give
the English translation for ``muncul'' such as ``come out'' as its
translation result.

\begin{figure}[t]
  \begin{center}
        \includegraphics[width=0.7\textwidth]{14-2ia4f2.eps}
  \end{center}
  \caption{Indonesian-Japanese Keyword Translation Schema}
  \label{fig:keyword_translation}
\end{figure}

The English translation results are then translated into Japanese using
an English-Japanese dictionary. The English translation results also
include inflected words, not only basic words. For example, the English
translation pair from Indonesian-English dictionary for ``obat-obatan''
is ``medicines'', while the term in the English-Japanese dictionary is
``medicine''. Therefore, in the English matching, it searched the same
English word or the basic word of the English translation.  

In Indonesian, a noun phrase has the reverse word position of that in
English. For example, ``ozone hole'' is translated as ``lubang ozon''
(ozone$=$ozon, hole$=$lubang) in Indonesian. Therefore, in English
translation, besides a word-by-word translation, we also searched for
the reversed English word pair as a single term in an English-Japanese
dictionary. This strategy reduced the number of translation
candidates. An example of a keyword translation process in the
transitive translation with bilingual dictionary is shown in
Table~\ref{table:translation_process}. In the query example, there are 3
word pairs treated as single terms in the English-Japanese dictionary:
ozone layer(オゾン層),ozone hole(オゾンホール) and human body (人体).
Other translations such as coating or stratum (the synonym for layer)
are eliminated as translation candidates.

\begin{table}[t]
  \caption{Illustration of Native (Indonesian) Keyword Translation Process}
  \label{table:translation_process}
        \begin{tabular}{|p{5zw}|p{4zw}|p{4zw}|p{3zw}|p{4zw}|p{4zw}|p{4zw}|p{4zw}|p{4zw}|}
\hline
      Indonesian query &
      \multicolumn{8}{l|}{
          \begin{minipage}[t]{39zw}
	Saya ingin belajar tentang akibat perusakan lapisan ozon
	dan pelebaran lubang ozon terhadap tubuh manusia
	($=$I want to learn about the effects of destruction of the
	ozone layer and expansion of the ozone hole have on the
	human body)
      \end{minipage}} \\ \hline
     \multicolumn{9}{|l|}{Indonesian Keywords} \\ \hline
     belajar & perusakan & lapisan & ozon & pelebaran & lubang & ozon &
     tubuh & manusia \\ \hline
     \multicolumn{9}{|l|}{English keywords as the Indonesian-English
     dictionary matching result} \\ \hline
     to study, to learn, to take up & damaging & coating, layer, stratum & ozone & widening, broadening & cavity, hole, hollow, perforation & ozone & body & human being, man, human \\ \hline
     \multicolumn{9}{|l|}{Japanese keywords as the English-Japanese
     dictionary matching result} \\ \hline
     〜を調べる,勉強する,研究する,学ぶ,… &
     損害を与える,不利な,有害な &
     \multicolumn{2}{p{8zw}|}{オゾン層 (ozone layer)} &
     拡大主義者,広がり &
     \multicolumn{2}{p{9zw}|}{オゾンホール (ozone hole)} &
     \multicolumn{2}{p{9zw}|}{人体 (human body)} \\ \hline
    \end{tabular}
\end{table}


Borrowed words are translated using an English-Japanese dictionary. The
English-Japanese dictionary is used because most of the borrowed words
in our query translation system come from English. Examples of borrowed
words in our query are ``Academy Awards'', ``Aurora'', ``Tang'',
``baseball'', ``Plum'', ``taping'', and ``Kubrick''.  

Even though using an English-Japanese dictionary may help with accurate 
translation of words, there are some proper nouns which can not be 
translated by this dictionary, such as ``Miyabe Miyuki'', 
``Miyazaki Hayao'', ``Honjo Manami'', etc.  These proper names 
come from Japanese words which are romanized. In the 
Japanese language, these proper names might be written
in one of the following scripts: kanji (Chinese character), hiragana
(cursive form), katakana (squared form) and romaji (roman alphabet). One
alphabetical word can be transliterated into more than one Japanese
word. For example, ``Miyabe'' can be transliterated into 宮部, 宮辺, みや
べ or ミヤベ. 宮部 and 宮辺 are written in kanji, みやべ is written in
hiragana, and ミヤベ is written in katakana. For hiragana and katakana
script, the borrowed word is translated using a pair list between
hiragana or katakana and its roman alphabet. These systems have a
one-to-one correspondence for pronunciation (syllables or phonemes),
something that can not be done for kanji. Therefore, in order to obtain
a kanji corresponding to borrowed words, we use a Japanese proper name
dictionary. Each term in the original proper name dictionary usually
consists of two words, the first and last names. For a wider selection
of the translation candidates, we separate each two-word term into two
terms. Even though the input word can not be found in the original
proper name dictionary (family name and first name), a match may still
be possible with the new proper name dictionary.  

Each of the above
translation processes also involves the stop word elimination process,
which aims to delete stop words or words that do not have significant
meaning in the documents retrieved. The stop word elimination is done at
every language step. First, Indonesian stop word elimination is applied
to an Indonesian query sentence to obtain Indonesian keywords. Second,
English stop word elimination is applied before English keywords are
translated into Japanese keywords.  Finally, Japanese stop word
elimination is done after the Japanese keywords are morphologically
analyzed by Chasen \cite{chasen}.

\begin{table}[t]
  \caption{Indonesian Query Examples}
  \label{table:query_examples}
  \begin{center}
    \begin{small}
    \begin{tabular}{|l|l|l|l|l|}\hline
     \multicolumn{1}{|c|}{Step} & \multicolumn{4}{|c|}{Result} \\ \hline 
     \multicolumn{1}{|p{8em}|}{Indonesian query} &
     \multicolumn{4}{|p{33em}|}{Saya ingin mengetahui metode untuk
     belajar bagaimana menari salsa ($=$I want to know the method of
     studying how to dance the salsa)} \\ \hline
     {} & \multicolumn{3}{|c|}{\underline{Native word translation}} &
     \multicolumn{1}{|p{6em}|}{\underline{Borrowed word translation}} \\
     \hline
     \multicolumn{1}{|p{9em}|}{Indonesian keyword} &
     metode & belajar & menari & salsa \\ \hline
     \multicolumn{1}{|p{9em}|}{English keyword (from Indonesian-English
     dictionary)} & method & \multicolumn{1}{|p{6em}|}{to learn, to
     study, to take up} & dance & salsa \\ \hline
     \multicolumn{1}{|p{9em}|}{Japanese keyword (from English-Japanese
     dictionary)} & \multicolumn{1}{|p{6em}|}{規則正しさ,筋道,秩序,方法}
     & \multicolumn{1}{|p{6em}|}{〜を調べる,勉強する,研究する,学ぶ,勉強,研
     究,調査,検討,書斎,〜を学ぶ,知る,わかる,暗記する,覚える,確認する,習
     う,突きとめる} & \multicolumn{1}{|p{6em}|}{舞踊,ダンス,ダンスパー
     ティー,
     バレエ,ダンスする,舞う,踊る,踊らされる,いいようにされる} &
     \multicolumn{1}{|p{10em}|}{サルサ,サルサのダンス} \\ \hline
     \multicolumn{1}{|p{9em}|}{Japanese keyword (after analyzed by
     Chasen)} & \multicolumn{1}{|p{6em}|}{規則正し,筋道,秩序,方法} &
     \multicolumn{1}{|p{6em}|}{調べる,勉強,研究,学ぶ,調査,検討,書斎,知る,
     わかる,暗記,覚える,確認,習う,突きとめる} &
     \multicolumn{1}{|p{6em}|}{舞踊,ダンス,ダンスパーティー,バレエ,舞う,
     踊る,踊ら} & サルサ \\ \hline
    \end{tabular}
    \end{small}
  \end{center}
\end{table}


Examples of the Indonesian-Japanese keyword translation system can be
seen in Table~\ref{table:query_examples}. Each word in the input query
is matched with the term in the Indonesian-English bilingual dictionary
and the stop word list. If the word is not included in the stop word
list and exists in the Indonesian-English bilingual dictionary, then it
is assumed to be a the native word and translated using
Indonesian-English and English-Japanese dictionaries. Words (in
Table~\ref{table:query_examples}) included in this type group are ``metode''
(method), ``belajar'' (to learn) and ``menari'' (dance). If the word is not
included in the stop word list and does not exist in the
Indonesian-English bilingual dictionary, then the word is assumed to be
a borrowed word and translated using an English-Japanese dictionary,
Japanese proper name dictionary and or by transliteration. For example,
``salsa'' is translated into サルサ using the English-Japanese
dictionary. The final Japanese keywords are then analyzed by Chasen and
inputted into the translation candidate filtering process which is
described in the following section.  

The keyword transitive translation
is used in 2 systems: 1) transitive translation to translate all words
in the query, and 2) transitive translation to translate OOV
(Indonesian) words by direct translation using an Indonesian-Japanese
dictionary and English-Japanese dictionary. We call the first method
transitive translation using bilingual dictionaries and the second
method combined translation (direct-transitive).

\subsection{Japanese Translation Candidate Filtering Process}
\label{Japanese Translation Candidate Filtering Process} The Japanese
translation candidate filtering process aims to select the most
appropriate among the Japanese translation candidates. In order to
select the best Japanese translation, rather than choosing only the
highest TF×IDF score or only the highest mutual information score among
all keyword lists, we combine both scores to select the highest mutual
information score among the top 3 TF×IDF scores. To avoid calculation
of all sequences, we selected 100 term-sequences by their mutual
information scores. The mutual information score is calculated per word
pair. First, we select the 100 (or less) best mutual information score
sequences among the translations of the first 2 Indonesian
keywords. These 100 best sequences joined with the 3rd keyword
translation set are recalculated to obtain the mutual information score
and reselected to yield the 100 best sequences for the 3 translation
sets. This step is repeated until all translation sets are covered.  For
a word sequence, the mutual information score is:
\begin{equation}
  I(t_1 \cdots t_n) = \sum^{n-1}_{i=1}\sum^{n}_{j=i+1}I(t_i,t_j)
    \label{eq1}
\end{equation}

\begin{table}[b]
  \caption{Translation Filtering Method}
  \label{table:Translation Filtering Method}
  \begin{center}
   \begin{tabular}{|l|l|l|l|l|}\hline
    \multicolumn{1}{|c|}{Step} & \multicolumn{4}{|c|}{Result} \\ \hline
    \multicolumn{1}{|p{10em}|}{Indonesian query} &
    \multicolumn{4}{|p{30em}|}{Saya ingin mengetahui metode untuk belajar
    bagaimana menari salsa (I want to know the method of studying how to
    dance the salsa)} \\ \hline
    \multicolumn{1}{|p{10em}|}{Japanese keyword (after analyzed by
     Chasen)} & \multicolumn{1}{|p{7em}|}{規則正し,筋道,秩序,方法} &
     \multicolumn{1}{|p{8em}|}{調べる,勉強,研究,学ぶ,調査,検討,書斎,知る,
     わかる,暗記,覚える,確認,習う,突きとめる} &
     \multicolumn{1}{|p{7em}|}{舞踊,ダンス,ダンスパーティー,バレエ,舞う,
     踊る,踊ら} & サルサ \\ \hline
    \multicolumn{1}{|p{10em}|}{Translation Combinations} &
    \multicolumn{4}{|p{30em}|}{(規則正し,調べる,舞踊,サルサ)(筋道,調べる,舞
    踊,サルサ)(秩序,調べる,舞踊,サルサ),etc} \\ \hline
    \multicolumn{1}{|p{10em}|}{Rank Sequences based on Mutual Information
    Score} &
    \multicolumn{4}{|p{30em}|}{1.(秩序,知る,踊る,サルサ)2.(秩序,研
    究,踊る,サルサ)3.(方法,わかる,ダンス,サルサ)4.(方法,覚える,ダンス,サル
    サ) 5.(秩序,分かる,踊る,サルサ)} \\ \hline
    \multicolumn{1}{|p{10em}|}{Select Queries with best TF×IDF score} &
    \multicolumn{4}{|l|}{(方法,わかる,ダンス,サルサ)} \\ \hline
   \end{tabular}
  \end{center}
\end{table}

$I(t_1 \cdots t_n )$ means the mutual information for a sequence of
words $t_1, t_2, \cdots t_n$. $I(t_i,t_j)$ means the mutual information
between two words $(t_i,t_j)$.  Here, a zero frequency word will have no
impact on the mutual information score of a word sequence.  The mutual
information score represents the relationship between word pairs in a
sequence, but not the relationship among all terms in a sequence at the
same time. Therefore, in the translation candidate filtering, we also
used the TF×IDF score to represent such a relationship.  

The next step
is to select a keyword list with the highest TF×IDF score among some
sequences with top mutual information scores. The TF×IDF score used
here is the relevance score between the document and the query
(Equation~\ref{eq2} from \cite{fujii2003}).
\begin{equation}
  \sum_t\{ \frac {TF_{t,i}}{{\frac{DL_i}{{\it avglen}} + TF_{i,j}}} \cdot
   log{\frac N{DF_i}}\}
    \label{eq2}
\end{equation}

$TF_{t,i}$ denotes the frequency of term $t$ appearing in document $i$.
$DF_t$ denotes the number of documents containing term $t$. $N$
indicates the total number of documents in the collection. $DL_{i}$
denotes the length of document $i$ (i.e., the number of characters
contained in $i$), and $\mathit{avglen}$ the average length of documents in the
collection.
 
Table~\ref{table:Translation Filtering Method} shows an example of the
keyword selection process after completion of the keyword translation
process (such as in Table~\ref{table:query_examples},
rewritten in Table~\ref{table:Translation Filtering Method}, $3^{rd}$ row,
the Japanese keyword translation result).  The translation combinations
($4^{th}$ row) and sequence rankings ($5^{th}$ row) are for all words 
(translation
sets of ``metode'', ``belajar'', ``dansa'', and ``salsa'') in the
query. Then, all resulting sequences (ranked by its mutual information
score) are executed in the IR system to obtain the TF×IDF score. The
final query chosen is the one which has the highest TF×IDF score ($6^{th}$
row) for the document retrieval results using translated query
candidates.


\section{Compared Method}
\label{Compared Method}
In the experiment, we compared our proposed method with other
translation methods. Table~\ref{table:5} lists all the compared methods
with its corresponding query label (as is of Figures in
Section~\ref{Experimental Result}).

\begin{table}[t]
  \caption{List of Compared Methods with its Corresponding Query Label
 Shown in Section~\ref{Experimental Result}}
  \label{table:5}
  \begin{center}
   \begin{tabular}{|l|l|}\hline
    \multicolumn{1}{|c|}{Method Description} &
    \multicolumn{1}{|c|}{Query label(Section\ref{Experimental Result})} \\ \hline
    
    \multicolumn{1}{|p{30em}|}{Indonesian-English-Japanese transitive
    translation with Machine Translation (Kataku and Excite)}
    & iej-mx    \\  \hline
    \multicolumn{1}{|p{30em}|}{Indonesian-English-Japanese transitive
    translation with Machine Translation (Kataku and Babelfish)}
    & iej-mb    \\ \hline
    \multicolumn{1}{|p{30em}|}{Indonesian-Japanese direct translation
    with existing Indonesian-Japanese dictionary}
    & ij    \\ \hline
     \multicolumn{1}{|p{30em}|}{Indonesian-Japanese direct translation
    with built-in Indonesian-Japanese dictionary}
    & ijn    \\ \hline
     \multicolumn{1}{|p{30em}|}{The Japanese keyword filtering with only
    using the mutual information score. It is marked by the query
    label's postfixes of $-I_n$ and $-I_{-n}$ $-I_n$ means that the
    Japanese keyword list is the $n^{th}$ rank Japanese keyword list of
    mutual information score. $-I_{-n}$ means that Japanese keyword list
    is the conjunct of $1^{st}$ rank until $n^{th}$ rank Japanese
    keyword list of mutual information score.}
    & \multicolumn{1}{|p{10em}|}{The query label's postfixes are $-I_n$ and
    $-I_n$} \\ \hline
    \multicolumn{1}{|p{30em}|}{Using the English keyword filtering by
    mutual information score. $-E_{-n}$ means that the English keywords
    are the conjunct of $1^{st}$ rank until $n^{th}$ rank English
    keyword list of mutual information score} &
    \multicolumn{1}{|p{10em}|}{The query label's infix is $-E\{-n\}$}
    \\ \hline
    \multicolumn{1}{|p{30em}|}{English-Japanese CLIR. The query label's
    notation is ej-xxx-yyy. xxx shows the keyword translation method such
    as ``man'' (the English keywords are selected manually from the
    English query sentence), ``mb'' (using Babelfish machine
    translation), ``mx'' (using Excite machine translation). yyy shows
    the keyword filtering method such as $I_n$, $I_{-n}$ and $IR_{-n}$
    (the keyword list chosen is the one with best TF×IDF score of its
    document collection among n highest rank of mutual in formation score)} &
    \multicolumn{1}{|p{10em}|}{Figure~\ref{fig:9}, the query label's notation is
   ej-xxx-yyy.} \\ \hline
   \end{tabular}
  \end{center}
\end{table}


\subsection{Transitive Translation using Machine Translation}
The first compared method is a transitive translation using MT (machine
translation). The Indonesian-Japanese transitive translation using MT
has a schema similar to Indonesian-Japanese transitive translation using
a bilingual dictionary. Instead of using the available
Indonesian-English and English-Japanese dictionaries, the Indonesian
queries are translated using online Indonesian-English MT Kataku \cite{kataku}
and 2 online English-Japanese MTs Babelfish \cite{babelfish}-Excite \cite{excite}. The example
of the Indonesian-Japanese translation result using machine translation
method is shown in Table~\ref{table:6}.

\begin{table}[t]
  \caption{Examples of Indonesian-Japanese Translation using Machine
 Translation Method}
  \label{table:6}
  \begin{center}
   \begin{tabular}{|l|l|}\hline
    \multicolumn{1}{|p{10em}|}{Indonesian query}
    &    \multicolumn{1}{|p{30em}|}{Saya ingin belajar tentang akibat
    perusakan lapisan ozon dan pelebaran lubang 
    ozon terhadap tubuh manusia ($=$I want to learn about the
    effects of destruction of the ozone layer and expansion of the ozone
    hole have on the human body)}   \\  \hline
    \multicolumn{1}{|p{10em}|}{Translated English (Kataku engine)}
    &    \multicolumn{1}{|p{30em}|}{I wanted to study about resulting
    from destruction and the widening of the ozone hole of the layer of
    ozone of the human body}   \\  \hline
    \multicolumn{2}{|l|}{Translated Japanese Sentence} \\ \hline
    \multicolumn{1}{|p{10em}|}{-Excite Engine (iej-mx)}
    &    \multicolumn{1}{|p{30em}|}{私は人体のオゾンの層のオゾンホールの
    破壊から生じて、広くなるのに関して研究したかったです}   \\  \hline
     \multicolumn{1}{|p{10em}|}{-Babelfish Engine(iej-mb)}
    &    \multicolumn{1}{|p{30em}|}{私は人体のオゾンの層のオゾン穴の破壊
    そして広がることに起因について調査したいと思った}   \\  \hline
     \multicolumn{2}{|l|}{Japanese Keywords} \\ \hline
     \multicolumn{1}{|p{10em}|}{-Excite Engine}
    &    \multicolumn{1}{|p{30em}|}{破壊,人体,オゾン,層,ホール,広く,起因,勉強}   \\  \hline
     \multicolumn{1}{|p{10em}|}{-Babelfish Engine}
    &    \multicolumn{1}{|p{30em}|}{人体,オゾン,層,穴,破壊,広がる,起因,調査,思っう}   \\  \hline
     \multicolumn{2}{|l|}{} \\ \hline
    \multicolumn{1}{|p{10em}|}{Indonesian query:}
    &    \multicolumn{1}{|p{30em}|}{Saya ingin mengetahui metode untuk
    belajar bagaimana menari salsa ($=$I want to know the method of
    studying how to dance the salsa)}   \\  \hline
    \multicolumn{1}{|p{10em}|}{Translated English (Kataku engine)}
    &    \multicolumn{1}{|p{30em}|}{I wanted to know the method of
    studying how danced salsa}   \\  \hline
    \multicolumn{2}{|l|}{Translated Japanese Sentence} \\ \hline
    \multicolumn{1}{|p{10em}|}{-Excite Engine (iej-mx)}
    &    \multicolumn{1}{|p{30em}|}{私は、どのようにを研究するか方法がサ
    ルサを踊ったのをしりたかったです}   \\  \hline
     \multicolumn{1}{|p{8em}|}{-Babelfish Engine(iej-mb)}
    &    \multicolumn{1}{|p{30em}|}{私はいかに踊られたサルサ調査する方法
    を知りたいと思った}   \\  \hline
     \multicolumn{2}{|l|}{Japanese Keywords} \\ \hline
     \multicolumn{1}{|p{10em}|}{-Excite Engine}
    &    \multicolumn{1}{|p{30em}|}{勉強,方法,サルサ,ダンス}   \\  \hline
     \multicolumn{1}{|p{10em}|}{-Babelfish Engine}
    &    \multicolumn{1}{|p{30em}|}{いかに,踊ら,サルサ,調査,方法,思っう}   \\  \hline
   \end{tabular}
  \end{center}
\end{table}


\subsection{Direct Translation using Existing Indonesian-Japanese Dictionary}
\label{Direct Translation using Existing Indonesian-Japanese Dictionary}
The second comparison method is a direct translation with an
Indonesian-Japanese dictionary. This direct translation also has a
schema similar to the transitive translation using a bilingual
dictionary (Figure~\ref{fig:keyword_translation}). The difference is
that in translation of an Indonesian keyword, only one dictionary is
used, rather than two; in this case, we use an Indonesian-Japanese
bilingual dictionary (14,823 words) with fewer words than the
Indonesian-English (29,054 words) and English-Japanese (556,237 words)
dictionaries. We also did some experiments in direct translation by
reducing the Indonesian-Japanese dictionary into various sizes (3000,
5000 and 8857 words). Table~\ref{table:7} shows the example of the
Indonesian-Japanese translation result with direct translation method
using Indonesia-Japanese dictionary.

\begin{table}[t]
  \hangcaption{Examples of Indonesian-Japanese Translation with Direct
 Translation using Existing Indonesian-Japanese Dictionary}
  \label{table:7}
   \begin{tabular}{|l|l|}\hline
      \multicolumn{1}{|p{13em}|}{Indonesian query}
    &    \multicolumn{1}{|p{30em}|}{Saya ingin belajar tentang akibat
    perusakan lapisan ozon dan pelebaran lubang 
    ozon terhadap tubuh manusia ($=$I want to learn about the
    effects of destruction of the ozone layer and expansion of the ozone
    hole have on the human body)}   \\  \hline
     \multicolumn{2}{|l|}{Japanese Keywords} \\ \hline
    \multicolumn{1}{|p{13em}|}{- 3000 dictionary size}
    &    \multicolumn{1}{|p{30em}|}{学ぶ,習う,学習,破壊,行為,層,階層,オゾン,
    pelebaran, lubang,体,肉体,人間}   \\  \hline
     \multicolumn{1}{|p{13em}|}{- 5000 dictionary size}
    &    \multicolumn{1}{|p{30em}|}{学ぶ,習う,学習,破壊,行為,層,階層,オゾン,
    pelebaran, ホール,穴,体,肉体,人間}   \\  \hline
      \multicolumn{1}{|p{13em}|}{- 8857 dictionary size}
    &    \multicolumn{1}{|p{30em}|}{学ぶ,習う,学習,破壊,行為,層,階層,オゾン,
    拡張,ホール,穴,体,肉体,人間}   \\  \hline
      \multicolumn{1}{|p{13em}|}{- 14,823 dictionary size (ij)}
    &    \multicolumn{1}{|p{30em}|}{学ぶ,習う,学習,破壊,行為,層,階層,オゾン,
    拡張,ホール,穴,体,肉体,人間}   \\  \hline
    
     \multicolumn{2}{|l|}{} \\ \hline
    \multicolumn{1}{|p{13em}|}{Indonesian query:}
    &    \multicolumn{1}{|p{30em}|}{Saya ingin mengetahui metode untuk
    belajar bagaimana menari salsa ($=$I want to know the method of
    studying how to dance the salsa)}   \\  \hline
     \multicolumn{2}{|l|}{Japanese Keywords} \\ \hline
    \multicolumn{1}{|p{13em}|}{- 3000 dictionary size}
    &    \multicolumn{1}{|p{30em}|}{方法,メソッド,学ぶ,習う,学習,サルサ,ダンス}   \\  \hline
     \multicolumn{1}{|p{13em}|}{- 5000 dictionary size}
    &    \multicolumn{1}{|p{30em}|}{方法,メソッド,学ぶ,習う,学習,サルサ,ダンス}   \\  \hline
      \multicolumn{1}{|p{13em}|}{- 8857 dictionary size}
    &    \multicolumn{1}{|p{30em}|}{方法,メソッド,学ぶ,習う,学習,サルサ,ダンス}   \\  \hline
      \multicolumn{1}{|p{13em}|}{- 14,823 dictionary size (ij)}
    &    \multicolumn{1}{|p{30em}|}{方法,メソッド,学ぶ,習う,学習,サルサ,ダンス}   \\  \hline
   \end{tabular}
\end{table}


 
\subsection{Direct Translation using Built-in Indonesian-Japanese Dictionary}
\label{Direct Translation using Built-in Indonesian-Japanese Dictionary}
We also compared the transitive translation results with those of direct
translation using our built-in Indonesian-Japanese dictionary. We call
it direct translation because although the Indonesian-Japanese
dictionary was made from the Indonesian-English and Japanese-English
dictionaries in advance, the query translation process uses only the
dictionary which yields different Japanese translations compared with
the transitive translation.  

When building the Indonesian-Japanese
dictionary from the Indonesian-English and Japanese-English
dictionaries, explosion of possible translation pairs arises.  To select
the correct pair, we used the ``one-time inverse consultation''
score\footnote{For each Indonesian word, we look up all English
translations, and see how many match the English translations of the
original Japanese word (translation candidate). This is called the ``one
time inverse consultation''.}  such as in \cite{tanaka1994}.  We also
used WordNet to get more English translation candidates. The complete
procedure is as follows:

\begin{enumerate}
 \item Do word matching between English translation (from
       Indonesian-English dictionary) and English word in the
       Japanese-English dictionary. If the English term is phrase and
       the matching word could not be found, then the English terms will
       be normalized by eliminating certain words (``to'', ``a'',
       ``an'', ``the'', ``to be'', ``kind of'').
 \item For every Japanese translation, a ``one time inverse
       consultation'' score is calculated.   
       English translation for every
       Japanese candidate is matched with the English translation for
       Indonesian words. If the matched word is more than one then it is
       accepted as Indonesian-Japanese pair. But if it is not, then the
       English translation will be added by its synonym taken from
       WordNet. The ``one-time inverse consultation'' score of the new
       English words is recalculated.
\end{enumerate}

The example of the Indonesian-Japanese translation result using built-in
Indonesian Japanese dictionary is shown in Table~\ref{table:8}.

\begin{table}[t]
  \hangcaption{Examples of Indonesian-Japanese Translation with Direct
 Translation using Built-in Indonesian-Japanese Dictionary}
  \label{table:8}
    \begin{tabular}{|l|l|}\hline
     \multicolumn{1}{|p{13em}|}{Indonesian query}
     &    \multicolumn{1}{|p{30em}|}{Saya ingin belajar tentang akibat
     perusakan lapisan ozon dan pelebaran lubang 
     ozon terhadap tubuh manusia ($=$I want to learn about the
     effects of destruction of the ozone layer and expansion of the ozone
     hole have on the human body)}   \\  \hline
      \multicolumn{1}{|p{13em}|}{Japanese keywords (ijn):}
     &    \multicolumn{1}{|p{30em}|}{わかる,研究,与える,層,オゾン,広げる,穴,空
     洞,団体,本体,集団,死体,人物,人,人間,人類}   \\  \hline
     \multicolumn{2}{|l|}{} \\ \hline
     \multicolumn{1}{|p{13em}|}{Indonesian query:}
     &    \multicolumn{1}{|p{30em}|}{Saya ingin mengetahui metode untuk
     belajar bagaimana menari salsa ($=$I want to know the method of
     studying how to dance the salsa)}   \\  \hline
       \multicolumn{1}{|p{13em}|}{Japanese keywords (ijn):}
     &    \multicolumn{1}{|p{30em}|}{秩序,方法,研究,わかる,踊る,ダンス,曲,パー
     ティー,舞う,バレエ,舞踊,サルサ}   \\  \hline
    \end{tabular}
\end{table}
\begin{table}[t]
 \hangcaption{Examples of Indonesian-Japanese Translation with Japanese
 Keyword Selection using Mutual Information Score only}
 \label{table:9}
 \begin{center}
  \begin{tabular}{|l|l|}\hline
    \multicolumn{1}{|p{12em}|}{Indonesian query}
     &    \multicolumn{1}{|p{28em}|}{Saya ingin belajar tentang akibat
     perusakan lapisan ozon dan pelebaran lubang ozon dan pelebaran
     lubang ozon terhadap tubuh manusia ($=$I want to learn about the
     effects of destruction of the ozone layer and expansion of the ozone
     hole have on the human body)}   \\  \hline
      \multicolumn{1}{|p{12em}|}{Japanese keywords (iej-I1):}
     &    \multicolumn{1}{|p{28em}|}{確認,与える,オゾン層,広がり,オゾン
   ホール,人体}   \\  \hline
   \multicolumn{1}{|p{12em}|}{Japanese keywords (iej-I-3):}
     &    \multicolumn{1}{|p{28em}|}{する,確認,覚える,与える,オゾン層,広がり,
   オゾンホール,人体}   \\  \hline
     \multicolumn{2}{|l|}{} \\ \hline
     \multicolumn{1}{|p{12em}|}{Indonesian query:}
     &    \multicolumn{1}{|p{28em}|}{Saya ingin mengetahui metode untuk
     belajar bagaimana menari salsa ($=$I want to know the method of
     studying how to dance the salsa)}   \\  \hline
       \multicolumn{1}{|p{12em}|}{Japanese keywords (iej-I1):}
     &    \multicolumn{1}{|p{28em}|}{方法,する,ダンス,サルサ}   \\
   \hline
    \multicolumn{1}{|p{12em}|}{Japanese keywords (iej-I-3):}
     &    \multicolumn{1}{|p{28em}|}{秩序,方法,する,知る,踊る,サルサ}   \\  \hline
  \end{tabular}
 \end{center}
\end{table}



\subsection{Japanese Keyword Selection using only Mutual Information  Score in Transitive Translation using Bilingual Dictionary}
\label{Japanese Keyword Selection using only Mutual Information Score in
Transitive Translation using Bilingual Dictionary} We also compared our
proposed keyword selection method with the Japanese keyword selection
based on mutual information score only. There are two keyword selection
schemas. In the first schema, only one keyword list among the ranked
keyword lists is selected. In the second, all keywords from the first
rank keyword list until the $n^{th}$-rank keyword list are grouped into one
keyword list. For the baseline (iej), we used the Indonesian-Japanese
transitive translation using a bilingual dictionary without keyword
selection. Table~\ref{table:9} shows some translation examples for the
Indonesian-Japanese transitive translation (bilingual dictionary) with
Japanese keyword selection using mutual information score only. Queries
with postfix ``I$n$'' (first schema) and postfix ``I-$n$'' (second schema)
in Section~\ref{Experimental Result} show the experiment results.


\subsection{English Keyword Selection using Mutual Information Score in Transitive Translation using Bilingual Dictionary}
\label{English Keyword Selection using Mutual Information Score in Transitive Translation using Bilingual Dictionary}

Another comparison method is a transitive translation with English
keyword selection based on mutual information taken from monolingual
English corpus. The English keywords are selected based on its mutual
information score. The English keywords selected are used as the input
for the English-Japanese translation. Table~\ref{table:10} shows the
example of Indonesian-Japanese translation with English keyword
filtering using mutual information score.

\begin{table}[b]
  \hangcaption{Examples of Indonesian-Japanese Translation (Bilingual
 Dictionary) with English Keyword Selection}
  \label{table:10}
   \begin{center}
    \begin{tabular}{|l|l|}\hline
     \multicolumn{1}{|p{10em}|}{Indonesian query}
     &    \multicolumn{1}{|p{30em}|}{Saya ingin belajar tentang akibat
     perusakan lapisan ozon dan pelebaran lubang ozon dan pelebaran
     lubang ozon terhadap tubuh manusia ($=$I want to learn about the
     effects of destruction of the ozone layer and expansion of the ozone
     hole have on the human body)}   \\  \hline
      \multicolumn{1}{|p{10em}|}{Japanese keywords (iej-E):}
     &    \multicolumn{1}{|p{30em}|}{検討,学ぶ,調査,調べる,研究,勉強,与える,有
     害,不利,損害,オゾン,層,拡大,主義者,ホール,人体}   \\  \hline
     \multicolumn{2}{|l|}{} \\ \hline
     \multicolumn{1}{|p{10em}|}{Indonesian query:}
     &    \multicolumn{1}{|p{30em}|}{Saya ingin mengetahui metode untuk
     belajar bagaimana menari salsa ($=$I want to know the method of
     studying how to dance the salsa)}   \\  \hline
       \multicolumn{1}{|p{10em}|}{Japanese keywords (iej-E):}
     &    \multicolumn{1}{|p{30em}|}{秩序,方法,知る,わかる,覚える,確認,学
     ぶ,習う,踊る,ダンス,パーティー,バレエ,舞踊,舞う,サルサ}   \\  \hline
    \end{tabular}
   \end{center}
\end{table}


\subsection{English-Japanese Translation}
\label{English-Japanese Translation} The English-Japanese translation is
done to compare the performance reduction caused by the
Indonesian-English translation. Methods used in the English-Japanese
query translation are machine translation and word-by-word translation
using a bilingual English-Japanese dictionary. The schemas for using the
machine translation and the dictionary are similar to those described in
Section~\ref{Direct Translation using Existing Indonesian-Japanese
Dictionary} and Section~\ref{Direct Translation using Existing
Indonesian-Japanese Dictionary} respectively. The machine translation
systems used here are Babelfish and Excite engines. Table~\ref{table:11}
shows the example of the English-Japanese translation result in the
English-Japanese CLIR.

\begin{table}[t]
  \caption{Examples of Indonesian-Japanese Translation of
 English-Japanese CLIR}
  \label{table:11}
   \begin{center}
    \begin{tabular}{|l|l|}\hline
     \multicolumn{1}{|p{10em}|}{Indonesian query}
     &    \multicolumn{1}{|p{30em}|}{I want to learn about the
     effects of destruction of the ozone layer and expansion of the ozone
     hole have on the human body}   \\  \hline
      \multicolumn{2}{|l|}{Japanese Keywords} \\ \hline
      \multicolumn{1}{|p{10em}|}{-English-Japanese dictionary (ej):}
     &    \multicolumn{1}{|p{30em}|}{学ぶ,知る,わかる,暗記,覚える,確認,習う,突
     きとめる,個人,資産,駆除,絶滅,倒壊,破壊,破滅,原因,撲滅,滅亡,オゾン,層,ホール,膨張,
     発展,拡張,人体}   \\  \hline
     \multicolumn{1}{|p{10em}|}{-Excite engine (ej-mx):}
     &    \multicolumn{1}{|p{30em}|}{オゾン,ホール,層,破壊,拡大,人体,上,効果,学
     び}   \\  \hline
     \multicolumn{1}{|p{10em}|}{-Babelfish engine (ej-mb):}
     &    \multicolumn{1}{|p{30em}|}{オゾン,層,効果,学び,思い,穴,拡張,人体}   \\  \hline
     \multicolumn{2}{|l|}{} \\ \hline
     \multicolumn{1}{|p{10em}|}{Indonesian query:}
     &    \multicolumn{1}{|p{30em}|}{I want to know the method of
     studying how to dance the salsa}   \\  \hline
     \multicolumn{2}{|l|}{Japanese Keywords} \\ \hline
      \multicolumn{1}{|p{10em}|}{-English-Japanese dictionary (ej):}
     &    \multicolumn{1}{|p{30em}|}{規則正しさ,筋道,秩序,方法,学習,学問,舞踊,
     ダンス,パーティー,バレエ,舞う,踊る,踊らされる,サルサ}   \\  \hline
     \multicolumn{1}{|p{10em}|}{-Excite engine (ej-mx):}
     &    \multicolumn{1}{|p{30em}|}{サルサ,踊る,方法,学ぶ}   \\  \hline
     \multicolumn{1}{|p{10em}|}{-Babelfish engine (ej-mb):}
     &    \multicolumn{1}{|p{30em}|}{サルサ,踊る,方法,学ぶ,為,思う}   \\  \hline
    \end{tabular}
   \end{center}
\end{table}



\section{Experiments}

\subsection{Experimental Data}
\label{Experimental Data} We measure our query translation performance
by the IR score achieved by a CLIR system because CLIR is a real
application and includes the performance of keyword expansion. For
this, we did not use word translation accuracy as for the performance of
word-to-word translation, since a one-to-one translation rate is not
suitable, given so many semantically equivalent words.  Our CLIR
experiments are conducted on NTCIR-3 Web Retrieval Task data (100 GB
Japanese documents), in which the Japanese queries and translated
English queries were prepared. The Indonesian queries (47 queries) are
manually translated from English queries. The 47 queries contain 528
Indonesian words (225 are not stop words), 35 English borrowed words,
and 16 transliterated Japanese words (proper nouns). The IR system
\cite{fujii2003} is borrowed from Atsushi Fujii (Tsukuba
University). The query translation system resources are as follows:

\begin{itemize}
 \item Indonesian-English dictionary KEBI \cite{kebi}, 29,054 Indonesian words
 \item English-Japanese dictionary Eijirou \cite{eijiro}, 556,237 English words
 \item Indonesian-Japanese dictionary \cite{kamusjpina}, 14,823 Indonesian words
 \item English stop word list, combined from \cite{fox1990} and \cite{zu2004}
 \item English morphology rules, implement WordNet \cite{wordnet} description
 \item Indonesian morphology rules, restricted only for word repetition, posfix
 -nya and -i
 \item Japanese morphological analyzer Chasen \cite{chasen}
 \item Japanese proper name dictionary\footnote{This dictionary contains 61,629 Japanese person names.}
 \item Mainichi Shinbun newspaper corpus \cite{mainichi}
 \item Daily Yomiuri Online (in English) newspaper corpus \cite{yomiuri}
 \item Indonesian newspaper corpus\footnote{
	Articles downloaded from \texttt{http://ilps.science.uva.nl/Resources/BI/} were used.}
\end{itemize}

The Mainichi Shinbun newspaper corpus is used as the data resource in
the mutual information score calculation between Japanese keywords. The
Daily Yomiuri Online newspaper corpus is used as the data resource in
the mutual information score calculation between English keywords. The
Indonesian newspaper corpus is used to reduce the vocabulary size of the
original Indonesian-Japanese dictionary.


\subsection{Experimental Result}
\label{Experimental Result} In these experiments, we compared the IR
score of each translation method. The IR scores are shown as Mean
Average Precision (MAP) scores. Each query group has 4 MAP scores: RL
(highly relevant document as correct answer with hyperlink information
used), RC (highly relevant document as correct answer), PL (partially
relevant document as correct answer with hyperlink information used),
and PC (partially relevant document as correct answer).

Figure~\ref{fig:3} shows the IR scores of queries translated with basic
translation methods such as the bilingual dictionary or machine
translation, without any enhanced process. All translation candidates
are grouped together and used as the query input for the IR system.

\begin{figure}[t]
 \begin{center}
   \includegraphics[width=\textwidth]{figure3.eps}
 \end{center}
 \caption{IR Score with Indonesian-Japanese Baseline Translation}
 \label{fig:3}
\end{figure}

With only bilingual dictionaries (Indonesian-Japanese and
English-Japanese), the proposed method (iej and ij-iej) gave an IR score
lower than for the transitive translation using machine translation
(iej-mx and iej-mb). The combination between direct and transitive
translation achieved a higher IR result than the direct translation
(ij), but the improvement was not significant. The direct translation
with built-in dictionary (ijn) achieved the lowest IR score which gives
conclusion that the new dictionary (Indonesian-Japanese) has lower
coverage than the two source dictionaries (Indonesian-English and
English-Japanese). The main baseline here is ``iej'', transitive
translation using bilingual dictionaries without any borrowed word
translation and without the keyword selection. The transitive
translation with machine translation (iej-mx and iej-mb) scored higher
(IR) than other translation methods. The highest CLIR score in the
baseline translation only achieved 31\% (iej-mx, MAP score on RL$=$0,0306) compared to the monolingual IR (jp, MAP score on RL$=$0.0985). The dictionary based transitive translation (iej, MAP score on
RL$=$0.0138) and the direct-transitive translation (ij-iej, MAP score on
RL$=$0.0218) achieved 14\% and 22\% compared to the monolingual IR,
respectively.  In Figure~\ref{fig:3}, the translation is only done for
original Indonesian words. This left many OOV words that are borrowed
words. In order to enhance the IR score, these borrowed words are
translated using the supporting resources (the English-Japanese
dictionary, the Japanese proper name dictionary and the Japanese common
noun dictionary). Figure~\ref{fig:4} shows the IR score by translating
the borrowed words. In Figure~\ref{fig:4}, by translating the borrowed
words in the query, each translation method improved the IR score
obtained by the baseline methods in Figure~\ref{fig:3}.

\begin{figure}[b]
  \begin{center}
    \includegraphics[width=0.7\textwidth]{figure4.eps}
  \end{center}
  \caption{IR Score of Indonesian-Japanese CLIR (with Borrowed Word Translation)}
  \label{fig:4}
\end{figure}


The most significant improvement is the direct Indonesian-Japanese
translation.  A combined translation (ij-iej) showed a lower IR score
than the direct translation (ij). We assumed that this lower IR score is
because the combined translation gives too many translation results and
leads to the retrieval of irrelevant documents. This reason is also
applied to the transitive translation (iej) that scored lowest (IR)
among all translation results.  

To reduce the translation candidates
yielded by the translation methods, we performed a keyword selection on
the translation result (see keyword selection details in
section~\ref{Japanese Translation Candidate Filtering Process}). We
experimented with 2 kinds of keyword selection: 1) Japanese keyword
selection, and 2) English keyword selection. Figure~\ref{fig:5} shows
the Japanese keyword selection impact on the IR score and
Figure~\ref{fig:7} shows the IR score achieved by the English and/or
Japanese keyword selection.  

\begin{figure}[b]
 \begin{center}
   \includegraphics[height=0.4\textheight]{figure5.eps}
 \end{center}
 \hangcaption{IR Score of Indonesian-Japanese CLIR (with Borrowed Word
 Translation and Japanese Keyword Selection) for All Queries (47 Queries)}
 \label{fig:5}
\end{figure}

Query label's notation used in
Figure~\ref{fig:5} is xxx-yyy. The prefix ``xxx'' shows the keyword
translation method such as written in Figure~\ref{fig:3}, for example:
iej, iej-mb, etc. The postfix ``yyy'' shows the keyword filtering
method. Figure~\ref{fig:5} shows that the use of keyword selection based
on the combination of mutual information and TF×IDF score (iej-IR-$n$)
yielded significant IR score improvement for the transitive
translation. The proposed transitive translation (iej-IR-10) improved
the IR (RL) score of the baseline method of transitive translation (iej)
from 0.0138 to 0.0371. The $t$-test showed that iej-IR-10 significantly
increased the baseline method (iej) with a 97.5\% confidence level,
T(68)$=$1.92, p$<$0.03. $t$-test also showed that, compared to other
baseline systems, the proposed transitive translation (iej-IR-10) can
increase the IR score at 85\% (T(84)$=$104, p$<$0.15), 69\% (T(86)$=$0.49,
p$<$0.31), 91\% (T(83)$=$1.35, p$<$0.09), and 93\% (T(70)$=$1.49, p$<$0.07)
confidence level for iej-mb, iej-mx, ij and ij-iej, respectively. The IR
score achieved by the transitive translation using bilingual dictionary
was better than those of transitive machine translation (iej-mb-IR and
iej-mx-IR) or direct translation (ijn-IR and ij-IR-5).


Another proposed method, a combination of direct and transitive
translation (ij-iej), achieved the best IR score among all the
translation methods (transitive machine translation, direct translation
and transitive translation using bilingual dictionaries). The proposed
combination translation method (ij-iej-IR-30) improved the IR (RL) score
of the baseline combination translation (ij-iej) from 0.0218 to
0.0486. The $t$-test showed that the proposed combination translation
significantly improved the IR score of the baseline ij-iej with a 98\%
confidence level, T(69)$=$2.09, p$<$0.02. Compared to other baseline
systems, $t$-test showed that the proposed combination translation method
(ij-iej-IR-30) significantly improved the IR score at 95\% (T(83)$=$1.66, p$<$0.05), 
86\% (T(85)$=$1.087, p$<$0.14), 97\% (T(82)$=$1.91, p$<$0.03)
and 99\% (T(67)$=$2.38, p$<$0.005) confidence level for iej-mb, iej-mx, ij
and iej, respectively.  

Figure~\ref{fig:6} shows the IR score of
Indonesian-Japanese CLIR for queries with in-vocabulary words (42
queries). The IR score achievement pattern in Figure~\ref{fig:6} equals
with the achievement pattern for all queries (Figure~\ref{fig:5}). In
Figure~\ref{fig:6}, ij-iej translation achieved the highest IR score of
0.0555 which is 56\% of the monolingual IR.

\begin{figure}[b]
  \begin{center}
    \includegraphics[width=0.8\textwidth,height=0.3\textheight]{figure6.eps}
  \end{center}
  \hangcaption{IR Score of Indonesian-Japanese CLIR (with Borrowed Word
  Translation and Japanese Keyword Selection) for Query with
  In-Vocabulary Words (42 Queries)}
  \label{fig:6}
\end{figure}

\begin{figure}[t]
 \begin{center}
   \includegraphics[width=0.8\textwidth]{figure7.eps}
 \end{center}
 \hangcaption{IR Score of Indonesian-Japanese CLIR with Translation useing
 English and/or Japanese Keyword Selection}
 \label{fig:7}
\end{figure}

Figure~\ref{fig:7} shows the impact of English (pivot language) keyword
selection on the transitive translation. The method is described in
Section~\ref{Japanese Keyword Selection using only Mutual Information
Score in Transitive Translation using Bilingual Dictionary}. The
experimental result shows that using keyword selection on the English
keywords failed to yield a significant improvement in translation.

Figure~\ref{fig:8} shows the IR score of English-Japanese CLIR, which
has 4 translation groups: ej-man (English-Japanese translation using
bilingual dictionary where the English keywords are selected manually),
ej-mb (English-Japanese translation using Babelfish engine), ej-mx
(English-Japanese translation using Excite engine) and ej
(English-Japanese translation using bilingual dictionary).

\begin{figure}[t]
 \begin{center}
   \includegraphics[width=0.8\textwidth]{figure8.eps}
 \end{center}
 \caption{IR Score of English-Japanese CLIR}
 \label{fig:8}
\end{figure}
\begin{figure}[t]
 \begin{center}
       \includegraphics[width=0.8\textwidth]{14-2ia4f9.eps}
 \end{center}
 \caption{IR Score of CLIR using Bilingual Dictionary-based Translation
 and its Dictionary Size}
 \label{fig:9}
\end{figure}



Compared to the Japanese monolingual IR in Figure~\ref{fig:3}, the
English-Japanese CLIR with bilingual-dictionary based translation
(ej-IR-30, MAP score on RL$=$0.0467) achieved 47\% performance. The
Indonesian-Japanese CLIR with transitive translation (iej-IR-30 in
Figure~\ref{fig:5}, MAP score on RL$=$0.0371) achieved 38\% performance
of the Japanese monolingual IR (MAP score on RL$=$0.0985); and the
Indonesian-Japanese CLIR with combined translation between direct and
transitive translation (ij-iej-IR-30 in Figure~\ref{fig:5}, MAP score on
RL$=$0.0486) achieved 49\% IR score of the Japanese monolingual IR,
comparable with the English-Japanese CLIR using the bilingual
dictionary.  

Figure~\ref{fig:9} shows the highest IR score of the CLIR
using bilingual-dictionary based translation (ijn, ij, iej and ij-iej)
and the vocabulary size of each translation. Even though the built-in
Indonesian-Japanese dictionary has a greater vocabulary than the
existing Indonesian-Japanese dictionary, the IR score of the translation
using the built-in Indonesian-Japanese dictionary (ijn) is lower than
that of the translation using the existing dictionary (ij). We assume
that the Japanese keyword selection in the dictionary building process
is not able to select appropriate Japanese translations. The ij-iej
translation uses a merged dictionary, resulted from the existing
Indonesian-Japanese and Indonesian-English dictionaries. The dictionary
size is higher than the existing Indonesian-Japanese (ij) and
Indonesian-English (iej) dictionaries because there are some Indonesian
OOV words of the Indonesian-Japanese dictionary exist in the
Indonesian-English dictionary and vice versa.

Figure~\ref{fig:10} shows the CLIR score with its OOV word number for
CLIR with direct translation using the existing Indonesian-Japanese
bilingual dictionary (ij). Dictionary words are reduced to 3000, 5000
and 8857, respectively. In other words, there are 4 dictionaries with
differing numbers of words; 3000, 5000, 8857 and the complete
Indonesian-Japanese dictionary of 14,823 words. The reduction is done by
selecting the most frequent Indonesian words in an Indonesian newspaper
corpus. Figure~\ref{fig:10} shows that the larger the OOV word number
one translation yielded, the lower IR score it achieved. It shows that
dictionary quality plays a significant role in a CLIR.

\begin{figure}[b]
\vspace{\baselineskip}
 \begin{center}
       \includegraphics[width=273.85pt,height=147.32pt]{14-2ia4f10.eps}
 \end{center}
 \hangcaption{IR Score of CLIR using Indonesian-Japanese Direct Translation
 (ij) with its Word Number and the OOV Word Number}
 \label{fig:10}
\end{figure}



\subsection{Keyword Comparison}
\label{Keyword Comparison}

All figures in Section~\ref{Experimental Result} show the IR score
achieved by each translation method. By comparing the IR score, each
translation result is compared with all words with the same semantic
meaning (one-to-all comparison). We also did a one-to-one comparison
(Table~\ref{table:12}) by comparing the translation result with the
keyword list of the monolingual query (jp). The comparison can be seen
in Table~\ref{table:12}.  Table~\ref{table:12} lists the keyword
comparison between the translation result and the original Japanese
keyword, as indicated by precision and recall scores. There is obviously
no direct correspondence of the precision and recall scores with the IR
score. Even though the combined translation (ij-iej-IR30) with the
highest IR score showed the highest recall and precision score, other
results show different comparisons. For example, the iej-IR10
(transitive translation using bilingual dictionary) had lower recall and
precision scores than ij-IR-5 (direct translation), yet the IR score
achieved by iej-IR-10 was higher than the one achieved by ij-IR-5. We
assumed that this is because the keyword comparison treated the main
keyword and the complement keyword equally, while the main and
complement keywords had a different effect on the information retrieval
score. For example, the query ``Find documents describing how to make
chiffon cake'' has ``chiffon cake'' as the main keyword and ``how to
make'' as the complement of the main keyword.  If the translation system
resulted in only a correct translation of the complement keywords (in
the example, the ``how to make'' word), then the precision and recall
scores of the keyword comparison would be increased, whereas the IR
score would not.

\begin{table}[t]
  \caption{Keyword Comparison between Translation Result and the
 Original Japanese Keyword List}
 \label{table:12}
 \begin{center}
    \begin{tabular}{|l|l||l|l|}\hline
     Query Label & \multicolumn{1}{c|}{Precision} &
     \multicolumn{1}{c|}{Recall} &  \multicolumn{1}{c|}{IR score(RL)} \\
     \hline
     \multicolumn{4}{|l|}{{\it Baseline method}} \\ \hline
     iej-mx & \multicolumn{1}{c|}{22.82\%} &
     \multicolumn{1}{c|}{42.18\%} &  \multicolumn{1}{c|}{0.0306} \\ \hline
     iej-mb & \multicolumn{1}{c|}{21.66\%} &
     \multicolumn{1}{c|}{34.6\%} &  \multicolumn{1}{c|}{0.0197} \\
     \hline
     ij& \multicolumn{1}{c|}{17.49\%} &
     \multicolumn{1}{c|}{33.65\%} &  \multicolumn{1}{c|}{0.0074} \\ \hline
     iej & \multicolumn{1}{c|}{3.63\%} &
     \multicolumn{1}{c|}{40.75\%} &  \multicolumn{1}{c|}{0.0138} \\ \hline
     ij-iej& \multicolumn{1}{c|}{10.78\%} &
     \multicolumn{1}{c|}{39.81\%} &  \multicolumn{1}{c|}{0.0218} \\ \hline
     \multicolumn{4}{|l|}{{\it Compared method}} \\ \hline
     iej-mx-IR & \multicolumn{1}{c|}{23.71\%} &
     \multicolumn{1}{c|}{43.6\%} &  \multicolumn{1}{c|}{0.0336} \\ \hline
     iej-mb-IR & \multicolumn{1}{c|}{23.72\%} &
     \multicolumn{1}{c|}{37.44\%} &  \multicolumn{1}{c|}{0.0238} \\
     \hline
     ij-IR & \multicolumn{1}{c|}{30.12\%} &
     \multicolumn{1}{c|}{35.55\%} &  \multicolumn{1}{c|}{0.0366} \\ \hline
     jp(monolingual)& \multicolumn{1}{c|}{100\%} &
     \multicolumn{1}{c|}{100\%} &  \multicolumn{1}{c|}{0.0985} \\ \hline
     \multicolumn{4}{|l|}{{\it Proposed method}} \\ \hline
     iej-IR10 & \multicolumn{1}{c|}{25.48\%} &
     \multicolumn{1}{c|}{31.28\%} &  \multicolumn{1}{c|}{0.0371} \\ \hline
     ijiej-IR30 & \multicolumn{1}{c|}{37.05\%} &
     \multicolumn{1}{c|}{44.08\%} &  \multicolumn{1}{c|}{0.0486} \\
     \hline
    \end{tabular}
 \end{center}
\end{table}




\section{Conclusions and Future Works}
\label{Conclusions and Future Works}

We presented a translation method that is suitable for queries with a
limited data resource language such as Indonesian. Compared to other
types of translation, such as transitive translation using machine
translation and direct translation using bilingual dictionary (the
source-target dictionary is a poor bilingual dictionary), our transitive
translation and the combined translation (direct translation and
transitive translation) achieved higher IR scores. In our
Indonesian-Japanese CLIR, the transitive translation achieved a 38\%
performance of the monolingual IR and the combined translation achieved
a 49\% performance of the monolingual IR, which is comparable with the
English-Japanese CLIR.  

The two important methods in our transitive
translation are the borrowed word translation and the keyword selection
method. The borrowed word translation method can reduce the number of
OOV from 50 to 5 words using a pivot-target (English-Japanese) bilingual
dictionary and target (Japanese) proper name dictionary. The keyword
selection using the combination of mutual information score and TF×IDF
score gives a significant improvement over the baseline transitive
translation.  The other important method, combining transitive and
direct translation using bilingual dictionaries, also improved the CLIR
performance, and $t$-test showed that it significantly increased the
baseline of transitive translation with a 99\% confidence level.  

We believe that our system can be easily adapted to be able to accept input
queries written in other minor languages. Some tools needed for the
modification are the bilingual dictionary between the query language and
English, the morphological rule for the stemming, and the stop word list
(in the query language) which can be easily translated from the English
stop words.  

As for future projects, we will build an
Indonesian-Japanese CLQA (Cross Language Question Answering) using
architecture similar to the Indonesian-Japanese CLIR.  In the CLQA, we
will attempt to deal with the poor data resource problem of the query
language.

\acknowledgment

We would like to thank Dr. Atsushi Fujii (Tsukuba University) for the
use of the IR Engine in our research.  This work was partially supported
by The 21st Century COE Program ``Intelligent Human Sensing''.

\bibliographystyle{jnlpbbl_1.1}
\begin{thebibliography}{}

\bibitem[\protect\BCAY{Ballesteros \BBA\ Croft}{Ballesteros \BBA\
  Croft}{1998}]{ballesteros1998}
Ballesteros, L.\BBACOMMA\ \BBA\ Croft, W.~B. \BBOP 1998\BBCP.
\newblock \BBOQ Resolving ambiguity for cross-language retrieval\BBCQ\
\newblock In {\Bem SIGIR '98: Proceedings of the 21st annual international ACM
  SIGIR conference on Research and development in information retrieval},
  \mbox{\BPGS\ 64--71}\ New York, NY, USA. ACM Press.

\bibitem[\protect\BCAY{Ballesteros}{Ballesteros}{2000}]{ballesteros2000}
Ballesteros, L.~A. \BBOP 2000\BBCP.
\newblock \BBOQ Cross-Language Retrieval via Transitive Translation\BBCQ\
\newblock In Croft, W.~B.\BED, {\Bem Advances in Information Retrieval},
  \mbox{\BPGS\ 203--230}. Kluwer Academic Publishers.

\bibitem[\protect\BCAY{Eguchi, Oyama, Ishida, Kando, \BBA\ Kuriyama}{Eguchi
  et~al.}{2003}]{NTCIR3WEB}
Eguchi, K., Oyama, K., Ishida, E., Kando, N., \BBA\ Kuriyama, K. \BBOP
  2003\BBCP.
\newblock \BBOQ Overview of Web Retrieval Task at the Third NTCIR
  Workshop\BBCQ\
\newblock In {\Bem Proceedings of the Third NTCIR Workshop}.
\newblock
  \url{http://research.nii.ac.jp/ntcir/workshop/OnlineProceedings3/NTCIR3-OV-W
EB-EguchiK.pdf}.

\bibitem[\protect\BCAY{{Excite Japan}}{{Excite Japan}}{\unskip}]{excite}
{Excite Japan}.
\newblock \BBOQ Excite Machine Translation\BBCQ\
\newblock \url{http://www.excite.co.jp/world/}.

\bibitem[\protect\BCAY{Federico \BBA\ Bertoldi}{Federico \BBA\
  Bertoldi}{2002}]{federico2002}
Federico, M.\BBACOMMA\ \BBA\ Bertoldi, N. \BBOP 2002\BBCP.
\newblock \BBOQ Statistical cross-language information retrieval using n-best
  query translations\BBCQ\
\newblock In {\Bem SIGIR '02: Proceedings of the 25th annual international ACM
  SIGIR conference on Research and development in information retrieval},
  \mbox{\BPGS\ 167--174}\ New York, NY, USA. ACM Press.

\bibitem[\protect\BCAY{Fox}{Fox}{1990}]{fox1990}
Fox, C. \BBOP 1990\BBCP.
\newblock \BBOQ A stop list for general text\BBCQ\
\newblock {\Bem SIGIR Forum}, {\Bbf 24}  (4), \mbox{\BPGS\ 19--21}.
\newblock Issue 2 Fall 89/Winter 90.

\bibitem[\protect\BCAY{Fujii \BBA\ Ishikawa}{Fujii \BBA\
  Ishikawa}{2003}]{fujii2003}
Fujii, A.\BBACOMMA\ \BBA\ Ishikawa, T. \BBOP 2003\BBCP.
\newblock \BBOQ {NTCIR}-3 Cross-Language {IR} Experiments at {ULIS}\BBCQ\
\newblock In {\Bem Proceedings of the Third NTCIR Workshop}.
\newblock
  \url{http://research.nii.ac.jp/ntcir/workshop/OnlineProceedings3/NTCIR3-CLIR
-FujiiA.pdf}.

\bibitem[\protect\BCAY{Gao, Nie, Xun, Zhang, Zhou, \BBA\ Huang}{Gao
  et~al.}{2001}]{gao2001}
Gao, J., Nie, J.-Y., Xun, E., Zhang, J., Zhou, M., \BBA\ Huang, C. \BBOP
  2001\BBCP.
\newblock \BBOQ Improving query translation for cross-language information
  retrieval using statistical models\BBCQ\
\newblock In {\Bem SIGIR '01: Proceedings of the 24th annual international ACM
  SIGIR conference on Research and development in information retrieval},
  \mbox{\BPGS\ 96--104}\ New York, NY, USA. ACM Press.

\bibitem[\protect\BCAY{Gollins \BBA\ Sanderson}{Gollins \BBA\
  Sanderson}{2001}]{gollins2001}
Gollins, T.\BBACOMMA\ \BBA\ Sanderson, M. \BBOP 2001\BBCP.
\newblock \BBOQ Improving cross language retrieval with triangulated
  translation\BBCQ\
\newblock In {\Bem SIGIR '01: Proceedings of the 24th annual international ACM
  SIGIR conference on Research and development in information retrieval},
  \mbox{\BPGS\ 90--95}\ New York, NY, USA. ACM Press.

\bibitem[\protect\BCAY{{Indonesian Agency for The Assessment and Application of
  Technology}}{{Indonesian Agency for The Assessment and Application of
  Technology}}{}]{kebi}
{Indonesian Agency for The Assessment and Application of Technology}.
\newblock \BBOQ KEBI, Kamus Elektronik Bahasa Indonesia\BBCQ\
\newblock \url{http://nlp.aia.bppt.go.id/kebi/}.

\bibitem[\protect\BCAY{Kishida \BBA\ Kando}{Kishida \BBA\
  Kando}{2004}]{kishida2004}
Kishida, K.\BBACOMMA\ \BBA\ Kando, N. \BBOP 2004\BBCP.
\newblock \BBOQ Two-stage refinement of query translation in a pivot language
  approach to cross-lingual information retrieval: {A}n experiment at {CLEF}
  2003\BBCQ\
\newblock In {\Bem Bridging Languages for Question Answering: DIOGENE at CLEF
  2003}, Lecture Notes in Computer Science, \mbox{\BPGS\ 253--262}. Springer,
  Berlin/ Heidelberg.

\bibitem[\protect\BCAY{Kosasih}{Kosasih}{2003}]{kosasih2003}
Kosasih, E. \BBOP 2003\BBCP.
\newblock {\Bem Kompetensi Ketatabahasaan dan Kesusastraan, Cermat Berbahasa
  Indonesia}.
\newblock CV. YRAMA WIDYA, Bandung, Indonesia.
\newblock (in Indonesian).

\bibitem[\protect\BCAY{{Mainichi Shimbun Co.}}{{Mainichi Shimbun Co.}}{1994
  1996}]{mainichi}
{Mainichi Shimbun Co.} \BBOP 1994--1996\BBCP.
\newblock {\Bem CD-ROM Data Sets 1993--1995}.
\newblock Nichigai Associates Co.

\bibitem[\protect\BCAY{Matsumoto, Kitauchi, Yamashita, Hirano, Matsuda,
  Takaoka, \BBA\ Asahara}{Matsumoto et~al.}{2000}]{chasen}
Matsumoto, Y., Kitauchi, A., Yamashita, T., Hirano, Y., Matsuda, H., Takaoka,
  K., \BBA\ Asahara, M. \BBOP 2000\BBCP.
\newblock \BBOQ {M}orphological {A}nalysis {S}ystem {C}ha{S}en version 2.2.1
  {M}anual\BBCQ\
\newblock \url{http://chasen.aist-nara.ac.jp/chasen/doc/chasen-2.2.1.pdf}.

\bibitem[\protect\BCAY{Michibata}{Michibata}{2002}]{eijiro}
Michibata, H.\BED\ \BBOP 2002\BBCP.
\newblock {\Bem Eijiro}.
\newblock ALC.
\newblock (in Japanese).

\bibitem[\protect\BCAY{Miller}{Miller}{1995}]{wordnet}
Miller, G.~A. \BBOP 1995\BBCP.
\newblock \BBOQ WordNet: a lexical database for English\BBCQ\
\newblock {\Bem Commun. ACM}, {\Bbf 38}  (11), \mbox{\BPGS\ 39--41}.

\bibitem[\protect\BCAY{Mirna}{Mirna}{2000}]{mirna2000}
Mirna, A. \BBOP 2000\BBCP.
\newblock \BBOQ Using Statistical Term Similarity for Sense Disambiguationin
  Cross-Language Information Retrieval\BBCQ\
\newblock {\Bem Infomation Retrieval}, {\Bbf 2}  (1), \mbox{\BPGS\ 71--82}.

\bibitem[\protect\BCAY{{Overture Services, Inc}}{{Overture Services,
  Inc}}{\unskip}]{babelfish}
{Overture Services, Inc}.
\newblock \BBOQ Altavista Babelfish Machine Translation\BBCQ\
\newblock \url{http://www.altavista.com/babelfish/}.

\bibitem[\protect\BCAY{Qu, Grefenstette, \BBA\ Evans}{Qu et~al.}{2002}]{qu2002}
Qu, Y., Grefenstette, G., \BBA\ Evans, D.~A. \BBOP 2002\BBCP.
\newblock \BBOQ Resolving Translation Ambiguity using Monolingual Corpora\BBCQ\
\newblock In Peters, C., Braschler, M., Gonzalo, J., \BBA\ Kluck, M.\BEDS,
  {\Bem Advanced in Cross-Language Information Retrieval (CLEF2002)},
  \mbox{\BPGS\ 223--241}. Springer, Berlin / Heidelberg.

\bibitem[\protect\BCAY{{Sanggar Bahasa Indonesia Proyek}}{{Sanggar Bahasa
  Indonesia Proyek}}{2000}]{kamusjpina}
{Sanggar Bahasa Indonesia Proyek} \BBOP 2000\BBCP.
\newblock \BBOQ KMSMINI2000\BBCQ\
\newblock
  \url{http://m1.ryu.titech.ac.jp/~indonesia/todai/dokumen/kamusjpina.pdf}.

\bibitem[\protect\BCAY{Tanaka \BBA\ Umemura}{Tanaka \BBA\
  Umemura}{1994}]{tanaka1994}
Tanaka, K.\BBACOMMA\ \BBA\ Umemura, K. \BBOP 1994\BBCP.
\newblock \BBOQ Construction of a bilingual dictionary intermediated by a third
  language\BBCQ\
\newblock In {\Bem Proceedings of the 15th conference on Computational
  linguistics}, \lowercase{\BVOL}~1, \mbox{\BPGS\ 297--303}\ Morristown, NJ,
  USA. Association for Computational Linguistics.

\bibitem[\protect\BCAY{{Toggle Text}}{{Toggle Text}}{\unskip}]{kataku}
{Toggle Text}.
\newblock \BBOQ Kataku Automatic Translation System\BBCQ\
\newblock \url{http://www.toggletext.com/kataku_trial.php}.

\bibitem[\protect\BCAY{{Yomiuri Shimbun Co.}}{{Yomiuri Shimbun
  Co.}}{2001}]{yomiuri}
{Yomiuri Shimbun Co.} \BBOP 2001\BBCP.
\newblock {\Bem Article Data of Daily Yomiuri 2001}.
\newblock Nihon Database Kaihatsu Co.
\newblock \url{http://www.ndk.co.jp/yomiuri/e_yomiuri/e_index.html}.

\bibitem[\protect\BCAY{Zu, Ohyama, Wakabayashi, \BBA\ Kimura}{Zu
  et~al.}{2004}]{zu2004}
Zu, G., Ohyama, W., Wakabayashi, T., \BBA\ Kimura, F. \BBOP 2004\BBCP.
\newblock \BBOQ Automatic text classification of {E}nglish newswire articles
  based on statistical classification techniques\BBCQ\
\newblock {\Bem IEEJ Transactions Electronics, Information and Systems}, {\Bbf
  124--C}  (3), \mbox{\BPGS\ 852--860}.

\end{thebibliography}


\begin{biography}

\bioauthor{Ayu Purwarianti}{
  Graduated from Bandung Institute of Technology for her Bachelor 
  and Master degree in 1998 and 2002, respectively. Now, she has been 
  studying at Toyohashi University of Technology as doctoral student 
  since 2004. Her research interest is in natural language processing 
  area, especially for Indonesian language. 
}

\bioauthor{土屋 雅稔}{
  1998年京都大学工学部 電気工学科第二学科 卒業.
  2004年京都大学大学院 情報学研究科 知能情報学専攻 博士課程単位認定退学.
  京都大学修士(情報学).
  2004年より
  豊橋技術科学大学 情報メディア基盤センター 助手.
  自然言語処理に関する研究に従事.
}

\bioauthor{中川 聖一}{
  1976年京都大学大学院博士課程修了.
  同年,京都大学情報工学科助手.1980年豊橋技術科学大学情報工学系講師.
  1990年教授.1985--1986年カーネギメロン大学客員研究員.
  音声情報処理,自然言語処理,人工知能の研究に従事.
  工学博士.
  1977年電子通信学会論文賞,1988年IETE最優秀論文賞,
  2001年電子情報通信学会論文賞,各受賞.
  電子情報通信学会フェロー.
  著書「確率モデルによる音声認識」(電子情報通信学会編),
  「音声聴覚と神経回路網モデル」(共著,オーム社),
  「情報理論の基礎と応用」(近代科学社),
  「パターン情報処理」(丸善),
  「Spoken Language Systems」(編著,IOS Press)など.
}
\end{biography}







\biodate


\end{document}
