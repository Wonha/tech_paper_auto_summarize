\documentstyle[ling-macros,ascmac,epsf,jnlpbbl]{jnlp_j_b5}




\renewcommand{\theenums}{}

\newcommand{\utt}[2]{}

\newcommand{\da}{}
\newcommand{\ua}{}

\newtheorem{df}{}
\newtheorem{pr}{}
\newtheorem{co}{}
\newtheorem{as}{}

\newenvironment{definit}[1]{}{}
\newenvironment{principle}[1]{}{}
\newenvironment{constr}[1]{}{}
\newenvironment{assum}[1]{}{}

\newtheorem{obs}{}
\newtheorem{hosoku}{}

\newcommand{\figlabel}[1]{}
\newcommand{\tablelabel}[1]{}
\newcommand{\eqlabel}[1]{}
\newcommand{\deflabel}[1]{}
\newcommand{\constrlabel}[1]{}
\newcommand{\prilabel}[1]{}
\newcommand{\exslabel}[1]{}

\newcommand{\figref}[1]{}
\newcommand{\tableref}[1]{}
\newcommand{\eqref}[1]{}
\newcommand{\defref}[1]{}
\newcommand{\constrref}[1]{}
\newcommand{\priref}[1]{}
\newcommand{\exsref}[1]{}

\newcommand{\asmlabel}[1]{}
\newcommand{\asmref}[1]{}
\newcommand{\obslabel}[1]{}
\newcommand{\obsref}[1]{}
\newcommand{\hosokulabel}[1]{}
\newcommand{\hosokuref}[1]{}

\newcommand{\lw}[1]{}

\def\susumu{}


\setcounter{page}{61}
\setcounter{巻数}{4}
\setcounter{号数}{4}
\setcounter{年}{1997}
\setcounter{月}{10}
\受付{1996}{8}{20}
\再受付{1997}{3}{26}
\採録{1997}{7}{18}

\setcounter{secnumdepth}{2}

\title{
日本語の自由会話における談話構造の推定 \\ 〜因果関係を表す接続詞の場合〜
}
\author{西澤 信一郎\affiref{fujitsu} \and 中川 裕志\affiref{YNU}}

\headauthor{西澤 信一郎・中川 裕志}
\headtitle{
日本語の自由会話における談話構造の推定 〜因果関係を表す接続詞の場合〜
}

\affilabel{fujitsu}{富士通株式会社}
{Fujitsu Limited}
\affilabel{YNU}{横浜国立大学 電子情報工学科}
{Division of Electrical and Computer Engineering, Faculty of
Enginnering, Yokohama National University}

\jabstract{
日本語の談話理解を考える際には,文脈,すなわち「会話の流れ」を把握する必
要がある.
一般的に日本語では,「会話の流れ」を明示する語として,順接・逆接・話題の
転換・因果性,などを表す接続(助)詞が用いられることが多い.
これらの語は,スケジュール設定など何らかの話題・目的が存在する会話だけで
はなく,雑談などの場合でも,聞き手が「会話の流れ」を把握するために利用し
ているものと考えられる.
本技術資料では,「だって」や「から」などの接続表現によって,因果関係の前
件および後件の関係が談話中で明示されている場合を対象とし,そのような因果
関係が談話中で示す特徴について検討する.
この検討から,いくつかの観察結果が得られるが,それについて自由会話コーパ
スを用いた検討を行ない,実際にそのような特徴が成立することを確認した.
この結果は,接続表現に依存する前件と後件の順序関係,および前件と後件の隣
接性という2項目にまとめることができ,機械的処理による談話理解への足掛か
りと考えられる.
}

\jkeywords{
自由会話,談話構造,因果関係,接続詞,コーパス
}

\etitle{
A Method of Discourse Structure Understanding \\ in Japanese Task-Free
Conversation \\ for Causal Conjunction}
\eauthor{Shin'ichiro Nishizawa\affiref{fujitsu} \and Hiroshi
Nakagawa\affiref{YNU}}

\eabstract{
We discuss here how a discourse structure for a causal relation between
two or more utterances is linguistically expressed in Japanese task-free 
conversations.
When we have a conversation, we often use conjunctions as a sign of
causal relation, change of topic, etc.
So, these words are important to understand a discourse structure
because we often use these words to represent ``an intention of
utterance'' or ``a stream of discourse.''
Here, we discuss a case where causal conjunctions are used in Japanese
task-free conversations by examining our corpus of causal conversation,
and show how to identify relations, represented by these conjunctions,
among a few adjacent sentences systematically.
}

\ekeywords{
Japanese task-free conversation, discourse structure, causal relation,
Japanese conjunction, corpus
}

\begin{document}
\maketitle

\section{はじめに}

日本語の談話理解を考える際,文脈すなわち「会話の流れ」の認識は重要な要素
となる.
一般的に日本語では,「会話の流れ」を明示するために順接・逆接・話題転換・
因果性,などを表す接続(助)詞が用いられることが多い.
このことから,接続(助)詞を含む発話とそれと組になる発話,という関係を認識
することが,談話理解の基本となると考えられる.
これについては,マニュアルや論説文などのいわゆる書き言葉について,接続詞
や指示語などによる連接パターンを用いてテキストの構造解析を行なう手法
\cite{福本:文の連接関係解析,田中:文の連接パターン}や,対話中の質問--応答
を表す発話対の認識に関する研究\cite{高野:発話対の認識手法について}などが
ある.
これに対して本研究では,「だって」や「から」などの接続表現により因果関係
の前件及び後件の関係が談話中で明示されている場合を対象とし,そのような因
果関係が談話中でどのような特徴を伴って出現するのか,について検討する.
また,この検討結果を,特に課題を設定していない状況での会話(自由会話)によ
るコーパスを用いて検証する.
このような,文の意味内容に関する連接関係については,
\cite{Hobbs:StructureOfDiscourse}で因果関係その他いくつかの場合について
述べられているが,ここでは,接続表現により前件と後件の連接関係が明示され
ている場合を主な対象とするものである.
なお,本技術資料では,会話データとして\figref{コーパス例}のようなコーパ
スを用いる.
\begin{figure}[htbp]
 {\small
 \setlength{\baselineskip}{2.0mm}
 {\bf 会話24}
 \begin{enumerate} 
  \item O → P あのね、これでもいいんじゃん\da
  \item P → O わかった\da
  \item O → P えー、嘘で言ったんだよ\da
  \item E → O 何\ua
  \item O → E だって、牛乳入れろって言ってたらさー\da
  \item G → G 何か酒飲みたいなー\da
  \item K → G あっ、ありますよ\da
  \item G → K それ何\ua
  \item E → G モルツ\da
  \item P → G ウイスキー\da
  \item E → G うまいよ\da
 \end{enumerate}   
 }
\caption{コーパスの例}
\figlabel{コーパス例}
\end{figure}
このコーパスは,大学のあるサークルでの飲み会の席上で録音された雑談(課題
を特に設定していない自由会話)を,そこに同席した者がテキストに書きおこし
たものであり,全部で1980の発話を含む.
書きおこす際に,(1):発話の切れ目の認識\footnote{発話の切れ目は原則として
話し手の交代時としているが,会話に同席した者が,発話が区切れていると判断
した場合には,話し手の交代に関わりなく発話の切れ目としている.この時,発
話間には平均して約0.5秒のギャップがある.},(2):会話内容によるセグメント
分け,(3):話し手と聞き手のデータ追加,(4):発話の末尾の調子のデータ追加,
を行なっており,例えば,「O → P あのね、これでもいいんじゃん\da 」 \hspace{-.4em}とい
う発話では,話し手が``O''で聞き手が``P''であり,末尾が下がり調子の発話
であったことを示している.
また,このコーパスでは,因果関係を表すとされる接続詞「だから/だって」お
よび接続助詞(相当)「ので/から/のだから/のだもの」が用いられており,本論
文ではこれらに注目して考察を行なう.

\section{因果関係を表わす発話の性質}

\subsection{接続表現による発話順序への影響}

本技術資料では,因果関係を表わす接続表現として,接続詞「だから/だって」,
接続助詞もしくはそれ相当の表現として「ので/から/のだから/のだもの」
\footnote{以降ではこれらの語を「接続助詞類」と述べることにする.}を対象
とする.
ここではまず,これらが談話中でどのように因果関係の前件及び後件を関係付け
るかについて述べる.

はじめに,接続助詞類が用いられる場合であるが,発話の一例を\exsref{発話例
1}に示す\footnote{各発話は ``話し手→聞き手 『発話本体』 末尾の調子''と
表記し,末尾の調子は ``\da''が下がり調子,``\ua''が上がり調子を示す.}.

\eenumsentence{\exslabel{発話例1}
\item A → B このお酒,とっても美味しいんだから \da
\item A → B とりあえず,飲んでご覧なさいよ \da
}

この発話例のように,接続助詞類はそれを含む発話が因果関係の前件であること
を示す.これと対応する後件は\exsref{発話例1}では前件の後方に位置するが,
この順序は\exsref{発話例2}のように逆転してもよい.

\eenumsentence{\exslabel{発話例2}
\item A → B とりあえず,飲んでご覧なさいよ \da
\item A → B このお酒,とっても美味しいんだから \da
}

因果関係を表わす複文では,主節及び従属節の倒置現象が頻繁に生じるが,上記
の現象もこれに類するものであり,\exsref{発話例1},\exsref{発話例2}ともに
因果関係を認識可能である.

以上の考察は次のようにまとめられる.

\begin{screen}
 \begin{obs}\obslabel{接続助詞}
  接続助詞類による因果関係では,その前件を表わす発話と後件を表わす発話と
  の間の順序関係は(1)「前件→後件」,(2)「後件→前件」のいずれも可能であ
  る.
 \end{obs}
\end{screen}

次に,接続詞が用いられる場合である.
この場合,接続詞で関係付けられる発話は,基本的には接続詞を含む発話とその
前方に位置する発話である\cite{三上:現代語法序説,森田:基礎日本語2}.
例えば,「だから」の場合は\exsref{発話例3},「だって」の場合は\exsref{発
話例4}のような発話例が考えられる.

\eenumsentence{\exslabel{発話例3}
\item A → B とても面白い落語だったのよ \da
\item A → B だから,すごく笑っちゃった \da
}

\eenumsentence{\exslabel{発話例4}
\item A → B すごく笑っちゃった \da
\item A → B だって,とても面白い落語だったのよ \da
}

このように,「だから」の場合はそれを含む発話が後件を表わし,「だって」の
場合はそれを含む発話が前件となるが,どちらもそれぞれ接続詞を含む発話とそ
の前方に位置する発話とが関係付けられる.
この発話の順序は,「だから」の場合は逆転不可能であり,例えば\exsref{発話
例3}の発話の順序を入れ換えた\exsref{発話例5}において,\exsref{発話例5a}
の前件が\exsref{発話例5b}であるという解釈はできない\footnote{ ``$\ast$''
を含む発話例は,それが因果関係として解釈不可能であることを示す.}.

\eenumsentence{\exslabel{発話例5}
\item $\ast$ A → B だから,すごく笑っちゃった \da \exslabel{発話例5a}
\item $\ast$ A → B とても面白い落語だったのよ \da \exslabel{発話例5b}
}

一方,「だって」の場合は,\exsref{発話例6}のように,前件と後件の順序が逆
であっても因果関係が認識可能な場合がある.

\eenumsentence{\exslabel{発話例6}
\item A → B だって,とても面白くって \da \exslabel{発話例6a}
\item A → B すごく笑っちゃった \da \exslabel{発話例6b}
}

この場合,\exsref{発話例6a}を前件,\exsref{発話例6b}を後件とする因果関係
として解釈することが可能であるが,これには次の要素が影響しているものと考
えられる.

\begin{enumerate}
 \item 「だって」が因果関係の前件を示す標識であり,\exsref{発話例6}のよ
       うに順序を入れ替えた場合には「前件→後件」という発話順序となるこ
       と.
 \item 「だって」を含む発話が連用形で終わっており,「連用接続による因果
       関係の記述」という解釈が可能であること.
\end{enumerate}

前者について「だから」の場合と比較すると,\exsref{発話例3}のような「だか
ら」が用いられる因果関係では,\exsref{発話例5}のように順序を入れ替えた場
合は「後件→前件」という発話順序となってしまうことから解釈不可能となるが,
「だって」ではそうではない.
後者については,\exsref{発話例7}のように接続助詞類を用いるとさらにはっき
りする.

\eenumsentence{\exslabel{発話例7}
\item A → B だって,とても面白いんだもの \da \exslabel{発話例7a}
\item A → B すごく笑っちゃった \da \exslabel{発話例7b}
}

このような場合,前件と後件を関係付けるのは主に連用接続や接続助詞類の力で
あり,「だって」はそれを含む発話が因果関係の前件であることを強調する役目
を果たしていると考えることができる.
なお,「だから」を含む発話は因果関係の後件となるため,例えば\exsref{発話
例8}のように連用接続とした場合でもやはり二つの発話を因果関係として認識す
ることは不可能である.

\eenumsentence{\exslabel{発話例8}
\item $\ast$ A → B だから,すごく笑っちゃって \da \exslabel{発話例8a}
\item $\ast$ A → B とても面白い落語だったのよ \da \exslabel{発話例8b}
}

以上から,接続詞による因果関係では,\obsref{接続助詞}に対して次のような
ことがいえる.

\begin{screen}
 \begin{obs}\obslabel{接続詞}
  接続詞「だから」による因果関係では,「A.だからB.」の順に発話がなされ,
  Aが前件,Bが後件を表わす.またこの順序が逆転することはない.

  接続詞「だって」による因果関係では,「A.だってB.」の順に発話がなされ,
  Aが後件,Bが前件を表わす.またこの順序はBが連用形で終わっている場合や
  因果関係を表わす接続助詞類で終わっている場合は「だってB.A.」の順序に
  逆転可能である.この時,因果関係は主に連用形や接続助詞類によって示され,
  「だって」はそれを含む発話が因果関係の前件であることを強調する役目を果
  たす.
 \end{obs}
\end{screen}

\subsection{前件と後件の隣接性}

次に,\obsref{接続助詞},\obsref{接続詞}のような発話順序による因果関係の
前件及び後件を述べる発話間の距離,及びそれぞれの発話の話し手の関係につい
て検討する.

まず,発話間の距離としてもっとも基本的な場合として,前件及び後件を述べる
発話が隣接している場合が考えられる.発話の例をいくつか示す.

\eenumsentence{\exslabel{発話例a}
\item A → B これ,飲んでよ \da
\item A → B おいしいから \da
}

\eenumsentence{\exslabel{発話例b}
\item A → B 飲み過ぎてしまった \da
\item A → B だから,頭が痛くてたまらないんだ \da
}

このとき,両発話の話し手は同一である必要はない.例えば\exsref{発話例a}が
\exsref{発話例c}のように発話された場合は,「AおよびCがBに勧める」という
状況であると解釈可能である.

\eenumsentence{\exslabel{発話例c}
\item A → B これ,飲んでよ \da
\item C → B おいしいから \da
}

また,\exsref{発話例d}では「AおよびBが一緒に酒を飲んだ」という前提がある
ものとして解釈可能である.

\eenumsentence{\exslabel{発話例d}
\item A → B 飲み過ぎてしまった \da
\item B → A だから,頭が痛くてたまらないんだ \da
}

結果として,前件及び後件の発話間の距離に関しては,まず次のような事柄が考
えられる.

\begin{screen}
 \begin{obs}\obslabel{基本距離}
  因果関係の前件及び後件を表わすそれぞれの発話は,談話中で隣接して出現す
  る.この時,それぞれの発話の話し手は同一であっても異なっていてもよい.
 \end{obs}
\end{screen}

しかし,接続詞もしくは接続助詞類を含む発話と組となる発話が隣接して存在し
ない場合も考えられる.
これにはおおまかにわけて次のような場合があげられる.
\begin{itemize}
 \item[($\alpha$)] 前件もしくは後件のいずれかもしくは双方が複数の発話か
		   らなる場合.
 \item[($\beta$)] 前件と後件それぞれを表す発話の間に,質問あるいは同意を
		   表す発話が存在する場合.
 \item[($\gamma$)] 直接的に組となる発話は存在しないものの,発話の含意や
		   前提などから,因果関係の前件と後件の関係が間接的に判
		   明する場合.
\end{itemize}

まず,($\alpha$)の一例を示す.

\eenumsentence{\exslabel{発話例e}
\item A → B 飲んでみて \da \exslabel{発話例e1}
\item A → B 美味しいんだから \da \exslabel{発話例e2}
\item C → B そんなにアルコール強くないから \da \exslabel{発話例e3}
}

\exsref{発話例e1}と\exsref{発話例e2}は因果関係を表わす隣接発話の組である
が,一方\exsref{発話例e1}と\exsref{発話例e3}も,隣接していないものの,因
果関係を表わすと解釈可能である.
これは,\exsref{発話例e2}と\exsref{発話例e3}が前件を表わす一つのグループ
としてまとまっており,このグループと,後件を表わす\exsref{発話例e1}が隣
接していると捉えることができる.

また,($\alpha$)の例として,後件が複数の発話からなる場合の例が\exsref{発
話例ee}である.

\eenumsentence{\exslabel{発話例ee}
\item A → B 後かたづけは我々がやっておくから\da \exslabel{発話例ee1}
\item C → B 心配しなくていいよ\da \exslabel{発話例ee2}
\item A → B もう帰ってもいいよ\da \exslabel{発話例ee3}
}

この場合は,\exsref{発話例ee2}および\exsref{発話例ee3}がまとまって後件を
表しており,その前件が\exsref{発話例ee1}であると捉えることができる.

次に,($\beta$)の例として,質問が間に入るような発話例を次に示す.

\eenumsentence{\exslabel{発話例f}
\item A → B もう帰る \da \exslabel{発話例f1}
\item B → A どうして \ua \exslabel{発話例f2}
\item A → B だって,もう疲れたよ \da \exslabel{発話例f3}
}

この例では,表面上は\exsref{発話例f1}と\exsref{発話例f3}が因果関係を表し
ており,その間に存在する\exsref{発話例f2}は\exsref{発話例f1}への質問であ
る.
これは「B → A どうして,もう帰るの \ua 」という発話の後半(\exsref{発話
例f1}で述べられている内容)を省略したものである.
このことから,\exsref{発話例f3}は\exsref{発話例f2}において省略された部分
を後件とした時の前件であるとみなすことも可能である.
つまり,因果関係の前件に関与する発話\exsref{発話例f1},\exsref{発話例f2}
に対し,\exsref{発話例f3}によって後件が述べられているとみなすことができ
る.

なお,($\beta$)は形式的な特徴として「質問文に接続表現を含む発話が隣接す
る」と捉えられるが,この場合の例外として,「だから」を含む発話の前方に質
問を示す発話が隣接する場合があげられる.
その一例を\exsref{発話例外}に示す.

\eenumsentence{\exslabel{発話例外}
\item D → A 明日は打ち合わせがあるんですよね\ua
\item A → D そうだよ\da
\item C → A えっ\ua \exslabel{発話例外c}
\item A → C だから,明日は打ち合わせだってば.\da
\exslabel{発話例外d}
}
質問を示す発話と隣接するという点では\exsref{発話例f}や後述する\exsref{発
話例l}と同様であるが,\exsref{発話例外}ではこれが因果関係を述べると仮定
した場合に\exsref{発話例外d}の前件となるような意味内容を含む発話が述べら
れておらず,\exsref{発話例外c}から推測することも不可能である.
実際には,この発話例では\exsref{発話例外d}は\exsref{発話例外c}を発した人
物Mに対して「それは前に説明したはずだが,覚えていないのか」などのような
非難の態度を表明するものであり,因果関係についての発話ではないと考えられ
る\cite{白川:理由を表さない「カラ」}.
これについては,つぎの観察として述べることができる.

\begin{screen}
\begin{obs} \obslabel{例外}
 質問を示す発話に対し,「だから」が含まれる発話が隣接する場合,「だから」
 を含む発話が因果関係に関与しない場合がある.
\end{obs}
\end{screen}

($\beta$)のもう一つの場合として,同意を示す発話が挿入される例を次に示す.

\eenumsentence{\exslabel{発話例g}
\item A → B もう疲れた \da \exslabel{発話例g1}
\item B → A そうだね \da \exslabel{発話例g2}
\item A → B だから,帰るよ \da \exslabel{発話例g3}
}

この例でも,表面上は\exsref{発話例g1}と\exsref{発話例g3}が因果関係を表し
ている.さらに\exsref{発話例g2}は\exsref{発話例g1}で述べられている内容へ
の同意を示している.
この発話は,「B → A そうだね,私ももう疲れた \da 」のように言い換えるこ
とが可能であり,\exsref{発話例g1}による意味内容を含んでいるとみなすこと
ができる.
このことから,因果関係の前件の意味内容は\exsref{発話例g1},\exsref{発話
例g2}の双方に含まれ,それに対する後件が\exsref{発話例g3}で述べられている
とみなすことができる.

このように,因果関係の前件もしくは後件が複数の発話の意味内容に含まれてい
ると解釈可能な場合,それを{\bf 発話群}として次のように定義する.

\begin{screen}
 \begin{df}\deflabel{発話群}
  因果関係の前件もしくは後件が連続する複数の発話の意味内容に含まれている
  と解釈されるとき,それを発話群と呼ぶ.この時,発話群中には前件もしくは
  後件の意味内容を明示する発話が一つ以上必ず含まれる.
 \end{df}
\end{screen}

上の例文では,\exsref{発話例e2},\exsref{発話例e3}が前件となる発話群であ
り,双方の発話によって前件の意味内容が明示されている.
また,\exsref{発話例f1},\exsref{発話例f2}は後件となる発話群であり,
\exsref{発話例f1}が後件の意味内容を明示している.
さらに,\exsref{発話例g1},\exsref{発話例g2}も前件となる発話群であり,や
はり\exsref{発話例g1}が前件の意味内容を明示しているととらえることが可能
である.

そして,これまでの議論から\obsref{基本距離}は次のように言い換えることが
可能となる.

\begin{screen}
 \begin{obs}\obslabel{距離a}
  因果関係の前件及び後件が談話中で明示される場合,次のいずれかの構造をと
  ることが可能である.
 \begin{enumerate}
  \item 前件及び後件がそれぞれ一つの発話で表わされ,それらが隣接する.
  \item 前件あるいは後件のいずれかが一つの発話で,もう一方が発話群により
	表わされ,それらが隣接する.
  \item 前件及び後件の双方が発話群により表わされ,それらが隣接する.
 \end{enumerate}
 なお,すべての場合において,前件及び後件に関係する話し手についての制限
 はなく,前件と後件が同一話者による場合も異なる話者による場合も存在する.
 \end{obs}
\end{screen}

一方,($\gamma$)で述べたように,直接的に組となる発話は存在しないものの,
発話の含意あるいは前提から因果関係の前件と後件との関係が間接的に認識可能
な場合がある.
例えば,\exsref{発話例l}では,\exsref{発話例l1}の発話により「Bが飲んでい
ない」あるいは「Bが飲むことをストップしている」などという前提が導かれ,
これを後件として\exsref{発話例l2}が因果関係の前件を述べていると解釈可能
である.

\eenumsentence{\exslabel{発話例l}
\item A → B 今日はもう飲まないの \ua \exslabel{発話例l1}
\item B → A だってもう眠いし \da \exslabel{発話例l2}
}

これは,質問に対する返答となるという点では\exsref{発話例f}の場合と類似し
ているが,\exsref{発話例f}では\exsref{発話例f1}として因果関係の後件の意
味内容が明示されているのに対し,\exsref{発話例l}では\exsref{発話例l2}の
後件の意味内容が明示されていない点で異なるものであり,これらは区別して扱
う必要があると考えられる.
なお,この例では「質問--応答」という形式から,前件と後件との関係を認識す
るための間接的な手がかりとなる発話を比較的容易に認識可能であるが,例えば
「周囲の人がお酒を探している」など平叙文により述べられている発話状況に対
して「ここにお酒がたくさんあるから.」という発話がなされる場合も考えられ
る.この場合は,発話状況や文脈などから因果関係が存在するかどうかを推測す
る必要がある.

以上のように,接続詞あるいは接続助詞類が談話中で用いられ,それによる因果
関係の前件及び後件が発話中に明示されている場合は,\obsref{接続助詞}〜
\obsref{距離a}に述べた事柄が観察されると考えられる.
これについては,次章で実際にコーパスを用いて検証を行なう.

一方,接続詞あるいは接続助詞を含む発話に対し,それと対応する前件あるいは
後件を発話の含意あるいは前提をもとに推測する必要がある場合も存在する.
この現象は本技術資料での観察の対象外であり,次章での検証でもその主な対象
から省くこととする.

\section{コーパスによる検証}

ここでは,前章での議論をふまえ,実際の会話中に現れる発話の組について
\figref{コーパス例}に例を示したコーパスを用いて調べる.
このコーパスは1980発話からなり,因果関係を表す接続詞として「だから/だっ
て」,接続助詞類として「ので/から/のだから/のだもの」が用いられている発
話が121発話存在する.
これから,「頼むから,もう撮るのを止めて下さい」など通常の複文が発話され
ている16例を除いた105例のうち,因果関係の前件及び後件の双方が談話中に明
示されていると認められる例が61発話存在する.

これを次のように分類する.

\begin{itemize}
 \item \obsref{接続助詞},\obsref{接続詞}より,前件,後件の位置関係には
       次のような場合が考えられる.

       \begin{center}
	\begin{tabular}{rlcrl}
	 (a1) & 前件.だから---後件. & & (b1) & 前件---接続助詞.後件.
	 \\
	 (a2) & 後件.だって---前件. & & (b2) & 後件.前件---接続助詞.
	 \\
	 (a3) & だって---前件.後件. & & &
	\end{tabular}
       \end{center}

 \item \obsref{距離a}より,前件と後件の間の距離およびそれぞれの話し手に
       は次のような場合が考えられる.

       \begin{itemize}
	\item[(A)] 前件および後件がそれぞれ隣接する一つの発話で表され,
		   双方の話し手が同一人物である.
	\item[(B)] 前件および後件がそれぞれ隣接する一つの発話で表され,
		   双方の話し手は異なる人物である.
	\item[(C)] 前件もしくは後件のいずれかが発話群である.
       \end{itemize}

       さらに,上記の(C)を次のように分類する.
       \begin{itemize}
	\item[(C--$\alpha$)] 前件あるいは後件のいずれかが複数の発話からな
			     る場合.
	\item[(C--$\beta$)] 前件と後件それぞれを表す発話の間に,質問ある
			     いは同意を示す発話が存在する場合.
       \end{itemize}
\end{itemize}

分類の結果を\tableref{分類結果}に示す\footnote{\obsref{距離a}より「前件
および後件の双方が発話群である」という場合も考えられるが,今回用いたコー
パス中ではそのような例は認められなかったため,ここでは省略した.}.

\begin{table}[htbp]
\caption{因果関係が認められる発話の分類}
\tablelabel{分類結果}
\begin{center}
 \begin{tabular}{|c|r|r|r|r|r||r|} \hline
   & (a1) & (a2) & (a3) & (b1) & (b2) & 計\\ \hline
  (A) & 2 & 6 & 1 & 4 & 20 & 33 \\ \hline
  (B) & 2 & 4 & 0 & 0 & 6 & 12 \\ \hline
  (C-$\alpha$) & 1 & 2 & 0 & 1 & 4 & 8 \\ \hline
  (C-$\beta$) & 3 & 3 & 0 & 0 & 2 & 8 \\ \hline\hline
  計 & 8 & 15 & 1 & 5 & 32 & 61 \\ \hline
 \end{tabular}
\end{center}

\end{table}
この結果から,次のようなことがいえる.
まず,前件と後件の位置関係に関する\obsref{接続助詞},\obsref{接続詞}につ
いて分類した結果からであるが,
最も例が多いのは(b2)の32例であり,続いて(a2)の15例となっている.
因果関係を表す複文では,後件は主節によって述べられ,こちらに重点を置くた
め主節と従属節を倒置する,という現象が見られるが,これが(b2)に相当する
ものと考えられる.
一方,接続詞による因果関係では,前件と後件の間に接続詞が位置するような
(a1)および(a2)が一般的であるといえる.
これに対して,接続詞が前件と後件の間に明示されず,前件 → 後件の順序のみ
を満たす場合である(a3)は1例しか存在しない.
これを\exsref{実例1}に示すが,\obsref{接続詞}で述べたように,因果関係は
\exsref{実例1c}の「「ちきしょう,ばれたか」って出てきて,『は\ua 』って」
という部分での接続表現によって主に述べられているものと考えられる.
\eenumsentence{\exslabel{実例1}
\item B → E あ,テッチャンその場にいたのか\da
\item B → E なんだ\da
\item E → G だって,誰も気づいていないのに,いきなり平賀電信柱の陰から
「ちきしょう,ばれたか」って出てきて,「は\ua 」って\da \exslabel{実例
1c}
\item E → G すっげー面白かった\da
}

次に,前件と後件の隣接性に関する,\obsref{距離a}についての分類では,
(A)は33例,(B)は12例,(C)は16例となっている.
これから,「2つの発話が隣接し,かつ同一の話し手による発話である」という,
人間による因果関係の認識を促す要素が重なっている場合が最も数が多い,とい
う結果となる.
また,61発話のうち,因果関係が認められるものの,その前件と後件の位置関係
が隣接関係に収まらない例は全て(C--$\alpha$)もしくは(C--$\beta$)として分
類可能であったことから,\defref{発話群}で述べた発話群を考慮した場合,前
件と後件が発話もしくは発話群として明示されるような因果関係の場合は,それ
らは隣接関係にあるということが実際のコーパスからもいえる.

ところで,上記の分類の対象から除外した44発話は,おおまかに次のような場合
に分類される.

\begin{enumerate}
 \item 発話の含意もしくは前提などによって,因果関係が推定可能であると考
       えられる場合.… 18例.
 \item 因果関係の推定(認識)が不可能な場合.… 26例.
\end{enumerate}

前者の一例を次に示す.


\eenumsentence{\exslabel{実例3}
\item J → B 柿だけ食べてるの\ua \exslabel{実例3a}
\item B → J いや,向こう豆ばっかあるからさ\da \exslabel{実例3b}
}

これは,\exsref{発話例l}と同じように,\exsref{実例3a}から「柿だけ食べて
いる」という前提が導かれ,それに対する前件(理由)として\exsref{実例3b}が
述べられているというように因果関係が推定可能である.

なお,後者に属する例のうち,\obsref{例外}が確認される例が4例存在した.
その一例を\exsref{実例2}に示す.

\eenumsentence{\exslabel{実例2}
\item F → A 肝試しとか騒いじゃいけないって意味じゃないんですか\ua
\item A → F そうそうそう\da
\item M → A はい\ua \exslabel{実例2c}
\item A → M だから,肝試しとかうるさくするなってことだと思う\da
\exslabel{実例2d}
}

今回用いたコーパス中では,質問を示す発話に「だから」を含む発話が隣接した
場合は,すべて\obsref{例外}が確認されるような場合であった.

以上の結果は,\tableref{最終結果}としてまとめることができる.
この表から,今回用いたコーパスにおいて,接続詞もしくは接続表現が含まれる
発話すべてを対象とした場合,\obsref{接続助詞},\obsref{接続詞}および
\obsref{距離a}で述べた事柄は全体の50.4\%をカバーしており,人間により因果
関係が認識される95例に限ればその64.2\%に対して有効な内容であることがわか
る.


\begin{table}[htbp]
\caption{コーパス中の発話例の分類}
\tablelabel{最終結果}
\begin{center}
 \begin{tabular}{|c|c|r|} \hline
   & 通常の複文 & 16例(13.2\%) \\ \cline{2-3}
   & \obsref{接続助詞},\obsref{接続詞}および\obsref{距離a}が & \lw{61例
  (50.4\%)} \\%
  因果関係が認められる場合 & 認められる場合 & \\ \cline{2-3}
   & 発話の含意や前提により & \lw{18例(14.9\%)} \\%
   & 類推可能な場合 & \\ \hline
  \multicolumn{2}{|c|}{因果関係の認識が困難} & \lw{26例(21.5\%)} \\%
  \multicolumn{2}{|c|}{あるいは不可能な場合} & \\ \hline\hline
  \multicolumn{2}{|c|}{計} & 121例(100.0\%) \\ \hline
 \end{tabular}
\end{center}

\end{table}


\section{おわりに}

本技術資料では,接続詞や接続助詞類によって前件と後件が関係づけられるよう
な因果関係を対象とし,談話中で前件と後件が明示される場合にはどのような現
象が観察されるのか,について検討を行ない,その結果を\obsref{接続助詞}〜
\obsref{距離a}として述べた.
特に\obsref{距離a}では,\defref{発話群}で示した「発話群」を導入すること
により,因果関係の基本的構造が隣接関係として説明可能であることを示した.
次に,実際の会話コーパスを用いて上記の考察を検討し,コーパス中で対象とな
る発話全体(121例)の50.4\%,人間により因果関係が認識可能な場合(95例)に限
ると64.2\%の割合の発話例に対して\obsref{接続助詞}〜\obsref{距離a}で述べ
た事柄が確認されることを検証した.

なお,この検証作業は手作業によるものだが,最終的にはこれを機械的に行なえ
るシステム,つまり談話理解システムの構築に対して本技術資料で述べた考察を
適用することが考えられる.
現段階では,\tableref{最終結果}のうち「通常の複文」の場合および
「\obsref{接続助詞},\obsref{接続詞},\obsref{距離a}が認められる場合」に
ついては形態素解析の結果など発話の表層的な情報から因果関係の認識が可能で
あり,121発話に対して6割弱の場合については本技術資料での考察結果をもとに
した談話構造の解析が可能と考えられる.
ただし,発話の含意や前提を利用して因果関係を推定する必要がある場合や,人
間による因果関係の認識自体が困難あるいは不可能な場合が存在し,これらを発
話の表層情報から特定することは非常に困難であると考えられ,総合的な談話理
解システムの構築に際してはこの点の検討が課題となると考えられる.

\section*{謝辞}

本研究は,文部省科学研究費重点領域研究「音声対話」の補助を受けていること
を記し,御協力いただいた関係各位に感謝致します.

\bibliographystyle{jnlpbbl}
\bibliography{j-paper}

\begin{biography}
\biotitle{略歴}
\bioauthor{西澤 信一郎}{
1969 年生まれ.
1992 年横浜国立大学工学部卒業.
1997 年同大学大学院工学研究科博士課程修了.工学博士.
現在,富士通株式会社に勤務.情報処理学会および言語処理学会の会員.
}
\bioauthor{中川 裕志}{
1953 年生まれ.
1975 年東京大学工学部卒業.
1980 年同大学院博士課程修了.工学博士.
1980 年より横浜国立大学工学部勤務.
現在,同教授.
日本語の意味論,語用論,電子化マニュアル検索システム,マルチメディア検索,
情報検索,自動ハイパーテキスト化などの研究に従事.日本認知科学会,人工
知能学会などの会員.
}

\bioreceived{受付}
\biorevised{再受付}
\bioaccepted{採録}

\end{biography}

\end{document}
