\documentstyle[jnlpbbl,leqno,fleqn,lingmacros,macros]{jnlp_j_b5}
\setcounter{page}{23}
\setcounter{巻数}{4}
\setcounter{号数}{1}
\setcounter{年}{1997}
\setcounter{月}{1}
\受付{1996}{2}{8}
\再受付{1996}{3}{22}
\採録{1996}{4}{19}

\setcounter{secnumdepth}{2}

\title{統一モデルに基づく話し言葉の解析}
\author{伝 康晴\affiref{ATR}}

\headauthor{伝 康晴}
\headtitle{統一モデルに基づく話し言葉の解析}

\affilabel{ATR}{ATR音声翻訳通信研究所}
{ATR Interpreting Telecommunications Research Laboratories
 (現在,奈良先端科学技術大学院大学 情報科学研究科,
  Graduate School of Information Science, Nara Institute of Science and Technology)}

\jabstract{
近年の音声認識技術の進歩によって,話し言葉の解析は自然言語処
理の中心的なテーマの1つになりつつある.話し言葉の特徴は,言
い淀み,言い直し,省略などのさまざまな不適格性である.書き言
葉には見られないこれらの現象のために,従来の適格文の解析手法
はそのままでは話し言葉の解析には適用できない.本稿では,テキ
スト(漢字仮名混じり文)に書き起こされた日本語の話し言葉の文
からその文の格構造を取り出す構文・意味解析処理の中で,言い淀
み,言い直しなどの不適格性を適切に扱う手法について述べる.本
手法は,適格文と不適格文を統一的に扱う統一モデルに基づいてお
り,具体的には,係り受け解析の拡張によって実現される.まず,
音声対話コーパスからの実例をあげながら統一モデルの必要性を述
べ,次に,本手法の詳細を説明した後,その有効性を解析の実例を
あげるとともに実験システムの性能を評価することで示す.その結
果,さまざまな不適格性を含む複雑な話し言葉の文が,係り受け解
析を基本とする本手法によってうまく扱えることを示し,さらに,
定量的にも,試験文の約半数に完全に正しい依存構造が与えられる
ことを示す.
}

\jkeywords{話し言葉の解析,不適格性,統一モデル,係り受け解析}

\etitle{A Uniform Approach to\\ Spoken Language Analysis}
\eauthor{Yasuharu Den \affiref{ATR}} 

\eabstract{
Recent advances in speech processing technologies have made the analysis
of spoken language one of the central issues in natural
language processing. One big feature of spoken language, that distinguishes
it from written language, is that it is in various
ways ill-formed, containing hesitations, repairs, ellipses, and so on. This
makes it difficult to apply traditional linguistic-based methods to spoken
language analysis. This paper proposes a method for properly dealing
with ill-formedness, such as hesitations and repairs, in the course
of syntactic/semantic analysis of spoken Japanese where sentences transcribed
in Kanji-Kana characters are parsed and interpreted to obtain their
frame representations. The method is based on a uniform model,
which handles well- and ill-formed sentences in a uniform way,
and realized by extending traditional \hspace*{0.3mm}dependency \hspace*{0.3mm}analysis. \hspace*{0.4mm}We, \hspace*{0.3mm}first, \hspace*{0.3mm}show
\hspace*{0.2mm}the \hspace*{0.3mm}necessity \hspace*{0.3mm}of \hspace*{0.3mm}a \hspace*{0.3mm}uniform \hspace*{0.2mm}model\\ with illustrating motive examples
from our spoken dialogue corpus. Then, after providing the details
of the method, we show its effectiveness with both illustrating
examples of analyzing real sentences from the corpus and evaluating
the performance of an experimental system.
}
\ekeywords{Spoken language analysis, Ill-formedness, Uniform model,\\
Dependency analysis}
\begin{document}
\maketitle
\section{はじめに}\label{sec:Intro}

自然言語処理では,これまで書き言葉を対象として,さまざまな理論や技術が
開発されてきたが,話し言葉に関しては,ほとんど何もなされてこなかった.
しかし,近年の音声認識技術の進歩によって,話し言葉の解析は
自然言語処理の中心的なテーマの1つになりつつある.
音声翻訳,音声対話システム,マルチモーダル・インターフェースなどの
領域で,自然な発話を扱うための手法が研究され出している.

話し言葉の特徴は,言い淀み,言い直し,省略などのさまざまな
{\bf 不適格性}\,(ill-formedness)である.
例えば,(\ref{eq:Sentence1})には,(i)\,言い直し(「ほん」が「翻訳」に
言い直されている),(ii)\,助詞省略(「翻訳」の後の格助詞「を」が
省略されている)の2つの不適格性がある.
\enumsentence{\label{eq:Sentence1}
  ほん,翻訳入れます.}

書き言葉には見られないこれらの現象のために,従来の適格文の解析手法は
そのままでは話し言葉の解析には適用できない.
したがって,不適格性を扱うための手法を確立することが,話し言葉を対象と
した自然言語処理研究にとって必須である.
特に,不適格性を扱うための手法をその他の言語解析過程の中に
どのように組み込むかが,重要な課題となる.
本稿では,テキスト(漢字仮名混じり文)に書き起こされた日本語の話し言葉の
文からその文の格構造を取り出す構文・意味解析処理の中で,言い淀み,
言い直しなどの不適格性を適切に扱う手法について述べる.

不適格文を扱う手法の研究は,以下の3つのアプローチに大別できる.
\begin{description}
  \item[A. 不適格性を扱う個別的な手法] 話し言葉に特有の不適格性を
個別的な手法で扱う.
言い直しを扱う手法\cite{Hindle:ACL83-123,Bear:ACL92-56,
Nakatani:ACL93-46,佐川:情処論-35-1-46}や助詞省略を扱う手法
\cite{山本:情処論-33-11-1322}がある.
  \item[B. 不適格性を扱う一般的な手法] さまざまな不適格性を
一般的なモデルに基づいて扱う.
以下の2つのモデルに大別される.
  \begin{description}
    \item[B-1. 二段階モデル(two-stage model)に基づく手法]
まず,通常の適格文の解析手法で入力文を解析し,それが失敗した場合に,
不適格性を扱うための処理を起動する.
{\bf 部分解析法}\cite{Jensen:CL-9-3-147,McDonald:ANLP92-193}や
{\bf 制約緩和法}\cite{Weischedel:CL-9-3-161,Mellish:ACL89-102}がある.
    \item[B-2. 統一モデル(uniform model)に基づく手法]
適格文と不適格文との間に明確な区別をおかず,
両者を連続的なものととらえ統一的に扱う.
{\bf 優先意味論}に基づく手法\cite{Fass:CL-9-3-178}や
{\bf アブダクション}に基づく手法\cite{Hobbs:AI-63-69}がある.
  \end{description}
\end{description}

本稿では,以下にあげる理由により,統一モデルに基づく手法を用いる.
\begin{enumerate}
\renewcommand{\theenumi}{}
\renewcommand{\labelenumi}{}
  \item 不適格文の処理はしばしば,適格文の処理と同等な能力を必要とする.
例えば,言い直しを含む文において修復対象(言い直された部分)の範囲を
同定するのは,適格文において従属節の範囲を決めるのと同じ難しさがある.
したがって,不適格文を扱うために,従来適格文の処理に使われてきた手法を
拡張して使えることが望ましい.
  \item 不適格文と適格文が曖昧な場合がある.
例えば,(\ref{eq:Sentence1})の「ほん」はたまたま「本」と同じ字面である
ため,「本(に)翻訳(を)入れます」のような適格文としての解釈が可能になる.
適格文と不適格文が統一的に扱えないと,このような曖昧性は解消できない.
  \item 話し言葉(特に音声言語)の解析に必要な実時間処理は,不適格文を
処理するのに二段階の過程を経る二段階モデルでは実現できない.
これに対して,統一モデルでは,漸時的な処理が可能なので,実時間処理を
実現しやすい.
  \item 統一モデルは人間の言語処理モデルとしても妥当である.
人間はしばしば,文の途中であっても不適格性が生じたことに気がつく.
このことは,人間が適格文の処理と並行して,不適格性の検出のための処理を
行なっていることを示唆する.
\end{enumerate}

統一モデルを採用することにより,適格文におけるさまざまな問題(構造の
決定や文法・意味関係の付与といった問題)を解決するための手法を拡張する
ことで,不適格性の問題も同じ枠組の中で扱える.
より具体的には,言い淀み,言い直しなどを語と語の間のある種の依存関係と
考えることにより,{\bf 係り受け解析}の拡張として,適格性と不適格性を
統一的に扱う手法が実現される.

以下,まず\ref{sec:Ill-formed}\,節では,日本語の話し言葉における
さまざまな不適格性を,音声対話コーパスからの実例をあげながら説明し,
統一モデルの必要性を述べる.
次に\ref{sec:Uniform}\,節で,本稿で提案する統一モデルに基づく
話し言葉の解析手法を説明する.
\ref{sec:Evaluation}\,節では,解析の実例をあげるとともに実験システムの
性能を評価することで本手法の有効性を検討する.
さらに,その適用範囲についても明らかにする.
\ref{sec:Comparison}\,節では,従来の手法との比較を述べ,
最後に,\ref{sec:Conclude}\,節でまとめを述べる.
なお,話し言葉の解析を考える上で,音声情報の果たす役割は重要であるが,
本稿では音声処理の問題には立ち入らない.
\section{日本語の話し言葉における不適格性}\label{sec:Ill-formed}

日本語の話し言葉では,さまざまなレベルでさまざまな種類の
不適格性が生じる\cite{河原:情処-36-11-1027}.
ここでは,これらの不適格性のうち,本研究でターゲットとする言い淀み,
言い直し,繰り返し,助詞省略について,ATR対話データベース(ADD)\,
\cite{江原:ATR-TR-I-0186}から実例をあげながら説明する.

\subsection{言い淀み}\label{sec:Ill-formed:Hesitation}

{\bf 言い淀み}(hesitations)とは,「あのー」「えーと」などの意味を
持たない冗長語(間投語,つなぎ語ともいう)によって発話が滞ることである.
(\ref{eq:Hesitation})では太字が冗長語である.
\enumsentence{\label{eq:Hesitation}
  {\bf えーっと},そちら第一回の通訳電話国際会議の事務局でしょうか.}

\citeA{村上:IEICE-SP-91-100-71}によると,ADDでは,約50\,\%の文に
言い淀みが含まれる.
このうち約70\,\%では,頻度上位4種類の冗長語(「えー」「あのー」「あの」
「え」)が使われている\footnote{
  \citeA{中川:音響-51-3-202}による日本音響学会連続音声データベース
(ASJ)の分析では,言い淀みは1文あたり1.126回出現し,「え」「えと」
「あの」「あ」「ま」の頻度上位5種類の冗長語が出現総数の82\,\%を占める.
\citeauthor{中川:音響-51-3-202}は長母音や促音による違いを
無視しているので,\citeauthor{村上:IEICE-SP-91-100-71}の分析結果にも
この方略を適用すると,同じ5種類の冗長語のバリエーションで上位9種類,
約85\,\%を占めることになり,両者の分析はほぼ一致する.
}.
このことから,言い淀みは,(言語処理においては)主だった冗長語を辞書に
登録することでかなりの部分が扱えると思われる.
ただし,冗長語はしばしば他の単語と混同される(連体詞「あの」「その」
など)ことがあり,そういった場合に統一モデルは有効である.

\subsection{言い直し}\label{sec:Ill-formed:Repair}

{\bf 言い直し}(repairs)とは,言い誤りを訂正したり,より良い言い回しに
変更するために,話者自身が発話を中断してその部分を再度発話することで
ある.
\citeA{村上:IEICE-SP-91-100-71}によると,ADDでは,約10\,\%の文に
言い直しが含まれる\footnote{
  これは,\citeA{中川:音響-51-3-202}によるASJの分析結果(1文あたり
0.145回)や\citeA{Bear:ACL92-56}によるDARPAのATISタスク・コーパスの
分析結果(10語以上の文の10\,\%)ともほぼ一致する.
}.
(\ref{eq:Repair1})では太字が言い直しによって訂正された部分(修復対象)で
ある.
\enumsentence{\label{eq:Repair1}
  あの,この{\bf クレ},クレジットカードというのは}

言語処理における言い直しの扱いの問題点は以下のようにまとめられる.
\begin{enumerate}
\renewcommand{\theenumi}{}
\renewcommand{\labelenumi}{}
  \item 言い直しには言語的な手がかりがないので,検出が困難である.
言い直しに伴って言い淀みが生じることがしばしばあるが,言い淀みは元来
独立して起こり得るので,信頼できる手がかりにはならない\footnote{
  音声処理との統合を考えると,音響的・韻律的な手がかりが役立つ可能性は
ある\cite{O'Shaughnessy:ICSLP92-931,Nakatani:ACL93-46}.
}.
  \item 修復対象の範囲を同定するのが困難である.
例えば,(\ref{eq:Repair2})では「準備を」と「始めて」は修復対象に
含まれるが,「もう」と「ビデオの」は含まれない.
\enumsentence{\label{eq:Repair2}
  もう{\bf 準備を始めて},ビデオの準備を始めていきたい.}
  \item 言い直しは,話者に生じたトラブルの内容に応じて,多様な
パターンをとる.
例えば,(\ref{eq:Repair3}), (\ref{eq:Repair4}), (\ref{eq:Repair5})は
それぞれ,音韻的な原因(発語の途中でつまった),統語的な原因(誤った
助詞を選択した),意味的な原因(不適切な意味内容を選択した)による
言い直しと考えられる.
\enumsentence{\label{eq:Repair3}
  {\bf つう},通訳電話に関するさまざまな,あのー,方面}
\enumsentence{\label{eq:Repair4}
  あの,{\bf クレジットカードをね},あのー,クレジットカードの名前と
なんか,ナンバーを}
\enumsentence{\label{eq:Repair5}
  あの,会議ではもちろん{\bf 通訳},翻訳も入れます.}
\end{enumerate}

これらの問題点は,適格文の処理で生じる問題点と共通している.
例えば,(\ref{eq:Repair2})における修復対象の範囲の同定は,
(\ref{eq:Subord})における従属節(太字部分)の範囲の同定と同種の問題で
ある.
\enumsentence{\label{eq:Subord}
  もう{\bf 準備を始めてから}ずいぶん待ちました.}

さらに,\citeA{佐川:IPSJ-NL-94-100-73}が指摘したように,言い直しの
中には適格文と簡単には区別できないものがある.
例えば,(\ref{eq:Repair6})では,「きょう」が「協賛する」に言い直されて
いるのであるが,「きょう」がたまたま「今日」と同じ字面のため,適格文と
しての解釈も可能である.
\enumsentence{\label{eq:Repair6}
  {\bf きょう},協賛する学会会員}

\citeauthor{佐川:IPSJ-NL-94-100-73}によると,このような適格文との間で
曖昧性がある例は,ADDに含まれる言い直しのうちの約1割を占めている.
しかし,(\ref{eq:Repair5})のような文でも,助詞省略の可能性を考えると
曖昧になる(「通訳」の後に助詞を補うと「入れます」に係る可能性が
ある)ことから,この割合は実際にはもっと大きいと思われる.

これらは,言い直しのような不適格性を,適格文の処理手法を拡張することで
統一的に扱う必要があることを示唆する.

\subsection{繰り返し}\label{sec:Ill-formed:Repetition}

{\bf 繰り返し}(repetitions)は,言い直しの特別な場合である.
\citeA{村上:IEICE-SP-91-100-71}によると,ADDに含まれる言い直しの
14\,\%が繰り返しである.
(\ref{eq:Repetition})では太字が繰り返された部分である.
\enumsentence{\label{eq:Repetition}
  えー,{\bf 京都ロイヤルが},えー,京都ロイヤルが一番都心ですね.}

\subsection{助詞省略}\label{sec:Ill-formed:ParticleEllipsis}

{\bf 助詞省略}(particle ellipses)は,書き言葉には見られない,
話し言葉に特有の現象である.
\citeA{保坂:JSAI-SLUD-9203-1}によると,ADDにおいて,普通名詞文節の
12\,\%,代名詞文節の16\,\%で助詞が省略されている\footnote{
  \citeA{山本:情処論-33-11-1322}によると,ASJでは,助詞省略の割合は
全名詞文節の4\,\%である.
}.
(\ref{eq:Particle})では省略された助詞を括弧内に補ってある.
\enumsentence{\label{eq:Particle}
  会議{\bf [に]},参加する手続き{\bf [を]},ちょっと,お教え願えます
でしょうか.}

助詞省略は,構文・意味解析において大きな問題となる.
日本語では,通常,語と語の間の意味関係が格助詞によって明示されるので,
解析システムは格情報を頼りに意味関係を決定しようとする.
もし助詞が省略されていると,このような方法で意味関係を決定することが
できない.

しかし,格情報をあてにできない場面は,書き言葉においても生じる.
例えば,主格や目的格の名詞句が主題化されると,助詞が同一の
形式(「は」や「も」)に置き換わるので,もとの格が何であったかは簡単には
わからない.
さらに,日本語には関係代名詞がないので,被修飾名詞の関係節内での
意味役割(例えば(\ref{eq:Particle})での「手続き」と「参加する」の間の
意味関係)を決めるのに格情報を利用することはできない.

こういった場合には,単語の意味属性を頼りに意味関係の推定を行なうことに
なるが,助詞省略を含む句に対しても同じ方法が適用できる.
これも,適格文と不適格文を統一的に扱うモデルの妥当性を示している.

\vspace*{-0.2mm}
\subsection{その他の不適格性}\label{sec:Ill-formed:Others}
\vspace*{-0.2mm}

日本語の話し言葉には,これまで述べたもの以外にも,
述部の省略,倒置,言い誤りなどの不適格性があるが,これらについては,
(a)\,頻度が低い,(b)\,単純なヒューリスティクスで扱えることが多い,
(c)\,構文・意味解析の範囲を越えるなどの理由により,本稿では扱わない.
\vspace*{-0.2mm}
\section{統一モデルに基づく話し言葉の解析}\label{sec:Uniform}
\vspace*{-0.2mm}

\vspace*{-0.2mm}
\subsection{本手法の概要}\label{sec:Uniform:Overview}
\vspace*{-0.2mm}

本稿で提案する統一モデルに基づく話し言葉の解析手法を説明する.
本手法は,
\begin{enumerate}
\renewcommand{\theenumi}{}
\renewcommand{\labelenumi}{}
  \item 話し言葉におけるさまざまな制約を統一的に記述した規則群
  \item 入力文に対してそれらの規則を順次適用して最適な解釈を求める解釈器
\end{enumerate}
からなる.
すべての規則は{\bf 確定節}によって記述され,解釈器としては
{\bf アブダクション}が用いられる.
本稿では前者についてのみ述べる.
後者については,別稿\cite{伝:情処論-35-12-2734}を参照されたい.

本手法は,基本的には,係り受け解析の拡張である.
入力文の依存構造を生成するために,各語(文節)の間の依存関係を調べる.
例えば,(\ref{eq:Sentence2})に対する依存構造と各文節の間の
文法・意味関係は(\ref{eq:Depend1})のようになる.
\enumsentence{\label{eq:Sentence2}
  会議では翻訳も入れます.}
\enumsentence{\label{eq:Depend1}
  [\DP{loct\&de} 会議では\Q [\DP{obje\&accAct} 翻訳も\Q 入れます]]}
ここで,\Rel{loct}, \Rel{obje}は意味関係(それぞれ「場所」「対象」)を
表し,\Rel{de}, \Rel{accAct}は文法関係(それぞれ「デ格」
「目的格・能動態」)を表す\footnote{
  文法関係は,係り文節が任意格要素の場合は係り文節の格で表し,
必須格要素の場合は係り文節の格と受け文節の態の組合せで表す.
これは,後者では,態に依存して格交替が起こる可能性があるからである.
}.

依存構造の決定と文法・意味関係の付与は,{\bf 構造規則}と
{\bf 解釈規則}を参照しながら行なう.
この処理は,格情報のみに基づいて決定的に行なうのではなく,係り文節と
受け文節の間の意味的な結合の強さや文法関係の実現のしやすさ(例えば
「も」は「主格」になりやすいか「目的格」になりやすいか)などを
考慮して,さまざまな候補に{\bf 優先度}を与え,最終的に最も優先度の
高い組合せを見つけることによって行なう.
したがって,我々のフォーマリズムは,制約に基づく(constraint-based)と
いうよりは,{\bf 選好}に基づく(preference-based)ものである.

本手法では,通常の係り受け解析を拡張し,言い淀み,言い直しなども語と
語(文節と文節)の間の依存関係ととらえる.
例えば,言い直しを含む文(\ref{eq:Sentence3})の依存構造は
(\ref{eq:Depend2})のようになる.
\enumsentence{\label{eq:Sentence3}
  ほん,翻訳入れます.}
\enumsentence{\label{eq:Depend2}
  [\DP{obje\&accAct} [\DP{phonRepair} ほん\Q 翻訳]\Q 入れます]}
ここで,\Rel{phonRepair}は「ほん」と「翻訳」の間に音韻的な原因による
言い直し(以下「音韻的言い直し」)によって依存関係が生じていることを表す.

このように,不適格性を扱えるよう係り受け解析を拡張することによって,
適格文の最適な解釈を求める処理と不適格性を検出・修正する処理が同じ
道具だてで実現できるだけでなく,適格文と不適格文との間の曖昧性にも
対処できる.

\subsection{構文・意味解析の過程}\label{sec:Uniform:Process}

本手法による構文・意味解析の過程を簡単な例題を用いて説明する.
図\,\ref{fig:Process}\,は(\ref{eq:Sentence3})の解析過程である.
この文は,言い直しと助詞省略の2つの不適格性を含む.
さらに,言い直しは適格文との間で曖昧である(「ほん」は「本」と同じ字面).
\begin{figure}
\begin{center}
\mbox{
{\bf A:}\quad
ほん\Q 翻訳\Q 入れます}\\[\medskipamount]
{\Large$\Downarrow$}\rlap{\fbox{文節解析}}\\[\medskipamount]
\mbox{
\makebox(0,0)[lb]{\raisebox{1.6\baselineskip}{\bf B:}}
\begin{footnotesize}
\Feature{
  \Slot{phon} & \Value{ほん}\\
  \Slot{syn} & \Pair{\footnotesize
                     \Value{本},\Value{普通名詞},\Value{無},\Value{$-$}}\\
  \Slot{sem} & \Pair{\footnotesize
                     \Value{本},\Value{書物}}}
\Feature{
  \Slot{phon} & \Value{ほんやく}\\
  \Slot{syn} & \Pair{\footnotesize
                     \Value{翻訳},\Value{サ変名詞},\Value{無},\Value{$-$}}\\
  \Slot{sem} & \Pair{\footnotesize
                     \Value{翻訳},\Value{翻訳}}}
\Feature{
  \Slot{phon} & \Value{いれます}\\
  \Slot{syn} & \Pair{\footnotesize
                     \Value{入れる},\Value{がを動詞},\Value{基本},\Value{能動}}\\
  \Slot{sem} & \Pair{\footnotesize
                     \Value{入れる},\Value{授受}}}
\end{footnotesize}}\\[\medskipamount]
{\Large$\Downarrow$}\rlap{\fbox{依存構造解析}}\\[\medskipamount]
\mbox{
{\bf C:}\quad
[\DP{?} [\DP{?} ほん\Q 翻訳]\Q 入れます]\qquad
{\bf OR}\qquad
[\DP{?} ほん\Q [\DP{?} 翻訳\Q 入れます]]}\\[\medskipamount]
{\Large$\Downarrow$}\rlap{\fbox{依存関係解析}}\\[\medskipamount]
\mbox{
\raisebox{1.2\baselineskip}{\bf D:}
\DependTable{\Pair{ほん,\,翻訳}}{
  \Rel{of\&gen} & 0.0002\\
  \underline{\Rel{phonRepair}} & \underline{0.0053}}
\DependTable{\Pair{翻訳,\,入れます}}{
  \underline{\Rel{obje\&accAct}} & \underline{0.0134}\\
  \Rel{inst\&de} & 0.0031}
\DependTable{\Pair{ほん,\,入れます}}{
  \Rel{obje\&accAct} & 0.0004\\
  \Rel{loct\&ni} & 0.0029}}\\[\medskipamount]
{\Large$\Downarrow$}\rlap{\fbox{最適解選択}}\\[\smallskipamount]
\mbox{
{\bf E:}\quad
[\DP{obje\&accAct} [\DP{phonRepair} ほん\Q 翻訳]\Q 入れます]}
\end{center}
\caption{構文・意味解析の過程}\label{fig:Process}
\end{figure}

解析過程は以下の4つのステップからなる.
\begin{description}
  \item[文節解析] 入力文{\bf A}を素性構造で表現された文節の列{\bf B}に
変換する. 
  \item[依存構造解析] 文節の列{\bf B}に構造規則を適用し,可能な
依存構造の集合{\bf C}に変換する.
  \item[依存関係解析] 依存構造の集合{\bf C}に含まれる各依存関係に
ついて,解釈規則を適用し,解釈の候補{\bf D}を生成する.
依存関係解釈の候補のおのおのには,$[0,1]$間の実数値で表される優先度を
与える.
  \item[最適解選択] 最も優先度の大きい解釈({\bf D}の下線部分)を選択し,
文全体の依存構造と依存関係解釈{\bf E}を出力する.
\end{description}

この例では,{\bf B}の文節列に対して,{\bf C}の2つの依存構造が可能で
あり,その中に3つの依存関係が含まれる.
それぞれの依存関係に対する解釈の候補は{\bf D}のようになり,
下線を引いたものが最も優先度の大きい組合せとして選択される.
この過程において,助詞省略は「目的格」に解釈され,言い直しと
適格文(「本(に)入れます」)との曖昧性も解消されている.

\subsection{文節解析}\label{sec:Uniform:Bunsetsu}

文節解析への入力は,音声認識システムの出力として漢字仮名混じり文が
与えられることを想定している.
文節解析は,入力文を素性構造で表現された文節の列に変換する.
文節は,ちょうど1つの自立語といくつかの付属語からなる\footnote{
  特別な付属語として,繋辞(「です」など)と補文標識(「こと」など)がある.
これらは自立語の範疇を変更する(「会議」(名詞) $\to$
「会議です」(動詞))ので,{\bf 範疇変換語}とよぶ.
範疇変換語を含む文節に対しては,複数のレベルでの係り受けが可能である.
例えば,「言語学の会議です」は名詞レベル,「これは会議です」は
動詞レベルでの係り受けである.
}.
素性構造は,一般に以下の形式である.
\begin{equation}\label{eq:Feature}
  \Feature{
    \Slot{phon} & よみ\\
    \Slot{syn} & \Pair{語彙,範疇,形,態}\\
    \Slot{sem} & \Pair{概念,属性}}
\end{equation}

\begin{description}
  \item[よみ] 文節全体のよみを仮名で表したもの.
  \item[語彙] 文節内の自立語の見出し語.
  \item[範疇] 語彙の統語範疇.
「連体詞」「副詞」「普通名詞」「固有名詞」「が動詞」「がを動詞」
「が形容詞」などがある(用言は格パターンに応じて細分類されて
いる)\footnote{
  範疇変換語を含む文節の範疇はリストで表す.
例えば,「会議です」の範疇は「普通名詞 + 繋辞」である.
}.
  \item[形] 名詞文節の格や動詞文節の活用形.
名詞文節が助詞省略を含む場合は「無」で表す.
  \item[態] 動詞文節の態.
動詞文節以外は`\Value{$-$}'で表す.
  \item[概念] 文節内の自立語の見出し語.
  \item[属性] 概念の意味属性.
角川類語新辞典\cite{大野:角類新-81}の小分類を使用.
\end{description}

話し言葉では{\bf 不要語}を扱う必要がある.
不要語は以下のいずれかである.
\begin{itemize}
  \item 「あのー」「えーと」のような冗長語
  \item 発語が中断された語(中断語)
\end{itemize}
これらは,しばしば,適正な語と同じ字面になることがある(例えば
(\ref{eq:Sentence3})の中断語「ほん」は「本」と同じ字面).
その場合には,不要語と適正な語の両方の可能性を考慮することになる.
なお本稿では,不要語であっても,音声認識システムによって正しい
音韻列が与えられると仮定しているが,これは現在の音声認識技術から
考えると少し難しいかも知れない.

文節解析の出力である文節列は,以下の形式の要素式によって表現される.
\begin{equation}\label{eq:Bunsetsu}
  \Formula{\Var{Sym}}{\Var{Start},\Var{End},\Var{Feature}}
\end{equation}
\Var{Sym}は非終端記号であり,\Var{Start}, \Var{End}はそれぞれ,その
文節の開始位置,終了位置(確定節文法(DCG)の文字列引数に対応)を表し,
\Var{Feature}はその文節の素性構造を表す.
非終端記号\Var{Sym}は,統語範疇に依存して表\,\ref{tab:Sym} (付録)の
ように決まる.

\subsection{依存構造解析}\label{sec:Uniform:Struct}

依存構造解析は,(\ref{eq:Bunsetsu})の形式の要素式で表現された文節の
列に対して,構造規則を適用しながら,すべての可能な依存構造を生成する.
構造規則は,一般に以下の形式である\footnote{
  実際には,範疇変換を扱う構造規則が別にあり,これによって,「会議
です」の範疇が「普通名詞 + 繋辞」から「が動詞」に変換される.
変換前には名詞レベルでの係り受けが,変換後には動詞レベルでの係り受けが
可能である.
}.
\begin{equation}\label{eq:Struct}
\begin{array}[t]{@{}l@{}}
  \Formula{\Var{Sym2}}{X_0,X_2,F_2}\,\Implied\,
\Formula{\Var{Sym1}}{X_0,X_1,F_1},\,
\Formula{\Var{Sym2}}{X_1,X_2,F_2},\,
\Formula{\Var{Depend}}{F_1,F_2}.
\end{array}
\end{equation}

この規則は,係り文節(右辺の第1式)と受け文節(右辺の第2式)から
複合句(左辺)が作られることを表す.
複合句の素性構造は受け文節から引き継がれ,これは日本語が主要部後置
(head-final)型であることによる.
右辺の第3式は,係り文節と受け文節の間の依存関係を表す.
この依存関係は,依存関係解析において,適当な文法・意味関係もしくは
不適格性を示す関係に解釈される.
係り文節と受け文節の可能な組合せとその間の依存関係の解釈の種類は,
非終端記号によって表\,\ref{tab:Struct} (付録)のように制限される.

この規則は複合句に対しても再帰的に適用される.
複合句の素性構造は主要部の情報だけを持つので,依存関係は句の間の
関係ではなく,主要部文節の間の関係であることに注意せよ.
これは係り受け解析の特徴であり,調べるべき依存関係の数が文節数の二乗の
オーダに抑えられる.

\subsection{依存関係解析}\label{sec:Uniform:Depend}

依存関係解析は,すべての可能な依存構造に含まれる依存関係のおのおのに
対して,解釈規則を適用し,すべての可能な解釈の候補を生成する.
解釈規則は,(i)\,適格な依存関係に関する規則と(ii)\,不適格な
依存関係に関する規則の2種類からなる.

\subsubsection{適格な依存関係に関する解釈規則}
\label{sec:Uniform:Depend:Well-formed}

適格な依存関係に関する解釈規則は,一般に以下の形式である.
\begin{equation}\label{eq:Well-formed}
\begin{array}[t]{@{}l@{}}
  \Formula{\Var{Depend}}{F_1,F_2}\,\Implied\,
\Formula{\Var{Cond}}{F_1,F_2},\,
\Formula{\Var{SemRel}}{F_1,F_2},\,
\Formula{\Var{SynRel}}{F_1,F_2}.
\end{array}
\end{equation}

この規則は,依存関係\Var{Depend}が,ある条件\Var{Cond}のもとで,ある
意味関係\Var{SemRel}とある文法関係\Var{SynRel}に解釈できることを表す.
例えば,名詞文節と動詞文節の間の依存関係(\Rel{dep\_n\_v})は,
意味関係\Rel{obje} (「対象」)と文法関係\Rel{accAct}
(「目的格・能動態」)に解釈できる.
この際の条件としては,受け文節が目的格に対象を取る動詞(「がを動詞」
など)であること,受け文節の態が「能動」であること,係り文節の形が
目的格になれるもの(「を」「も」「無」など)であること,などが課される.
これらの条件は,単一化文法において,下位範疇化素性として記述されて
いるものと等価である.
適格な依存関係の一部を表\,\ref{tab:Well-formed} (付録)にあげる.

\subsubsection{不適格な依存関係に関する解釈規則}
\label{sec:Uniform:Depend:Ill-formed}

不適格な依存関係に関する解釈規則は,一般に以下の形式である.
\begin{equation}\label{eq:Ill-formed}
\begin{array}[t]{@{}l@{}}
  \Formula{\Var{Depend}}{F_1,F_2}\,\Implied\,
\Formula{\Var{Cond}}{F_1,F_2},\,
\Formula{\Var{IllRel}}{F_1,F_2}.
\end{array}
\end{equation}

この規則は,依存関係\Var{Depend}が,ある条件\Var{Cond}のもとで,ある
不適格性を示す関係\Var{IllRel}に解釈できることを表す.
例えば,不要語と名詞文節の間の依存関係(\Rel{dep\_nonlex\_any})は,
不適格な依存関係\Rel{phonRepair} (「音韻的言い直し」)に解釈できる.
この際の条件としては,係り文節のよみが受け文節のよみの部分(例えば
「ほん」と「ほんやく」)であることが課される.
不適格な依存関係を表\,\ref{tab:Ill-formed} (付録)にあげる.
ここでは,以下の5種類の不適格性を考えている.
\Rel{hest} (「言い淀み」), \Rel{phonRepair} (「音韻的言い直し」),
\Rel{synRepair} (「統語的言い直し」), \Rel{semRepair}
(「意味的言い直し」), \Rel{rept} (「繰り返し」).

\subsubsection{優先度}\label{sec:Uniform:Depend:Preference}

依存関係解釈の候補のおのおのには,$[0,1]$間の実数値で表される優先度が
与えられる.
優先度をどのように与えるかということは,話し言葉の構文・意味解析を
どのようなモデルに基づいて行なうかということとは,一応独立した問題で
あると考えられる.
しかし,本手法の有効性を検証するために,我々は一つの優先度計算法を
実験システムに実装した.
以下では,これについて簡単に説明する.
詳細は別稿\cite{伝:言処-投稿中}を参照されたい.

我々の優先度計算法は,{\bf コーパス}に基づく(corpus-based)手法である.
優先度は,その依存関係解釈が学習データ中でどのくらいの頻度で生じて
いるかに応じて与える.
すなわち,係り文節$\alpha$と受け文節$\beta$の間の依存関係解釈$\pi$の
優先度$P(\pi,\alpha,\beta)$は,次式で与えられる.
\begin{equation}\label{eq:Preference}
  P(\pi,\alpha,\beta) =
\frac{\mbox{\Formula{$\pi$}{\alpha,\beta}の頻度}}
     {\sum_{p,x,y}\mbox{\Formula{$p$}{x,y}の頻度}}
\end{equation}
分子は係\hspace{-0.1mm}り\hspace{-0.1mm}文節$\alpha$\hspace{-0.2mm}と受け文節$\beta$\hspace{-0.1mm}の間に解釈$\pi$\hspace{-0.1mm}を持つ
事例(\Formula{$\pi$}{\alpha,\beta}で表す)\hspace{-0.1mm}の頻度であり,
\hspace{-0.5mm}分母は学\\習データ中のすべての事例の頻度の総和である.

しかし,このままでは学習データの希薄性(data-sparseness)の問題を
避けられないので,分子の\,\Formula{$\pi$}{\alpha,\beta}\,の頻度を
計算する際に,完全に一致する事例だけでなく類似した事例の頻度も考慮する.
例えば,\Formula{\Rel{obje}}{翻訳,入れる}\,の頻度を計算する際に,
これと類似した事例\,\Formula{\Rel{obje}}{通訳,行なう}\,が
学習データ中にあれば,その頻度を考慮に入れるという具合である.

不適格な依存関係についても,例えば,言い直しのパターンの間の類似性を
定義し,上と同じ方法で優先度の計算を行なう.
類似性の定義などの詳細についてはここでは省略する.

\subsection{最適解選択}\label{sec:Uniform:Best}

最適解選択は,すべての可能な依存構造と依存関係解釈の候補から,
最も優先度が大きくなる組合せをみつけ,文全体の依存構造と
依存関係解釈を出力する(格構造はそこから生成できる).
\section{評価}\label{sec:Evaluation}

\subsection{解析の実例}\label{sec:Evaluation:Example}

以下では,本手法による構文・意味解析の実例をあげる.
例文はいずれもATR対話データベース(ADD)\,\cite{江原:ATR-TR-I-0186}から
とったものである.

\begin{Ex}\label{Ex:Example1}
\rm\
\begin{description}
  \item[入力] あの,会議ではもちろん通訳,翻訳も入れます.
  \item[出力]
[\DP{loct\&de} \SS{
 [\DP{hest}
  あの\Q
  会議では]\\\relax
 [\DP{advRel\&renyo} \SS{
  もちろん\\\relax
  [\DP{obje\&accAct}
   [\DP{semRepair}
    通訳\Q
    翻訳も]\Q
   入れます]]]}}
\end{description}
\end{Ex}

この文は言い淀み(「あの」)と言い直し(「通訳」$\to$「翻訳も」)を含む.
冗長語「あの」は直後の文節「会議では」に対して不適格な
依存関係\Rel{hest}で係る.
これはすべての言い淀みに共通した構造である.
この文の言い直しは,意味的な原因で生じており,不適格な
依存関係\Rel{semRepair}が付与される.
構造規則(\ref{eq:Struct})により,言い直しの訂正部分「翻訳も」が
主要部として上位の構造に伝わることに注意せよ.
これは,日本語においては,不適格性による構造も含めて,主要部後置の
性質が成り立つことを意味する.

\begin{Ex}\label{Ex:Example2}
\rm\
\begin{description}
  \item[入力]  あのー,言語学関係の方からのスピーチ,スピー,スピー,
スピーチのですね,申し込み
  \item[出力]
[\DP{from\&karano} \SS{
 [\DP{in\&gen}
  [\DP{hest}
   あのー\Q
   言語学関係の]\Q
  方からの]\\\relax
 [\DP{of\&gen} \SS{
  [\DP{synRepair} \SS{
   スピーチ\\\relax
   [\DP{phonRepair} \SS{
    スピー\\\relax
    [\DP{phonRepair}
     スピー\Q
     スピーチのですね]]]}}\\\relax
  申し込み]]}}
\end{description}
\end{Ex}

この文では言い直しが3つ連続して生じている(「スピーチ」「スピー」
「スピー」$\to$「スピーチのですね」).
これらはすべて訂正部分に係るように分析される.
その結果,連続する言い直しの部分は右分岐構造をなす.
これは,日本語の適格文の標準的な構造に合致する.

\begin{Ex}\label{Ex:Example3}
\rm\
\begin{description}
  \item[入力] えーっと,受領の通知は,受け取りの通知は
十二月三十一日までに出させていただきます.
  \item[出力]
[\DP{obje\&accAct} \SS{
 [\DP{rept} \SS{
  [\DP{of\&gen}
   [\DP{hest}
    えーっと\Q
    受領の]\Q
   通知は]\\\relax
  [\DP{of\&gen}
   受け取りの\Q
   通知は]]}\\\relax
 [\DP{tlim\&madeni}
  十二月三十一日までに\Q
  出させていただきます]]}
\end{description}
\end{Ex}

この文の言い直し(「受領の通知は」$\to$「受け取りの通知は」)では,
修復対象が単一の文節ではなく,複数の文節である.
しかし,修復対象の内部が適格な構造([\DP{of\&gen} 受領の 通知は])を
なしているため,係り受け解析を拡張した本手法では,大域的な
マッチング\cite{Kikui:ICSLP94-915}を用いることなく,修復対象の
範囲を同定できる.
修復対象の内部構造の適格性は,日本語においては,多くの場合に成り立つと
予想される.
実際,ADDの10対話を分析したところ,修復対象が(冗長語を除いて) 2文節
以上ある言い直し (14例)のすべてについて,適格な内部構造を与えることが
できた.

\begin{Ex}\label{Eq:Example4}
\rm\
\begin{description}
  \item[入力] 国際電話の,どうじ,コンピュータによる通訳,
コンピュータによる同時通訳に関する,あのー,会議
  \item[出力] \SS{
[\DP{on\&nikansuru}\\\QQ
 [\DP{of\&gen} \SS{
  国際電話の\\\relax
  [\DP{phonRepair} \SS{
   どうじ\\\relax
   [\DP{semRepair} \SS{
    [\DP{by\&niyoru}
     コンピュータによる\Q
     通訳]\\\relax
    [\DP{by\&niyoru}
     コンピュータによる\Q
     同時通訳に関する]]]]}}}\\\QQ
 [\DP{hest}
  あのー\Q
  会議]]}
\end{description}
\end{Ex}

この文では,言い直しが2つ連続して生じており(「どうじ」「コンピュータに
よる通訳」$\to$「コンピュータによる同時通訳に関する」),かつ,2番めの
言い直しの修復対象は複数の文節である.
さらに,言い直しの直前の文節「国際電話の」は2つの言い直しをはさんで
「コンピュータによる同時通訳に関する」に係る.
この例は一見かなり複雑にみえるが,上のように問題なく依存構造が
与えられる.
このような複雑な例でもうまく扱えるということは,係り受け解析を拡張した
本手法の一般性を示しているといえる.

\subsection{実験システムの性能}\label{sec:Evaluation:Experiment}

本手法の有効性を定量的に検証するために,実験システムを作成した.
優先度の計算法としては,\ref{sec:Uniform:Depend:Preference}\,節で
概略を述べたものを用いた.
実験は,ADDの10対話(662文)を対象として,交差検定(cross validation)で
行なった.
実験の詳細については,別稿\cite{伝:言処-投稿中}を参照されたい.

実験の結果,依存関係解釈の正解率は66\%,文全体の解釈の正解率は49\%で
あった(人手で付与したものと完全に一致したときのみ正解).
不適格性の検出・修正だけでなく,構文・意味解析の総合性能を評価して
いることを考えると,悪くない成績といえる.
一方,言い直しに注目すると,統語的言い直しでは再現率90\%,適合率47\%,
意味的言い直しでは再現率88\%,適合率32\%であった(音韻的言い直しでは,
修復対象と適正な語との曖昧性が生じなかったために,再現率,適合率
ともに100\%であった).

さらに,言い直しの解析の適合率の低さが統一モデルに起因するものかどうか
調べるために,この手法を二段階モデルに組み込んで比較した.
すなわち,言い直しに対する解釈規則を除いた規則群で第一段階の解析を
行ない,それが失敗した場合に,言い直しに対する解釈規則を加えて再解析を
行なう.
この結果,適合率は,統語的言い直しでは75\%と大きく改善されたが,
意味的言い直しでは33\%に改善されたに過ぎなかった.
逆に,再現率はそれぞれ,30\%, 12\%と大きく低下した.
これは,言い直しの多くが適格文として誤って解析されたことを表し(事実,
適格な依存関係の解析の適合率が二段階モデルでは低下した),
\citeA{佐川:IPSJ-NL-94-100-73}が観察した不適格文と適格文との曖昧性が
二段階モデルでは大きな問題となることがわかった.
これに対し,統一モデルでは,音響的・韻律的情報を用いることによって,
誤って言い直しと判断される例を除去できる可能性が
ある\cite{O'Shaughnessy:ICSLP92-931,Nakatani:ACL93-46}.

\subsection{適用範囲}\label{sec:Evaluation:Limitation}

本研究のターゲットである言い淀み,言い直し,繰り返し,助詞省略のうち,
本手法では扱えないものを以下にあげる.
\begin{enumerate}
\renewcommand{\theenumi}{}
\renewcommand{\labelenumi}{}
  \item\label{Limitation:1}
文節の途中で生じた言い淀みや言い直しは扱えない.
例えば,(\ref{eq:Limitation1})では,冗長語「あのー」が文節
「こちらが」の途中で生じている.
(\ref{eq:Limitation2})では,言い直しの訂正が文節の途中から生じている.
(\ref{eq:Limitation3})では,助詞を含む文節全体(「アブストラクトが」)の
代わりに助詞(「が」)だけが言い換えられている.
\enumsentence{\label{eq:Limitation1}
  こちら,{\bf あのー},が朝食の,えー,お値段.}
\enumsentence{\label{eq:Limitation2}
  えーと,この時点で,提出していただく{\bf だく}のは}
\enumsentence{\label{eq:Limitation3}
  あのー,アブストラクト{\bf を}があれば}

ADDの613対話に対する粗い見積りでは,言い淀み,言い直しが
文節の途中で生じる割合は,それぞれ,0.15\%, 3.7\%であった.
後者の半数は助詞の言い換えである.
  \item\label{Limitation:2}
複合的な原因によって生じた言い直しは扱えない.
例えば,(\ref{eq:Limitation4})は,音韻的な原因と意味的な原因が複合した
言い直しと考えられる(「おおさ」は「大阪」の発語を中断したものであり,
かつ,「大阪」自体も「京都」を言い誤ったものである).
\enumsentence{\label{eq:Limitation4}
  東京から,{\bf おおさ},えー,京都まで}
  \item\label{Limitation:3}
修復対象が2文節以上ある言い直しのうち,主要部が中断語であるものは
扱えない.
例えば,(\ref{eq:Limitation5})では,修復対象「これはきん」の主要部
「きん」は訂正部分の主要部「禁ぜられています」の中断語であると
思われるが,「きん」が不要語になってしまうため,修復対象の内部構造を
正しく解析できない.
\enumsentence{\label{eq:Limitation5}
  {\bf これはきん},えー,一応,これは,えー,禁ぜられていますんで}
\end{enumerate}

(\ref{Limitation:1})は,係り受け解析を基本とする本手法とは別に扱う
必要がある.
(\ref{Limitation:2}), (\ref{Limitation:3})は,中断語の扱いに関する
問題点であり,今後検討したい.

\section{従来の手法との比較}\label{sec:Comparison}

不適格文を扱う手法の研究は,\ref{sec:Intro}\,節で述べたように,
(A)\,不適格性を扱う個別的な手法,
(B-1)\,二段階モデルに基づく一般的な手法,
(B-2)\,統一モデルに基づく一般的な手法
の3つのアプローチに大別できる.
これらと本稿の手法との比較を述べる.

不適格性を扱う個別的な手法\cite{Hindle:ACL83-123,Bear:ACL92-56,
Nakatani:ACL93-46,佐川:情処論-35-1-46,山本:情処論-33-11-1322}に
おいては,言い直しの扱いなどが他の構文・意味解析過程とは
独立して論ぜられている.
これらの中には,再現率・適合率による評価において,本稿の手法より
優れているものもある(例えば,\citeA{Nakatani:ACL93-46}は言い直しの
検出で再現率83\%と適合率94\%を達成している)が,
これらの手法を構文・意味解析システムに組み込んだときに
どれだけ有効かは明らかでない\footnote{
  \citeauthor{Bear:ACL92-56}は後の報告\cite{Dowding:ACL93-54}に
おいて,言い直しの処理を構文・意味解析システムに組み込んだ際の
再現率・適合率がそれぞれ30\%, 62\%であったとしている.
}.
特に,\ref{sec:Ill-formed:Repair}\,節で述べたように,
日本語においては適格文と不適格文の曖昧性が生じる場合が多いので,
不適格性の処理を構文・意味解析過程の中にうまく組み込む方法を
考えることがとりわけ重要である.

一方,不適格性を扱う一般的な手法を話し言葉の不適格性の扱いに
具体的に適用した研究は見られない.
二段階モデルに基づく手法\cite{Jensen:CL-9-3-147,McDonald:ANLP92-193,
Weischedel:CL-9-3-161,Mellish:ACL89-102}の中には,話し言葉の処理に
適用できそうなものもある(例えば,チャート法の拡張によって
文中の余分な語の削除や欠落している語の挿入を
行なう\citeA{Mellish:ACL89-102}の手法)が,具体例がないので
その有効範囲は明らかでない.
また,統一モデルに基づく手法\cite{Fass:CL-9-3-178,Hobbs:AI-63-69}は,
動作原理と簡単な例の提示に留まっており,
本稿で示したような広範囲な実例への適用については述べられていない.
本稿の手法は,既存の統一モデルに基づく手法からアイデアを借用している
部分もある(例えば,確定節による規則の記述とアブダクションによる解釈は
\citeA{Hobbs:AI-63-69}のモデルの拡張である)が,話し言葉の不適格性を
扱うための具体的な規則や優先度計算の具体的な手法を与えている
点において,既存の手法に優る.

まとめると,本稿の手法は,不適格性を扱うための具体的な方法を与えつつ,
特定の不適格性の扱いにとどまらず構文・意味解析過程全体を考慮した
手法を実現している点が,従来の手法に対して優れているといえる.
\section{おわりに}\label{sec:Conclude}

本稿では,テキスト(漢字仮名混じり文)に書き起こされた日本語の話し言葉の
文からその文の格構造を取り出す構文・意味解析処理の中で,言い淀み,
言い直しなどの不適格性を適切に扱う手法について述べた.
本手法は,適格文と不適格文を統一的に扱う統一モデルに基づいており,
具体的には,係り受け解析の拡張によって実現されている.
本稿では,まず,音声対話コーパスからの実例をあげながら統一モデルの
必要性を述べ,次に,本手法の詳細を説明した後,その有効性を解析の
実例をあげるとともに実験システムの性能を評価することで示した.
その結果,さまざまな不適格性を含む複雑な話し言葉の文が,
係り受け解析を基本とする本手法によってうまく扱えることが示され,
さらに,定量的にも,試験文の約半数に完全に正しい依存構造が与えられる
ことが示された.
今後の課題としては,適用範囲の拡大とともに,音響的・韻律的情報を
利用した不適格性の解析の高精度化があげられる.
\bibliographystyle{jnlpbbl}
\bibliography{main}


\begin{biography}
\biotitle{略歴}
\bioauthor{伝 康晴}{
1988年京都大学工学部電気工学第二学科卒業.
1993年同大学大学院博士後期課程研究指導認定退学.
京都大学博士(工学).
1991年より2年間ATR自動翻訳電話研究所滞在研究員.
1993年国際電気通信基礎技術研究所入社,ATR音声翻訳通信研究所研究員.
1996年10月より奈良先端科学技術大学院大学情報科学研究科助教授,現在に至る.
計算言語学,認知科学の研究に従事.
日本認知科学会,日本ソフトウェア科学会,人工知能学会,情報処理学会,
言語処理学会各会員.}

\bioreceived{受付}
\biorevised{再受付}
\bioaccepted{採録}

\end{biography}


\appendix

\begin{table}[h]
\caption{非終端記号}
\label{tab:Sym}
\smallskip
\centering
\begin{tabular}{|c|l|}
  \hline
  \Var{Sym} & \hfil 統語範疇 \\\hline\hline
  \Rel{nonlex} & 不要語 \\\hline
  \Rel{adn} & 連体詞 \\\hline
  \Rel{adv} & 副詞 \\\hline
  \Rel{n} & 普通名詞,固有名詞などの名詞類 \\\hline
  \Rel{v} & が動詞,がを動詞,が形容詞などの動詞類 \\\hline
  \Rel{n\_v} & 名詞類 + 繋辞 \\\hline
  \Rel{v\_n} & 動詞類 + 補文標識 \\\hline
  \Rel{n\_v\_n} & 名詞類 + 繋辞 + 補文標識 \\\hline
  \Rel{v\_n\_v} & 動詞類 + 補文標識 + 繋辞  \\\hline
\end{tabular}
\end{table}
\begin{table}[h]
\caption{依存構造における係り文節と受け文節の組合せ}
\label{tab:Struct}
\vspace*{-\bigskipamount}
\centering
\begin{tabular}[t]{|c|c|c|}
  \hline
  \Var{Sym1} & \Var{Sym2} & \Var{Depend} \\\hline\hline
  \Rel{nonlex} & \Rel{nonlex}
  & \Rel{dep\_nonlex\_nonlex} \\\hline
  \Rel{nonlex} & \Rel{adn}, \Rel{adv}, \Rel{n}, \Rel{n\_v},
                 \Rel{n\_v\_n}, \Rel{v}, \Rel{v\_n},
                 \Rel{v\_n\_v}\,のいずれか
  & \Rel{dep\_nonlex\_any} \\\hline
  \Rel{adn} & \Rel{adn} & \Rel{dep\_adn\_adn} \\\hline
  \Rel{adn} & \Rel{n}, \Rel{n\_v}, \Rel{n\_v\_n}\,のいずれか
  & \Rel{dep\_adn\_n} \\\hline
  \Rel{adv} & \Rel{adv} & \Rel{dep\_adv\_adv} \\\hline
  \Rel{adv} & \Rel{v}, \Rel{v\_n}, \Rel{v\_n\_v}\,のいずれか
  & \Rel{dep\_adv\_v} \\\hline
  \Rel{n} & \Rel{n}, \Rel{n\_v}, \Rel{n\_v\_n}\,のいずれか
  & \Rel{dep\_n\_n} \\\hline
  \Rel{n} & \Rel{v}, \Rel{v\_n}, \Rel{v\_n\_v}\,のいずれか
  & \Rel{dep\_n\_v} \\\hline
  \Rel{v} & \Rel{n}, \Rel{n\_v}, \Rel{n\_v\_n}\,のいずれか
  & \Rel{dep\_v\_n} \\\hline
  \Rel{v} & \Rel{v}, \Rel{v\_n}, \Rel{v\_n\_v}\,のいずれか
  & \Rel{dep\_v\_v} \\\hline
\end{tabular}
\end{table}
\begin{table}[h]
\caption{適格な依存関係に関する解釈(一部)}
\label{tab:Well-formed}
\smallskip
\centering
\begin{tabular}{|c|c|c||l|}
  \hline
  \Var{Depend} & \Var{SemRel} & \Var{SynRel} & \hfil 例 \\\hline\hline
  \Rel{dep\_adn\_n}
  & \Rel{adnRel} & \Rel{rentai} & \Pair{この,\,会議} \\\hline
  \Rel{dep\_adv\_v}
  & \Rel{advRel} & \Rel{renyo} & \Pair{多分,\,簡単だ} \\\hline
  & \Rel{of} & \Rel{gen} & \Pair{日本の,\,首相} \\\cline{2-4}
  \Rel{dep\_n\_n}
  & \Rel{in} & \Rel{gen} & \Pair{京都の,\,ホテル} \\\cline{2-4}
  & \Rel{from} & \Rel{karano} & \Pair{大学からの,\,参加者} \\\hline
  & \Rel{agen} & \Rel{nomAct} & \Pair{学生が,\,研究する} \\\cline{2-4}
  & \Rel{obje} & \Rel{nomAct} & \Pair{会議が,\,興味深い} \\\cline{2-4}
  \Rel{dep\_n\_v}
  & \Rel{obje} & \Rel{accAct} & \Pair{日本語を,\,話す} \\\cline{2-4}
  & \Rel{obje} & \Rel{nomPass} & \Pair{日本語が,\,話される} \\\cline{2-4}
  & \Rel{loct} & \Rel{de} & \Pair{大学で,\,研究する} \\\hline
  \Rel{dep\_v\_n}
  & \Rel{that} & \Rel{toiu} & \Pair{分析するという,\,研究} \\\hline
  \Rel{dep\_v\_v}
  & \Rel{caus} & \Rel{node} & \Pair{興味深いので,\,聞く} \\\hline
\end{tabular}
\end{table}
\begin{table}[h]
\caption{不適格な依存関係に関する解釈}
\label{tab:Ill-formed}
\smallskip
\centering
\begin{tabular}{|c|c||l|}
  \hline
  \Var{Depend} & \Var{IllRel} & \hfil 例 \\\hline\hline
  \Rel{dep\_nonlex\_nonlex}
  & \Rel{hest} & \Pair{えーっと,\,あのー} \\\hline
  \Rel{dep\_nonlex\_any}
  & \Rel{hest} & \Pair{えーっと,\,そちら} \\\cline{2-3}
  & \Rel{phonRepair} & \Pair{つう,\,通訳電話} \\\hline
  \Rel{dep\_adn\_adn}
  & \Rel{semRepair} & \Pair{同じ,\,同一} \\\hline
  \Rel{dep\_adv\_adv}
  & \Rel{semRepair} & \Pair{直接に,\,簡単に} \\\hline
  & \Rel{synRepair} & \Pair{カードを,\,カードの} \\\cline{2-3}
  \Rel{dep\_n\_n}
  & \Rel{semRepair} & \Pair{通訳,\,翻訳} \\\cline{2-3}
  & \Rel{rept} & \Pair{通知は,\,通知は} \\\hline
  & \Rel{synRepair} & \Pair{持ち込んで,\,持ち込んでいただいて} \\\cline{2-3}
  \Rel{dep\_v\_v}
  & \Rel{semRepair} & \Pair{つもりです,\,予定です} \\\cline{2-3}
  & \Rel{rept} & \Pair{行くと,\,行くと} \\\hline
\end{tabular}
\end{table}
\end{document}
