\documentstyle[epsf,nlpbbl]{jnlp_e_b5}
\newcounter{example-number}
\newcommand{\example}[1]{}

\newenvironment{singlespace}{}{}

\newtheorem{princ}{}
\newenvironment{pr}[2]{}{}

\setcounter{page}{17}
\setcounter{巻数}{4}
\setcounter{号数}{4}
\setcounter{年}{1997}
\setcounter{月}{10}
\受付{August}{27}{1996}
\再受付{January}{17}{1997}
\採録{May}{6}{1997}

\setcounter{secnumdepth}{2}

\title{}
\author{}
\jkeywords{}

\etitle{A Method of Ordering English Adverbs \\
--- as exemplified in Japanese-to-English \\
    Machine Translation ---}

\eauthor{Kentaro Ogura\affiref{NTTCSlabs}   \and
         Francis Bond\affiref{NTTCSlabs}        \and
         Satoru Ikehara\affiref{TottoriUniv}}

\headauthor{Ogura,~K.~et~al.}
\headtitle{A Method of Ordering English Adverbs}

\affilabel{NTTCSlabs}
          {NTT Communication Science Laboratories, NTT}
          {NTT Communication Science Laboratories, NTT}
\affilabel{TottoriUniv}
          {Faculty of Engineering, Tottori University}
          {Faculty of Engineering, Tottori University}

\eabstract{
This paper proposes a new method for ordering English adverbs.  First, we
propose a classification of adverbs for English adverb generation.
Adverbs are classified into 41 classes by grammatical function
(adjuncts, subjuncts, disjuncts and conjuncts), meaning (process,
space, time etc.) and their default positions in sentences (initial,
medial, end, pre, and post).  
Then a method to order English adverbs correctly is described, 
using the proposed adverb classification and principles of
word ordering for adverbs (principles of ordering between adverbs and
other sentence constituents and principles of ordering between
adverbs).  In particular we give detailed rules for
deciding precedence when two adverbs have the same default position.
Exceptions to the default adverb generating process are also
described.  Finally, the proposed method is examined in three
experiments from the point of view of Japanese-to-English machine
translation.  The first experiment focuses on aspects of various
types of adverbs and a comparison of the proposed method and the
previous method.  The second experiment focuses on aspects of
quantitative coverage, and the third looks at aspects of practical use.  
The results show an accuracy of 97\% or more in all experiments
which highlights the efficiency of the proposed method.  The third
experiment, in particular, with an accuracy of 99\%, confirms that the
proposed method is effective in practical applications.
}

\ekeywords{adverbs, generation, ordering, machine translation}

\begin{document}
\maketitle 


\section{Introduction}
\label{sec:intr}

Adverbs\footnote{In this paper we treat idiomatic adverbial phrases,
  such as {\em on purpose} as adverbs.} express a wide variety of
meanings in sentences. An examination of adverb frequency in English
newspaper sentences showed that adverbs appeared 585 times in 1,000
sentences, that is, there was an adverb in one of every two sentences
on average.  Thus the correct generation of English adverbs is
important for accurate machine translation.  However, in natural
language processing, the study of adverbs has not developed very far
to date, in comparison with verbs and nouns, because adverbs often do not
construct the main parts of sentence meaning.
In addition they have various complex grammatical functions in
sentences, so they are very difficult to deal with in machine
translation.

Linguists have examined adverb grammatical functions and meanings in
detail \cite{Quirk:85,Otsuka:82,Rivero:92}.  In particular, conjuncts
and disjuncts, usually called sentence adverbs, are extensively
treated in \citeA{Greenbaum:83}.  Our method is based on studies by these
linguists about the grammatical functions and meanings of adverbs.
However, adverb studies by linguists show only proto-typical examples.
To get a more comprehensive view, we compiled a Japanese and English
bilingual corpus of sentences which contained 6,264 examples of
different English adverbs (2,402 different English adverbs) and which
also contained peripheral examples.\footnote {Some adverbs are very
difficult to classify.  But we must classify them to treat them with a
computer.} The corpus was collected from English-to-Japanese,
Japanese-to-English and English-to-English Dictionaries and grammar
books.  In the corpus, adverbs were classified according to our
classification of English adverbs. Bilingual data clarified the
different grammatical functions and meanings of the English adverbs.
Referring to the adverb corpus, we classified English adverbs in a
15,000-entry Japanese-to-English adverb transfer dictionary, for
machine translation.

There have been various studies of specific adverbs by linguists, such
as {\em even}\/\,\cite{Berckmans:93}, {\em still} and {\em
  already}\/\,\cite{vanderAuwera:93} and temporal adverb studies which
handle temporal semantics in sentences~\cite{Vlach:93}.  There are
also studies of adverb position in English in general~\cite{Ernst:84}
and positions of specific adverbs, such as {\em
  however}\/\,\cite{Sugiura:91}.
Studies of adverbs by linguists are written for fellow humans and  it
is difficult to transfer the results directly to natural language
processing. The adverb corpus is, therefore, used to examine and determine
the default position of adverbs and the most suitable order between adverbs 
with the same default position.   It is also used to determine the
principles of ordering between adverbs and other sentence constituents
and the principles of ordering between adverbs.

In natural language processing, however, few
studies\,\cite{Conlon:92,Glasbey:93} have considered adverbs.
\citeA{Conlon:92} aimed to decrease ambiguity in adverb meanings and
to select words during generation by applying information about adverb
semantics and syntax from linguistic studies to an adverb lexicon.
They did not use grammatical function clearly and do not show the
ordering between adverbs and other constituents in sentences.

  In research into adverb processing in machine translation, 
\citeA{Kamei:90} studied a syntactic analysis method for English-to-Japanese 
machine translation.
The technique uses the positions of English adverbs and the
relationship between an adverb modifier and its modificant.
However the authors do not show the effectiveness of the method for
English-to-Japanese machine translation quantitatively 
nor do they show the ordering between adverbs and 
other constituents in sentences.

There are four main problems of adverb processing in Japanese-to-English 
machine translation:

\begin{description}
  \item [Problems of analysis:] \mbox{}\\
    the multiplicity of adverb meanings 
  \item [Problems of transfer:]  \mbox{}\\
    differences in expression between Japanese and English for adverbial 
        meaning
  \item [Problems of generation:]  \mbox{}\\
    word ordering of English adverbs in English generation 
  \item [Problems of knowledge representation:] \mbox{}\\ 
    representation of adverbs in a
    computer\,\cite{Ogura:92,Shimazu:83,Iida:83,Moore:93}
   \vspace*{2mm}
\end{description}

In this paper, we focus our attention on the third problem: word
ordering in English adverb generation.  Among the four problem areas,
word ordering is comparatively easy to
treat in a superficial manner, but is much harder to study in detail.

In Section~\ref{sec:classification}, we classify English adverbs into
41 classes by their grammatical functions, meanings and default
positions in sentences.  In Section~\ref{sec:method}, we propose an
order for English adverbs based on their default positions and the
principles which govern the order of English adverbs.  
We present a word ordering method, based on our proposed classification, 
order and principles.  
In Section~\ref{sec:results}, we consider three sets of English sentences
generated in a test of the proposed word ordering method.  The
proposed method was tested in these three experiments by focusing on
aspects of various types of adverbs, comparing the proposed method and
a previous method, ascertaining the quantitative coverage, and
confirming its practical use. The results show an accuracy of 97\% or
more in all experiments and show the efficiency of the proposed
method.  Finally, Section~\ref{sec:conclusion} summarizes the main
values of the proposed method and suggests some directions for further
work.

This method has been implemented in the Japanese-to-English machine
translation system {\bf ALT-J/E} (the Automatic Language Translator
--- Japanese to English) \cite{Ikehara:89,Ikehara:91,Ogura:93}.

\section{Classification of English Adverbs}
\label{sec:classification}

\subsection{Basic Position}

  Adverbs (or Adverbials) can be put in different positions in a sentence. 
Figure~\ref{fig:position} illustrates the various possible positions of 
the adverbial {\em by then}.

\begin{singlespace}
\begin{figure}[htbp]
  \vspace*{-5mm}
  \begin{center}
    \leavevmode
    \small
    \begin{tabbing}
     {\em By then} the book must have been placed on the shelf. 
     \hspace{2cm} \= {\bf I}  \\
     The book {\em by then} must have been placed on the shelf.
     \hspace{2cm} \> iM \\
     The book must {\em by then} have been placed on the shelf. 
     \hspace{2cm} \> {\bf M}  \\
     The book must have {\em by then} been placed on the shelf. 
     \hspace{2cm} \> mM \\
     The book must have been {\em by then} placed on the shelf. 
     \hspace{2cm} \> eM \\
     The book must have been placed {\em by then} on the shelf. 
     \hspace{2cm} \> iE \\
     The book must have been placed on the shelf {\em by then}. 
     \hspace{2cm} \> {\bf E}  \\
    \end{tabbing}
  \end{center}
  \vspace*{-5mm}
  \caption{Positions of Adverbials [from (Quirk, et al., 1985:490)]}
  \label{fig:position}
\end{figure}
\end{singlespace}

In our adverb position system for English adverb generation, 5 positions 
are provided. 
The first is {\bf initial position}, the beginning of the sentence,
as illustrated by ``I'' in Figure~\ref{fig:position}. 
The second is {\bf medial position}, between the subject and predicate, or 
if auxiliary verbs are involved in the sentence just after the first 
auxiliary verb. It is illustrated by ``M'', ``iM'', ``mM'' and ``eM''.
Adverbs do not normally appear in positions ``iM'', ``mM'' and ``eM'', 
so they are not supported in our adverb position system. 
The third is {\bf end position}, after a predicate. 
It is illustrated by ``iE'' or ``E''. 
In our system, whether an adverb should be at ``iE'' or ``E'' is 
decided by its position order between adverbs 
(see Figure~\ref{fig:order-e} in \ref{subsubsec:order}).

{\bf Pre position} and {\bf post position} are provided for adverbs 
used as modifiers. 
A pre position adverb comes before its modificant and 
a post position adverb comes after its modificant.

\smallskip
\begin{quote}
 \small
  \begin{tabular}{@{}l@{$\;\;$}l@{$\;$}l@{}}
    {\bf pre position:} & (\example{ex:pre-position}) & 
    He had {\em only} \underline{two dollars}. \\
    {\bf post position:}& (\example{ex:post-position}) & 
    \underline{The Scots} {\em in particular} are very 
    proud of their separate nationality. \\
  \end{tabular}
\end{quote}

\subsection{Classification}

Adverbs usually have many grammatical functions and meanings,
especially adverbs which are used frequently in our daily life.
Normally differences in grammatical function and meaning are reflected
by the position in the sentence.

Example sentences using {\em again} are shown in
Figure~\ref{fig:again}.  When it is a conjunct, it appears in the initial
position, when it is an additive focusing subjunct, in the post position,
and when it is a time adjunct, in the end position.

\begin{singlespace}
\begin{figure}[tb]
  \begin{center}
    \leavevmode
    \small
    \begin{itemize}
       \item Conjunct (Initial position)\\
       (\example{ex:again1})
       {\em Again}, the psychologist can observe the child at work 
       and at play.
       \item Additive Focusing Subjunct (Post position)\\
       (\example{ex:again2})
       The psychologist, {\em again}, can observe the child at work 
       and at play. \\
       \ ( = The psychologist also can observe ...)
       \item Time Adjunct (End position)\\
       (\example{ex:again3})
       The psychologist can observe the child at work 
       and at play {\em again}. \\
       \ ( = The psychologist can observe ... once more.)
    \end{itemize}
  \end{center}
  \vspace*{-3mm}
  \caption{Example sentences with {\em again} [from (Greenbaum 1983:65)]}
  \label{fig:again}
  \vspace*{-3mm}
\end{figure}
\end{singlespace}

Other examples are shown in Figure~\ref{fig:frankly}.
In the case of {\em frankly}, it can be used as an adjunct, subjunct or 
disjunct. When it is a manner adjunct it appears in the end position, and
when it is a subjunct or a disjunct it appears in the initial position.

\begin{singlespace}
\begin{figure}[tbp]
  \begin{center}
    \leavevmode
    \small
    \begin{itemize}
       \item Manner Adjunct (End position)\\
       (\example{ex:frankly1})
       She told me {\em frankly} that she did not like him. \\
       \ ( = She told me in a frank manner ...)
       \item General Subject-orientation Item Subjunct (Initial position)\\
       (\example{ex:frankly2})
       {\em Frankly}, he explained his position to me. \\
       \ ( = He was frank when ...)
       \item Style Disjunct (Initial position)\\
       (\example{ex:frankly3})
       {\em Frankly}, I cannot afford to buy the car. \\
       \ ( = Frankly speaking, ...)
    \end{itemize}
  \end{center}
  \vspace*{-3mm}
  \caption{Example sentences with {\em frankly}}
  \label{fig:frankly}
\end{figure}
\end{singlespace}

Even if two adverbs have the same grammatical function and meaning,
their default positions in a sentence may be different.  For example,
the indefinite time-frequency adjunct {\em always} normally takes the
medial position, while the indefinite time-frequency adjunct {\em at
  all times}, which has the same meaning, is normally put in the end
position.

\smallskip
\begin{quote}
 \small
 (\example{ex:always}) No man is {\em always} wise. \\
 (\example{ex:at_all_times}) No man is wise {\em at all times}.
\end{quote}

To handle these linguistic phenomena, in this paper we classify adverbs 
by grammatical functions, meanings and default positions in sentences.

  Adverbs can be roughly divided into 
adjuncts (Figure~\ref{fig:adjuncts}), 
subjuncts (Figure~\ref{fig:subjuncts}),
disjuncts (Figure~\ref{fig:disjuncts}),
conjuncts (Figure~\ref{fig:conjuncts}) 
and post position numeral modifiers (Figure~\ref{fig:ppnm}),
according to their grammatical functions. 

\begin{figure}[htb]
\fbox{\parbox{.95\textwidth}{
    \small
  \begin{description}
    \begin{singlespace}
   \item [Adjuncts] - 
      \begin{description}
         \item [Manner] as always, easily, fast, patiently, steadily, violently
         \item [Means] by air, by force, collect, in order, legally, surgically
         \item [Instrument] microscopically (= with a microscope)
         \item [Position] around, behind, here, in, in the distance, out, there
         \item [Direction] down, east, from a distance, forward, left, 
                           round, upright
         \item [Time-position] - 
            \begin{description}
               \item [Medial position] already, finally, just, now, once, soon
               \item [End position] a long time ago, again, at noon, last week,
                                    late, previously, recently, tomorrow
            \end{description}
         \item [Span] - 
            \begin{description}
               \item [Medial position] momentarily, permanently, temporarily
               \item [End position] all day, for a while, 
                                    from beginning to end, in a day, overnight
            \end{description}
         \item [Time-frequency] - 
            \begin{description}
               \item [Definite] every day, nightly, once, twice, weekly
               \item [Indefinite] - 
                  \begin{description}
                     \item [Medial position] always, constantly, often, 
                                             rarely, sometimes, usually
                     \item [End position] again and again, at all times, 
                                          every now and again, now and then
                  \end{description}
            \end{description}
      \end{description}
    \end{singlespace}
  \end{description}
}}
  \caption{Adverb Classification: Adjuncts}
  \label{fig:adjuncts}
\end{figure}

\begin{figure}[htbp]
\fbox{\parbox{.95\textwidth}{
    \small
  \begin{description}
    \begin{singlespace}
   \item [Subjuncts] - 
      \begin{description}
         \item [Viewpoint] linguistically, logically, mathematically
         \item [Courtesy]  cordially, graciously, humbly, kindly, pray, please
         \item [Subject-orientation Item] - 
            \begin{description}
               \item [General] bitterly, resentfully, miserably, 
                               with great unease
               \item [Volitional] - 
                  \begin{description}
                     \item [Medial position] intentionally, purposely, 
                                             unwillingly
                     \item [End position] against one's will, on purpose
                  \end{description}
            \end{description}
         \item [Focusing] - 
            \begin{description}
               \item [Restrictive] - 
                  \begin{description}
                     \item [Pre position] chiefly, even, just, 
                                          only, specially
                     \item [Post position] above all,  
                                           in particular, precisely, sharp
                  \end{description}
               \item [Additive] - 
                  \begin{description}
                     \item [Pre position] even, still, yet
                     \item [Post position] again, alike, besides, more
                     \item [End position] as well, either, too
                     \item [Initial position] neither
                     \item [Medial position] also
                  \end{description}
            \end{description}
         \item [Emphasizers] - 
            \begin{description}
               \item [Pre position] actually, honestly, indeed, 
                                    obviously, really, surely
               \item [End position] all right, by all means, for certain, 
                                    without fail
            \end{description}
         \item [Amplifiers] - 
            \begin{description}
               \item [Pre position] absolutely, badly, completely, deeply, 
                                    much, so, very 
               \item [Post position] enough, ever, in the devil, in the world, 
                                     on earth
               \item [End position] all over, at all, to excess, 
                                    to the full, so much, worse
               \item [Pre-and-End position] a good deal, a great deal, a lot
            \end{description}
         \item [Downtoners] - 
            \begin{description}
               \item [Pre position] about, almost, considerably, hardly, 
                                    never, quite, rather
               \item [Post position] all told, altogether, at (the) best,
                                     enough, or so
               \item [End position] a little more, in the slightest, 
                                    to some extent
               \item [Pre-and-End position] a bit, a little, any, at all, 
                                            in the least, more
            \end{description}
      \end{description}
    \end{singlespace}
  \end{description}
}}
  \caption{Adverb Classification: Subjuncts}
  \label{fig:subjuncts}
\end{figure}

\begin{figure}[htbp]
\fbox{\parbox{.95\textwidth}{
    \small
  \begin{description}
   \begin{singlespace}
   \item [Disjuncts] - 
      \begin{description}
         \item [Style] briefly, frankly, generally speaking, in short, roughly
         \item [Content] - 
            \begin{description}
               \item [Degree of truth] actually, certainly, evidently, ok,
                                       probably, reportedly, sure enough
               \item [Value judgment] - 
                  \begin{description}
                     \item [Non-subject] amazingly, annoyingly, mercifully,
                                         remarkably, to one's relief
                     \item [Subject] foolishly, rightly, unjustly, wisely
                  \end{description}
            \end{description}
      \end{description}
    \end{singlespace}
  \end{description}
}}
  \caption{Adverb Classification: Disjuncts}
  \label{fig:disjuncts}
\end{figure}

\begin{figure}[htbp]
\fbox{\parbox{.95\textwidth}{
    \small
  \begin{description}
   \begin{singlespace}
   \item [Conjuncts] - 
      \begin{description}
         \item [Initial position] as a result, by the way, first, further, 
                                  for example, if so, in conclusion, instead, 
                                  namely, on the contrary, similarly, 
                                  to begin with, thus
         \item [End position] if any, if possible, though, considering
      \end{description}
   \end{singlespace}
  \end{description}
}}
  \caption{Adverb Classification: Conjuncts}
  \label{fig:conjuncts}
\end{figure}

\begin{figure}[tp]
\fbox{\parbox{.95\textwidth}{
    \small
  \begin{description}
   \begin{singlespace}
   \item [Post position numeral Modifiers] across, around, o'clock, through
   \end{singlespace}
  \end{description}
}}
  \caption{Adverb Classification:  Post position Numeral modifiers}
  \label{fig:ppnm}
\end{figure}

Adjuncts modify predicates and are central elements in sentence
structure, while subjuncts\mbox{}, disjuncts, conjuncts, and post position
numeral modifiers, are peripheral elements.  When considering order,
adjuncts need to be treated along with other case elements, such as
subjects and objects. We treat all adjuncts as case elements,
following \citeA{Shimazu:83}.  Subjuncts modify various levels of
sentence elements, for example, clauses, predicates and case elements,
and some subjuncts express modality.  They express the degree of their
modificant, and they play a subordinate role in comparison with other
clause elements. {\bf Pre-and-End position} subjuncts are
placed Pre position when they modify an adjective, passive voice verb
or adverb and in End position otherwise. {\em a little} in
(\ref{ex:pre-and-end-position1}) and (\ref{ex:pre-and-end-position2})
is an example of a pre-and-end position subjunct.

\begin{quote}
 \small
 (\example{ex:pre-and-end-position1})
 I'm {\em a little} {\bf tired}. \\
 (\example{ex:pre-and-end-position2})
 I {\bf know} him {\em a little}.
\end{quote}

Disjuncts and conjuncts are usually called
sentence adverbs.  Disjuncts modify sentences to show the speaker's
emotional attitudes or comment.  Conjuncts connect sentences or other
sentence elements and show the relationships of connected elements.

We also have the special class of post position numeral modifiers
which are not subjuncts, adjuncts, disjuncts or conjuncts.
They are used to modify noun phrases consisting of a numeral and a
unit.  They are located in the sentence after the noun phrase they
modify, for example:

\smallskip
\begin{quote}
 \small
 (\example{ex:numeral_modifier})
 The river is \underline{fifty yards} {\em across}.
\end{quote}

{\em Across} is a post position numeral modifier and it modifies the noun 
phrase ``fifty yards''.

\subsection{How to classify adverbs}

A problem in classifying adverbs is that adverbs have various meanings and
their division is not always clear. Grammar books examine proto-typical 
examples but often omit peripheral examples.

To solve this problem, 
we examined how adverbs were actually used in sentences 
in  English-to-Japanese dictionaries\,\cite{Takebayashi:90,Konishi:87}, 
Japanese-to-English dictionaries\,\cite{Kojima:90,Kondo:86}, 
English-to-English dictionaries\,\cite{LDOCE}, 
grammar books
\,\cite{Quirk:85,Quirk:72,Greenbaum:83,Leech:75,Okada:85,Hornby:77}
and handbooks of illustrative sentences
\,\cite{Horiguchi:83,Tada:77,Konishi:89}.
Any one dictionary does not have enough examples of adverbs and the examples 
may be unbalanced, therefore many dictionaries were examined. 
At present, 6,264 sentences using 2,402 different adverbs have been collected.

How adverbs are used in sentences can be examined by checking newspaper 
articles and manuals, but this requires a large volume of text to 
examine adverbs comprehensively, so we did not adopt this procedure.

We added adverb class information, determined from this 
classification, to adverbs in our dictionary (about 15,000 entries).\footnote{
In fact, in the implementation of ALT-J/E,
this information was added to the Japanese-to-English adverb dictionary
which shows correspondence between Japanese
adverbial expressions and English adverbs. Each dictionary entry
consists of the meaning expressed by a Japanese adverbial expression
and the meaning expressed by the corresponding English adverb.}

This adverb classification could be applied not only to adverbs but also
to prepositional\mbox{} phrases with adverbial usage. Prepositional phrases
express more diverse meanings than adverbs, so we expect that additional 
classes will be needed.

\section{Determining Word Order for English Adverbs}
\label{sec:method}

The basic adverb position in the sentence, that is, initial, medial,
end, pre or post position is given by the adverb classification in a
dictionary, once its meaning has been determined.  The precise
position is determined by the relations between the adverb and other
sentence elements.  The scope of the adverb is one of the important
factors influencing order, thus our method considers the scope of
adverbs. Context information such as topic, or new/old information, or
importance of information, can affect adverb positions, but this is
beyond the range of this paper.  Our method determines the default 
positions of adverbs.

First, we show the principles of ordering between adverbs and other
sentence constituents and the principles of ordering between adverbs.
In particular, we deal with cases in which two or more adverbs (or
adverbial phrases) occur in the same basic position (initial, medial,
end, pre or post position), and show how they should be ordered.
Finally we show how to process exceptions.
This method can be applied not only to adverbs but also to adverbials,
so we show examples of both. 

\subsection{Principles of ordering between adverbs and other sentence constituents}

The position of an adverb depends not only on the adverb's meaning but
also on the relationship between the adverb and the other sentence
elements.

\subsubsection{Order in sentence initial position}

  In the sentence initial position, the order between a conjunction,
an interjection and an initial position adverb is a problem.  We use
the following principle:

\begin{pr}{pr:conj-interj-initial}{}
  Conjunctions and interjections precede initial position adverbs. \\
  (Conjunctions precede  interjections.)
\end{pr}
\begin{quote}
 \small
 (\example{ex:i-position-adverb-and-conjunction})
 Children need many things, {\bf but} {\em above all} they need
 love.\footnote{Adverbs and adverbials are shown in italics, other
   relevant constituents are shown in bold font.}  \\
 \hspace{1cm}{\bf Conjunction} $<$  {\bf Initial position adverb}\footnote
{``{\bf A} $<$ {\bf B}'' shows the order at a given position. 
 ``{\bf A}'' comes before ``{\bf B}'' in the sentence.}\\
 (\example{ex:i-position-adverb-and-interjection1})
 Would you like a cup of tea? {\bf Yes}, {\em please}! \\
 \hspace{1cm}{\bf Interjection} $<$  
 {\bf Initial position adverb (Courtesy Subjunct)}\\
 (\example{ex:i-position-adverb-and-interjection2})
Can I use your telephone? {\em Sure}, {\bf go ahead}. \\
 \hspace{2cm}{\bf Initial position adverb (Content Disjunct with degree of truth)} \\
 \hspace{2cm} $<$~{\bf Interjection}
\end{quote}

 (\ref{ex:i-position-adverb-and-conjunction}) 
is an example which has an initial position adverb and a conjunction in 
a sentence.  
(\ref{ex:i-position-adverb-and-interjection1}) is a standard example 
which has an initial position adverb and an interjection.
(\ref{ex:i-position-adverb-and-interjection2}) is an exception to
Principle~\ref{pr:conj-interj-initial}.  Probably, it is because the
interjection {\it go ahead\/} is an idiom which retains verb
function. Idiomatic expressions tend to require exceptional 
treatments in generation.

\subsubsection{Order in sentence medial position}

In the sentence medial position, the order between adverbs, the predicate and
auxiliary verbs is a problem.


\begin{pr}{pr:predicate-medial}{}
    Medial position adverbs are placed just before the verb. \\
    If the predicate is the verb ``be'', \\
    then medial position adverbs are placed just after the verb. \\
    If there are one or more auxiliary verbs, \\
    then medial position adverbs are placed just after the first auxiliary 
    verb. \footnote{The later condition is stronger than former condition.}
\end{pr}

\begin{quote}
 \small
 (\example{ex:m-adverb-and-predicate1})
 She must {\em always} {\bf get up} at six. \\
 (\example{ex:m-adverb-and-predicate2})
 He {\bf is} {\em already} here. \\
 (\example{ex:m-adverb-and-predicate3})
 She {\bf would} {\em never} {\bf have} believed that story. \\
 (\example{ex:m-adverb-and-predicate4})
 He {\bf will} {\em soon} {\bf be} here. 
\end{quote}

(\ref{ex:m-adverb-and-predicate1}), (\ref{ex:m-adverb-and-predicate2}), 
(\ref{ex:m-adverb-and-predicate3}) and (\ref{ex:m-adverb-and-predicate4}) 
follow Principle \ref{pr:predicate-medial}. 

\subsubsection{Order in sentence end position}

In the generation of adjuncts, the order of adjuncts and noun phrase
case elements is important. Our method places complements (obligatory
or semi-obligatory case elements licensed by the predicate) closer to
the predicate than adjuncts (optional case elements),\footnote{The
Japanese-to-English machine translation system ALT-J/E has a pattern
dictionary which shows relationships between predicates and their case
elements.  Obligatory cases or semi-obligatory cases and their
positions are shown in the pattern dictionary entries.}  unless the
complement is a clause or a phrase which is modified by a clause.

\begin{pr}{pr:obligatory-optional}{}
    Optional case elements (adjuncts) come after 
    obligatory or semi-obligatory case elements (complements).\\
    Except:  non-subject case elements with embedded 
    clauses come after optional case elements.
\end{pr}

\begin{quote}
 \small
 (\example{ex:optional-case-and-obligatory case})
 I go {\bf to school} {\em slowly}. 
\end{quote}

In (\ref{ex:optional-case-and-obligatory case}), {\em to school} is a
``Direction adjunct'' and {\em slowly} is a ``Manner adjunct''.
Usually the order between ``Direct adjunct'' and ``Manner adjunct'' is:
\begin{flushleft}
  {\bf Manner~adjunct}~$<$~{\bf Direction~adjunct} (from
  Figure~\ref{fig:order-e}, Section~\ref{subsubsec:order})
\end{flushleft}
However, because {\it to school\/} is a semi-obligatory case element
of {\it go\/} and {\it slowly\/} is an optional case element, {\it to
  school\/} is placed before {\it slowly}.

\smallskip
\begin{quote}
 \small
 (\example{ex:adverb-with-embedded1})
 He studied them \underline{\em carefully} {\em that night}. \\
 (\example{ex:adverb-with-embedded2})
 He studied them {\em that night} 
 \underline{\em with the kind of care his wife had suggested}. 
\end{quote}

In (\ref{ex:adverb-with-embedded1}), {\em carefully} is a ``Manner
adjunct'' and {\em that night} is a ``Time-position adjunct'' and the
order between a ``Manner adjunct'' and a ``Time-position adjunct'' is:
\begin{flushleft}
  {\bf Manner adjunct} $<$ {\bf Time-position adjunct} (from
  Figure~\ref{fig:order-e}, Section~\ref{subsubsec:order})
\end{flushleft}
Therefore {\em carefully} comes before {\em that night}.  But in
(\ref{ex:adverb-with-embedded2}), {\em with the kind of care his wife
  had suggested} is a ``Process adjunct with embedded clause'', so the
exception is triggered and {\em that night} comes before {\em with the
  kind of care his wife had suggested}.


\subsection{Principles of ordering between adverbs}

When there are two or more adverbs in the same basic position, the
order is based on the scope of the adverbs.  We have prepared tables
of default scopes for each of the basic positions, which gives the
scope for most combinations of adverb classes (Figures~\ref{fig:order-i},
\ref{fig:order-m}, \ref{fig:order-e}, and~\ref{fig:order-pre}).  If
adverbs can be ordered by these tables, then we do so
(Principle~\ref{pr:proposed-order}).  Otherwise, for the combinations
of classes not given by default, we must determine the scope explicitly
(Principles~\ref{pr:scope} and~\ref{pr:inclusion}).

\subsubsection{Adverbs with the same basic position --- default scope}
\label{subsubsec:order}

\begin{pr}{pr:proposed-order}{}
    If two or more adverbs occur in the same default position
    consult the default orders given in Figures~\ref{fig:order-i},
    \ref{fig:order-m}, \ref{fig:order-e}, and~\ref{fig:order-pre}
    use the ordering given, if it exists.
\end{pr}

Figures~\ref{fig:order-i}--\ref{fig:order-post} show the default order
for adverbs in initial position, medial position, end position, pre
position and post position, for when two or more adverbs
come in the same basic position.  
The order reflects the scope of the adverbs. For example, conjuncts
usually have wider scope than disjuncts, so at the initial position
conjuncts come before disjuncts.

\begin{singlespace}    
\begin{figure}[htbp]
    \small
    {\bf Initial position} \\
    \hspace*{2cm} Conjuncts \\
    \hspace*{2cm} $<$ \\
    \hspace*{2cm} Style Disjuncts, \\
    \hspace*{2cm} Content Disjuncts with value judgment for non-subject, \\
    \hspace*{2cm} Content Disjuncts with value judgment for subject, \\
    \hspace*{2cm} Content Disjuncts with degree of truth \\
    \hspace*{2cm} $<$ \\
    \hspace*{2cm} Viewpoint Subjuncts, \\
    \hspace*{2cm} General Subject-orientation Item Subjuncts, \\
    \hspace*{2cm} Courtesy Subjuncts \\
    \hspace*{2cm} $<$ \\
    \hspace*{2cm} Additive Focusing Subjuncts
  \caption{Default Order of Initial Position Adverbs}
  \label{fig:order-i}
  \vspace*{-5mm}
\end{figure}
  \end{singlespace}

\begin{singlespace}    
\begin{figure}[htbp]
    \small
    {\bf Medial position} \\
    \hspace*{2cm} Additive Focusing Subjuncts \\
    \hspace*{2cm} $<$ \\
    \hspace*{2cm} Indefinite Time-frequency Adjuncts \\
    \hspace*{2cm} $<$ \\
    \hspace*{2cm} Time-position Adjuncts \\
    \hspace*{2cm} $<$ \\
    \hspace*{2cm} Span Adjuncts, \\
    \hspace*{2cm} $<$ \\
    \hspace*{2cm} Volitional Subject-orientation Item Subjuncts, \\
    \hspace*{2cm} Courtesy Subjuncts 
  \caption{Default Order of Medial Position Adverbs}
  \label{fig:order-m}
\end{figure}
\end{singlespace}

\begin{singlespace}    
\begin{figure}[htbp]
  \small
    {\bf End position} \\
    \hspace*{2cm} Amplifiers, Downtoners \\
    \hspace*{2cm} $<$ \\
    \hspace*{2cm} Manner, Means, Instrument Adjuncts \\
    \hspace*{2cm} $<$ \\
    \hspace*{2cm} Direction Adjuncts \\
    \hspace*{2cm} $<$ \\
    \hspace*{2cm} Position Adjuncts \\
    \hspace*{2cm} $<$ \\
    \hspace*{2cm} Span Adjuncts \\
    \hspace*{2cm} $<$ \\
    \hspace*{2cm} Time-frequency Adjuncts, \\
    \hspace*{2cm} $<$ \\
    \hspace*{2cm} Time-position Adjuncts \\
    \hspace*{2cm} $<$ \\
    \hspace*{2cm} Volitional Subject-orientation Item Subjuncts \\
    \hspace*{2cm} $<$ \\
    \hspace*{2cm} Emphasizers \\
    \hspace*{2cm} $<$ \\
    \hspace*{2cm} Additive Focusing subjuncts \\
    \hspace*{2cm} $<$ \\
    \hspace*{2cm} Conjuncts
  \caption{Default Order of End Position Adverbs}
  \label{fig:order-e}
\end{figure}
\end{singlespace}

\begin{singlespace}    
\begin{figure}[htbp]
    \small
    {\bf Pre position} \\
    \hspace*{2cm} Restrictive Focusing Subjuncts \\
    \hspace*{2cm} $<$ \\
    \hspace*{2cm} Additive Focusing Subjuncts \\
    \hspace*{2cm} $<$ \\
    \hspace*{2cm} Emphasizers \\
    \hspace*{2cm} $<$ \\
    \hspace*{2cm} Downtoners \\
    \hspace*{2cm} $<$ \\
    \hspace*{2cm} Amplifiers \\
    \vspace*{-2mm}
    \caption{Default Order of Pre Position Adverbs}
    \label{fig:order-pre}
\end{figure}
\end{singlespace}

\begin{figure}[htbp]
    \leavevmode
    \small    
    {\bf Post position} (no default order)\\
    \hspace*{2cm} Focusing Subjuncts, Amplifiers, Numeral modifiers 
    \caption{Default Order of Post Position Adverbs}
    \label{fig:order-post}
\end{figure}

These default orders were constructed after examining English corpora
about adverbs (about 7,000 sentences) and handmade corpora in which
English native speakers composed sentences which involve two or more
adverbs in the same position.  We also considered the scope of adverbs
to determine the order.

For example, {\em by then} in Figure~\ref{fig:position} is an end position 
time-position adjunct and {\em on the shelf} is a position adjunct, so 
according to the order at end position shown in 
Figure~\ref{fig:order-e} ``Position Adjunct'' comes before 
``Time-position Adjunct''. 
The proposed ordering system generates the sentence as follows.

\smallskip
\begin{quote}
 \small
 (\example{ex:e-position-order})
 The book must have been placed 
 \underline{\em on the shelf} \underline{\em by then}. \\
 {\bf Position Adjunct} $<$ {\bf Time-position Adjunct} (Fig~\ref{fig:order-e})
 
\end{quote}

(\ref{ex:m-position-order}), (\ref{ex:i-position-order}) and
(\ref{ex:pre-position-order}) are examples of the proposed ordering
system, in medial position, initial position and pre position
respectively.  Note, there is no default order for post position
adverbs.


\paragraph{Medial Position:}
\begin{quote}
 \small
 (\example{ex:m-position-order})
 He is {\em \underline{still} \underline{purposely}} against our plan. \\
 \footnotesize
 {\bf Time-position Adjunct} $<$ 
 {\bf Volitional Subject-orientation Item Subjunct} (Fig~\ref{fig:order-m})
 \vspace*{-2mm}
\end{quote}

\paragraph{Initial Position:}
\begin{quote}
 \small
 (\example{ex:i-position-order})
 There are two reasons why I think she cannot graduate from college. \\
 \hspace*{5mm} \underline{\em First}, \underline{\em probably} 
               she failed the examination. \\
 \hspace*{5mm} Second, she has no time to study for the next examination. \\
 \footnotesize
 {\bf Conjunct} $<$ {\bf Content Disjunct with degree of truth} (Fig~\ref{fig:order-i})
  
 \vspace*{-2mm}
\end{quote}

\paragraph{Pre Position:}
\begin{quote}
 \small
 (\example{ex:pre-position-order})
  This room is \underline{\em just} \underline{\em about} big enough. \\
 \footnotesize
 {\bf Restrictive Focusing Subjunct} $<$ {\bf Downtoner} (Fig~\ref{fig:order-pre})
  
 \vspace*{-2mm}
\end{quote}

\subsubsection{Adverbs with the same basic position --- dynamic scope}

Some adverb classes have no order between them, such as manner, means
and instrument adjuncts in end position, or two adverbs of the same
class.  In such cases, this system orders the English sentence by 
the following two principles (Principle 5 and 6).

\begin{pr}{pr:scope}{}
    If two or more adverbs occur in the same default position and
    there is no default order, order according to the scope.
\end{pr}

For example, in Japanese-to-English machine translation, when the
order between adverbs is not given by Principle
\ref{pr:proposed-order}, the scope of the Japanese adverbials which
correspond to the English adverbs may be available.  If one Japanese
adverbial expression is put nearer to the predicate than the other
Japanese adverbial expression, the English translated adverb is also
put nearer to the English translated predicate than the other
translated adverb.  This uses a heuristic that the scope
of the source language will be preserved in the target language by
translation. (\ref{ex:using-J-order1}), (\ref{ex:using-J-order2}) and
(\ref{ex:using-J-order3}) are examples in which adverbs are placed by
Principle \ref{pr:scope}. \vspace*{2mm} 

\begin{singlespace} 
\begin{tabular}{clllll}
\small 
(\example{ex:using-J-order1}) 
& Jpn:   & {\em fuyu-niwa} & taiy\={o}-wa        & {\em hayaku} & shizumu \\ 
& Gloss: & {\em in winter} & the sun {\sc [TOP]} & {\em early}  & sets \\ 
& Eng:   & \multicolumn{4}{l} 
         {The sun sets \underline{\em early} \underline{\em in winter}.}
\end{tabular}

\begin{tabular}{clllll}
 \small
(\example{ex:using-J-order2})
& Jpn:   & kare-wa           & {\em asa}              & {\em hayaku} 
         & shuppatsu-shi-ta \\
& Gloss: & He {\sc [TOP]}    & {\em in the morning}   & {\em early}
         & started \\
& Eng:   & \multicolumn{4}{l}
         {He started \underline{\em early} \underline{\em in the morning}.}
\vspace*{3mm}
\end{tabular}

\begin{tabular}{cllll}
 \small
(\example{ex:using-J-order3})
& Jpn:   & Kurushii omoi-wo shi     & ikan-to-wa omoi-nagara-mo 
         & sono k\={o}fu-tachi-wa   \\
&        & ch\={u}sei-wo omonji, 
         & jibun-tachi-no kumiai-ni & hantai-shita \\
& Gloss: & {\em Painfully}          & {\em resentfully} 
         & the miners {\sc [TOP]}       \\
&        & stand by their loyalty
         & their own conference     & went against \\
& Eng:   & \multicolumn {3}{l}
         {\underline{\em Painfully}, \underline{\em resentfully}, 
         the miners have stood by their loyalty,}\\
&        & \multicolumn {3}{l}
         {and gone against their own conference.}
\vspace*{3mm}
\end{tabular}
\end{singlespace}


In (\ref{ex:using-J-order1}) and (\ref{ex:using-J-order2}) {\em
  early}, {\em in winter} and {\em in the morning} are all ``End
position Time-position adjuncts''.  In Japanese (the source language),
{\em hayaku} ``early'' is put nearer to the predicate than either
{\em fuyu-niwa} ``in winter'' and {\em asa} ``in the morning''.
Therefore {\em early} has a narrower scope, and we put it
nearer to the English translated predicates (``sets'' and ``started'') than
{\em in winter} and {\em in the morning} respectively.\footnote 
{In (\ref{ex:using-J-order2}) a possible interpretation is that {\em early} 
modifies {\em in the morning}, thus {\em early} comes just before 
{\em in the morning}.} 
In (\ref{ex:using-J-order3}) {\em painfully} and {\em resentfully} are 
``General Subject-orientation Item subjuncts'',
{\em Kurushii omoi-wo shi} ``painfully'' comes before 
{\em Ikan-to-wa omoi-nagara-mo} ``resentfully'', thus in the translation
{\em painfully} comes before {\em resentfully}.


A more specific rule is applied when there are two or more adverbials
in the same basic position, they are both ``Time-position adjunct''
or ``Position adjunct'' and one includes the other.

\begin{pr}{pr:inclusion}{}
     When there are two or more adverbials in the same basic position 
     and they are both ``Time-position adjunct'' or  ``Position adjunct'',
     an included adverbial must be placed before an including adverbial.
\end{pr}

The following sentences exemplify this, taken from  \citeA{Hornby:77}.

\begin{singlespace}
\begin{quote}
 \small
{\bf [End position Time-position adjuncts]} \\
 (\example{ex:order-of-time-position1})
I saw the film \underline{\em on Tuesday evening} \underline{\em last week}. \\
 (\example{ex:order-of-time-position2})
I'll meet you \underline{\em at three o'clock} \underline{\em tomorrow}. \\
 (\example{ex:order-of-time-position3})
We arrived \underline{\em at five o'clock} \underline{\em yesterday afternoon}.
\end{quote}
\end{singlespace}

\begin{singlespace}
\begin{quote}
 \small
{\bf [Position adjuncts]} \\
 (\example{ex:order-of-Position1})
He lives \underline{\em in a small village} \underline{\em in Kent}. \\
 (\example{ex:order-of-Position2})
We spent the holidays \underline{\em in a cottage} 
\underline{\em in the mountains}.
\end{quote}
\end{singlespace}

For example, in (\ref{ex:order-of-time-position1})
{\em on Tuesday evening} is a narrower time span than {\em last week}, 
that is, {\em on Tuesday evening} is included {\em last week}, 
thus {\em on Tuesday evening} comes before {\em last week}.
In (\ref{ex:order-of-Position1}) 
{\em in a small village} is a narrower space than {\em in Kent}, 
that is, {\em a small village} is {\em in Kent}, 
thus {\em in a small village} comes before {\em in Kent}. 

\subsubsection{Ordering between adverbs of different basic positions}

It is possible that two adverbs with different basic positions
(initial, medial, or end position and pre or post position), may be
generated in the same position in a sentence.  When the possible order
is between an initial or end position adverb and a pre or post
position adverb, each adverb modifies a different sentence element, so
the order is determined by the modificants.
But when the decision is between a medial position adverb and a pre position
adverb, they both modify the predicate, and thus have the same
modificant.  In this case, we consider the strength of the connection
between the adverb and the predicate.  A pre position adverb is more
tightly coupled to the predicate than a medial position adverb, so pre
position adverbs will be placed nearer to the predicate than medial
position adverbs.  We express these rules in
principle~\ref{pr:medial-pre}.

\begin{pr}{pr:medial-pre}{}
  Adverbs are grouped with the constituent they modify.  A medial
  position adverb comes before a pre position subjunct when both
  adverbs modify the same predicate.
\end{pr}

In (\ref{ex:i-position-and-pre-position}) {\em probably} modifies the
sentence and {\em even} modifies ``Bob'', the subject of the sentence.
An initial position adverb must come before a sentence subject, and
{\em even} is part of the subject, so {\em probably} comes before  {\em
  even}. (\ref{ex:m-position-and-pre-position}) is an example of
order between a medial position adverb and a pre position adverb.

\begin{singlespace}
\begin{quote}
 \small
 (\example{ex:i-position-and-pre-position}) 
\underline{\em Probably} \underline{\em even} Bob will pass the examination. \\
 {\bf Content Disjunct with degree} $<$ 
              {\bf Pre position Additive Focusing Subjunct}\\
 ({\bf initial position adverb} $<$ {\bf pre position adverb})\\
              (\example{ex:m-position-and-pre-position})
  The price raise is \underline{\em still} \underline{\em too} small. \\
 {\bf Medial position Time-position Adjunct} $<$ 
              {\bf Pre position Amplifier}
  
 \vspace*{-3mm}
\end{quote}
\end{singlespace}

\subsection{Exceptional adverb generating process}

This analysis includes special handling for exceptions to the above
principles.  Figure~\ref{fig:exception} shows the exceptional adverb
generating process that has been implemented in {\bf ALT-J/E}.

\begin{singlespace}
\begin{figure}[b]
    \leavevmode
    \small    
 \begin{description}
  \item [Courtesy Subjunct (1)] \mbox{} \\
        {\em kindly}, {\em please} and {\em graciously} are generated 
        at initial position in an imperative sentence. \\
        (\example{ex:courtesy-subjunct-exception1})
        {\em Please} come in. \\
        (\example{ex:courtesy-subjunct})
        Will you {\em please} show me the way to the station?
  \item [Courtesy Subjunct (2)] \mbox{}\\
        In a to-infinitive phrase with {\em not}, 
        a courtesy subjunct is put before ``to''. \\
        (\example{ex:courtesy-subjunct-exception2})
        I asked him {\em please} {\bf not to} do that.
  \item [End position Adjunct] \mbox{} \\
        If an adjunct pre-modifies 
        a present participle, a past participle, 
        a gerund or an adjective 
        which further pre-modifies something, 
        the adjunct is put just before its modificant\\
        (\example{ex:predicate-modifier-modifier})
        a {\em newly} {\bf discovered} land.
  \item [Conjunct] \mbox{} \\
        If a conjunct connects phrases, it is put between the phrases. \\
        (\example{ex:conjunct-exception})
        It happened {\bf three years later}, {\em that is}, {\bf in 1955}.
 \end{description}
    \caption{Exceptional Processing for English Adverb Generation}
    \label{fig:exception}
\end{figure}
\end{singlespace}

In (\ref{ex:courtesy-subjunct-exception1}) {\em please} is placed in 
the initial position though {\em please} is a ``Courtesy subjunct'', because
the sentence is an imperative sentence. In (\ref{ex:courtesy-subjunct}) 
{\em please} is placed in the medial position.
In (\ref{ex:courtesy-subjunct-exception2}) {\em please} modifies ``do'' 
with ``not'' which is to-infinitive, thus {\em please} is just put before 
``not to do''. 
Sometimes end position adjuncts such as ``Time-position adjunct'' and 
``Manner adjunct'' are not placed in the end position.
In (\ref{ex:predicate-modifier-modifier}), a ``Time-position adjunct'' 
{\em newly} pre-modifies the past participle ``discovered'' which
pre-modifies ``land''.
In (\ref{ex:conjunct-exception}) {\em that is} connects adverbial phrases 
and it shows ``in 1955'' is an apposition of ``three years later'', thus 
{\em that is} is put between them.

Other exceptional processing is needed for interrogative adverbs and 
relative adverbs. Figure~\ref{fig:related-processing} shows 
the exceptional processing which is not directly adverb processing.

\begin{singlespace}
\begin{figure}[t]
    \leavevmode
    \small    
 \begin{description}
  \item [Processing of interrogative] \mbox{} \\
    (\example{ex:interrogative-adverb1})
    {\em Why} didn't you come to our party? ; interrogative adverb \\
    (\example{ex:interrogative-adverb2})
    How {\em well} do you know him?         ; a part of interrogative element
  \item [Processing of relative] \mbox{} \\
    (\example{ex:relative-adverb})
    Monday is the day {\em when} I have the most work to do. ; relative adverb
 \end{description}
    \caption{Processing related to English Adverb Generation}
    \label{fig:related-processing}
\end{figure}
\end{singlespace}

Interrogative adverbs (or interrogative elements) and relative adverbs must 
be put in the initial position of the sentences, such as in
(\ref{ex:interrogative-adverb1}), (\ref{ex:interrogative-adverb2}) and
(\ref{ex:relative-adverb}).

\section{Experimental Results}
\label{sec:results}

The proposed method was tested in three experiments focusing on aspects of
various types of adverbs and a comparison of the proposed method and the
previous method, quantitative coverage, and practical
use. All experiments were made under conditions in which the right English 
adverbs were given, so we tested word order, not word selection.

\subsection{Experiment 1}

The first experiment had the Japanese-to-English machine 
translation system ALT-J/E translate Japanese sentences to test various 
English adverb functions\,\cite{Ikehara:94}. The goal was to confirm 
that this adverb ordering method could handle various types of 
English adverbs.

The experiment considered 200 arbitrary sentences which ALT-J/E was
known to analyze correctly.  This method was compared to a previous
version of ALT-J/E which did not use adverbs' grammatical functions
and meanings but only the basic adverb positions.  The experiment was
performed with sentences out of context.  The result is as follows.

\begin{singlespace}
\begin{table}[ht]
\begin{center}
\caption{Results of Experiment 1}
 \begin{tabular}{|l|c|c|}
 \hline
 Method           &  Accuracy rate &  Correctly ordered sentences \\ \hline
 This Method      &       98\%     &         196/200              \\ \hline
 Previous Method  &       86\%     &         172/200              \\ \hline
 \end{tabular}
\vspace*{-1mm}
\end{center}
 \begin{quote}
  \small
 \hspace*{2cm} Improved Sentences: 27 sentences (13.5\%) \\
 \hspace*{2cm} Sentences changed for the worse: 3 sentences (1.5\%)
 \end{quote}
\end{table}
\end{singlespace}

An accuracy of 98\% in ordering accuracy was achieved compared to 86\%
achieved by the previous method.\footnote{We believe the previous method is 
comparable with the common treatment of adverb ordering in current 
machine translation systems.}

The first improvement came from the fact\hspace{2mm}that\hspace{2mm}the proposed method took
account of the grammatical functions of adverbs.  The previous methods
could not handle subjuncts adequately, but the proposed method can.
The second improvement comes from correctly
ordering the adverbs when two or more appear in the same position.
Of the sentences that changed for the worse, two contain 
verbal idioms.  The problem was in
generating the idiomatic expression.  \vspace*{2mm}

\begin{singlespace}
\begin{tabular}{ll}
 \small
 {\bf This Method      } &  (\example{ex:fail-1-1})
 * I have {\em completely} \underline{got wet}. \\
 {\bf Human Translation} &  (\example{ex:fail-1-1r}) 
 I have \underline{got} {\em completely} \underline{wet}. \\[2mm]
 
 {\bf This Method      } & (\example{ex:fail-1-2})
 * This cloud looks {\em even} like an airplane. \\
 {\bf Human Translation} & (\example{ex:fail-1-2r})
 This cloud {\em even} looks like an airplane.
 
 \vspace*{5mm}
\end{tabular} 
\end{singlespace}

In (\ref{ex:fail-1-1}), the pre position amplifier {\em completely}
modifies ``got wet'', thus the proposed method placed {\em completely}
just before ``got wet''. However, ``get'' is only a light verb, the main 
meaning of ``get wet'' is carried by ``wet'', 
as a result {\em completely} must be put just before ``wet''.
In (\ref{ex:fail-1-2}), the pre position additive focusing subjunct
{\em even} modifies the prepositional phrase ``like an airplane'', but
``look'' and ``like'' are strongly connected, thus {\em even} can not
be placed just before ``like an airplane''. These linguistic phenomena
can be handled correctly by reinforcement of the generation of
idiomatic expressions.

The remaining sentences were sentences whose Japanese originals were
successfully analyzed at a shallow level but were not successfully
analyzed at a deep level; as a result the focusing subjuncts modified
the wrong modificants.  \vspace*{4mm}

\begin{singlespace} 
\begin{tabular}{ll} \small {\bf This
Method      } & (\example{ex:fail-1-3}) * I had {\em exactly} got up
at 6:00. \\ {\bf Human Translation} & (\example{ex:fail-1-3r}) I had
got up {\em exactly} at 6:00.  \vspace*{2mm} 
\vspace*{5mm} \end{tabular} 
\end{singlespace}

\subsection{Experiment 2}

The second experiment was performed to test the quantitative coverage
of the proposed method.\footnote{The case that a Japanese adverb is
  translated into an English adverb is relatively simple for
  Japanese-to-English machine translation.  Thus the result of the
  second experiment shows the immediate effect of implementing our
  method.} It was tested on 1,906 Japanese sentences with at least one
Japanese adverb taken from the ``Dictionary of Basic Japanese Usage
for Foreigners'' \,\cite{ACAJ:90}. The sentences were translated by a
human translator.  We manually examined whether the English adverbs in
the translation (1,035 English adverbs) would be generated correctly
using the proposed method.  The experiment was also performed with
sentences out of context.  Results of experiment 2 are shown in
Figure~\ref{fig:result2}.

\begin{singlespace}
\begin{figure}[htb]
    \leavevmode
    \small    
    {\bf Examined Objects} \\
    \begin{tabular}{lrl}
    Japanese adverb entries:          &       362 & words \\
    sentences:                        &     1,906 & sentences \\
    (sentences with two adverbs:      &         7 & sentences) \\
    examined Japanese adverbs:        &     1,913 & \\
    English adverbs (or adverbials)   &           &            \\
    translated from Japanese adverbs: &     1,053 & 
    \end{tabular} \\
    {\bf Accuracy rate}  97.3\% \\
    {\bf Adverbs generated in incorrect positions:}  28 (2.7\%) \\
    \begin{tabular}{lr}
    absolutely incorrect position: & 12 (1.1\%) \\
    strange position\footnotemark:
                                   & 16 (1.5\%)
    \end{tabular}\\
    {\bf Adverbs generated in suboptimal but acceptable position:} 7 (0.7\%)
    \caption{Results of Experiment 2}
    \vspace*{3mm}
    \label{fig:result2}
\end{figure}
\end{singlespace}

In this experiment 97\% accuracy in the adverb order was achieved and
it was confirmed that the proposed word ordering method can handle a
large amount of adverbs correctly. 

However, there are some adverbs which might be incorrectly placed. 
The adverbs generated\mbox{} in incorrect positions are adverbs which  
have many possible positions for one meaning
and it is difficult to determine a default position for them.
For example, {\em soon} which we classified as a medial position 
time-position adverb also appears at the end position in example sentences. 
{\em Soon} must be placed at the end position, when the sentence 
is a imperative sentence.  

\begin{singlespace}
\begin{quote}
 \small
 (\example{ex:fail-2-1})
 * I have something for you to do, so \underline{please} {\em soon} come. \\
 (\example{ex:fail-2-1r})
 I have something for you to do, so \underline{please} come {\em soon}.
\end{quote}
\end{singlespace}

\footnotetext{This position gives a possible, but unlikely interpretation}

Other cases are when the position of the adverb gives rise to ambiguity.
For example we classified {\em especially} as a pre position 
restrictive focusing subjunct. 
Therefore when {\em especially} modifies ``many shrines 
and temples'', the proposed method placed it before 
``many shrines and temples''. 
This could be interpreted as {\em especially} modifies ``many''.
It would be better in the medial position in this example.

\begin{singlespace}
\begin{quote}
 \small
 (\example{ex:fail2-2})
 * There are many things I want to see while I'm in Japan, \\
 \hspace*{5mm}
 I want to see {\em especially} \underline{many shrines and temples}. \\
 (\example{ex:fail2-2r})
 There are many things I want to see while I'm in Japan, \\
 \hspace*{5mm}
 I {\em especially} want to see \underline{many shrines and temples}.
\end{quote}
\end{singlespace}

We need to investigate under what conditions adverbs should not be 
generated in the default position.
Exceptional processing also needs to be refined, especially for 
special adverbs.

\subsection{Experiment 3}

The third experiment was performed to test the applicability of the
proposed method for practical use. 
The proposed method is applied to 525 English adverbs whilst 
translating 1,000 Japanese sentences from newspaper articles into English.  
The results showed  99\% accuracy in the ordering of the adverbs.  
Detailed results of experiment 3 are shown in Figure~\ref{fig:result3}. 
The accuracy of experiment 3 is higher than the accuracies of experiment 1 
and 2, thus this confirms that the proposed method is effective 
in practical applications.

In Figure~\ref{fig:non-evaluated_adverb}, we show the adverbs which
were not evaluated in experiment 3.
They are generated by using other processing instead of the processing 
described in Section~\ref{sec:method}.
Figure~\ref{fig:Failures-Ex3} shows the sentence in which the proposed method
placed adverbs in the wrong position. Figure~\ref{fig:acceptable-Ex3} shows 
the examples which are acceptable but not in the best position.

\begin{singlespace}
\begin{figure}[htb]
    \leavevmode
    \small    
    {\bf Examined Objects} from newspaper articles on industry and economics\\
 \begin{tabular}{lr}
    examined sentences:                    & 1,000 \\
    adverbs in English sentences:          &   646 \\
    evaluated adverbs:        &   525 \\
    non-evaluated adverbs:    &   121 
 \end{tabular} \\
 \begin{tabular}{lr}
    {\bf Accuracy rate (acceptable position)} & 99.4\% (522/525) \\
    {\bf Accuracy rate (best position)}       & 98.5\% (517/525)
 \end{tabular}
\vspace*{1mm}
    \caption{Results of Experiment 3}
    \label{fig:result3}
\end{figure}
\end{singlespace}

\begin{singlespace}
\begin{figure}[htb]
    \leavevmode
    \small    
    {\bf non-evaluated adverbs} \\
\hspace*{1cm} idiom : 70 \\
\hspace*{2cm}   verb+adverb type verbal idioms ;  {\em pull apart} \\
\hspace*{2cm}   other type idiom ; 
                 {\em not only ... but (also), neither ... or} \\
\hspace*{1cm} appositive structure : 48 \\
\hspace*{2cm}   {\em such as} \\
\hspace*{1cm} comparative structure: 2 \\
\hspace*{2cm}   {\em (as brisk) as usual, (as) much (as possible)} \\
\hspace*{1cm} noun phrase modifier: 2 \\
\hspace*{2cm}   {\em (by the end of June) next year,} \\
\hspace*{2cm}   {\em (two days) off} 
\vspace*{1mm}
\caption{Non-evaluated adverbs in Experiment 3}
\label{fig:non-evaluated_adverb}
\end{figure}
\end{singlespace}

\begin{singlespace}
\begin{figure}[htbp]
    \leavevmode
\begin{description}
 \small
 \item[This Method\ \ \ \ \ \ \ ]
As well as making itself known {\em better}, the company will increase 
its capital supply and actively open business hotels and restaurant chains. 
 \item[Human Translation]
As well as making itself {\em better} known, the company will increase 
its capital supply and actively open business hotels and restaurant chains. 
 \item[] ({\em better}: Amplifier, End position) 
\vspace*{2mm}
\end{description} 
\begin{description}
 \small
 \item[This Method\ \ \ \ \ \ \ ]
Tokyo Nissan Motors, a major dealer of Nissan automobiles, will  
establish a monitoring system, outside company channels {\em actively}, 
to raise the quality of customer service in its stores. 
 \item[Human Translation]
Tokyo Nissan Motors, a major dealer of Nissan automobiles, will 
{\em actively} establish a monitoring system, outside company channels, 
to raise the quality of customer service in its stores. 
 \item [] ({\em actively}:  Manner Adjunct, End position)
\vspace*{1mm}
\end{description} 
\begin{description}
 \small
 \item [This Method\ \ \ \ \ \ \ ]
Fujitsu Laboratory has developed the measuring device to measure 
the delay time between input signal entering and 
the output signal coming {\em precisely}. 
 \item [Human Translation]
Fujitsu Laboratory has developed the measuring device to measure 
{\em precisely} the delay time between input signal entering and 
the output signal coming.
 \item [] ({\em precisely}: Restrictive Focusing Subjunct, Post position)
\vspace*{2mm}
\end{description} 
    \caption{Failures in Experiment 3}
    \label{fig:Failures-Ex3}
\end{figure}
\end{singlespace}

\begin{singlespace} 
\begin{figure}[htbp]
    \leavevmode
\begin{description}
 \small
 \item[This Method\ \ \ \ \ \ \ ]
Olympus will use this opportunity to pursue {\em earnestly} the development of 
information machinery it started at the end of last year. 
 \item[Human Translation]
Olympus will use this opportunity to {\em earnestly} pursue the development of 
information machinery it started at the end of last year. 
 \item[] ({\em earnestly}: Manner Adjunct, End position)
\vspace*{2mm}
\end{description} 
\begin{description}
 \small
 \item[This Method\ \ \ \ \ \ \ ]
The company has been receiving many orders from finance companies, 
but will now start selling {\em actively} to the increasingly 
international manufacturing industry. 
 \item[Human Translation]
The company has been receiving many orders from finance companies, 
but will now start {\em actively} selling to the increasingly 
international manufacturing industry. 
 \item[] ({\em actively}:  Manner Adjunct, End position)
\end{description} 
    \caption{Examples of acceptable position in Experiment 3}
    \label{fig:acceptable-Ex3}
\end{figure}
\end{singlespace}

\section{Conclusion}
\label{sec:conclusion}

\vspace*{-5mm}

A new classification (41 classes) of adverbs based on the grammatical
functions and meanings of adverbs and their basic positions is
proposed for the generation of English adverbs. After examining the actual
use and position of adverbs in English, principles are proposed which
then enable a default order of adverbs to be compiled. 
The proposed method has been tested both by using the ALT-J/E machine 
translation system and also by manually reordering human translations
according to the proposed order, and evaluating the results.

The effectiveness of our English adverb ordering method is shown, in
Japanese-to-English machine translation, based on the classification
of adverbs and the default ordering principles for adverbs.
When correct English adverbs are selected after Japanese analysis and
Japanese-to-English transfer, about 97\% or more of adverbs can be put
in the correct position by this method.  
The accuracy rate of the proposed method improved by 12\% compared with 
the previous adverb ordering method.
In the experiment to confirm the practical use of the proposed method,
it achieved an accuracy of 99\%, which confirms its efficiency with 
real problems.

The proposed method is not restricted to Machine Translation, but can
be used for English generation in general, both for adverbs and
adverbial prepositional phrases.  To achieve a more precise order of
adverbs, the generation of idioms must be improved and the
generation of special adverbs like ``however'', ``already'' and ``still'' 
must be refined.  We must also refine adverb
classification, for example, developing more detailed classes of
conjuncts.  To apply this method to context sensitive text where
marked order may be important, we will also need to consider old/new
information, topic and importance of the information.


\acknowledgment

The authors wish to thank Mr. Mizuno and other members of NTT Software for 
implementing this method into English adverb generation and the members of 
NTT Advanced Technology for data collection. They also thank to the members 
of the MT research group for valuable discussion and Prof. Roland Sussex 
for reading the paper and giving constructive suggestions.

\begin{thebibliography}{}

\bibitem[\protect\BCAY{Asano et~al.}{Asano et~al.}{1990}]{ACAJ:90}
Asano, T.\BBACOMMA\  et~al.\BEDS. \BBOP 1990\BBCP.
\newblock {\Bem Dictionary of Basic Japanese Usage for Foreigners\/} (3rd
  \BEd).
\newblock Japanese Finance Ministry Publishing Agency.
\newblock (in Japanese){Agency for Cultural Affairs of Japan}.

\bibitem[\protect\BCAY{Berckmans}{Berckmans}{1993}]{Berckmans:93}
Berckmans, P. \BBOP 1993\BBCP.
\newblock \BBOQ The Quantifier Theory of Even\BBCQ\
\newblock {\Bem Linguistics and Philosophy}, {\Bem 16}, \BPGS\ 589--611.

\bibitem[\protect\BCAY{Conlon \BBA\ Evans}{Conlon \BBA\
  Evans}{1992}]{Conlon:92}
Conlon, S. P.-N.\BBACOMMA\  \BBA\ Evans, M. \BBOP 1992\BBCP.
\newblock \BBOQ Can Computers Handle Adverbs?\BBCQ\
\newblock In {\Bem 14th International Conference on Computational Linguistics
  (COLING-92)}, \BPGS\ 1192--1196\ Nantes, France.

\bibitem[\protect\BCAY{Ernst}{Ernst}{1984}]{Ernst:84}
Ernst, T.~B. \BBOP 1984\BBCP.
\newblock {\Bem Towards an integrated theory of adverb position in English}.
\newblock Ph.D.\ thesis, Indiana University, Bloomington, Indiana, USA.

\bibitem[\protect\BCAY{Glasbey}{Glasbey}{1993}]{Glasbey:93}
Glasbey, S. \BBOP 1993\BBCP.
\newblock \BBOQ A Computational Treatment of Sentence-Final `then'\BBCQ\
\newblock In {\Bem Sixth Conference of the European Chapter of the Association
  for Computational Linguistics (EACL-93)}, \BPGS\ 158--167\ Utrecht, Holland.

\bibitem[\protect\BCAY{Greenbaum}{Greenbaum}{1983}]{Greenbaum:83}
Greenbaum, S. \BBOP 1983\BBCP.
\newblock {\Bem Studies in English adverbial usage}.
\newblock Kenkyusha Ltd.
\newblock (Japanese Translation).

\bibitem[\protect\BCAY{Horiguchi, Ikeda, \BBA\ Kasuya}{Horiguchi
  et~al.}{1983}]{Horiguchi:83}
Horiguchi, T., Ikeda, S., \BBA\ Kasuya, H.\BEDS. \BBOP 1983\BBCP.
\newblock {\Bem A Handbook of Illustrative Sentences of English --- Idiom
  ---\/} (2nd \BEd).
\newblock Nihon Tosho Lib.
\newblock (in Japanese).

\bibitem[\protect\BCAY{Hornby}{Hornby}{1977}]{Hornby:77}
Hornby, A.~S. \BBOP 1977\BBCP.
\newblock {\Bem Guide to Patterns and Usage in English\/} (2nd \BEd).
\newblock Oxford University Press.
\newblock Japanese translation.

\bibitem[\protect\BCAY{Iida, Ogura, \BBA\ Nomura}{Iida et~al.}{1983}]{Iida:83}
Iida, H., Ogura, K., \BBA\ Nomura, H. \BBOP 1983\BBCP.
\newblock \BBOQ {E}nglish Analysis in Machine Translation System {LUTE}\BBCQ\
\newblock {\Bem IPSJ SIG Notes, NLP}, {\Bem 35\/}(4).
\newblock (in Japanese).

\bibitem[\protect\BCAY{Ikehara, Miyazaki, Shirai, \BBA\ Yokoo}{Ikehara
  et~al.}{1989}]{Ikehara:89}
Ikehara, S., Miyazaki, M., Shirai, S., \BBA\ Yokoo, A. \BBOP 1989\BBCP.
\newblock \BBOQ An Approach to Machine Translation Method Based on
  {C}onstructive {P}rocess {T}heory\BBCQ\
\newblock {\Bem Review of ECL}, {\Bem 37\/}(1), \BPGS\ 34--49.

\bibitem[\protect\BCAY{Ikehara, Shirai, \BBA\ Ogura}{Ikehara
  et~al.}{1994}]{Ikehara:94}
Ikehara, S., Shirai, S., \BBA\ Ogura, K. \BBOP 1994\BBCP.
\newblock \BBOQ Criteria for Evaluating the Linguistic Quality of {J}apanese to
  {E}nglish Machine Translations\BBCQ\
\newblock {\Bem Journal of Japanese Society for Artificial Intelligence}, {\Bem
  9\/}(4), \BPGS\ 569--579.
\newblock (in Japanese).

\bibitem[\protect\BCAY{Ikehara, Shirai, Yokoo, \BBA\ Nakaiwa}{Ikehara
  et~al.}{1991}]{Ikehara:91}
Ikehara, S., Shirai, S., Yokoo, A., \BBA\ Nakaiwa, H. \BBOP 1991\BBCP.
\newblock \BBOQ Toward an {MT} System without Pre-Editing --- {E}ffects of New
  Methods in {ALT-J/E} ---\BBCQ\
\newblock In {\Bem Third Machine Translation Summit: MT Summit III}, \BPGS\
  101--106\ Washinton,D.C., USA.

\bibitem[\protect\BCAY{Kamei, Okumara, \BBA\ Muraki}{Kamei
  et~al.}{1990}]{Kamei:90}
Kamei, S., Okumara, A., \BBA\ Muraki, K. \BBOP 1990\BBCP.
\newblock \BBOQ Syntax of {E}nglish Adverb\BBCQ\
\newblock In {\Bem 43rd Transactions of IPSJ}.
\newblock (in Japanese).

\bibitem[\protect\BCAY{Kojima \BBA\ Takebayashi}{Kojima \BBA\
  Takebayashi}{1990}]{Kojima:90}
Kojima, Y.\BBACOMMA\  \BBA\ Takebayashi, S.\BEDS. \BBOP 1990\BBCP.
\newblock {\Bem Kenkyusha's Lighthouse Japanese-English Dictionary\/} (2nd
  \BEd).
\newblock Kenkyusha.
\newblock (in Japanese).

\bibitem[\protect\BCAY{Kondo \BBA\ Takano}{Kondo \BBA\ Takano}{1986}]{Kondo:86}
Kondo, I.\BBACOMMA\  \BBA\ Takano, F.\BEDS. \BBOP 1986\BBCP.
\newblock {\Bem Shogakukan Progressive Japanese-English Dictionary}.
\newblock Shogakukan.
\newblock (in Japanese).

\bibitem[\protect\BCAY{Konishi}{Konishi}{1989}]{Konishi:89}
Konishi, T.\BED. \BBOP 1989\BBCP.
\newblock {\Bem A Dictionary of English Word Grammar on Adjectives and
  Adverbs}.
\newblock Kenkyusha.
\newblock (in Japanese).

\bibitem[\protect\BCAY{Konishi, Yasui, \BBA\ Kunihiro}{Konishi
  et~al.}{1987}]{Konishi:87}
Konishi, T., Yasui, M., \BBA\ Kunihiro, T.\BEDS. \BBOP 1987\BBCP.
\newblock {\Bem Shogakukan Progressive English-Japanese Dictionary\/} (2nd
  \BEd).
\newblock Shogakukan.
\newblock (in Japanese).

\bibitem[\protect\BCAY{Leech \BBA\ Svartvik}{Leech \BBA\
  Svartvik}{1975}]{Leech:75}
Leech, G.\BBACOMMA\  \BBA\ Svartvik, J. \BBOP 1975\BBCP.
\newblock {\Bem A Communicative Grammar of English}.
\newblock Longman.

\bibitem[\protect\BCAY{Moor}{Moor}{1993}]{Moore:93}
Moor, R.~C. \BBOP 1993\BBCP.
\newblock \BBOQ Events, situations, and adverbs\BBCQ\
\newblock In Bates, M.\BBACOMMA\  \BBA\ Weischedel, R.~M.\BEDS, {\Bem
  Challenges in Natural Language Processing}, \BPGS\ 135--145. Cambridge
  University Press.

\bibitem[\protect\BCAY{Ogura}{Ogura}{1992}]{Ogura:92}
Ogura, K. \BBOP 1992\BBCP.
\newblock \BBOQ Adverb (Phrase) in {E}nglish Semantic Structural
  Representation\BBCQ\
\newblock In {\Bem 43rd Transactions of IPSJ}.
\newblock (in Japanese).

\bibitem[\protect\BCAY{Ogura, Yokoo, Shirai, \BBA\ Ikehara}{Ogura
  et~al.}{1993}]{Ogura:93}
Ogura, K., Yokoo, A., Shirai, S., \BBA\ Ikehara, S. \BBOP 1993\BBCP.
\newblock \BBOQ {J}apanese to {E}nglish Machine Translation and
  Dictionaries\BBCQ\
\newblock In {\Bem 44th Congress of the International Astronautical
  Federation}\ Graz, Austria.

\bibitem[\protect\BCAY{Okada}{Okada}{1985}]{Okada:85}
Okada, N. \BBOP 1985\BBCP.
\newblock {\Bem Adverbs and Parenthetic Sentences}.
\newblock New Selections in English Grammar 9. Taishukan.
\newblock (in Japanese).

\bibitem[\protect\BCAY{Otsuka \BBA\ Nakajima}{Otsuka \BBA\
  Nakajima}{1982}]{Otsuka:82}
Otsuka, T.\BBACOMMA\  \BBA\ Nakajima, F.\BEDS. \BBOP 1982\BBCP.
\newblock {\Bem The Kenkyusha Dictionary of English Linguistics and Philology}.
\newblock Kenkyusha.
\newblock (in Japanese).

\bibitem[\protect\BCAY{Procter}{Procter}{1978}]{LDOCE}
Procter, P.\BED. \BBOP 1978\BBCP.
\newblock {\Bem Longman Dictionary of Contemporary English}.
\newblock Longman.

\bibitem[\protect\BCAY{Quirk, Greenbaum, Leech, \BBA\ Svartvik}{Quirk
  et~al.}{1972}]{Quirk:72}
Quirk, R., Greenbaum, S., Leech, G., \BBA\ Svartvik, J. \BBOP 1972\BBCP.
\newblock {\Bem A Grammar of Contemporary English}.
\newblock Longman, London.

\bibitem[\protect\BCAY{Quirk, Greenbaum, Leech, \BBA\ Svartvik}{Quirk
  et~al.}{1985}]{Quirk:85}
Quirk, R., Greenbaum, S., Leech, G., \BBA\ Svartvik, J. \BBOP 1985\BBCP.
\newblock {\Bem A Comprehensive Grammar of the English Language}.
\newblock Longman, London.

\bibitem[\protect\BCAY{Rivero}{Rivero}{1992}]{Rivero:92}
Rivero, M.-L. \BBOP 1992\BBCP.
\newblock \BBOQ Adverb Incorporation and the Syntax of Adverb in Modern
  {G}reek\BBCQ\
\newblock {\Bem Linguistics and Philosophy}, {\Bem 15}, \BPGS\ 289--331.

\bibitem[\protect\BCAY{Shimazu, Naito, \BBA\ Nomura}{Shimazu
  et~al.}{1983}]{Shimazu:83}
Shimazu, A., Naito, S., \BBA\ Nomura, H. \BBOP 1983\BBCP.
\newblock \BBOQ {J}apanese Language Semantic Analyzer based on an Extended Case
  Frame Model\BBCQ\
\newblock In {\Bem Proceedings of 8th IJCAI}.

\bibitem[\protect\BCAY{Sugiura}{Sugiura}{1991}]{Sugiura:91}
Sugiura, M. \BBOP 1991\BBCP.
\newblock \BBOQ {The Distribution Environment of the Connective ``however'' and
  the Principle of Its Position --- Based on the LOB Corpus ---}\BBCQ\
\newblock {\Bem Journal of Language and Culture Chubu University Junior
  College}, {\Bem 2}, \BPGS\ 47--63.

\bibitem[\protect\BCAY{Tada}{Tada}{1977}]{Tada:77}
Tada, K. \BBOP 1977\BBCP.
\newblock {\Bem A Handbook of English Adverbial Phrase Usage}.
\newblock Taishukan.
\newblock (in Japanese).

\bibitem[\protect\BCAY{Takebayashi \BBA\ Kojima}{Takebayashi \BBA\
  Kojima}{1990}]{Takebayashi:90}
Takebayashi, S.\BBACOMMA\  \BBA\ Kojima, Y.\BEDS. \BBOP 1990\BBCP.
\newblock {\Bem Kenkyusha's Lighthouse English-Japanese Dictionary\/} (2nd
  \BEd).
\newblock Kenkyusha.
\newblock (in Japanese).

\bibitem[\protect\BCAY{van~der Auwera}{van~der Auwera}{1993}]{vanderAuwera:93}
van~der Auwera, J. \BBOP 1993\BBCP.
\newblock \BBOQ {`Already' and `Still': Beyond Duality}\BBCQ\
\newblock {\Bem Linguistics and Philoso-phy}, {\Bem 16}, \BPGS\ 613--653.

\bibitem[\protect\BCAY{Vlach}{Vlach}{1993}]{Vlach:93}
Vlach, F. \BBOP 1993\BBCP.
\newblock \BBOQ Temporal Adverbials, Tenses and the Perfect\BBCQ\
\newblock {\Bem Linguistics and Philosophy}, {\Bem 16}, \BPGS\ 231--283.

\end{thebibliography}

\begin{biography}

\biotitle{}
\bioauthor{Kentaro Ogura}
{Kentaro Ogura received a B.E. and an M.E. in 
administrative engineering from Keio University, in 1978 and 1980 
respectively. In 1980, he joined the NTT Musashino Electrical 
Communication Laboratories. From 1986 to 1990, he worked at ATR 
Interpreting Telephony Research Laboratories. He is now a senior 
research scientist at the NTT Communication Science Laboratories. 
He has been engaged in machine translation research since 1980. 
He was awarded the Dissertation Prize in 1995 from the JSAI.
He is a member of the IPSJ, the IEICE and the JSAI.}

\bioauthor{Francis Bond} {Francis Bond received a B.A. in Japanese and
  mathematics from the University of Queensland in 1988 followed by a
  B.Eng in electrical systems engineering in 1990.  He joined NTT in
  1991 and is currently researching machine translation in the NTT
  Communication Science Laboratories.  He is a member of the IEEE, the
  ANLP and the Australian Linguistic Society.  His current research
  interests include lexical structure,  machine translation,
  and corpus linguistics.} 

\bioauthor{Satoru Ikehara}
{Satoru Ikehara received a B.E., an M.E. and a Dr. Eng.  
from Osaka University in 1967, 1969 and 1983. From 1969 to 1996, 
he was a researcher at the NTT Electrical Communication Laboratories
where he developed a formal algebraic manipulation language, traffic theories 
and several natural language processing systems. 
From 1996, he is a professor of Information and Knowledge Engineering 
at Tottori University and also a visiting professor of Stanford University.
He was awarded the Dissertation Prizes in 1982 and 1993 from the IPSJ, 
in 1995 from JSAI and got the Achievement Award in 1995 from the Information 
Center for Science and Technology.
He is a member of the IPSJ, the IEICE, the JSAI and the AAMT. }

\bioreceived{Received}
\biorevised{Revised}
\bioaccepted{Accepted}

\end{biography}

\end{document}
