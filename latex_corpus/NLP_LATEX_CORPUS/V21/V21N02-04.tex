    \documentclass[japanese]{jnlp_1.4}
\usepackage{jnlpbbl_1.3}
\usepackage[dvips]{graphicx}
\usepackage{amsmath}
\usepackage{udline}
\setulminsep{1.2ex}{0.2ex}
\let\underline


 \usepackage{linguex}
\usepackage{multirow}


\Volume{21}
\Number{2}
\Month{April}
\Year{2014}

\received{2013}{9}{20}
\revised{2013}{12}{3}
\accepted{2014}{1}{17}

\setcounter{page}{213}


\jtitle{多様な文書の書き始めに対する\\意味関係タグ付きコーパスの構築とその分析}
\jauthor{萩行 正嗣\affiref{Author_1} \and 河原 大輔\affiref{Author_1} \and 黒橋 禎夫\affiref{Author_1}}
\jabstract{
現在,自然言語処理では意味解析の本格的な取り組みが始まりつつある.
意味解析の研究には意味関係を付与したコーパスが必要であるが,従来の意味関係のタグ付きコーパスは新聞記事を中心に整備されてきた.
しかし,文書には多様なジャンル,文体が存在し,その中には新聞記事では出現しないような言語現象も出現する.
本研究では,従来のタグ付け基準では扱われてこなかった現象に対して新たなタグ付け基準を設定した.
Webを利用することで多様な文書の書き始めからなる意味関係タグ付きコーパスを構築し,その分析を行った.
}
\jkeywords{タグ付きコーパス,多様な文書,述語項構造,照応関係,文書の著者・読者}

\etitle{Building and Analyzing a Diverse Document Leads Corpus Annotated with Semantic Relations }
\eauthor{Masatsugu Hangyo\affiref{Author_1} \and Daisuke Kawahara\affiref{Author_1} \and Sadao Kurohashi\affiref{Author_1}} 
\eabstract{
Recently, there have been active studies of semantic analysis in the field of natural language processing.
To study semantic analysis, a corpus annotated with semantic relations is required.
Although existing corpora annotated with semantic relations have been restricted to newspaper articles, there are texts of various genres and styles containing linguistic expressions that are missing in newspaper articles.
In this paper, we define annotation criteria for linguistic phenomena which have not been treated using existing criteria.
We have built a diverse  document leads corpus annotated with semantic relations.
We report the statistics of this corpus. 
}
\ekeywords{Annotated Corpus, Diverse Documents, Predicate-Argument Structure, Anaphora Relation, Author and Reader of a Document}

\headauthor{萩行,河原,黒橋}
\headtitle{多様な文書の書き始めに対する意味関係タグ付きコーパスの構築とその分析}

\affilabel{Author_1}{京都大学大学院情報学研究科}{Graduate School of Informatics, Kyoto University}



\begin{document}
\maketitle


\section{はじめに}

現在,自然言語処理では意味解析の本格的な取り組みが始まりつつある.
意味解析には様々なタスクがあるが,その中でも文書中の要素間の関係性を明らかにする述語項構造解析と照応解析は最も基本的かつ重要なタスクである.
本稿ではこの両者をまとめて意味関係解析と呼ぶこととする.
述語項構造解析では用言とそれが取る項の関係を明らかにすることで,表層の係り受けより深い関係を扱う.
照応解析では文章中の表現間の関係を明らかにすることで,係り受け関係にない表現間の関係を扱う.
意味関係解析の研究では,意味関係を人手で付与したタグ付きコーパスが評価およびその分析において必要不可欠といえる.

意味関係およびそのタグ付けを以下の例\ref{意味・談話関係のタグ付け例}で説明する.

\ex.\let\oldalph\let\alph
\label{意味・談話関係のタグ付け例}  今日はソフマップ京都に行きました。\\\label{意味・談話関係のタグ付け例a}
 \hspace*{4ex}
$\left(
\begin{tabular}{@{}l@{}}
行きました$\leftarrow$ガ:[著者],ニ:ソフマップ京都\\
\end{tabular}
\right)$\\
時計を買いたかったのですが、この店舗は扱っていませんでした。\\
 \hspace*{4ex}
$\left(
\begin{tabular}{@{}l@{}}
買いたかった$\leftarrow$ガ:[著者],ヲ:時計\\
店舗$\leftarrow$=:ソフマップ京都\\
扱っていませんでした$\leftarrow$ガ:店舗,ヲ:時計 \label{意味・談話関係のタグ付け例b}
\end{tabular}
\right)$\\
時計を売っているお店をコメントで教えてください。\\
 \hspace*{4ex}
$\left(
\begin{tabular}{@{}l@{}}
 時計$\leftarrow$=:時計\\
売っている$\leftarrow$ガ:お店,ヲ:時計\\
教えてください$\leftarrow$ガ:[読者],ヲ:お店,ニ:[著者]
\label{意味・談話関係のタグ付け例c}
\end{tabular}
\right)$
\global\let\alph

ここでA$\leftarrow${\textit rel}:BはAに{\textit rel}という関係でBというタグを付与することを表す.
{\textit rel}が「ガ」「ヲ」「ニ」などの場合はAが述語項構造の{\textit rel}格の項としてBをとることを表わし,「=」はAがBと照応関係にあることを表す.
また以降の例では議論に関係しないタグについては省略する場合がある.

照応関係とは談話中のある表現(照応詞)が別の表現(照応先)を指す現象である\footnote{照応に類似した概念として共参照が存在する.共参照とは複数の表現が同じ実体を指す現象であるが,照応として表現できるものがほとんどなので,本論文では特に断りがない限り照応として扱う.}.
ここでは,「店舗」に「=:ソフマップ京都」というタグを付与することで,この照応関係を表現している.
述語項構造は述語とその項の関係を表したもので,例\ref{意味・談話関係のタグ付け例b}の「扱っていませんでした」に対してガ格の項が「店舗」,ヲ格の項が「時計」という関係である.
ここで,ヲ格の「時計」は省略されており,一般に{\bf ゼロ照応}と呼ばれる関係にあるが,ゼロ照応も述語項構造の一部として扱う.
またゼロ照応では照応先が文章中に出現しない{\bf 外界ゼロ照応}と呼ばれる現象がある.
例えば,例\ref{意味・談話関係のタグ付け例a}の「行きました」や「買いたかった」のガ格の項はこの文章の著者であるが,この著者を指す表現は文章中には出現しない.
外界の照応先として[著者],[読者],[不特定-人]\footnote{以降,外界の照応先は[ ]で囲う.}などを設定することで,外界ゼロ照応を含めた述語項構造のタグ付けを行う.


これまでの日本語の意味関係解析の研究で主に用いられてきたのは意味関係を付与した新聞記事コーパスであった\cite{KTC,NTC}.
しかし,テキストには新聞記事以外にも百科事典や日記,小説など多様なジャンルがある.
これらの多様なテキストの中には依頼表現,敬語表現など新聞記事ではあまり出現しない言語現象も出現し,意味関係と密接に関係している.
例えば例\ref{意味・談話関係のタグ付け例}の「買いたかった」のガ格が[著者]となることは意志表現に,「教えてください」のガ格が[読者],ニ格が[著者]になることは依頼表現に密接に関係している.
このような言語現象と意味関係の関係を明らかにするためには,多様なテキストからなるタグ付きコーパスの構築とその分析が必要となる.
そこで本研究ではニュース記事,百科事典記事,blog,商用ページなどを含むWebページをタグ付け対象として利用することで,多様なジャンル,文体の文書からなる意味関係タグ付きコーパスの作成を行う.

上述のように,本研究のタグ付け対象には新聞記事ではあまり出現しない言語現象が含まれる.
その中でも特に大きなものとして文章の著者・読者の存在が挙げられる.
著者や読者は,省略されやすい,モダリティや敬語などと密接に関係するなど,他の談話要素とは異なった振る舞いをする.
新聞記事では,客観的事実を報じる内容がほとんどのため,社説を除くと記事の著者や読者が談話中に出現することはほとんどない.
そのため,従来のタグ付け基準では[著者]や[読者]などを外界の照応先として定義していたが,具体的なタグ付け基準についてはあまり議論されてこなかった.
一方,本研究で扱うWebではblog記事や通販ページ,マニュアルなど著者や読者が談話中に出現する文書が多く含まれ,その中には従来のタグ付け基準では想定していなかった言語現象および意味関係が出現する.
そのため,著者・読者が出現する文書でのタグ付け上の問題点を分析し,タグ付け基準を設けることが重要となる.

著者・読者が出現する文書へのタグ付けでの1つ目の問題は,文章中で著者・読者に対応する表現である.

\ex. \underline{僕}は京都に行きたいのですが,\underline{皆さん}のお勧めの場所があったら\underline{教えてください}。\\\label{例:著者・読者表現}
\hspace*{4ex}
$\left(
\begin{tabular}{@{}l@{}}
僕$\leftarrow$=:[著者]\\
皆さん$\leftarrow$=:[読者]\\
教えてください$\leftarrow$ガ:皆さん,ヲ:場所,ニ:僕
\end{tabular}
\right)$

例\ref{例:著者・読者表現}では,「僕」は著者に対応し,「皆さん」は読者に対応した表現となっている.
本研究ではこのような著者や読者に対応する表現を{\bf 著者表現},{\bf 読者表現}と呼ぶこととする.
著者表現,読者表現は外界ゼロ照応における[著者]や[読者]と同様に談話中で特別な振る舞いをする.
例えば例\ref{例:著者・読者表現}の「教えてください」のように,依頼表現の動作主は読者表現に,依頼表現の受け手は著者表現になりやすい.
本研究で扱う文書は多様な著者,読者からなり,著者読者,読者表現も人称代名詞だけでなく,固有表現や役割表現など様々な表現で言及され,語の表層的な情報だけからは簡単に判別できない.
そこで本研究では著者表現,読者表現をタグ付けし,著者・読者の談話中での振る舞いについて調査した.


2つ目の問題は項を明示していない表現に対する述語項構造のタグ付けである.
日本語では一般的な事柄に対して述べる場合には,動作主や受け手などを明示しない表現が用いられることが多い.
従来の新聞記事を対象としたタグ付けでは,[不特定-人]を動作主などとすることでタグ付けを行ってきた.
一方,著者・読者が談話中に出現する場合には,一般的な事項について述べる場合でも動作主などを著者や読者と解釈できる場合が存在する.

\ex. ブログに記事を書き込んで、インターネット上で\underline{公開する}のはとても簡単です。\label{曖昧性}\\
\hspace*{4ex}(公開する$\leftarrow$ガ:[著者] ? [読者] ? [不特定-人],ヲ:記事)

例\ref{曖昧性}の「公開する」の動作主であるガ格は,不特定の人が行える一般論であるが,著者自身の経験とも読者が将来する行為とも解釈することができ,作業者の解釈によりタグ付けに一貫性を欠くこととなる.
本研究ではこのような曖昧性が生じる表現を分類し,タグ付けの基準を設定した.


本研究の目的である多様な文書を含むタグ付きコーパスの構築を行うためには,多数の文書に対してタグ付け作業を行う必要がある.
この際,1文書あたりの作業量が問題となる.
形態素,構文関係のタグ付けは文単位で独立であり,文書が長くなっても作業量は文数に対して線形にしか増加しない.
一方,意味関係のタグ付けでは文をまたぐ関係を扱うため,文書が長くなると作業者が考慮すべき要素が組み合わせ的に増加する.
このため1文書あたりの作業時間が長くなり,文書全体にタグ付けを行うと,タグ付けできる文書数が限られてしまう.
そこで,先頭の数文に限定してタグ付けを行うことで1文書あたりの作業量を抑える.
意味関係解析では既に解析した前方の文の解析結果を利用する場合があり,先頭の解析誤りが後続文の解析に悪影響を与える.
先頭数文に限定したコーパスを作ることで,文書の先頭の解析精度を上げることが期待でき,全体での精度向上にも寄与できると考えられる.



本論文では,2節でコーパスを構成する文書の収集について述べ,3節で一般的な意味関係のタグ付けについて述べる.
4節では著者・読者表現に対するタグ付け,5節では複数の解釈が可能な表現に対するタグ付けについて述べる.
6節でタグ付けされたコーパスの性質について議論し,7節で関連研究について述べ,8節でまとめとする.


\section{タグ付与対象の文書の収集}

従来,意味関係タグ付きコーパスの構築は新聞記事を中心に行われてきた\cite{KTC,NTC}.
しかし,新聞記事にはほとんど出現しない言語現象も存在し,そのような言語現象を研究するためには多様な文書を対象としたコーパスを構築する必要がある.
本研究ではドメインなどを限定せずにWebを利用することで多様な文書を収集する.
多様性を確保するためには,1文書あたりの作業負荷を低くする必要があるので,各文書の先頭3文にタグ付けを限定する.
現在1,000文書のタグ付けが完了している.

タグ付け対象を先頭3文とした理由は,以下の理由による.
本研究では意味関係のうち特にゼロ照応関係を重視している.
ゼロ照応における照応先の位置を京都大学テキストコーパス\cite{KTC}および\cite{sasano-kurohashi:2011:IJCNLP-2011}が実験で使用したWebコーパスについて調査した結果を表\ref{照応先の出現位置}に示す.
この結果から,ゼロ照応関係は1文前までで約70\%,2文前までで約80\%に出現しており,ゼロ照応関係については先頭から3文までを扱うことで,多くの現象を収集できると考えられる.
そのため本研究ではタグ付けする文数を3文とした.

\begin{table}[b]
\caption{照応先の出現位置}
\label{照応先の出現位置}
\input{ca04table01.txt}
\end{table}

本研究では,Webに存在する文書をタグ付け対象とすることで,多様な文書からなるコーパスの構築を目的とするが,Web上に存在する日本語の現象を網羅することやWebに存在する文書の分布を反映することについては重視していない.
これは,以下の2つの問題による.
1つ目の問題は,Web上には意味関係タグの定義および付与が困難な文書が多数存在することである.
本研究では,京都大学テキストコーパスで定義された意味関係とその拡張である著者・読者表現のタグ付けを行う.
京都大学テキストコーパスでは,新聞記事をタグ付け対象としており,そのタグ付け基準は以下のようなテキストを前提としていると言える.
\begin{itemize}
\item 本文のみで内容を理解できる
\item 形態素や文節の単位が認定できる程度に固い文体で記述されている
\item 1文書は1人の著者により記述されている
\end{itemize}
本研究でも同様の基準でタグ付けを行うため,上記の条件を満たす文書のみをタグ付け対象として扱う.
そのため,Webに存在する文書のうち以下のような文書をタグ付けから除くこととなり,そこに含まれる言語現象については扱うことができない.
\begin{description}
\item [イラストや写真などを参照する必要のある文書] 本文のテキストのみだけでは,意味関係を推測できない
\item [AAや顔文字などが含まれる文書] AAや顔文字などはテキストで表現されるが,文をまたぐことや中に言葉が入っていることが多く,範囲の定義が困難
\item [掲示板やチャットなど対話形式の文書] 著者が一貫しないので,著者・読者表現のタグ付けが困難であり,発言者情報や投稿の区切りなどの情報を付与する必要がある
\end{description}
2つ目の問題は,Web文書の真の分布が不明なことである.
Webには誰でも文書をアップロードすることができる一方でクローリングを回避する手段が存在するなど,Web上の文書を網羅的に収集することは困難である.また,網羅的に収集することができたとしても,自動生成されたテキスト,引用・盗用されたテキストの存在などにより,意味関係コーパスとして利用するには不適当なものが大量に含まれると考えられる(404 not foundのページが多数含まれるなど).
これらの問題から,本研究ではタグ付け対象のWebにおける網羅性などを目指すことはせず,タグ付け可能な文書に対して効率よく大量の文書にタグ付けを行うことを目標とした.

上記のようにWebに存在する文書には,コーパスとして利用するには不適切な文書も多数存在している.
これらのうちテキストのみでは内容の理解が困難な文書の定義や扱いについては\ref{意味・談話関係の理解が困難な文書の判定}節にて詳しく述べる.
一方,過度にくだけた文体で記述された文書などテキストの内容からタグ付けが困難な文書の定義や扱いについては\ref{タグ付けに不適切な文書の判定}節にて詳しく述べる.

これらの不適な文書を全て人手で確認し,選別することは非常にコストがかかる.
そのため,まず簡単なルールで自動フィルタリングを行い,その後残った文書を人手で確認しコーパスとして適切な文書についてのみタグ付けの作業を行うこととした.
ルールにおける自動フィルタリングにより除外される文書にはタグ付けに適当なものも多く含まれる.
しかし,Web文書には大量の不適切文書が含まれるため,自動フィルタリングを行わない場合には,作業者が大量の文書の確認を行うことになる.
また,上述のように本研究の目的は偏りなくWeb文書を収集することではなく大量の文書のタグ付けを行うことである.
そこで本研究では人手によるフィルタリングの作業を減らし,タグ付け作業に時間を割くために,自動フィルタリングを行う.


本研究では以下の手順でコーパスの構築を行った.
\begin{enumerate}
\item \cite{Kawahara2006}の手法によりWebからクローリングされたHTMLファイルから日本語文を抽出.
	   \begin{enumerate}
		\item 文字コード情報から日本語のWebページ候補を判定.
		\item 助詞「が」「を」「に」「は」「の」「で」を0.5\%以上含むWebページを日本語Webページと判定.
		\item 句点および$<$br$>$,$<$p$>$タグにより文単位に分割.
		\item ひらがな,カタカナ,漢字の割合が60\%以上の文のみを日本語文として抽出.
	   \end{enumerate}
\item 各ファイルで抽出された最初の日本語文から連続して抽出された日本語文を日本語文書として抽出する.
\item 抽出された日本語文書の1文目が見出しかを自動判定(\ref{意味・談話関係の理解が困難な文書の判定}節で述べる).
	   \begin{description}
		\item[見出しを持つ] 見出しに続く3文をタグ付け対象として抽出.見出しを除いた3文で内容が理解できるかを自動判定.
		\item[見出しを持たない] 先頭から3文をタグ付け対象として抽出.
	   \end{description}
\item 抽出された3文に対してルールによるフィルタリング(\ref{タグ付けに不適切な文書の判定}節で述べる).
\item 人手によるフィルタリング.\label{手順:人手フィルタリング}
\item 人手によるタグ付け.\label{手順:人手タグ}
\end{enumerate}
なお,クローリングの際には日本語のWebページかの判定は行っているが,それ以外のドメインや内容によるフィルタリングは行っていない.
また,実際には(\ref{手順:人手フィルタリング})の人手によるフィルタリングは,タグ付けの際に作業者が不適と判断した文書をタグ付けしないことで行う.



\subsection{テキストのみからは意味関係の理解が困難な文書の判定}
\label{意味・談話関係の理解が困難な文書の判定}

発話や文書などの言語使用はある場・状況において行われ,場・状況は基本的に話者・著者と聴者・読者の間で共有されている.
また,発話や文書の内容は場・状況となんらかの連続性を持っている.
Webページにおいては,どのようなWebサイト内に掲載された文書なのか,またサイト内でどのような位置付けにある文書なのか,などがこれにあたる.

形態素・構文レベルのタグ付きコーパスでは,各文を独立に扱うので,このような場・状況との連続性を考慮する必要はない.
しかし,意味関係コーパスにおいては,この問題を考慮する必要がある.
本研究ではコーパスとしてはテキストだけを扱うため,このような場・状況の情報がなくても意味関係を理解可能な文書のみをコーパスに含める.
例えば,ニュース記事であれば,その文体からニュース記事であることが分かり,多くの場合その記事に記載されている内容はテキストのみから理解することが可能である.
一方で,製品紹介ページ内の「使用上の注意」などのページの場合には,製品自体の知識がない場合には理解することが困難なことが多く,コーパスに含む文書としては不適である.このような文書はタグ付けの前に人手によりコーパスから取り除く.

テキストは見出しを持つ場合があり,その見出しは場・状況との連続性において重要な役割を持つ場合がある.
しかし,見出しは名詞句の連続など通常の文として成立していないものも少なくないため本研究ではタグ付け対象から除く.


本研究では,文書が見出しをもつかどうかを自動的に判定する.
WebにはHTMLタグなどの構造情報があるが,見出しを指定する$<$h$>$タグ以外で見出しが記述される場合があり,一方で$<$h$>$タグでマークアップされていても見出しではない場合もある.
そこでHTMLのタグを用いずテキストの内容から見出しの判定を行う.
1文目が句点で終わっていない場合または体言止めの場合に1文目を見出しと判定し,それ以外の場合には見出しなしとする.
見出しなしの場合には先頭3文をタグ付け対象として抽出し, 1文目が見出しの文書の場合には見出しを除いた後続の3文をタグ付け対象として抽出する.
ただし,見出しを除くと意味関係の理解が困難になると考えられる文書は以下の手順で除去する.


\begin{figure}[b]
\begin{center}
\includegraphics{21-2iaCA4f1.eps}
\end{center}
\caption{見出しの内容語が本文中に出現しない例}
\label{見出しが本文中に出現しない例}
\end{figure}
\begin{figure}[b]
\begin{center}
\includegraphics{21-2iaCA4f2.eps}
\end{center}
\caption{見出しの内容語が先頭3文中に出現する例}
\label{見出しの要素が先頭3文中に出現する例}
\end{figure}
\begin{figure}[b]
\begin{center}
\includegraphics{21-2iaCA4f3.eps}
\end{center}
\caption{見出しの内容語が先頭3文以外に出現する例}
\label{見出しを除くと意味・談話関係の理解が困難になる例}
\end{figure}

図\ref{見出しが本文中に出現しない例}のようにblog記事の見出しが日付けの場合など,見出しの内容が本文の内容にほとんど関係ない場合には,見出しを除いても本文の意味関係の理解に影響を与えないと考えられる.
このように本文に関係ない見出しの場合,見出し中の内容語が以降の文書中に出現しないと考えられる.
見出し中の内容語が文書中に出現する場合でも,先頭3文中に出現する場合には,見出しを除いても先頭3文の意味関係は理解できると考えられる.
図\ref{見出しの要素が先頭3文中に出現する例}の例では1文目が要約の役割を果たしており,見出し中の内容語が全て先頭3文に出現している.
このような場合には見出しを除いても先頭3文の理解は可能であると考えられる.
一方で見出し中の内容語が先頭3文以外に出現した場合には,コーパスとして利用する先頭3文だけで見出しの情報が復元できず,意味関係の理解が困難となると考えられる.
図\ref{見出しを除くと意味・談話関係の理解が困難になる例}の例では見出しに含まれる「売布神社」が6文目に出現している.
しかし先頭3文には「売布神社」は出現せず,先頭3文だけでは「売布神社」に向かうという意味関係の理解が困難である.
そこで見出し中の内容語が先頭3文以外に出現する文書は見出しを除くと先頭3文の意味関係の理解が困難になるとし,自動で除去する.
この後残された文書に対してもタグ付けの際に人手による判定を行い,抽出された3文だけでは意味関係が理解できない場合にはコーパスから除去する.


\subsection{タグ付けに不適切な文書の判定}
\label{タグ付けに不適切な文書の判定}

Webから収集された文書には様々なものがあり,タグ付けを行うには不適切な文書も含まれる.
本研究では以下のいずれかに該当するものはタグ付けが困難であるとして,コーパスに含めない.
\begin{description}
 \item[理解に専門知識を必要とする] 理解に専門的な知識を必要とする文書は作業者が理解できない場合があり,正しいタグ付けが困難である
 \item[文章に意味的連続性がない] 収集された文書には本来は離れた位置にレンダリングされるテキストを連続したテキストとして抽出してしまったものが含まれる.このような文書は文をまたぐ意味関係のタグ付けができない
 \item[過度にくだけた文体で記述されている] 過度にくだけた表現はタグ付けの基本単位となる形態素のタグ付けが困難である
\end{description}

 \begin{table}[b]
\vspace{-0.5\Cvs}
  \caption{ストップフレーズ}
  \label{ストップフレーズの例}
\input{ca04table02.txt}
\end{table}

これらを除くために,まずタグ付け対象となる先頭3文の中に以下の要素を含む文書を自動で除去する.
\begin{itemize}
 \item 体言止めの文:修辞的な文や箇条書きの一部であることが多い
 \item 句点で終わっていない文:テキストの抜き出し誤りであることが多い
 \item 10文節以上ある文:過度にくだけた文体や非文は形態素解析において過分割される場合が多く,文節数も過度に多くなる傾向にある
 \item ローマ字:略語や伏せ字,専門用語であることが多い
 \item 表\ref{ストップフレーズの例}のストップフレーズ:自動生成ページやWeb独特の表現を除くため
\end{itemize}
また,ミラーページや引用ページを除去するために,編集距離が50以下の文書ペアがあった場合には一方を除去する.
この作業において50字以下の文書は全て削除されるが,3文で50字以下のテキストではほとんど意味関係が理解できないため全て削除しても問題ないと考えられる.
自動判定の結果残った不適切な文書はタグ付けの前に人手で除去する.


\section{タグ付け}
\label{タグ付け内容と基準}

\subsection{タグ付け内容}

本コーパスに対して形態素,係り受け関係,固有表現,述語項構造,照応関係のタグ付けを行う.
本研究の焦点は意味関係(述語項構造,照応関係)のタグ付けであるが,そのためにはタグ付け単位の設定などのために形態素,係り受け関係のタグ付けが必要となる.
固有表現は意味関係のタグ付けには必要ないが,意味関係解析の際には重要な手掛かりとなるのでタグ付けを行う.
これらのタグ付けは原則的に京都大学テキストコーパス\cite{KTC}とIREX\footnote{http://nlp.cs.nyu.edu/irex/NE/df990214.txt}の基準に準拠して付与し,一部では基準を変更した.
本節ではこれらの基準のうち,本コーパスにおいて重要となる部分および本コーパスで基準を変更した点について述べる.


述語項構造と照応関係のタグ付けの単位として,京都大学テキストコーパスと同様に,基本句を設定する.
基本句とは自立語1語を核として,前後の付属語を付加した形態素列である.
例\ref{複合語の例}に基本句単位での分割の例を示す.
述語項構造と照応関係の情報は基本句ごとに付与し, 述語項構造の項や照応関係の照応先も基本句とする.
項や照応先が複合語の場合には,その主辞の基本句を照応先とする.
例\ref{複合語の例}では,下線部の「党」の照応先は「国民新党」なので,その主辞の基本句である「新党」を照応先としてタグ付けする.

\ex. 7月/17日、/国民/新党/災害/対策/事務/局長と/して、/\underline{党}を/代表して/現地へ/向かいました。\label{複合語の例}\\
\hspace*{4ex}(党$\leftarrow$=:新党)

述語項構造は基本的に京都大学テキストコーパスと同様の基準で付与する.
格はガ格,ヲ格,ニ格などの表層格と時間,修飾,外の関係などの関係を表す格として定義され,項としては直接係り受け関係にある項,文章内ゼロ照応の項,外界ゼロ照応の項の3種類がある.
直接係り受け関係にある項,文章内ゼロ照応の項については,文章中の基本句から選択する.
外界ゼロ照応では表\ref{外界ゼロ照応の照応先}に示す5種類の照応先の中から選択する.
ここで,「不特定-人」は不特定の人だけでなく,文章中で言及されていない人全てを指す.
述語項構造のタグ付け対象は述語のみでなく,事態性を持つ体言に対してもタグを付与する.


京都大学テキストコーパスでは,二重主語構文に対するタグ付けとしてガ 2 格を設定し,以下の例のようにタグ付けを行っている.

\ex. 彼はビールが\underline{飲みたい}。\\
\hspace*{4ex}(飲みたい$\leftarrow$ガ 2:彼,ガ:ビール)

京都大学テキストコーパスの基準では,例\ref{象}では「象が長い」とは言えないので,「象」は「長い」のガ 2 格と扱わないこととなっている.
「象」は「長い」の主題にあたる役割を持っているが,この基準では述語項構造として「象」と「長い」が関係を持つことを表現できない.
そこで,本コーパスでは主題を表す表現の場合にはガ 2 格とすることとした.

\ex. 象は鼻が\underline{長い}。\label{象}\\
\hspace*{4ex}(長い$\leftarrow$ガ 2:象,ガ:鼻)


\begin{table}[t]
\caption{外界ゼロ照応の照応先の一覧と例}
\label{外界ゼロ照応の照応先}
\input{ca04table03.txt}
\end{table}


照応関係のタグ付けは京都大学テキストコーパスに準拠する.
京都大学テキストコーパスでは,照応関係を「=」(共参照関係),「ノ」(AのBと言い換えられる橋渡し照応),「≒」(それ以外)の3つに分けてタグ付けを行っている.
また,照応関係は体言同士だけでなく述語同士および体言・述語間に対してもタグ付けされている.

京都大学テキストコーパスでは,ある基本句のある格に対して複数の項を付与するために「AND」「OR」「?」の3つのタイプを定義している.
「AND」は「AおよびBが〜」のように付与された項が並列の関係にあり,これらが共に行われる表現に対して利用される.
例\ref{AND例}では「太郎」「花子」が共に「学校に行った」のでこれらを「AND」の関係で付与する.

\ex. 太郎と花子は学校に\underline{行った}。\label{AND例}\\
\hspace*{4ex}(行った$\leftarrow$ガ:太郎 AND 花子)

「OR」は「AまたはBが〜」のように付与され項が並列の関係にあり,どちらかが行われる表現に対して利用される.
例\ref{OR例}では「持っていく」のは「太郎」または「花子」のどちらかであるので「OR」の関係で付与する.

\ex. 太郎か花子が\underline{持っていきます}。\label{OR例}\\
\hspace*{4ex}(持っていきます$\leftarrow$ガ:太郎 OR 花子)

「?」は文脈だけからは,複数の候補から実際の項を特定できない場合に付与される.
例\ref{?例}では,「撤廃する」の主格は「高知県」,「橋本知事」,[不特定:人](高知県議員や職員)のいずれにも解釈できるので,「?」の関係で付与する.

\ex. 高知県の橋本知事は$\cdots$国籍条項を\underline{撤廃する}方針を明らかにした。\label{?例}\\
\hspace*{4ex}(撤廃する$\leftarrow$ガ:高知県 ? 橋本知事 ? 不特定:人)



\section{著者・読者表現}
\label{著者・読者表現}

談話において文書の著者・読者は特別な要素であり
他の談話要素と異なった振舞いをする.
従来の新聞記事コーパスでは,表\ref{外界ゼロ照応の照応先}で示したように文章中に出現しない外界ゼロ照応先として著者や読者などの要素を考慮していた.
しかし,著者・読者は著者・読者表現として文章中に記述される場合がある.

\ex. \underline{私}の担当するお客様に褒めて頂きました。\label{文章内例}\\
\hspace*{4ex}$\left(
\begin{tabular}{@{}l@{}}
褒めて頂きました$\leftarrow$ガ:私,ニ:お客様\\
私$\leftarrow$=:[著者]
\end{tabular}
\right)$

例えば,例\ref{文章内例}では「私」が著者表現として文章中に記述されている.このような場合,従来のコーパスでは他の談話要素と同様の文章内ゼロ照応として扱い,著者や読者として特別には扱ってこなかった.
しかし,文章中での著者や読者の振る舞いを調査するためには,このような文章中に記述された著者・読者表現の振る舞いも調査する必要がある.

本研究では,例\ref{文章内例}の「私」が著者表現であることを共参照としてタグ付けすることとする.

本研究で扱う文書は多様な著者によって多様な読者に向けて記述されており,著者・読者表現は人称代名詞に限らず様々な表現で記述される.
例えば例\ref{こま}の「こま」のように固有名である場合や「主婦」や「母」などのように立場や役職などである場合が存在する.

\ex. \let\oldalph\let\alph
\label{こま} 東京都に住む「お気楽\underline{主婦}」\underline{こま}です。\\
\hspace*{4ex}$\left(
\begin{tabular}{@{}l@{}}
主婦$\leftarrow$=:[著者]\\
こま$\leftarrow$=:主婦\\
\end{tabular}
\right)$\\
0歳と6歳の男の子の\underline{母}をしてます。\\
\hspace*{4ex}$\left(
\begin{tabular}{@{}l@{}}
母$\leftarrow$=:主婦
\end{tabular}
\right)$\global\let\alph

本研究では人称代名詞に限らず,文書の著者・読者に対応する表現全てを著者・読者表現としてタグ付けを行った.

著者・読者表現に対しては外界照応のタグとして「=:[著者]」,「=:[読者]」のタグを付与する.
著者・読者表現が複合語の場合にはその主辞となる基本句に対して付与する.
著者・読者は各文書で1人と仮定し,文書中で「=:[著者]」,「=:[読者]」それぞれ最大でも1基本句にしか付与しないこととする.
共参照関係にあり著者・読者が複数回言及されている場合には,原則として初出となる著者表現に対して付与することとする.
例\ref{こま}では下線部の3つの表現が著者表現だが,「主婦」に対して「=:[著者]」とタグ付けしている.



\subsection{著者表現}

本節では,著者表現を付与する際に問題となる,組織やホームページを指す表現の扱いについて述べる.

企業など組織のホームページでは組織自身が人格や主体性をもっているかのように記述されることが多い.
そのような場合には,実際の著者はホームページの管理者などであると考えられるが,その組織を著者として扱いタグを付与することとする.
例\ref{病院}ではサイト管理者が「神戸徳洲会病院」を代表して記述していると考えられるので,その主辞である「病院」に対し「=:[著者]」を付与する.

\ex. \underline{神戸徳洲会病院}では地域の医療機関との連携を大切にしています。\\\label{病院}
\hspace*{4ex}(病院$\leftarrow$=:[著者])\\
ご来院の際は、是非かかりつけの先生の紹介状をお持ち下さい。\\
紹介状を持参頂いた患者様は、優先的に診察させて頂きます。

また,例\ref{結婚}のようにWebサイト自体を指す表現においても同様に扱う.

\ex. \underline{結婚応援サイト}は、皆さんの素敵な人生のパートナー探しを応援します。\\\label{結婚}
\hspace*{4ex}(サイト$\leftarrow$=:[著者])

 店舗のページなどでは店舗を表す表現と店長や店員を表す表現が共に出現する場合がある.
このような場合には,店舗と店長・店員のどちらが著者的に振る舞っているかを判断してタグ付けを行う.
例\ref{店員}では店舗が著者的なので「スタッフ」ではなく「館」に「=:著者」を付与する.

\ex. \underline{タウンロフト館}の店舗情報をお伝えします。\\\label{店員}
\hspace*{4ex}(館$\leftarrow$=:[著者])\\
ご来店予定の際にアクセスでお困りでしたら、\underline{当店スタッフ}までお気軽にご連絡下さい。\\
\hspace*{4ex}$\left(
\begin{tabular}{@{}l@{}}
当店$\leftarrow$=:館\\
スタッフ$\leftarrow$ノ:当店
\end{tabular}
\right)$

一方,例\ref{店長}では,店長である「かおりん」が著者として店舗を紹介しているので「かおりん」に対して「=:著者」を付与する.

\ex. 『ソブレ』アマゾン店,店長の\underline{かおりん}です。\\\label{店長}
\hspace*{4ex}(かおりん$\leftarrow$=:[著者])\\
新商品の情報や、かおりん日記を相棒☆みかんと一緒に紹介します。


\subsection{読者表現}

本コーパスで扱う文書はWebから収集されたものであり,不特定多数の人間が閲覧できる状態である.
そのため厳密に常に読者を指す表現といえるのは二人称代名詞のみといえる.
例\ref{皆さん}では,「皆さん」は二人称代名詞の敬語表現であり,読者表現としてタグ付けを行う.

\ex. \underline{皆さん}は初詣はどこに行かれたでしょうか?\\\label{皆さん}
\hspace*{4ex}(皆さん$\leftarrow$=:[読者])

一方,不特定の人が閲覧できる状態であっても,多くの文書では著者が主な読者として想定する対象が存在する. 
本研究ではそのような対象を指す表現も読者表現にあたると定義してタグ付けを行う.
例\ref{ぽすれん}は「ぽすれん登録会員」に対するガイドラインであるので,「ぽすれん登録会員」を読者表現とし,その主辞である「会員」に「=:[読者]」を付与する.

\ex. \underline{ぽすれん登録会員}がコミュニティサービスをご利用いただくには、本ガイドラインの内容を承諾いただくことが条件となります。\\\label{ぽすれん}
\hspace*{4ex}(会員$\leftarrow$=:[読者])

一方,例\ref{方}では「写真を撮られた方」は著者にとって想定している読者のうちの一部であり,読者全体を想定した表現ではないので「方」は読者表現としては扱わない.

\ex. 桜の下で写真を撮られた\underline{方}も多いのではないでしょうか。\\\label{方}





\section{複数の解釈が可能な表現に対するタグ付け}
\label{複数付与基準}

日本語では用言の動作主や受け手にあたる格要素が明示されない表現が用いられることがある.
京都大学テキストコーパスでは明示されていない格要素の候補が文章中に出現する場合には\ref{タグ付け内容と基準}節で説明した「?」による複数付与によってタグ付けを行っている.
また,候補が文章中の表現にないような場合でも,京都大学テキストコーパスが対象とする新聞記事では[不特定-人]を格要素としてタグ付けすればよい場合がほとんどである.
一方,Webテキストのように著者・読者が談話構造中に出現する場合には,この明示されていない格要素を[不特定-人]だけでなく[著者]や[読者]としても解釈できる場合が多くある.


本研究では,複数の解釈ができる場合には「?」の関係で解釈可能な全ての項を付与することとする.
複数解釈可能な典型的表現についてはマニュアルを作成し,作業者に例示を行った.
例示した内容は付録\ref{付録:複数付与基準}に示した.
本節では,[著者],[読者]および[不特定-人]を格要素として解釈する際の基準について説明する.
なお,以降の例では[著者],[読者]および[不特定-人]を例として紹介するが,[著者],[読者]については\ref{著者・読者表現}節で述べた著者表現,読者表現も同様に扱うものとする.


\subsection{[不特定-人]を付与する基準}

行為が一般論といえる場合,著者・読者以外で文章中で言及されていない人を指す場合には[不特定-人]を付与する.

例\ref{不特定-焙煎}では,一般論と言えるので動作主にあたるガ格に[不特定-人]を付与する.

\ex. コーヒー生豆とは\underline{焙煎する}前の裸の状態の豆をいい、グリーンコーヒーとも呼ばれています。\label{不特定-焙煎}\\
\hspace*{4ex}(焙煎する$\leftarrow$ガ:[不特定-人],ヲ:豆)

例\ref{不特定-2}では,文章中で言及されていないメールマガジンの会員が受け手と言えるのでニ格に[不特定-人]を付与する.
この例では「是非ご登録ください」と書かれていることから,読者はまだメールマガジンの会員でないと考えられるので,[読者]は付与しない.

\ex. メールマガジンではお得な情報を\underline{お送りしています}。是非ご登録ください。\label{不特定-2}\\
\hspace*{4ex}(お送りしています$\leftarrow$ガ:[著者],ニ:[不特定-人])


\subsection{[著者]を付与する基準}

著者自身が実行したことがある,著者自身にもあてはまると解釈できる場合には[著者]を付与する.

例\ref{著者-2}では,一般論といえるが,著者(鉄道会社)にもあてはまると解釈できるのでガ格に[不特定-人]に加えて[著者]も付与する.

\ex. 線路は列車の安全を確保し、快適な乗り心地を維持する状態に\underline{整備しておかなければなりません}。\label{著者-2}\\
\hspace*{4ex}(整備しておかねばなりません$\leftarrow$ガ:[著者] ? [不特定-人],ヲ:線路,ニ:状態)

例\ref{田楽}では,一般論とも言えるが,著者自身が「源流を辿った」経験があるとも解釈できるのでガ格に「[著者] ? [不特定-人]」を付与する.

\ex.しかし名前からも察することができるように、源流を\underline{辿れば}「田楽」に行き当たる。\label{田楽}\\
\hspace*{4ex}(辿れば$\leftarrow$ガ:[著者] ? [不特定-人],ヲ:源流)


\subsection{[読者]を付与する基準}

依頼表現など読者に働きかけをする表現,読者に対して何かを勧めている表現の場合には[読者]を付与する.
何かを勧める表現の場合,対象となる用言だけでなく,周辺の文脈も含めて判断する.

例\ref{読者-1}では,読者に対して依頼しているのでガ格に[読者]を付与する.

\ex. メールの際は必ず名前を\underline{添えてください}。\label{読者-1}\\
\hspace*{4ex}(添えてください$\leftarrow$ガ:[読者])

例\ref{読者-2}は通販サイト内の文である.ここで,「選択できます」自体は一般論と言えるが,ページ全体として読者に通販の利用を勧めていると解釈できるので,ガ格に[読者]および[不特定-人]を付与する.

\ex. 分割払いなど、多彩なお支払い方法から\underline{選択できます}。詳しくはガイドをご参照ください。\label{読者-2}\\
\hspace*{4ex}(選択できます$\leftarrow$ガ:[読者] ? [不特定-人])

例\ref{著者-1}では,読者に勧めていると解釈できるのでガ格に[読者]を付与している.
また,一般論とも著者自身の経験とも解釈できるので[著者]および[不特定-人]も付与している.

\ex. ブログに記事を書き込んで、インターネット上で\underline{公開する}のはとても簡単です。\label{著者-1}\\
\hspace*{4ex}(公開する$\leftarrow$ガ:[著者] ? [読者] ? [不特定-人],ヲ:記事)

例\ref{読者-3}では著者が読者を勧誘する表現になっているので[読者]を付与する.
Webサイトを通してのやりとりであるが,説明の過程で著者も同時に見ていると仮定して「AND」で付与する.

\ex. まずは株式市場の分類を\underline{見てみましょう}。\label{読者-3}\\
\hspace*{4ex}(見てみましょう$\leftarrow$ガ:[著者] AND [読者])



\section{作成されたコーパス}

現在までに,3人の作業者により1,000文書のタグ付け作業が終了している.
本節ではコーパスを作成した手順について説明し,その後作成されたコーパスの統計量およびその性質について議論を行う.

作成されたコーパスの統計量およびその性質についての議論では,まず,コーパスの基本的な統計と文体などの性質について議論する.
次に,著者・読者の談話への出現とその振る舞いについて議論する.
これらの議論において必要に応じて新聞記事コーパスである京都大学テキストコーパスとの比較を行う.
最後に作業者間でのタグ付けの一致度について議論する.


\subsection{タグ付け作業の手順および環境}

タグ付け作業の際にはまず形態素解析器JUMAN ver.6.0\footnote{http://nlp.ist.i.kyoto-u.ac.jp/index.php?JUMAN},構文解析器KNP ver.3.01\footnote{http://nlp.ist.i.kyoto-u.ac.jp/index.php?KNP}のデフォルト設定により自動でタグ付けを行い,その後GUIのツールを利用してタグの付与および自動付与されたタグの修正を行った.各文書に対して一人の作業者が作業した後に別の作業者が内容の確認・修正を行った.
タグ付けの際には作業者に与えられた情報は,タグ付け対象となる3文のテキストおよびそのテキストがWeb上から収集されたという情報だけである.

作業者は3名であり,全員がコーパスへのタグ付け作業の経験者である.作業開始前に京都大学コーパスのマニュアル\footnote{http://nlp.ist.i.kyoto-u.ac.jp/nl-resource/corpus/KyotoCorpus4.0/doc/syn\_guideline.pdf および http://\linebreak[2]nlp.\linebreak[2]ist.\linebreak[2]i.\linebreak[2]kyoto-u.\linebreak[2]ac.jp/\linebreak[2]nl-resource/\linebreak[2]corpus/\linebreak[2]KyotoCorpus4.0/\linebreak[2]doc/\linebreak[2]rel\_guideline.pdf},著者・読者表現の定義および例を配布した.事前作業として,3人が同一の50文書に対してタグ付けを行い,特に著者・読者表現に対してのタグ付けの疑問点の確認および基準の修正を行った.
その後,1,000記事に対してタグ付け作業を行ったところ,\ref{複数付与基準}節で述べた複数解釈可能な表現が問題となることが分かった.そこで,作業者を交えてタグ付け基準について検討し,その結果を付録\ref{付録:複数付与基準}として配布した.
新たな基準に基づいて上記1,000記事の修正作業を行った.
現在は5,000記事を目標として作業を進行中である.
作業中においても,タグ付けの疑問点等については,筆者らと相談のうえで作業を進めている.

作成されたコーパスには文書情報として文書を取得したURLを付与する予定である.なお,コーパスにタグ付けされた意味関係はテキストのみに基いており,意味関係コーパスとしてはURL情報は必須的なものではない.


\subsection{コーパスの統計量}

作成されたコーパス記事の統計を表\ref{コーパスの統計}に示す.比較のため京都大学テキストコーパスの統計も合わせて示した.本コーパスでは1文あたりの形態素数が約17個であり,京都大学テキストコーパスの約26個と比較して1文あたりの形態素数が少ない傾向にある.
本コーパスでは意味関係のタグ付け対象である基本句のうち約2/3に対して,何らかの意味関係が付与された.

\begin{table}[b]
\caption{コーパスの統計}
\label{コーパスの統計}
\input{ca04table04.txt}
\end{table}
\begin{table}[b]
\caption{モダリティが出現する文の割合}
\label{モダリティ}
\input{ca04table05.txt}
\end{table}

文体の差異を調査するために,両コーパスにおいてモダリティ,敬語表現を含む文の割合を表\ref{モダリティ},表\ref{敬語}に示す.なお,モダリティ,敬語表現はKNPにより自動で付与されたものである.また,「全て」はいずれかのモダリティ,敬語表現が含まれた文の割合を示す.
表\ref{モダリティ}から本コーパスには依頼,勧誘,命令,意志など著者から読者への働きかけを持つモダリティが多く含まれる.意志のモダリティは京都大学テキストコーパスにも多く含まれているが,これは発言の引用内での使用が多かったためである.
逆に,京都大学テキストコーパスでは評価:強や認識-証拠性などが多く含まれている\footnote{評価:強の例としては「関係を無視した暴言と\underline{言わざるを得ない}。」,認識-証拠性の例としては「海部政権誕生の願望が\underline{込められているようだ}。」がある.}.これらのモダリティは報道記事や社説で広く使われる表現であり,本コーパスとの文体の差を示していると言える.
表\ref{敬語}から,本コーパスでは80\%近い文で何らかの敬語表現が使用されていることが分かる.尊敬表現,謙譲表現も高い割合で使用されており,本コーパスでは読者の存在を意識した文書が多く含まれると考えられる.


\begin{table}[b]
\caption{敬語が出現する文の割合}
\label{敬語}
\input{ca04table06.txt}
\end{table}
\begin{table}[b]
\caption{人手による記事タイプの分類}
\label{記事分類}
\input{ca04table07.txt}
\vspace{-1\Cvs}
\end{table}


本コーパスではドメインなどを限定せずに文書をWebから収集したため多様な文書が含まれている.
タグ付けされた文書の傾向を調べるため,タグ付けされた文書を人手で13種類に分類した.
その分類結果が表\ref{記事分類}である.
表\ref{記事分類}から企業・店舗ページ,ブログ・個人ページ,辞典・解説記事を中心に多様な文書がタグ付けされたことが分かる.
さらに,同じ企業・店舗ページであっても,企業のページだけでなく,学校や公共機関,地方自治体のページなど様々なページから収集された文書が含まれている.
また,タグ付けされた文書の中には企業ページ内の広報用blogのような,一意にジャンル分けすることが難しいものも存在した\footnote{今回は企業・店舗ページに分類した.}.



\subsection{著者・読者表現}

タグ付けされたコーパスにおける著者,読者の文書ごとの出現数を表\ref{ドキュメントごとの一人称・二人称の出現}に示す.
 「出現あり」のうち「表現あり」は文書中に著者・読者表現としてタグ付けされた表現のある文書の数を表す.
「表現なし」は著者・読者表現はないが外界ゼロ照応の照応先として出現している文書の数を表す.
著者の場合は約7割,読者の場合は約5割の文書において談話に出現することが分かる.
また,著者,読者ともに多くの文書において,外界ゼロ照応の照応先としてのみ出現することが分かる.

\begin{table}[b]
\caption{文書ごとの著者・読者の出現}
\label{ドキュメントごとの一人称・二人称の出現}
\begin{center}
\input{ca04table08.txt}
\end{table}
\begin{table}[b]
\begin{minipage}{0.45\hsize}
\input{ca04table09.txt}
\end{minipage}
\begin{minipage}{0.45\hsize}
\input{ca04table10.txt}
\end{minipage}
\vspace{-1\Cvs}
\end{table}

著者・読者表現として使われた語を主辞のJUMAN代表表記により調査した結果,著者表現は145種類,読者表現は25種類の表現が存在した.
その例と出現回数を表\ref{著者表現の例}と表\ref{読者表現の例}に示す\footnote{代表表記は「皆様/みなさま」のような形式で表現されるが,表では「皆様」にあたる部分のみを表示している.}.
なお,ここでは著者・読者表現と共参照関係にある表現も著者・読者表現として扱った\footnote{例\ref{こま}であれば,「主婦」「こま」「母」を全て著者表現とした.}.
著者表現では,「私」が56回と一番多く使われている.これはブログ記事において特に多く使用されていた.
「私」や「僕」などのブログで使われると思われる表現では,「わたし」「あたし」,「ぼく」「ボク」などの若干くだけた表記も用いられていた.
しかし,「私」の場合「私」が56回中53回,「僕」では「僕」が11回中7回と,多くは漢字での表記が用いられていた.
また,「弊社」「当社」など企業が自社を表す表現も多く見られた.
「管理人」「主婦」「監督」などの立場を表す表現や「協会」「病院」などの組織を表す表現,「ローソン」「真理子」など固有名など多様な表現で出現することが分かる.
またコーパス全体で1度しか出現しなかった表現が106表現,2度しか出現しなかった表現が24表現と,文書固有の著者表現も多かった.
読者表現では二人称代名詞の敬語表現である「皆様」や「皆さん」が多く出現した.
これはWebページで読者を想定するのは企業ページの商品販売サイトが多いため,読者に対して敬語を用いることが多いためである.
これらの表現でも「皆様」であれば「みなさま」「皆さま」,「皆さん」では「みなさん」などの異表記も用いられていた.
これらの表現では,上述の一人称代名詞と異なり,「皆様」が26回中17回,「皆さん」が7回中4回であり,比較的様々な表記が用いられていた.
また「客」や「会員」など企業ページで想定される読者を指す表現も多く見られた.
「生徒」「ドライバー」「市民」など文書特有の読者を想定する表現も見られる.
著者,読者両方の表現で用いられるものとしては「自分」が見られた.


\subsection{ゼロ照応関係}

\begin{table}[b]
\caption{本コーパスにおけるゼロ照応の個数}
\label{ゼロ照応の個数}
\input{ca04table11.txt}
\end{table}

タグ付けされたゼロ照応の個数を表\ref{ゼロ照応の個数}に示す.
また,文章内ゼロ照応の照応先の内訳を表\ref{本コーパスの文章内ゼロ照応の内訳}に外界ゼロ照応の照応先の内訳を表\ref{本コーパスの外界ゼロ照応の内訳}に示す.
なお,表\ref{本コーパスの文章内ゼロ照応の内訳}で著者,読者とは,ゼロ代名詞の照応先が著者,読者表現または著者,読者表現と共参照関係であることを表す\footnote{例\ref{こま}であれば,「主婦」「こま」「母」が照応詞になる場合に著者に分類される.}.
表\ref{ゼロ照応の個数}から特にガ格においてゼロ照応が多いことが分かる.
また,ガ格,ニ格,ガ 2 格において外界ゼロ照応の割り合いが高いことが分かる.
表\ref{本コーパスの文章内ゼロ照応の内訳}と表\ref{本コーパスの外界ゼロ照応の内訳}から他の格に比べてガ格,ガ 2 格において著者が照応先になる割合が高いことが分かる.このことから用言の動作主が著者であることが多いことが分かる.
一方,ニ格は他の格に比べて読者が照応先となることが多い.これは「[著者]ガ [読者]ニ お勧めする」や「[著者]ガ [読者]ニ 販売しています」といった,著者が読者に何らかの働きかけをする表現が多いためと考えられる.

\begin{table}[b]
\caption{本コーパスの文章内ゼロ照応の内訳}
\label{本コーパスの文章内ゼロ照応の内訳}
\input{ca04table12.txt}
\end{table}
\begin{table}[b]
\caption{本コーパスの外界ゼロ照応の内訳}
\label{本コーパスの外界ゼロ照応の内訳}
\input{ca04table13.txt}
\end{table}
\begin{table}[b]
\caption{京都大学テキストコーパスにおけるゼロ照応の個数}
\label{KTCのゼロ照応の個数}
\input{ca04table14.txt}
\end{table}

比較のために京都大学テキストコーパスにおけるゼロ照応の個数を表\ref{KTCのゼロ照応の個数}に,外界ゼロ照応の内訳を表\ref{KTCの外界ゼロ照応の内訳}に示す.京都大学テキストコーパスには著者・読者表現が付与されていないので,文章内照応の内訳は調査できなかった.
この比較から,本コーパスでは外界ゼロ照応の割合が京都大学テキストコーパスに比べて非常に高いことが分かる.特にガ格,ニ格,ガ 2 格においてその傾向が顕著である.これらの格では外界ゼロ照応の照応先を比較すると,本コーパスにおいて[著者]や[読者]が多いが,京都大学テキストコーパスではほとんどない.
新聞記事では文書の著者や読者が談話に登場することはほとんどないが,Web文書では頻繁に登場する.
この違いがゼロ照応の照応先としても表れているといえる.

\begin{table}[t]
\caption{京都大学テキストコーパスにおける外界ゼロ照応の内訳}
\label{KTCの外界ゼロ照応の内訳}
\input{ca04table15.txt}
\end{table}
\begin{table}[t]
\caption{複数付与のタグが付与された関係数}
\label{複数付与数}
\input{ca04table16.txt}
\end{table}

\ref{複数付与基準}節で示した複数の解釈が可能な表現に対するタグ付けを調査するために,[著者],[読者],[不特定-人]のいずれかが付与された項とそのうち複数が付与された項の数を表\ref{複数付与数}に示す.
表\ref{複数付与数}より,[著者],[読者],[不特定-人]のいずれかが付与された項のうち約13\%が複数の解釈が可能であることが分かる.また,[不特定-人]が付与されたもののうち約半数において複数の解釈が可能となっている.


\subsection{作業者間一致度}

著者・読者表現および述語項構造のタグ付けの一致度を調査するために,3人の作業者が100記事に対してタグ付けを行った.
これらのタグ付けに必要な形態素,構文関係および共参照関係については3人の作業者の相互確認のうえであらかじめタグ付けを行い,その後独立に著者・読者表現および述語項構造のタグ付けを行った.

著者・読者表現の一致度は文書単位で一方の作業者を正解とした場合のF1スコアにより求めた.
その結果を表\ref{著者・読者表現一致度}に示す.

著者・読者表現が一致しなかったものを確認したところ,ほとんどの事例では作業者による作業ミスと考えられるものであった.実際の作業では全ての文書に対して異なる作業者による確認作業を行っているので,このようなものは取り除かれると考えられる.

\begin{table}[t]
\caption{著者・読者表現一致度}
\label{著者・読者表現一致度}
\input{ca04table17.txt}
\end{table}

一方,作業者の判断のゆれが原因と考えられるものとしては例\ref{スタッフサービス}があった.
この文書では一人の作業者のみが「スタッフサービス」が著者表現と判断し,他の二人は著者表現なしと判断した.「スタッフサービス」が著者表現とした作業者は,「スタッフサービス」が人材派遣サービスの企業名だと判断し,著者表現なしと判断した作業者は派遣業一般の言い換えと判断したと考えられる.このような一般的な名詞とも著者表現ともとれる表現は3文の文脈のみからは判断が難しいことが分かる.

\ex. \underline{スタッフサービス}には一般事務だけではなく、医療機関専門に派遣されるスタッフサービスメディカルもあります。\label{スタッフサービス}

また,タグ付け作業時に基準を設定しなかったことによるずれとしては例\ref{猫}があった.
この文書では「私」がモニターの上で過ごすと書かれていることから,猫などを擬人的に扱ったブログであると考えられる.
実際の著者は飼い主であると考えられ,このような場合に著者表現をどのように扱うかを定義していなかったために,作業者間で「私」を著者として扱うかの判断が分かれた.

\ex. \label{猫}
台風が通り過ぎるたびに寒くなっていきますね。
\underline{私}は暖かい場所を求めて会社の中を彷徨います。
今日はこのモニターの上で過ごすことにしましょう。

同様に,一人称視点の小説などで主人公を表す表現でも同様の問題が起こると考えられ,今後は実際の著者以外の人物が著者的に振る舞う場合の著者表現について定義する必要がある.

述語項構造の一致度は格ごとに以下の式で計算した.
{\allowdisplaybreaks
\begin{align*}
F1(B;A,\mathit{rel}) &= \frac{2\times \mathit{Recall}(B;A,rel)\times \mathit{Presicion}(B;,rel)}{\mathit{Recall}(B;A,rel) + \mathit{Presicion}(B; A,rel)}\\[1ex]
\mathit{Recall}(B;A,\mathit{rel}) & =  \frac{\displaystyle\sum^{}_{\mathit{pred}\in \textit{anno-pred}(A,rel)} \frac{|\mathit{anno}(A,\mathit{rel},\mathit{pred})\bigcap \mathit{anno}(B,\mathit{rel},\mathit{pred})|}{|\mathit{anno}(A,\mathit{rel},\mathit{pred})|}}{|\textit{anno-pred}(A,\mathit{rel})|}\\[1ex]
\mathit{Precision}(B;A,\mathit{rel}) & =  \frac{\displaystyle\sum^{}_{\mathit{pred}\in \textit{anno-pred}(B,\mathit{rel})} \frac{|\mathit{anno}(A,\mathit{rel},\mathit{pred})\bigcap \mathit{anno}(B,\mathit{rel},\mathit{pred})|}{|\mathit{anno}(B,\mathit{rel},\mathit{pred})|}}{|\textit{anno-pred}(B,\mathit{rel})|}
\end{align*}
}
ここで,$\textit{anno-pred}(A,\mathit{rel})$は作業者Aが \textit{rel}(e.g., ガ,ヲ,ニ…)という格を付与した基本句の集合を表し,$\mathit{anno}(A,\mathit{rel},\mathit{pred})$は作業者Aが基本句 \textit{pred} に \textit{rel} の格で付与した項の集合とする.
$\mathit{anno}(A,\mathit{rel},\mathit{pred})$が複数の項からなる集合の場合,本来「AND」「OR」「?」の関係を持つが,一致度の調査では考慮しなかった.
なお,$\mathit{Recall}(B;A,\mathit{rel})$,$\mathit{Precision}(B;A,\mathit{rel})$は精度と再現率の用言ごとのマクロ平均と言える.

\begin{table}[b]
\caption{用言の述語項構造の一致度}
\label{用言一致度}
\input{ca04table18.txt}
\end{table}
\begin{table}[b]
\caption{動作性体言の述語項構造の一致度}
\label{体言一致度}
\input{ca04table19.txt}
\end{table}

表\ref{用言一致度}と表\ref{体言一致度}に用言と動作性を持つ体言に対するタグ付けの一致度の平均を示す.
全体として係り受け関係にある項で一致度が高い傾向にある.
特に用言のガ格,ヲ格,ニ格で高い傾向にあるが,これらの格の場合には助詞として格が明示されていることが多いためである.
文章内ゼロ照応と外界ゼロ照応の項では格によって差はあるがおおむね似たような一致度であり,その一致度は係り受けのある項よりも低い.
また用言の一致度に比べ動作性体言の一致度が低い傾向にあることが分かる.

用言において作業者間のタグ付けが一致しないものでは,用言が取る格が一致していないものが多くあった.
このようなものは大きく分けて3種類に分類することができる.
一つ目は項は同じものを付与しているが,付与する格が異なるものである.
例\ref{春雨}では「春雨がくせがない」「くせが春雨にない」と2通りの表現が可能なため,作業者によってタグ付けが分かれた.

\ex. くせの\underline{ない}春雨は、サラダ・和えもの・炒めもの・鍋物と様々な料理に使えます。\label{春雨}
\a. (ない$\leftarrow$ガ:くせ,ニ:春雨)\label{春雨a}
\b. (ない$\leftarrow$ガ 2:春雨,ガ:くせ)

このようなずれはガ 2 格で多く見られたが,例\ref{降る}のようなニ格とデ格のずれなどでも見られた.

\ex. 唐松岳に行くつもりだったが、ライブカメラで現地の様子を確認すると、もう雨が\underline{降っている}。\label{降る}
\a. (降っている$\leftarrow$ガ:雨,ニ:唐松岳)\label{降るa}
\b. (降っている$\leftarrow$ガ:雨,デ:唐松岳)

このようなずれがあった場合,例\ref{春雨}のようなガ 2 格に関するずれの場合には,ガ 2 格以外を優先することとし,例\ref{春雨}では\ref{春雨a}とタグ付けした.それ以外の場合には,どちらも間違いとは言えない場合には,どちらがより自然な表現かを作業者間で多数決を行うこととし,例\ref{降る}では\ref{降るa}をタグ付けした.


二つ目は用言の解釈が分かれたものである.
例\ref{イメージ}の「イメージさせる」では,他動詞として考えるとニ格として[不特定-人]をとる.しかし,ニ格をとらずに「色合いが海をイメージさせる」として,「色合い」の性質を表す表現としても解釈できる.
そのため,作業者間ではニ格に[不特定-人]を付与するか何も付与しないかで判断が分かれた.

\ex. 床板には深い海を\underline{イメージさせる}色合いのガラスを落とし込んでおります。\label{イメージ}
\a. (イメージさせる$\leftarrow$ガ:色合い,ヲ:海,ニ:[不特定-人])\label{イメージa}
\b.(イメージさせる$\leftarrow$ガ:色合い,ヲ:海)

同様に例\ref{得られた}でも「得られた」を可能と解釈するか,受け身と解釈するかでタグ付けが分かれた.

\ex. ここに今までに\underline{得られた}資料の一部を公表し、広く皆さまからの資料提供を願っております。\label{得られた}
\a. (得られた$\leftarrow$ガ:[著者],ヲ:資料)\label{得られたa}
\b. (得られた$\leftarrow$ガ:資料)

このような場合には項を取る格が多くなる方を選択することとした.
これはタグ付けされる項が多い方が述語項構造の持つ情報量が多くなるためである.
例\ref{イメージ}では\ref{イメージa}を,例\ref{得られた}では\ref{得られたa}をタグ付けした.

三つ目は必須的な格への[不特定-人],[不特定-物]などの付与漏れである.
これらは文章中に出現しない項であり,意識的に用言の格構造を考えなければ必須的な格であっても見落しやすいと考えられる.
例\ref{載せたい}の「載せたい」ではニ格は必須的な格であり[不特定-物]\footnote{「載せる」対象はこのサイトであるが,文章中に表現が存在しないので[不特定-物]とする.}を付与する必要がある.
タグ付けでは一人の作業者のみがニ格に何も付与しておらずこの作業者の見落しといえる.

\ex. 私の作詞の作品や身近の出来事や政治経済の事を\underline{載せたい}と思います。\label{載せたい}\\
\hspace*{4ex}(載せたい$\leftarrow$ガ:私,ヲ:作品 AND 出来事 AND 事,ニ:[不特定-物])

このような誤りについては,作業の際に複数の作業者による確認を行うことで訂正することが可能であると考えられる.



本研究で定義した複数の解釈が可能な表現に対するタグ付けが一致していないものはほとんど見られなかった.
一致していなかったもののうち,文脈からは判断が難しいために作業者間の解釈が分かれたものとして例\ref{サイコ}がある.
例\ref{サイコ}の「判断する」では,著者が「サイコロジカルライン」を読者に勧めている,と解釈すると\ref{サイコa}のようにタグ付けすることとなる.
一方,単なる「サイコロジカルライン」の説明と解釈する場合でも,「投資家」や投資についての研究者([不特定-人])が利用する手法だと解釈すれば\ref{サイコb}のようにタグ付けし,研究者のみが利用する手法だと解釈すれば\ref{サイコc}のようにタグ付けすることとなる.
どの解釈が正しいかは今回タグ付け対象とした3文だけからは困難である.
そこで今回はそのような場合には解釈可能なタグを全て付けることとし,\ref{サイコd}のようにタグ付けを行った.
一方,文書全体にタグ付けする際などには後続文の内容から解釈が一意に定まると考えられるので,格要素が明示されていない表現へのタグ付けの基準自体には問題はないといえる.

\ex. サイコロジカルとは、日本語に訳すと『心理的』という意味です。\\\label{サイコ}
サイコロジカルラインは、投資家心理に基づいて、買われすぎか売られすぎかを\underline{判断する}時に利用します。\\
直近12日間で、終値が前日の株価を上回った確率を示すのが一般的です。
\a.  
  (判断する$\leftarrow$[著者] ? [読者] ? [不特定-人])\label{サイコa}
\b. (判断する$\leftarrow$[不特定-人] ? 投資家)\label{サイコb}
\b. (判断する$\leftarrow$[不特定-人])\label{サイコc}
\b. (判断する$\leftarrow$[著者] ? [読者] ? [不特定-人] ? 投資家)\label{サイコd}
\z.


動作性体言に対するタグ付けの一致度は用言に対するタグ付けに比べて低くなっている.
これは体言は動作性を持つ場合にのみ用言としての述語項構造のタグ付けを行うが,作業者によって体言が動作性を持つかの基準が異なっていたことである.
例えば例\ref{トンカツ}では一人の作業者のみが動作性を持つとして\ref{トンカツa}のように述語項構造を付与したが,他の作業者は体言として\ref{トンカツb}のようにタグを付与した.
この場合にも,項を取る格が多くなる方を選択することとし\ref{トンカツa}をタグ付けした.

\ex.我々日本人は、生のキャベツの千切りをトンカツの\underline{付け合わせ}にしている。\label{トンカツ}
\a. (付け合わせ$\leftarrow$ガ:日本人,ヲ:千切り,ニ:トンカツ)\label{トンカツa}
\b. (付け合わせ$\leftarrow$ノ:トンカツ)\label{トンカツb}


\section{関連研究}

日本語の述語項構造および照応関係タグ付きコーパスとしては,京都大学テキストコーパス\cite{KTC}とNAISTテキストコーパス\cite{NTC}があり,述語項構造解析や照応解析の研究に利用されている\cite{笹野2008b,imamura-saito-izumi:2009:Short,iida-poesio:2011:ACL-HLT2011}.
これらのコーパスは1995年の毎日新聞に述語項構造および照応関係を付与したコーパスである.
新聞記事は内容が報道と社説に限られており,文体も統一されているため,新聞記事以外の意味関係解析への適応には不向きである.

様々なジャンルからなる日本語コーパスとしては現代日本語書き言葉均衡コーパス (BCCWJ)\footnote{http://www.ninjal.ac.jp/corpus\_center/bccwj/}がある.
このコーパスは書籍,雑誌などの出版物やインターネット上のテキストなどからなるコーパスである.
このコーパスでは,書籍などについては幅広いジャンルのテキストから構築されているが,インターネット上のテキストは掲示板やブログなどに限定されている.
このためインターネット上に多数存在する企業ページや通販ページなどはコーパスには含まれない.
また,BCCWJに意味関係を付与する研究も行われている.
一つ目は\cite{JapaneseFrameNet}によるBCCWJに日本語FrameNetで定義された意味フレーム情報,意味役割,述語項構造を記述する試みである.
この研究ではBCCWJのコアデータに含まれる用言と事態性名詞に対して項構造の記述を行っている.
しかしFrameNetではゼロ代名詞の有無は述語項構造に含まれるものの,先行詞が同一文内にない場合にはその照応先の情報を付与していない.
また,照応関係の情報も付与されておらず,文をまたぐ意味関係の情報は付与されていない.
二つ目は\cite{小町2012bccwj}による,述語項構造と照応関係のアノテーションである.
この研究では,NAISTテキストコーパスと同様の基準で述語項構造と照応関係をタグ付けしている.
述語項構造についてはNAISTテキストコーパスと同様にガ格,ヲ格,ニ格など限られた格にしか付与されていない.
しかし,NAISTテキストコーパスでは付与されている橋渡し照応などの関係は付与されていない.


日本語以外で複数のジャンルに渡って意味関係を扱ったコーパスとしては,OntoNote \cite{hovy-EtAl:2006:HLT-NAACL06-Short}やZ-corpus \cite{Z-corpus},LMC (Live Memories Corpus) \cite{LMC}などがある.
OntoNoteは英語,中国語,アラビア語の新聞記事,放送原稿,Webページなどからなるコーパスで,コーパスに含まれる一部のテキストは複数言語による対訳コーパスとなっている.
構文木,述語項構造,語義,オントロジー,共参照,固有表現などが付与されている.

Z-corpusはスペイン語の法律書,教科書,百科事典記事に対しゼロ照応の情報を付与したコーパスである.
ゼロ照応のみを扱っており,前方照応や述語項構造の情報は付与されていない.
スペイン語ではゼロ照応は主語のみに発生するため,述語項構造の情報とは独立にゼロ照応の情報を記述できるためである.

LMCはイタリア語のWikipediaとblogに照応関係のタグ付けをしたコーパスである.
照応関係としてゼロ照応も扱っているが,述語項構造は扱っていない.
イタリア語もゼロ照応は主語のみに発生するので,このコーパスではゼロ照応の起こった用言を照応詞としてタグ付けしている.


\section{まとめ}

本研究ではWebを利用することで多様な文書からなる意味関係タグ付きコーパスを構築した.
本研究では意味関係のタグとして,述語項構造と照応関係の付与を行った.
また,文書の著者・読者に着目し,その表現に対してタグ付けを行った.
タグ付けを先頭3文に限定することで1文書あたりの作業量を減らし,1,000文書へのタグ付けを行った.
タグ付けされた文書を人手で確認した結果,ブログ記事,企業ページなど多様な文書が含まれていた.
構築されたコーパスを分析した結果,多くの文書において談話に著者・読者が出現し,多様な著者・読者表現で記述されること,また特にゼロ照応において重要な役割を持つことを確かめた.

コーパス作成は5,000文書を目標として現在も作業中である.
完成後は研究利用を前提としての公開を予定している.



\acknowledgment
本コーパスのタグ付け作業に協力していただいた,石川真奈見氏,二階堂奈月氏,堀内マリ香氏に心から感謝致します.


\bibliographystyle{jnlpbbl_1.5}
\begin{thebibliography}{}

\bibitem[\protect\BCAY{Hovy, Marcus, Palmer, Ramshaw, \BBA\ Weischedel}{Hovy
  et~al.}{2006}]{hovy-EtAl:2006:HLT-NAACL06-Short}
Hovy, E., Marcus, M., Palmer, M., Ramshaw, L., \BBA\ Weischedel, R. \BBOP
  2006\BBCP.
\newblock \BBOQ OntoNotes: The 90\% Solution.\BBCQ\
\newblock In {\Bem Proceedings of the Human Language Technology Conference of
  the NAACL, Companion Volume: Short Papers}, \mbox{\BPGS\ 57--60}, New York
  City, USA. Association for Computational Linguistics.

\bibitem[\protect\BCAY{飯田\JBA 小町\JBA 井之上\JBA 乾\JBA 松本}{飯田 \Jetal
  }{2010}]{NTC}
飯田龍\JBA 小町守\JBA 井之上直也\JBA 乾健太郎\JBA 松本裕治 \BBOP 2010\BBCP.
\newblock 述語項構造と照応関係のアノテーション:NAIST
  テキストコーパス構築の経験から.\
\newblock \Jem{自然言語処理}, {\Bbf 17}  (2), \mbox{\BPGS\ 25--50}.

\bibitem[\protect\BCAY{Iida \BBA\ Poesio}{Iida \BBA\
  Poesio}{2011}]{iida-poesio:2011:ACL-HLT2011}
Iida, R.\BBACOMMA\ \BBA\ Poesio, M. \BBOP 2011\BBCP.
\newblock \BBOQ A Cross-Lingual ILP Solution to Zero Anaphora Resolution.\BBCQ\
\newblock In {\Bem Proceedings of the 49th Annual Meeting of the Association
  for Computational Linguistics: Human Language Technologies}, \mbox{\BPGS\
  804--813}, Portland, Oregon, USA. Association for Computational Linguistics.

\bibitem[\protect\BCAY{Imamura, Saito, \BBA\ Izumi}{Imamura
  et~al.}{2009}]{imamura-saito-izumi:2009:Short}
Imamura, K., Saito, K., \BBA\ Izumi, T. \BBOP 2009\BBCP.
\newblock \BBOQ Discriminative Approach to Predicate-Argument Structure
  Analysis with Zero-Anaphora Resolution.\BBCQ\
\newblock In {\Bem Proceedings of the ACL-IJCNLP 2009 Conference Short Papers},
  \mbox{\BPGS\ 85--88}, Suntec, Singapore. Association for Computational
  Linguistics.

\bibitem[\protect\BCAY{河原\JBA 黒橋\JBA 橋田}{河原 \Jetal }{2002}]{KTC}
河原大輔\JBA 黒橋禎夫\JBA 橋田浩一 \BBOP 2002\BBCP.
\newblock 「関係」タグ付きコーパスの作成.\
\newblock \Jem{言語処理学会第 8 回年次大会}, \mbox{\BPGS\ 495--498}.

\bibitem[\protect\BCAY{Kawahara \BBA\ Kurohashi}{Kawahara \BBA\
  Kurohashi}{2006}]{Kawahara2006}
Kawahara, D.\BBACOMMA\ \BBA\ Kurohashi, S. \BBOP 2006\BBCP.
\newblock \BBOQ Case Frame Compilation from the Web using High-Performance
  Computing.\BBCQ\
\newblock In {\Bem Proceedings of the 5th International Conference on Language
  Resources and Evaluation}, \mbox{\BPGS\ 67--73}.

\bibitem[\protect\BCAY{小町\JBA 飯田}{小町\JBA 飯田}{2011}]{小町2012bccwj}
小町守\JBA 飯田龍 \BBOP 2011\BBCP.
\newblock BCCWJに対する述語項構造と照応関係のアノテーション.\
\newblock \Jem{日本語コーパス平成22年度公開ワークショップ}, \mbox{\BPGS\
  325--330}.

\bibitem[\protect\BCAY{小原}{小原}{2011}]{JapaneseFrameNet}
小原京子 \BBOP 2011\BBCP.
\newblock
  日本語フレームネットの全文テキストアノテーション:BCCWJへの意味フレーム付与の試み.\
\newblock \Jem{言語処理学会第17回年次大会}, \mbox{\BPGS\ 703--704}.

\bibitem[\protect\BCAY{Rello \BBA\ Ilisei}{Rello \BBA\ Ilisei}{2009}]{Z-corpus}
Rello, L.\BBACOMMA\ \BBA\ Ilisei, I. \BBOP 2009\BBCP.
\newblock \BBOQ A Comparative Study of Spanish Zero Pronoun Distribution.\BBCQ\
\newblock In {\Bem Proceedings of the International Symposium on Data and Sense
  Mining, Machine Translation and Controlled Languages (ISMTCL)}, \mbox{\BPGS\
  209--214}.

\bibitem[\protect\BCAY{Rodr\'iguez, Delogu, Versley, Stemle, \BBA\
  Poesio}{Rodr\'iguez et~al.}{2010}]{LMC}
Rodr\'iguez, K.~J., Delogu, F., Versley, Y., Stemle, E.~W., \BBA\ Poesio, M.
  \BBOP 2010\BBCP.
\newblock \BBOQ Anaphoric Annotation of Wikipedia and Blogs in the Live
  Memories Corpus.\BBCQ\
\newblock In {\Bem Proceedings of the Seventh Conference on International
  Language Resources and Evaluation (LREC'10)}, \mbox{\BPGS\ 157--163},
  Valletta, Malta.

\bibitem[\protect\BCAY{笹野\JBA 黒橋}{笹野\JBA 黒橋}{2008}]{笹野2008b}
笹野遼平\JBA 黒橋禎夫 \BBOP 2008\BBCP.
\newblock 自動獲得した名詞関係辞書に基づく共参照解析の高度化.\
\newblock \Jem{自然言語処理}, {\Bbf 15}  (5), \mbox{\BPGS\ 99--118}.

\bibitem[\protect\BCAY{Sasano \BBA\ Kurohashi}{Sasano \BBA\
  Kurohashi}{2011}]{sasano-kurohashi:2011:IJCNLP-2011}
Sasano, R.\BBACOMMA\ \BBA\ Kurohashi, S. \BBOP 2011\BBCP.
\newblock \BBOQ A Discriminative Approach to Japanese Zero Anaphora Resolution
  with Large-scale Lexicalized Case Frames.\BBCQ\
\newblock In {\Bem Proceedings of 5th International Joint Conference on Natural
  Language Processing}, \mbox{\BPGS\ 758--766}, Chiang Mai, Thailand. Asian
  Federation of Natural Language Processing.

\end{thebibliography}


\appendix

\section{複数解釈可能な表現の例とタグ付け基準}\label{付録:複数付与基準}

\ref{複数付与基準}節で示した複数解釈可能な表現のタグ付けにおいて,典型的表現として作業者に例示したものを示す.
なお\ref{複数付与基準}節と同様に[著者],[読者]は著者表現,読者表現に対しても同様に考えることとする.
以降の例で文頭に括弧内で書かれている内容は,例文の書かれている文脈についての情報である.

\subsection*{著者の考えや経験を述べた表現}

著者の経験や考えを述べた表現において[著者]だけでなく[不特定-人]を付与するかどうかは以下のように判断する.

\paragraph{[著者]のみを付与する場合:}
著者のみが当てはまり,他の人にはあてはまらないと考えられる場合には動作主にあたる格に[著者]のみを付与する.
具体的には,ブログにおける著者の自身の出来事や感想,謙譲表現の主体などがあたる.

\ex. 今回始めて\underline{訪店しましたが}、素敵なお店だと\underline{思いました}。\label{著者のみ1}\\
\hspace*{4ex}
$\left(
\begin{tabular}{@{}l@{}}
訪店しましたが$\leftarrow$ガ:[著者]\\
思いました$\leftarrow$ガ:[著者]\\
\end{tabular}
\right)$

\paragraph{[著者]および[不特定-人]を付与する場合:}
以下のような場合には動作主にあたる格に「[著者] ? [不特定-人]」を付与する.
\begin{description}
\item [・一般論だが著者にもあてはまる] 例\ref{?著者+不特定-1}では,「整備しておかなければならない」のは一般論だが,この文書では著者は鉄道会社と考えられ,著者自身にもあてはまると解釈できる. 

\ex. 線路は列車の安全を確保し、快適な乗り心地を維持する状態に\underline{整備しておかなければなりません}。\label{?著者+不特定-1}\\
\hspace*{4ex}(整備しておかねばなりません$\leftarrow$ガ:[著者] ? [不特定-人])

\item [・著者自身が経験したことで,一般にもあてはまる] 例\ref{?著者+不特定-2}では,著者自身が鹿を見た経験を持つと考えられるが,「できます」という表現により一般の人にもあてはまると解釈できる.

   \ex. (ブログ内にて)蓼科ではいたるとことで鹿を\underline{見ることができます}。\label{?著者+不特定-2}\\
\hspace*{4ex}(見ることができます$\leftarrow$ガ:[著者] ? [不特定-人])

\item [・著者自身が実行したか分からない]例\ref{?著者+不特定-3}では,文脈だけからは乳牛を育てている著者が飼料を栽培しているのか,飼料販売会社が栽培しているのかが分からない.

\ex. 乳牛のエサは有機肥料を用いて\underline{栽培した}飼料を使用しています。\label{?著者+不特定-3}\\
\hspace*{4ex}(栽培した$\leftarrow$ガ:[著者] ? [不特定-人])

\end{description}

\subsection*{読むこと自体が行為の受け手となる表現}

「紹介します」「お教えします」などの表現は,読むこと自体がその行為の受け手となるため特別に扱う必要がある.

\paragraph{[読者]のみを付与する場合:}
「紹介します」「お教えします」の具体的な内容が直後に書かれている場合には受け手にあたる格に[読者]のみを付与する.
この場合には,その文章を読んだ人(読者)のみが受け手となるからである.

\ex. 今まで中々結婚にたどりつけなかった理由を\underline{お教えしましょう}。\label{?サービス-4}\\
\hspace*{4ex}(お教えしましょう$\leftarrow$ガ:[著者],ニ:[読者])\\
幸せな結婚の為にはまず、あなた自身の内面と向き合う必要があります。

\paragraph{[読者]および[不特定-人]を付与する場合:}
「紹介します」「お教えします」などがWebサイト全体のことを指している場合には,受け手にあたる格要素に「[読者] ? [不特定-人]」を付与する.
これは,Webサイトにおける「当サイトの紹介」ページなどでは,そのページを読んだだけではそのサイト全体を読んでいるとは限らないためである.

\ex. このページでは、免許取得のための講座を\underline{ご紹介していきます}。\label{?サービス-1}\\
\hspace*{4ex}(ご紹介していきます$\leftarrow$ガ:[著者],ニ:[読者] ? [不特定-人])

\ex. 見ごたえのある作品を当サイトにて\underline{ご紹介させて頂いております}。\label{?サービス-2}\\
\hspace*{4ex}(ご紹介させて頂いております$\leftarrow$ガ:[著者],ニ:[読者] ? [不特定-人])

\subsection*{勧誘的表現}

\paragraph{[著者]および[読者]を付与する場合:}
勧誘した行為を著者も行うと解釈できる場合には[著者]および[読者]を付与する.この場合には著者と読者が同時に行うと考えられるので「AND」の関係とする.
例\ref{勧誘-1}のような場合には,著者が読者に株式市場の分類を見るように促し,また説明のため著者も見ていると考えられる.Webサイトを通してのやりとりであり実際に同時に見るわけではないが,説明の過程で同時に見ていると仮定して「AND」で付与する.

\ex. まずは株式市場の分類を\underline{見てみましょう}。\label{勧誘-1}\\
\hspace*{4ex}(見てみましょう$\leftarrow$ガ:[著者] AND [読者])

\paragraph{[読者]のみを付与する場合:}
勧誘した行為を著者が行わないと解釈できる場合には[読者]のみを付与する.
例\ref{勧誘-2}や例\ref{勧誘-3}では,著者は読者に勧めているが,著者自身は実行しないと考えられるので[読者]のみを付与する.

\ex. (旅行会社のサイト内で) 様々な無人島が沖縄にはありますので、いろいろ\underline{チャレンジしてみましょう}。\label{勧誘-2}\\
\hspace*{4ex}(チャレンジしてみましょう$\leftarrow$ガ:[読者])

\ex. リフレックスで不動産を\underline{売却してみませんか}。\label{勧誘-3}\\
\hspace*{4ex}(売却してみませんか$\leftarrow$ガ:[読者])

\subsection*{使役的表現}

使役表現や依頼表現では[読者]のみを付与する.

\ex. 直接\underline{予約してください}。\label{使役-1}\\
\hspace*{4ex}(予約してください$\leftarrow$ガ:[読者])

\ex. メールの際は必ず名前を\underline{添えてください}。\label{使役-2}\\
\hspace*{4ex}(添えてください$\leftarrow$ガ:[読者])

\subsection*{会員制のサービス,メールマガジン等の紹介}

\paragraph{[読者]および[不特定-人]を付与する場合:}
読者を会員などと仮定している場合には,そのサービスは会員である[読者]および読者以外の会員([不特定-人])が利用できるので「[読者] ? [不特定:人]」を付与する.

\ex. (会員制の通販のページで)分割払いなど、多彩なお支払い方法から\underline{選択できます}。詳しくはガイドをご参照ください。\label{サービス-1}\\
\hspace*{4ex}(選択できます$\leftarrow$ガ:[読者] ? [不特定-人])

\paragraph{[不特定-人]のみを付与する場合:}
読者をまだサービスに加入していない人と仮定し,そのサービスを紹介している場合には,この段階では読者はサービスを受けていないので[不特定-人]のみを付与する.

\ex. メールマガジンではお得な情報を\underline{お送りしています}。是非ご登録ください。\label{サービス-2}\\
\hspace*{4ex}(お送りしています$\leftarrow$ニ:[不特定-人])

\subsection*{著者が読者に勧めている表現}

一般的な事項であるが,著者が読者に勧めているような表現の場合には[読者]および[不特定-人]を付与する.また,そこに[著者]を加えるかは以下のように判断する.

\paragraph{[著者],[読者]および[不特定-人]を付与する場合:}
著者自身も実行した経験から勧めていると考えられる場合には,「[著者] ? [読者] ? [不特定-人]」を動作主にあたる格に付与する.

\ex. ブログに記事を書き込んで、インターネット上で\underline{公開する}のはとても簡単です。\label{勧める-1}\\
\hspace*{4ex}(公開する$\leftarrow$ガ:[著者] ? [読者] ? [不特定-人],ヲ:ブログ)

\paragraph{[読者]および[不特定-人]を付与する場合:}
著者が自社製品を紹介している場合など,著者自身は実行していないと考えられる場合には,「[読者] ? [不特定-人]」を動作主にあたる格に付与する.

\ex. 吊るし紐付きですので、部屋に吊るして\underline{飾る}事もできます。\label{勧める-2}\\
\hspace*{4ex}(飾る$\leftarrow$ガ:[読者] ? [不特定-人])

\begin{biography}
\bioauthor{萩行 正嗣}{
2008年京都大学工学部電気電子工学科卒業.2010年同大学大学院情報学研究科修士課程修了.現在,同大学院博士後期課程在学中.日本語ゼロ照応解析の研究に従事.
}
\bioauthor{河原 大輔}{
1997年京都大学工学部電気工学第二学科卒業.1999年同大学院修士課程修了.
2002年同大学院博士課程単位取得認定退学.東京大学大学院情報理工学系研究
科学術研究支援員,独立行政法人情報通信研究機構研究員,同主任研究員を経
て,2010年より京都大学大学院情報学研究科准教授.自然言語処理,知識処理
の研究に従事.博士(情報学).言語処理学会,情報処理学会,人工知能学会,
電子情報通信学会,ACL,各会員.
}
\bioauthor{黒橋 禎夫}{
1994年京都大学大学院工学研究科電気工学第二専攻博士課程修了.博士(工学).2006年4月より京都大学大学院情報学研究科教授.自然言語処理,知識情報処理の研究に従事.言語処理学会10周年記念論文賞等を受賞.
}

\end{biography}


\biodate



\end{document}
