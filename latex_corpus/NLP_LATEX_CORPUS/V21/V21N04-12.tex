    \documentclass[english]{jnlp_1.4_rep}

\usepackage{jnlpbbl_1.3}
\usepackage[dvipdfm]{graphicx}
\usepackage{udline}
\setulminsep{1.2ex}{0.2ex}
\usepackage{array}


\Volume{21}
\Number{4}
\Month{September}
\Year{2014}

\received{2012}{4}{18}
\revised{2012}{7}{15}
\accepted{2012}{7}{26}

\setcounter{page}{941}

\etitle{Particle Error Correction from Small Error Data \\for Japanese Learners\footnotetext{\llap{*~}This article has been partially revised for better understanding of overseas readers.}}
\eauthor{Kenji Imamura\affiref{NTT} \and Kuniko Saito\affiref{NTT}\affiref{NTTCOM} \and Kugatsu Sadamitsu\affiref{NTT} \and Hitoshi Nishikawa\affiref{NTT}} 
\eabstract{
This paper shows how to correct the grammatical errors of Japanese
particles made by Japanese learners. Our method is based on
discriminative sequence conversion, which converts one sequence of
words into another and corrects particle errors by substitution,
insertion, or deletion. However, it is difficult to collect large
learners' corpora. We solve this problem with a discriminative
learning framework that uses the following two methods. First,
language model probabilities obtained from large, raw text corpora are
combined with $n$-gram binary features obtained from learners'
corpora. This method is applied to measure the accuracy of Japanese
sentences. Second, automatically generated pseudo-error sentences are
added to learners' corpora to enrich the corpora
directly. Furthermore, we apply domain adaptation, in which the
pseudo-error sentences (the source domain) are adapted to the real
error sentences (the target domain). Experiments show that the recall
rate is improved using both language model probabilities and $n$-gram
binary features. Stable improvement is achieved using pseudo-error
sentences with domain adaptation.}
\ekeywords{Particle Error Correction, Language Models, Pseudo-Error Generation, \linebreak
	Domain Adaptation, Discriminative Sequence Conversion}

\headauthor{Imamura et al.}
\headtitle{Particle Error Correction from Small Error Data for Japanese Learners}

\affilabel{NTT}{}{NTT Media Intelligence Laboratories, NTT Corporation}
\affilabel{NTTCOM}{}{Presently with NTT Communications Corporation}

\Reprint[T]{Vol.~19, No.~5, pp.~381--400}

\begin{document}

\maketitle

\section{Introduction}

Case markers in a sentence are represented by postpositional particles
in Japanese. Incorrect usage of particles causes serious communication
errors because readers cannot understand the content of sentences
through the case markers. For example, it is unclear as to what must
be deleted in the following sentence.

\begin{minipage}[t]{\textwidth}
\begin{center}
\begin{tabular}{cccccc}
\textit{mail} & \textbf{o} & \textit{todoi} & \textit{tara} 
& \textit{sakujo} & \textit{onegai-shi-masu} \\
mail & \textbf{ACC.} & arrive & when & delete & please \\
\multicolumn{6}{c}
{``When $\phi$ has arrived an e-mail, please delete it.''} \\
\end{tabular}
\end{center}
\end{minipage}

If the accusative particle \textit{o} is replaced by a nominative one
\textit{ga}, it becomes clear that the writer wants the e-mail to be
deleted (``When the e-mail has arrived, please delete it.'').  Such
particle errors frequently occur in sentences written by Japanese
learners (c.f., Section \ref{sec-particle-errors}).

Automatic error correction of texts written by Japanese learners is
essential in education as well as business. For example, some Japanese
corporations outsource the computer software 
\linebreak
development to foreign
companies, especially China and India, using Japanese or English as
the communication language. However, Japanese is a foreign language
for most development companies, and texts such as e-mails can contain
many errors that inhibit effective communication. Thus, automatic
error correction is regarded as a key solution
\cite{Ohki:ParticleError2011j,Suenaga:ErrorCorrection2012j}.



This paper presents a method that can automatically correct Japanese
particle errors. This task corresponds to preposition error correction
in English. For English error correction, many studies employ
classifiers, which select the appropriate prepositions/articles by
restricting the error types to articles and frequent prepositions
\cite{gamon:2010:NAACLHLT,HAN10.821,rozovskaya-roth:2011:ACL-HLT2011}.

On the contrary, \citeA{mizumoto-EtAl:2011:IJCNLP-2011} proposed
machine-translation-based error correction. This approach can handle
all error types by converting learners' sentences into the correct
version. Although the target of this paper is particle error, we
employ a similar approach based on sequence conversion
\cite{Imamura:MorphTrans2011} since it offers excellent
scalability.


However, the machine-translation-based approach requires a sufficient
number of parallel sentences, which consist of learners' erroneous
sentences and their manually corrected ones, to learn the
models. However, collecting a large number of such pairs is difficult
because of high costs. To avoid this problem, we propose the following
two methods.


\begin{enumerate}

\item Using Large Raw Text Corpora \\
(Combination of Language Model Probabilities and Binary Features)

Because one half of the parallel sentences are correct Japanese, they
can be easily obtained from raw text corpora. Thus, we regard a large
text corpus as a set of corrected sentences and incorporate it into
the error corrector models. The text corpus is used for computing
language model probabilities. The error corrector models are jointly
optimized with binary features acquired from the parallel sentences by
using discriminative learning. We expect that the error correction
coverage will be improved because the degree of sentence correctness
is measured by language model probabilities, even if the errors rarely
appear in the parallel sentences.

\item Expansion of the Parallel Sentences Using Pseudo-Error Sentences \\
(and Application of Domain Adaptation)

Since collecting learners' erroneous sentences is not easy, we
automatically generate pseudo-error sentences, which mimic learners'
real errors, by applying error patterns to correct sentences
\shortcite{rozovskaya-roth:2010:NAACLHLT}. An additional training
corpus consists of the pairs of pseudo-error sentences and their
source sentences. Note that the pseudo-error sentences do not
completely reflect the error distribution of learners'
sentences. Therefore, we further apply a domain adaptation technique
that regards the pseudo-errors and real errors as the source and
target domains, respectively. We expect stable improvement if the
error distributions are partially different.

\end{enumerate}

The remainder of this paper is organized as follows. In Section
\ref{sec-particle-errors}, we analyze errors in the sentences written
by Japanese learners. Section \ref{sec-conversion} describes our error
correction method and use of raw text corpora. Section
\ref{sec-pseudo-sentences} describes the expansion of parallel
sentences using pseudo-error sentences and domain adaptation. In
Section \ref{sec-experiments}, we conduct experiments to confirm error
correction accuracy. Section \ref{sec-related-work} introduces related
studies, and Section \ref{sec-conclusion} concludes this study.


\section{Errors in Sentences Written by Japanese Learners}
\label{sec-particle-errors}

In order to analyze error types, we first collected erroneous examples
written by Chinese native speakers who are learning Japanese.


Thirty seven subjects, who learnt Japanese while they are/were at
engineering universities in Japan, participated in the experiment. The
periods spent living in Japan ranged from six months to six years. We
provided 80 English sentences obtained from Linux manuals and 24
figures to each subject (104 tasks in total), and they re-wrote the
sentences in Japanese (Hereafter ``learners' sentences''). As a
result, 2,770 learners' sentences were collected. Each sentence was
revised by Japanese native speakers (called the correct sentences). To
make the revisions, a bare minimum of error correction was applied to
the point at which the sentences become grammatically/pragmatically
correct, while retaining the meaning. In other words, only fatal
errors were corrected from the viewpoint of Japanese
grammar.\footnote{Vocabulary errors were corrected even if the
sentences were grammatically acceptable.}


\subsection{Error Categorization and Distribution}

We categorized the errors into three major types, grammatical,
vocabulary, and surface form errors, and set sub-categories, as shown
in Table \ref{tbl-error-class}.

\begin{table}[t]
\caption{Error Categorization and Examples}
\label{tbl-error-class}
\input{03table01.txt}
\small
X and Y of ``*X/Y'' in this table denote the error and its correction, respectively.
\par
\end{table}

The rewriters revised 2,171 of 2,770 sentences. The unchanged
sentences included 559 sentences without errors and 40 sentence
fragments. The following analyses were applied to the 2,171 revised
sentences.


The number of corrected errors was 4,916 words/phrases (2.26 per
sentence). From the perspective of distribution by categories, most
were grammatical errors (54\%), followed by 
\linebreak
vocabulary errors (28\%)
and surface form errors (16\%). The rest were compound errors. From
the perspective of sub-categories, the most frequent errors were
located in particles or auxiliary verbs (33\%), followed by
transliteration (11\%) and synonyms (10\%).



\subsection{Analyses and Discussion}

With regard to the most frequent errors, particle errors appear widely
in texts written by non-native Japanese speakers regardless of their
first language. This is because particles are special grammatical
units in Japanese,\footnote{Particles are also used in Korean.} and
have various functions. For example, case particles specify the case
of noun phrases such as nominative, accusative, and possessive. Topic
particles work as topic markers, and can combine with case
particles. There are other types such as conjunctive and sentence-end
particles. Therefore, appropriate usage is difficult for most
non-native speakers, and effective teaching should include correction
of particle errors.


Since fallibility depends on the particle, we calculated the frequency
of particle errors. Table~\ref{tbl-particle-errors} shows the top 10
particle errors.

Correction type in Table \ref{tbl-particle-errors} denotes the edit
operation needed to correct the errors: insertion (INS), deletion
(DEL), or substitution (SUB). From the viewpoint of edit operation,
substitutions were the most frequent (74\%), particle omissions (INS)
were 17\%, and excess particles (DEL) were 9\%.



\begin{table}[t]
\caption{Frequent Particle Errors (Top 10)}
\label{tbl-particle-errors}
\input{03table02.txt}
\vspace{1\Cvs}
\end{table}

The following are examples of substitution, insertion, and
deletion. Rank 1 in Table \ref{tbl-particle-errors} implies that the
topic particle \textit{wa} should appear only once in a sentence. If
it appears twice, one of them should be replaced by a case particle
\textit{ga} ({\it mondai- \underline{*wa/ga} aru-toki- \underline{wa}
...} ``if you have problem ...''). Rank 3 shows an example of
insertion. Lack of the possessive particle \textit{no} between a
cardinal number and a noun was frequent ({\it futatsu *$\phi$/-no
file} ``two files''). Rank 8 shows an example of deletion. Many
learners appended the possessive particle \textit{no} after verbs or
adjectives, which modify a noun ({\it chiisai *-no/$\phi$ e} ``small
picture'').


The above analyses confirmed that particle error correction is
effective for Japanese learners. These errors are corrected by
applying the edit operations of substitution, insertion, and deletion.



\section{Error Correction by Discriminative Sequence Conversion}
\label{sec-conversion}

This section describes error correction using discriminative sequence
conversion. Our error correction method converts learners' word
sequences into correct sequences (sentences are segmented into words
in advance by a morphological analyzer). The method is similar to
phrase-based statistical machine translation (PBSMT), but has the
following three differences: 1) it adopts conditional random fields,
2) it allows insertion and deletion, and 3) combines $n$-gram binary
features and language model probabilities. Unlike the classification
approach (e.g., \shortcite{suzuki-toutanova:2006:COLACL}), the
conversion approach can correct multiple errors of all types in a
sentence.


\subsection{Basic Procedure}

We apply a morpheme conversion approach that converts the results of a
speech recognizer into word sequences for language analyzer processing
\shortcite{Imamura:MorphTrans2011}. It corrects particle errors in the
input sentences as follows.

\begin{figure}[b]
\begin{center}
\includegraphics{21-4ia3f1.eps}
\end{center}
\caption{Example of Phrase Lattice (bold lines denote the correct sequence)}
\label{fig-lattice}
\end{figure}


\begin{itemize}

\item First, all modification candidates are obtained by referring to
a phrase table. This table, called the confusion set
\shortcite{rozovskaya-roth:2010:EMNLP} in the error correction task,
stores pairs of incorrect and correct particles (Table
\ref{tbl-particle-errors}).\footnote{Table \ref{tbl-particle-errors}
shows a part of the phrase table. As we will describe in Section
\ref{sec-experimental-settings}, all words whose part-of-speech is
\texttt{Particle} according to the ipadic-2.7.0 dictionary are
processed.}  The candidates are packed into a lattice structure,
called the phrase lattice (Figure \ref{fig-lattice}).  To deal with
unchanged words, it also copies the input words and inserts them into
the phrase lattice.

\item Next, the best phrase sequence in the phrase lattice is
identified on the basis of the conditional random fields (CRFs)
\cite{Lafferty:CRF2001}. The Viterbi algorithm is used for
decoding because error correction does not change the word
order. Although the phrase lattice includes obviously ungrammatical
sequences (e.g., a sequence in which identical particles \textit{o}
are adjacent), we do not prune them, and search for the best sequence
according to the model.

\item While training, word alignment is carried out by dynamic
programming matching. From the alignment results, the phrase table is
constructed by acquiring particle errors, and the CRF models are
trained using the alignment results as supervised data.\footnote{We
implemented a CRF learner using Okazaki's \texttt{libLBFGS} as the
optimizer.\\ http://www.chokkan.org/software/liblbfgs/}

\end{itemize}


\subsection{Insertion/Deletion}

The error correction in this paper uses insertion and deletion,
whereas phrase-based SMTs generally translate a sentence using only
substitution. Since an insertion can be regarded as replacing an empty
word with an actual word and deletion is the replacement of an actual
word with an empty one, we treat these operations as substitution
without distinction while learning/applying the CRF models.


However, insertion is a high cost operation because it may occur at
any location and can cause the lattice size to explode. To avoid this
problem, we permit one only word insertion and only immediately after
nouns. This restriction makes a few errors impossible to correct
(e.g., it becomes impossible to insert the topic particle \textit{wa}
immediately after the dative particle \textit{ni}).


Note that substitution can be represented as a sequence of insertion
and deletion. In this paper, the supervised data for the CRF models
are created while consecutive insertion and deletion are bundled as
substitution. During error correction, multiple candidates consisting
of the same surface forms exist in the phrase lattice; one is
represented by substitution and the other is a sequence of insertion
and substitution. However, substitution is selected in almost all
cases by the learnt models.


\subsection{Features}

In this paper, we use mapping features and link features. The former
measure the correspondence between input and output words (similar to
the translation models of PBSMT), while the latter measure the fluency
of the output word sequence (similar to language models). Figure
\ref{fig-features} illustrates the features, and Table
\ref{tbl-templates} shows the feature templates.

Many natural language processing (NLP) tasks that employ
discriminative models (e.g., named entity recognition) use not only
the targeted word but also its neighboring words as features. We
employ the same approach. The focused phrase and its two neighboring
words in the input are regarded as the window. The mapping features
are defined as the pairs of the output phrase and uni-, bi-, and
tri-grams in the window. All mapping features are binary.

Link features are detailed in the next section.

\begin{figure}[t]
\begin{center}
\includegraphics{21-4ia3f2.eps}
\caption{Mapping and Link Features}
\label{fig-features}
\end{center}
\end{figure}

\begin{table}[t]
\caption{Feature Templates}
\label{tbl-templates}
\input{03table03.txt}
\vspace{4pt}\small
$\delta(\cdot)$ in the table denotes a binary function that returns 1 
iff all arguments match the function, and 0 otherwise.
\par
\end{table}


\subsection{Using Raw Text Corpora and Incorporation into Link Features}

The link features are important for the error correction task because
the system has to judge output accuracy. The correct sentences can be
easily obtained from raw text corpora. Using these corpora, we mix the
two types of features, as follows, and optimize their weights in the
CRF framework. We leverage a characteristic that discriminative models
can handle features that depend on each other.

\begin{itemize}
\item \textbf{$N$-gram binary features}

$N$-grams of the output words, from 1 to 3, are used as binary
features. These are obtained from the correct side of a training
corpus (parallel sentences). Since the weights of individual features
are optimized considering all features (including the mapping
features), fine-tuning can be achieved. The accuracy becomes almost
perfect on the training corpus. In other words, we can certainly
correct errors in new texts if the same error patterns are present in
the training corpus.

\item \textbf{Language model probability}

This is a logarithmic value (real value) of the $n$-gram probability
of the output word sequence. One feature weight is assigned. The
$n$-gram language model can be constructed from large text corpora,
i.e., it does not require parallel sentences. The language model
probabilities provide grammatical correctness scores, regardless of
whether the errors in the new texts appeared in the training corpus or
not.
\end{itemize}

Incorporating binary and real features yields a rough approximation of
generative models in semi-supervised conditional models
\cite{suzuki-EtAl:2009:EMNLP,Suzuki:SemiSupervised2010j}. It can
appropriately correct new sentences while maintaining high accuracy on
the training corpus.


\section{Expansion of Parallel Sentences Using Pseudo-error Sentences}
\label{sec-pseudo-sentences}

The error corrector described in Section \ref{sec-conversion} requires
parallel sentences, which correspond to bilingual sentences in machine
translation. However, it is difficult to collect a sufficiently large
set of such sentences. We resolve this problem by using pseudo-error
sentences to expand them. In this section, we describe the generation
of pseudo-error sentences and their application using domain
adaptation.


\subsection{Pseudo-Error Generation}

As we described before, correct sentences, which are halves of the
parallel sentences, can be easily acquired from raw text corpora.  If we
can generate errors that mimic learners' sentences, we can obtain new
parallel sentences.


We utilize the method of
\shortciteA{rozovskaya-roth:2010:NAACLHLT}. Concretely, when particles
appear in the correct sentence, they are replaced by incorrect ones in
a probabilistic manner by applying the phrase table (which stores the
error patterns) in the opposite direction.  The error generation
probabilities are relative frequencies on the training corpus (i.e.,
the real error corpus). Namely,
\begin{equation}
P_{\mathit{error}}(f|e)  =  \frac{C(f,e)}{C(e)},
\end{equation}
where $P_{\mathit{error}}(f|e)$ denotes the error generation probability,
$C(f,e)$ denotes the co-occurrence frequency of the correct particle
$e$ and its error particle $f$ in the real error corpus, and $C(e)$
denotes the frequency of correct particle $e$ in the corpus.

The models are learnt using both real error corpus and pseudo-error
corpus.


\subsection{Domain Adaptation by Feature Augmentation}
\label{sec-domain-adaptation}

Although the error generation probabilities are computed from the real
error corpus, the error distribution is not exactly the same. To fit
the pseudo-errors to the real errors better, we apply a domain
adaptation technique. Namely, we regard the pseudo-error corpus as the
source domain and the real error corpus as the target domain, and
models that fit the target domain are learnt.


In this paper, we use \shortciteA{daumeiii:2007:ACLMain}'s feature
augmentation method for domain adaptation. This method learns the
target domain models by expanding its feature space, and has
the same effect; that is, the models for the source domain are
regarded as prior distribution for the target domain. In addition, it
eliminates the need to change the learning algorithm.


We briefly review the feature augmentation. The feature space is
segmented into three parts: common, source, and target. The features
extracted from the source domain data ($D_s$) are deployed to the
common and source spaces, and those from the target domain data
($D_t$) are deployed to the common and target spaces. Namely, the
feature space is tripled (Figure \ref{fig-augment}).

\begin{figure}[t]
\begin{center}
\includegraphics{21-4ia3f3.eps}
\end{center}
\caption{Feature Augmentation}
\label{fig-augment}
\end{figure}

Parameter estimation is carried out in the usual way on the above
feature space. Consequently, the weights of the common features are
emphasized if the features are consistent between the source and
target. With regard to domain-dependent features (i.e., inconsistent
between the source and target), the weights in the source or the
target space are emphasized. With respect to the features that only
appear in the source or target domain, the weights in the common and
domain-dependent spaces are emphasized.


Figure \ref{fig-augment} shows an example of feature
augmentation. Here, we simplify the task to just asking whether the
case particle \textit{ga} should be replaced with \textit{o} or left
unchanged. We assume that the following three features are acquired
from the source and target domain data (we assume template No. 11 in
Table \ref{tbl-templates}).

\begin{itemize}
\item[A)] ``{\it Kinou \underline{ga} riyou}'' appears in both source
and target domains, and \textit{ga} is replaced with \textit{o}.

\item[B)] ``{\it Data \underline{ga} henkou}'' appears in both
source and target domains.  However, it is unchanged in the source
domain and is replaced with \textit{o} in the target domain data.

\item[C)] ``{\it Kansuu \underline{ga} jikkou}'' only appears in the
source domain data.
\end{itemize}

When we estimate parameters on the above feature space, the weights of
A) in the common space are emphasized because it is consistent between
the domains. On the contrary, the weights of B) in the common space
become zero, and those in the source or target space are emphasized,
depending on the correction, because the domains are
inconsistent. Furthermore, the weights of C) in the common and source
spaces are emphasized.

Error correction uses only features in the common and target
spaces. The error distribution approaches the real error distribution
because the weights of the features are optimized to the target
domain. In addition, it becomes robust against new sentences because
the common features acquired from the source domain can be used even
when they do not appear in the target domain. From the examples in
Figure \ref{fig-augment}, the features of C), which only appeared in
the source domain data, can be used to enhance error correction.


\section{Experiments}
\label{sec-experiments}

\subsection{Experimental Settings}
\label{sec-experimental-settings}

\paragraph{Target Particles: }

The target particles in this paper are words whose part-of-speech
(POS) tag is \texttt{Particle}, according to the ipadic-2.7.0 morpheme
dictionary. The dictionary includes not only case particles but also
topic (focus), adverbial, conjunctive, sentence-ending, and parallel
particles (236 particles in total). Thirty-eight particles in the
dictionary, which appear in learners' corpus and were incorrectly used
by the learners at least once, are targeted in the
experiments.\footnote{In comparison with
\shortciteA{suzuki-toutanova:2006:COLACL}'s study, which handled 10
case particles and topic particle \textit{wa}, we exclude dative
particle \textit{e} and include combined particle \textit{ni-yori}
instead of \textit{yori} alone. In addition, we handle topic particle
\textit{wa} as an individual word, whereas
\shortciteA{suzuki-toutanova:2006:COLACL}'s study regarded a sequence
of case and topic particles as one word.}


\paragraph{Learners' Corpus (Real Error Corpus): }

The learners' corpus used in the experiments consists of 2,770
parallel sentences (104 tasks) collected in Section
\ref{sec-particle-errors}. From these sentences, only particle errors
were retained; the other errors were corrected by copying the
corresponding parts of the correct sentence. Therefore, the parallel
sentences for the experiments contain only particle errors. If just
the POS tags of words were different (i.e., surface forms were
identical) between the pairs, we did not regard them as errors, and
POS tags of the correct sentence were copied to the learner's
sentence. The number of incorrect particles was 1,087 (8.0\%) of
13,534. Note that most particles did not need to be revised. The
number of pair types of incorrect particles and their correct ones was
132 (SUB: 95, INS:14, DEL: 23). All sentences in the experiments were
segmented and tagged by the Japanese morphological analyzer
MeCab,\footnote{http://mecab.sourceforge.net/} using the ipadic-2.7.0
morpheme dictionary. The word information consisted of the surface
form and its POS tag.


\paragraph{Language Model: }

This was constructed from Japanese Wikipedia articles on computers and
Japanese Linux manuals, 527,151 sentences in total. SRILM
\cite{Stolcke:SRILM2011} was used to train a trigram
model. During model construction, the modified Kneser-Ney
discounting and interpolation were used for backoff smoothing, and
unknown unigrams were retained as pseudo-word \texttt{<unk>}.


\paragraph{Pseudo-error Corpus: }

The pseudo-errors were generated using 10,000 sentences randomly
selected from the corpus for the language model. We changed the
scaling factor of error generation probabilities from 0.0 (i.e., no
errors) to 2.0, while the relative frequency in the real error corpus
was taken as 1.0.


\paragraph{Evaluation Metrics: }

Five-fold cross-validation on the real error corpus was used. We used
two metrics as follows.


\begin{enumerate}
\item Precision and recall rates, and F-measures of error correction
by the systems. We compared only surface forms of the words for score
computation.

\item Most particles did not need to be revised in this task. The
system may excessively revise particles that do not need to be
corrected (over-correction). Therefore, we employ another metric
called relative improvement, which represents the difference between
the number of accurately corrected error particles and that of
over-corrected particles (i.e., those that did not need to be
corrected). This is a practical metric because it denotes the number
of particles that human rewriters do not need to revise after system
correction. The relative improvement becomes zero if the system does
not change any particle.
\end{enumerate}


\subsection{Experiment 1: Using Raw Text Corpora}
\label{sec-exp-raw-corpora}

First, we evaluate the effects of language model probability using raw
text corpora in error correction. The following three methods were
compared.

\begin{itemize}
\item \textbf{Proposed Method:}
Incorporating the language model probability as a link feature with
$n$-gram binary features.

\item \textbf{$N$-gram Binary Features:}
Only $n$-gram binary features are used as link features (i.e., the
language model probability is not used).

\item \textbf{Language Model Probability:}
Only the language model probability is used as the link feature (i.e.,
$n$-gram binary features are not used).
\end{itemize}

Table \ref{tbl-exp-results} shows the results.  In the table,
\mbox{\dag} denotes a significant difference between the proposed
method and the $n$-gram binary feature method, and \mbox{\S} denotes a
significant difference between the proposed method and the language
model probability method ($p < 0.05$).\footnote{For the precision and
recall rates, we used Pearson's $\chi^2$ test.  For the relative
improvement, we used Student's $t$-test.}

\begin{table}[b]
\caption{Results for Various Link Features}
\label{tbl-exp-results}
\input{03table04.txt}
\end{table}

Focusing on precision rates, the proposed and $n$-gram binary feature
methods yielded the same accuracies, while the language model
probability method yielded lower accuracy than the other
methods. Focusing on recall rates, the proposed method had
significantly better accuracy than the other two methods (from 9.9\%
and 11.2\% to 18.9\%). This indicated that the proposed method had the
highest F-measure. Comparing with the $n$-gram binary features and the
language model probability methods, the recall rate of the language
model probability method was slightly higher. Consequently, F-measure
scores, in decreasing order, were the proposed method, the language
model probability method, and the $n$-gram binary feature method.

However, focusing on relative improvements, the improvement of the
language model probability method was degraded (i.e., over-correction
occurred) even though the recall rate was higher than that of the
$n$-gram binary feature method. This is because of the characteristic
of this task; about 92\% of the particles did not need
modification. Improving the recall rate sometimes causes
over-correction.

The proposed method had better relative improvement than the other two
methods. However, the difference between the proposed and $n$-gram
binary feature methods was not significant. We suppose that the
$n$-gram binary features are especially effective in correcting
certain errors, and have an advantage from the perspective of relative
improvement. The proposed method improved the recall rate while
maintaining the precision rate, and thus increased relative
improvement. We conclude that combining the language model probability
with the $n$-gram binary features is effective in improving error
correction accuracy.


\subsection{Experiment 2: Expansion of Parallel Sentences Using
Pseudo-error Sentences}
\label{sec-exp-adaptation}

Next, we evaluate the effect of introducing pseudo-error
sentences. The experiments were carried out by changing the usage of
pseudo-error sentences; the link features are those proposed in
Section \ref{sec-exp-raw-corpora}.

\begin{figure}[b]
\begin{center}
\includegraphics{21-4ia3f4.eps}
\caption{Recall-Precision Curve (the scaling factor of error generation probabilities is 1.0)}
\label{fig-graph1}
\end{center}
\end{figure}

Figure \ref{fig-graph1} plots the precision/recall curves for the
following four combinations of training corpora and method. Note that
each precision rate in Figure \ref{fig-graph1} was, at the specified
recall rate, achieved by selecting corrected particles from high score
corrections.

\begin{itemize}
\item \textbf{TRG:}
The models were trained using only the real error corpus (baseline).

\item \textbf{SRC:}
The models were trained using only the pseudo-error corpus.

\item \textbf{ALL:}
The models were trained using the real error and pseudo-error corpora
by simply adding them.

\item \textbf{AUG:} 
The proposed method.  Feature augmentation was realized by regarding
the pseudo-errors as the source domain and the real errors as the
target domain.
\end{itemize}

The SRC case, which uses only the pseudo-error sentences, did not
match the precision of TRG. The ALL case matched the precision of TRG
at high recall rates. AUG, the proposed method, achieved higher
precision than TRG at high recall rates. At the recall rate of 18\%,
the precision rate of AUG was 55.4\%, while that of TRG was 50.5\%
(the significance was $p = 0.16$). Note that the precision rate of SRC
was 35.6\% at the recall rate of 18\%, which is better than random
correction.

\begin{figure}[t]
\begin{center}
\includegraphics{21-4ia3f5.eps}
\caption{Relative Improvement among Error Generation Probabilities}
\label{fig-graph2}
\end{center}
\end{figure}

Figure \ref{fig-graph2} shows the relative improvement of each method
according to error generation probabilities. In this experiment, ALL
achieved higher improvement than TRG at scaling factors of error
generation probabilities ranging from 0.0 (no errors) to 0.6. Although
the improvements were high, we have to control the error generation
probability because the improvements in the SRC case fell as the
scaling factor was raised. On the other hand, AUG achieved stable
improvement regardless of the error generation probability. The
relative improvement at a scaling factor of 1.0 was significantly
improved because the improvement of TRG was +28 and that of AUG was
+59 ($p < 0.05$). Thus, we can conclude that domain adaptation to
pseudo-error sentences is the preferred approach.


\subsection{Examples of Error Correction}

In the second experiment (c.f., Section \ref{sec-exp-adaptation}), the
precision and recall rates of the proposed method (AUG) were 54.8\%
(210/383) and 19.3\% (210/1,087), respectively, when the scaling factor
of error generation probabilities was 1.0. The precision rate of 55\%
was not high enough because 45\% of system modifications should be
corrected again. However, some usages were clearly incorrect from the
grammatical perspective, while others were acceptable because
alternative particles can be accepted depending on the
context. Accordingly, we carried out a subjective evaluation.

One hundred and seventy-three particles, which the system modified
without matching the answer, were evaluated by one evaluator. Note
that 151 of them were over-corrections, where the answers were not
modified from learners' sentences. We asked the evaluator whether the
system modification would be acceptable, i.e., grammatically and
semantically identical to the answer. As a result, 103 of the 173
particles were acceptable. Therefore, the precision rate including
acceptable modification became 81.7\% ($(210 + 103) / 383$).

Table \ref{tbl-correct-examples} shows examples of system
corrections. Successful corrections were realized by substitution,
insertion, and deletion. The acceptable examples included substitution
from topic particle \textit{wa} to case particle \textit{ga} (No. 4),
and dividing a compound noun by inserting the correct particle
(No. 5). Unacceptable examples included the following: idiomatic
expression was over-corrected (No. 7), the case particle for the
passive voice was replaced with that for the active voice (No. 8),
correction is impossible without knowledge of Linux's free command
(No. 10). No. 9 is similar to No. 4. However, it was judged
unacceptable because ``\textit{watashi-tachi} (we)'' and
``\textit{anata} (you)'' are concord. Therefore, the same particles
should be used. Features used in this paper only reflect local context
of particles. Uncorrectable particle errors will persist unless global
context features are introduced.

\begin{table}[t]
\caption{Examples of Error Correction by Proposed Method}
\label{tbl-correct-examples}
\input{03table05.txt}
\end{table}


\section{Related Studies}
\label{sec-related-work}

Particle error correction for Japanese learners has been researched
for a long time. Recently, \shortciteA{suzuki-toutanova:2006:COLACL}
used maximum entropy (ME) classifiers to restore particles (mainly
case markers) omitted from sentences. They inserted the appropriate
particle at the given position of the parsed and tagged
sentences. \shortciteA{Ohki:ParticleError2011j} detected incorrect
usage of particles in a learner's sentence. They tagged and parsed the
erroneous sentence and detected the incorrect usage (including lack of
particles) by the support vector machines (SVMs), which used the
features of neighboring words and dependency structure. Only detection
was carried out.


In English preposition/article correction, \shortciteA{HAN10.821}
corrected preposition errors using ME classifiers, where the features
were neighboring words and head words acquired by parsers. Similarly,
\shortciteA{gamon:2010:NAACLHLT} proposed a method to detect and
correct preposition and article errors. The detection was based on ME
classifiers, and the correction was based on decision
trees. \shortciteA{rozovskaya-roth:2010:EMNLP} corrected preposition
errors using classifiers based on averaged perceptron models. These
studies corrected errors by classifiers by restricting error types to
those of particles, prepositions, and articles.


However, \shortciteA{mizumoto-EtAl:2011:IJCNLP-2011} proposed an error
correction method that did not restrict the error type. They collected
numerous pairs of learners' sentences and their native correction from
a social network service, and corrected errors by using a phrase-based
SMT. Note that they did not segment learners' sentences into words
because they included errors. The erroneous sentences were segmented
into characters. Compared to our study, the sequence conversion used
in this paper is similar to that used in phrase-based SMT. Therefore,
our method can handle all error types, i.e., it does not need to
restrict errors to particles. However, our method assumes that the
input sentences are segmented into words in advance. The accuracies of
tagging and parsing might decrease when erroneous sentences are
analyzed. We need to investigate word segmentation methods for
erroneous sentences \cite{Fujino:ErrorMorphAnalysis2012j}.


From the perspective of correct sentence models,
\shortciteA{suzuki-toutanova:2006:COLACL,Ohki:ParticleError2011j,HAN10.821,rozovskaya-roth:2010:EMNLP}
modeled correct sentences using $n$-gram binary
features. \shortciteA{gamon:2010:NAACLHLT,mizumoto-EtAl:2011:IJCNLP-2011}
modeled them as language model probabilities. In this paper, we
combined both features and optimized them together with mapping
features. As a consequence, the recall rates were improved.


From the perspective of learners' sentence usage, all studies reported
that the error correction accuracies improved by incorporating models
that represent trends in learners' errors with models of native
speakers' texts
\cite{HAN10.821,gamon:2010:NAACLHLT,rozovskaya-roth:2010:EMNLP,Kasahara:CaseParticleCorrection2012j}. The
method in this paper encompasses the above findings by using mapping
features that represent learners' error tendencies.


\shortciteA{rozovskaya-roth:2010:NAACLHLT} have proposed the use of
pseudo-error sentences, which mimic learners' sentences.  They
reported that error correction accuracy was improved by adding
pseudo-error sentences, which have the same distribution as learners'
sentences. However, the best error generation method depended on the
data (concretely, the learner's native language), and error generation
should be manually controlled. In this paper, we adjust the difference
between pseudo-errors and real errors using the domain adaptation
technique; hence, stable improvement is achieved.


\shortciteA{dahlmeier-ng:2012:EMNLP-CoNLL} proposed a decoder that can
correct multiple errors in a sentence. The decoder handles not only
preposition/article errors but also typos, punctuation errors, and
number agreement errors. The correction hypotheses are generated using
classifiers and rules, which depend on the error type, and an input
sentence is rewritten by hill climbing. The decoder maintains multiple
hypotheses while decoding in order to reduce the risk of falling into
local optima. Our method maintains all hypotheses in the phrase
lattice and uses the Viterbi algorithm to search for the best
combination. The result is the optimal solution from the viewpoint of
the models.


We described how most particles do not need to be corrected, and that
improving the recall rate does not directly affect the relative
improvement in this task. This phenomenon, called the imbalanced data
problem, is a major problem when applying machine learning techniques
to practical tasks (c.f., the survey paper
\shortcite{He:Imbalanced2009}). Many solutions have been proposed to
address this problem. For example, sampling methods aim to balance
data by decreasing majority data or increasing minority data. The
minimum Bayes risk methods learn models with different costs,
depending on the majority and minority classification errors (in this
task, correction errors). We need to investigate what methods can be
applied to our task. Note that the pseudo-error sentences proposed in
this paper are different from the over-sampling method because our
purpose is to increase the training data without changing error
distribution.


\section{Conclusion}
\label{sec-conclusion}

In this paper, we propose a method to correct particle errors in the
sentences of Japanese learners. In the error correction task, it is
difficult to collect sufficient number of pairs of learners' and
correct sentences. To avoid this problem, we first combine the binary
features acquired from parallel sentences (small scale) and the
language model probability derived from large, raw text corpora. By
optimizing the above features via discriminative learning, we improve
the recall rate of error correction. In addition, we generate
pseudo-error sentences, which mimic learners' sentences and add them
to parallel sentences. By incorporating domain adaptation, stable
improvement is achieved.


Our discriminative sequence conversion can handle not only particle
errors but also all other error types. In the future, we aim to apply
this method to as many other error types as possible.


\acknowledgment

A part of this study was presented at \textit{the 50th Annual Meeting
of the Association for Computational Linguistics}
\cite{imamura-EtAl:2012:ACL2012short}. The authors are grateful
to Dr.~Tomoko Izumi, who provided advice on the English translation of
Japanese grammar. The authors appreciate helpful comments given by
anonymous reviewers.


\bibliographystyle{jnlpbbl_1.5}
\begin{thebibliography}{}

\bibitem[\protect\BCAY{Dahlmeier \BBA\ Ng}{Dahlmeier \BBA\
  Ng}{2012}]{dahlmeier-ng:2012:EMNLP-CoNLL}
Dahlmeier, D.\BBACOMMA\ \BBA\ Ng, H.~T. \BBOP 2012\BBCP.
\newblock \BBOQ A Beam-Search Decoder for Grammatical Error Correction.\BBCQ\
\newblock In {\Bem Proceedings of the 2012 Joint Conference on Empirical
  Methods in Natural Language Processing and Computational Natural Language
  Learning (EMNLP-CoNLL 2012)}, \mbox{\BPGS\ 568--578}, Jeju Island, Korea.

\bibitem[\protect\BCAY{Daume}{Daume}{2007}]{daumeiii:2007:ACLMain}
Daume, III, H. \BBOP 2007\BBCP.
\newblock \BBOQ Frustratingly Easy Domain Adaptation.\BBCQ\
\newblock In {\Bem Proceedings of the 45th Annual Meeting of the Association of
  Computational Linguistics (ACL-2007)}, \mbox{\BPGS\ 256--263}, Prague, Czech
  Republic.

\bibitem[\protect\BCAY{Fujino, Mizumoto, Komachi, Nagata, \BBA\
  Matsumoto}{Fujino et~al.}{2012}]{Fujino:ErrorMorphAnalysis2012j}
Fujino, T., Mizumoto, T., Komachi, M., Nagata, M., \BBA\ Matsumoto, Y. \BBOP
  2012\BBCP.
\newblock ``Word Segmentation toward Error Correction of Documents Written by
  Japanese
  Learners(日本語学習者の作文の誤り訂正に向けた単語分割)\inhibitglue''.\
\newblock In {\Bem Proceedings of the 18th Annual Meeting of the Association
  for Natural Language Processing}, \mbox{\BPGS\ 26--29}, Hiroshima, Japan.

\bibitem[\protect\BCAY{Gamon}{Gamon}{2010}]{gamon:2010:NAACLHLT}
Gamon, M. \BBOP 2010\BBCP.
\newblock \BBOQ Using Mostly Native Data to Correct Errors in Learners'
  Writing.\BBCQ\
\newblock In {\Bem Human Language Technologies: The 2010 Annual Conference of
  the North American Chapter of the Association for Computational Linguistics
  (HLT-ACL 2010)}, \mbox{\BPGS\ 163--171}, Los Angeles, California.

\bibitem[\protect\BCAY{Han, Tetreault, Lee, \BBA\ Ha}{Han
  et~al.}{2010}]{HAN10.821}
Han, N.-R., Tetreault, J., Lee, S.-H., \BBA\ Ha, J.-Y. \BBOP 2010\BBCP.
\newblock \BBOQ Using an Error-Annotated Learner Corpus to Develop an {ESL/EFL}
  Error Correction System.\BBCQ\
\newblock In {\Bem Proceedings of the Seventh International Conference on
  Language Resources and Evaluation (LREC'10)}, Valletta, Malta.

\bibitem[\protect\BCAY{He \BBA\ Garcia}{He \BBA\
  Garcia}{2009}]{He:Imbalanced2009}
He, H.\BBACOMMA\ \BBA\ Garcia, E.~A. \BBOP 2009\BBCP.
\newblock \BBOQ Learning from Imbalanced Data.\BBCQ\
\newblock {\Bem IEEE Transactions on Knowledge and Data Engineering}, {\Bbf 21}
   (9), \mbox{\BPGS\ 1263--1284}.

\bibitem[\protect\BCAY{Imamura, Izumi, Sadamitsu, Saito, Kobashikawa, \BBA\
  Masataki}{Imamura et~al.}{2011}]{Imamura:MorphTrans2011}
Imamura, K., Izumi, T., Sadamitsu, K., Saito, K., Kobashikawa, S., \BBA\
  Masataki, H. \BBOP 2011\BBCP.
\newblock \BBOQ Morpheme Conversion for Connecting Speech Recognizer and
  Language Analyzers in Unsegmented Languages.\BBCQ\
\newblock In {\Bem Proceedings of Interspeech 2011}, \mbox{\BPGS\ 1405--1408},
  Florence, Italy.

\bibitem[\protect\BCAY{Imamura, Saito, Sadamitsu, \BBA\ Nishikawa}{Imamura
  et~al.}{2012}]{imamura-EtAl:2012:ACL2012short}
Imamura, K., Saito, K., Sadamitsu, K., \BBA\ Nishikawa, H. \BBOP 2012\BBCP.
\newblock \BBOQ Grammar Error Correction Using Pseudo-Error Sentences and
  Domain Adaptation.\BBCQ\
\newblock In {\Bem Proceedings of the 50th Annual Meeting of the Association
  for Computational Linguistics (ACL-2012) Volume 2: Short Papers},
  \mbox{\BPGS\ 388--392}, Jeju Island, Korea.

\bibitem[\protect\BCAY{Kasahara, Fujino, Komachi, Nagata, \BBA\
  Matsumoto}{Kasahara et~al.}{2012}]{Kasahara:CaseParticleCorrection2012j}
Kasahara, S., Fujino, T., Komachi, M., Nagata, M., \BBA\ Matsumoto, Y. \BBOP
  2012\BBCP.
\newblock ``Case Particle Correction that Reflects Error Tendency of Japanese
  Learners(日本語学習者の誤り傾向を反映した格助詞訂正)\inhibitglue''.\
\newblock In {\Bem Proceedings of the 18th Annual Meeting of the Association
  for Natural Language Processing}, \mbox{\BPGS\ 14--17}, Hiroshima, Japan.

\bibitem[\protect\BCAY{Lafferty, McCallum, \BBA\ Pereira}{Lafferty
  et~al.}{2001}]{Lafferty:CRF2001}
Lafferty, J., McCallum, A., \BBA\ Pereira, F. \BBOP 2001\BBCP.
\newblock \BBOQ Conditional Random Fields: Probabilistic Models for Segmenting
  and Labeling Sequence Data.\BBCQ\
\newblock In {\Bem Proceedings of the 18th \mbox{International} Conference on
  Machine Learning (ICML-2001)}, \mbox{\BPGS\ 282--289}, Williamstown,
  \mbox{Massachusetts}.

\bibitem[\protect\BCAY{Mizumoto, Komachi, Nagata, \BBA\ Matsumoto}{Mizumoto
  et~al.}{2011}]{mizumoto-EtAl:2011:IJCNLP-2011}
Mizumoto, T., Komachi, M., Nagata, M., \BBA\ Matsumoto, Y. \BBOP 2011\BBCP.
\newblock \BBOQ Mining Revision Log of Language Learning {SNS} for Automated
  {Japanese} Error Correction of Second Language Learners.\BBCQ\
\newblock In {\Bem Proceedings of 5th International Joint Conference on Natural
  Language \mbox{Processing} (IJCNLP-2011)}, \mbox{\BPGS\ 147--155}, Chiang
  Mai, Thailand.

\bibitem[\protect\BCAY{Ohki, Oyama, Kitauchi, Suenaga, \BBA\ Matsumoto}{Ohki
  et~al.}{2011}]{Ohki:ParticleError2011j}
Ohki, M., Oyama, H., Kitauchi, A., Suenaga, T., \BBA\ Matsumoto, Y. \BBOP
  2011\BBCP.
\newblock ``Judgement of Incorrect Usage of Particles in Documents for System
  Development Written by Non-natvie Japanese
  Speakers(非日本語母国語話者の作成するシステム開発文書を対象とした助詞の誤用判定)\inhibitglue''.\
\newblock In {\Bem Proceedings of the 17th Annual Meeting of the Association
  for Natural Language Processing}, \mbox{\BPGS\ 1047--1050}, Toyohashi, Japan.

\bibitem[\protect\BCAY{Rozovskaya \BBA\ Roth}{Rozovskaya \BBA\
  Roth}{2010a}]{rozovskaya-roth:2010:EMNLP}
Rozovskaya, A.\BBACOMMA\ \BBA\ Roth, D. \BBOP 2010a\BBCP.
\newblock \BBOQ Generating Confusion Sets for Context-Sensitive \mbox{Error
  Correction}.\BBCQ\
\newblock In {\Bem Proceedings of the 2010 Conference on Empirical Methods in
  \mbox{Natural} Language Processing (EMNLP-2010)}, \mbox{\BPGS\ 961--970},
  Cambridge, Massachusetts.

\bibitem[\protect\BCAY{Rozovskaya \BBA\ Roth}{Rozovskaya \BBA\
  Roth}{2010b}]{rozovskaya-roth:2010:NAACLHLT}
Rozovskaya, A.\BBACOMMA\ \BBA\ Roth, D. \BBOP 2010b\BBCP.
\newblock \BBOQ Training Paradigms for Correcting Errors in Grammar and
  Usage.\BBCQ\
\newblock In {\Bem Human Language Technologies: The 2010 Annual Conference of
  the North American Chapter of the Association for Computational Linguistics
  (HLT-ACL 2010)}, \mbox{\BPGS\ 154--162}, Los Angeles, California.

\bibitem[\protect\BCAY{Rozovskaya \BBA\ Roth}{Rozovskaya \BBA\
  Roth}{2011}]{rozovskaya-roth:2011:ACL-HLT2011}
Rozovskaya, A.\BBACOMMA\ \BBA\ Roth, D. \BBOP 2011\BBCP.
\newblock \BBOQ Algorithm Selection and Model Adaptation for {ESL} Correction
  Tasks.\BBCQ\
\newblock In {\Bem Proceedings of the 49th Annual Meeting of the Association
  for Computational Linguistics: Human Language Techologies (ACL-HLT 2011)},
  \mbox{\BPGS\ 924--933}, Portland, Oregon.

\bibitem[\protect\BCAY{Stolcke, Zheng, Wang, \BBA\ Abrash}{Stolcke
  et~al.}{2011}]{Stolcke:SRILM2011}
Stolcke, A., Zheng, J., Wang, W., \BBA\ Abrash, V. \BBOP 2011\BBCP.
\newblock \BBOQ {SRILM} at Sixteen: Update and Outlook.\BBCQ\
\newblock In {\Bem Proceedings of IEEE Automatic Speech Recognition and
  Understanding Workshop (ASRU 2011)}, Waikoloa, Hawaii.

\bibitem[\protect\BCAY{Suenaga \BBA\ Matsushima}{Suenaga \BBA\
  Matsushima}{2012}]{Suenaga:ErrorCorrection2012j}
Suenaga, T.\BBACOMMA\ \BBA\ Matsushima, T. \BBOP 2012\BBCP.
\newblock ``Optimimal Prediction Method Using Hierarchical $N$-gram Models
  Based on Bayes Decision Theory and its Application to Japanese Input
  Support(ベイズ決定理論にもとづく階層Nグラムを用いた最適予測法と日本語入力支援技術への応用)\inhibitglue''.\
\newblock In {\Bem Proceedings of the 18th Annual Meeting of the Association
  for Natural Language Processing}, \mbox{\BPGS\ 6--9}, Hiroshima, Japan.

\bibitem[\protect\BCAY{Suzuki \BBA\ Toutanova}{Suzuki \BBA\
  Toutanova}{2006}]{suzuki-toutanova:2006:COLACL}
Suzuki, H.\BBACOMMA\ \BBA\ Toutanova, K. \BBOP 2006\BBCP.
\newblock \BBOQ Learning to Predict Case Markers in Japanese.\BBCQ\
\newblock In {\Bem Proceedings of the 21st International Conference on
  Computational Linguistics and 44th Annual Meeting of the Association for
  Computational Linguistics (COLING-ACL 2006)}, \mbox{\BPGS\ 1049--1056},
  Sydney, Australia.

\bibitem[\protect\BCAY{Suzuki \BBA\ Isozaki}{Suzuki \BBA\
  Isozaki}{2010}]{Suzuki:SemiSupervised2010j}
Suzuki, J.\BBACOMMA\ \BBA\ Isozaki, H. \BBOP 2010\BBCP.
\newblock ``Performance Validation of Dependency Parsing Using Large Unlabelled
  Data(大規模ラベルなしデータを利用した係り受け解析の性能検証)\inhibitglue''.\
\newblock In {\Bem Proceedings of the 16th Annual Meeting of the Association
  for Natural Language Processing}, \mbox{\BPGS\ 19--22}, Tokyo, Japan.

\bibitem[\protect\BCAY{Suzuki, Isozaki, Carreras, \BBA\ Collins}{Suzuki
  et~al.}{2009}]{suzuki-EtAl:2009:EMNLP}
Suzuki, J., Isozaki, H., Carreras, X., \BBA\ Collins, M. \BBOP 2009\BBCP.
\newblock \BBOQ An Empirical Study of Semi-supervised Structured Conditional
  Models for Dependency Parsing.\BBCQ\
\newblock In {\Bem Proceedings of the 2009 Conference on Empirical Methods in
  Natural Language Processing (EMNLP-2009)}, \mbox{\BPGS\ 551--560}, Singapore.

\end{thebibliography}


\vspace{-1\Cvs}
\begin{biography}

\bioauthor[:]{Kenji Imamura}{
graduated from Chiba University, Chiba, Japan, in 1985 and received
the Ph.D. degree in engineering from the Nara Institute of Science and
Technology, Nara, Japan, in 2004.  He joined the Nippon Telegraph and
Telephone Corporation (NTT) in 1985.  He was with the NTT Software
Corporation from 1995 to 1998 and the Advanced Telecommunication
Research Institute International (ATR), Kyoto, Japan, from 2000 to
2006.  He is currently with the NTT Media Intelligence Laboratories,
Yokosuka, Japan, as a Senior Research Scientist.  His research
interests include machine translation, parsing, and predicate-argument
structure analysis.  He is a member of the ACL, the IEICE, the IPSJ,
and the NLP. }

\bioauthor[:]{Kuniko Saito}{
Senior Manager, Innovative Architecture Center, NTT Communications
Corporation.  She received the B.S. and M.S. degrees in chemistry from
the University of Tokyo in 1996 and 1998, respectively. She joined NTT
Information and Communication Systems Laboratories in 1998 and then
moved to NTT Cyber Space Laboratories (now NTT Media Intelligence
Laboratories).  Her research focuses on part-of-speech tagging, named
entity recognition, and term extraction. She is a member of the IPSJ
and the NLP. }

\bioauthor[:]{Kugatsu Sadamitsu}{
Researcher, Audio, Speech, and Language Media Project, NTT Media
Intelligence Laboratories.  He received the B.E., M.E., and
Ph.D. degrees in engineering from Tsukuba University, Ibaraki, in
2004, 2006, and 2009, respectively. He joined NTT Cyber Space
Laboratories (now NTT Media Intelligence Laboratories) in 2009. His
current research interests include NLP and machine learning. He is a
member of the Association for Natural Language Processing and the
Information Processing Society of Japan (IPSJ). }

\bioauthor[:]{Hitoshi Nishikawa}{
received B.A. (Policy Management) and M.M.G. from the Keio University
in 2006 and 2008, respectively. He received his Dr.~Eng.\ from Nara
Institute of Science and Technology in 2013. He is currently a
researcher at the NTT Media Intelligence Laboratories. His research
interests include NLP, especially automatic summarization. He is a
member of the Association for Computational Linguistics, the
Association for Natural Language Processing, the Japanese Society for
Artificial Intelligence and the Information Processing Society of
Japan. }
\end{biography}

\biodate


\end{document}
