    \documentclass[english]{jnlp_1.4_rep}

\usepackage{jnlpbbl_1.3}
\usepackage[dvipdfm]{graphicx}
\usepackage{udline}
\setulminsep{1.2ex}{0.2ex}
\usepackage{amsmath}

\usepackage{gb4e}
\noautomath
\newcommand{\sm}[1]{}
\newcommand{\nom}{}
\newcommand{\acc}{}
\newcommand{\dat}{}
\newcommand{\abl}{}
\newcommand{\gen}{}
\newcommand{\loc}{}
\newcommand{\cmi}{}
\newcommand{\enu}{}
\newcommand{\cnj}{}
\newcommand{\all}{}
\newcommand{\TOP}{}
\newcommand{\TM}{}


\Volume{21}
\Number{4}
\Month{September}
\Year{2014}

\received{2004}{8}{30}
\revised{2004}{12}{15}
\accepted{2005}{1}{15}

\setcounter{page}{679}

\etitle{Gradual Fertilization of Case Frames\footnotetext{\llap{*~}This article has been partially revised for better understanding of overseas readers.}}
\eauthor{Daisuke Kawahara\affiref{Author_1} \and Sadao Kurohashi\affiref{Author_1}} 
\eabstract{
 This article proposes an automatic method of gradually constructing case
 frames. First, a large raw corpus is parsed, and base case frames
 are constructed from reliable predicate-argument examples in the
 parsing results. Second, case analysis based on the base case frames
 is applied to the large corpus, and the case frames are upgraded by
 incorporating newly acquired information. Case frames are gradually
 fertilized in this way. We constructed case frames from 26 years of newspaper
 articles consisting of approximately 26 million sentences. The case frames are
 evaluated manually as well as through syntactic and case
 analyses. These results presented the effectiveness of the constructed
 case frames.
}
\ekeywords{case frame, corpus, case structure analysis}

\headauthor{Kawahara and Kurohashi}
\headtitle{Gradual Fertilization of Case Frames}

\affilabel{Author_1}{}{Graduate School of Information Science and Technology, The University of Tokyo}

\Reprint[T]{Vol.~12, No.~2, pp.~109--131}

\begin{document}

\maketitle

\section{Introduction}

In order to allow computers to understand natural language text, it is
necessary at least to capture the relationships between words in
text. Analyzing such relationships requires wide-coverage knowledge
similar to the intrinsic linguistic knowledge shared by people.

One area of such knowledge is \textbf{case frames}. Case frames represent the
relationships between a predicate and its arguments. For instance, the
following example is a case frame for the Japanese verb
``\textit{tsumu}'' (load):\footnote{In this article, we use the
following abbreviations: \nom~(nominative), \acc~(accusative),
\dat~(dative), \loc~(locative), \abl~(ablative), \gen~(genitive), and \TOP~(topic marker).}
\begin{exe}
\ex \label{Example::CaseFrame1}
 \gll {\{\textit{jugyouin, untenshu, $\cdots$}\}-\textit{ga}} {\{\textit{kuruma, torakku, $\cdots$}\}-\textit{ni}} {\{\textit{nimotsu, busshi}\}-\textit{wo}} {\textit{tsumu}} \\
      {\{employee, driver, $\cdots$\}-\nom} {\{car, truck, $\cdots$\}-\dat} {\{baggage, supply\}-\acc} {load} \\
\end{exe}

We apply dependency parsing to a large raw corpus, extract
predicate-argument structures without syntactic ambiguities from the
parses, and then cluster them to automatically produce case frames \cite{Kawahara2001}.
The biggest problem with case frame compilation is the semantic ambiguity of predicates. For example, for
the verb ``\textit{tsumu},'' it is necessary to discriminate between the
expressions with different meanings, such as ``\textit{nimotsu-wo tsumu}'' (load
baggage) and ``\textit{keiken-wo tsumu}'' (accumulate experience).

Our method solves this problem by pairing a predicate with its
closest argument. For example, the following case frame (\ref{Example::CaseFrame2}) is acquired as a different one from case frame
(\ref{Example::CaseFrame1}) by handling expressions of
``\textit{tsumu}'' with the pairs ``\textit{nimotsu-wo tsumu}'' (load
baggage), ``\textit{busshi-wo tsumu}'' (load supply),
``\textit{keiken-wo tsumu}'' (accumulate experience) and
``\textit{taiken-wo tsumu}'' (accumulate experience).
\begin{exe}
\ex \label{Example::CaseFrame2}
 \gll {\{\textit{senshu, jugyouin, $\cdots$}\}-\textit{ga}} {\{\textit{keiken, taiken}\}-\textit{wo}} {\textit{tsumu}} \\
      {\{player, employee, $\cdots$\}-\nom} {\{experience, experience\}-\acc} {accumulate} \\
\end{exe}

The above method only uses instances without syntactic ambiguities, and
thus basically 
\linebreak
collects arguments with case-marking
postpositions. Therefore, acquired case frames cannot handle the
complex expressions underlined in the following sentences.
\begin{exe}
\ex \label{例::象は鼻が長い}
 \begin{xlist}
 \ex
  \gll {\textit{\underline{zou-wa}}} {\textit{hana-ga}} {\textit{nagai}} \\
       {elephant-\TOP} {nose-\nom} {long} \\
  \trans (As for an elephant, its nose is long.)
 \ex
  \gll {\textit{sakana-wo}} {\textit{yaku}} {\textit{\underline{kemuri}}} \\
       {fish-\acc} {broil} {smoke} \\
  \trans (smoke that arises when (someone) broils fish)
 \ex
  \gll {\textit{\underline{jimintou-no}}} {\textit{shiji-wo}} {\textit{eru}} \\
       {Liberal Democratic Party-\gen} {support-\acc} {obtain} \\
  \trans ((someone) obtained Liberal Democratic Party's support.)
 \end{xlist}
\end{exe}
Sentence (\ref{例::象は鼻が長い}a) has a double nominative
construction and the underlined part is the second nominative. The word
``\textit{kemuri}'' (smoke) in Sentence (\ref{例::象は鼻が長い}b) does
not have a directly case-marked relationship, such as nominative and
accusative, to ``\textit{yaku}'' (broil), but rather has an
indirect relationship called a ``non-gapping relationship.'' Sentence (\ref{例::象は鼻が長い}c)
can be interpreted as ``\textit{jimintou-kara shiji-wo eru}'' (obtain a support from
the Liberal Democratic Party) by human, where the postpositions
``\textit{no}'' (\gen) and ``\textit{kara}'' (\abl) are
interchangeable (case alternation between ``\textit{no}'' and
``\textit{kara}'').

This article describes a method for automatically compiling case frames by
gradually acquiring reliable information for analysis of the
above phenomena. We first apply dependency parsing to a corpus, extract
reliable predicate-argument structures, and cluster them to produce
base case frames. We then apply case structure analysis based on the
base case frames and acquire new reliable information to fertilize
the case frames. The new information extracted by the case structure
analysis includes information about double nominative constructions
and non-gapping relationships. To address the issue of case alternation,
we merge case slots according to the similarities between case slots in
a case frame.


\section{Target Expressions}
\label{章::日本語表現}

This section describes expressions problematic for case frame
compilation and case structure analysis.

In Japanese, postpositions such as ``\textit{ga}'' (\nom), ``\textit{wo}'' (\acc) and
``\textit{ni}'' (\dat) represent relationships between predicates and their
arguments. However, the following expressions obscure such relationships:


\noindent
\underline{\textbf{Topic-marked phrases}} \ (phrases with a postposition
``\textit{wa}'' or ``\textit{mo}''; hereafter TM phrases)

\begin{exe}
\ex
 \begin{xlist}
 \ex
  \gll {\textit{\underline{kuruma\vphantom{y}}-wa}} {\textit{\underline{\underline{hayai}}}} {\ \ \ \ (\textit{ga})} \\
       {car-\TOP} {fast} \\
  \trans (Cars are fast.)
 \ex
  \gll {\textit{\underline{hon\vphantom{y}}-mo}} {\textit{\underline{\underline{yonda}}}} {\ \ \ \ (\textit{wo})} \\
       {book-also} {read} \\
  \trans ((someone) read also a book.)
 \end{xlist}
\end{exe}

\noindent
\underline{\textbf{Clausal modifiees}} \ (phrases modified by a clause)

\begin{exe}
\ex
 \begin{xlist}
 \ex
  \gll {\textit{\underline{\underline{hayai}}}} {\textit{\underline{kuruma\vphantom{y}}}} {\ \ \ \ (\textit{ga})} \\
       {fast} {car} \\
  \trans (a fast car)
 \ex
  \gll {\textit{\underline{\underline{yonda}}}} {\textit{\underline{hon\vphantom{y}}}} {\ \ \ \ (\textit{wo})} \\
       {read} {book} \\
  \trans (the book that (someone) read)
 \end{xlist}
\end{exe}
A relation that a predicate (doubly underlined) has with the underlined expression is
indicated in the right parenthesis.


The TM phrases and clausal modifiees in the following
sentences do not have simple case markers, such as ``\textit{ga}'' and
``\textit{wo}.''


\noindent
\underline{\textbf{Double nominative constructions}}

\noindent
These constructions contain a TM phrase, a phrase with
the postposition ``\textit{ga}'' and a predicate, in that order. The
TM phrase is interpreted as a ``\textit{ga}'' argument (as will be
described below), so the predicate actually has two ``\textit{ga}'' arguments:

\begin{exe}
\ex \label{Example::DoubleNominative}
 \gll {\textit{kono}} {\textit{\underline{kuruma}-wa}} {\textit{enjin-ga}} {\textit{\underline{\underline{yoi}}}} \\
      {this} {car-\TOP} {engine-\nom} {good} \\
 \trans (As for this car, its engine is good.)
\end{exe}

\vspace{-3pt}
In this sentence, the case of ``\textit{kuruma}'' (car) is
``\textit{ga}'' because we can say ``\textit{kuruma-ga yoi}'' (car is
good). Therefore, ``\textit{yoi}'' (good) takes two ``\textit{ga}''
arguments, ``\textit{kuruma}'' and ``\textit{enjin}'' (engine), and
``\textit{kuruma}'' is an outer nominative.


\noindent
\underline{\textbf{Non-gapping relations}}

\noindent
This is the case where the relation between a clausal modifiee and a
predicate cannot be expressed by surface case marking.

\vspace{-3pt}
\begin{exe}
\ex
 \gll {\textit{sakana-wo}} {\textit{\underline{\underline{yaku}}}} {\textit{\underline{kemuri\vphantom{y}}}} \\
      {fish-\acc} {broil} {smoke} \\
 \trans (smoke that arises when (someone) broils fish)
\end{exe}
\vspace{-3pt}

Since ``\textit{kemuri}'' (smoke) in this sentence means smoke that
arises when broiling fish, the relation between ``\textit{kemuri}''
(smoke) and ``\textit{yaku}'' (broil) cannot be expressed by any
case-marking postpositions.


\noindent
\underline{\textbf{Case alternation}}

\noindent
Various cases are used to express the same meaning.

\vspace{-3pt}
\begin{exe}
\ex
 \begin{xlist}
 \ex
  \gll {\textit{syakaitou-ga}} {\textit{\underline{shinshintou}-no}} {\textit{shiji-wo}} {\textit{\underline{\underline{eru}}}} \\
       {Social Democratic Party-\nom} {New Frontier Party-\gen} {support} {obtain} \\
  \trans (Social Democratic Party obtained the New Frontier Party's support.)
 \ex
  \gll {\textit{syakaitou-ga}} {\textit{\underline{shinshintou}-kara}} {\textit{shiji-wo}} {\textit{\underline{\underline{eru}}}} \\
       {Social Democratic Party-\nom} {New Frontier Party-\abl} {support} {obtain} \\
  \trans (Social Democratic Party obtained a support of the New Frontier Party.)
 \end{xlist}
\end{exe}
\vspace{-3pt}

In the second sentence, ``\textit{shinshintou}'' (New Frontier Party) has a
``\textit{kara}'' (\abl) relation to ``\textit{eru}'' (obtain). A
``\textit{no}'' (\gen) postposition is used to express the same meaning
in the first sentence, and this phenomenon is called a case alternation
between ``\textit{no}'' and ``\textit{kara}.''

\vspace{-3pt}
\begin{exe}
\ex
 \begin{xlist}
 \ex
  \gll {\textit{kono}} {\textit{\underline{kuruma}-no}} {\textit{enjin-ga}} {\textit{\underline{\underline{yoi}}}} \\
       {this} {car-\gen} {engine-\nom} {good} \\
  \trans (This car's engine is good.)
 \ex
  \gll {\textit{kono}} {\textit{\underline{kuruma}-wa}} {\textit{enjin-ga}} {\textit{\underline{\underline{yoi}}}} \\
       {this} {car-\TOP} {engine-\nom} {good} \\
  \trans (As for this car, its engine is good.)
 \end{xlist}
\end{exe}

We can express the same meaning as an outer nominative phrase, similar to
example (\ref{Example::DoubleNominative}), by using a \textit{no} phrase in many cases.
This is a case alternation between ``\textit{no}'' (\gen) and outer nominative.

In this work, we treat a noun phrase marked with 
``\textit{no}'' that depends on the closest argument as a kind of argument
of a predicate. Such a noun phrase does not depend on a predicate but
has a strong semantic relationship with a predicate and sometimes causes 
case alternation. It is also located at a distance of one from
the predicate-argument pair in the same way as other
arguments. Thus, many important phrases occur in this position.


\section{Compilation of Base Case Frames and Case Structure Analysis using Base Case Frames}

We automatically compiled case frames from a large raw corpus by
coupling a predicate and its closest argument \cite{Kawahara2001}. We
call these \textbf{base case frames}. This section first
outlines the compilation method for the base case
frames and then explains a method for case structure analysis based on
these case frames.


\subsection{Compilation of Base Case Frames}
\label{Section::1次格フレーム辞書の構築}

The biggest problem in automatic case frame compilation is predicate
sense ambiguity. Predi\-cates with different meanings should have
different case frames, but it is hard to disambiguate predicate senses
very precisely. To address this problem, we distinguish
predicate-argument structures, which are collected from a large corpus,
by coupling a predicate and its closest argument. That is,
predicate-argument structures are not distinguished by single predicates
such as ``\textit{naru}'' (make/become) and ``\textit{tsumu}''
(load/accumulate), but by pairs such as ``\textit{tomodachi-ni naru}'' (make a
friend), ``\textit{byoki-ni naru}'' (become sick), ``\textit{nimotsu-wo tsumu}'' (load baggage),
and ``\textit{keiken-wo tsumu}'' (accumulate experience).

This process creates separate case frames that have almost the same
meaning or usage. For example, ``\textit{nimotsu-wo tsumu}'' (load
baggage) and ``\textit{busshi-wo tsumu}'' (load supply) are separate
case frames. To merge these similar case frames and increase coverage of
the case frame, we cluster them.

We employ the following procedure for the automatic case frame compilation:
\begin{enumerate}
 \item A large raw corpus is parsed by a Japanese parser, KNP \cite{Kurohashi1994}, and reliable
       predicate-argument structures are extracted from the parses.
       \label{1次格フレーム::構文的曖昧性のない述語項構造抽出}
 \item The extracted predicate-argument structures are bundled according
       to a predicate and its closest argument to generate initial case
       frames. Arguments with a topic marker, such as ``\textit{wa}'' or ``\textit{mo}'' and
       clausal modifiees are discarded, because their case markers
       cannot be interpreted by syntactic analysis.
       \label{1次格フレーム::用言とその直前の格要素でまとめる}
 \item The initial case frames are clustered using a similarity
       measure shown in the Appendix to produce the base case frames. The similarity is
       calculated by using a thesaurus.
\end{enumerate}
Figure \ref{図::1次格フレーム構築} shows the procedure for compiling case
frames for the verb ``\textit{tsumu}'' (load/accumulate). Reliable
predicate-argument structures in the first step (\ref{1次格フレーム::構文的曖昧性のない述語項構造抽出}) are extracted from phrases that have only
one candidate head according to the rules of KNP.
For example, the underlined parts are extracted from the following sentences as reliable predicate-argument structures.
\begin{exe}
\ex \label{例::構文的曖昧性のない述語項構造}
 \begin{xlist}
 \ex
  \gll {\textit{ookuno}} {\textit{\underline{rensyu-wo}}} {\textit{\underline{tsunda-ga}},} {\textit{yosen-wa}} $\cdots$ \\
       {many} {practice-\acc} {accumulate} {preliminary match-\TOP} \\
  \trans ((someone) accumulated many practices, but $\cdots$)
 \ex
  \gll {\textit{\underline{nimotsu-wo}}} {\textit{\underline{tsunda}}} {\textit{\underline{torakku}-ga}} {\textit{douro-wo}} {\textit{hashiri}} $\cdots$ \\
       {baggage-\acc} {loaded} {truck-\nom} {road-\acc} {run} \\
  \trans (a truck that loaded baggage runs $\cdots$)
 \end{xlist}
\end{exe}
In example (\ref{例::構文的曖昧性のない述語項構造}a), the conjunction ``-\textit{ga}'' (but) is
recognized as a strong boundary and ``\textit{tsunda}'' (loaded) is the only candidate
head of ``\textit{rensyu-wo}'' (practice-\acc).
In example (\ref{例::構文的曖昧性のない述語項構造}b),
``\textit{torakku-ga}'' (truck-\nom) is the only candidate head of the
clausal modifier ``\textit{nimotsu-wo tsunda}'' and thus is extracted.

\begin{figure}[b]
  \begin{center}
  \includegraphics{21-4ia9f1.eps}
 \end{center}
  \caption{Compilation of base case frames}
  \label{図::1次格フレーム構築}
\end{figure}

We evaluated the extraction process for syntactically unambiguous
predicate-argument 
\linebreak
structures using the Kyoto University Text Corpus
\cite{Kurohashi1998}. While the accuracy of dependencies in all the
predicate-argument structures was 90.9\%, that of extracted
predicate-argument structures was 98.3\%. The coverage of extracted
predicate-argument 
\linebreak
structures against all the predicate-argument
structures was 20.7\%. Major errors were caused by the failed recognition of
coordinate structures and copulas.

We compiled base case frames from a newspaper corpus comprising a volume
of 26 years (12 years of Mainichi Shimbun and 14 years of Yomiuri Shimbun).
These case frames consisted of approximately 18,000 predicates. The
number of average case frames per a predicate was approximately 17.9.


\subsection{Case Structure Analysis using Base Case Frames}
\label{章::格解析}

This section describes a method for case structure analysis using the
base case frames. The basic process of case structure analysis is to
align arguments in an input sentence with case frames. This process is
performed simultaneously with dependency parsing. That is, case
structure analysis is applied to each possible dependency structure to
assess the appropriateness of the dependency structure based on the
score of case structure analysis. Finally, the dependency and case
structure with the highest score is generated as output \cite{Kuro-IEICE1994}.

In this study, case structure analysis is applied to each predicate in an input
sentence. The analysis consists of two steps: case frame selection and matching
input arguments with case slots in the selected case frame. The
rest of this subsection describes these steps in detail.


\subsubsection*{Case frame selection}

Since each predicate has multiple case frames, it is necessary to select
the case frame corresponding to the usage of an input sentence. As
stated in Section \ref{Section::1次格フレーム辞書の構築}, the closest
argument plays an important role in determining the usage of a
predicate.  In particular, this tendency is strong when the closest case
is ``\textit{wo}'' (\acc) or ``\textit{ni}'' (\dat). On the other hand, an expression
``\sm{agent}\footnote{In this article, we represent a semantic marker in
a thesaurus as \sm{}.} \textit{ga motomeru}'' (\sm{agent} requires), for example,
does not provide enough cues to determine the usage of the predicate and
select an appropriate case frame. By considering these factors, we
impose the following conditions on the selection of a corresponding case
frame:

\begin{description}
 \item[Conditions for input expressions] The input predicate has the
	    closest argument $C$, and $C$ and the closest case
	    $cm$ meet one of the following conditions:
	    \label{格フレーム選択アルゴリズム::直前条件}
	    \begin{itemize}
	     \item $cm$ is ``\textit{wo}'' or ``\textit{ni}.''
	     \item $C$ does not belong to the semantic marker \sm{agent}.
	    \end{itemize}
 \item[Conditions of case frames] A case frame that has $cm$ and has
	    examples in $cm$ with similarities to $C$ over a
	    predetermined threshold is
	    selected. If there is more than one such case frame, the
	    case frame that has the highest level of similarity is selected.
\end{description}


``Similarity'' for the purpose of this process is defined as the highest
similarity value between the closest argument and each case example in the case
slot. The similarity between the two examples is calculated by equation
(\ref{Formula::NttSimilarity}) in the Appendix. The similarity threshold is
empirically set to 0.70.

If there is no case frame that satisfies the above conditions, all the
case frames of the input verb are processed by the following matching
step, and the best-scored case frame is selected.


\subsubsection*{Matching input arguments with case slots in the
selected case frame}
\label{Section::MatchingInputWIthCaseFrames}

We match arguments of the target predicate with case slots in the
selected case frame. When an argument has a case marker, it must be
assigned to the case slot with the same case marker. When an argument
is a TM phrase or a clausal modifiee, which does not
have a case marker, it can be assigned to one of the case slots in the
following table.
\vspace{1\Cvs}

\begin{center}
 \begin{tabular}{@{ }l@{ : }l}
\hline
  TM phrases & \textit{ga}, \textit{wo} \\
  clausal modifiees    & \textit{ga}, \textit{wo}, \textit{ni} \\
\hline
 \end{tabular}
\end{center}
\vspace{1\Cvs}

\noindent
The conditions above may produce multiple matching patterns. In this
case, the one with the best score is selected. The score of a matching
pattern is defined as the sum of the scores of case assignments. The score
of case assignments is the best similarity between the input argument
and examples in the case slots. This similarity measuring
function is calculated using the Japanese thesaurus as described in the subsection
``Case Frame Selection.''

Consider the following examples of case structure analysis.
\begin{exe}
 \ex
 \gll \textit{syorui-wa} \textit{gyousya-ni} \textit{watashita} \\
      {document-\TOP} {company-\dat} {hand in} \\
 \trans ((someone) handed in the document to the company)
\end{exe}
For this example, the following case frame is selected.\footnote{``CS'' denotes a case slot. The CS
with ``*'' mark means that it is the closest case.}
\vspace{1\Cvs}

\begin{center}
 \begin{tabular}{c|l|l|l}
\hline
  & \multicolumn{1}{c|}{CS}   & \multicolumn{1}{c|}{examples in English} & \multicolumn{1}{c}{input} \\
\hline
  & \textit{ga} & boss, wife, staff, clerk & --- \\
\raisebox{0pt}[0pt][0pt]{\begin{tabular}{@{}c@{}}\textit{watasu}\\(hand in)\end{tabular}} 
	& \textit{wo}   & \sm{quantity}yen, copy, document, $\cdots$ & \textit{syorui} (document) \\
  & \textit{ni}*  & dealer, company, firm, bank & \textit{gyousya} (company) \\
\hline
\end{tabular}
\end{center}
\vspace{1\Cvs}

\noindent
In this example, ``\textit{gyousya-ni}'' ({company-\dat}) matches the ``\textit{ni}''
case in the case frame. As for ``\textit{syorui}'' (document), which is a TM
phrase, its corresponding case slot is determined to be the \textit{wo} case
slot, because it has a higher value of similarities to the ``\textit{wo}'' case examples 
than to the ``\textit{ga}'' case examples.

\begin{exe}
 \ex
 \gll \textit{doitsugo-mo} \textit{hanaseru} \textit{sensei} \\
      {German-\TOP} {can speak} teacher \\
 \trans (the teacher who can speak German)
\end{exe}

This example does not satisfy the conditions for case frame
selection, and so the matching process is performed for each case
frame of ``\textit{hanasu}'' (speak). As a result, the following case
frame, which has the highest total score, and the case correspondences
are adopted.
\vspace{1\Cvs}

\begin{center}
\begin{tabular}{l|l|l|l}
\hline
 & \multicolumn{1}{c|}{CS} & \multicolumn{1}{c|}{examples in English} & \multicolumn{1}{c}{input} \\
\hline
\raisebox{-8pt}[0pt][0pt]{\begin{tabular}{@{}c@{}}\textit{hanaseru}\\(can speak)\end{tabular}} 
	& \textit{ga} & parent, teacher, member, $\cdots$ & \textit{sensei} (teacher) \\
 & \textit{wo}* & language, foreign language, $\cdots$ & \textit{doitsugo} (German) \\
\hline
\end{tabular}
\end{center}


\section{Fertilization of the Base Case Frames}

We apply case structure analysis based on the base case frames to the
large corpus, and upgrade the case frames by incorporating newly acquired
information from the case analysis results. The new information consists
of outer nominative and non-gapping relationships. Furthermore, to address
case alternation phenomena, case slot similarity is judged using 
similarities between case examples.

\begin{figure}[b]
 \begin{center}
  \includegraphics{21-4ia9f2.eps}
 \end{center}
  \caption{Outline of the fertilization process}
  \label{Figure::Outline}
\end{figure}

The procedure is as follows (Figure \ref{Figure::Outline}):

\begin{enumerate}
 \item The base case frames are acquired by the method shown in
       Section \ref{Section::1次格フレーム辞書の構築}. \label{Fertilization::parse}
 \item Case analysis utilizing the case frames acquired in phase
       \ref{Fertilization::parse} is applied to the large corpus, and
       examples of the outer nominative are collected from case analysis
       results. \label{Fertilization::double-nominative}
 \item Then, case analysis utilizing the case frames acquired in phase
       \ref{Fertilization::double-nominative} is applied to the large
       corpus, and examples of non-gapping relationships are similarly collected.
 \item Case similarities are judged to handle case alternation.
\end{enumerate}

In the following sections, each step is described in detail.


\subsection{Collecting Instances of the Outer Nominative}

In the base case frame construction described in Section
\ref{Section::1次格フレーム辞書の構築}, the TM phrase was discarded because
its case marker could not be understood by parsing. In example
(\ref{Example::DoubleNominative}), ``\textit{enjin ga yoi}'' (the engine
is good) is used to build the base case frame, but the TM phrase
``\textit{kuruma wa}'' (car-\TOP) is not used.

Case analysis based on the base case frames shows how 
TM phrases can be addressed. Correspondence to the outer nominative cannot be understood by
the case slot matching, but it can be done indirectly. If the TM cannot correspond to
any case slots in the base case frame, the TM can be regarded as
an outer nominative. For example, in the case of
(\ref{Example::DoubleNominative}), since the case frame of
``\textit{enjin ga yoi}'' (the engine is good) has only a nominative
slot that corresponds to ``\textit{enjin},'' the TM of ``\textit{kuruma
wa}'' cannot correspond to any case slots and is recognized as an outer
nominative. On the other hand, in the expression ``\textit{hon-wa
kare-ga kaita}'' (as for the book, he wrote (it)), 
the TM of ``\textit{hon wa}'' is recognized as accusative, because
``\textit{hon}'' (book) is similar to the examples in the accusative
slot. We can distinguish and collect outer nominative examples in this
way.

We apply the following procedure to each sentence containing both a TM
and ``\textit{ga}.'' To reduce the influence of parsing errors, these
sentences are collected under the condition that a
TM phrase has no candidates of its modifying head without its verb.

\begin{enumerate}
 \item We apply case analysis to the verb that is a head of a TM phrase.
       If the verb does not have the closest case component and cannot
       select a case frame, we stop processing this sentence and proceed
       to the next one. In this phase, the TM phrase is not made to
       correspond with a case of the selected case frame.
 \item If the case frame does not have any cases lacking a
       correspondence with the case components in the input, the TM
       cannot appear in any case slots and is thus regarded as an outer
       nominative. This TM phrase is added to outer nominative examples
       of the case frame.
\end{enumerate}

The following is an example of this process.

\begin{exe}
 \ex
 \gll \textit{nagai} \textit{\underline{sumo-wa}} \textit{ashikoshi-ni} \textit{futan-ga} \textit{\underline{\underline{kakaru}}} \\
      long {sumo-\TOP} {legs and loins-\dat} {burden-\acc} impose \\
 \trans (Long sumo imposes a burden on legs and loins.)
\end{exe}

\noindent
Case analysis of this example chooses the following case frame
``\textit{futan ga kakaru}'' (impose a burden).
\vspace{1\Cvs}

\begin{center}
\begin{tabular}{l|l|l|l}
\hline
 & \multicolumn{1}{c|}{CS}  & \multicolumn{1}{c|}{examples in English} & \multicolumn{1}{c}{input} \\
\hline
\raisebox{-8pt}[0pt][0pt]{\begin{tabular}{@{}c@{}}\textit{kakaru}\\(impose)\end{tabular}} 
	& \textit{ga}* & burden & \textit{futan} (burden) \\
 & \textit{ni}  & heart, legs, loins, $\cdots$ \ & \textit{ashikoshi} (legs and loins) \\
\hline
\end{tabular}
\end{center}
\vspace{1\Cvs}

\noindent
Here, \textit{futan} (burden) and \textit{ashikoshi} (legs and loins)
correspond to the nominative and dative slots, respectively, in the case
frame; and \textit{sumo} does not correspond to any case marker. Accordingly, the TM of
``\textit{sumo wa}'' is recognized as an outer nominative, and ``\textit{sumo}''
is added to outer nominative examples of the case frame ``\textit{futan
ga kakaru}.''

This process generated outer nominatives in 16,431 case frames (of 733 verbs).


\subsection{Collecting Instances of Non-gapping Relationships}

Examples of non-gapping relationships can be collected in a similar way as
outer nominatives. When a clausal modifiee contains a non-gapping relationship, it
will not resemble any examples of any cases in the case frame,
because these constructed case frames do not include examples of 
non-gapping relationships. From this point of view, we apply the
following procedure to each example sentence containing a modifying
clause. To reduce the influence of parsing errors, the collection
process for example sentences obeys the condition that a verb in
a clause has no candidates of its modifying head without its clausal
modifiee (``$\cdots$ [modifying verb] N$_1$ \textit{no} N$_2$'' is not
collected).

\begin{enumerate}
 \item We apply case analysis to a verb contained in a
       modifying clause. If the verb does not have the closest case
       component and cannot select a case frame, we quit processing this
       sentence and proceed to the next one. In this phase, the
       clausal modifiee is not made to correspond with a case of the
       selected case frame.
 \item If the similarity between the clausal modifiee and examples of any
       cases not corresponding to input case components
       does not exceed a predetermined threshold, this clausal modifiee is added to
       examples of non-gapping relationships in the case frame. We set the
       threshold empirically at 0.3.
\end{enumerate}

The following is an example of this process.

\begin{exe}
 \ex \label{Example::soto-cf}
 \gll \textit{gyoumu-wo} \textit{\underline{\underline{itonamu}}} \textit{\underline{menkyo}-wo} \textit{syutoku-shita} \\
      {business-\acc} {carry on} {license-\acc} get \\
 \trans ((someone) got a license to carry on business.)
\end{exe}

\noindent
Case analysis of this example selects the following case frame
``\{\textit{gyomu, business}\} \textit{wo itonamu}'' (carry on
\{~work, business~\}).

\vspace{1\Cvs}
\begin{center}
\begin{tabular}{l|l|l|c}
\hline
 & \multicolumn{1}{c|}{CS} & \multicolumn{1}{c|}{examples in English} & input \\
\hline
\raisebox{-8pt}[0pt][0pt]{\begin{tabular}{@{}c@{}}\textit{itonamu}\\(carry on)\end{tabular}} 
	& \textit{ga} & bank, company, $\cdots$ & --- \\
 & \textit{wo}* & work, business & \textit{gyoumu} (business) \\
\hline
\end{tabular}
\end{center}
\vspace{1\Cvs}

\noindent
The nominative slot of this case frame has no corresponding case component 
in the input, so the clausal modifiee, ``\textit{menkyo}''
(license), is checked to determine whether it corresponds to any of the nominative case
examples. In this case, the similarity between ``\textit{menkyo}'' (license)
and examples of the nominative is not very high. Consequently, 
``\textit{menkyo}'' (license) is recognized as having a non-gapping relationship, and
``\textit{menkyo}'' is added to examples of non-gapping relationships in the case
frame ``\{\textit{gyomu, business}\} \textit{wo itonamu}.''

\begin{exe}
 \ex \label{Example::soto-all}
 \gll \textit{ihouni} \textit{denwa gyoumu-wo} \textit{\underline{\underline{itonande-ita}}} \textit{\underline{utagai}} \\
      illegally {telephone business-\acc} {carry on} suspect \\
 \trans (suspect that (someone) carried on telephone business illegally)
\end{exe}

For this sentence, the same case frame is also selected. Since
``\textit{utagai}'' (suspect) is likewise dissimilar to the nominative case
examples, it is added to case examples of non-gapping relationships in the
case frame.

This process found non-gapping relationships in 23,094 case frames (of 637 verbs).


\subsubsection*{Collecting instances of non-gapping relationships for all case frames}

Non-gapping relationship words with wide distribution across verbs can
be considered to have non-gapping relationships for all verbs or case
frames. We add these words to examples of non-gapping relationships in all
the case frames. For example, five verbs have ``\textit{menkyo}'' (license)
(example (\ref{Example::soto-cf})) in their non-gapping relationships, and
303 verbs have ``\textit{utagai}'' (suspect) (example
(\ref{Example::soto-all})). We consequently judge ``\textit{utagai}'' to
have a non-gapping relationship for all the case frames. We call such a word
a \textbf{global non-gapping word}.

We treated words as global non-gapping words if they have non-gapping
relationships for more than 100 predicates. We acquired 73 global
non-gapping words. The following are a few examples:
\begin{quote}
 ``\textit{kanousei}'' (possibility), ``\textit{hitsuyou}'' (necessity),
 ``\textit{kekka}'' (result), ``\textit{houshin}'' (course), ``\textit{kesu}''
 (case), ``\textit{kangae}'' (thought), ``\textit{yotei}'' (schedule),
 ``\textit{mitoushi}'' (outlook), ``\textit{keikaku}'' (plan), ``\textit{mikomi}'' (chance), $\cdots$
\end{quote}

By the above process, non-gapping relationship words for a case frame were
obtained for 828 predicates and 34,243 case frames.


\subsection{Case Similarity Judgment}

To deal with case alternation, we applied the following process to every case
frame with outer nominative and non-gapping relationships.
\begin{enumerate}
 \item The value of similarities between every possible pair of cases is
       calculated. The calculation is the average
       of similarities between all combinations of case examples.
       However, similarities between pairs of basic cases,
       such as (\textit{ga}, \textit{wo}), (\textit{ga}, \textit{ni}),
       (\textit{wo}, \textit{ni}), are not calculated.
 \item A pair whose similarities exceed a predetermined threshold is judged to be
       similar, and is merged into one case. We set this threshold
       empirically at 0.8.
\end{enumerate}

The following example shows this process applied to
``\textit{\{setsumei, syakumei\} wo motomeru}'' (demand
\{explanation, excuse\}).

\vspace{0.5\Cvs}
\begin{center}
\begin{tabular}{l|l|l}
\hline
  & \multicolumn{1}{c|}{CS} & \multicolumn{1}{c}{examples in English} \\
\hline
 & \textit{ga} & committee, group, $\cdots$  \\
 & \textit{wo}* & explanation, excuse \\
\raisebox{0pt}[0pt][0pt]{\begin{tabular}{@{}c@{}}\textit{motomeru}\\(demand)\end{tabular}} 
	& \textit{ni} & government, president, $\cdots$ \\
 & \textit{ni-tsuite} & progress, condition, state, $\cdots$ \\
 & \textit{no} & progress, reason, content, $\cdots$ \\
\hline
\end{tabular}
\end{center}
\vspace{0.5\Cvs}

\noindent
In this case frame, the examples of ``\textit{no}'' are similar to those of
``\textit{ni-tsuite}'' (about), and the similarity between them is very
high, 0.94, so these case examples are merged into a new case
``\textit{no/ni-tsuite}'' (of/about).

Through this process, 1,449 pairs of similar cases are merged. An NP with
``\textit{no}'' modifying a case component can be analyzed by this merging.


\section{Post-processing of Case Frames}

This article constructs case frames from a large volume of newspaper
\pagebreak
corpora. These corpora are very large, but the sparse data problem still
arises. In particular, some commonly used everyday verbs, such as
``\textit{taberu}'' (eat) or ``\textit{suwaru}'' (sit), have incomplete
case frames without obligatory case slots. In this section, a method of
correcting and discarding such case slots is described. In addition, 
a case slot for which there are few case examples is discarded, because
such a case slot has a weak relationship with its verb.


\subsection{Modification and Deletion of Incomplete Case Frames}

The automatically constructed case frames contain incomplete case
frames without obligatory case slots. For example, the transitive verb
``\textit{kanetsu}'' (heat) should have an accusative slot, but some case frames
of ``\textit{kanetsu}'' do not have a \textit{wo} (\acc) case
slot. This problem is caused by data sparseness, and can occur
particularly in less-common case frames. To avoid this problem, case
frames without an obligatory case slot are deleted. A case slot is judged
as obligatory for a verb if many case frames of the verb have the
case slot. All the case slots except those for ``\textit{ga}'' are checked in
this way. Since all case frames can be considered to have the
``\textit{ga}'' case slot, it is conversely
supplemented for case frames without the ``\textit{ga}'' case slot.

An obligatory case slot $cm$ for a verb is judged by the ratio of case
frames with $cm$ to all the case frames of the verb.  This ratio is
called \textit{case ratio}. The case ratio is calculated for every case
slot except ``\textit{ga}.'' A case frame without a case slot whose
case ratio is higher than a predetermined threshold is discarded. The threshold is
empirically set to 0.90.

For example, in ``\textit{hikiwatasu}'' (transfer), the case
ratio of the ``\textit{wo}'' case slot is 1,314/1,342 (0.979), and the following
case frames without the ``\textit{wo}'' case slot are deleted.

\begin{exe}
\ex
 \begin{xlist}
 \ex
  \gll {\{\textit{kainushi}\}-\textit{ga}} {\{\textit{kencho}\}-\textit{de}} {\{\textit{Yonago}\}-\textit{no}} {\{\textit{hokenjo}\}-\textit{ni}} {\textit{hikiwatasu}} \\
       {\{keeper\}-\nom} {\{office\}-\loc} {\{Yonago\}-\gen} {\{health center\}-\dat} {transfer} \\
 \ex
  \gll {\{\textit{Taiwan, keisatsucho}\}-\textit{no}} {\{\textit{fune}\}-\textit{ni}} {\textit{hikiwatasu}} \\
       {\{Taiwan, National Police Agency\}-\gen} {\{ship\}-\dat} {transfer} \\
 \ex
  \gll {\{\textit{ken}\}-\textit{no}} {\{\textit{doubutsuen}\}-\textit{ni}} {\textit{hikiwatasu}} \\
       {\{prefecture\}-\gen} {\{zoo\}-\dat} {transfer} \\
 \end{xlist}
\end{exe}

These incomplete case frames are caused by low-frequency expressions (six
or seven occurrences in the corpus), in which case components of ``\textit{wo}'' case are
omitted.\footnote{In the base case frame construction, we do not use
examples that occur less than five times.}


\subsection{Selection of Obligatory Case Slots}

For each case frame, obligatory case slots are selected on the basis of
\pagebreak
the frequency of case examples. The following criteria are employed for
obligatory case slot selection.
\begin{itemize}
 \item The ``\textit{ga}'' case slot is obligatory, because all the verbs
       can be considered to have ``\textit{ga}'' (\nom) case slots.
 \item A case slot with a high case ratio is obligatory, even if only
       one example occurs.
 \item An outer nominative case slot is obligatory for a case frame with
       one or more case examples in a slot. The non-gapping case slot
       is not obligatory, but is unique to a case frame. This case slot
       is regarded as obligatory for that case frame, as well as an outer nominative case slot.
 \item A case slot whose frequency is larger than the established threshold is
       regarded as \linebreak obligatory. The threshold is defined as $2\sqrt{mf}$,
       where $mf$ is the number of examples of the closest case slot.
\end{itemize}

The following example is a case frame of ``\textit{hikiwatasu}''
(transfer). For this case frame, ``\textit{ga}'' and the closest case
slot ``\textit{ni}'' are obligatory, and ``\textit{wo}'' is judged as
obligatory because of its high case ratio. The other case slots are not
obligatory, because their frequencies are smaller than the threshold (34.8).

\vspace{0.5\Cvs}
\begin{center}
\begin{tabular}{l|l@{ }l|r|l}
\hline
  & \multicolumn{2}{c|}{CS} & \multicolumn{1}{c|}{freq} & \multicolumn{1}{c}{examples in English} \\
\hline
 & ○ & \textit{ga} & 2 & student, person \\
 & ○ & \textit{wo} & 6 & person, man, employee, passenger \\
 & ○ & \textit{ni}* & 302 & official, member, staff \\
\raisebox{0pt}[0pt][0pt]{\begin{tabular}{@{}c@{}}\textit{hikiwatasu}\\(transfer)\end{tabular}} 
	&    & \textit{de}  & 25 & red-handed, station, platform \\
 &    & \textit{no}  & 2 & firefighting, station \\
 &    & \textit{wo-tsuujite} & 1 & station employee \\
 &    & \textit{toshite} & 3 & red-handed \\
\hline
\end{tabular}
\end{center}


\section{Evaluation of Acquired Case Frames}
\label{章::格フレーム辞書評価}

We evaluated the acquired case frames in three ways. The first
one is a manual evaluation of newly obtained outer nominative
examples, non-gapping relationships, and case similarity judgments. The
second one is a manual evaluation of the acquired case
frames. The last evaluation is an experiment of syntactic and case
structure analysis to confirm the effectiveness of the acquired case
frames.


\subsection{Evaluation of Instances Acquired by Fertilization of Case Frames}

We selected 20 case frames (40 case frames in total) containing an outer
\pagebreak
nominative or a non-gapping relation, respectively, and manually
evaluated these instances. We also selected 20 case frames containing
similar cases, and manually evaluated pairs of similar cases. The rest
of this section describes the details of this process.


\subsubsection*{Evaluation of Instances of Outer Nominatives}

We manually judged whether examples acquired as outer nominatives
satisfy the characteristics of outer nominatives described in Section
\ref{章::日本語表現}. There were 367 tokens and 259 types of instances
of outer nominatives in the 20 target case frames, and all of them were
judged to be correctly identified. Two of the evaluated case frames are
shown below:\footnote{``\textit{ga2}'' denotes an outer
nominative case slot.}

\vspace{1\Cvs}
\begin{center}
\begin{tabular}{c|l|l}
\hline
  & \multicolumn{1}{c|}{CS} & \multicolumn{1}{c}{examples in English} \\
\hline
\raisebox{-8pt}[0pt][0pt]{\begin{tabular}{@{}c@{}}\textit{hikui} \\ (low)\end{tabular}}
	& \textit{ga}*  & height, altitude, length, $\cdots$ \\
 & \textit{ga2} & he, person, youth, I, man, tribe, bed, horse, home, $\cdots$ \\
\hline
\raisebox{-8pt}[0pt][0pt]{\begin{tabular}{@{}c@{}}\textit{tsuzuku} \\ (continue)\end{tabular}}
	& \textit{ga}* & intense heat, heat, lingering summer heat, extreme heat, $\cdots$ \\
  & \textit{ga2} & Okinawa, state, prefecture, West Japan, Hokkaido, ship, Philippine, $\cdots$ \\
\hline
\end{tabular}
\end{center}



\subsubsection*{Evaluation of Instances of Non-gapping Relationships}

We manually judged whether an instance recognized as having a non-gapping
relationship satisfies the characteristics of non-gapping relationships
described in Section \ref{章::日本語表現}. There were 330 tokens and 73 types of
non-gapping relationship instances in the target 20 case frames, and out of
these, 298 tokens (90.3\%) and 55 types (75.3\%) were judged to be correct.
Two of the evaluated case frames are shown below:

\vspace{1\Cvs}
\noindent
\resizebox{\textwidth}{!}{
\begin{tabular}{l|l|l}
\hline
 & \multicolumn{1}{c|}{CS} & \multicolumn{1}{c}{examples in English} \\
\hline
 & \textit{ga}  & child, person \\
 & \textit{wo}* & card, passbook, ticket, coupon, license, notebook, certificate, $\cdots$ \\
\raisebox{0pt}[0pt][0pt]{\begin{tabular}{@{}c@{}}\textit{nusumu} \\ (steal)\end{tabular}}
	& \textit{no}  & name, \sm{quantity} yen, colleague, bank, customer, male, female, $\cdots$ \\
 & \textit{kara} & car, house, car, office, wallet, room, bag, person, guest, suit, $\cdots$ \\
 & non-gapping & suspect, way \\
\hline
 & \textit{ga}  & bereaved family, group, governor, Ministry of Finance, $\cdots$ \\
\raisebox{-8pt}[0pt][0pt]{\begin{tabular}{@{}c@{}}\textit{hantai} \\ (oppose)\end{tabular}}
	& \textit{ni}* & move, movement, relocation, translation, transfer \\
 & \textit{no}  & naval port, base, maneuver, hospital, function, institution \\
 & non-gapping & gathering$_{×}$, meeting$_{×}$, problem, resolution, declaration, $\cdots$ \\
\hline
\end{tabular}
}
\vspace{1\Cvs}

For the case frame of ``\textit{nusumu}'' (steal), ``\textit{utagai}''
(suspect) and ``\textit{teguchi}'' (way) were acquired as instances of
non-gapping relationships. ``\textit{Utagai}'' was also regarded as
a global non-gapping word and ``\textit{teguchi}'' was recognized as an
instance of a non-gapping relationship in this specific case frame. For the case
frame of ``\textit{hantai}'' (oppose), ``\textit{shukai}'' (gathering)
and ``\textit{kaigi}'' (meeting) were judged as incorrectly identified
instances of non-gapping relationships. These instances were extracted from the following
sentences.

\begin{exe}
\ex
 \begin{xlist}
 \ex
  \gll {\textit{shutoiten-ni}} {\textit{danko}} {\textit{\underline{\underline{hantai-suru}}}} {\textit{\underline{shukai}-wo}} {\textit{kaisai-suru}} \\
       {capital relocation-\dat} {firmly} {oppose} {gathering-\acc} {hold} \\
  \trans ((someone) holds a gathering that firmly opposes the capital relocation.)
 \ex
  \gll {\textit{hikoujo-no}} {\textit{isetsu-ni}} {\textit{\underline{\underline{hantai-suru}}}} {\textit{\underline{kaigi}-ga}} {\textit{kaisai}} \\
       {airport-\gen} {relocation-\dat} {oppose} {meeting-\nom} {hold} \\
   \trans (A meeting that opposes the airport relocation was held.)
 \end{xlist}
\end{exe}

The correct relation that the words ``\textit{shukai}'' (gathering)
and ``\textit{kaigi}'' (meeting) have with ``\textit{hantai}''
(oppose) is ``\textit{ga}'' (\nom). However, since the similarity
between the two words and ``\textit{dantai}'' (group), the most
similar word in the ``\textit{ga}'' case slot, is 0.17 and 0.14 in the
NTT thesaurus, respectively, these words are judged as instances of non-gapping
relationships.  To solve this problem, we need to employ a better
thesaurus or a larger corpus to collect instances such as
``\textit{shukai}'' (gathering) and ``\textit{kaigi}'' (meeting) as
instances of ``\textit{ga}'' case slot.


\subsubsection*{Evaluation of Case Similarity Judgments}

We manually judged whether a pair of similar cases was correctly
identified. The criterion for this judgment is that the two cases in a
pair can be interchangeable. Sixteen pairs out of 20 were judged as
correct. Two of the evaluated case frames are shown below:

\vspace{1\Cvs}
\noindent
\resizebox{\textwidth}{!}{
\begin{tabular}{l|l|l}
\hline
 & \multicolumn{1}{c|}{CS} & \multicolumn{1}{c}{examples in English} \\
\hline
 & \textit{ga}  & director, head, president, manager, representative, $\cdots$ \\
 & \textit{wo}  & response, problem, procedure, circumstance, plan \\
\raisebox{0pt}[0pt][0pt]{\begin{tabular}{@{}c@{}}\textit{kyougi} \\ (discuss)\end{tabular}}
	& \textit{to}* & president, prime minister, foreign minister, leader, assemblyman, $\cdots$ \\
 & \textit{de}  & phone call, official residence, problem, headquarter, Kiev, Dili \\
 & \textit{ni-tsuite} & problem, circumstance, response, content, removal, $\cdots$ \\
\hline
 & \textit{ga} & Democratic Party \\
\raisebox{-8pt}[0pt][0pt]{\begin{tabular}{@{}c@{}}\textit{kimeru} \\ (determine)\end{tabular}}
	& \textit{wo}* & inauguration, reappointment, remaining \\
 & \textit{no} & chairman, president, head, \sm{quantity} person, director, dean, $\cdots$ \\
 & \textit{ni} & chairman, successor, head, manager, president, director, $\cdots$ \\
\hline
\end{tabular}
}
\vspace{1\Cvs}

The similar case pair of ``\textit{wo}'' (\acc) and
``\textit{ni-tsuite}'' (about) was correctly detected in the case frame
of ``\textit{kyougi}'' (discuss). The similar case pair of
``\textit{ni}'' (\dat) and ``\textit{no}'' (\gen) was incorrectly
detected in the case frame of ``\textit{kimeru}'' (determine). Although
the instances in these case slots are similar, they play different roles
and can appear at the same time, as in the following sentence.
\begin{exe}
\ex
 \gll {\textit{shin\underline{kaicho}-ni}} {\textit{Okada}} {\textit{fuku\underline{kaicho}-no}} {\textit{shunin-wo}} {\textit{\underline{\underline{kimeta}}}} \\
      {chairman-\dat} {Okada} {vice chairman-\gen} {appointment-\acc} {determine} \\
 \trans ((someone) determined the appointment of Vice Chairman Okada as a chairman)
\end{exe}
Such co-occurrence information may be used to improve the accuracy of
case similarity judgment.


\subsection{Manual Evaluation of Case Frames}
\label{Section::ManualCaseFrameEvaluation}

We evaluated the case frames that are automatically constructed as
described in the previous section. We selected 20 verbs randomly, and
manually evaluated them. The case frame evaluation was performed
according to the following criteria.

\begin{itemize}
 \item Verb usage is disambiguated by the verb and its closest case
       component. That is to say, there are no different meanings or
       case slot patterns in a case frame.
 \item A case frame must have obligatory case slots, such
       as the ``\textit{wo}'' case slot for transitive verbs.
 \item A case slot, except the closest one, may contain an ineligible
       example. If this happens for the closest case slot, the case
       frame does not satisfy the first condition and is judged as incorrect.
\end{itemize}

Table \ref{Evaluation::CF} shows the 20 evaluated verbs and their
accuracies.  For example, ``\textit{oru}'' has 32 case frames, and out
of these, 31 case frames were judged as correct. In total, 1,021 out of 1,070
(95.4\%) case frames were correct. Tables \ref{Evaluation::CF-Oru} and
\ref{Evaluation::CF-Hiyasu} show the evaluation of all the case
frames of ``\textit{oru}'' (break) and ``\textit{hiyasu}'' (cool),
respectively. In the leftmost of the tables, correct case frames are
marked by ``○,'' and incorrect case frames are marked by ``×.''

49 case frames judged as incorrect did not satisfy the first condition; 
in these case frames, expressions with different meanings were merged. 
For instance, ``\textit{hiyasu}'' (cool) had the following case
frame:
\begin{exe}
\ex
 \gll {\{\textit{kan, geraku, $\cdots$}\}-\textit{ga}} {\{\textit{shinri, keiki, maindo}\}-\textit{wo}} \textit{hiyasu} \\
      {\{sense, decline, $\cdots$\}-\nom} {\{mentality, business, mind\}-\acc} cool \\
\end{exe}
In this case frame, two different expressions, ``\textit{keiki-wo
hiyasu}'' (decline business) and ``\textit{maindo-wo hiyasu}'' (cool
down one's mind) are incorrectly merged, because the words
``\textit{keiki}'' (business) and ``\textit{maindo}'' (mind) are similar
in the thesaurus. To overcome this problem, a better thesaurus should be
employed or a word sense disambiguation process should be incorporated
into the case frame construction.

\begin{table}[t]
 \caption{Evaluation result of case frames}
 \label{Evaluation::CF}
\input{09table01.txt}
\end{table}

On the other hand, ``\textit{oru}'' (break) had the following case frame:

\begin{exe}
 \ex
 \gll {\{\sm{quantity}, \textit{josei}, $\cdots$\}-\textit{ga}}
 {\{\sm{clause}, \textit{tyousei}, $\cdots$\}-\textit{ni}}
 {\{\textit{atama, kubi, $\cdots$}\}-\textit{no}}
 {\{\textit{hone, rokkotsu, $\cdots$}\}-\textit{wo}} \textit{oru} \\
 {\{\sm{quantity}, woman, $\cdots$\}-\nom} {\{\sm{clause}, adjustment, $\cdots$\}-\dat}
 {\{head, neck, $\cdots$\}-\gen} {\{bone, rib, $\cdots$\}-\acc} break \\
\end{exe}
This case frame contains different meanings: to break a bone and to have
difficulty. The current method could not disambiguate these meanings. To
avoid this problem, it is necessary to consider not only the closest
case component but also the wider context of the potential case frame.

\begin{table}[t]
\caption{Evaluation of case frames of ``\textit{oru}''}
 \label{Evaluation::CF-Oru}
\input{09table02.txt}
\end{table}

\begin{table}[t]
 \caption{Evaluation of case frames of ``\textit{hiyasu}''}
 \label{Evaluation::CF-Hiyasu}
\input{09table03.txt}
\end{table}


\subsection{Evaluation through Analyses}

To examine the usefulness of the automatically constructed case frames,
we evaluated the case frames through syntactic and case analyses. For
this evaluation, we utilized ``Relevance-tagged Corpus''
\cite{Kawahara2002c}, in which syntactic relations, case/ellipsis
relations, coreferences, and relations between nouns are annotated.
We applied syntactic and case analysis based on the constructed case
frames to 94 articles (1,100 sentences) in the corpus, and evaluated
their analysis results.


\subsubsection*{Evaluation through Syntactic Analysis}

Table \ref{結果::構文解析の精度} shows the experimental results of the
syntactic analysis based on the automatically constructed case frames.
The baseline in this table represents the accuracy of the original
syntactic analyzer without case frames. The overall accuracy increased
by 0.2\%; in particular, the accuracy of clausal modifiees increased
by 0.8\%. This improvement was due to the selectional preference of the
case frames.


\subsubsection*{Evaluation through Case Analysis}

In this section, the case analysis based on the fertilized case frames
is evaluated. The fertilized case frames have outer nominatives and
non-gapping relationships, and the correspondence table of TM phrases and
clausal modifiees (described in Section
\ref{Section::MatchingInputWIthCaseFrames}) can now be updated as follows:

\vspace{1\Cvs}
\begin{center}
\begin{tabular}{l@{ : }l}
\hline
  TM phrases & \textit{ga}, \textit{wo}, \textit{ga2} \\
  clausal modifiees & \textit{ga}, \textit{wo}, \textit{ni}, \textit{ga2}, non-gapping \\
\hline
\end{tabular}
\end{center}
\vspace{1\Cvs}

\noindent
If a TM phrase has no corresponding case slot in a case frame, the
relation to its head verb is analyzed as \textit{ga2}.

\begin{table}[b]
  \caption{Accuracy of syntactic analysis}
  \label{結果::構文解析の精度}
\input{09table04.txt}
\end{table}

\begin{table}[b]
  \caption{Case analysis accuracy}
  \label{Result::CaseAnalysisAccuracy}
\input{09table05.txt}
\end{table}
\begin{table}[b]
  \caption{Non-gapping relation accuracy}
  \label{Result::Non-GappingRelationAccuracy}
\input{09table06.txt}
\end{table}

We applied case analysis to 94 articles in the ``Relevance-tagged
Corpus,'' and evaluated the analysis results by comparing them with the
gold standard of the corpus. To evaluate the real case analysis without
the influence of parsing errors, we input the correct structure of the
corpus sentences to the analyzer.  The accuracy of clausal modifiees and
TM phrases is shown in Table \ref{Result::CaseAnalysisAccuracy}, and the
accuracy of non-gapping relationships is shown in Table
\ref{Result::Non-GappingRelationAccuracy}. The baseline of these tables
is that if a clausal modifiee belongs to a non-gapping noun dictionary
wherein nouns are written as always having non-gapping relationships as
clausal modifiees, it is analyzed as a non-gapping relationship.

The accuracy of clausal modifiees increased by 3.0\%. This shows
the effectiveness of our fertilization process. However, the accuracy of TM
phrases did not increase. This is attributed to the low frequency of 
TM phrases that were analyzed using added outer nominative examples.


\section{Related Work}

Many approaches have been proposed in recent years for learning case
frames from a raw corpus for English \cite{Brent1993,Ushioda1993,Manning1993,Briscoe1997,Korhonen2003}.
The English language does not have omitted arguments and thus the
problem is how to judge an argument as a core or an adjunct for a predicate.
This judgment is conducted by calculating the relatedness between a
predicate and case frames based on statistical information. Learned case
frames are not examples but syntactic patterns, such as a verb taking a
noun phrase and a prepositional phrase. That is, these approaches
collect usages of predicates and do not consider the variety of
predicate usages.

In Japanese, methods for learning case frames from a syntactically
tagged corpus have been proposed \cite{Higashi1996,Utsuro1997}. Since
these methods used a small-scale tagged corpus, they provide
insufficient coverage for the resulting case frames. \citeA{Haruno1995} proposed a method
for learning case frames based on semantic markers from a corpus. 
Since he targeted 11 verbs and used examples
manually extracted from a one-year volume of a newspaper corpus, he
had no problem with coverage. However, it would be difficult to
increase the number of verbs and create practical case frames. Another
difference from our case frames is that case frames obtained by
these methods generalized arguments as semantic markers. These methods
handled predicate ambiguities as follows. Higashi et al. (1996) induced case
frames for each verbal category assigned to a verb based on the EDR corpus.
Utsuro et al. (1997) and Haruno (1995) determined the degree of generalization
of arguments on the basis of machine learning and information
compression, respectively, in the process of clustering. However, these
approaches are thought to not be appropriate for case frame compilation
in terms of accuracy.

Some related work has analyzed clausal modifiees and TM
phrases. Most of this work used statistical information or machine learning
techniques \cite{Baldwin1999,Torisawa2001,Murata2005,Abekawa2005}.
\citeA{Abekawa2005} claimed that a case frame-based method cannot judge non-gapping
relationships precisely and depended on an
elimination process to assign non-gapping relationships in the absence of 
predicate-argument relations. Our proposed method uses this
kind of elimination process to compile fertilized case frames, but
considers examples of both non-gapping and predicate-argument relationships in the case structure analysis step based on
the case frames. In practice, we achieved higher accuracy with the
system based on the fertilized case frames than on the base case
frames. From this result, we can see that we successfully overcame
the defect identified by Abekawa and Okumura (2005).


\section{Conclusions}

This article described a method for gradually extracting reliable
information from a corpus and compiling case frames. The resulting case
frames contain information about outer nominative cases of double
nominative constructions, non-gapping relationships of clausal modifiees, and
case similarity. We compiled case frames from a corpus of a 26-year volume
of newspaper articles, comprising approximately 26 million sentences,
and evaluated the resulting case frames in two ways: manually and through
syntactic and case structure analyses. The results confirm that the case
frames have a good enough accuracy to perform robust case structure
analysis. One of the causes of the low accuracy of ellipsis analysis,
which is a subsequent step of case structure analysis, is that some
complex expressions, such as non-gapping relationships and double nominative
constructions, cannot be analyzed precisely. That is, if a clausal
modifiee whose real relationship to the predicate is a non-gapping one is judged as
nominative, it is impossible to detect an ellipsis at the nominative case.
Our case frames will contribute to the improvement of ellipsis and
anaphora resolution.


\bibliographystyle{jnlpbbl_1.5}
\begin{thebibliography}{}

\bibitem[\protect\BCAY{Abekawa \BBA\ Okumura}{Abekawa \BBA\
  Okumura}{2005}]{Abekawa2005}
Abekawa, T.\BBACOMMA\ \BBA\ Okumura, M. \BBOP 2005\BBCP.
\newblock \BBOQ Corpus-Based Analysis of Japanese Relative Clause
  Constructions.\BBCQ\
\newblock In {\Bem Proceedings of the 2nd International Joint Conference on
  Natural Language Processing}, \mbox{\BPGS\ 46--57}.

\bibitem[\protect\BCAY{Baldwin, Tokunaga, \BBA\ Tanaka}{Baldwin
  et~al.}{1999}]{Baldwin1999}
Baldwin, T., Tokunaga, T., \BBA\ Tanaka, H. \BBOP 1999\BBCP.
\newblock \BBOQ The Parameter-based Analysis of {J}apanese Relative Clause
  Constructions.\BBCQ\
\newblock In {\Bem IPSJ SIG Notes 1999-NL-134}, \mbox{\BPGS\ 55--62}.

\bibitem[\protect\BCAY{Brent}{Brent}{1993}]{Brent1993}
Brent, M. \BBOP 1993\BBCP.
\newblock \BBOQ From Grammar to Lexicon: Unsupervised Learning of Lexical
  Syntax.\BBCQ\
\newblock {\Bem Computational Linguistics}, {\Bbf 19}  (2), \mbox{\BPGS\
  243--262}.

\bibitem[\protect\BCAY{Briscoe \BBA\ Carroll}{Briscoe \BBA\
  Carroll}{1997}]{Briscoe1997}
Briscoe, T.\BBACOMMA\ \BBA\ Carroll, J. \BBOP 1997\BBCP.
\newblock \BBOQ Automatic Extraction of Subcategorization from Corpora.\BBCQ\
\newblock In {\Bem Proceedings of the 5th Conference on Applied Natural
  Language Processing}, \mbox{\BPGS\ 356--363}.

\bibitem[\protect\BCAY{Haruno}{Haruno}{1995}]{Haruno1995}
Haruno, M. \BBOP 1995\BBCP.
\newblock \BBOQ A Case Frame Learning Method for {J}apanese Polysemous
  Verbs.\BBCQ\
\newblock In {\Bem Proceedings of the AAAI Spring Symposium: Representation and
  Acquisition of Lexical Knowledge: Polysemy, Ambiguity, and Generativity},
  \mbox{\BPGS\ 45--50}.

\bibitem[\protect\BCAY{Higashi, Mine, \BBA\ Amamiya}{Higashi
  et~al.}{1996}]{Higashi1996}
Higashi, M., Mine, T., \BBA\ Amamiya, M. \BBOP 1996\BBCP.
\newblock \BBOQ Sense Discrimination of Verbs Using EDR Dictionary.\BBCQ\
\newblock In {\Bem In Proceedings of the Institute of Electronics, Information
  and Communication Engineers, Natural Language Understanding and Models of
  Communication, NLC96-36}, \mbox{\BPGS\ 39--44}.
\newblock (in Japanese).

\bibitem[\protect\BCAY{Ikehara, Miyazaki, Shirai, Yokoo, Nakaiwa, Ogura, Oyama,
  \BBA\ Hayashi}{Ikehara et~al.}{1997}]{NTT}
Ikehara, S., Miyazaki, M., Shirai, S., Yokoo, A., Nakaiwa, H., Ogura, K.,
  Oyama, Y., \BBA\ Hayashi, Y.\BEDS\ \BBOP 1997\BBCP.
\newblock {\Bem {J}apanese Lexicon (\textit{Nihongo Goi Taikei})}.
\newblock Iwanami Publishing.
\newblock (in Japanese).

\bibitem[\protect\BCAY{Kawahara \BBA\ Kurohashi}{Kawahara \BBA\
  Kurohashi}{2001}]{Kawahara2001}
Kawahara, D.\BBACOMMA\ \BBA\ Kurohashi, S. \BBOP 2001\BBCP.
\newblock \BBOQ {J}apanese Case Frame Construction by Coupling the Verb and its
  Closest Case Component.\BBCQ\
\newblock In {\Bem Proceedings of the Human Language Technology Conference},
  \mbox{\BPGS\ 204--210}.

\bibitem[\protect\BCAY{Kawahara, Kurohashi, \BBA\ Hasida}{Kawahara
  et~al.}{2002}]{Kawahara2002c}
Kawahara, D., Kurohashi, S., \BBA\ Hasida, K. \BBOP 2002\BBCP.
\newblock \BBOQ Construction of a {J}apanese Relevance-tagged Corpus.\BBCQ\
\newblock In {\Bem Proceedings of the 3rd International Conference on Language
  Resources and Evaluation}, \mbox{\BPGS\ 2008--2013}.

\bibitem[\protect\BCAY{Korhonen \BBA\ Preiss}{Korhonen \BBA\
  Preiss}{2003}]{Korhonen2003}
Korhonen, A.\BBACOMMA\ \BBA\ Preiss, J. \BBOP 2003\BBCP.
\newblock \BBOQ Improving Subcategorization Acquisition using Word Sense
  Disambiguation.\BBCQ\
\newblock In {\Bem Proceedings of the 41st Annual Meeting of the Association
  for Computational Linguistics}, \mbox{\BPGS\ 48--55}.

\bibitem[\protect\BCAY{Kurohashi \BBA\ Nagao}{Kurohashi \BBA\
  Nagao}{1994a}]{Kuro-IEICE1994}
Kurohashi, S.\BBACOMMA\ \BBA\ Nagao, M. \BBOP 1994a\BBCP.
\newblock \BBOQ A Method of Case Structure Analysis for {J}apanese Sentences
  based on Examples in Case Frame Dictionary.\BBCQ\
\newblock {\Bem IEICE Transactions on Information and Systems}, {\Bbf E77-D}
  (2), \mbox{\BPGS\ 227--239}.

\bibitem[\protect\BCAY{Kurohashi \BBA\ Nagao}{Kurohashi \BBA\
  Nagao}{1994b}]{Kurohashi1994}
Kurohashi, S.\BBACOMMA\ \BBA\ Nagao, M. \BBOP 1994b\BBCP.
\newblock \BBOQ A Syntactic Analysis Method of Long {J}apanese Sentences based
  on the Detection of Conjunctive Structures.\BBCQ\
\newblock {\Bem Computational Linguistics}, {\Bbf 20}  (4), \mbox{\BPGS\
  507--534}.

\bibitem[\protect\BCAY{Kurohashi \BBA\ Nagao}{Kurohashi \BBA\
  Nagao}{1998}]{Kurohashi1998}
Kurohashi, S.\BBACOMMA\ \BBA\ Nagao, M. \BBOP 1998\BBCP.
\newblock \BBOQ Building a {J}apanese Parsed Corpus while Improving the Parsing
  System.\BBCQ\
\newblock In {\Bem Proceedings of the 1st International Conference on Language
  Resources and Evaluation}, \mbox{\BPGS\ 719--724}.

\bibitem[\protect\BCAY{Manning}{Manning}{1993}]{Manning1993}
Manning, C.~D. \BBOP 1993\BBCP.
\newblock \BBOQ Automatic Acquisition of a Large Subcategorization Dictionary
  from Corpora.\BBCQ\
\newblock In {\Bem Proceedings of the 31st Annual Meeting of the Association
  for Computational Linguistics}, \mbox{\BPGS\ 235--242}.

\bibitem[\protect\BCAY{Murata \BBA\ Isahara}{Murata \BBA\
  Isahara}{2005}]{Murata2005}
Murata, M.\BBACOMMA\ \BBA\ Isahara, H. \BBOP 2005\BBCP.
\newblock \BBOQ {J}apanese Case Analysis Based on Machine Learning Method that
  Uses Borrowed Supervised Data.\BBCQ\
\newblock In {\Bem Proceedings of 2005 IEEE International Conference on Natural
  Language Processing and Knowledge Engineering}, \mbox{\BPGS\ 774--779}.

\bibitem[\protect\BCAY{Torisawa}{Torisawa}{2001}]{Torisawa2001}
Torisawa, K. \BBOP 2001\BBCP.
\newblock \BBOQ An Unsupervised Method for Canonicalization of {J}apanese
  Postpositions.\BBCQ\
\newblock In {\Bem Proceedings of the 6th Natural Language Processing Pacific
  Rim Symposium}, \mbox{\BPGS\ 211--218}.

\bibitem[\protect\BCAY{Ushioda, Evans, Gibson, \BBA\ Waibel}{Ushioda
  et~al.}{1993}]{Ushioda1993}
Ushioda, A., Evans, D., Gibson, T., \BBA\ Waibel, A. \BBOP 1993\BBCP.
\newblock \BBOQ The Automatic Acquisition of Frequencies of Verb
  Subcategorization Frames from Tagged Corpora.\BBCQ\
\newblock In {\Bem Proceedings of the Workshop on Acquisition of Lexical
  Knowledge from Text}, \mbox{\BPGS\ 95--106}.

\bibitem[\protect\BCAY{Utsuro, Miyata, \BBA\ Matsumoto}{Utsuro
  et~al.}{1997}]{Utsuro1997}
Utsuro, T., Miyata, T., \BBA\ Matsumoto, Y. \BBOP 1997\BBCP.
\newblock \BBOQ Maximum Entropy Model Learning of Subcategorization
  Preference.\BBCQ\
\newblock In {\Bem Proceedings of the 5th Workshop on Very Large Corpora},
  \mbox{\BPGS\ 246--260}.

\end{thebibliography}


\appendix


\section*{Similarity between Case Frames}

We define the similarity between two case frames, $F_1$ and $F_2$, as
the product of similarity of case slots and similarity of instance
words. To calculate this similarity, two case frames, $F_1$ and $F_2$,
are first aligned according to the agreement of case slots. Suppose the
result of the case slot alignment of $F_1$ and $F_2$ is as follows:
\begin{eqnarray*}
 \begin{array}{ccccccc}
  F_1 : & C_{11}, & C_{12}, & \cdots, & C_{1l}, & \cdots, & C_{1m} \\
        & \updownarrow & \updownarrow  & & \updownarrow \\
  F_2 : & C_{21}, & C_{22}, & \cdots, & C_{2l}, & \cdots, & C_{2n}.
 \end{array}
\end{eqnarray*}

The similarity between two words, $e_1$ and $e_2,$ is calculated by
using the NTT thesaurus \cite{NTT} as follows:
\begin{gather}
 sim_e(e_1, e_2) = max_{x \in s_1, y \in s_2} \, sim(x, y), 
\label{Formula::NttSimilarity} \\
 sim(x, y) = \frac{2L}{l_{x}+l_{y}}, \nonumber
\end{gather}
where $x, y$ are semantic markers, and $s_1, s_2$ are sets of semantic
markers of $e_1, e_2$, respectively.\footnote{In several cases, nouns have
many semantic markers in the NTT thesaurus.} $l_{x}, l_{y}$ are the depths
of $x, y$ in the thesaurus, and the depth of their lowest (most
specific) common node is $L$. If $x$ and $y$ are in the same node of the
thesaurus, the similarity is 1.0, the maximum score based on this
criterion.

The similarity between two case slots, $C_{1i}$ and $C_{2i}$ in $F_1$
and $F_2$, respectively, is defined as the following equation by finding the most
similar instance in the set of instances in the corresponding case slot
for each instance and averaging these similarities:
\begin{equation}
\begin{aligned}[b]
 & CaseSim(C_{1i},C_{2i})  \\
 & \quad =\frac{\sum_{e_1 \in C_{1i}}{\displaystyle |e_1| \cdot \max \{ sim(e_1,e_2) | e_2 \in C_{2i} \}} + 
	\sum_{e_2 \in C_{2i}}{\displaystyle |e_2| \cdot \max \{ sim(e_1,e_2) | e_1 \in C_{1i} \}}}
	{\sum_{e_1 \in C_{1i}}{\displaystyle |e_1|} + \sum_{ e_2 \in C_{2i} }{\displaystyle |e_2|}}.
\end{aligned}
\label{Formula::CaseSimilarity}
\end{equation}

We consider a frequent case slot to be important, and add a weight to a
case slot according to the square root of the product of the number of
instances appearing in the case slot. The weighted case similarity
$\mathit{WeightedCaseSim}(F_1,F_2)$ is calculated as follows:
\begin{equation}
\begin{aligned}[b]
 & \mathit{WeightedCaseSim}(F_1,F_2) \\
 & \quad = \frac
  {
  \sum_{i=1}^{l} \displaystyle \sqrt{|C_{1i}||C_{2i}|} \cdot CaseSim(C_{1i},C_{2i})
  }
  {
  \sum_{i=1}^{l} \displaystyle \sqrt{|C_{1i}||C_{2i}|}
  },
\end{aligned}
\end{equation}
where $e_1$ and $e_2$ are instances of $C_{1i}$ and $C_{2i}$,
respectively, and
$|e_1|$ is the frequency of $e_1$. $|C_{1i}|$ and $|C_{2i}|$ are the
number of instances in $C_{1i}$ and $C_{2i}$, respectively.

On the other hand, the similarity of case slots is defined as the square
root of the product of the number of aligned instances divided by the
number of all instances for each case slot in the case frames $F_1$ and
$F_2$. It is calculated as follows:
\begin{equation}
 Alignment(F_1,F_2) =
  \sqrt{
  \frac{\sum_{i = 1}^{l}|C_{1i}|}{ \sum_{i=1}^{m} |C_{1i}|} \times
  \frac{\sum_{i = 1}^{l}|C_{2i}|}{ \sum_{i=1}^{n} |C_{2i}|}
  }.
\end{equation}

Finally, the similarity between two case frames is calculated as follows:
\begin{equation}
\begin{aligned}[b]
 & \mathit{CaseFrameSim}(F_1,F_2)  \\
 & \quad = \mathit{WeightedCaseSim}(F_1,F_2) \times Alignment(F_1,F_2).
\end{aligned}
\end{equation}


\begin{biography}

\bioauthor[:]{Daisuke Kawahara}{
Daisuke Kawahara received his B.S. and M.S. degrees in Electronic Science and
Engineering from Kyoto University in 1997 and 1999, respectively. He is
currently a research associate of the Graduate School of Information
Science and Technology at the University of Tokyo. His research
interests center on natural language processing, in particular syntactic
analysis and anaphora resolution.
}

\bioauthor[:]{Sadao Kurohashi}{
Sadao Kurohashi received a Ph.D. in Electrical Engineering from Kyoto
University in 1994. He is currently an associate professor of the
Graduate School of Information Science and Technology at the University
of Tokyo. His research interests include natural language processing
and knowledge information processing.
}
\end{biography}

\biodate


\end{document}
