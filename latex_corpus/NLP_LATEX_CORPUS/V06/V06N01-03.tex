
\documentstyle[epsf,jnlpbbl]{jnlp_j_b5}

\setcounter{page}{43}
\setcounter{巻数}{6}
\setcounter{号数}{1}
\setcounter{年}{1999}
\setcounter{月}{1}
\受付{1998}{5}{18}
\再受付{1998}{8}{24}
\採録{1998}{10}{2}

\newcounter{textcounter}
\newenvironment{TEXT}{}{}

\newenvironment{NEWS}{}{}

\newcounter{algocounter}
\newcounter{algocnt}[algocounter]
\newenvironment{ALGO}{}{}

\newcounter{expercounter}
\newenvironment{CONFIG}{}{}


\title{表題へのつながりに基づく文の重要度評価}
\author{吉見 毅彦\affiref{SHARP/KOBE} \and 奥西 稔幸\affiref{SHARP}
\and 山路 孝浩\affiref{SHARP} \and 福持 陽士\affiref{SHARP}}

\headauthor{吉見,奥西,山路,福持}
\headtitle{表題へのつながりに基づく文の重要度評価}

\affilabel{SHARP/KOBE}{シャープ(株)ソフト事業推進センター / 神戸大学 
大学院自然科学研究科}{Software Business Development Center, SHARP
Corp. / Graduate School of Science and Technology, Kobe University}
\affilabel{SHARP}{シャープ(株) ソフト事業推進センター}
{Software Business Development Center, SHARP Corp.}

\jabstract{本稿では,表層的な情報を手がかりとして文と文のつながりの強
さを評価し,その強さに基づいて重要な文を選び出す手法を提案する.
文の重要度の評価に際して,表題はテキスト中で最も重要な文であり,
重要な文へのつながりが強い文ほど重要な文であるという仮定を置き,
文から表題へのつながりの強さをその文の重要度とする.
二つの文のつながりの強さは,人称代名詞による前方照応と,同一辞書
見出し語による語彙的なつながりに着目して評価する.
平均で29.0文から成る英文テキスト80編を対象とした実験では,
文選択率を25\%に設定したとき,従来手法による精度を上回る再現率78.2\%,
適合率57.7\%の精度を得,比較的短いテキストに対して提案手法が有効である
ことを確認した.}

\jkeywords{抄録,重要文選択,照応,語彙的なつながり}

\etitle{Evaluation of Importance of Sentences \\
based on Connectivity to the Title}
\eauthor{Takehiko Yoshimi\affiref{SHARP/KOBE} \and
Toshiyuki Okunishi\affiref{SHARP} \and 
Takahiro Yamaji\affiref{SHARP} \and
Yoji Fukumochi\affiref{SHARP}} 

\eabstract{This paper proposes a method of selecting important
sentences from a text based on the evaluation of the connectivity
between sentences by using surface information.
We assume that the title of a text is the most concise statement
which expresses the most essential information of the text, and
that the closer a sentence relates to an important sentence,
the more important this sentence is.
The importance of a sentence is defined as the connectivity between
the sentence and the title.
The connectivity between two sentences is measured based on
correference between a pronoun and a preceding (pro)noun, and 
on lexical cohesion of lexical items.
In an experiment with 80 English texts, 
which consist of an average of 29.0 sentences, the proposed method has
marked 78.2\% recall and 57.7\% precision, with the selection ratio
being 25\%.
The recall and precision values surpass those achieved by conventional
methods, which means that our method is more effective in abridging
relatively short texts.}

\ekeywords{Abridgement, Selection of Important Sentences,
Correference, Lexical Cohesion}

\begin{document}
\maketitle

\section{はじめに}
\label{sec:introduction}

電子化テキストの急増などに伴い,近年,テキストから要点を抜き出す
重要文選択技術の必要性が高まってきている.
このような要請に現状の技術レベルで応えるためには,表層的な情報を有効に
利用することが必要である.
これまでに提案されている表層情報に基づく手法では,文の重要度の評価が主
に,
1) 文に占める重要語の割合,
2) 段落の冒頭,末尾などのテキスト中での文の出現位置,
3) 事実を述べた文,書き手の見解を述べた文などの文種,
4) あらかじめ用意したテンプレートとの類似性
などの評価基準のいずれか,またはこれらを組み合わせた基準に基づいて行
なわれる\cite{Luhn58,Edmundson69,Kita87,Suzuki88,Mase89,Salton94,Brandow95,Matsuo95,Sato95,Yamamoto95,Watanabe96,Zechner96,FukumotoF97,Nakao97}.

本稿では,表層的な情報を手がかりとして文と文のつながりの強さを評価し,
その強さに基づいて文の重要度を決定する手法を提案する.
提案する手法では文の重要度に関して次の仮定を置く.
\begin{enumerate}
\item 表題はテキスト中で最も重要な文である.
\item 重要な文とのつながりが強ければ強いほど,その文は重要である.
\end{enumerate}
表題はテキストの最も重要な情報を伝える表現であるため,それだけで最も
簡潔な抄録になりえるが,多くの場合それだけでは情報量が十分でない.
従って,不足情報を補う文を選び出すことが必要となるが,そのような文は,
表題への直接的なつながりまたは他の文を介しての間接的なつながりが強い
文であると考えられる.
このような考え方に基づいて,文から表題へのつながりの強さをその文の重要
度とする.
文と文のつながりの強さを評価するために次の二つの現象に着目する.
\begin{enumerate}
\item 人称代名詞と先行(代)名詞の前方照応
\item 同一辞書見出し語による語彙的なつながり
\end{enumerate}

重要文を選択するために文間のつながりを解析する従来の手法としては,
1) 接続表現を手がかりとして修辞構造
を解析し,その結果に基づいて文の重要度を評価する手法
\cite{Mase89,Ono94}や,
2) 本稿と同じく,語彙的なつながりに着目した手法
\cite{Hoey91,Collier94,FukumotoJ97,Sasaki93}がある.
文と文をつなぐ言語的手段には,照応,代用,省略,接続表現の使用,語彙
的なつながりがある\cite{Halliday76,Jelinek95}が,接続表現の使用頻度は
あまり高くない
\footnote{文献\cite{Halliday76}で調査された七編のテキストでは,照
応,代用,省略,接続表現の使用,語彙的なつながりの割合は,それぞれ,
32\%,4\%,10\%,12\%,42\%である\cite{Hoey91}.}.
このため,前者の手法には,接続表現だけでは文間のつながりを解析するため
の手がかりとしては十分でないという問題点がある.
後者の手法では,
使用頻度が比較的高い
照応を手がかりとして利用していない.

\section{テキストの結束を維持する手段}
\label{sec:coherence}

適格なテキストでは通常,文と文の間につながりがある.
二つの文をつなぐ言語的手段のうち照応と語彙的なつながりは,他の結束維持
手段よりも頻繁に見られる.

照応は,二つのテキスト構成要素が一つの事象に言及することによってテキス
トの結束を生む手段である.
前方照応では,ある要素の解釈が,テキスト中でその要素より前方に現れる先
行要素に依存して決まる.
ある要素$Y$とその先行要素$X$の間で照応が成り立つためには,
1) $Y$は$X$を縮約した言語形式であり,
2) $Y$の意味と$X$の意味は矛盾してはならない\cite{Jelinek95}.
例えば,代名詞は名詞句を,名詞句は分詞節を,分詞節は文をそれぞれ縮約し
た言語形式である.
次のテキスト\ref{TEXT:dismiss}\,では斜体の表現の意味は互いに矛盾しな
いので,それらはいずれも同一事象を指しているとみなせる.
\begin{TEXT}
\text
{\it The Soviet National Emergency Committee dismissed President
Gorbachov from office.} 
As well as {\it dismissing the President}, the Committee
embarked upon choosing his replacement. 
{\it Gorbachov's dismissal} is bound to put Western policies
vis-a-vis the Soviet Union into great turmoil.
{\it It} will have grave repercussions on the exchange
rates.\label{TEXT:dismiss}
\end{TEXT}

語彙的なつながりでは,照応と異なり,二つのテキスト構成要素が同一事象
に言及しているとは限らない\cite{Halliday76}.
次のテキスト\ref{TEXT:boy_same}\,では第二文の``boy''は先行文の``boy''
と同じ少年に言及しているが,テキスト\ref{TEXT:boy_exclude}\,では別の少年
に言及している.
\begin{TEXT}
\text There's a {\it boy} climbing that tree.
The {\it boy}'s going to fall if he doesn't take care.
\label{TEXT:boy_same} 
\text There's a {\it boy} climbing that tree.
And there's another {\it boy} standing underneath.
\label{TEXT:boy_exclude}
\end{TEXT}
テキスト\ref{TEXT:boy_same}\,と\ref{TEXT:boy_exclude}\,では同一辞書見出
し語が繰り返されているが,類義語や上位概念語などの使用によって語彙的な
つながりが生じることもある.
次のテキスト\ref{TEXT:boy_lad}\,では``boy''の類義語``lad''が用い
られている.
\begin{TEXT}
\text There's a {\it boy} climbing that tree.
The {\it lad}'s going to fall if he doesn't take care.
\label{TEXT:boy_lad} 
\end{TEXT}

\vspace{-3mm}
\section{文の重要度の評価}
\label{sec:importance}

\subsection{テキスト構造と文の重要度に関する仮定}
\label{sec:importance:assumption}

本稿では,テキストを構成する文$S_1,S_2,\cdots,S_n$の間で次の条件が
成り立つと仮定する.
\begin{enumerate}
\item 冒頭文$S_1$はどの文にもつながらない.
\item $S_1$以外の各文$S_j$について,$S_j$が直接つながる先行文$S_i (i<j)$が
唯一つ存在する.
\end{enumerate}
この仮定は,二つの文(の構成要素)のつながりに,後続文(の構成要素)から先
行文(の構成要素)への方向性があることを意味する.
この方向性に対する反例として後方照応\cite{Hirst81}があるが,後方照応が
用いられることは希である
\footnote{提案手法の開発に際して訓練用に用いた英文テキスト20編では,代名
詞による照応のうち後方照応は3\%に満たなかった.}.
また,この仮定に従えば,文が同時に複数の先行文に直接つながることはない
ので,テキスト構造は,図\ref{fig:texttree}\,に示すように,冒頭文$S_1$
を根節点とする木で表される.
\begin{figure}[htbp]
\begin{center}
\epsfile{file=texttree.eps,width=0.7\columnwidth}
\end{center}
\caption{文の重要度の評価}
\label{fig:texttree}
\end{figure}

\ref{sec:introduction}\,節で述べたように,本稿では,文の重要度の評価を,
1) 表題はテキスト中で最も重要な文であり,2) 重要な文へのつながりが強い
文ほど重要な文であるという仮定に基づいて行なう.
この仮定は次のように具体化できる.
\begin{enumerate}
\item テキストの冒頭文$S_1$は,多くの場合,そのテキストの表題であるので,
$S_1$にはテキスト全体で最大の重要度を与える.
\item 冒頭文$S_1$以外の文$S_j$の重要度は$S_j$から先行文$S_i$へのつながりの
強さ(関連度)と$S_i$の重要度によって決まると考え,文$S_j$の重要度を求め
る式を次のように定める.
\begin{equation}
S_jの重要度 = \max_{i<j}\{S_iの重要度 \times S_iとS_jの関連度\}
\label{eq:importance}
\end{equation}
\end{enumerate}
文の重要度を(\ref{eq:importance})式で求めることにすると,
テキストの冒頭から順に処理を行なっていけば,テキストを構成する文すべて
の重要度が決定できるが,そのためには,二つの文の関連度をどのよう
にして求めるかを定めなければならない.

\subsection{二文間の関連度の評価}
\label{sec:importance:relevance}

提案手法への入力はテキストの形態素解析結果である.
形態素解析によってテキスト中の各語の辞書見出し語と品詞が得られる.
今回利用した形態素解析系からの出力では品詞は一意に決定されている.
以降,品詞が名詞,人称代名詞,動詞,形容詞,副詞のいずれかである辞
書見出し語を重要語と呼ぶ.

文$S_j$の先行文$S_i$へのつながりの強さ(関連度)を求める式を次のように定
める.
\begin{equation}
S_iとS_jの関連度 = \frac{S_j中の重要語のうちS_iの題述中の重要語につ
ながるものの重みの和}{S_iの題述中の重要語の数}
\label{eq:relevance}
\end{equation}

(\ref{eq:relevance})式の意味は
\ref{sec:importance:relevance:anaphora}\,節以降で説明する.
二つの重要語の間につながりがあるかどうかの判定は,
人称代名詞と先行(代)名詞の前方照応を検出すること
(\ref{sec:importance:relevance:anaphora}\,節)と,
同一辞書見出し語による語彙的なつながりを検出すること
(\ref{sec:importance:relevance:lexical}\,節)によって行なう.
重要語への重み付けについては
\ref{sec:importance:relevance:title_weight}\,節で述べ,
本稿でいう文の題述(rheme)の定義は
\ref{sec:importance:relevance:rheme}\,節で与える.

\vspace{-1mm}
\subsubsection{人称代名詞と先行(代)名詞の照応の検出}
\label{sec:importance:relevance:anaphora}
\vspace{-1mm}

人称代名詞と先行名詞または先行代名詞との照応を検出するためには,両者の
人称,性,数,意味素性をそれぞれ照合する必要がある.
しかし,今回は,名詞の性と意味素性が記述されていない辞書を用いたので,
照応の検出は両者の人称,数をそれぞれ照合することによって行なった.

しばしば指摘されるように,代名詞との間で照応が成り立つ先行(代)名詞は,
その代名詞を含む文$S_j$あるいは$S_j$の直前の文$S_{j-1}$に現れ
ることが多い\footnote{訓練テキスト20編では,人称代名詞
による前方照応のうち96\%がこのような事例であった.}ので,先行(代)名詞
の検索対象文を$S_j$と$S_{j-1}$に限定する.
検索は$S_j$,$S_{j-1}$の順で行ない,$S_j$中の(代)名詞との照合が成功した
場合は,$S_{j-1}$に対する処理は行なわない.

\vspace{-1mm}
\subsubsection{重要語の語彙的なつながりの検出}
\label{sec:importance:relevance:lexical}
\vspace{-1mm}

二つの文に現れる重要語が文字列として一致するとき,両者の間に語彙的なつ
ながりがあるとみなす.
ここでは,\ref{sec:coherence}\,節で述べたような,二つの語が同一事象に言及
しているかどうかの区別は行なわない.
文字列照合において,照合対象が両方とも単語である場合は,二つの重要語が
完全に文字列一致したときに限り照合成功とみなすが,照合対象の両方または
いずれか一方が辞書に登録されている連語である場合は,両者が前方一致また
は後方一致したときも照合成功とみなす.
例えば,``put pressure on''と``put''は前方一致で,``cabinet meeting''
と``meeting''は後方一致で照合が成功する.

二つの文がある一定の距離以上離れていると,それらに含まれる重要語の文字
列照合が成功しても二つの文の間に直接的なつながりはないと考えられる.
このため,二文間の距離に関して制限を設ける.
提案手法を開発する際に訓練用として用いた英文テキスト20編において,
文字列照合が成功する重要語(人称代名詞は除く)を含む二つの文の間の距離と,
その重要語が二つの文を直接つなぐ役割を実際に果たしているかどうかとの関
連を調べた結果に基づいて,処理対象範囲を文$S_j$から五文前までの先行文
$S_i (j-5 \le i < j)$とする.

直観的には,単に処理対象範囲を制限するだけでなく,文字列照合が成功する
重要語を含む二文間の距離に応じて照合結果に重み付けを行なう方が自然か
もしれない.
このため,訓練テキストを対象とした実験において,
文$S_j$から五文前までの先行文$S_i$の範囲で,
二つの文の距離が離れるにつれてつながりの強さが弱まるように重み付けを試
みた.
しかし,重み付けを行なわない場合の再現率と適合率を上回る結果は得られな
かった.
このため,本稿では処理範囲を制限するに留める.

\vspace{-1mm}
\subsubsection{表題語への重み付け}
\label{sec:importance:relevance:title_weight}
\vspace{-1mm}

テキストの表題中に現れる重要語(以降,表題語と呼ぶ)は,そのテキストに
おいて重要な情報を伝えると考えられる.
従って,表題語を含む文の重要度を大き
くするために,他の重要語に与える重みの値よりも大きな値を与える
こと\cite{Edmundson69,Mase89,Watanabe96}が適切である.
本稿では,表題語への重み付けを行なう際にテキスト中での表題語の出
現頻度を考慮する.
すなわち次のような仮定を置く.
\begin{quote}
表題語を含む文の重要性は,表題語がテキスト中に頻繁に現れる場合は,表題
語を含まない文の重要性に比べて特に高いわけではないが,表題語がテキスト
中に希にしか現れない場合には,表題語を含まない文に比べて特に高くなる.
\end{quote}
訓練テキスト20編を分析した結果に基づいて,表題語を含む文の数がテキス
トの総文数の$1/4$以下である場合に限り,表題語の重みを$w (> 1)$とする.
表題語以外の重要語の重みは常に1とする.
\[
重要語kwの重み = \left\{
\begin{array}{lp{0.6\columnwidth}}
w (> 1) & $kw$が表題語であり,かつ$kw$を含む文の数が総文数の1/4以下の場合 \\
1 	& その他
\end{array}
\right.
\]
重み$w$の具体的な値は,訓練テキストを対象とした実験で再現率と適合率が
できるだけ高くなるように調整し,最終的に$w=5$とした.

\vspace{-1mm}
\subsubsection{先行文の題述へのつながり}
\label{sec:importance:relevance:rheme}

テキストは,通常,先行文$S_i$における題述(rheme)が文$S_j$においてその
主題(theme)として受け継がれ,それに新たな情報が付け加わるという形で展
開する
\cite{Givon79}.
従って,文$S_j$の先行文$S_i$へのつながりの強さの評価を,$S_j$が$S_i$
の題述をどれだけ多く主題として受け継いでいるかに基づいて行なう.

主題と題述は,文の前半部分が主題,後半部分が題述というように文中の位置
で区別されることが多い\cite{Fukuchi85}が,本稿では,文中の位置ではなく,
関連文とのつながりに基づいて区別する.
ここで,$S_j$の関連文とは,\ref{sec:importance:assumption}\,節の
(\ref{eq:importance})式において,\{$S_i$の重要度$\times$$S_i$と$S_j$の
関連度\}の値が最大となるときの先行文$S_i$を意味する.
この値を最大にする先行文が複数存在する場合は,$S_j$との距離が最も近い
ものを関連文と呼ぶ.
関連文とのつながりに基づいて主題と題述を次のように定める.
\begin{quote}
文$S_j$の主題は,$S_j$中の重要語のうち$S_j$の関連文中の重要語につなが
るものから構成され,
文$S_j$の題述は,つながらない重要語から構成される.
ただし,関連文を持たない冒頭文$S_1$では,それに含まれる重要語すべてが
題述を構成する.
\end{quote}
例えば,図\ref{fig:texttree}\,において,括弧\{と\}で括った英大文字を各文
に現れる重要語とすると,各文の主題と題述は表\ref{tab:theme-rheme}\,のよ
うに分けられる.
\begin{table}[htbp]
\caption{
図\protect\ref{fig:texttree}\,の各文の主題と題述}
\label{tab:theme-rheme}
\begin{center}
\begin{tabular}{|c||c|l|l|}\hline
文 & 関連文 & \multicolumn{1}{|c}{主題} & \multicolumn{1}{|c|}{題述}
\\\hline\hline
$S_1$     & ---	  & \multicolumn{1}{c|}{---} & A,B,C \\
$S_2$     & $S_1$ & A	  		     & D,E   \\
$S_3$     & $S_2$ & A,D,E		     & F     \\
$S_4$     & $S_1$ & B,C   		     & G     \\
:         & :     & \multicolumn{1}{c|}{:}   & \multicolumn{1}{c|}{:} \\
$S_{j-1}$ & $S_2$ & D     		     & H     \\\hline
\end{tabular}
\end{center}
\end{table}

\subsection{重要文選択手順と処理例}
\label{sec:importance:algorithm}

\ref{sec:importance:assumption}\,節と
\ref{sec:importance:relevance}\,節で述べた考え方に従って重要文を選ぶ処
理は図\ref{fig:algorithm}\,のようにまとめられる.
\begin{figure}[htbp]
\samepage
\begin{center}
\fbox{
\begin{minipage}{0.9\columnwidth}
\vspace*{0.5em}
\setcounter{algocounter}{0}
\begin{ALGO}
\step 入力を形態素解析する.
\step 表題語への重み付け処理を行なう.
\step 冒頭文$S_1$の重要度を次式で求める.
\[ S_1の重要度 = \frac{S_1中の重要語の重みの和}{S_1の重要語の数} \]
\label{ALGO:init}
\step 各文$S_j (j=2,3,\cdots,n)$について,
$S_j$から五文前までの先行文$S_i$の範囲$(j-5 \le i < j)$で,
\ref{sec:importance:assumption}\,節の(\ref{eq:importance})式と
\ref{sec:importance:relevance}\,節の(\ref{eq:relevance})式に従って重要度
を求める.
\step あらかじめ定められた数だけ文を重要度の順に選択し,それらをテキス
トでの出現順に出力する.
\end{ALGO}
\vspace*{0.5em}
\end{minipage}
}
\end{center}
\caption{重要文選択手順}
\label{fig:algorithm}
\end{figure}
例として,図\ref{fig:example}\,のテキストを処理して得られる結果を表
\ref{tab:example_result}\,に示す.
このテキストでは,表題語``amorphous'',``Si'',``TFT''を含む文はそれ
ぞれ三文,三文,五文存在し\footnote{表題も含めて数えている.},いずれ
もテキスト総文数10文の$1/4$を越えるので,表題語への重み付けは行なわれ
ない.
表\ref{tab:example_result}\,の「つながり語」欄に現れる記号$\phi$は,
先行文の題述中の重要語につながる重要語が存在しなかったことを意味する.
このテキストからは,文選択率が25\%に設定されているとき,表題$S_1$,
$S_1$につながる文$S_4$,$S_4$につながる文$S_6$の三文が重要文として選び
出される.
\begin{figure}[htbp]
\samepage
\begin{center}
\fbox{
\begin{minipage}{0.9\columnwidth}
\vspace*{0.5em}
\small{
\begin{NEWS}
\item[$S_1$] Amorphous Si TFT
\item[$S_2$] Active matrix LCDs which are typically used in products
such as LCD color TVs are controlled by a switching element known as a
thin-film transistor or thin-film diode placed at each pixel. 
\item[$S_3$] The fundamental concept was revealed in 1961 by RCA of
America, a U.S. company, but basic research only began in the 1970's.
\item[$S_4$] Amorphous Si TFT LCDs introduced in 1979 and 1980 have
become the mainstream for today's active matrix displays.  
\item[$S_5$] These units place an active element at each pixel, and
taking advantage of the non-linearity of the active element, are able
to apply sufficient drive-voltage margin to the liquid crystal itself,
even with the increase in the number of scan lines. 
\item[$S_6$] As shown in Figure 1, TFT LCDs that use amorphous Si
thin-film transistors (TFTs) as the active elements are becoming the
mainstream today, and full-color displays achieving contrast ratios of
100:1 and which compare favorably to CRTs are being developed. 
\item[$S_7$] The driver electronics for TFT LCDs consist of data-line
drive circuitry that applies display signals to the data lines (source
drivers) and scanning line drive circuitry that applies scanning
signals to the gate lines (gate drivers).  
\item[$S_8$] A signal control circuit to control these operations and
a power supply circuit complete the system. 
\item[$S_9$] Liquid crystal materials used in TFT LCDs are TN (twisted
nematic) liquid crystals, but despite the fact that pixel counts have
increased and a drive element is placed at each pixel, we have still
been able to rapidly increase the contrast, viewing angle, and image
quality of these displays. 
\item[$S_{10}$] However, manufacturing technologies to fabricate
several hundred thousand such elements onto the surface of a large
screen are extremely problematic, and the fundamental approach
developed in 1987 is still being used today. 
\end{NEWS}
}
\vspace*{0.5em}
\end{minipage}
}
\end{center}
\caption{テキスト例}
\label{fig:example}
\end{figure}
\begin{table}[htbp]
\caption{図\protect\ref{fig:example}\,のテキストに対する処理結果}
\label{tab:example_result}
\begin{center}
\small{
\begin{tabular}{|c||c|l|c|c|c|}\hline
文&関連文&\multicolumn{1}{|c|}{つながり語}&関連度&重要度&選択順位
\\\hline\hline 
$S_1$   &---  &\multicolumn{1}{|c|}{---}&---&1&1\\
$S_2$   &---  &$\phi$&0&0&---\\
$S_3$   &---  &$\phi$&0&0&---\\
$S_4$   &$S_1$&amorphous, Si, TFT&3/3&1&2\\
$S_5$   &$S_4$&active&1/9&1/9&7\\
$S_6$   &$S_4$&LCD, active, become, mainstream, today, display&6/9&2/3&3\\
$S_7$   &$S_4$&LCD, display&2/9&2/9&4\\
$S_8$   &$S_7$&signal&1/13&2/117&8\\
$S_9$   &$S_4$&LCD, display&2/9&2/9&5\\
$S_{10}$&$S_6$&element, develop, use&3/16&1/8&6\\\hline
\end{tabular}
}
\end{center}
\end{table}

\section{実験と考察}
\label{sec:experiment}

重要文選択実験には英文報道記事100編を用いた.
100編のテキストを訓練用の20編と試験用の80編に分けた.
まず,訓練テキスト20編を対象として実験を繰り返し,再現率と適合率ができ
るだけ高くなるように,\ref{sec:importance:relevance:title_weight}\,節で
述べた表題語の重みを調整した.
次に,訓練テキストを対象とした実験で最も高い再現率と適合率が得られた設
定で,試験テキスト80編を対象として実験を行なった.

テキストの総文数は,訓練テキストの場合,最も短いもので15文,最も長いも
ので36文,一テキスト当たりの平均では26.2文であり,
試験テキストの場合,それぞれ12文,64文,29.0文であった.

各テキストについて,第三者(一名)によって重要と判断された文を,選択すべ
き正解文とした.
人手による正解文の選択では,システムが行なっているような各文についての
選択順位付けは行なわず,テキスト中の各文についてそれが重要な文であるか
そうでないかを判断するに留めた.
正解文の数は,訓練テキストの場合,平均で元テキストの総文数の20.8\%であ
り,試験テキストの場合17.9\%であった.

\subsection{訓練テキストでの実験結果}

\ref{sec:importance:relevance:title_weight}\,節で述べた表題語への重み付
けに関して次のような三種類の設定で,各訓練テキストについて正解文と同じ
数だけ文を選択した場合の平均精度(再現率と適合率は同じ値となる)を表
\ref{tab:training}\,に示す.
\begin{CONFIG}
\config
表題語を含む文の数がテキスト総文数の$1/4$以下である場合に限り,
表題語の重みを5とする.表題語以外の重要語の重みは1とする.
\label{CONFIG:freq}
\config
表題語の重みをその出現頻度に関係なく常に5とする.表題語以外の重要語の
重みは1とする.
\label{CONFIG:always}
\config
表題語の重みを他の重要語の重みと同じ1とする.
\label{CONFIG:none}
\end{CONFIG}

\newpage

\begin{table}[htbp]
\caption{訓練テキスト20編での実験結果}
\label{tab:training}
\begin{center}
\begin{tabular}{|c||c|c|c|} \hline
設定 & \ref{CONFIG:freq} & \ref{CONFIG:always} &
\ref{CONFIG:none}\\\hline\hline 
精度 & 71.0\% & 70.0\% & 62.5\% \\\hline
\end{tabular}
\end{center}
\end{table}
表\ref{tab:training}\,によれば,設定\ref{CONFIG:freq}\,での精度が
最も高くなっており,
\ref{sec:importance:relevance:title_weight}\,節で示した,出現頻度を
考慮した表題語への重み付けが有効であることがわかる.

\subsection{試験テキストでの実験結果}

訓練テキストを対象とした実験で最も高い再現率と適合率が得られた設定で,
80編の各試験テキストについて正解文と同じ数だけ文を選択した場合の精度は,
平均で72.3\%であった.
各テキストごとの精度分布を図\ref{fig:distri}\,に示す.
\begin{figure}[htbp]
\begin{center}
\input{distri.tex}
\end{center}
\caption{提案手法による精度分布}
\label{fig:distri}
\end{figure}

文選択率を5\%から100\%まで五刻みで変化させたときの平均再現率と平均適合率
の変化の様子を図\ref{fig:rec_pre}\,に示す.図\ref{fig:rec_pre}\,には,精
度比較のために実装した重要語密度法による実験結果を併せて示す.重要語密度
法に関して改良手法が提案されている\cite{Suzuki88}が,ここでは次式で文$S$
の重要度を評価した.

\[ 文Sの重要度 = \frac{文S中の各重要語のテキスト全体での出現頻度の和}
{文S中の重要語の数} \]
図\ref{fig:rec_pre}\,によれば,一般的な抄録において適切な文選択率であ
るとされる20\%から30\%までの付近で,特に,提案手法の精度が重要語密度法
の精度を大きく上回っている.
\begin{figure}[htbp]
\begin{center}
\input{denst_source.tex}
\end{center}
\caption{提案手法と重要語密度法の精度比較}
\label{fig:rec_pre}
\end{figure}

提案手法の精度と,インターネット上で試用可能なシステムAと,市販されて
いる三つのシステムB,C,Dの精度を比較した.
それぞれの平均再現率と平均適合率を表\ref{tab:comparison}\,に示す.
システムA,B,C,Dの文選択率は,各システムの既定状態で選ばれた文の数と
テキストの総文数から逆算したものである.
提案手法の文選択率は,四システムの文選択率とほぼ同じである25\%とした.
表\ref{tab:comparison}\,によれば,一般ユーザに利用されている実動システム
の精度を提案手法の精度が上回っており,提案手法の実用的な抄録システムと
しての有効性が示されている.

\begin{table}[htbp]
\caption{提案手法と他の実動システムの精度比較}
\label{tab:comparison}
\begin{center}
\begin{tabular}{|c||r|r|r|} \hline
 & \multicolumn{1}{c}{再現率} &
\multicolumn{1}{|c|}{適合率} & 
\multicolumn{1}{|c|}{文選択率} \\\hline\hline
提案手法    & 78.2\% & 57.7\% & 25\% \\%\hline
システムA & 72.3\% & 52.6\% & 26\% \\%\hline
システムB & 61.7\% & 39.5\% & 29\% \\%\hline
システムC & 61.4\% & 40.9\% & 29\% \\%\hline
システムD & 57.5\% & 42.2\% & 27\% \\\hline
\end{tabular}
\end{center}
\end{table}

\subsection{考察}
\label{sec:experiment:discussion}

提案手法によって正解文に与えられた重要度が小さく,正解文が選択されなかっ
た原因を分析した.
ここでは代表的な原因を二つ挙げる.
一つは,辞書見出し語の文字列照合では,語彙的なつながりが捉えられなかっ
たことである.
あるテキストでは,``shooting''と``gunfire''の類義関係が把握できないた
め,``gunfire''を含む正解文はどの先行文にもつながらないとみなされ,
重要文として選択できなかった.
このような語彙的なつながりを捉えるためにはシソーラスが必要となるが,他
のテキストでは,辞書見出し語の文字列照合の代わりに語基(base)の文字列照
合を行なえば,つながりが捉えられる可能性もあった.
例えば,``announce''と``announcement''は,辞書見出し語としては異なるが
語基は同一であるので,文字列照合が成功するだろう.

本研究では,一般ユーザに利用される実動システムへの組み込みを前提として,
高速な処理を実現することを目標の一つとした.
実動システムでは,プロトタイプシステムと異なり,重要文選択の精度と共
に処理速度も重要視される.
シソーラスの検索に比べて,文字列照合は処理効率の点で有利である.

正解文に十分大きい重要度が与えられなかったもう一つの原因は,テキストが
複数のサブトピックから構成されていることであった.
一般に,トピックが切り替わると,それまでとは異なった語彙が用いられるよ
うになる.
このため,提案手法のように同一辞書見出し語による語彙的なつながり(と人
称代名詞による前方照応)に基づいて文と文のつながりを評価する手法では,
トピックが切り替わる文から先行文へのつながりが弱いと判定され,トピック
切り替わり文に対して与えられる重要度は小さくなる.
従って,トピック切り替わり文が正解文であるようなテキストでは,高い精度
を得ることが難しくなる.

\section{おわりに}

本稿では,人称代名詞による前方照応と,同一辞書見出し語による語彙的なつ
ながりを検出することによって,テキストを構成する各文と表題との直接的な
つながりまたは他の文を介しての間接的なつながりの強さを評価し,その
強さに基づいて各文の重要度を決定する手法を提案した.
平均で29.0文から成る英文テキスト80編を対象とした実験では,
文選択率を25\%に設定したとき,再現率78.2\%,適合率57.7\%の精度を得,提案
手法が比較的短いテキストに対して有効であることを確認した.

複数のサブトピックから成るような比較的長いテキストの扱いは今後の課題で
ある.
同一辞書見出し語の出現頻度と出現分布を利用してトピックの切り替わりを検
出し\cite{Hearst97},各サブトピックごとに提案手法を適用すると,長いテ
キストに対してどの程度の精度が得られるかを今後検証したい.


\bibliographystyle{jnlpbbl}
\bibliography{v06n1_03}

\begin{biography}
\biotitle{略歴}
\bioauthor{吉見 毅彦}
{1987年電気通信大学大学院計算機科学専攻修士課程修了.
現在,シャープ(株)ソフト事業推進センターにて機械翻訳システムの研究開発
に従事.
在職のまま,1996年より神戸大学大学院自然科学研究科博士課程在学中.}

\bioauthor{奥西 稔幸}
{1984年大阪大学基礎工学部情報工学科卒業.
同年シャープ(株)に入社.
1985〜89年(財)新世代コンピュータ技術開発機構に出向.
現在,同社情報システム事業本部ソフト事業推進センターに勤務.
機械翻訳システムの研究開発に従事.}

\bioauthor{山路 孝浩}
{1990年大阪市立大学理学部数学科修士課程修了.
同年シャープ(株)に入社.
1993〜95年(財)新世代コンピュータ技術開発機構に出向.
現在,同社OAシステム事業部においてワープロの開発に携わる.}

\bioauthor{福持 陽士}
{1982年インディアナ大学言語学部応用言語学科修士課程修了.
翌年,シャープ(株)に入社.
現在,情報システム事業本部ソフト事業推進センター副参事.
機械翻訳システムの研究開発に従事.}

\bioreceived{受付}
\biorevised{再受付}
\bioaccepted{採録}

\end{biography}

\end{document}

