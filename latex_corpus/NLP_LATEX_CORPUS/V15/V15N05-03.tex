    \documentclass[japanese]{jnlp_1.4}
\usepackage{jnlpbbl_1.1}
\usepackage[dvips]{graphicx}
\usepackage{amsmath}
\usepackage{hangcaption_jnlp}
\usepackage{udline}
\setulminsep{1.2ex}{0.2ex}
\let\uline
\let\underline
\usepackage{ascmac}
\usepackage{here}


\Volume{15}
\Number{5}
\Month{October}
\Year{2008}

\received{2007}{12}{26}
\revised{2008}{4}{27}
\accepted{2008}{5}{28}

\setcounter{page}{45}


\jtitle{整合性を考慮した物語要約システムの構築}
\jauthor{横 野   光\affiref{Okayama_uni}\affiref{Tokyo_ins}}
\jabstract{
 物語は複数の話題で構成された文書である.内容の理解にはこの展開していく
 話題を正しく把握しなければならず,
 そのために原文書の代わりに用いられる要約は特に整合性を重視する必要が
 ある.
 本稿では整合性として話題の繋がりに着目した物語要約手法を提案する.
 提案手法では,まず
 物語を主題に着目した話題単位に分割し,登場人物に着目した重要度によって
 要約として抽出する話題を決定する.その後,話題間の整合性を保つために,
 話題間の状況の変化を示す文を補完する.
 提案手法の有効性を確認するため実際の物語を対象とした被験者の主観的評価
 による比較実験を行った.整合性を考慮しな
 いtf$\cdot$idfを用いた重要文抽出に比べて,提案手法の方が内容の理解にお
 いて良好な結果を得ることができた.
}
\jkeywords{物語要約,報知的要約, 整合性,登場人物抽出, 話題抽出}

\etitle{Construction of story summarization system toward producing coherent summary consistent \\
	with story line}
\eauthor{Hikaru Yokono\affiref{Okayama_uni}\affiref{Tokyo_ins}} 
\eabstract{
 Since a story consists of several scenes and topics, for making a
 summary of a story, it is essential to get hold of relations between
 topics. This means that to make a coherent summary is a key issue for
 informative summary of a story. On the basis of this background, in
 this paper, the author proposes a method to produce a coherent summary
 of story focusing on extracting (1) topic block that consists of
 sentences that may be written on the same topic, and (2) complement
 sentences that may express change of scenes. They are extracted on the
 basis of automatic topic recognition and identification of
 characters. The experimental results of summarization for 9 stories
 show the proposed method produces easier-to-follow summaries than those
 of a tf$\cdot$idf based model.
}
\ekeywords{Story summarization, Informative summary, Coherence, Character extraction, Topic extraction}

\headauthor{横野}
\headtitle{整合性を考慮した物語要約システムの構築}


\affilabel{Okayama_uni}{岡山大学大学院自然科学研究科}{
	Graduate School of Natural Science and Technology, Okayama University}
\affilabel{Tokyo_ins}{現:東京工業大学精密工学研究所}{The present address: Precision and Intelligence Laboratory, Tokyo Institute of Technology}



\begin{document}
\maketitle


 \section{はじめに}\label{hajime}

 インターネットの拡大により大量の文書情報が入手可能となった現在において,
 ユーザが自分の望む情報を手早く手に入れるための要素技術として要約が重要
 となってきている.近年の自動要約の研究では新聞記事や論説文,議事録,特
 許文書を対象とするものが多い.こうした文書は論理的な構造を持つため,そ
 の文書構造を利用した要約手法が提案され,一定の成果が上げられている
 \cite{yamamoto1995,hatayama2002}.

 一方で,より多くの人がインターネットを使うようになり,Web上で多くの文芸作品が公開され,自
 由に読むことができるようになった.さらに,著作権の切れた文学作品を電子テキスト化し公開している
 青空文庫\footnote{http://www.aozora.gr.jp}のようなインターネット電子図
 書館も存在している.こうした背景から電子化された多くの文学作品や物語か
 ら好みに応じた,読みたい作品を探す手段としての要約(指示的要約)の必要性があると考え
 られる.また,近年``あらすじ本''と呼ばれる複数の文学作品のあらすじを
 まとめて紹介している本が出版されていることから,その内容を簡潔にまとめた原文書の代
 わりとして機能する要約(報知的要約)まで必要とされていることが伺える.物
 語の指示的要約には結末を含まず,物語の展開においてある程度重要な箇所
 を含んでいることが必要とされる.これに対して,重要な箇所の推定は物語全体
 の構成を把握することが必要である.よって本研究では物語に対して報知的要
 約を作成する手法の構築を目標とする.これにより同時に指示的要約もカバー
 することができると考える.

物語は登場人物が遭遇した出来事と登場人物の行動の描写で構成さ
 れている.出来事は基本的に時系列順に記述されるため,論説文に見られる
 ような,主張する事柄を中心としてその前後に根拠や前提を配置するといった
 論理的な構成はほとんど存在しない.
 さらに,論説文では著者の主張が述べられている箇所が重要であるとされ,“〜する必
 要がある”や“〜すべきである”といった文末表現を手がかり語として要約作
 成に利用することができる.しかし,物語ではどの箇所が重要であるかは全体
 の流れや他の箇所との関係から決定されるため,そのような手がかり語を定義
 することができない.
 従って,新聞記事や論説文を対象としているような要約手法では物語の要約に適応しない
 と考えられる.

 また,新聞記事の要約では背景となる前提知識を読者が保有しているために
 文章の繋がりが悪くてもある程度は推測によって補完することができるため,
 要約中に記事中の重要文がいくつか存在すれば要約として機能する.これに対して物語では背景とな
 る前提知識は物語固有であることが多いため,その要約は対象とする物語の重要な要素を含むだけでなく
 要約中の整合性まで考慮しなければ要約として十分に機能することができない.
 整合性とは文書の意味的なまとまりの良さのことであり,本稿では話題間の繋
 がりの良さのことを示す.

 

 本研究では話題の繋がりに焦点を置いた物語要約システムを構築する.物語は登
 場人物の行動を中心に展開していくことから,まず登場人物を自動抽出して,それ
 を軸に話題にまとまりのある重要箇所(\ref{method}章参照)を取り出す.さら
 に重要箇所間の繋がりを補完し読みやすさを向上させるために,局所的重要度
 を測定し重要箇所間の連結を考慮した文抽出を提案する.本手法を評価するた
 めに物語9作品を用いた複数人による人手の要約評価を行い,ベースラインとし
 てtf$\cdot$idfを利用した重要文抽出手法との比較を行う.

 \section{関連研究}\label{related}

 自動要約の主な手法の一つに重要文抽出がある.これはテキスト中の文に対し
 て重要度を計算し,その上位から要約率分だけ抽出するという手法である.重要度の計算にはtf$\cdot$idfを用いる手法\cite{hirao2001}や,単語の共起を利
 用する手法\cite{sunayama2000}などがある.
 
 これらの手法においては重要度は文そのものにしか着目しておらず,抽出された文同
 士の関係については考慮されていない.そのため生成された要約には以下のよ
 うな問題が生じる\cite{okumura}.
 \begin{enumerate}
  \item 要約文に代名詞が含まれていても,その先行詞が要約中に存在しない場
	合がある.
  \item 全体を通して首尾一貫性に欠ける要約が生成される可能性がある.
 \end{enumerate}

 1.の問題に対しては照応解析を行い,代名詞をその先行詞に置換するなどといっ
 た対応で解消できる.2.の問題は生成された要約の読みやすさにも関係してい
 る.読みやすい要約の作成は自動要約の研究において大きな課題であり,様々な研
 究が行われている.

 話題の繋がりを考慮した先行研究も存在する.市丸ら\cite{itimaru2005}は文中で共
 起する単語を話題と定義し,その連想による文間,段落間の繋がりに着目した
 手法を,館林ら\cite{tatebayasi2006}はテキストをセグメントに分割し,そのセグメン
 ト内での重要性とセグメント間の結束性に関する重要性を同時に考慮した要約
 手法を提案し,それぞれ一定の成果を上げている.要約全体のまとまりを考慮
 しているという点では,本研究はこれらの研究と関連があるが,より物語に即
 しており,単語の共起ではなく登場人物とその行動に着目しているという点で異なっ
 ている.

 

 自動要約を含め現在行われている言語処理の研究では新聞記事や論説文を対象
 としているものが多いが,物語を対象とした研究も行われている.物語の構造
 に関する研究として,石井ら\cite{isii2006}は登場人物に着目した事象
 をネットワークで繋ぐことによる構造化を,馬場ら\cite{baba2007}は物語テキ
 ストの柔軟な検索のために人物に着目したモデル化を行っている.
 また,小林\cite{kobayashi2007}は場所,時間,登場人物に着目した場面境界の認定
 手法を提案している.

 一方,物語を対象とした要約の研究に関して,
 Kazantseva~\cite{Anna}は短編小説の指示的な要約の作成を目的としたSVMによる重要
 文抽出を行っている.また,横野\cite{yokono2007}は登場人物の感情表現に着
 目した文抽出による要約を,村上ら\cite{murakami2004}は命題間の関係に着目
 した要約手法を,山本ら\cite{syousetu}は隣接文間の結束性に基づいた提案している.
 
 他にも,物語の登場人物
 に着目した手法としてLehnertの提案したplot unitによるものがある
 \cite{plotunit}.これは物語の構造の中心は登場人物の感情的な反応と感情の状
 態,それらの因果関係にあるという考えに基づいている.この要約手法では物語中の出
来事に対して登場人物が良い感情を抱いたか悪
 い感情を抱いたかという情報を付与し,それらを組み合わせて物語の構造を組
 み立て,要約を作成する.しかし,実現に関しては出来事に対して登場人物がどのような感情
 を抱いたかと推定することは困難である.

 本研究は物語の登場人物に着目するという点ではLehnertと同様であるが,行動
 から登場人物の感情を推測するといった深い意味処理は行わない.また,整合
 性に焦点を当てているという点で他の物語の自動要約手法とは異なってい
 る.


 \section{整合性を考慮した要約文の抽出}\label{method}

 \ref{hajime}章で述べたように整合性とは意味的な文書のまとまりの良さのこ
 とであり,これを実現するためには適切に話題間の繋がりを扱うことが重要であると考え
 る.ここで話題とは同じ対象について述べている一つのまとまりのことである.
 物語では時間の経過や舞台となる場所の移動によって区切られる場面という
 大きなまとまりが考えられるが,本稿では話題を場面よりも小さな単位と捉え,
 一つの場面は複数の話題から構成されていると考える.

 本稿で提案する手法の流れを図\ref{flow}に示す.
 \begin{figure}[b]
  \begin{center}
   \includegraphics{15-5ia3f1.eps}
  \end{center}
 \caption{提案手法の流れ}
  \label{flow}
 \end{figure}

 提案手法は要約生成のための3つの手順と,その手順において利用する人物辞書
 を構築するための登場人物抽出で構成される.登場人物抽出では人物辞書の他
 に未知語として出現した登場人物表現を登録した形態素辞書を作成する.こ
 の形態素辞書は要約生成の各手順で行う処理に利用する.

 要約生成に関する手順について説明する.
 
 まず一つ目は物語中にいくつか存在する場面内
 において話題が一貫した箇所(以降{\bf トピック・ブロック}と呼ぶ)があると仮定
 して,トピック・ブロック抽出を行う.文を単位とする既存の要約手法
 では文間の関係が考慮されず,そのために一連の事象が複数の文に渡って記
 述されている場合に正しく要約を抽出することができない.よってトピック・
 ブロックの導入により話題が一貫した単位を要約の最小単位として,不用意に
 話題を切断しないことが期待できる.

 二つ目としてトピック・ブロックから要約に必要となる重要なトピック・ブロッ
 ク(以降{\bf 重要ブロック}と呼ぶ)を選択する
 ために,登場人物辞書を用いて登場人物を軸とした重要度を提案する.これは
 登場人物が物語のストーリーの理解に重要な要素である\cite{brian1981}とい
 う観点から,ト
 ピック・ブロック中に登場人物の活動が記述されている文があるものを重要と
 する.
 
 三つ目として重要箇所として取り出されたトピック・ブロック間の話題の整合
 性を補完するための文(以降{\bf ブロック連結文}と呼ぶ)を抽出する.これは
 重要箇所の抽出というタスクは取り出した文の整合性やまとまり
 を求めるというタスクと直交していることが原因であり,重要ブロックのみを
 取り出しても,ブロック間に不整合が生じるためである.この処理は文を単位
 として行う.
 
 これは不整合の解消には場面の切り替えや主人公の移動などの話題間の繋がり
 を示唆す
 る文が物語には存在するはずで,それらは大抵 1 文単位で記述されていること
 が期待できるからである(図\ref{segments}).

\begin{figure}[b]
  \begin{center}
   \includegraphics{15-5ia3f2.eps}
  \end{center}
  \caption{重要ブロックとブロック連結文}
  \label{segments}
 \end{figure}

 生成する要約の要約率と重要ブロックとブロック連結文の割合はパラメータと
 して人手で設定する.重要ブロックの割合を大きくすれば物語に重要な文を多
 く含む要約を生成し,
 逆にブロック連結文の割合を大きくすれば,整合性のある要約を生成する.

 次の
 \ref{extract}節で登場人物抽出について述べた後,\ref{topic}節でトピック・
 ブロック抽出について述べる.\ref{select}節で重要ブロック抽出について説
 明し,\ref{connectsentence}節でブロック連結文の抽出について説明する.



  \subsection{登場人物抽出}\label{extract}

  登場人物を抽出するというタスクにおける問題は以下の2点である.
  \begin{list}{}{}
   \item[(a).] 登場人物はその物語に固有の表現であるものが多く,それらは単語と
	 して一般的な辞書に登録されていない.
   \item[(b).] (a)と関連しているが『さるかに合戦』に登場する``うす''のように人
	 名でないものが人物化することがある.
  \end{list}
  従って,登場人物の抽出に際して人名辞書などの既存の辞書に頼った方法
  では抽出漏れが生じてしまう.

  そこで本研究では物語文から次の2段階の手法で登場人物を推定する.
  \begin{enumerate}
   \item 文字列中から登場人物候補を文字種情報と形態素情報を利用して獲得
	 し,辞書登録を行う.
   \item 登録した辞書を用いた係り受け解析器で文を解析し,登場人物候補の単
	 語の係り先の動詞の意味から人物表現を同定する.
  \end{enumerate}

  
  馬場ら\cite{baba2007}は形態素解析で得られた品詞情報を利用した人物抽出の手法を採用し
  ている.また網羅性の向上のために人名辞典から人名を収集し形態
  素辞書に登録している.この手法は対象とする物語のジャンルを限定すれば有
  効である.これに対して本研究では対象のジャンルを限定しない.そのため,
  辞書を利用した手法ではなく,物語における登場人物表現の出現の仕方に着目した
  手法を提案する.

  上記の2段階の手法について具体的に記述する.まず物語文から登場人物候補を取り出すために文字n-gramを利用して頻度の高
  い部分文字列を獲得する.n-gramの単位を形態素ではなく文字にしている理由
  は,登場人物表現が辞書に登録されていないために生じる形態素解析誤りの影
  響を抑えるためである.これによって例えば『さるかに合戦』で``〜とこがには〜''など
  の表記から``こがに''という正しい候補を獲得することができる.

  次に,得られた部分文字列に対してMeCab\footnote{http://mecab.sourceforge.net}による
  形態素解析を行ってから,以下のフィルタリングを行い残ったものを登場人物候補と
  してMeCabの辞書に登録する.
  \begin{itemize}
   \item 平仮名,カタカナの小文字から始まっていない
   \item 句読点,カギ括弧,``!'',``?''を含まない
   \item 文字列中に接続詞,連体詞,代名詞を含まない
   \item 複数の形態素で構成されている
   \item 文章中に``は''を接続した形で出現する
  \end{itemize}
  上記のルールにより,単語としてあり得ない文字列や,例えば``そして杜子春'',``杜子春は''などのように人
  物表現の周辺でよく使われる表現を含んだ文字列を人物候補から削除できる.

  フィルタリングで残った候補に対して,同頻度で出現したn-gram文字列で
  他の文字列の部分文字列であるような候補は削除し,残った候補をMeCabの辞
  書に``名詞,固有名詞,一般''として追加する.これにより既に形態素辞書に登
  録されてある人物候補と同様に物語固有の人物表現を扱うことができる.ここ
  で作成した形態素辞書は以降で説明する要約生成に関する処理にも利用する.

  ここで登場人物が現れる際の文の特徴について考える.登場人物は物語において重要な役割を担っているため,主題として取り上げら
  れることが多い\cite{danwabunseki}.また,登場人物は意志を持つものであ
  るため,意志
  性のある動作を記述している文の動作主として現れることが多い.意志性のあ
  る動作とはその動作の実現に対して動作主が能動的に関与しているとみなせる
  動作のことである.
  この傾向を利用して,文中の名詞が登場人物として出現しているかどうかを推定する.
  具体的には物語の地の文に対して,更新した辞書を用いた
  CaboCha\footnote{http://www.chasen.org/\~{}taku/software/cabocha/}による
  係り受け解析を行い,主題化する機能を持つ副助詞``は''を後接して複数回出現する名詞で,その文の述部となる動詞に一つでも意志性のあ
  る動作を示すものがあれば,その名詞を登場人物と推定する.
  意志性のある動作を示す動詞の判定には日本語和語動詞LCS \cite{ecs}を利用
  した.これは日本語和語動詞を中心に約1000個の動詞に関してその語彙概念構造
  を記述を試みたものであり,実際にLCSが付与されている延べ742項
  目の動詞を利用した\footnote{意志性のある動詞として,例えば``話す'',
  ``押す'',``立つ''などがあった.}.『風の又三郎』に対する人物抽出の例を付
  録\ref{humanexample}に示す.

  人物抽出の精度を計るために\ref{eval}章の評価実験
  で用いた物語9編を使用した予備実験を行った.複数の被験者に登場人物を列
  挙してもらい,そ
  のうち過半数の被験者が登場人物と認定したものを正解とした.部分文字列を
  獲得する際のnの範囲は2〜10とし,出現頻度が2以上の部分文字列のみを対象
  とした.実験の結果,
  平均の再現率は0.74,適合率は0.68となった.人手で作成した正解の例と実験
  の詳細な結果を付録\ref{humexp}に示す.

  \subsection{トピック・ブロックの抽出}\label{topic}

  物語文に対して話題がある程度まとまった範囲で出現する部分を仮定しトピック・
  ブロックとして抽出する.

  ある文書の話題は何か,そしてその話題の切れ目はどこかという課題は大変難
  しい問題であり,例えば単語の共起と単語の重要度を利用した手法
  \cite{hirao2000}や,語の反復距離を使用した手法\cite{nakano2006}などが
  提案されているが決定的な手法はまだ確立していない.

  本稿では作者が記した1文の表現に着目し,1文単位の主題を仮定してそれが連
  続する文をまとめることでトピック・ブロックを作成する.厳密な連続でト
  ピック・ブロックを作成すると,小さなトピック・ブロックが多く作られ,本
  来は 1 つのまとまりであるべき箇所が過分割される可能性がある.これを防ぐ
  ため,ウィンドウ幅を設定し,その中での連続を見る.本稿ではこのウィンド
  ウ幅を前2文とした.従って,別の主題の文を間に挟んで同じ主題を持つ文は同
  じトピック・ブロックに属する.

  1文の主題とは文中に出てきた単語で表現するならば主語が有力な候補であり,
  特に副助詞``は''は主語を取り上げて主題化する機能を有している.そこで副助詞``は''
  が後接する名詞を1文の主題と見なし,ない場合
  は助詞``が''が後接する名詞を主題とする.ただし,``今日は'',``このごろ
  は''など``は''と接続して副詞的な働きを持つ名詞は主題とは考えにくいので,MeCabを利用し
  て副詞可能となる名詞は排除する.また,主題は省略されやすく,同じ主題が
  連続する場合は後に続く文において明示的に現れないことが多い
  \cite{nariyama2002}.従って,文中に助詞``は''と``が''がない場合は,そ
  れが省略されているとみなし
  前文における主題をその文の主題とする.

  
  助詞``は'',``が''を後接している名詞が代名詞の場合,センタリング理論に基づいた単純
  な照応解析を行う\cite{ishizaki}.具体的には代名詞が出現した文の1つ前の
  文に出現する名詞を先行詞候補とし,その候補が接続する助詞について以下の
  優先順に従って先行詞を決定する.

\makeatletter
\def\footnotemark{}
\def\footnotetext{}
\makeatother

\vspace*{0.5\baselineskip}
  \begin{screen}
   \begin{center}
    は>が>に>を>その他\footnotemark
   \end{center}
  \end{screen}
\footnotetext[\thefootnote]{格助詞の``で''や副助詞などがこれに含まれる}
\vspace*{-0.5\baselineskip}

\noindent
具体例を『杜子春』(芥川龍之介)の一部から示す.

\vspace*{0.5\baselineskip}
  \begin{screen}
   {\small
   するとどこからやって来たか、突然\underline{彼}の\underline{前}へ\underline{足}を止めた、
   \underline{片目眇}の\underline{老人}があります。

   {\bf それ}が夕日の光を浴びて、大きな影を門へ落すと、じっと杜子春の顔を見な
   がら、「お前は何を考えているのだ」と、横柄に声をかけました。
   }
   \end{screen}

\noindent
   2文目にある代名詞``それ''の先行詞の候補として1文目から下線部の5個の名
   詞が挙げられる.このうち上記の優先順から助詞``は''を後接している``老
   人''を``それ''の先行詞とみなす.

   センタリング理論に基づいた照応解析の精度を計るため,本稿の評価実験で
   用いた物語9編を対象に予備実験を行った.助詞``は'',``が''を後接する
   代名詞に対して出力された先行詞を人手で正否を判定した.実験の結果,照
   応解析の平均の精度は0.41(134/330)であった.

  1文ごとの主題を抽出した後,
 
  設定した幅のウィンドウの中で
 連続した主題の文をまとめることでトピック・ブロックを作成する.このと
  きウィンドウ幅内で全く同じ主題が連続した文だけでなく,同じ主題に挟まれ
  た文も同一のトピック・ブロックに含める.これは1文ごとの主題は主語を手
  がかりに獲得しているが,話の展開においては連続しない場合があるためであ
  る.

  上記の操作はすべて会話文は考慮せず,いわゆる地の文のみを対象とする.こ
  のとき会話文は直前の地の文と同じトピック・ブロックに属することとする.こ
  れは会話の途中で話題が転換することはほとんどないと考えられるからである.


  例えば,『赤ずきんちゃん』(グリム兄弟)において以下に示す部分で
  は途中でおばあさんの台詞が入っているが,おおかみを中心に話が
  展開している.従って,この部分をひとまとまりのトピック・ブロックとみなす.

\vspace*{0.5\baselineskip}
  \begin{screen}
   {\small
 ところが、このあいだに、すきをねらって、\underline{おおかみ}(主題)は、
   すたこらすたこら、おばあさんのおうちへかけていきました。そして、とん
   とん、戸をたたきました。\footnotemark\\
「おや、どなた。」\\
「赤ずきんちゃんよ。お菓子とぶどう酒を、おみまいにもって来たのよ。あけてちょうだい。」\\
「とっ手をおしておくれ。おばあさんはご病気でよわっていて、おきられないのだよ。」\\
 \underline{おおかみ}(主題)は、とっ手をおしました。
}
  \end{screen}
  \footnotetext[\thefootnote]{前処理の照応解析でこの文の主題を「おおか
  み」と判定している.}
\vspace*{-0.5\baselineskip}


\noindent
  さらに1文が多重のトピック・ブロックに所属することを認める.
  具体的にはある文$s_i$に対してその文からウィンドウ幅分だけ前の文のトピック・ブロッ
  ク
  中で主題が同じ文が存在すれば,その文が属するトピック・ブロックに$s_i$も属する
  とする.同時にその文と$s_i$間の文も同じトピック・ブロックに属する.これは,ある文が示す主題には助詞``は''や``が''によって明示的
  に示されるものと文脈から推定できるものの2種類が考えられ,その文はそれぞ
  れの主題で構成される話題の一部分とみなせると考えるからである.

  ウィ
  ンドウ内に主題が同じ文がなければその文を新しいトピック・ブロックに加え
  る.
  例をあげて説明する.文$s_0$から$s_4$があり,その主題が表\ref{example}
  のようであったとき,ここから抽出されるトピック・ブロックは図
  \ref{tbexample}のようになる.

   \begin{table}[b]
    \caption{文と推定された各文の主題の例}
    \label{example}
   \begin{center}
\input{03table01.txt}
   \end{center}
  \end{table}

  \begin{figure}[t]
\input{03fig03.txt}
\caption{抽出されたトピック・ブロック}
\label{tbexample}
  \end{figure}

  文$s_0$,$s_1$は前に同じ主題を持つ文が存在しないためそれぞれ新しいトピッ
  ク・ブロックに割り当てる.文$s_2$はウィンドウ幅内に同じ主題を持つ文$s_0$が存在するため,その文
  が属するトピック・ブロック$P_0$に加える.さらにその間にある文$s_1$も$P_0$に加える.一方,文$s_4$は文$s_1$と同じ主題を持つが,$s_1$がウィンドウ幅内にない
  ので新しいトピック・ブロックに属することになる.


  \subsection{重要ブロックの決定}\label{select}

  抽出したトピック・ブロックについて物語におけるストーリーへの関与度を評
  価し,関与しているものが高いものを重要ブロックとして要約に採用する.
  手法としてはまず重要度の評価尺度を導入し(\ref{importance}節参照),次に
  物語の構成や接続詞情報を考慮して重要ブロックの抽出を行う
 (\ref{extracttb}節参照).


  \subsubsection{重要度計算}\label{importance}

  物語においては,登場人物が行動を起こし,それによって状況が変化するような
  箇所が重要であると考えられる.これに基づいてトピック・ブロックの重要度
  を定義する.まずトピック・ブロック内の各文のスコア付けについて説明し,
  その後ブロック全体の重要度を求める.トピック・ブロック内を構成する文に
  対して,登場人物や動作主の変化に関する動詞が含まれている場合は評価が高
  くなるようにスコア付けを行う.具体的には以下の式で決定する.
  \[
   Score_{sentence}(s)=\sum_{w \in s}(Terms(w) \cdot tw(w,s)) \cdot weight(s)
  \]
  $Score_{sentence(s)}$は文$s$に対するスコアであり,$w$は$s$中に出現する単語である.$Terms(w)$は単語$w$の文書全体におけ
  る重要度を示しており,$w$が登場人物である場合にその頻度を重要度とする.
  \[
   Terms(w)=  \begin{cases}
    文書Dにおける単語wの出現頻度 & (wが登場人物表現)\\
    0 & (上記以外)\\
   \end{cases}
  \]
  $tw(w,s)$は単語$w$の文$s$における主題性に基づく重みである.
  \[
   tw(w,s)= \begin{cases}
    1.2 & (単語wが文sの主題)\\
    1 & (上記以外)\\
   \end{cases}
  \]
  $weight(s)$は文$s$の述部における動詞に着目して,動作主体が変化を伴うよ
  うな動詞であれば重要度を高くするように設定する.この判定には竹内ら
  \cite{takeuti2006}の語彙概念構造辞書を利用する.この辞書では``主体の変
  化''という分類があり,例えば``寝る'',``振り向く''といった動詞が分類さ
  れている\footnote{この辞書には4153の動詞の動詞に対して分類が記述されて
  おり,このうち``主体の変化''の分類に属する動詞は635ある.}.
  \[
   weight(s)= \begin{cases}
    1.2&(文s中の動詞の意味に``主体の変化''が含まれている)\\
    1&(上記以外)\\
   \end{cases}
  \]
  トピック・ブロック$P$の重要度$Score_{tb}(P)$は上記で決めた文の重要度を
  利用して以下の式で求める.
  \[
   Score_{tb}(P)=\frac{\sum\limits_{s_i\in P}Score_{sentence}(s_i)}{|P|}
  \]
  ここで$|P|$は$P$中の文の数である.これにより正規化することで長さに依存
  しないスコアとなる.


  \subsubsection{重要ブロックの抽出}\label{extracttb}

  トピック・ブロックに対して求めた重要度に基づいて重要ブロックを決定する.
  基本的には要約率を満たすまで重要度の高いトピック・ブロックから選んでい
  くが,その前にどのような物語でも比較的重要と考えられるトピック・ブロッ
  クを重要ブロックとして選ぶ.

  物語では,登場人物表現が最初に出現する文はその登場人物の説明であること
  が多く,登場人物表現が最後に出現する文はその登場人物の最終状況を
  説明している文であることが多い.そこで,重要度の高いものからトピック・ブロックを選択する前に,
  登場人物表現が最初に出現するトピック・ブロックと最後に出現するトピック・ブロッ
  クを重要ブロックとして抽出し,残りのトピック・ブロックに対して重要度の上位から順に要約率分だけ抽出する.

  物語のあらすじにとって関与が低い要素を省くために,抽出した重要ブロックに
  対して文の連接関係に着目した要約を行う.ここで不必要な要素とは同じこと
  を繰り返して述べていたり,補足説明をしている要素のことである.
  連接関係の推定には接続詞と文末表現を利用する.接続関係の分類は永野
  \cite{bunsyouron}による7種類の関係(表\ref{connect})を利用した.

  \begin{table}[t]
    \caption{連接関係の種類}\label{connect}
   \begin{center}
\input{03table02.txt}
   \end{center}  
  \end{table}
  \begin{table}[t]
    \caption{連接関係と対応する接続詞と文末表現の例}\label{conexample}
   \begin{center}
\input{03table03.txt}
   \end{center}
  \end{table}

  具体的な手法について述べる.あらかじめ接続詞と文末表現のそれぞれに
  対して,それが示す連接関係との対応を作成しておく.一部の関係について対応の例を表
  \ref{conexample}に示す.
隣接する2文に対して,
  後文に出現する接続詞と文末表現から作成した対応を利用し連接関係を推定す
  る.後文に連接関係の推定に使用する接続詞や文末表現がなかった場合は2文
  間の連接関係は``展開''とする.推定した連接関係が``補足'',``同格''
  と判定されたものについて,その後文を削除する. 


  \subsection{ブロック連結文の抽出}\label{connectsentence}

  隣接する重要ブロック間での整合性を保つために,その間において生じた状況
  の変化を示すような文を挿入する.状況の変化とは時間経過や場所移動による
  場面転換や,登場人物の心境や取り巻く環境の変化などのことである.物語中
  でこのような状況の変化が起こったとき,それを示すような文がなければ読者
  はそれに気付かず,読み進めていくにつれて違和感を覚えたり,物語本文を読
  んだときに得られる理解と異なる理解をする可能性がある.このような
  状況の変化を示す文をブロック連結文と呼び,要約に挿入することで重要ブロック間の整合性を損なわないようにする.

  本稿では重要ブロックと原文においてその間に存在するトピック・ブロックで
  の登場人物表現の出現を基にした局所的な重要度を定義し,これを利用してブロッ
  ク連結文を抽出する.


  \subsubsection{局所的重要度の計算}

  隣接する重要ブロックにおいて片方に出現し,且つ,もう片方には出現していない
  ような登場人物があれば,その登場人物は重要ブロック間の状況の変化に関わっ
  ていると考えられる.そのような登場人物について記述してある文をブ
  ロック連結文として抽出するために,隣接する重要ブロック間での出現の仕方
  によって登場人物の局所的な重要度を決定し,それを基にして重要ブロック間における
  文の重要度を決定する.

  隣接する重要ブロック$P_i=\{s_j,\dots,s_k\}$,
  $P_{i+1}=\{s_m,\dots,s_n\} (j<k<m<n)$に対して,この間における登場人物
  $t$の局所的重要度を以下の式で求める.
  \[
   Terms_{local}(t,P_i,P_{i+1})=\sum_{s_j\le s \le s_n}freq(t,s) \cdot lw(t,P_i,P_{i+1})
  \]
  $freq(t,s)$は文$s$における登場人物表現$t$の出現頻度であり,
  $lw(t,P_i,P_{i+1})$は単語$t$がセグメントの連結に対する有効性を表す重みであ
  る.$P_i$と$P_{i+1}$間の文の集合を$IP_i=\{s_{k+1},\dots ,s_{m-1}\}$と
  する.
  
  \[
   lw(t,P_i,P_{i+1})=\left\{
       \begin{array}{lp{20zw}}
    1 & 単語tが$IP_i$で出現し,かつ,重要ブロック$P_i,P_{i+1}$のいずれか一方にのみ出現する\\
    0.5 & 単語tが重要ブロック$P_i,P_{i+1}$と$IP_i$の全てで出現する\\
    0 & 上記以外 \\
   \end{array}
\right\}
  \]

  この単語の重要度と\ref{importance}節で利用した動詞の意味分類を用いて,
  重要ブロック$P_i$,$P_{i+1}$間の
  文$s_i$の局所的重要度$Score_{local}(s_i,P_i,P_{i+1})$を以下のように定
  める.
  \[
   Score_{local}(s_i,P_i,P_{i+1})=\sum_{t\in s_i}Terms_{local}(t,P_i,P_{i+1}) \cdot weight(s_i)
  \]
  $weight(s_i)$は重要ブロックの決定で用いた動詞の意味による重みである.
  これは動詞の意味分類において動作主体の変化を表す動詞として分類されてい
  るものであり,これらの動詞が出現した文には重要ブロック間で起きた状況の
  変化が記述されてある可能性が高いためスコアを少し上げる.


  \subsubsection{ブロック連結文抽出}

  局所的な重要度に基づいて抽出するブロック連結文を決定する.要約として
  抽出した隣接する重要ブロック$P_k$,$P_{k+1}$について,その間にある文の
  局所的重要度を上記の式$Score_{local}(s_{i},P_i,P_{i+1})$によって計算し,下記に示す
  ブロック連結文の量$CS(IP_k)$以下で$Score_{local}(s_{i},P_i,P_{i+1})$の高いものから順
  にブロック連結文として抽出する.

  ブロック連結文は要約として抽出された重要ブロックに属していない文から選択する.生成する
  要約の要約率$x$とする.要約率は原文書の文字数に対する要約文書の文字数
  の割合で与えられ,以下の式から求める.
  \[
   要約率=\frac{要約文書の文字数}{原文書の文字数}
  \]

要約における重要ブロックの割合を$\alpha$とすると,
  隣接する重
  要ブロック$P_k$,$P_{k+1}$間の文集合$IP_k$に対して取り出すべき文の量
  $CS(IP_k)$は以下のようになる.
  \[
  CS(IP_k)=\frac{(1-\alpha)x}{1-\alpha x}\cdot |IP_k|
  \]
  ここで$|IP_k|$は$IP_k$に属する文の量である.割合$\alpha$を大きくすると
  限られた要約率の中で重要ブロックを多く抽出することを意味し,反対に
  $\alpha$を小さくすると整合性を重視して連結文を多く取り出すことを意味す
  る.

   \begin{figure}[t]
\input{03fig04.txt}
\vspace{-1\baselineskip}
\caption{要約に抽出されたブロック連結文の例(『雪の夜』(小林多喜二))}\label{conrei}
  \end{figure}

  本手法によって生成した要約中に出現するブロック連結文の例を図\ref{conrei}に示す.下線部が抽出
  したブロック連結文であり,その前後は要約として抽出した重要ブロックの一部である.図
  \ref{conrei}に示すように“郊外の家へ帰ろうと思った。”という部分から主
  人公の龍介が“通りを出た。”までに汽車に乗り損ねたことが推論できる内容
  が補完されている.


 \section{評価}\label{eval}

 提案手法の有用性を確認するために実験を行った.報知的な要約では原文書の内容をどれだけ正確に抽出できて
 いるかということが重要となる.本稿では,要約の整合性と原文書の適切な部分が要約として抽出できてい
 るかを被験者による実験によって確認する.

 形態素解析器MeCabやChaSen,係り受け解析器CaboChaなどの既存の
 言語処理ツー
 ルは基本的に現代仮名遣いと口語文法に則って書かれた文を処理の対象として
 いる.このため,実験の対象とする物語は,青空文庫で入手で
 きる口語文法と現代仮名遣いで書かれている3人称視点の物語9作品とした(付録 \ref{data}).3人称視点の物語に限定している理由は,1人称視点で書かれた物
 語では,地の文に語り手となる登場人物の心理描写が交じり,登場人物の行動
 を述べる文との判別が困難であると考えたためである.

 比較手法にはtf$\cdot$idfによる重要文抽出(以降tf$\cdot$idf法と呼ぶ)を採用した.tf$\cdot$idf
 は単語の重要度を決定する手法の一つであり,ある文書集合において特定の文書
 に出現する単語を重要であるとみなす.本稿では,文中に出現
 する名詞の重要度をtf$\cdot$idfによって求め,その和をその文の重要度とし,重要度の高いものから順に要約率を満たすまで文を抽出することで要約を生成する.
 文書$d$における名詞$t$に対する重要度$weight(t,d)$は以下の式で求める.
 \[
  weight(t,d)=tf(t,d) \cdot (log\left(\frac{N}{df(t)}\right)+1)
 \]
$tf(t,d)$は文書$d$中の名詞$t$の出現頻度であり,$df(t)$は文書集合中で名詞
$t$が出現している文書数,$N$は文書集合に含まれる文書の数である.本実験
で利用する文書集合は上記で挙げた物語9作品とした(N=9).


 \subsection{評価方法}

 要約の評価でよく用いられる方法はあらかじめ人手で正解データを作成してお
 き,それとシステムの出力を比較するというものである.この方法は正解デー
 タとなる人手で作成された要約が個人による差異が小さければ有効である.

 本稿では人手による要約の個人差を計るため,3人の被験者に重要文抽出による要約
 率30\%の要約と要約率50\%の要約,本手法で作成したトピック・ブロック
 単位での重要箇所抽出による要約率30\%の要約
 を作成してもらい,それぞれのkappa値\cite{jean1996}を算出した.その結果を表
 \ref{kappa}に示す.
 kappa値が0.7を越えると個人差が少ない良いデータだと一般的に考えられてい
 る.
 これに従うと,人手による要約は個人差が大きく,唯一の正解データを作成す
 ることは困難であると言える.

 また,30\%要約に関して文単位での要約とトピック・ブロック単位での要約を
 比較すると,文単位での要約におけるkappa値に比べてトピック・ブロック単位
 での要約におけるkappa値の方が作品ごとの差が大きいという結果になった.

 \begin{table}[t]
   \caption{人手による要約のkappa値}\label{kappa}
  \begin{center}
\input{03table04.txt}
  \end{center}
 \end{table}

 そこで本稿では正解データを利用した評価ではなく,被験者による主観的な評
 価を行った.実験項目は以下の通りである.
 \begin{itemize}
  \item 整合性に関する評価
  \item 内容理解に関する評価
 \end{itemize}

 本研究において整合性とは話題のまとまりの良さとみなしている.話題のまと
 まりの良い文書とは,読者がその文書を読んで予測できる展開と文書の実際の
 展開が似ているような文書だと考えられる.逆に話題のまとまりが悪い文書で
 は読者の予測と文書の実際の展開が異なることがあり,このとき読者はその文
 書が読みにくいと感じることがある.そこで本稿では整合
 性の評価として,10人の被験者に各作品毎に提案手法による要約と
 tf$\cdot$idf法による要約の2種類の要約を提示し,読みやすさの評価,つまり要約文が話として繋がっ
 ているかどうかを評価し,良いと思う方を選択するよう指示した.どちらでも
 ないという選択肢は設けず,各作品に対して必ずどちらかを選択してもらうよ
 うにした.

 後者の内容理解に関する評価では被験者が物語の内容を知っていると正しい評
 価ができないということが考えられる.そこで実験を行う際に物語の内容を知らない被験者を選び,
 全ての作品に対して評価を行った.また,提案手法かtf$\cdot$idf法のどちらか
 片方の要約を読んでから,もう片方の評価を行うと後者の評価の際に前者で読んだ要
 約から得られた知識を被験者が無意識に使ってしまう可能性がある.そのため,
 被験者にはどちらか片方の要約しか提示しない.
 この実験では1作品あたり提案手法による要
 約に対する評価とtf$\cdot$idf法による要約に対する評価のそれぞれについて8人
 の被験者で行っている.内容理解に関する評価は以下の手順で行った.

 \begin{list}{}{}
  \item[{\bf 作業1.}] 対象の作品に対する提案手法か比較手法のどちらか一方
		      の要約を読み,そのあらすじを自由筆記で作成する.こ
		      こで被験者には要約が提案手法によるものか
		      tf$\cdot$idf法によ
		      るものかは知らせない.
  \item[{\bf 作業2.}] 原文書を読んで,作業1で書いたあらすじを被験者が自
		      分で修正する
 \end{list}

 これは報知的要約が理想的なものであるならば,そこから得られる理解と原文
 書を読んだときに得られる理解は同じであるという仮定に基づいている. 上記
 の作業2において修正数が少なければ,その要約は良いと評価できる.

 自由筆記でのあらすじ作成では,単文で記述することと,名詞
 に関しては要約内に出現した単語のみを使用することを被験者に指示した.修
 正の方法は以下の3種類のどれかに従うように指示した.
 \begin{itemize}
  \item 訂正(1文中の一部を追加削除)
  \item 追加(1文を追加)
  \item 削除(1文を削除)
 \end{itemize}
 作成するあらすじには個人差が生じるが,これは修正数をあらすじ中の文の数
 で正規化することで対処する.

 評価に使用した要約の要約率は提案手法,tf$\cdot$idf法ともに30\%に設定
 した.また提案手法では重要ブロックとブロック連結文の割合を設定する必要
 がある.
 本研究では物語の内容を簡単に把握できるための報知的要約の生成を
 目的としている.そのため要約には物語に重要な要素を多く含んでい
 る必要がある.しかし,重要な要素を羅列するだけでは文書としてのまとまり
 に欠け,それが内容の理解の妨げになる可能性がある.このことから重要な要
 素を中心として構成され,文書のまとまりを考慮した要約が良い要約であると
 考える.これを反映した要約を生成するため,本稿では生成する要約における
 重要ブロックの割合は0.7と設定した.
 
 
 \subsection{結果と考察}\label{result}

 整合性の評価の結果を表\ref{yomiyasusa}に示す.
  数字は各手法の方が読みやすいと判定した人の数である.個人差によるばらつ
 きを考慮して,9作品の平均で整合性を評価すると提案手法がtf$\cdot$idfより
 も優れた結果を示していることが分かる.
 『風の又三郎』では評価が良くなかっ
 たが,これは後の内容理解での考察にも関係するが会話文の中に物語の展開に
 関わる表現が入っているため本手法ではうまく扱うことができなかったのが原
 因である.

 \begin{table}[b]
   \caption{整合性の評価結果}\label{yomiyasusa}
  \begin{center}
\input{03table05.txt}
  \end{center}
 \end{table}
 

 内容理解の評価の結果を表\ref{rikai}に示す.
あらすじの平均文数は被験者が
 作成したあらすじの文数の平均を取ったものである.評価値は以下の式に基づ
 いて決定した.
 \[
  評価値=\frac{(訂正数 \cdot 1+追加数 \cdot 2+削除数 \cdot 0.5)}{被験者が作業1で答えたあらすじの
  文数}
 \]
 修正の際に,新たに文を追加したということは内容の理解に必要な情報が要約
 には含まれていなかったことを示している.文の訂正は必要な情報は現れてい
 たもののその解釈で誤りがあったことを示している.このことから追加が最
 も重大な修正といえる.これを反映するために各修正項目にペナルティをかけ
 る.上記の評価式は修正項目にペナルティをかけた値の合計を修正前のあらす
 じの文数で正規化したものである.評価値が小さいほど要約を読んだときに得
 られる理解と原文書を読んだときに得られる理解の差が小さいと考える.

 \begin{table}[b]
   \caption{内容理解の評価結果}\label{rikai}
  \begin{center}
\input{03table06.txt}
  \end{center}
 \end{table}

 実験の結果,表\ref{rikai}の9作品に対する評価値の総合計から,提案手法の方が
 tf$\cdot$idf法よりも内容理解において原文との差異が少なく,被験者に対して
 よりよい理解を与えたことが示された.しかしながら,
 提案手法が大きく勝るというものではなく,またいくつかの作品では劣ってい
 る部分もある.これらについて考察を述べる.

 提案手法では重要ブロックの抽出の際に会話文の内容を考慮していない.その
 ため,物語の内容に対して会話文が重要である箇所があってもその部分を正し
 く抽出することができない.一方,tf$\cdot$idf法では地の文や会話文に関係なく,単語
 の重要度から抽出する文を決定しているので,会話文の中に重要度の高い語が
 含まれていればそれを抽出することができる.例として『風の又三郎』に対す
 る提案手法による要約(図\ref{teianrei})と,tf$\cdot$idf法による要約(図
 \ref{tfidfrei})を示す.この例ではtf$\cdot$idf法の方では物語の内容にとって重要と
 考えられる先生の台詞を会話文から抽出している.

 \begin{figure}[b]
\input{03fig05.txt}
\vspace{-1\baselineskip}
\caption{提案手法による要約の例}\label{teianrei}
 \end{figure}
 \begin{figure}[b]
 \input{03fig06.txt}
\vspace{-1\baselineskip}
\caption{tf$\cdot$idfによる要約の例}\label{tfidfrei}
 \end{figure}
 \begin{figure}[b]
 \input{03fig07.txt}
\vspace{-1\baselineskip}
\caption{提案手法による『名人伝』の要約(抜粋)}\label{meijintei}
 \end{figure}


 この他に提案手法がtf$\cdot$idf法に劣ってしまう要因として,重要箇所抽出の際
 に本来要約に必要とされるべき箇所が抽出できなかったということが考えられ
 る.提案手法ではトピック・ブロックを単位として重要箇所抽出
 を行っている.このため,重要箇所抽出で正しく重要ブロックが抽出できなかっ
 たとき,内容理解に関する評価において被験者が該当部分に関する全ての記述
 を行うことになる.これに対してtf$\cdot$idf法では文を単位として抽出を行っ
 ているため,重要な箇所が全て抽出できなくてもある程度の文は抽出できるこ
 とがある.この場合,内容理解の評価で被験者が追加する項目は提案手法と比
 べて少なくなり,結果としてtf$\cdot$idf法の方が良い評価になる.
 例えば,『名人伝』には主人公である紀昌とその師である飛衛が対決する場面が
 ある.提案手法による要約(図\ref{meijintei})ではこの場面が重要箇所と見なされずに抽出できて
 いないが,tf$\cdot$idf法では文単位で抽出されるため,ある程度は要約に出現
 している(図\ref{meijintf}).本稿で行った内容理解に関する実験において被
 験者はこの場面を重要と判断してあらすじに加えているが,提案手法による要
 約に対してはその場面に該当する記述を全て加筆している.一方,
 tf$\cdot$idf法による要約に対しては記述を補完するだけでよく,修
 正のために追加する文の数はtf$\cdot$idf法による要約の方が少なくなる.その
 結果,内容理解の評価ではtf$\cdot$idf法の方が良いという結果になったと考え
 られる.

\begin{figure}[t]
\input{03fig08.txt}
\vspace{-1\baselineskip}
\caption{tf$\cdot$idf法による『名人伝』の要約(抜粋)}\label{meijintf}
 \end{figure}

 しかし,整合性の評価において『名人伝』では提案手法が大きく上回ってい
 る.これはtf$\cdot$idf法による要約では細かな欠損が多く,それらが被験者に
 対して何度も違和感を生じさせ,結果として提案手法による要約の方が読みや
 すいという評価になったと考えられる.

 tf$\cdot$idf法では物語においてどのような単語が重要かという判断
 を文書集合に依存しているとみなせる.従って,一般的に使われるような語が
 ある物語において重要な要素であった場合,tf$\cdot$idfではこれを重要な語とみ
 なせないことがある.この点においてtf$\cdot$idf法による要約は作品によっ
 ては不安定になることが予測される.実際,表\ref{rikai}の評価値の分散をみ
 ると提案手法に比べてtf$\cdot$idf法の方が分散が大きく,内容理解に貢献でき
 る場合と劣る場合の差が小さくない.

 提案手法は整合性も考慮したつなぎの文も含めて要約率を達成して出力してい
 る一方で,tf$\cdot$idf法による要約では特徴的な名詞を多く含む文を出力している.よって
 tf$\cdot$idfは同じ要約率という制約で読みやすさを捨てた分,内容に関する
 文をより多
 く出力していると考えられる.この点が内容理解で大きな差が出なかった原因
 の一つである.

 しかしながら,その内容理解の質は提案手法とtf$\cdot$idf法による要約手法では同じでないと考えら
 れる.表\ref{rikai}において,被験者が作成したあらすじの文数を見ると全ての作品において提案手法の方が
 tf$\cdot$idf法よりも倍近く上回っている.
 
 もし物語の内容を十分理解していれば,より詳細なあらすじを作成することが
 できる.このとき,作成されたあらすじの文の数は多くなると考えられる.従っ
 て,提案手法による要約から作成されたあらすじの文の数が多いということは,
 tf$\cdot$idf法による要約に比べて,提案手法による要約はあらすじの作成に
 貢献する文を多く含んでいるとみなすことができる.本研究では物語の内容理
 解のための要約の生成を目的としており,このことから提案手法の方がより質
 の高い要約を生成できると考えられる.

 物語では何を重要と見なすかは人によって異なることが多い.そのため,人手
 で要約を作った場合でも個人差が大きく,また物語は個々の特徴が違うため,物語によ
 る違いも大きいと考えられる.表\ref{yomiyasusa},表\ref{rikai}から複数の人
 間による判断である合計値の平均と,そのばらつき具合である分散のいずれに
 おいて提案手法の方が上回っており,このことから提案手法はtf$\cdot$idf法に
 よる要約と比べて物語の違いによる影響を受けにくく,安定した要約を生成で
 きるという点で有効であると考えられる.


 
 本研究では重要ブロック間のつながりを補完するためにブロック連
 結文を導入している.その有効性を調べるため,ブロック連結文を挿入せずに
 重要ブロックの抽出のみで作成
 した要約と提案手法による要約とで整合性に関する比較評価を行った.

 \begin{table}[b]
   \caption{ブロック連結文の有無による整合性の比較評価} \label{yomiyasusaforalpha}
  \begin{center}
\input{03table07.txt}
  \end{center}
 \end{table}

 表\ref{yomiyasusaforalpha}は整合性についての評価結果である.9作品を対象
 にした平均では有意な差は見られなかった.これは短い物語(『杜子春』,
 『鼻』,『名人伝』(160文前後))に対して既に重要ブロックで十分な整合性が
 ありブロック連結文を挿入することで冗長となり評価を下げたと考えられる.
 よって短い物語に対してはブロック連結文を適用しないなどの手法を取り入れ
 ることができれば有意な差が得られると考えられる.

 一方,内容理解に対する評価では表\ref{comparerikai}に示すようにブロック
 連結文を適用することで評価値が0.86から0.84に減少し,向上が見られた.これは総合的には
 ブロック連結文が内容を補完したため内容理解に対して有効に働いたと考えら
 れる.
 
 提案手法の性能の限界を調べるために人物を人手で抽出し,かつ,トピック・
 ブロックの決定のための主題を人手で特定した場合の提案手法による要約について内容理解の評価実験を
 行った.その結果を表\ref{comparerikai}に示す.

 \begin{table}[t]
   \caption{内容理解の評価に関する手法の比較}\label{comparerikai}
  \begin{center}
\input{03table08.txt}
  \end{center}
 \end{table}

 提案手法による要約,重要ブロック抽出による要約と比較して,人手で前処理を行った場合での要約に対する評価がもっとも良かっ
 た.提案手法で
 は照応解析をセンタリング理論に基づいた単純な方法で行っており,実
 験の対象の物語に対する精度が0.41と低い.従って,照応解析の精度を上げる
 ことで提案手法の性能の向上が見込める.
また表\ref{yomiyasusaforalpha}においてブロック連結文の有無だけでは整合性
 の向上は全作品に対して明らかではなかったが表\ref{yomiyasusa}に示すとお
 り,重要ブロックが一貫した話題で構成されていればtf$\cdot$idf法に比べて
 全作品に対して整合性が勝ることを示した.よって表\ref{rikai}や表
 \ref{comparerikai}に示されるように内容理解において提案手法に基づく要約
 がtf$\cdot$idf法による要約に対して優れた結果が得られたのは結局,整合性
 を指向して一貫性のある要約を出力できたことが理由であると考えられる.




 \section{おわりに}

 本稿では整合性を考慮した物語の要約手法を提案した.整合性は読者が内容を
 正しく把握するために重要な要素であり,特に複数の話
 題について述べている文書では話題間の整合性が明らかでないと読者への負担
 が大きくなる.

 物語では場面転換などで話の状況の変化が何回も発生する.この状況の変化を把握で
 きていないと,実際に物語で述べられている状況と読者が理解している状況に
 差異が生じ,結果として間違った理解をしてしまう可能性がある.状況の変化に
 は場所の移動や時間の経過などがあるが,本手法は特にその状況に登場してい
 る人物の移り変わりに着目した.従来の重要箇所抽出による要約に,隣接する抽出した箇所間での整合性を保つよ
 うに,人物の移り変わりに関する文を挿入することで整合性のある読みやすい
 要約を作成した.

 tf$\cdot$idf法による要約との比較実験では,被験者の内容の理解の程
 度に関してわずかではあるが提案手法の方が原文から得られる内容を反
 映している要約を作成できることを示した.被験者による整合性の評価で行った読み
 やすさに関する実験では提案手法の方が読みやすいという評価が得られた.

 また,ブロック連結文の有無の違いに関する評価実験から,ブロック連結文だ
 けでなく重要ブロックとの両方の効果によって要約文全体の読みやすさ(整合性)
 の向上に対して有効であることを示した.さらに提案手法は内容理解において
 整合性を考慮していないtf$\cdot$idf法に比べて優れていることが評価結果か
 ら示された.

このことから整合性を考慮することでより内容を理解しやすい要約が生成できる
 と考えられる.

 

 提案手法ではトピック・ブロックの重要度計算に登場人物とその行動を考慮し
 たが,\ref{result}節で示したように会話文を考慮する必要があ
 る.
 
 
 また,より詳細な話題間の繋がりを計算するために,場所の移動や時間経過などの場面転換表現も考
 慮に入れる必要がある.これらが今後の課題である.


\bibliographystyle{jnlpbbl_1.3}
\begin{thebibliography}{}

\bibitem[\protect\BCAY{Carletta}{Carletta}{1996}]{jean1996}
Carletta, J. \BBOP 1996\BBCP.
\newblock \BBOQ Assessing Agreement on Classification Tasks: The kappa
  Statistic\BBCQ\
\newblock {\Bem Computational Linguistics}, {\Bbf 22}  (2), \mbox{\BPGS\
  249--254}.

\bibitem[\protect\BCAY{Kazantseva}{Kazantseva}{2006}]{Anna}
Kazantseva, A. \BBOP 2006\BBCP.
\newblock \BBOQ Automatic Summarization of Short Fiction\BBCQ\
\newblock In {\Bem Master thesis Ottawa-Carleton Institute for Computer Science
  School of Information Technology and Engineering, University of Ottawa}.

\bibitem[\protect\BCAY{Lehnert}{Lehnert}{1981}]{plotunit}
Lehnert, W.~G. \BBOP 1981\BBCP.
\newblock \BBOQ Plot units and narrative summarization\BBCQ\
\newblock {\Bem Cognitive Science}, {\Bbf 5}, \mbox{\BPGS\ 293--331}.

\bibitem[\protect\BCAY{Nariyama}{Nariyama}{2002}]{nariyama2002}
Nariyama, S. \BBOP 2002\BBCP.
\newblock \BBOQ Grammar for ellipsis resolution in Japanese\BBCQ\
\newblock In {\Bem the 9th International conference on Theoretical and
  methdological Issue in Machine Translation}.

\bibitem[\protect\BCAY{Reiser}{Reiser}{1981}]{brian1981}
Reiser, B.~J. \BBOP 1981\BBCP.
\newblock \BBOQ Character Tracking and the Understanding of Narratives\BBCQ\
\newblock In {\Bem 7th International Joint Conference on Artificial
  Intelligence}.

\bibitem[\protect\BCAY{馬場\JBA 藤井}{馬場\JBA 藤井}{2007}]{baba2007}
馬場こづえ\JBA 藤井敦 \BBOP 2007\BBCP.
\newblock \JBOQ 小説テキストを対象とした人物情報の抽出と体系化\JBCQ\
\newblock \Jem{言語処理学会第13回年次報告}.

\bibitem[\protect\BCAY{市丸\JBA 日高}{市丸\JBA 日高}{2005}]{itimaru2005}
市丸夏樹\JBA 日高達 \BBOP 2005\BBCP.
\newblock \JBOQ 要約文の話題の流れの最大化による自動要約\JBCQ\
\newblock \Jem{自然言語処理}, {\Bbf 12}  (6), \mbox{\BPGS\ 45--61}.

\bibitem[\protect\BCAY{石崎\JBA 伝}{石崎\JBA 伝}{2001}]{ishizaki}
石崎雅人\JBA 伝康晴 \BBOP 2001\BBCP.
\newblock \Jem{談話と対話}.
\newblock 東京大学出版会.

\bibitem[\protect\BCAY{奥村\JBA 難波}{奥村\JBA 難波}{2005}]{okumura}
奥村学\JBA 難波英嗣 \BBOP 2005\BBCP.
\newblock \Jem{テキスト自動要約}.
\newblock オーム社.

\bibitem[\protect\BCAY{永野}{永野}{1986}]{bunsyouron}
永野賢 \BBOP 1986\BBCP.
\newblock \Jem{文章論総説}.
\newblock 朝倉書店.

\bibitem[\protect\BCAY{横野}{横野}{2007}]{yokono2007}
横野光 \BBOP 2007\BBCP.
\newblock \JBOQ 登場人物の感情表現に着目した物語要約\JBCQ\
\newblock \Jem{言語処理学会第13回年次報告}.

\bibitem[\protect\BCAY{竹内\JBA 乾\JBA 藤田}{竹内\Jetal }{2006}]{takeuti2006}
竹内孔一\JBA 乾健太郎\JBA 藤田篤 \BBOP 2006\BBCP.
\newblock \JBOQ 語彙概念構造に基づく日本語動詞の統語・意味特性の記述\JBCQ\
\newblock \Jem{レキシコンフォーラム}, 2. ひつじ書房.

\bibitem[\protect\BCAY{加藤\JBA 畠山\JBA 坂本\JBA 伊藤}{加藤\Jetal
  }{2005}]{ecs}
加藤恒昭\JBA 畠山真一\JBA 坂本浩\JBA 伊藤たかね \BBOP 2005\BBCP.
\newblock \JBOQ 日本語和語動詞に関する語彙概念構造辞書構築の試み\JBCQ\
\newblock \Jem{言語処理学会第11回年次報告}.

\bibitem[\protect\BCAY{中野\JBA 足立\JBA 牧野}{中野\Jetal }{2006}]{nakano2006}
中野滋徳\JBA 足立顕\JBA 牧野武則 \BBOP 2006\BBCP.
\newblock \JBOQ 語の反復距離に基づく段落境界の認定\JBCQ\
\newblock \Jem{自然言語処理}, {\Bbf 13}  (2), \mbox{\BPGS\ 3--26}.

\bibitem[\protect\BCAY{館林\JBA 原口}{館林\JBA 原口}{2006}]{tatebayasi2006}
館林俊平\JBA 原口誠 \BBOP 2006\BBCP.
\newblock \JBOQ セグメント間の接続関係を考慮した文書要約に関する一考察\JBCQ\
\newblock \Jem{情報処理学会研究報告 NL-175}.

\bibitem[\protect\BCAY{メイナード}{メイナード}{1997}]{danwabunseki}
メイナード泉子・K. \BBOP 1997\BBCP.
\newblock \Jem{談話分析の可能性}.
\newblock くろしお出版.

\bibitem[\protect\BCAY{小林}{小林}{2007}]{kobayashi2007}
小林聡 \BBOP 2007\BBCP.
\newblock \JBOQ 場・時・人に着目した物語のシーン分割手法\JBCQ\
\newblock \Jem{情報処理学会研究報告 NL-179}.

\bibitem[\protect\BCAY{村上\JBA 上之薗\JBA 榎津\JBA 古宮}{村上\Jetal
  }{2004}]{murakami2004}
村上聡\JBA 上之薗和宏\JBA 榎津秀次\JBA 古宮誠一 \BBOP 2004\BBCP.
\newblock \JBOQ 物語の自動要約\JBCQ\
\newblock In {\Bem The 18th Annual Conference of the Japanese Society for
  Artificial Intelligence}.

\bibitem[\protect\BCAY{砂山\JBA 谷内田}{砂山\JBA 谷内田}{2000}]{sunayama2000}
砂山渡\JBA 谷内田正彦 \BBOP 2000\BBCP.
\newblock \JBOQ
  文章要約のための特徴キーワードの発見による重要文抽出法—展望台システム—\JBCQ\
\newblock \Jem{情報処理学会研究報告 NL-135}.

\bibitem[\protect\BCAY{平尾\JBA 前田\JBA 松本}{平尾\Jetal }{2001}]{hirao2001}
平尾努\JBA 前田英作\JBA 松本裕治 \BBOP 2001\BBCP.
\newblock \JBOQ Support Vector Machine による重要文抽出\JBCQ\
\newblock \Jem{情報処理学会研究報告情報学基礎 63-16}.

\bibitem[\protect\BCAY{平尾\JBA 北内\JBA 木谷}{平尾\Jetal }{2000}]{hirao2000}
平尾努\JBA 北内啓\JBA 木谷強 \BBOP 2000\BBCP.
\newblock \JBOQ
  語彙的結束性と単語重要度に基づくテキストセグメンテーション\JBCQ\
\newblock \Jem{情報処理学会論文誌}, {\Bbf 41}  (SIG3), \mbox{\BPGS\ 24--36}.

\bibitem[\protect\BCAY{畑山\JBA 松尾\JBA 白井}{畑山\Jetal
  }{2002}]{hatayama2002}
畑山満美子\JBA 松尾義博\JBA 白井諭 \BBOP 2002\BBCP.
\newblock \JBOQ 重要語句抽出による新聞記事自動要約\JBCQ\
\newblock \Jem{自然言語処理}, {\Bbf 9}  (4).

\bibitem[\protect\BCAY{山本\JBA 増山\JBA 酒井}{山本\Jetal }{2006}]{syousetu}
山本悠二\JBA 増山繁\JBA 酒井浩之 \BBOP 2006\BBCP.
\newblock \JBOQ 小説自動要約のための隣接文間の結束性判定手法\JBCQ\
\newblock \Jem{言語処理学会第12回年次報告}.

\bibitem[\protect\BCAY{石井\JBA 小方}{石井\JBA 小方}{2006}]{isii2006}
石井理恵\JBA 小方孝 \BBOP 2006\BBCP.
\newblock \JBOQ
  登場人物の履歴情報からの物語ネットワークの構成とそれを利用した物語の作成—ハ
イパーコミックの一般化と自動化に向けて—\JBCQ\
\newblock In {\Bem The 20th Annual Conference of the Japanese Society for
  Artificial Intelligence}.

\bibitem[\protect\BCAY{山本\JBA 増山\JBA 内藤}{山本\Jetal
  }{1995}]{yamamoto1995}
山本和英\JBA 増山繁\JBA 内藤昭三 \BBOP 1995\BBCP.
\newblock \JBOQ 文章内構造を複合的に利用した論説文要約システムGREEN\JBCQ\
\newblock \Jem{自然言語処理}, {\Bbf 1}  (2).

\end{thebibliography}


\begin{biography}
\bioauthor{横野  光(正会員)}{
2003年岡山大学工学部情報工学科卒.2008年同大大学院自然科学研究科産業創成
工学専攻単位取得退学.同年東京工業大学精密工学研究所研究員,現在に至る.
修士(工学).自然言語処理の研究に従事.情報処理学会会員.
}


\end{biography}


\biodate



\clearpage
\appendix

\section{実験に使用した小説}\label{data}

  \begin{table}[H]
    \caption{実験に使用した小説}
   \begin{center}
\input{03table09.txt}
    \end{center} 
  \end{table}
\footnotetext[\thefootnote]{台詞はカギ括弧でくくられたまとまりを1    
    文とみなす.}


\section{登場人物抽出の例}\label{humanexample}

 \begin{figure}[H]
  \begin{center}
   \includegraphics{15-5ia3f9.eps}
  \end{center}
  \caption{人物候補辞書作成の例}
  \label{ngram2dic}
 \end{figure}
\clearpage

 \begin{figure}[t]
  \begin{center}
   \includegraphics{15-5ia3f10.eps}
  \end{center}
\caption{意志性による人物抽出の例}
  \label{volitional}
 \end{figure}
\begin{figure}[t]
\input{03fig11.txt}
\vspace{-1\baselineskip}
\caption{非人物と判定された候補の文の例}
\label{nonvolexample}
\end{figure}
\begin{figure}[t]
\input{03fig12.txt}
\vspace{-1\baselineskip}
\caption{人物と判定された候補の文の例}
\label{volexample}
\end{figure}


\section{人物抽出の予備実験結果}\label{humexp}

  \begin{table}[H]
   \caption{人物抽出の正解データの例}
   \begin{center}
\input{03table10.txt}
   \end{center}
  \end{table}

\clearpage

  \begin{table}[H]
   \caption{人物抽出の実験結果}\label{humresult}
   \begin{center}
\input{03table11.txt}
   \end{center}
  \end{table}


\end{document}


