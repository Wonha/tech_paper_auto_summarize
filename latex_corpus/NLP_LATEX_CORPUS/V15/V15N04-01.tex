    \documentclass[japanese]{jnlp_1.4}
\usepackage{jnlpbbl_1.1}
\usepackage[dvips]{graphicx}
\usepackage{amsmath}
\usepackage{hangcaption_jnlp}
\usepackage{udline}
\setulminsep{1.2ex}{0.2ex}
\let\underline

\Volume{15}
\Number{4}
\Month{September}
\Year{2008}

\received{2007}{10}{10}
\revised{2008}{1}{25}
\accepted{2008}{3}{18}

\setcounter{page}{3}

\jtitle{歴史オントロジー構築のための史料からの人物情報抽出}
\jauthor{石川 徹也\affiref{Author_1} \and 北内  啓\affiref{Author_2} \and 城塚 音也\affiref{Author_2}}
\jabstract{
本研究の目的は,歴史資料(史料)を対象に歴史知識の構造化の基盤となる
「歴史オントロジー」を構築するシステムを開発し,広く提供することによっ
て歴史学の発展に寄与することにある.この目標を具体的に検証するために,
昭和15年に時の帝国学士院において始められた明治前日本科学史の編纂成果で
ある『明治前日本科学史』(刊本全28巻)の全文を日本学士院の許諾の下に電
子化し,明治前の日本の科学技術を創成してきた科学技術者に関する属性およ
び業績の情報を抽出することにより,前近代日本の人物情報データベースの構
築を試みる.人物の属性として人名とそれに対する役職名と地名を,人物の業
績として人名とそれに対する書名を,いずれもパターンマッチングなどのルー
ルベースの手法によって抽出する.『明治前日本科学史総説・年表』を対象と
した性能評価を行った結果,人名,人名とその役職名,および人名とその地名
について,F値で0.8を超える結果が得られた.
}
\jkeywords{情報抽出,歴史情報,史料,歴史知識学,オントロジー}

\etitle{Extraction of Person Information from Historical \\
	Materials for Building Historical Ontology}\eauthor{Tetsuya Ishikawa\affiref{Author_1} \and Akira Kitauchi\affiref{Author_2} \and Otoya Shirotsuka\affiref{Author_2}} 
\eabstract{
Our goal of this study is to contribute to the progress in historical
science by developing a system for building a historical ontology from
historical materials and making it available to the public. We
digitize all the books of ``Meiji-mae Nippon Kagaku-shi'' (Pre-modern
Japanese History of Science and Technology) published by Nippon
Gakushiin (The Japan Academy), and extract the attribution and the
works of scientists and engineers from the books to build a database
of person information in pre-modern Japanese history. We extract the
names of persons, positions, places, and books as the attribution and
the works of persons by pattern matching. The experimental results
show that the F-measures for the names of persons, positions, and
places are over 0.8.
}
\ekeywords{Information Extraction, Historical Information, Historical Materials, \\
	Knowledge-based Historical Science, Ontology}

\headauthor{石川,北内,城塚}
\headtitle{歴史オントロジー構築のための史料からの人物情報抽出}

\affilabel{Author_1}{東京大学史料編纂所前近代日本史情報国際センター}{International Center for Digitization of Pre-modern Japanese Sources, Historiographical Institute, University of Tokyo}
\affilabel{Author_2}{株式会社NTTデータ技術開発本部ビジネスインテリジェンス推進センタ}{Business Intelligence Deployment Center, Research and Development Headquarters, NTT DATA Corporation}


\def\|{}


\begin{document}
\maketitle



\section{はじめに}

本研究の目的は,歴史資料(史料)から歴史情報を自動抽出する方式を確立す
ること,および歴史知識を構造化するためにその抽出結果を歴史オントロジー
として構築し,提供することにある.歴史研究は史料内容の解読から始まる.
そのために史料の収集・翻刻(楷書化)・解読の作業が伴う.ただし,史料の
形態・記述は多様であり,翻刻・解読には相当の知識と経験を必要とする.国
内には未解読の史料が未だ多数存在する.一方,これまでに解読された結果に
ついても電子化されていない,あるいは機関・個人など個別に存在するために
各史料を共用できないという問題があり,歴史事象の関連性の解明,すなわち
歴史研究の推進そのものに支障をきたしている.この種の問題解決のために,
すなわち歴史知識の構造化のために歴史オントロジーの提供が求められている.

われわれは,歴史研究のより一層の推進を目的として「歴史オントロジー構築
プロジェクト」を実施している.本プロジェクトは,史料を電子化する,史料
に記載されている情報を抽出,構造化して歴史オントロジーを構築する,歴史
オントロジーを利用した検索・参照システムを構築するという3つの手順によっ
て構成されている.本プロジェクトを具現化するための史料として『明治前日
本科学史』(日本学士院編・刊行,全28巻)を対象に歴史オントロジーを構築
する.当刊行史料は,明治前日本科学史の編纂を目的に昭和15年に帝国学士院
において企図され,昭和35年に最初の巻が出版され,昭和57年に28巻目の刊行
によって現在,完結している.本史料をより有効に活用するため,全巻の電子
化および研究目的の利用・提供に関して日本学士院の許諾を得て,電子化に着
手した.本史料は公的性が高く,歴史研究の推進という本研究の目的に適合す
るものである.本史料から日本の科学技術を創成してきた明治前の人物に関す
る情報を抽出,構造化することにより歴史オントロジーを構築する.

本研究では,プロジェクトの第一歩として,『明治前日本科学史』のうちの1 
巻『明治前日本科学史総説・年表』の本文を電子化したテキストから,人物の
属性として人名とそれに対する役職名と地名,人物の業績として人名とそれに
対する書名を抽出する.機械学習に基づく情報抽出によって十分な精度を得る
には大量の正解データを作成する必要があり,多大な時間がかかることから,
本研究ではルールベースの手法によって人物に関する情報を抽出する.

本稿では,2章で歴史オントロジー構築プロジェクトの全体像を示す.3章で人
物に関する情報を抽出する手法について説明し,4章で実際に評価実験を行い,
その結果を考察する.最後に5章で本研究の結論を示す.


\section{歴史オントロジー構築プロジェクト}

歴史研究において,史料はその手がかりとなる重要な資源であるにも関わらず,
多くの史料が電子化されておらず,また電子化されていたとしても利用しやす
い形で提供されていない場合が多いという問題がある.東京大学史料編纂所で
は史料をデータベース化し,キーワードなどの条件で検索可能なシステムを提
供する取り組みを進めており,これまでにその一部が「東京大学史料編纂所デー
タベースSHIPS」\cite{ships}として公開されている.このシステムで提供さ
れているデータベースには,史料中の文や図が掲載されている箇所(巻やペー
ジなど),記述された出来事の日付,図に描かれた人物の名前などのメタデー
タが付与されているものもある.しかし,これらのメタデータのほとんどを人
手で付与しているため,データベースの構築に多大な時間を費やしている.ま
た,上記以外のメタデータ,たとえば出来事が発生した場所,人物の役職や著
作といった情報はほとんど付与されていないため,「ある人物が執筆した著作
の年代順一覧」などのように,検索条件や出力内容に様々な種類の情報を指定
した複雑かつ柔軟な検索を行うことができない.

われわれは歴史研究のより一層の推進を目的として,史料に記載されている多
様な情報をより効率的に抽出,構造化して歴史オントロジーを構築することに
より,広範な歴史情報を様々な形で利用可能とするためのプロジェクト「歴史
オントロジー構築プロジェクト」を進めている.本章では,本プロジェクトの
全体構成,および本プロジェクトにおける歴史オントロジーの詳細について説
明する.



\subsection{プロジェクトの全体構成}

本プロジェクトは,史料の電子化,電子化されたデータからの歴史オントロジー
の構築,歴史オントロジーを利用した史料の検索・参照システムの構築という
3つの手順によって構成されている(図\ref{fig:arch}).

まず,史料を電子化する.対象となる史料は印刷物として刊行されている『明
治前日本科学史』全28巻である.史料の電子化においては,その利用目的に応
じた電子化方式を確立する必要がある.たとえば,史料の見た目をそのまま復
元すればよいのであれば高い解像度でスキャンした画像を蓄積すればよいが,
それだけではキーワードによる検索ができない.また,史料中の文を単にテキ
スト化するだけでは,年代や人名などの情報を利用した検索や図の参照ができ
ない.複雑かつ柔軟な検索を実現するには,史料に記述された文や掲載されて
いる表や図について,できるだけ論理的な構造を保持したままXMLなどの構造
化文書の形式で電子化する必要がある.たとえば文については,文字のテキス
ト化における外字の表現や,文のテキスト化における章や節,箇条書きなどの
論理構造の表現といった課題がある.また,表や図については,表構造の表現,
タイトルや説明文との関係付け,文中で参照している箇所との関係付けなどが
検討課題となる.

\begin{figure}[t]
\begin{center}
\includegraphics{15-4ia1f1.eps}
\caption{歴史オントロジー構築プロジェクトの全体構成}
\label{fig:arch}
\end{center}
\end{figure}

次に,電子化されたデータから歴史情報を抽出し,構造化することにより歴史
オントロジーを構築する.歴史オントロジーの内容と構築手順については次節
以降で述べる.

最後に,歴史オントロジーを利用した史料の検索・参照システムを構築する.
歴史オントロジーがもつ情報を最大限に活用し,歴史研究に必要とされる様々
な観点での検索の実現,利用者が必要とする情報を分かりやすく表示する検索
結果の可視化,使いやすいユーザインタフェースなどが検討課題としてあげら
れる.また,検索・参照機能を提供するだけではなく,歴史オントロジーその
ものをRDFなどの形式で公開し,利用者が自由に利用できるようにすることも
計画している.


\subsection{歴史オントロジーの定義}

本プロジェクトの対象となる史料『明治前日本科学史』には,主に明治よりも
前の時代における日本の科学・技術の成果に関する史実(歴史的事実)が,図
や表とともに各事項について時系列で記述されている.史実としては,各時代
の科学・技術の推移やその内容,および科学者の業績などの人物に関する情報
(以下,人物情報)が大部分を占める.特に人物情報は歴史研究におけるニー
ズが高く,人物情報を検索・参照可能とすることは歴史研究への貢献度が高い
と考える.そこで,人物情報を可能な限り漏れなく抽出することを本プロジェ
クトの目標のひとつとする.ただし,史料中には一部史実でない記述,たとえ
ば推測,疑問,感想などが記述されており,これらは抽出対象外とする.『明
治前日本科学史総説・年表』における科学者の業績に関する記述部分の抜粋を
図\ref{fig:desc}に示す.

\begin{figure}[t]
\begin{center}
\includegraphics{15-4ia1f2.eps}
\caption{『明治前日本科学史総説・年表』における科学者の業績に関する記述部分の抜粋}
\label{fig:desc}
\end{center}
\end{figure}

抽出した情報を様々な形で利用可能とするためは,資料中の文や図などをその
まま抽出して,キーワードで検索できるようにしたり一覧表示したりするだけ
では不十分であり,人名で検索したり書名を一覧表示したりといった,様々な
種類の概念に基づく検索や参照が必要となる.この要件を実現するため,各種
の概念を史料から抽出し,それらを構造化してオントロジーとして蓄積する.
以降,史料から抽出した人物情報を格納したオントロジーを「歴史オントロ
ジー」とよぶ.

オントロジーについては,哲学をはじめとした様々な観点からの定義が提案さ
れている\cite{mizoguchi1997}が,人工知能分野においては「概念間の関係を
記述することによって知識を体系化したもの」と定義づけられることが多い.
たとえばStandard Upper Ontology Working Group\cite{suo}では,オントロ
ジーは概念(concepts),関係(relations),公理(axioms)の3つの要素によって
構成されており,これらの構成要素によって,ある分野に関する事物や構造を
記述できるとしている.本稿では,この説明に基づいて歴史オントロジーを定
義する.

歴史オントロジーで対象とする分野は日本の科学史である.「概念」は,人物
情報を構成する各要素である.人物情報は,人物自体を指す情報,人物の属性,
人物の業績の3種類に大きく分類される.人物自体を指す情報としては人名,
写真,似顔絵などが,人物の属性としては役職,出身地,生没年,家族関係な
どが,業績としては著作,建造物,訪問先などがあげられる.したがって,歴
史オントロジーにおける「概念」,すなわち人物情報の構成要素としては,人
物,写真,役職,年代,書籍,建造物,場所などがあり,概念どうしを結ぶ関
係としてはたとえば以下のものがあげられる.
\begin{itemize}
\item 人物から写真への「写真」という関係
\item 人物から年代への「生年」という関係
\item 人物から書籍への「著作」という関係
\item 書籍から年代への「発行年」という関係
\end{itemize}
本プロジェクトで抽出対象とする歴史オントロジーの概念や関係の全体像につ
いては現在検討中であるが,これまでに候補としてあがっている概念どうしを
結ぶ関係の一部を表\ref{table:ontology}に示す.なお,表
\ref{table:ontology} において概念間の階層関係はない.

\begin{table}[t]
\caption{歴史オントロジーにおける,概念どうしを結ぶ関係(一部)}
\label{table:ontology}
\begin{center}
\input{01table01.txt}
\end{center}
\end{table}

本プロジェクトでは基本的に各概念のインスタンスをその名前で表し,写真な
どの画像データは「人物」という概念とは別の概念として「写真」などの関係
で結ぶ.したがって,人物情報のうち,人名は「人物」という概念のインスタ
ンスとして,それ以外の情報は概念のインスタンスどうしを結ぶ関係として表
現される.

「公理」は概念や関係が満たす制約条件であり,概念間の階層関係や,ある概
念がもつ各関係の数(たとえば「一人の人物は生年を一つだけもつ」)などが
あげられる.



\subsection{歴史オントロジーの構築手順}

歴史オントロジー構築の最終目標は,『明治前日本科学史』全28巻を電子化し,
そこから人物情報を高精度に抽出することである.しかし,以下のような問題
があるため,それを一度に実施するのは膨大な時間がかかる上に非効率的であ
る.

\begin{itemize}
\item 人物情報には様々なものがあり,抽出すべき情報を決めるのには時間がかかる.
\item 全28巻の電子化,特に写真や図表などの画像を電子化するのには時間がかかる.
\item 高精度な抽出方式を確立するには,仮説と検証を繰り返す必要があり時間がかかる.
\end{itemize}

そこで,以下の手順で段階的に歴史オントロジーを構築する.

\begin{enumerate}
\item 数巻程度の本文をテキスト形式で電子化する.
\item 電子化されたテキストを対象として,人物情報のうち特に歴史研究に必要とされる情報を高精度に抽出する方式を確立する.
\item 全巻について,テキストと画像の両方を電子化する.
\item 全巻の電子化データを対象として,(2)と同様に人物情報のうち特に歴史研究に必要とされる情報を抽出する.
\item 抽出する人物情報を拡大し,全巻を対象として様々な人物情報を高精度に抽出する方式を確立する.
\end{enumerate}

われわれは現在,上記の手順のうち(1), (2)について取り組みを進めている.
また(3)のうち,画像の電子化方法について検討中である.



\section{人物情報の抽出}
本研究では,『明治前日本科学史』全28巻のうちの1巻『明治前日本科学史総
説・年表』を対象として,人名,人物の属性として人名とそれに対する役職名
と地名,および人物の業績として人名とそれに対する書名を抽出する.これら
は基本的かつ重要な情報であり,また史料中の記述量も多いため評価実験で多
くの知見が得られる見込みが高い.役職名には,「将軍」のような官職(役人
の職業)の名前と「医師」のような一般の職業名の2種類がある.地名には,
その人物の出身地,所属する組織の地域,国籍などがあり,歴史研究への活用
のためにはそれらが区別されている方が好ましいが,本研究では区別せず,す
べて地名として抽出する.ある人物の役職名や地名は時間の経過にしたがって
変化する場合があるため,一人に対し複数の役職名や地名が抽出されることが
ある.したがって,人物の役職名と地名の抽出結果は,〈人名,役職名〉,
〈人名,地名〉のいずれかの組の列として表現される.書名についても,一人
が複数の書籍を書く場合があるため,人物が書いた書籍の抽出結果は,〈人名,
書名〉の組の列として表現される.

人名のような固有表現を抽出する方法については,大きく分けてルールベース
の手法\cite{rau1991}と機械学習に基づく手法
\cite{asahara2003,maccallum2003}の2種類が提案されている.機械学習に基
づく手法は,学習のための正解データが必要となる.史料には人名の索引が掲
載されているが,科学者のみが掲載の対象となっており政治家の人名は含まれ
ない.また,史料の文中には姓や名のみが出現する場合がある.索引の人名は
姓名(フルネーム)のみであるため,姓や名を人名として抽出するのは困難で
ある.したがって,史料の正解データを作成するのには膨大な時間を必要とす
る.IREX\cite{sekine2000} の公開データなどの人名タグ付きコーパスを正解
データとして利用する方法もある.しかし,上記のコーパスは1994年から1995 
年の新聞記事を対象としている.一方,史料中の人名は主に明治よりも前の時
代のものであり,人名を構成する文字や形態素が大きく異なるため,高い抽出
精度の実現を期待できない.そこで,本研究では,ルールベースの手法により
人名を抽出する.

また,人物に対する役職名のような関係を抽出する情報抽出についても,ルー
ルベースの手法と機械学習に基づく手法\cite{sudo2003,greenwood2005} の2 
種類があるが,上記の固有表現抽出と同様の理由で,ルールベースの手法によ
り人物の属性や業績を抽出する.

人名の抽出手順としては,人手で作成した形態素列の抽出パターンを利用した
パターンマッチングによって人名を抽出したあと,大域的情報を利用してさら
なる人名の抽出と名寄せを行う.また,人物の属性と業績もパターンマッチン
グによって抽出する.これらの手法について以下に説明する.


\subsection{形態素列のパターンマッチングによる情報抽出}

\begin{figure}[b]
\begin{center}
\includegraphics{15-4ia1f3.eps}
\caption{形態素列のパターンマッチングによる情報抽出手順}
\label{fig:matching}
\end{center}
\end{figure}

図\ref{fig:matching}に示す手順で形態素列のパターンマッチングによる情報
抽出を実行する.まず,史料中の文に対して形態素解析を実施する.その結果
に対して,各形態素の出現形,基本形,品詞,字種などの情報を用いて,正規
表現に似た形式で形態素列を表現したパターンにマッチする形態素列を抽出す
る.このとき,形態素列のパターンがパターンマッチング処理に埋め込まれて
いると,パターンの修正にともなうパターンマッチング処理の修正に時間がか
かる.そこで,形態素列のパターンとパターンマッチング処理を分離し,パター
ンのみを修正すれば,それに応じたパターンマッチング処理が実行できるよう
にした.たとえば,図\ref{fig:matching}の抽出パターンの1行目は,「名詞— 
固有名詞—地域」の品詞をもつ形態素と基本形が「都」「府」「県」のいずれ
かの形態素からなる形態素列(たとえば「岩手県」),あるいは基本形が「北
海道」である形態素のいずれかにマッチし,\|PREF|というタグを付与するこ
とを意味する.

人物の属性や業績のように,固有表現どうしの関係を抽出する際は,固有表現
(人名,役職名,地名,書名)を抽出するパターンとそれらをまとめあげるパ
ターンを作成し,固有表現を抽出した後でそれらをまとめあげるという二段階
の処理を行う.

パターンマッチングによる情報抽出については,情報抽出に関する評価型会議
であるMUC \cite{grishman1996}をはじめとして様々な手法が提案されている.
たとえば,\cite{nishino1998}では本稿と同様に人名とその職業名を抽出して
おり,その際,職業名のリストや人名の直後に出現する「さん」といった語句
を手がかり句を利用している.われわれの抽出パターンでは,このような手が
かり句に加えて,人名などの固有表現を構成する形態素の特徴も利用している.
固有表現を抽出するパターンに利用した主な特徴を表\ref{table:feature} に
示す.なお,「パターンの例」は実際のパターンを分かりやすいように一部書
き換えてある.たとえば,表\ref{table:feature} の上から2番目のパターン
の例では,構成する形態素の字種を利用することにより,「ウィリアム・アダ
ムズ」のようなカタカナを含む人名を抽出できる.また,『明治前日本科学史』
には古い時代の人名が数多く出現し,形態素解析誤りが頻繁に発生する.たと
えば,「桂川甫周」を形態素解析すると,「桂川」は品詞が「人名—姓」の形
態素となるが,「甫」と「周」はそれぞれ別の形態素に分割されてしまう.こ
のような場合,表\ref{table:feature}の上から3番目のパターンの例のように,
形態素の文字数を指定することにより,「桂川甫周」を人名として正しく抽出
できる.

\begin{table}[b]
\caption{形態素列のパターンに利用した主な特徴}
\label{table:feature}
\begin{center}
\input{01table02.txt}
\end{center}
\end{table}

固有表現をまとめあげるパターンについては,固有表現と固有表現の間やその
前後に出現する形態素の特徴をもとにパターンを作成した.人名と地名をまと
めあげるパターンとそれにマッチする形態素列の例を表\ref{table:pattern} 
に示す.パターンの例において,\|$PERSON|,\|$PLACE|はそれぞれすでに抽
出された人名,地名の形態素列を表す.

固有表現を抽出するパターンとそれらをまとめあげるパターンをあわせ,全部
で約90個の抽出パターンを作成した.



\subsection{大域的な情報を利用した情報抽出と名寄せ}

人手で作成した形態素列のパターンを利用したパターンマッチングによって人
名を抽出する場合,すべての人名を抽出するためのパターンを網羅的に記述す
ることは困難であり,多くの抽出漏れが発生してしまう.そこで,パターンマッ
チングによる抽出結果に対し,大域的な情報を利用して抽出漏れを削減する.

\begin{table}[b]
\caption{人名と地名をまとめあげるパターンの例}
\label{table:pattern}
\begin{center}
\input{01table03.txt}
\end{center}
\end{table}
\begin{figure}[b]
\begin{center}
\includegraphics{15-4ia1f4.eps}
\caption{大域的な情報を利用した情報抽出と名寄せの例}
\label{fig:global}
\end{center}
\end{figure}


史料中に出現する人名は,初出時は姓名(フルネーム)で出現し,その後方で
同一人物を表す人名が姓名,姓,名のいずれかの形で出現することが多い.そ
こで,形態素列のパターンマッチングによる人名の抽出結果に対し,さらに追
加で人名の候補となる形態素列をパターンマッチングにより抽出し,その中で
人名となる形態素を判定する.人名の候補となる形態素列として,漢字のみで
構成される形態素が連続する形態素列,およびカタカナのみで構成される形態
素列と「・」(ナカグロ)が交互に出現する形態素列を抽出する.抽出された
人名の候補のうち,以下のいずれかの条件を満たすものを人名と判定する.

\begin{description}
\item[条件a] 任意の箇所に出現する人名と同じ文字列の形態素列
\item[条件b] ある箇所に出現する人名よりも後ろに出現し,かつその人名の
先頭または末尾の部分文字列となっている形態素列
\end{description}

条件bに合致する形態素については,その前に出現する(フルネームの)人名
と同一人物であるという判定(名寄せ)も同時に行う.

大域的な情報を利用した情報抽出と名寄せの例を図\ref{fig:global}に示す.
条件aにより,一番上の「藤林普山」が人名と判定される.また条件bにより,
「良沢」と「藤林」が人名と判定され,さらに「良沢」は「前野良沢」と,
「藤林」は「藤林普山」とそれぞれ同一人物であると判定される.



\section{評価実験}

史料からの人名,人物の属性,人物の業績の抽出精度を評価する実験を行った.


\subsection{実験条件}

評価用データとして『明治前日本科学史総説・年表』の本文(5096文,約21万
8千字)を使用し,人名,人物の属性(人名とそれに対する役職名と地名),
人物の業績(人名とそれに対する書名)の抽出精度を評価した.評価のための
正解作成にあたっては,システムが抽出した誤りや漏れを人手で修正すること
により,すべての情報を人手で抽出するのと比べて短時間で正解を作成するこ
とができた.

人名については,パターンマッチングのみを行った場合と,パターンマッチン
グのあと大域的な情報を使って抽出漏れを削減した場合の精度を評価した.ま
た,現代の文書を対象に機械学習を行う手法として,
CaboCha\footnote{http://chasen.org/\~{}taku/software/cabocha/}の固有表現
抽出機能において,CaboChaに付属する,毎日新聞記事のタグ付きコーパスで
学習したモデルを使って人名を抽出した場合との精度を比較した
\footnote{CaboChaの固有表現抽出機能は地名も抽出可能だが,今回は比較対
象としなかった.今後の課題としたい.}.パターンマッチング,CaboChaとも
形態素解析にはChaSen\footnote{http://chasen.naist.jp/hiki/ChaSen/}を利
用した.

人物の属性と人物の業績については,パターンマッチングのみを行った場合の
精度を評価した.人物の属性については,人名とそれに対する役職名として
〈人名,役職名〉の組を,人名とそれに対する地名として〈人名,地名〉の組
を抽出し,その抽出精度を評価した.また人物の業績については,人名とそれ
に対する書名として〈人名,書名〉の組の抽出精度を評価した.

評価尺度として,以下の式で算出される再現率,適合率,F値を測定した.
\[
再現率R = \frac{一致件数}{正解件数}, \hspace{1em}
適合率P = \frac{一致件数}{出力件数}, \hspace{1em}
F値 = \frac{2PR}{P+R}
\]
上記の件数の算出にあたっては,人名については出現箇所を区別し,人物の属
性と人物の業績については出現箇所を区別せず算出した.たとえば人名の正解
については,同じ人名が複数の箇所に出現する場合はそれぞれ別のものとして
正解件数をカウントした.また,出力された人名を正解と比較し,同じ人名が
同じ箇所に出現する場合にのみ一致するとして,一致件数を算出した.逆に人
物の業績については,同じ〈人名,書名〉の組が複数の箇所から抽出された場
合,それらをまとめて1とカウントした.


\subsection{実験結果}

人名の抽出結果について,パターンマッチングのみを行った場合,パターンマッ
チングのあと大域的な情報を利用した場合,およびCaboChaを使った場合の結
果を表\ref{table:result_person}に示す.もっとも精度の高いものを太文字
の数字で表している.また,役職名,地名,書名の抽出結果について,パター
ンマッチングのみを行った場合の結果を表\ref{table:result_other}に示す.

\begin{table}[b]
\caption{人名の抽出結果の比較}
\label{table:result_person}
\begin{center}
\input{01table04.txt}
\end{center}
\end{table}
\begin{table}[b]
\caption{役職名,地名,書名の抽出結果(パターンマッチングのみ)}
\label{table:result_other}
\begin{center}
\input{01table05.txt}
\end{center}
\end{table}


\subsection{考察}

人名の抽出精度については,パターンマッチングのあと大域的な情報を使った
場合がもっとも高い精度であった.パターンマッチングのみの場合の精度と比
較すると,適合率が若干下がったものの,再現率が大幅に向上しており,大域
的な情報を利用することによって抽出漏れを削減できたことが分かる.

CaboChaを使った場合のF値は0.702であった.現代の日本語の文書を対象とし
て機械学習を行った場合の人名の抽出精度は0.87前後と報告されている
\cite{asahara2003}.本実験の抽出対象文書である史料に出現する人名は,現
代の人名と比べて人名を構成する文字や形態素が大きく異なることが,
CaboChaを使った場合の抽出精度が低かった原因と考える.

\begin{table}[t]
\caption{役職名の抽出漏れの例}
\label{table:loss_position}
\begin{center}
\input{01table06.txt}
\end{center}
\end{table}
\begin{table}[t]
\caption{地名の抽出漏れの例}
\label{table:loss_place}
\begin{center}
\input{01table07.txt}
\end{center}
\end{table}
\begin{table}[t]
\caption{書名の抽出漏れの例}
\label{table:loss_book}
\begin{center}
\input{01table08.txt}
\end{center}
\end{table}

役職名,地名,書名の抽出精度はいずれも,適合率に比べて再現率が低かった.
役職名,地名,書名それぞれの抽出漏れの例を表\ref{table:loss_position},
\ref{table:loss_place}, \ref{table:loss_book}に示す.「抽出漏れの箇所」
欄において,太字部分が抽出された人名を表し,[](カギ括弧)で囲まれた
語句が抽出できなかった(つまり抽出漏れの)役職名,地名,書名を表してい
る.「解決方法」欄には抽出するための方法の案を示した.役職名,地名,書
名とも,形態素列の抽出パターンの追加で抽出可能となる抽出漏れだけでなく,
係り受け解析や照応解析など,形態素解析以外の自然言語処理が必要とされる
抽出漏れもあることが分かる.たとえば係り受け解析については,パターン中
に形態素間の係り受け関係も記述できるようにし,形態素列のパターンにマッ
チする形態素列を求めたあと,それらが係り受け関係にあるものを求めるとい
う方法がある.表\ref{table:loss_book}の上から3番目の例の場合,
「\|<人名>"は"|」と「\|<書名> "を書いた"|」という形態素列のパターンに
マッチする形態素列として「松永良弼は」と「『円理乾坤之巻』を書いた」を
それぞれ抽出したあと,前者の形態素列「松永良弼は」が後者の形態素列の先
頭の文節「『円理乾坤之巻』を」に係ることから,松永良弼の著書として『円
理乾坤之巻』を抽出できる.

役職名,地名,書名それぞれの抽出精度を比較すると,書名の再現率が役職名,
地名に比べて低かった.それぞれの抽出漏れを分析したところ,役職名,地名
と比較して,書名は人名と離れた位置に出現する場合が多いことが分かった.
このような抽出漏れは,形態素列のパターンでは正しく抽出することができず,
係り受け解析や照応解析が必要となるものが多いことが,書名の再現率が低かっ
た原因だと考える.



\section{おわりに}

歴史資料を対象として歴史オントロジーを構築するシステムを開発するための
第一歩として,『明治前日本科学史』の一部を電子化し,史料中の科学技術者
に関する属性および業績の情報を抽出することにより,前近代日本の人物情報
データベースの構築を試みた.人名とそれに対する役職名,地名,書名をルー
ルベースの手法によって抽出する方法を提案し,『明治前日本科学史総説・年
表』を対象とした精度評価を行った結果,人名,人名とそれに対する役職名,
人名とそれに対する地名についてはF値で0.8を超える結果が得られた.

今後の課題としては,抽出精度を向上させるために,機械学習によって情報抽
出を行うこと,係り受け解析や照応解析の結果を形態素列の抽出パターンや機
械学習に利用することを考えている.また,人物の属性や業績として抽出する
情報を拡大するとともに,抽出対象データについても『明治前日本科学史総説・
年表』以外の巻を対象とした抽出と評価を行う予定である.





\bibliographystyle{jnlpbbl_1.3}
\begin{thebibliography}{}

\bibitem[\protect\BCAY{Asahara \BBA\ Matsumoto}{Asahara \BBA\
  Matsumoto}{2003}]{asahara2003}
Asahara, M.\BBACOMMA\ \BBA\ Matsumoto, Y. \BBOP 2003\BBCP.
\newblock \BBOQ Japanese Named Entity Extraction with Redundant Morphological
  Analysis\BBCQ\
\newblock In {\Bem Proceedings of the 2003 Conference of the North American
  Chapter of the Association for Computational Linguistics on Human Language
  Technology (HLT-NAACL)}, \mbox{\BPGS\ 8--15}.

\bibitem[\protect\BCAY{Greenwood, Stevenson, Guo, Harkema, \BBA\
  Roberts}{Greenwood et~al.}{2005}]{greenwood2005}
Greenwood, M.~A., Stevenson, M., Guo, Y., Harkema, H., \BBA\ Roberts, A. \BBOP
  2005\BBCP.
\newblock \BBOQ Automatically Acquiring a Linguistically Motivated Genic
  Interaction Extraction System\BBCQ\
\newblock In {\Bem Proceedings of the 4th Learning Language in Logic Workshop
  (LLL05)}, \mbox{\BPGS\ 46--52}.

\bibitem[\protect\BCAY{Grishman \BBA\ Sundheim}{Grishman \BBA\
  Sundheim}{1996}]{grishman1996}
Grishman, R.\BBACOMMA\ \BBA\ Sundheim, B. \BBOP 1996\BBCP.
\newblock \BBOQ Message Understanding Conference-6: A Brief History\BBCQ\
\newblock In {\Bem Proceedings of the 16th Conference on Computational
  Linguistics (COLING-96)}, \mbox{\BPGS\ 466--471}.

\bibitem[\protect\BCAY{IEEE}{IEEE}{2003}]{suo}
IEEE.
\newblock \BBOQ IEEE P1600.1 Standard Upper Ontology Working Group (SUO
  WG)\BBCQ, \Turl{http://suo.ieee.org/}.

\bibitem[\protect\BCAY{McCallum \BBA\ Li}{McCallum \BBA\
  Li}{2003}]{maccallum2003}
McCallum, A.\BBACOMMA\ \BBA\ Li, W. \BBOP 2003\BBCP.
\newblock \BBOQ Early Results for Named Entity Recognition with Conditional
  Random Fields, Feature Induction and Web-Enhanced Lexicons\BBCQ\
\newblock In {\Bem Proceedings of The Seventh Conference on Natural Language
  Learning (CoNLL-2003), {\upshape\textbf{4}}}, \mbox{\BPGS\ 188--191}.

\bibitem[\protect\BCAY{Mizoguchi \BBA\ Ikeda}{Mizoguchi \BBA\
  Ikeda}{1997}]{mizoguchi1997}
Mizoguchi, R.\BBACOMMA\ \BBA\ Ikeda, M. \BBOP 1997\BBCP.
\newblock \BBOQ Towards Ontology Engineering\BBCQ\
\newblock In {\Bem Proceedings of Joint Pacific Asian Conference on Expert
  Systems/Singapore International Conference on Intelligent Systems
  (PACES/SPICIS '97)}, \mbox{\BPGS\ 259--266}.

\bibitem[\protect\BCAY{西野\JBA 落谷}{西野\JBA 落谷}{1998}]{nishino1998}
西野文人\JBA 落谷亮 \BBOP 1998\BBCP.
\newblock \JBOQ 新聞記事からの人物・企業情報の抽出\JBCQ\
\newblock \Jem{情報処理学会自然言語処理研究会 (NL-127)}, \mbox{\BPGS\
  125--132}.

\bibitem[\protect\BCAY{Rau}{Rau}{1991}]{rau1991}
Rau, L.~F. \BBOP 1991\BBCP.
\newblock \BBOQ Extracting Company Names from Text\BBCQ\
\newblock In {\Bem Proceedings of Seventh IEEE Conference on Artificial
  Intelligence Applications}, \mbox{\BPGS\ 29--32}.

\bibitem[\protect\BCAY{Sekine \BBA\ Isahara}{Sekine \BBA\
  Isahara}{2000}]{sekine2000}
Sekine, S.\BBACOMMA\ \BBA\ Isahara, H. \BBOP 2000\BBCP.
\newblock \BBOQ IREX: IR and IE Evaluation Project in Japanese\BBCQ\
\newblock In {\Bem Proceedings of the 2nd International Conference on Language
  Resources and Evaluation (LREC-2000)}, \mbox{\BPGS\ 1475--1480}.

\bibitem[\protect\BCAY{Sudo, Sekine, \BBA\ Grishman}{Sudo
  et~al.}{2003}]{sudo2003}
Sudo, K., Sekine, S., \BBA\ Grishman, R. \BBOP 2003\BBCP.
\newblock \BBOQ An Improved Extraction Pattern Representation Model for
  Automatic IE Pattern Acquisition\BBCQ\
\newblock In {\Bem Proceedings of the 41st Annual Meeting on Association for
  Computational Linguistics}, \mbox{\BPGS\ 224--231}.

\bibitem[\protect\BCAY{東京大学史料編纂所}{東京大学史料編纂所}{2006}]{ships}
東京大学史料編纂所.
\newblock \JBOQ 東京大学史料編纂所データベースSHIPS\JBCQ,
  \Turl{http://www.hi.u-tokyo.{\linebreak}ac.jp/ships/}.

\end{thebibliography}

\begin{biography}
\bioauthor{石川 徹也}{
1971年慶応義塾大学大学院修士課程修了.富士フイルム(株)足柄研究所,図
書館短期大学,図書館情報大学,文部省在外研究員 (UCLA, IU), 筑波大学等
を経て,現在,東京大学史料編纂所前近代日本史情報国際センター特任教授
(研究開発主査).歴史知識学の創成研究に従事.工学博士(早稲田大学),
筑波大学名誉教授,筑波大学大学院図書館情報メディア研究科共同研究員.
情報文化学会学会賞(2005年8月),言語処理学会優秀発表賞(2006年3月),
E\"{u}gen Wuster Special Prize(UNESCO, 2006年7月).
}
\bioauthor{北内  啓}{
1998年奈良先端科学技術大学院大学情報科学研究科博士前期課程修了.
同年,NTTデータ通信(株)入社.
現在,(株)NTTデータ技術開発本部において自然言語処理の研究開発に従事.
言語処理学会,情報処理学会各会員.
}
\bioauthor{城塚 音也}{
1988年東京大学文学部言語学科卒業.同年日本電信電話株式会社入社,
音声言語,知識処理技術の研究開発に従事.
現在,(株)NTTデータ技術開発本部において自然言語処理の研究開発を担当.
}

\end{biography}

\biodate


\end{document}
