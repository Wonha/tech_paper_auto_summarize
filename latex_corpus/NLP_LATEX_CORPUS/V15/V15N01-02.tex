    \documentclass[japanese]{jnlp_1.4}
\usepackage{jnlpbbl_1.1}
\usepackage[dvips]{graphicx}
\usepackage{hangcaption_jnlp}
\setlength{\captionwidth}{\textwidth}
\usepackage{udline}
\setulwidth{0.25pt}\setulminsep{1.2ex}{0.4427pt}
\let\underline
\usepackage{booktabs,colortbl}

\Volume{15}
\Number{1}
\Month{Jan.}
\Year{2008}
\received{2007}{4}{9}
\accepted{2007}{12}{15}

\setcounter{page}{23}

\jtitle{日本語テキストから手話テキストへの機械翻訳の試み}
\jauthor{松本 忠博\affiref{GUIS} \and 池田 尚志\affiref{GUIS}}

\jabstract{
  手話は言語でありろう者の母語である.手話と音声言語の間のコミュニケー
  ションには手話通訳が必要となるが,手話通訳士の数は圧倒的に不足してい
  る.両言語間のコミュニケーションを支援する技術が期待される.本論文は
  日本語と手話との間の機械翻訳を目指して,その一つのステップとして,日
  本語テキストから手話テキストへの機械翻訳を試みたものである.
  機械翻訳をはじめとする自然言語処理技術はテキストを対象としているが,
  手話には文字による表現がないため,それらを手話にそのまま適用すること
  ができない.我々は言語処理に適した日本手話の表記法を導入することで,
  音声言語間の翻訳と同様に,日本語テキストから手話テキストへの機械翻訳
  を試みた.日本語から種々の言語への機械翻訳を目的として開発中のパター
  ン変換型機械翻訳エンジンjawをシステムのベースに用いている.目的言語で
  ある手話の内部表現構造を設定し,日本語テキストを手話の表現構造へ変換
  する翻訳規則と,表現構造から手話テキストを生成する線状化規則を与える
  ことで実験的な翻訳システムを作成した.日本手話のビデオ教材等から例文
  を抽出し,その翻訳に必要な規則を与えることで,日本語から手話に特徴的
  な表現を含んだ手話テキストへの翻訳が可能であることを確認するとともに,
  現状の問題点を分析した.}

\jkeywords{機械翻訳,日本手話,日本語,手話表記法}

\etitle{An Approach to Machine Translation\\
 from Japanese Text to JSL Text}

\eauthor{Tadahiro Matsumoto\affiref{GUIS} \and Takashi Ikeda\affiref{GUIS}}

\eabstract{
  In this paper we present an approach to machine translation from
  Japanese to JSL (Japanese Sign Language).  There is no standard way
  of writing JSL, and that brings complexity and difficulty on natural
  language processing for JSL.  We defined a Japanese gloss-based
  notation system for JSL.  This notation system enables us to divide
  the MT process into two stages, namely, the text-to-text translation
  stage (Japanese to JSL text) and text-to-motion synthesis stage (JSL
  text to JSL motion).  Our current focus is on the former stage.
  This notation also allows us to apply the existing MT techniques to
  JSL.  We implemented a pilot MT system using \textit{jaw}, a pattern
  transfer-based MT engine that we have been developing to construct
  MT systems from Japanese to various (oral) languages.  To test the
  validity of our approach and to find the problems of it, we
  conducted a translation experiment with sentences extracted from
  videos for JSL learners.  }

\ekeywords{Machine translation, Japanese Sign Language, Japanese,
  notation system}

\headauthor{松本,池田}
\headtitle{日本語テキストから手話テキストへの機械翻訳の試み}

\affilabel{GUIS}{岐阜大学工学部応用情報学科}
	{Department of Information Science, Faculty of Engineering, Gifu University}


\newenvironment{ex}{}{}

\newenvironment{format}{}{}

\newcounter{example}
\newcommand{\exn}{}

    \setcounter{secnumdepth}{2}

\begin{document}
\maketitle



\section{はじめに}

手話はろう者の間で生まれ広がった自然言語であり,ろう者にとっての第一言
語である\cite{Yonekawa2002}.そのため手話による情報アクセスやサービス
の提供はろう者の社会参加にとって重要であるが,手話通訳者は不足しており,
病院や職場,学校などで手話通訳を必要とする人々に十分な通訳サービスが提
供されているとはいえない.これらを支援するシステムの実現が期待されてい
る.

音声言語では機械翻訳をはじめとして,言語活動を支援するさまざまの自然言
語処理技術が研究開発されている.ところが,手話はこれまで自然言語処理の
領域では研究対象としてほとんど取り上げられていない.手話には広く一般に
受け入れられた文字による表現(テキスト表現)が存在しないため,これまで
のテキストを対象とした自然言語処理技術が手話に対して適用できないことが
その要因としてあげられる.そこで我々は,手話言語をテキストとして書き留
める方法について検討し,「日本語援用手話表記法」を提案し
た\cite{Matsumoto2006,Matsumoto2005c,Ikeda2006,Matsumoto2004a,Matsumoto2004b,Matsumoto2005a,Matsumoto2005b}
.本論文では,この表記法
で表現された手話を目的言語とする日本語—手話機械翻訳システムについて述べ
る.手話テキストから手話動画像等への変換(音声言語におけるテキスト音声
合成に相当する)もまた大きな課題であるが,本論文ではこの課題は扱わない.
手話のテキスト表現を導入したことにより,手話テキストから手話動画像等へ
の変換を,テキスト音声合成の問題と同じように,本研究とは別の一つの大き
な問題領域としてとらえることができる.このように音声言語の翻訳の場合と
同じように翻訳過程を二つの領域にモジュール化することによって,手話の翻
訳の問題が過度に複雑になることを避けることができ
る.
\nocite{Matsumoto2005a,Matsumoto2005b,Matsumoto2004a,Matsumoto2004b}

\ref{sec:JSL}節で述べるように日本の手話には日本手話と日本語対応手話およ
び中間型手話がある.これらの間には必ずしも明確な境界があるわけではない
が,本論文で対象として念頭に置いているのは日本手話である.日本手話は日
本語の影響を強く受けているものの,日本語とは別の言語である.語彙は日本
語と1対1に対応しておらず,文法的にも独自の体系を持っている.例えば,日
本語において内容語に後置される機能語や前置される修飾語が,手話では独立
した単語としてではなく,内容語を表す手の動きや位置の変化(内容語の語形
変化),顔の表情などによって表現される場合がある.また,動詞の主語・目
的語・道具などの内容語も,動詞を表す手の形や動きの変化として動詞の中に
組み込まれる場合がある.したがって,日本手話への翻訳は単に日本語の単語
を手話単語に置き換えるだけでは不十分であり,外国語への翻訳と同等の仕組
みが必要となる.

本研究では,日本語から種々の言語への翻訳を目的として開発が進められてい
るパターン変換型機械翻訳エンジンjaw \cite{Shie2004}を核とし,手話に対す
る計算機内部での表現構造,日本語から手話表現構造への翻訳規則,表現構造
から手話テキストへの線状化関数を与えることにより,日本語から手話への機
械翻訳システムjaw/SLの作成を試みた.

以下,2節では目的言語である手話と,我々が定義した手話表記法の概略を述
べる.3節で機械翻訳エンジンjawの翻訳方式について,4節で手話を目的言語
とした翻訳システムjaw/SLについて述べ,5節で翻訳実験と現状の問題点に
ついて述べる.

\section{手話とそのテキスト表現}


\subsection{手話について}
\label{sec:JSL}

手話は手の形,位置,動き,顔の表情など複数の要素を組み合わせて意味を伝
達する言語である.各要素をパラメータのように変化させることによっても様々
な意味が表現される.身体だけでなく,話者の周りの空間も文法的に利用され
る.

日本手話は日本のろう者の間で使われている手話言語である.手話通訳に期待
される事柄についての調査\cite{Shirasawa2002}において,「頭の中で日本語
を考えなくてもよく,手話として自然に頭に入ってくること」,「(日本語の)
口話に頼らなくても十分に内容が伝わってくること」,「日本語の口形は日本
語借用部分でのみ用い,それ以外では手話口形を用いること」などがあげられ
ていることからも,日本語とは異なる体系を持つ言語であることがうかがえる.

日本手話のほかに,日本で一般に手話と呼ばれるものに「日本語対応手話」が
ある.これにはさまざまな考え方があり,定義は一つではないが,多くの場合,
日本手話が日本語とは異なる別の体系を持つ手話言語(手話を第一言語とする
ろう者同士が日常使う手話)をさすのに対し,日本語対応手話は,手話単語を
使用してはいるが,文法や語彙などが日本語的な表現になっているものをさす
\cite{Yonekawa2004}.日本語の語順に手話単語を並べただけのものから,手
話的な表現を部分的に取り入れたものまで幅がある.両者が入り混じったもの
は中間型手話とも呼ばれる.一般に日本語を話しながら表現され,手話独自の
口型や表情は省かれる.


本研究では日本手話を目的言語とするが,手話には地域差や,世代・集団(ろ
う学校/普通学校)・失聴時期・日本語能力などによる個人差が見られ,今のと
ころ標準日本手話というものも存在しない\cite{Nakamura2002,Inaba1998}.同
じ話者(ろう者)が,聞き手に応じて無意識に手話を切り替える場合もある.
我々は日本手話を対象とした文献や教材・ビデオ映像から翻訳規則を取得した.
手話文はいずれも日本語の口話を伴わないものだが,その中にも日本語的な表
現が混ざっている可能性はある.我々が取り上げた個々の手話文が厳密な意味
での日本手話であるか否かという点についてはあるいは議論があるかもしれな
いが,我々はその点にはあまりこだわらず,本論文ではおおよその分類として
日本手話という呼び方を用いている.与えられた日本語文とその手話訳によっ
て,システムに組み込まれる翻訳規則は変化するが,同じ機械翻訳の枠組みの
中で対処できるものと考える.



\subsection{日本語援用手話表記法}
\label{sec:notation}

ここでは機械翻訳結果として出力される日本手話の文を,テキストとして表現
するのに用いる手話表記法の概略を述べる.詳しくは文
献\cite{Matsumoto2006}を参照されたい.

従来の手話表記法\cite{Prillwitz2004,Sutton2002,Ichikawa2001}は,その多
くが音声言語における発音記号のように手話の動作を記述するものであり,翻
訳の問題と動作合成の問題とを分離するという目的には適していない.本表記
法では個々の手話単語の動作の詳細よりも,動作によって表される語彙内容
や文法的な機能の記述に重点を置いた.それによって,動作の詳細に立ち入ら
ずに手話を言語として処理するのに適した表現方法となった.機械翻訳のため
の内部的な中間表現ということではなく,音声言語のテキストと同様,手話文
をテキストの形で書き留め,利用することを念頭に置いている.

日本語から手話への機械翻訳の研究には,徳田・奥村(1998)\nocite{Tokuda1998}や
黒川ら\cite{Fujishige1997,Hirata2003,Ikeda2003,Miyashita2004}の研究があ
る.いずれも手話表記法を定義して翻訳に利用しているが,日本語の口話を伴
い,基本的に日本語と同じ語順で表現される日本語対応手話/中間型手話を対
象としているため,これらをそのまま日本手話の記述に用いるには記述能力の
点で問題があった\cite{Matsumoto2006}.また,後者\cite{Ikeda2003}では手
話画像生成のために有効な形式として手話表記法が導入されているが,入力日
本語文の解析結果を部分的に含んだ中間表現的なものであり,手話言語に対す
るテキスト表現というものではなかった.今のところ,日本手話を目的言語と
した機械翻訳の研究は見あたらない.

日本手話では顔の表情や頭の動きなどの非手指要素が,文法標識などの言語的
に重要な役割を果たしている.本研究では,非手指要素や単語の語形変化によっ
て表される語彙的・文法的な情報もテキストとして表現できるような表記法を
定義することによって,これらの要素を翻訳結果として出力できるようにした.



\subsubsection{手話単語}

手話単語は手話単語名と語形変化パラメータにより,次のように記述する
\footnote{手話単語の表記は拡張BNFを用いてやや詳しく記述すると以下のよ
  うになる.
   \\
  \begin{minipage}{1.0\linewidth}
\begin{tabbing}
\qquad\qquad  手話\=単語 ::= 手話単語名 [ ``\texttt{[}'' 手形 ``\texttt{]}'' ] [
``\texttt{(}'' [ 空間 ] [ ``\texttt{;}'' 修飾 ] ``\texttt{)}'' ] \\
\> 空間  ::= 位置 $\mid$ 方向 \\
\> 方向  ::= ( [ 位置 ] ``→''  位置 ) $\mid$ ( 位置 ``→'' [ 位置 ]
) \\
\> 位置  ::= ``1'' $\mid$ ``2'' $\mid$ ``3'' $\mid$ ``4'' $\mid$ ``x'' $\mid$ ``y'' $\mid$ ``R'' $\mid$
``L'' $\mid$ ``C'' $\mid$ $\cdots$
\end{tabbing}
  \end{minipage}
}.

\vspace{-0.5\baselineskip}
\begin{format}
  \texttt{手話単語名[ 手形 ]( 空間;修飾 )}
\end{format}
\vspace{-0.5\baselineskip}

手話単語名は手話単語を識別するための識別子である.便宜上その単語の意味
に近い日本語の語句\footnote{現在は基本的に『日本語—手話辞
  典』\cite{JISLS1997}のイラスト名を単語名として利用している.そこにな
  い単語やあまり一般的でない名前については一般的と思われる名前を使って
  いる.}を単語名として用いるが,日本語と手話の語彙は1対1に対応していな
いため,単語名と単語の意味とが一致しない場合もある.

語形変化パラメータには,その単語の基本形(辞書形)からの変化によって表
される付加的な語彙的・文法的情報を,「手形」「空間」「修飾」の各要素に
分けて記述する.手形パラメータには手の形による数詞等の情報,空間パラメー
タには単語が表現される空間上の位置や動詞の方向によって表される人称や格
関係,修飾パラメータには動作の強弱・大小・緩急・反復などによって表され
る修飾内容を記述する.修飾パラメータにも日本語の語句を援用している.なお,基
本形から変化しない要素については記述を省略する.以下に表記例を示
す\footnote{空間パラメータに現れる文字 `1',`2',`$x$'は,空間上の位置
  を表すと同時に,人称を表している(`1' は1人称,`2' は2人称,それ以外
  の `3', `4', `$x$', `$y$', `$L$', '$R$' などは3人称).人称と位置,格
  関係等については\ref{sec:param}節で述べる.\\ なお,手形パラメータに
  現れる `1',`2', `3' などの数字は人称ではなくその数を示す手話単語(の
  手形)を表す.}.


\begin{ex}
  (1) \> \texttt{人[3]}   \> ;3人(手話単語〈3〉の手形で〈人〉を表現)  \\
  (2) \> {\tt 話す(2→1)} \> ;あなたが私に言う
  (動詞の方向(=格関係)の表示)   \\
  (3) \> \texttt{友達($x$) もらう($x$→1)} \= ;友達からもらう
  (名詞の位置と動詞の始点の一致)   \\
  (4) \> {\tt 過去(;とても)} \> ;ずっと前 (〈過去〉の強調)
\end{ex}



\subsubsection{単語の合成}

単語の合成は,単語の逐次的な合成(複合語),左右の手で異なる単語を同
時に表現する同時的な合成,1つの単語を表現した後,片手をそのまま残して,
他方の手で別の単語を表現する半同時的な合成の3つに分けて,それぞれ以下
のように記述する.
\begin{ex}\noindent
(5) \> 手話—サークル \> ;手話サークル (逐次的合成) \\ 
(6) \> 電話 \verb+|+ 仕事  \> ;電話しながら仕事をする (同時的合成) \\
(7) \> 家($x$) / 帰る(→$x$) \> ;家に帰る(半同時的合成)
\end{ex}

\subsubsection{非手指要素と句読点}

手話では顔の表情や頭の動きなどの非手指要素が,文法的にも重要な機能を持
つことが知られている.単語の並びが同じでも,平叙文と疑問文では非手指要
素が異なるため,実際の会話では区別ができる.木村・市田(1995)
\nocite{Kimura1995}は,日本手話における疑問文,話題化,条件節,
同意を求める表現などの非手指動作について述べており,例えば話題化では,
話題化される語句に眉上げとあご引きの動作が伴うとしている.このよう
な非手指文法標識についても,その機能をテキストとして明示的に記述する.


次の表記は,単語列に非手指要素{\it NMS}が伴うことを表す.

\begin{format}
  \textrm{\{$<${\it NMS}$>$ 単語列\}}
\end{format}

\textit{NMS}の部分には,\texttt{t}(話題化),\texttt{cond}(条件節
),\texttt{cleft}(分裂文)など,
\pagebreak
非手指要素による文法標識を表す文字列
を指定する.ただし,疑問文を表す非手指要素は文末の記号「?」で表す.こ
のほか,通常の文末は「。」,文法的な区切りは「\texttt{,}」「\texttt{;}」
で表す.これらは動作的には,うなずきや瞬き,時間的な間合いによって表さ
れる.

\begin{ex}
  (8) \>  \tt\textrm{\{}$<$t$>$ 私 家族 \textrm{\}} 人[4]。
  \> ;私の家族は4人です.
\end{ex}

また,助動詞には「\texttt{\~{}}」を前置する.手話の助動詞のほとんどが内容
語としての用法を併せ持っているため,この記号により助動詞的用法であるこ
とを明示する.

\begin{ex}
(9) \>  \texttt{行く \~{}いらない}  \> ;行かなくてもいい
\end{ex}

\section{日本語から多言語への機械翻訳エンジンjaw}

翻訳システムの核となる機械翻訳エンジンjawについては謝ら(2004)が既に述べ
ているが,その後の進展もあるためここで改めてその翻訳方式について簡潔に
述べる.


jawは日本語から他の任意の言語への翻訳を目的とした,パターン変換に基づく
機械翻訳エンジンである.日本語パターンとそれに対する変換規則を用意する
ことによって,いろいろの目的言語に対応することができる.これまでにjawを
用いて中国語・ベトナム語・ミャンマー語・シンハラ語を目的言語とする機械
翻訳について研究が行われている(図\ref{fig:Chinese-Vietnamese}).以下
では中国語への翻訳を例に用いて述べる.

\begin{figure}[b]
  \centering
      \includegraphics{15-1ia4f1.eps}
  \caption{jawを用いた機械翻訳システムの出力例}
  \label{fig:Chinese-Vietnamese}
\end{figure}



\subsection{表現構造を介した翻訳の流れ}

\begin{figure}[t]
  \centering
      \includegraphics{15-1ia4f2.eps}
  \caption{jawによる翻訳の流れ}
  \label{fig:jaw}
\end{figure}

jawによる機械翻訳の流れを図\ref{fig:jaw}に示す.入力された日本語文に対
して,日本語解析器ibukiCおよびibukiS(山田他2006)\nocite{Yamada2006}を用
いて形態素・文節構造・係り受け構造の各解析を行なった後,目的言語の表現
構造を介して目的言語テキストを生成する.表現構造は目的言語の文の表現要
素(単語など)を表すC++ 言語のオブジェクトであり,表現要素に対する訳語や,
関連する他の表現要素へのリンクなどの情報をその属性として保持する.


日本語文の解析結果に対してパターン翻訳規則と機能語翻訳規則を適用するこ
とにより,日本語文を目的言語の表現構造へ変換する.パターン翻訳規則は,
入力日本語文の係り受け構造と関係データベース(RDB)上の日本語パターンを
照合し,マッチした箇所を目的言語の表現構造に変換するC++ の関数群であり,
主として文の骨格となる命題部分の表現構造を組み立てる.機能語翻訳規則は,
モダリティ等を表す機能語の翻訳として,表現要素オブジェクトに情報を設定
する.

各表現要素オブジェクトには,それを一次元のテキストに変換するためのメソッ
ド(線状化関数と呼ぶ)が,クラス(品詞等)毎に定義されており,この線状
化関数の呼び出しにより表現構造から目的言語テキストが生成される.



\subsection{目的言語表現構造への変換}

日本語の文は基本的に,命題的な内容とモダリティ等で構成される.命題的な
内容は,パターン翻訳規則により目的言語の表現構造へ変換する.機能語で表
されるモダリティ等については,命題部分とは分離して機能語翻訳規則で処理
することも,パターンに含めて命題部分と一緒に処理することも可能である.


\subsubsection{パターン翻訳規則による表現構造の生成}

照合に用いる日本語パターンにはキーワードとなる語が必ず一つだけ存在する.
パターンの種類は,キーワードの種類(内容語/機能語),および,キーワード
文節の他の文節との係り受け関係によって,
次の3種類に分類される(図\ref{fig:JapanesePatterns}).


\begin{description}
\item[(a) 基本型:]
  受け側の語句の内容語をキーワードとするパターン.
  
\item[(b) 追加型(内容語):] 係り側の語句の内容語をキーワードとするパター
  ン.
\item[(c) 追加型(機能語):] 係り側の語句の機能語をキーワードとするパター
  ン
(文や文節を接続する機能語をキーワードとするパターン).
\end{description}

基本型は従来の結合価パターンと同様のパターンであるが,次のような特徴が
ある.
\begin{itemize}
\item キーワード文節の機能語に対する条件も指定できる(図
  \ref{fig:baseType}左)
\item 2階層以上の深さを持ったパターンも記述できる(大域パターン.図
  \ref{fig:baseType}中央)
\item 名詞や副詞など,動詞以外の語に対しても記述できる(
  図\ref{fig:baseType}右)
\end{itemize}

\begin{figure}[b]
  \centering
      \includegraphics{15-1ia4f3.eps}
  \caption{日本語パターンの種類.二重枠で囲まれた語句はそのパターンのキーワード}
  \label{fig:JapanesePatterns}
\end{figure}
\begin{figure}[b]
  \centering
      \includegraphics{15-1ia4f4.eps}
  \caption{基本型パターンの例(従来の結合価パターンとの違い)}
  \label{fig:baseType}
\end{figure}


日本語パターンは表\ref{tab:patterns}のような形式でRDBに格納される.各
パターンは1文節1レコードで記述され,そのうち1つがキーワードを含む.各レ
コードには,そのパターン内での文節番号,係り先文節番号,内容語条件(意
味属性\footnote{意味属性は『日本語語彙大系』\cite{Ikehara1999}を参考に
  した.}または字面),機能語条件,省略可能文節かどうかのフラグのほか,
パターン内文節の語順についての制限の有無,(動詞キーワードの場合)受動
態としての使用可能性などの情報が登録されている.


\begin{table}[t]
  \caption{RDB上の日本語パターンの例(概略)}
  \label{tab:patterns}
  \centering
\input{04table1.txt}
\end{table}

入力日本語文の解析で得られた文節係り受け構造の各部と,これら日本語パター
ンとの照合は,まず基本型パターンを用いて次のように行う.

\begin{enumerate}
\item 根の内容語をキーとして,RDB上の日本語パターンを検索する.
\item 得られたパターンの各子ノードに対して機能語条件をチェックし,候補
  を絞り込む.
\item 各子ノードに対して日本語パターンとの照合を行い(子ノードを根とし
  て,再帰的に照合する),照合できたパターンを内容語条件で絞り込む.
\end{enumerate}

基本型パターンとの照合の後,照合されずに残ったノード(文節)があれば,
その部分に対して追加型パターンとの照合を試みる.このような照合を根から
葉に向かって再帰的に行う.このようにして,入力文の木構造を覆うことので
きる日本語パターンの組み合わせをすべて求める.使用したパターンの種類や
数,パターンの持つ条件(内容語条件,機能語条件)の厳しさ,適用された意
味属性の距離(意味カテゴリの階層構造における距離)などからコストを算出
して最適解を求める.

日本語パターンにはそれぞれ,そのパターンにマッチした入力日本語文の(部
分)構造を,目的言語の表現構造に変換するパターン変換規則が定義されている.
この変換規則は,目的言語の表現要素オブジェクトを生成し,そのデータ
メンバに訳語や他の表現要素へのリンクなどの属性を書き込むプログラムであ
る.図\ref{fig:ITtoET}に追加型(内容語)の日本語パターンの変換規則によっ
て生成される目的言語の表現構造の例を示す.

\begin{figure}[t]
  \centering
      \includegraphics{15-1ia4f5.eps}
\hangcaption{「Nのあおりを食ってV」の日
      本語パターン(左上)と,それに対する変換規則によって生成される中
      国語の表現構造(右).破線部分は別のパターン翻訳規則によって生成
      されるオブジェクトを表す.}
  \label{fig:ITtoET}
\end{figure}

パターン変換規則(C++ プログラム)は,専用のエディタ (jawEditor) を使っ
て,日本語パターンとともにフォームに必要事項を記入することで自動生成さ
れる.場合によっては人手で書く,あるいは,修正することも可能である.
図\ref{fig:TransferRule}に,「Nのあおりを食ってV」のパターン
(表\ref{tab:patterns}下部)に対する変換規則入力フォームを示す.図のよう
に,jawEditorでは多階層の規則が記述できる.

\begin{figure}[t]
  \centering
      \includegraphics{15-1ia4f6.eps}
  \caption{jawEditorによる翻訳規則の記述}
  \label{fig:TransferRule}
\end{figure}



\subsubsection{機能語翻訳規則}
\label{sec:fw}

パターン翻訳規則によって文の骨格となる命題的な内容を目的言語の表現構造
へ変換した後,機能語翻訳規則により,用言に後続してモダリティ等を表す助
動詞や体言/用言に後続する取り立て詞の翻訳に対応する各種の情報を,表現
要素オブジェクトに設定する.

入力された日本語文の各文節は,文節構造解析器ibukiCによって内容語と機能
語,係り先情報などに分割される.機能語はさらにその機能と語順に応じて,
複数のグループに分割して出力される(図\ref{fig:ibukiC_fw}).これら機能
語や係り先の情報は,命題部分のパターン変換時に,一旦そのまま目的言語の
表現構造に受け渡される.パターン変換終了後,それらに対して機能語翻訳規
則を適用することによって,目的言語での表現に必要となる情報が表現構造
オブジェクトのデータメンバに設定される.

\begin{figure}[t]
  \centering
      \includegraphics{15-1ia4f7.eps}
  \caption{ibukiCによる文節構造解析結果の概略(左:体言文節,右:用言文節)}
  \label{fig:ibukiC_fw}
\end{figure}
\begin{table}[t]
  \caption{機能語翻訳規則テーブル(1対1対応)}
  \label{tab:fwTable1}
  \centering
\input{04table2.txt}
\end{table}
\begin{table}[t]
  \caption{「も」に対する機能語翻訳規則テーブルSP\_mo(訳語選択用)}
  \label{tab:fwTable2}
  \centering
\input{04table3.txt}
\end{table}

機能語翻訳規則は表\ref{tab:fwTable1}(日本語と目的言語の機能語
が1対1に対応する場合)や表\ref{tab:fwTable2}(機能語の訳し分けが必要な
場合)のような表形式で,分割された機能語要素ごとに記述する.
例えば,表\ref{tab:fwTable1}の先頭の規則は,用言に後続する機能語要素
1が「たい」だけなら,\texttt{mModeC}
のデータメンバ\texttt{aux\_verb}に訳語 ``想'' を設定することを示してい
る.一方,「ない」「ている」については,訳語が1対1に決まらないため,そ
れぞれ専用の訳語選択テーブル\texttt{SP\_nai},\texttt{SP\_teiru}を参照
することを示している.
表\ref{tab:fwTable2} (\texttt{SP\_mo}) は体言に後続する取り立て詞「も」
に対する14番目の規則である.規則は番号順に実行され,規則1〜13の条
件がいずれも満たされなかった場合に限り,14番目の規則が適用される.この
例では,検査対象となる表現構造オブジェクト\texttt{it}(体言)が主語であ
る場合に,データメンバ\texttt{postSubject}(主語の後ろの位置)に,訳
語 ``也'' を設定するという規則を表している.

機能語翻訳規則は機能語に対する訳語の選択のほか,語順の決定で必要となる
文型(受身文,使役文など)や文のムードの判定なども行う.RDB上
に表形式で記述された機能語翻訳規則は,命題部分の翻訳規則と同様,C++ プロ
グラム(動的ライブラリ)に自動変換される.


\subsection{目的言語テキストの生成}

表現要素オブジェクトが持つ線状化関数の呼び出しにより,表現構造から目的
言語テキストが生成される.
始めにjawが,文の述語(複文の場合は主節の述語)を表す表現要素オブジェク
トの線状化関数を呼び出すと,そのオブジェクトはそれ自身の訳語を生成する
ほか,それに係る名詞や副詞,従属節の述語などの線状化関数を,目的言語の
語順に従って呼び出していく.呼び出された表現要素に係る表現要素があった
場合も同様に,その表現要素の線状化関数が呼び出され,訳文が生成される.


\section{日本手話テキストへの機械翻訳システムjaw/SL}

前節ではjawを用いた機械翻訳システムすべてに共通する事柄について述べた.
ここでは,手話を目的言語とした機械翻訳を実現するために設定した手話の表
現構造,翻訳規則,線状化関数について述べる.



\subsection{日本手話の表現構造のためのクラス設計}

手話単語の品詞分類については議論があるが\footnote{例えば田上
  ら(1979)\nocite{Tanokami1979}は,名詞・動詞・形容詞を区別をせず,これ
  らを自用詞と呼ぶ単一の品詞に分類している.},ここでは音声言語と同様の品
詞を想定し,表現要素オブジェクトのクラス階層を図\ref{fig:class}のように
設定した.

\begin{figure}[b]
  \centering
      \includegraphics{15-1ia4f8.eps}
  \caption{手話文の表現構造を構成するオブジェクトのクラス階層の概略}
  \label{fig:class}
\end{figure}

主なクラスのデータメンバの例を表\ref{tab:member}に示す.ここ
で,TObjectクラスはjawの目的言語に共通する基底クラスである.それを根と
する部分表現構造を線状化して得られた目的言語テキストを保持するとともに,
対応する日本語の機能語情報を保持する.Signクラスは手話単語共通の基底ク
ラスで,手話単語名と単語の語形変化パラメータ(手形・空間・修飾)などを
持つ.Propositionクラスは文の大枠を決める述語を表し,格要素となる名詞
(句)や,述語を修飾する副詞,従属節へのリンク,テンス・モダリティなど
の情報を持つ.Nounクラスは名詞を表すクラスであり,数量や複合名詞などを
表すクラスの基底クラスでもある.名詞を修飾・限定する表現要素へのポイン
タを持つ.


\begin{table}[t]
  \caption{手話の表現要素クラスとそのデータメンバの例}
  \label{tab:member}
\input{04table4.txt}
\end{table}



\subsection{語形変化による表現への翻訳}
\label{sec:param}

日本語では機能語や修飾語などの独立した単語として表現される情報が,手話
では主となる単語の語形(手の形・位置・動き)の変化によって表現される場
合がある.例えば,動作の主体/対象といった格関係が動詞の手の動きの向きに
より表現される場合や,様態・程度・アスペクトなどが動詞/形容詞の動きの変
化により表現される場合がある.



\subsubsection{格関係(名詞の位置と動詞の方向)}
\label{sec:case-relation}

手話では名詞の人称と話者の回りの空間上の位置が対応づけられてい
る\cite{Matsumoto2001,Baker-Shenk1980}.1人称と2人称はそれぞれ話し手と
聞き手の位置に固定されており,会話の場にいない人や物,場所などの3人称は
その他の位置に割り当てられる\footnote{標準的には,人は話し手の斜め前方,
  物や場所は話し手と聞き手の中間に配置され
  る\cite{Matsumoto2001}.}(図\ref{fig:personalLocations}).

\begin{figure}[b]
  \centering
      \includegraphics{15-1ia4f9.eps}
  \caption{人称と位置}
  \label{fig:personalLocations}
\end{figure}

主体や対象の人称(位置)に呼応して手の運動の向きが変化する動詞は一致動
詞 (agreement verb) と呼ばれ
る\cite{Ichida1999,Sutton-Spence1999,Sandler2006}.日本語では名詞に後置
された格助詞によって表される格関係が,一致動詞を使った文では,名詞の位
置と動詞の(手の動きや指先の)方向によって表示されることにな
る\footnote{一方,人称といった文法的な情報ではなく,動詞が表す動作の軌
  道や動作が行われる場所など,現実世界での動きや位置関係に呼応して手の
  動きや位置が変化する動詞は空間動詞 (spatial verb) あるいはclassifier
  predicateなどと呼ばれる.例えば,「上方に立つ」,「右方に立つ」は動詞
  〈立つ〉をそれぞれ通常より高い位置,体の右側で表現することにより表さ
  れる.ただし,一致動詞が空間動詞として使われる場合もあり,明確な境界
  があるわけではない.}.例えば,「AがBをしかる」,「AがBに言う」,
「AがBへ行く」では,動詞の手の動きは基本的にみな名詞Aの位置から名詞Bの
位置へ向かう.

\begin{figure}[t]
  \centering
\input{04fig10.txt}
\caption{一致動詞へのパターン翻訳規則の例}
  \label{fig:shikaru}
\end{figure}

一致動詞への翻訳を行うためのパターン翻訳規則の例を図\ref{fig:shikaru}に
示す.この翻訳規則により,「AがBをしかる」における名詞Aは動作主格
 (mAgent), 名詞Bは目標格 (mGoal) に設定される.また,一致動詞であるこ
とを示すフラグ (mIsAgreement) にtrueが設定される.「AがBに言う」,
「AがBへ行く」の場合も同様に,名詞AをmAgent,名詞BをmGloalに設定する.
これら3つの例において名詞Bは,日本語ではそれぞれ異なる格だが,手話の格
をこのように捉えることで,一致動詞の方向が通常mAgentからmGoalへ向かうこ
とになり,処理が簡単化できる\footnote{ただし,「AがCからBへ行く」のよう
  に,起点 (mSource) が指定されている場合には,mAgentではなくmSourceの
  名詞位置を動詞の始点とする.}.



一致動詞に対する線状化関数では,名詞オブジェクトmAgentおよびmGoalの
人称情報 (mPerson) を元に,その名詞の表現位置 (mParam.position) と動詞
の始点・終点 (mParam.direction) を決定する.1人称,2人称の場合はそれぞ
れ `1', `2' (位置定数)を,3人称の場合は `$x$',`$y$' 等(位置変数)
を設定する.ただし,身体の一部を背景として利用するために表現位置あるい
は始点/終点が固定される単語については,固定部分の位置指定は行わない.

位置を決定した後,手話の語順(\ref{sec:linearization}節)に沿って名詞・
動詞の訳語を生成する.一致動詞では1人称と2人称の名詞は動詞の方向によっ
て表示されるため,「私」「あなた」といった単語は省かれることが多い.例
えば,「あなたが私に言う」は ``話す(2→1)'' という動詞とその向きだけで
表現できる.このような名詞の省略も考慮して線状化を行う.


\subsubsection{様態やアスペクトなどの修飾表現}

音声言語では副詞として語彙化される修飾概念が,手話では述語の語形変化と
して表現される場合が多い.例えば「\underline{だんだん}暗くなる」は,手
話単語〈暗い〉を徐々に動かすことで表現される\cite{Yonekawa2005}.語形変
化で表現される修飾内容としては,様態(ゆっくり,激しく),アスペクト
(ずっと,しばしば),程度(とても,少し)などがあげられる.


日本語援用手話表記法では,今のところ語形変化によって表される修飾内容は,
被修飾語の修飾パラメータにその内容を次のように日本語の語句で指定すること
により表現している.

\begin{center}
  \begin{tabular}{lll}
    「\underline{ずっと}前」 & ⇒ & {\tt 過去(;とても) } \\
    「\underline{だんだん}暗くなる」 & ⇒ &  {\tt 暗い(;徐々に) } 
  \end{tabular}
\end{center}

副詞を述語の語形変化へ翻訳するための日本語パターンと翻訳規則の例を
図\ref{fig:sugoku}に示す.この規則により,「すごく」が係る状態述語の語
形変化パラメータ(修飾要素)に ``とても'' が設定される.なお,述語の語
形変化ではなく,独立した手話単語〈とても〉を用いて表現される場合もある
が,その場合にはPropositionのメンバmAdverbialに〈とても〉を設定する.


\begin{figure}[t]
  \centering
\input{04fig11.txt}
\hangcaption{副詞を述語の語形変化として翻訳するためのパターン翻訳規則の例.
	qualifiersは述語に係る修飾内容(複数可)を保持する.}
\label{fig:sugoku} 
\end{figure}

日本語の1つの単語が,手話では修飾パラメータを含んだ表現となる場合もある.
例えば,「さっき」は〈{\tt 過去(;少し)}〉,「真っ赤」は〈{\tt 赤い(;と
  ても)}〉と表現される.これらはその翻訳規則において各単語の修飾パラメー
タに値を直接設定する.

このほか,日本語では動詞・助動詞によって表されるアスペクト等の情報が,
語形変化で表現される場合もある.それらは,動詞・助動詞の翻訳規則で修飾
パラメータに値を設定する.

\subsection{手話の助動詞と機能語翻訳規則}
\label{sec:functionwords}

手話にも日本語の助動詞のように,動詞の後ろに置かれ,モダリティやアス
ペクトなどを表す単語が存在す
る\cite{Ichida2000,Ichida2005b,Matsumoto2001}.
その例を表\ref{tab:aux}に示す.

\begin{table}[t]
  \caption{手話の助動詞の例}
  \label{tab:aux}
  \centering
\input{04table5.txt}
\end{table}

用言に後続する機能語の翻訳規則の例を図\ref{fig:FWTransRule}に示す.同
図左側の表は日本語の機能語を,それに1対1に対応する訳語に置き換える規則
の例であり,右側の表は訳し分けが必要な場合の例である.この例では,日本
語の終助詞「か」が文末にあり,かつ,機能語が「ます」等を含むとき,
「か」を直接手話単語に置き換えるのではなく,文のムード(Propositionオ
ブジェクトが持つmModeSLオブジェクトのデータメンバmoodの値)を
interrogative(疑問)に設定している.この値は線状化の段階で,語順や
句読点の種類を決定する際に参照される.

\begin{figure}[t]
  \centering
      \includegraphics{15-1ia4f12.eps}
  \caption{機能語翻訳規則の例}
  \label{fig:FWTransRule}
\end{figure}


\subsection{線状化関数}
\label{sec:linearization}

最後に,命題部分と機能語部分の変換により得られた手話言語の表現構造から,
語順の決定,非手指文法標識の付加,一致動詞における方向や名詞の位置の決
定等を行い,手話テキストを生成する.この処理は先に述べたように各表現要
素オブジェクトが持つ線状化関数で行う.



\subsubsection{語順}

日本手話の基本的な文は,話題(主題)とそれに関する陳述で形成されている
(topic-comment構造).これは日本語やBSL(イギリス手話),ASL(アメリカ
手話)などと同様である\cite{Nakamura2002,Sutton-Spence1999}.話題は文頭
に置かれ,話題化のための非手指要素による文法標識を伴う.〈昨日〉〈今〉
のような語によって,時間的な枠組みが文頭で設定される場合も多い.また,
日本手話の基本語順はS-O-V(主語—目的語—動詞)と言われており,日本語と同
様,述語は原則として文末に配置される\cite{Kimura1995,Matsumoto2001}.述
語には助動詞が後続し,さらに,動作主や述語の対象などを示す指差し
(pronoun copy) が付加される場合もある.話題と述語以外の要素はこれら
の間に置かれるが,前述のように動作主や対象は動詞の方向によって表わされ,
単語として表現されない場合もある.また,疑問詞を含む疑問文では,疑問詞
が文末に置かれる場合が多い\cite{Yonekawa2005}.談話の先頭以外の文は,
「ところで」,「次に」,「しかし」,「だから」などの接続的な語で始まる
ことも少なくない\cite{Tanokami1979}.

jaw/SLでは,Propositionオブジェクトがデータメンバとして持つ表現要素を,
基本的に次のような順序でテキスト化している.

\begin{quote}
  接続的な語
  → 話題化された名詞(句)・時間的枠組みを設定する語
  → その他の格要素となる名詞(句)
  → (述語に前置される)副詞
  → 述語
  → (述語に後置される)副詞
  → 助動詞
  → (疑問文での)疑問詞.
\end{quote}

修飾語は被修飾語に前置される場合と後置される場合とが併存している.修飾
語の語順について松本(2001)\nocite{Matsumoto2001}は,原則的には中心的な
語が付随的な語に先行するのが手話の自然な語順だが,倒置によって修飾語を
強調する効果があること,そして,日本語の影響を受けて修飾語の前置が受け
入れられ,単語によっては原則の方が廃れていったという考えを述べている.
しかし,〈とても〉〈最高〉〈最低〉など程度を表す副詞は現在でも前置する
と不自然であり,大きさや形状を表す修飾語も後置が一般的
としている.一方,市田(1998)\nocite{Ichida1998}は,名詞句内の語順は,形
容詞—名詞,関係節—名詞,属格—名詞であり,形容詞や関係節が名詞に後置され
ているように見える例は,主要部内在型関係節という構造を利用した表現であ
ると説明している.副詞の語順については言及していない.

jaw/SLでは,修飾語ごとに前置/後置が分かれるものと仮定して,修飾語の表
現要素クラスに前置/後置を指定するためのデータメンバを設け,個々に語順
を指定することにした.

このほか,数と単位の語順/表現は,(1)単位を数に後置する(金額など),
(2)単位を数に前置する(年齢など),(3)数を手形で表し,単位を動きで
表す(年数など),(4)片手で数,もう一方の手で単位を表す(人数など)と
いうように対象により異なっている.手話の表現構造レベルでは,これら語
順/表現の違いを,数量表現クラスQuantityの持つ単位格納用メンバ
(mPostfixedUnit,mMovementUnitなど)の違いとして表現しており,メンバご
とに異なる線状化コードを割り当てている.




\subsubsection{非手指要素による文法標識}

話題化や疑問文,条件節などを表す非手指要素による文法標識のテキスト化も
線状化関数で行う.

話題化された名詞の検出は,Propositionクラスの線状化関数内で,mAgent,
mObject等の格要素に対して表\ref{tab:fwrule_ha}のような機能語翻訳規則を
適用することにより行う.この規則では,助詞「は」を伴う名詞のうち自称の
代名詞でないものに話題化の印が付けられる.表中の\texttt{it}は述語の格要
素となる名詞オブジェクトへのポインタである.このテーブルには2つの規則
(id=1およびid=99)が登録されており,idの小さいものから順に適用される.規
則1により,名詞オブジェクトの日本語における意味属性(メンバ変
数\texttt{mSemanticAttr})の値が10007〜10013の範囲(僕,私など自称の代
名詞)である場合には話題化フラグ(メンバ変数\texttt{topic})にfalseが設
定され,その他の場合には規則99によりtrueが設定される.話題化された名詞
(句)には,話題化の非手指文法標識を表す記号 ``\{$<$t$>$ $\cdots$
\}'' を付加して線状化する.

機能語の翻訳によって文が疑問のムードを持つと判断された場合は,格要素に
疑問詞が存在するならその格の線状化を文末まで遅延し,文末記号を標準の「。」
から疑問の非手指文法標識を表す「?」に変更する.

\begin{table}[t]
\caption{助詞「は」に対する機能語翻訳規則(訳語選択テーブルSP\_wa).}
  \label{tab:fwrule_ha}
  \centering
\input{04table6.txt}
\end{table}



\section{翻訳実験}

手話学習教材から取得した文を題材として翻訳実験を行った.
まだオープンテストによって精度を評価するような段階にはなく,取得
した文の翻訳に必要な規則を実際に与えてシステムを構築していく過程で,現
状の問題点を明らかにしていくことをねらいとした.



\subsection{実験方法}

\begin{table}[b]
  \caption{『手話ジャーナル』における手話文の自然な日本語訳と構
    造訳}
  \label{tab:struct}
  \centering
\input{04table7.txt}
\end{table}

翻訳実験の題材には主として日本手話のビデオ教材『手話ジャーナル 初級教
材No.~1 \& 2』\cite{SignFactory1997,SignFactory1999}を用いた(以下,単
に「手話ジャーナル」と記す).各巻にはそれぞれ4人のろう者がカメラに向かって家
族や日常生活について日本手話で語った映像が記録されている.日本語の音声
や字幕は一切含まれていないが,全文(約720文)に対する自然な日本語訳と,
手話の構造に沿った日本語訳(手話ジャーナルでは ``構造訳'' と呼んで
いる)が記載された小冊子が付属している.表\ref{tab:struct}に自然な日本
語訳と構造訳の例を示す.このほか,手話の教育番組\cite{Yonekawa2006}から
の文も補助的に使用した.これらはすべて会話文/話し言葉である(手話には
文字がないので書き言葉は存在しないことからくる必然である).

基本的に手話文に対する ``構造訳'' を入力日本語テキストとし,手話映像
を\ref{sec:notation}節の日本語援用表記法で書き取った手話テキストを正解
とした.教材に含まれる文から80文を抽出し,正解を出力するために必要な日
本語パターンとその変換規則,機能語翻訳規則,線状化関数をシステムに与え
た.談話の一部を抜き出して利用した文もある.

手話ジャーナルの映像を対象に行った調査\cite{Matsumoto2006}におい
て,\ref{sec:notation}節の表記法による記述が困難と判断された表現を含む
文が全体の1割程度あったが,これらは対象から除外した.また,現在jawの翻
訳単位は文であるため,空間上の位置と名詞との対応関係が複数の文をまたい
で持続するような文も除外した.このほか,言い間違いを訂正した
り,付け加えたりしている箇所を含む文も80文には含まれていない.
ただし,翻訳規則の検討や問題点の分析には翻訳対象以外の文も考慮に入れた.

なお,入力文に対する文節構造解析や文節係り受け解析に誤りが生じた場合は,
誤り箇所を人手で修正した.



\subsection{実験結果と考察}

実験の結果,正解どおりの手話文を導くことができたのは80文中48文,正解ど
おりではないが正しい手話文が得られたと考えられるものが20文,正しく翻訳
できない表現を含むものが12文あった.

次のような項目については,意図したとおり出力させることができた(
括弧内の数字は表\ref{tab:transexamples}の文番号).

\begin{table}[t]
    \caption{正解を導けた翻訳結果の例
      \label{tab:transexamples}}
\input{04table8.txt}
\end{table}

\begin{itemize}
\item 格関係による動詞の方向変化と名詞の位置変化(
  1,4,14,17,24など)
\item 疑問文,話題化,条件節等の非手指文法標識への翻訳(2, 5, 7, 20な
  ど)
\item 用言に後続する機能語によるモダリティやアスペクトなどの
  翻訳(4, 5, 18, 20など)
\item 手話単語の(半)同時的な組み合わせによる表現(2, 12など)
\item 語順(数詞と単位,疑問詞,話題化,修飾語と被修飾語)
  (9, 22など)
\item 手形変化による語義の変化(手形で数詞,動きで単位を表す場合)(
   5, 9, 20)
\item 語形変化による副詞の表現(6)
\item 複文(5)
\item 指文字による単語表現(23)
\end{itemize}

また,次のような表現については,正解と異なっていても正しく翻訳できたと
判断した.

\setcounter{example}{0}
\begin{table}[t]
  \caption{正解どおりではないが正しい手話文が得られたと判断した翻訳結果の例}
  \label{tab:transexamples2}
\input{04table9.txt}
\end{table}


\begin{itemize}
  \renewcommand{\labelenumi}{}
\item 時間・金額の表現:一般に時刻の表現では,時間を表す数詞に手話単語
  〈時間〉を前置するとされるが,〈時間〉が省かれる例が多く見られた(表
  \ref{tab:transexamples2}:例文1).金
  額の表現でも同様に,通常,数詞に後置される単位〈円〉が省かれる例が見
  られた(同表例文2).翻訳結果では基本的にこれらの単位を省略せずに出力してい
  る\footnote{ただし,話題の中に〈時間〉という単語が現れた場合は,数詞
    の前の〈時間〉を省くようにしている.}.
\item 祖父・祖母の表現:手話単語〈祖父〉と〈祖母〉は,人差し指でほおに
  触れる動作(肉親を表すといわれる)を行った後,それぞれ手話単語〈おじ
  いさん〉または〈おばあさん〉を表現することで表される.ほおに触れる動
  作が省略される例が多く見られたが,試作システムでは基本的に省かず表現
  している(例文3).
\item 文末の指差し:手話では文末に人や物,場所と対応づけられた位置への
  指差し(代名詞)が現れる場合がある.例えば〈私 朝 苦手 私〉=「私
  は朝が苦手です.」のように文中の代名詞が文末で繰り返されこともあれば,
  〈起きる 難しい 私〉=「(私は)起きられないのです.」のように,文
  末だけに現れることもある.文末の指差しには,だれ(何)についての話な
  のかを明確にする働きがあると推測される.しかし,同じ日本語「私は朝が苦
  手です.」が〈私 朝 苦手〉と文末の代名詞なしで表現される例もあり,
  また,例文全体から見ても使用しない文の方が多いため,試作システムでは
  文末の代名詞を出力していない(例文4).
\item 接続助詞「ので」:「ので」に相当する手話単語〈ので〉は省略され,
  うなずきだけで表現される場合もあるが,システムの出力結果は〈ので〉を
  省略していない(例文5).

\end{itemize}




一方,適切な訳を出力することが難しい表現も多く存在した.以下,現状での
主な問題点について述べる.



\subsubsection{省略・文脈に伴う問題点}

\begin{table}[t]
  \caption{問題の残った表現を含む例文の一部}
  \label{tab:problems}
\input{04table10.txt}
\end{table}

表\ref{tab:problems}の例1では,動詞〈断る〉が複数変化(終点を3人称の位
置範囲で変えながら,2, 3回繰り返す)して,複数の相手に
対して「断る」ことを表している.手話では動作主や対象等の数に呼応して動
詞の語形が変化す
る\cite{Oka2005,Matsumoto2001,Sandler2006,Sutton-Spence1999}が,この例
文のように入力日本語文が数についての情報を持たない場合,正解を導けなかっ
た.また,例2の〈しかる〉は,「ことが多い」という言葉から,〈しかる〉と
いう行動が複数回行われることが分かるが,その対象が同一人物か複数かまで
は判断できない.手話ではその違いが動詞の方向の違いとして表現される.こ
れらを正確に処理するには,その文で明示的に述べられていない情報について
文脈から取得し補完する必要がある.現状では文脈を考慮した翻訳には対応し
ていないために,このような動詞の変化を正しく出力することができていない.

一般に,日本語に比べて手話では物事をより具体的に表現する傾向があるため,
日本語から手話への翻訳では入力文中に明示されていない情報を正しく補わな
ければ,手話として不自然な表現や間違った表現になる可能性がある.
数の一致はその一例といえる.不足した情報をいかに補足するかが問題である.



\subsubsection{グループ化や対比のための空間の利用}

表\ref{tab:problems}の例3と4は重文で,それぞれ前半と後半の節で表現位置
が変わる.例3では母と父の仕事,例4では家族内でのろう者と聴者のグループ
ごとに表現位置を分けることによって,話の内容を視覚的に分かりやすく伝え
ている.しかし,重文が常にこのように表現されるわけではなく,現状ではこ
のような空間の利用についての判断が機械的に行えていない.


\subsubsection{機能語(ながら,られる,た,ている)の翻訳}
\begin{itemize}
\item 「ながら」:複数の行為を並行して行うことを表す接続助詞「ながら」
  は,手話では単語として表現されない.各行為を表す動詞を実際に(部分的
  に)同時に表現する(例5),動詞を交互に連続的に表現する(例6),非手
  指動作を使って同時性を表現する(例7の「食べながら仕事をする」は,食べ
  物を噛む仕草をしながら〈仕事〉を表現)といった方法が見られた.動詞の
  組み合わせによって,「ながら」の表現方法が異なると考えられるが,現状
  ではその訳し分けはできていない.

  〈食べる〉や〈眠い〉の非手指動作部分(口の動き,顔の表情)が次の動詞
  と同時に表現されることによっても同時性が表されているが,そのような表
  現に対する表記法上の問題もある.


\item 「た」:過去や完了を表す日本語の助動詞「た」に相当する手話単語に
  は〈た〉と〈終わる〉がある.だが,『手話ジャーナル』の例文中に出現し
  た「た」270例のうち,これらの単語が使用されたのは11例であり,その他の
  例では単語として表現されていなかった\footnote{一方,手話ニュース等の
    改まった場での報告では,過去を表すためにこれらの語が頻繁に用いられ
    るようである.}.例えば「亡くなった」は21例中,2例だけが〈死ぬ 終
  わる〉と表現されていた.〈た〉と〈終わる〉はいずれも単純な過去という
  より,完了的な意味合いが強いとされるため,話者の主観によって表現が異
  なることも考えられる.現状では,一部の慣用的な表現を除いて,「た」に
  対する訳語を出力していない.逆に,例8の「昼食をとって」のように,日本
  語では「た」が現れない箇所での〈終わる〉の出力も現状ではできていな
  い.

  なお,単語として過去を表現しない場合でも,動詞を表現する際,口形が
  「た」になる例が散見された.
  また,〈終わる〉には,過去・完了の助動詞的な用法の他に,動詞「終わる」
  や名詞「終わり」としての用法があり,「仕事が終わる」「食べ終わる」
  「おしまい」など,動詞や名詞としての用法は多数見られた.

\item 「ている」:動きの継続・進行中・習慣や動きの結果の状態を表す「て
  いる」は,『手話ジャーナル』の例文中に145回出現し,〈最中〉を使った表
  現が7例,〈いる〉と〈ある〉がそれぞれ2例ずつあった.その他の「ている」
  は単語として表現されなかった.「(寝ないで)起きている」のように動詞
  〈目覚める〉を長めに表現することで状態の継続を表す例もあった.現状で
  は,明確な翻訳規則を見出すことができず,「ているところだ」など一部
  の慣用的な表現を除いて「ている」に対する訳語を出力していない.

\item 「られる」:受身の「られる」に対応する手話単語は存在しな
  い\footnote{可能の「られる」には〈できる〉が対応する.}.動詞の方向の
  変更(一致動詞の場合)や,格要素の入れ替え,機能語条件として「られる」
  を持つ動詞パターンの追加によって翻訳できた例文もある.しかし,「(桜
  の木でびっしり)囲まれている」を,立てた指の動きと表情で,楕円状に木
  が密に並んでいる様子を表現する例や,「(ビールを)飲まされる」を,コッ
  プを渡され嫌々飲む様子で表現する例は,日本語と手話との構造の隔たりが
  大きく,次に述べる「言い換え」や表記法の問題と絡んで翻訳することがで
  きていない.

\end{itemize}


\subsubsection{言い換えの問題}

『手話ジャーナル』からの例文は手話の構造に近い「構造訳」を入力文とした
が,実用的な観点からは「自然な日本語文」から「手話への翻訳に適した日本
語文(構造訳)」への言い換えをシステム側で行う必要がある.現状では,次
のように構造訳からさらに(文脈を考慮して)手を加え,「(正しく翻訳可能
な文)」のようにしなければ正しい手話テキストに翻訳できない例もあった.

\begin{ex}
   \> 将来飼ってみたいのは大きな犬です。  \> \rule{16zw}{0pt} \=
  (自然な日本語文) \\
   \>  将来飼ってみたいのは\underline{何かというと},大きな犬\underline{が欲しい}です。
  \>\> (構造訳)  \\
  \>  将来\underline{欲しい}のは何かというと,大きな犬が欲しいです。
   \>\> (正しく翻訳可能な文) \\
    \> \{$<$cleft$>$ 将来 好き 何\}, 犬 大きい 好き。
   \>\> (手話文)
\end{ex}

また,実験では主に手話文の日本語訳を入力としたが,現実の翻訳では手話の
語彙不足も問題である.機能語や副詞については,単語として存在しなくても,
語形変化・非手指動作・同時表現などで表される場合があるため単純に比較で
きないものの,市販の手話辞典の語彙は数千語しかなく,日本語のそれに比べ
て著しく少ない.指文字(50音等に対する手指表現)を使って単語の音を伝え
ることはできるが,聞き手がその言葉を知らなければ意味は伝わらない.その
ため(固有名詞を除き)類義語で代用するか,手話語彙の範囲で意味を説明す
るといった処理が必要となる\footnote{例えば,田上
  ら(1979)\nocite{Tanokami1979}は,「弟はテレビばかり見ている」の「ばか
  り」を各地の手話通訳者が〈だいたい〉,〈毎日〉,〈たくさん〉,〈一生
  懸命〉,〈だけ〉などの手話単語で代用して訳したという例をあげている.}.
この点からも,手話への翻訳において言い換え技術の導入は今後検討すべき大
きな課題といえる.


\subsubsection{表記法,その他の問題}

今回の実験では,単語を用いないパントマイム的な表現や,単語を用いていて
も,その動きや表現位置が現実世界での動きや位置関係を再現するように変化
する自由度の高い表現(空間動詞)など,\ref{sec:notation}節の表記法によ
る記述が難しい例文は対象としなかった.例えば「(隣で)向かい合って座っ
ていた(男女)」は,両手のそれぞれで表した〈座る〉の手形を向かい合わせ
にし,話者の右側に配置することで,男女間および男女と話者の間の位置関係
が簡潔に表現された.このような表現は視覚的で分かりやすい,手話らしい表
現であり,その生成は手話への機械翻訳において重要であると考えられる.し
かしながら,現状ではそれらをどう記述し,機械的に生成するか,難しい問題
である.

口形の表記と生成についても検討が必要と考えられる.現状では口形を単語の
一部と見なし,明示的に表記していないが,実験に用いた例文では,省略され
た時間の単位を口形で補う例,過去を表す口形,同じ手話単語(手指要素)で
口形だけが異なる例などが見られた.このような口形の語彙的な働きについて
調査・整理する必要がある.


\section{おわりに}


日本語—日本手話機械翻訳システム構築のための最初のステップとして,パター
ン変換型機械翻訳エンジンjawを核としたルールベースの翻訳システムを試作し
た.入力は日本語テキスト,出力は我々が提案した日本語援用手話表記法であ
る.このアプローチの有効性と現状の問題点を明らかにするため,日本手話の
ビデオ教材等を題材とした翻訳実験を行った.
格関係による動詞の方向や名詞の位置の変化,話題化や疑問を表す非手指文法
標識など,手話に特徴的な言語要素を含む手話文が生成可能であることが確認
できた.しかし,数の一致,言い換え,表記法上の問題など未解決の問題も
多く,課題は山積している.また,音声言語と比較すると手話の言語学的解析
はまだ十分行われているとは言えず,そのことも翻訳システムを構築する上で
の大きな課題である.手話言語学の領域での今後の進展に期待したい.


\acknowledgment

本研究を進めるにあたり,手話に関してご教示いただいた愛知医科大学原大介
先生,岐阜県立岐阜聾学校鈴村博司先生・長瀬さゆり先生,岐阜大学池谷尚剛
先生に深く感謝いたします.なお,本研究の一部は科学研究費補助金・基盤研
究C(課題番号18500111)により行われました.


    \bibliographystyle{jnlpbbl_1.3}
\begin{thebibliography}{}

\bibitem[\protect\BCAY{Baker-Shenk \BBA\ Cokely}{Baker-Shenk \BBA\
  Cokely}{1980}]{Baker-Shenk1980}
Baker-Shenk, C.\BBACOMMA\ \BBA\ Cokely, D. \BBOP 1980\BBCP.
\newblock {\Bem American {S}ign {L}anguage, A Teacher's Resource Text on
  Grammar and Culture}.
\newblock Clerc Books, Gallaudet University Press.

\bibitem[\protect\BCAY{藤重\JBA 黒川}{藤重\JBA 黒川}{1997}]{Fujishige1997}
藤重栄一\JBA 黒川隆夫 \BBOP 1997\BBCP.
\newblock \JBOQ
  意味ネットワークを媒介とする日本語・手話翻訳のための日本語処理\JBCQ\
\newblock \Jem{計測自動制御学会ヒューマン・インタフェース部会Human Interface
  News and Report}, {\Bbf 12}  (1), \mbox{\BPGS\ 45--50}.

\bibitem[\protect\BCAY{平田\JBA 池田\JBA 岩田\JBA 黒川}{平田\Jetal
  }{2003}]{Hirata2003}
平田麗湖\JBA 池田隆二\JBA 岩田圭介\JBA 黒川隆夫 \BBOP 2003\BBCP.
\newblock \JBOQ
  中間型手話における名詞表現位置に関する文法とその日本語手話翻訳への導入\JBCQ\
\newblock \Jem{ヒューマンインタフェースシンポジウム2003論文集}, \mbox{\BPGS\
  293--296}.

\bibitem[\protect\BCAY{市田}{市田}{1998}]{Ichida1998}
市田泰弘 \BBOP 1998\BBCP.
\newblock \JBOQ 日本手話の名詞句内の語順について\JBCQ\
\newblock \Jem{日本手話学会 第24回大会論文集}, \mbox{\BPGS\ 50--53}.

\bibitem[\protect\BCAY{市田}{市田}{1999}]{Ichida1999}
市田泰弘 \BBOP 1999\BBCP.
\newblock \JBOQ 日本手話一致動詞パラダイムの再検討—「順向・反転」「4
  人称」の導入から見えてくるもの—\JBCQ\
\newblock \Jem{日本手話学会 第25回大会論文集}, \mbox{\BPGS\ 34--37}.

\bibitem[\protect\BCAY{市田\JBA 川畑}{市田\JBA 川畑}{2000}]{Ichida2000}
市田泰弘\JBA 川畑裕子 \BBOP 2000\BBCP.
\newblock \JBOQ 日本手話の助動詞について\JBCQ\
\newblock \Jem{日本手話学会 第26回大会論文集}, \mbox{\BPGS\ 6--7}.

\bibitem[\protect\BCAY{市田}{市田}{2005}]{Ichida2005b}
市田泰弘 \BBOP 2005\BBCP.
\newblock \JBOQ 手話の言語学(11)
  文法化—日本手話の文法(7)「助動詞,否定語,構文レベルの文法化」\JBCQ\
\newblock \Jem{月刊言語}, {\Bbf 34}  (11), \mbox{\BPGS\ 88--96}.

\bibitem[\protect\BCAY{市川}{市川}{2001}]{Ichikawa2001}
市川熹 \BBOP 2001\BBCP.
\newblock \JBOQ 手話表記法sIGNDEX\JBCQ\
\newblock \Jem{手話コミュニケーション研究}, {\Bbf \rule{0pt}{1pt}}  (39),
  \mbox{\BPGS\ 17--23}.

\bibitem[\protect\BCAY{池田\JBA 岩田\JBA 黒川}{池田\Jetal }{2003}]{Ikeda2003}
池田隆二\JBA 岩田圭介\JBA 黒川隆夫 \BBOP 2003\BBCP.
\newblock \JBOQ
  日本語手話翻訳のための言語変換とそこにおける語形変化規則の処理\JBCQ\
\newblock \Jem{ヒューマンインタフェース学会研究報告集}, {\Bbf 5}  (1),
  \mbox{\BPGS\ 19--24}.

\bibitem[\protect\BCAY{池田}{池田}{2006}]{Ikeda2006}
池田尚志 \BBOP 2006\BBCP.
\newblock \JBOQ
  言語バリアフリーな社会を目指して—日本語テキストから手話テキストへの機械翻訳
—\JBCQ\
\newblock \Jem{放送文化基金報(HBF)}, {\Bbf \rule{0pt}{1pt}}  (69),
  \mbox{\BPGS\ 30--31}.

\bibitem[\protect\BCAY{池原\JBA 宮崎\JBA 白井\JBA 横尾\JBA 中岩\JBA 小倉\JBA
  大山芳史\JBA 林}{池原\Jetal }{1999}]{Ikehara1999}
池原悟\JBA 宮崎正弘\JBA 白井諭\JBA 横尾昭男\JBA 中岩浩巳\JBA 小倉健太郎\JBA
  大山芳史\JBA 林良彦 \BBOP 1999\BBCP.
\newblock \Jem{日本語語彙大系}.
\newblock 岩波書店.

\bibitem[\protect\BCAY{稲葉}{稲葉}{1998}]{Inaba1998}
稲葉通太 \BBOP 1998\BBCP.
\newblock \Jem{アクセス! ろう者の手話:言語としての手話入門}.
\newblock 明石書店.

\bibitem[\protect\BCAY{日本手話研究所}{日本手話研究所}{1997}]{JISLS1997}
日本手話研究所\JED\ \BBOP 1997\BBCP.
\newblock \Jem{日本語-手話辞典}.
\newblock 全日本ろうあ連盟.

\bibitem[\protect\BCAY{木村\JBA 市田}{木村\JBA 市田}{1995}]{Kimura1995}
木村晴美\JBA 市田泰弘 \BBOP 1995\BBCP.
\newblock \Jem{はじめての手話—初歩からやさしく学べる手話の本}.
\newblock 日本文芸社.

\bibitem[\protect\BCAY{松本}{松本}{2001}]{Matsumoto2001}
松本晶行 \BBOP 2001\BBCP.
\newblock \Jem{実感的手話文法試論}.
\newblock 全日本ろうあ連盟.

\bibitem[\protect\BCAY{Matsumoto, Tanaka, Yoshida, Imai, \BBA\ Ikeda}{Matsumoto
  et~al.}{2004}]{Matsumoto2004a}
Matsumoto, T., Tanaka, N., Yoshida, A., Imai, Y., \BBA\ Ikeda, T. \BBOP
  2004\BBCP.
\newblock \BBOQ The first step toward a machine translation system from
  {J}apanese text to sign language\BBCQ\
\newblock In {\Bem Proceedings of Asian Symposium on Natural Language
  Processing to Overcome Language Barriers}, \mbox{\BPGS\ 61--66}.

\bibitem[\protect\BCAY{松本\JBA 田中\JBA 吉田\JBA 谷口\JBA 池田尚志}{松本\Jetal
  }{2004}]{Matsumoto2004b}
松本忠博\JBA 田中伸明\JBA 吉田鑑地\JBA 谷口真代\JBA 池田尚志 \BBOP 2004\BBCP.
\newblock \JBOQ
  手話の表記法とテキストレベルの日本語—手話機械翻訳システムの試みについて\JBCQ\
\newblock \Jem{信学技報}, {\Bbf 104}  (316), \mbox{\BPGS\ 7--12}.

\bibitem[\protect\BCAY{Matsumoto, Taniguchi, Yoshida, Tanaka, \BBA\
  Ikeda}{Matsumoto et~al.}{2005}]{Matsumoto2005b}
Matsumoto, T., Taniguchi, M., Yoshida, A., Tanaka, N., \BBA\ Ikeda, T. \BBOP
  2005\BBCP.
\newblock \BBOQ A proposal of a notation system for {J}apanese {S}ign
  {L}anguage and machine translation from Japanese text to sign language
  text\BBCQ\
\newblock In {\Bem Proceedings of the Conference Pacific Association for
  Computational Linguistics (PACLING 2005)}, \mbox{\BPGS\ 218--225}.

\bibitem[\protect\BCAY{松本\JBA 谷口\JBA 吉田\JBA 田中\JBA 池田}{松本\Jetal
  }{2005}]{Matsumoto2005a}
松本忠博\JBA 谷口真代\JBA 吉田鑑地\JBA 田中伸明\JBA 池田尚志 \BBOP 2005\BBCP.
\newblock \JBOQ
  日本語—手話機械翻訳システムに向けて—テキストレベルの翻訳系の試作と簡単な例
文の翻訳—\JBCQ\
\newblock \Jem{信学技報}, {\Bbf 104}  (637), \mbox{\BPGS\ 43--48}.

\bibitem[\protect\BCAY{松本\JBA 池田}{松本\JBA 池田}{2005}]{Matsumoto2005c}
松本忠博\JBA 池田尚志 \BBOP 2005\BBCP.
\newblock \JBOQ 日本語から手話への機械翻訳のための手話表記法の試み\JBCQ\
\newblock \Jem{手話コミュニケーション研究}, {\Bbf \rule{0pt}{1pt}}  (57),
  \mbox{\BPGS\ 31--37}.

\bibitem[\protect\BCAY{松本\JBA 原田\JBA 原\JBA 池田}{松本\Jetal
  }{2006}]{Matsumoto2006}
松本忠博\JBA 原田大樹\JBA 原大介\JBA 池田尚志 \BBOP 2006\BBCP.
\newblock \JBOQ 日本語を援用した日本手話表記法の試み\JBCQ\
\newblock \Jem{自然言語処理}, {\Bbf 13}  (3), \mbox{\BPGS\ 177--200}.

\bibitem[\protect\BCAY{宮下\JBA 高橋\JBA 森本\JBA 黒川}{宮下\Jetal
  }{2004}]{Miyashita2004}
宮下純一\JBA 高橋正己\JBA 森本一成\JBA 黒川隆夫 \BBOP 2004\BBCP.
\newblock \JBOQ 日本語手話翻訳における重文,複文,埋め込み文の言語処理\JBCQ\
\newblock \Jem{ヒューマンインタフェースシンポジウム2004論文集}, \mbox{\BPGS\
  33--38}.

\bibitem[\protect\BCAY{Nakamura}{Nakamura}{2002}]{Nakamura2002}
Nakamura, K.
\newblock \BBOQ About Japanese Sign Language\BBCQ,
  \Turl{http://www.deaflibrary.org/\allowbreak jsl.html}.

\bibitem[\protect\BCAY{岡}{岡}{2005}]{Oka2005}
岡典栄 \BBOP 2005\BBCP.
\newblock \JBOQ 日本手話の動詞における数の一致\JBCQ\
\newblock \Jem{手話コミュニケーション研究}, {\Bbf \rule{0pt}{1pt}}  (57),
  \mbox{\BPGS\ 18--26}.

\bibitem[\protect\BCAY{Prillwitz et~al.}{Prillwitzet~al.
  }{2004}]{Prillwitz2004}
Prillwitz, S. et.~al.
\newblock \BBOQ Sign Language Notation System\BBCQ,
  \Turl{http://www.sign-lang.unihamburg.de\allowbreak /Projects/HamNoSys.html}.

\bibitem[\protect\BCAY{Sandler \BBA\ Lillo-Martin}{Sandler \BBA\
  Lillo-Martin}{2006}]{Sandler2006}
Sandler, W.\BBACOMMA\ \BBA\ Lillo-Martin, D. \BBOP 2006\BBCP.
\newblock {\Bem Sign Language and Linguistic Universals}.
\newblock Cambridge University Press.

\bibitem[\protect\BCAY{謝\JBA 今井\JBA 池田}{謝\Jetal }{2004}]{Shie2004}
謝軍\JBA 今井啓允\JBA 池田尚志 \BBOP 2004\BBCP.
\newblock \JBOQ 日中機械翻訳システムjaw/Chineseにおける変換・生成の方式\JBCQ\
\newblock \Jem{自然言語処理}, {\Bbf 11}  (1), \mbox{\BPGS\ 43--80}.

\bibitem[\protect\BCAY{白澤}{白澤}{2002}]{Shirasawa2002}
白澤麻弓 \BBOP 2002\BBCP.
\newblock \JBOQ 手話通訳に対する期待の内容に関する研究\JBCQ\
\newblock \Jem{日本手話学会第28回大会論文集}.

\bibitem[\protect\BCAY{SignFactory}{SignFactory}{1997}]{SignFactory1997}
SignFactory \BBOP 1997\BBCP.
\newblock \Jem{手話ジャーナル初級教材 No.~1 (VHSビデオ)}.
\newblock ワールドパイオニア.

\bibitem[\protect\BCAY{SignFactory}{SignFactory}{1999}]{SignFactory1999}
SignFactory \BBOP 1999\BBCP.
\newblock \Jem{手話ジャーナル初級教材 No.~2 (VHSビデオ)}.
\newblock ワールドパイオニア.

\bibitem[\protect\BCAY{Sutton}{Sutton}{2002}]{Sutton2002}
Sutton, V. \BBOP 2002\BBCP.
\newblock {\Bem Lessons In SignWriting, Textbook \& Workbook, 3rd ed.}
\newblock The Deaf Action Committee for SignWriting.

\bibitem[\protect\BCAY{Sutton-Spence \BBA\ Woll}{Sutton-Spence \BBA\
  Woll}{1999}]{Sutton-Spence1999}
Sutton-Spence, R.\BBACOMMA\ \BBA\ Woll, B. \BBOP 1999\BBCP.
\newblock {\Bem The Linguistics of British Sign Language---An Introduction}.
\newblock Cambridge University Press.

\bibitem[\protect\BCAY{田上\JBA 森\JBA 立野}{田上\Jetal }{1979}]{Tanokami1979}
田上隆司\JBA 森明子\JBA 立野美奈子 \BBOP 1979\BBCP.
\newblock \Jem{手話の世界}.
\newblock 日本放送出版協会.

\bibitem[\protect\BCAY{徳田\JBA 奥村}{徳田\JBA 奥村}{1998}]{Tokuda1998}
徳田昌晃\JBA 奥村学 \BBOP 1998\BBCP.
\newblock \JBOQ
  日本語から手話への機械翻訳における手話単語辞書の補完方法について\JBCQ\
\newblock \Jem{情報処理学会論文誌}, {\Bbf 39}  (3), \mbox{\BPGS\ 542--550}.

\bibitem[\protect\BCAY{山田\JBA 高松\JBA 石原\JBA 水野\JBA 大口\JBA 佐藤\JBA
  松本\JBA 池田}{山田\Jetal }{2006}]{Yamada2006}
山田佳裕\JBA 高松大地\JBA 石原吉晃\JBA 水野智美\JBA 大口智也\JBA 佐藤芳秀\JBA
  松本忠博\JBA 池田尚志 \BBOP 2006\BBCP.
\newblock \JBOQ 日本語文解析システムibukiC/Sについて\JBCQ\
\newblock \Jem{言語処理学会第12回年次大会発表論文集}, \mbox{\BPGS\ 280--283}.

\bibitem[\protect\BCAY{米川}{米川}{2002}]{Yonekawa2002}
米川明彦 \BBOP 2002\BBCP.
\newblock \Jem{手話ということば—もう一つの日本の言語}.
\newblock PHP新書.

\bibitem[\protect\BCAY{米川\JBA 井崎}{米川\JBA 井崎}{2004}]{Yonekawa2004}
米川明彦\JBA 井崎哲也 \BBOP 2004\BBCP.
\newblock \Jem{NHKみんなの手話4--6月号}.
\newblock 日本放送出版協会.

\bibitem[\protect\BCAY{米川\JBA 井崎}{米川\JBA 井崎}{2005}]{Yonekawa2005}
米川明彦\JBA 井崎哲也 \BBOP 2005\BBCP.
\newblock \Jem{NHKみんなの手話4--6月号,7--9月号}.
\newblock 日本放送出版協会.

\bibitem[\protect\BCAY{米川\JBA 井崎}{米川\JBA 井崎}{2006}]{Yonekawa2006}
米川明彦\JBA 井崎哲也 \BBOP 2006\BBCP.
\newblock \Jem{NHKみんなの手話4--6月号,7--9月号}.
\newblock 日本放送出版協会.

\end{thebibliography}


\begin{biography}
  \bioauthor{松本 忠博}{
    1985年岐阜大学工学部電子工学科卒業.1987年同大学院修士課程修了.現
    在,同大学工学部応用情報学科助教.自然言語処理の研究に従事.言語処
    理学会,情報処理学会,電子情報通信学会,日本ソフトウェア科学会,日
    本手話学会,ヒューマンインタフェース学会各会員.
  }

  \bioauthor{池田 尚志}{
    1968年東京大学教養学部基礎科学科卒業.同年工業技術院電子技術総合研
    究所入所.制御部情報制御研究室,知能情報部自然言語研究室に所
    属.1991年岐阜大学工学部電子情報工学科教授.現在,同応用情報学科教
    授.工博.自然言語処理の研究に従事.情報処理学会,電子情報通信学会,
    人工知能学会,言語処理学会,各会員.
  }
\end{biography}


\biodate

\clearpage



































\clearpage











\end{document}
