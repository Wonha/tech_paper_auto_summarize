    \documentclass[japanese]{jnlp_1.4}
\usepackage{jnlpbbl_1.1}
\usepackage[dvips]{graphicx}
\usepackage{amsmath}
\usepackage{hangcaption_jnlp}
\usepackage{udline}
\setulminsep{1.2ex}{0.2ex}
\let\underline



\Volume{15}
\Number{5}
\Month{October}
\Year{2008}
\received{2007}{10}{31}
\revised{2008}{2}{21}
\rerevised{2008}{5}{12}
\accepted{2008}{7}{24}

\setcounter{page}{23}


\jtitle{中間言語を用いたインドネシア語—日本語対訳辞書の拡充}
\jauthor{土屋 雅稔\affiref{tutimc} \and 脇田 敏行\affiref{tutics} \and Ayu Purwarianti\affiref{tutics} \and 中川 聖一\affiref{tutics}}
\jabstract{
  2つの言語に関わる言語横断の言語処理を実現するには,その言語対を対象とす
  る豊富な言語資源が必要である.対訳辞書は,そのような言語資源の中でも特
  に重要であるが,あらゆる言語対に対して大規模な対訳辞書が利用できるわけ
  ではなく,小規模な対訳辞書しか利用できないような言語対も多い.
  そこで本論文では,ある言語対についての既存の小規模な対訳辞書を,その言
  語対と中間言語の言語資源を利用して大規模な対訳辞書に拡充する方法を提案
  する.
  提案法では,対象となる2つの言語のコーパスから得られた言語の異なる共起ベ
  クトルを,種辞書に基づいて比較して,対象となる2つの言語と中間言語の2種
  類の対訳辞書を用いて得られた訳語候補を選択する情報として用いる.
  実際に,小規模なインドネシア語-日本語辞書を,大規模なインドネシア語-英
  語辞書と英語—日本語辞書に基づいて拡充する実験を行い,拡充された辞書が言
  語横断情報検索の精度を向上させるのに役立つことを示した.
}
\jkeywords{訳語獲得}

\etitle{Expanding Indonesian-Japanese Translation Dictionary Using Pivot Language}
\eauthor{Masatoshi Tsuchiya\affiref{tutimc} \and Toshiyuki Wakita\affiref{tutics} \and Ayu Purwarianti\affiref{tutics} \and \\
	Seiichi Nakagawa\affiref{tutics}} 
\eabstract{
  Cross-lingual language resources are necessary to realize
  cross-lingual natural language processing.  A large translation
  dictionary is especially important as such a resource, however, large
  dictionaries are available for few language pairs and small ones are
  only available for most language pairs.
  We propose a novel method to expand a small existing translation
  dictionary to a large translation dictionary using a pivot language.
  Cooccurrence vectors in the source language and ones in the
  destination language are compared based on the small existing
  translation dictionary, and provide information to select appropriate
  translations among translation candidates gotten from transitive
  translation using two translation dictionaries.
  Experiments that expand the Indonesian-Japanese dictionary using the
  English language as a pivot language show that the proposed method
  can improve performance of a real CLIR system.
}
\ekeywords{Acquisition of Translation Words}

\headauthor{土屋,脇田,Purwarianti, 中川}
\headtitle{中間言語を用いたインドネシア語—日本語対訳辞書の拡充}

\affilabel{tutimc}{豊橋技術科学大学情報メディア基盤センター}{
	Information and Media Center, Toyohashi University of Technology}
\affilabel{tutics}{豊橋技術科学大学情報工学系}{
	Department of Information and Computer Sciences, Toyohashi University of Technology}

\usepackage{url}
\newcommand{\tabref}[1]{}
\newcommand{\figref}[1]{}
\newcommand{\eqnref}[1]{}


\begin{document}
\maketitle




\section{はじめに}
\label{sec:intro}

言語横断情報検索や言語横断質問応答,機械翻訳などの2つの言語に関わる処理
を実現するには,その言語対に対する大規模対訳辞書などの言語横断言語資源が
必要である.情報流通技術の発達に伴って,様々な言語で記述された情報を活用
することが可能となりつつあり,複数言語を対象とする自然言語処理技術はます
ます重要な課題となることが予想される.しかし,世界には数多くの言語が存在
するため,あらゆる言語対を対象として豊富な言語資源を整備することは,非現
実的である.
現実には,需要の大きい一部の言語対については大規模な言語資源が利用できる
が,それ以外の多くの言語対については,小規模な対訳辞書しか利用できない場
合が多い.

もし,新規の言語対に対して対訳辞書を自動的に構築することができれば,この
ような状況を改善するのに非常に役立つと考えられるが,広く知られている通り,
完全に自動的に高精度の対訳辞書を構築することはかなり困難である.
そのため,本論文では,新規対訳辞書の自動構築というタスクに代わって,既存の
小規模な対訳辞書を拡充するというタスクに着目する.
まず,入力言語,中間言語,出力言語という3つの言語を考えた時,入力言語から
出力言語への小規模な対訳辞書(以後,{\bfseries 種辞書}と呼ぶ)と,入力言語
から中間言語への大規模な辞書および,中間言語から出力言語への大規模な辞書
という3つの辞書が利用できる状況を考える.
この時,種辞書を拡充するというタスクは,以下の2つの条件を満たす語の訳語を
推定するというタスクとして定義される.第1に,その語は,種辞書には登録され
ていない未知語である.第2に,入力言語から中間言語への対訳辞書と中間言語か
ら出力言語への対訳辞書の両方を参照することにより,その語の出力言語上での
訳語候補が得られる.

タスクの設定から明らかに,種辞書の拡充というタスクは,2つの仮定に依存して
いる.まず,(a)小規模な種辞書が存在しなければならず,次に,(b)先に述べた
条件を満たす適切な中間言語が存在しなければならない.
最初の仮定(a)から,完全に新規の言語対に対しては,このタスク設定は適用でき
ないという制限が発生する.しかし,最近のネットワークとコンピュータの発達
にともない,そのような完全に新規の言語対は少なくなりつつあり,非常に小規
模な対訳辞書でも良ければ,多くの言語対について対訳辞書が利用できるように
なってきている.
また,英語を中間言語として考えると,多くの言語対について,後の仮定(b)が成
り立つことは経験的に知られている.
したがって,種辞書の拡充というタスクは,対訳辞書の自動構築よりも多くの仮
定に依存していることは事実であるが,この仮定は多くの場合に問題にならない
と考えられ,かつ,これらの仮定を導入することによって利用可能となる知識を
用いれば,より簡単に訳語推定が可能になると期待される.

種辞書の拡充というタスクは,新規対訳辞書の自動構築や,既存辞書に登録され
ていない新規な未知語に対する訳語の推定といった関連研究とは,2つの点で異なっ
ている.
\cite{日仏対訳辞書}は,英語を中間言語として利用し,和英辞書と英和辞書およ
び英仏辞書と仏英辞書という4種類の辞書を利用して,新規の和仏対訳辞書を作成
する方法を提案している.このような新規対訳辞書の自動構築というタスクでは,
対象とする言語対についてはまったく対訳辞書が存在しない状況を想定しており,
入力言語—出力言語の対訳辞書から得られる情報を考慮することは行われていない.
それに対して,本論文で提案する辞書拡充というタスクは,小規模な種辞書から
得られる情報をなるべく有効に利用しようとしている点で,先行研究とは異なる.
\cite{ウェブから対訳を推定}は,既存の対訳辞書に登録されていない新規な未知
語を対象として,大規模なコンパラブルコーパスなどを用いて訳語の推定を行っ
ている.このような研究は,既存の対訳辞書から得られる情報を用いているとい
う点では,辞書拡充というタスクと類似している.しかし,このような新規な未
知語の多くは名詞であるため,多くの先行研究では,未知の名詞の訳語推定に特
化した検討がされている.それに対して,非常に小規模な種辞書の拡充を行うに
は,名詞のみの訳語推定では不十分であり,動詞・形容詞などについても訳語の
推定を行う必要が生じる.この問題については,\ref{subsec:辞書の分析}節で再
び議論する.

以下,\ref{sec:expansion}節では,中間言語を用いて対訳辞書を拡充する方法を
提案する.\ref{sec:experiment}節では,入力言語をインドネシア語,中間言語
を英語,出力言語を日本語として,対訳辞書の拡充を行った実験について報告す
る.特に,拡充された辞書を,実際の言語横断情報検索システムに組み込んで,
評価した結果について報告する.\ref{sec:related_works}節では関連研究につい
て述べ,最後に結論を述べる.


\section{中間言語を用いた対訳辞書の拡充}
\label{sec:expansion}

\subsection{対訳辞書の拡充}


ある入力言語から,ある出力言語への対訳辞書を作成するとき,以下のような状
況を仮定する.
\begin{quote}
  \textbf{仮定:}ある中間言語を考えると,入力言語から出力言語への小規
  模な対訳辞書(以下,\textbf{種辞書}と呼ぶ)と,入力言語から中間言語へ
  の大規模な対訳辞書および中間言語から出力言語への大規模な対訳辞書が存在
  する.
\end{quote}
このような仮定は,英語を中間言語とすると,かなり多くの言語対に対して成り
立つことが期待できる.

本論文では,上述の仮定の下で,入力言語から中間言語への大規模対訳辞書には
登録されているが,種辞書には登録されていない語の訳語を推定することによっ
て,種辞書を拡充するというタスク(対訳辞書の拡充)を扱う.


\subsection{中間言語と共起ベクトルを用いた拡充方法}
\label{subsec:提案手法}

本論文で提案する拡充方法は,以下の2段階からなる.
\begin{enumerate}
  \item 入力言語のコーパスを用いて,翻訳したい単語と,種辞書に登録されて
	いる見出し語の単語共起ベクトルを作成する.次に,その単語共起ベク
	トルを,種辞書を用いて,出力言語上のベクトルに変換する.
  \item 入力言語から中間言語への対訳辞書と,中間言語から出力言語への対訳
	辞書を利用して,訳語候補を列挙する.出力言語のコーパスを用いて,
	それぞれの訳語候補について単語共起ベクトルを作成し,前段階で得ら
	れたベクトルとの類似度に基づいて,訳語を決定する.
\end{enumerate}
提案手法の概略を,\figref{fig:提案手法}に示す.

\begin{figure}[b]
  \begin{center}
   \includegraphics{15-5ia2f1.eps}
    \caption{提案手法}
    \label{fig:提案手法}
  \end{center}
\end{figure}

最初に,入力言語上の単語共起ベクトルを,種辞書を用いて,出力言語上のベク
トルに変換する.言語を問わず,コーパス上における単語$w_{i},w_{j}$の共起
頻度は$f(w_{i},w_{j})$と表す.
種辞書$D$の全見出し語を$x_{i}(i=1,2,\ldots,n)$とすると,
入力言語の単語$x_{s}$の共起ベクトル$\bf{v}(x_{s})$は,次式のように表
される.
\begin{equation}
  \bf{v}(x_{s}) = (f(x_{s},x_{1}),\ldots,f(x_{s},x_{n}))
    \label{eq:src_vector}
\end{equation}
つまり,この共起ベクトル$\bf{v}(x_{s})$の各次元は,入力言語の単語と対応し
ている.この共起ベクトル$\bf{v}(x_{s})$を,種辞書を用いて,各次元要素が出
力言語の単語と対応するようなベクトル$\bf{v}_{t}(x_{s})$に変換する.
\begin{equation}
  \bf{v}_{t}(x_{s}) = (f_{t}(x_{s},z_{1}),\ldots,f_{t}(x_{s},z_{m}))
    \label{eq:trans_vector}
\end{equation}
ここで,$z_{j}(j=1,2,\ldots,m)$は,種辞書に現れる全ての訳語である.また,
$f_{t}(x_{s},z_{k})$は,単語$x_{s}$に関する入力言語コーパス上の共起頻度
を,出力言語の単語$z_{j}$との共起の程度を示すように変換する関数であり,
次のように定義する.
\begin{equation}
  f_{t}(x_{s},z_{j}) = \sum_{i=1}^{n}f(x_{s},x_{i})\cdot\delta(x_{i},z_{j})
\end{equation}
ここで,$\delta(x_{i},z_{j})$は,単語$z_{j}$が単語$x_{i}$の訳語であるか
どうかを示す関数であり,単語$x_{i}$を種辞書で調べたときに得られる訳語集
合を$D(x_{i})$とすると,次式によって表される.
\begin{equation}
  \delta(x_{i},z_{j}) = \begin{cases}
			   1 & \mbox{if}\ z_{j}\in D(x_{i})\\
			   0 & \mbox{otherwise}
			 \end{cases}
\end{equation}

次に,訳語候補を,以下の手順で列挙する.最初に,入力言語の単語$x_{s}$につ
いて,入力言語から中間言語への対訳辞書を検索して,
中間言語上の訳語集合$\bf{Y}_{s}$を得る.
続いて,得られた訳語$\bf{y}_{s}\in\bf{Y}_{s}$を用いて,中間言語から出力言
語への対訳辞書を検索し,出力言語上の訳語候補集合$\bf{Z}_{s}$を得る.
ただし,訳語$\bf{y}_{s}$は,一般に,複数の語からなることがあり得る.通常
の対訳辞書において,入力言語の単語$x_{s}$に相当する単語が中間言語上に存在
する場合には,その相当する単語を訳語として使うことができるが,相当する単
語が中間言語上に存在しない場合には,複数の語からなる句または説明文の形の
訳語が使われる.そこで,訳語$\bf{y}_{s}$が単語1語からなっていた場合は,そ
のまま中間言語から出力言語への対訳辞書を検索し,訳語$\bf{y}_{s}$が複数の
語からなっていた場合は,訳語$\bf{y}_{s}$を構成する語全てを1つずつ用いて中
間言語から出力言語への対訳辞書を検索し,得られた全ての訳語候補集合の和集
合を訳語$\bf{y}_{s}$の訳語候補集合とする.
訳語候補$\bf{z}_{s}\in \bf{Z}_{s}$についても同様に,複数の語からなること
があり得る.ある訳語候補$\bf{z}_{s}$が
単語列$z_{s}^{1}z_{s}^{2}\cdots{}z_{s}^{l}$である時,
訳語候補$\bf{z}_{s}$の共起ベクトルを構成語全てを用いて次式のように求める.
\begin{equation}
  \bf{u}(\bf{z}_{s}) = \left(\sum_{k=1}^{l}f(z_{s}^{k},z_{1})\
				,\ldots,\ \sum_{k=1}^{l}f(z_{s}^{k},z_{m})\right)
    \label{eq:dst_vector}
\end{equation}
ベクトル$\bf{v}_{t}(x_{s})$とベクトル$\bf{u}(\bf{z}_{s})$のcosine類似度
$s(\bf{v}_{t}(x_{s}),\bf{u}(\bf{z}_{s}))$を計算し,適当な条件を満たした訳
語候補$\bf{z}_{s}$を,単語$x_{s}$の訳語として出力する.本論文では,条件と
して,(1)類似度の大きい訳語候補から順に出力する,(2)類似度が適当な閾値よ
り大きい訳語候補を出力する,という2通りの方法を考える.この評価については,
\ref{subsec:閾値の比較}節で述べる.

\paragraph{例}\

入力言語としてインドネシア語,出力言語として日本語,中間言語として
英語を用いた場合,提案手法による対訳辞書の拡充が,どのようにして行われる
かを具体例を用いて示す.
拡充対象となる語は,インドネシア語—日本語辞書(種辞書)には登録されていない
が,インドネシア語—英語辞書には登録されている語である\footnote{実際には,
正解訳語の判定を安定して行うために,元々の辞書に登録されている語を一部取
り除いた辞書を種辞書,取り除いた語をテスト単語として実験を行った.詳細に
ついては,\ref{subsec:condition}節を参照.}.そのような語 `peradaban' を
インドネシア語—英語辞書を用いて翻訳すると,2つの英訳語 `civilization', 
`culture' が得られる.次に,この2つの英訳語を,英語—日本語辞書を用いて翻
訳すると,13通りの訳語候補が得られる.
\begin{quote}
  文明,
  文明人,
  文化,
  文化生活,
  教化,
  開化,
  教養,
  都会,
  人口密集地,
  養成する,
  培養する,
  培養,
  菌株
\end{quote}

これらの訳語候補に対して\eqnref{eq:dst_vector}によって求めたベクトル
$u(\mbox{文明}),u(\mbox{文明人}),\ldots,u(\mbox{菌株})$と,
\eqnref{eq:trans_vector}によって求めたベクトル
$v_{t}(\mbox{`peradaban'})$とのcosine類似度と,それぞれの訳語候補が実際に
訳語として正しいかどうかを人手で判定した結果を,\tabref{tbl:example}に示
す.
この場合,13通りの訳語候補から類似度順に上位3個の訳語候補を訳語として出力
すると,出力された3個の訳語中で正しい訳語は1個だけであるから,精度は33\%
となり,正しい2個の訳語中で出力された訳語は1個だけであるから,再現率は
50\%となる.類似度が0.2より大きい訳語候補を選択した場合には,精度は20\%,
再現率は100\%である.
この手順を,対象となる全ての語に対して行うと,インドネシア語—英語辞書と英
語—日本語辞書によって訳語候補が見つかる全ての語を含む拡充された辞書が得ら
れる.

\begin{table}[t]
  \caption{インドネシア語単語 `peradaban' に対する訳語候補の例}
  \label{tbl:example}
  \begin{center}
\input{02table01.txt}
  \end{center}
\end{table}


\subsection{共起ベクトルの補正}

共起ベクトルを求めるとき,単純な共起頻度$f(w_{i},w_{j})$を用いる代わりに,
適当な補正を加える方法が有効である可能性がある.その方法として,本論文で
は,共起頻度を補正する方法と,Latent Semantic Analysis (LSA) に基づいてベ
クトルを変換する方法の2通りを検討する.

まず,情報検索においてしばしば用いられる$TF\cdot IDF$の考え方を応用して,
以下の2通りの補正された頻度を,単純な共起頻度$f(w_{i},w_{j})$の代わりに用
いる方法を比較する.
\begin{align}
  f_{\rm IDF}(w_{i},w_{j}) &= \frac{f(w_{i},w_{j})}{df(w_{j})} \label{eq:IDF}\\
  f_{\rm TFIDF}(w_{i},w_{j}) &= \frac{f(w_{i},w_{j})\cdot{}tf(w_{j})}{df(w_{j})} \label{eq:TFIDF}
\end{align}
ここで,$tf(w)$は,ある単語$w$の単語出現頻度であり,$df(w)$は,ある単語
$w$の文書出現頻度である.

LSA に基づく方法では,まず,日本語コーパス上での単語—文書共起行列$A$を求
める.この行列$A$の$i$行$j$列の要素は,単語$w_{i}$ の文書$d_{j}$中におけ
る頻度である.この時,行列$A$ の行数は,種辞書に出現する語数$m$に等しく,
列数は,コーパスに含まれる文書数$d$に等しい.
このような行列$A$は,次式のように,3つの行列$U,D,V$に特異値分解することが
できる.
\begin{equation}
  A_{m\times d}=U_{m\times r}D_{r\times r}V^{\rm T}_{d\times r}
\end{equation}
ただし,$r$は行列$A$の階数である.
この時,適当な小さい階数$r'$(ただし,$r'< r$)を選ぶと,階数$r'$における行
列$A$の最適近似は次式によって表される.
\begin{equation}
  U'_{m\times r'}D'_{r'\times r'}V'^{\rm T}_{d\times r'}
\end{equation}
ただし,$U',D',V'$の各要素は,それぞれ,行列$U,D,V$の対応する要素と等し
い.
ここで,左特異行列$U'$の各行は,その行に対応する語が,単語—文書という共起
の観点から見て,他の語とどのように類似しているかを表すベクトルと考えるこ
とができる.
このようにして得られた語の類似性を表すベクトルを用いて,
\eqnref{eq:trans_vector}で求められたベクトル$v_{t}(x_{s})$と,
\eqnref{eq:dst_vector}で求められたベクトル$u(z_{s})$を,以下のように補正
する.
\begin{align}
  v_{\mathrm{LSA}}(x_{s}) &= v_{t}(x_{s})\ U' \label{eq:LSA1} \\
  u_{\mathrm{LSA}}(z_{s}) &= u(z_{s})\ U' \label{eq:LSA2}
\end{align}
このように補正することにより,類似語と共起している場合の訳語選択を,より
適切に行えるようになる可能性がある.

これらの補正方法の評価については,\ref{subsec:補正の比較}節で述べる.


\section{評価実験}
\label{sec:experiment}

入力言語をインドネシア語,中間言語を英語,出力言語を日本語として,対訳辞
書の拡充を行った実験について述べる.

\subsection{実験条件}
\label{subsec:condition}

本論文では,日本語コーパスとして,毎日新聞CD-ROM(1993年$\sim$1995年)を,
形態素解析器MeCab~\cite{mecab}で形態素解析したデータを用いた.共起ベクトル
を用いて単語間の意味的な類似度を測定するには,なるべく類似したドメインに
対するコーパスの方が良い結果が得られると予想される.しかし,インドネシア
語に対する既存の言語資源は大変少ないため,インドネシア国内向けに編集・公
開されているウェブ新聞\footnote{\url{http://www.kompas.com/},
\url{http://www.tempointeraktif.com/}}の記事を,インドネシア語コーパスと
して用いた.各コーパスの諸元は\tabref{tbl:コーパスの諸元}の通りである.

\begin{table}[b]
  \caption{コーパスの諸元}
  \label{tbl:コーパスの諸元}
  \begin{center}
\input{02table02.txt}
  \end{center}
\end{table}
\begin{table}[b]
  \caption{語彙サイズと品詞分布}
  \label{tbl:語彙サイズと品詞分布}
  \begin{center}
\input{02table03.txt}
  \end{center}
\end{table}


インドネシア語から日本語への対訳辞書としては\cite{IEDIC}を,インドネシア
語から英語への対訳辞書としては\cite{IEDIC}を,英和辞書としては英辞郎
\cite{EJDIC}を用いた.各辞書の語彙サイズを\tabref{tbl:語彙サイズと品詞分
布}に示す\footnote{\tabref{tbl:語彙サイズと品詞分布}において,「その他」
は,品詞情報が付与されていない見出し語全て(複数語からなる慣用句など)を含
む.}.

対訳辞書の拡充手法を正確に評価するには,インドネシア語から英語への対訳辞
書に収録されているが,種辞書には収録されていない語,つまり実際の拡充対象
となる語を対象として,各種評価を行う必要がある.しかし,そのような語につ
いては正解訳語のリストが存在せず,安定した評価が難しい.そのため,本論文
では,インドネシア語から日本語への対訳辞書から,500個の訳語対をテスト用と
して取り出し,残りの訳語対のみを登録した辞書を種辞書として実験を行う.こ
の時,インドネシア語コーパスを用いて,拡充対象となる語の頻度分布を調査し,
得られた頻度分布と,テスト用訳語対のインドネシア語単語の頻度分布が概ね等
しくなるように,テスト用訳語対を選択した.結果を\figref{fig:テスト単語}に
示す.
また,選択されたテスト単語(500語)に対する訳語候補数を\tabref{tbl:テスト単
語対の諸元}に示す.インドネシア語1語に対して英語訳語候補は平均1.73個,日
本語訳語候補は平均3.03個存在する.

\begin{figure}[b]
  \begin{center}
   \includegraphics{15-5ia2f2.eps}
  \end{center}
  \caption{インドネシア語—日本語の訳語対のインドネシア語コーパス中の頻度分布}
  \label{fig:テスト単語}
\end{figure}
\begin{table}[b]
  \caption{テスト単語対の諸元}
  \label{tbl:テスト単語対の諸元}
  \begin{center}
\input{02table04.txt}
  \end{center}
\end{table}

評価尺度としては,次式によって定義される精度,再現率,$F_{\beta=1}$値,訳
語含有率を用いた.
{\allowdisplaybreaks
\begin{align*}
  \mbox{精度} &= \frac{a}{b} \\
  \mbox{再現率} &= \frac{c}{d} \\
  F_{\beta=1} &= \frac{2\times\mbox{精度}\times\mbox{再現率}}{\mbox{精度}+\mbox{再現率}} \\
  \mbox{訳語含有率} &= \frac{e}{f}
\end{align*}
}
ただし,出力された候補の内,正解と判定された候補の数を$a$,出力された候補
の総数を$b$とする.また,正解の内,出力された正解の数を$c$,正解の総数を
$d$とする\footnote{正解と同義の表現が出力された場合は,人手で判定を行った.
そのとき,1つの正解に対して複数の出力が対応付けられ,$a$と$c$が等しくなら
ないことがある.}.$e$は出力された候補中に少なくとも1つの正解が含まれてい
たテスト単語の数,$f$はテスト単語の総数(500)である.


\subsection{出力する訳語の選択方法による比較}
\label{subsec:閾値の比較}

\begin{figure}[b]
  \begin{center}
   \includegraphics{15-5ia2f3.eps}
  \end{center}
  \caption{出力する訳語の選択方法による比較}
  \label{fig:閾値の比較}
\end{figure}

類似度によって整列された訳語候補リストから,どの部分を訳語候補として出力
することが適切かを検討する.上位$n$候補を取り出した場合と,類似度$s$が閾
値より大きい候補を取り出した場合の精度・再現率を\figref{fig:閾値の比較}に
示す.
\figref{fig:閾値の比較}では,ベースラインとして,3通りの訳語選択方法を想
定している.第1の方法および第2の方法は,逆引きによって訳語候補を選択する
手法\cite{日仏対訳辞書}である.逆引き用辞書としては,\cite{EJDIC}に含まれ
ている和英辞書を使い,中間言語(英語)上で一致度を求めた\footnote{この方法
は,\cite{日仏対訳辞書}では「1回逆引き法」として言及されている方法であ
る.}.提案法と同様に,一致度の上位$n$候補を取り出した場合を第1のベースラ
イン,一致度が閾値$x$より大きい候補を取り出した場合を第2のベースラインと
する.第3のベースラインは,まったく訳語選択を行わずに,インドネシア語から
英語への対訳辞書と英和辞書を検索して得られた日本語訳語候補全てを訳語とし
て選択する手法である.

\figref{fig:閾値の比較}より,提案法には,類似度と閾値を比較して訳語を選択
すると,低い閾値を用いた場合には,良い精度が得られず,高い閾値を用いた場
合には,精度は改善されるが,出力される訳語が極端に少なくなってしまう問題
があることが分かる.
つまり,提案法に対しては,適当な閾値と比較して訳語を選択する方法よりも,
上位$n$候補を選択する方法が適している.適切な$n$は,拡充した辞書を利用す
る実際の応用アプリケーションによって変化すると予想されるが,本論文では,
最も良い$F_{\beta=1}$値が得られた$n=3$を用いることにする.
上位$n$候補を訳語として選択した場合の提案法は,逆引きを用いた2通りのベー
スラインに対して,全ての評価尺度で優っている.また,提案法は,まったく訳
語選択を行わないベースラインに対して,精度および$F_{\beta=1}$値で優ってい
る.





\subsection{共起ベクトルの補正方法の比較}
\label{subsec:補正の比較}

単純な共起頻度を用いて共起ベクトルを求めた場合,\eqnref{eq:IDF}のように文
書出現頻度を用いて共起頻度を補正して共起ベクトルを求めた場合,
\eqnref{eq:TFIDF}のように単語出現頻度と文書出現頻度を用いて共起頻度を補正
して共起ベクトルを求めた場合,さらに\eqnref{eq:LSA1}と\eqnref{eq:LSA2}の
ようにLSAに基づいて共起ベクトルを補正した場合を比較した.LSAに基づく方法
では,$10\sim 500$の範囲で$r'$の適切な値を実験的に求めたところ,最も良い
結果が得られた$r'=500$を選んだ.結果を\tabref{tbl:補正の比較}に示す.表よ
り,これらの補正による効果は殆んど観察されず,単純さから,共起頻度を用い
て共起ベクトルを求める方法が良い.

\begin{table}[b]
  \caption{共起頻度の補正方法による比較}
  \label{tbl:補正の比較}
  \begin{center}
\input{02table05.txt}
  \end{center}
\end{table}

以下,このような結果が得られた理由について考察する.ある訳語候補集合から,
1つの訳語候補を訳語として選択するか否かを判定する場合,その候補と共起する
語の重要度は,その語が,一般的にどのように振る舞うかによって決まるのでは
なく,その語が,訳語候補集合に含まれる他の候補と,その候補とを区別するの
に役立つか否かによって決まると予想される.
それに対して,\eqnref{eq:IDF},\eqnref{eq:TFIDF}および\eqnref{eq:LSA1}と
\eqnref{eq:LSA2}はいずれも,語の一般的な振舞いにのみ注目して補正を行って
いるため,効果が得られなかったのではないかと考えられる.



\subsection{品詞別の比較}
\label{subsec:辞書の分析}

\tabref{tbl:語彙サイズと品詞分布}より,2つの辞書の品詞別分類に大きな差は
なく,約7,000語から約30,000語にインドネシア語の語彙が拡大するとき,名詞ば
かりが増えるのではなく,動詞・形容詞についてもほぼ均等に増加していること
が分かる.そのため,小規模な種辞書を拡充する場合には,未知語(殆んどが名
詞)に対する訳語獲得とは異なり,名詞だけではなく,動詞や形容詞についても
訳語を推定する必要がある.

\begin{figure}[b]
  \begin{center}
   \includegraphics{15-5ia2f4.eps}
  \end{center}
  \caption{品詞による比較}
  \label{fig:品詞による比較}
\end{figure}

このような性質を持つ辞書拡充タスクに対する提案法の有効性を検討するため,
テスト用の訳語対集合を,インドネシア語の品詞によって名詞・動詞・形容詞の
3つに分類し,それぞれに対して訳語推定を行った.結果を\figref{fig:品詞によ
る比較}に示す.
ベースラインは,逆引きによって訳語候補を選択する手法($n=10$)と,全く訳語
候補の選択を行わない手法の2つである.
\figref{fig:品詞による比較}より,提案法は,ベースラインと比較して品詞によ
る性能の変化が小さく,辞書拡充タスクに適していることが分かる.


\subsection{頻度別の比較}

テスト用の訳語対集合を,インドネシア語の開発用コーパス上の頻度によって分
類した場合の結果を\figref{fig:頻度の比較}に示す.頻度が精度・再現率・訳語
含有率に与えている影響は,それほど大きくない.
したがって,本来の目的である種辞書中に存在しない単語に対する訳語も,同程
度の精度で得られると期待できる.実際に,そのような単語を対象として行った
実験結果については,\ref{subsec:CLIR_experiment}節で述べる.

\begin{figure}[b]
  \begin{center}
   \includegraphics{15-5ia2f5.eps}
  \end{center}
  \caption{頻度による比較}
  \label{fig:頻度の比較}
\end{figure}


\subsection{種辞書の大きさによる比較}
\label{subsec:seed_compare}

次に,種辞書の大きさが,訳語推定に対して与える影響について検討する.種辞
書の訳語対を,インドネシア語の開発用コーパス上の頻度順によって整列し,上
位$n$対のみを残すことによって,$n$対からなる小規模な種辞書を作成した.こ
の種辞書を用いて,テスト用訳語対に対して訳語候補の上位3候補までを出力した
場合の精度の変化を\figref{fig:種辞書の大きさによる精度の変化}に示す.図よ
り,訳語推定を行うには3,000語程度の種辞書が必要であり,特に1,000語未満の種
辞書を用いると極端に推定精度が悪化することが分かる.

\begin{figure}[t]
  \begin{center}
   \includegraphics{15-5ia2f6.eps}
  \end{center}
  \caption{種辞書の大きさによる精度の変化}
  \label{fig:種辞書の大きさによる精度の変化}
\end{figure}


\subsection{言語横断情報検索における効果}
\label{subsec:CLIR_experiment}

言語横断情報検索は,
\pagebreak
大規模な対訳辞書を必要とする典型的な自然言語処理技術
の1つであり,他の言語横断なタスクに比べて,カバー率の高い対訳辞書を特に必
要とするタスクである.
上位3候補を訳語として出力する提案法は,僅かな精度の低下と引き換えに,大き
な再現率の改善を得ている(\figref{fig:閾値の比較}).よって,提案法は,言
語横断情報検索に適していると考えられるので,本節では,現実の言語横断情報
検索システムにおける提案法の効果を検討する.
具体的には,インドネシア語—日本語の言語横断情報検索タスクを対象として,既
存の小規模なインドネシア語—日本語辞書を提案法によって拡充した場合に得られ
る効果について述べる.

最初に,対象とするインドネシア語—日本語の言語横断情報検索タスクの内容に
ついて説明する.
このタスクは,日本国内の事象についてのインドネシア語の質問文を入力とし,
その質問文に対する適切な情報を含む日本語文書を出力とする.
評価用テストセットコレクションおよび評価尺度としては,NTCIR3 ウェブ情報検
索タスク\cite{eguchi_NTCIR3}において使用されたものを用いる.
ただし,このテストセットコレクションは,日本語の文書集合(約100\,GB)と,この
文書集合中に回答が含まれている日本語の質問文(47個),および,その英訳文か
らなり,インドネシア語の質問文は用意されていない.そのため,英訳された質
問文を,日本に留学中のインドネシア語を母語とする大学院生(2名)に翻訳しても
らい,インドネシア語の質問文を用意した.
NTCIR3 ウェブ情報検索タスクでは,評価尺度として Mean Average Precision
(MAP)値を用いる.ただし,出力された文書の評価にあたっては,
one-click-distance document model を採用し,同時に,正解の関連の度合いに
ついても2段階の評価を行っている\footnote{正確には3段階の評価を行っている
が,実際に検討しているのは2段階のみである.}.そのため,以下の4種類のMAP
値が評価尺度として用いられる.
\begin{itemize}
  \item RC: ハイパーリンクを考慮することなく,正解と強く関連している文書
	の精度
  \item RL: ハイパーリンクを考慮すると,正解と強く関連している文書の精度
  \item PC: ハイパーリンクを考慮することなく,正解と部分的に関連している
	文書の精度
  \item PL: ハイパーリンクを考慮すると,正解と部分的に関連している文書の
	精度
\end{itemize}

言語横断情報検索システムとしては,\cite{IJCLIR}を用いる.
このシステムの検索手順は以下の通りである.まず最初に,入力されたインドネ
シア語質問文からインドネシア語キーワードを抽出する.次に,インドネシア語
キーワードを,対訳辞書を用いて,日本語キーワードに翻訳する.最後に,日本
語キーワードを用いて,日本語文書集合を検索し,条件に合致する文書を出力す
る.この部分には,\cite{fujii03b}による情報検索システムをそのまま用いて
いる.
このように,本システムは,言語横断情報検索システムとして,対訳辞書を用い
た非常に基本的な構成を採用しており,対訳辞書による違いが検討しやすいと考
える.

次に,提案手法によって種辞書を拡充した場合の効果について検討する.質問文
中に含まれる単語の異なり数は301であり,その内,種辞書に含まれない未知語の
異なり数は106(35\%)である.それに対して,提案手法によって辞書の拡充を行い,
20,457語の見出し語からなる辞書を作成したところ,28語について訳語を得るこ
とができ,未知語の異なり数は78(26\%)まで減少した.
この28語は,テスト単語を取り除く前の元々のインドネシア語—日本語辞書にも含
まれていなかった完全な未知語である.日本に留学中のインドネシア語を母語と
する大学院生(1名)に,この28語に対して出力された訳語の正解判定を依頼したと
ころ,約52\%の精度で正しい訳語が出力されていることが分かった.この結果は,
\ref{subsec:閾値の比較}節における結果と概ね一致していることから,本論文の
提案手法は完全な未知語についても有効である.

\begin{table}[b]
\caption{言語横断情報検索における効果}
\label{tbl:IR実験結果}
\begin{center}
\begin{small}
\input{02table06.txt}
\end{small}
\end{center}
\end{table}

辞書を変更した場合の情報検索性能の変化を\tabref{tbl:IR実験結果}に示す.
ここで,手法(1)〜(3)が既存の比較手法であり,手法(4)〜(6)が提案手
法によって拡充した辞書を辞書を用いた結果である.
また,手法(3)と手法(6)は,他の手法とは異なり,\cite{IJCLIR}によって提案さ
れた方法を用いて,対訳辞書を用いて得られた訳語候補の絞り込みを行っている.
この絞り込みには,訳語候補の日本語コーパスにおける相互情報量と,その訳語
候補を用いて検索して発見された文書の信頼度が,組み合わせて用いられる.こ
の絞り込みは,検索時に同時に行わざるを得ないため,システムの処理速度の低
下が欠点である.
手法(7)は,\cite{日仏対訳辞書}の手法を,インドネシア語—英語—日本語につい
て適用して作成した辞書を用いた結果である.

\tabref{tbl:IR実験結果}では明らかに,手法(7)がもっとも性能が悪い.このよ
うに,\cite{日仏対訳辞書}の手法は,元とする辞書の品質によっては,非常に低
品質の辞書しか得られないという問題が生じることがある.

手法(1)と手法(4),手法(2)と手法(5)を比較すると,いずれの場合も,拡充した
辞書を用いることによって情報検索性能が改善している.したがって,本提案手
法を用いて拡充した辞書は,情報検索において有用と考えられる.
手法(3)と手法(6)の間には,性能の差は殆んどなく,訳語候補の絞り込みを行う
と,本提案手法によって辞書を拡充した効果が現れなくなることが分かる.しか
し,手法(5)は,訳語候補の絞り込みを行っていないにも関わらず,手法(3)およ
び手法(6)とほぼ同等の性能を達成している.
よって,本提案手法を用いて拡充した辞書は情報検索において有用であり,同時
に,訳語候補の絞り込みという高負荷な処理を行うことなしに,訳語候補の絞り
込みを行った場合と同等の性能を達成できる.

\begin{figure}[b]
  \begin{center}
   \includegraphics{15-5ia2f7.eps}
  \end{center}
  \caption{種辞書の大きさが情報検索性能に与える影響}
  \label{fig:RL_comparison}
\end{figure}

\ref{subsec:seed_compare}節で述べた方法により,見出し語数の異なる種辞書を
複数用意し,種辞書の大きさが検索性能に与える影響を調べた.
結果を\figref{fig:RL_comparison}に示す.
手法(2), (5)はともに,種辞書に含まれている見出し語の数が増えるにつれて性能
が改善しているが,種辞書に含まれる見出し語の数が少ない場合には,手法(2)と
手法(5)の差が大きくなっている.
このように,本提案手法は,種辞書に含まれる見出し語数が少ない場合には,種
辞書をそのまま使う場合に比べて効果が大きいと考えられる.


\section{関連研究}
\label{sec:related_works}

\ref{sec:intro}節でも述べた通り,本研究と関連が深いタスクとして,2つの方
向がある.第1は,新規の言語対に対する対訳辞書を自動構築するという研究であ
り,第2は,既存の対訳辞書に登録されていない未知語に対する訳語を推定すると
いう研究である.

新規の言語対に対する対訳辞書を自動構築する研究は,さらに,大きく2つのアプ
ローチに分けることができる.
第1のアプローチは,既存の対訳辞書をまったく仮定せず,対象としている言語対
のコーパスから直接に対訳辞書を構築しようとするアプローチである.
第2のアプローチは,ある中間言語を導入して,その中間言語との間の対訳辞書を
利用することにより,新規言語対の対訳辞書を構築するというアプローチである.

第1のアプローチでは,基本的に,単語の周辺の文脈を何らかの方法で表現し,入
力言語と出力言語で類似した文脈に出現する語を訳語対とする.
例えば,\cite{fung95}は,英語と中国語を対象として,ある単語の直前と直後に
現れる単語の種類数を求め,その種類数によって単語の文脈的な特異性を表し,
良く似た特異性を備えた英語単語と中国語単語とを訳語対としてまとめるという
方法を提案している.
\cite{rapp95}は,英語とドイツ語を対象として,非常に基本的な単語の訳語対
(6個)とコンパラブルコーパスを用意しておき,これらの訳語との共起頻度に基づ
いて定義した単語間の類似度を用いて訳語対を求める方法を提案している.

新規対訳辞書を自動構築する第2のアプローチとしては,本研究と同様に,英語を
中間言語として用いる試みが幾つか報告されている.
\cite{日仏対訳辞書}は,英語を中間言語として利用して,和仏対訳辞書を作成す
る方法を提案している.この方法では,和英辞書と英仏辞書を利用して,日本語
単語に対するフランス語訳語候補を獲得し,仏英辞書と英和辞書を利用して,得
られたフランス語単語に対する日本語訳語候補を調べる(逆引きを行う)ことに
よって,訳語候補の絞り込みを行い,訳語推定精度を改善している.この方法で,
名詞を対象とした場合の精度は76\%,再現率は44\%である.
白井ら\cite{shirai01}は,田中らと同様の方法を用いて,英語を中間言語として
日本語と韓国語の対訳辞書を作成している.
Bondら\cite{bond01}も同様に,英語を中間言語として利用して,日本語とマレー
語の対訳辞書を作成している.
張ら\cite{日中対訳辞書}は,英語を中間言語として利用して,日中対訳辞書を
作成する方法を提案している.この方法では,和英辞書と英中辞書を利用して,
日本語単語に対する中国語訳語候補を獲得し,日本語と中国語の品詞情報と漢字
情報を利用して訳語候補の順位付けを行っている.この方法で,第1位に順位付
けられた訳語候補のみを出力した場合の精度は81.4\%である.
これらの先行研究は,対象となる言語対の辞書が全く存在しない状況を想定して
おり,入力言語—出力言語の対訳辞書から得られる情報を考慮することは行われて
いない.それに対して,本論文の手法では,対象となる言語対について小規模な
種辞書が存在する状況を想定しており,その種辞書から得られる情報をなるべく
有効に利用しようとしている点で,これらの先行研究とは異なる.

既存の対訳辞書には含まれていない語について訳語推定を行い,かつ,その推定
にあたっては既存の対訳辞書を最大限に利用しようするという2つの点において,
本論文で提案する種辞書の拡充というタスクと,未知語の訳語推定というタスク
は関連が深い.
例えば,\cite{tanaka02}は,対象となる言語対のコンパラブルコーパスを用意し,
周辺に共起する単語を文脈ベクトルとして表現し,文脈ベクトルの類似度を求め
て訳語を推定するという方法を提案している.ただし,推定対象は,複合名詞に
限られており,動詞や形容詞には対応していない.
\cite{kaji01}も,類似の方法を提案し,複合語と単純語の両方に対して評価を行っ
ている.ただし,\cite{kaji01}は,非常に大規模な既存の対訳辞書(50,000語)を
用いている点で,本論文とは問題設定が異なっていると考えられる.
本論文で提案している辞書の拡充というタスクにもっとも近い問題設定としては,
\cite{tanaka96,fung98,chiao02,gaussier04}がある.例えば,\cite{tanaka96}
は,英語と日本語のコンパラブルコーパスと小規模な対訳辞書を用意し,英語コー
パス上で観測された単語共起と,日本語コーパス上で観測された単語共起とを比
較して,単語共起として類似した振る舞いをしている単語を訳語として選択する
という方法を提案している.
本論文の提案手法は,入力言語から中間言語への対訳辞書と,中間言語から出力
言語への対訳辞書の情報をも利用することによって,より小さい種辞書で,より
再現率の高い訳語推定を行っている.
また,本論文では,得られた訳語を人手で判定して評価を行うだけでなく,実際
の言語横断情報検索システムに組み込んだ性能評価を行っている.


\section{むすび}
本論文では,インドネシア語—英語辞書および英語—日本語辞書を利用して訳語候
補を取り出し,インドネシア語コーパスと日本語コーパスの共起情報を用いて訳
語候補の絞り込みを行って,小規模なインドネシア語—日本語辞書を拡充する方法
を提案した.提案手法を用いて実際に辞書を拡充したところ,上位3候補を出力し
た場合には精度45.9\%,再現率60.7\%で拡充することができた.
さらに,実際のインドネシア語—日本語言語横断情報検索システムに,提案手法を
用いて拡充した辞書を組み込んで実験を行った.この実験により,提案手法によっ
て拡充された辞書は,実際の言語横断情報検索システムにとって有用であり,特
別な訳語絞り込み手法を適用することなしに,訳語絞り込み手法を適用した場合
と同等の性能を得ることができることを示した.

この提案手法は,実際の言語横断情報検索システムにおいて有効であることが示
されているが,収録されている見出し語は約20,000語とまだ少なく,より大規模
な辞書への拡充が必要と予想される.そのためには,本提案手法によって拡充さ
れた辞書とコーパスを用いて,中間言語には依存せずに辞書を拡充するブートス
トラップ的な方法の検討が必要と考えられる.


\begin{thebibliography}{}

\bibitem[\protect\BCAY{{Agency for The Assessment and Application of
  Technology}}{{Agency for The Assessment and Application of
  Technology}}{}]{IEDIC}
{Agency for The Assessment and Application of Technology} \BBOP ?\BBCP.
\newblock \BBOQ Kamus Elektornik Bahasa Indonesia\BBCQ\
\newblock \url{http://nlp.aia.bppt.go.id/kebi}.

\bibitem[\protect\BCAY{Bond, Yamazaki, Sulong, \BBA\ Ogura}{Bond
  et~al.}{2001}]{bond01}
Bond, F., Yamazaki, T., Sulong, R.~B., \BBA\ Ogura, K. \BBOP 2001\BBCP.
\newblock \BBOQ Design and {C}onstruction of a machine-tractable
  {J}apanese-{M}alay {L}exicon\BBCQ\
\newblock \Jem{言語処理学会第7回年次大会発表論文集}, \mbox{\BPGS\ 62--65}.

\bibitem[\protect\BCAY{Chiao \BBA\ Zweigenbaum}{Chiao \BBA\
  Zweigenbaum}{2002}]{chiao02}
Chiao, Y.-C.\BBACOMMA\ \BBA\ Zweigenbaum, P. \BBOP 2002\BBCP.
\newblock \BBOQ Looking for candidate translational equivalents in specialized,
  comparable corpora\BBCQ\
\newblock In {\Bem Proceedings of the 19th international conference on
  Computational linguistics}, \mbox{\BPGS\ 1--5}\ Morristown, NJ, USA.
  Association for Computational Linguistics.

\bibitem[\protect\BCAY{Eguchi, Oyama, Ishida, Kando, \BBA\ Kuriyama}{Eguchi
  et~al.}{2003}]{eguchi_NTCIR3}
Eguchi, K., Oyama, K., Ishida, E., Kando, N., \BBA\ Kuriyama, K. \BBOP
  2003\BBCP.
\newblock \BBOQ Overview of the Web Retrieval Task at the Third {NTCIR}
  Workshop\BBCQ\
\newblock In {\Bem Proceedings of the Third NTCIR Workshop on research in
  Information Retrieval, Automatic Text Summarization and Question Answering}.

\bibitem[\protect\BCAY{Fujii \BBA\ Ishikawa}{Fujii \BBA\
  Ishikawa}{2003}]{fujii03b}
Fujii, A.\BBACOMMA\ \BBA\ Ishikawa, T. \BBOP 2003\BBCP.
\newblock \BBOQ {NTCIR}-3 Cross-Language {IR} Experiments at {ULIS}\BBCQ\
\newblock In {\Bem Proceedings of the Third NTCIR Workshop}.
\newblock
  \url{http://research.nii.ac.jp/ntcir/workshop/OnlineProceedings3/NTCIR3-CLIR
-FujiiA.pdf}.

\bibitem[\protect\BCAY{Fung}{Fung}{1995}]{fung95}
Fung, P. \BBOP 1995\BBCP.
\newblock \BBOQ Compiling Bilingual Lexicon Entries from a Non-Parallel
  {E}nglish-{C}hinese Corpus\BBCQ\
\newblock In Yarovsky, D.\BBACOMMA\ \BBA\ Church, K.\BEDS, {\Bem Proceedings of
  the Third Workshop on Very Large Corpora}, \mbox{\BPGS\ 173--183}\ Somerset,
  New Jersey. Association for Computational Linguistics.

\bibitem[\protect\BCAY{Fung \BBA\ Yee}{Fung \BBA\ Yee}{1998}]{fung98}
Fung, P.\BBACOMMA\ \BBA\ Yee, L.~Y. \BBOP 1998\BBCP.
\newblock \BBOQ An IR approach for translating new words from nonparallel,
  comparable texts\BBCQ\
\newblock In {\Bem Proceedings of the 17th international conference on
  Computational linguistics}, \mbox{\BPGS\ 414--420}\ Morristown, NJ, USA.
  Association for Computational Linguistics.

\bibitem[\protect\BCAY{Gaussier, Renders, Matveeva, Goutte, \BBA\
  Dejean}{Gaussier et~al.}{2004}]{gaussier04}
Gaussier, E., Renders, J., Matveeva, I., Goutte, C., \BBA\ Dejean, H. \BBOP
  2004\BBCP.
\newblock \BBOQ A Geometric View on Bilingual Lexicon Extraction from
  Comparable Corpora\BBCQ\
\newblock In {\Bem Proceedings of the 42nd Meeting of the Association for
  Computational Linguistics (ACL'04), Main Volume}, \mbox{\BPGS\ 526--533}\
  Barcelona, Spain.

\bibitem[\protect\BCAY{Purwarianti, Tsuchiya, \BBA\ Nakagawa}{Purwarianti
  et~al.}{2007}]{IJCLIR}
Purwarianti, A., Tsuchiya, M., \BBA\ Nakagawa, S. \BBOP 2007\BBCP.
\newblock \BBOQ Indonesian-Japanese Transitive Translation using English for
  CLIR\BBCQ\
\newblock \Jem{自然言語処理}, {\Bbf 14}  (2), \mbox{\BPGS\ 95--123}.

\bibitem[\protect\BCAY{Rapp}{Rapp}{1995}]{rapp95}
Rapp, R. \BBOP 1995\BBCP.
\newblock \BBOQ Identifying Word nanslations in Non-Parallel Texts\BBCQ\
\newblock In {\Bem Proceedings of the 33rd Annual Meeting of the Association
  for Computational Linguistics}, \mbox{\BPGS\ 320--322}.

\bibitem[\protect\BCAY{Shirai \BBA\ Yamamoto}{Shirai \BBA\
  Yamamoto}{2001}]{shirai01}
Shirai, S.\BBACOMMA\ \BBA\ Yamamoto, K. \BBOP 2001\BBCP.
\newblock \BBOQ Linking English Words in Two Bilingual Dictionaries to Generate
  Another Language Pair Dictionary\BBCQ\
\newblock In {\Bem Proceedings of ICCPOL2001}, \mbox{\BPGS\ 174--179}.

\bibitem[\protect\BCAY{Tanaka \BBA\ Iwasaki}{Tanaka \BBA\
  Iwasaki}{1996}]{tanaka96}
Tanaka, K.\BBACOMMA\ \BBA\ Iwasaki, H. \BBOP 1996\BBCP.
\newblock \BBOQ Extraction of lexical translations from non-aligned
  corpora\BBCQ\
\newblock In {\Bem Proceedings of the 16th conference on Computational
  linguistics}, \mbox{\BPGS\ 580--585}\ Morristown, NJ, USA. Association for
  Computational Linguistics.

\bibitem[\protect\BCAY{Tanaka}{Tanaka}{2002}]{tanaka02}
Tanaka, T. \BBOP 2002\BBCP.
\newblock \BBOQ Measuring the similarity between compound nouns in different
  languages using non-parallel corpora\BBCQ\
\newblock In {\Bem Proceedings of the 19th international conference on
  Computational linguistics}, \mbox{\BPGS\ 1--7}\ Morristown, NJ, USA.
  Association for Computational Linguistics.

\bibitem[\protect\BCAY{梶博行\JBA 相薗敏子}{梶博行\JBA 相薗敏子}{2001}]{kaji01}
梶博行\JBA 相薗敏子 \BBOP 2001\BBCP.
\newblock \JBOQ 共起語集合の類似度に基づく対訳コーパスからの対訳語抽出\JBCQ\
\newblock \Jem{情報処理学会論文誌}, {\Bbf 42}  (9), \mbox{\BPGS\ 2248--2258}.

\bibitem[\protect\BCAY{田中\JBA 梅村\JBA 岩崎}{田中\Jetal
  }{1996}]{日仏対訳辞書}
田中久美子\JBA 梅村恭司\JBA 岩崎英哉 \BBOP 1996\BBCP.
\newblock \JBOQ 第3言語を介した対訳辞書の作成\JBCQ\
\newblock \Jem{情報処理学会論文誌}, {\Bbf 39}  (6), \mbox{\BPGS\ 1915--1924}.

\bibitem[\protect\BCAY{張\JBA 馬\JBA 井佐原}{張\Jetal }{2005}]{日中対訳辞書}
張玉潔\JBA 馬青\JBA 井佐原均 \BBOP 2005\BBCP.
\newblock \JBOQ 英語を介した日中対訳辞書の自動構築\JBCQ\
\newblock \Jem{自然言語処理}, {\Bbf 12}  (2), \mbox{\BPGS\ 63--85}.

\bibitem[\protect\BCAY{工藤}{工藤}{2006}]{mecab}
工藤拓 \BBOP 2006\BBCP.
\newblock \JBOQ 形態素解析器 {M}e{C}ab\JBCQ\
\newblock \url{http://chasen.org/~taku/software/mecab/}.

\bibitem[\protect\BCAY{道端秀樹}{道端秀樹}{2002}]{EJDIC}
道端秀樹\JED\ \BBOP 2002\BBCP.
\newblock \Jem{英辞朗}.
\newblock アルク.

\bibitem[\protect\BCAY{宇津呂\JBA 日野\JBA 堀内\JBA 中川}{宇津呂\Jetal
  }{2005}]{ウェブから対訳を推定}
宇津呂武仁\JBA 日野浩平\JBA 堀内貴司\JBA 中川聖一 \BBOP 2005\BBCP.
\newblock \JBOQ 日英関連報道記事を用いた訳語対応推定\JBCQ\
\newblock \Jem{自然言語処理}, {\Bbf 12}  (5), \mbox{\BPGS\ 43--69}.

\end{thebibliography}



\begin{biography}
\bioauthor{土屋 雅稔}{
  1998年京都大学工学部電気工学第二学科卒業.
  2004年京都大学大学院情報学研究科知能情報学専攻博士課程単位認定退学.
  博士(情報学).
  2004年豊橋技術科学大学情報処理センター助手.
  2007年より
  豊橋技術科学大学情報メディア基盤センター助教.
  自然言語処理に関する研究に従事.
}

\bioauthor{脇田 敏行}{
  2007年豊橋技術科学大学情報工学系卒業.
}

\bioauthor[:]{Ayu Purwarianti}{
  Graduated from Toyohashi University of Technology for her Dr. of
  Eng. degree in 2007. Since 2008, she has been joining Bandung
  Institute of Technology as a Research Associate in the School of
  Informatics and Electrical Engineering. Her research interest is in
  natural language processing area especially for Indonesian language.
}

\bioauthor{中川 聖一}{
  1976年京都大学大学院博士課程修了.
  同年,京都大学情報工学科助手.1980年豊橋技術科学大学情報工学系講師.
  1990年教授.1985--1986年カーネギメロン大学客員研究員.
  音声情報処理,自然言語処理,人工知能の研究に従事.
  工学博士.
  1977年電子通信学会論文賞,1988年IETE最優秀論文賞,
  2001年電子情報通信学会論文賞,各受賞.
  電子情報通信学会フェロー.
情報処理学会フェロー.
  著書「確率モデルによる音声認識」(電子情報通信学会編),
  「音声聴覚と神経回路網モデル」(共著,オーム社),
  「情報理論の基礎と応用」(近代科学社),
  「パターン情報処理」(丸善),
  「Spoken Language Systems」(編著,IOS Press)など.
}
\end{biography}

\biodate


\clearpage









































\end{document}

