


\documentstyle[epsf,jnlpbbl]{jnlp_j_b5}

\setcounter{page}{19}
\setcounter{巻数}{2}
\setcounter{号数}{4}
\受付{1994}{10}{17}
\再受付{1995}{2}{2}
\採録{1995}{2}{22}
\setcounter{年}{1995}
\setcounter{月}{10}

\newcommand{\jap}[1]{}
\newcommand{\eng}[1]{}
\newtheorem{pr}{}
\newtheorem{co}{}
\newtheorem{default}{}
\newtheorem{defi}{}

\makeatletter
\newcounter{enums}



\def\enumsentence{}

\long\def\@enumsentence[#1]#2{}

\newcounter{tempcnt}

\newcommand{\ex}[1]{}

\def\@item[#1]{}


\newcounter{enumsi}


\def\theenumsi{}


\def\@mklab#1{}
\def\enummklab#1{}
\def\enummakelabel#1{}

\def\eenumsentence{}

\long\def\@eenumsentence[#1]#2{}

\makeatother
\setcounter{secnumdepth}{2}

\title{日本語マニュアル文におけるテイル,テアル,テオク,\\テミル,テシマウの語用論}
\author{中川 裕志\affiref{YNU} \and 森 辰則\affiref{YNU}}

\headauthor{中川裕志・森辰則}
\headtitle{日本語マニュアル文におけるテイルなどの語用論}

\affilabel{YNU}{横浜国立大学工学部 電子情報工学科}
{Department of Electronics and Computer Engineering, Yokohama National University}

\jabstract{
本論文では,テイル,テアル,テシマウ,テオク,テミルといったアスペクト辞の
マニュアル文における意味を検討する.これらのアスペクト辞は,時間的なアス
ペクトを表す他に,書き手の態度などのいわゆるモダリティをも表現することが
あるので,モダリティについての解釈から,アスペクト辞の隣接する動詞句の主語
に関する制約を明らかにする.さらに,実際にマニュアル文から例文を集め,提
案する制約の正当性を検証する.このような制約は,省略された主語などの推定
に役立ち,マニュアル文からの知識抽出や機械翻訳に応用できる.}

\jkeywords{日本語,マニュアル,語用論,テイル,テアル,テシマウ,
テオク,テミル}

\etitle{Pragmatic Analysis of Aspect Morphemes \\in Manual Sentences in Japanese}
\eauthor{Hiroshi Nakagawa \affiref{YNU} \and Tatsunori Mori\affiref{YNU}} 

\eabstract{
In this paper, we analyze the pragmatic aspects of Japanese verbal
suffixes {\it teiru}, {\it tearu}, {\it tesimau}, {\it teoku} and {\it
temiru} used in manual sentences in Japnanese.  These suffixes mainly
express temporal structure of events but also express some kinds of
modality such as a writer's attitude.  By analyzing these kinds of
modality used in manual, we propose the constraints as for the possible
referents of an agent of action described by a verb phrase
subcategorized by one of these suffixes.  The proposed constraints are
proven to be valid by examining a number of manual sentences.  }

\ekeywords{Japanese, Manual, Pragmatics, {\it teiru}, {\it tearu}, {\it
tesimau}, {\it teoku}, {\it temiru}}

\begin{document}
\maketitle


\section{はじめに}
\label{intro}
日本語マニュアル文では,次のような文をしばしば見かける.
\enumsentence{\label{10}
長期間留守をするときは,必ず電源を切っておきます.}
この文では,主節の主語が省略されているが,その指示対象
はこの機械の利用者であると読める.この読みにはアスペクト辞テオク(実際は
「ておきます」)が関与している.なぜなら,主節のアスペクト辞をテオクから
テイルに変えてみると,
\enumsentence{\label{20}
長期間留守をするときは,必ず電源を切っています.}
マニュアルの文としては既に少し違和感があるが,少なくとも主節の省略された
主語は利用者とは解釈しにくくなっているからである.もう少し別の例として,
\enumsentence{\label{30}
それでもうまく動かないときは,別のドライブから立ち上げてみます.}
では,主節の省略されている主語は,その機械の利用者であ
ると読める.このように解釈できるのは,主節のアスペクト辞テミルが影響
している.仮に,「みます」を「います」や「あります」にすると,マニュアル
の文としてはおかしな文になってしまう.

これらの例文で示したように,まず第一にマニュアル文においても,主語は頻繁
に省略されていること,第二に省略された主語の指示対象が利用者なのかメーカー
なのか,対象の機械やシステムなのかは,テイル,テアルなどのアスペクト辞の
意味のうち,時間的アスペクトではないモダリティの意味に依存する度合が高い
ことが分かる.後の節で述べることを少し先取りしていうと,a.利用者,メー
カー,機械などの動作が通常,意志的になされるかどうかと,b.文に記述され
ている動作が意志性を持つかどうか,のマッチングによって,省略されている主
語が誰であるかが制約されている,というのが本論文の主な主張である.このよ
うなモダリティの意味として,意志性の他に準備性,試行性などが考えられる.
そして,意志性などとアスペクト辞の間に密接な関係があることが,主語とアス
ペクト辞の間の依存性として立ち現れてくる,という筋立てになる.なお,受身
文まで考えると,このような考え方はむしろ動作などの主体に対して適用される
ものである.そこで,以下では考察の対象を主語ではなく
\cite{仁田:日本語の格を求めて}
のいう「主(ぬし)」とする.すなわち,仁田の分類ではより広く(a)対象に変化
を与える主体,(b)知覚,認知,思考などの主体,(c)事象発生の起因的な引き起
こし手,(d)発生物,現象,(e)属性,性質の持ち主を含む.したがって,場合に
よってはカラやデでマークされることもありうる.若干,複雑になったが簡単に
言えば,能動文の場合は主語であり,受身文の場合は対応する能動文の主語にな
るものと考えられる.以下ではこれを{\dg 主}と呼ぶことにする.そして,省略
されている場合に{\dg 主}になれる可能性のあるものを考える場合には、この考え方を
基準とした.

マニュアル文の機械翻訳などの処理においては,省略された{\dg 主}の指示対象の同
定は重要な作業である.したがって,そのためには本論文で展開するような分析
が重要になる.具体的には,本論文では,マニュアル文において,省略された{\dg 主}
の指示対象とアスペクト辞の関係を分析することによって,両者の間にある語
用論的な制約を明らかにする.さて,このような制約は,省略された{\dg 主}などの
推定に役立ち,マニュアル文からの知識抽出や機械翻訳の基礎になる知見を与え
るものである.さらに,実際にマニュアル文から例文を集め,提案する制約を検
証する.なお,本論文で対象としているマニュアル文は,機械やシステムの操作
手順を記述する文で,特に if-then 型のルールや利用者がすべき,ないしは,
してはいけない動作や利用者にできる動作などを表現するような文である.した
がって,「ひとことで言ってしまえば」のような記述法についての記述はここで
は扱わない.


\section{マニュアル文に登場する人物と事物}

マニュアル文であると,参照される人物や事物に何らかの制限があるが,ここで
特に重要なのは,{\dg 主}の候補が大別して以下の4カテゴリーに限定されること
である.もちろん,特定のマニュアルにおいてはより多様な候補が考えられるが,
それらは,この4カテゴリーのいずれかに属することになる.

\begin{description}
  \item[メーカー] 製造,マニュアルの記述,販売後のサービスという行為
などを行なう. マニュアルライターを含むので,マニュアル文の書き
手であり,言語的には話し手に相当する.
  \item[システム] マニュアル文の記述対象である機械の全体または一部を表す.

  \item[利用者] システムの利用者であると同時に,マニュアル文の読み手であり,
言語的には聞き手に相当する.
  \item[メーカー・利用者以外の第三者] 
システムへの侵入者など利用者にとって不利になる人物や, 共同利用者などを
指す. 
\end{description}

次にこれらの各カテゴリーが固有に持っている特質について調べてみよう.
まず,以下では動作および特定の状態にあり続けることをまとめて{\dg 行動}と
呼ぶことにする.
さて,メーカーは次のような特質を持つ.第一に,メーカーの行動は全て意志的
に行なわれるものである.さらに,利用者の利益を目的とした行動のみを行ない,
かつその行動に誤ったもの,無意志的なものはない.また,メーカーが誰か他者
からの指導ないしは指令によって行動するのではなく,独立した意志決定によっ
て行動できる.つまり,メーカーの持つ意志性は非常に強いので,これを表すた
めに,独立性という新たな素性を加えることにする.メーカーは,製品出荷時以
前に全ての試行的行動(つまり製品の試験)は終えているはずである.つまり,利
用者が製品を使用するときにはメーカーの試行的行動はない.さらに,なんらか
の将来の事態に備えての準備的行動,例えばシステムの初期化処理など,を行な
うこともありうる.これらのモダリティの意味を素性の束の形でまとめると次の
ようになる.

\enumsentence{\label{maker}
メーカー$:$\\
$[$意志性$:+]$,$[$利用者の利益$:+]$,$[$準備性$:\pm]$,$[$試行性$:-]$,$[$独立性$: +]$}

ただし,ここで,素性値$+$は,その素性が必ずあることを示し,素性値$-$は,
その素性が決\\してないことを示す.また,素性値$\pm$は$+$,$-$のいずれの可能性も
あるとする.

次はシステムである.現状ではシステムは,無情物であり意志的な行動をしない
のが一般的である.また,準備的行動も行なわない.ただし,高度なソフトウェ
アなどの有情物とみなせるシステム,例えば,複雑かつ高級
なソフトウェアや知的システム,はこの限りでなく,意志的行動や準備的行動も
行ない得る.また,現状では知的システムといえども,システム自身が独立した
意志決定によって行動したり,試験的あるいは試行的行動を行なうところまでは
進んでいない.一方,利用者の利益という点に関して言えば,基本的には利用者
の利益を計るような行動をするが,利用者が誤った操作をしたときには利用者に
とって好ましくない行動をとることもありうる.以上,まとめると,モダリティ
の素性としては次のようになる.ただし,知的システムの場合とそうでない場合
のふたつに分けている.

\enumsentence{\label{sys}
知的ではないシステム$:$\\
$[$意志性$:-]$,$[$利用者の利益$:\pm]$,$[$準備性$:-]$,$[$試行性$:-]$,$[$独立性$:-]$}

\enumsentence{\label{intlsys}
知的システム$:$\\
$[$意志性$:\pm]$,$[$利用者の利益$:\pm]$,$[$準備性$:\pm]$,$[$試行性$:-]$,$[$独立性$:-]$}

次はシステムの利用者である.通常は意志的な行動を行なう.しかし,誤った操
作を無意識に行なうこともありえる.その場合には,無意志の行動と考え
る.また,準備的行動,さらに頻繁に試行的行動も行なうであろう.もちろん,
準備的でない行動,試行的でない行動も行なう.また,基本的には自分の利益の
ために行動するが,誤った操作によって不利益を被ることともある.また,明ら
かに利用者は,メーカーと同様に独立した意志決定をできる人物である.このよ
うにみてくると,利用者は,意志性,準備性,試行性,独立性,利用者の利益の
いずれの素性に対しても $+,-$のいずれもあり得る.結局,システムのモダリティ
の素性は次のようになる.

\enumsentence{\label{user}
利用者$:$\\
$[$意志性$:\pm]$,$[$利用者の利益$:\pm]$,$[$準備性$:\pm]$,$[$試行性$:\pm]$,$[$独立性$:\pm]$}


第3者としては,共同利用者あるいはシステムに害をなす侵入者などが考えられ
る.これらも,利用者と同様に人間であり,その意味素性は利用者と同じと考え
られる.

\enumsentence{\label{3P}
第3者$:$\\
$[$意志性$:\pm]$,$[$利用者の利益$:\pm]$,$[$準備性$:\pm]$,$[$試行性$:\pm]$,$[$独立性$:\pm]$}



ところで,ここで述べたようなモダリティを表す素性としてシステムの意志とい
うものを考えようとすると,そもそも今まで定義をきちんとせずに言語学の領域
で使用されている定義と同じ意味で使っていた「意志」の定義が問題になってく
る.例えば,計算機システムの利用者が端末からコマンドを入力して何らかの操
作をする場合を考えてみよう.利用者はコマンドと利用者が最終的に得る結果に
は関与しているが,実際のソフトウェアの内部の動きまでは意志的に操作してい
るわけではない.そこで,ソフトウェア内部の利用者が関知しない動きは利用者
の意志によるのではないということになる.これをシステム自身の意志的行動と
するのは少し強引である.そこで,行動のイニシアチブは他者にあるにせよ,他
者からは予測できない自律的な行動であることまで意志性の定義に含める.こ
の考察の結果として,意志性を以下のように定義しておく.

\begin{defi}\label{isi}
ある行動が{\dg 主}の意志的行動であるのは,(a)その行動を{\dg 主}が関知し,(b)
直接に制御でき,(c)その{\dg 主}の行動が自律的である場合である.
\end{defi}

この定義によれば,複雑なシステムの内部動作でかつ,その内部動作が外部から
は関知ないしは予測できない場合,加えて外部から見える行動が予測困難な挙
動は,システム自身の意志的行動とみなしてもよいことになろう.ただし,ここ
では単なる意志性に加え,独立した意志決定を行なうというモダリティの素性と
して,独立性を導入しているが,定義\ref{isi}によれば,このような独立性ま
では意志性に含まれていないことに注目されたい.


\section{アスペクト辞の意味と,動詞との隣接について}
\label{semcon}

前節では,マニュアルに登場する利用者,システム,メーカーなどの行動が意志
的であるかどうかについて議論した.そこで,本節では,1.節で述べたように,
意志性などの素性を介して動詞句と{\dg 主}を結びつけるための第二の
準備として,動詞句にアスペクト辞が接続した場合の意志性について検討する.

\subsection{アスペクト辞の意味}
\label{seman}

まず,ここでは従来の言語学による考察
\cite{寺村:日本語のシンタクスと意味2}
に従って,各アスペクト辞の素性について述べる.なお,独自に素性を持たず,
隣接する動詞によってテイルなどのアスペクト辞が付く動詞句全体の素性が決まる
場合には素性値は定義しない.

\noindent {\dg テイル}\\広く知られていることであるが,テイルの時間的アス
ペクトとしては継続と完了の用法がある.テイルの完了の用法では純粋に既然の
結果を示す.さらに,テイルの場合,それ自身には前節で述べたような準備性な
どのモダリティとしての意味合いはない.また,テイル自身が意志性ないしは無
意志性を示すこともない.意志性は後で述べるように隣接する動詞の意志性に依
存している.また,他の素性についても隣接する動詞にしたがって決まる.よっ
て,テイル独自には,意志性,などの素性は定義されないので,テイルのモダリ
ティの素性は,次のように空になる.

\enumsentence{\label{f-teiru}
テイル$:[\;]$}

\noindent {\dg テアル}\\時間的アスペクトとしてはテアルは完了の用法のみで
ある.テイルと異なり,完了とは言ってもテアルは,人が何かに対して働きかけ
る行動の結果の存在,つまり,誰かの意志的行動によってもたらされた現
在の状態を表すことである.この意味からテアル自体に意志性を表す力があると
言える.
\enumsentence{絵が壁にかけてある.}
のように,行動をした人物に全く注意が払わ
れない用法もあるが,その行動自体は意志的に行なわれたものに相違ない.さら
に注意すべきことは,壁に絵をかける行動は,この文だけを見る限りにおいては,
その行動をした人の独立した意志によって行なわれたことを妨げるものはない.
仮に,従属節を追加することによって,
{\dg 主}の独立した意志決定でない次の文にすると
\enumsentence{\label{tearu10}$??$上司に言われて絵が壁にかけてある.}
絵を壁にかけるという行動を上司に命令されたとは読めず,
相当受け入れにくい文になってしまう.
\footnote{テイルの場合
\enumsentence{上司に言われて絵を壁にかけている.}
と比較すると良く分かる.この文では,明らかに上司に言われた通りの行動をし
ていると読める.}
従って,独立性を$+$にしておく.その他の素性に関しては,
テアル独自には決められない.
よって,テアルのモダリティの素性は次のようになる.

\enumsentence{\label{f-tearu}
テアル$:$
$[$意志性$:+]$,$[$独立性$:+]$}

\noindent {\dg テオク}\\テオクもテアルと同様に,完了の用法である.さらに,
次の例文に見られるように,話し手の意志的な準備を表す場合がある.
\enumsentence{\label{teoku1}お客さんが来るので,掃除をしておく.}
したがって,\hspace*{-0.2mm}テ\hspace*{-0.2mm}オ\hspace*{-0.2mm}ク自身によって準備性は$+$になりうる.
\hspace*{-0.2mm}独立性については,\hspace*{-0.2mm}次の例文を見ると,
\enumsentence{上司に言われて絵を壁にかけておく.}
絵をかける行動は上司の命令であると解釈できるので,独立性$+$とはいえない.
ただし,
\enumsentence{絵を壁にかけておく.}
では,\hspace*{-0.2mm}絵をか\hspace*{-0.2mm}け\hspace*{-0.2mm}る\hspace*{-0.2mm}こ\hspace*{-0.2mm}と\hspace*{-0.2mm}を誰かに命令されたとまでは読めないから,\hspace*{-0.2mm}独立性$-$と
も確定できず,結局,隣接する動詞や文脈によることになる.
以上まとめると,テオクのモダリティの素性は次のようになる.
\enumsentence{\label{f-teoku}
テオク$:$
$[$意志性$:+]$,$[$準備性$:+]$}
\noindent {\dg テミル}\\テミルは,時間的アスペクトとしては未完了の意味だ
が,次の例文に見られるように,{\dg 主}がある行動を試行的に(もちろん意志的
に)行なうことを表す.
\enumsentence{\label{temiru1}
今度は,中華料理を作ってみよう.}
明らかに,テミル独自では準備性は表さない.また,独立性については,次の例
文によれば,
\enumsentence{上司に言われて絵を壁にかけてみる.}
上司に命令された行動を行なうように読め,「上司に言われて」の部分を
削除すると独立に意志決定したように読める.筆者の語感では,テオクよりは独
立性が顕在化しているようにも感じられる.しかし,テアルの場合の
(\ref{tearu10})のような非文性にまではいかないので,テオクと同様にテミル
自身は独立性に関与していないとするのが妥当であろう.よって,テミルの
モダリティの素性は次のようになる.
\enumsentence{\label{f-temiru}
テミル$:$
$[$意志性$:+]$,$[$試行性$:+]$}
\noindent {\dg テシマウ}\\
テシマウは基本的には完了の用法であるが,これに文脈などにより「話
し手にとって予想外の事象である」という話し手の{主観的な評価が加わる場合が
ある.この評価により,被害性や予想外性が表現される場合がある.
この話し手の評価が含まれない完了を単純完了と呼ぶことにする.また,予想外
性や被害性が意味に入っている完了を,予想外完了,被害完了などと呼ぶことに
する.以下に例を示す.
\enumsentence{宿題をやってしまえば,後はひまだ.-- 単純完了}
\enumsentence{勉強をしなかったので,試験に落ちてしまった. -- 被害完了}
被害完了の例文では「落ちる」ということに意志性はないが,だからといって
被害完了一般に意志性がないとは言い切れない.例えば,
\enumsentence{知らずに,誤ったコマンドを入力してしまった.}
では,コマンド入力そのものは意志的に行なわれたと考えられる.その他の素性
としては,明らかに準備性や独立性についての情報はテシマウ自身では与えない.
よって,テシマウのモダリティの素性は次のようになる.
\enumsentence{\label{f-tesimau}
テシマウ$:$
$[$被害性$:\pm]$}
なお,アスペクト辞の持つモダリティにまで射程を広げると,テヤル,テクレル,
テモラウ,テイク,テクル などの視点に関与する補助動詞との関連もでそうだ
が,これらは視点に関する研究\cite{久野78,大江75}のなかでその意味論が詳し
く研究されているので,ここでは対象としない.

\subsection{動詞との隣接について}

前節で述べたように,アスペクト辞の付く動詞句のモダリティは隣接する動詞に
依存する場合があることを述べた.ここでは,最も重要なモダリティである意志
性について,動詞がアスペクト辞に隣接した場合にどのようになるかについて考察
する.ただし,ここでいう隣接とは,日本語句構造文法\cite{郡司94a}に述べら
れている概念であり,アスペクト辞は用言を直前に持つという隣接素性を持つ.
換言すれば,ここでは,隣接素性の値となる用言を動詞に絞って考えるわけであ
り,そのような動詞をアスペクト辞に隣接する動詞と呼ぶ.

まず,動詞が意志性を持つとは,その動詞が{\dg 主}による意志的な行動を表して
いることであり,動詞が無意志性であるとは,{\dg 主}による意志的でない行動を
表していることである.アスペクト辞の持つ意志性については既に前節で考察し
た通りである.

前節の結果を利用して,動詞自体の意志性と,アスペクト辞が隣接して形成され
る動詞句の意志性の間の関係を調べてみる.

まず,テイルは,それ自身が意志性を持たないが,意志性の動詞(例
えば,「落す」)にも,無意志性の動詞(例えば,「落ちる」)にも付き,その動
詞句全体としての意志性は,隣接する動詞の意志性を引き継ぐ.例えば,
\enumsentence{落ちている.}
では,全体として無意志の行動ないしは結果状態を表すのに対し,
\enumsentence{落としている.}
では,全体として意志的行動が継続していることを表す.

テアルは,
\enumsentence{落してある}
のように意志性の他動詞だけを隣接する.無意志の動詞につけると,
\enumsentence{$\ast$落ちてある}
のようにおかしな文になってしまう.

テオク,テミルについては,意志性のある動詞に付くのはもちろんである
が,先に述べたテオク,テミル自体の意志性により,たとえ意志的,無意志的の
両方の用法がある動詞についても{\dg 主}の意志性が発現する.例えば,
\enumsentence{落ちておく}
\enumsentence{落ちてみる}
という文では,「わざと落ちた」という意味になるが,これはもともと,無意志
の動詞である「落ちる」にも極弱いながら意志的用法が存在し\cite{IPAL},そ
れがテオク,テミルで発現したと考えられる.真に無意志にしか解釈できない動
詞,例えば「そびえる」では,「そびえてみる」「そびえておく」は全く非文に
なる.

テシマウの場合は少し厄介である.まず,無意志性の動詞なら,テシマウをつけ
た動詞句も無意志性がある.例えば,
\enumsentence{落ちてしまう.}
ところが,次の例で示すように,意志的な動詞につくと,自分の意志とは無関係
に起こったような意味合いがでてくる.
\enumsentence{落としてしまう.}
もちろん,この場合でも意志的に「落した」という読みも可能である.よく考え
ると,「落す」という動詞自体にも,次の例のような無意志的な用法もある.
\enumsentence{しまった.財布を落した.}
したがって,結局,テシマウのついた動詞句は,動詞の元来意味していた意志性,
無意志性をそのまま引き継いでいると考えるのが妥当であろう.
まとめると表\ref{table0}になる.

\begin{table}
\caption{動詞の意志性とアスペクト辞付きの動詞句の意志性の関係}
\begin{center}
\begin{tabular}{|c||c|c|}
\hline アスペクト辞 & 動詞 & 動詞+アスペクト辞 \\\hline\hline
テイル & 意志性 &  意志性 \\& 無意志性 & 無意志性 \\\hline
テアル & 意志性 & 意志性 \\\hline
テオク & 意志性 & 意志性 \\\hline
テミル & 意志性 & 意志性 \\\hline
テシマウ & 意志性 &  意志性 \\& 無意志性 & 無意志性 \\\hline
\end{tabular}
\end{center}
\label{table0}
\end{table}

\section{マニュアル文における{\dg 主}についての制約}
\label{manual}

いよいよ,アスペクト辞の付く動詞句のモダリティの素性と,{\dg 主}になる可
能性のあるメーカー,システム,利用者など{\dg 主}の候補者とのモダリティのマッ
チングによって得られる{\dg 主}についての制約を解析する.つまり,動詞句が
絶対持たないモダリティ素性を持つ{\dg 主}候補者は,実際はその動詞句の{\dg 
主}にはなり得ない,という考え方で{\dg 主}に関する制約を求める.これは,
単一化文法における単一化操作をモダリティ素性に適用したと見なせる.これは
,{\dg 主}のゼロ代名詞照応に役立つ.
以下に個別のアスペクト辞について調べていく.

\noindent {\dg テイル}\\ (\ref{f-teiru})に示したようにテイル自身はモダリ
ティ素性はなく,意志性については表\ref{table0}から分かるように,隣接する
動詞の意志性ないしは無意志性を引き継ぐ.そこで,テイルが付く動詞句につい
ては,\hspace*{-0.3mm}意志性\hspace*{-0.2mm}$+$\hspace*{-0.2mm}の場合と\hspace*{-0.2mm}$-$\hspace*{-0.2mm}の場合に分けて論ずる.\hspace*{-0.3mm}意志性\hspace*{-0.2mm}$+$\hspace*{-0.2mm}であると,\hspace*{-0.3mm}それ
に矛盾しない素性を持つのは,
(\ref{maker})(\ref{sys})(\ref{intlsys})(\ref{user})(\ref{3P})に記述され
ている素性から見て,メーカー,知的なシステム,利用者,第3者であり,これ
らがテイルの付く動詞句の{\dg 主}になりうる.一方,意志性が$-$であると,
(\ref{maker})によりメーカーが \hspace*{-0.2mm}意志性$:+$なので,\hspace*{-0.2mm}メーカーだけが,\hspace*{-0.2mm}{\dg 主}にな
れない.\hspace*{-0.2mm}いくつか例文をあげておこう.例えば,次の例は,「行なう」が意志的
な動詞なので少なくとも{\dg 主}はシステムではない.直観では{\dg 主}が利用者と読める.

\enumsentence{
初期設定を行なっているとき,...
}
最近では複雑なソフトウェアなど知的とみなされるシステムもあるようで,その
場合はシステムにもなりうる.例えば,次に示される例がそうである.
\enumsentence{入力ファイルを処理している様子を直接示します.}この文では,
「ファイルを処理している」の部分の{\dg 主}は省略されているが,直観的には,なん
らかのソフトウェアが{\dg 主}であろう.実際,この行動は,先ほど定義\ref{isi}
で述べたかなり複雑なシステムの内部行動といえる.もっとも,現状では,この
ような例は少数派である.

\noindent {\dg テアル}\\テ\hspace*{-0.2mm}アルのモ\hspace*{-0.2mm}ダ\hspace*{-0.2mm}リ\hspace*{-0.2mm}テ\hspace*{-0.3mm}ィ\hspace*{-0.2mm}素性(\ref{f-tearu})で特徴的な
のは意志性\hspace*{-0.05mm}$+$\hspace*{-0.05mm}に加えて独立性も\hspace*{-0.05mm}$+$\hspace*{-0.05mm}な\hspace*{-0.2mm}こ\hspace*{-0.2mm}とである.\hspace*{-0.5mm}この素性と矛盾しない{\dg 主}
候補者は,各候補のモダリティ素性(\ref{maker})-(\ref{3P})によれば,メーカー,
利用者,第3者だけである.したがって,現状ではシステムはたとえ知的であっ
ても,やはり利用者のイニシアチブによって行動するので,テアルの付く動詞句
の{\dg 主}にはなれないと考えられる.例えば,
\enumsentence{ \label{sen:12}
調節ダイヤルが右端にセットしてある場合,....}
では,直観的にはセットするという行動をしたのは少なくともシステム自身とは解釈できない.

\noindent {\dg テオク}\\テオクの場合は,\hspace*{-0.2mm}意志性\hspace*{-0.05mm}$+$\hspace*{-0.05mm}に加えて準備性\hspace*{-0.05mm}$+$\hspace*{-0.05mm}である
点が特徴である.\hspace*{-0.2mm}このモダリティ素性に矛盾しない{\dg 主}候補者は,各候補のモ
ダリティ素性(\ref{maker})-(\ref{3P})によれば,メーカー,知的システム,利
用者,第3者だけである.
例えば,次の例文では直観的には省略された{\dg 主}は利用者と解釈できる.
\enumsentence{このような場合には,変更をする可能性のあるすべてのところに$\ast \ast \ast$を書いておく.}
ただし,非常に知的なシステムで,システ
ム自身が利用者のために何かを準備しておいてやるようなシステムでは,{\dg 主}
がシステムになる場合も可能であろう.
例えば,
\enumsentence{
システムは,そのファイルに更新の情報を書き込んでおくのである.
}
では,{\dg 主}がシステムであり,それが利用者に役に立つことをしてくれている
という意味合いがある.

\noindent {\dg テミル}\\テミルでは,\hspace*{-0.2mm}意志性\hspace*{-0.05mm}$+$に加えて,\hspace*{-0.2mm}試行性\hspace*{-0.05mm}$+$\hspace*{-0.05mm}が重要で
ある.\hspace*{-0.2mm}このモダリティ素性に矛盾しない{\dg 主}候補者は,各候補のモダリティ素
性(\ref{maker})-(\ref{3P})によれば,利用者,第3者だけである.繰り返しに
なるがメーカーが{\dg 主}になれないのは,マニュアルが使われるのはメーカー出
荷後であり,その時点までにメーカーは全ての試験や試行的行動を完了している
はずだからである.そして,テミルが試行的な操作の実施であるということを考
慮すると,{\dg 主}が利用者である場合には,メーカーが行動例の説明の代わりに,
メーカーから利用者へ操作法の指示を出していると考えられる.例文としては,
次のようなものがある.
\enumsentence{\label{miru10} 誤っているときは,このコマンドを次の2つの
記号の あいだに入れてみるとよい.}
\enumsentence{\label{miru11} ファイルの内容を変更して,次のように変えて
みる.} 
(\ref{miru10})では利用者が{\dg 主}である.(\ref{miru11})動詞の言い切り形だ
が,動詞の言い切りでは,命令の用法もあり,その場合だと,利用者にその行動
を行なうことを指示しているという意味もあり,この場合がそうである.

以上の考察を,アスペクト辞の付いた動詞句における{\dg 主}についての制約とし
て表\ref{table:actor}にまとめて示す.
{\small
\vspace*{-1mm}
\begin{table}[htb]
\caption{{\dg 主}についての制約}
 \begin{center}
   \begin{tabular}{|l||cc|c|c|c|c|}
     \hline
     アスペ&意志&アス&
     \multicolumn{4}{c|}{{\dg 主}}\\
     \cline{4-7}
クト辞         &性              &     ペクト           &    
     メー & シス & 利用 & 第三\\
&&& カー & テム & 者 & 者\\
     \hline

     テイル   & 無& 完了 & × &○&○&○\\   
              & 有& 完了   &○ &知&○&○\\   
              & 無& 継続 & ×&○&○&○\\
              & 有& 継続   & ○&知&○&○\\
     \hline
     テアル   & 有 & 完了 & ○&×&○&○\\       
     \hline
     テオク   & 有 & 完了 & ○&知&○&○\\  
     \hline
     テミル   & 有 & 未完了 & ×  &×&○& ○ \\   
     \hline
   \end{tabular}\\
\vspace*{1mm}
ただし,{\dg 主}に関する制約は ○ は{\dg 主}になりうること,× は{\dg 主}には
ならないことを示す.また,「知」は知的なシステムの場合のみ可能であること
を示す.\\
 \end{center}
\label{table:actor}
\end{table}
}

次にこの表\ref{table:actor}に記述されている制約が実際の例文でどの程度満
たされているかを調べる.調査は、論文末尾に記載したマニュアルから本論文で
扱っているアスペクト辞を含む合計 1495文を集めて行なった.その内訳は,テ
イルを含む文が 1015文,テアルを含む文が 13文,テオクを含む文が 147文,テ
ミルを含む文が 101文,テシマウを含む文が 219文である.ただし,一文のなか
に複数のアスペクト辞が現れた場合は別々に数えることにする.まず,テイル,
テアル,テオク,テミルを含む文について,アスペクト辞のつく動詞句の意志性
および,その動詞句の{\dg 主}についてまとめたのが表\ref{table:actodata}である.
ここで,表面的に{\dg 主}が現れている場合は,それに基づいて判断した.しかし,
省略されている場合は,文脈などを考慮して適切な{\dg 主}を補った上で判断した.

{\small
\vspace*{-1.5mm}
\begin{table}[htb]
\caption{例文における{\dg 主}}
 \begin{center}
   \begin{tabular}{|l||cc|c|c|c|c|}
     \hline
     アスペ&意志&アス&
     \multicolumn{4}{c|}{{\dg 主}}\\
     \cline{4-7}
クト辞         &性              &     ペクト           &    
     メー & シス & 利用 & 第三\\
&&& カー & テム & 者 & 者\\
     \hline
     テイル   & 無& 完了 & 0 & 303 & 18 & 0\\   
              & 有& 完了 & 149 & 57$^{\ast}$ & 143 & 62 \\   
              & 無& 継続 & 0 & 120 & 0 & 0 \\
              & 有& 継続 & 13 & 70$^{\ast}$ & 69 & 11 \\
     \hline
     テアル   & 有 & 完了 & 7 & 0 & 6 & 0 \\             
     \hline
     テオク   & 有 & 完了 &  0 & 8$^{\ast}$ & 139$\dagger$ & 0 \\  
     \hline
     テミル   & 有 & 未完了 & 0  & 0 & 101$\dagger$ & 0 \\   
     \hline
   \end{tabular}\\
\vspace{1mm}
$^{\ast},\dagger$については後述.
 \end{center}
\label{table:actodata}
\end{table}
}
この結果を詳しく分析してみる.
まず,テアルで{\dg 主}がメーカーである例は,次のような文であり,
出荷以前の初期設定を表している.
\footnote{特に断らない場合は,本論文における例文は筆者らの作例である.}
\enumsentence{コマンドの名前は,入力の簡略化のためではなく,意味を
表すために付けてあるので,長めになっている. 
}
テイルの場合,無意志的行動の{\dg 主}が利用者や第三者である例もないが,
利用者などが,無意志的のうちに継続して行なってしまう行動もありえないわけ
ではないと考えられる.実際には,そのような無意志行動は少ないということで
あろう.
メーカーのモダリティ素性(\ref{maker})によれば,メーカーに無意志的行動はな
い.実際の例文でも無意志用法の動詞の{\dg 主}がメーカーになる例は皆無である
ことが,表\ref{table:actodata}により示されている.

テミルがつく場合の表\ref{table:actodata}の$\dagger$に示した{\dg 主}が利用
者である例は,テミルについては18例は利用者独自の行動記述してあり,
残りの83例は操作を指示する次のような例である.
\enumsentence{
では,初期設定をしてみましょう.
}
また,テオクがつく場合の表\ref{table:actodata}の$\dagger$に示した{\dg 主}が利用
者である例は,次のような利用者への操作の指示の例である.
\enumsentence{
不要なファイルを削除しておきましょう.
 }
この結果からみると,言語学的考察からは予測できないことであるが,特にテオ
クは,実質的には利用者への行動指示という使われ方が標準的であることがわか
る.
次に,システムが{\dg 主}になる場合について考察する.基本的には知的でないシ
ステムは意志的行動はしない.したがって,システムが意志的行動である場合は,
そのシステムが知的なものかどうかを調べる必要がある.
なお,表3の$^{\ast}$に示した
テイルあるいはテオクがつく場合にシステムが意志的行動をする例は,
すべて UNIX や \LaTeX\ のような高度なソフトウェアシステムの場合である.
第三者が{\dg 主}になる場合は例文自体が極端に少なく,制約において
{\dg 主}となれる場合でも例が見つかっていない場合もあった.しかし,以上のよう
な分析を除いては,表\ref{table:actor}の制約は確認できたといえる.

\subsection{動詞の意志性$+,-$の判定法}

表\ref{table0}に示したように,テイルの場合,意志性があるかどうかは,隣接
する動詞によって決まるため,表\ref{table:actor}の制約を適用するには,各々
の動詞の意志性を判断する手段が要求される.本研究では,動詞の辞書として,
情報処理振興協会で開発された計算機日本語基本動詞辞書IPAL
\cite{IPAL}を利用している.IPALの辞書によると,動詞は,{\dg 主}によって意
志的に行ないうる行動を表す意志動詞と, {\dg 主}による意志的な行動を表して
はいない無意志動詞(分類1,2)に 分類され, 意志動詞については,意志動詞と
してのみ用いられる動詞(分類3b)と, 無意志動詞としても用いられるもの(3a)
に分類される. これにより動詞は四種類に分類される. そこで,分類1,2なら 
テイルの付く動詞句の素性を$[$意志性$:-]$として処理できる.また,分類
3bなら$[$意志性$:+]$として処理できる.したがって理解システムの作成にあたっ
ては,表\ref{table:actor}を直接適用できる. 問題は,かなり多数存在する分
類3aの動詞である.実際,1015例中858例がこれに該当する.この種の動詞は意志
性$+,-$の各々に共起する表現などにより意志性を定めなければならない.そこで,
集めたマニュアル文のうち,テイルの付く動詞句を含む文について,どのような
場合に意志性$+$で,どのような場合に意志性$-$になるかを調べてみた.その結
果,以下のようなことが分かった.\\

\noindent {\bf a.}他動詞用法の全く無い自動詞の場合,\hspace{-0.3mm}182例中173例(95.1\%)が意志性$-$の用法で
あった.\hspace{-0.3mm}し\\たがって,自動詞の場合は動詞句の$[$意志性$:-]$というデフォール
ト規則が設定できる.\\
\noindent {\bf b.}能動に用いらる他動詞\cite{IPAL}の受動態は44例中38例(86.4\%)が意
志性$+$であった.また,残りの6例では,ニヨッテ,ニ などの助詞によって{\dg 主}
語が明示されていた.よって,能動の動詞の受動態は意志性$+$というデフォー
ルト規則をおき,明示された{\dg 主}があるときには,それを優先させればよい.\\
\noindent {\bf c.}サ変動詞の場合,509例中60例が「表示する/されている」と
いう表現であり,この場合は{\dg 主}がシステムと解釈できるから,「表示する」
を特殊な語彙として扱うことにし,以下では残りのサ変動詞449例について考察
する.このうち,208例が受動態だが,意志性$+$が205例(98.5\%)なので,サ変
動詞の受動態は意志性$+$というデフォールト規則が成り立つ.また,能動態サ
変動詞については,サ変動詞を構成する名詞の性質によりおおよそ次のように言
える.
\begin{itemize}
\item 
サ変動詞「X する」を「Xをする」と言い替えられるものは177例中135例
(76.2\%)が意志性$+$であり,意志性$+$というデフォールト規則が成り立つ.
\item 
サ変動詞「X する」を「Xをする」と言い替えられないものは64例中56例
(87.5\%)が意志性$-$であるので,意志性$-$というデフォールト規則が成り立つ.
\end{itemize}

\noindent {\bf d.}上記a,b,c以外の他動詞の能動体では,はっ
きりした傾向がない.111例中,意志性$+$が43例,$-$が68例であり,個別の動
詞ないし文脈に依存している.\\
\noindent {\bf e.}上記以外の動詞としては,「備える」など自動詞にも他動詞
にも用いられるものが12例あったが,意志性は文脈に依存しておりデフォールト
規則を立てることはできない.\\

ここでは,以上の場合についてデフォールト規則を観察した.これ
らのデフォールト規則からはずれるものは特殊な語彙として辞書登録し,文脈や
領域知識を用いて意志性を判断することになる.しかし,上記のカテゴリーの
858例においてデフォールト規則で77.7\%の動詞の意志性を判断でき,また IPAL 
の分類から意志性が判断できる分類1,2,3bを加えると,全体としては81.1\%が
機械的に意志性を判断できる.したがって,効率的な機械的な処理の可能性がう
かがわれる.

\section{テシマウの被害性に関する考察}

(\ref{f-tesimau})で,テシマウには予想外性(話し手にとって事象が予想外であ
る)という評価から転じて被害性を表わしうるモダリティ素性があることを述べたが,
ここでより詳細に検討する.

(\ref{maker})の $[$利用者の利益$:+]$にも示したように,メーカーはシステム
の全てについて把握した上\\で,利用者の利益を意図している.このことから,メー
カーは商品を買った利用者に利益は与えるが,被害は与えないということになる.
したがって,確実に言えるのは次の制約になる.

\begin{co}\label{teiru5}
テシマウが被害性の意味,すなわち $[$利用者の利益$:-]$なる素性を持つ場合
は,{\dg 主}はメーカーにならない.
\end{co}

ただし,(\ref{f-tesimau})でも示したように,テシマウには単純な完了を表す
場合もあるので,テシマウがつくだけで,{\dg 主}がメーカーにならないとは言い
切れない.一方,システムも,それ自身は利用者の利便のためにあるものであり,
かつ出荷以前に十分検査されているのが普通であるから,予想外の行動をするこ
とは通常ありえない.ただし,利用者の誤った操作によって利用者にとっては予
想外の,そして多くの場合は利用者にとっては好ましくない行動結果をシステム
が示すことがある.この場合は{\dg 主}はシステムになる.
例文をみよう.
\enumsentence{
 決められた手順を飛ばしてしまうと,エラーが発生する恐れがあります.}
では,手順を飛ばしたのは利用者であると解釈される.主節で,利用者にとって
被害になることが記述されており,従属節はその原因の行動であり,利用者にとっ
ては被害となる行動である.
\enumsentence{ディスケットを入れ換えずにこのコマンドを実行してしまうと,
ディスクが壊れてしまう可能性がある.}
この文の従属節のテシマウも同様に利用者にとって被害となる行動を表している.
なお,主節のテシマウは{\dg 主}がシステム側の一要素であるディスクであるが,
システムは利用者の利益素性は$+,-$とも可能なのでテシマウが使われていてもよい.

これらを表の形でまとめると,表\ref{sima6}のようになる.なお,メーカー,
システム,利用者,第三者のいずれの場合も,被害感のない完了は理論的には否
定されないが,被害感のない単なる完了であればわざわざテシマウを使う意味は
なく,あまり現れないと考えられる.したがって,被害感のない完了になる場合
は少ないと考えられる.

{\small
\vspace{-1mm}
\begin{table}[htb]
\caption{テシマウの被害性と{\dg 主}についての制約}
 \begin{center}
   \begin{tabular}{|l||c|c|c|c|c|}
     \hline
     被害性&意志性&
     \multicolumn{4}{c|}{{\dg 主}}\\
     \cline{3-6}
      & &       メー & シス & 利用 & 第三\\
      & &  カー & テム & 者 & 者 \\
     \hline
     被害性の & 無 & $\triangle$&$\triangle$&$\triangle$&$\triangle$\\   
             ない完了           & 有  &$\triangle$ &$\triangle$&$\triangle$&$\triangle$\\   
     \hline
     被害性の & 無 &×&○&○&○\\   
             ある完了          & 有  &×&知&○&○\\
     \hline
   \end{tabular}\\
\vspace{1mm}
ただし,$\triangle$ は可能ではあるが,頻度が低いことを示す.
 \end{center}
\label{sima6}
\end{table}
}

表\ref{sima6}のシステムが意志的行動の{\dg 主}になれる,という点は注意が必
要である.すなわち,前に述べたように,非知的システムではシステムは行動
{\dg 主}にはなれないが,知的システムなら{\dg 主}になりうるわけである.これらの制
約を実際のマニュアル文について調べたのが表\ref{sima6data}である.


{\small
\begin{table}[htb]
\caption{例文における{\dg 主}}
 \begin{center}
   \begin{tabular}{|l||c|c|c|c|c|}
     \hline
     被害性&意志性&
     \multicolumn{4}{c|}{{\dg 主}}\\
     \cline{3-6}
      & &       メー & シス & 利用 & 第三\\
      & &  カー & テム & 者 & 者 \\
    \hline
     被害性の & 無 & 0 & 3 & 1 & 0\\   
     ない完了 & 有  & 0 & 2 & 11 & 0\\   
     \hline
     被害性の & 無 & 0 & 61 & 48 & 3\\   
     ある完了 & 有  & 0 & 63 & 16  & 11 \\
     \hline
   \end{tabular}
 \end{center}
\label{sima6data}
\end{table}
}

このデータは,上記の理論的考察結果によく一致しているといえよう.また,一
般に被害性のない完了ではテシマウを使う必要がないということも,このデータ
から裏付けられる.少数派である被害性のない完了としては,次のようなものが
考えられる.

\enumsentence{
この考え方の実用的な面は,一度正しいプログラムができてしまうと,応用プログラムを作るのがずっと簡単になる点である.} 


また,表\ref{sima6}についても述べたように,システムが意志性のある行動の
{\dg 主}になるのはいずれも UNIX や \LaTeX\ のような知的なシステムの場合であ
ることが確認された.

さて,被害性の有無は,主観によって判断したが,計算機でこの理論をマニュア
ル理解システムにおける{\dg 主}同定のために使用しようとすると,被害性の有無
を形式的にあるいは言語の表現から知る方法が必要である.そこで,被害性があ
ると判断された202例文について調べてみると,以下のことが分かった.\\

\noindent {\bf a.} 受身形にシマウがつく場合--- 32例\\
例えば,次の例で,これは,システムが{\dg 主}であるが,利用者の誤った操作に
より利用者に被害が及ぶ,という例である.
\enumsentence{
      絶対セクタへの書込みは,OSを通さないで
      行なわれるので,不用意に書き込むとディスクの内容が壊されてしまう
      場合がある.
}
\noindent {\bf b.}ナッテシマウ という表現の場合 --- 34例\\
\noindent {\bf c.}シマッタという表現の場合 --- 42例\\
\noindent {\bf d.}特定の動詞との共起 --- 34例\\
次のような例である.「失う」「破壊する」「占有する」「ハングする」「忘れ
る」「失敗する」「破る」「間違える」「困る」など.\\
\noindent {\bf e.}特定の副詞(節,句)との共起 --- 15例\\
次の副詞である.「誤って」「最悪の場合」「うっかり」「無意識に」「間違って」\\
\noindent {\bf f.}特定の名詞句 --- 11例\\
「ミスタイプ」など\\

なお,これらの文例については{\bf a,b,c,d,e,f}の要素が1文の中に重複している
場合もある.重複を考慮すると,これらの語彙によって156例は被害性が認識で
きる.一方,被害性のない完了では,これらの表現は一切使用されていない.ま
た,逆に被害性のない完了だけで使用される表現としては,「デキテシマウ」な
どの可能を表す表現(4例)がある.さて,これらについては,予め要注意語彙と
してシステムに登録しておき,テシマウとの共起を調べることにより処置できる
から,計算機での実現が可能であろう.ただし,これ以外にも,まったく文脈あ
るいは記述内容によってしか判断できない場合もある.例えば,
\enumsentence{
部外者がパスワードを探り当ててしまえば,
}
実際には,表\ref{sima6data}に示すように,テシマウの場合被害性がある場合
が大部分であり,計算機による処理の場合は,上記の特定の表現がなくても被害
性ありというデフォールトの判断をしておけばよいと考えられる.このような分
析により,テシマウの被害性の有無が文の表現だけから明らかになれば,表
4 の制約を適用して,メーカーを{\dg 主}の候補から除外できる.

\section{おわりに}
\label{end}マニュアル文において,アスペクト辞 テイル,テアル,テオク,テ
ミル,テシマウが表すモダリティによって,そのアスペクト辞がつく動詞句の{\dg 主}が誰であるかの制約を言語学的考察により明らかにした.次に,その制約が
実際のマニュアル文で成立していることを検証した.より強い制約やデフォール
ト規則の発見,日本語マニュアル理解システムへの本研究成果の実装などが今後
の課題である.

\vspace{0.6cm}
\hspace*{-0.5cm}
{\large\dg 統計データをとるために使用したマニュアルの出典一覧} 
\vspace{0.2cm}

\hspace*{-0.5cm} LASER SHOT {\small B406E} レーザービームプリンタ操作説明書.\ Canon,\ 1991.
\vspace{0.2cm}

\hspace*{-0.5cm} アスキー出版局編著.\ 標準MS-DOSハンドブック.\ アスキー出版局,\ 1984.
\vspace{0.3cm}

\hspace*{-0.5cm} I.Bratko著,\ 安部憲広訳.\ Prolog への入門.\ 近代科学社,\ 1990.
\vspace{0.2cm}

\hspace*{-0.5cm} 坂本文.\ たのしいUNIX.\ 株式会社アスキー,\ 1990.
\vspace{0.2cm}

\hspace*{-0.5cm} システムのネットワークと管理.\ 日本サン・マイクロシステムズ株式会社,\ 1991.
\vspace{0.2cm}

\hspace*{-0.5cm} Leslie Lamport著,\ 大野俊治他\ 訳.\ 文書処理システム\LaTeX\.\ 株式会社アスキー,\ 1990.
\vspace{0.2cm}

\hspace*{-0.5cm} トヨタマーク2取扱書.\ トヨタ自動車株式会社,\ 1988.

\section*{謝辞}
本研究において,例文収集,統計データの作成に力を尽くしてくれた本学電子情
報工学科大学院生 近藤靖司君に感謝いたします.


\bibliographystyle{jnlpbbl}
\bibliography{jpaper}

\begin{biography}
\biotitle{略歴}
\bioauthor{中川 裕志}{
1953年生.1975年東京大学工学部卒業.1980年東京大学大学院博士課程修了.工学博士.現在横浜国立大学工学部電子情報工学科教授.自然言語処理,日本語の意味論・語用論などの研究に従事.日本認知科学会,人工知能学会などの会員.}
\bioauthor{森 辰則}{
1986年横浜国立大学工学部卒業.1991年同大大学院工学研究科博士課程修了.
工学博士.1991年より横浜国立大学工学部勤務.現在,同助教授.計算言語学,
自然言語処理システムの研究に従事.情報処理学会,人工知能学会,日本認知
科学会,日本ソフトウェア科学会各会員.
}

\bioreceived{受付}
\biorevised{再受付}
\bioaccepted{採録}

\end{biography}

\end{document}
