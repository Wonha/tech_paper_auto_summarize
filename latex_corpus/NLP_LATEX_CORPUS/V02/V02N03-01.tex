\documentstyle[epsf,jnlpbbl]{jnlp_j_b5}

\setcounter{page}{3}
\setcounter{巻数}{2}
\setcounter{号数}{3}
\受付{1994}{6}{9}
\再受付{1994}{11}{21}
\採録{1995}{1}{12}
\setcounter{年}{1995}
\setcounter{月}{7}

\setcounter{secnumdepth}{2}

\unitlength=1mm

\title{言語過程説に基づく日本語品詞の体系化とその効用}
\author{宮崎正弘\hspace{-8mm}\affiref{Niigata}
 \and 白井諭\hspace{-14mm}\affiref{NTT} \and 池原悟\affiref{NTT}}

\headauthor{宮崎 正弘, 白井 諭, 池原 悟}
\headtitle{言語過程説に基づく日本語品詞の体系化とその効用}

\affilabel{Niigata}{新潟大学工学部情報工学科}
{Dept. of Information Engineering, Faculty of Engineering,
   Niigata University}

\affilabel{NTT}{NTT コミュニケーション科学研究所}
{NTT Communication Science Laboratories}

\jabstract{
  三浦文法は、時枝誠記により提唱され三浦つとむにより発展的に継承された
言語過程説に基づく日本語文法である。言語過程説によれば、言語は対象−認
識−表現の過程的構造をもち、対象のあり方が話者の認識を通して表現されて
いる。本論文では、三浦文法に基づいて体系化した日本語品詞体系および形態
素処理用の文法記述形式を提案し、日本語の形態素処理や構文解析におけるそ
の有効性を論じた。日本語の単語を、対象の種類とその捉え方に着目し、約4
00通りの階層化された品詞に分類して、きめ細かい品詞体系を作成した。本
論文で提案した品詞体系と形態素処理用文法記述形式に基づき、実際に形態素
処理用の日本語文法を構築した結果によれば、本文法記述形式により例外的な
規則も含めて文法を簡潔に記述できるだけでなく、拡張性の点でも優れている
ことが分かった。本品詞体系により、三浦の入れ子構造に基づく意味と整合性
の良い日本語構文解析が実現できるものと期待される。} 

\jkeywords{形態素処理, 構文解析, 日本語文法, 品詞, 言語過程説, 三浦文法}

\etitle{A Japanese Syntactic Category System Based on\\
 the Constructive Process Theory and its Use}

\eauthor{Masahiro Miyazaki \affiref{Niigata} \and Satoshi Shirai \affiref{NTT}
 \and Satoru Ikehara \affiref{NTT}}

\eabstract{ Miura grammar is a Japanese grammar based on the
Constructive Process Theory\\proposed \hspace*{0.7mm}by \hspace*{0.7mm}M.Tokieda, \hspace*{0.4mm}and \hspace*{0.4mm}developed \hspace*{0.4mm}by
T.Miura. \hspace*{0.7mm}In \hspace*{0.7mm}this \hspace*{0.7mm}theory, \hspace*{0.7mm}language \hspace*{0.7mm}is\\composed of three processes:
object, recognition and expression. These processes are \hspace*{0.25mm}combined \hspace*{0.25mm}by
\hspace*{0.5mm}the \hspace*{0.5mm}law \hspace*{0.5mm}of \hspace*{0.5mm}causality. \hspace*{1.5mm}The \hspace*{0.4mm}state \hspace*{0.25mm}of \hspace*{0.25mm}an \hspace*{0.25mm}object \hspace*{0.25mm}is \hspace*{0.25mm}reflected \hspace*{0.25mm}in \hspace*{0.25mm}the 
speaker's recognition, and the way the speaker recognizes, the object
gives rise to an expression.
This paper proposes a Japanese syntactic category system (part of
speech system) based on Miura grammar and formal description method of
grammar rules for morphological processing, and discusses its use in
Japanese morphological processing and syntactic analysis.
Japanese words are classified into 400 hierarchical syntactic
categories from the viewpoints of the class of the object itself and
the manner of the speaker's recognition.
The results of designing Japanese grammar rules for morphological
processing using the proposed syntactic categories system and formal
description method, show that it is easy to design and improve grammar
rules, including nongeneral rules, by the proposed method.
The proposed syntactic category system can be used to develop
Japanese syntactic analysis, using nested structure models based on
Miura grammar, without a gap between syntactic and semantic
analysis. }

\ekeywords{morphological processing, syntactic analysis, Japanese grammar, part of speech, the Constructive  Process Theory, Miura grammar}

\begin{document}
\maketitle

\section{まえがき}

言語表現には万人に共通する対象のあり方がそのまま表現されているわけでな
く, 対象のあり方が話者の認識(対象の見方, 捉え方, 話者の感情・意志・判
断などの対象に立ち向かう話者の心的状況)を通して表現されている(言語が
対象−認識−表現からなる過程的構造をもつ)ことは, 国語学者・時枝誠記に
よって提唱された言語過程説\cite{Tokieda1941,Tokieda1950}として知られて
いる. 時枝の言語過程説によれば, 言語表
現は以下のように主体的表現(辞)と客体的表現(詞)に分けられ, 文は, 辞
が詞を重層的に包み込んだ入れ子型構造(図1参照)で表される. 
\begin{itemize}
\item \underline{主体的表現}:話者の主観的な感情・要求・意志・判断などを直
接的に表現し
たものであり, 日本語では助詞・助動詞(陳述を表す零記号, すなわち図1に
示すように肯定判断を表すが, 表現としては省略された助動詞を含む)・感動
詞・接続詞・陳述副詞で表される. 
\item \underline{客体的表現}:話者が対象を概念化して捉
えた表現で, 日本語では名詞・動詞・形容詞・副詞・連体詞・接辞で表される. 
主観的な感情・意志などであっても, それが話者の対象として捉えられたもの
であれば概念化し, 客体的表現として表される.
\end{itemize}

時枝の言語過程説, およびそれに基づく日本語文法体系(時枝文法)を発展的
に継承したのが三浦つとむである. 三浦は, 時枝が指摘した主体的表現と客体
的表現の言語表現上の違いなどを継承しつつ, 時枝が言語の意味を主体的意味
作用(主体が対象を認識する仕方)として, 話者の活動そのものに求めていた
のを排し, 意味は表現自体がもっている客観的な関係(言語規範によって表現
に固定された対象と認識の関係, 詳細は2章を参照のこと)であるとした関係
意味論\footnote{対象, 表現, 話者などのような言語上の実体ではなく, それ
らの関係で意味を定義する考え方は状況意味論\cite{Barwise1983}と共通する
点がある. しかし, 状況意味論が「言語に関する社会的な約束事である言語規
範に媒介された表現の意味」と「表現の置かれた(発話された)場の意味」と
を区別せず, むしろ「場の表現」の側から意味を説明しているのに対して, 三
浦文法は両者を分けている. }  \cite{Miura1977,Ikehara1991}を提唱し, そ
れに基づく新しい日本語文法, 三浦文法
\cite{Miura1967a,Miura1967b,Miura1972,Miura1975,Miura1976}を提案してい
る. 三浦文法は, 細部についての分析が及んでいない部分も多々ある未完成な
文法であるが, 従来の自然言語処理の研究では見逃されていた人間の認識機構
を組み込んだより高度な自然言語処理系を実現するための新しい視点を与えて
くれるものと期待される
\cite{Ikehara1987,Ikehara1992,Miyazaki1992a,Miyazaki1992b}.

 そこで, 上記のようなより高度な自然言語処理系を実現するための第一歩と
して, 三浦文法に基づく日本語形態素処理系を実現することを目指し, 三浦文
法をベースに日本語の品詞の体系化を行い, 規則の追加・修正が容易で拡張性
に富む形態素処理用文法を構築した. 本論文では, まず三浦文法の基本的な考
え方について述べ, 次にそれに基づき作成した日本語の品詞体系, および品詞
分類基準を示すと共に, 形態素処理用の新しい文法記述形式を提案する. さら
にそれらの有効性を論じる.

\begin{figure}
\begin{center}
	
	
	
\epsfile{file=fig1.eps,width=63.5mm}
\end{center}
\vspace*{-0.2mm}
\caption{時枝の入れ子型構造}
\label{fig:tokieda}
\end{figure}
\begin{figure}
\vspace*{-0.2mm}
\begin{center}
	
	
	
\epsfile{file=fig2.eps,width=56.0mm}
\end{center}
\vspace*{-0.2mm}
\caption{言語過程説(三浦)の言語モデル}
\label{fig:miura}
\end{figure}
\section{三浦文法の言語観}

時枝は, 「言語が対象−認識−表現からなる過程的構造をもつ」という言語
過程説を提唱し, それを基に時枝文法という独自の日本語文法を構築した. 時
枝文法によれば, 言語表現は話者の主観的な感情・要求・意志・判断などを直
接的に表現した主体的表現と話者が対象を概念化して表現した客体的表現に分
けられ, 文は主体的表現が客体的表現を重層的に包み込んだ入れ子型構造で表
わされる. 時枝は, 言語の本質を主体の概念作用にあるとし, 言語の意味を主
体の把握の仕方, すなわち対象に対する意味作用そのものとした. 従って, 言
語表現に伴う言語規範(言語に関する社会的な約束事)とそれによる媒介の過
程が無視され, 認識を対象のあり方の反映とみる立場が貫かれなくなってしま
い, 言語による情報の伝達について, ソシュールのラングのような個人的な能
力に基礎づけるところまで後退している.

 これに対して, 三浦は言語の意味を対象/認識/表現の関係として捉えるこ
となど, 時枝の言語過程説にいくつかの修正を加え, 独自の理論的展開を図っ
た. 三浦によれば, 音声や文字の種類に結び付き固定された対象と認識の客観
的な関係が言語の意味である. 語は使われて(表現となって)始めて意味(関
係)を生じる. 従って, 表現が存在すれば意味は存在し, 表現(文字, 音声)
が消滅すれば言語規範に固定されていた対象と認識の関係, すなわち意味も消
滅する. 対象や認識そのものは意味ではなく, 意味を形成する実体である. 対
象や認識が消滅しても, 表現がある限り意味は存在する. 意味は話者や聞き手
の側にあるのではなく, 言語表現そのものに客観的に存在する. 三浦の言語過
程説における言語モデルを図2に示す. 

 三浦は, 時枝の提起した“主体の客体化”(1人称の代名詞は主体そのもの
ではなく, 主体が客体化されたものとみなすこと)の問題を, 対象の認識の立
場から発展させ, 主体の観念的自己分裂と視点の移動という観点から主体を捉
えた. 三浦によれば, 一人称の表現は見たところ, 自分と話者が同一の人間で
あるが, これを対象として捉えているということは, 対象から独立して対象に
立ち向かっている別の人間(主体の観念的自己分裂によって生じた観念的話者
“もう一人の自分”)が存在していると考えるのである. 三浦は, このような
観念的話者による視点の移動を表すものとして, 観念的世界が多重化した入れ
子構造の世界の中を自己分裂によって生じた観念的話者が移動する入れ子構造
モデル(図3参照)を提案している. このモデルによれば, たとえば, 過去の
否定表現は, (過去の仮想世界/過去の現実世界/現在の現実世界)の三重の
入れ子構造で表される. 
\begin{figure}
\begin{center}
	
	
	
\epsfile{file=fig3.eps,width=96.5mm}
\end{center}
\caption{三浦の入れ子構造モデル}\label{fig:miura2}
\end{figure}

\section{日本語の品詞体系}

 日本語は, 膠着言語に分類される言語であり, 小さな単位要素が次々と付着
して表現を形成していくという特徴を持つ. これらの単位要素が結合し, 表現
構造を形成していく過程には一定の手順がある. 言語過程説によれば, 日本語
の表現は客体的表現と主体的表現が入れ子になった構造として捉えることがで
きる. 

  ここで, 表現の元となる対象世界を構成する一つの事象は, 実体・
属性・関係の3要素から構成される. これらに対する話者の認識を言語規範を
介して表現に結び付けるときに最も基本となるのは, 概念化された対象(実体・
属性・関係)とそれを表現する単語(詞)との対応関係, ならびに概念化され
た対象に対する話者自身(主体)のあり方と単語(辞)との関係である. 前者
に対して詞が選択され, 後者においてそれに辞が付加される. このようにして
概念化された対象および主体と単語との結び付きが形成されると, 次にそれら
の相互関係が構造化され, 認識された構造と表現構造との対応づけが行われる. 
この過程で単語と単語が統語規則に従って構造化され, 文が形成される
\cite{Ikehara1990}. 

 三浦文法では他の多くの文法とは異なり, 上記のような過程
により形成される日本語文において, 表現に用いられる単語を文構成上の機能
や単語が表す内容で分類するのではなく, 対象の種類とその捉え方で分類する. 
以下, 三浦文法に基づく品詞分類の基本的考え方
\cite{Miyazaki1992a,Ikehara1990,Shirai1992}
について述べ, それに基づき作成した日本語の品詞体系を示す.

\subsection{品詞の大分類}

単語をまず以下のように客体/主体の観点から\underline{詞}と\underline
{辞}に分ける. 詞をさらに分類すると, 一\\つの事象を表現するうえで必須であ
る語とそうでない語の2種類がある. 事象表現に必須である語は, 表現の対象
が実体か実体の属性かにより体言と用言に分けられる.

\underline{体言}は実体・属性・関係からなる対象のう
ち実体を概念化したものである. 実体は物理的実体と観念的実体がある. ま
た, 実体は立体的な構造を持ち, 種々の側面があるため, どの側面から取り
上げるかによって使用される体言も異なってくる. また, 実体の構造に対応
して体言間も構造的な関係を持つ. \hspace*{0.4mm}対象に立ち向かう主体\hspace*{0.2mm}(話者)も客体化
して捉えた時は詞\hspace*{0.2mm}(体言)が用い\\られる. \hspace*{-0.2mm}普通の\hspace*{-0.2mm}\underline{名詞}が実体の
あり方を捉えたものであるのに対して, \hspace*{-0.2mm}\underline{代名詞}
は実体と主体との特\\殊な関係を表現す
る. \hspace*{-0.2mm}主体と対象の関係としては, \hspace*{-0.1mm}1.話者と話者の関係, \hspace*{-0.1mm}2.話者と聞き手の関係,
3.話者と話題となる事物・場所方角・人間などとの関係の3種の関係が
ある. 実体と実体, 属性と属性, 実体と属性の間には種々の関係が存在する. 
関係自体は, 感覚的存在ではないので関係自体を概念的に対象化して名詞と
して用い, 種々の関係は「\underline{上下}(の)関係」や
「\underline{親子}のつなが\\り」などのように表現することが多い. 

 実体の属性を概念化する\hspace*{-0.2mm}\underline{用言}\hspace*{-0.1mm}は, 動的属性を表
す\hspace*{-0.2mm}\underline{動詞}\hspace*{-0.1mm}と静的属性を表す\hspace*{-0.2mm}\underline{形容詞}\hspace*{-0.1mm}に分けられる.\\ 
属性も, これを固定的に実体
化して捉えた場合は, 「大きさ」や「動き」などのように名詞化する. 

 事象表現に必須でない語は, 属性に属性を付加する\hspace*{-0.2mm}\underline{副詞}\hspace*{-0.1mm}
と実体に属性を付加する\hspace*{-0.2mm}\underline{連体詞}\hspace*{-0.1mm}
に分\\類される. なお, 「\underline{もっと}右」の例のような名詞を修飾する語
については, 名詞の中の属性把握の部分を取り上げ, それに属性を付加して
いるとみなすことができるため副詞とする. 

 話者の感情・意志・判断など対象に対する立場や対象から引き起こされる話
者自身に関する認識を表す辞としては助詞や助動詞が用いられる. 

 \underline{助詞}は対象(実体)に立ち向かう話者の立場を直接表現する. 
「花咲く」と
言えば「花」と「咲く」との間の客観的な関係を捉えたものと見ることができ
るが, この関係は変わらないものの, 「花\hspace*{-0.2mm}\underline{が}\hspace*{-0.1mm}咲く」
「花\hspace*{-0.2mm}\underline{は}\hspace*{-0.1mm}咲く」「花\hspace*{-0.2mm}\underline{も}\hspace*{-0.1mm}咲く」
と言えば, 「花」に対する話者の立場が変化してくる. 

 このように, 助詞が実体に対する話者の捉え方, すなわち, 対象(もの)と
主体との関係に関する主体自身の認識を表すのに対して, \underline{助動詞}
は対象(こと)
との関係において話者自身の立場を表現するものと見ることができる. 人間の
認識は現実の世界だけを相手にするだけでなく, 想像によって過去の世界や未
来の世界, 空想の世界などさまざまな世界に行き来する. このような話者の見
る対象世界に対する話者の感情・意志・判断などを直接表現したものが助動詞
である. 

 他に, 話者の感情や意志などを直接表現する\underline{感動詞}, 話者によ
る事象間の関係認識を表現する\underline{接続詞}, および話者の主観を強調
する\underline{陳述副詞}が辞に分類される.

 以上の11品詞の他に, 他の語(接辞承接語)に付加して別の意味や品詞性
を付与する詞である\hspace*{-0.1mm}\underline{接辞}\hspace*{-0.1mm}, 文中に出現する句読点や繰返し記号な
どの\hspace*{-0.1mm}\underline{記号類}\hspace*{-0.1mm}の二つを加えて, 図4に示すように合計13の品詞に
大分類した.

\begin{figure}
\begin{center}
	
	
	
\epsfile{file=fig4.eps,width=95.5mm}
\end{center}
\caption{三浦文法による日本語の品詞体系(大分類)}


\label{fig:hinshi}
\end{figure}

\subsection{品詞分類の細分化}

 単語間の文法的接続関係の検定を精密に行い, 形態素処理の精度を向上させ
るため, 品詞の大分類を細分化して品詞を約400通りに分類した. 細分化の
ポイントを以下に示す. \\
(1) 名詞

実体を同種の他の実体と共通の側面, すなわちその一般性の側面で捉えた認
識を表す\underline{普通名詞}, 
実体をその固有性の側面で捉えた認識を表す\underline{固有名詞}
(\underline{\underline{地名}}, \underline{\underline{人名}}, 
\underline{\underline{組織名}}, \underline{\underline{その他の固有名詞}}),
 動的属性を固定的に実体化して捉え
た認識を表す\underline{動作性名詞}
(\underline{\underline{サ変動詞型名詞}}, 
\underline{\underline{連用形名詞}}, \underline{\underline{その他の動作性転生
名詞}}), 静的属性を固定的に実体化して捉えた認識を表す\underline{状態性名詞}
(\underline{\underline{静詞}}<
形容動詞語幹に対応・ダ型とタルト型がある>, \underline{\underline{状態性転生
名詞}}<形容詞転生名詞\( << \)例\hspace*{-0.15mm}:\hspace*{-0.2mm}寒さ, 厚み\(>>\)・静詞転生名詞
\( << \)例\hspace*{-0.1mm}:\hspace*{-0.2mm}親切さ\(>>\)>, \hspace*{-0.2mm}\underline{\underline{連体詞型名詞}}\hspace*{-0.1mm}<例\hspace*{-0.15mm}:\hspace*{-0.3mm}
大型, 急性>), 対象を具\\体的に取り上げることができなかったり, 取り上げ
る必要がない場合などに, 対象を最も抽象化して捉えた認識を表す
\hspace*{-0.2mm}\underline{形式名詞}\hspace*{-0.2mm}
(例\hspace*{-0.2mm}:\hspace*{-0.2mm}もの, こと, ため, ところ, とき, まま, 際, 場合|の・ん<準体助詞
に対応>|よう<比況の助動詞「よう(だ)」に対応>|そう<伝聞の助動詞
「そう(だ)」に対応>|みたい, ふう)に細分化した. その他に特殊なもの
として, 具体的な数や数量\\など\hspace*{-0.2mm}(例\hspace*{-0.2mm}:\hspace*{-0.2mm}1, 2, ・\hspace{-.5em}・\hspace{-.5em}・, 2本)\hspace*{-0.2mm}単位性の認識を表す
\underline{数詞}, 
属性に属性を付加する副詞としても用い\\られる\underline{副詞型名詞}(
\underline{\underline{時詞}}<例:
今日, 従来>, \underline{\underline{その他の副詞型名詞}}
<例:全て, みんな, 多数, 一部>)を設定した. 

 なお, 静詞については格助詞(に・の・を)や肯定判断の助動詞「だ」の連
体形「な」が後接するか否かを区別できるようにさらに細分化した.\\
(2) 代名詞\\
~~\underline{人称代名詞}(例:私, 彼), \underline{指示代名詞}
(例:これ, あれ), \underline{疑問代名詞}(例:だれ, どれ)
の区分を導入した. \\
(3) 動詞

\underline{本動詞}/\hspace*{-0.2mm}\underline{形式動詞}
\hspace*{-0.2mm}の区別(3.4の(9)参照), 活用型(\hspace*{-0.2mm}\underline{\underline{五段}}/
\hspace*{-0.2mm}\underline{\underline{一段}}/\hspace*{-0.2mm}\underline{\underline{サ変}}/
\hspace*{-0.2mm}\underline{\underline{カ変}})の区別, \\五段動詞に対する活用行の導入により
細分化した. 

 またサ変動詞は, 単独で用いる「する」「〜する(例:開発する, 対する)」
「〜ずる(例:論ずる)」を区別できるようにした. さらに, 五段/一段動詞
のうち, 例外的活用をするもの(例:行く→行った, 有る→有らない×, なさ
る→なさいます・なさい, 問う→問うた, くれる→くれ)は, 別品詞として区
別できるようにした. \\
(4) 形容詞\\
 ~~ウ音便の形態(例:にくい→にく\underline{う}, あさい
→あそ\underline{う}, 美しい→美し\underline{ゅう})により細分化した.\\
(5) 副詞

属性をさらに具体的な面から捉えて別な語と結び付け叙述を立体化する
\underline{情態副詞}
(例:がたがた, ピカピカ), 属性をさらに抽象的な面から捉えて別な語
と結び付け叙述の程度を表す\underline{程度副詞}
(例:ずっと, かなり)に細分化した. 

 さらに, 格助詞(に・と・の)や肯定判断の助動詞(だ・です)が後接する
か否かを区別できるように細分化した. \\
(6) 連体詞\\
~~指示代名詞が連体詞化した\underline{指示連体詞}
(例:この, その), 疑問代名詞が連
体詞化した\underline{疑問連体詞}
(例:どの, どんな), 形容詞が連体詞化した\underline{形容詞的連体詞}
(例:大きな), それ以外で外延の制約を表す\underline{限定連体詞}
(例:ある, あらゆる)に細分化した.\\
(7) 接辞

接辞承接語との接続形態により\underline{接頭辞}/\underline{接尾辞}
/\underline{接中辞}(例:〜対〜)に中分類した. 

 接頭辞については, 接辞承接語の品詞により\underline{\underline{名詞接続型}}
/\underline{\underline{動詞接続型}}(例:ぶち)/\\
\underline{\underline{形容詞接続型接頭辞}}
(例:もの|こ, ま|お)に小分類した. 

 接尾辞については, 接辞承接語+接尾辞で構成される複合語の品詞により
\underline{\underline{名詞型}}/\underline{\underline{動詞型}}
(例:れる, られる, せる, させる|がる, ぶる, めく, づく, 
つく, じみる, 過ぎる|込む, 始める, 終わる, 続ける, きる)/
\underline{\underline{形容詞型接尾辞}}
(例:たい|らしい, がましい, (っ)ぽい|やすい, よい, にくい, づら
い, がたい)に小分類した. 

 さらに, \cite{Nomura1978} を参考に接辞のより細かな文法的・意味的属性によ
り以下のように細分化した. \\
[細分化された名詞接続型接頭辞とその例]\\
<普通名詞型>県, 女, 核\\
<固有名詞型>東, 新, 奥\\
<動詞型>超, 反\\
<形容詞型>大, 新\\
<連体詞型>各, 全, 同\\
<副詞型>再, 最, 既\\
<否定型>無, 不, 非, 未\\
<前置助数詞型>約, 第\\
<敬意添加型>御, ご\\
[細分化された名詞型接尾辞とその例]\\
<普通名詞型>者, 人, たち\\
<固有名詞型>市, さん, 屋, 号\\
<動作性名詞型>\\
\hspace*{3.5mm} \(<<\)サ変動詞型名詞型\(>>\)化, 視|\\
\hspace*{3.5mm} \(<<\)連用形名詞型\(>>\)行き, 沿い|\\
\hspace*{3.5mm} \(<<\)その他\(>>\)発, 製\\<状態性名詞型>\\
\hspace*{3.5mm} \(<<\)ダ型静詞型\(>>\)そう, げ, 的|\\
\hspace*{3.5mm} \(<<\)タルト型静詞型\(>>\)然|\\
\hspace*{3.5mm} \(<<\)状態性転生名詞型\(>>\)さ, み, け|\\
\hspace*{3.5mm} \(<<\)連体詞型名詞型\(>>\)性, 用, 風, 型, 式\\
<後置助数詞型>個, 番, %, \\
<助数詞承接型>強, 台, 目\\
<副詞型名詞型>\\
\hspace*{3.5mm} \(<<\)時詞型\(>>\)前, 後, 間, 内, 中, 時, がてら|\\
\hspace*{3.5mm} \(<<\)その他\(>>\)上\\
<代名詞型>たち, ら, 自身\\
 なお, 各型の接尾辞は細分化された名詞・動詞・形容詞の品詞体系と対応付
けられている(例:普通名詞型接尾辞「者」は普通名詞に対応する). \\
(8) 接続詞\\
~~接続対象により\underline{文接続詞}
(例:しかし, ただし)/\underline{句接続詞}
(例:または, および)の区別を導入した. \\
(9) 感動詞\\
~~話者の呼びかけや感情を表す\underline{感嘆詞}
(例:さあ, おや, まあ), 相手の言葉に対する聞き手の応答を表す
\underline{応答詞}(例:はい, ええ, いいえ)に細分化した. \\
(10) 助動詞\\
~~話者の\hspace*{-0.2mm}\underline{肯定判断}\hspace*{-0.2mm}(だ, ある, です, ます, φ<零判断辞>)/
\hspace*{-0.2mm}\underline{否定判断}\hspace*{-0.2mm}
(ない, ぬ, まい)/\hspace*{-0.2mm}\underline{既定判断}\\[回想・確認]
(た<たり, て>, だ<だり, で>)/\underline{未定判断}
[推量・意志](う, よう, らしい, べし)を表す助動詞に細分化した. \\
(11) 助詞

実体のあり方の認識を表す\hspace*{-0.2mm}\underline{格助詞}
(が, を, に, へ, と, から, より, まで, 
で, をば, って, して\\|の), 認識に対する陳述の要求を表す\hspace*{-0.2mm}\underline{係助詞}
(は, こそ, も, さえ, すら, でも, とて, しか, しも, \\ぞ, して), 実体や
認識に対する観念的前提の付加を表す\underline{副助詞}
(は|など, なんか, なんて|まで, のみ,
(っ)きり, くらい, ぐらい, だけ, ばかり, ほど, とも, ずつ|や, やら, か, 
なり, なりと), 認識内容の確認を表す\underline{間投助詞}
(ね(ぇ), さ(ぁ), よ(ぉ), 
な(ぁ), の(ぉ)|ってば, ったら, って|や, よ|だ, です, と)の他, 事象
間の関係づけを行う\underline{接続助詞}と話者の感情を伝達する
\underline{終助詞}に中分類した. 

 さらに, 接続助詞は接続の型\cite{Ikehara1990,Minami1974,Minami1993}, 
終助詞は伝達の方向\cite{Ikehara1990,Shirai1992,Saeki1983}により, 
それぞれ, 以下に示すように3通りに細分
化した. \\
[接続助詞の細分化]\\
<同時型>つつ, ながら\\
<条件型>ば, と, に, ながら\\
<展開型>が, から, けれど(も), けど(も), し\\
[終助詞の細分化]\\
<話者方向>\\
\hspace*{3.5mm}
 \(<<\)強意\(>>\)ぜ, ぞ, わ, ね(ぇ), さ, よ, な, とも, ってば, ったら, って, 
っと, い, や|\\
\hspace*{3.5mm} \(<<\)驚き\(>>\)わ\\
<相手方向>\\
\hspace*{3.5mm} \(<<\)疑問\(>>\)か, かしら, や, っけ|\\
\hspace*{3.5mm} \(<<\)命令・勧誘\(>>\)な, ねぇ, い, たら, だら|\\
\hspace*{3.5mm} \(<<\)禁止\(>>\)な|\\
\hspace*{3.5mm} \(<<\)伝聞\(>>\)と, って|\\
\hspace*{3.5mm} \(<<\)確認\(>>\)ね(ぇ), さ, よ|\\
\hspace*{3.5mm} \(<<\)婉曲\(>>\)が, けれど(も), けど(も)\\
<不定方向>\\
\hspace*{3.5mm} \(<<\)詠嘆\(>>\)なぁ, わぁ, のぉ, に|\\
\hspace*{3.5mm} \(<<\)不確定\(>>\)やら\\
(12) 記号類\\
~~日本語文中に現れる記号類を, その機能に着目して以下のように細分化した. \\
\underline{句点相当記号}(例:~。~?~!~.)\\
\underline{読点相当記号}(例:~、~,)\\
\underline{中点相当記号}(例:~・ <空白>)\\
\underline{引用符}(例:~「 」『 』‘ ’“ ”)\\
\underline{括弧類}(例:~〈〉《》【】[]{}())\\
\underline{補足記号類}(例:~… 〓 −)\\
\underline{文頭記号}(例:◎○◇▽●☆)\\
\underline{数式関連記号}(例:~,~.~〜 — ‐+−±×÷=≠< >≦ ≧*/)\\
\underline{繰返し記号}(例:~ゝ~ゞ~ヽ~ヾ~々)\\
\underline{その他の特殊記号}(例:~;~:~@~#)

\subsection{活用形の扱い}
 動詞, 形容詞, 動詞型接尾辞, 形容詞型接尾辞のような活用語の活用形は, 
従来の学校文法における6活用形を基本とし, 以下の変更を加えた. \\
(1) 未然形を以下の2通りに細分化した. \\
\hspace*{3mm}・ 未然形1:推量形[〜う, 〜よう]\\
\hspace*{3mm}・ 未然形2:否定形[〜ぬ, 〜ない]\\
(2) 連用形を以下の3通りに細分化した. \\
\hspace*{3mm}・ 連用形1:連用中止形[〜, 〜ます]・連用修飾形\\
\hspace*{3mm}・ 連用形2:音便形[〜た, 〜だ」\\
\hspace*{3mm}・ 連用形3:形容詞ウ音便形[〜ございます]\\
(3) 形容詞のカリ活用語尾は, 以下のように扱う. \\
\hspace*{3mm}・ かろ(未然形1)→く(形容詞語尾・連用形1)\\
\hspace*{3mm}+あろ(助動詞「ある」の未然形1)\\
\hspace*{3mm}・ かっ(連用形2)→く(形容詞語尾・連用形1)\\
\hspace*{3mm}+あっ(助動詞「ある」の連用形2)\\
(4) タルト型形容動詞活用語尾は, 以下のように扱う. \\
\hspace*{3mm}・ と(連用形1)→と(格助詞)\\
\hspace*{3mm}・ たる(連体形)→と(格助詞)\\
\hspace*{3mm}+ある(形式動詞「ある」の連体形)

\subsection{学校文法との主な相違点}
 3.2〜3.3で示した品詞体系と学校文法との主要な相違点は, 以下の通
 りである. \\
(1) 形容動詞を独立した品詞とはせず, 名詞(静詞)+助動詞(肯定判断)「だ」
/名詞(静詞)+格助詞「に」とした.\\
(2) 受身・使役の助動詞(れる, られる, せる, させる)は動的属性を付与する
詞とし, 動詞型接尾辞とした.\\
(3) 希望の助動詞(たい)は静的属性を付与する詞とし, 形容詞型接尾辞とした. \\
(4) 伝聞の助動詞(そうだ), 比況の助動詞(ようだ), 様相の助動詞(そうだ)
は助動詞とせず, それぞれ, 形式名詞(そう, よう)/静詞型接尾辞(そう)
+肯定判断の助動詞(だ)とした.\\
(5) 準体助詞(の), 終助詞(の)は形式名詞とした. \\
(6) 接続助詞(ので, のに), 終助詞(のだ)はそれぞれ, 形式名詞(の)+
[格助詞(で)/肯定判断の助動詞(だ)の連用形1(で)]/格助詞
(に)/肯定判断の助動詞(だ)とした.\\
(7) 接続助詞(て, で, たり, だり)は既定判断の助動詞(た, だ)の
連用形1とした. \\
(8) 補助動詞(ある), 補助形容詞(ない)はそれぞれ, 肯定判断の助動詞, 
否定判断の助動詞とした.

 例:本で\underline{ある}/\underline{ない}, \\
 \hspace*{12mm}静かで\underline{ある}/\underline{ない}, \\
 \hspace*{12mm}重く\underline{ない}\\
 \hspace*{12mm}書いて\underline{ある}/\underline{ない}\\
(9) 既定判断の助動詞の連用形1(て, で)に後接する動詞(いる, みる, く
れる, あげる, くる, もらう, やる, しまう, おく, いく, 下さる, いただ
く, …), 形容詞連用形1/[静詞+格助詞(に)]に後接する動詞(する, 
なる), およびサ変動詞型名詞/連用形名詞に後接する動詞(する, できる, 
下さる, なさる, 致す, 申す, 申し上げる, いただく, 願う, たまう, …)
は, 形式動詞とする.

 例:走って\underline{いる}, 美しく\underline{なる}, \\
 \hspace*{12mm}静かに\underline{なる}, 開発\underline{する}

\section{形態素処理用の文法記述形式}

 形態素処理における隣接単語間の文法的な接続検定には, 通常各単語の辞書
情報に前接コードと後接コードを持たせ, \hspace*{0.4mm}その二つの情報から前接コードと後
接コード間の接続の可否を示す\\マトリックス形式の接続テーブルを用いること
が多い. しかし, 規則が簡潔に分かりやすく記述されておらず, 例外的な接続
をする単語に対しては, 従来の規則との整合性を保ちながら, 新しい前接・後
接コードを付与したり, 辞書情報を変更しなければならず, 規則の追加・修正
が容易でなく, 規則のメンテナンス性が悪い. また, 「良\hspace*{-0.2mm}\underline{さ}\hspace*{-0.1mm}そう
だ」(形容詞に「そうだ」が接続する場合, 通常形容詞語幹に直接「そうだ」
が接続する<例:\hspace*{-0.2mm}楽しそうだ>が, 「良い・無い」は接尾\\辞「さ」を介して接
続する)のように, 2項関係だけでなく, 3項関係もチェックしなければなら
ないような例外的な接続に対応できない.

 そこで, このような問題を解決するものとして, 以下に述べるような形態素
処理用の文法記述形式(接続ルール)を提案する. この接続ルールは基本的に
は, ある品詞\(P_{0}\)の直後(接尾辞などで直前の語の品詞などが問題となる
場合には直前)に文法的に接続可能な全ての品詞 \(q_{1},q_{2},---q_{m}\)
をリスト形式で記述し, 
そのリストと\(p_{0}\)を対にして,
\(\underline{((p_{0})((q_{1})(q_{2})---(q_{m})))}\)
の形で定義したもの(ルール文)の集合であり, 必要に応じて3項以上の関係
にも簡単に拡張できる.

 文法記述においては, 規則の追加・修正が容易であること, 例外的な規則を
記述しやすいこと, 規則を簡潔に分かりやすく記述できることなどを考慮し, 
以下のような点を工夫した. \\
(1) 接続ルールの記述量の削減
\begin{itemize}
\item \(\underline{((s_{0}):((r_{1})(r_{2})---(r_{n})))}\)の形
[\(s_{0}\):定義文番号(=定義文識別子 \underline{F}と通番),
\(r_{j}\):品詞]の定義文を導入し, ルール文の\((q_{i})\)の位置に
\((s_{0})\)がある場合, \((s_{0})\)を\((r_{1})(r_{2})---(r_{n})\) で置
換可であることを表す.

\item \(\underline{((p_{i})=(q_{j}))}\)
の形[\(p_{i},p_{j}\):品詞]の同格文を導入し, \(p_{i}\)は\(q_{j}\)
と同じ接続をすることを表す.

\item 文頭, 文末, 句境界[句は詞(接頭辞以外の接辞・形式名詞・形式動詞は
除く), 接続詞, 感動詞, 陳述副詞で始まる. 用言・体言が後接するか否か
により4種の句境界に分類]には, 文頭, 文末, 句境界であることを表す架
空単語が存在するものとし, それらに特別の品詞を設定した. それらは, ルー
ル文, 定義文などにおける規則の記述において, 通常の品詞と同様に扱
う. 日本語では, 単語間の文法的接続条件は, 句内に比べて句境界では厳し
くなく, 句境界をまたがって接続可能な多くの品詞対がある. このような
句境界をまたがって接続する品詞を句境界架空単語を介して接続するとみな
すことにより, 接続ルールの記述量を大幅に削減できるだけでなく, これを
句境界の検出情報としても利用できる.

\item 4桁の16進数で表示される品詞コード(上位3桁は品詞<1桁目は品詞
の大分類を表す>, 下位1桁は活用形を表す)の2〜4桁目には, ワイルド
カード文字(X,Y,Z<0〜fの任意の値>/x,y,z:<1〜fの任意の値>)が使
え, これによりいくつかの品詞コードをまとめて記述できる. また,
X,Y,Z,x,y,z のとる値を1xyz/\underline{y=1,3,4の}ように指定できる. 
\end{itemize}
(2) 例外的な規則の記述
\begin{itemize}
\item 品詞では表現できず, 各単語に依存するような例外的規則をも前接・後接
コードなどを用いずに記述するため, 品詞\(q_{i}\)と共に単語の字面(表記のゆれ
がある場合にはそれらの代表形である標準表記)を接続ルールに記述できる
ようにした.  例:(\(q_{i}\) \underline{"同じ"})

\item 形容詞のウ音便において形容詞の語幹の末尾が変化する場合
(例:あ\hspace*{-0.2mm}\underline{さ}\hspace*{-0.2mm}い→あ\hspace*{-0.2mm}\underline{そ}\hspace*{-0.2mm}う), 
単\hspace*{0.05mm}語\hspace*{0.05mm}辞\hspace*{0.05mm}書\hspace*{0.05mm}中\hspace*{0.05mm}の\hspace*{0.05mm}各\hspace*{0.05mm}単\hspace*{0.05mm}語\hspace*{0.05mm}の\hspace*{0.05mm}素\hspace*{0.05mm}性\hspace*{0.05mm}情\hspace*{0.05mm}報\hspace*{0.05mm}を\hspace*{0.05mm}用\hspace*{0.05mm}い\hspace*{0.05mm}た\hspace*{0.05mm}素\hspace*{0.05mm}性\hspace*{0.05mm}チ\hspace*{0.05mm}ェ\hspace*{0.05mm}ッ\hspace*{0.05mm}ク\hspace*{0.05mm}に\hspace*{0.05mm}よ\hspace*{0.05mm}り\hspace*{0.05mm}接
続判定を行う必要がある場\\合(例:ある形容詞語幹に接尾辞<\hspace{-0.2mm}さ, \hspace{-0.2mm}み, \hspace{-0.2mm}・\hspace{-.5em}・\hspace{-.5em}・\hspace{-0.1mm}>\hspace{-0.1mm}
が\hspace*{-0.05mm}接\hspace*{-0.05mm}続\hspace*{-0.05mm}す\hspace*{-0.05mm}る\hspace*{-0.05mm}か\hspace*{-0.05mm}否\hspace*{-0.05mm}か\hspace*{-0.05mm})\hspace*{-0.05mm}など例外的処理を行\\うため, 接続ルールからある種の手
続きの起動を指示するような情報(手続き識別子 \underline{\$}と手続き名)
を品詞\(q_{i}\)と共に接続ルール中に記述できるようにした. 
例:(\(q_{i}\)/\underline{\$uonbin})

\item 規則の適用性や優先度を示す接続確率(確率識別子 \underline{*}と確
率値<0〜1>, 本情報を省略した場合は1とみなす)を導入し, 例外的な接続
をする場合にその規則の適用性が変化したり, 通常, あまり適用されない規則
であることなどを記述できるようにした.

 例:\(((p_{0})((q_{1})(q_{2})--(q_{i}~\underline{*0.5})--(q_{m})))\)・\hspace{-.6em}・\hspace{-.6em}・\hspace{-.6em}・\hspace{-.6em}・\hspace{-.6em}・(a)\\
~~~~~\(((p_{0})((q_{1})(q_{2})---(q_{m}))~\underline{*0.1})\) ・\hspace{-.6em}・\hspace{-.6em}・\hspace{-.6em}・\hspace{-.6em}・\hspace{-.6em}・\hspace{-.6em}・\hspace{-.6em}・\hspace{-.6em}・\hspace{-.6em}・\hspace{-.6em}・\hspace{-.6em}・(b)\\
~~~~~~~(a):品詞\(p_{0}\)と品詞\(q_{i}\)の接続確率=0.5 \\
~~~~~~~(b):品詞\(p_{0}\)と品詞\(q_{i}\)(i=1〜m)の接続確率=0.1
\end{itemize}
(3) 接続ルールの拡張性
\begin{itemize}
\item  品詞\hspace*{-0.1mm}\(p_{0}\)\hspace*{-0.1mm}に関する一般ルール\hspace*{-0.1mm}\(((p_{0})((q_{1})(q_{2})---(q_{m})))\)\hspace*{-0.1mm}
 の他に, 品詞\hspace*{-0.1mm}\(p_{0}\)\hspace*{-0.1mm}の例
外ルール(単\\語の字面が指定されたもの, 3項以上の関係を記述したものな
ど)がある場合, 例外ルールを一般ルールの前に置いて一般ルールより高い
優先順位を与え, 形態素処理における接続検定において, 一般ルールより先に
例外ルールを先に適用し, 例外ルールの適用失敗時に初めて一般ルールを適
用することとした. これにより, 既存の一般ルールを修正することなく例
外ルールを追加することができる. また, 一般ルール自身の修正も, \((q_{i})\)の
変更・例外化(単語の字面指定化など), 新しい\((q_{i})\)の追加などにより, 他
のルール文に影響を与えることなく局所的修正で済ませることができる. 以
上により, 接続ルールの拡張が容易となった.

\item 接続ルールには, 前接/後接コードを用いないため, これらのコードの付
与, 単語辞書への収録が不要となり, 規則の追加, 修正が容易になった.
\end{itemize}

 ここで, 接続ルールは人間にとって見やすく, メンテしやすく作られている
が, これをそのまま形態素処理における単語接続検定に用いると, 処理効率が
低下する. そこで, 接続ルールをこれと等価で単語接続検定が高速に行える形
式の接続表{\((\underline{p_{i}},\underline{q_{j}}\),(\underline{\underline{
\(p_{i}\)の字面}})(\underline{\underline{\(q_{j}\)の字面}}),
(接続確率),(\(p_{i},q_{j}\)の手続き名))}
\hspace*{-0.2mm}<\(p_{i},q_{j}\)はそれぞれ前接品詞・後接品詞で, 
ワイルドカード文字や句境\\界架空
単語の品詞を用いず, \hspace*{-0.1mm}定義文・同格文による品詞の置換済みである, ( )は省
略可, \hspace*{-0.1mm}\underline{ }\hspace*{-0.1mm}や\hspace*{-0.1mm}\underline{\underline{ }}\hspace*{-0.1mm}\\の部分をそれぞれ主キー,
 副キーとして接続表を検索できるよう
になっている. 3項以上の関係は別形式で記述>に変換することとし, 接続ルー
ルから接続表を自動生成するツール(接続表ジェネレータ)を作成し, 形態素
処理系において, 接続表を用いて単語接続検定を高速に行えるようにした.


\section{本品詞体系の効用}

 本品詞体系は, 三浦文法をベースに作成されており, 構文構造として, 従来
の句構造や係り受け構造とは異なった三浦の入れ子構造を想定している. 従っ
て, 三浦の入れ子構造と親和性が良く, 以下のような日本語処理系の実現が
期待できる. \\
(1)多目的利用型の日本語形態素処理用文法

きめ細かい品詞分類に基づく形態素処理用の日本語文法により, 単語間の文
法的接続チェックを厳密に行えるので, 本日本語文法を正しい文の解析だけで
なく, 正しい文の生成や入力文の誤り検出など多目的に利用可能となる.

3で提案した約400通り(大分類=13)の品詞体系に基づき, 実際に網
羅的な日本語形態素処理用文法を作成した. 種々の日本語表現を含む日英機械
翻訳用機能試験文(3300文)\cite{Ikehara1990} を用いて形態素解析シ
ステム\cite{Takahashi1993} 上で形態素解析実験を行った. 
本実験においては, 例外的な規則をも含む文法の記述のしやすさ, 規則の追加・
修正など文法の拡張のしやすさなどを, 文法規則や辞書の修正, 文法規則や辞
書への単語の追加を行いながら確認した.

 その結果, 例外的な規則も含めて文法を簡潔に記述できるため, 文法規則が
コンパクトとなるだけでなく(ルール数=374), 規則の追加・修正が容易
で拡張性の点で優れていることなど, 本品詞体系および形態素処理用の文法記
述形式の有効性を確認できた. また, 本実験に用いた3300の機能試験文に
は日本語の種々の表現を網羅しているため, 本実験によりかなり精度の良い文
法を実現できたと考えている. \\
(2)意味と整合性のよい構文解析
\begin{figure}
\begin{center}
	
	
	
\epsfile{file=fig5.eps,width=104.0mm}
\end{center}
\caption{構文解析結果と意味との整合性}
\label{fig:syntax}
\end{figure}

現在, 主流となっている文節構文論(学校文法)に基づく日本語パーザでは
構文解析結果が一般に意味と整合性が良くなく, 時枝文法風の構文解析の方が
解析結果に則って意味がうまく説明できることが指摘されている
\cite{Mizutani1993}.
本文法は時枝文法を発展的に継承した三浦文法を採用しているので, 意味と整
合性の良い構文解析を行うことができる.

 例えば, 「山を下り, 村に着いた」は, 学校文法風に解析すれば, 図5(a)
のような意味的におかしい解析結果を得るが, 三浦文法風に解析すれば, 図5
(b)のように意味的に正しい解析ができる(助動詞「た」のスコープは, 
(b)の場合動詞「下る」と「着く」を含む文全体となるが, (a)の場合動
詞「着く」のみとなる). 

また, 「太郎は今日山を下り, 村に着いた」は, 
学校文法風の係り受け解析では, 図5(c)のように「太郎は」「今日」の係
り先は「下り」か「着いた」のどちらか一方となるが(通常, 係り受け解析で
は係り受けの曖昧さの爆発的増大を抑止するため, 係り受けの非交差条件と係
り先は1つであるという制約をもうけている), 三浦文法風に解析すれば, 図
5(d)のように「太郎は」「今日」が共に「下り」「着い」の両方に係って
いるという意味的にも正しい解析結果を入れ子構造により自然に表現すること
ができる. \\
(3)微妙なニュアンスの違いを解析できる構文・意味解析

助詞の使い分けによって生じる微妙なニュアンスの違いは, 格助詞を単に用
言に対する実体の関係(格関係)として捉えている限り解析できない. 三浦の
助詞論\cite{Miura1967a,Miura1967b,Miura1972,Miura1975,Miura1976}によれ
ば, 格助詞「が」は実体の個別性, 係助詞「は」は実体の普遍性, 副助詞「は」
は実体の特殊性を表す. 例えば, 「鳥が飛ぶ」は発話者の目前にいるインスタ
ンス(個体)としての「鳥」を取り上げているのに対して, 「鳥は飛ぶ」はク
ラス(種)としての「鳥」を取り上げているし, 「太郎は学生です」と「太郎
が学生です」は微妙なニュアンスの違いがある\cite{Miyazaki1993}.

 次に, 連続した辞が付加されている述語の構造を考える. 「読みませんでし
た」と「読まなかったです」を例にとり両者の構造を比較する. ただし, 「読
む」はその否定形「読まない」と比べ肯定と見ることができるので, このよう
な肯定をφ判断辞で表現し, また「なかっ」は「ない+ある」と分けて考える. 
この上で, 三浦が指摘した観念的世界の多重化に基づいて述語構造を考察する
と図6のようになる. すなわち, 「読む」の後に, 肯定・否定・肯定・既定・
肯定の順に話者の判断が重畳されている点ではまったく同じ構造となっている. 
ただし, 陽に表現された語とφ判断辞の違いから, 丁寧さと断定の度合いに微
妙な差が生じていると考えられる.
\begin{figure}
\begin{center}
	
	
	
\epsfile{file=fig6.eps,width=100.0mm}
\end{center}

\caption{連続した辞が付加された述語の入れ子構造}
\label{fig:jutsugo}

\end{figure}

 このように, 三浦文法の品詞体系やそれに基づく入れ子構造により, 助詞の
使い分けや連続した辞によって表現される微妙なニュアンスの違いを解析でき
るようになる. \\
(4)二つの品詞性のある語を自然に扱える構文解析

一語が二つの品詞性を持つ場合(一語が体と用を兼ねて使われる場合等)の
例として「本を読みはしない」をとりあげる. 話者は「本を読む」という事象
を取り上げ, 「は」で特殊性という主体判断を下した後, その動作に対して否
定の判断を下している. ここで, 事象の特殊性を表すために, 取り上げた事象
全体の捉え直しも行われ, 実体化(体言化)が行われている. すなわち, この
表現は図7(a)のような入れ子構造と見ることができる. 「読む」は二重線
の内側の世界では動詞として働いているが, その外の世界の構成要素で体言の
一部分を構成していると考えられる.
\begin{figure}
\begin{center}
	
	
	
\epsfile{file=fig7.eps,width=114.5mm}
\end{center}
\caption{二つの品詞性のある語を含む文の入れ子構造}
\label{fig:futatsu}
\end{figure}
\begin{figure}
\begin{center}
	
	
	
\epsfile{file=fig8.eps,width=94.0mm}
\end{center}
\caption{入れ子破りが起こる文の入れ子構造}
\label{fig:yaburi}
\end{figure}

 このように, 実際の表現の場面では, ある品詞属性を持つ単語が組み合わさっ
て文要素が構成されるという単純な図式では説明できないと思われるものを図
7(a)〜(b)のように入れ子構造により自然に扱うことができる. \\
(5)入れ子破りの表現を解析できる構文解析

係り受けが交差し入れ子破りが生じる場合, 句構造解析では構文木が生成さ
れない. 一方, 係り受け解析では, 係り受け構造が得られるが, 係り受けの曖
昧さが爆発的に増大してしまう. 三浦の入れ子構造では, 「本を決して読まな
い」「うなぎを浜松に食べに電車で行った」など入れ子破りが生じる場合, 主
体表現である陳述副詞「決して」と否定の助動詞「ない」との呼応, および格
要素「浜松に」が直近の動詞「食べる」に係らず, 後方の動詞「行く」に係る
という点に着目して, 図8(a)〜(b)のように入れ子構造化できる. \\
(6)意味の単位としてのフレーズが切り出し可能

統語構造と意味は一体化したものであり, これを独立に扱おうとすれば, 構
造のもつ意味が欠落する. 各部分はそれを含む上位の構造の中に位置づけられ
て始めて意味を持つ. 従って, 部分を全体の中で位置づけて解析を進めること
が必要である. 三浦の入れ子構造モデルでは, 内側の入れ子はそれを包含する
より大きな入れ子の部分となっており, 上位の構造の中に自然に位置づけられ
ている. そこで, このような入れ子を意味のまとまった単位(フレーズや慣用
表現)に翻訳しようとするフレーズ翻訳方式や多段翻訳方式
\cite{Ikehara1987,Ikehara1992}
などにおける意味の単位とすることができると考える.

\section{むすび}
 時枝文法を発展的に継承した日本語文法である三浦文法に基づき, 単語を対
象の捉え方で分類することにより, 日本語の品詞の体系化を行い, 品詞を約4
00通り(大分類数:13)に分類した品詞体系を作成すると共に, 規則の追
加・修正が容易で拡張性に富む形態素処理用の文法記述形式を提案し, それら
の有効性を論じた. 

 本論文で提案した品詞体系に基づき, 実際に網羅的な日
本語形態素処理用文法(ルール数=374)を作成し, 種々の日本語表現を網
羅した日英機械翻訳用機能試験文(3300文)を用いた形態素解析実験によ
り, 本文法の改良を進めると共に, 本品詞体系および形態素処理用の文法記述
形式の有効性を確認した. 今後, 本文法を組込んだ形態素解析システムの定量
的評価を進めると共に, 構文解析用文法規則を作成する予定である.

\vspace*{10mm}
\acknowledgment
接続表ジェネレータの作成や形態素解析実験にご協力いただ
いた, 高橋大和君(現NTTコミュニケーション科学研究所勤務), 宍倉祐司
君, 前川忠嘉君をはじめとする, 新潟大学工学部情報工学科・宮崎研究室の学
生諸君に深謝する. 

\begin{thebibliography}{}

\bibitem[\protect\BCAY{Barwise \BBA\ Perry}{Barwise \BBA\
  Perry}{1983}]{Barwise1983}
Barwise, J.\BBACOMMA\  \BBA\ Perry, J. \BBOP 1983\BBCP.
\newblock {\JBOQ Situation and Attitudes\JBCQ}
\newblock MIT Press.

\bibitem[\protect\BCAY{池原, 宮崎, 白井, 林}{池原, 宮崎, 白井, 林}{1987}]{Ikehara1987}
池原, 宮崎, 白井, 林 \BBOP 1987\BBCP.
\newblock \JBOQ 言語における話者の認識と多段翻訳方式\JBCQ\
\newblock \Jem{情処論}, {\Bbf 28}  (12), 1269--1279.

\bibitem[\protect\BCAY{池原, 白井}{池原, 白井}{1990}]{Ikehara1990}
池原, 白井 \BBOP 1990\BBCP.
\newblock \JBOQ 日英機械翻訳機能試験項目の体系化\JBCQ\
\newblock \Jem{信学技報}, {\Bbf NLC90-43}, 17--24.

\bibitem[\protect\BCAY{池原}{池原}{1991}]{Ikehara1991}
池原悟 \BBOP 1991\BBCP.
\newblock \JBOQ 言語表現の意味\JBCQ\
\newblock \Jem{人工知能学会誌}, {\Bbf 6}  (2), 290--291.

\bibitem[\protect\BCAY{池原, 宮崎, 白井}{池原, 宮崎, 白井 }{1992}]{Ikehara1992}
池原, 宮崎, 白井 \BBOP 1992\BBCP.
\newblock \JBOQ 言語過程説から見た多段翻訳方式の意義\JBCQ\
\newblock
  \Jem{自然言語処理の新しい応用シンポジウム論文集,ソフトウェア科学会/電子情報
通信学会}, 139--140.

\bibitem[\protect\BCAY{南}{南}{1974}]{Minami1974}
南不二男 \BBOP 1974\BBCP.
\newblock \Jem{現代日本語の構造}.
\newblock 大修館書店.

\bibitem[\protect\BCAY{南}{南}{1993}]{Minami1993}
南不二男 \BBOP 1993\BBCP.
\newblock \Jem{現代日本語文法の輪郭}.
\newblock 大修館書店.

\bibitem[\protect\BCAY{三浦}{三浦}{1967a}]{Miura1967a}
三浦つとむ \BBOP 1967a\BBCP.
\newblock \Jem{認識と言語の理論, 第一部}.
\newblock 勁草書房.

\bibitem[\protect\BCAY{三浦}{三浦}{1967b}]{Miura1967b}
三浦つとむ \BBOP 1967b\BBCP.
\newblock \Jem{認識と言語の理論, 第二部}.
\newblock 勁草書房.

\bibitem[\protect\BCAY{三浦}{三浦}{1972}]{Miura1972}
三浦つとむ \BBOP 1972\BBCP.
\newblock \Jem{認識と言語の理論, 第三部}.
\newblock 勁草書房.

\bibitem[\protect\BCAY{三浦}{三浦}{1975}]{Miura1975}
三浦つとむ \BBOP 1975\BBCP.
\newblock \Jem{日本語の文法}.
\newblock 勁草書房.

\bibitem[\protect\BCAY{三浦}{三浦}{1976}]{Miura1976}
三浦つとむ \BBOP 1976\BBCP.
\newblock \Jem{日本語とはどういう言語か}.
\newblock 講談社.

\bibitem[\protect\BCAY{三浦}{三浦}{1977}]{Miura1977}
三浦つとむ \BBOP 1977\BBCP.
\newblock \Jem{言語学と記号学}.
\newblock 勁草書房.

\bibitem[\protect\BCAY{宮崎, 池原, 白井}{宮崎, 池原, 白井}{1992}]{Miyazaki1992a}
宮崎, 池原, 白井 \BBOP 1992\BBCP.
\newblock \JBOQ 言語の過程的構造と自然言語処理\JBCQ\
\newblock
  \Jem{自然言語処理の新しい応用シンポジウム論文集,ソフトウェア科学会/電子情報
通信学会}, 60--69.

\bibitem[\protect\BCAY{宮崎}{宮崎}{1992}]{Miyazaki1992b}
宮崎正弘 \BBOP 1992\BBCP.
\newblock \JBOQ 言語を理解するコンピュータ 自然言語技術の展望\JBCQ\
\newblock \Jem{コンピュートロール}.
\newblock コロナ社, (37), 75-81.

\bibitem[\protect\BCAY{宮崎, 高橋}{宮崎, 高橋}{1993}]{Miyazaki1993}
宮崎, 高橋 \BBOP 1993\BBCP.
\newblock \JBOQ 話者の対象認識過程からみた助詞「が」「は」の意味分析\JBCQ\
\newblock \Jem{情報処理学会第46回全国大会},  {\Bbf 3} (1B--8).

\bibitem[\protect\BCAY{水谷}{水谷}{1993}]{Mizutani1993}
水谷静夫 \BBOP 1993\BBCP.
\newblock \JBOQ 意味・構文の関係を考へる九十例\JBCQ\
\newblock \Jem{計量国語学}, {\Bbf 19}  (1), 1--14.

\bibitem[\protect\BCAY{野村}{野村}{1978}]{Nomura1978}
野村雅昭 \BBOP 1978\BBCP.
\newblock \JBOQ 接辞性字音語基の性格\JBCQ\
\newblock \Jem{国立国語研究所報告, (61)}, 102--138.

\bibitem[\protect\BCAY{佐伯}{佐伯}{1983}]{Saeki1983}
佐伯哲夫 \BBOP 1983\BBCP.
\newblock \JBOQ 語順と意味\JBCQ\
\newblock \Jem{日本語学}, {\Bbf 2}  (12), 30--38.

\bibitem[\protect\BCAY{白井, 宮崎, 池原}{白井, 宮崎, 池原}{1992}]{Shirai1992}
白井, 宮崎, 池原 \BBOP 1992\BBCP.
\newblock \JBOQ 言語過程説から見た日本語述語の構造\JBCQ\
\newblock
  \Jem{自然言語処理の新しい応用シンポジウム論文集,ソフトウェア科学会/電子情報
通信学会}, 141--142.

\bibitem[\protect\BCAY{高橋, 佐野, 宍倉, 前川, 宮崎}{高橋\Jetal
  }{1993}]{Takahashi1993}
高橋, 佐野, 宍倉, 前川, 宮崎 \BBOP 1993\BBCP.
\newblock \JBOQ 頑健性を目指した日本語形態素解析システムの試作\JBCQ\
\newblock \Jem{自然言語処理における実働シンポジウム論文集,
  電子情報通信学会/ソフトウェア科学会}, 1--8.

\bibitem[\protect\BCAY{時枝}{時枝}{1941}]{Tokieda1941}
時枝誠記 \BBOP 1941\BBCP.
\newblock \Jem{国語学原論}.
\newblock 岩波書店.

\bibitem[\protect\BCAY{時枝}{時枝}{1950}]{Tokieda1950}
時枝誠記 \BBOP 1950\BBCP.
\newblock \Jem{日本文法口語篇}.
\newblock 岩波書店.

\end{thebibliography}
\newpage
\section*{付録}

\begin{center}
	
	
	
\epsfile{file=app1.eps,width=84.0mm}
\end{center}
\newpage
\begin{center}
	
	
	
\epsfile{file=app2.eps,width=84.0mm}
\end{center}
\newpage
\begin{center}
	
	
	
\epsfile{file=app3.eps,width=81.0mm}
\end{center}

\begin{biography}
\biotitle{略歴}
\bioauthor{宮崎 正弘}{
1969年東京工業大学工学部電気工学科
卒業.同年日本電信電話公社に入社.以来,電気通
信研究所において大型コンピュータDIPSの開発,
コンピュータシステムの性能評価法の研究,日本文
音声出力システムや機械翻訳などの自然言語処理の
研究に従事.1989年より新潟大学工学部情報工学科
教授.自然言語理解,機械翻訳,辞書・シソーラス
など自然言語処理用言語知識の体系化などの研究に
従事.工学博士.電子情報通信学会,情報処理学会,
人工知能学会,各会員}

\bioauthor{池原 悟}{
1967年大阪大学基礎工学部電気工学科卒
業.1969年同大学大学院修士課程修了.同年日本電
信電話公社に入社.以来,電気通信研究所において
数式処理,トラヒック理論,自然言語処理の研究に
従事.現在,NTTコミュニケーション科学研究所
池原研究グループ・リーダ(主幹研究員).工学博
士.1982年情報処理学会論文賞,1993年情報処理学
会研究賞受賞.電子情報通信学会,情報処理学会,
人工知能学会,各会員}

\bioauthor{白井 諭}{
1978年大阪大学工学部通信工学科卒業.
1980年同大学大学院博士前期課程修了.同年日本電
信電話公社入社.以来,電気通信研究所において日
英機械翻訳を中心とする自然言語処理の研究に従事.
現在,NTTコミュニケーション科学研究所主任研
究員.電子情報通信学会,情報処理学会,各会員}

\bioreceived{受付}
\biorevised{再受付}
\bioaccepted{採録}

\end{biography}

\end{document}



