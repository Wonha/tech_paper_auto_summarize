\documentstyle[jtheapa,ling-macros,treemacros,epsf]{jnlp_j_b5_old}

\input{prepictex}
\input{pictex}
\input{postpictex}
\input{fig2tex}


\def\atari(#1,#2){}

\newtheorem{df}{}
\newtheorem{pr}{}
\newtheorem{co}{}

\newenvironment{definit}[1]{}{}
\newenvironment{principle}[1]{}{}
\newenvironment{constr}[1]{}{}

\newcommand{\zero}[1]{}
\newcommand{\prole}[2]{}
\newcommand{\pzero}[2]{}

\newcommand{\figlabel}[1]{}
\newcommand{\tablelabel}[1]{}
\newcommand{\eqlabel}[1]{}
\newcommand{\deflabel}[1]{}
\newcommand{\constrlabel}[1]{}
\newcommand{\prilabel}[1]{}
\newcommand{\exslabel}[1]{}

\newcommand{\figref}[1]{}
\newcommand{\tableref}[1]{}
\newcommand{\eqref}[1]{}
\newcommand{\defref}[1]{}
\newcommand{\constrref}[1]{}
\newcommand{\priref}[1]{}
\newcommand{\exsref}[1]{}

\newcommand{\lw}[1]{}

\renewcommand{\theenums}{}


\setcounter{page}{19}
\setcounter{巻数}{2}
\setcounter{号数}{1}
\受付{1994}{5}{27}
\再受付{1994}{9}{5}
\採録{1994}{11}{25}

\setcounter{secnumdepth}{2}

\title{主観的動機に関する意味および語用論的制約を利用した \\ 日本語複文
の理解システム \\ 〜「ので」「のに」による接続を中心として〜}
\author{西沢 信一郎\affiref{YNU} \and 中川 裕志\affiref{YNU}}

\headauthor{西沢 信一郎・中川 裕志}
\headtitle{主観的動機に関する意味および語用論的制約を利用した日本語複文
の理解システム}

\affilabel{YNU}{横浜国立大学工学部 電子情報工学科}
{Division of Electrical and Computer Engineering, Faculty of
Enginnering, Yokohama National University}

\jabstract{
我々は,接続助詞「ので」による順接の複文と接続助詞「のに」による逆接の複
文を対象とする理解システムを計算機上に構築することを目的とする.この際に
は,ゼロ代名詞の照応の解析が重要な問題となるが,文献\cite{中川:複文の意
味論,COLING94}にあるように,本論文で扱う形式の複文では動機保持者という語
用論的役割を新たに定義し用いることにより,従属節と主節それぞれで設定され
る意味役割や語用論的役割の間の関係を制約として記述することができる.そこ
で,日本語の複文に対する形態素解析や構文解析の結果を素性構造で記述し,こ
の結果に対して制約論理プログラミングの手法を用いることにより意味および語
用論的役割間の制約を解消し,ゼロ代名詞照応などを分析する理解システムを計
算機上に構築した.
}

\jkeywords{順接および逆接の複文,ゼロ代名詞の照応,動機保持者,意味およ
び語用論的役割,制約論理プログラミング}

\etitle{A zero anaphora resolution system \\ for Japanese complex
sentences \\ based on semantic and pragmatic constraints}
\eauthor{Shin'ichiro Nishizawa\affiref{YNU} \and Hiroshi
Nakagawa\affiref{YNU}}

\eabstract{
Our aim is to construct a system which is able to deal with semantics of
Japanese complex sentences.  One of the most important problem to be
solved is the zero anaphora resolution.  According to (Nakagawa, 1994;
Nakagawa \& Nishizawa, 1994), we can use a new pragmatic role called
{\it motivated\/} to bridge semantic roles of subordinate and those of
main clauses.  The new role constrains the relation among semantic and
pragmatic roles within subordinate or main clause for Japanese complex
sentences.  So, we construct the system that treats this type of
relation by constraint logic programming, and we use feature structures
to formalize these constraints in unification grammar formalism.  We
also describe how this system works.
}

\ekeywords{Japanese complex sentence, zero anaphora resolution,
motivated role, semantic and pragmatic constraint, constraint logic
programming}

\begin{document}
\maketitle

\section{はじめに}

我々が目標とするのは,日本語の複文の理解システムである.このようなシステ
ムにおいては,{\bf ゼロ代名詞}の照応の解析が重要な問題となり,例えば「の
で」「から」などで接続された複文におけるゼロ代名詞照応の解析は,構文論,
意味論,語用論の総合的な利用が要求される.
文献\cite{中川:複文の意味論,COLING94}では,複文中に設定される意味および
談話役割を用いた制約条件という形でこの問題を取り扱うことが提案されている.
これは,ゼロ代名詞と対応する役割 (動作主,経験者など) だけではなく,語用
論的な役割 (観察者など) の照応にも言及する制約であり,これによって,意味
論および語用論を統合した形での複文の意味解析が可能である.

ところで,意味役割や語用論的役割の照応解析の結果は,役割間での照応関係と
いう形で得られるが (例えば, ``観察者 = 動作主''など),実際にそれらの役
割がどのような対象を指示するかは文脈情報を利用しないと決定できない場合が
多い.
つまり,各役割を変数とみなした場合,変数の値が決定されているわけではない
が,別の変数との関係づけがなされている,という情報を解析の途中および結果
として扱う必要がある.
このような場合に用いられる方法論の一つとして,制約論理プログラミング
\cite{橋田:情報の部分性}が考えられる.この場合,変数の間の関係 (同値関係
など) をその変数の持つ制約とみなすことにより,適用が可能である.

そこで,著者らは,まず形態素解析システム JUMAN\cite{松本:NewJUMANmanual}
および構文解析システム SAX\cite{松本:NewSAXmanual}を用い,その結果得られ
る素性構造に制約論理プログラミングの手法を用いて,ゼロ代名詞照応などを分
析する理解システムを構築した.この理解システムでは,プログラム変換の手法
を用いた制約変換システム\cite{森:否定情報の扱える制約システム}を利用して
いる.
このシステムで扱える文は,例えば「花子が暑がったので窓を開けた.」など,
文献\cite{中川:複文の意味論,COLING94}で扱った複文の一部であり,日本語文
全体からみてもその対象は非常に限定されるが,文献\cite{中川:複文の意味論,
COLING94}で扱われている他の文,例えば「叱られたので,反省文を書かせた.」,
「病気で苦しかったのに,会社を休めなかった.」などについても,本論文で述
べる手法により,処理が可能である.
また,他の種類の複文,例えば「傷が痛いのなら,病院に行く.」など従属節が
条件節になるような複文に関しても,節間の制約を適切に記述できれば,本論文
での手法の応用は可能である.

なお,本システムに類する研究であるが,まず,本システムで参考としている
ような日本語文の構造をもとにし,LFG (語彙機能文法) の枠組を用いて記述
したシステムが,文献\cite{水野:日本語の文の構造}に述べられている.
これは,日本語文の発話構造を叙述部分と陳述部分に分けて階層化し,それを 
LFG によって記述するものである.この構造は本論文で参考としている日本語
の階層構造 (後で述べる) に類似したものであり,さらに,その枠組上で複文
の構造的な特徴についても議論がなされている.
しかし,その検討の対象が構文解析のレベルに限定されており,本論文による
システムで扱っているような,意味役割や語用論的役割の照応解析といったレ
ベルまでは扱っていない点が異なる.

また,名詞や代名詞の照応解析を対象とした研究としては,例えば,文献
\cite{清水:日本語談話の照応解決}で,視点や焦点といった語用論的概念を用
いた議論が,解析システムの構築を前提としてなされている.
しかし,第一文の解析結果を用いて,第二文以降に現れる名詞や代名詞の照応
解析を行なう,という議論がなされており,本論文で扱うような,従属節と主
節という構造が一文中に現れるような場合の,その一文中での照応関係の解析
を行なう,というものではない.

\section{日本語複文に関する制約および素性構造による表現}
\label{節:複文の制約}

本論文で述べるシステムにより意味解析が可能となる日本語の複文は, 
\tableref{システムの対象}に示すような種類のものである.例えば,次のよう
な例文である.

\begin{table}[htbp]
\caption{システムの解析対象となる複文}
\tablelabel{システムの対象}
\begin{center}
\begin{tabular}{|c||c|} \hline
        {\bf 接続助詞} & ので (順接), のに (逆接) \\ \hline
        {\bf 従属節の形式} & 主観形容詞,主観形容詞 + 「がる」,受動態 
\\ \hline
        {\bf 主節の形式} & 意志的動作の記述 (能動態),使役態 \\ \hline
\end{tabular}
\end{center}
\end{table}

\enumsentence{\exslabel{複文a} 寒かったので,窓を閉めた.}
\enumsentence{\exslabel{複文aa} 寒がったので,窓を閉めた.}
\enumsentence{\exslabel{複文c} 奈緒美は寒がったのに窓を閉めなかった.}
\enumsentence{\exslabel{複文d} オモチャを壊されたのに,作り直さなかった.}

これらの例文は,\exsref{複文a},\exsref{複文aa}が順接の複文,\exsref{複
文c},\exsref{複文d}が逆接の複文といわれるものであり,それぞれ次のような
意味を表していると考えることができる.

\begin{itemize}
        \item 「ので」 (「から」) による順接文 … 従属節での記述内容を原
因として起こった動作・状態が主節での記述内容である.
        \item 「のに」による逆接文 … 従属節での記述内容から予想される結
果とは異なる動作・状態が主節での記述内容である.
\end{itemize}
つまり,これらの複文は全て「原因・理由--結果」を表しているといえる.その
ため,複文の意味を考える際にも,この「因果性」という要素を扱わなければな
らない\footnote{「とき」などを接続助詞とする,時・場所を表す従属節の場合
は,この「因果性」が希薄なため,因果性に基礎をおく制約が有効でない.その
ため,別の方法を検討することが必要である.}.
そこで,次のように{\bf 動機保持者}という意味および語用論的性質を持つ役割
を定義し,従属節と主節との間で意味役割もしくは語用論的役割を橋渡しする役
割を担わせる.

\begin{definit}{動機保持者} \deflabel{動機保持者}
動機保持者とは,従属節で記述される状況によって,主節中で記述される何らか
の動作もしくは状態を引き起こすに十分の動機を持つ人物を指す.
\end{definit}

動機保持者を用いることにより,\tableref{システムの対象}に示した種類の複
文では,意味的な因果性を\tableref{従属節の制約}および\tableref{主節の制
約} のような意味もしくは語用論的役割の間の制約の形で記述することができる
\cite{中川:複文の意味論,COLING94}.
ここで,本論文で扱う制約に必要な意味および語用論的役割の定義を以下に示す.

\begin{itemize}
        \item 動作主,経験者,受動者,対象 … いわゆる $\theta$役割
\footnote{GB 理論における $\theta$役割は,述語の要求する項に対して意味的
情報を与えるものである\cite{Sells:ContemporarySyntacticTheories}.例えば,
{\it hit} (叩く) という述語は,Agent (動作主) と Patient (受動者) という 
$\theta$役割を与える.}に対応する.
        \item 観察者 … 命題部で記述される状況を直接もしくは間接的に観察
する人物のうち,経験者以外の人物を指す\footnote{文献\cite{斎藤:心情述語
の語用論的分析,大江:日英語の比較研究}などの議論をもとにした役割であり,
「悲しい」「痛い」などの主観形容詞に「がる」がついて「悲しがる」「痛がる」
となった時,主観形容詞で表される状態を外部から観察している人物を表す.}.
        \item 被影響者 … 受動態で記述される動作・作用の影響を受ける人物
を指し,
        \begin{itemize}
                \item 直接受動文の場合,接尾辞「られ」が支配する動詞句で
の受動者と同一人物を指す.
        \item 間接受動文の場合,接尾辞「られ」の主格すなわち被害者を指す.
\end{itemize}
        \item 使役者 … 使役態の文の主格に対応する意味役割であり,使役態
で記述されている事態を直接もしくは間接的にひきおこす人物を指す\cite{寺村:
日本語のシンタクスと意味1}.例えば,「母親が赤ん坊にミルクを飲ませる」と
いう文の場合,主格である「母親」が使役者である.
\end{itemize}
なお,以降で,意味もしくは語用論的役割の照応関係を記述する場合, 
``\prole{役割名}{\small 設定された節}''という表記を用いる\footnote{例え
ば, ``\prole{動作主}{主節}''とは,主節中に設定された動作主を表す.}.
\begin{table}[htbp]
\caption{従属節中の制約}
\tablelabel{従属節の制約}
\begin{center}
\begin{tabular}{|c|c|c|} \hline
        {\lw {\bf 従属節の形式}} & \multicolumn{2}{c|}{\bf 動機保持者に
なりうる意味および語用論的役割} \\ \cline{2-3}
         & \multicolumn{1}{c|}{\bf 順接} & \multicolumn{1}{c|}{\bf 逆接}
\\ \hline \hline
        主観形容詞 & \prole{経験者}{\small 従属節} & 
\prole{経験者}{\small 従属節} \\ \hline
        主観形容詞 & {\lw {\prole{観察者}{\small 従属節}}} & \prole{観察
者}{\small 従属節} または \\ 
        ~~+ 「がる」& & \prole{経験者}{\small 従属節} \\ \hline
        受動態 & \prole{被影響者}{\small 従属節} & 
\prole{被影響者}{\small 従属節} \\ \hline
\end{tabular}
\end{center}
\end{table}

\begin{table}[htbp]
\caption{主節での制約}
\tablelabel{主節の制約}
\begin{center}
\begin{tabular}{|c|c|} \hline
        {\bf 主節の形式} & {\bf 動機保持者に関する制約} \\ \hline \hline
        意志的動作 & 動機保持者 = \prole{動作主}{\small 主節} \\ \hline
        使役態 & 動機保持者 = \prole{使役者}{\small 主節} \\ \hline
\end{tabular}
\end{center}
\end{table}
先ほどの例文を用いて,これらの制約がどのように適用されるかを見てみる.
ただし,本論文では,「は」は主格を主題として取り立てる場合のみについて考
えることとする.
\begin{itemize}
        \item[\protect\exsref{複文a}] 寒かったので,窓を閉めた.
\end{itemize}
この文では,主節の「窓を閉めた」という動作を行なった人物は,従属節で「寒
い」と感じた人であり,この二つの意味役割が動機保持者を介して一致する.
\begin{itemize}
        \item[\protect\exsref{複文aa}] 寒がったので,窓を閉めた.
\end{itemize}
一方この文では,誰かが寒がった状況を観察した人物 (観察者) が動機保持者と
なり,これが主節の動作主と一致する.

\begin{itemize}
        \item[\protect\exsref{複文c}] 奈緒美は寒がったのに窓を閉めなかっ
た.
\end{itemize}
従属節では,誰かが寒がった状況を観察した人物が動機保持者となり,これが主
節の動作主と一致する.また,「奈緒美は」は,主節の主格を主題として取り立
てられているとするため,結局「奈緒美」が誰かが寒がっている状況を観察し,
その結果として主節の動作主となる解釈が得られる.
また,\tableref{従属節の制約}により,逆接文の持つもう一つの解釈として,
「寒がった」人物,つまり「寒い」という状況の経験者が動機保持者となり,こ
れが主節の動作主と一致する場合もある.この時も,「奈緒美は」は主節の主格
となるため,「奈緒美」自身が寒がり,かつ主節の動作主となることになる.

\begin{itemize}
        \item[\protect\exsref{複文d}] オモチャを壊されたのに,作り直さな
かった.
\end{itemize}
この文の従属節は間接受身の解釈となる.例えば,「私」がオモチャの所有者で
あり,そのオモチャを「壊される」という動作の被影響者となる.そして,その
人物 (ここでは「私」) が\tableref{従属節の制約}の制約によって動機保持者
となり,主節の動作「作り直す」の動作主となる.

以上で述べた制約について重要な点は,文献\cite{中川:複文の意味論,
COLING94}で提案された動機保持者に関する制約が,従属節における動機保持者
の決まり方,主節における動機保持者の結び付き先を,各々,従属節内,主節内
において局所的に与えている点である.
この局所性により,動機保持者が関与するゼロ代名詞照応の計算で考慮すべき領
域が主節あるいは従属節の内部に限定されるため,計算の効率向上に大きく寄与
し,また,句構造文法など,既成の文法体系の上でこの制約を利用したシステム
を構築する際には都合がよい.

さて,ここで,以上のような複文の解釈結果を素性構造で記述することについて
述べる.
そのために,まず,本論文で扱う「ので (から)」「のに」による複文の階層構
造を,\figref{複文の階層構造}のように考える.
すると,例えば\exsref{複文aa}は,複合事象のレベルまでの構造が
\figref{複文の階層構造の例}となる.
この階層構造は,文献\cite{郡司:制約に基づく文法}での議論をもとにしており,
本論文では接続助詞「ので(から)」「のに」によって形成される従属節中に様相
辞が存在しない場合を扱う\footnote{「北海道は寒いらしいので,上着を持って
行こう.」など,従属節中に「らしい」「そうだ」が存在するような文も考えら
れるが,これは様相辞の意味論に関する問題を含んでおり,本論文の対象外とす
る.}ことなどから,「ので(から)」「のに」が事象レベルに接続して従属節を
形成すると考えることにより得られる.

\begin{figure}[hbtp]
\begin{center}
\unitlength=.04ex
\tree{
\node{発話}
        {\Ln4{判断}
                {\Ln4{主題}}
                {\Rn4{意見}
                        {\Ln4{複合事象}
                                {\Ln7{従属事象}
                                        {\Ln4{事象}}
                                        {\Rn4{接続}}
                                }
                                {\Rn7{事象}
                                        {\Ln4{過程}
                                                {\Ln4{行動/状態}
                                                        {\Ln6{主体/対象}}
                                                        {\Rn6{動作/様子}
                                                                {\Ln4{受け手}}
                                                                {\Rn4{動作/様子}
                                                                        {\Ln4{動作}}
                                                                        {\Rn4{様態}}
                                                                }
                                                        }
                                                }
                                                {\Rn4{相}}
                                        }
                                        {\Rn4{時制}}
                                }
                        }
                        {\Rn4{様相}}
                }
        }
        {\Rn4{陳述}}
}
\end{center}
\caption{複文の基本的な階層構造}
\figlabel{複文の階層構造}
\end{figure}

\begin{figure}[htbp]
\begin{center}
\unitlength=.04ex
\tree{
\node{複合事象}
        {\Ln7{従属事象}
                {\Ln4{事象}{\lf{寒がった}}}
                {\Rn4{接続}{\lf{ので}}}
        }
        {\Rn7{事象}{\lf{窓を閉めた}}}
}
\end{center}
\caption{複文の階層構造の例 : 「寒がったので窓を閉めた」}
\figlabel{複文の階層構造の例}
\end{figure}

次に,複文の意味素性構造について考える.
日本語の複文の素性構造表現,特にその意味素性に関する素性構造表現の定式化
は今だ明確になされていないのが現状である.
例えば,文献 \cite{Tonoike:HierarchicalClauseStructure}では,「ために」
という接続詞の構造を\figref{「ために」の構造}のように表している
\footnote{図中で {\bf adjacent} が従属節,{\bf dep} が主節を表している.}.
このように,{\bf sem} に関しては抽象的な説明となっており, ``{\it a
cause of\/}''という意味が実際にどのような構造をとるかについては述べられ
ていない.
しかし,本論文では節間の意味的なつながりに触れており,\figref{「ために」
の構造}における {\bf sem} にあたる情報をどのような構造で表すかという問題
を扱わなければならない.
そこで,ここでは,複文の意味素性として\figref{複文の意味素性}のような構
造を考える.

\begin{figure}[htbp]
\footnotesize
\begin{center}
\vspace{-0.2mm}
\outerfs{
\bf adjacent: & \outerfs{
        \bf head: & \outerfs{
                \bf pos: & v \\
                \bf tense: & +tensed
        } \\
        \bf sem: & {\it a cause of\/} \fbox{1}
} \\
\bf dep: & \outerfs{
        \bf head: & \outerfs{
                \bf pos: & v \\
                \bf tense: & +tensed
        } \\
        \bf sem: & \fbox{1}
}
}
\end{center}
\caption{文献\protect\cite{Tonoike:HierarchicalClauseStructure}による
「ために」の素性構造}
\figlabel{「ために」の構造}
\end{figure}
\begin{figure}[htbp]
\begin{center}
\outerfs{
        \bf 意味主辞: & 主節の意味主辞 \\
        \bf 意味修飾辞: & \{ \protect\figref{修飾辞素性}に示す素性構造 
\}
}
\end{center}
\caption{複文の意味素性構造}
\figlabel{複文の意味素性}
\end{figure}
\begin{figure}[htbp]
\begin{center}
\vspace{-0.2mm}
\outerfs{
        \bf 意味主辞: & \outerfs{
                \bf 修飾関係: & 従属節 \\
                \bf 接続: & 接続助詞名 \\
                \bf 接続関係: & 順接もしくは逆接 \\
                \bf 動機保持者: & \fbox{1} \\
                \bf 事象: & 従属節の \\ 
                 & 事象レベルでの \\
                 & 意味主辞
        } \\
        \bf 意味修飾辞: & \{~\}
}
\end{center}
\caption{修飾辞素性}
\figlabel{修飾辞素性}
\vspace{-0.1mm}
\end{figure}

まず,複文の意味主辞として主節の意味主辞をとる.\vspace{-0.1mm}意味修飾辞としては,従属
節を示す\figref{修飾辞素性}のような素性構造をとる.\vspace{-0.1mm}

\figref{修飾辞素性}では,意味主辞として,修飾関係,接続,接続関係という
素性を用意している.\vspace{-0.1mm}
修飾関\\係素性の値は,従属節を表すために ``従属節''とし,接続素性は,その
値として ``ので'', ``のに''のような接続助詞名をとる.\vspace{-0.1mm}また,接続関係素性
の値は,接続助詞が「ので」の場合は ``順接'',「のに」の場合は ``逆接''と
する.\vspace{-0.1mm}さらに,動機保持者の参照する人物を値としてとる動機保持者素性と,従
属節の事象レベルでの意味主辞を値としてとる事象素性も意味主辞中に設定する.\vspace{-0.1mm}
なお,事象という素性を用いるのは,前記の\figref{複文の階層構造}の構造に
よる.\vspace{-0.1mm}

\newsavebox{\myboxaaa}
\sbox{\myboxaaa}{
\footnotesize
\outerfs{
        Soa: \outerfs{
                relation: 閉める \\
                動作主: \fbox{1} \\
                対象: \fbox{窓}
        } \\
        時制: 基準時以前 \\
        認め方: 肯定
}
}
\begin{figure}[htbp]
\footnotesize
\begin{quote}
\fbox{Main}=
\outerfs{
判断: \outerfs{
        意見: \outerfs{
                事象: \usebox{\myboxaaa}
                }
        }
}
\end{quote}
\caption{主節の意味主辞素性の値}
\figlabel{主節のSEM}
\end{figure}

ここで,文\exsref{複文aa}の意味素性について考えてみる.主節の意味素性は,
\figref{主節のSEM}に示すようになり,従属節の意味素性は,\figref{従属節の
SEM}に示すようになる\footnote{素性構造中で,\fbox{exp} は,実際に経験者
の対象となる人物の意味情報を表す素性構造を参照する.}.
ただし,\figref{従属節のSEM}の ``OBSERVE''という relation は,\hspace{0.05mm}接尾辞「がる」により導入される関係であり,「がる」\hspace{0.05mm}に\hspace{0.05mm}よ\hspace{0.05mm}っ\hspace{0.05mm}て\hspace{0.05mm}導\hspace{0.05mm}入\hspace{0.05mm}さ\hspace{0.05mm}れ\hspace{0.05mm}る\hspace{0.05mm}役\hspace{0.05mm}割\hspace{0.05mm}が
``観察者: \fbox{2}'',観察された状況が ``Soa: [ ・\hspace{-.5em}・\hspace{-.5em}・ ]''である.
\newsavebox{\myboxa}
\sbox{\myboxa}{
\footnotesize
\outerfs{
        relation: OBSERVE \\
        Soa: \outerfs{
                relation: 寒い \\
                経験者: \fbox{exp}
        } \\
        観察者: \fbox{2}
}
}
\begin{figure}[htbp]
\footnotesize
\begin{quote}
\fbox{Sub}=
\outerfs{
        意味主辞: \outerfs{
                修飾関係: 従属節 \\
                接続: ので \\
                接続関係: 順接 \\
                動機保持者: \fbox{2} \\
                事象: \outerfs{
                        Soa: \usebox{\myboxa} \\
                        時制: 基準時以前 \\
                        認め方: 肯定
                }
        } \\
        意味修飾辞: \{~\}
}
\end{quote}
\caption{従属節の意味素性の値}
\figlabel{従属節のSEM}
\end{figure}

\newsavebox{\myboxc}
\sbox{\myboxc}{
\footnotesize
\outerfs{
        relation: OBSERVE \\
        Soa: \outerfs{
                relation: 寒い \\
                経験者: \fbox{exp}
        } \\
        観察者: \fbox{1}
}
}
\newsavebox{\myboxb}
\sbox{\myboxb}{
\footnotesize
\outerfs{
        意味主辞: \outerfs{
                修飾関係: 従属節 \\
                接続: ので \\
                接続関係: 順接 \\
                動機保持者: \fbox{1} \\
                事象: \outerfs{
                        Soa: \usebox{\myboxc} \\
                        時制: 基準時以前 \\
                        認め方: 肯定
                }
        } \\
        意味修飾辞: \{~\}
}
}
\newsavebox{\myboxbbb}
\sbox{\myboxbbb}{
\footnotesize
\outerfs{
        Soa: \outerfs{
                relation: 閉める \\
                動作主: \fbox{1} \\
                対象: \fbox{窓}
        } \\
        時制: 基準時以前 \\
        認め方: 肯定
}}
\begin{figure}[htbp]
\begin{center}
\footnotesize
\fbox{SEM}=\outerfs{
        意味主辞: \outerfs{
                判断: \outerfs{
                        意見: & \outerfs{
                                事象: \usebox{\myboxbbb}
                        }
                }
        } \\
        意味修飾辞: $\left\{ \usebox{\myboxb} \right\}$
}
\end{center}
\caption{単一化による複文の素性構造}
\figlabel{単一化による複文の素性構造}
\end{figure}

\fbox{Sub} 中では,タグ \fbox{2} の参照関係によって, ``動機保持者 =
\prole{観察者}{\small 従属節}''という制約が記述されている.
ただし,これらから単一化により\figref{複文の意味素性}の素性構造を組み立
てる際に, ``動機保持者 = \prole{動作主}{\small 主節}''という
\tableref{主節の制約}の制約に対応する ``\fbox{1} = \fbox{2}''という制約
も素性構造に反映させる.ここで,\fbox{1} は,主節の動作主を参照するタグ
であり,\fbox{2} は,従属節の観察者を参照するタグである.このようにして
組み立てた複文\exsref{複文aa}の意味素性を\figref{単一化による複文の素性
構造}に示す.ここでは,\fbox{2} は全て \fbox{1} に置き換わっている.

\section{制約変換による日本語複文の意味解析システム}

以上のように,意味および語用論的役割の照応関係に関する意味解析を行なうた
めには,各役割 (変数) の同値関係などの情報を解析中に扱わなければならず,
その結果も各変数間の関係で得られることが多い.このような問題を扱うための
方法論の一つに制約論理プログラミングがある\cite{橋田:情報の部分性}.
一方,JPSG\cite{Gunji:JPSG}における下位範疇化原理などの文法的な原理も,
一種の制約条件とみなせることから\cite{郡司:制約に基づく文法},制約論理プ
ログラミングの手法を取り入れた,複文の意味解析システムの構築が考えられる.

\begin{figure}[htbp]
\begin{center}
        \input{system.tex}
\end{center}
\caption{日本語文の意味解析システム}
\figlabel{システム構成}
\end{figure}

\iffalse
どの部分に語用論的制約を記述し,
どの様にしてそれを利用するか,について
ここで説明する.
\fi

そこで今回,著者らは,形態素解析システム JUMAN\cite{松本:NewJUMANmanual},
構文解析システム SAX\cite{松本:NewSAXmanual}を利用し,さらに制約変換シス
テム\cite{森:否定情報の扱える制約システム}を併用するシステムを構築した.
システムの構成の概略は\figref{システム構成}のようになる.
図中において,JUMAN および SAX は既存のシステムである.
我々は,ここに,SAX で用いる文法規則として\figref{複文の階層構造}の階層
構造を基にした DCG を,さらに意味解析のための制約を記述した意味辞書を新
しく設けた.さらに,これらを利用して解析を行なうために SAX と制約変換シ
ステムを DCG の補強項に記述した Prolog のプログラムを通じて接続したもの
が,本システムである.

このシステムに日本語文 (単文,「ので」「のに」による複文) が入力されると,
JUMAN により形態素解析され,SAX にその結果が渡される.そして,構文解析さ
れるわけだが,ここで,\ref{節:複文の制約}~節の\tableref{従属節の制約},
\tableref{主節の制約}で挙げた制約を用いて,意味役割などの照応解析を行な
う.
本システムでは,\tableref{従属節の制約},\tableref{主節の制約}の語用論的
制約を,各語彙の意味辞書として素性構造の形式で記述する.構文解析を行ない
つつ,これらの情報を制約変換システムにより変換し,意味および語用論的役割
の照応解析を行なうことが,本システムの目的である.

\subsection{意味辞書} \label{節:意味辞書}

以上のように,本システムでは,意味辞書に記述した制約を変換しながら解析が
進む.
このため,各語彙についての意味辞書の記述,特に「ので」「のに」という接続
助詞の意味辞書の記述が本システムにとってのもっとも重要な点となる.

\begin{figure}[htbp]
\begin{quote}
\setlength{\baselineskip}{4.2mm}
\begin{verbatim}
閉める(動詞,F,Kform,
    [F=[主辞:[品詞:動詞,
              活用形:Kform,
              使役形態:ニ使役,
              文法格:[主格:'+',対格:'+',与格:'-'],
              接尾可能様態辞:[passive:'+',causative:'+',
                              observe:'-',desire:'+']],
        文法格内容:[主格:[意味:X],対格:[意味:Y]],
        見出し:閉める,
        slash:[],
        近接:[],
        下位範疇化:[[主辞:[品詞:格助詞,
                           格標識:が,
                           文法役割:subject],
                     依存語彙:[品詞:動詞,見出し:閉める],
                     意味:X],
                    [主辞:[品詞:格助詞,
                           格標識:を,
                           文法役割:object],
                     依存語彙:[品詞:動詞,見出し:閉める],
                     意味:Y]],
        意味:[意味主辞:[soa:[relation:閉める,
                             動作主:X,
                             対象:Y#[意味主辞:[animate:'-']]],
                        時制:T],
              意味修飾辞:[]]],
    constraint_Tense(Kform,T)]).

\end{verbatim}
\end{quote}
\caption{動詞「閉める」の意味辞書}
\figlabel{動詞の意味辞書例}
\end{figure}

本システムにおける意味辞書は,意味役割などがどのように設定されるかといっ
たいわゆる意味情報に加えて,用言における下位範疇化情報など,文法情報に属
するものも併せて素性構造の形式を用いて記述する.
例えば,動詞「閉める」の意味辞書は\figref{動詞の意味辞書例}のように記述
する.
なお, ``{\tt X\#[*]}''という表記は,素性構造 ``{\tt [*]}''を {\tt X} と
いう名前のタグを用いて参照できることを示す.

この辞書には,「閉める」が格助詞「が」を伴う後置詞句を subject として下
位範疇化し,その意味役割は動作主となること,格助詞「を」を伴う後置詞句を 
object として下位範疇化し,その意味役割は対象となること,が記述してある.
時制については {\tt constraint\_Tense/2} という制約を用いて,用言の活用
語尾によって「基準時」(現在形の場合) もしくは「基準時以前」(過去形の場合) 
という値をとるように記述してある.

\begin{figure}[htbp]
\begin{quote}
\setlength{\baselineskip}{4.2mm}
\begin{verbatim}
のだ(助動詞,F,ダ列タ系連用テ形,
     [F=[主辞:[品詞:助動詞,
               接続関係:順接,
               修飾関係:従属節,
               依存:Depend],
         見出し:ので,
         近接:Adjacent,
         意味:Sem],
     constraint_NoDe(Depend,Adjacent,Sem)]).
\end{verbatim}
\end{quote}

\caption{接続助詞「ので」の意味辞書}
\figlabel{接続助詞の意味辞書例}
\end{figure}

\begin{figure}[htbp]
\begin{quote}
\setlength{\baselineskip}{4.2mm}
\begin{verbatim}
constraint_NoDe(
    [主辞:[品詞:動詞,
           態:能動],
     意味:[意味主辞:[事象:[soa:[動作主:Motiv]]]]],
    [[主辞:[品詞:動詞性接尾辞,
            態:observe],
      意味:[意味主辞:Sem#[事象:[soa:[観察者:Motiv]]]]]],
    [意味主辞:Sem#[接続:ので,修飾関係:従属節,
                   接続関係:順接,動機保持者:Motiv],
     意味修飾辞:[]]).
\end{verbatim}
\end{quote}

\caption{「ので」による動機保持者に関する制約の記述}
\figlabel{従属節制約例}
\end{figure}

また,「ので」「のに」による複文を解析するために,これら接続助詞の意味辞
書を記述する必要がある.ここでは,「ので」の意味辞書を\figref{接続助詞の
意味辞書例}に示す\footnote{品詞が助動詞となっているのは,JUMAN\cite{松本:
NewJUMANmanual}による品詞分類をそのまま用いているためである.}.
辞書項目中において,近接素性の値が,「ので」がつく従属節の内容を示す素性
構造となる.また,依存素性の値が,「〜なので」という従属節をとる主節の内
容を示す素性構造となる.

ところで,依存素性や近接素性の値は,主節や従属節の記述形式によって変化す
るものであるため,辞書中では制約を用いてその値を記述する.
これらの値を与えるのが制約 {\tt constraint\_NoDe/3} であり,第一引数が依
存素性の値,第二引数が近接素性の値,第三引数が「ので」の意味素性の値とな
る.
ここでは,従属節が主観形容詞 + 「がる」による記述,主節が意志的動作記述 
(能動態) の場合の制約について,\figref{従属節制約例}に示す.この記述によ
り, ``\prole{観察者}{\small 従属節} = 動機保持者''および ``\prole{動作
主}{\small 主節} = 動機保持者''という制約が変数 {\tt Motiv} を用いて表さ
れている.

\begin{figure}[htbp]
\newsavebox{\conjboxc}
\sbox{\conjboxc}{
\scriptsize
\outerfs{
        \bf 主辞 : & \fbox{H3}\outerfs{
                \bf 依存 : & \outerfs{
                        \bf 主辞 : & \fbox{H2} \\
                        \bf 意味 : & \fbox{S2}
                }
        } \\
        \bf 意味 : & \fbox{S3} \\
        \bf 近接 : & $\left\{\begin{array}{l}
                        \mbox{\outerfs{
                                \bf 主辞 : & \fbox{H1} \\
                                \bf 意味 : & \fbox{S1}
                        }}
                \end{array}\right\}$
        }
}
\newsavebox{\subboxc}
\sbox{\subboxc}{
\scriptsize
\outerfs{
        \bf 主辞 : & \fbox{H1} \\
        \bf 意味 : & \fbox{S1}
}}
\newsavebox{\mainboxc}
\sbox{\mainboxc}{
\scriptsize
\outerfs{
        \bf 主辞 : & \fbox{H2} \\
        \bf 意味 : & \fbox{S2}
}}
\newsavebox{\subordboxc}
\sbox{\subordboxc}{
\scriptsize
\outerfs{
        \bf 主辞 : & \fbox{H3}\outerfs{
                \bf 依存 : & \outerfs{
                        \bf 主辞 : & \fbox{H2} \\
                        \bf 意味 : & \fbox{S2}
                }
        } \\
        \bf 意味 : & \fbox{S1} + \fbox{S3} \\
        \bf 近接 : & \{~\}
}}
\newsavebox{\compboxc}
\sbox{\compboxc}{
\scriptsize
\outerfs{
        \bf 主辞 : & \fbox{H2} \\
        \bf 意味 : & \fbox{S1} + \fbox{S2} + \fbox{S3}
}}
\begin{center}
\unitlength=.08ex
\tree{
\node{\usebox{\compboxc}}
        {\Ln8{\usebox{\subordboxc}}
                {\Ln7{\usebox{\subboxc}}\lf{\scriptsize 従属節の内容}}
                {\Rn7{\usebox{\conjboxc}}\lf{\scriptsize ので}}
        }
        {\Rn8{\usebox{\mainboxc}}\lf{\scriptsize 主節の内容}}
}
\end{center}
\caption{接続助詞「ので」による素性値の共有関係}
\figlabel{共有関係}
\end{figure}

このように,依存素性および近接素性を用いることによって,接続助詞「ので」
による複文は複合事象レベルにおいて\figref{共有関係}のような構造および素
性値の共有関係を持つこととなる
\cite{Tonoike:HierarchicalClauseStructure,JPSGOverView}.なお,接続助詞
「のに」の場合も同様にして扱う.

\subsection{接続助詞「ので」による日本語複文の解析} \label{節:日本語複文
の解析}

本システムは,SAX において DCG を用いることにより構文解析を行ない,同時
に制約変換システムと情報のやりとりをすることによって意味解析を行なうもの
である.
このため,文の解析は\figref{複文の階層構造}のような構造を持つ構文解析木
がボトムアップに生成されるように進む.
複文を解析した場合には,従属節が従属事象として解析され,主節が事象として
解析され,これらを複合事象としてまとめることによって複文となる,というよ
うに解析が進む.

\begin{figure}[htbp]

\begin{center}
  \epsfile{file=fig14.eps,width=85mm}
\end{center}


\caption{構文解析木の出力例}
\figlabel{例文の構文木}
\end{figure}

この節では,「寒がったので,窓を閉めた」という簡単な複文を例にとり,本シ
ステムにおける解析がどのように行なわれるのかについて述べる.
なお,本システムにおけるこの文の構文解析木は,\figref{例文の構文木}のよ
うに得られる.

まず,従属節「寒がったので」の部分の解析について述べる.
これは, ``{\tt 従属事象 --> 事象,接続}''という DCG により,「寒がった」
という事象と「ので」という接続とからなると解析される.
このとき,「寒がった」という部分の解析結果として\figref{従属節部分1}の素
性構造がえられる.また「ので」の意味辞書より,従属節が主観形容詞 + 「が
る」であり主節が動作動詞の能動態の場合として\figref{従属節部分2}の素性構
造がそれぞれ得られる\footnote{以下で示す各素性構造はそれぞれ独立したもの
であり,素性構造中のタグ ({\tt F1, F2, …}) も素性構造毎に独立している.}.

\figref{従属節部分1}の素性構造では,意味素性の中に経験者と対象と観察者と
いう役割が設定されている.
また,経験者については,タグ {\tt F9} によって主格の後置詞句の持つ意味素
性の値を参照し,対象はタグ {\tt F12} によって対格の後置詞句の持つ意味素
性の値を参照する.

\figref{従属節部分2}では,意味素性の値として,\figref{修飾辞素性}に対応
する素性構造をとる.また,近接素性の値は従属節の形式 (この場合は主観形容
詞 + 「がる」) に対応する値であり,主辞素性中の依存素性の値は主節の形式 
(ここでは動作動詞による能動態) に対応している.

\begin{figure}[htbp]

\begin{center}
  
  \epsfile{file=fig15.eps,width=116mm}
\end{center}


\caption{事象「寒がった」の素性構造}
\figlabel{従属節部分1}
\end{figure}

\begin{figure}[htbp]

\begin{center}
  
  \epsfile{file=fig16.eps,width=109mm}
\end{center}


\caption{接続助詞「ので」の素性構造}
\figlabel{従属節部分2}
\end{figure}

そして,\figref{従属節部分1}の素性構造と,\figref{従属節部分2}中の近接素
性の値となっている素性構造との単一化が行なわれ,従属節「寒がったので」の
解析結果として\figref{従属節部分3}の素性構造が得られる.この素性構造中の
意味素性の値は,\figref{従属節のSEM}に示した \fbox{Sub} に相当し,タグ 
{\tt F21} によって,観察者と動機保持者が同じ値を参照することを表している.

\begin{figure}[htbp]

\begin{center}
  
  \epsfile{file=fig17.eps,width=109mm}
\end{center}


\caption{従属節「寒がったので」の素性構造}
\figlabel{従属節部分3}
\end{figure}



次に,主節の「窓を閉めた」という部分が,事象として同様に解析される.
その結果として,\figref{主節部分1}の素性構造が得られる.
ここでは,意味素性の中に動作主と対象という役割が設定されており,タグ 
{\tt F16} によって対象が対格である「窓」の意味素性を参照すること,および
タグ {\tt F8} によって動作主が主格の後置詞句の意味素性の値を参照すること
が示されている.

\begin{figure}[htbp]

\begin{center}
  
  \epsfile{file=fig18.eps,width=85mm}
\end{center}


\caption{主節「窓を閉めた」の素性構造}
\figlabel{主節部分1}
\end{figure}

そして,\figref{従属節部分3},\figref{主節部分1}の素性構造が, ``{\tt 複
合事象 --> 従属事象,事象}''という DCG によって組み合わされる.この時,
\figref{従属節部分3}の依存素性の値と\figref{主節部分1}の素性構造を単一化
することで,複文全体の意味素性が形成される.
その結果として,\figref{全文}に,全文の解析結果として得られる素性構造を
示す.

\begin{figure}[htbp]

\begin{center}
  
  \epsfile{file=fig19.eps,width=140mm}
\end{center}


\caption{「寒がったので窓を閉めた」の解析結果}
\figlabel{全文}
\vspace*{7cm}
\end{figure}

\begin{figure}[htbp]
\begin{center}
  
  \epsfile{file=fig20.eps,width=134mm}
\end{center}


\caption{「寒がったのに窓を閉めなかった」の解析結果}
\figlabel{逆接全文}
\end{figure}

\section{おわりに}

本論文では順接の「ので」による複文の解析を例にとり,日本語の複文の意味解
析システムの構成および動作について説明した.
前節で述べたように,「ので」による順接の複文に関しては,\tableref{従属節
の制約}や\tableref{主節の制約}に示した制約を意味辞書および補強項に記述す
る事により,意味解析が可能になる.
同じように,逆接を表わす接続助詞「のに」についても,意味辞書を前節の
\figref{接続助詞の意味辞書例}のように記述することによって解析を行なうこ
とが可能である.例として,「寒がったのに窓を閉めなかった」という文の解析
結果を\figref{逆接全文}に示す.

この文の場合,\tableref{従属節の制約}および\tableref{主節の制約}の制約か
ら,1) \prole{観察者}{\small 従属節} = 動機保持者 = \prole{動作主}
{\small 主節},2) \prole{経験者}{\small 従属節} = 動機保持者 = \prole{動
作主}{\small 主節} の二通りの解析結果が存在する.\figref{逆接全文}では,
最初の素性構造において 1) の結果をタグ {\tt F8} で,二つめの素性構造にお
いて 2) の結果をタグ {\tt F8} を用いて表している.

このように,意味および語用論的役割に関する制約を\tableref{従属節の制約}
や\tableref{主節の制約}に示したように節ごとの局所的な制約として記述する
ことにより,句構造文法をベースとしたシステムに制約変換システムを組み合わ
せる (\figref{システム構成}参照) という手法で,接続助詞「ので」「のに」
による順接および逆接の複文の意味解析システムを計算機上に構築した.
さらに,「れば」「たら」「なら」により従属節が条件節となる複文などについ
ても,同様の手法で扱えると考えられ,現在検討中である.


\section*{謝辞}
本研究において,日本語の複文の理解システムの試作を進めるにあたり,Prolog 
による制約変換システムを提供して頂き,また,その利用や,理解システム試作
に関する全般的なアドバイスを頂いた横浜国立大学工学部の森 辰則講師に感謝
します.
また,本研究には,文部省科学研究費重点領域研究「音声言語」により経済的サ
ポートを受けていることを記し,関係各位に感謝いたします.


\bibliographystyle{jtheapa}
\bibliography{jpaper}

\newpage
\begin{biography}
\biotitle{略歴}
\bioauthor{西沢 信一郎}{
1969 年生.1992 年横浜国立大学工学部卒業.1994 年横浜国立大学大学院工学
研究科博士課程前期修了.現在,横浜国立大学大学院工学研究科博士課程後期に
在学中.現在の主な研究テーマは自然言語処理.情報処理学会の学生会員.
}
\bioauthor{中川 裕志}{
1953 年生.1975 年東京大学工学部卒業.1980 年東京大学大学院修了.工学博
士.現在,横浜国立大学工学部電子情報工学科助教授.現在の主たる研究テーマ
は自然言語処理および日本語の語用論.日本認知科学会,人工知能学会などの会
員.
}

\bioreceived{受付}
\biorevised{再受付}
\bioaccepted{採録}

\end{biography}

\end{document}

