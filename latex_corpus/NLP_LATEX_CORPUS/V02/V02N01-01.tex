



\documentstyle[epsf,jtheapa]{jnlp_j_b5_old}


\setcounter{page}{3}
\setcounter{巻数}{2}
\setcounter{号数}{1}
\受付{1994}{2}{21}
\再受付{1994}{8}{26}
\採録{1994}{10}{11}

\setcounter{secnumdepth}{2}

\title{日英機械翻訳における利用者登録語の意味属性の自動推定}
\author{池原 悟\affiref{CS} \and 白井 諭\affiref{CS}
  \and 横尾 昭男\affiref{CS} \and Francis \sc Bond\affiref{CS} \and 
  小見 佳恵\affiref{NATC}}

\headauthor{池原・白井・横尾・{\sc Bond}・小見}
\headtitle{利用者登録語の意味属性の自動推定}

\affilabel{CS}{NTTコミュニケーション科学研究所}
{NTT Communication Science Laboratories}
\affilabel{NATC}{NTTアドバンステクノロジ}
{NTT Advanced Technology Corporation}

\jabstract{機械翻訳システムを使用して現実の文書を翻訳する場合, 通常, 
翻訳対象文書に合った利用者辞書が必要となる. 特に, 
高品質翻訳を狙った機械翻訳システムでは, 各単語に対して, 約2,000種
以上の分解精度を持つ単語意味属性の付与が必要であると言われており, 一般
の利用者が, このような精密な情報を付与するのは困難であった.  
そこで本論文では, 利用者が登録したい日本語名詞 (複合名詞を含む) と
英語訳語を与えるだけで, システムがシステム辞書の知識を応用して, 名詞
種別を自動的に判定し, それに応じた単語の意味属性を付与する
方法を提案する. 本方式を, 新聞記事102文とソフトウエア設計書105文
の翻訳に必要な利用者辞書作成に適用した結果, 自動推定方式では, 
専門家の付与した意味属性よりも多くの属性が付与されるが, 40〜80\%の
再現率が得られることが分かった. また, 人手で作成した利用者辞書を使用す
る場合と同等の訳文品質が得られることが分かった. 以上の結果, 利用者辞書
作成への単語の登録において, 最も熟練度の要求される単語意味属性付与作業
を自動化できる見通しとなった. }

\jkeywords{機械翻訳, 利用者登録語, 意味属性, 自動推定}



\etitle{Automatic Determination of Semantic Attributes\\
    for User Defined Words\\
    in Japanese to English Machine Translation}


\eauthor{Satoru Ikehara\affiref{CS} \and Satoshi Shirai\affiref{CS}
\and Akio Yokoo\affiref{CS} \and Francis Bond\affiref{CS} 
\and Yoshie Omi\affiref{NATC}} 

\eabstract{User dictionaries are important for practical machine
  translation.  However it is difficult for users to enter the
  detailed semantic attributes that a system may require. This paper
  proposes a method of automatically determining semantic attributes
  for noun pairs entered by users. The method compares the Japanese
  and English words with words already in the system dictionary. When
  this method was applied to words in two user dictionaries (for
  newspaper articles and software manuals) it generated more
  attributes than trained humans did, with a recall rate of 40--80\%.
  Evaluation showed that translations using the machine-made
  dictionary were similar in quality to translations using the
  human-made dictionary.  Thus automatic determination of semantic
  attributes removes the need for highly trained lexographers to make
  user dictionaries.}

\ekeywords{machine translation, user dictionary, semantic category,
  automatic determination}

\begin{document}
\maketitle



\section{はじめに}
\label{sec:hajime}

機械翻訳システムを使用する時, 利用者はシステム辞書に登録されていない
単語や, 登録されているが, 訳語が不適切な単語に対して, 利用者辞書を作成
して使用することが多い\cite{Carbonell:1992}. しかし, 辞書に新しく単語を登録する際は, 登
録する語の見出し語, 訳語の他に, 文法的, 意味的な種々の情報を付与する必
要がある. 高い翻訳品質を狙ったシステムほど, 利用者辞書にも詳細で正確な
情報を必要としており\cite{Ikehara:1993,Utsuro:1992}, 素人の利用者がそれらの情報を正しく付与する
のは簡単でない\footnote{単語意味属性を付与するには, 通常のシステムの意
味属性を理解していることが必要であるが, 一般の利用者には簡単でない. }. 
例えば, 日英機械翻訳システムALT-J/Eでは, 意味解析のため約3,000種の精密
な意味属性体系\footnote{単語の意味的用法を分類したもので, 
各要素となる名詞に着目した動詞の訳し分けにおいて, ほぼ必要十分といえる
意味属性分解能が約2,000種類であることを示し, 実際に名詞の意味属性
を3,000種に分類している. 詳細は\cite{Ikehara:1993}を参照のこと. }を持っており, 利用者辞書
の単語を登録する際は, 各単語にこの意味属性体系に従って意味的用法 (一
般に複数) を指定する必要がある\cite{Ikehara:1989b,Ikehara:1989a}. この作業は熟練を要し, 一般の利用
者には困難であるため, 従来から自動化への期待が大きかった. 

そこで本論文では, 利用者登録語の特性に着目し, 利用者が登録したい見出
し語 (単一名詞または複合名詞) に対して英語訳語を与えるだけで, システム
がシステム辞書の知識を応用して, 名詞種別を自動的に判定し, 名詞種別に応
じた単語の意味属性を推定して付与する方法を提案する. また, 自動推定した
利用者辞書を使用した翻訳実験によって, 方式の効果を確認する. 

具体的には, 名詞を対象に, 与えられた見出し語と訳語から主名詞と名詞種
別 (一般名詞, 固有名詞) を判定し, それぞれの場合に必要な単語意味属性
を自動推定する方法を示す. また, 適用実験では, まず, 本方式を, 新聞記
事102文とソフトウエア設計書105文の翻訳に必要な利用者辞書の作成に適用
して, 自動推定した単語意味属性と辞書専門家の付与した単語意味属性を比
較し, 精度の比較を行う. 次に, これらの意味属性が翻訳結果に与える影響
を調べるため,  (1) 意味属性のない利用者辞書を使用する場合,  (2) 自
動推定した意味属性を使用する場合,  (3) 専門家が利用者登録語の見出し
語と訳語を見て付与した意味属性を使用する場合,  (4) 正しい意味属性
 (専門家が翻訳実験により適切性を最終的に確認した意味属性) を使用した
場合, の4つの場合について翻訳実験を行う. 

\vspace{-0.2mm}
\section{システム辞書と利用者辞書}
\label{sec:dic}
\subsection{ALT-J/Eの意味辞書の構成}
\label{sec:2.1}
 ここでは, 機械翻訳システム側であらかじめ用意された辞書をシステム辞書, 
利用者が作成して使用する辞書を利用者辞書と呼ぶ. 日英機械翻訳システム
ALT-J/Eのシステム辞書と利用者辞書および単語意味属性の関係を図
1に示す.  
{\unitlength=1mm
\begin{figure}[htbp]
  \begin{center}
    
    
    
\epsfile{file=fig1.eps,width=134mm}
\vspace{-0.2mm}
  \end{center}
  \caption{ALT-J/Eの意味属性体系と意味辞書}
  \label{fig:1}
\end{figure}
}

\begin{description}
\vspace{-0.2mm}
\item [ (1) 意味辞書の種類]
ALT-J/Eでは意味解析を実現するため, これらの辞書に単語意味属性
を使用した意味情報が登録されるようになっている. 意味情報を記載した辞
書を意味辞書と呼ぶ. 現在, 実装されている意味辞書は単語意味辞書と, 構文
意味辞書の2種類からなる. 単語意味辞書は日本語単語の意味的用法を記述し
た辞書 (日本語解析用の40万語辞書と訳語決定用の38万語辞書) であり, 構
文意味辞書は, 用言毎の日本語文型とそれに対応する英語文型を収録した辞書
 (13,000文型) である. システムがあらかじめ用意したこれらの単語ま
たは文型が不足したとき, もしくは不適切なときは, 同種の辞書を利用者が利用
者辞書として作成して使用する.  

\item [ (2) 単語意味属性の種類]
ALT-J/Eの単語意味属性には一般名詞意味属性 (2,800種) , 固有名詞意
味属性 (200種) , 用言意味属性 (100種) の3種類がある. 固有名詞意
味属性は, 一般名詞意味属性の一部を取り出して, 複合語解析の観点から詳細
化したものであり, 属性名の数は一般名詞意味属性の数より少ないが, 分類精
度は詳細である.  

単語意味辞書の一般名詞には一般名詞意味属性 (一般に複数個) が, 固有名
詞には一般名詞意味属性と固有名詞意味属性の両者 (いずれも複数個) が付与
される. 用言意味属性は構文意味辞書に登録された文型パターンの主用言に付
与される\cite{Nakaiwa:1992}.  

\end{description}

\subsection{利用者登録語の特性}
\label{sec:2.2}

本論文では, 名詞 (一単語名詞または複合名詞) の利用者辞書への登録を考
える. 通常の機械翻訳システムでは, 一般語 (一般名詞) についてはほぼ漏れ
なくシステム辞書に収録されるが, 専門用語や固有名詞などは余り収録されて
いない場合が多い. ALT-J/Eの場合は, 新聞記事で使用される語を中心
に多数 (延べ50万語) の固有名詞, 専門用語なども収録されているが, 全て
を網羅することは不可能であり, 必ずしも十分とは言えない.  

従って, 通常, 利用者辞書に登録される語は,  (1) 原文に現れた専門用語
や固有名詞, 利用者固有の技術用語で, システム辞書に登録されていないため
未知語となった語, もしくは (2) システム辞書に登録されているが, 訳語が
適切でない語の2種類に大別される. 後者の単語意味属性は既にシステム辞書
に登録されているため, 通常改めて登録する必要はないのに対して, 前者は登
録語が複合名詞で, その構成要素の一部がシステム辞書に登録されていなかっ
たため未知語となったものが多い. このようにシステム辞書は, 多く
の場合, 利用者辞書登録語と関係する情報を持つ場合が多いので, その情報を
利用すれば, 多くの利用者登録語の意味属性は自動付与できると期待できる. 


\section{意味属性推定の方法}
\label{sec:3}

{\unitlength=1mm
\begin{figure}[htbp]
  \begin{center}
    
    
    
\epsfile{file=fig2.eps,width=55mm}
  \end{center}
  \caption{意味属性自動推定の手順}
  \label{fig:2}
\end{figure}
}

利用者登録語の日本語表記と英語訳語が与えられたとき, 機械翻訳システム
に装備されたシステム辞書の情報を使って, 登録語の意味属性を推定する方法
を図\ref{fig:2}に示す\footnote{意味属性は, あらかじめシステムで決められた体系を
使用する. その意味属性体系に不足や不適切な部分があっても, 本方式で修
正改良することは考えない. これは, 利用者辞書作成は通常システム運用時
に行なわれるものであり, この段階では, 意味属性体系の変更に伴って生じ
るシステム辞書や翻訳プログラムの修正は通常困難と考えられることからで
ある. }.  

利用者登録語の単語意味属性を推定する手順は, 主名詞の判定, 名詞種別
 (固有名詞, 一般名詞) の判定, 固有名詞意味属性の推定 (固有名詞の場合) , 
一般名詞意味属性の推定 (一般名詞, 固有名詞双方の場合) の手順からなる.  



\subsection{主名詞の判定方法}
\label{sec:3.1}

利用者辞書に登録される見出し語は, 単一の名詞もしくは複数単語から構成
される複合名詞のいずれかとし, 訳語は単一の単語, 名詞連続の
複合語, 名詞句のいずれかとする. 見出し語, 訳語を構成する単語のうち, 中
心的な意味を担う名詞を主名詞と呼ぶ.  

通常, 登録語の単語意味属性は主名詞の単語意味属性と一致することが多い
と考えられる. また, システム辞書の中に利用者辞書登録語の見出し語または
訳語の一致する語が存在する可能性に比べて, 利用者辞書登録語の主名詞が存
在する可能性は高い. 従って, 主名詞に着目すれば, 登録語の意味属性を推定
できる可能性が大きい.  

日本語名詞は語形変化しないため, システム辞書の見出し語と利用者登録語
の主名詞を含む部分とを直接比較し, システム辞書内から必要な情報を引き出
すことができる. これに対して, 英語名詞は複合語内などで屈折による語形変
化を伴うことがあるため, 主名詞を含む部分とシステム辞書の英語訳語を直接
比較することはできない. そこで, ここでは, システム辞書の訳語との比較が
可能となるよう, 利用者登録語の英語訳語に対して主名詞を抽出する.  

[英語主名詞の判定手順]
\begin{description}
\item[(1)] まず, 訳語が単語一語で構成されるときは, その語を主名詞とする. 
\item[(2)] 次に, 訳語が2語以上の語から構成されている場合は, まず, 訳語の全
体が, システム辞書に登録されているか否かを調べ, 登録されている場合は, 
訳語全体を主名詞とする. 

\item[(3)] 登録されていない場合は, 名詞句 (訳語) を構成する単語の中から主名
詞を推定する. この場合, 英語訳語は名詞連続複合語または修飾語や句を伴っ
た名詞句で構成されていると考えられる. 前者の場合は, 最後の名詞が主
名詞になるのに対して, 後者の場合では, 修飾語句は主名詞の前方だけで
なく後方に来る場合のあることを考慮する必要がある. 通常, 後方修飾は
前置詞, 関係詞で導かれることを考慮して, 以下の方法で主名詞を選定す
る.  \begin{itemize}
\item 訳語中にin, on, withなどの前置詞, またはthat, which   などの関係詞
 (ストップワード) があるか否かを調べ, ある場合は, それの語以下の語を削
除する.  
\item 次に, 残った英語全体に対して英語辞書引きを行い, 辞書内に一致する語が
あれば, それを主名詞とする. 
\item 一致する語のないときは, 前方から一語ずつ落としながら (修飾語を外しな
がら) , 残った語に対して英語辞書引きを行い, 辞書と一致した語 (ま
たは語の組) を主名詞とする. 外せる修飾語がなくなったときは, 残っ
た語を主名詞とする. 
\end{itemize}
\end{description}


\subsection{名詞種別の判定方法}
\label{sec:3.2}

前に述べたように, 一般名詞には, 一般名詞意味属性を付与すればよいのに
対して, 固有名詞には一般名詞意味属性と固有名詞意味属性の両方を付与する
ことが必要である. そのため, 利用者登録語が固有名詞か一般名詞かの判定を
行う必要がある. この判定は, 利用者にとって比較的容易な作業であるが, 利
用者の負担を少しでも削減することを狙って, 自動化の方法を考える. 

日本語表現では, 一般名詞と固有名詞は通常, 表記上区別されないのに対し
て, 英語表現では, 固有名詞の先頭文字は大文字で書かれる点に特徴がある. 
そこで, 登録された単語の英語側の表記に着目し, 訳語が1単語のときは, 先
頭文字1文字が大文字の場合は固有名詞とし, それ以外は一般名詞とする. 
複数の単語から構成される訳語のときは, 各単語の先頭1文字が大文字の場合
は, 固有名詞とする. 訳語にすべて大文字からなる単語が含まれる場合は, 
それ以外の単語がすべて固有名詞と判定されるときは全体を固有名詞とし, そ
れ以外は一般名詞とする. 


\subsection{意味属性推定の方法}
\label{sec:3.3}

利用者登録語の見出し語, 訳語, 訳語の主名詞と, システムに既に準備され
ている日英対照辞書の内容を比較して, 利用者登録語の単語意味属性を推定す
る. 利用者登録語が一般名詞の場合は, 日英対照辞書に登録された一般名詞を
検索の対象として, 一般名詞意味属性を推定するのに対して, 利用者登録語が
固有名詞の場合は, 日英対照辞書に登録された固有名詞を検索の対象として, 
一般名詞意味属性と固有名詞意味属性を推定する.  \vspace{-0.1mm}

以下, 利用者登録語の見出し語から意味属性を推定する方法と訳語から推定
する方法を示すが, これらの手順は一般名詞意味属性の場合と固有名詞意味属
性の場合に共通である.  \vspace{-0.1mm}

また, 意味属性をなるべく漏れなく抽出するため, 見出し語と訳語のそれぞれ
に対して下記の手順を適用する. なお, 事項の順序は任意である. 
\vspace{-0.2mm}
\subsubsection{見出し語 (日本語表記) から推定する方法}
\vspace{-0.2mm}
日英対照辞書を検索し, 利用者登録語の見出し語と一致する見出し語が日英
対照辞書の登録語にある場合は, 既に登録された訳語が適切でないため, 訳
語を変えるのが利用者辞書登録の目的である場合が多いと考えられるから, 
単語意味属性の変更はしないものとし, 日英対照辞書に記載された単語意味
属性を利用者登録語の単語意味属性とする. \vspace{-0.1mm}

利用者登録語の見出し語と一致する見出し語が日英対照辞書の登録語にない
場合は, 利用者登録語の後方からの最長一致法で, 再度, 日英対照辞書を検索
する. カタカナ語を除き, 2文字以上が, 日英対照辞書の見出し語と部分一致
 (カタカナ語の場合は単語単位で一致) すれば, 日英対照辞書の意味属性を利
用者登録語の意味属性とする. \vspace{-0.1mm}

例えば表\ref{tab:1}で, 利用者登録語の「治療」, 「放射線治療」は, 
システム辞書 (表\ref{tab:2})に「治療」があるので, 意味属性は《治療》となる. \vspace{-0.1mm}

\begin{table}
\hspace{-0.5cm}
  \begin{minipage}{8.8cm}
  \caption{利用者辞書の例}
  \label{tab:1}
  \begin{center}
    \leavevmode
    \begin{tabular}{|p{2.3cm}|p{2.3cm}|l|}
      \hline
      \multicolumn{1}{|c|}{日本語見出し} & \multicolumn{1}{|c|}{英語訳} & 単語意味属性 \\
      \multicolumn{1}{|c|}{ (利用者付与) } & \multicolumn{1}{|c|}{ (利用者付与) } &  (推定結果)  \\\hline
      治療 & cure & 《治療》 \\ \hline
      放射治療 & radiotherapy & 《治療》 \\ \hline
      手当て & {\it treatment\/} & 《治療》 \\ \hline
      医療 & medical {\it treatment\mbox{}\/} & 《治療》 \\ \hline
      数値\bf 制御ロボット & numerically  {\it controlled robot\/} & 《産業機器》 \\ \hline
      照明付き\bf 机 & {\it desk\/} with light unit & 《家具》 \\ \hline
    \end{tabular}
  \end{center}
  \end{minipage}
  \begin{minipage}{5.3cm}
  \caption{システム辞書の例\\ (日英対照辞書) }
  \label{tab:2}
  \begin{center}
    \leavevmode
    \begin{tabular}{|l|p{1.2cm}|l|}
      \hline
      日本語& \multicolumn{1}{|c|}{英語訳} & 単語意味属性 \\
      見出し & & \\ \hline
      治療 & treatment & 《治療》 \\ \hline
      制御 & controlled & 《産業機器》 \\ 
      ロボット & robot & \\ \hline
      机 & desk & 《家具》 \\\hline
    \end{tabular}
  \end{center}
\end{minipage}
\end{table}



\subsubsection{訳語 (英語表記) から推定する方法}

日英対照辞書の訳語の中に, 利用者登録語の訳語と一致する語がある場合は, 
その訳語に対応する見出し語は, 利用者登録語の見出し語と同義語の場合が多
いと考えられるので, 日英対照辞書に登録された意味属性を, そのまま利用者
登録語の意味属性とする.  

利用者登録語の訳語と一致する訳語が日英対照辞書の登録語にない場合は, 
 (1) の場合と同様, 再度, 日英対照辞書を検索する. その中に, 利用者登録
語の主名詞もしくは主名詞を含む訳語部分が, 日英対照辞書の訳語にあれば, 
日英対照辞書の意味属性を利用者登録語の意味属性とする. 但し, 利用者登録
語と日英対照辞書の訳語が同一の主名詞を持つ場合でも, 語形が異なる場合が
あるので, 主名詞は可能な語形変化 (単数複数など) をさせながら, 照合を行
う\footnote{具体的には, まず, 抽出された主名詞のシステム辞書内での有無
を調べ, それが発見されないときに, 主名詞を語形変化させ再度, システム
辞書を検索する. これにより, 語形変化によって意味の変わる単語の場合な
どで, システム辞書内から, なるべくもとの語形と一致する単語が抽出され
る. }.  

例えば, 表\ref{tab:1}で, 利用者登録語の「手当」, 「医療」は, その訳語 (または
主名詞訳語) 「treatment」がシステム辞書 (表\ref{tab:2}) にあるので, 意味属性は
《治療》となる.  

以上の方法では, システム辞書には一般に複数の意味属性が付与されている
こと, 日本語表記だけでなく英語表記からも意味属性が抽出されるため, 一般
に一語に対して複数の意味属性が抽出されることになる. 利用者辞書は特定の
翻訳対象に対して指定して使用されるため, 用語の用法が限られる特徴がある. 
従って, 実際の用法が意味属性として与えられていれば, それ以外の用法が多
少付与されていても, 副作用は少ないと期待される. そこで, 意味属性として
は, 得られた意味属性すべてを登録する. 但し, 同一の単語意味属性が重複し
て抽出された場合は, 重複を取って登録する.  


\section{意味属性推定精度の評価}
\label{sec:4}


\subsection{実験の条件}
\label{sec:4.1}

表\ref{tab:3}に示すような新聞記事文とソフトウエア設計書の日本文に対して前章の
方法を適用し, 自動推定の精度を評価する. 具体的には, 以下の3つの場合に
分けて, 得られた単語意味属性を比較評価する.  
\begin{description}
\item[(1)自動推定方式による場合] 
与えられた見出し語, 訳語のペアに対して, 前章の方法で単語意味属性を
付与する.  
\item [(2)人手付与方式の場合]
意味属性体系に精通した辞書担当のアナリストが, 与えられた見出し
語, 訳語のペアを見て, 単語意味属性を付与する.  
\item [(3)最適意味属性の場合]
(2)で作成した利用者辞書を使用して対象文の翻訳実験を行い, その結果
を見て意味属性の修正追加を行う. 最終的に翻訳結果が最適となるま
でこの作業を繰り返して, 意味属性を定める. この方法で得られた意
味属性を, 最適値と仮定する.  
\end{description}

\begin{table}[htbp]
  \caption{実験対象文の特性}
  \label{tab:3}
  \begin{center}
    \leavevmode
    \begin{tabular}{|c|c|c|c|}
      \hline
      \multicolumn{2}{|c|}{項目} & 新聞記事 & ソフト設計書 \\ \hline
      \multicolumn{2}{|c|}{対象文数 (文) }  & 102文 & 105文 \\ \hline
      \multicolumn{2}{|c|}{平均文字数 (文字/文) }  & 42文字 & 40文字 \\ \hline
      \multicolumn{2}{|c|}{平均単語数 (単語/文) }  & 21.2単語 & 16.0単語 \\ \hline
       \raisebox{-6pt}{利用者辞書} & 一般名詞 & 28語 & 98語 \\ \cline{2-4}
       \raisebox{-6pt}{登録語数} & 固有名詞 & 49語 & 7語 \\ \cline{2-4}
      & 合計 & 77語 & 106語 \\ \hline
      \multicolumn{2}{|c|}{利用者登録語を含む文数}  & 53文 & 93文 \\ \hline
    \end{tabular}
  \end{center}
\end{table}


\subsection{名詞種別自動判定精度}
\label{sec:4.2}

前章の3種類の意味属性付与方式で得られた名詞種別の判定精度を表\ref{tab:4}に示
す.  

\begin{table}[htbp]
  \caption{名詞種別の判定結果}
  \label{tab:4}
  \begin{center}
    \leavevmode
    \begin{tabular}{|c|c|c|c|c|c|c|}
      \hline
      標本 & \multicolumn{3}{|c|}{新聞記事} & \multicolumn{3}{|c|}{ソフトウエア設計書} \\\hline
      属性種別 & 自動判定 & 人手判定 & 最適解 & 自動判定 & 人手判定 & 最適解 \\\hline
      一般名詞 & 31語 & 30語 & 28語 & 93語 & 99語 & 98語 \\\cline{2-7}
      意味属性 & (27語) & (27語) & & (90語) & (97語) & \\\hline
      固有名詞 & 46語 & 47語 & 49語 & 12語 & 6語 & 7語 \\\cline{2-7}
      意味属性 & (45語) & (46語) & & (4語) & (5語) &  \\\hline
      \raisebox{-6pt}{合計} & 77語 & 77語 & 77語 & 105語 & 105語 & 105語 \\\cline{2-7}
      & (72語) & (73語) & & (94語) & (102語) & \\\hline
      判定正解率 & 93.5\% & 94.8\% & 100\% & 89.5\% & 97.1\% & 100\% \\\hline
    \end{tabular}

    \vspace{1.5mm} ()内の数は, 正しい判定の数を示す. 
  \end{center}
\end{table}


新聞記事の場合, 自動判定方式では, 利用者登録語全体77語のうち, 判定
の正しかった名詞は一般名詞27語, 固有名詞45語の合計72語で, 正解率
は93.5\%であった. 人手付与方式では, 一般名詞27語, 固有名詞語46
語を正しく判定し, 正解率は94.8\%であった. これに対して, 設計書の場
合は, 自動判定法の正解率89.5\%, 人手付与方式の正解率は97.1\%で
あった. 

自動判定で, 一般名詞を誤って固有名詞と判定した語は,「郵政大臣」,
「中部圏」,「GE」,「IGS」, 「汎用GS」などであった. 逆に, 固有
名詞を誤って一般名詞と判定したのは,「PC9800」,「VOS3.2」,
「X.25プロトコル」などであった. 

以上から, 文書の種類によって多少の差はあるが, 自動判定方式で人手判定
方式と大差のない結果が得られることがわかった.  

判定に失敗した約10\%の名詞について考えると, 固有名詞には固有名詞意
味属性のほかに一般名詞意味属性も付与することになっているため, 一般名詞
を固有名詞と判定した語 (新聞記事1語, 設計書語8語) の場合は, 一般名詞
意味属性も付与されることになり, 訳文品質への影響は殆どないと期待される. 
しかし, 逆に, 固有名詞を一般名詞と判定した語 (新聞記事4語, 設計書3語) 
には, 固有名詞意味属性が付与されないので, その語が複合語構成要素として
使用された場合, 影響がでると考えられる.  


\subsection{意味属性自動推定精度}
\label{sec:4.3}



\begin{table}[htbp]
  \caption{単語別にみた単語意味属性の自動付与品質}
  \label{tab:5}
  \begin{center}
    \leavevmode
    \begin{tabular}{|l|l|c|c|c|c|c|c|}
      \hline
      \multicolumn{4}{|r|}{属性種別} & \multicolumn{2}{|c|}{新聞記事} &  \multicolumn{2}{|c|}{ソフトウエア設計書}  \\  \hline
       \multicolumn{4}{|r|}{} & 一般名詞 & 固有名詞 & 一般名詞 & 固有名詞 \\ 
       \multicolumn{4}{|c|}{比較項目} & 意味属性 & 意味属性 & 意味属性 & 意味属性 \\ \hline
       \multicolumn{4}{|c|}{属性付与の必要な語数} & 77語(100\%) & 49語(100\%) & 105語(100\%) & 7語(100\%) \\ \cline{2-8}
       \hspace{4mm} & \multicolumn{2}{|c|}{属性が付与された} & 自動 & 73語(94.8\%) & 47語(95.9\%) & 88語(83.8\%) & 3語(42.9\%) \\
       & \multicolumn{2}{|c|}{語数の合計} & 人手 & 77語(100\%) & 47語(95.9\%) & 100語(95.2\%) & 5語(71.4\%) \\ \cline{2-8}
       & \hspace{4mm} & そのうち & 自動 & 38語(49.4\%) & 42語(85.7\%) & 3語(2.9\%) & 2語(28.6\%) \\
       & & 全属性が正解 & 人手 & 44語(57.1\%) & 42語(85.7\%) & 50語(47.6\%) & 1語(14.3\%) \\ \cline{3-8}
       & &  そのうち & 自動 & 21語(27.3\%) & 0語(0.0\%) & 41語(39.2\%) & 0語(0.0\%) \\
       & & 余分に付与 & 人手 & 9語(11.7\%) & 0語(0.0\%) & 11語(10.5\%) & 1語(14.3\%) \\  \cline{3-8}
       & & そのうち & 自動 & 4語(5.2\%) & 0語(0.0\%) & 18語(17.1\%) & 0語(0.0\%) \\
       & & 一部付与不足 & 人手 & 8語(10.4\%) & 0語(0.0\%) & 27語(25.7\%) & 2語(28.6\%) \\ \cline{3-8}
       & & そのうち & 自動 & 10語(13.0\%) & 5語(10.2\%) & 26語(24.8\%) & 1語(14.3\%) \\
       & & 全てが誤り & 人手 & 16語(20.8\%) & 4語(8.2\%) & 12語(11.4\%) & 1語(14.3\%) \\ \cline{2-8}
       & \multicolumn{2}{|c|}{自動付与され} & 自動 & 4語(5.2\%) & 2語(4.3\%) & 17語(16.2\%) & 4語(57.1\%) \\
       & \multicolumn{2}{|c|}{なかった語数} & 人手 & 0語(0.0\%) & 2語(4.3\%) & 5語(4.8\%) & 2語(28.6\%) \\ \hline
       \multicolumn{4}{|c|}{属性付与の必要ない語数} & 0語 & 28語 & 0語 & 98語 \\ \cline{2-8}
       & \multicolumn{2}{|c|}{\raisebox{-6pt}[0pt][0pt]{属性付与された語数}} & 自動 & 0語 & 1語 & 0語 & 8語(8.2\%) \\
       & \multicolumn{2}{|c|}{} & 人手 & 0語 & 1語 & 0語 & 1語(0.1\%) \\ \hline
    \end{tabular}
\vspace{-3mm} 
  \end{center}
\end{table}

\begin{table}[htbp]
  \caption{属性数から見た自動推定と人手付与の比較}
  \label{tab:6}
\vspace{-2mm} 
  \begin{center}
    \leavevmode
    \begin{tabular}{|c|c|c|c|c|c|c|}
      \hline
      標本 & \multicolumn{3}{|c|}{新聞記事} & \multicolumn{3}{|c|}{ソフトウエア設計書} \\ \hline
      属性付与 & 自動推定 & 人手付与 & 最適解の & 自動推定 & 人手付与 & 最適解の \\
      の方法 & で付与し & の属性数 & 属性数 & で付与し & の属性数 & 属性数 \\ \cline{1-1}
      属性の種類 & た属性数 & & & た属性数 & & \\ \hline
      一般名詞意味属性 & 194件 & 110件 & 127件 & 341件 & 130件 & 191件 \\
      & 74+21件 & 93+17件 & -- & 67+20件 & 74+17件 & -- \\ \hline
      固有名詞意味属性 & 46件 & 43件 & 48件 & 12件 & 7件 & 7件 \\
      & 42+0件 & 42+1件 & -- & 2+0件 & 1+2件 & -- \\ \hline
      合計 & 240件 & 153件 & 175件 & 353件 & 137件 & 198件 \\
      & 116+21件 & 135+18件 & -- & 63+22件 & 75+19件 & -- \\ \hline
    \end{tabular}

    \vspace{1.3mm} 下段の数字の説明:nnn+mmm\\
    nnn=付与された属性の内, 最適解と一致する属性の数\\
    mmm=自動付与された属性が最適値の近傍 (上位または下位) にあるものの数を示す. 
\vspace{-2mm} 
  \end{center}
\end{table}
 
\begin{table}[htbp]
  \caption{自動付与した意味属性の適合率と再現率}
  \label{tab:7}
  \begin{center}
    \leavevmode
    \begin{tabular}{|c|c|c|c|c|c|}
      \hline
      \multicolumn{2}{|c|}{標本} & \multicolumn{2}{|c|}{新聞記事} & \multicolumn{2}{|c|}{ソフトウエア設計書} \\\hline
      \multicolumn{2}{|c|}{意味属性種別} & 適合率 & 再現率 & 適合率 & 再現率 \\\hline
      一般名詞 & 自動付与方式 & 38.1\% & 58.3\% & 19.6\% & 35.1\% \\
      意味属性 & & (49.0\%) & (74.8\%) & (25.5\%) & (45.5\%) \\\cline{2-6}
      & 人手付与方式 & 84.5\% & 73.2\% & 56.9\% & 38.7\% \\
      & & (100\%) & (86.6\%) & (70.0\%) & (47.6\%) \\\hline
      固有名詞 & 自動付与方式 & 91.3\% & 87.5\% & 16.7\% & 28.6\% \\
      意味属性 & & (同上) & (同上) & (同上) & (同上) \\\cline{2-6}
      & 人手付与方式 & 97.6\% & 87.5\% & 14.3\% & 14.3\% \\
      & & (100\%) & (89.6\%) & (42.9\%) & (42.9\%) \\\hline
      全体 & 自動付与方式 & 48.3\% & 66.3\% & 17.8\% & 31.8\% \\
      & & (57.5\%) & (78.3\%) & (24.0\%) & (42.9\%) \\\cline{2-6}
      & 人手付与方式 & 88.2\% & 77.1\% & 54.7\% & 37.9\% \\
      & & (100\%) & (87.4\%) & (68.6\%) & (47.5\%) \\ \hline
    \end{tabular}

    \vspace{1.5mm} ()内の数字は, 最適意味属性の近傍 (上位下位) も正解とした場合を示す. 
  \end{center}
\end{table}

単語別にみたときの自動推定とアナリスト付与の結果を表\ref{tab:5}, 付与された意
味属性全体の数とその内訳を表\ref{tab:6}に示す. アナリストの付与した意味属性が正
解であると考えたときの適合率と再現率は, 表\ref{tab:6}から表\ref{tab:7}の通り求められる. 
これらより以下のことが分かる. \vspace{-0.2mm}

\begin{description}
\item[(1)] 意味属性自動推定のアルゴリズムは, システム辞書の情報を手がかりに働
くため, 利用者登録語の全てに意味属性が付与されるとは限らない. これ
に対して, 実験結果では, 意味属性付与の必要な単語延べ238語に対し
て, 意味属性が自動推定された語数は211語であり, その割合 (88.7\%) 
は大きい. これは利用者登録語に関連する語の情報が, 既にシステ
ム辞書に豊富に存在することを示している.  \vspace*{-0.2mm}
\item[(2)] 単語毎に見たとき, 正解以外の余分の意味属性が付与された語も多いため, 
適合率はあまり高くないが, 再現率を見ると, 新聞記事の場合は8割近く, 
設計書の場合は約4割を得ている. 従って, 3. 3節
に述べた理由から自動推定の効果は十分あると予測される. 
\item[(3)] ソフトウエア設計書の場合, 固有名詞の意味属性の精度かなり低い. しか
し, この場合, 固有名詞の数は少数であること, 固有名詞でも一般名詞意
味属性は付与されることから, 訳文品質への影響は少ないと思われる. 
\end{description}


\section{訳文品質の向上効果}
\label{sec:5}

\subsection{実験の条件}
\label{sec:5.1}

利用者登録語に対する意味属性自動推定の効果を調べるため, 前章と同一の
試験文 (新聞記事102文, ソフトウエア設計書105文) を対象に, 前章で
得られた利用者辞書を用いて, 翻訳実験を行った. 実験は以下の4つの場合に
分けて実施した.  
\begin{description}
\item[場合1] 単語意味属性の付与されない利用者辞書を使用した場合
\item[場合2] 自動推定方式により付与した意味属性を使用した場合
\item[場合3] 人手付与方式により付与した意味属性を使用した場合
\item[場合4] 最適意味属性を使用した場合
\end{description}


\subsection{実験結果}
\label{sec:5.2}

\begin{table}[htbp]
  \caption{訳文品質の比較評価}
  \label{tab:8}
  \begin{center}
    \leavevmode
    \begin{tabular}{|c|c|c|c|c|c|}
      \hline
      \multicolumn{2}{|c|}{訳文品質の比較} & \multicolumn{2}{|c|}{新聞記事} & \multicolumn{2}{|c|}{ソフトウエア設計書} \\ \hline
      \multicolumn{2}{|c|}{意味属性付与の方法} & 訳文合格率 & 品質変化* & 訳文合格率 & 品質変化* \\  \hline
      場合1 & 意味属性付与無し & 56.7\% & 0.0\% & 65.7\% & 0.0\% \\\hline
      場合2 & 自動推定方式 & 69.6\% & +16.7\% & 71.4\% & +10.5\% \\\hline
      場合3 & 人手付与方式 & 71.5\% & +21.6\% & 71.4\% & +15.2\% \\\hline
      場合4 & 最適意味属性 & 72.5\% & +25.5\% & 73.3\% & +23.8\% \\\hline
    \end{tabular}

    \vspace{1.5mm} * 10点満点評価で1点以上, 訳文品質に変化のあった文の割合を示す. \vspace{-1mm} 
  \end{center}
\end{table}


上記の4つの場合の翻訳結果を表\ref{tab:8}に示す. この表より以下のこ
とが分かる.  

\begin{description}
\item[(1)] 自動推定された単語意味属性を使用した場合, 意味属性を付与しなかった場                            
合に比べて, 訳文合格率は, 新聞記事の場合約10\%, ソフトウエア設計書                            
の場合約6\%向上した.                                                                             
\item[(2)] これらの値は, いずれも, 人手付与方式によって得られる効果と
大差ない値である.  
\item[(3)] 最適意味属性を使用した場合は, 人手付与方式よりさらに1〜3\%高い
訳文品質向上率が得られている.  
\end{description}


\subsection{考察}
\label{sec:5.3}

\subsubsection{訳文品質向上効果について}

最適意味属性を決定する繰り返し実験のコストを考えると, 上記で得られ
た結果は, 十分満足できる値である. 経験的に言って, 機械システムの改良に
より10\%の翻訳率向上を得ることは容易ではない. 機械翻訳の実用レベルの
品質は70〜80\%以上と考えられるから, 訳文品質が50〜60\%の現状の
システムでは, 10\%前後の翻訳率の向上は大きな効果といえる. 

\subsubsection{対象文による効果の違い}
新聞記事文の場合に比べて, ソフトウエア設計書の場合は, 訳文品質向上効
果が少ない. この理由は以下の通りと考えられる. すなわち, 新聞記事文では, 
一般語を組み合わせた複合語が利用者辞書登録語となる場合が多く, 主名詞が, 
既にシステム辞書に登録されていることが多いため, 必要な意味属性が付与さ
れやすい. これに対して, ソフトウエア設計書では, 意味不明な英字略語やカ
タカナ語の登録が多く, システム辞書から適切な意味属性を抽出するのが困難
な場合が多い.  

しかし, 後者の場合は, 人手付与の場合も, 適切な意味属性付与は簡単とは
言えず, 意味属性付与の効果は, 前者に比べて少ないことを考えると, 両者の
実験から, 本方式では, 人手付与に近い効果が得られたと言える.  

\subsubsection{方式の有効範囲について}
本実験では, 3,000種の意味属性を使用したが, 本方式は意味属性の数
によらず適用可能である. 方式の適用性は, システム辞書の充実性に依存する
点が大きいと考えられる. 特に, 一般語に関する見出し語の網羅性が保証され, 
登録語に対してそのシステムで定められた意味属性が漏れなく付与されているこ
とが大切と思われる.  

但し, 意味属性付与の効果は, 意味属性体系自体の構成概念 (何を狙ってど
んな方針で体系化するか) や分類精度\footnote{\citeA{Ikehara:1993}におい
て, 日英機械翻訳では, 格フレームを使用して動詞を訳し分ける (一部の動
詞を除く) には, 格要素の意味マーカをおおよそ2,000種類程度に分類
すれば良いことが報告されている. 従って, 3,000通りの分類を用いた
本実験は, 動詞の意味による訳し分けの点から見て, 意味属性分類能の必
要十分と見られる領域での実験と考えらる. } (どれだけ細かく分類するか) , 品質
などにも強く依存しており, 使用する意味属性体系が異なれば, 意味属性付与
の効果そのものが本実験の場合と異なることになる. しかし, 本実験の結果から, 
自動付与方式では, システム辞書が充実していれば, 人手付与の場合に近い効
果が得られることが期待される.  

\subsubsection{その他}

新聞記事の場合, 自動推定方式で訳文品質を向上できなかった3文を見ると, 
その原因は, 名詞種別の判定誤りが1件, 正解の意味属性の上位または下位の
属性を選択したものが, それぞれ1件であった. 本方式では, 名詞の種別も自
動判定しているが, 誤りの例から見て, 名詞種別と意味属性の単純な分類 (上
位2〜3段程度) を利用者に依頼することができれば, これらの誤りは, ほぼ
防ぐことができると推定される.  

以上の結果, 従来, 利用者が利用者辞書を作成する際, 最も熟練の必要な単語
意味属性の付与作業を自動化できる展望が得られた.  


\section{あとがき}
\label{sec:6}


利用者辞書に登録する利用者登録語の見出し語 (日本語) と訳語 (英語) が
与えられたとき, 機械翻訳システムに既に存在する情報を利用して, その単語
意味属性を自動的に推定する方法を提案した. また, 本方式を新聞記事102
文, ソフトウエア設計書105文の翻訳に必要な利用者辞書の作成に適用し, 
推定された単語意味属性の精度, 最終的な翻訳結果に与える影響などを評価し
た.  

その結果, 自動推定された単語意味属性は, 専門家が実験の繰り返しによっ
て決定した意味属性 (最適意味属性) の40〜80\%を再現していることが
分かった. この値は, 専門家が自動推定と同一の条件で人手付与方式により付
与した意味属性の再現率 (50〜90\%) よりは若干 (〜10\%) 低いが, 
十分効果の期待できる値である.  

また, 自動推定された単語意味属性を使用した翻訳実験では, 意味属性を付
与しなかった場合に比べて, 訳文合格率は6〜13\%向上し, 人手付与方式の
場合と同等の品質が得られることが分かった. この品質は, 最適意味属性を使
用した場合に比べても, 2〜3\%しか低下しない値であり, 最適意味属性を決
定する繰り返し実験のコストを考えると, 十分満足できる値である.  

これらの結果, 従来, 利用者が利用者辞書を作成する際, 最も熟練の必要な
単語意味属性の付与作業を自動化できる展望が得られた. 今後は, 対訳コーパ
スから, 利用者辞書登録の必要な単語の見出し語と訳語を自動抽出し, 利用者
辞書全体を自動生成する方法について研究を進める予定である.  


\bibliographystyle{jtheapa}
\bibliography{/home/ecl/bond/Bib/ling}

\begin{biography}
\biotitle{略歴}
\bioauthor{池原 悟}{
1967年大阪大学基礎工学部電気工学科卒業. 1969年同大学大学院修士課程終了. 
同年日本電信電話公社に入社. 以来, 電気通信研究所において数式処理, トラ
ヒック理論, 自然言語処理の研究に従事. 現在, NTTコミュニケーション
科学研究所池原研究グループ・リーダ (主幹研究員) . 工学博士. 1982年情報
処理学会論文賞, 1993年情報処理学会研究賞受賞. 電子情報通信学会, 情報処
理学会, 人工知能学会, 各会員. }

\bioauthor{白井 諭}{
1978年大阪大学工学部通信工学科卒業. 1980年同大学院博士前期課程修了. 同
年日本電信電話公社入社. 現在, NTTコミュニケーション科学研究所主任
研究員. 日英機械翻訳を中心とする自然言語処理の研究に従事. 電子情報通
信学会, 情報処理学会, 各会員. }

\bioauthor{横尾 昭男}{
1980年電気通信大学電気通信学部電子計算機学科卒業. 1982年同大学院電子計
算機学専攻修士課程終了. 同年日本電信電話公社に入社. 現在, NTTコミュ
ニケーション科学研究所勤務. この間, 自然言語処理の研究に従事. 現在, 日
英機械翻訳システムにおける日英構造変換処理や翻訳辞書の研究に従事. 情報
処理学会, 電子情報通信学会, 人工知能学会, ACL, 各会員. }

\bioauthor{Francis Bond}{Francis Bond received a B.A. in Japanese and
  mathematics from the University of Queensland in 1988 followed by a
  B.Eng in electrical systems engineering in 1990.  He joined NTT in
  1991 and is currently researching machine translation in the NTT
  Communication Science Laboratories.  He is a member of ALS, IEEE, 
  IPSJ and NLP.  }

\bioauthor{小見 佳恵}{
1977年鶴見大学文学部日本文学科卒業. 1988年NTT技術移転株式会社 (現・NTT
アドバンステクノロジ株式会社) 入社. 現在, 情報技術部担当課長. 日
英機械翻訳システムを中心に自然言語処理における言語データベースの構築, 
言語現象の研究に従事. }

\bioreceived{受付}
\biorevised{再受付}
\bioaccepted{採録}

\end{biography}

\end{document}




