\documentstyle[epsf,nlpbbl]{jnlp_e_b5}
\setcounter{page}{17}
\setcounter{巻数}{5}
\setcounter{号数}{4}
\setcounter{年}{1998}
\setcounter{月}{10}
\受付{October}{30}{1997}
\再受付{April}{2}{1998}
\再々受付{June}{18}{1998}
\採録{July}{10}{1998}

\setcounter{secnumdepth}{2}

\title{}
\author{}
\jkeywords{}

\etitle{The Application of Classification Trees to\\
 Bunsetsu Segmentation of Japanese Sentences}

\eauthor{Yujie Zhang\affiref{ElecUniv}   \and
         Kazuhiko Ozeki\affiref{ElecUniv} }

\headauthor{Zhang and Ozeki}
\headtitle{The Application of Classification Trees to Bunsetsu Segmentation}

\affilabel{ElecUniv}
          {The University of Electro-Communications,Department of Computer Science and Information}
          {The University of Electro-Communications,Department of Computer Science and Information}

\eabstract{
In conventional bunsetsu segmentation methods  for  Japanese sentences,
segmentation rules have been given manually. This causes
difficulties in maintaining the consistency of the rules,
and in deciding an efficient order of rule application.
This paper proposes a method of automatic bunsetsu segmentation
using a classification tree,
by which knowledge about bunsetsu boundaries is automatically
acquired from a corpus, and an  efficient order of rule application 
 is realized automatically.
 It can adapt quickly to a new system of parts of speech, and
also to a new task domain without
the need for changing the algorithm.
Generation of classification
trees for bunsetsu segmentation and  evaluation experiments  were
carried out on an ATR corpus and an EDR corpus.
The  segmentation  accuracy of 98.9\%   was achieved   for the ATR
corpus, and 96.2\% for the EDR corpus. The method was compared  with
a simple rule-based method and  the Bayes decision rule on the ATR  corpus. 
The proposed method outperformed  the 
rule-based method when the training data size was larger than about 20
sentences, and outperformed the Bayes decision rule over the whole range of
training data sizes.
The superiority of the proposed method   was more evident
over the former when the training data size was larger,  and  over the latter 
 when the training data size was  smaller.
}

\ekeywords{classification tree, machine learning,
linguistic knowledge, corpus, bunsetsu segmentation}

\begin{document}

\maketitle

\section{Introduction}
{\em Bunsetsu},  which is comprised of a content word
with or without being followed by a string of function words, is a
convenient
unit for dependency structure analysis of Japanese.
Since there are no spaces indicating bunsetsu boundaries in the
orthographic writing of Japanese, 
a sentence must be segmented into bunsetsu's prior to
dependency structure analysis.
According to the elementary definition of bunsetsu \,(Nagao Ed. 1984, p.12), 
such  segmentation might look simple.
There are, in reality, many factors that complicate the problem.
For example,\ Chinese characters can be concatenated to form a compound
word.
Some nouns and verbs have functions different from their original
ones. For example, the noun ``$yoru$''(``$night$'')    functions also as
 an adverb,  and the verb ``$aru$''(``$exist$'')  functions also as a 
 formal verb, but some systems of parts of speech do not
distinguish between common noun and temporal noun, or  between verb
and formal verb.
Also, there are many idiomatic usages of morpheme concatenations.
All these matters cause difficulties in detecting bunsetsu
boundaries.
Moreover, there is no system of parts of speech 
in Japanese
that has received a general consensus.
This situation gives rise to another obstacle to
establishing a standard method of bunsetsu segmentation.

There have been two approaches to the
bunsetsu segmentation problem: one based on an
automaton representing a definition of bunsetsu \,(Fujio and Matsumoto 1997), 
and the other based on a set of
handicraft rules\,(Suzuki 1996). In the former approach, one has to give a
definition of bunsetsu manually. The latter
involves handiwork in getting knowledge about
bunsetsu boundaries. Thus both approaches
have problems in keeping consistency, coverage and optimality
of manually obtained knowledge.
When  the system of parts of speech and/or the task domain are
changed, one has to repeat the whole manual process
to get new knowledge, which is rather laborious.

This paper proposes a method of bunsetsu segmentation using a
classification tree, by which knowledge about bunsetsu boundaries
is automatically acquired from a small amount of training data (about 30 
to 100 sentences)(Zhang and Ozeki\ \   1997). 
 It can adapt quickly to a new system of parts of speech, and
also to a new task domain without the need for changing the algorithm.
 
Section 2 reviews the classification tree technique briefly. Section 3 
gives an
account of how to apply the classification tree technique to the 
bunsetsu segmentation problem. Section 4 describes experiments on an
ATR corpus, including generation and evaluation of a classification
tree,
together with  comparisons of the proposed method with a  simple rule-based
method and the Bayes decision
rule. Section 5 is
devoted to the description of experiments on an EDR corpus with 
detailed error analysis on each leaf of the generated tree.
Section 6 summarizes the results, and suggests some directions for
further improvement.
\section{Classification Tree}
A classification tree is a binary tree that classifies objects
into classes\,(Breiman et al. 1984, Safavian and Landgrebe 1991). This technique has been well studied 
in such fields as pattern recognition and machine learning.
Through an automatic generation of a classification tree,
one can rapidly acquire underlying regularities in a large amount of
data,
which are difficult or even impossible for a human to capture by
intuition.
 
The process of classification by a tree is shown in Fig.1.
Associated with each non-terminal node of a classification tree is a 
{\it test}.
If an object passes a test, it reaches ``yes'' 
child-node; otherwise ``no'' child-node.
The process is repeated until the 
object reaches some terminal node or
{\em leaf},  which has been assigned to a class label.
Only two classes, $c_{1}$ and $c_{2}$, will be considered here.
\begin{center}
\epsfile{file=fig1.ps,width=250bp,height=330bp}
\end{center}
\begin{center}
\vspace{-30mm}
{\bf Fig.1}\ A classification tree for two classes $c_{1}$ and
$c_{2}$. An oval shows a non-terminal node, and a square a leaf. In
each non-terminal node, a test is performed on input objects. A leaf is assigned to a
class label.\vspace*{3mm}\\
\end{center}

In order to grow a classification tree from the  root, 
objects with class labels,
 or {\em training} objects, are necessary. Also a finite set
of  tests must be prepared based on  problem-specific knowledge. 
Suppose that a tree has been grown to  some size.
It consists of three kinds of nodes: non-terminal nodes,
leaves, and active nodes yet to be processed.
 Let $t$ be an active node,
 and $L(t)$ the number of training objects that reach  $t$,
among which  the number of objects belonging to $c_{i}$
is denoted as $L_{i}(t)$ ($i=1, 2$).
Then the impurity of $t$, {\em Gini} index\,(Gelfand, Ravishankar, and 
Delp 1991), 
is defined as
\begin{displaymath}
I(t)= \frac{L_{1}(t)}{L(t)}\cdot \frac{L_{2}(t)}{L(t)}.
\end{displaymath}
Because of the constraint $L_{1}(t)/L(t)
+L_{2}(t)/L(t)=1$, $I(t)$ attains the maximum value of
$1/4$
(=$1/2 \times 1/2$)  when
$L_{1}(t)/L(t)= L_{2}(t)/L(t)=1/2.$  

 If the impurity is lower than a prescribed threshold $I_0$, then
$t$ is decided to be a leaf. Otherwise all the tests
are tried exhaustively. 
By means of a test, the training objects that reach
$t$ are divided into those that reach the ``yes'' child-node $t_{yes}$
and those that reach the ``no'' child-node $t_{no}$. Let $L(t_{x})$
denote the number of training
objects that reach child-node $t_{x}$, where $x$ equals ``yes''
 or ``no''.  Then the test that maximizes the reduction  of  impurity
\begin{displaymath}
\bigtriangleup I(t)=
 I(t)-\frac{L(t_{yes})}{L(t)}I(t_{yes})-\frac{L(t_{no})}{L(t)}I(t_{no})
\end{displaymath}
is selected as the test associated with $t$,
 and new active nodes, $t_{yes}$ and $t_{no}$,
are appended under $t$.
If there is no test that reduces  the impurity of $t$,
 then $t$ is decided to be
a leaf. With the root as the initial active node,
the above procedure is recursively iterated until
all the active nodes are turned into non-terminal nodes or leaves. 
A leaf $t$ is assigned to class label $c_{i}$ 
if the majority of training objects
that reach $t$ belong to class $c_{i}$ ($i=1, 2$).

\section{Application of Classification Trees to Bunsetsu Segmentation}
By morphological analysis, a sentence is segmented into morphemes. 
The attribute values of
each morpheme such as the part of speech  and 
the orthographic expression are also obtained.
An object to be classified
here is a  pair of consecutive morphemes  (left morpheme and right
morpheme, henceforth) in a sentence with their
attribute values.
The purpose of classification is to decide whether the boundary
between two consecutive morphemes (boundary in focus, henceforth)  
is a bunsetsu boundary;
this is a classification problem for two classes.
A sentence consisting of $N$ morphemes yields $N-1$
objects.

Among the attributes of a morpheme, the part of speech is considered
to be
most important. In some cases, however, the part  of speech alone does
not provide enough information
for bunsetsu boundary detection.
Let us take ``$ni\ aru$'' and ``$te\ aru$'' as examples. The boundary
between ``$ni$'' and ``$aru$'' should be a bunsetsu boundary because 
this ``$aru$''  functions as a verb, whereas  the boundary
between ``$te$'' and ``$aru$'' should not be a bunsetsu boundary because
this  ``aru''  functions as a formal  verb. Some systems of
parts of speech, however, do not distinguish between verb and
formal verb. By such a system both of ``$ni\ aru$'' and ``$te\ aru$''
are tagged as ``particle verb'', which has no information 
about the distinction between the above two cases.
Therefore the orthographic
expression is employed as another test attribute for some range of
morphemes
that are selected by a preliminary experiment. Also the wild card
``$*$'' is introduced as a symbol to match any attribute values. 

Let $pos_{i}$ be a part of speech, and $W(pos_{i})$ the set of
orthographic expressions, including ``$*$'', of morphemes that belong
to $pos_{i}$ and are selected by the preliminary experiment.
Then the set of the pairs of $pos_{i}$ and its orthographic expressions
is
 denoted as $\{pos_{i}\}\times W(pos_{i})$. Let $S$ be the set of all such 
pairs
plus $(*,\,*)$:
\begin{displaymath}
S= [ \cup_{i} \{pos_{i}\}\times W(pos_{i})] \cup \{(*,\,*)\}.
\end{displaymath}
Then $S\times S$  is employed as the set of tests (classification rules)  in this work.
A test takes the form 
$\langle $ ($pos_{1}, e_{1}$ )($pos_{2}, *$) $\rangle $, for example.
An object will pass this test
if the part of speech of the left morpheme  
 equals $pos_{1}$, its orthographic
expression equals $e_{1}$, and the part of speech of the right
morpheme
equals $pos_{2}$.

In the process of growing a classification tree,
 an active node $t$ is decided to be a
leaf if the condition $I(t) \leq I_0$ is satisfied,
 or if there is no test that reduces the impurity
of $t$. Since the maximum value of the impurity is $1/4$ as stated in
the previous section,\ the
threshold $I_0$ is set at a value  between 0 and 1/4.
A threshold value  $I_0 \neq 0 $ allows  some amount of impurity
lower than $I_0$ to remain at leaves, thereby controlling  the over-growth of 
the classification tree.
\section{Experiments on  ATR Corpus}
By the procedure described above, classification trees
for bunsetsu segmentation  were generated and  evaluated 
on an ATR corpus\,(Abe et al. 1990).
The ATR corpus contains 503 sentences taken from
newspapers, magazines, and etc. The sentences 
are  labeled with the bunsetsu boundary, and the morphemes are labeled with 
the part of speech.
For the sake  of open experiments,  randomly selected 100 sentences  were
used  as  evaluation data, which  had 1298 objects,\ and the remaining
403 sentences as  training data, from which training data of different sizes
 were  defined as in Table 1.
\begin{center}
{\bf Table\ 1} Training data of different sizes.
\vspace{10pt}\\
\begin{tabular}{c|c|c|c|c|c|c|c|c|c|c}
\#Sentences&5&10&20&30&40&50&100&200&300&403\\ \hline
\#Objects& 90 &145&280 &411   &540   &673   & 1332 &2624   &4138   &5697 
\end{tabular}
\end{center}
The attribute used for test here was the part of speech only; 
the orthographic expression
was not used because the ATR corpus's part of speech system
distinguishes formal verbs and verbs. There are 25 parts of speech in the ATR corpus.
\begin{flushleft}
{\large \bf 4.1\ \ Optimum Value for the Threshold  $I_0$ }
\end{flushleft}
  The threshold $I_0$ controls the growth of a tree in the training
stage.
A smaller value of $I_0$ will produce more leaves, resulting in 
finer classification of the training objects. However, there is a
possibility
that too fine classification of the training objects causes over-fitting,
thereby giving an adverse effect on generalization power  of the
generated tree.
In order to decide the optimal value of  $I_0$, classification trees
were generated with various values of $I_0$  ranging from $10^{-5}$ to 
0.24, and 
for training data of different sizes: 5, 50, and 403 
sentences.  Table 2  shows the error rates of the generated trees for 
  the evaluation data.
\begin{center}
{\bf Table\ 2}\ \ Error rate\ (\%) vs. $I_0$ for training data of different sizes.
\vspace*{2mm}\\
\begin{tabular}{|c|c|c|c|c|c|c|c|c|}
\hline
\makebox[30mm]{Training Data}&\multicolumn{8}{c|}{$I_0$}\\ 
\cline{2-9}
\makebox[30mm]{Size(\#Sentences)}
&\makebox[9mm]{$10^{-5}$}&\makebox[9mm]{$10^{-4}$}&\makebox[9mm]{$10^{-3}$}&\makebox[9mm]{$10^{-2}$}&\makebox[9mm]{$10^{-1}$}&\makebox[9mm]{$0.15$}&\makebox[9mm]{$0.20$}&\makebox[9mm]{$0.24$}\\ \hline
   5   & 5.7  & 5.7  &5.7  & 5.7 & 11.8 & 14.4 & 23.4 & 23.4 \\ \hline
   50  & 2.3  & 2.3  & 2.3 & 3.4 & 8.3  & 12.1 & 23.4 & 23.4   \\ \hline
   403 & 1.1  & 1.1  & 1.2 & 2.0 & 8.3  & 12.1 & 14.7 & 23.4  \\ \hline
\end{tabular}
\vspace*{10mm}
\\
\end{center}

As far as this experiment shows, a  smaller value of $I_0$ gives  a
better result for any training data size.
Although we can not draw  a general  conclusion immediately from this
result that the
optimal value of $I_0$ is always 0, we  tentatively  set $I_0$ at 0 in the
following experiments. This means that an active node
is to be split into child-nodes as long as there is a test that reduces the
impurity of the node.
\begin{flushleft}
{\large \bf 4.2\ \ Generation and Evaluation of a Classification Tree }
\end{flushleft}
As an  illustration, we take up  a    classification
tree generated by  training data of the largest size, 403 sentences.
The tree is formed with  77 nodes, in which 39 are leaves.
A part of the tree near the root is illustrated in  Fig.2.
There was a tendency that the tests related to parts of speech with  higher
frequencies appeared in the nodes closer to the root.
 Thus it can be said that an efficient order of tests
was realized automatically.
\begin{table}[t]
\begin{center}
\epsfile{file=fig2.ps,width=350bp,height=220bp}
\end{center}
\begin{center}
{\bf Fig.2}\ Part of the tree generated on the ATR corpus.
\end{center}
\end{table}
Table\ 3 shows the classification 
result of the tree on the evaluation data. The symbol $Y$ signifies that the boundary in
focus
is a bunsetsu boundary, and $N$ signifies that is not.
The arrow $\rightarrow$ denotes the classification operation by the
tree.
So, $|Y\rightarrow N|$ means the number of bunsetsu boundaries
which were not judged as boundaries.

\vspace{10pt}
\begin{center}
{\bf Table\ 3} Classification  result for the ATR corpus using the
classification tree method.
\vspace{10mm}\\
\small
\begin{tabular}{c|c|c|c|c|c}
\#Objects&$|Y\rightarrow Y|$&$|N\rightarrow N|$&$|Y\rightarrow N|$&
$|N\rightarrow Y|$&Accuracy\%\\ \hline
1298&521&763&13&1&98.9
\end{tabular}
\end{center}
\vspace{10pt}
\normalsize
About 93\% of the errors were of the type $Y\rightarrow N$.
It was found that most of the errors of this type came from
the boundaries   between  common nouns. The generated tree 
decided a boundary  between common nouns not to be a bunsetsu
boundary. However, many of the first common nouns
in such boundaries  were in fact those that
functioned as adverbs, and this kind of objects were labeled as
 bunsetsu boundaries in the corpus. Thus, the coarseness of 
sub-categorization of noun was a part of the causes of the errors.
\begin{flushleft}
{\large \bf 4.3\ \ Comparison with a  Rule-Based Method  }
\end{flushleft}
It will be meaningful to compare
the performance of the proposed method with that of a rule-based
method. The basic structure of a bunsetsu can be defined by using
the four categories (prefix, suffix, content word, function word) 
as follows (Nagao Ed. 1984, p.12):
\[\ (\ [Prefix]\ (Content\ Word\ )\ [Suffix]\ )^*[Function\ Word\
]^*,\]
where ``[\ ]'' denotes  that the item is optional, 
 and ``$*$''  denotes that the item appears once or 
more than once. Based on this definition, we have constructed  a simple,
basic rule for bunsetsu segmentation as  shown in Table 4.
The boundary between content words  may or may not be a bunsetsu
boundary. Therefore, two rules were considered: Rule I makes a
bunsetsu boundary between content words, and Rule II does not.\\
\begin{table}[t]
\begin{center}
{\bf Table\ 4}\ \  Segmentation rule based on the definition of
bunsetsu. ``Y'' signifies ``Make a bunsetsu boundary'', and ``N'' 
signifies ``Don't make a bunsetsu boundary'' between the two
categories. The blank is  a combination of categories which violates
the definition of bunsetsu.
\vspace{5mm}\\
\small
\begin{tabular}{|c|c|c|c|c|}
\hline
Left &\multicolumn{4}{c|}{Right Category}\\
\cline{2-5}
Category& Prefix  &Content Word  &Suffix   &Function Word\\ \hline
Prefix &  & N &   &\\ \hline
Content Word & Y  &Y/N  & N  &N\\ \hline
Suffix &Y  & Y &   &N\\ \hline
Function Word & Y  &Y  &   &N\\ \hline
\end{tabular}
\end{center}
\end{table}
\vspace{5pt}\\
Each of the parts of speech of ATR corpus was assigned to one of the
four  categories (Nagao Ed. 1984, p.92 ), and bunsetsu segmentation 
experiments were carried out using the same evaluation data as in 4.2.
The accuracy of 95.9\% for Rule I and 91.1\% for Rule II were
obtained.
 
Since Rule I gave a better result than Rule II,  the performance of
Rule I was compared with that of the classification tree method.
Fig.3 shows  that when the training data size is larger than about 20 
sentences, the classification tree method outperforms the rule-based
method.  The superiority of the classification tree method is more 
evident when the training data size is larger.
\begin{flushleft}
{\large \bf 4.4\ \ Comparison with the Bayes Decision Rule  }
\end{flushleft}
It will also be of interest to compare the performance of the
classification tree method with that  of a purely statistical learning 
and decision method. For such a method, we take up the Bayes decision
rule here.
 The Bayes decision
rule is well known for its minimum error property\,(Fukunaga 1990). That is, if the
class
conditional probability distribution of objects to be classified and
the $a\ priori$ class probability distribution are known exactly, then it is
possible
to classify objects with smaller errors than any other
methods. However, those
probability distributions are not known in general, and have to be
estimated
from training data. Thus, it is possible that other methods outperform 
the Bayes decision rule despite its optimum property in the ideal case.
  The Bayes decision rule for bunsetsu segmentation is implemented in
the
following way. 

Let $(pos_1, pos_2)$ be a boundary in focus of which
the left and right morphemes have parts of speech, $pos_1$ and $pos_2$,
respectively. Thus
$P(Y|\ pos_1, pos_2)$ denotes the probability that the boundary $(pos_1, pos_2)$ 
is a
bunsetsu boundary. The probability $P(Y|\ pos_1, pos_2)$ can be estimated
from the
co-occurrence count as:
\[P(Y\ |\ pos_1,pos_2) \approx \frac{Count[\ (pos_1,pos_2)\ is\ a\
bunsetsu\ boundary]}{Count[\ (
pos_1,pos_2)\ ]},
\]
where $Count[X]$ denotes the number of occurrences of the event $X$ in the training data.
In the same way, \ $P(N\ |\ pos_1,pos_2)$,  the  probability that the
boundary $(pos_1,pos_2)$ is not   a
bunsetsu boundary, can  be estimated.
The boundary $(pos_1,pos_2)$ is decided to be  a bunsetsu boundary
if and only if
\[ P(Y\ |\ pos_1,pos_2) \geq  P(N\ |\ pos_1,pos_2).\]
The amount of training data to estimate the probability distributions
was varied according to Table 1.

We tried two different ways to deal with  the case $Count[\ (pos_1,pos_2)\ ]=0$: one that makes a bunsetsu boundary between such $pos_1$ and $\ pos_2$,
and the other that  does not.
Because the former gave a better performance than the latter when the
training data size was larger than 50 sentences, 
the result using  the former  was 
compared  with that of the classification tree method.\\
\begin{table}[t]
\hspace*{10mm}
\epsfile{file=fig3.ps,width=350bp,height=250bp}
\begin{center}
{\bf Fig.3}\ Comparison of error rates among the classification
tree method,\ the rule-based method and the Bayes decision rule. 
\end{center}
\end{table}
  Fig.3 shows that for any training data size, the classification tree method
outperforms   the Bayes decision rule. The superiority of the
classification tree method
becomes
more remarkable as the training data size decreases. Training data of 
30 sentences is enough to achieve the error rate of 2.4\%; for that 
training data size, the error rate of the Bayes decision rule is
 5.5\%. This shows that the classification tree method has a strong power of
generalizing the knowledge beyond the training data so that unseen
data can  be
classified with high accuracy.

  In order to check the statistical significance of the performance
difference between the classification tree method and the Bayes decision rule, 
the $a\ posteriori$ probability method\,(Ozeki 1997) was employed. In this method, the
superiority of  one method to the other
is measured by the $a\ posteriori$ probability  of the event that the true 
classification rate of one method  is higher than that of the other, given the
classification results of the two methods. It utilizes, not only the number of objects classified
correctly by each method, but also the number of objects classified
correctly by  both methods.
The results are shown in Table 5.

\begin{table}[b]
\begin{center}
{\bf Table\ 5} The superiority of the classification tree method to
the Bayes decision rule for  training  data of different sizes.
\vspace{15pt}\\
\begin{tabular}{c|c|c|c|c}
\#Sentences&5&50&200&403\\ \hline
Superiority$(\%)$& 100 & 100 & 99 & 96
\end{tabular}
\end{center}
\end{table}

This confirms that the
classification tree
method is surely superior to the Bayes decision rule, and that the
superiority becomes more prominent when the training data size becomes smaller.

One problem of the Bayes decision rule is that it is vulnerable to 
the sparseness of training data; when $Count[\ (pos_1,pos_2)\ ]$
is equal to 0 or close to 0,  the estimation of $P(Y/N\ |\ pos_1,\
pos_2)$  becomes impossible or unreliable.  
Various methods  to alleviate this problem are conceivable. For example,  
we could  estimate  $P(Y/N\ |\ pos_1,\ pos_2)$ with 
$\alpha P(Y/N\ |\ pos_1,*) + (1- \alpha ) P(Y/N\ |\ *, pos_2)$, where $\alpha$
is  a suitably
chosen weight. It is easy to show that when $\alpha =1/2$, 
 this linear interpolation  method is equivalent to a decision method
based on comparison of the four probabilities
$P(Y/N\ |\ pos_1,*)$ and $P(Y/N\ |\ *,\ pos_2)$.
We could also use a coarser
classification of the set of all the $(pos_1,pos_2)$ to replace
$P(Y/N\ |\ pos_1,pos_2)$ with  $P(Y/N\ |\ \Phi(pos_1,pos_2))$, where
$\Phi(pos_1,pos_2)$ is the
class to which $(pos_1,pos_2)$ belongs. 
All these methods might improve the performance of the Bayes
decision rule.
However, it is difficult to 
decide which one   should serve as a baseline.
For this reason we have compared the proposed method with the Bayes
decision
rule in its most primitive form, and left the problem of its
improvement untouched.

\section{Experiments on  EDR Corpus}
In order to see the influence of
different systems of parts of speech and
different sentence materials on the results of classification tree,
experiments on an EDR corpus\,(EDR 1996) were also carried out.
\begin{flushleft}
{\large \bf 5.1\ \  Labels and Parts of Speech in EDR  Corpus}
\end{flushleft}
The EDR corpus has  no labels indicating bunsetsu
boundaries. Instead, it has detailed information about the
syntactic structure of sentences.
 By utilizing this information and the definition of
bunsetsu\,(Nagao Ed. 1984, p.12), 400 sentences were labeled with the bunsetsu
boundary. Then
6984 objects for training were extracted from 
randomly selected 200 sentences, and
7110 objects for evaluation from the rest.
The sub-categorization of noun
in the EDR corpus seemed too coarse for the present
purpose. Therefore it was augmented by using the semantic identifier,
which was common to the corpus and the dictionary. The resulting
number of parts of speech was 19, in which noun was sub-categorized
into common noun,\ proper noun,\ numerals,\ temporal noun
and formal noun.

In the EDR corpus, sub-categorization of particle is coarser than that
in the ATR corpus. Moreover, there is no such part of speech  as
formal  verb, 
which is employed as a part of speech  in the ATR corpus. 
Although a large portion  of boundaries between particles  and verbs  are
bunsetsu boundaries,\ quite a few  of them  are not. The  boundary 
between a particle and a verb that functions as a formal  verb,
for example, is not a bunsetsu boundary.
 Therefore, as  was explained in Section 3, 
the test of the form $\langle (particle,\, \ast )(verb,\, \ast)
\rangle $
has no power to distinguish between a bunsetsu boundary and
a non bunsetsu boundary;
the part of speech
information alone, especially for boundaries related to particles and verbs, is  not enough
for bunsetsu segmentation. For that reason, the orthographic
expression
was also used as an attribute for test
together with the part of speech. 
In this experiment the orthographic expressions of  
morphemes, ``$aru$''\ (``$exist$''),\ ``$iru$''\ (``$exist$''),\
``$kuru$''\ (``$come$'') for verb ,  ``$nai$'' (``$not$'')  for adjective,
and ``$te$'',\ ``$de$'' for particle, were selected.
\begin{flushleft}
{\large \bf 5.2\ \ Experimental  Result}
\end{flushleft}
A classification tree with 175 nodes was generated in this experiment.
It was found that segmentation rules related   to the 
orthographic  expressions  employed were extracted.
For example, the boundary between a particle and a verb 
was decided not to be a bunsetsu boundary if the orthographic
expression of the particle equals ``$de$'' and that of the verb 
equals ``$aru$''.

It was observed that the tree generated on the 
EDR corpus (EDR tree, henceforth)
 acquired new segmentation rules that were not acquired in
the tree generated on the ATR corpus (ATR tree, henceforth):
\begin{enumerate}
\item The boundary between a temporal noun and a common noun
was decided to be a bunsetsu boundary, while it was not by the ATR tree. 
(Because the ATR corpus has no such part of speech
as temporal noun, it makes no
distinction between a temporal noun and a common noun.)
\item The boundary between an auxiliary verb and a common noun was
decided to be a bunsetsu boundary, and  
the  boundary between an auxiliary verb and a formal noun was not.
The ATR corpus  makes no distinction between a common noun and a formal noun.
\item The boundary between a particle and a common noun was decided
to be a bunsetsu boundary, 
and   the boundary between a particle and a formal noun was
not. The ATR corpus  makes no distinction between 
a common noun and a formal noun.
\item The boundary between two common nouns was
decided to be a bunsetsu boundary, and the boundary
between two proper nouns was not, while neither was decided to be a 
bunsetsu boundary  by the ATR tree. 
(The ATR corpus does not contain enough amount of data
for extracting such a segmentation rule.)
\item The EDR tree acquired 12 rules related to symbols, while
the ATR tree acquired no such rules.
(There were  no occurrences
of symbols although it has a category  ``symbol''  as a part of speech 
in the ATR corpus.)
\end{enumerate}
Thus, the classification tree extracted the new segmentation rules
exploiting  the sub-categorization of noun and an increased amount of 
training materials  available in  the EDR corpus.
In this way
a classification tree can adapt to a new system of parts of speech
and a new task domain.

The 200 sentences of the evaluation data were segmented by the tree, 
and the classification  result was obtained as in  Table\ 6.
\vspace{10pt}
\begin{center}
\small
{\bf Table\ 6} Classification result  for the EDR Corpus.
\vspace{10pt}\\
\begin{tabular}{c|c|c|c|c|c}
\#Objects&$|Y\rightarrow Y|$&$|N\rightarrow N|$&$|Y\rightarrow N|$&
$|N\rightarrow Y|$&Accuracy\%\\ \hline
7110&2502&4341&62&205&96.2
\end{tabular}
\end{center}

\vspace{5pt}
\begin{flushleft}
{\large \bf 5.3\ \ Error Analysis on  Leaves }
\end{flushleft}

It is noted that the accuracy  of 96.2\% in Table 6 is a little lower
than that  for   the ATR corpus in Table 3. In order to find out the causes of errors,
every leaf $t$ of the EDR tree was examined in detail by measuring three quantities:
the local error rate,\ i.e. the error rate  at leaf $t$, defined as
\vspace{3pt}
\[LER(t)\ = \frac{Number\ of\ evaluation\ objects\ misclassified\ at\ 
t}{Number\ of\ evaluation\ objects\ that\ reach\ t},\]
the number of training objects that reach $t\ (NTO(t))$,\ and the impurity of
$t$ on the training objects$(IMP(t))$.
Fig.4 and Fig.5 are scatter plots of $LER(t)$ versus $NTO\ (t)$,
and $LER(t)$ versus $IMP(t)$, respectively.
\begin{table}[t]
\begin{center}	
\epsfile{file=fig4.ps,width=350bp,height=250bp}
\end{center}
\begin{center}
{\bf Fig.4}\  Local error rate vs. number of training objects  at
leaves. The black marks show leaves with $IMP=0$.
\end{center}
\end{table}
\begin{table}[t]
\begin{center}
\epsfile{file=fig5.ps,width=350bp,height=250bp}
\end{center}
\begin{center}
{\bf Fig.5}\  Local error rate vs. impurity of leaves on  training objects. 
\end{center}
\end{table}

 Fig.4 shows that a leaf formed 
with  a smaller number of training objects is likely to have a higher
local error
rate. Fig.5 indicates that a leaf with  greater impurity on training 
objects tends to have a higher local error rate. 
The point $IMP= 0$ is an
exception. At this point the local error rates distribute from
0 to 100\%. The black marks in Fig.4 show the leaves with $IMP=0$. As 
seen   here, a leaf with $IMP= 0$ is very reliable as long as there are
enough number of training objects reaching that leaf. However, leaves
formed with less than 30 training objects are unreliable
and the local error rates  become unpredictable. 

 These results lead to a conclusion that the classification accuracy
of a leaf 
is affected by both the number of training objects
reaching the leaf and the impurity of the leaf on the training
objects.

 It was found that in the EDR tree there were quite a number of leaves 
formed with small number  of training objects  or with high impurity. It was just
at these leaves that many of the errors occurred. The high impurity
remained  because there was no test that could reduce  it. The information
carried in two consecutive morphemes  and 
the attribute values tested in this experiment could not make 
further reduction on impurity. So one solution to this problem would be to
widen the window size and  test a longer  morpheme sequence, and  to test  more  attribute values.
There were, in fact, some cases where
testing more than two morphemes and adding some other orthographic 
expressions, such as  ``$oku$'' (``$put$'') and  ''$iku$'' (``$go$''), would improve the result. 
Some leaves with high impurity, however, obviously resulted from
mislabeling in the corpus. A leaf formed with very small number  of
training objects  was often a result of
mislabeling in the corpus or rare exceptions. 
\section{Conclusion}
A bunsetsu segmentation method using a classification tree was
proposed. It has been
shown that linguistic knowledge about bunsetsu boundaries can be
acquired automatically by this method. It can adapt quickly to a new 
system of parts of speech and
also to a new task domain. An efficient order of tests is
automatically realized under the criterion of maximum impurity reduction. All these
are advantages of the proposed method over conventional handicraft methods. The
classification accuracy  of 98.9\%  was achieved for the ATR corpus, and 96.2\% for the EDR corpus.

The performance of  the proposed method  was compared  with a 
rule-based method and 
the Bayes decision rule.
The classification tree method 
outperformed the rule-based method when the training data size was
larger than about 20 sentences,  and the
superiority was more prominent when the training data size was larger.
The classification tree method outperformed the Bayes decision rule
over the whole range of training data sizes, and the superiority 
was more prominent when the training data size was
smaller.  Training data of as little as 30
sentences  was enough to achieve 2.4\% error rate, whereas
 in  the Bayes decision rule the error rate was
 5.5\%  for  the same training data.
This fact is a good
evidence that the method has a strong power of generalizing the
knowledge beyond the 
training data.
Furthermore, the knowledge acquired by the classification tree method
is easier to interpret and more compact to store than that acquired by the Bayes
decision rule.

  Error analysis was conducted by evaluating the local error rates.
From this analysis it has been made clear  that a leaf formed with  
a smaller number  
of training objects is likely to have a higher  local error  rate. In order to
prevent the formation of a  leaf with a small number  of training
objects, the
control method for growing a classification tree must be improved. The 
threshold $I_0$ is a parameter intended to control the growth of a
tree, but it  did not work in these experiments. When  the set of
tests becomes larger, over-growth may occur.
In such a case, the threshold $I_0$ may function  to  control the  
over-growth of the classification tree. 
Another solution to the over-growth  problem may be to incorporate another quantity,
the number of training objects  that reach an active node, into the stopping condition
to  control the splitting of the node.

  Another conclusion from the error analysis is that a leaf with higher 
impurity on the
training objects is also likely to cause a higher  local error rate.
On a leaf with high impurity, the difference between the majority
and the minority is small, which makes the selected class  label
unreliable.
In order to further reduce the impurity, new testing information is
necessary.
Testing  a morpheme sequence longer than two morphemes is one such possibility. Another
solution may be
to improve the set of orthographic expressions for the attribute
values. 
Therefore it is desirable to develop a method for automatic
acquisition of morphemes
whose orthographic expressions are effective in bunsetsu boundary
detection.

\acknowledgment

The authors are grateful to the anonymous referees for the detailed and
constructive comments which were useful to improve the paper.

\begin{thebibliography}{19}
\bibitem{Abe90} Abe, M.,  {\it et al.}(1990).\ \ {\it Speech Database User's Manual.}
 {\rm ATR Interpreting Telephony Research Laboratories}. (in
Japanese).
\bibitem{Breiman84} Breiman, L., Friedman, J. H., Olshen, R. A. and Stone,
C. J.(1984).\ \ {\it Classification  and Regression Trees.} {\rm
Chapman \& Hall}.
\bibitem{EDR96} EDR(1996). \ \ {\it Specification of EDR Electronic Dictionary}
\,{\rm Ver.1.5.}\,
Japan Electronic Dictionary Research Institute. (in Japanese).
\bibitem{Fujio97} Fujio, M. and Matsumoto, Y.(1997).\ \ ``Statistical
Japanese Dependency Structure Analysis Using an
EDR Bracketed Corpus.''
{\rm In} {\it Proc. of Symposium on Applications of EDR Electronic
Dictionary}, 49-55. (in Japanese).
\bibitem{Fukunaga90} Fukunaga, K.(1990). \ {\it Introduction to Statistical Pattern
Recognition(2nd edition).}
{\rm Academic Press}. 
\bibitem{Gelfand91} Gelfand, S. B., Ravishankar, C. S. 
and Delp, E. D.(1991).\ ``An Iterative Growing and Pruning Algorithm
for Classification Tree Design." {\it IEEE Trans. PAMI}, 
13(2), 163-174.
\bibitem{Nagao84} Nagao, M. Ed.(1984).\ \ {\it Japanese Language Information
Processing.} {\rm The Institute of Electronics, Information and
Communication Engineers}. (in Japanese).
\bibitem{Ozeki97} Ozeki, K.(1997). ``
Performance Comparison of Recognition Systems Based on the a
Posteriori Probability of the True Recognition Rate.''
{\it Technical Report of IEICE.},\ SP97-62, 35-42.(in
Japanese).
\bibitem{Safavian91} Safavian, S. R. and Landgrebe, D.(1991). ``
A Survey of Decision Tree Classifier Methodology.''{\it IEEE Trans. on 
Systems, Man, and Cybernetics}, 21(3), 660-674.
\bibitem{Suzuki96} Suzuki, M.(1996). ``
Japanese Sentence Segmentation Algorithm Using Character Patterns
Based on the Statistical Investigation." 
{\it IEICE Trans. on Information and Systems},
J79-D-II(7), 1236-1243. (in Japanese).
\bibitem{Zhang97} Zhang, Y. and Ozeki, K.(1997).\ ``Automatic Bunsetsu Segmentation of
Japanese Sentences Using a Classification Tree
.'' {\it Technical Report of Information Processing Society of Japan}, 
97(85), 1-8. (in Japanese).

\end{thebibliography}

\begin{biography}

\biotitle{}

\bioauthor{Yujie Zhang}
{
Yujie Zhang received the B.E. degree from Northern-Jiaotong  University
in 1983, and the M.E. degree from  Graduate School of Academia Sinica
in 1986, both in computer science.
From 1986 to 1992, she worked at  the Institute of Computing Technology,
 Academia  Sinica,
where she was involved in the development of a machine translation  system.
 She received the 
National Award for the Progress of Science and Technology in China in 1995.
She is currently pursuing the doctoral degree at Department of Computer
 Science and
 Information Mathematics, the  University of Electro-Communications.
Her current research interests include corpus linguistics and  natural
 language processing.
}

\bioauthor{Kazuhiko Ozeki}
{Kazuhiko Ozeki received the B.E. degree in electrical engineering
from the University of Tokyo in  1965. He joined  Japan Broadcasting
 Corporation in  1965. He was
a visiting researcher at Department of Phonetics and Linguistics,
 the University of Edinburgh in 1968-1969.
He received the Dr. Eng. degree from the University of Tokyo in 1977.
Since 1989, he has been a professor of applied computer science
 at Department of Computer Science and Information Mathematics,
the  University of Electro-Communications. 
 His research activities are in the areas of 
 statistical methods for spoken language processing 
and theory of pattern recognition and learning. He received the Paper Award 
from IEICE in 1984. He is a member of IEICE, IPSJ, ASJ, and IEEE.
}

\bioreceived{Received}
\biorevised{Revised}
\biorerevised{Rerevised}
\bioaccepted{Accepted}

\end{biography}

\end{document}


