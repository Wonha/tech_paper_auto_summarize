



\documentstyle[epsf,jnlpbbl]{jnlp_j_b5}

\setcounter{page}{127}
\setcounter{巻数}{5}
\setcounter{号数}{4}
\setcounter{年}{1998}
\setcounter{月}{10}
\受付{1997}{10}{16}
\採録{1998}{4}{10}

\setcounter{secnumdepth}{2}

\title{日韓機械翻訳システムの現状分析および開発への提言}
\author{金泰完 \affiref{ETRI} \and 崔杞鮮 \affiref{KAIST}}

\headauthor{金泰完・崔杞鮮}
\headtitle{日韓機械翻訳システムの現状分析および開発への提言} 

\affilabel{ETRI}{ETRI コンピュタ\&ソフトウェアー研究所, 自然語処理研究部}
{Natural Language Information Processing Department, Systems Engineering Research Institute,Korea}
\affilabel{KAIST}{韓国科学技術院 電算学科}
{Department of Computer Scinece, KAIST,Korea}


\jabstract{
\quad
本技術資料は, 現在入手可能な日韓機械翻訳システムを対象に翻訳品質の評価を行い, 
日韓機械翻訳システムの現状および技術水準を把握, 今後の研究方向についてのいくつかの提言を行うことを目的とする. 
現在, 韓国国内で発表, あるいは発売されている日韓機械翻訳システムの中で, 入手可能な四つの製品に対して
ユーザサイドからの翻訳品質の分析と言語学的な解決範囲を把握するための対照言語学的誤謬分析を行う. 
さらに, \cite{choiandkim}と比較することにより日韓翻訳システムの性能向上の度合いを比較する. 
これにより, 日韓機械翻訳システムの性能向上のための長期・短期課題を考える. 本技術資料は, 対象にした各々のシステムの
優劣のランク付けを目的とするものではないことをあらかじめ断っておく. 
本技術資料での評価は限られた観点からの分析に基づいたものであるからである. }

\jkeywords{日韓機械翻訳システム, 翻訳システム評価, 翻訳品質評価, 対照言語学的評価}

\etitle{Analysis of Current Commercial\\
Japanese to Korean\\
 Machine Translation Systems\\
and Suggestions for Future Development}
\eauthor{Tae Wan Kim \affiref{ETRI} \and Key Sun Choi \affiref{KAIST}} 

\eabstract{
\quad
This Report describes the status and performance of current Janpanese-to-Korean Machine Translation Systems. 
And some suggestions for developing the better systems are made. This result is made by analyzing the latest version 
of four commercial Japanese-to-Korean Machine Translation Systems in Korea which have been produced until Feburary 1997. 
Declarative evaluation is executed in the view of\break 
user side to measure the translation quality. Typological evaluation is tried  
to probe the linguistic coverage of current commercial systems. Operational evaluation is performed in the view of user interface. 
And progress evaluation is executed by comparing the result with the result reported at \cite{choiandkim}. This report does not 
intend to rank the relative standing of the systems. The evaluations are executed in the range of interest of this report.}

\ekeywords{Japanese to Korean Machine Translation System,  machine translation system evaluation, 
Korean to Japanese Machine Translation System,  declarative evaluation, typological evaluation,
operational evaluation, progress eval\-uation}

\begin{document}
\maketitle


\section{まえがき}
韓国において日本語は技術の分野のみではなく, 経済などの他の分野に
おいても英語に次ぐ重要な言語の一つになっている. しかし, 日本語が自由に
操れる人は少ない. このような背景により, 機械翻訳に関する研究が韓国に
紹介され始めた80年代の初めから日韓機械翻訳に対する期待はかなり高い
状況であった. このような期待が実り, 90年代に入り, 韓国, あるいは日本で
開発された使用可能な日韓機械翻訳システム5種が市販されるようになった. 
しかし, 現在市販されている商用日韓機械翻訳システムは, 日本語と韓国語の
言語構造の類似点などによる一般ユーザたちの高い期待とは裏腹にその翻訳品質は
低いレベルにとどまっている. このような現実を踏まえ, 日韓機械翻訳システムの
活性化を達成するために, 現在の日韓機械翻訳システムが持っている問題点を
客観的に分析, 評価し, その問題点の在処を解明し, 解決法を探す必要がある. 
そのためには, 現在の機械翻訳システムに対する客観的分析と評価が前提となる.  
本技術資料は, 四つの商用日韓機械翻訳システムを分析・評価し, 技術的現象を把握, 
その問題点を分析することにより今後の開発作業に有効ないくつかの提言を行うのに目的がある. 
このような努力の一環として, 筆者は\cite{choiandkim}を発表した. しかし, その後, 
韓国国内では\cite{choiandkim}で評価対象にした各システムのアップグレードや新しい
システムの出現という状況の変化があったのため, 現時点での分析・評価と\cite{choiandkim}で
明らかになった問題点とを比較することにより, 解決された問題と未解決の問題がどのようなもの
であるかを把握, 短期的解決課題と長期的解決課題の性格をより明確にする必要が出てきた. 
翻訳システムの評価には様々な側面からの評価が必要であり, 多様な評価法方が提案されている
\cite{dijk,White,Whiteandconnel,井佐原}. 本技術資料では\cite{arnold}で提示された
ユーザサイドからの翻訳品質の評価といえるDeclarative Evaluation, 開発側からの評価であるといえる
Typological Evaluation, 経済的立場からのシステムの効用性の評価であるといえるOperational Evaluationの
三つの立場からの評価とシステムの性能向上度評価といえるProgress Evaluationを行う. 
評価のための評価にならないよう, 実際の生活で機械翻訳が用いられるという状況を作るため, 
評価対象文を市販されている98種の日本語で書かれた文庫本から直接抽出し評価を行った. 

今回の評価結果と\cite{choiandkim}を比較すると, 開発者側からの言語学的処理範囲の評価と
いえるTypological Evaluationではシステムによっては多少改善されたものが見られるが, 機械翻訳
処理技術の最も重要な部分であるといえる翻訳技術そのものには大きな進展は見られない. と同時に, 
ユーザ側からの翻訳品質の向上も\cite{choiandkim}とあまり変わらないことが明らかになった. 
これは今までの日韓機械翻訳システムの開発で用いられた方法である一般の文法書と一般辞書に
基づく演繹的翻訳規則および知識水準ではこれ以上の発展は期待できないことを物語るものであると
考えられる. この限界を乗り越えるためには実際の人間の言語生活で用いられる日本語—韓国語間の
大量の対訳用例集の構築とそれを用いた日本語と韓国語の客観的で一貫性のある翻訳モデルの確立, 
大量の用例に基づく帰納的翻訳規則および知識の開発と蓄積が前提となる必要がある. 
効用性の評価といえるOperational Evaluationではすべてのシステムが韓国語Windows95で運用される
ようになり, 日本語原文入力ツールの支援, インターネット翻訳支援などというように大きく進展したといえる. 

\section{韓国国内の日韓機械翻訳システム}
現在, 韓国国内で市販, あるいは発表されたシステムを表~\ref{jkmtinkorea}に示す. この中で  「ATLAS/JK」は, 
1990年に実用化された初の日韓機械翻訳システムである.  「ATLAS/JK」は, 富士通のATLAS翻訳システムを
基に韓国語への変換, 生成のための文法および辞書の開発を韓国のシステム工学研究所が行った. 
ハードウェアプラットフォームは, 富士通の大型コンピュータ(M series)であった.  「ATLAS/JK」の応用分野はJICSTの
論文抄録の翻訳であったが, 現在でも韓国の研究開発情報センター(KORDIC)によりパソコン通信を通じて一般ユーザに
日本のJOIS DBオンライン翻訳サービスが提供されている. 


\begin{table}
\begin{center}
\caption{\label{jkmtinkorea} 韓国国内の日韓機械翻訳システム} 
\epsfile{file=129.eps}
\end{center}
\end{table}	


「J-Seoul」は, 日本の高電社により開発された製品であり, 最初はNEC PC 9800シリーズでしか運用できなかったが, 
最近ではIBM互換機, 韓国語WINDOWS95でも使えるようになり, IBM互換機が主流である韓国市場への進出の地盤を
広めたといえる. 「ハングルカナ」は韓国のチャンシンコンピュータにより開発されたもので, やはりIBM互換機, 
韓国語WINDOWS95上で動くようになっている. 「名品」は, 日本の日立情報ネットワークにより開発されたもので
IBM互換機, 韓国語WINDOWS95上で動くようになっている. 「オギョンバクサ」は, 韓国のユニソフトにより開発されたもので, 
IBM互換機, 韓国語WINDOWS95で動くようになっている. 
本技術資料ではこれら五つの製品の中で入手可能な四つのシステムを対象に1997年3月現在最も最新バージョンである
「J-Seoul Business Package」(1996.10), 「ハングルカナ 3.0」(1997.2), 「名品GOLD」(1997.2), 「オギョンバクサ 1.52」(1997.2)を
対象に分析を行った. 

\section{評価方法の考察}
\subsection{評価方法の概要}
機械翻訳システムの評価については様様翻訳品質に関する評価方法は大変難しい問題である\cite{king}. その理由として, 

\vspace{0.5cm}

\begin{itemize}
\item 評価とは本質的に難しい問題である
\item 機械翻訳の様々な使用目的によって評価基準の設定を変えなければならない
\item 評価基準の客観的な設定が難しい
\end{itemize}

\vspace{0.5cm}

という三つをあげることができる. しかし, このような限界を持つにも関わらず, 機械翻訳システムに関する評価の必要性はシステム開発と
いう立場からこれまで多くの研究がなされてきた. これら研究を総合してみると, 評価には大きく三つの観点が必要である\cite{arnold}. 

\vspace{0.5cm}

\begin{itemize}
\item[(1)] Declarative Evaluation
\item[(2)] Typological Evaluation
\item[(3)] Operational Evaluation
\end{itemize}

\vspace{0.3cm}

以上の三つの以外に開発者の立場で重要なものとして

\vspace{0.3cm}

\begin{itemize}
\item[(4)] Progress Evaluation
\end{itemize}

\vspace{0.3cm}

がある. 

\begin{table}
\begin{center}
\caption{\label{declevalmethod} 理解度測定(declarative evaluation)のための代表的な三つの方法}
\epsfile{file=131.eps}
\end{center}
\end{table}

\subsection{Declarative Evaluation}
 Declarative Evaluationとは, 翻訳結果に対する理解度, 正確性, 信頼性など, ユーザ側から見た「翻訳品質」に関する評価であり, 
表~\ref{declevalmethod}に示したように三つの方法が用いられている. \cite{dijk}では, 最も安定性が高く, 多く用いられている方法として4〜5個のスケールを
持つ  「rating on intelligibility scale」法を勧めている. スケールが細かすぎると判断に苦しむことになり, 偏差が広がる可能性が高く, 
小さすぎると弁別力が低下する恐れがあるからである. 

\subsection{Typological Evaluation}
Typological Evaluationとは, 翻訳システムの言語学的側面からの解決範囲, すなわち, システム開発者側からの評価であるといえる. 
一般的にあらゆる言語現象を網羅するTest Suiteを利用し, 機械翻訳された結果からシステムの言語学的解決可能範囲を評価する. 
しかし, この方法は, あらゆる言語現象を網羅する有効なTest Suiteそのものを作ること自体が難しい作業であること,  Test Suiteで実験し
ようとする任意の言語現象についてのシステムの動作が実際のテキストでは他の言語現象と複合的に現れるため同一言語現象について
均一的に動作するという保障がないのが問題点として指摘されている \cite{king,arnold}. 

\subsection{Operational Evaluation}
Operational Evaluationとは, 特定のユーザが特定のシステムを使うことにより得られる経済的利益という観点からの評価であり, 
ユーザの機械翻訳システムの利用目的, 経済性によって評価結果は大きく異なる. 例えば, Browsabilityを重視するユーザは満足
している機械翻訳システムでもLinguistic Accuracyを重視するユーザにとってはとても使えないといった評価が下される場合もあり得る. 
また, 組織の使用目的, 要求環境, 維持補修費用, 拡張可能性, 速度など様々な観点からの評価が必要である. 

\subsection{Progress Evaluation}
Declaraive Evaluation, Typological Evaluation, Operational Evaluationが現在のシステムの状況を評価するものであるとすれば
Progress Evaluationは一つのシステムに対する通時的発展段階を把握するためのものである. これは, システム開発のそれぞれの
段階で評価を行い, 問題点および解決法を見出そうとするものである. 

\begin{figure}
\begin{center}
\epsfile{file=132.eps,height=53mm}
\caption{\label{japantext}日本語標本テキストの選定および韓国語翻訳テキストの作成}
\end{center}
\end{figure}

\section{評価の準備および方法}
\subsection{日本語標本テキストの選定および韓国語翻訳テキストの作成}
評価のための評価ではなく, 翻訳システムが実際の生活で使われている状況に近い状態で分析・評価をするために評価対象文を
実際市販されている98種の日本語で書かれた文庫本から直接抽出し, 評価を行った(図~\ref{japantext}参照). 
日本で刊行された98種の文庫本の本文中100番目の文から119番目の文までの各20個ずつの文をOCRを用いてテキストデータに落とし込み, 脱字や誤字を人間が修正した
1960個の文(98 TEXT×20文/TEXT=1960文)を対象に四つの機械翻訳システムを用い, システムごとに機械翻訳文を抽出した. 
各機械翻訳システムは翻訳結果としてそれぞれ98個の韓国語テキストを出力する(日本語原文の抽出に使用した98種の文庫本の
目録は\cite{choiandkim}を参照). 

\subsection{Declarative Evaluation}
本技術資料では\cite{dijk}で推奨する "rating on intelligibility scale" 法を用いる. 互いに関連性を持つと
考えられる次の三つの項目から六つのスケールを持つ基準表を作成した.  その具体的内容を表~\ref{declcrit}に示す. 

\vspace{0.3cm}

\begin{itemize}
\item テキスト全体に対する理解度
\item テキストを構成する文それぞれに対する評価
\item 後編集の量
\end{itemize}

\vspace{0.5cm}

機械翻訳システムから出力された計392個(4つのシステム×98個の日本語テキスト = 392個の韓国語テキスト)の韓国語テキスト
各システムごとの翻訳文の評価には少なくとも1週間の間をおいて評価を行った. 間をおかず評価作業を続けた場合, 前回の評価の
点数が記憶され評価に影響を与えやすく, 点数が甘かったり辛かったりするからである. 

\begin{table}
\begin{center}
\caption{\label{declcrit}Declarative Evaluationの"rating on intelligibility scale"のための評価基準表}
\epsfile{file=133.eps}
\end{center}
\end{table}

\begin{figure}
\begin{center}
\epsfile{file=134-1.eps}
\caption{\label{typoflow}Typological Evaluation方法}
\end{center}
\end{figure}


\subsection{Typological Evaluation方法}
Typological Evaluationのためには, 日本語の文法的, 語法的言語現象が韓国語の翻訳にどのような影響を与えるかを検討する必要がある. 
Typological Evaluationの典型的方法は,  Test Suiteを用いる方法であるが, 3.3で述べたように様々な問題点を持つ. 
本技術資料ではより現実的な分析・評価のためにTest Suiteを用いず, 実生活で用いられる日本語標本テキストと韓国語翻訳結果テキストを
用い図~\ref{typoflow}のような方法で評価を行った結果, 表~\ref{typoresult}の誤謬の内訳に
見られるような結果が得られた. 

\begin{figure}
\begin{center}
\epsfile{file=134-2.eps}
\caption{\label{jkmtconf}日韓機械翻訳システムの一般的誤謬}
\end{center}
\end{figure}

Typological Evaluationには日本語原文テキストを基準に20個のテキスト(20テキスト×20の文/テキスト=400文), 
対応する韓国語翻訳文テキストを基準に80個のテキスト(日本語原文20のテキスト×四つのシステム=韓国語80のテキスト;
80テキスト×20の文=1600文)のみを用いた. その理由を次に示す. 

\newpage

\begin{itemize}
 \item[一.] Declarative Evaluationが低い順にTypological Evaluationを実施
	    した結果, 日本語原文テキストを基準に12番目のテキストを分析し
	    た後から20番目のテキストまで新しい誤謬のパターンが発見されず, 
	    誤謬のパターンの度数のみが増加した. 誤謬のパターンの度数も誤
	    謬パターン別の度数とほぼ同じ比率で増加した. このような観察結
	    果に対する信頼度を検証するために任意に20個の韓国語テキストを
	    選び日本語原本テキストと比較分析した. その結果同じ結論を得る
	    ことができた. 
 \item[二.] 分析の客観性を保つために, 一人の人間が評価を担当する必要があ
	    るが, 分析の量が多く, 個人の能力の限界を超えている. 
\end{itemize}

\subsection{Operational Evaluation}
 3.4で見たように, Operational Evaluationは, ユーザの使用目的によって違うので客観的な評価が困難である. したがって, 
本技術資料では, Operational Evaluationに関しては, 対象とした四つの日韓機械翻訳システムのユーザインターフェスという
側面からの評価のみを行うことにした. 

\subsection{Progress Evaluation }
Progress Evaluationとは, 本来一つのシステムが発展していく各段階において各段階毎に問題点の分析および性能向上の
ための評価を指す. 対象とするシステムについて十分な知識を持つ開発者でないと客観的な評価が不可能である. さらに, 
本技術資料は今回対象とした四つのシステムのそれぞれに対して開発戦略を立てるためのものではないため一年前に行った
日韓機械翻訳システム分析\cite{choiandkim}での結果を基に, 誤謬パターンの変化, 誤謬パターンの頻度の変化などを
比べることにより日韓機械翻訳システム全般という観点からその向上の度合いを分析した. 

\section{評価の実施および結果分析}
\subsection{Declarative Evaluation}
評価結果は表~\ref{declmark}の通りである. 最も高い点数をマークしたのは2.88点であり, これは翻訳結果を何回か繰り返し読んだ後, 
若干の推測を交えれば全体の意味が把握できるといったレベルである. 

\begin{table}
\begin{center}
\caption{\label{declmark} Declarative Evaluation結果}
\epsfile{file=135.eps}
\end{center}
\end{table}

最も点数の低かった10つのテキストと最も点数の高かった10つのテキストを分析
すると大変興味深い事実が発見された. 表~\ref{tbtextlist}で見られるように
該当テキストのジャンル, 文体, 文の長さ, 漢字使用率, 平仮名使用率, 片仮名
使用率を比較すると


\begin{itemize}
 \item[一.] 翻訳品質は文の長さにあまり影響されない. 
 \item[二.] 漢字使用率が高く平仮名使用率が低いほど翻訳品質が良い. 
 \item[三.] 翻訳する文のジャンルよりは文体に翻訳品質はもっと影響される. 
\end{itemize}

\vspace{0.5cm}

\begin{flushleft}
ことが分かった. 特に, 著者が形式に束縛されず自由に書いた散文のテキストが
比較的低い点数をマークし, 著者が丁重な表現を用い, 形式に拘った固い表現の
多いテキストは比較的高い得点をもらった. 
\end{flushleft}

\begin{table}
\begin{center}
\caption{\label{tbtextlist} 最高および最低10位の翻訳品質テキストリスト}
\epsfile{file=136.eps}
\end{center}
\end{table}

\subsection{Typological Evaluation}
分析結果を表~\ref{typoresult}に示す. 分析を行うにあたって韓国語翻訳文の誤謬がどのシステムモジュールで発生したかについてはそれぞれの
システムの開発者でない限り明確に特定することはできないため, ここでは対照言語学的立場から日本語の言語的特徴を中心に
できる限り客観的に分析を行おうと努力した. ただし, 本来評価の持つ難しさから若干の判断誤謬もあり得る. しかし, 全体的に
見るとほとんど正確であるといってよい. 評価には多くの時間が費やされた. 特に, ある誤謬が発生した時, その原因を突き止めるのに
大変とまどった. 例えば, 韓国語翻訳文で誤謬が発見されたとき, それが未登録語による分析誤謬なのか, 翻訳単位認識誤謬による
間違いなのか判断に迷った. このような場合, 常識的判断により, 当然登録されるべき単語について誤謬が発生した場合は分析誤謬
として, そうでない場合は, 未翻訳として処理した. 以下, 誤謬の具体例を示す. 

\begin{table}
\begin{center}
\caption{\label{typoresult} Typological Evaluation結果}
\epsfile{file=138.eps}
\end{center}
\end{table}
\vspace{1cm}

\subsubsection{未翻訳}
翻訳されずそのままの日本語が翻訳結果文に出力されている場合を未翻訳誤謬と
分類した. 

\vspace{0.5cm}

\begin{tabular}{ll}
 (原文)& 自身がない人 \\
 (翻訳結果) &  mauleui わんぱく aesongidul \\
 (正しい翻訳)& mauleui jangnankuroki aesongidul\\
\end{tabular}

\raggedbottom

\subsubsection{分析誤謬}
日本語を韓国語に翻訳する時, 一般的に同じ単位の韓国語に対応する翻訳単位を想定することができる. このような単位を正確に抽出できなかった場合は翻訳単位認識誤謬とした. また, 翻訳単位は正しくても品詞決定の段階で誤謬が発生し, 誤った翻訳結果が得られた場合には品詞判定誤謬とした. 
\vspace{0.3cm}

\begin{quote}
\begin{flushleft}
{\it 翻訳単位認識誤謬の例}
\end{flushleft}

\vspace{0.3cm}
\begin{tabular}{ll}
 (原文)    &   自信がない人 \\
 (翻訳結果) &  jasin  irado duleo saram(自信 がな い 人) \\
 (正しい翻訳) & jasini  eobnun saram \\
\end{tabular}

\vspace{0.3cm}
\begin{tabular}{ll}
 (原文)  &     いつもこもっているのだった \\
 (翻訳結果) &  eonjena o dulgo itnun geotida\\
 & (いつも こ もって いる のだった) \\
 (正しい翻訳) & eonjena jaukhan geotieotda \\
\end{tabular}

\vspace{0.5cm}

\begin{flushleft}
{\it 品詞判定誤謬の例}
\end{flushleft}
\vspace{0.5cm}

\begin{tabular}{ll}
 (原文)      & 当然の\underline{扱い}ではあるが \\
 (翻訳結果)  & dangyeoneui \underline{chooigubhae} iginun hajiman(verb) \\
 (正しい翻訳)& dangyeoneui \underline{chooigub}ikinun hajiman(noun) \\
\end{tabular}

\vspace{0.5cm}
\begin{tabular}{ll}
 (原文)      & \underline{一定}ではない \\
 (翻訳結果)  & \underline{iljeong}i anida(noun) \\
 (正しい翻訳)& \underline{iljeongha}ji anta(adjective) \\
\end{tabular}
\end{quote}

\vspace{0.5cm}

\begin{flushleft}
{\bf 分析誤謬の考察}
\end{flushleft}
 
 分析誤謬として判定されたもののほとんどは日本語の仮名文字列で発生している. このような傾向は四つのシステムで共通して見られ, 
システムによって多少違いはあるものの四つのシステムともに仮名文字列の分析において不安定な傾向を見せている. 
分析誤謬として判定された結果を詳しく分析すると, あるシステムでは品詞のレベルで左右接続の可能/不可能すらも検査していないものが
あった. また, 「品詞判断誤謬」にはほとんどが「述語の連用形」と「連用形の名詞的用法」, 「名詞」と「形容動詞の語幹」で発生している.  
日本語の場合, 形容動詞であるか名詞であるか, また, 動詞連用形であるか動詞の連用形の名詞的用法であるかによって韓国語への翻訳は大きく異なる. 

\subsubsection{多義性による誤謬}
翻訳単位の設定, 品詞の決定が正しく行われたのにも関わらず, 意味的に全く違った対訳語を当ててしまった場合を多義性による誤謬と判断した. 特に, 名詞の場合は, 韓国語には存在しない語になっているものをも多義性誤謬として分類した.  また, 助詞の場合, あるシステムは可能性がある対訳語を"( )"の中に複数個入れて出力している. この場合, 読む時大きな無理がなければ誤謬として扱わなかった. このような場合を全部誤謬にするとほとんどの多義性を持つ助詞が誤謬となり, 全部を誤謬がないとすると分析の意味がなくなるためである. 

\begin{flushleft}
 誤謬の例
\end{flushleft}

\begin{quote}

{\it 動詞の誤謬}

\vspace{0.5cm}

\begin{tabular}{ll}
 (原文)     &  食事を\underline{とる} \\
 (翻訳結果) &  siksalul \underline{jabda(grasp)} \\
 (正しい翻訳) & siksalul \underline{hada(do)} \\
\end{tabular}

\vspace{0.5cm}

\begin{tabular}{ll}
 (原文)      & 見直しを\underline{迫る} \\
 (翻訳結果)  & jaepyungkalul \underline{dagaoda(come)} \\
 (正しい翻訳)& jaepyungkalul \underline{kangyohada(force to do)} \\
\end{tabular}

\vspace{0.5cm}

\begin{flushleft}
{\it 名詞の誤謬}
\end{flushleft}

\begin{tabular}{ll}
  (原文)      & 用事でやらされるのは大いに\underline{苦手}だった \\
  (翻訳結果)  & \underline{geobukhan sangdae(a man difficult to manage)} \\
  (正しい翻訳)& \underline{golchitgeori(difficult thing)} \\
\end{tabular}

\vspace{0.5cm}

\begin{flushleft}
{\it 形容詞誤謬}
\end{flushleft}

\begin{tabular}{ll}
  (原文)      & \underline{古い}しきたり \\
  (翻訳結果)  & \underline{nalgun(worn-out)} goanrye \\
  (正しい翻訳)& \underline{oraen(long-continued)} goanrye \\
\end{tabular}

\vspace{0.5cm}

\begin{flushleft}
{\it 副詞の誤謬}
\end{flushleft}

\begin{tabular}{ll}
  (原文)      & \underline{くまなく}案内してもらった \\
  (翻訳結果)  & \underline{dduryothage(sharp distinct)} \\
  (正しい翻訳)& \underline{jasehi(in detail)} \\
\end{tabular}

\vspace{0.5cm}

\newpage

\begin{flushleft}
{\it 数詞の誤謬}
\end{flushleft}

\begin{tabular}{ll}
  (原文)      & 二十六 \\
  (翻訳結果)  & 2+6 \\
  (正しい翻訳)& 26 \\
\end{tabular}

\vspace{0.5cm}

\begin{tabular}{ll}
  (原文)      & 第\underline{2次}世界大戦 \\
  (翻訳結果)  & je \underline{second-order} segyedaejeon \\
  (正しい翻訳)& je \underline{2cha} segyedaejeon \\
\end{tabular}

\vspace{0.5cm}

\begin{flushleft}
{\it 連体詞誤謬}
\end{flushleft}

\begin{tabular}{ll}
  (原文)      & \underline{ある}時 \\
  (翻訳結果)  & \underline{itnun(to be)} dda \\
  (正しい翻訳)& \underline{eonu(uncertain)} ddae \\
\end{tabular}

\vspace{0.5cm}

\begin{flushleft}
{\it 助詞の誤謬}
\end{flushleft}

\begin{tabular}{ll}
  (原文)      & 女性の増加\underline{で}家庭での \\
  (翻訳結果)  & jungga\underline{eso(場所)} \\
  (正しい翻訳)& jungga\underline{ro(原因)} \\
\end{tabular}

\vspace{0.5cm}

\begin{flushleft}
{\it 受け身/可能/自発の誤謬}
\end{flushleft}

\begin{tabular}{ll}
  (原文)      & 密接不離の関係にあることが知\underline{られる} \\
  (翻訳結果)  & al\underline{ryeojida(passive)} \\
  (正しい翻訳)& al \underline{soo itda(can do)} \\
\end{tabular}

\vspace{0.5cm}

\begin{flushleft}
{\it 「よう」の誤謬}
\end{flushleft}

\begin{tabular}{ll}
  (原文)      & 考え込む\underline{よう}にしながら \\
  (翻訳結果)  & saengkakdul\underline{dorok(in order to)} hamyunseo \\
  (正しい翻訳)& saengkake jamkin \underline{duti(as)} hamyeonseo \\
\end{tabular}
\end{quote}

\vspace{0.5cm}

\begin{flushleft}
{\bf 多義性誤謬の考察}
\end{flushleft}

分析誤謬と同じく漢字表記を用いない仮名文字列で誤謬が多く見られた. さらに, 漢字で表記されている動詞の場合も多義語に
関する配慮が見られず, 一つの対訳しか当てられていない. これは動詞の多義性を解決する技法を用いていないことを示している. 
名詞に関しては, 一般の辞書の見出し語で最も頻繁に用いるものとして第一項目にあがっている語を機械的に対訳語として
用いているようである. したがって, 文脈とあわない対訳が多く発見される. 日韓翻訳の開発の中で最も早い時期から先決課題として
叫ばれていた助詞の多義性に関しては「に」と「と」で最も誤謬が多く発見された. 特に「と」の場合は, 「引用, 仮定, 時間的前後」などの
意味をあらわし, そのいずれかを決定するのが難しい. 「J-Seoul」と「名品」は各システムごとに助詞の多義性を解決するための技法を導入し, 
できる限り一つの対訳語だけを出していると見なされるものの, いずれのシステムも満足のいくレベルには達していない. 

\subsubsection{対訳語選定の誤謬}
翻訳単位の認識, 品詞決定が正しく行われ, その単語だけを取り出してみると意味的にも間違っておらず, 多義性の誤謬とも言えないが
文全体からみて意味が通じない場合を対訳語選定の誤謬として分類した. 

\begin{quote}
\begin{flushleft}
{\it 不自然な否定表現}
\end{flushleft}

\begin{tabular}{ll}
 (原文)      & それが\underline{言えない} \\
 (翻訳結果)  & gugeotul malhal \underline{soo itji anta} \\
 (正しい翻訳) & gugeotul malhal \underline{soo eobda} \\
\end{tabular}

\vspace{0.5cm}

\begin{flushleft}
{\it 不自然な受け身表現}
\end{flushleft}

\begin{tabular}{ll}
  (原文)     &  八十ページも\underline{費やされ}ているのに \\
  (翻訳結果) &  \underline{sobihaejigo} \\
  (正しい翻訳) & \underline{halaedoigo} \\
\end{tabular}

\vspace{0.5cm}

\begin{flushleft}
{\it 不自然な使役表現}
\end{flushleft}

\begin{tabular}{ll}
 (原文)     &  \underline{静かめ}ながら \\
 (翻訳結果)  & \underline{joyonghi ha}myunseo(自発) \\
 (正しい翻訳) & \underline{joyonghage ha}myunseo(使役) \\
\end{tabular}

\vspace{0.5cm}

\begin{flushleft}
{\it 不自然なやりもらい表現}
\end{flushleft}

\begin{tabular}{ll}
  (原文)     &  出してみて\underline{いただきたい} \\
  (翻訳結果)  & naeeoboa \underline{joosigo sipda} \\
  (正しい翻訳) & naeeoboa \underline{joosibsio} \\
\end{tabular}
\end{quote}

\vspace{0.5cm}

\begin{flushleft}
{\it 不自然な対訳語}
\end{flushleft}

主に名詞に多い. 韓国語でもあるのはあるが極めてその使用例が希であり, 特に
文の中で不自然であると感じられるものと「不自然な〜」に分類できないものを
「不自然な対訳語誤謬」に分類した. 

\vspace{0.5cm}
\begin{tabular}{ll}
 (原文)     &  大司教の\underline{もと}で \\
 (翻訳結果)  & daejoogyoeui \underline{batang(foundation)}eseo \\
 (正しい翻訳) & daejoogyoeui \underline{mit(beneath)}eseo \\
\end{tabular}

\vspace{0.5cm}
\begin{tabular}{ll}
 (原文)     &  内分泌を\underline{介して}微妙に調節されている \\
 (翻訳結果)  & naeboonbigyelul \underline{kiooeu(insert)} ... \\
 (正しい翻訳) & naeboonbigyelul \underline{maegaero hayeo(interpose)} ... \\
\end{tabular}



\begin{flushleft}
{\bf 対訳語選定の誤謬の考察}
\end{flushleft}

対訳語誤謬は, 分析誤謬などに比べ比較的軽い誤謬である. しかし, これもユー
ザの立場から見ると数回繰り返し読んだり原文と照らし合わせながらでないと意
味を正確に把握できないという点で解決すべき課題であることには間違いない. 

\subsubsection{韓国語生成誤謬}
表~\ref{typoresult}でも示したように上記以外の韓国語生成時の誤謬をここにまとめた. 

\begin{flushleft}
 誤謬の例
\end{flushleft}
\begin{quote}

\begin{flushleft}
{\it 状態/動作}
\end{flushleft}

\begin{tabular}{ll}
 (原文)     &  見ていない \\
 (翻訳結果) &  bo\underline{a it(状態)}ji anta \\
 (正しい翻訳) & bo\underline{go it(動作)}ji anta \\
\end{tabular}

\vspace{0.5cm}

\begin{flushleft}
{\it テンス}
\end{flushleft}

\begin{tabular}{ll}
 (原文)    &  来るの\underline{だった} \\
 (翻訳結果) &  onun geot\underline{ida(現在)} \\
 (正しい翻訳) & onun geot\underline{ieotda(過去)} \\
\end{tabular}

\vspace{0.5cm}

\begin{flushleft}
{\it 慣用表現}
\end{flushleft}

\begin{tabular}{ll}
 (原文)     &  \underline{口をはさむ} \\
 (翻訳結果)  & \underline{ibul kida(insert mouth)} \\
 (正しい翻訳) & \underline{chamgyeonul hada(interfere)} \\
\end{tabular}

\vspace{0.5cm}

\begin{flushleft}
{\it 定型表現}
\end{flushleft}

\begin{tabular}{ll}
 (原文)      & \underline{のために} \\
 (翻訳結果)  & \underline{ki ddaemoone(reason)/ki eooihaeseo(purpose)} \\
 (正しい翻訳) & \underline{ki eooihaeseo(purpose)/ki ddaemoone(reason)} \\
\end{tabular}

\vspace{0.5cm}

\begin{flushleft}
{\it 韓国語冠形形語尾の生成}
\end{flushleft}

\begin{tabular}{ll}
  (原文)      & す\underline{る}時 \\
  (翻訳結果)  & ha\underline{nun} ddae(time something is being done) \\
  (正しい翻訳) & ha\underline{l}ddae(time to do) \\
\end{tabular}

\vspace{0.5cm}

\begin{flushleft}
{\it 韓国語綴り}
\end{flushleft}

\begin{quote}
韓国語の活用, 音韻縮約, 媒介母音挿入など, 韓国語の文法・用法から外れる
場合. 
\end{quote}

\end{quote}


\subsection{Operational Evaluation}
今回対象にした四つのシステムはユーザの使用上の便宜を図るためにある程度の
ユーザインターフェースとユーティリテイを提供している. 表~\ref{function}
に四つの商用日韓機械翻訳システムが提供する機能を示した. 四つのシステム全
てが翻訳原文と翻訳結果文の編集のためのツールを持っている. 日本語入力環境
が具備されていない韓国では日本語文章作成機能と日本語文書構成コードの内部
翻訳処理コードへの変換機能が提供されなければならない. 日本語文書の新規作
成のためには「ハングルカナ」を除く他の三つのシステムは日本語入力ツールを
提供している. 「ハングルカナ」は内部処理コードがKSC5601コードであり, 韓
国語Windows95で提供されるKSC5601日本語コード入力コードをそのまま使ってい
る. コード変換機能は表~\ref{function}で見られる通り, 各システムの内部コー
ドへの変換機能を持っている. 翻訳処理と関連し, 未登録漢字や片仮名を簡単に
ハングル音読変換を用い出力するか, あるいは該当漢字や片仮名を原文通りに出
力させるか, 翻訳処理の途中, 未登録語登録や多義性解消のための翻訳機とユー
ザとの相互作用を支援すべきかなどに関するオプションを備えている. 辞書と関
連し, 基本的に辞書の新規語登録, 更新, 削除などを支援している. 翻訳処理速
度は, 平均文の長さ62.03文字/文, 64個の文を翻訳する時に最も速いものが
7秒, 最も遅いものが2分10秒程度かかった. 翻訳速度の面では, ユーザが不
便さを感じないほどにまでなっているといえそうである. 

\begin{table}
\begin{center}
\caption{\label{function}各システムの機能一覧表}
\epsfile{file=144.eps}
\end{center}
\end{table}

\subsection{Progress Evaluation}
一年前に行った評価と現在の評価とを比較すると, 最も目立つ発展を見せたのは
「ハングルカナ」である. 一年前の翻訳品質分析では「ハングルカナ」が機械翻
訳というより単純な漢語のハングル音読への変換と“について”, “において”
のような表現の単純置換のレベルに過ぎず, 未翻訳誤謬が相当発見された. しか
し, 現在の翻訳品質は未翻訳誤謬が相当減り, 翻訳結果も相当なレベルに達して
いる. これは, 辞書の見出し語の拡充にその発展の原因がある. 一年前に未翻訳, 
形態素分析誤謬と判定されたほとんどの語が辞書に登録されていなかったからで
あったが, 辞書にこれら未登録語を登録することにより問題が大幅に改善された
といえる. 一年前の「ハングルカナ」が Declarative Evaluationで何回読み返
しても意味が分からないといったレベルの低い点数(1.95点)をマークしたのに対
し, 現在は何回か読むと全体の意味が分かるといったレベルにまで向上した. こ
のような事実は, 機械翻訳における辞書の占める役割の重要さを物語るものであ
るといえる.  「名品」もやはり\cite{choiandkim}で問題として指摘された平仮
名列の分析誤謬においてある程度改善された. しかし, Declarative Evaluation
の点数には変化があまり見られない(一年前 - 2.37, 現在 - 2.39). その理由は
\cite{choiandkim}でも述べたように情報量は概念語に従属されるのに比べ概念
語部分に対する誤謬には大きな改善のあとが見られなかったことによる. 
「J-Seoul」はあまり変化が見られない. ただ, "記号前の終止表現", 助詞 "と"
の多義性誤謬などで改善点が見られるが Declarative Evaluationでより良い点
数をもらうには改善が不十分であった. 「オギョンバクサ」は未翻訳, 分析, 多
義性, 韓国語生成の誤謬など全体に渡って多くの誤謬が見られる. これは 
\cite{choiandkim}で「ハングルカナ」の誤謬が未翻訳誤謬に集中していたため
辞書を拡充することにより大きく翻訳誤謬が減ったのに対し, 「オギョンバクサ」
は誤謬が多岐にわたっているため改善にはなお多くの努力が必要であるといえる. 
この他, 四つのシステムに共通していえることとして, 記号の前の終止表現, 助
詞 "と"の多義性の誤謬などはある程度改善されたが, 根本的な解決策を用いた
というより翻訳のcoverageの広い対訳語を生成するといった程度の改善であると
いえる. 

  以上を総合すると,  \cite{choiandkim}以降, 技術的には大きな変化は無く, 辞書や文法の拡張やチュウニングだけで
問題を解決しようとしたと判断される. 



\section{結論}

\subsection{開発への提言}

各システムの翻訳結果を分析した結果, 次のような点が明らかになった. 

\begin{itemize}
\item
韓国語と日本語との類似性, すなわち, 文の構造, 語彙などの 類似性に依存しすぎている
\item
正確な分析に基づく翻訳になっていない 
\item
名詞はもちろん動詞においても日本語の漢字表記語をそのまま韓国語の漢字音に変換するといった極めて単純な翻訳に依存している
\item
固有名詞, 日本語の和語, カタカナ表記外来語などが辞書に登録されていない場合が多い
\item
述部の翻訳において複雑な形態素の結合に十分対応できていない
\item
日本語独特の造語法に十分対応できていない
\item
状態/動作に関する区別など, 言語学的にはほぼ解決の糸口がつかめつつある問題に関しても注意が払われていない音韻の縮約, 
媒介母音の挿入, 述語の活用など, 韓国語の綴りに十分対応していない
\item
動詞と名詞の多義性への配慮の欠如
\end{itemize}

\newpage

\begin{flushleft}
このような問題を解決するため次のような課題を提案する. 
\end{flushleft}

\begin{flushleft}
{\bf 短期課題}
\end{flushleft}

\begin{enumerate}
\item 辞書の拡張 \\
 主に外来語の表記を担当するカタカナ, 漢字で表記される固有名詞, 日本語の和語をそのまま韓国語の発音に機械的に変換するといった極めて単純な処理に依存せず, これらについては辞書に登録して正確な韓国語の対訳を生成すべきである.  実際の実験結果, 未翻訳誤謬のほとんどは人名, 地名, 団体名などといった固有名詞の未登録により発生している. 特に, 最近では日韓機械翻訳機の使い道がWWW情報の翻訳にまで拡大していく傾向にあることを考えると, WWWで頻繁に用いられる語彙に対する用語調査と登録が必要である. 
\item 韓国語表記法に合った翻訳文の生成 \\
 韓国語は活用が多く, 分かち書きをする言語である. したがって, 活用や分かち書き誤謬はユーザの翻訳結果の理解に大きな障害となり, 時には意味上の誤謬まで生み出す結果にもなり得る. このような韓国語生成技術はある程度確立されている. このような技術の積極的な導入が必要である. 
\item 日本語分析誤謬の解決 \\
 分析性能の向上は必ず解決しなけらばならない最も重要な課題である. これが不安な状況でその後の多義性誤謬のための技法などは効果が得られないからである. 効用性が立証された技法を利用して日本語分析の性能を向上させなければならない. 特に, 主にひらかな文字列で表現される機能語部分の分析に力を注ぐ必要がある. 
\end{enumerate}

\begin{flushleft}
{\bf 中長期的課題}
\end{flushleft}


\begin{enumerate}
\item
大量の日韓対訳コーパスの構築 \\
 現在の日韓機械翻訳システムは一般の文法書と一般辞書から文法と辞書を構築する演繹的(deductive)な方法を採用して来た. 
したがって, 一般の文法書に記述されている定型化された言語現象についてはある程度性能を発揮するが, 実際の言語生活で
用いられる日本語に対しては対応しきれない面を数多くもつ. これは現在の日韓機械翻訳技術の限界でもある. この限界を乗り越え, 
日韓機械翻訳技術を向上させていくためには実際の人間の言語を反映する大量の日韓対訳コーパスの構築が必要である. 大量の
日韓対訳コーパスから日本語と韓国語の類似性, 相違性についての分析と翻訳知識の帰納的構築によりより実生活で用いられる
日本語や韓国語の言語現象に機械翻訳が対応できるようになるであろう. 
\item
日韓翻訳知識自動獲得ツールの開発 \\
 人間による翻訳知識の構築は多くの費用と時間が必要であり, 構築された翻訳知識の一貫性を見出すことも困難である. したがって, 
大量の日韓対訳コーパスを用い翻訳に必要な各種の言語情報を自動的に抽出できるツールの開発によりより客観的でパワーフルな
翻訳知識の構築が必要である. 
\item
日韓機械翻訳のための意味情報構築および活用 \\
 現在の日韓機械翻訳システムは形態素あるいは常用句のレベルで直接機械翻訳方式を採用している. 意味の処理においては
意味素性分類に基づく格構造あるいは連語情報を用いた多義性の解消技法(Park 1995)を採用しているが, 意味素性の分類, 
日本語-韓国語の単語間の意味範囲の違いなどについての研究が立ち後れており期待通りの結果が得られていない. 
多義性の処理について各システムが実際どれほどの処理能力を持っているのかを把握するために「乗る」「のる」という動詞を
例に実験した. その結果を図~\ref{seman}に示す. 図~\ref{seman}でも見られるように定型的で単純な文構造ではある程度多義性が解消されているように
見える. しかし, 実際のテキストでは全くといって良いほどその機能を発揮していない. これは多義性の解消のための意味情報と
その処理技法が断片的で十分でないことを物語るものである. 大量の日韓対訳コーパスを用い客観的で一貫性のある意味情報の
獲得と日本語-韓国語の単語間の意味範囲の違いを究明することにより正確な意味分析と対訳語選定が可能な環境を作る必要がある. 
\end{enumerate}

\begin{figure}
\begin{center}
\caption{\label{seman}単純定型文に対する多義性処理結果(名品)}
\epsfile{file=147.eps}
\end{center}
\end{figure}



\subsection{おわりに}
 本技術資料では, 四つの商用システムを対象に直接翻訳実験を行った結果に基づきその翻訳品質について評価した. 
その結果, 現在のシステムは, 片仮名, 日本語の漢語, 和語の韓国語への単純音訳, 一対一の対訳語の選択などといったレベルでの
翻訳であることが分かった. これは結果的に現在のシステムが日韓の言語的類似性に頼りすぎていることを物語るものである. 
さらに, すでに応用可能な技術の導入にも消極的であることが分かった. 今後は両言語の類似性より両言語の相違点に焦点を
当てるべきである. 
  現在のシステムも入力テキストに関する何らかの制限も設けず汎用として販売するよりは4.2の“Declarative Evaluation”で
明らかになったように, 使用目的, 翻訳対象文の制限(漢字表記が多く用いられる政治, 経済, 法律, 科学 関連 乾燥体文書)などといった
使用分野を特定すれば機械翻訳システムの利用環境を十分向上させることができ, 一般ユーザからも肯定的な評価を得ることができると
思われる. このような翻訳品質および使用範囲に関する努力とユーザの便宜性を考慮した様々な機能とツールを支援するための研究も
おろそかには出来な課題の一つである. 
商品化されたシステムの出現は市場からの技術改善要求をフィドバックしながら技術の向上がはかれるという技術進化の一つの過程に
位置する. 今回の調査結果を発展の過程から評価するなら, その間の進展について高く評価すべきである. 




\bibliographystyle{jnlpbbl}
\bibliography{v05n4_08}







\begin{biography}
\biotitle{略歴}
\bioauthor{金泰完}{
1985年韓国漢陽大学校大学院電子工学科修士課程卒業.
1998年現在韓国科学技術院電算学科博士課程.
1985年2月よりETRI コンピュタ, ソフトウェアー研究所, 自然語処理研究部勤務.
機械翻訳,自然言語処理,知識情報処理の研究に従事.
言語処理学会等の会員.
}

\bioauthor{崔杞鮮}{
1986年韓国科学技術院電算学科卒業.工学博士.
韓国科学技術院電算学科助教授,副教授を経て,1998年より韓国科学技術院電算学科教授.
1987--1988年 NEC,C\&C Information Research Laboratories 招請研究員.
1997年 Stanford大学,CSLI 訪問教授.
韓国語情報処理,機械翻訳,専門用語,認知科学,多国語情報検索,知識情報処理等の研究に従事.
韓国文化体育部,国立図書館,文化芸術財団諮問委員.
CPCOL,言語処理学会 編集委員.
現在 FIPA,TC6 議長.}


\bioreceived{受付}

\bioaccepted{採録}



\end{biography}


\end{document}

