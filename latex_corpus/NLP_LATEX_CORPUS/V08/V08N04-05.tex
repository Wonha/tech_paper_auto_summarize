



\documentstyle[epsf,jnlpbbl,here]{jnlp_j}

\setcounter{page}{71}
\setcounter{巻数}{8}
\setcounter{号数}{4}
\setcounter{年}{2001}
\setcounter{月}{10}
\受付{2000}{12}{15}
\再受付{2001}{4}{23}
\採録{2001}{6}{29}
\setcounter{secnumdepth}{2}

\title{日本語読み上げ文の係り受け解析における\\韻律的特徴量の有効性}
\author{廣瀬 幸由\affiref{SONY} \and 尾関 和彦\affiref{UEC}
 \and 高木 一幸\affiref{UEC}}

\headauthor{廣瀬, 尾関, 高木}
\headtitle{係り受け解析における韻律的特徴量の有効性}

\affilabel{SONY}{ソニー株式会社}
{Sony Corporation}
\affilabel{UEC}{電気通信大学}
{The University of Electro-Communications}

\jabstract{
韻律には発話が文字化されると失われてしまう情報が含まれているが,
そのような情報は発話文の構文解析に有効である可能性がある.
我々のグループでは,以前の研究で12種類の韻律的特徴量を取り上げ,それらと
係り受け距離の関係を表現する統計モデルを構成した.そして,
そのモデルを組み込んだ係り受け解析器を用い,韻律情報が実際に読み上げ文
の係り受け解析に有効であることを示した.本研究では新たな特徴量を加えて
24種類の韻律的特徴量を取り上げ,有効な特徴量を広い範囲で探索した.
また, 統計モデルを特徴量の現実の分布によりよく当てはまるように
修正した.その結果,
ATR 503文データベースを用いたオープン実験において,韻律的特徴量を
用いることにより,係り受け解析の文正解率が 21.2\%向上した.これは, 
我々のグループの以前の実験における向上率より4.0ポイント高い.特徴量の中で
ポーズ長はクローズド実験においてもオープン実験においても非常に有効であっ
たが,これと併用したときの,ピッチやパワー,話速等に関連する他の特徴量
の有効性はオープン実験においてはあまり明らかでなかった.
}

\jkeywords{韻律情報,韻律的特徴量,係り受け解析,総ペナルティ最小化法}

\etitle{Effectiveness of Prosodic Features in Dependency
\\Analysis of Read Japanese Sentences}
\eauthor{Yukiyoshi Hirose \affiref{SONY} \and Kazuhiko
Ozeki \affiref{UEC} \and Kazuyuki Takagi \affiref{UEC}}

\eabstract{
Prosody contains information that is lost when utterances are
transcribed into letters or characters. Such information may be
useful for syntactic analysis of spoken sentences. In our previous work,
we took up 12 prosodic features, and made a statistical model to
represent the relationship between those features and dependency
distances. Then, using a dependency analyzer that incorporates the
model, we have shown that prosodic information is in fact effective
for dependency analysis of read Japanese sentences. In the present
work, we employed 24 features including new ones, and conducted an
extensive search for effective ones. Also, the statistical model
was modified to better
fit the actual distributions of the feature values. 
As a result, in open experiments
using the ATR 503-sentence database, the correct parsing rate was
improved by 21.2\%  with the use of the prosodic features. This figure 
is 4.0 points higher than the improvement in the previous experiment 
of our group.
 Among the features, the duration of pause was definitely
effective in both the open and the closed experiments, while
the effectiveness of
other features related to the pitch, the power, and the speaking 
rate, when used together
 with the duration of pause, was not clear in the open experiments.}

\ekeywords{prosodic information,prosodic feature, dependency analysis,
minimum penalty parser}

\begin{document}
\thispagestyle{plain}
\maketitle



\section{まえがき}
音声認識技術の進歩により,最近は文章入力を音声で行うことも可能に
なって来ている.文章を音声で入力する場合には,音声を文字化すると
失われてしまう韻律のような情報も言語処理に利用できる可能性がある.
韻律には,多様な情報が含まれているが,その中で構文情報に着目した
研究がこれまでにいくつか行われている.

\cite{UYE}は
読み上げ文のポーズやイントネーションを観察し,それらが文の
構文構造と関連を持つことを明らかにした.この結果は,もし韻律情報が
得られるならば,それを構文解析のための知識源の一つとして利用できる
可能性を示唆している.\cite{KOM}は韻律情報を用いて
隣接句間の結合度を
定義し,結合度の弱い句境界から順に分割して行くことにより,構文木
に似た構造が得られることを示した.また,
\cite{SEK}は隣接句間の修飾
関係の有無の判定に韻律情報が有効であることを報告している.これらの研究は,
韻律と構文構造の関係を取り扱ってはいるが,実際に韻律情報を通常の意味の
構文解析に利用したものではない.これに対して,\cite{EGU}は
5種類の韻律的特徴量を取り上げ,それらと係り受け距離の統計的な関係を,
総ペナルティ最小化法\cite{OZE-1}を用いて係り受け解析を行う際の
ペナルティ関数に組み込むことにより,韻律情報を用いない場合に比べて
解析精度が向上することを見い出した.そして,そこで取り上げられた
韻律的特徴量の中では文節間のポーズ長が最も有効であることを報告している.
その後,同じ枠組みの中で韻律的特徴量の種類を増やし,また
対象話者数を拡大して,特徴量の有効な組合せを求める研究や,
特徴量の話者独立性に関する検討が行われている\cite{KOU-1,OZE-2,OZE-3,OZE-4}.

総ペナルティ最小化法を用いたこれら一連の研究においては,
韻律的特徴量が正規分布することが仮定されている.しかし,実際の
分布は正規分布とはかなり異なっている.したがって,特徴量の
分布を近似するための分布関数を改良することにより,韻律情報をより有効に
利用できる可能性がある.また,これまでに取り上げられていない
韻律的特徴量の中に有効性の高いものがある可能性もある.
そこで本研究では,まず韻律的特徴量,特に最も有効とされるポーズ長に
対する分布関数の改良を試みた.また,韻律的特徴量を従来の12種類
\cite{OZE-4}から
24種類に増やし,日本語読み上げ文の係り受け解析におけるそれらの
有効性を実験的に検討した\cite{HIR}.

\section{係り受け解析}

日本語文の構文構造は,文節間の広義の修飾・被修飾関係である「係り受け」
という考え方に基づいて記述することができる\cite{HAS}.すなわち
日本語文の構文構造は,文中のどの文節がどの文節に係るかを指定する
ことにより決定される.いま,文を文節列 $x_{1} x_{2} \cdots x_{m}$ 
で表し,文節 $x_{i}$ の係り先,すなわち $x_{i}$ を受ける唯一の文節を
 $x_{c(i)}$ で表せば,$c$ は $\{1, 2, \ldots , m-1\}$ から
  $\{2, 3, \ldots , m\}$ への写像となる.この写像は次の性質を
持つ\cite{YOS}.
\vspace{5mm}
\begin{itemize}
\item 後方唯一性:\ $i< c(i)$\ \ ($i=1,2,\ldots ,m-1$)\\
(唯一性は $c$ が写像であるということに,すでに含まれている.)
\item 非交差性:\ $i< j$ ならば,
$c(i)\leq j$ または $c(j) \leq c(i)$\ \ ($i, j=1,2,\ldots ,m-1$)
\end{itemize}
\vspace{5mm}
上の2つの条件を満たす写像 $c$ を,ここでは $x_{1} x_{2} \cdots x_{m}$ 
の上の係り受け構造という.係り受け構造 $c$ が定まっているとき,
$(x_{i}, x_{c(i)})$ を係り受け文節対という.
また,$c(i)-i$ を文節 $x_{i}$ と $x_{c(i)}$ 
の間の係り受け距離,あるいは単に $x_{i}$ の
係り受け距離という.$(x_{i},\ x_{j})$ が係り受け文節対であるか否かに
関わらず,$j-i$ を $x_{i}$ と $x_{j}$ の間の文節間距離,あるいは
単に距離という.

文長 $m$ が3以上の場合には複数の係り受け構造が存在するが,
そのすべてが妥当な構文構造を表すわけではない.したがって,さらに
制約条件を加え,係り受け構造の中から妥当な構文構造を見い出す必要が
ある.古典的な係り受け解析においては,文節 $x, y$ の属性値によって
$x$ が $y$ に係ることが許されるか否かが決まっていると考え,これを
制約条件として用いることが多かった.しかし,それだけでは大きな
構文的曖昧性が残るので,最近では「許されるか否か」の2値情報では
なく「許される程度」を考え,
それを確率,整合度,選好度など\cite{FJO,EHA,UTU}の実数値で表す
ことが試みられている.本研究で用いた係り受け解析法である
「総ペナルティ最小化法」では,$x$ が $y$ に係ることの困難さを,
非負の実数値を取るペナルティ関数 $F(x, y)$ で表す.そして,総ペナルティ
\begin{equation}
\sum_{i=1}^{m-1} F(x_{i}, x_{c(i)}) 
\end{equation}
が最小になる係り受け構造 $c$ を見い出す.この問題は動的計画法の
原理に基づき効率良く解くことができる\cite{OZE-1}.

ペナルティ関数 $F(x, y)$ には種々の言語的知識を組み込むことが
できる.本研究では,学習データから得られる韻律と係り受け距離に
関する統計的知識を組み込む.

本研究では,この他に,韻律情報を用いない場合の解析精度を知るため,
「決定論的解析法」と呼ばれる係り受け解析法\cite{KUR}を用いた.
この方法では,2文節間に係り受けが許されるか否かの2値情報に基づき,
文末文節から順にその文節を受ける文節を決定していく.その文節を
受けることができる文節が複数存在する場合は,最も距離が近い文節を
受け文節として採用する.

\section{データベース}
本研究で使用したATR音声データベース(セットB)
\cite{ATR}について
簡単に説明する.
このデータベースには,新聞,雑誌,小説,手紙,教科書等の
出版物から抽出された503文が含まれている.これらの文はAから
Jまでの10グループに分けられており,各グループには50文 (グループ
Jだけは53文)が含まれている.総文節数は3425であり,文末の文節は
係り先を持たないので,全部で 2922 の係り受け文節対が存在する.
各文には,表記の他に品詞情報や各文節の係り受け距離などを表すラベルが
付されている.このような言語情報の他に,このデータベースには
これらの文を読み上げた音声データが含まれており,音韻や
ポーズの位置などを示すラベルが付けられている.
データベース全体としては,
男性6名,女性4名のアナウンサー/ナレーターの音声データが含まれて
いるが,
本研究では,その中の男性2名(MHT,MTK),女性2名(FKN,FYM)
の音声データを用いた.これらの発声者はすべてナレーターである.
MHT,FKN,FYM の3名については,ピッチ(基本周波数)データが
データベースに含まれているので,それを利用した.
MTK については,ピッチデータが含まれていなかったので,
ラグ窓法\cite{SAG}により抽出した.

\section{韻律情報}

係り受け解析とは,文中の各文節がどの文節に係るかを定めることで
あるから,各文節の係り受け距離に関する何らかの情報があれば,それは
係り受け解析のための有効な情報となる.したがって,各文節の韻律的特徴量と
係り受け距離の間の関係が分かれば,韻律的特徴量は係り受け解析の
ための有効な情報となるはずである.本研究では,このような考え方に
基づいて韻律情報の利用を図る.

\subsection{係り受け距離とポーズ長}
まず,係り受け距離と関係がある韻律的特徴量の例として,着目している文節と
その直後の文節の間のポーズ長を取り上げる.
ポーズ長は,\cite{EGU}および,その後の一連の研究で採用された
韻律的特徴量の中で,
最も有効性が高いと報告されているものである.

図 \ref{pau1} は,男性話者 MHT について,1 から 5 まで
の係り受け距離ごとに,ポーズ長の相対頻度分布を示したものである.
このグラフから,どの係り受け距離に対しても,頻度が一度極端に
少なくなった後で再び上昇する傾向のあることが分かる.(以後,この
ように落ち込んだ部分を「ディップ」と呼ぶ.)
そして,係り受け距離によって固有の分布を持つことが知られる.
このことは,ポーズ長が係り受け距離に関する情報を含んでいることを
意味している.
なお,参考までに係り受け距離そのものの頻度分布を
表\ref{kakarifreq} に示す. 
これより,係り受け距離が大きくなると頻度が急激に減少し,
1 から 3 までの係り受け距離が全体の
 90\%以上を占めていることが分かる.
\begin{figure}
\begin{center}
\atari(130,89.3)
\caption{ポーズ長の相対頻度分布 (男性話者 MHT).\\(係り受け距離1,2に対し
てはポーズ長0の相対頻度はスケールの上限を越えている.)}
\label{pau1}
\end{center}
\end{figure}

\small
\begin{table}
\caption{係り受け距離の頻度分布}
\label{kakarifreq}
\begin{center}
\begin{tabular}{|c||c|c|c|c|c|c|c|c|c|c||c|}
\hline
係り受け距離&1   &   2&  3&  4& 5& 6& 7&8&9&10&計\\\hline
文節数&1909&500 &253&126& 73& 35& 13&  9&  3&1&2922  \\
相対頻度(\%)&65.3&17.1&8.7&4.3&2.5&1.2&0.4&0.3&0.1&0.0&100.0\\
\hline
\end{tabular}
\end{center}
\end{table}
\normalsize

\subsection{韻律的特徴量}
文節 $X$ に対する韻律的特徴量は,文節 $X$と,主としてその直後の
文節 $Y$ が持つ物理量の
相対的関係から抽出する.測定する物理量は,ポーズ長,ピッチ曲線,パワー曲線,
発話速度などである.ポーズ長はそのまま特徴量とするが,ピッチ曲線や
パワー曲線からは何に着目するかによって
種々の特徴量が抽出できる.発話速度に関しても,どの部分の速度に着目する
かによって異なった特徴量が得られる.本研究では,以下のような24種類の
特徴量を取り上げた.ピッチやパワーの値は対数をとっている.
図\ref{prosody3}は1$\sim$16 の特徴量を模式的に示したものである.
図中の番号は上記の番号に対応している.
19$\sim$24 は,アクセントコマンドやフレーズコマンド
\cite{FUJ}を,
東京大学新領域創成科学研究科 広瀬研究室で作成された
韻律解析プログラム``{\it PROSODY}''\cite{MIN}によって推定し,
それらから求めた特徴量である.
\vspace{5mm}

\begin{itemize}
\item[1),2)] $X$,$Y$ 間のポーズ長(ポーズ長),
および$X$ の直前のポーズ長(ポーズ長前)
\item[3),4)] $X$ の末尾の母音継続時間長(母音長前),
および$Y$ の先頭の母音継続時間長(母音長後)
\item[5)] $X$ のピッチ曲線にあてはめた回帰直線の傾き(ピッチ傾)
\item[6),7)]$X$ のピッチ曲線の最大値までの,
および最大値後の回帰直線の傾き(それぞれ,ピッチ傾前,ピッチ傾後)
\item[8)] $X$ の回帰直線の終端値と $Y$ の回帰直線の始端値
の差(ピッチ差1)
\item[9)] ピッチ曲線を文節内の最大値で分割し,それぞれに
あてはめた回帰直線によるピッチの差(ピッチ差2)
\item[10)]$X$, $Y$ のそれぞれに対するピッチ曲線の平均値
の差(ピッチ平均)
\end{itemize}

パワーについても上記 {\it 5-10} と同様に抽出
\begin{itemize}
\item[11)](パワー傾),12)(パワー傾前),13)(パワー傾後)
\item[14)](パワー差1),15)(パワー差2),16)(パワー平均)
\end{itemize}
\vspace{5mm}
\begin{itemize}
\item[17),18)]\ \  $X$,および $Y$ の 1秒当たりの
平均モーラ数(それぞれ,平均モーラ数1,平均モーラ数2)
\item[19),20)]\ \  $X$ の直前,および直後にある
フレーズコマンドの大きさ(それぞれ,フレーズコマンド前,
フレーズコマンド後)
\item[21)] $X$ の終端から $X$ の直後にあるフレーズコマンドまでの
時間(フレーズコマンド時間)
\item[22)] X の直前,および直後にあるフレーズコマンドの
大きさの差(フレーズコマンド差1)
\item[23)] 22) を $X$ の継続長で割ったもの(フレーズコマンド差2)
\item[24)] $X$ 内のアクセントコマンドの数を $X$ の継続長で割った
もの(アクセントコマンド数)
\end{itemize}
\vspace{5mm}

これらの中で,1), 3), 4), 5), 8), 9), 10), 11), 14), 15), 16) 
の11種類の特徴量は\cite{OZE-4}で取り上げられたものである.
なお,同論文では 3) と 4) の比も特徴量として取り上げられているが,
明確な有効性は報告されていないので,本研究では取り上げなかった.
19) $\sim$ 22)において,$X$ の直前(直後)のフレーズコマンドとは,
$X$ の始端(終端)から一定の時間的しきい値内にあるフレーズコマンドの
中で始端(終端)に最も近いものである.そのようなフレーズコマンドが
存在しない場合には,その大きさを0とした.

\begin{figure}
\begin{center}
\atari(130.1,43.5)
\atari(65.8,43.5)
\caption{韻律的特徴量の模式図}
\label{prosody3}
\end{center}
\end{figure}

\section{ペナルティ関数}
\vspace{-1mm}
総ペナルティ最小化法を用いて韻律的特徴量の有効性を
調べるため,文節 $x$ が文節 $y$ に係ることの困難さを表す
ペナルティ関数 $F(x,\ y)$ を,学習データから得られる韻律的特徴量と
係り受け距離に関する統計的知識に基づいて定義する.

まず,文中のある文節の係り受け距離を $d$ とし,その文節に対する
 $n$ 個の韻律的特徴量を成分とする
ベクトルを $\mbox{\boldmath$p$}_n=(p_1,\ldots,p_n)$ とする.そして,
$\mbox{\boldmath$p$}_n$ が与えられたときの $d$ の条件付き確率
を $P(d|\mbox{\boldmath$p$}_n)$ とする.
$P(d|\mbox{\boldmath$p$}_n)$ はベイズの定理により

\begin{equation}
P(d|\mbox{\boldmath$p$}_n) =
\frac{P(\mbox{\boldmath$p$}_n|d)P(d)}
{ \sum_{d} P(\mbox{\boldmath$p$}_n|d)P(d)}
\end{equation}
と書き直すことができる.
したがって,$P(\mbox{\boldmath$p$}_n|d)$ と $P(d)$ が
分かれば $P(d|\mbox{\boldmath$p$}_n)$ が求められる.
$P(\mbox{\boldmath$p$}_n|d)$ は,第$i$特徴量に対する条件付き確率分布 
$P_{i}(\ \cdot\ \mid d)$ を学習データ中の係り受け距離 $d$ の文節に対する
第$i$特徴量の実際の分布から推定し,また,それらの独立性を仮定して

\begin{equation}
P(\mbox{\boldmath$p$}_n|d) = \prod_{i=1}^{n}P_{i}(p_{i}|d)
\end{equation}

\noindent
により推定する.$P_{i}(\ \cdot\ \mid d)$ の具体的な推定法については後で
述べる.また,学習データ中の係り受け距離が $d$ である文節数を
$N_d$ とすれば,$P(d)$ は

\begin{equation}
P(d) = \frac{N_d}{\displaystyle\sum_{d} N_d}
\end{equation}
により推定できる.

さて,文節 $x$ が文節 $y$ に係ることができるか否かは,それらを
構成する形態素によってかなりの程度定まっている.
そこで,これを「係り受け規則」として表し,文節 $x$ が文節 $y$ に係るこ
とがその規則によって許されないときは $\infty$ のペナルティを与える.
また,
許されるときは $P(d|\mbox{\boldmath$p$}_n)$ を用いてペナルティを定める
ことにする.すなわち,$d(x, y)$ を $x$ と $y$ の間の距離として,
ペナルティ関数 $F(x,y)$ を次のように定義する\cite{EGU}.

\begin{equation}
F(x,y) = \left\{
\begin{array}{ll}
\infty,       & x\ \mbox{が}\ y\ \mbox{に係ることが規則によって許さ
れない場合} \\
-\log P(d(x,y)|\mbox{\boldmath$p$}_n), & x\ \mbox{が}
\ y\ \mbox{に係ることが規則によって許される場合} 
\end{array}\right.
\label{eqn:6}
\end{equation}
係り受け規則は\cite{KUR}の考え方に基づいて
人手で作成したもの\cite{KOU-1}を用いた.この係り受け規則は,
決定論的解析法においても共通に使用する.
使用したデータベースに対するこの
規則の係り受け被覆率,すなわち,データベース中のラベルによって
示される2922個の係り受け文節対の中で,
この規則により係り受けが許される文節対の割合は92.6\%であった.
また,文被覆率,すなわち,503文の中でラベルによって示される係り受け
構造がこの規則により許される文の割合は 73.0\% であった.

\section{韻律的特徴量の有効性}
係り受け解析における韻律的特徴量の有効性を文正解率,すなわち評価文の中で
解析結果がデータベースのラベルで示される係り受け構造と一致する文の
割合によって評価する.
また,学習データと評価データの組合わせを変えたときの文正解率と
係り受け正解率の違いを,それぞれ文長と係り受け距離ごとに観察する.
\subsection{学習データと評価データ}
3節に述べたデータベースを
学習データと評価データに用いた.そのときの条件を表\ref{cond:1}に示す.
表中の Exp(i) はクローズド実験のための条件である.
使用できるデータ量が少ない場合のオープン実験においては,
学習データと評価データの役割を入れ替えて複数回の実験を
行う,クロス・バリデーション\cite{JEL,MAN}
を用いるのがよいとされている.
クロス・バリデーションにも,単純なものから複雑なものまで
種々の変形が考えられるが,ここで用いるデータ量や研究目的に
対してどれが
最適であるかは,現時点では不明であるので,
ここでは,全データを学習データと
評価データに分割する仕方を変えたデータセットを2組用意するという
単純な方法を採用した.学習データと
評価データの最適な分割比率も不明であるが,学習データを
多めに取り,学習データと評価データの量の比が約7対3となるように分割
した.また,2つのデータセットの評価データは,重なりがなく,
読み上げ方が異なる
可能性が大きい部分を選ぶこととした.すなわち,
データセット Exp(ii)では 503文
リストの最初の3グループを評価データ,残りを学習データとし,一方,
Exp(iii)ではリストの
最後の3グループを評価データ,残りを学習データとしている.
これらのデータセットに対する解析結果の平均を
韻律的特徴量の有効性の評価値とすると共に,それぞれに対する結果の
違いも観察することとした.
なお,全て話者依存実験である.すなわち,学習と評価は
同一話者の文音声を用いて,話者ごとに行っている.

\begin{table}
\caption{学習データと評価データに関する実験条件}
\label{cond:1}
\begin{center}
\begin{tabular}{|c|c|c|}
\hline
&
学習データ & 評価データ\\
\hline
Exp(i)&A-J(503 文,3425 文節)&A-J(503 文,3425 文節)\\
Exp(ii)&D-J(353 文,2409 文節)&A-C(150 文,1016 文節)\\
Exp(iii)&A-G(350 文,2505 文節)&H-J(153 文,920 文節)\\
\hline
\end{tabular}
\end{center}
\end{table}


\subsection{ポーズ長に対する分布関数}

韻律的特徴量の中で,
特にポーズ長が係り受け解析に有効であることが報告されている
\cite{EGU,KOU-1,KOU-2}.
また,ポーズ長は,
図 \ref{pau1} に示したように,係り受け距離によって特異な分布を持っている.
そこで,まずポーズ長のみに着目して,使用する分布関数を係り受け距離ごとに
変え,どのような分布関数の組合せが有効であるかを調べた.
実験条件は Exp(i) である.

取り上げた分布関数は,正規分布,ポアソン分布,指数分布,
および相対頻度分布である.係り受け距離 1,2,3 に対しては,
これらの分布関数の全ての組合わせを試みた.
また,係り受け距離 4 以上に対しては,全て相対頻度分布を
用いた.相対頻度分布は,各特徴量の最大値と最小値の間を30分割して
求めた.

相対頻度分布以外の分布関数の平均値パラメータを推定するとき,
\begin{itemize}
\item[(A)] 全てのデータから平均値パラメータを推定
\item[(B)] ディップより大きい値を持つデータのみから
平均値パラメータを推定
\end{itemize}
の2つの場合を比較すると,(B)の方が文正解率が高いことが予備実験において
観察された.そこで,(A),(B) それぞれの場合において文正解率が上位であった
分布関数の組合わせを3つずつ選んだ.
その結果を表3に掲げる.$C_{1}\sim C_{3}$ が場合(A),
$C_{4}\sim C_{6}$ が場合(B)から選んだものである.

\begin{table}
\begin{center}
\caption{係り受け距離1$\sim$3に対するポーズ長の分布関数の組合せ.\\
係り受け距離4以上に対しては,すべて相対頻度分布を使用.}
\label{cond:2}
\small
\begin{tabular}{cl}
\hline
\hline
$C_1$& 係り受け距離 $1\sim3$ の全てに対して正規分布で近似.\\
$C_2$& 係り受け距離 $1\sim3$ の全てに対してポアソン分布で近似.\\
$C_3$& 係り受け距離 1 に対しては指数分布で近似.
 $2, 3$ に対しては正規分布で近似.\\
\hline
$C_4$& 係り受け距離 $1\sim3$ の全てに対して正規分布で近似.\\
$C_5$& 係り受け距離 2 に対しては相対頻度分布を使用.
 1, 3 に対しては正規分布で近似.\\
$C_6$& 係り受け距離 1, 2 に対しては正規分布で近似.
 3 に対してはポアソン分布で近似.\\
\hline
\hline
\end{tabular}
\normalsize
\end{center}
\end{table}


これらの分布関数の組合せに対する文正解率を表\ref{res:3} に示す.
同表において``距離情報''は,式(\ref{eqn:6})において
$P(d(x,y)|\mbox{\boldmath$p$}_n)$ の代りに,その事前確率,
すなわち文節間距離の確率 $P(d(x,y))$ を
使用したときの文正解率である.
また,``決定論的''は,決定論的解析法を用いた時の文正解率を表す.

$C_{1}$ の文正解率は,係り受け距離4以上に対しても
正規分布を適用した場合と全く同じであった.したがって,
係り受け距離4以上に対して従来の正規分布の代りに相対頻度分布を
用いても,文正解率は向上も低下もしなかったことになる.

\begin{table}[htbp]
\begin{center}
\caption{分布関数の組合せに対する文正解率(\%)}
\label{res:3}
\small
\begin{tabular}{|c|c|c|c|c|c|}
\hline
組合せ	&MHT	&MTK	&FKN	&FYM	&平均\\
\hline
\hline
$C_1$   &57.7   &57.3   &54.5   &55.3   &56.2\\
$C_2$   &57.9   &56.9   &57.1   &55.5   &56.9\\
$C_3$   &56.7   &56.7   &55.3   &56.1   &55.8\\
\hline
$C_4$   &59.6   &57.3   &57.3   &55.5   &57.4\\
$C_5$   &59.4   &57.3   &57.1   &55.7   &57.4\\
$C_6$   &60.2   &56.9   &57.3   &55.7   &57.5\\
\hline
距離情報&	&	&	&	&52.3\\
\hline
決定論的&	&	&	&	&47.3\\
\hline
\end{tabular}
\normalsize
\end{center}
\end{table}

組合せ $C_4$,$C_5$,$C_6$ の中
では文正解率の違いはあまり見られない.しかし,
組合せ $C_4$,$C_5$,$C_6$ の方が,組合せ $C_1$,$C_2$,$C_3$ よりも全
体的に正解率が高い.すなわち,平均値パラメータの推定法として,方法(A)
より方法(B)の方が良い.この理由は解明できていないが,ポーズ長は値が
0のデータが多く,またディップを持つという特異な分布によるものと思われ
る.

表\ref{res:3} の 4 名の話者の文正解率の平均から,文正解率が,
係り受け規則に係り受け距離の頻度情報を加えることで 5 ポイント,さらに
ポーズ情報を用いることで約 5 ポイント向上することが分かる.

\subsection{ポーズ長と他の一つの韻律的特徴量の組合せ}

組合せ $C_4$,$C_5$,$C_6$ に対する文正解率は話者により高低があるので,
どの組合せが最適かが明確でない.そこで,ポーズ長にもう一
つ韻律的特徴量を組合せて結果を比較する.
ポーズ長以外の韻律的特徴量の分布関数としては,相対頻度分布を用いた.

表\ref{res:4} は,ポーズ長と他の一つの特徴量の組合せで,話者 4 名の
文正解率の平均値が高かったものを
上から順に示している.実験条件は Exp(i) である.


\begin{table}
\begin{center}
\caption{ポーズ長と他の一つの特徴量を組合わせたときの文正解率(\%)}
\label{res:4}
\begin{tabular}{|c|c||c|c|c|c|c|}
\hline
ポーズ長分布関数  & 組合せる特徴量 &MHT &MTK &FKN &FYM  &平均\\
\hline
\hline
$C_5$	&平均モーラ数2	&
     60.4 & 58.4 & 58.1 & 57.1	&58.5\\
$C_5$	&ピッチ傾	&
     60.6 & 57.1 & 57.7 & 56.1	&57.9\\
$C_4$	&平均モーラ数2	&
     60.6 & 57.1 & 57.7 & 56.1	&57.9\\
$C_5$	&パワー差1	&
     59.4 & 57.3 & 57.9 & 56.5	&57.8\\
$C_4$	&パワー差2	&
     60.4 & 57.7 & 57.3 & 55.7	&57.8\\
$C_4$	&パワー傾前	&
     60.4 & 57.1 & 57.3 & 56.3	&57.8\\
$C_6$	&平均モーラ数2	&
     59.8 & 57.5 & 57.7 & 56.3	&57.8\\
$C_5$	&アクセントコマンド数&
     60.4 & 58.1 & 57.1 & 55.5	&57.8\\
\hline
\end{tabular}
\end{center}
\end{table}

表\ref{res:4} の結果より,組合せ $C_5$ によってポーズ情報を利用し,
もう一つの特徴量として ``平均モーラ数2'' すなわち,注目してい
る文節の次の文節の平均モーラ数を利用した場合に,
ほとんどの話者で最高の結果が得ら
れた.したがって,ポーズ長の分布関数の組合せ $C_5$ が,ポーズ情報を
利用する上で効果があることが分かった.このことから,以後の
実験では,組合せ $C_5$ を利用する.

\subsection{韻律的特徴量の組合わせ(クローズド実験)}
ポーズ長以外の有効な特徴量を探索するため,
次のように一つずつ特徴量を追加して係り受け解析実験を行った.
ただし,実験条件は Exp(i) である.
また,25 番目の韻律的特徴量として,「``パワー傾前''
と``パワー傾後''の併用」を追加した.
これは,これらの特徴量の併用による相乗効果を
期待したためである.以下で,$T$ は特徴量の全体からなる
集合,$S$ はその中で実際に係り受け解析に使用する特徴量の
集合を表す.
\vspace{5mm}
\begin{itemize}
\item[(1)] $S:= \{\mbox{ポーズ長}\}$ とし,$S$ を用いて係り受け解析を行う.
\item[(2)] 今までに用いてない (すなわち,$T-S$ に含まれる) 特徴量の
一つを $f$ とし,$S \cup \{f\}$ を用いて係り受け解析を行う.
\item[(3)] $f$ を $T-S$ の中で動かしたときの $S\cup \{f\}$ による
文正解率の最大値が $S$ に
よる文正解率より高くなければ終了する.
高ければ,そのときの最大値を与える $f$ を $f_{0}$ とし,
$S:= S \cup \{f_{0}\}$ とする.もし,$S= T$(すなわち,特徴量を
使い切った)ならば終了する.
$S \not = T$ ならば (2) に戻る.
\end{itemize}
\vspace{5mm}
特徴量の組合せを表\ref{cond:3} に,実験結果を表\ref{res:5} に
示す.距離情報を用いることにより平均文正解率が
5.0ポイント向上し,それがポーズ長を用いることによって,
さらに 5.1ポイント
向上している.残りの10種類の特徴量の追加による平均文正解率の向上は
3.3ポイントである.この結果,決定論的解析法の結果をベースラインとしたときの
文正解率の向上率は28.3\%となる.これはこれまでの向上率22.0\% 
\cite{KOU-1,OZE-4}より
6.3ポイント高い.

\begin{table}
\begin{center}
\caption{韻律的特徴量の組合せ}
\label{cond:3}
\begin{tabular}{cl}
\hline
\hline
特徴量の組合せ & 使用する特徴量\\
\hline
$C_a$ & ``ポーズ長''のみ\\
$C_b$ & $C_a$ と``平均モーラ数2''\\
$C_c$ & $C_b$ と``パワー差1''\\
$C_d$ & $C_c$ と``ピッチ傾前''\\
$C_e$ & $C_d$ と``パワー傾前後の併用''\\
$C_f$ & $C_e$ と``アクセントコマンド数''\\
$C_g$ & $C_f$ と``母音長後''\\
$C_h$ & $C_g$ と``フレーズコマンド差2''\\
$C_i$ & $C_h$ と``パワー差2''\\
$C_j$ & $C_i$ と``ピッチ差2''\\
$C_k$ & $C_j$ と``ポーズ長前''\\
\hline
\hline
\end{tabular}
\end{center}
\end{table}

\begin{table}
\begin{center}
\caption{韻律的特徴量の組合わせに対する文正解率(\%)}
\label{res:5}
\begin{tabular}{|c|c|c|c|c|c|}
\hline
組合せ	&MHT	&MTK	&FKN	&FYM	&平均\\
\hline
\hline
$C_a$ &     59.4  &57.3  &57.1  &55.7	&57.4\\
$C_b$ &     60.4  &58.4  &58.1  &57.1	&58.5\\
$C_c$ &     60.4  &59.0  &58.1  &57.3	&58.7\\
$C_d$ &     61.0  &59.2  &57.9  &57.3	&58.9\\
$C_e$ &     61.6  &58.4  &58.3  &58.3	&59.2\\
$C_f$ &     62.6  &59.8  &58.3  &58.3	&59.8\\
$C_g$ &     63.0  &59.2  &59.2  &58.1	&59.9\\
$C_h$ &     63.0  &60.0  &59.6  &59.2	&60.5\\
$C_i$ &     63.4  &60.8  &58.6  &59.4	&60.5\\
$C_j$ &     63.2  &60.8  &58.6  &59.6	&60.6\\
$C_k$ &     63.8  &60.4  &59.2  &59.4	&60.7\\
\hline
距離情報&	&	&	&	&52.3\\
\hline
決定論的&	&	&	&	&47.3\\
\hline
\end{tabular}
\end{center}
\end{table}
\newpage
\subsection{韻律的特徴量の組合わせ(オープン実験)}

クローズド実験と同様に特徴量を順次追加することにより,有効な特徴量の
探索を行った.ただし,実験条件は Exp(ii), Exp(iii) であり,
結果は,それらの文正解率の平均で示す.


特徴量の組合せを表\ref{cond:4} に,Exp(ii),Exp(iii) の実験結
果の平均を表\ref{res:8} に示す.このように,
オープン実験においては3種類の特徴量が有効であり,
距離情報で 5.0ポイント,ポーズ長でさらに 4.7ポイント,
2つの特徴量の
追加によりさらに 0.8ポイントの文正解率の向上が認められる.
決定論的解析法の結果をベースラインとしたときの
文正解率の向上率は21.2\%である.これは,これまでの向上率17.2\% 
\cite{KOU-1}より4.0ポイント高い.

\begin{table}
\begin{center}
\caption{韻律的特徴量の組合せ}
\label{cond:4}
\begin{tabular}{cl}
\hline
\hline
特徴量の組合せ & 使用する韻律的特徴量\\
\hline
$C_A$ & ポーズ長のみ\\
$C_B$ & $C_A$ と``パワー差1''\\
$C_C$ & $C_B$ と``母音長後''\\
\hline
\hline
\end{tabular}
\end{center}
\end{table}
\begin{table}
\begin{center}
\caption{韻律的特徴量の組合わせに対する文正解率(\%) (Exp(ii)と
Exp(iii) の平均)}
\label{res:8}
\begin{tabular}{|c|c|c|c|c|c|}
\hline
組合せ   	&MHT	&MTK	&FKN	&FYM& 平均\\
\hline
\hline
$C_A$     	&60.4  &59.8  &59.7  &56.8&	59.2\\
$C_B$     	&62.3  &60.0  &59.8  &57.4&	59.9\\
$C_C$     	&62.3  &61.3  &59.1  &57.4&	60.0\\
\hline
距離情報	&	&	&	&&54.5\\
\hline
決定論的	&	&	&	&&49.5\\
\hline
\end{tabular}
\end{center}
\end{table}
しかしながら,クローズド実験に比べ,文正解率の向上に寄与する特徴量の
数が少ない.また,正解率の向上は,ほとんどポーズ長によるものであり,
それと併用したときの,ピッチ,パワー,話速などに関連する特徴量の
寄与はあまり明らかでなかった.
\subsection{文長と文正解率}

クローズド実験,およびオープン実験において,それぞれ文正解率が
一番高かった特徴量の組合せに対する
文長(文節数)と文正解率の関係を
表\ref{res:9},表\ref{res:10},表\ref{res:11} に示す.
文長11以上については,オープン実験における評価文の数が少なく,
信頼性が低いと思われるので省略した.
クローズド実験,オープン実験共に,当然ながら
文長が長くなるほど文正解率が下がっているが,ほとんどの場合,韻
律情報を使った方が,決定論的解析法よりも良い結果が得られている.

韻律情報を用いた場合の話者平均正解率は,Exp(ii)(表\ref{res:10})においては,
文が長くなったとき比較的緩やかに低下するが,Exp(iii)(表\ref{res:11})に
おいては,
それより急に低下する.この傾向は決定論的解析法を用いたときにも
見られるので,Exp(ii)の評価文セットよりも
Exp(iii)の評価文セットの方が,長い文に対する解析の困難度が
高いと考えられる.しかし,韻律情報を用いたときの全文話者平均正解率は
Exp(ii)(58.7\%)よりもExp(iii)(61.4\%)の方が高い.この原因は,
Exp(ii)の評価文セットには,話者平均正解率が全文話者平均正解率を
下回る長さ7以上の文が 
Exp(iii)の評価文セットより多く存在するためと考えられる.ところが
決定論的解析法の全文平均正解率は,Exp(ii)(50.0\%)の方が 
Exp(iii)(49.0\%)
より高いので,韻律情報を用いたときのExp(ii)とExp(iii)の
全文話者平均正解率の違いは,文長の分布の違いだけに帰せられる
ものではなく,韻律情報の効果の違いが関係していると考えられる.
また,話者別に見ると,Exp(ii)の方がExp(iii)より
全文平均正解率が高い話者もいれば,逆の話者もいる.話者FYMは,Exp(ii)において
もExp(iii)においても,全文平均正解率が全文話者平均正解率より低い.
FYMが読み上げた音声は,他の話者より発話速度が速く,ポーズ数が少なく,
平均ポーズ長も短いことが知られており\cite{OZE-4},
このことがFYMに対する全文平均正解率の低さに関係があると思われる.
以上のように,解析結果は,評価文が本来持っている解析の
困難さ,韻律情報の効果,学習データと評価データの組合せ,
読み上げ方など,多くの要因によって影響を受けると推察される.
しかし,上に述べた観察結果が単なる統計的ばらつきによるものではない
ことを確認するためには,より多くのデータが必要と思われる.

\begin{table}
\begin{center}
\caption{文長に対する文正解率(\%) (クローズド実験)(Exp(i))}
\label{res:9}
\begin{tabular}{|c|c|c|c|c|c|c|c|c|c||c|}
\hline
文長		&2&3&4&5&6&7&8&9&10&全文平均 \\
\hline
\hline
文数		&4&17&38&80&110&86&58&58&23& 503\\
\hline
MHT		&100&94.1&89.5&75.0&73.6&58.1&50.0&48.3&56.5&63.8\\
MTK		&100&94.1&78.9&75.0&73.6&50.0&43.1&46.6&39.1&60.4\\
FKN		&100&76.5&84.2&75.0&66.4&53.5&44.8&46.6&34.8&59.2\\
FYM		&100&94.1&84.2&67.5&68.2&46.5&50.0&46.6&56.5&59.4\\
話者平均            &100&89.7&84.2&73.1&70.5&52.0&47.0&47.0&46.7&60.7\\
\hline
決定論的	&100&82.4&73.7&63.8&57.3&37.2&31.0&32.8&30.4&47.3\\
\hline
\end{tabular}
\end{center}
\end{table}

\begin{table}
\begin{center}
\caption{文長に対する文正解率(\%) (オープン実験)(Exp(ii))}
\label{res:10}
\begin{tabular}{|c|c|c|c|c|c|c|c|c|c||c|}
\hline
文長		&2&3&4&5&6&7&8&9&10&全文平均 \\
\hline
\hline
文数		&2&3&10&28&30&26&19&20&4&150\\
\hline
MHT		&100&100  & 90.0&75.0&80.0&53.8&47.4&55.0&50.0&63.3\\
MTK		&100&100  & 80.0&82.1&73.3&42.3&47.4&50.0&50.0&60.0\\
FKN		&100& 66.7&100  &71.4&70.0&46.2&36.8&55.0& 0.0&56.7\\
FYM		&100&100  & 90.0&57.1&66.7&42.3&47.4&50.0&25.0&54.7\\
話者平均            &100& 91.7& 90.0&71.4&72.5&46.2&44.8&52.5&31.3&58.7\\
\hline
決定論的	&100&100&80.0&57.1&63.3&38.5&36.8&35.0&50.0&50.0\\
\hline
\end{tabular}
\end{center}
\end{table}
\vspace{-5mm}
\begin{table}
\begin{center}
\caption{文長に対する文正解率(\%) (オープン実験)(Exp(iii))}
\label{res:11}
\begin{tabular}{|c|c|c|c|c|c|c|c|c|c||c|}
\hline
文長		&2&3&4&5&6&7&8&9&10&全文平均 \\
\hline
\hline
文数		&2&10&19&32&37&24&13&10&2&153 \\
\hline
MHT		&100&90.0&84.2&78.1&62.2&37.5&38.5&40.0&50.0&61.4\\
MTK		&100&90.0&84.2&87.5&59.5&37.5&38.5&40.0&0.0&62.7\\
FKN		&100&80.0&89.5&84.4&59.5&41.7&38.5&30.0&0.0&61.4\\
FYM		&100&100 &78.9&78.1&56.8&33.3&38.5&40.0&50.0&60.1\\
話者平均            &100&90.0&84.2&82.0&59.5&37.5&38.5&37.5&25.0&61.4\\
\hline
決定論的	&100&80.0&73.7&68.8&43.2&29.2&23.1&30.0&0.0&49.0\\
\hline
\end{tabular}
\end{center}
\end{table}

\subsection{係り受け距離と係り受け正解率}

6.5節と同じ特徴量の組合せに対する,
係り受け距離と係り受け正解率の関係を
表\ref{res:12},表\ref{res:13},表\ref{res:14} に示す.
ただし係り受け正解率とは,評価文中の文末を除く全文節の中で,
解析結果による係り先がデータベースのラベルで示される係り先と一致する
ものの割合である.
ほとんどの場合において,韻律情報を用いた方が決定論的解析法
より高い係り受け正解率が得られている.

Exp(ii)(表\ref{res:13})とExp(iii)(表\ref{res:14})を比較すると
係り受け距離6のところで正解率に
かなりの差が見られる.文が長くなったときのExp(ii)とExp(iii)の
文正解率の差は,このことが関係しているかも知れない.
その他の点については,Exp(ii)とExp(iii)で顕著な傾向の違いは見られない.
FYMは,Exp(ii)においてもExp(iii)においても,ほとんどの係り受け距離に対
して係り受け正解率が平均より低い.
全文節話者平均で見るとExp(ii)の方がExp(iii)より係り受け正解率が高いが,
全文話者正解率はExp(iii)の方が高い.傾向として
は,係り受け正解率が高くなるほど文正解率も高くなるはずであるが,
この結果が示すように完全な単調性はない.

\begin{table}
\begin{center}
\caption{係り受け距離に対する係り受け正解率(\%) (クローズド実験)(Exp(i))}
\label{res:12}
\footnotesize
\begin{tabular}{|c|c|c|c|c|c|c|c|c|c|c||c|}
\hline
係り受け距離		&1&2&3&4&5&6&7&8&9&10&合計/全文節平均\\
\hline
\hline
文節数		&1909&500&253&126&73&35&13&9&3&1&2922\\
\hline
MHT		&95.0&83.8&90.9&82.5&71.2&62.9&92.3&88.9&0.0&0.0&91.0\\
MTK		&94.7&85.0&86.6&76.2&68.5&65.7&76.9&66.7&0.0&0.0&90.2\\
FKN		&95.1&85.4&86.2&73.0&63.0&60.0&69.2&88.9&0.0&0.0&90.2\\
FYM		&95.8&80.4&86.6&74.6&65.8&48.6&84.6&55.6&0.0&0.0&89.8\\
話者平均            &95.2&83.7&87.6&76.6&67.1&59.3&80.8&75.0&0.0&0.0&90.3\\
\hline
決定論的	&94.0&79.4&76.3&57.1&39.7&17.1&15.4&0.0&0.0&0.0&85.3\\
\hline
\end{tabular}
\normalsize
\end{center}
\end{table}

\begin{table}
\begin{center}
\caption{係り受け距離に対する係り受け正解率(\%) (オープン実験)(Exp(ii))}
\label{res:13}
\footnotesize
\begin{tabular}{|c|c|c|c|c|c|c|c|c||c|}
\hline
係り受け距離		&1&2&3&4&5&6&7&8&合計/全文節平均\\
\hline
\hline
文節数		&566&161&74&30&19&10&3&3&866\\
\hline
MHT		&94.7&83.9&91.9&80.0&57.9&40.0&0.0&0.0&89.8\\
MTK		&94.7&85.1&86.5&76.7&47.4&30.0&0.0&0.0&89.1\\
FKN		&95.4&80.1&89.2&73.3&52.6&30.0&0.0&0.0&88.9\\
FYM		&96.5&78.9&87.8&66.7&42.1&30.0&33.3&0.0&88.9\\
話者平均            &95.3&82.0&88.9&74.2&50.0&32.5&8.3&0.0&89.2\\
\hline
決定論的	&93.6&83.2&74.3&60.0&42.1&20.0&0.0&0.0&86.3\\
\hline
\end{tabular}
\normalsize
\end{center}
\end{table}

\begin{table}
\begin{center}
\caption{係り受け距離に対する係り受け正解率(\%) (オープン実験)(Exp(iii))}
\label{res:14}
\footnotesize
\begin{tabular}{|c|c|c|c|c|c|c|c|c|c|c||c|}
\hline
係り受け距離		&1&2&3&4&5&6&7&8&9&10&合計/全文節平均\\
\hline
\hline
文節数		&511&129&61&35&17&9&1&2&1&1&767\\
\hline
MHT		&95.7&79.8&88.5&80.0&47.1&22.2&0.0&0.0&0.0&0.0&89.2\\
MTK		&95.7&86.0&88.5&68.6&47.1&11.1&0.0&0.0&0.0&0.0&89.6\\
FKN		&96.1&83.7&86.9&62.9&29.4&11.1&0.0&0.0&0.0&0.0&88.7\\
FYM		&97.1&78.3&82.0&57.1&58.8&11.1&0.0&0.0&0.0&0.0&88.4\\
話者平均            &96.2&82.0&86.5&67.2&45.6&13.9&0.0&0.0&0.0&0.0&89.0\\
\hline
決定論的	&94.3&77.5&73.8&57.1&23.5&11.1&0.0&0.0&0.0&0.0&85.0\\
\hline
\end{tabular}
\normalsize
\end{center}
\vspace{3mm}
\end{table}

\section{あとがき}
24種類の韻律的特徴量を取り上げ,係り受け解析に有効な特徴量を求める
ための広範な探索を行った.また,特徴量の現実の分布をより良く近似するために,
分布関数の改良を試みた.
その結果,決定論的解析法をベースラインにしたとき,韻律的特徴量を
用いることによる文正解率の向上率は,従来の向上率に比べて,
クローズド実験において 6.3ポイント,オープン実験において 4.0ポ
イント高い値が得られた.

ポーズ長はクローズド実験においてもオープン実験においても非常に
有効であったが,これと併用したときの,ピッチ,パワー,話速などに
関連する特徴量の有効性は,オープン実験においてはあまり明らかでは
なかった.しかし,これは特徴量の抽出法やその利用法に問題があるためかも
知れないので,このことから直ちにピッチ,パワー,話速などに
関連する特徴量が構文情報を含まないと結論付けることはできない.
ピッチやパワーに関連する特徴量がそれぞれ単独ではある程度有効である
ことが知られている\cite{EGU,OZE-3}ので,
ポーズ長と併用したときの
有効性があまり認められなかったのは,
これらの特徴量がポーズ長と強い相関を持つためかも知れない.
また,各文節は固有のピッチやパワーのパターンを持っている.
したがって,本研究で使用した
韻律的特徴量には,係り受け距離に関する情報だけでなく,
文節が変ることによるピッチやパワーの
変動も同時に含まれている.もし,それらを分離できればさらに
有効な特徴量が抽出できる可能性もある.また,ポーズ長の
有効性は確認されているものの,その特異な分布を良く近似するには
到っておらず,改良の余地が残されている.

学習データと評価データの組合せを変えた2つのオープン実験の結果の
観察から,
文正解率は,評価文が本来持っている解析の
困難さ,韻律情報の効果,学習データと評価データの組合せ,
読み上げ方など,多くの要因に依存することが推察された.
しかし,今回用いた503文のデータではデータ量が少ないため,
この観察結果が
単なる統計的なばらつきによるものか,必然性のあることなのかを
明確に区別することは困難であった.
今後より多くのデータを用いてこれらの点を確認したい.

本研究で用いた係り受け規則の文被覆率は 73.0\% である.したがって,
この規則を用いる限り,文正解率はこの数字を越えることはできない.
本研究では,韻律的特徴量を用いた場合の文正解率の向上量だけを問題に
したので,文被覆率にはあまり注意を払わなかった.しかし,文被覆率の
低さが有効な特徴量を見い出す妨げになっている可能性も否定できない.
したがって,今後は,係り受け規則の改良も含めて検討する必要がある.
\vspace{5mm}
\acknowledgment
藤崎モデルに基づく韻律解析プログラムを提供して下さった,
東京大学新領域創成科学研究科 広瀬啓吉教授に深く
感謝致します.また,本研究は高坂和之氏(現在日本電気(株)勤務)の
電気通信大学大学院在学中の研究に負うところが多いことを記して,
感謝の意を表します.



\bibliographystyle{jnlpbbl}
\bibliography{288}

\newpage
\begin{biography}
\biotitle{略歴}
\bioauthor{廣瀬 幸由}{
1998 年電気通信大学電気通信学部情報工学科卒業.
2000 年同大学院修士課程修了.
在学中,音声言語の研究に従事.現在,ソニー株式会社勤務.言語処理学会会
員.}

\bioauthor{尾関 和彦}{
1965 年東京大学工学部電気工学科卒業.同年,日本放送協会入社.
1968 年より1年間エジンバラ大学客員研究員.
音声言語処理の研究に従事.
電子通信学会第41回論文賞受賞.
現在,電気通信大学電気通信学部情報通信工学科教授.
工学博士.
言語処理学会,日本音響学会,電子情報通信学会,情報処理学会,ISCA,
IEEE 各会員.
}

\bioauthor{高木 一幸}{
1987 年筑波大学第3学群情報学類卒業.
1989 年筑波大学理工学研究科修士課程修了.同年,日本IBM入社.
1995 年筑波大学工学研究科博士課程修了.
音声言語処理の研究に従事.
現在,電気通信大学電気通信学部情報通信工学科助手.
博士(工学).
日本音響学会,電子情報通信学会,情報処理学会,人工知能学会 各会員
}

\bioreceived{受付}
\biorevised{再受付}
\bioaccepted{採録}

\end{biography}

\end{document}
