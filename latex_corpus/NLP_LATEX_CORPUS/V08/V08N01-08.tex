\documentstyle[jnlpbbl,epsf]{jnlp_j}

\setcounter{page}{175}
\setcounter{巻数}{8}
\setcounter{号数}{1}
\setcounter{年}{2001}
\setcounter{月}{1}
\受付{2000}{7}{12}
\再受付{2000}{9}{20}
\採録{2000}{10}{10}
\setcounter{secnumdepth}{2}

\newtheorem{example}{}
\renewcommand{\theexample}{}

\title{対話者の社会的役割を利用した訳し分け手法}
\etitle{Transfer Method Using\\ Dialogue Participants' Social Role}

\author{
       山田 節夫\affiref{ATR}\affiref{NTT} \and
       隅田 英一郎\affiref{ATR} \and 
       柏岡 秀紀\affiref{ATR}
       }
\eauthor{
         Setsuo Yamada \affiref{ATR}\affiref{NTT} \and     
         Eiichiro Sumita \affiref{ATR} \and
         Hideki Kashioka \affiref{ATR}
         }

\headtitle{対話者の社会的役割を利用した訳し分け手法}
\headauthor{山田 節夫・隅田 英一郎・柏岡 秀紀}

\affilabel{ATR}
    {ATR音声言語通信研究所}
    {ATR Spoken Language Translation Research Laboratories}

\affilabel{NTT}
    {現在, NTTコミュニケーション科学基礎研究所}
    {Current affiliation: NTT Communication Science Laboratories}

\jkeywords{音声翻訳, 話し言葉, 対話者の情報, 訳語選択}
\ekeywords{Speech translation, spoken language, information on
  dialogue participants, word selection}



\jabstract{
音声翻訳を介した対話をより自然なものにするためには, 原言語を解析するだ
けでは取得困難な『言語外情報』を利用することが有効である. 例えば, 『対
話者の社会的役割』を使用した翻訳は対話をより自然にする. 本論文では, 特
にこの『対話者の社会的役割』に着目し, この役割情報を利用して, 適切な丁
寧度の翻訳にする手法を提案する. 既存の変換ルールや辞書にこの役割情報に
応じた修正を加えることによって訳を変える. 実際に英日翻訳における変換ルー
ルや辞書に『対話者の社会的役割』に応じたルールやエントリーを登録し, そ
の際に参照していない未訓練の23会話(344発声)を使って実験をした.
その結果, 丁寧表現にすべき発声に対して, 再現率が65\%, 適合率が86\%
であった. したがって, 本手法は, 音声翻訳を使って自然な対話を行うために
は効果的であり実現性も高い. さらに, 対話者の性別情報など他の言語外情報
や英日以外の言語対に対する本手法の適用可能性についても考察する.
}


\eabstract{
This paper proposes to improve translation quality by using
information on dialogue participants that is easily obtained from
outside the translation component. If we wish to make a conversation
smooth with the dialogue translation system, it is important to use
not only linguistic information, which comes from the source language,
but also extra-linguistic information, which is shared between the
participants of the conversation. We incorporated the participants'
social role into transfer rules and dictionary entries. We conducted
an experiment with 23 unseen dialogues (344 utterances) using an
English-to-Japanese translation. The experiment demonstrated that
recall and precision for expressions which should be polite, are 65\%
and 86\%, respectively. Thus, our simple and easy-to-implement method
is effective, which is a key technology enabling smooth conversation
with a dialogue translation. Additionally, this paper discusses the
useful information such as the participants' gender, and how our
method could apply information on dialogue participants to other
language pairs.
}

\begin{document}
\maketitle

\section{はじめに}

最近様々な音声翻訳が提案されている
\cite{Bub:1997,Kurematsu:1996,Rayner:1997b,Rose:1998,Sumita:1999,Yang:1997,Vidal:1997}. 
これらの音声翻訳を使って対話を自然に進めるためには, 原言語を解析して得
られる言語情報の他に言語外情報も使う必要がある. 例えば, 対話者
\footnote{本論文では, 2者間で会話をすることを対話と呼び, その対話に参
加する者を対話者と呼ぶ. すなわち, 対話者は話し手と聞き手の両者のこと
を指す. }に関する情報(社会的役割や性別等)は, 原言語を解析するだけ
では取得困難な情報であるが, これらの情報を使うことによって, より自然な対
話が可能となる. 

言語外情報を利用する翻訳手法は幾つか提案されている. 例えば, 文献
\cite{Horiguchi:1997}では, 「spoken language pragmatic information」を
使った翻訳手法を, また, 文献\cite{Mima:1997a}では, 「situational
information」を使った手法を提案している. 両文献とも言語外情報を利用し
た手法であり, 文献\cite{Mima:1997a}では机上評価もしているが, 実際の翻
訳システムには適用していない. 言語外情報である「pragmatic adaptation」
を実際に人と機械とのインターフェースへの利用に試みている文献
\cite{LuperFoy:1998}もあるが, これも音声翻訳には適用していない. これら
提案の全ての言語外情報を実際の音声翻訳上で利用するには課題が多くあり, 
解決するのは時間がかかると考えられる. 

そこで, 本論文では, 以下の理由により, 上記言語外情報の中でも特に話し手
の役割(以降, 本論文では社会的役割のことを役割と記述する)に着目し, 実
際の音声翻訳に容易に適用可能な手法について述べる. 
\begin{itemize}
\item 音声翻訳において, 話し手の役割にふさわしい表現で喋ったほうが対話
は違和感なく進む. 例えば, 受付業務で音声翻訳を利用した場合, 「受付」
\footnote{本論文では, 対話者の役割である「受付」をサービス提供者, すな
わち, 銀行の窓口, 旅行会社の受付, ホテルのフロント等のことを意味し, 
「客」はサービス享受者を意味している. }が『丁寧』に喋ったほうが「客」
には自然に聞こえる. 
\item 音声翻訳では, そのインターフェース(例えば, マイク)によって, 対
話者が「受付」か否かの情報が容易に誤りなく入手できる. 
\end{itemize}

本論文では, 変換ルールと対訳辞書に, 話し手の役割に応じたルールや辞書エ
ントリーを追加することによって, 翻訳結果を制御する手法を提案する. 

英日翻訳において, 旅行会話の未訓練(ルール作成時に参照していない)23会
話(344発声\footnote{一度に喋った単位を発声と呼び, 一文で完結すること
もあり, 複数の文となることもある. })を対象に実験し, 『丁寧』表現にす
べきかどうかという観点で評価した. その結果, 丁寧表現にすべき発声に対し
て, 再現率が65\%, 適合率が86\%となった. さらに, 再現率と適合率を下げた
原因のうち簡単な問題を解決すれば, 再現率が86\%, 適合率が96\%になること
を机上で確認した. したがって, 本手法は, 音声翻訳を使って自然な対話を行
うためには効果的であり実現性が高いと言える. 

以下, 2章で『話し手の役割』と『丁寧さ』についての調査, 3章で本手法の詳
細について説明し, 4章で『話し手の役割』が「受付」の場合に関する実験と
その結果について述べ, 本手法が音声翻訳において有効であることを示す. 5
章で, 音声翻訳における言語外情報の利用について, また, 他の言語対への適
用について考察し, 最後に6章でまとめる. 

なお, 本論文は, 文献\cite{Yamada:2000}をもとにさらに調査検討し, まとめ
たものである. 

\section{『話し手の役割』と『丁寧さ』}
\label{sec:role and politeness}

音声翻訳を使った窓口などでの対話をより自然にするために, 訳文の『丁寧さ』
が重要である. 我々は, 『話し手の役割』と『丁寧さ』について着目し, その
関係について調査した. 『丁寧さ』を特に考慮せずに一般的な訳を出す機械翻
訳が出力した結果に対して, 『話し手の役割』が「受付」である場合に, 訳を
より丁寧な表現にすることが望まれるかという観点で調査を行った. 実際の調
査に用いた音声翻訳は, 変換主導型翻訳\cite{Sumita:1999} ({\bf
T}ransfer-{\bf D}riven {\bf M}achine {\bf T}ranslation, 略してTDMT)の
英日版で, 対象は旅行会話の中で『話し手の役割』が「受付」である1409発声
とした. その結果, 約70\%(952)の発声は丁寧な表現を使ったほうが良いと
判断された. したがって, 音声翻訳を使って, より自然な対話を実現するため
に『話し手の役割』に応じて『丁寧さ』を変えるのは, 有用であると言える. 

丁寧な表現を使ったほうが良いと判断された発声には, 様々な種類の表現が含
まれている. これを英語表現の種類別(動詞, 名詞等の品詞, 頻出表現等)に分
類した(表\ref{tab:num of exp}). この英語表現は1発声中に1種類とは限ら
ず, 複数の種類を含むことが多い. 例えば, 例\ref{ex:if}に示すように, 
「受付」が``\underline{Mr}. Suzuki, \underline{if} you'll
\underline{wait} \underline{a second}, I'll \underline{call} right
now''と喋った時の翻訳は, 「標準」訳の``鈴木さん, 少し待ってくれたらす
ぐ電話します''より, 丁寧表現を使って``鈴木様, 少々お待ちいただけました
らすぐ電話致します''とするほうが良い. この発声の中には, 敬称(Mr.), 接
続詞(if), 動詞(wait, call), 名詞(second)が含まれている.

\vspace{3mm}
\begin{center}
\fbox{\begin{minipage}{120mm}
\begin{example} \
  \label{ex:if}
\end{example}
  \tabcolsep=2mm
  \begin{tabular}{ll}
      種類: & {\bf 敬称, 接続詞, 動詞, 名詞} \\
      英語: & \underline{Mr}. Suzuki, \underline{if} you'll \underline{wait}
      \underline{a second}, I'll \underline{call} right now \\
      標準: & 鈴木\underline{さん}, \underline{少し} \underline{待って}
      \underline{くれたら}すぐ\underline{電話します} \\
      丁寧: & 鈴木\underline{様}, \underline{少々} \underline{お待ち}
      \underline{いただけましたら}すぐ\underline{電話致します} \\
  \end{tabular}
  \tabcolsep=0mm
\end{minipage}}
\end{center}
\vspace{3mm}

表\ref{tab:num of exp}を見ると, 少なくとも今回調査した旅行会話では, そ
の種類によって丁寧にすべき方法が違うことが分かる. しかし, {\bf いずれ
も従来の翻訳システムの枠組み, すなわち, 変換ルールや対訳辞書に条件を加
えることで丁寧表現に変えるとができる.} つまり, 『話し手の役割』に応じ
たルールや辞書エントリーを既存の変換ルールや対訳辞書に追加すれば, 適切
に丁寧表現が訳出可能となる.

\tabcolsep=2mm
\begin{table}[htbp]
  \begin{center}
  \caption{英日翻訳における丁寧表現の種類}
  \label{tab:num of exp}
  \begin{tabular}{|c|r||c|c|c|} \hline
    \multicolumn{2}{|c||}{} & \multicolumn{3}{|c|}{例} \\ \hline
    種類 & 件数 & 英語 & 標準 & 丁寧 \\ \hline \hline

    動詞 & 692 & accept & 受け付ける & お受けする \\ \hline
    名詞 & 331 & child & 子供 & お子様 \\ \hline
    頻出表現 & 261 & good bye & さようなら & 失礼致します \\ \hline
    敬称 & 94 & Mr. & さん & 様 \\ \hline
    形容詞 & 78 & fine & 良い & 結構 \\ \hline
    代名詞 & 59 & you & あなた & お客様 \\ \hline
    助動詞 & 56 & can & できる & 頂ける \\ \hline
    接続詞 & 27 & if & たら & ましたら \\ \hline
    副詞 & 6 & alone & 一名 & おひとり \\ \hline
    前置詞 & 4 & with & 一緒 & ご一緒 \\ \hline \hline

    合計 & 1608 & \multicolumn{3}{|c|}{} \\ \hline
  \end{tabular}
\end {center}
\end{table}

上記の1409発声は「受付」の『丁寧さ』に着目して調査したが, 発声内容によっ
ては, 話し手が「客」の訳と「標準」の訳が違う場合もあり得る. 例えば, 例
\ref{ex:very good}の場合, ``Very good, please let me confirm them''は, 
話し手が「受付」では, ``\underline{承知しました} 確認させて\underline
{いただきます}'', 話し手が「客」では, ``\underline{それで結構です} 確
認させて\underline{下さい}''とするのが良い.

\begin{center}
\fbox{\begin{minipage}{100mm}
\begin{example} \
  \label{ex:very good}
\end{example}
  \tabcolsep=2mm
  \begin{tabular}{ll}
    英語: & \underline{Very good}, please \underline{let} me confirm them \\ 
    標準: & {\it \underline{分かりました}, 確認させて\underline{下さい}} \\ 
    受付: & {\it \underline{承知致しました}, 確認させて
      \underline{いただきます}} \\ 
    客: & {\it \underline{それで結構です}, 確認させて\underline{下さい}} \\ 
  \end{tabular}
  \tabcolsep=0mm
\end{minipage}}
\end{center}

\section{話し手の役割情報を翻訳知識に組み込む方法}

本章では, 話し手の役割情報をどのように既存の変換ルールや対訳辞書に導入
するかについて述べる. 本手法は変換ルールや対訳辞書を利用している一般の
機械翻訳に適用可能であるが, その有効性を確かめるために, 本論文では翻訳
システムとしてATRで開発した話し言葉翻訳システムTDMTを利用した. 最初に
TDMTについて簡単に説明し, 次に, 話し手の役割情報を組み込んだルールや辞
書エントリーについて述べる.

\subsection{TDMT(Transfer-Driven Machine Translation)}

TDMTは図\ref{fig:transfer rule}に示す変換ルールを使って, 左から右への
ボトムアップ型チャート解析法で構文解析を行っている
\cite{Furuse:1999}. 様々な変換ルールによって, 入力文の構文構造が複数の
候補となった時は, 意味距離計算によって絞られる. 意味距離はシソーラスに
よって定義されたものを用いている.

変換ルールは, 図\ref{fig:transfer rule}のように, 原言語のパターン, 目
的言語のパターン, 原言語の用例からなる. 原言語のパターンは変項と構成素
境界からなる. ここでの変項は, XやYなどの大文字の記号で表し構成素に対応
する. 構成素境界は, 機能語あるいは左の構成素の品詞と右の構成素の品詞を
付けた記号(品詞バイグラムマーカと呼ぶ)である. 品詞バイグラムマーカは, 
構文解析をする前に対象の構成素間に挿入される. 例えば, ``accept''が{\bf
V}erbで``payment''が{\bf C}ommon {\bf N}ounなので, 例\ref{ex:accept}で
は, 品詞バイグラムマーカ$\langle V$--$CN\rangle$が, ``accept''と
``payment''の間に挿入される. 目的言語のパターンは, 原言語のパターンに
対応した変項と訳語から構成される. ここでの変項は, xやyなどの小文字の記
号で表す. 例えば, xは原言語のパターンでの変項Xに対応している. 原言語の
用例は, パターン作成時に参照した文において実際に出現した語である.

\begin{center}
  \fbox{
    \begin{minipage}{115mm}
      \begin{example} \
        \label{ex:accept}
      \end{example}
   
    \tabcolsep=1mm
    \begin{tabular}{ll}
      英語: & \underline{We} \underline{accept} \underline{payment}
      by credit card \\
      標準: & {\it 私\underline{達}はクレジットカードでの支払いを
        \underline{受け付け}ます} \\
      丁寧: & {\it 私\underline{供}はクレジットカードでの
        \underline{お}支払いを\underline{お受けし}ます} \\
    \end{tabular}
    \tabcolsep=0mm
  \end{minipage}}
\end{center}

図\ref{fig:ex rule}では, 原言語のパターンは(X $\langle V$--$CN
\rangle$ Y), 目的言語のパターンは(y ``を'' x)や(y ``に~'' x), 目的言語
の用例は, ((``accept'') (``payment'')) や((``take'') (``picture''))で
ある. ((``accept'') (``payment''))は, 例\ref{ex:accept}の「標準」訳に
由来する用例であるが, 他の用例は他の訓練文から来ている. 図\ref{fig:ex
rule}の変換ルールは, 原言語が(X $\langle V$--$CN \rangle$ Y)にマッチし
たら, XとYに対応した入力が「意味的に最も近いか, または, 全く同じ」用例
の組が付いている目的言語のパターンを選択せよという意味である. 例えば, 
入力``accept $\langle V$--$CN \rangle$ payment''に「最も近いか, または, 
全く同じ」用例の組は((``accept'') (``payment''))なので, 目的言語のパター
ンは(y ``を'' x)が選択される. このようにして, 適切な目的言語のパターン
が変換ルールによって選ばれる. 

\begin{figure}[htb]
  \begin{center}
    \begin{tabular}{cc}

\begin{minipage}{60mm}
\begin{center}
\fbox{\begin{minipage}{45mm}
\begin{tabbing}

(\=(\=( \kill

(原言語のパターン) \\
$\Rightarrow$ \\
((目的言語のパターン 1) \\
\>((原言語の用例 1) \\
\>\>(原言語の用例 2) \\
\>\> ... ) \\
\>(目的言語のパターン 2) \\
\> ... )

\end{tabbing}
\end{minipage}}    
  \caption{変換ルールのフォーマット}
  \label{fig:transfer rule}
\end{center}
\end{minipage} &

\begin{minipage}{60mm}
\begin{center}
\fbox{\begin{minipage}{45mm}
\begin{tabbing}

(\=(\=( \kill

(X $\langle V$--$CN \rangle$ Y) \\
$\Rightarrow$ \\
((y ``を'' x) \\
\>(((``accept'') (``payment'')) \\
\>\>((``take'') (``picture''))) \\
\>(y ``に'' x) \\
\>(((``take'') (``bus'')) \\
\>\>((``get'') (``sunstroke''))) \\
)

\end{tabbing}
    \end{minipage}}
  \caption{変換ルールの例(英日版)}
  \label{fig:ex rule}
\end{center}
\end{minipage}

\end{tabular}
\end{center}
\end{figure}

目的言語のパターンが選ばれた後, 図\ref{fig:transfer dic}に示したフォー
マットの対訳辞書を用いて, そのパターンに応じた訳を決定する. 例えば, 
``accept $\langle V$--$CN \rangle$ payment''の場合は, ``支払いを受け付
ける~''という訳になるが, ``支払い''と``受け付ける''は図\ref{fig:ex dic}
の対訳辞書の辞書引きの結果から, また, ``~を~''は目的言語のパターン(y ``
を'' x)から来ている. 



\begin{figure}[htb]
  \begin{center}
    \begin{tabular}{cc}

\begin{minipage}{60mm}
\begin{center}
\fbox{\begin{minipage}{60mm}
\begin{tabbing}

(\=(\=( \kill

((原言語の単語) $\rightarrow$ (目的言語の単語) \\
\> ... )

\end{tabbing}
\end{minipage}}
  \caption{対訳辞書のフォーマット}
  \label{fig:transfer dic}
\end{center}
\end{minipage} &

\begin{minipage}{60mm}
\begin{center}
\fbox{\begin{minipage}{60mm}
\begin{tabbing}

(\=(\=( \kill

((``accept'') $\rightarrow$ (``受け付ける'') \\
\>(``payment'') $\rightarrow$ (``支払い''))

\end{tabbing}
\end{minipage}}
  \caption{対訳辞書の例(英日版)}
  \label{fig:ex dic}
\end{center}
\end{minipage}

\end{tabular}
\end{center}
\end{figure}


\begin{figure}[htb]
  \begin{center}
    \fbox{\begin{minipage}{45mm}
      \begin{tabbing}

(\=(\=(\=( \kill

(原言語のパターン) \\
$\Rightarrow$ \\
(({\bf(目的言語のパターン $1_1$) :パターンの条件 $1_1$} \\
\>\>{\bf (目的言語のパターン $1_2$) :パターンの条件 $1_2$} \\
\>\> ... \\
\>\>{\bf (目的言語のパターン $1_n$) :デフォルト}) \\
\>((原言語の用例 1) \\
\>\> ... ) \\
\>(({\bf(原言語の用例 1) $\rightarrow$ (目的言語の単語 $1_1$) :単語の条件 $1_1$}\\
\>\>\>{\bf(原言語の用例 1) $\rightarrow$ (目的言語の単語 $1_2$) :単語の条件 $1_2$}\\
\>\>\> ... \\
\>\>\>{\bf(原言語の用例 1) $\rightarrow$ (目的言語の単語 $1_m$) :デフォルト})\\
\>\> ... ) \\
(({\bf (目的言語のパターン $2_1$) :パターンの条件 $2_1$} \\
\> ... )))

\end{tabbing}
\end{minipage}}    
  \caption{対話者の情報に関する条件付き変換ルールのフォーマット}
  \label{fig:rule with clerk}
\end{center}

\vspace*{2mm}

  \begin{center}
    \fbox{\begin{minipage}{45mm}
      \begin{tabbing}

(\=(\=(\=( \kill

(({\bf (原言語の単語 1) $\rightarrow$ (目的言語の単語 $1_1$) :条件 $1_1$} \\
\>\>{\bf (原言語の単語 1) $\rightarrow$ (目的言語の単語 $1_2$) :条件 $1_2$} \\
\>\> ... \\
\>\>{\bf (原言語の単語 1) $\rightarrow$ (目的言語の単語 $1_k$) :デフォルト}) \\
\> ... )

\end{tabbing}
\end{minipage}}    
  \caption{対話者の情報に関する条件付き対訳辞書のフォーマット}
  \label{fig:dic with clerk}
\end{center}
\end{figure}


\subsection{対話者の情報に関する条件付きルール及び辞書エントリ}
\label{subsec:rule and dic with role}

対話者の情報に関する条件を変換ルールや対訳辞書の中で記述できるように, 
既存の変換ルールや対訳辞書のフォーマットを図\ref{fig:rule with clerk}
と図\ref{fig:dic with clerk}のように変えた. 図\ref{fig:rule with
clerk}では, 目的言語のパターンはそのパターンの条件で, また, 目的言語の
単語はその単語の条件で対話者の情報を参照できる. つまり, ``~目的言語のパ
ターン $1_1$''は, ``パターンの条件 $1_1$''に記述してある条件で選ばれ, 
``原言語の用例1''の対訳は, ``単語の条件 $1_1$''に記述してある条件で``
目的言語の単語$1_1$''が選ばれる. 目的言語パターンに条件が付く場合の例
を図\ref{fig:ex1 rule with role}に示し, 原言語の用例に条件が付く場合の
例を図\ref{fig:ex2 rule with role}に示す. 

図\ref{fig:ex1 rule with role}に示すように, ``パターンの条件 $1_1$''を
``:s-role clerk''と定義することができる. これは, 話し手の役割(s-role)
が「受付」(clerk)という条件である. したがって, もし話し手が「受付」
と分かれば, ``if''は``いただけましたら''と訳される. これは, 例えば, 
\ref{sec:role and politeness}章の例\ref{ex:if}に示すように, 入力が``if
you'll wait a second''の場合, 「標準」では, ``少し待って\underline {く
れたら}''となるが, 話し手が「受付」なら``少し待って\underline{いただけ
ましたら}''と訳出される. 

一方, 図\ref{fig:ex2 rule with role}に示すように, ``:単語の条件
$1_1$''を``:s-role clerk''と定義することもできる. この場合も, もし話し
手が「受付」と分かれば, ``accept''は``お受けする''と訳される. ここで注
意する点は, 話し手が「受付」であっても, 常に``accept''の訳は``お受けす
る''ではないことである. つまり, 図\ref{fig:rule with clerk}の``目的言
語の単語$1_1$''は, そこで定義されている変換ルール内のみで有効である. 
もしも話し手の役割に応じた訳が常に同じであれば, 図\ref{fig:dic with
clerk}に示すフォーマットの対訳辞書に登録する. また, 必要ならば, 同じ変
換ルール内で, 同時に, 目的言語パターンと原言語の用例に条件を付けること
もできる. 

図\ref{fig:rule with clerk}及び図\ref{fig:dic with clerk}の``:デフォ
ルト''は, どのような条件にも合わなかった時にマッチする条件である. 例え
ば最初に``:パターンの条件$1_1$'', 次に``:パターンの条件$1_2$''という
ように, 条件は上から順に調べられ, 全て満足されない場合にデフォルトのも
のが選択される. 

\begin{figure}[htb]

  \begin{center}
    \fbox{\begin{minipage}{45mm}
\begin{tabbing}
(\=(\=(\=(\=( \kill

(``if'' ``you'' X) \\
$\Rightarrow$ \\
(({\bf(x ``いただけましたら'') :s-role clerk} \\
\>\>{\bf(x ``くれたら'')}) \\
\>(((``wait'')) \\
\>\>((``find''))))

\end{tabbing}
    \end{minipage}}
  \caption{話し手の役割情報に関する条件付き変換ルールの例1(英日版)}
  \label{fig:ex1 rule with role}
  \end{center}

\vspace{5mm}

  \begin{center}
    \fbox{\begin{minipage}{45mm}
\begin{tabbing}
(\=(\=(\=(\=( \kill

(X $\langle V$--$CN \rangle$ Y) \\
$\Rightarrow$ \\
((y ``を'' x) \\
\>(((``accept'') (``payment'')) \\
\>\>((``take'') (``picture''))) \\
\>((\bf{(``accept'') $\rightarrow$ (``お受けする'') :s-role clerk} \\
\>\>\>\bf{(``accept'') $\rightarrow$ (``受け付ける'')})))

\end{tabbing}
    \end{minipage}}
  \caption{話し手の役割情報に関する条件付き変換ルールの例2(英日版)}
  \label{fig:ex2 rule with role}
  \end{center}

\vspace{5mm}

  \begin{center}
    \fbox{\begin{minipage}{60mm}
\begin{tabbing}

(\=(\=(\=( \kill

(({\bf (``payment'') $\rightarrow$ (``お支払'') :s-role clerk} \\
\>\>{\bf (``payment'') $\rightarrow$ (``支払い'')}) \\
\>({\bf (``we'') $\rightarrow$ (``私供'') :s-role clerk} \\
\>\>{\bf (``we'') $\rightarrow$ (``私達'')}))

\end{tabbing}
    \end{minipage}}
  \caption{話し手の役割情報に関する条件付き対訳辞書の例(英日版)}
  \label{fig:ex dic with clerk}
  \end{center}

\end{figure}

図\ref{fig:ex2 rule with role}のルール, 図\ref{fig:ex dic with clerk}
の辞書エントリーを用いると, 例\ref{ex:accept}の発声は, 話し手が「受付」
の場合, 最終的には``私供はクレジットカードでのお支払いをお受けします''
と訳出される. 

どの語を丁寧表現にすべきかは機械的には判断できないが, ほとんどの丁寧表
現にすべき語は, 上記の方法, すなわち, 目的言語パターン, 原言語の用例に
対する訳, 対訳辞書によって, 話者の役割に応じた訳し分けができる. 基本的
には, 次の基準で解決方法を決めている.
\begin{itemize}
  \item 訳し分けすべき語が機能語ならば, 目的言語パターン (例:図
\ref{fig:ex1 rule with role})
  \item 内容語で, 役割以外の条件も利用して訳が決まる時は, 原言語の用例
に対する訳 (例:図\ref{fig:ex2 rule with role})
  \item 内容語で, 常に役割の条件のみで訳が決まる時は, 対訳辞書 (例:図
\ref{fig:ex dic with clerk})
\end{itemize}

例えば, \ref{sec:role and politeness}章の表\ref{tab:num of exp}に挙げ
た例については, 表\ref{tab:solution of exp}に示した方法で丁寧表現にす
ることができる. だたし, 表\ref{tab:solution of exp}に示したのは例であっ
て, その種類の全てが表\ref{tab:solution of exp}に示した解決方法で訳し
分けができるとは限らない. 例えば, ``child''は辞書に登録することで適切
に訳し分け可能だが, 表\ref{tab:num of exp}で示した名詞は全て辞書によっ
て訳し分けができるわけではない. 

\tabcolsep=2mm
\begin{table}[htbp]
  \begin{center}
  \caption{英日翻訳における丁寧表現の訳出方法の例}
  \label{tab:solution of exp}
  \begin{tabular}{|c|c|c||c|} \hline
   \multicolumn{3}{|c||}{例} & \\ \hline
    英語 & 標準 & 丁寧 & 解決方法 \\ \hline \hline

    accept & 受け付ける & お受けする & 用例 \\ \hline
    child & 子供 & お子様 & 辞書 \\ \hline
    good bye & さようなら & 失礼致します & 辞書 \\ \hline
    Mr. & さん & 様 & パターン \\ \hline
    fine & 良い & 結構 & 用例 \\ \hline
    you & あなた & お客様 & 辞書 \\ \hline
    can & できる & 頂ける & パターン \\ \hline
    if & たら & ましたら & パターン \\ \hline
    alone & 一名 & おひとり & 用例 \\ \hline
    with & 一緒 & ご一緒 & パターン \\ \hline
  \end{tabular}
\end {center}
\end{table}

\section{実験}

今回の実験で用いたTDMT(英日版)は, 話し手の役割を考慮する前では, 約
1,500の変換ルール, 約8,000の対訳辞書エントリーが登録されていた. 「受付」
用に『丁寧さ』を向上させるために, \ref{sec:role and politeness}章の表
\ref{tab:num of exp}に基づいて, 約300ルール, 約40の辞書エントリーを追
加, 修正した. 内容は, 例えば, \ref{subsec:rule and dic with role}節の
表\ref{tab:solution of exp}に示した通りである. 表\ref{tab:num of exp}
の合計数よりもルール数が少ないのは, 1つのルールの中に, 複数の目的言語
パターンや原言語の用例に対する訳が含まれているからである. 

「受付」用に修正した変換ルールと対訳辞書を用いて未訓練である23対話(344
発声)を対象に実験を行った. 音声翻訳への入力は本来音声であるが, 今回の
実験では, 翻訳のみの評価を行うため書き起こしテキストを形態素解析したも
のを入力とし, 同時に対話者の役割情報\footnote{本データはATRのデータベー
ス\cite{Takezawa:1999}の一部である. これは, 「受付」と「客」との間の模
擬対話を実施し, その音声データを書き起こしてデータベース化したもので, 
収録時の設定にある対話者の情報を実験に用いた.}も与えた.

実験の評価をするために, 実験に使った23対話(このうち, 話し手が「受付」
は199発声, 「客」は145発声)に対して, 話し手の役割情報を適用する前の
TDMTの翻訳結果を2つに分類した. 一方は丁寧にすべき表現を含む場合, もう
一方は丁寧にすべき表現を含まない場合である. 表\ref{tab:correct-data}に
その調査結果を示す. 表現を変える必要がある発声(104)は全て話し手の役割
が「受付」であったが, 表現を変える必要がない発声(166)のうち43発声は
「受付」で, 123発声は「客」であった. 

\begin{table}[htbp]
\tabcolsep=2mm
  \begin{center}
    \caption{表現を変える必要性の調査結果}
    \label{tab:correct-data}
    \begin{tabular}{c|cl}
      表現を変える必要性 & 発声数 & (「受付」,「客」)  \\ \hline
      あり               & 104 & (104, \hspace{4mm}0) \\
      なし               & 166 & (\hspace{2mm}43, 123)  \\ \hline
      評価対象外\footnotemark[5]  & \hspace{2mm}74 &\\ \hline \hline
      合計                        & 344 & 
    \end{tabular} \\
  \end{center}
\end{table}

\footnotetext[5]{翻訳結果のうち74発声に関しては, 『丁寧さ』を判断でき
ないほどの翻訳の質であったので, 本論文では対象外とした. }

同じ対象文に対して, 話し手の役割情報を適用する前と後の翻訳結果を『丁寧
さ』という観点で印象がどう変ったかで評価した. 表\ref{tab:correct-data}
の表現を変える必要性の有無別にまとめたものを表\ref{tab:eval data}に示
す. 

\tabcolsep=1mm
\begin{table}[htb]
  \begin{center}
    \caption{話し手の役割情報を使った翻訳の評価結果}
    \label{tab:eval data}
    \begin{tabular}{c||c|c|c|c}
      表現を変える必要性  & 差分 & \multicolumn{2}{c|}{印象} & 計 \\ \hline
               &            & 良い & 68 & \\
          あり &       あり & 同じ & \hspace{2mm}5  & \hspace{2mm}76  \\ 
          104  &            & 悪い & \hspace{2mm}3  &     \\ \cline{2-5}
               &       なし & \multicolumn{2}{c|}{} &  \hspace{2mm}28  \\ \hline \hline
               &            & 良い & \hspace{2mm}0  &     \\
          なし &       あり & 同じ & \hspace{2mm}3  & \hspace{4mm}3  \\ 
          166  &            & 悪い & \hspace{2mm}0  &     \\ \cline{2-5}
               &       なし & \multicolumn{2}{c|}{} & 163  \\ \hline
     \multicolumn{5}{l}{} \\
     \multicolumn{5}{l}{良い:翻訳結果の印象が良くなった} \\
     \multicolumn{5}{l}{同じ:翻訳結果の印象が同じだった} \\
     \multicolumn{5}{l}{悪い:翻訳結果の印象が悪くなった} \\
    \end{tabular}
  \end{center}

\end{table}
\tabcolsep=0mm

以下に示す再現率と適合率で本手法を評価した. 


$$ \mbox{再現率} =  \frac{\mbox{\small 翻訳結果が良くなった発声数}}
                         {\mbox{\small 丁寧表現にするべき発声数}} $$


$$ \mbox{適合率} = \frac{\mbox{\small 翻訳結果が良くなった発声数}}
                        {\mbox{\small 翻訳結果に差分のあった発声数}} $$

\vspace{5mm}

その結果, 再現率は65\% ($ = 68 \div 104 $) , 適合率は86\%($ = 68 \div
(76+3) $)であった. 表現を変える必要があった発声(104)で, 話し手の役割
情報を適用する前と後とで翻訳結果が完全に一致しなかった(翻訳結果に差分
があった)発声は, 76発声であった. そのうち印象が良くなった68発声が望ま
れた効果である. 表現を変える必要がなかった発声(166)で翻訳結果に差分
がなかったものは, 163発声であった. これらは副作用がなかったという意味
で望ましいものである. 他の結果は, 再現率, 適合率を下げている. 

印象が悪くなった3件の原因は, 実験に使った翻訳の日本語生成処理における
実装の問題に起因するもので, 本論文では無視できる. また, 翻訳結果に差分
があった中で印象が同等のもの, 表現を変える必要があるのに翻訳結果に差分
がなかったものの原因は, 大きく2種類に分けることができた. 1つは, 実験に
用いた翻訳システムでは, その動作主が誰(何)なのかが分からないことに原因
がある. 話し手の役割が「受付」であっても, 丁寧にする対象の語の動作主が
「受付」の場合は丁寧な表現を使わないことがあるからである. もう1つは, 
「受付」用のルールや辞書エントリーがまだ不十分なことに原因がある.

「受付」用のルールや辞書エントリー不足については比較的容易に解決できる
問題であるのでこれを解決し, 日本語生成処理の問題を無視したとすると, 再
現率と適合率を向上させることができる. 調査したところ, 再現率が86\%, 適
合率が96\%まで向上できることが分かった.

\section{考察}

本章では, 音声翻訳における言語外情報の有用性と, 他の言語対に対する本手
法の適用可能性について述べる. 

\subsection{言語外情報}

現時点では, 一般の言語外情報を音声翻訳で活用することは困難である. しか
し, 以下に示すような幾つかの言語外情報は容易に得ることができ, 音声翻訳
で十分活用できる. 

\hspace{-4mm}{\bf 対話者の役割}

音声翻訳のインターフェース(例えばマイク)によって, 誤りなく対話者の役
割を同定することができる. 対話者の役割が「受付」と「客」の場合, 英日翻
訳システムで有用であることは本論文で述べた. 

また, 文献\cite{Yamamoto:1999}では, 日本語対話文の格要素省略補完をする
決定木の属性に対話者の役割が活用されている. 

\vspace{5mm}

\hspace{-4mm}{\bf 対話者の性別}

音声認識において性別を高精度で判別できることが報告されている
\cite{Takezawa:1998,Naito:1998}. 英日翻訳の場合, 例えば, 女性が話す場
合, 文末に「わ」を付けると女性らしい表現となるように, 話者の性別は, 男
性または女性固有の表現に訳出するのに有効である. 

\vspace{5mm}

このように簡単に得ることができる言語外情報を音声翻訳に適用することは, 
人が違和感なく対話を進めるためには重要である. 上記に挙げた言語外情報は
現時点でも容易に得ることができるが, 将来は, さらに多様な情報を得ること
が期待でき, 音声翻訳にも有用であると考えられる. 例えば, もし1人1台, 常
に携帯端末を持ち歩くことになれば, その端末と通信することによって, 年齢
などの情報を得るのは可能である. そうなると, 聞き手が子供であれば, 複雑
な表現も簡単に翻訳するといった高度な展開も期待される. 


\subsection{他の言語対への利用}

これまで, 主に目的言語が日本語の場合, 話し手の役割情報が丁寧表現を訳し
分けるのに有用であることは述べた. 他の言語対について, 言語外情報とそれ
を活用できる処理(言語現象)について述べる. 

\hspace{-4mm}{\bf 日英}

現在の日英翻訳が正しく翻訳できる旅行会話の発声を対象に, 話し手の役割に
よって英語表現がどう変化するかを調査した. 表\ref{tab:ex of e-ex}にその
例を示す. 

\tabcolsep=2mm
\begin{table}[htbp]
  \begin{center}
    \caption{英語表現が変化する例}
    \label{tab:ex of e-ex}
    \begin{tabular}{|ll|} \hline

      種類: & {\bf 表現方法の違い} \\
      日本語: & 遅くなっても\underline{構いません} \\
      標準: & {\it \underline{I don't mind} if you are late} \\
      受付: & {\it \underline{It's okay} if you are late} \\[1mm] \hline

      種類: & {\bf 主語の違い, 敬称の追加} \\
      日本語: & はい, できます \\
      標準: & {\it Yes, \underline{I} can} \\
      受付: & {\it Yes \underline{sir-madam},\underline{we} can}\\ \hline

      種類: & {\bf 主語の省略} \\
      日本語: & 一人五十ドルぐらいです \\
      標準: & {\it \underline{It's} about fifty dollars per person} \\
      客: & {\it About fifty dollars per person}\\ \hline

      種類: & {\bf 代名詞の違い} \\
      日本語: & 前日までには必ず連絡を下さい \\
      標準: & {\it Please make sure to call
        \underline{me} by the day before} \\
      受付: & {\it Please make sure to call
        \underline{us} by the day before} \\ \hline
    \end{tabular}\\
  \end{center}
\end{table}
\tabcolsep=0mm

この表\ref{tab:ex of e-ex}を見ると, 日英の場合も話し手の役割によって適
した表現が存在することが分かる. これらも, 我々の提案手法により訳し分け
が可能である. また, ``敬称の追加''の場合, 表\ref{tab:ex of e-ex}の「受
付」では, ``sir-madam''が追加されているが, 聞き手の性別が分かれば,
``sir'' か``madam'' かに訳出したほうが良い. これは, 例えば, 図
\ref{fig:ex rule with role and gender}のように, ルールの条件部に
``:s-role clerk : h-gender male''(話し手の役割(s-role)が「受付」
(clerk)で聞き手の性別(h-gender)が「男性」(male))と指定をすれば, 本手法
で``sir''と訳出することが可能である. さらに, 性別情報だけの利用として
は, 例えば, 図\ref{fig:ex rule with gender}のような変換ルールを作成す
れば, ``様''に対して, もしも聞き手が「男性」であれば``Mr.'', 「女性」
であれば``Ms.''  と英訳可能になる.

\begin{figure}[htb]
  \begin{center}
    \fbox{\begin{minipage}{45mm}
\begin{tabbing}
(\=(\=(\=(\=( \kill

(``はい'' X) \\
$\Rightarrow$ \\
(({\bf (``Yes sir,'' x) :s-role clerk :h-gender male} \\
\>\>{\bf (``Yes madam,'' x) :s-role clerk :h-gender female} \\
\>\>{\bf (``Yes sir-madam,'' x) :s-role clerk} \\
\>\>{\bf (``Yes,'' x)}) \\
\>(((``できる'')) \\
\>\>((``そう''))))

\end{tabbing}
    \end{minipage}}
  \caption{話し手の役割と聞き手の性別情報に関する条件付き変換ルールの例(日英版)}
  \label{fig:ex rule with role and gender}
  \end{center}
\end{figure}


\begin{figure}[htb]
  \begin{center}
    \fbox{\begin{minipage}{45mm}
\begin{tabbing}
(\=(\=(\=(\=( \kill

(X ''様'') \\
$\Rightarrow$ \\
(({\bf (``Mr.'' x) :h-gender male} \\
\>\>{\bf (``Ms.'' x) :h-gender female} \\
\>\>{\bf (``Mr-ms.'' x)}) \\
\>(((``鈴木'')) \\
\>\>((``佐藤''))))

\end{tabbing}
    \end{minipage}}
  \caption{聞き手の性別情報に関する条件付き変換ルールの例(日英版)}
  \label{fig:ex rule with gender}
  \end{center}
\end{figure}


\hspace{-4mm}{\bf 日中}

現在の日中翻訳が正しく翻訳できる旅行会話の発声を対象に, 話し手の役割に
よって中国語表現がどう変化するかを調査した. その結果, 以下のような場合
に中国語表現に違いが出ることが分かった. 

\begin{enumerate}
  \item 日本語の表現が曖昧な場合
  \item 話し手によって丁寧度が変る場合
\end{enumerate}

(1)に関しては, 例えば日本語表現が``〜をお願いします''の場合, 一般には
``〜したい''という意味の中国語に翻訳されるが, もしも話し手が「受付」で
あれば, ``〜して下さい''という意味の中国語に翻訳される. (2)に関しては, 
話し手の役割が不明な時は, 翻訳システムではより多く使われる表現を訳出せ
ざるを得ないが, 実際は, 話し手が「受付」ならより丁寧な表現に, また, 
「客」なら客らしい表現の中国語に翻訳される. 例えば, ``お待たせしました''
の場合, 機械翻訳システムは丁寧表現を使って訳出するが, 話し手が「客」の
場合には少し丁寧度を下げた表現のほうが自然な中国語となる. これらの訳し
分けも, 我々の提案方法により実現できる. 

\vspace{5mm}

以上のように, 対話者の情報は, 英日に関する丁寧表現の訳し分けだけでなく, 
他の言語対に対しても重要な情報であることが分かった. また, 訳し分けに関
して, 我々の提案方法は, 言語対によらずに容易に適用可能であるので, 一般
的な手法であると言える. 

\section{おわりに}

原言語を解析するだけでは取得困難な対話者の役割や性別情報を使った翻訳手
法について述べた. 本手法では, 対話者の役割や性別情報を利用して訳し分け
ができる. これらの情報は音声翻訳の中では容易に得ることができる. 今回利
用した役割は「受付」と「客」であり, 話し手が「受付」の時はより丁寧な表
現になる.

旅行会話を対象とした英日翻訳に「受付」用のルールと辞書エントリーを追加
した. 翻訳システムにとってオープンである23対話(344発声)を対象に評価し
たところ, 丁寧表現にするべき発声に対して, 再現率が65\%, 適合率が86\%の
結果となった. 再現率と適合率を下げる原因のうち簡単に解決できるもの, す
なわち, 「受付」用のルールや辞書エントリー不足の問題を解決すれば, 再現
率が86\%, 適合率が96\%に向上することを机上調査により確認している. した
がって, 本手法は, 音声翻訳を使って自然な対話を行うためには効果的であり
実現性が高いと言える. また, 対話者の性別情報など他の言語外情報や英日以
外の言語対に対する本手法の適用可能性についても考察し, 本手法が言語対に
よらない, 一般的な手法である見通しを得た.

話し手の役割だけでなく, 発声中の訳し分け対象である語の動作主が分からな
いと『丁寧さ』を決めるのが困難な場合がある. そこで, 動作主を決める方法
を検討する予定である. また, より精度良く翻訳結果を出力させるために, 
「受付」用のルールや辞書エントリーをより一層充実させていく予定である. 
さらに, 他の言語対への適用や, 容易に得られそうな他の言語外情報を音声翻
訳で利用する手法についても検討していく予定である. 

\acknowledgment

本研究の実験や調査に協力いただいたATR音声言語通信研究所第三研究室の
翻訳知識作成者の皆様に感謝致します. 

\bibliographystyle{jnlpbbl} 

\bibliography{jpaper}

\begin{biography}

\biotitle{略歴}
\bioauthor{山田 節夫}{
1990年東京電機大学理工学部情報科学科卒業. 1992年同大学大学
院情報科学専攻修士課程修了. 同年日本電信電話株式会社入社. 
1997年よりATR音声翻訳通信研究所, ATR音声言語通信研究所へ出向. 
2000年日本電信電話株式会社へ復帰. 現在NTTコミュニケーション
科学基礎研究所. 
自然言語処理, 特に機械翻訳の研究に従事. 情報処理学会会員. 
}
\bioauthor{隅田 英一郎}{
1982年電気通信大学大学院計算幾科学専攻修士課程修了. 
1999年京都大学工学博士. 
ATR音声言語通信研究所主任研究員. 
自然言語処理, 機械翻訳, 情報検索, 並列処理の研究に従事. 
情報処理学会, 電子情報通信学会, ACL各会員. 
}
\bioauthor{柏岡 秀紀}{
1993年大阪大学大学院基礎工学研究科博士後期課程修了. 博士(工学). 
同年ATR音声翻訳通信研究所入社. 1998年同研究所主任研究員. 
1999年奈良先端科学技術大学院大学情報学研究科客員助教授. 
2000年ATR音声言語通信研究所主任研究員. 自然言語処理, 
機械翻訳の研究に従事. 言語処理学会, 情報処理学会, 
人工知能学会, 認知科学会, 各会員. 
}

\bioreceived{受付}
\biorevised{再受付}
\bioaccepted{採録}

\end{biography}

\end{document}
