



\documentstyle[epsf,jnlpbbl]{jnlp_j_b5}

\setcounter{page}{85}
\setcounter{巻数}{8}
\setcounter{号数}{1}
\setcounter{年}{2001}
\setcounter{月}{1}
\受付{2000}{5}{16}
\再受付{2000}{7}{12}
\採録{2000}{10}{10}

\setcounter{secnumdepth}{2}

\title{情報検索システムの統計的手法による特徴と精度の分析}
\author{佐々木 稔\affiref{Tdoc} \and 北 研二\affiref{Toku}}

\headauthor{佐々木 稔・北 研二}
\headtitle{情報検索システムの統計的手法による特徴と精度の分析}

\affilabel{Tdoc}{徳島大学大学院工学研究科}
{Graduate School of Engineering, University of Tokushima}
\affilabel{Toku}{徳島大学工学部}
{Faculty of Engineering, University of Tokushima}


\jabstract{
本論文では,IREXワークショップにおける情報検索課題(IR)の本試験の結果,
および,参加したすべてのIRシステムについてのアンケートをもとに,
平均適合率,再現率・適合率曲線を直線回帰させた傾きと切片が
システムに用いられた手法とどのような相関関係をもっているのかを調査し,
それぞれの手法がシステムの性能に与える影響の大きさを示した.
その結果,多くの手法について,再現率0.0での適合率の値と適合率の減少量に
トレードオフの関係が存在し,検索システムに用いる手法の選択の難しさが現れた.
また,NARRATIVEタグの使用有無により,同様に相関関係を調査し,
NARRATIVEタグの有効性とシステムの性能に与える影響の大きさを示した.
その結果,NARRATIVEタグを利用する場合,それに適した有効な手法を
選択することが重要であることが分かった.
}

\jkeywords{情報検索,IREXワークショップ,性能評価,適合率,再現率}

\etitle{Analysis of the Characteristics and the \\
	Efficiency of Information Retrieval Systems \\
	by Statistical Method}
\eauthor{Minoru Sasaki\affiref{Tdoc} \and
	Kenji Kita\affiref{Toku}}

\eabstract{
In evaluating the effectiveness of information retrieval (IR) and
extraction system, the most common method is to compare two retrieval
methods and decide if one system measurably achieves better results than the
other.  However, it is difficult for researchers to compare more than
two retrieval methods because there are many participants in IR task
in IREX workshop.
In this paper, we evaluate the characteristics and the effectiveness
of the IR systems using a statistical method based on the results
of the IR formal run and questionnaires of systems.  Comparisons of
systems deal with effects on the performance such as indexing, 
querying and retrieval model.
The results confirm the effectiveness of this evaluation method because 
phrases relates to the performance better than words.
There is a trade-off relation between the precision value at 0.0 and 
decrease rate in many systems and this result indicates the difficulty 
of the choice of techniques in system.
We also evaluate correlations between the efficiency and the characteristics 
of the systems with both a short and long versions of the topics.
A result of this evaluation shows that it is important 
to select effective methods for the long version of topics.
}

\ekeywords{Information Retrieval, IREX Workshop, Evaluation, 
Precision, Recall}

\begin{document}
\maketitle


\section{はじめに}
情報検索の分野は,欧米において過去数十年の間に,英語を中心とした文書を対象に
研究が盛んに進められ,高速な文字列検索アルゴリズムや自動索引づけなどに
多くの成果が得られた.これらの技術が基礎となり,大規模な文書集合に対する
検索技術,新しい評価技術の向上を目的として,
TREC (Text REtrieval Conference)
\footnote{TRECワークショップホームページ: http://trec.nist.gov}
などのコンテストが開催され,
新しい技術の開発やこれまでの技術の改良などが活発に行われている.

日本においても,情報抽出,検索技術に関する研究が盛んに行われ,
数多くの優れた日本語情報検索システムが提案されている.
このようなシステムを評価するための日本語テストコレクションの
整備も進み\cite{kitani},個々の検索システムを容易に評価できるようになった.
さらに,共通のデータベース,プラットフォームにおけるシステム評価の
場として,IREX (Information Retrieval and Extraction Exercise)ワークショップ
\footnote{IREXワークショップホームページ: 
http://cs.nyu.edu/cs/projects/proteus/irex/}
が開催された.
このワークショップには,情報検索(IR)と情報抽出(NE)の各課題に対して
数多くのシステムが参加し,全体的な評価を通して様々な議論が行われた.

IREXの目標のひとつとして,共通の基準における各検索システムの評価を基にした
問題点の共有と,それによるこの分野の飛躍的な進歩,発展がある
\cite{sekine99}.
一般的に,情報検索システムの性能評価をする際には,
提案された手法を利用したシステムと利用していないシステムとの比較を行う.
比較する際,一つのシステムからみると,参加した数多くのシステムにおける
評価結果の違いから,研究の新しい方向性や発展性が発見できる.
しかし,IREXでは,数多くのシステムが参加しているため,
ふたつのシステム間の比較実験では実験回数が莫大となり,
共通点,相違点の整理が複雑になってしまう.
また,他の比較手法として,使用されたシステムとは別の基準システムを作り,
比較を行う手法も提案されている\cite{Hull93}.
しかし,その場合,システム間の相違点が多くなり,
直接的に何が精度向上の原因であるのかをとらえることが難しくなる.
したがって,すべての検索システムを対象として,システムの構成要素を
評価すると同時に,全体的なシステムの検索精度を評価するようなシステム指向の
評価方法が必要となる.
このような全体的な評価は,問題点を発見,解決するための議論を進める上で
重要な課題であると考えられ,
TRECやIREXでは様々な
評価が行われている\cite{Lagergren98}\cite{Voorhees98SIGIR}\cite{Matuo99}.

本論文では,IREXにおけるIR課題の本試験の結果,および参加した各システムに
ついての,参加者が回答したアンケート結果を参考にして,
IR課題におけるシステムの特徴と精度の関連性を
独自の統計的な手法を用いて分析する.
これまでは,手法を利用したシステムと利用していないシステムとの実験結果を
比較することによって,その手法の有効性が評価されていた.
これに対し,我々の提案する評価手法は,数多くのシステムにおける
検索結果を基にして,システムに用いられた手法との関連性を客観的な
相関係数として表し,検索システムに対し有効な手法を明確にしている.
このような検索システムに対し有効な手法を示す評価は,これまで
TREC7においても行われているが,比較に用いられたすべてのシステムで
再現率・適合率曲線の違いがほとんど無い条件の下で行われている\cite{Voorhees98}.
この条件において,比較に用いたシステムが利用した手法が示されているが,
客観的にその手法が有効かどうかの判断は難しい.
その点で,我々の評価手法はどのような再現率・適合率曲線に対しても
客観的に有効な手法を示すことができる.
さらに,我々の評価手法は,検索結果でのランクの上位に,
関連のある文書を数多く検索するための有効な手法を示すことができる.
この分析においては,IRシステムのアンケートの中で
システムの性能に大きく影響する次の3点
\begin{itemize}
\item 索引づけ,索引構造
\item 検索式の生成
\item 検索モデル,ランクづけ
\end{itemize}
に注目して,これらの要素を実現するために用いられた手法が
検索精度とどの程度関連があるのかを調査する.

IREXでは,検索課題
\footnote{検索課題の例としては,以下のようなものがある.\\
$<$TOPIC$>$\\
$<$TOPIC-ID$>$1001$<$/TOPIC-ID$>$\\
$<$DESCRIPTION$>$企業合併$<$/DESCRIPTION$>$\\
$<$NARRATIVE$>$記事には企業合併成立の発表が述べられており、その合併に参加する企業の名前が認定できる事。また、合併企業の分野、目的など具体的内容のいずれかが認定できる事。企業合併は企業併合、企業統合、企業買収も含む。$<$/NARRATIVE$>$\\
$<$/TOPIC$>$}
に,検索要求を簡潔に表現したDESCRIPTIONタグと,
人間が判断可能な程度の詳細な検索要求の記述をした
NARRATIVEタグが用いられている.
通常,WWWサイトなどに存在する検索エンジンに入力される索引語の数は2,3語と
少ないために,DESCRIPTIONタグのみを検索実験に考慮する方が実用的である.
しかし,DESCRIPTIONタグのみを利用した場合には曖昧さが生じてしまい,
人間が可能な限り正確に検索できるという点においては,
詳細に書かれているNARRATIVEタグの方が重要な情報であるといえる.
実際,TRECなどにおいても,このような検索要求の長さに対する精度への影響が
議論され,NARRATIVEタグの使用による精度の違いが分析されている
\cite{Voorhees97}\cite{Voorhees98}\cite{Hull96}.
このようなことから,IREXにおいても,検索式を作成する際のNARRATIVEタグの
使用有無により,検索システムに与える影響が変化するものと考えられる.
このことを明らかにするため,検索システムにおけるNARRATIVEタグの利用有無により
shortとlongに分け,それぞれの平均適合率と相関の高いシステムの特徴を調べる.

再現率・適合率曲線に対し単回帰分析を行い直線として近似した場合,
その切片が大きい時,ランクの上位に適合する文書を検索できる
確率が高いと考えられる.また,傾きが平行に近いほど,システムは再現率の増加
とともに起こる適合率の減少を抑えることができると考えられる.
そこで,検索結果を平均して得られた再現率・適合率曲線に
単回帰分析を行い直線として近似し,
その切片と傾きがさまざまな手法のなかでどの手法に関連性が強いのかを調べる.
また,同様に,shortとlongにおける切片と傾きとの相関が
高いシステムの特徴を調べる.
これらを分析することにより,本試験に参加したすべてのシステムで,
検索質問をshortとlongに分けたそれぞれの場合に対して,
傾き,切片から総合的に,どの手法と関連性が強いかを考察する.


\vspace*{-0.3cm}
\section{再現率と適合率の関係}
IRシステムの評価には,一般的に適合率(Precision)と再現率(Recall)が使用される
\cite{lewis2}\cite{Witten}.
適合率は,システムが検索した文書に対する,検索した正解文書数の割合であり,
検索の能力を表している.
再現率は,全正解文書数に対するシステムが検索した正解文書数の割合であり,
システムがすべての適合する文書のうちどの程度実際に検索可能かという
検索の幅を表している.
再現率と適合率はそれぞれ個別に用いてもシステムの評価を行うことができるが,
ランクづけを行う検索システムでは,一般的に再現率・適合率曲線が用いられている.

\begin{figure}[t]
	\begin{center}
                \atari(110,76)
		
	\end{center}
	\caption{A判定のみの再現率・適合率曲線}
	\label{re_pre}
\end{figure}
IREXワークショップにおけるIR課題は,2年分の新聞記事から検索課題に書かれた
検索要求に関連する文書を検索するもので,予備試験と本試験が行われた.
予備試験は課題数が6課題あり,評価結果は非公開で,本試験は課題数が30課題あり,
評価結果は実際の団体名が分からないように各団体をシステムIDにより表し,
公開している.
検索結果の判定基準にはA,B,Cの3種類あり,A判定は記事の主題が検索課題に
関連している場合,B判定はA判定のように記事の主題には関連性がないが,
記事の一部が関連している場合,C判定は何も関連していない場合という
判断基準になっている.
そのIRの本試験に参加したすべてのシステムにおける,A判定
のみを正解とした検索結果を図\ref{re_pre}に示す.
このグラフにおいて,一つの曲線を示す`1103a'などの文字列は
システムIDを表している.
このグラフから分かるように,再現率と適合率の関係は大きく分けて2種類あると
考えられる.
一つ目は,適合率が再現率の増加に対し,直線的に減少する関係である.
これは,ほとんどのシステムに当てはまる傾向で,特に再現率が0.0での
適合率の値が高いシステムがこのようなグラフになっている.
もう一つはグラフが下に凸の曲線を描くように適合率が減少する関係である.
このような曲線は,検索の結果上位にランクされている文書に適合する文書が
少ないため,再現率が少ない値において適合率の変動が激しくなっていると
考えられる.
システムについてのアンケートから,この曲線になる直接的な原因を
調査したが,特にシステムに共通して用いられている手法は存在しなかった.

また,多くのシステムが再現率0.0における適合率が0.7を超えており,
ランク上位に適合する文書が検索される確率が高くなっている.
再現率0.0における適合率の値はランク上位に適合する文書を検索できるのかを表す
尺度で,高い値をもつシステムほどユーザの探している情報が検索されていると
考えることができる.
ランクの 1 位で検索された文書が適合する文書であると,
その時点で再現率0.0での適合率は1.0となる.
適合しない文書の場合には適合率は1.0にはならず,以降の検索結果の
ランクに従って適合率に大きな変動が生じる.
変動の大きさはあるものの,その後適合する文書が適合しない文書に比べ
数多く検索されると,適合率は高い値となる.

しかし,再現率0.0での適合率の大きさには関係なく,
再現率・適合率曲線のグラフは必ず単調に減少する特徴を持っている.
ランクに関係なく適合する文書が連続して検索される,
または適合する文書の割合が適合しない文書に比べ大きい場合には,
再現率の増加に対する適合率の減少量が少なくなる.
ランクの上位にいくらか適合しない文書を検索していたとしても,途中で適合する文書
を連続して検索し,適合しない文書の数に比べて多くなることで,
適合率の減少が少なくなり,傾きの最大値である0に近くなる.
これより,適合する文書を効率良く検索するためには,グラフの減少量を少なくする
必要があり,これはシステムを評価する上で重要な尺度であると考えられる.

\clearpage
\section{評価実験}
IREXワークショップにおける,IR本試験の結果を公表する方法については,
団体名を実名で公表するのではなく,各団体に割り当てられたID番号で
結果を公表する方法がとられた.
このため,検索結果に対してどのような手法が用いられているのか,
研究的な内容を対応づけることが難しくなっている.
システムの詳細を知る手段として,検索結果と同時に提出した
IRシステムアンケートがある.
このアンケートにより,各システムがどのような手法を利用したかが
理解できるようになっている.

IRシステムアンケートは,各システムについてどのような手法を用いて
本試験の検索結果を出したのかを回答したものであり,主に次の項目がある.
\begin{itemize}
\item 索引づけとそのデータ構造
\item 検索式の作成
\item 検索を実行する環境
\item 検索モデル
\item その他
\end{itemize}
これらの項目の回答を集計して,システムが用いた手法の内,
主要な55個の手法に注目した.
これらの手法を平均適合率などの数値と比較するために,
手法の使用の有無により数値を割り当てる.
本実験では,各システムが用いた手法には1を,用いていない手法
については0を割り当て,数値データに変換した.
このときアンケートの中で「はい」,「いいえ」で答えられない質問項目については,
システムにただひとつしか用いられていない手法でもひとつの手法として
数値を割り当て,システムの違いを明確に定めることにした.
このように変換をしたデータに平均適合率を付け足してひとつの行列にして,
用いられた手法と平均適合率との相関係数を求めた.
相関とは,変数 $x$,$y$ において一方の変化が他方の変化にある傾向を
伴うとき,$x$ と $y$ の間に相関関係があるという.相関には次の3種類がある.
\begin{itemize}
\item[] 正の相関:一方の値が大きくなると,他方の値も大きくなる.
\item[] 負の相関:一方の値が小さくなると,他方の値も小さくなる.
\item[] 無相関:2つの値に明白な関係がみられない.
\end{itemize}
この相関を数値的に表したものに相関係数があり,
2つの変数の相互関係の程度を表したものである.
本実験では,相関の有無を明確にするために,相関係数の絶対値が0.5を超える手法は
相関が認められるとして,システムが用いた手法に対して評価を行った.

\subsection{平均適合率とシステムの関連}
平均適合率とシステムとの相関係数を求めた結果,相関係数の高かった主な
システムの特徴を表\ref{pre_sys}に示す.
相関係数の高い手法は,全体的に正の相関を持っているが,
すべての手法に対し,相関係数の絶対値が0.5を下回っている.
表\ref{pre_sys}において,相関係数の絶対値の高い
LSI(Latent Semantic Indexing)やIDF(Inverse Document Frequency)については,
若干の相関は見られるものの,0.5に満たしていないため,平均適合率との間に
明確な相関を認めることができなかった.

\begin{table}[t]
\renewcommand{\arraystretch}{}
\caption{平均適合率と相関の高い主なシステムの特徴}
\centering
\small
\label{pre_sys}
\begin{tabular}{lcc} 
システムの特徴 		&   平均  & 相関係数 \\ \hline
LSI    			& 0.04545 & -0.49600 \\
IDF 			& 0.86364 &  0.49341 \\
レレバンスフィードバック& 0.22727 &  0.45499 \\
名詞 			& 0.45455 & -0.42376 \\
フレーズ 		& 0.31818 &  0.41665 \\
NEGタグ 		& 0.36364 &  0.40655 \\
ロバートソン法   	& 0.09091 &  0.40427 \\
文書の長さ 		& 0.45455 &  0.38297 \\
名詞以外の品詞 		& 0.40909 & -0.38053 \\
シソーラス 		& 0.13636 &  0.33728 \\
DESCRIPTOINタグ 	& 0.90909 & -0.33728 \\
文字列形態素インデクス 	& 0.09091 &  0.32822 \\
BM25 			& 0.09091 &  0.32822 \\ \hline
\end{tabular}
\end{table}

また,この表\ref{pre_sys}を見ると,DESCRIPTOINタグから索引語を抽出する手法と
NEGタグなどのようにNARRATIVEタグを利用した手法が混在していることが分かる.
実際に,NARRATIVEタグ自体を利用することについては,
平均適合率との相関係数が $-0.03864$ と低く,NARRATIVEタグは
他の2つのタグに比べ,適合率との関連性が少ない結果となった.
これは,NARRATIVEタグの中には比較的長い文章が存在し,検索に重要な
索引語が存在するのと同時に,一般的に広い分野で使われる索引語も
比較的多く存在しているため,このタグを使う際には注意が必要であると考えられる.
このようなことから,IREXワークショップに参加したシステムを
検索課題を簡潔に表現したDESCRIPTOINタグのみを用いたシステムと
比較的長い文章が存在し,NEGタグ
\footnote{NARRATIVEタグ中において,「〜を除く」などの否定的な表現を示すもの}
が存在するNARRATIVEタグを
同時に用いたシステムに分けて評価を行う.
これにより,相関の認められる手法がより顕著に現れ,
より有効な評価をすることができると考えられる.



\clearpage
\subsection{shortおよびlongでの平均適合率とシステムの関連} \label{slheikin}
DESCRIPTIONタグのみを用い,NARRATIVEタグを利用しない場合をshort,
検索課題をすべて使用した場合をlongとして,それぞれを用いたシステムにおける
平均適合率と利用した手法との関連性を比較する.
そのために,short,longそれぞれにおける平均適合率と
システムが用いた手法との相関係数を求め,
NARRATIVEタグを利用しない場合にはどのような手法が有効であるか,
また,NARRATIVEタグを利用した場合についても同様に有効な手法に
どのようなものがあるかを調査する.
その結果,shortを用いたシステム,longを用いたシステムの平均適合率との
相関係数が高かった主なシステムの特徴を,
それぞれ表\ref{sho_sys}と表\ref{lon_sys}に示す.

\begin{table}[t]
\renewcommand{\arraystretch}{}
\caption{shortの平均適合率と相関の高い主なシステムの特徴}
\centering
\small
\label{sho_sys}
\begin{tabular}{lcc} 
システムの特徴 		&   平均  & 相関係数 \\ \hline
IDF    			& 0.88888 &  0.64416 \\
LSI 			& 0.11111 & -0.64416 \\
フレーズ 		& 0.55555 &  0.61796 \\
レレバンスフィードバック& 0.33333 &  0.57326 \\
確率と情報量		& 0.33333 &  0.56095 \\
名詞	 		& 0.66667 & -0.54080 \\
名詞以外の品詞 		& 0.66667 & -0.54080 \\
構文解析 		& 0.22222 & -0.43654 \\
NEGタグ		 	& 0.22222 &  0.42834 \\
シソーラス		& 0.22222 &  0.42834 \\
単語		 	& 0.77778 & -0.42834 \\ \hline
\end{tabular}
\end{table}

\begin{table}[t]
\renewcommand{\arraystretch}{}
\caption{longの平均適合率と相関の高い主なシステムの特徴}
\centering
\small
\label{lon_sys}
\begin{tabular}{lcc} 
システムの特徴 		&   平均  & 相関係数 \\ \hline
固有名詞    		& 0.15385 &  0.74158 \\
ロバートソン法  	& 0.15385 &  0.74158 \\
構文的な手がかり	& 0.15385 &  0.74158 \\
文書の長さ		& 0.46154 &  0.68955 \\
NEGタグ			& 0.46154 &  0.48372 \\
名詞	 		& 0.30769 & -0.41387 \\
面の情報 		& 0.15385 & -0.41384 \\
IDF	 		& 0.84615 &  0.39269 \\
単語		 	& 0.61538 &  0.37213 \\
形態素解析		& 0.15385 & -0.36284 \\ \hline
\end{tabular}
\end{table}

表\ref{sho_sys}と表\ref{lon_sys}から共通して言えることは,
shortとlongに分ける前の表\ref{pre_sys}と比較すると
相関係数が全体的に高くなっているということである.
すなわち,shortとlongに分けることによって,それぞれの検索システム
に対して,平均適合率により関係深いシステムの特徴が顕著になっている.


shortで平均適合率と相関が最も高かったのは,LSIとIDFである.
LSIは負の相関を持ち,検索精度を下げる傾向があるため,
今回の本試験での結果においては,LSIの利用や索引づけに改良が必要であった.
平均適合率と直接関連のある手法に,文書の重みづけとしてはIDF,索引語には
意味を限定しやすいフレーズが,共に正の相関を持ち検索精度を上げる傾向がある.
フレーズを用いるとき,索引語が数多く存在しない場合があるため,
レレバンスフィードバックを用いて検索式を拡張させることで,
性能の良いシステムが構築できると考えられる.
名詞や名詞以外という品詞情報を用いて索引づけを行う方法は,
どちらも負の相関が高くなった.これは,元々索引語の数などのように情報量が
少ない上に,一般的な単語まで取り出していたために,
検索精度が下がる傾向があると考えられる.

longで平均適合率と相関が最も高かったのは,固有名詞である.
これは,NARRATIVEタグにある文章が比較的長いめ,索引づけの手法の中でも
特定分野にしか出現しない語を抽出できる固有名詞の相関が高くなったと考えられる.
また,ロバートソン法や構文的な手がかり,文書の長さといった
文書そのものの違いや特徴を明確にする重みづけの手法も相関が高かった.
shortでは平均適合率に関連性があったIDFは,
longの場合,これらの手法より相関が低かった.
このことより,索引語に対する重みづけ手法は関連性が低いと考えられる.

表\ref{sho_sys}に見られる手法と表\ref{pre_sys}に見られる手法を比較すると,
同じような手法が全体に見受けられた.
すなわち,shortの場合は,一般的な検索システムの平均適合率に
関連性のある手法がそのまま関連性があるということができる.
longの場合は,表\ref{lon_sys}に示した相関と
全体の結果とを比較すると,全体的に正の相関を持つものが多かったが,
表\ref{pre_sys}では見られなかった手法が多く現れた.
 	
これらの結果から, NARRATIVEタグの使用有無によって検索システムの
平均適合率に関連性のある手法が変化することが分かった.
したがって,NARRATIVEタグを用いる際には,
平均適合率に関連性のある,適した手法を選択する必要があると考えられる.

\subsection{回帰式とシステムの関連}
検索結果を平均して得られた,再現率と適合率の関係のデータを
単回帰分析を用いて直線近似を行う.
計算の結果,傾き(回帰係数)や切片(定数項)に対して
相関の高かった手法を,それぞれ表\ref{coe_sys}と表\ref{con_sys}に示す.
直線回帰を行ったときの決定係数は最大で0.99446,
最小で0.69193となり,22個のシステムにおける平均が0.942と高い値となり,
再現率・適合率曲線を直線で近似することが妥当であることを表している.

\begin{table}[htb]
\renewcommand{\arraystretch}{}
\caption{回帰係数と相関の高い主なシステムの特徴}
\centering
\small
\label{coe_sys}
\begin{tabular}{lcc} 
システムの特徴 		&   平均  & 相関係数 \\ \hline
文書の長さ		& 0.45455 & -0.60396 \\
構文解析		& 0.04545 & -0.49385 \\
IDF		 	& 0.86364 & -0.44741 \\
固有名詞(索引)		& 0.22727 & -0.43614 \\
固有名詞(検索式)	& 0.22727 & -0.36891 \\
単語	 		& 0.68182 & -0.30771 \\
面の情報	 	& 0.09091 &  0.30267 \\
LSI	 		& 0.04545 &  0.30140 \\ \hline
\end{tabular}
\end{table}

\begin{table}[htb]
\renewcommand{\arraystretch}{}
\caption{回帰直線の定数項と相関の高い主なシステムの特徴}
\centering
\small
\label{con_sys}
\begin{tabular}{lcc} 
システムの特徴 		&   平均  & 相関係数\\ \hline
IDF		 	& 0.86364 &  0.60051\\
文書の長さ		& 0.45455 &  0.55917\\
LSI	 		& 0.04545 & -0.55286\\
ロバートソン法  	& 0.09091 &  0.44812\\
NEGタグ 		& 0.36364 &  0.42826\\
フレーズ 		& 0.31818 &  0.39632\\
固有名詞(検索式)	& 0.22727 &  0.39059\\
レレバンスフィードバック& 0.22727 &  0.37556\\
面の情報	 	& 0.09091 & -0.36234\\
照合文字列長		& 0.33400 & -0.34285\\ \hline
\end{tabular}
\end{table}

傾きと最も相関係数の高いものは文書の長さで,
相関係数が $-0.60396$ と相関の認められる数値で,傾きを下げる傾向がある.
これは,検索した文書数が増えるにつれて,適合する文書を検索する割合が少なくなる
ことを表している.
特に,傾き,切片共に関連が大きいIDFと文書の長さは重みづけ手法であり,
文書間での違いや特徴を明確にしているため,
検索システムの構築において,重みづけ手法が特に重要であると考えることができる.

切片と最も相関係数の高いものはIDFで,次いで,文書の長さ,LSIとなっている.
全体的に正の相関を持つ手法が多く,切片を上げる傾向がある.
しかし,これらの手法には,傾きにおいて負の相関の高い手法と共通するものが
多く存在している.
これより,これらの手法に対して,切片の大きさと
傾きの大きさにはトレードオフの関係が存在していることが分かる.


\clearpage
\subsection{shortおよびlongでの回帰式とシステムの関連}
NARRATIVEタグの有効性,有効な利用方法について考えるため,
\ref{slheikin}節と同様にshortとlongに分割し,評価を行った.
傾きや切片に対して相関の高かった手法を,
shortの場合をそれぞれ表\ref{coe_s_sys},表\ref{con_s_sys}に,
longの場合をそれぞれ表\ref{coe_l_sys},表\ref{con_l_sys}に示す.

\begin{table}[t]
\renewcommand{\arraystretch}{}
\caption{shortの回帰係数と相関の高い主なシステムの特徴}
\centering
\small
\label{coe_s_sys}
\begin{tabular}{lcc} 
システムの特徴 		&   平均  & 相関係数 \\ \hline
シソーラス		& 0.12500 &  0.75160 \\
LSI      		& 0.12500 &  0.75160 \\
自動検索質問拡張	& 0.12500 &  0.75160 \\
IDF      		& 0.75000 & -0.74788 \\
フレーズ        	& 0.37500 & -0.57108 \\
文字列形態素インデクス  & 0.25000 & -0.51490 \\
確率と情報量	 	& 0.25000 & -0.51490 \\
BM25	 		& 0.25000 & -0.51490 \\ \hline
\end{tabular}
\end{table}

\begin{table}[t]
\renewcommand{\arraystretch}{}
\caption{shortの回帰直線の定数項と相関の高い主なシステムの特徴}
\centering
\small
\label{con_s_sys}
\begin{tabular}{lcc} 
システムの特徴 		&   平均  & 相関係数\\ \hline
IDF		 	& 0.75000 &  0.60051\\
シソーラス		& 0.12500 &  0.55917\\
LSI	 		& 0.12500 & -0.55286\\
自動検索質問拡張        & 0.12500 &  0.44812\\
文字列形態素インデクス 	& 0.25000 &  0.42826\\
確率と情報量		& 0.25000 &  0.39632\\
BM25            	& 0.25000 &  0.39059\\ \hline
\end{tabular}
\end{table}

\subsubsection*{shortにおける関連性}
shortの場合,傾きと最も相関係数の高い手法は,シソーラス,LSI,自動
検索質問拡張で,これら3つは等しい値で,正の高い相関を持つ.
しかし,全体的には負の相関を持つ手法が多く,傾きを下げる傾向がある.
shortでは情報が少ないために,索引づけされた語をあらかじめ準備した
知識集合であるシソーラスで拡張することにより,
傾きを上げる高い正の相関が得られたものと考えられる.
さらに,LSI,フレーズ,自動検索質問拡張,文字列形態素インデクスも
シソーラスと同様に索引づけの手法で,相関係数は比較的高い値になっている.
このことから,検索に有効な索引語の選択やLSIによる意味的な表現形式を
得る手法が,傾きに深く関わっているといえる.

また,切片と最も相関係数の高い手法はIDFで,次いで,シソーラス,
LSIとなっている.
こちらは,全体的に正の相関を持つものが多く,切片を上げる傾向がある.
傾きとの相関の高い手法と共通するものが多く,切片でも索引づけの手法が
深く関連しているといえる.
シソーラスや自動検索質問拡張は傾き,切片ともに正の相関係数を持ち,
shortの場合,シソーラスが切片と傾きに最も関連の深い手法だと考えられる.
しかし,LSI,IDFなどは,傾きでは正の相関,切片では負の相関を持っているために,
ここでも傾きと切片の間にトレードオフの関係が存在することが分かる.

\begin{table}[t]
\renewcommand{\arraystretch}{}
\caption{longの回帰係数と相関の高い主なシステムの特徴}
\centering
\small
\label{coe_l_sys}
\begin{tabular}{lcc} 
システムの特徴 		&   平均  & 相関係数 \\ \hline
ベクトル		& 0.14285 & -0.60621 \\
文書の長さ 		& 0.14285 & -0.60621 \\
索引語の長さ	        & 0.14285 &  0.59558 \\
IDF      		& 0.42857 & -0.58837 \\
フレーズ        	& 0.28571 & -0.47094 \\
インデクス(ベクトル)    & 0.92857 &  0.38171 \\
TF       	 	& 0.07142 & -0.38171 \\
構文的な手がかり	& 0.07142 & -0.38171 \\ \hline
\end{tabular}
\end{table}

\begin{table}[t]
\renewcommand{\arraystretch}{}
\caption{longの回帰直線の定数項と相関の高い主なシステムの特徴}
\centering
\small
\label{con_l_sys}
\begin{tabular}{lcc} 
システムの特徴 		&   平均  & 相関係数\\ \hline
フレーズ		& 0.28571 &  0.75913\\
ベクトル		& 0.14285 &  0.62619\\
文書の長さ 		& 0.14285 &  0.62619\\
索引語の長さ      	& 0.14285 & -0.60014\\
NEGタグ             	& 0.57142 &  0.52856\\
IDF      		& 0.42857 &  0.49117\\
名詞             	& 0.28571 & -0.46559\\ 
n-gram           	& 0.28571 &  0.36211\\ \hline
\end{tabular}
\end{table}

\subsubsection*{longにおける関連性}
longの場合,傾きと最も相関係数の高い手法は,ベクトル,文書の長さで,
次いで索引語の長さとなる.
shortと同様に,全体的に負の相関を持つ手法が多く,
ベクトルなどは傾きを下げる傾向がある.
また,検索要求全体における傾き,切片と同様に,
文書の長さ,索引語の長さ,IDFなどの重みづけ手法が
深く関わっていることが分かる.
さらに,ベクトルは文書または検索要求全体を用いた手法であるため,
longを扱う際の索引語の増加に従って,関連性がより顕著に現れたと考えられる.

切片と最も相関係数の高い手法はフレーズで,次いで,ベクトルと文書の長さが
高い相関を持っている.
short同様,正の相関を持つものが多く,切片を上げる傾向がみられるが,
これらのほとんどは重みづけの手法である.
しかし,これらの手法は,傾きと負の相関の高い手法と多数共通しているため,
longにおいてもトレードオフの関係が存在していることが明らかになった.
また,検索要求の長いlongの場合,正の相関が高いNEGタグは,
傾きとの相関係数が低く,切片との相関係数は高い.
したがって,NEGタグの利用は,
ランクの上位に適合する文書を検索する有効な手段であると考えることができる.


\section{まとめ}
本論文では,IREXワークショップにおけるIRの本試験の結果,
および,参加したすべてのIRシステムについてのアンケートを基に,
平均適合率,再現率・適合率曲線を直線回帰させた傾きと切片が
IRシステムに用いられた手法とどのような相関関係をもっているのかを調査し,
それぞれの手法がシステムの性能に与える影響の大きさを示した.
その結果,名詞や名詞以外の品詞単語を用いる以上にフレーズが性能向上に関係あり,
複数の単語を組み合わせることで意味が限定され,
精度に良い影響を与えることを確認することができた.
また,再現率・適合率曲線の切片と傾きにトレードオフの関係が
多くの手法に見られ,検索システムに用いる手法の選択の難しさが
現れる結果となった.

さらに,NARRATIVEタグの使用有無によりshortとlongに分け,
平均適合率との相関関係,また,再現率・適合率曲線を
直線回帰させた傾きと切片にそれぞれどのような相関関係があるかを調査し,
システムの性能に与える影響の大きさを示した.
その結果,分ける前と比較して,全体的に相関が高くなり,
関連が大きいシステムの特徴が顕著に現れるようになった.
また,NARRATIVEタグを利用する場合,それに適した有効な手法を
選択することが重要であることが分かった.
shortでは,分ける前と比較して,索引語そのものに対する索引づけの手法への
関連性がより顕著に現れた.
その中でも,シソーラスや自動検索質問拡張は検索性能を上げる関連の
深い手法であった.
一方,longでは文書全体に対する重みづけの手法が性能向上に関係があった.
形態素解析やそれにより抽出される名詞単語を用いるよりも,
フレーズや固有名詞などを用いる方が性能向上に関連が深く,
より要求する意味が明確になっている.
このことは形態素解析により索引語を選択する難しさを表現し,
形態素から索引語をより有効的に選択する手法が必要であることを示している.

最後に,情報検索システムの用いる手法の選択の難しさを克服するため,
本論文におけるデータが今後の研究開発で広く利用され,
情報検索システムの分野の進歩と発展につながることを期待したい.

\subsection*{謝辞}
本論文をまとめる機会を与えてくださり,有用なデータを提供して下さった
IREX実行委員の方々,及び,IREX-IRの本試験に参加した方々と
判定者の方々に心から感謝したい.



\bibliographystyle{jnlpbbl}
\bibliography{sankou}

\begin{biography}
\biotitle{略歴}
\bioauthor{佐々木 稔}{
1973年生.1996年徳島大学工学部知能情報工学科卒業.
1998年徳島大学大学院博士前期課程修了.
同年,徳島大学大学院博士後期課程入学,現在に至る.
機械学習,情報検索等の研究に従事.
情報処理学会会員.
}
\bioauthor{北 研二}{
1957年生.1981年早稲田大学理工学部数学科卒業.
1983年から 1992年まで沖電気工業(株)勤務.
この間,1987年から 1992年まで ATR 自動翻訳電話研究所に出向.
1992年 9月から徳島大学工学部勤務.現在,同教授.工学博士.
確率・統計的自然言語処理,情報検索等の研究に従事.
情報処理学会,電子情報通信学会,日本音響学会,日本言語学会,
計量国語学会,ACL 各会員.
}
\bioreceived{受付}
\biorevised{再受付}
\bioaccepted{採録}

\end{biography}

\end{document}

