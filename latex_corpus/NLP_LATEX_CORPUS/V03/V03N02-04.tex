


\documentstyle[jnlpbbl]{jnlp_j_b5}

\newcommand{\susumu}{}

\setcounter{page}{57}
\setcounter{巻数}{3}
\setcounter{号数}{2}
\setcounter{年}{1996}
\setcounter{月}{4}
\受付{1995}{7}{10}
\採録{1995}{9}{8}
\newcommand{\jap}[1]{}
\newcommand{\eng}[1]{}
\newtheorem{obs}{}
\newtheorem{kousatu}{}
\makeatletter
\newcounter{enums}



\def\enumsentence{}

\long\def\@enumsentence[#1]#2{}

\newcounter{tempcnt}

\newcommand{\ex}[1]{}

\def\@item[#1]{}


\newcounter{enumsi}


\def\theenumsi{}


\def\@mklab#1{}
\def\enummklab#1{}
\def\enummakelabel#1{}

\def\eenumsentence{}

\long\def\@eenumsentence[#1]#2{}

\makeatother
\setcounter{secnumdepth}{2}

\title{順接複文における主語の共参照関係の分析}
\author{中川 裕志\affiref{YNU}}

\headauthor{中川裕志}
\headtitle{順接複文における主語の共参照}

\affilabel{YNU}{横浜国立大学工学部 電子情報工学科}
{Department of Electronics and Computer Engineering, Yokohama National University}

\jabstract{日本語においては主語が頻繁に省略されるため,省略された主語す
なわちゼロ主語の指示対象同定が重要である.複文は従属節と主節からなるので,
主節主語と従属節主語がある.したがって,複文の理解に不可欠なゼロ主語の指
示対象同定の問題は,2段階に分けて考えるべきである.第一の段階では,主節
主語と従属節主語が同じ指示対象を持つかどうか,すなわち共参照関係にあるか
どうかの分析である.第二の段階では,第一段階で得られた共参照関係を利用し
て,実際のゼロ主語の指示対象同定を行なう.このうち,第一の共参照関係の有
無は,複文のゼロ主語の扱いにおいて固有の問題であり,本論文ではこの第一の
問題について主として小説に現れるノデ,カラで接続される順接複文について分
析した.分析は,主節および従属節の述語の意味をIPAの動詞形容詞辞書の分類
に従って分類し,各々の述語がどのような分類の場合において共参照するかしな
いかを調べた.この結果,共参照関係の同定に有力であるいくつかのデフォール
ト規則を見い出した.}

\jkeywords{日本語,複文,主語の共参照関係}

\etitle{Analysis of Coreference between Subjects of \\Main and
Subordinate Clauses for Japanese \\{\it NODE} and {\it KARA} Complex Sentences}
\eauthor{Hiroshi Nakagawa \affiref{YNU}} 

\eabstract{
Since complex sentences consist of more than two clauses, we have
more than two subjects in one complex sentence. Therefore it is
important to recognize coreference relations among these subjects for
identifying the referents of these subjects. We focus here on subjects
coreference relations of Japanese complex sentences conjoined by {\it
node} or {\it kara}, both of them mean {\it because} in English. The
reason why this is important is that in Japanese subjects are very
frequently omitted, in other words come to be zero subjects. In this
circumstance, to resolve zero anaphora is one of key factors for
understanding or translating. For this purpose we use semantic features
of predicates of main and subordinate clause respectively, and 
calculate the frequencies for each combination of every semantic feature
of subordinate clause and those of main clause. By analyzing these
frequencies we find several tendencies, for instance, if a subordinate
predicate is linguistic activity like speaking, then the subject of main
clause is distinct from the subject of subordinate clause in its
referent. We also account for these tendencies based on linguistic
and cognitive aspects of human beings.
 }

\ekeywords{Japanese, complex sentences, coreference of subjects}

\begin{document}
\maketitle


\section{はじめに}
日本語の理解において省略された部分の指示対象を同定することは必須である.
特に,日本語においては主語が頻繁に省略されるため,省略された主語の指示対
象同定が重要である.省略された述語の必須格をゼロ代名詞と呼ぶ.主語は多く
の場合,述語の必須格であるから,ここでは省略された主語をゼロ主語と呼ぶこ
とにする.ここでは特に,日本語の複文におけるゼロ主語の指示対象同定の問題
を扱う.日本語の談話における省略現象については久野の分析
\cite{久野:日本文法研究,久野78}
以来,言語学や自然言語処理の分野で様々な提案がなされてい
る.この中でも実際の計算モデルという点では, centering に関連するもの
\cite{Kameyama88,WIC90}が重要である.しかし,これらは主として談話につい
ての分析やモデルである.したがって,複文に固有のゼロ主語の指示対象同定と
いう観点からすればきめの粗い点もある\cite{中川動機95,中川ので95}.したがって,
本論文では主としてノデ,カラで接続される順接複文について,複文のゼロ主語
に固有の問題について扱う.ノデ文については,既に\cite{中川動機95,中川ので95}に
おいて,構文的ないしは語用論的な観点から分析している.そこで,ここでは意
味論的観点からの分析について述べる.

複文は従属節と主節からなるので,主節主語と従属節主語がある.複文の理解に
不可欠なゼロ主語の指示対象同定の問題は,2段階に分けて考えるべきである.
第一の段階では,主節主語と従属節主語が同じ指示対象を持つかどうか,すなわ
ち共参照関係にあるかどうかの分析である.第二の段階では,第一段階で得られ
た共参照関係を利用して,実際のゼロ主語の指示対象同定を行なう.このうち,
第一の共参照関係の有無は,複文のゼロ主語の扱いにおいて固有の問題であり,
本論文ではこの問題について考察していく.

さて,主語という概念は一見極めて構文的なものであるが,久野の視点論\cite
{久野78}で述べられているように実は語用論的に強い制限を受けるものである.
例えば,授受補助動詞ヤル,クレルや,受身文における主語などは視点に関する
制約を受けている.このような制約が複文とりわけノデ文においてどのように影
響するかについては\cite{中川動機95}で詳しく述べている.ここでは,見方を変え
て,意味論的な観点から分析するので,ゼロ主語の問題のうち視点に係わる部
分を排除しなければならない.そこで,能動文においては直接主語を扱うが,受
身文においては対応する能動文の主語を考察対象とする.また,授受補助動詞の
影響については,ここでの意味論的分析と抵触する場合については例外として扱う
ことにする.

なお,ここでの意味論的分析の結果は必ずしも構文的制約のように
例外を許さない固いものではない.文脈などの影響により覆されうるものであり,
その意味ではデフォールト規則である.ただし,その場合でも文の第一の読みの
候補を与える点では実質的に役立つものであろう.

さて,この論文での分析の対象とする文は,主として小説に現れる順接複文(一
部,週刊誌から採取)である.具体的には以下の週刊誌,小説に記載されていた
全ての順接複文を対象とした.

\noindent 週間朝日 1994年 6月17日号,6月24日号,7月1日号

\noindent 三島由紀夫,鹿鳴館,新潮文庫, 1984

\noindent 星新一,ようこそ地球さん,新潮文庫,1992

\noindent 夏目漱石,三四郎,角川文庫,1951

\noindent 吉本ばなな,うたかた,福武文庫,1991

\noindent カフカ/高橋義孝訳,変身,新潮文庫,1952

\noindent 宗田理,殺人コンテクスト,角川文庫,1985

\noindent 宮本輝,優駿(上),新潮文庫,1988

\bigskip

このような対象を選んだ理由は,物理的な世界の記述を行なう文ばかりで
なく,人間の心理などを記述した文をも分析の対象としたいからである.実際週
刊誌よりは小説の方が人間の心理を表現した文が多い傾向がある.ただし,週刊
誌においても人間心理を記述した文もあるし,逆に小説でも物理的世界の因果関
係を記述した文も多い.

次に,分析の方法論について述べる.分析の方法の一方の極は,全て論文著者の
言語的直観に基づいて作例を主体にして考察する方法である.ただし,この場合
非文性の判断や指示対象に関して客観的なデータであるかどうか疑問が残ってし
まう可能性もないではない.もう一方の極は,大規模なコーパスに対して人間の
言語的直観に頼らず統計的処理の方法で統計的性質を抽出するものである.後者
の方法はいろいろな分野に関する十分な量のデータがあればある程度の結果を出
すことは可能であろう.ただし,通常,文は対象領域や(小説,新聞,論文,技
術文書などという)ジャンルによって性質を異にする.そこで,コー
パスから得られた結果はそのコーパスの採取元になるジャンルに依存した結果に
なる.これらの問題点に加え,単なる統計的結果だけでは,その結果の応用範囲
の可能性や,結果の拡張性などについては何も分からない.そこでここでは,両
極の中間を採る.すなわち,まず第一に筆者らが収録した小規模なコーパスに対
してその分布状況を調べることにより何らかの傾向を見い出す.次に,このよう
にして得られた傾向に対して言語学的な説明を試みる.これによって,見い出さ
れた傾向の妥当性,応用や拡張の可能性が推測できる.

具体的には,従属節と主節の述語の性質を基礎に,主節主語と従属節主語の一致,
不一致という共参照関係を調べる.このような述語の性質として,動詞に関して
は,IPAL動詞辞書~\cite{IPALverb}にある意味的分類,ヴォイスによる分類,
ムード(意志性)
による分類を利用する.形容詞,形容動詞に関しては
IPAL形容詞辞書~\cite{IPALadj}にある分類,とりわけ
IPAL形容詞辞書~\cite{IPALadj}にある意味分類のうち心理,感情,感覚を表すも
のに関しては快不快の素性を,属性の評価に関しては良否の素性を利用する.例
えば,
\enumsentence{淋しいので,電話をかける.}
という文では,従属節に「感情-不
快」という性質を与え,主節に「意志的な能動の動詞」という性質を与える.また,
主節主語と従属節主語の一致,不一致については人手で判断する.このようにし
て与える従属節と主節の性質および主語の一致不一致の組合せが実例文において
どのように分布するかを調べ,そこに何か特徴的な分布が見い出されれば,その
原因について考察するという方法を採る.

\section{順接複文の性質}\label{node-kara}

本節では,この論文で対象としている順接複文すなわちノデ文,カラ文の意味論
的性質についてまとめておく.第一に順接複文は因果性を記述しており,その従
属節は原因,理由を示している.しかし,ノデ文とカラ文で微妙な差があり
\cite{日本語の複文構造,カラノデ},そ
れが主節主語と従属節主語の共参照関係に影響を与える可能性がある.以下で,
ノデ文とカラ文の意味について説明する.

\bigskip

\noindent {\dg ノデ文:} 従属節と主節とも話し手の主観的評価を離れて事実と
みなしている.よって,因果性は記述された世界の中に内在する. つまり,因
果性は,主節の主語が従属節の事態を評価した結果何らかの動作をしたり状態に
なったりするという形で現れるものである.

\noindent {\dg カラ文:} 主節の内容も従属節の内容も基本的には話し手が外部
から評価したものである.したがってカラ文の場合,因果性はむしろ話し手によっ
て認識されたものであるといえる.もちろん,ノデ文と同じく因果性がカラ文で
記述された世界に内在している場合も多い.

\bigskip

後の節で述べることを先取りすると,実際の例文を調べて得られる観察の妥当性
や拡張性の検討に当たって,これらのノデとカラの意味は中心的役割を果たす.
ただし,目下のところ,主節主語と従属節主語の共参照関係に関しては,次のよ
うな考察ができる.すなわち,ノデ文では主として主節主語,またカラ文では主
として話し手という差はあるものの,いずれも従属節で記述されている事態を観
察ないしは感覚し評価する人物がいる.また,カラ文の話し手が主節主語になっ
ている場合も多く,ノデ文と同じように記述された世界に内在する因果性を記述
することも多い.よって順接複文であるノデ文全ておよびカラ文のかなりの部分
の性質として次のことがいえる.

\bigskip

\noindent {\bf 1.} 従属節に記述されている事態を外部から観察可能なら,主
節主語と従属節主語は不一致でもよい.

\noindent {\bf 2.} 従属節に記述されている事態を外部から観察不可能なら,
主節主語と従属節主語は一致しなければならない.

\bigskip

なお,\cite{中川動機95,中川ので95}では,この性質を利用してノデ文の共参照
関係を語用論的に分析している.この論文では,外部からの観察可能性と述語の
意味との関連に着目することになる.


\section{IPALの述語の素性}\label{section3}

ここでは,IPALの動詞辞書\cite{IPALverb}および形容詞辞書\cite{IPALadj}に
記載されている述語の素性のうち,本論文で利用しているものについてまとめて
おく.まず,動詞の意味素性としては主として次の2点に着目する.

\begin{enumerate}
\item ヴォイスによる以下の4分類.能動(例:殺す ),相互(例:並ぶ),中動(例:走る),受動(例:習う).\\
注意すべき点は,能動では,主語から発する行為が他に及ぶ点である.したがっ
て,能動の場合は外部からの観察可能性が高い.一方,中動の場合は,主語の行
為が他には及ばないから外部からの観察可能性は低い.また,受動の場合は,主
語自身は原則的にはなんの行為もしていないわけで,同じく外部からの観察可能
性はさらに低いといえる.実際,収録した例文の中に受動詞が使われているもの
はなかった.

\item 意味的分類.大きくは状態と動作に二分され,さらに存在,所有,移動
  などに細分される.30種類近い細分類があるので,ここでは適当にまとめた
  分類を用いた.

\bigskip\hspace*{-3zw}\hspace*{-1mm}
ただし,部分的には次の意志性に関する性質も利用している.

\item 意志性による以下の分類.
1: 命令形なし(例:そびえる).2: 願望のみを表す(例:咲く).以上のふたつは常
に無意志である.3a: 命令をも表す(例:落す).基本的には意志性があるが,無
意志の用法もある.3b: 命令を表し,意志性の用法だけである(例:探す).
1,2,3bのタイプは意志性の有無については表層的な語彙からだけで判断できる.
しかし,3aはその判断は文脈などに依存するため意志性の有無を人手で判断しな
ければならない.そこで,本論文では3aタイプに関しては,意志性の判断を人手
で行なった.

\item この他に各項(ガ格,ヲ格,ニ格)の意味素性などの情報も記載されているが,これは対応する名詞の素性などとも関連してくるから,ここでは利用しなかった.

\end{enumerate}

\bigskip

形容詞,形容動詞に関しては,次のような素性に着目する.

\begin{enumerate}
\item 評価.属性の語義の形容詞は好ましい状態を表す場合は「良」,好ましくない状態を表す場合は「否」と書く.どちらでもない場合は単に「評価」と書くことにする.

\item 快・不快.感覚や感情を表す場合,話し手が快と感じるものなら「快」,不快に感じるものなら「不快」と書く.どちらともいえない場合は「心理」と書くことにする.

\end{enumerate}

なお,名詞\hspace{-0.3mm}$+$ \hspace*{-1mm}ダ は状態とする.形容動詞かどうかは,副詞「非常に」をつけら
れるかどうかでテストする.

\section{例文の分析}

この節ではノデ文とカラ文を,主節主語と従属節主語が一致,不一致の各々の場
合について,従属節の述語の性質と主節の述語の性質を表にした結果を示し,そ
こから観察される傾向について検討する.

\subsection{従属節が形容詞,形容動詞である場合の分布}

まず,動詞は,能動,中動,相互,受動(実際は例なし)に粗く分類し,形容詞,
形容動詞に関しては前節で述べたように 快 不快 心理,良 否 評価 に細分類し
た分布表を示す.また,以下では紛らわしくない場合は,主節主語と従属節主語
が一致する場合を単に「主語の一致」,不一致の場合を単に「主語の不一致」と
書くことにする.なお,以下の各表の各データにおいて 数字1/数字2 という記
述は,数字1がノデ文の個数,数字2がカラ文の個数を意味する.表のデータが空
欄であるのは 0/0 すなわちノデ,カラ文とも0個であることを表す.また,例文
数はノデ文が187例,カラ文が440例である.また,主語が一致するのは,ノデ文
が72例,カラ文が142例,主語が不一致なのはノデ文が115例,カラ文が298例で,
全体としてはノデ文とカラ文は同じような傾向である.

\begin{center}
\frame{\parbox{5mm}{\centering\shortstack{従\\属\\節}}
\begin{tabular}{|l||c|c|c|c|c|c|c|c|c|c|c|c|}
       &\multicolumn{12}{c}{主節}\\ \hline
& 能動  & 中動  & 相互  & 授受  & 使役  & 快  & 心理  & 不快  & 良  & 評価  & 否  & 状態 \\ \hline \hline
能動 & 0/4 & 5/10 &     &     &     &     &     &     &     & 0/2 & 0/1 & 0/3\\ \hline 
中動 & 3/8 & 33/37 & 1/0 & 2/2 & 1/0 &     & 0/1 & 2/2 &     & 2/1 & 0/1 & 0/4\\ \hline 
相互 &     & 1/0 &     &     &     &     &     &     &     &     &     &     \\ \hline 
授受 &     & 2/1 &     &     &     &     &     &     &     &     &     &     \\ \hline 
使役 &     &     &     &     &     &     &     &     &     &     &     &     \\ \hline 
快 &     &     &     &     &     &     &     &     &     &     &     &     \\ \hline 
心理 & 1/5 & 0/2 &     & 0/1 &     &     & 0/2 &     &     & 0/2 &     &     \\ \hline 
不快 & 0/2 & 5/4 &     & 0/1 &     &     &     &     &     & 0/1 &     &     \\ \hline 
良 &     & 1/1 & 0/1 &     &     &     &     &     &     &     &     &     \\ \hline 
評価 & 1/2 & 6/7 &     &     &     &     &     & 1/0 &     &     &     & 0/1 \\ \hline 
否 & 1/0 & 1/1 &     &     &     &     &     &     &     &     &     &     \\ \hline 
状態 & 1/6 & 1/13 & 1/2 &     &     &     & 0/2 &     &     & 0/5 & 0/1 & 0/3\\% \hline revised by shirai
\end{tabular}}\\
ノデ文の合計 72 例 / カラ文の合計 142 例\\
表 1. 形容(動)詞の性質を細分した分布(主語が一致の場合)\\
\end{center}
\vspace{1zh}
\begin{center}
\frame{\parbox{5mm}{\centering\shortstack{従\\属\\節}}
\begin{tabular}{|l||c|c|c|c|c|c|c|c|c|c|c|c|}
       &\multicolumn{12}{c}{主節}\\ \hline
& 能動  & 中動  & 相互  & 授受  & 使役  & 快  & 心理  & 不快  & 良  & 評価  & 否  & 状態 \\ \hline \hline
能動 & 0/7 & 7/24 & 1/0 & 0/3 &     &     & 0/1 &     & 0/1 & 0/12 &     & 0/2\\ \hline 
中動 & 9/31 & 48/62 &     & 1/8 & 0/1 & 1/1 & 0/2 & 1/1 & 2/3 & 3/20 & 0/1 & 0/13 \\ \hline 
相互 &     & 1/0 &     &     &     &     &     &     &     &     & 1/0 &     \\ \hline 
授受 & 0/1 & 1/2 &     & 0/1 &     &     &     &     &     &     &     &    \\ \hline 
使役 &     &     &     &     &     &     &     &     &     & 0/1 &     &     \\ \hline 
快 &     & 1/0 &     &     &     &     &     &     &     &     &     &     \\ \hline 
心理 & 0/3 & 1/0 & 0/1 & 0/1 & 0/1 &     &     &     &     & 0/1 &     &     \\ \hline 
不快 &     & 3/1 &     &     &     &     &     &     &     &     &     &     \\ \hline 
良 &     & 0/2 &     &     &     & 1/0 &     &     &     &     &     &     \\ \hline 
評価 & 0/11 & 6/10 &     &     &     &     & 1/3 & 1/0 &     & 0/2 & 0/1 & 0/2 \\ \hline 
否 & 4/2 & 3/4 &     &     &     &     &     &     &     & 1/1 &     &     \\ \hline 
状態 & 3/9 & 8/26 &     & 2/2 &     &     & 0/2 & 0/1 & 2/1 & 1/5 &     & 1/7\\% \hline revised by shirai
\end{tabular}}\\
ノデ文合計 115 例 / カラ文合計 298 例\\
表 2. 形容(動)詞の性質を細分した分布(主語が不一致の場合)\\
\end{center}

以下ではこれらの表から得られる観察およびそれに対する言語学的な考察を行なう.

これらの表によると,まずノデ文では,従属節が能動の動詞の場合,主節に形容詞,
形容動詞,状態を表す動詞がこない.しかし,この現象についてもう少し深く考
察してみる.まず,主節と従属節の主語が一致する場合について考えてみよう.
この場合,従属節で記述される自分の意志的な動作が原因となって自分が持つに
至った感情や感覚を主節として表現することはおかしい.なぜなら,その動作自
体が主節の主語の意志的なものであり,その動作の結果をある程度予想している
はずだからである.例えば,
\enumsentence{\label{1}$?$人を殺したので,恐ろしい.}
は主語が一致とすると解釈すると若干違和感がある.ただし,不一致なら,
おかしくない.これは次の文を見ればより明らかであろう.
\enumsentence{\label{2}おとなしそうに見えた隣人が人を殺したので,私は恐
ろしい.}
では,なぜ不一致ではおかしくないか.他人の動作であれば,意志的な動作であっ
ても,その動作に対する自分の感情や感覚を表す形容(動)詞で表現することは極
ありふれたことだからである.ところで,自分の動作の結果に対する感情や感覚
であっても,それが予想外に湧き上がってきた場合,すなわち状態の変化を表す
場合は,不自然ではない.つまり,主節が動詞であれば許容できる表現となるで
あろう.例えば,
\enumsentence{\label{3}人を殺したので,恐ろしくなった.}
なら,主語が一致でも不一致でもおかしくない.さて,このような考察は従属節
の動詞の意志性だけを利用して導いている.したがって,従属節の述語が能動の動詞
のみならず中動の動詞であっても,意志性のものであれば同じ制約があるはずである.
実際,例文を調べてみると,ノデ文では中動の動詞の場合でも,意志性がある場合は
主語が一致する例はない.

\begin{obs}\label{5}
{\dg ノデ文の場合:}
従属節の動詞が意志性であり,主節の述語が感情ないし感覚形容(動)詞の場合は,
主語が一致しない.ただし,主節の述語が動詞なら,感情や感覚を表す場合でも
主語が一致しうる.
\end{obs}

カラ文では,表1,2から分かるように,従属節が能動の動詞あるいは中動の動詞で主節が形容
(動)詞の例が多数ある.例えば,次のような一致および不一致の例である.
\enumsentence{\label{k1}赤ん坊は乳母になついたから,大丈夫だ.}
\enumsentence{\label{k2}何度追い払ってもついて来るから,嫌だ.}
\ref{node-kara} 節で述べたようにカラ文は従属節,主節ともその話し手の立場
から評価されたものである.つまり,主語にとってはあずかり知らない二つの事
象を話し手が原因と結果と認識して発話してよいわけである.例えば,
(\ref{k1})では,主語である赤ん坊自身が大丈夫かどうか関知していることはこ
の文で言いたいことではない.あくまで,原因と結果という認識は話し手がした
ものである.したがって,形容(動)詞で記述するような静的な状態が主節で記述
されかつ,主語が一致してもよくなるわけである.
しかし,表1,2を見ると,従属節が能動の動詞あるいは中動の動詞で主節が形容(動)詞の
場合,主語が一致の場合8例に対し不一致の場合42例と不一致が圧倒的に多い.
自分自身が何かの動作をしたことが理由になって,自分自身を評価したり感情を
持ったりするという状態は考えにくいということは一般的にいえる.一方,自分
自身の動作でなければ,それを評価したり,それに対して何らかの感情などを持
つことはなんら不自然ではない.よって,カラ文が話し手の視点から因果性を認
識するといっても,主語は一致しにくくなるのであろう.

\begin{obs}\label{5k}
{\dg カラ文の場合:}
従属節が能動の動詞あるいは中動の動詞であり,
主節が形容{\rm (}動{\rm )}詞だと,主語は一致しにくい.
\end{obs}

次の観察は表1,2から直接得られたものである.

\begin{obs}\label{01}
カラ文では,従属節が心理を表す形容{\rm (}動{\em )}詞の場合は
主語が一致しやすい.
\end{obs}

心理は本来,主観的であり,主節の主語が他人の心理を読んで何かをするという
事態は考えにくいという制限があると考えられる.したがって,ノデ文の場合は
主語の一致は当然予測されることであるが,話し手から因果関係を認識するカラ
文においても同様の制限が働いているのであろう.例えば,
\enumsentence{\label{02}会いたい ので/から,会いに出かけた.}のような文
である.

表1,2より次の観察も得られる.
\begin{obs}\label{10}
ノデ文の場合,従属節が不快あるいは否だと,主節は能動あるいは中動の動詞で
ある場合が一致,不一致のいずれの場合も多い.
\end{obs}
例えば,次のような例である.
\enumsentence{\label{20}苦しいので,薬を飲む.}
この例のように不快な状態からの脱出するための意志的動作をする場合に対応し
ている場合が多い.ただし,従属節が主節の主語に不快を与えて,その結果,主
節が「怒る」などの無意志的な動作を記述する場合もある.実際ここで集めた例
文を調べてみると,主節が意志的な動詞の場合は一致,無意志的動詞の場合は不
一致という結果である.したがって,観察~\ref{10} を一歩進め,次の考察が得
られる.

\begin{kousatu}\label{30}
ノデ文では従属節が不快の場合,主節が意志的な動詞なら主語は一致し,主節が無意志的な
動詞なら主語は不一致である.
\end{kousatu}

主語が一致の場合は(\ref{20})が例文になるが,不一致の場合は次のような例である.
\enumsentence{\label{31}相手がひどく横柄なので,ぼくはむっとした.}
一方,従属節が快あるいは良の場合は,その状態から脱出しようという意志は働
かないので,主節に意志性の動詞は来にくいと考えられる.事実,表 1,2では
このような組合せの例はない.ただし,全く不可能かといえばそうとも言い切れ
ない.例えば,
\enumsentence{\label{40}その宿を好きだったので,もう一度泊りに出かけた.}
のように,快あるいは良の状態を続けよう,ないしは繰り返そうという場合があ
りうる.

次に従属節,主節とも述語が形容(動)詞のノデ文について考えてみる.表 1,2に
おいてはこのようなケースは稀である.一般的に形容(動)詞は属性や状態を
表す.ある属性や状態が原因になって何らの変化もなしに別の属性や状態になる
ことはない.したがって,従属節と主節の双方において属性や状態が記述される
ことは考えにくいわけである.しかし,全く不可能というわけではなく,
主語が一致と不一致の例として各々,
\enumsentence{\label{50}私はその手の話には興味がないので,うんざりだ.}
\enumsentence{\label{60}電車があまりに混雑しているので,気分が悪い.}
のような例は可能である.実際,実例でこのタイプの文はこのような例であった.

ただし,これで全て尽きているというわけではなく,主語が一致の場合と不一致
の場合についてもう少し細かい分類を見て考察してみる必要がある.まず,一致
の場合だが,同一の主語が矛盾する感情や評価を同時に持つことはありえない.
よって次の考察が得られる.
\begin{kousatu}\label{70}
ノデ文では主語が一致する場合,従属節が快あるいは良,かつ主節が不快あるい
は否という組合せはありない.同様に従属節が不快あるいは否,かつ主節が快あ
るいは良という組合せはありえない.
\end{kousatu}
\vspace{-0.3mm}
このような組合せは収集した例文にも存在しない.ただし,主語が不一致だと,
ある人にとっての不快は別人(例えば敵)にとっての快という場合もあるから,
考察~\ref{70} のような組合せは矛盾ではなく,文として可能である.例えば,
\enumsentence{\label{80}同僚のガールフレンドがあまりに美しいので,
私はねたましかった.}
などという文が可能である.

一方,カラ文では従属節が形容(動)詞によって不快や否を表す例自体がほとんど
ない.感情,感覚などを表す心理的な述語は,そもそも主観的であり当事者(意
味役割としては経験者)自身の立場からしか記述できないとされている.カラ文
が外部の話し手の立場で記述していることを考えれば,従属節が主観的な快・不
快を表す場合が少ないことは納得できる.

ところで,従属節が形容(動)詞で評価の場合,主語の性質を調
べると次のような観察が得られた.

\begin{obs}\label{90}
ノデ文,カラ文とも従属節が形容{\rm (}動{\rm )}詞で評価だと,
従属節主語が無情物だと主
語が不一致である.
\end{obs}

例えば,
\enumsentence{\label{92}紛争地域の出張が多いから,家族が心配する.}
である.従属節主語が無情物とくになにかの状況だったりすると,その結果を被
るのは,その無情物そのものではなく,周りにいる人物や別の物である.なぜな
ら,無情物は意志的に動作しないから,無情物の評価が同一の無情物への別の評
価なり状態変化なりを生むとは考えにくい.よって,観察~\ref{90} になると考
えられる.この観察を少し拡張して考えると,従属節の主語が有情物とくに人間
であっても,その動作なり様子なりが主節の主語ないしは話し手から観察された
ような場合はやはり主語が一致しない.例えば,
\enumsentence{\label{91}あまりに参加者が多いので,驚いた.}
のような例はかなり多い.このような例は従属節だけを見て主語の一致,不一致
を予想することは難しいが,主節の動詞が感情を表す場合はあてはまる場合が多
い.よって,次のようになる.
\begin{kousatu}\label{94}
従属節の主語が有情物であっても,主節が感情を表す
述語の場合は,主語は一致しない場合がある.
\end{kousatu}
実際の例文では,ノデ文主語が一致しない傾向が強いことが確かめられたが,カ
ラ文では次のような一致の例も多く,必ずしもその傾向は見られない.
\enumsentence{\label{95}自分のやり方に自信を持っているから,他人の非難は
気にならない.}
これもやはりノデとカラの意味の差によると考えられる.つま
り,ノデ文では主節主語が従属節の事態を観察して主節に記述される感情を持つ
わけだから主語は一致しにくい.一方,カラ文では,従属節と主節の因果関係は
話し手の認識による.したがって,主節の主語がある感情を持ったことが,実は
主節の主語が従属節の事態を観察したこと以外の経験から得られたものでも,話
し手が両者を因果関係にあると認識しさえすればよい.よって,主語が一致して
も不都合はない.

なお,これらの表には現れていないが,実際の例文においてノデ文には無意志の
能動の動詞(IPALの意志性による分類の 1 および 2)は現れない.しかし,実際には
次のような例が可能であろう.

\enumsentence{前回の試合で勝ったので,敵を侮ってしまった.}こ
の例は主語が一致しているが,不一致の例も容易に作れる.よって,能動の動詞の無
意志性は今のところ決定的な要因とは言えない.

\subsection{主節および従属節の述語が動詞の場合}

前節の表1,2から,主節,従属節とも動詞の場合が非常に多いことが分かる.そ
こでこの節では,主節,従属節とも述語が動詞の場合について検討する.動詞の
分類は第~\ref{section3} 節で述べた意味的分類を利用する.また,意志性に係わ
る観察については適宜説明していく.以下の表3,4に例文の分布を示す.

\begin{center}
\frame{\parbox{5mm}{\centering\shortstack{従\\属\\節}}
\begin{tabular}{|l||c|c|c|c|c|c|c|c|c|}
       &\multicolumn{9}{c}{主節}\\ \hline
& \makebox[7mm]{存在} & \makebox[7mm]{関係} & \makebox[7mm]{単純}
& \makebox[7mm]{抽象} & \makebox[7mm]{動き} & \makebox[7mm]{生理}
& \makebox[7mm]{知覚} & \makebox[7mm]{言語} & \makebox[7mm]{他}
\\ \hline \hline 
存在所有 &     &     &     & 1/0 & 1/3 &     & 1/2 &     & 1/0 \\ \hline 
関係認定 &     &     &     &     & 1/0 &     &     &     &     \\ \hline 
単純状態 &     &     & 0/1 & 1/0 & 1/1 &     & 1/0 &     &     \\ \hline 
抽象的関係 &     &     &     & 1/0 & 0/2 &     & 2/0 &     &     \\ \hline 
動き &     &     &     &     & 5/14 & 1/2 & 1/8 & 0/2 &     \\ \hline 
生理 &     &     &     &     & 0/1 &     & 1/0 & 1/0 &     \\ \hline 
知覚心理 &     &     &     &     & 6/4 & 2/0 & 7/6 & 2/1 &     \\ \hline 
言語活動 &     &     &     &     &     &     &     &     &     \\ \hline 
その他 & 1/0 &     &     &     & 2/2 &     & 1/0 & 1/0 & 0/1 \\
\end{tabular}}\\
ノデ文合計 42例 / カラ文合計 52 例\\
表3 主節,従属節とも動詞であり,主語が一致する場合の分布\\
\end{center}


\vspace{2zh}

\begin{center}
\frame{\parbox{5mm}{\centering\shortstack{従\\属\\節}}
\begin{tabular}{|l||c|c|c|c|c|c|c|c|c|}
       &\multicolumn{9}{c}{主節}\\ \hline
& \makebox[7mm]{存在} & \makebox[7mm]{関係} & \makebox[7mm]{単純}
& \makebox[7mm]{抽象} & \makebox[7mm]{動き} & \makebox[7mm]{生理}
& \makebox[7mm]{知覚} & \makebox[7mm]{言語} & \makebox[7mm]{他}
\\ \hline \hline 
存在所有 &     &     & 1/2 & 0/1 & 3/9 &     & 4/5 & 0/2 & 0/3 \\ \hline 
関係認定 &     &     &     & 0/1 & 0/1 &     &     &     &     \\ \hline 
単純状態 & 1/0 &     &     &     &     &     & 1/1 &     &     \\ \hline 
抽象的関係 & 0/1 & 0/1 &     &     & 1/2 &     & 4/2 & 2/2 &     \\ \hline 
動き & 0/2 & 1/0 & 1/2 & 4/1 & 8/31 &     & 8/7 & 1/6 & 4/3 \\ \hline 
生理 &     &     &     &     &     &     & 1/0 &     &     \\ \hline 
知覚心理 & 0/1 &     & 0/1 & 0/1 & 1/2 & 0/1 & 0/3 & 0/1 & 1/1 \\ \hline 
言語活動 &     &     &     &     & 5/1 & 2/0 & 4/5 & 2/1 & 0/1 \\ \hline 
その他 & 1/0 &     &     & 0/1 & 3/8 &     & 1/1 & 0/3 & 0/2 \\
\end{tabular}}\\
ノデ文合計 65例 / カラ文合計 120 例\\
表4 主節,従属節とも動詞であり,主語が不一致の場合の分布\\
\end{center}

なお,表3,4で「動き」という欄は,意味素性が,移動,とりわけ出発帰着,出
現発生,設置,離脱,着脱,接触,加力,および,消滅,生産,もようがえ,の
全部をまとめたものである.また,「知覚心理」は知覚および心理という意味素
性をまとめたものである.

まず,従属節が存在所有の場合について説明する.表3,4には直接現れていない
が,ノデ文の不一致の場合8例すべておよびカラ文の不一致の場合22例中21例は
「ある」「いる」という動詞である.例えば,次のような例である.
\enumsentence{\label{1001}会議があるので,出かけた.}
\enumsentence{\label{1002}先客がいるので,ぼくは外で待っていた.}
(アスペクト辞である「ている」「てある」ではなく,本動詞の「いる」「ある」
である.)一方,ノデ文では一致の場合は「ある」「いる」は全く現
れず,「持つ」などの意志性を持ち得る動詞である.また,カラ文では一致の5
例中4例が「ある」,1例が「持つ」であった.この結果について少し考察してみ
る.「ある」の場合,通常,主語は有情物特に人間にはならない.したがって主
節主語が人間なら明らかに主語は不一致になる.ただし,例外として「子供があ
る」のような表現がある.しかし,この場合も下記の「いる」の場合と同じ理由
で主語は一致しない.では,主節,従属節とも同じ無情物でありうるかどうかに
ついて考えてみる.石や本などの無情物がある場所にあることが原因になって,
それ自体の状態変化を引き起こせるかどうか,という問題である.これは実際に
は可能であって,例えば,
\enumsentence{\label{100}その食物は長い間冷蔵庫の外にあったので,腐った.}
などは可能である.したがって,不一致は主節の主語が有情物の場合に限られる
であろう.次に,「いる」だが,これは明らかに有情物しか主語にならない.主
節の主語が無情物なら明らかに主語は不一致だから,主節の主語も有情物の場合
について考察すればよい.主節の主語がその複文が記述する状況に身を置くのは
自明である.したがって,もし主語が一致し,従属節で「いる」が使われると,
上記の記述された状況に身を置くという自明のことをわざわざ従属節で述べ立て
ることになり,明らかに冗長であるのみならず原因を示す従属節はなんの情
報も与えていないことになる.よって,主語は一致しないという結論が得られる.

上記の考察をまとめた次の考察は有用であろう.

\begin{kousatu}\label{110}~\\
1. 従属節の動詞が「ある」の場合,主節が有情物なら主語は一致しない.\\
2. 従属節の動詞が「いる」の場合,主語は一致しない.
\end{kousatu}

表から明らかに読みとれるノデ文における観察として次のものがある.

\begin{obs}\label{120}
ノデ文の場合,従属節が知覚思考あるいは心理を表す動詞だと,主語が一致しやすい.
\end{obs}
例えば,次のような例がありうる.
\enumsentence{\label{121}ぼくは,そのことを知らなかったので,びっくりした.}
知覚,思考,心理などは本来主語の内的な状態であり,それを外部から観察する
特殊な状況がなければ,知覚,思考,心理などの中動の動詞で表される状態を経験し
た人自身が,それを理由に何らかの動作なりをするというのが普通であろう.し
たがって,観察~\ref{120} は一般的に成立すると考えても良い.ただし,知覚な
どを外部から観察可能とする言語的表現としては,「そうだ」などの様相の助
動詞があり,このような場合は観察~\ref{120} の例外となる.例えば,
\enumsentence{\label{125}じっくり考えているそうなので,我々ももう少し待とう.}
また,知覚の場合,主語が知覚の主体でない「見える」「聞こえる」のような動
詞があり,この場合も主語の一致という点からは例外である.例えば,
\enumsentence{\label{122}ぼくはあこがれの大陸が見えたので,感激した.}
ただし,文法的なガ格でなく,知覚の主体を問題にするなら観察~\ref{120} に類
似の傾向が成立する.したがって,より洗練すると次の方がよい.
\begin{kousatu}\label{123}
従属節が知覚思考あるいは心理を表す動詞だと,従属節における知覚,思考など
の心理状態の主体{\rm (}意味役割としては経験者格{\rm )}が
主節の主語に一致する.
\end{kousatu}
この観察については\cite{中川ので95}に詳しく述べられている.
なお,カラ文の場合は観察~\ref{120} 自体がノデ文の場合ほど明確に成立しない.
つまり,従属節の文法的ガ格が知覚・思考の主体でない(\ref{122})のタイプの
例が多い.もちろん,従属節の経験者格に注目した考察~\ref{123} には多くの例
が当てはまるが,カラ文の場合この考察にも反する次のような例が存在する.
\enumsentence{\label{k121}彼らがやらないから,私がやった.}
これらのことは,ノデ文に比べて,カ
ラ文のほうが従属節の事態を話し手の立場からより客観的に記述していることの
現れであろう.つまり,本来主観的な知覚などの心理的経験をより客観的に記述
した場合はカラ文を使うということであろう.

一方,従属節が言語活動の場合は次のような観察が得られる.

\begin{obs}\label{130}
ノデ文,カラ文の双方において,従属節が言語活動を表す動詞だと,主語は不一
致である.
\end{obs}
例えば,
\enumsentence{\label{131}知らないと言ったので,それ以上追求しなかった.}
本来,言語活動は外部に現れる事象であり,かつ他人を意識したものだから,外
部からの観察可能性が高く,不一致となる.ただし,例外的ケースとして,自分の
発言を後から振り返るような場合は,一致することもできる.例えば,
\enumsentence{\label{132}まずいことを言ったので,後悔している.}
では,主語が一致している.また,将来の発言を予想しての文,
\enumsentence{\label{150}明日は長い時間しゃべるので,今日は早めに休む.}
でも,主語が一致しており,これらは観察~\ref{130} の例外である.

次に従属節で,移動などを含む「動き」が記述されている場合につい
て検討する.この場合も,言語活動と同じように考えられる.つまり,「動き」
も動作であることからして外部に現れる事象であり,外部からの観察可能性が高
いので,言語活動と同じような傾向を示すと予想される.ただし,「動き」の場
合,言語活動と異なるのは必ずしも他人を意識した動作だけではない点である.
したがって,主語の一致は言語活動ほど強い制約ではないと考えられる.ただし
ノデ文に限っては集めた例文では,主語が一致する場合5例は全て従属節がタ形
であり,主節と従属節の時刻が異なる.よって,次のような観察となる.

\begin{obs}\label{140}
ノデ文では従属節が動きを表す動詞で主語は一致だと,
過去形{\rm (}タ形{\rm )}である.
\end{obs}
例えば次のような例がありうる.
\enumsentence{\label{141}朝早く出発したので,昼のうちに到着した.}
ただし,不一致も全く不可能というわけではなく,
\enumsentence{\label{142}あまりにたくさんの人が来たので,驚いた.}
のような例も作例できる.
カラ文では主語が不一致の場合の過去形15例,非過去形33例,主語が一致の場合
の過去形10例,非過去形15例で,特段の傾向は見られない.したがって,このよ
うな現象に関してはノデ文との差が際だっている.

次に,観察~\ref{140} に関して,従属節が動作動詞の場合の時制の影響について
考察してみる.主節の主語が従属節の表す事態を知覚,あるいは感覚し評価して
から,それに対応する行動を起こす.したがって,主節主語と従属節主語が一致
している場合は,従属節で記述された自分の自身の動作を評価する時間が必要で
ある.よって,従属節の参照時刻は主節の参照時刻より以前になる.日本語では,
従属節のタ形は主節の時刻より以前であることを示す.よって,従属節はタ形に
なるのが一般的である.しかし,次のような例文もある.
\enumsentence{アメリカへ留学する ので/から,英語の勉強をした.}
この場合でも,留学が決まったのは,英語の勉強をするより以前である.カラ文
で従属節が非過去形で主語が一致するのは,このようなタイプの文が多い.この
文のように主節と従属節の時刻の差は表層だけからは分からないから,上記の時
制の影響の分析を機械的に利用することは困難である.もちろん,不一致なら主
節と従属節の時刻について特に制約はない.

\section{おわりに}

以上,この論文では実例文の調査から得られた観察に言語学的考察を加えるとい
う方法で,順接複文の主節と従属節の主語の共参照関係に関するいくつかの観察
を提案した.この観察は計算機上へ日本語理解システムを作る際に,複文の省略
された主語の指示対象を同定する場合に直接役立つ言語学的知識である.ただし,
このような応用を考えるに当たっては,考慮すべき問題点がいくつかあるので,
ここではそれについて述べる.

最後に,ここまで述べてきたような研究をするにあたって,IPALの辞書を利用す
る場合の問題点について述べる.動詞の意志性を利用した分析を行ない,これは
かなり有力であることが分かった.しかし,意志性の有無は,1,2,3bタイプなら
表層の語彙から機械的に判断できるが,多数存在する3aタイプは意志,無意志両
方の可能性があるので,人手で判断しなければならなかった.意志性は文脈依存
的である部分も多く,自動的な判定が難しい.よって機械的な処理においては大
きな問題になる.
次に問題であったのは多数存在するサ変動詞の意味分類をどのように扱うかであ
る.動詞性接尾辞スルだけでは意味分類を決定できないので,ここではサ変動詞
を構成する名詞の意味分類によって人手で判断した.これは,名詞の意味分類を
記述した辞書が整備されれば,機械的にできるようになるであろう.

なお,今回の分析では動詞句を構成しうる様相助動詞,クレル,ヤルなどの視点
に関する表現,については考慮しなかった.これらについては\cite{中川動機
95}において,主語の共参照関係にどのように影響するかについて分析している
が,今回の述語の意味に基づく分析とどのように関係するか,また共参照関係の
決定への寄与がどちらがどの程度の割合かなどを検討する必要がある.このよう
な検討を経て順接複文の理解システムの基本的設計を行なっていくことが今後の
課題として重要である.



\vspace*{7mm}
\acknowledgment

例文の収集および統計処理に尽力してくれた横浜国立大学の木村啓一君,俵正樹君,山本恵理子さんに感謝いたします.



\bibliographystyle{jnlpbbl}
\bibliography{jpaper}

\begin{biography}
\biotitle{略歴}
\bioauthor{中川 裕志}{
1953年生.1975年東京大学工学部卒業.1980年東京大学大学院博士課程修了.工学博士.現在横浜国立大学工学部電子情報工学科教授.自然言語処理,日本語の意味論・語用論などの研究に従事.日本認知科学会,人工知能学会などの会員.}

\bioreceived{受付}
\bioaccepted{採録}

\end{biography}

\end{document}
