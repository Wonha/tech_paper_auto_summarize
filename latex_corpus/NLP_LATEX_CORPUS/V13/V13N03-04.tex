    \documentstyle[graphicx,ascmac,jumoline,linguex,jnlpbbl]{jnlp_j_b5_2e}


\setcounter{page}{91}
\setcounter{巻数}{13}
\setcounter{号数}{3}
\setcounter{年}{2006}
\setcounter{月}{7}
\受付{2005}{11}{4}
\再受付{2006}{1}{25}
\採録{2006}{2}{26}

    \setcounter{secnumdepth}{3}
\def\dg{}

\setlength\UnderlineDepth{0.8pt}

\title{談話構造解析に基づくスライドの自動生成}
\authorC{柴田 知秀\affiref{UT} \and 黒橋 禎夫\affiref{KU}}
\headauthor{柴田,黒橋}
\headtitle{談話構造解析に基づくスライドの自動生成}
\affilabel{UT}{東京大学大学院情報理工学系研究科}
	{Graduate School of Information Science and Technology, University of Tokyo}
\affilabel{KU}{京都大学大学院情報学研究科}
	{Graduate School of Informatics, Kyoto University}
\jabstract{
本稿では,テキストから要約スライドを自動生成する手法を提案する.本稿で生
成するスライドは,入力テキストから抽出したテキストの箇条書きからなる.そ
れらに適切なインデントを与えるには,対比関係や詳細化関係などといった文ま
たは節間の関係を解析する必要がある.本手法では,まず,接続詞などの手がか
り表現,語連鎖の検出,二文間の類似度の三つの観点を用いてテキストの談話構
造を解析する.そして,テキストから主題部・非主題部を抽出・整形し,抽出し
たテキストのインデントを談話構造に基づいて決定することにより,スライドを
生成する.実験を行なったところ,入力テキストよりもかなり見やすいスライド
を自動生成できることが確認された.
}
\jkeywords{談話構造解析,主題抽出,文簡約,自動プレゼンテーション}

\etitle{Automatic Slide Generation Based on \\Discourse Structure Analysis}
\eauthor{Tomohide Shibata\affiref{UT} \and Sadao Kurohashi\affiref{KU}} 
\eabstract{
In this paper, we describe a method for automatically generating summary
slides from a text. The slide consists of itemizations of extracted
texts, and to determine their indentation, we need to analyze relations
between sentences/clauses, such as contrast and elaboration. We first
analyze the discourse structure of the text by considering three types
of information: cue phrases, identification of word chain and similarity
between two sentences. Then, we extract topic/non-topic parts from the
text and generate the slide by placing the extracted texts, whose
indentations are controlled according to the discourse structure. Our
experiments demonstrate that generated slides are far easier to read in
comparison with original texts.
}
\ekeywords{discourse structure analysis, topic extraction, sentence reduction, automatic presentation}

\begin{document}
\maketitle
\thispagestyle{empty}


\section{はじめに}
\label{sec:intro}

スライドを用いたプレゼンテーションは,意見を人々に伝えるのに大変効果的で
あり,学会やビジネスといった様々な場面において利用されている.近年,
PowerPointやKeynoteといったプレゼンテーションスライドの作成支援をするソ
フトが開発・整備されてきているが,一からスライドを作成することは依然とし
て大変な作業である.

そこで,科学技術論文や新聞記事からプレゼンテーションスライドを自動(また
は半自動)で生成する手法が研究されている.Utiyamaらは,GDAタグで意味情報・
文章構造がタグ付けされた新聞記事を入力としてプレゼンテーションスライドを
自動生成している\cite{Utiyama99}.また,安村らは,科学技術論文の\TeX ソー
スを入力として,プレゼンテーション作成を支援するソフトウェアを開発してい
る\cite{Yasumura03j}.しかし,いずれの研究においても,入力テキストに文章
構造がタグ付けされている必要があり,入力テキストを用意することにコストが
かかってしまう.

\begin{figure}[t]
\fbox{
\begin{minipage}[t]{\hsize}
大阪と神戸を結ぶJR神戸線,阪急電鉄神戸線,阪神電鉄本線の3線の不通によ
り,一日45万人,ラッシュ時最大1時間12万人の足が奪われた.JR西日本
東海道・福知山・山陽線,阪急宝塚・今津・伊丹線,神戸電鉄有馬線の不通区間
については,震災直後から代替バスによる輸送が行われた. 国道2号線が開通
した1月23日から,同国道と山手幹線を使って,大阪〜神戸間の代替バス輸送
が実施された. 1月28日からは,国道2号,43号線に代替バス優先レーン
が設置され,効率的・円滑な運行が確保された.
\end{minipage}
}
\caption{入力テキストの例}
\label{fig:text_example}
\end{figure}

\begin{figure}[t]
\begin{center}
\begin{minipage}[t]{\hsize}
\begin{shadebox}
\vspace{2mm}
\begin{center}鉄道の復旧\end{center}
\begin{itemize}
\item 大阪と神戸を結ぶJR神戸線,阪急電鉄神戸線,阪神電鉄本線の3線の不通
\begin{itemize}
\item 一日45万人,ラッシュ時最大1時間12万人の足が奪われた
\end{itemize}

\item JR西日本東海道・福知山・山陽線,阪急宝塚・今津・伊丹線,神戸電鉄有馬線の不通区間
\begin{itemize}
 \item 震災直後から
  \begin{itemize}
   \item 代替バスによる輸送
  \end{itemize}
 \item 国道2号線が開通した1月23日から
   \begin{itemize}
    \item 同国道と山手幹線を使って,大阪〜神戸間の代替バス輸送が実施
   \end{itemize}
 \item 1月28日から
       \begin{itemize}
    \item 国道2号,43号線に代替バス優先レーンが設置され,効率的・円滑な
	  運行が確保
 \end{itemize}
\end{itemize}
\end{itemize}
\vspace{2mm}
\end{shadebox}

 \end{minipage}
\end{center}
\caption{自動生成されたスライドの例}
\label{fig:slide_example}
\end{figure}

本稿では,生テキストからスライドを自動生成する手法を提案する.入力テキス
トの例を図\ref{fig:text_example},それから自動生成されたスライドの例を
図\ref{fig:slide_example}\,に示す.本稿で生成するスライドは,入力テキストか
ら抽出したテキストの箇条書きから構成される.箇条書きを使うことによって,
テキストの構造を視覚的に訴えることができる.例えば,インデントが同じ要素
を並べることで並列/対比関係を表わすことや,インデントを下げることによっ
て詳細な内容を表わすことなどといったことが可能となる.従って,生成するス
ライドにおいて,箇条書きに適切なインデントを与えるには,入力テキストにお
ける,対比/並列関係や詳細化関係などといった文または節間の関係を解析する
必要がある.本稿では,入力テキストの談話構造を解析し,入力テキストから抽
出・整形されたテキストを箇条書きにし,そのインデントを入力テキストの談話
構造に基づいて決定することによりスライドを生成する.生成されたスライドは
入力テキストに比べて見やすいものにすることができる.特に,テキストに大き
な並列や対比の構造があると,見やすいスライドを生成することができる.
図\ref{fig:slide_example}\,の例では,「震災直後から」,「国道2号線が開通し
た1月23日から」,「1月28日から」の対比の関係が解析され,それらが同じイン
デントで表示されることにより見やすいスライドが生成されている.また,
図\ref{fig:slide_example}\,の例の「震災直後から」と「代替バスによる輸送」の
ように,各文から主題を取り出し,主題部と非主題部を分けて出力することによ
り,スライドを見やすくしている.特に対比関係の場合,何が対比されているの
かが明確になる.

本稿で提案するスライド生成の手法の概要を以下に示す.

\begin{enumerate}
 \item 入力文をJUMAN/KNPで形態素解析,構文解析,格解析する.

 \item 入力文を談話構造解析の基本単位である節に分割し,表層表現に基づいて談
       話構造解析を行なう.

 \item 入力文から主題部・非主題部を抽出し,不要部分の削除,文末の整形を
       行なう.

 \item 談話構造解析結果に基づき,抽出した主題部・非主題部を配置すること
       によりスライドを生成する.
\end{enumerate}


また,我々の手法は,プレゼンテーションスライドの作成支援を行なうだけでな
く,自動プレゼンテーションを生成することができる.すなわち,テキストを入
力とし,自動生成したスライドを提示しながら,テキストを音声合成で読み上げ
ることにより,自動でプレゼンテーションを行なう.我々はこのシステムのこと
を,「text-to-presentation システム」と呼んでいる
(図\ref{fig:presentation_system}).難解な語や長い複合語は音声合成の入力に適
しているとはいえないので,Kajiらの言い換え手法\cite{Kaji02,Kaji04}で書き
言葉を話し言葉に自動変換してから音声合成に入力することにより,音声合成の
不自然さを低減する.

\begin{figure*}[t]
 \begin{center}
    \includegraphics[scale=0.55]{ttps-j.eps}
\caption{text-to-presentationシステム}
\label{fig:presentation_system}
 \end{center} 

\end{figure*}

本稿の構成は以下のようになっている.\ref{sec:ds_analysis}\,章で談話構造解
析について述べ,\ref{sec:topic_extract}\,章で入力テキストからスライドに表
示するテキストを抽出する方法について述べ,\ref{sec:output_slide}\,章でスラ
イドの生成方法を述べる.そして,\ref{sec:evaluation}\,章で実装した
text-to-presentation システムと,自動スライド生成の実験の結果を報告する.
\ref{sec:related_work}\,章で関連研究について述べ,\ref{sec:conclusion}\,章で
まとめとする.

\section{談話構造解析}
\label{sec:ds_analysis}

この章では,テキストの談話構造を解析する手法を述べる.まず,談話構造のモ
デルを説明し,次に談話構造を解析する手順について説明する.
\subsection{談話構造のモデル}
\label{subsec:ds_model}

談話構造を図\ref{fig:discourse_structure}\,に示すようなものにモデル化した.
図において,矢印,ラベルはそれぞれ,文(S)または節(C)の接続,結束関係を意
味する.このモデルでは,初期状態として初期節点を設けており,文が初期節点
に接続する時の結束関係を``初期化''とし,この文から新しい話題が始まること
を意味する.

\begin{figure*}[t]
 \begin{center}
     \includegraphics[scale=.48]{ds_model-j.eps}
\caption{談話構造のモデル}
\label{fig:discourse_structure}
 \end{center} 
\end{figure*}

節と文を談話構造の基本単位とし,以下にあげる二種の結束関係を考える.どの
ような結束関係を考えるかは研究者によって異なるが,スライドを自動生成する
ためにはこれらで十分であると考えた.

\begin{enumerate}
 \item 一文内における節間の関係 (4種類)

 \begin{quote}
   並列,対比,順接,逆接
 \end{quote}

 \item 二文間の関係 (11種類)
 \begin{quote}
並列,対比,理由,条件,主題連鎖,焦点主題連鎖,詳細化,理由,原因結果,例提示,質問応答  
\end{quote}
\end{enumerate}


\vspace{5mm}

次節から,談話構造解析について述べる.解析の手順は\cite{Kurohashi94j}に基
づいている.解析は入力文一文ずつ行ない,談話構造を逐次的に構築する.まず,
入力文を節に分割し,節間の関係を解析する.次に,すでに入力した文の中で,
入力文と最も関連する文とその間の結束関係を様々な手がかりをもとに決定する.
図\ref{fig:discourse_structure}\,の例は,1文目から順番に談話構造を解析して
いき,4文目までの構造が決定され,5文目の解析を行なっている様子を示してお
り,様々な文・結束関係の確信度を計算した結果,最も高い確信度を得た4 文目
と詳細化の関係で接続すると解析されている.


\subsection{主題属性の付与}
\label{subsec:topic_feature}

まず,入力文をJUMANで形態素解析,KNPで構文・格解析を行なう.その後,
\ref{subsec:dpnd_in_s}\,節で述べる対比関係の抽出,
\ref{subsec:relation_two_sentences}\,節で述べる主題・焦点の抽出,
\ref{sec:topic_extract}\,章で述べる主題部・非主題部の抽出のために,あらか
じめ,主題となりうる文節に主題属性を付与しておく.以下にあげるようなパター
ンを満たす文節に主題属性を付与する.パターンは形態素を単位に記述し,入力
文の形態素解析結果と照合する.

\begin{itemize}
 \item 延焼速度\Underline{は},...
 \item インナーシティ\Underline{では},...
 \item 出火原因の判明した火災\Underline{において},...
 \item 3線の不通\Underline{により}, ...

\end{itemize}

また,以下のパターンの場合は〜の部分に主題属性を付与する.

\begin{quote}
…\{する/した\}\{の/とき\}は〜\{だ/になる\}
\end{quote}

以下の例では,「安否情報など」に主題属性を付与する.

\ex. 震災直後に被災者が必要と\Underline{したのは},家族や友人・知人の消息に関する安否情報など\underline{\underline{だった}}.

なお,ここで付与した主題属性を利用して,
\ref{subsec:dpnd_in_s}\,節や\ref{subsec:relation_two_sentences}\,節
で対比・並列関係の解析を行なうが,
関係を解析した結果,新たに主題属性が付与される場合がある.

\subsection{入力文の節への分割}
\label{subsec:divideintonodes}

談話構造において,何をその基本単位とするかは研究者によって様々な定義がな
されてきた.\cite{Polanyi88,Kurohashi94j}では文,\cite{Longacre83}では節,
\cite{Grosz86,Marcu99a,Marcu99b}では独自に定義された単位(clause-like
unit)が談話構造の基本単位として採用されている.本研究では,スライドに配
置する箇条書きが適切な長さとなるように,節に分割する基準を,南
\cite{Minami93}の従属節の分類に応じて以下のように設定した\footnote{本論
文では,一つの述語からなるまとまりではなく,ここで定義したものを節と呼ぶ
ことにする.}.なお,節は談話構造解析の基本単位であると同時に,
\ref{sec:topic_extract}\,章で説明する主題部・非主題部の抽出の基本単位でも
ある.

\begin{figure*}[t]
 \begin{center}
     \includegraphics[scale=.49]{clause-j.eps}
\caption{文の節への分割}
\label{fig:discourse_unit}
 \end{center} 
\end{figure*}

\begin{description}
 \item [レベルC] (例: 〜が,〜けれども) : 必ず分割する
 \item [レベルB] (例: 〜て,〜し) : 前後の文節列が類似している場合,また
	     は,節の文字数が閾\linebreak[4]値\footnote{この閾値はスライドの横幅とフォ
	     ントのサイズによって決定される.}以上の場合に分割する
 \item [レベルA] (例: 〜ながら,〜つつ) : 分割しない 
\end{description}

レベルBにおける節の分割において利用している文節列の類似度は,KNPで並列構
造を検出するために計算している任意の2つの文節列の類似度を用いる.任意の2 
つの文節列の類似度計算は,まず,あらゆる2文節の類似度を語の一致,品詞の
一致,シソーラス\cite{NTT}による類似度などにより計算し,その上で,DPマッ
チングでそれらの文節間の類似度を組み合わせることにより行なわれる
\cite{KNP94}.

\ref{ex:koube-level}\,の例では,「達したが」のレベルがCなので節に分割し,
\ref{ex:hyakunin-level}\,の例では「減り」のレベルがBで,前後の文節列が類似
しているので節に分割する.\ref{ex:shinsai-level}\,の例では,前後の文節列が
類似していないので,「なく」で分割せず,「迫られて」もレベルがAなので分
割を行なわない.


 \ex. [神戸市には,他都市,業界等からの仮設トイレ支援が約3,000基に 
 \Underline{達したが,} $_{レベルC}$] [受入れのための仮置き場の確保が
 大きな課題となった.]\label{ex:koube-level}

\ex. [100人に1基行き渡った段階で設置についての苦情はかなり
\Underline {減り,}$_{レベルB}$]  [75人に1基達成できた段階では苦
情が殆どなくなった.]\label{ex:hyakunin-level}

\ex. [震災時の環境保全については事前の具体的な対応策等が\Underline
{なく,}$_{レベルB}$ 必要に\Underline{迫られて}$_{レベルA}$進めざるを得なかった.]\label{ex:shinsai-level}

\subsection{一文内の節間の関係の解析}
\label{subsec:dpnd_in_s}

入力文を節に分割した後,一文内の節間の結束関係を求める.まず,各節の親の
節を構文解析結果に基づいて決定する.すなわち,各節の親を,節の最終文節の
係り先の文節を含む節とする.次に,節間の結束関係を以下の基準で決定する.

\begin{itemize}
 \item 二節が類似している場合
 \begin{itemize}
  \item 並列
  \item 対比
 \end{itemize}
 \item 類似していない場合
  \begin{itemize} 
   \item 順接 (〜て,(連用形))
   \item 逆接 (〜が,〜けれども)
  \end{itemize}
\end{itemize}

まず,二節が類似していない場合,結束関係を順接または逆接とし,順接である
か逆接であるかは節末の形態素列のパターンで認識する.

二節が類似している場合,結束関係を対比または並列とする.一般に並列の関係
の場合,人または物がある二つの属性を持ち,対比の関係の場合,二つの異なる
人または物が類似した属性を持つ.従って,二つの節において,主題属性が付与
された二文節が,二節の類似度を計算する際のDPマッチングのパス上で対応関係
にあり,かつ,それらの類似度が閾値以上である場合,結束関係を対比とし,そ
うでない場合を並列とする.

以下の例では,二節が類似しており,主題属性が付与された「当初は」と「3月
末までは」が類似しているので,対比の関係とする.

\ex. [代替バス利用者は,\Underline{当初は}1日あたり3〜5万人であった
が,] [\Underline{3月末までは}1日約20万人が利用した.]\label{ex:bus}

また,どちらか一方の節の文節に主題属性が付与されており,もう一方の節の文
節には主題属性が付与されていない場合でも,類似度が高い場合は対比関係とす
る.以下の例では,二節が類似しており,主題属性が付与された,「(75人に
1基達成できた)段階では」と主題属性のついていない「(100人に1基行き渡っ
た) 段階で」が類似しているので,結束関係を対比とし,「(100人に1基行
き渡った) 段階で」に主題属性を付与する.

\ex. [100人に1基行き渡った\Underline{段階で}設置についての苦情はかな
り減り,] [75人に1基達成できた\underline{\underline{段階では}}苦情が殆ど
なくなった.]

以下の例では,主題属性の付与された「パソコン通信ニフティサーブでは」とDP 
マッチングで対応付けられた「ボランティア情報」との間の類似度が閾値以下な
ので,結束関係を並列とする.

\ex. [\Underline{パソコン通信ニフティサーブでは}「地震情報コーナー」が
開設され,] [ボランティア情報,安否情報,行政情報など各種の情報提供に
用いられた.]


\subsection{二文間の関係の検出}
\label{subsec:relation_two_sentences}

二文間の関係は,種々の表層的手がかりをもとに,各入力文に対して,関係をも
つ以前の文(接続文)とその間の結束関係を逐次的に求める.新しい話題が導入さ
れた後に古くなった話題に接続することはないという仮定をおき,入力文は談話
構造の一番最後の子供の文にのみ接続可能と考える.
図\ref{fig:discourse_structure}\,では,文5は初期節点,文3,文4に接続可能とな
り,文1,文2との接続を許さない.そして,さまざまな接続可能文との間のさま
ざまな結束関係を考慮し,(1)手がかり表現,(2)語連鎖,(3)二文の類似度の3つ
の観点から確信度を計算し,最終的に最も高い確信度を得た関係を採用する.以
下,これらの3つの観点について順に詳しく述べる.

\begin{enumerate}
\item 手がかり表現

      種々の結束関係を示す,接続詞などの表層的な手がかり表現を認識し,そ
      の結束関係への確信度を得るために,表\ref{tab:rule}\,に示すようなルー
      ルを用意した.表\ref{tab:rule}\,において,接続可能文パターン,入力文
      パターンは,それぞれに対する表層表現,文間の結束関係\linebreak[4]([ ] で括られ
      たもの)などのパターン,適用範囲とはどれだけ離れた文との関係まで考
      えるかである.適用範囲において,「1」は接続可能文と入力文が隣接し
      ている場合のみルールが適用されることを,「*」はルールの適用に制限
      がないことをそれぞれ意味する.ルールが一致した場合には,指定された
      結束関係欄の関係に対して,確信度欄の点数が与えられる.この確信度は
      経験的に決定した.


\begin{table}[t]
\caption{談話構造解析のルール}
\label{tab:rule}
\begin{center}
\begin{small}
\begin{tabular}{@{\ }l@{\ }|@{\ }l@{\ }|@{\ }c@{\ }||@{\ }c@{\ }|@{\
 }c@{\ }} \hline
\begin{tabular}{c}
接続可能文 \\ パターン
\end{tabular}
 & 
\begin{tabular}{c}
入力文 \\ パターン
\end{tabular}
&
適用範囲
 &
 結束関係 & 確信度
 \\ \hline
 〜 & さて〜 & * & 初期化 & 10 \\ 
 〜 & そして〜 & 1 & 並列 & 5 \\ 
 第一に〜 & 第二に〜 & * & 並列 & 30 \\ 
 \verb|[|並列\verb|]| & さらに〜 & 1 & 並列 & 40 \\ 
 〜 & むしろ〜 & 1 & 対比 & 30 \\ 
 〜 & すなわち〜 & 1 & 詳細化 & 30 \\ 
 〜 & 〜からだ & 1 & 理由 & 30   \\ \hline
\end{tabular}
\end{small}
\end{center}
\end{table}

\item 語連鎖の検出

      一般に文は主題を示す部分(主題部)とそれ以外の部分(非主題部)に分ける
      ことができ,主題を\ref{subsec:topic_feature}\,節で付与した主題属性の
      ついている文節から名詞をとり出したもの,焦点を非主題部の名詞とする.
      そして,二文間で,主題と主題,焦点と主題に語の連鎖(同一の語/句の出
      現)がある時は,それぞれ,主題連鎖,焦点主題連鎖の結束関係に確信度
      を与える.語連鎖は,完全一致と部分一致を考え,完全一致の場合は確信
      度15点を,部分一致の場合は確信度10点を与える.

\item 二文間の類似度

      二文が並列または対比の関係にある場合,それらはある種の類似性を持つ
      と仮定することができる.二文間の類似度は,一文内の節の係り受け関係
      の所で述べた,任意の文節列間の類似度計算の方法で計算することができ
      る.そして,一文内の対比/並列の関係の検出と同じように,二文におけ
      る主題が類似している場合は対比の関係に,類似していない場合は並列の
      関係に確信度を与える.

      以下の例では,二文が類似しており,主題属性が付与されていない
      文\ref{ex:23}\,の「1月23日から」と,主題属性が付与されている
      文\ref{ex:28}\,の「1月28日から」に高い類似が認められるので,対比の
      関係に確信度を与え,「1月23日から」に主題属性を付与する.

\ex. \a. 国道2号線が開通した\Underline{1月23日から,}大阪〜神戸間の代替バス輸送が実施された. \label{ex:23}
 \b. \Underline{1月28日からは,}国道2号,43号線に代替バス優先レー
      ンが設置され,効率的・円滑な運行が確保された.\label{ex:28}\\
\par

\end{enumerate}

\section{スライドに表示するテキストの抽出}
\label{sec:topic_extract}

この章では,入力テキストから,スライドに表示するテキストを抽出する手法を
説明する.\ref{subsec:relation_two_sentences}\,節で述べたように,文は主題
部と非主題部から成る.入力テキストから文を抽出してそのままスライドに配置
するのではなく,主題部と非主題部を分けてスライドに配置することにより,ス
ライドを見やすいものとする.また,非主題部は一般に長いことが多いので,非
主題部の簡約を行なうことにより,スライドを見やすくする.主題部と非主題部
の抽出は,\ref{subsec:divideintonodes}\,節で分割した節を基本単位として行な
う.一連の解析の様子を図\ref{fig:topic_extract}\,に示す.

\begin{figure*}[t]
 \begin{center}
     \includegraphics[scale=.4]{extract_topic-j.eps}
\caption{主題部・非主題部の抽出と非主題部の簡約}
\label{fig:topic_extract}
 \end{center} 
\end{figure*}


\subsection{主題部・非主題部の抽出}
\label{subsec:extract_topic}

\ref{subsec:topic_feature}\,節で付与した主題属性をもとに,主題部の抽出を行
なう.主題属性が付与された文節から構文木を子の方向にたどって句
を抽出し,それを主題部とし,残りの部分を非主題部とする.以下の例では,「延
焼速度」が主題部として抽出され,「おおむね20〜40\,m/h程度で,」が非主
題部となる.

\ex. \Underline{延焼速度は}おおむね20〜40\,m/h程度で,

節に主題属性が付与された文節が複数存在する場合は,そのうち一番前にあるも
ののみを抽出する.以下の例では,「震災初日の被災地内では」と「視聴は」に
主題属性が付与されているが,一番前の「震災初日の被災地内では」を主題部と
して抽出し,残りを非主題部とする.

\ex. 震災初日の被災地内\underline{\underline{では}}停電などによりテレビの視聴\Underline{は}ほとんどできず,\label{ex:double_topic}

ただし,主題属性のついているもので,対比関係にあると解析されたものは必ず
抽出する.以下の例では,前の節と後の節が対比の関係にあり,「神戸市では」
に主題属性に主題属性が付与されており,主題属性の付与された「当初は」と
「1月22日には」が対比の関係にあると解析されている.このような場合,前
の節からは,「神戸市では」と「当初は」の両方を主題として抽出する.

\ex. [\Underline{神戸市では},\Underline{当初は} 仮設トイレ300基程度
で足りると考えていたが,] [\Underline{1月22日には} 「仮設トイレ対策
本部」を設置し対応することとなった.]

\subsection{非主題部の簡約}

スライドを見やすいものとするためには,できるだけ入力テキストの情報を保持
した上で,テキストを簡約する必要がある.本研究では,(1)構文解析結果に基
づく不要な語あるいは語句の削除,(2)節末の用言の整形により,非主題部の簡
約を行なう.

\begin{enumerate}
\item 構文解析結果から不要部分の削除

構文解析結果から以下の不要な語句の削除を行なう.

\begin{itemize}
 \item 接続詞
 \item 副詞
 \item レベル:Aの節
 \item 副詞句

  例) \Underline{バッテリー切れによる利用不能のほか},救援・復旧関係者による被災地外から大量持ち込みによる輻輳の発生で利用できなくなった.

 \item 同格 : 節末の用言の子の文節に「〜など」があれば削除する.

  例) \Underline{農林水産省,国土庁など}国の各機関

\end{itemize}

\item 節末の用言の整形

次のようなルールにより節末の用言の整形を行なう.

\begin{itemize}
 \item サ変名詞 + する/された $\rightarrow$ サ変名詞

 例) 実施された $\rightarrow$ 実施

 \item サ変名詞 + が行われた $\rightarrow$ サ変名詞

 例) 輸送が行われた $\rightarrow$ 輸送

 \item 名詞 + 判定詞 $\rightarrow$ 名詞

 例) 無被害であった $\rightarrow$ 無被害

 \item ナ形容詞/ナノ形容詞 $\rightarrow$ 活用語尾を削除

 例) 軽微であった $\rightarrow$ 軽微
\end{itemize}

ただし,節末の用言に否定表現を含む場合は,否定表現を削除してしまわないよ
      うに,否定表現より後の部分の整形を行なう.

\end{enumerate}


\begin{center}
\begin{table}[t]
\caption{重要説明表現} \label{significant_words}
\begin{center}
 \begin{tabular}[tb]{l|l}
 \hline
 格 & 用言 \\ \hline
 ガ格 & 重要だ,本質をつく,エッセンスだ,\\
  & ポイント,望ましい,鍵だ,大切だ,\\
  & 有益だ,必要だ,指摘された \\ \hline
 ヲ格 & 重視する,重要視する,明らかにする,\\
  & 明確にする,取り上げる \\ \hline
 ニ格 & 着目する,重点を置く,注目する \\ \hline
 \end{tabular}
\end{center}
\end{table}
\end{center}

\subsection{強調表示}
\label{subsec:emphasize}

節末の用言が表\ref{significant_words}\,にあげるもので,かつ,指定された格
を持つ場合,重要表現とみなし,次節で説明するスライドの出力の際に,この節
の非主題部の強調表示を行なう.以下の例では,格解析の結果,「ことも」がガ
格と解釈され,「指摘された」がガ格を持つことになるので,スライドの生成の
際に強調表示を行なう.

\ex. 大規模な供給施設が液状化地域に設置されていなかった\Underline{ことも}
指摘された.


\section{スライドの生成}
\label{sec:output_slide}

図\ref{fig:slide_output}\,に示すように,\ref{sec:ds_analysis}\,章で解析した
談話構造に基づいて,\ref{sec:topic_extract}\,章で抽出した主題部,非主題部
を次にあげるルールでスライドに配置する(図において,T,Nはそれぞれ主題部,
非主題部を表す).


\begin{figure*}[t]
 \begin{center}
     \includegraphics[scale=.45]{slide_output2-j.eps}
\caption{談話構造に基づくスライドの生成}
\label{fig:slide_output}
 \end{center} 
\end{figure*}

\begin{itemize}
 \item メタデータなどで,入力テキストにタイトルがある場合,そのままスラ
       イドのタイトルとする.タイトルがない場合は,入力文の最初の主題部
       をスライドのタイトルとする.

\item 文の先頭から順番に,各節において,主題部があれば出力し,インデント
       を一つ下げて次の行に,非主題部を出力する.主題部がなければ,非主
       題部だけを出力する.

       以下の例のように,節に主題が二つある場合は,一つ目の主題部を出力
       し,インデントを一つ下げて次の行に二つ目の主題部を出力し,さらにイ
       ンデントを一つ下げて次の行に非主題部を出力する.

\ex. \Underline{神戸市では},\Underline{当初は}仮設トイレ300基程度で足りると考えていたが, 
\vspace{-3mm}
\begin{center}$\Downarrow$\end{center}
\vspace{-3mm}
\begin{minipage}[t]{0.75\hsize}
\begin{itemize}
 \item 神戸市
   \begin{itemize}
     \item 当初
    \begin{itemize}
      \item 仮設トイレ300基程度で足りると考えていた
    \end{itemize}
\end{itemize}
\end{itemize}
\end{minipage}

\vspace{4mm}
また,\ref{subsec:emphasize}\,節の処理により,非主題部が重要表現とみなされ
       ている場合は,強調表示を行なう.
       
\item 節のインデントのレベルを親との結束関係に応じて以下のように設定する.

  \begin{itemize}
   \item \textbf{初期化:} インデントレベルを0にする.

   \item \textbf{並列/対比:} 同じにする.

   \item \textbf{主題連鎖:} 主題部が親と同じ場合は,インデントを下げずに
	 非主題部だけを出力する.主題部が異なっている場合は,インデント
	 を下げて,主題部と非主題部を出力する.

 \item \textbf{その他:} 親に対してインデントを一つ下げて出力する.

 \end{itemize}
\end{itemize}

出力される箇条書きの行数が閾値以上となる場合,各スライドの行数が閾値以下
になるように分割し,複数のスライドを生成する.

また,多くの研究者が指摘しているように,一般に談話構造の根に近い方が文の
重要度が高いと考えられるので,談話構造は要約生成のための手がかりとなりう
る\cite{Ono94,Marcu99a}.従って,根に近い文から抽出したテキストをスライ
ドに出力し,談話構造木のある深さ以上の文から抽出したテキストはスライドに
出力しないといった処理を行なうことにより,スライドに表示するテキストの量
を制御することができる.しかし,自動生成したスライドを音声合成とともにユー
ザに提示する場合,音声に対応するテキストが全くないとユーザが違和感を感じ
てしまうので,上記の処理を行なわなかった.

\section{実装と評価}
\label{sec:evaluation}


\subsection{text-to-presentationシステムの実装}

\begin{figure*}[t]
 \begin{center}
     \includegraphics[scale=.29]{screenshot.eps}
\caption{システムのスクリーンショット}
\label{fig:screenshot}
 \end{center} 
\end{figure*}


この節では,実装したtext-to-presentationシステムについて説明する.このシ
ステムでは,ユーザは自然言語でクエリを入力すると,クエリに関するプレゼン
テーションを閲覧することができる.本稿では,テキスト集合として,阪神淡路
大震災教訓資料集\footnote{http://www.hanshin-awaji.or.jp/kyoukun/}を用い
た.この資料集はHTMLで書かれており,HTMLタグを手がかりとして400テキスト
に自動分割することによりテキスト集合とした.各テキストにはタイトルが付与
されており,スライドのタイトルとして利用した.テキストの平均文数は3.7,
一文あたりの平均文字数は50であった.

システムはまず,Kiyotaらの手法を用いて,ユーザからのクエリと最も類似した
テキストを検索する\cite{Kiyota02}.その後,書き言葉を話し言葉に変換して
音声合成に入力し,同時に,本稿で述べた手法で生成したスライドをユーザに提
示する.図\ref{fig:screenshot}\,に示すように,本システムはWebブラウザ上で
動作する.テキストから複数のスライドが生成された場合は音声合成と同期して
スライドの表示を切り換える.

\subsection{評価と考察}

「ボランティアの役割」「火災の原因にはどのようなものがありますか」などと
いったユーザからの30クエリから検索されたテキストからスライドを生成し,談
話構造解析と生成されたスライドの評価を行なった.書き言葉からの話し言葉へ
の変換とテキスト検索の評価に関しては,それぞれ\cite{Kaji02,Kaji04},
\cite{Kiyota02}に譲る.入力テキストと自動生成されたスライドのサイズを比
較した平均圧縮率は0.797であった.

\begin{center}
\begin{table}[t]
\small
\caption{談話構造解析の精度}
\label{tab:evaluation}
\begin{center}
 \begin{tabular}[tb]{c|c}
 \hline
  & 精度 \\ \hline
 節間の関係 & 30 / 39 (76.9\%)  \\ \hline
 文間の関係 & 60 / 89 (67.4\%)  \\ \hline
 \end{tabular}
\end{center}
\end{table}
\end{center}

\subsubsection{談話構造解析の評価}

談話構造解析の精度を表\ref{tab:evaluation}\,に示す.評価は,節または文の接
続先と結束関係が正しいかで行なった.談話構造解析の主な誤り原因を以下に示
す.

\paragraph{語連鎖の検出もれ}

以下の例では1文目の「震災」と2文目の「地震」の関係が捉えられず,2文目で
初期化されてしまっている.

\ex. \a. \Underline{震災}直後には,神戸市によって神戸市外語大のホームページに被害写真が掲載され,海外に被害の大きさを知らせた.
 \b. パソコン通信ニフティサーブでは「\Underline{地震}情報コーナー」が開設され,ボランティア情報,安否情報,行政情報など各種の情報提供に用いられた.

この問題には,国語辞典やシソーラスを用いて表現のずれを認識することにより
対処することができると考えられる.

また,本稿で扱ったテキストは書き言葉のため,語が省略されているために語連
鎖が捉えられない例はそれほどなかったが,以下のような例では,「延焼速度」
には「火災の」が省略されており,この関係が捉えられていないため,二文間の
関係を主題連鎖と解析することができず,初期化されてしまった.

\ex. \a. このうち焼損面積10,000平方メートル以上の火災は,特に神戸市長田区などで集中的に発生した.
\b. \Underline{延焼速度は}おおむね20〜40\,m/h程度で,過去の都市大火事例等と比較して極めて遅かった.

この問題には省略・照応解析を行なうことで対処する予定である.

\paragraph{対比関係の検出もれ}

名詞と句/節などが対比の関係にある時に,対比関係を検出できないことがあっ
た.以下の例では,「震災直後に」と「震災から1週間程度を経ると」の対比関
係を検出できなかった.

\ex. \a. \Underline{震災直後に}被災者が必要としたのは,地震の規模や発生場所,被害状況などの被害情報,家族や友人・知人の消息に関する安否情報などだった.\\
 \b. \Underline{震災から1週間程度を経ると},長期的な生活に関わる情報として,住宅やり災証明を始めとする各種申請などの情報も求められた.

また,以下の例では,「当初の」,「時間とともに」の対比関係が検出できなかっ
た.

\ex. ボランティアの\Underline{当初の}役割は,医療,食糧・物資配給,
高齢者等の安否確認,避難所運営等だったが,\Underline{時間とともに},物資
配分,引っ越し・修理,高齢者・障害者のケアなどへと変化していった.\\

この問題には,「時間とともに」や「震災から1週間程度を経ると」が時間に関
する表現であることを認識するためのルールを用意した上で,節/文の類似度を
計算することで対処することができると考えられる.

また,本稿で対象とした地震ドメインのテキストにはシソーラスにない語が含ま
れており,語の類似度を正しく計算できないことがあった.それが原因となり,
文/節の類似度を正しく計算できないことがあった.この問題には地震ドメイン
におけるシソーラスを人手で用意するか,コーパスから自動構築することにより
対処できると考えられる.


\subsubsection{自動スライドの出力例と評価}

次に,自動生成したスライドの評価を行なった.評価の基準は,生成されたスラ
イドがユーザの理解を妨げるものとなっていないかどうかとし,スライドのイン
デント,主題の抽出,文簡約などが適切であるかどうかをもとに評価を行なった.
生成した30スライドについて筆者らが評価したところ,15枚については自然であ
り,12枚は少し不自然なところが含まれており,3 つは全体的に不自然であると
いう結果であった.出力例を図\ref{fig:slide_example1}\,と
図\ref{fig:slide_example2}\,に示す.図\ref{fig:slide_example1}\,の例では,「断
水」,「停電」,「都市ガスの供給停止」の対比関係が正しく解析され,また,
「明かりに不自由しながらの診察・治療が行われ」と「手動の人工呼吸器を押し
続ける姿も見られた」の並列関係で項目を分割することにより,見やすいスライ
ドとなっている.また,2文目の「医療用水のほか」の簡約や,「(治療)が行な
われ」の整形などが行なわれている.

\begin{figure}[t]
\begin{center}
\fbox{
\begin{minipage}[t]{\hsize}
断水により,水の調達に苦慮した医療機関が多かった.断水の影響には,医療用
水のほか,ボイラー用水や,コンプレッサー・自家用発電機等の冷却水が得られ
ないという面もあった.停電により,明かりに不自由しながらの診察・治療が行
われ,手動の人工呼吸器を押し続ける姿も見られた.都市ガスの供給停止により,
入院患者の食事提供に影響があった病院もある.
\end{minipage}
}

\vspace{2mm}
$\Downarrow$
\vspace{2mm}

\begin{minipage}[t]{\hsize}
\begin{shadebox}
\vspace{2mm}
\begin{center}被災地医療機関\end{center}
\begin{itemize}
  \item 断水
  \begin{itemize}
   \item 水の調達に苦慮した医療機関が多かった
   \item 断水の影響
	 \begin{itemize}
        \item ボイラー用水や,コンプレッサー・自家用発電機等の冷却水が得られないという面もあった
	 \end{itemize}
  \end{itemize}
  \item 停電
   \begin{itemize}
    \item 明かりに不自由しながらの診察・治療
    \item 手動の人工呼吸器を押し続ける姿も見られた
   \end{itemize}
  \item 都市ガスの供給停止
   \begin{itemize}
    \item 入院患者の食事提供に影響があった病院もある
   \end{itemize}
\end{itemize}
\vspace{2mm}
\end{shadebox}
\end{minipage}
\caption{出力例1}
 \label{fig:slide_example1}
\end{center}
\end{figure}

また,図\ref{fig:slide_example2}\,の例では,1文目の主題「代替バス利用者」
が抽出され,「当初」,「バスレーン設置後」,「3月末」の対比関係が正しく
解析されている.しかし,2文目では,「代替バス」と「バスレーンの設置後」
が対比していると間違って解析されており,正しくは,「当初」と「バスレーン
の設置後」が対比関係にある.この対比関係を正しく解析できるようにした上で
さらにこのスライドをよくするには,2文目からは「利用時間」という主題を取
り出し,1文目の「代替バス利用者」と対比させるのが望ましいが,これはかな
り難しい処理であるといえる.

\begin{figure}[t]
\begin{center}
\fbox{
\begin{minipage}[t]{\hsize}
代替バス利用者は,当初は1日あたり3〜5万人であったが,バスレーン設置後
は上昇し,3月末までは1日約20万人が利用した.当初,代替バスは交通渋滞
に巻き込まれ,通行に多くの時間を要したが,バスレーンの設置後は約半分の所
要時間に短縮されるなど,徐々に時間は短縮された.
\end{minipage}
}

\vspace{2mm}
$\Downarrow$
\vspace{2mm}

\begin{minipage}[t]{\hsize}
\begin{shadebox}
\vspace{2mm}
\begin{center}鉄道の復旧\end{center}
\begin{itemize}
 \item 代替バス利用者
  \begin{itemize}
   \item 当初
    \begin{itemize}
     \item 1日あたり3〜5万人
    \end{itemize}
   \item バスレーン設置後
    \begin{itemize}
     \item 上昇
    \end{itemize}
   \item 3月末
    \begin{itemize}
     \item 1日約20万人が利用
    \end{itemize}
  \end{itemize}
    \item 代替バス
     \begin{itemize}
      \item 交通渋滞に巻き込まれ,通行に多くの時間を要した
     \end{itemize}
    \item バスレーンの設置後
    \begin{itemize} 
      \item 約半分の所要時間に短縮されるなど,時間は短縮
     \end{itemize}
    \end{itemize}
\vspace{2mm}
\end{shadebox}
\end{minipage}
\caption{出力例2}
 \label{fig:slide_example2}
\end{center}
\end{figure}

自動生成されたスライドで不自然な部分のほとんどは,談話構造解析誤りによる
インデントのずれによるものであった.\ref{sec:output_slide}\,章で説明したス
ライド生成のヒューリスティックルールによる大きな誤りはなく,また,主題部・
非主題部の抽出や非主題部の簡約にも大きな誤りは見受けられなかった.現在の
文簡約は構文解析結果を基に行なう比較的シンプルなモデルであるが,十分機能
しているといえる.しかし,以下のように固有表現にかかる連体節などは削除し
た方がよりよいスライドになると考えられるので,今後は固有表現抽出を行ない,
このような簡約を行なう予定である.

\ex. \Underline{大阪と神戸を結ぶ} \underline{\underline{JR神戸線,阪急電鉄神戸線,阪神電鉄本線}}の3線の不通により,一日45万人,ラッシュ時最大1時間12万人の足が奪われた.

\ex. \Underline{兵庫県下随一の3次救急医療機関である}
\underline{\underline{神戸市立中央市民病院}}は,市街地と島を結ぶ神戸大橋
の不通により震災直後の救急患者の受け入れがあまりできなかった.

たとえ自動スライドにインデントのずれや抽出したテキストが不自然であるといっ
た誤りが少しあったとしても,入力テキストを音声合成と自動スライドのマルチ
モーダルに変換することは,ユーザに入力テキストをそのまま提示するよりもは
るかによいことが実験により示された.特に,テキストに大きな並列や対比関係
がある場合は,入力テキストよりも見やすいスライドを生成できることが確認さ
れた.


\section{関連研究}
\label{sec:related_work}

Utiyamaらは,GDAで意味情報・文章構造がタグ付けされた文書からスライドショー
を生成する手法を提案している\cite{Utiyama99}.GDAタグとは,文書に意味論
的構造や語用論的構造を与えるもので,人手で付与される.まず,共参照を示す
タグから文章構造をボトムアップに決定する.そして,重要なトピックを抽出し,
各トピックに対して関連する文を集め,それらを箇条書きにして一枚のスライド
を生成する.GDAタグを用いることにより,ある程度長い文章についても文章構
造を解析し,スライドを生成することができるが,GDAタグを付与するコストは
大きなものとなる.

安村らは,論文からプレゼンテーション資料の作成支援を行なっている
\cite{Yasumura03j}.まず,論文中の各セクションに対して,使用するスライド
の枚数を割り当て,そして,個々のスライドに対してレイアウトを決定し,論文
中から抽出した文や図表といったオブジェクトを配置している.しかし,この研
究ではTF*IDF法で重要文を抽出しており,文章構造の解析や文簡約は行なわれて
いない.また,入力は\TeX 形式の論文に限られており,本研究のように生テキ
ストからスライドを生成することができない.

次に,個別の処理に関連する研究をあげる.まず,談話構造解析の分野でよく知
られているものとして,Marcuらの研究がある
\cite{Marcu99b,Marcu00,Carlson01}.彼らは談話構造タグ付きのコーパスを作
成し,機械学習の手法を用いることにより談話構造解析を行なっている.彼らの
手法には精度の向上が見られるが,談話構造タグ付きコーパスを作成するにはコ
ストがかかってしまう.これに対して,我々の談話構造解析は一般的なヒューリ
スティックルールに基づいている.我々のシステムの確信度などはもともと比較
的少数の科学技術文章を対象に経験的に定めたものであるが,そのままの設定で
地震ドメインのテキストに対しても,スライドを生成するのに十分な精度を達成
しているといえる.従って,地震ドメインで談話構造タグ付きコーパスを作成し,
機械学習を行なう必要はないと考えている.

また,文末表現の整形で関連するものとして,\cite{Yamamoto05}の研究がある.
この研究では,体言止めや助詞止めといった文末表現に着目し,新聞記事の表現
を,新幹線車内や街頭での電光掲示板で流れるニュースで使われる表現に変換す
る手法を提案している.手法は我々と同じルールベースで,本研究で扱っている
ものよりも多くのパターンを利用しているが,誤り例も報告されており,我々の
扱ったパターンでも十分であると考えている.

Jingらは,自動要約の質を向上させるために,新聞記事とそこから作られた人間
による要約のペアから文簡約の手法を学習している\cite{Jing00}.本研究にお
いても,論文とプレゼンテーションスライドのペアから文の対応関係をとる研究
\cite{Hayama05}を利用して,このアイデアを適用し得ると考えられる.


\section{おわりに}

\label{sec:conclusion}

本稿では,テキストからスライドを自動生成する手法について述べた.スライド
生成は,談話構造解析,主題部と非主題部の抽出と簡約によるスライドに出力す
るテキストの抽出,抽出したテキストの適切配置からなる.地震教訓集を入力テ
キストとして実験を行なったところ,生成されたスライドは入力テキストよりも
かなり見やすいものであることが確認された.また,テキストを入力として自動
プレゼンテーションを行なう,text-to-presentationシステムの実装を行なった.

今後は,談話構造解析,主題の抽出,文簡約などの精度を高めるとともに,実装
したtext-to-presentationシステムに会話エージェントを統合しシステムの質を
向上させる予定である.また,システム全体が自然なプレゼンテーションである
かや,ユーザの理解の向上に貢献するかについては今後,評価実験を行なう予定
である.

\bibliographystyle{jnlpbbl}
\begin{thebibliography}{}

\bibitem[\protect\BCAY{Carlson, Marcu, \BBA\ Okurowski}{Carlson
  et~al.}{2001}]{Carlson01}
Carlson, L., Marcu, D., \BBA\ Okurowski, M.~E. \BBOP 2001\BBCP.
\newblock \BBOQ Building a Discourse-Tagged Corpus in the Framework of
  Rhetorical Structure Theory\BBCQ\
\newblock In {\Bem Proceedings of the 2nd SIGDIAL Workshop on Discourse and
  Dialogue}.

\bibitem[\protect\BCAY{Grosz \BBA\ Sidner}{Grosz \BBA\ Sidner}{1986}]{Grosz86}
Grosz, B.~J.\BBACOMMA\ \BBA\ Sidner, C.~L. \BBOP 1986\BBCP.
\newblock \BBOQ Attention, intentions, and the structure of discourse\BBCQ\
\newblock {\Bem Computational Linguistic}, {\Bbf 12}, \mbox{\BPGS\ 175--204}.

\bibitem[\protect\BCAY{Hayama, Nanba, \BBA\ Kunifuji}{Hayama
  et~al.}{2005}]{Hayama05}
Hayama, T., Nanba, H., \BBA\ Kunifuji, S. \BBOP 2005\BBCP.
\newblock \BBOQ Alignment between a Technical Paper and Presentation Sheets
  Using Hidden Markov Model\BBCQ\
\newblock In {\Bem Proceedings of the 2005 Internatinal Conference on Active
  Media Technology}.

\bibitem[\protect\BCAY{Jing}{Jing}{2000}]{Jing00}
Jing, H. \BBOP 2000\BBCP.
\newblock \BBOQ Sentence Reduction for Automatic Text Summarization\BBCQ\
\newblock In {\Bem Proceedings of the sixth conference on Applied natural
  language processing}, \mbox{\BPGS\ 310--315}.

\bibitem[\protect\BCAY{Kaji, Kawahara, Kurohashi, \BBA\ Sato}{Kaji
  et~al.}{2002}]{Kaji02}
Kaji, N., Kawahara, D., Kurohashi, S., \BBA\ Sato, S. \BBOP 2002\BBCP.
\newblock \BBOQ Verb Paraphrase based on Case Frame Alignment\BBCQ\
\newblock In {\Bem Proceedings of the 40th Annual Meeting of the Association
  for Computational Linguistics}, \mbox{\BPGS\ 215--222}.

\bibitem[\protect\BCAY{Kaji, Okamoto, \BBA\ Kurohashi}{Kaji
  et~al.}{2004}]{Kaji04}
Kaji, N., Okamoto, M., \BBA\ Kurohashi, S. \BBOP 2004\BBCP.
\newblock \BBOQ Paraphrasing Predicates from Written Language to Spoken
  Language using the Web\BBCQ\
\newblock In {\Bem Proceedings of the Human Language Technology Conference},
  \mbox{\BPGS\ 241--248}.

\bibitem[\protect\BCAY{Kiyota, Kurohashi, \BBA\ Kido}{Kiyota
  et~al.}{2002}]{Kiyota02}
Kiyota, Y., Kurohashi, S., \BBA\ Kido, F. \BBOP 2002\BBCP.
\newblock \BBOQ Dialog Navigator: A Question Answering System based on Large
  Text Knowledge Base\BBCQ\
\newblock In {\Bem Proceedings of 19th COLING}, \mbox{\BPGS\ 460--466}.

\bibitem[\protect\BCAY{Kurohashi \BBA\ Nagao}{Kurohashi \BBA\
  Nagao}{1994}]{KNP94}
Kurohashi, S.\BBACOMMA\ \BBA\ Nagao, M. \BBOP 1994\BBCP.
\newblock \BBOQ A syntactic analysis method of long Japanese sentences based on
  the detection of conjunctive structures\BBCQ\
\newblock {\Bem Computational Linguistics}, {\Bbf 20}  (4).

\bibitem[\protect\BCAY{Longacre}{Longacre}{1983}]{Longacre83}
Longacre, R. \BBOP 1983\BBCP.
\newblock {\Bem The Grammar of Discourse}.
\newblock New York: Plenum Press.

\bibitem[\protect\BCAY{Marcu}{Marcu}{1999a}]{Marcu99b}
Marcu, D. \BBOP 1999a\BBCP.
\newblock \BBOQ A decision-based approach to rhetorical parsing\BBCQ\
\newblock In {\Bem Proceedings of the 39th Annual Meeting of the Association
  for Computational Linguistics}, \mbox{\BPGS\ 365--372}.

\bibitem[\protect\BCAY{Marcu}{Marcu}{1999b}]{Marcu99a}
Marcu, D. \BBOP 1999b\BBCP.
\newblock \BBOQ Discourse trees are good indicators of importance in text\BBCQ\
\newblock In I.Mani\BBACOMMA\ \BBA\ M.Maybury\BEDS, {\Bem Advances in Automatic
  Text Summarization}, \mbox{\BPGS\ 123--136}. The MIT Press.

\bibitem[\protect\BCAY{Marcu}{Marcu}{2000}]{Marcu00}
Marcu, D. \BBOP 2000\BBCP.
\newblock \BBOQ The Rhetorical Parsing of Unrestricted Texts: A Surface-Based
  Approach\BBCQ\
\newblock {\Bem Computational Linguistics}, {\Bbf 26}  (3), \mbox{\BPGS\
  395--448}.

\bibitem[\protect\BCAY{NTTコミュニケーション科学研究所}{NTTコミュニケーション
科学研究所}{1997}]{NTT}
NTTコミュニケーション科学研究所 \BBOP 1997\BBCP.
\newblock \JBOQ 日本語語彙大系\JBCQ\
\newblock 岩波書店.

\bibitem[\protect\BCAY{Ono, Sumita, \BBA\ Miike}{Ono et~al.}{1994}]{Ono94}
Ono, K., Sumita, K., \BBA\ Miike, S. \BBOP 1994\BBCP.
\newblock \BBOQ Abstract generation based on rhetorical structure
  extraction\BBCQ\
\newblock In {\Bem Proceedings of the 15th COLING}, \mbox{\BPGS\ 344--348}.

\bibitem[\protect\BCAY{Polanyi}{Polanyi}{1988}]{Polanyi88}
Polanyi, L. \BBOP 1988\BBCP.
\newblock \BBOQ A formal model of the structure of discourse\BBCQ\
\newblock {\Bem Jounnal of Pragmatics}, {\Bbf 12}, \mbox{\BPGS\ 601--638}.

\bibitem[\protect\BCAY{Utiyama \BBA\ Hasida}{Utiyama \BBA\
  Hasida}{1999}]{Utiyama99}
Utiyama, M.\BBACOMMA\ \BBA\ Hasida, K. \BBOP 1999\BBCP.
\newblock \BBOQ Automatic Slide Presentation from Semantically Annotated
  Documents\BBCQ\
\newblock In {\Bem 1999 ACL Workshop on Coreference and Its Applications}.

\bibitem[\protect\BCAY{安村禎明\JBA 武市雅司\JBA 新田克己}{安村禎明\Jetal
  }{2003}]{Yasumura03j}
安村禎明\JBA 武市雅司\JBA 新田克己 \BBOP 2003\BBCP.
\newblock \JBOQ 論文からのプレゼンテーション資料の作成支援\JBCQ\
\newblock \Jem{人工知能学会論文誌}, {\Bbf 18}  (4), \mbox{\BPGS\ 212--220}.

\bibitem[\protect\BCAY{山本和英\JBA 池田諭史\JBA 大橋一輝}{山本和英\Jetal
  }{2005}]{Yamamoto05}
山本和英\JBA 池田諭史\JBA 大橋一輝 \BBOP 2005\BBCP.
\newblock \JBOQ 「新幹線要約」のための文末の整形\JBCQ\
\newblock \Jem{自然言語処理}, {\Bbf 12}  (6), \mbox{\BPGS\ 85--111}.

\bibitem[\protect\BCAY{黒橋\JBA 長尾}{黒橋\JBA 長尾}{1994}]{Kurohashi94j}
黒橋禎夫\JBA 長尾眞 \BBOP 1994\BBCP.
\newblock \JBOQ 表層表現中の情報に基づく文章構造の自動抽出\JBCQ\
\newblock \Jem{自然言語処理}, {\Bbf 1}  (1), \mbox{\BPGS\ 3--20}.

\bibitem[\protect\BCAY{南不二男}{南不二男}{1993}]{Minami93}
南不二男 \BBOP 1993\BBCP.
\newblock \Jem{現代日本語文法の輪郭}.
\newblock 大修館書店.

\end{thebibliography}

\begin{biography}
\biotitle{略歴}
\bioauthor{柴田 知秀}{
2002年東京大学工学部電子情報工学科卒業.2004年東京大学大学院情報理工学系研究科修士課程修了.現在,東京大学大学院情報理工学系研究科博士課程在学中.自然言語処理の研究に従事.
}
\bioauthor{黒橋 禎夫}{
1994年京都大学大学院工学研究科電気工学第二専攻博士課程修了.
博士(工学).2006年4月より京都大学大学院情報学研究科教授.自然言語処
理,知識情報処理の研究に従事.
}

\bioreceived{受付}
\biorevised{再受付}
\bioaccepted{採録}
\end{biography}
\end{document}
