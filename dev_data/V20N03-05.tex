    \documentclass[japanese]{jnlp_1.4}
\usepackage{jnlpbbl_1.3}
\usepackage[dvips]{graphicx}
\usepackage{amsmath}
\usepackage{hangcaption_jnlp}
\usepackage{udline}
\setulminsep{1.2ex}{0.2ex}
\let\underline

\usepackage{array}
\newcommand{\addspan}[1]{}
\newcommand{\delspan}[1]{}
\newcommand{\ulinej}[1]{}

\usepackage{mdwlist}


\Volume{20}
\Number{3}
\Month{June}
\Year{2013}

\received{2012}{12}{14}
\revised{2013}{2}{22}
\accepted{2013}{3}{29}

\setcounter{page}{423}

\jtitle{返信・非公式リツイートに基づくツイート空間の論述構造解析}
\jauthor{大和田裕亮\affiref{tokyo_univ} \and 水野 淳太\affiref{nict} \and 岡崎 直観\affiref{tohoku_univ}\affiref{jst} \and 乾 健太郎\affiref{tohoku_univ} \and 石塚  満\affiref{tokyo_univ}}
\jabstract{
東日本大震災では安否確認や被災者支援のためにTwitterが活躍したが,一方
で多種多様な情報が流通し,混乱を招いた.我々は,情報の信憑性や重要性を
評価するには,ツイート空間の論述的な構造を解析・可視化し,情報の「裏」を
取ることが大切だと考えている.本稿では,ツイートの返信および非公式
\addspan{リツイート}(以下,両者をまとめて返信と略す)に着目し,ツイート間の論述的な関係を認識する
手法を提案する.具体的には,返信ツイートによって,投稿者の「同
意」「反論」「疑問」などの態度が表明されると考え,これらの態度を推定する
分類器を教師有り学習で構築する.評価実験では,返信ツイートで表明
される態度の推定性能を報告する.さらに,本手法が直接的に返信関係の
ないツイート間の論述的な関係の推定にも応用できることを示し,ツイート間の
含意関係認識に基づくアプローチとの比較を行う.
}
\jkeywords{Twitter, 態度推定,教師あり学習,ネットワーク構造,含意関係認識}

\etitle{Analyzing the Statement Structure on Twitter using Replies to Tweets and Quoted Tweets}
\eauthor{Yusuke Owada\affiref{tokyo_univ} \and Junta Mizuno\affiref{nict} \and Naoaki Okazaki\affiref{tohoku_univ}\affiref{jst} \and Kentaro Inui\affiref{tohoku_univ} \and Mitsuru Ishizuka\affiref{tokyo_univ}} 
\eabstract{
Although Twitter played an important role in supporting victims of the 2011 Tohoku earthquake and tsunami disaster, 
we encountered a number of situations in which the vast flow of unauthorized information was problematics. 
To assess the credibility and importance of a piece of information, 
we find that it is important to analyze the statement structure on Twitter and to understand the background of information. 
In this study, we propose a method for analyzing the statement relation between a tweet and its reply or quoted tweet. 
More specifically, we assume that a reply or quoted tweet expresses a statement relation (e.g., {\it agreement}, {\it rebuttal}, {\it question}, {\it other}) toward the target tweet, 
and we build a classifier for predicting a statement relation for a given pair of tweets. 
The experimental results report the performance of the classifier for predicting statement relations. 
In addition, we demonstrate that the proposed method can be applied to analyze statement relations between tweets that have no direct reply/quoting link, 
and we compare the proposed approach with the previous method based on textual entailment.\pagebreak}
\ekeywords{Twitter, Statement Relation, Supervised Learning, Network Structure, Textual Entailment}

\headauthor{大和田,水野,岡崎,乾,石塚}
\headtitle{返信・非公式リツイートに基づくツイート空間の論述構造解析}

\affilabel{tokyo_univ}{東京大学}{The University of Tokyo}
\affilabel{nict}{独立行政法人情報通信研究機構}{National Institute of Information and Communications Technology (NICT)}
\affilabel{tohoku_univ}{東北大学}{Tohoku University}
\affilabel{jst}{科学技術振興機構さきがけ}{Japan Science and Technology Agency (JST)}



\begin{document}
\maketitle


\section{はじめに}

近年,TwitterやFacebookなどのソーシャルメディアが社会において大きな存在感を示している.
特に,Twitterは情報発信の手軽さやリアルタイム性が魅力であり,有名人のニュース,スポーツなどの国際試合の勝利,災害の発生などの速報,アメリカ大統領選挙に代表される選挙活動,アラブの春(2010年,2011年)やイギリスの暴動(2011年)など,社会に大きな影響を与えるメディアになっている.
2011年3月に発生した東日本大震災においても,安否確認や被災者支援のために,ソーシャルメディアが活躍した.

Twitter上ではリアルタイムな情報交換が行われているが,
誤った情報や噂も故意に,あるいは故意ではなくとも広まってしまうことがある.
東日本大震災での有名な例としては,「コスモ石油の火災に伴い有害物質の雨が降る」や「地震で孤立している宮城県花山村に救助が来ず,赤ちゃんや老人が餓死している」などの誤情報の拡散が挙げられる.
このような誤情報の拡散は無用な混乱を招くだけでなく,健康被害や風評被害などの 2 次的な損害をもたらす.
1923年に発生した関東大震災の時も,根拠のない風説や流言が広まったと言われているが,科学技術がこれほど進歩した2011年でも,流言を防げなかった.

このような反省から,Twitter上の情報の\addspan{信憑性}を判断する技術に注目が集まっている.
しかしながら,情報の\addspan{信憑性}をコンピュータが自動的に判断するのは,技術面および実用面において困難が伴う.
コンピュータが情報の\addspan{信憑性}を推定するには,大量の知識を使って自動推論を行う必要があるが,
実用に耐えうる知識獲得や推論手法はまだ確立できていない.
また,情報の\addspan{信憑性}は人間にも分からないことが多い.
例えば,「ひまわりは土壌の放射性セシウムの除去に効果がある」という情報が間違いであることは,震災後に実際にひまわりを植えて実験するまで検証できなかった.
さらに,我々は情報の\addspan{信憑性}と効用のトレードオフを考えて行動決定している.
ある情報の\addspan{信憑性}が低くても,その情報を信じなかったことによるリスクが高ければ,その情報を信じて行動するのは妥当な選択と言える.

そこで,我々はツイートの\addspan{信憑性}を直接判断するのではなく,
そのツイートの情報の「裏」を取るようなツイートを提示することで,情報の価値判断を支援することを考えている.
図\ref{fig:map}に「イソジンを飲めば甲状腺がんを防げる」という内容のツイート(中心)に対する,周囲の反応の例を示した.
このツイートに対して,同意する意見,反対する意見などを提示することで,この情報の根拠や問題点,他人の判断などが明らかになる.
例えば,図\ref{fig:map}左上のツイート「これって本当???」は,中心のツイートに対して疑問を呈しており,
図\ref{fig:map}左下のツイート「これデマです. RT @ttaro:イソジンを飲めば甲状腺がんを防げるよ.」は,中心のツイートに対して反論を行っている.
これらのツイート間の関係情報を用いれば,中心のツイートに対して多くの反論・疑問が寄せられているため,中心のツイートの信憑性は怪しいと判断したり,
右下のツイートのURLの情報を読むことで,追加情報を得ることができる.

\begin{figure}[t]
  \begin{center}
   \includegraphics{20-3ia16f1.eps}
  \end{center}
  \caption{返信・非公式リツイート,もしくは内容に基づくツイート間の論述関係}
  \label{fig:map}
\vspace{-0.5\Cvs}
\end{figure}

Twitterにおいて特徴的なのは,ツイート間に返信\footnote{メールで返信を行うときに,返信元の内容を消去してから返信内容を書く状況に相当する.
Twitterのメタデータ上では,どのツイートに対して返信を行ったのかという情報が残されている.}
や非公式リツイート\footnote{メールで返信を行うときに,返信元の内容を引用したままにしておく状況に相当する.
\addspan{元のツイートをそのままの形でフォロワーに送る
(公式)リツイートとは異なりTwitterが提供している機能ではないが,サードパーティ製のクライアントでサポートされており頻繁に利用されている.}}
などの\addspan{形式を取った投稿が可能な}点である.
例えば,図\ref{fig:map}左上のツイートは中心のツイートに対する発言であること,図\ref{fig:map}左下と右上のツイートは中心のツイートを引用したことが記されている.
これに対し,図\ref{fig:map}右下のツイートは,返信や非公式リツイートの
\addspan{形式を取っていないため,中心のツイートを見て投稿されたものかは不明である.}

本研究では,返信や非公式リツイートの形式を取ったツイート(返信ツイート)に着目し,
ツイート間の論述的な関係を認識する手法を提案する.具体的には,返信ツイートによって,投稿者の「同意」「反論」「疑問」などの態度が表明されると考え,
これらの態度を推定する分類器を教師有り学習で構築する.評価実験では,返信で表明される態度の推定性能を報告する.
さらに,既存の含意関係認識器をこのタスクに適用し,直接的に返信関係のないツイート間の論述的な関係の推定を行い,その実験結果を報告する.



\section{関連研究}\label{sec:Related}

\subsection{デマ分析}

Web情報の信憑性を判断する研究はこれまでにも多く行われてきているが,
近年はTwitterを対象としたものも多い.
Twitter上に流れるデマを見つけ出す研究はその典型である.
Castilloら\cite{Castillo_2011}は,手動で信憑性のラベル付けをしたデータを学習させ,ツイートやアカウントの特徴を用いてTwitter上の情報の信憑性を自動で判別する分類器を訓練し,高い性能を得た.彼らは流行しているトピックをニュースとその他に分類し,ニュースが正しいか間違っているかを分類している.

また,ソーシャルネットワークの関係をグラフ化し,解析する研究も盛んである.
Twitterのグラフ関係を利用したマイニングを行っているものとして,\addspan{Gupta}らの研究\cite{Gupta_2012}が挙げられる.Guptaらは,Castilloらと同様の分類器を用いた手法と,グラフ上の最適化の組み合わせにより,
Twitter上で観測されたイベントの信憑性の推定を行った.彼らは信憑性を,イベントが起こった確からしさとして扱っている.アカウント・ツイート・イベントに分類器の結果に基づく信憑性の初期値を与え,それらをノードとするネットワークを構成し,リンク関係を用いて信憑性の値を更新していくことで,最終的にイベントの真偽を判定する.

これらの研究は,いずれもデマかそうでないかという真偽判定の問題を扱っている.
しかしながら,情報の真偽を判断する事はそもそも難しく,真偽が存在しない情報に対応できないという問題点がある.我々はそのような立場ではなく,価値判断を行いたい情報の周辺を整理することで,真偽情報の受け手側の行動決定を支援することを狙っている.


\subsection{感情分析}

本研究は,返信ツイートによって,投稿者の「同意」「反論」「疑問」などの態度が表明されると仮定しているため,評判分析,感情分析に関する研究とも関わりが深い.
感情分析に関する研究の中で近年特に盛んなのは,極性分類と呼ばれるタスクである.
極性分類では,単語や文を単位としてポジティブもしくはネガティブに分類するが,本研究では,ツイートを単位とした極性分類が必要になる.

Speriosuら\cite{Speriosu_2011}はグラフ上での値の更新を利用したツイートの感情分類を提案している.
グラフを構成するノードとしてはユーザやツイートの他に,単語Nグラムや分類器の学習に使用した感情表現などを加えている.
ツイート及び分類器の学習に用いた素性をシードとして各ラベルの確率重みを初期値とし,それらのラベル(ポジティブとネガティブの 2 値)をグラフを通して拡散させることで,ネットワーク構造を利用しない手法を上回る性能を得ている.
なお,ユーザのフォロー関係も利用しているが,その有効性については示せていない.


\subsection{情報のグルーピングと構造の可視化}

このアプローチの代表的なものとして,WISDOM \cite{Akamine_2009}や言論マップ\cite{水野_2011}が挙げられる.
これらの研究は,あるトピックに関連する言明を整理して表示することで,ユーザの信憑性判断支援を行う.
WISDOMでは,評価極性(良い/悪い)に基づき意見を分類する.
評価極性による意見の分類を行う研究は,Turneyの研究\cite{Turney_2002}以降,他にも数多く存在する.
極性としては良い/悪い,あるいは正の感情/負の感情が用いられる.
元々は製品に対するレビューなどを大雑把に分類するのが目的であったが,近年では,評価対象をより明確にした極性分類が盛んである.
例えば,「Windows7はVistaよりもずっと良い」という文から,Windows7に対する良い/正の感情とVistaに対する悪い/負の感情を区別して抽出したいという要求がある.
Jiangらは,評価対象との構文関係に関するルールを素性に用いた上で,リツイートなどの関連ツイートを利用することにより,評価対象を明確にした極性分類の性能を向上させた\cite{Jiang_2011}.

水野ら~\cite{水野_2011}の言論マップでは,与えられたクエリと検索対象文の間の意味的関係,
すなわち同意するか反対するかに基づいて意見を分類している.
言論マップでは,良い・悪いという絶対的な評価極性で分類するのではなく,
クエリ文と任意の文が与えられたときに,その関係を文間関係認識で推定している.
この文間関係認識は,近年盛んに行われている含意関係認識\cite{Dagan_2005}を対立や根拠などの関係に拡張したもので,根拠関係などのより広範な分類を扱うことにより,言論構造の把握を目指している.

本論文に最も近いものはHassanらの研究\cite{Hassan_2012}である.
彼らは,ディベートサイトにおける各議論トピックへの参加者のグループ分けを行った.
まずユーザ間のポストをpositiveかnegativeかに分類し,
同じ意見を持つ者同士はpositiveなやり取りが多く,違う意見を持つ者同士はnegativeなやり取りが多いという
仮定に基づき,最終的に同じ意見を持つ者同士をまとめたグループを出力した.
これらの先行研究は,いずれも内容に基づくポジネガ分析が主体である.
しかしながら,字数が制限されくだけた文体が多いツイートに対し,これらの技術をそのまま適用するのは容易ではない.
本研究では,Twitter特有のグラフ関係を利用することで,内容ベースの手法だけでは難しい言論の整理を行う.


\section{返信・非公式リツイートの態度分類}

本研究では,返信の態度分類を行うことで,
ツイート間の論述構造を解析する手法を提案する.
返信では,返信先の主張に対する態度や,
ユーザそのものに対する態度が表明される.
これらの態度を分類することで,ツイート内容やユーザ間の関係を用いて
ツイート空間の構造を解析することができる.
本節では返信で表明される態度分類の手法について述べる.

\addspan{
\subsection{本稿における各投稿形式の区分}

本稿における表記の使い分けを明確にするため,Twitterにおけるいくつかの投稿形式を整理する.
\begin{description}
 \item[返信] @で始まる投稿であり,特定のツイートに対する返信として投稿すること.Twitterの提供する機能であり,関係が記録されている. 
 \item[公式リツイート] ある投稿をそのままの形でフォロワーに拡散すること.Twitterの提供する機能である.
 \item[非公式リツイート] ある投稿を必要に応じて編集しつつ,自身のコメントを付加して投稿すること.
 「(自分の投稿)[RQ]T @(返信元のアカウント名)(返信元の投稿)」のような形式を取る.
 Twitterの提供する機能ではないが,使用者が多い.
 なお,自身のコメントを付加していないものについても非公式リツイートとして扱う.
\end{description}
本稿では,「返信」という表記は,特に断りのない限り「返信」と「非公式リツイート」の二つをまとめたものを表す.
分けて扱う必要がある場合においては,前者を「返信」,後者を「引用」と表記する.
また,単に「リツイート」「RT」と表記した場合は「公式リツイート」を指すものとする.
}


\subsection{問題設定}
\label{sec:classification_setting}

返信には様々な意図のツイートが存在する.
相手の発言に同調,あるいは反発するツイートもあれば,
相手や周囲に疑問を投げかけるツイート,
引用して情報を補足するツイートなどもある.
ツイート空間の論述構造を分析するためには,
これらの多様なツイートを,その投稿の意図に従いいくつかのグループに分類する必要がある.
そこで本研究では,返信で表明される態度を,
「同意」「反論」「疑問」「その他」の4クラスに分類するタスクを考える.
\addspan{
それぞれのクラスの定義と,その例を以下に示す.例は,上のツイートに対して,
下のツイートが返信である.
}

\begin{description}
 \item[同意] 主張の支持\addspan{(\ref{ex:agree1})(\ref{ex:agree2})}や感情的な同調(感謝や崇拝\addspan{(\ref{ex:agree3})}も含む)など,返信先のツ
	    イートに対して明確な同意の意図が感じられるもの.
	    \begin{enumerate}
     \item \addspan{…コスモ石油千葉製油所LPGタンクの爆発により、千葉県、近隣圏に在住の方に有害物質が雨などと一緒に飛散するという虚偽のチェーンメールが送られています。千葉県消防地震防災課に確認したところ、そのようなことはないと確認できました。…\\ 
	$\Rightarrow$ 本当ですか 安心です} \label{ex:agree1}
	     \item \addspan{コスモ石油が否定 「火災で有害物質降る」のメール連鎖 http://...\\
	$\Rightarrow$ デマです。みんな冷静になろう。} \label{ex:agree2}
	     \item \addspan{天皇陛下は、宮内庁が京都御所に避難してほしいと要望した
		   のを…お断りになったそうじゃないか。…私は心から尊敬
		   します。\\
	$\Rightarrow$ 天皇陛下!万歳!\(T-T)/}
		   \label{ex:agree3}
	\suspend{enumerate}
 \item[反論] 主張の否定\addspan{(\ref{ex:counter1})}や感情的な反発
	     \addspan{(\ref{ex:counter2})(\ref{ex:counter3})}など,明確な反論の意図が感じられる
	     もの.発言者に対し強く注意を促すようなもの\addspan{(\ref{ex:counter4})}も含める.
	    \resume{enumerate}
	     \item \addspan{…コスモ石油の爆発で有害物質の雨が降る件はデマ。
		   広げてしまった方はツイート削除の上、訂正を/コスモ石油
		   が否定… \\
	$\Rightarrow$ 火災で壊滅した、コスモ
		   石油のコンビナートにあった科学燃料の詳細を記者会見で発
		   表して報道しろ!「危険というのはデマです。」なんて情報
		   で納得するわけないだろ!} \label{ex:counter1}
	     \item \addspan{…東京まで健康被害が現れるようなPuは飛んでこない… RT @YYY: もしプルトニウムが漏れてたら東京から
		   逃げますか?…\\
	$\Rightarrow$(怒)}
		   \label{ex:counter2}
	     \item \addspan{…日本政府は事故の重大性をまったく認識していない。
		   今すぐに多国籍軍を総動員して封じ込めないとチェルノブイ
		   リ以上の被害が出る\\
	$\Rightarrow$ 馬鹿左翼、煽るな。}
		   \label{ex:counter3}
             \item \addspan{…食料がありません。脱水、低血糖が徐々にきています
                   。デマじゃない…見捨てないで下さい【from茨城県鹿島コン
                   ビナート地区】\\
	$\Rightarrow$ せめて公式RTしてください
                   。元の発信者の名前消すとか非公式RTよりひどい。救助を必
                   要としてる人が、誰だかわからなくなると思いませんか?そ
                   のせいで救助が遅れたらと想像できませんか?}
                   \label{ex:counter4}
	    \suspend{enumerate}
 \item[疑問] 返信先に対して情報を要求している\addspan{(\ref{ex:question1})}が,明確な反論とは言えないも
	    の.情報源を要求するようなもの\addspan{(\ref{ex:question2})}や,引用部に対する疑問の吐露な
	    ども含まれる.
	    \resume{enumerate}
	     \item \addspan{…コスモ石油の爆発で有害物質の雨が降る件はデマ。
		   広げてしまった方はツイート削除の上、訂正を/コスモ石油
		   が否定… \\
	$\Rightarrow$ JFEケミカル等含めた火災
		   による被害が無いと言うことでよろしいですか?}
		   \label{ex:question1}
	     \item \addspan{…コスモ石油の爆発により有害物質が雲などに付着し、
		   雨などといっしょに降るので…コピペとかして皆さんに知
		   らせてください!! \\
	$\Rightarrow$ NHK のニュースで
		   は今のところ有毒物質が発生することはないと言っています
		   が、あなたのツイートのソースは何ですか?}
		   \label{ex:question2}
	    \suspend{enumerate}
 \item[その他] 上記のどれにも分類できないもの.
	    \resume{enumerate}
	     \item \addspan{千葉県、近隣圏に在住の方に有害物質が雨などと一緒に飛散
		   するという虚偽のチェーンメールが送られています。…\\
	$\Rightarrow$ そうであったとしても、雨カッパとかは持ってた方が良いよね。} \label{ex:other1}
	     \item \addspan{…千葉県、近隣圏に在住の方に有害物質が雨などと一
		   緒に飛散するという虚偽のチェーンメールが \\
		   $\Rightarrow$ 硫黄分の多い原油が燃えると酸性雨につなが
		   る可能性があるので、それに尾ひれはひれがついたものと推
		   測します。} \label{ex:other2}
	    \end{enumerate}
\end{description}

これらの4クラスを設計した意図について\addspan{,上記の例を参照しつつ}述べる.
ツイート空間の整理に必須なのが,対立構造の抽出である.
そのために「同意」と「反論」の2クラスを設定する.
「同意」のクラスによって結びつけられたツイート群は,
何らかの主張やユーザに対して同様の態度を表明し,
「反論」のクラスによって結びつけられたツイート群は,
何らかの主張やユーザに対して異なる態度を表明する.
この2クラスを設定することで,ツイート空間を,
同じ態度を表明するツイートクラスタの集合として整理できる.
なお,「同意」や「反論」については態度が明確なもののみを分類する.
前述したように,これら2クラスがツイート間の論述構造を解析する上で中心的な役割を果たすため,曖昧な態度のツイートをこれらの2クラスに分類しないように注意する必要がある.
\addspan{
(\ref{ex:agree1})(\ref{ex:agree2})(\ref{ex:counter1})(\ref{ex:other1})(\ref{ex:other2})は,
「コスモ石油の爆発で有害物質の雨が降るというのはデマ」という主旨のツイートに対する返信である.
(\ref{ex:agree1})は主旨を受け入れているので「同意」,(\ref{ex:counter1})
は納得しないことを主張しているので「反論」に分類する.
(\ref{ex:agree2})は「デマです」と返信しているのでやや紛らわしい例である
が,「冷静になろう」という記述から「危険はない」と考えていることが読み取れ,
返信先のツイートの内容に「同意」していると判断できる.
(\ref{ex:other1})は,主旨を受け入れてはいるが,完全に安心はしていないこ
とが読み取れることから,同意とは言えず,「その他」に分類する.
また,(\ref{ex:other2})は主旨を受け入れてはいるが,返信先の情報への同意
よりも自分の考えを表明するためのツイートと読み取れる.このように補足の意図を持つツイートも「その他」に分類する.
}
\addspan{
(\ref{ex:counter4})は発言者に対し強く注意を促すものの例である.
この例のように不用意な拡散をたしなめるものは,拡散しようとするユーザへの反論としてとらえられる.
}

「同意」と「反論」の2クラスは,著者間の対立関係を表すのに対して,「疑問」
と「その他」の2クラスは,著者間に対立構造は存在しない.従って,ツイート空
間において重要なツイートとしてツイートAがあるとき,それに同意または反論す
るツイートは重要な存在であるが,その他の関係にあるツイートは,論述構造を
明らかにする上で重要ではない.一方で疑問関係は,論述構造を整理する上で重
要である.情報の信憑性を判断する際に,その情報に疑問を持つユーザの存在は,
信憑性が低いことを示唆していると考えられるためである.さらに,その疑問に
対する回答は,有益な情報となる場合が多いと考えられる.
(\ref{ex:question1})は純粋な疑問,すなわち質問の例である.この質問に対す
る回答があれば,被害状況を詳しく把握できる可能性がある.次に,
(\ref{ex:question2})は情報源を求めている例である.これは,懐疑的な返信で
はあるが,明確な反論とはいえないため,疑問に分類する.



\addspan{
(\ref{ex:agree2})(\ref{ex:counter2})(\ref{ex:counter3})のような感情的な同調あるいは反発のツイートは,内容(引用の場合は付加部分)には有用な情報は含まれていないが,
}
誰が誰に同調,あるいは反発したかというユーザ同士の関係を推測することは,論述構造分析の助けになると考えられる.
したがって本研究においては,感情に基づくと考えられるツイートについても,「同意」や「反論」として同等に扱い分類を行なっている.


\subsection{返信ツイートのアノテーション}

返信ツイートの分類を行うにあたり,正解データを準備する.
本稿で行う実験で使用するデータは,ホットリンク社より提供された,
2011年の\addspan{3月11日から3月29日まで}
のツイートデータ(以下hottoコーパス)\footnote{http://www.hottolink.co.jp/press/936}である\footnote{東日本大震災ワークショップで提供されたツイートデータを利用したかったが,返信の情報が含まれていなかったため,断念した.}.
hottoコーパスには,\#tsunamiや\#jishinなど震災に関連するハッシュタグまたはキーワードが含まれるツイートと,
そのツイートを投稿したユーザやそのプロフィール情報などが収録されている.
収集対象ユーザ数は約100万人,ツイート数は約2億1千万ツイート\addspan{であり,本
研究では全てのデータを利用した.}
また,各ツイートには公式リツイート(以下RT)や返信の関係が含まれており,これらはそのまま利用した.

\begin{figure}[b]
  \begin{center}
   \includegraphics{20-3ia16f2.eps}
  \end{center}
  \caption{非公式リツイートのリンク情報の復元}
  \label{fig:recognizing-qts}
\end{figure}

ただし,hottoコーパスには引用関係が記録されていないので,以下の方法で復元した(図\ref{fig:recognizing-qts}参照).
ツイートに``\texttt{[RQ]T username}''というパターンが出てくるとき,そのパターンに続く部分(図\ref{fig:recognizing-qts}左上で,
「現在、」から始まる網掛けの部分)をどこかのツイートから引用したと考える.
そして,ユーザ``\texttt{username}''のツイートの中で,投稿日時が引用ツイートよりも前のものをすべて検索する.
これらのツイート群に対し,引用部のテキストとの類似度を\addspan{文字トライグラム}のオーバーラップ係数で計算し,これが最大になるものを引用元と推定する.
\addspan{
オーバーラップ係数は,文字トライグラムの集合XとYに対し,以下の式で定義される.
{
\[
オーバーラップ係数 = \frac{|X \cap Y|}{\min \{|X|, |Y|\}}
\]
}}
なお,`\texttt{[RQ]T username1 comment1 [RQ]T username2 comment2} ''のように引用が何段階が行われているケースの場合,
引用元と考え検索対象とするのは``\texttt{username1}''のツイートのみである.
図\ref{fig:recognizing-qts}では,\addspan{あるユーザ}のツイートを検索した結果,右側真ん中のツイートの類似度が最も高かったため,左上のツイートの引用元と推定している.
ただし,オーバーラップ係数の最大値が0.4未満である場合は引用元が見つからなかったとする.

返信関係のアノテーションには,震災の際に出回ったデマ一覧\footnote{「東北関東大震災に関するデマまとめ」のまとめ http://togetter.com/li/112556}
を参考に,20個のトピックを選択した.
\addspan{トピックは,デマの内容を端的に説明する短い文である.}
デマ一覧を利用したのは,東日本大震災の時に情報の信憑性で問題に上がったことと,反論関係が比較的多い割合で含まれると考えたからである.
\addspan{それぞれのトピックに関連するツイートを収集するために,クエリとしてトピック中の単語を設定する.}
\addspan{選んだトピックとクエリを表\ref{tab:20query}に示す.}
\addspan{これらのトピックは,クエリで検索できるツイートから,十分な数の返信を抽出できるように選択した.}
具体的には,後で述べるように,返信・引用が各100個ずつ以上取得できるような\addspan{トピックと}クエリを選んだ.
各クエリ\addspan{で}hottoコーパスを検索し,リツイート数が多い順に並べる.
上位のツイートから順に,返信ツイートを取得する.
なお,あるユーザ間で相互の返信が続いている場合には,反論の応酬であるなど重要な論述構造が存在する可能性があり,有用なデータとなるため,
返信ツイートが続いている限り全て取得している.
このようにして,各クエリを用い返信・引用をそれぞれ100個ずつ,20トピックで計\addspan{4000}ツイートを集める.
\addspan{この中に}重複するツイートがあった場合には一つを残して削除し,トピック毎の返信ツイートを補充した.
以上により,計4000個の返信ツイート(以降データセットA)を準備した.

\begin{table}[t]
\caption{選出した20個のトピックと,検索に用いたクエリ}
\label{tab:20query}
\input{16table01.txt}
\vspace{0.5zw}\small
  ※表中に示されているトピックは,全て虚偽と確認されており事実ではない.\par
\end{table}

データセットAに対し,三人のアノテーター(以下X・Y・Z)の手により,
前項で述べた方針に従い4クラスのラベル付けを行った.
アノテーションの一致度合いを表\ref{tab:class_annotation}に示す.
4000ツイートのうち,3人のラベルが一致したツイートは2690ツイートであった.
ラベルの一致度を評価するため,ペアワイズにCohenのカッパ係数を計算した.
\[
\frac{p_0 - p_c}{1 - p_c}
\]
ただし $p_0$ は対応クラスの出現数の一致率, $p_c$ は対応クラスの偶然の一致率である.
XY間が0.660,XZ間が0.621,YZ間が0.641となり,十分な一致と見なせる.
3人のアノテーションが一致した2690ツイートを集めたものをデータセットBとする.

\begin{table}[t]
\caption{データセットAにおけるアノテーションの一致度}
\label{tab:class_annotation}
\input{16table02.txt}
\end{table}


\subsection{分類手法}
\label{subsec:class_method}

返信の態度を教師有り学習で分類する手法について述べる.
学習に用いる素性は大まかに 3 種類に分かれ,ツイート内容に関す\addspan{る}素性,ツイート間の素性,ユーザ間の素性である.

ツイート内容の素性は,単語ユニグラム,単語バイグラム,URL数,ハッシュタグ
数,デマ否定単語の有無,反論表現との一致度である.これらの素性では,返信ツイートで
あればツイート全文,引用ツイートであれば追加部分(ツイート全体から引用箇
所以降を除いたもの)を対象とし,本文中のアカウント名やURL,ハッシュタグを
正規表現により取り除く.取り除いたアカウント・URL・ハッシュタグに対しては,
それぞれに対し0個・1個・2個以上を含むという3値の素性を作る.この設定は,
\addspan{実験に用いたツイートデータを観察した結果,}
URLを1個も含まない場合と1個以上含む場合には大きな差があると見られ,また,1個
だけ含む場合と多数を含む場合にも大きな差があると見られる一方で,3個か4
個かには大きな差はない\addspan{と見られたため導入した.}
不要箇所を取り除いた後の本文に対して,
MeCab \cite{Kudo04applyingconditional}\footnote{http://mecab.googlecode.com/svn/trunk/mecab/doc/index.html}
による形態素解析を行い,単語ユニグラムと単語バイグラムを抽出する.

\addspan{デマ否定単語の有無}とは,ツイート本文が\addspan{デマの否定を示す単語}を含むかどうかを表すもので,\addspan{デマ否定単語}は「デマ」「ガセ」「誤報」「虚報」「削除」「訂正」の6つとした.
これらの単語は必ずしもデマを否定していることを示すとは限らないが,
返信で用いられる場合では,デマの否定に用いられるケースが多く,
反論の抽出に有効と考えられる.

反論表現との一致度とは,あらかじめ作成した反論表現辞書を用い,
反論に固有の表現をどの程度含んでいるかを表す.
反論表現辞書は,データセットAの中から反論を表すと思われる表現を人手で抽出して作成する.
それらの表現にはデマを否定する表現も含まれるが,
相手に対する感情的な反発を表す表現なども含まれる.
表\ref{tab:反論表現}に反論表現の例を示す.

\begin{table}[b]
\vspace{-0.5\Cvs}
\caption{反論表現の例}
\label{tab:反論表現}
\input{16table03.txt}
\vspace{-0.5\Cvs}
\end{table}

なお,この二種類の表現は厳密に区別できるわけではない.
例えば,表\ref{tab:反論表現}のその他の表現例の「余計なこと書くな」については,
「余計なこと」が指す内容がデマの可能性もある.
実際にはこれら二種類を分けて扱う必要はないが,反論表現をたくさん集めることが,
デマによる否定以外の反論を認識する上で有効である.
また,一見すると反論を表す表現であるが,反論表現辞書に含めるべきでないものも存在する.
例えば,「それはデマです」は反論表現として挙げた一方で,
「デマです」「デマらしいです」は反論表現にはふさわしくないことがある.
例えば,以下のようなケースが存在する.
\begin{quote}
デマです,注意! RT @XXX: (デマを否定する情報) RT @YYY: (デマ情報) \\
デマらしいです RT @XXX: (デマを否定する情報) RT @YYY: (デマ情報)
\end{quote}
この場合,引用元との間には共にデマを否定しているという“同意”の関係が成り立っているため,
反論表現としてはふさわしくない.
このように,伝聞形で使われそうな表現などは反論表現辞書から除外している.
実際に収集した100個の反論表現は付録に載せた.

以上のように作成した反論表現辞書を用い,反論表現との一致度を求め,素性として使用する.
まず,反論表現辞書中の各反論表現に対して,その単語バイグラムをどの程度含むかを求める.
例えば,反論表現辞書が「それ/は/デマ/です」「これ/デマ/ね」の二つからなる場合を考える.
「デマ/です」を含むツイートがあった場合,「それ/は/デマ/です」の3個のバイグラムのうち1個を含むので1/3,
「これ/デマ/ね」の2個のバイグラムは1個も含まないので0となる.
これらのうちの最大値(この場合は1/3)を,反論表現との一致度とする.
実際には反論表現は100個あるので,それら100個に対して計算を行い,最大値を反論表現との一致度としている.

返信先のツイートとの間\addspan{の関係についての}素性
では,ツイートのタイプ,空文,単語バイグラムのコサイン類似度を使用する.
ツイートのタイプとは,引用かどうかを表す.
空文とは,引用でありながらも自身のコメントを付加していないものに発火する素性である.
単語バイグラムのコサイン類似度とは,
返信ツイートの単語バイグラム(引用ツイートの場合は,ツイート全体から引用箇所を除いた部分)と,
返信先ツイートの単語バイグラムのコサイン類似度である.

返信先のユーザとの間\addspan{の関係についての}素性
として,
返信の回数,返信の方向性,RTの回数,RTの方向性,
共通のツイートをRTした回数,共通のURLをツイートに含んだ回数,
共通のツイートに返信した回数を利用する.
返信の回数は,hottoコーパスの中から,返信している 2 ユーザ間で
返信が行われた回数である.
方向性とは,2 ユーザ間の返信の履歴を調べ,
双方向のやりとりがある,返信相手への返信のみがある,
返信がない,
のいずれかを指す.
RTの回数や方向性についても同様である.
これらの素性はユーザ間の関係から
返信による態度を推測するために使われる.
例えば,RTが相互でたくさん行われている場合,
新たな発言も「同意」である可能性が高い\addspan{と考えられる}.
返信が一方的なものであった場合,
それらの発言は「反論」である可能性がある\addspan{と考えられる}.
また,共通のツイートをRTした回数,共通のURLをツイートに含んだ回数,
共通のツイートに返信した回数の3つは,第三者を媒介して表れる素性である.
例えば,よく共通のツイートをRTしているユーザ同士での返信があった場合は,「同意」の可能性が高い\addspan{と考えられる}.
\addspan{これらの素性の設定における仮定の妥当性は,\ref{subsec:cross}節で検証する.}

\addspan{
以上,各素性の設計について述べたが,
本研究において特に重要と考えているのは,反論表現辞書との一致度および構造的特徴の利用である.
ここでいう構造的特徴とは,返信か引用かや,返信先のユーザとの間の関係についての各素性であり,これらはTwitter上の返信構造により生じる素性である.
また,反論表現辞書は,前述したように返信で用いられることの多い表現を元に作成しており,これらもまたTwitter上の返信構造により生じる素性と言える.
次節の評価では,返信構造により生じるこれらの素性がどの程度性能に寄与するのかについても示す.
}



\section{返信・非公式リツイートの態度分類の評価}

本節では,前節で述べた返信の態度を分類する手法について性能を評価する.


\subsection{交差検定による性能評価}
\label{subsec:cross}

前節でアノテーション済のデータセットBに対し,前述した4クラスに識別する多クラス分類を行う.
本実験では,分類器として最大エントロピーモデルを用いる.
分類器の実装として,Classias\footnote{Classiasのホームページ: http://www.chokkan.org/software/classias/}のpegasos.logistic(L2正則化ロジスティック回帰)を使用した.
実験では,2トピックずつ計10個のデータに分割し,10分割交差検定を行った.
表\ref{tab:10cross2}に,クラス毎の\addspan{精度}・再現率・\addspan{$F_1$値}と,全体の\addspan{正答率}を10分割のマイクロ平均\addspan{で}示した.
\addspan{
なお,$F_1$値については全クラスのMacro平均も示した.
}
\addspan{
$F_1$値の定義は以下の通りである.
{
\[
F_1 = \frac{2*精度*再現率}{精度*再現率}
\]
}}

\begin{table}[b]
\caption{返信ツイートの分類結果(データセットB)}
\label{tab:10cross2}
\input{16table04.txt}
\end{table}

各クラス毎の性能について見ると,
「同意」と「疑問」に対してある程度高い分類性能が得られた一方で,
「反論」と「その他」の分類性能は低くなっている.
「その他」については,他の3クラスと異なりツイートの意図が広範囲にわたるため,
共通の素性を得にくくなっているためと考えられる.
素性のクラスに対する特定性の指標として,以下の数式を導入する.
素性$f$におけるクラス$c_0$の特定性は,各クラス$c$の学習結果の重み$w_{fc}$に対して
\begin{equation}
c_0の特定性 = \frac{exp(w_{f_{c_0}})}{exp(\sum_{c \neq c_0} w_{f_c})}\\
\end{equation}
で表す.\addspan{この時,特定性は0.333を上回るほどそのクラスに有効な素性であることを表す.}
データセットBに含まれる全てのデータで学習した場合に,
「その他」を除く各クラス毎に特定性が上位の素性5個ずつを表\ref{fig:features}に示す.


\begin{table}[t]
\caption{各クラスで特定性の高い素性}
\label{fig:features}
\input{16table05.txt}
\end{table}


「同意」と「疑問」の2クラスについては,
それぞれを表すような特徴が比較的うまく抽出されていることが分かる.
「同意」については,「!」・URLが0個である・感謝を表す表現などが有効な素性である.
\addspan{これらは,Nグラムを直接利用する程度の言語処理で組み込める素性であり,
同意を表す素性としても直感に合うものになっている.}
また,バイグラムのコサイン類似度の素性も上位に来ていることから,元のツイートと同じような内容を
繰り返す場合にも同意である可能性が高いと考えられる.
\addspan{なお,素性の導入の際においた「RTの回数が多ければ同意」「共通のツイートをRTした回数が多ければ同意」
などの仮定は,素性の表す同意の特定性がそれぞれ0.335・0.336であり,有効ではなかった.}
「疑問」については,疑問符や問いかけの表現が上位に来ている.
これらもまたNグラムの利用程度で組み込める素性であり,疑問を表す素性としても直感に合う.
一方,「反論」については,反論表現との一致度が有用であるが,
それらを除くとあまり反論に特有の単語とは思えない特徴が多く並んでおり,
反論を示唆する表現は,Nグラムなどの単純な抽出法ではなかなか見つからないことが分かる.
反論の認識に比較的有効であった素性としては,
「返信相手への返信のみがある」,すなわち返信が一方通行であるという特徴\addspan{(特定性0.437)}が挙げられる.
なお,反論に関してもう少し特定性の低いものを見ていくと,「お前」などの表現が見つかる.
しかしながら,このような蔑称の類が登場する事例はさほど多くなかったので,反論に関する学習が十分に行えなかったと考えられる.


\subsection{各素性の有効性}

本手法において特徴的なのは,
反論に特有な表現を持つ事例を分類するための反論表現辞書と,
そのような表現を持たない事例を分類するための構造的特徴を用いたことである.
そこで,これらの有効性を評価するべく,以下の比較を行った.
単語ユニグラム・単語バイグラム・元ツイートとのコサイン類似度・
URL数,ハッシュタグ数のみを特徴として使用する場合をベースライン (Base) とする.
ベースラインにデマ表現と反論表現辞書との一致度を加えた場合を反論表現あり (+Con-Exp),
構造的特徴(返信か引用か・共通のRTやURL引用回数・相互のRTや返信回数・相互のRTや返信の方向性)を
加えた場合を構造的特徴あり (+Structure),
反論表現辞書と構造的特徴を加えた場合を全使用 (+All) とする.
これらの 4 つの場合において,それぞれ分類器を構築し,性能を比較したのが図\ref{fig:使用する素性による性能変化}である.
本研究では,「反論」の認識が重要であると考え,
マクロ\addspan{$F_1$値},「反論」クラスの\addspan{$F_1$値},\addspan{正答率}の 3 指標で性能を比較した.

\begin{figure}[b]
  \begin{center}
   \includegraphics{20-3ia16f3.eps}
  \end{center}
  \caption{使用する素性による性能変化}
  \label{fig:使用する素性による性能変化}
\end{figure}

\addspan{まず,Baseに対する+Con-Exp,および+Structureに対する+Allの性能から,}
反論表現辞書は反論クラスの識別を中心として大きく寄与することが分かる.
反論表現辞書を充実させることで,分類器の性能をさらに改善することができると考えられる.
\addspan{次に,Baseに対する+Structure,および+Con-Expに対する+Allの性能について,
それぞれ「反論」$F_1$値が向上していることから,}
構造的特徴は反論クラスの識別性能を多少上げるのに貢献していると言える.
\addspan{一方で他クラスの識別の失敗例が増えることにより全体の正答率はやや低下しており,}今後の検討が必要である.
例えば,返信の回数・方向性については,分類対象のツイートに関連するように
トピックや時系列を限定することで,より正確にツイート間の関係に繋がる特徴となる可能性がある.


\subsection{考察}

本研究の目的であるツイート間の論述構造解析では,「反論」を高精度で識別することが重要である.
そこで,反論の識別に失敗した事例を調査・分析した.
ツイートで反論を行うパターンは,大きく\addspan{3}種類に分けられる.
1 つ目は,発言者への反論であり,2 つ目は返信先の内容への反論である.
\addspan{そして 3 つ目は,発言者と返信先の内容の両方に対する,いわば複合的な反論である.
これら 3 種類の反論の例を以下に示す.}

\begin{enumerate}
 \item @XXXXXX 偉い学者さんなら、もっともらしい事言ってないで国に話し通すなり、
 非難勧告するなり、どうにかしろよ。なんにも出来ないなら不用意に被災者の不安煽る様な事言うな。 \textbf{発言者への反論}
 \item 私が言うのも変ですが水で冷やしている限りメルトダウンはしません。
 問題が水が送れるか否かです。 RT @XXXXXX 水で冷やして「炉心溶融・メルトダウン」を停止できるという原子炉の専門家はいない。 \textbf{内容への反論}
 \addspan{
 \item @XXXXXX 防護ケース(格納容器のことか?)が万が一破裂しても圧力容器があるので、即炉心が外界に露出ということではない。誤解を生むRTは控えて、RT元は吟味していただきたい。 \textbf{複合的な反論}
 }
\end{enumerate}

\addspan{
データセットBに含まれる263個の反論のうち,発言者への反論が81個,内容への反論が137個,複合的な反論が45個であった.
それぞれの反論についてどの程度識別することができたのかを評価するため,
図\ref{fig:使用する素性による性能変化}における各特徴を使用した時毎の再現率を図\ref{fig:反論種類毎の性能}に示す.
}

\begin{figure}[b]
  \begin{center}
   \includegraphics{20-3ia16f4.eps}
  \end{center}
  \caption{反論の種類毎の再現率}
  \label{fig:反論種類毎の性能}
\end{figure}

\addspan{発言者への反論については,}返信先の内容に依らない表現が使われることが多いため,
反論表現の抽出が有効である.
\addspan{上記(1)の例では,}「不安煽る様な事言うな」という文は内容の影響を受けずに,反論を表すために広く用いられる.
本研究では,集めた返信ツイートの中からこのような表現を集め,反論表現辞書とした.
反論表現を持つという特徴を使用することで,分類の精度を上げることができる.
\addspan{このことは,図\ref{fig:反論種類毎の性能}において,反論表現を使用した際に再現率が大きく上昇していることから分かる.}

\addspan{次に,内容への反論については,}返信先の内容によりツイートに含まれる表現は千差万別であり,
内容を理解しなければうまく分類できない.
この例では,返信先が「水で冷やしてもメルトダウンを防げない」
という内容であることを理解し,「水で冷やしている限りメルトダウンしない」という主張が反論関係にあることを
認識する必要があるが,このような事例を提案手法で分類することは難しい.
このような問題を解くためのアプローチとしては,
返信先の主張の対象が「水で冷やしてもメルトダウンを防げない」であることを認識し,
それに対し,返信の「水で冷やしている限りメルトダウンしない」という主張が対立関係にあることを認識しなければならない.
これは,対象依存の感情分析\cite{Jiang_2011}や言論マップ\cite{水野_2011}の矛盾認識で取り扱う事項である.
本稿ではそのようなアプローチを取らずに,Twitter上のネットワークにおける構造的特徴を用いて態度の分類を行っている.
\addspan{
図\ref{fig:反論種類毎の性能}からは,構造的特徴の利用により,内容への反論の識別数が若干向上していることが読みとれるが,
十分な改善とは言えずさらなる検討が必要である.
}

\addspan{
最後に,複合的な反論については,基本的な特徴のみで比較的よく識別できており,
反論表現の利用によりさらに性能が向上している.
これは,複合的な反論の中でも発言者への反論部分から特有の表現をとらえているためである.
基本的な特徴のみでよく識別できている理由としては,複合的な反論はある程度の長さがあり,
また定型的な文章で書かれていることが多く,それらに共通する有効な素性が抽出されやすいためと考えられる.
}


\section{一般的なツイート間関係認識への拡張}

本節では,\addspan{返信に限らない全てのツイート間の関係認識を行うことを考える.
その際,前節までで述べた}返信の態度分類を,直接的に返信関係のない一般のツイー
ト間の論述関係分析に応用する手法について述べる.一般に,返信のツイー
トが全ツイートに占める割合は非常に少ないため,前節までで作成した分類器を
そのまま適用するのは難しい.特に,反論表現辞書の有効性は低いと考えられる.
反論表現辞書に含まれる表現は,返信先への態度を表す際に利用される
ことの多い表現であり,返信になっていない一般のツイートに含まれる可
能性は低い.

\addspan{
返信の関係にない一般のツイート間に対して4クラスの分類を行うには,その言語
的な内容を利用することが考えられる.例えば,ツイート間の反論関係は,含意
関係認識課題~\cite{Dagan_2005}における文間の矛盾関係
 (RTE3~\cite{GiampiccoloRTE32007}) に相当する.文間の矛盾関係の認識に取り
組んだMarneffeら~\cite{MarneffeIdentifying2011}は,RTE3のテストデータに対
して精度22.95\%,再現率19.44\%を達成した.また,日本語を対象とした場合,
NTCIR9-RITE~\cite{ShimaRITE12011}およびNTCIR10-RITE2において,矛盾関係を
含む含意認識課題であるMC (multi class)タスクが取り扱われており,RITE2の
フォーマルランにおける矛盾関係認識性能は,精度52.17\%,再現率19.67\%が最
高性能であった\footnote{http://www.cl.ecei.tohoku.ac.jp/rite2/doku.php?id=wiki:results}
.これらの認識性能が示す通り,言語的な内容に基づく文間の矛盾関係の認識は
容易ではない.
}


そこで本節では,\addspan{返信態度の分類器}をツイート間の関係分析に拡張する手法につ
いて述べる.具体的には,ツイートペアが与えられた時に,両ツイートおよび両
ユーザ,および関連するツイート・ユーザに拡張したネットワークを作り,
ネットワーク内に存在する返信関係を分類器で解く.その結果から,元のツイー
トペアの間の関係を推測する.さらに,\addspan{後述する文間関係認識}を用いて,ツイー
ト間の関係を推測する手法についても述べ,それぞれの分類性能や差異について
考察する.


\subsection{問題設定}

一般のツイートの内容や意図は広範囲に渡る.意見の主張に始まり,ニュースな
どの拡散,さらに他愛もないつぶやきや独り言の類も多い.これら全てのツイー
トを整理することはできない.したがって,整理が完了した際の有用性を考慮し
維持しながら範囲を狭めることが必要である.そこで,
\addspan{\ref{sec:classification_setting}節で定義した,トピックに関連する
ツイート群を収拾し,}それらを整理するタスクを考える.\addspan{同じトピッ
クに属するツイート間には,そのトピックに対する同意,反論などの観点から関
係を付与できると考えられる.各ツイートがあるトピックに属するか否かを決定
する手法については,本稿の手法の対象外であり,人手で行う.}

本節では,同じ\addspan{トピックに属する}ツイートのペアが与えられた場合に,関係を分
類するという問題を解く.\addspan{分類する関係は},返信の関係分類の際と同じく,「同
意」,「反論」,「疑問」,「その他」の四つとする.

\begin{description}
 \item[同意] 同様の主張や感情を示すなど,明確な同意の意図が感じられるもの.
 \item[反論] 対立する主張や感情を示すなど,明確な反論の意図が感じられるもの.
 \item[疑問] 情報・情報源の要求や,疑問の吐露など.
 \item[その他] 上記のどれにも分類できないもの.
\end{description}

これらの分類基準は,返信の場合と似ているが全く同じにはならない.
前述したように,ツイート間の関係は,
トピックに関する\addspan{主張間の関係}としてとらえられる.
返信の関係が存在しないツイートペアにおいては,
お互いに対する単純な賛成や反対,感情的な同調や反発などは表れにくい.
また,お互いに対する直接的な質問もまず存在しないため,
「疑問」はかなり出現頻度が少ないと予想されるが,
存在した場合には重要であると考え,\addspan{分類対象の関係に含める}.
なお,ツイート間のトピックが明らかに異なる場合についても,この4クラスに分類しようとすれば
「その他」のラベルを付けることは可能である.
しかし,「その他」のツイートはツイート空間の整理には有用でないため,
ツイート間のトピックが明らかに異なる事例は除外する.


\subsection{同\addspan{トピック}ツイートペアのアノテーション}

上記のタスクを行うにあたり,正解データを準備する.
データはhottoコーパスより集める.
まず,11個のトピックを用意する.
これらのトピックは返信の分類のために集めたときのもの\addspan{(表
\ref{tab:20query})}の中から選択した.
対応するクエリを用いて検索されたツイートのうち,RT数が上位のものから30件ずつ集める.
RT数上位のものから順に集めるのは,それらのツイートは多くのユーザの目に触れるものであり,
論述構造を分析・整理する意義が大きいためである.
\addspan{ただし,疑問を呈するツイートのRT数が上位に来ることは少ないため,
この集め方では疑問の関係は少ないデータとなる.}
クエリは,該当トピックをうまく集められそうなものを選んでいるが,
中にはトピックに関連するツイートだけを集めるようなクエリの調整が難しく,
実際にトピックに対応するツイートの数は30未満になるものもある.
トピック,クエリ,トピックに関連するツイートの数を表\ref{tab:11query}に示す.

\begin{table}[b]
\caption{一般ツイートペア分類実験の11トピック}
\label{tab:11query}
\input{16table06.txt}
\vspace{0.5zw}
\small 対応ツイート数とは,収集用クエリで抽出したRT数上位30件のうち,トピックに対応するツイートの数を指す.\\
※表中に示されているトピックは,全て虚偽と確認されており事実ではない.\par
\end{table}

これらの11トピックに対し,ペアワイズな全てのツイートの組み合わせを考える.
例えば,「イソジンを飲むと放射能対策になる」のトピックに関しては,
27個のツイートが存在するので,その組み合わせは全部で351個ある.
これらの組み合わせについて,前述した4クラスのラベル付けを行う.
ラベル付けは,\addspan{アノテーション経験の豊富な一人のアノテーター}の手により行った.
アノテーションの結果を表\ref{tab:pair_annotation}に示す.
「同意」の割合が高く,予想した通り「疑問」の割合は非常に少ないことが分かる.
また,このアノテーション済のデータを,以降データセットCと呼ぶ.

\begin{table}[t]
\caption{トピック毎のアノテーション結果}
\label{tab:pair_annotation}
\input{16table07.txt}
\vspace{0.5zw}\small
※表中に示されているトピックは,全て虚偽と確認されており事実ではない.\par
\end{table}


\subsection{ネットワーク拡張に基づく分類手法}

同じトピックの中でのツイートペアの関係分類を,返信関係分類器を用いて行う手法について述べる.
大まかな方針としては,直接の返信の関係が見当たらない場合,
ツイートやユーザからなるネットワークを拡張し,返信の関係を探す.
前節で述べた反論表現辞書などを用いることで,
返信関係にないツイートペアに比べ格段に関係が推測しやすい.
そして,返信分類器の出力の結果を元のツイートペア間の関係の推測に用いることとする.
まず,直接返信の関係になっている場合が最も直接的な経路である(図 \ref{fig:direct_repqt}).
図のリンクについている名称は関係を表している.
リンク関係には「author」「RT」「返信」の三種類がある.
authorはツイートとそのツイートの投稿者の間に,
RTはツイートとそのツイートをリツイートしたユーザの間に生じる関係である.
このケースでは,前節の分類器の結果をそのまま使用する.
分類器の学習には,データセットBの全データ(2690個)を用いる.

次に,その周囲のツイートやユーザまで拡張したネットワークを考え,考えられる様々な経路で元のツイートペアを結ぶ.
その経路内に返信・引用関係が存在する場合,前節で提案した分類器を用いて各クラスのスコアを算出する.
\addspan{
最後に各クラス毎にスコアの和を計算し,最もスコアの高いクラスに分類する.
}
元のツイートペアを結ぶ経路として以下のものを導入する.
まず,両ツイートをリツイートしているユーザがいる場合を考える(図
\ref{fig:both_RT}).
本稿では,リツイートは同意であるという仮定を置いている.
しかしながら,反論の意図を持ってリツイートしている場合もあるため,
片方をリツイートしたユーザ数に対し,両方をリツイートしたユーザ数の割合が大きいときほど
同意とみなしやすいと考え,\addspan{aをリツイートしたユーザ集合とbをリツイートしたユーザ集合の間の}ジャッカード係数を同意のスコアとする.
次に,\addspan{片方のツイートの投稿者が,もう片方のツイートに対し他のツイートで返信している場合}
を考える(図\ref{fig:other_repqt}).
このケースでは,a--b間の関係をa1--b間が表していると考え,
スコアには,ロジスティック回帰\addspan{により求められた各クラス毎}の確率値を用いる.
さらに,片方をリツイートしてもう片方に返信しているユーザがいる場合
を考える(図\ref{fig:RT_and_repqt}).
このケースでは,リツイートは同意であるという仮定に基づき,
a--b間の関係をc1--b間が表していると考え,同様に返信分類器の出力(\addspan{ロジスティック回帰により求められた各クラス毎の}確率値)を用いる.

\begin{figure}[t]
 \begin{minipage}[t]{0.45\textwidth}
  \begin{center}
   \includegraphics{20-3ia16f5.eps}
  \end{center}
  \caption{直接返信のパターン}
  \label{fig:direct_repqt}
 \end{minipage}
\hfill
 \begin{minipage}[t]{0.45\textwidth}
  \begin{center}
   \includegraphics{20-3ia16f6.eps}
  \end{center}
  \caption{両方をリツイートするパターン}
  \label{fig:both_RT}
 \end{minipage}
\end{figure}
\begin{figure}[t]
\setlength{\captionwidth}{0.45\textwidth}
 \begin{minipage}[t]{0.45\textwidth}
  \begin{center}
   \includegraphics{20-3ia16f7.eps}
  \end{center}
  \caption{他のツイートで返信のパターン}
  \label{fig:other_repqt}
 \end{minipage}
\hfill
 \begin{minipage}[t]{0.45\textwidth}
  \begin{center}
   \includegraphics{20-3ia16f8.eps}
  \end{center}
  \hangcaption{片方をリツイート・もう片方に返信のパターン}
  \label{fig:RT_and_repqt}
 \end{minipage}
\end{figure}

以上のようなルールは,要求に応じて拡張が可能である.
例えば,aをリツイートしたユーザCとbをリツイートしたユーザDの間に返信関係があった場合に,
分類器によるその関係の出力を用いることができる.
ただし,ルールを増やすほど精度は下がると考えられるため,本稿では上記のルールに限定する.



\subsection{文間関係認識に基づく分類手法}

本節では,2文を与えたときにその間の意味的関係を返す文間関係認識器を用いて
ツイート間の関係を分類する手法について述べる.文間関係認識の部分課題に位
置づけられる含意関係認識は,前述の通りワークショップが開催されるなど広く
研究されている.含意関係認識課題で取り扱われている関係の種類を以下に示す.
\begin{description}
 \item[含意] 一方の文(テキスト $T$)を真としたとき,他方の文(仮説 $H$)も真
	    であると判断できる\addspan{文対}
	    \begin{description}
	     \item[$T$] 川端康成は「雪国」などの作品でノーベル文学賞を受賞した
	     \item[$H$] 川端康成は「雪国」の著者である
	    \end{description}
 \item[矛盾] 文中の事象が同時に成立し得ない\addspan{文対}
	    \begin{description}
	     \item[$T$] モーツァルトはザルツブルグで生まれた
	     \item[$H$] モーツァルトはウィーンで生まれた
	    \end{description}
 \item[その他] 上記以外\addspan{の文対}
	    \begin{description}
             \item[\addspan{$T$}] \addspan{川端康成は小説家である}
             \item[\addspan{$H$}] \addspan{川端康成は大阪で生まれた}
	    \end{description}
\end{description}

本稿で取り扱う4つの関係とは,含意が同意に,矛盾が反論に対応するが,同意
は厳密な含意ではなく,反論は矛盾しているとは限らない.
そこで,水野ら\cite{水野_2011}の同意,対立関係を対象とした文間関係認識器
を用いる.彼らの定義する同意,対立関係は,本稿で定義する同意,反論関係と
ほぼ対応する.\addspan{疑問関係は取り扱えないため,同意,反論,その他の3種類の関
係に分類する.}

まず,\addspan{図\ref{fig:srr_example}を用いて,}彼らの手法を簡潔に述べる.次に,本
課題のために変更した点について述べる.

\begin{figure}[b]
 \centering
   \includegraphics{20-3ia16f9.eps}
 \caption{文間関係認識例}
 \label{fig:srr_example}
\end{figure}

\begin{description}
 \item[言語解析] 入力された2文それぞれに対して,形態素解析
	    \cite{Kudo04applyingconditional},係り受け解析\cite{cabocha},
	    述語項構造解析\cite{watanabe10jointwsdsrl},拡張モダリティ解
	    析\cite{eguchi10nlp},評価極性判定を行う.\addspan{図
	    \ref{fig:srr_example}では,係り受け構造を文の上下の矢印で示
	    す.}

 \item[\addspan{仮説対応部分の同定}] \addspan{テキスト中には,仮説の内容と関連の低い情報も含
	    まれている.そこで,まず,テキスト中で仮説と内容的に対応する
	    部分を同定する.後段の関係分類では,対応する名詞間や述語間の
	    関係を考慮することで,文間の関係を同定する.テキスト中で仮説
	    に対応する部分の同定は,文節アライメント,局所構造アライメン
	    ト,文節アライメントの選択という3段階で行われる.以下にそれぞ
	    れの手続きを示す.}
	    \begin{description}
	     \item[\addspan{文節アライメント}] \addspan{文節中に含まれる内容語の類似・関連性
			に基づいて文節間に文節アライメントを付与する.文
			節中の内容語が類似しているとき,それらが含まれる
			文節間にアライメントを付与する.図
			\ref{fig:srr_example}では,「イソジンで($H$)」と
			「イソジンを($T$)」が,イソジンという内容語が共通
			することからアライメントされる.}
			
                        \addspan{
			内容語の類似・関連性は,その表層だけでなく,日本
			語WordNet~\cite{bond_09_enhancing}や動詞含意関係
			データベース~\cite{hashimoto09emnlp}を用いて,意
			味的な類似度にも基づいて判断される.図
			\ref{fig:srr_example}において,「防ぐ($H$)」と
			「回避できる($T$)」は,意味的に類似しているためア
			ライメントされる.}

	     \item[\addspan{局所構造アライメント}] \addspan{文節アライメントによってアライ
			メントされた文節の中には,文全体の意味を考えると
			アライメントすべきではない場合がある.例えば,
			「イソジンを飲んで被曝を防ぐ」と「ワカメの味噌汁
			を飲むと良い」という2文を考える.2文間で「飲む」
			が共通しているが,その対象は異なっているため,ア
			ライメントすべきではない.}

			\addspan{
                        この問題に対して,文中の依存構造および述語項構造
			を対応付けるのが局所構造アライメントである.図
			\ref{fig:srr_example}において,文節アライメントさ
			れる「被曝を($H$) \--\footnote{「\--」 で前後の単
			語を含む文節が文節アライメントされることを示す}
			被曝を($T$)」と,「防ぐ($H$) \-- 回避できる($T$)」
			に着目する.これらは$H$側,$T$側のいずれにも依存
			構造が存在し,この構造間にアライメントを付与する
			のが局所構造アライメントである.「イソジンで
			($H$) \-- イソジンを($T$)」と「防ぐ($H$) \-- 回避
			できる($T$)」は,$H$側には依存構造が存在するが,
			$T$側には直接の依存関係は存在しない.しかしながら,
			「飲むと($T$)」を介して依存構造の繋がりが存在する.
			本研究では水野らと同様に,4文節まで介しても良いと
			いう上限を設けた.}

	     \item[\addspan{文節アライメントの選択}] \addspan{文節アライメントされた文節対の
			うち,局所構造アライメントもされている文節対のみ
			を選択する.}
	    \end{description}

 \item[関係分類] アライメント結果を入力として,まず,同意および対立と,そ
	    の他との分類を行い,次に,同意と対立の分類を行う.1つめの分類
	    は,仮説側の文節が全て\addspan{テキストにアライメントされたか}によって
	    判断される.2つめの分類について,対応付けられた述語が,否定の
	    関係にあるか,評価極性が異なるか,反義の関係にある場合,対立
	    に分類され,いずれにも当てはまらないものは同意に分類される.
\end{description}
本研究で対象としている,\addspan{ツイートデータ}には,述語間の否定,評価極性,反義
\addspan{に基づいて分類可能な対立文対}だけでなく,「〜というのはデマです」や「〜と
いう事実はありません」といった\addspan{述語}の後方での否定・反論も少なくない.そこ
で,このような表現を否定表現と呼び,対応付けられた述語よりも後方に否定表
現が現われた場合,同意ならば対立に,対立ならば同意に関係を反転させる工程
を付け加えた.本研究で利用した否定表現は,「誤り,ねつ造,誇張,嘘,誤報,
デマ,今のところない,ソースがありません,事実はありません,根も葉もない
噂,根拠がない,ニュースはない」である.これらは,評価データ以外のツイー
トから収集した.表\ref{tab:反論表現}に示される反論表現とは,一般ツイート
中の表現である点が異なるが,重なる表現も存在する.\addspan{図
\ref{fig:srr_example}において,$T$側の「回避できる」は,その後方の「とい
うのはデマだ」という表現によって否定されており,この2文は対立関係に分類さ
れる.}

彼らの文間関係認識手法は,\addspan{仮説}が単文(一つの述語といくつかの項からなる文)
であることを前提としている.しかし,ツイートに含まれる文は,単文であるこ
とは少ない.そこで,\addspan{まず,トピックと2つのツイートの間の関係をそれぞれ同定
し,次にこれら2つの関係に基づいてツイート間の関係を同定する.このとき,
トピックが仮説に,ツイートがテキストに対応する.図 \ref{fig:rel_classification_1}に,トピックを用いてツイート間の関係を求め
る手法を示す.まず,関係を求めたい2つのツイートに対して,ツイートとトピッ
クとの間の関係を求める.その際,ツイートは,まず文分割され,次に収集用ク
エリが全て含まれる文のみが選択される.そのような文が複数存在する場合は,
一番後ろの一文を選択する.図\ref{fig:rel_classification_1}は,中心にトピッ
ク,上下にツイートから選択された文を表している.上側の文はいず
れも同意で,下側の文は,一方が同意で,他方が対立に分類されている.最後に,
求められた2つの関係から,ツイート間の関係を分類する.トピックとツイートと
の間の関係から導かれるツイート間の関係の組み合わせを,以下に示す.}
\addspan{
\begin{enumerate}
 \item 2つの関係のうち1つでも「その他」 $\Rightarrow$ 「その他」
 \item 2つの関係がいずれも「同意」 $\Rightarrow$ 「同意」
 \item 2つの関係がいずれも「対立」 $\Rightarrow$ 「同意」
 \item 2つの関係が異なる,すなわち「同意・対立」または,「対立・同意」
       $\Rightarrow$ 「反論」
\end{enumerate}
}
以降,本手法を「文間関係認識手法1」と呼ぶ.

\begin{figure}[b]
 \centering
   \includegraphics{20-3ia16f10.eps}
 \caption{文間関係認識手法1}
 \label{fig:rel_classification_1}
\end{figure}

ツイート間の関係を直接同定する手法も評価する.まず,文間関係認識手法1と同
様に,ツイートを文分割し,収集用クエリを含む文を選択する.次に,関係を求
めたい任意の2ツイートについて,選択された2文間の関係を求める.同定された
文間関係は,同意はそのまま同意関係に対応し,対立は反論関係に対応する.以
降,本手法を「文間関係認識手法2」と呼ぶ.


\section{\addspan{一般的なツイート間の関係認識結果}}

本節では,前節で述べた一般のツイート間の関係認識手法の評価を行う.本研究
の目的は,図\ref{fig:map}のように,ツイート間の論述関係を可視化し,ユーザ
に情報の「裏」を提示し,情報の価値判断を支援することである.誤分類事例が
多いと,正しく情報の裏を取ることができないため,高い精度を実現することが
重要であり,その上で再現率を上げていく戦略をとる.本稿では,論述関係の可
視化を実現するために必要な程度の数のツイートに対して,正しく関係分類が行
えたかに着目して評価する.また,ネットワーク拡張手法,文間関係認識手法が
それぞれどのように有効であったかを考察する.


\subsection{各トピックにおける評価}

\begin{table}[b]
\vspace{-0.8\Cvs}
\caption{ツイート間関係分類結果(精度)} 
\label{tab:rel_res_prec}
\input{16table08.txt}
\vspace{0.5zw}
\small ※表中に示されているトピックは,全て虚偽と確認されており事実ではない.\par
\end{table}

正解の関係ラベルを付与したデータセットCの各ツイートペアについて,関係分
類実験を行った.ネットワーク拡張手法について,図\ref{fig:direct_repqt}〜
図\ref{fig:RT_and_repqt}に示すいずれのパターンにも当てはまらない場合は,
関係を出力しない.

実験結果の精度を表\ref{tab:rel_res_prec}に,再現率を表
\ref{tab:rel_res_rec}に,全トピックのマクロ平均と併せて示す.N/Aは,その
関係に分類された事例が存在しなかったことを示す.

\begin{table}[t]
\caption{ツイート間関係分類結果(再現率)}
\label{tab:rel_res_rec}
\input{16table09.txt}
\par\vspace{0.5zw}\small
※表中に示されているトピックは,全て虚偽と確認されており事実ではない.\par
\end{table}


ネットワーク拡張手法は,簡単なパターンの追加だけで,分類された事例は少な
いものの,直接の返信関係にないツイート間の関係分類を行うことがで
きた.文間関係認識に基づく手法について,一部のトピックについては,多くの
事例を高精度で分類できている.いずれの手法でも分類された事例数の少なかっ
たトピックとして,「東大が入学を取り消し」,「埼玉の水道水に異物が混入し
危険」などがある.


\subsection{考察}

\addspan{
本節では,まず,ネットワーク拡張手法の分類結果について考察し,文間関係認
識手法1および2との比較を行う.次に,文間関係認識手法1と2について比較する.
}

\begin{table}[t]
\caption{ネットワーク拡張手法の各パターンによる判定の精度}
 \label{tab:各パターンの精度}
\input{16table10.txt}
\par\vspace{0.5zw}\small
各トピックについて,上の行が同意,下の行が反論関係の分類性能を示し,手法ごとの精度を示す.\\
※表中に示されているトピックは,全て虚偽と確認されており事実ではない.\par
\end{table}

\addspan{
ネットワーク拡張手法について,図\ref{fig:direct_repqt}から図
\ref{fig:RT_and_repqt}の各パターンごとの分類性能を表\ref{tab:各パターンの精度}に示す.表の各列は,各パターンを単独で使用して,ツイートa,b間の関係
を分類した場合の分類精度を示している.
ただし,両方をリツイートするパターン(図\ref{fig:both_RT})は,
それぞれのツイートをリツイートしたユーザ集合間のジャッカード係数を同意のスコアとするものであるため,
閾値0.01以上の場合のみ同意と判定することとしている.
このとき,直接返信のパターン(図
\ref{fig:direct_repqt})は,返信態度の分類器をそのまま適用した場合と同等
である.反論関係は「福島から避難した子供には教科書が配布されない」に対し
て1件の誤分類が存在するのみであったが,同意関係への分類精度はマクロ平均で
0.953と高かった.次に,拡張した各パターンの結果について,まず,両方をリツ
イートするパターン(図\ref{fig:both_RT})について,本研究ではリツイートは同意関係であるという仮定を置いているため,
本パターンでは反論関係への分類は行われない.同意関係への分類精度は比較的
高く,本パターンの有効性が示されている.今後は,リツイート後のツイートも
参照し,同意関係ではないリツイートも考慮することで,同意関係の分類精度向
上,反論関係への分類も行うことが考えられる.
次に,他のツイートで返信のパターン(図\ref{fig:other_repqt})は,適合する
場合が少なかっために,分類された事例も少なかった.分類性能は,反論事例の
分類精度が最も高かったが,分類された事例数が少ないことが問題である.
最後に,片方をリツイート・もう片方に返信のパターンは,分類された事例は
同意・反論とも多かった.
分類精度については,同意関係は,両方をリツイートするパターン・他のツイー
トで返信するパターンと同程度の精度で分類することができたが,
反論関係の分類精度はあまり高くなかった.
}

\addspan{ネットワーク拡張手法の分類性能を向上させるためには,}パターンの拡張や,
ネットワークを構成するツイートやユーザの増加が考えられる.前者は,ネットワークをより広範囲にわたって探索して,
関係を探してくるようにすることによって,分類可能事例を増やす.
ただし,ノイズの多いパターンを採用することで精度は下がる可能性が高い.例
えば,元のツイート間のパスが長くなるほど,そのリンク関係の信用度も下がる
と考えられる.本研究で目指すツイート空間の整理は,大量のツイートをまとめ
るよりは,少数でも重要なツイートを高精度で分類,整理することが有効である
と考えているので,ルールの拡張は最適点を見つける問題に帰着することになる.
一方で,後者の対策である,ネットワーク探索の対象となるツイートそのものを増やすことは有効であると
考えられる.処理を適用するツイートを増やすことで,パターンを拡張することなく,
新しいリンクが見つかる可能性は高い.

\addspan{ネットワーク拡張手法と,文間関係認識手法1および2を比較して,ネットワーク
拡張手法でのみ正しく分類できた事例として,「救援物資の空中投下は法律で禁
止されている」の反論事例の一つを以下に示す.}
\begin{description}
 \item[a] 【デマ33】 「日本では物資の空中投下が認められていない」 そん
	    なことはありません。
 \item[b] なんと驚いた情報です!日本では物資の空中投下が認められていない
	    んだそう!とっくに自衛隊が孤立被災者に実施してると思ってた。
	    これでは本当に孤立者が死んでしまう。救出前にヘリで食糧を落と
	    して何が悪いんだろう。わたしは今これを知り怒りで全身が震えて
	    ます。みなさんリツイートをお願い!
\end{description}
\addspan{a,bのいずれも,「法律」については言及されておらず,さらにaは括弧による引
用に対して否定する構造となっていたため,文間関係認識手法1,2のいずれも反
論関係に分類できなかった.ネットワーク拡張手法では,片方をリツイート・も
う片方に返信のパターン(図\ref{fig:RT_and_repqt})によって分類された.パ
ターン中のc1として使われたツイート,すなわちaをリツイートしたユーザによる
ツイートbへの返信は全部で7ツイート存在した.
そのうちの3つを以下に示す.}


\begin{description}
\item[c1] @XXXXX 【お願いします】問題のツイート削除してください!このままでは本
当に物資の空中投下が出来ないと誤解されます。物資が遅れている、投下より安
全な輸送を優先している、これは事実です。しかしツイッター上では議論がずれ
てしまってる現状をご理解ください!
 \item[] @XXXXX このツイートはデマです。すぐに削除してください。情報に正確性の
無いものを拡散させないようお願いします。ツイートでの発言も、常識を忘れな
いで下さい。拡散させたいときは、ちゃんと調べて情報元などをしっかり確認し
た上でして下さい。迷惑になります。
 \item[] @XXXXX これデマだそうですね 投下始まったそうで ''孤立状態の被災者
に対しヘリコプターを使った食料の投下も始まった''
http://www.xxx./xxx.html こんな時にデマ
流さないで!
\end{description}
\addspan{
これらのツイートは全て反論であり,他の4ツイートも全て反論であった.
返信の分類器によりこれらを正しく「反論」と分類できたため,最終的に出力されるスコアも「反論」が最も高く,
}
a, b間の関係を正しく分類することができた.

\addspan{
次に,文間関係認識手法1と2の結果について比較する.全体的には,手法2の方
が精度,再現率ともに優れていた.同意関係について手法2の方が優れていた事
例について分析したところ,2ツイートがいずれも同一の文を引用している事例
が大多数であった.以下に例を示す.
}
\addspan{
\begin{description}
 \item[ツイート1] ツイッターもテレビもメディアを扱う人は考えねばならない
	    ことがある。誤報について\ulinej{福島・双葉病院患者置き去り報
	    道の悪意。}医師・看護師は患者を見捨てたりしていなかった
 \item[ツイート2] 記事:\ulinej{福島・双葉病院患者置き去り報道の悪意。}医
	    師・看護師は患者を見捨てたりしていなかった—誤報流した新聞は全
	    力で訂正しろ!
\end{description}
}
\addspan{
ツイート1と2で,下線部は共通している.下線部は,これらのツイートが言及し
ている記事のタイトルであり,この共通部分の関係が同意と分類された結果,ツ
イート間の関係も同意と正しく分類された.手法1では,ツイート中にトピックと
同意関係にある1文が存在しないため,対応できなかった.
}
\addspan{
反論関係について手法2の方が優れていたトピックとして「救援物資の空中投下は
法律で禁止されている」があげられる.以下の例では,下線部の関係が対立に分
類されたことで,ツイート間の関係が反論に正しく分類された.
}
\addspan{
\begin{description}
 \item[ツイート3] 孤立してる被災地に物資が届いていない。不足マップ作って
	    空と陸から支援できないものか?知り合いは徒歩でおにぎりを届け
	    てるらしい。\ulinej{ヘリで空中投下できると良い。}インターネッ
	    ト接続設備も情報のやりとりの為に必須。
 \item[ツイート4] \ulinej{自衛隊が物資の空中投下が出来ないという噂はデマ
	    だったのでマスコミのみなさんのヘリで孤立した避難場所への物資
	    の空中投下はできないのでしょうか?}政府で許可が下りないのなら
	    即非常措置をお願いしたい。
\end{description}
}
\addspan{
手法1で反論関係に分類されるのは,一方のツイート中にトピックに同意する文が
含まれ,他方のツイート中に対立する文が含まれている場合である.上記の例で
は,ツイート3でトピックに関連する文は,対象物である「物資」が省略されてお
り,「法律」に対する言及もなされていないため,トピックに対して同意
関係に分類されなかった.そのため,ツイート間の関係もその他に分類された.
}

\addspan{一方で,反論関係について手法1の方が優れていたトピックとして「双
葉病院の医師が患者を置き去りにした」が挙げられる.以下の例において,ツイー
ト5,6の下線部は,手法2では,ツイート5中の「見捨てていません」に相当する
表現がツイート6に含まれていなかったため,対立関係に分類できなかったが,手
法1ではトピックとの関係分類において,それぞれ対立,同意に分類され,反論関
係に分類することができた.}
\addspan{
\begin{description}
 \item[ツイート5] TUFの報道番組が大熊町双葉病院の職員は患者を残して逃げた
	    と報道しましたが違います。事実は職員は一生懸命患者を搬送して
	    から避難しました。これが真実です。私の父は双葉病院の主任の 1 人です。
\ulinej{双葉病院の職員は患者を見捨てていません。}ご理
	    解ご…
 \item[ツイート6] \ulinej{\mbox{福島第1原発の10キロ圏内にあり}避難指示が出た同町
	    の双葉病院で、患者を避難させるため自衛隊が到着した際、\mbox{病院内は高齢の入院患者128人だけで、}
	医師や病院職員らがいなかったこと
	    が分かった。}最低だよ酷すぎる。この病院から避難所に移送された
	    患者が14人も亡くなってる
\end{description}
}

以上より,手法2が有利なトピックは,2つのツイートに類似した表現が含まれる
場合であることが分かった.特に,両方のツイートが同じ文を引用し,それに対
する意見を述べているようなトピックにおいて顕著であった.一方で手法1は,
トピックを適切に作文できれば,2つのツイートが類似していない場合に有利で
あることが分かった.


水野らの文間関係認識の問題点は,複数文同士の関係分類に\addspan{直接は}対応できない
ことである.\addspan{そのため,手法1および2では,文分割を行った.}
\addspan{
今後は,文をまたぐ照応解析や共参照解析を行うことで,関係分類の精度をより向上させ,再現率
も向上させることが考えられる.
}一般的に文をまたぐ照応解析\addspan{や共参照解析}の実
現は難しいが,ツイートを対象とした場合,\addspan{参照先}となる単語が最大でも140文字
以内に含まれていることを活用することで,問題の難易度を下げられる可能性が
ある.


\subsection{論述構造の可視化}

今回の実験で得た結果を元に,「イソジンを飲むと放射能対策になる」のトピックに関する
ツイート空間の論述構造グラフを作成した.
作成の手順は以下である.
まず,「イソジンを飲むと放射能対策になる」に対応する\addspan{351}個のツイートペアに対し,
ネットワーク拡張による手法と文間関係認識による手法のそれぞれで関係分類を試みる.
1つ以上の手法で「反論」か「疑問」と識別したツイートペアについてはそれらの関係で結ぶ.
\addspan{
関係が結ばれたツイートペアを全て集めたものを,グラフに載せるツイートの集合とする.
次に,そのツイート集合内に生じる全てのツイート間のペアのうち,
すでに「反論」か「疑問」の関係がついたペア以外については,}
1つ以上の手法で「同意」と識別されたものがあれば,「同意」関係で結ぶ.
以上の流れで論述構造グラフを作成することで,
「反論」や「疑問」といった重要な関係を抽出しつつ,
無闇にツイート数を増やすことのないグラフを得ることができる.

今回の実験結果を元に作成できたグラフを図\ref{fig:graph}に示す.

\begin{figure}[t]
  \begin{center}
   \includegraphics{20-3ia16f11.eps}
  \end{center}
  \caption{「イソジンを飲むと放射能対策になる」の論述構造グラフ}
  \label{fig:graph}
\end{figure}

グラフ中の\addspan{双方向矢印}は「反論」,\addspan{結合点が丸になっている線}は「同意」と分類された関係を表し,
実線はネットワーク拡張による手法,破線は文間関係認識による手法で分類した関係を表す.
\addspan{なお,図\ref{fig:graph}中のツイート間では,どちらの手法でも分類できた例はなかった.}
ただし,これらの分類結果が誤っていた関係には×印を付与してある.
この実験では,「疑問」を識別することはできなかった.
なお,図を描くにあたっては,最も多くの関係を持つツイート「イソジン飲まないで!放射性ヨウ素が集まるのを抑制する効果なし…」を便宜上中心に据えた.

このグラフから以下のことが言える.
まず,中心ツイートと,図の左上にある二つのツイート(「甲状腺に問題がない人なら…」「ちょ,イソジン等内服以外の…」)は
三角形に結ばれているが,これはネットワーク拡張による手法と文間関係認識が相補的に寄与することで実現している.
\addspan{
ツイート間の関係を独立に表示するだけでなく,3ツイート以上の相互の関係を示すことで,
ユーザにとっての分かりやすさやシステムの信頼性を高められる可能性がある.
}
このように,\addspan{精度を重視した}両手法を組み合わせることで,\addspan{システムの信頼性を高め,説得力のある論述構造をユーザに提供できる可能性が高まる.}
また,中心のツイートと,その下のツイート(「薄めたイソジンならOK?…」)の間の反論関係を取得している点も特徴である.
グラフを作る際に集めたツイートは「イソジンを飲むと放射能対策になる」というトピックに関連するツイートであり,
このトピックに対する立場で言えば,中心のツイートとその下のツイートは反論関係にあるとは言えない.
しかしながら,海藻が効果があるかどうかという点で見ればこの二つのツイートは反論関係にあり,取得したい情報である.
このように多様な情報の論述構造を取得するのが,本研究の目指すところである.

現状の課題として,クラス分類を行った関係に対して誤りが多いことが挙げられる.
正しく分類されている関係のみを見ることができれば,ユーザは十分な情報を得ることができるが,
間違って分類されている関係が混在していると,情報を取得するにあたり混乱を招くこととなる.
より実用性のある論述構造を得るには,分類の精度を高めなければならない.
また,「疑問」の関係にあるツイートやURLを含むツイートを取得し,論述構造グラフに組み込むことも目指している.



\section{おわりに}

本研究では,震災時など多様な情報がTwitter上で氾濫するような状況を想定し,
ツイート空間の論述構造を解析する構想を示した.
多岐に渡るツイートを整理することは容易ではないが,
我々はまず返信で表明される態度に着目し,教師あり学習による4クラスへの分類を行った.
反論表現辞書や構造的特徴を用いることで性能を向上できることを示し,
特に重要な論述関係である「反論」の識別性能の\addspan{$F_1$値}で0.472,
4クラス全体の正答率で0.751という性能を得た.

続いて,一般のツイート間の論述関係を分類する手法として,
ネットワークの拡張による手法と,文間関係認識による手法を比較した.
前者は,あるツイート間の関係を間接的に表すと思われる
ツイートやユーザの関係を探索し,必要に応じて返信の分類器を適用することで,
元のツイートペア間の関係を推測する手法である.
それに対し後者は,ツイート間の関係を直接言語処理で解く手法であり,
従来研究の含意関係認識器を本稿のタスク用にカスタマイズして使用した.
これら二つの\addspan{手法}は相補的に作用することが期待される.
実験では,各手法において得意な\addspan{性質を持つツイートペアの関係分類で高い性能を}発揮した他,
適用するトピック次第では両手法が相補的に活用され,
有用な論述構造グラフが得られることを示した.
元来Twitterの持つ強力な情報伝達力を活用するためには,
本研究の提案手法は有効であると考えている.

今後の課題としては,返信の態度分類器の性能をさらに高める必要がある.
具体的には,反論表現辞書の拡張および,より有効な構造的特徴の開発が必要と
なる.さらに,より有用な論述構造グラフ作成に向けては,ネットワークの拡張
による手法と,文間関係認識による手法を共に改善させていくことが必要である.
具体的には,前者はより有用なヒューリスティックスの実装とそれに基づく広範
なネットワークの探索,およびそのためのデータ整備を行う.後者については,
複数文同士の関係分類\addspan{において,文をまたいだ照応解析や共参照解析技術を取り
入れること}が重要であると考えている.

本研究が,先の大震災のような危難の中で,
Twitter上の情報活用に一役買うことになれば幸いである.


\acknowledgment
本研究は,文部科学省科研費 (23240018),文部科学省科研費 (23700159),
およびJST 戦略的創造研究推進事業さきがけの一環として行われた.
本研究で使用したデータは,株式会社ホットリンクより提供された.

\bibliographystyle{jnlpbbl_1.5}
\begin{thebibliography}{}

\bibitem[\protect\BCAY{Akamine, Kawahara, Kato, Nakagawa, Inui, Kurohashi,
  \BBA\ Kidawara}{Akamine et~al.}{2009}]{Akamine_2009}
Akamine, S., Kawahara, D., Kato, Y., Nakagawa, T., Inui, K., Kurohashi, S.,
  \BBA\ Kidawara, Y. \BBOP 2009\BBCP.
\newblock \BBOQ Wisdom: A web information credibility analysis
  systematic.\BBCQ\
\newblock In {\Bem Proceedings fo the International Conference on ACL-IJCNLP
  2009 Software Demonstrations}, \mbox{\BPGS\ 1--4}.

\bibitem[\protect\BCAY{Bond, Isahara, Fujita, Uchimoto, Kuribayashi, \BBA\
  Kanzaki}{Bond et~al.}{2009}]{bond_09_enhancing}
Bond, F., Isahara, H., Fujita, S., Uchimoto, K., Kuribayashi, T., \BBA\
  Kanzaki, K. \BBOP 2009\BBCP.
\newblock \BBOQ Enhancing the {J}apanese {WordNet}.\BBCQ\
\newblock In {\Bem Proceedings of the 7th Workshop on Asian Language
  Resources}, \mbox{\BPGS\ 1--8}.

\bibitem[\protect\BCAY{Castillo, Mendoza, \BBA\ Poblete}{Castillo
  et~al.}{2011}]{Castillo_2011}
Castillo, C., Mendoza, M., \BBA\ Poblete, B. \BBOP 2011\BBCP.
\newblock \BBOQ Information credibility on twitter.\BBCQ\
\newblock In {\Bem Proceedings of the 20th International Conf. on World Wide
  Web (WWW2011)}, \mbox{\BPGS\ 675--684}.

\bibitem[\protect\BCAY{Dagan, Glickman, \BBA\ Magnini}{Dagan
  et~al.}{2005}]{Dagan_2005}
Dagan, I., Glickman, O., \BBA\ Magnini, B. \BBOP 2005\BBCP.
\newblock \BBOQ The pascal recognising textual entailment challenge.\BBCQ\
\newblock In {\Bem First Machine Learning Challenges Workshop}, \mbox{\BPGS\
  177--190}.

\bibitem[\protect\BCAY{de~Marneffe, Rafferty, \BBA\ Manning}{de~Marneffe
  et~al.}{2011}]{MarneffeIdentifying2011}
de~Marneffe, M.-C., Rafferty, A.~R., \BBA\ Manning, C.~D. \BBOP 2011\BBCP.
\newblock \BBOQ {Identifying Conflicting Information in Texts}.\BBCQ\
\newblock In {\Bem Handbook of Natural Language Processing and Machine
  Translation: DARPA Global Autonomous Language Exploitation}. Springer.

\bibitem[\protect\BCAY{江口\JBA 松吉\JBA 佐尾\JBA 乾\JBA 松本}{江口 \Jetal
  }{2010}]{eguchi10nlp}
江口萌\JBA 松吉俊\JBA 佐尾ちとせ\JBA 乾健太郎\JBA 松本裕治 \BBOP 2010\BBCP.
\newblock モダリティ、真偽情報、価値情報を統合した拡張モダリティ解析.\
\newblock \Jem{言語処理学会第16回年次大会発表論文集 E3-8}, \mbox{\BPGS\
  852--855}.

\bibitem[\protect\BCAY{Giampiccolo, Magnini, Dagan, \BBA\ Dolan}{Giampiccolo
  et~al.}{2007}]{GiampiccoloRTE32007}
Giampiccolo, D., Magnini, B., Dagan, I., \BBA\ Dolan, B. \BBOP 2007\BBCP.
\newblock \BBOQ The third PASCAL recognizing textual entailment
  challenge.\BBCQ\
\newblock In {\Bem Proceedings of the ACL-PASCAL Workshop on Textual Entailment
  and Paraphrasing}, \mbox{\BPGS\ 1--9}.

\bibitem[\protect\BCAY{Gupta, Zhao, \BBA\ Han}{Gupta et~al.}{2012}]{Gupta_2012}
Gupta, M., Zhao, P., \BBA\ Han, J. \BBOP 2012\BBCP.
\newblock \BBOQ Evaluating event credibility on twitter.\BBCQ\
\newblock In {\Bem Proceedings of the 12th International Conf. on Data Mining
  (SDM2012)}, \mbox{\BPGS\ 153--164}.

\bibitem[\protect\BCAY{Hashimoto, Torisawa, Kuroda, Murata, \BBA\
  Kazama}{Hashimoto et~al.}{2009}]{hashimoto09emnlp}
Hashimoto, C., Torisawa, K., Kuroda, K., Murata, M., \BBA\ Kazama, J. \BBOP
  2009\BBCP.
\newblock \BBOQ Large-Scale Verb Entailment Acquisition from the Web.\BBCQ\
\newblock In {\Bem Conference on Empirical Methods in Natural Language
  Processing (EMNLP2009)}, \mbox{\BPGS\ 1172--1181}.

\bibitem[\protect\BCAY{Hassan, Abu-Jbara, \BBA\ Radev}{Hassan
  et~al.}{2011}]{Hassan_2012}
Hassan, A., Abu-Jbara, A., \BBA\ Radev, D. \BBOP 2011\BBCP.
\newblock \BBOQ Detecting subgroups in online discussions by modeling positive
  and negative relations among participants.\BBCQ\
\newblock In {\Bem Proceedings of the 9th Empirical Methods in Natural Language
  Processing and Computational Natural Language Learning (EMNLP-CoNLL2012)},
  \mbox{\BPGS\ 59--70}.

\bibitem[\protect\BCAY{Jiang, Yu, Zhou, Liu, \BBA\ Zhao}{Jiang
  et~al.}{2011}]{Jiang_2011}
Jiang, L., Yu, M., Zhou, M., Liu, X., \BBA\ Zhao, T. \BBOP 2011\BBCP.
\newblock \BBOQ Target-dependent twitter sentiment classification.\BBCQ\
\newblock In {\Bem Proceedings of the 49th Annual Meeting of the Association
  for Computational Linguistics (ACL2011)}, \mbox{\BPGS\ 151--160}.

\bibitem[\protect\BCAY{Kudo \BBA\ Matsumoto}{Kudo \BBA\
  Matsumoto}{2002}]{cabocha}
Kudo, T.\BBACOMMA\ \BBA\ Matsumoto, Y. \BBOP 2002\BBCP.
\newblock \BBOQ Japanese Dependency Analysis using Cascaded Chunking.\BBCQ\
\newblock In {\Bem CoNLL 2002: Proceedings of the 6th Conference on Natural
  Language Learning 2002 (COLING 2002 Post-Conference Workshops)}, \mbox{\BPGS\
  63--69}.

\bibitem[\protect\BCAY{Kudo, Yamamoto, \BBA\ Matsumoto}{Kudo
  et~al.}{2004}]{Kudo04applyingconditional}
Kudo, T., Yamamoto, K., \BBA\ Matsumoto, Y. \BBOP 2004\BBCP.
\newblock \BBOQ Applying conditional random fields to Japanese morphological
  analysis.\BBCQ\
\newblock In {\Bem Proceedings of the 1st Empirical Methods in Natural Language
  Processing (EMNLP2004)}, \mbox{\BPGS\ 230--237}.

\bibitem[\protect\BCAY{水野\JBA 渡邉\JBA 村上\JBA 乾\JBA 松本}{水野 \Jetal
  }{2011}]{水野_2011}
水野淳太\JBA 渡邉陽太郎\JBA 村上浩司\JBA 乾健太郎\JBA 松本裕治 \BBOP 2011\BBCP.
\newblock 文間関係認識に基づく賛成・反対意見の俯瞰.\
\newblock \Jem{情報処理学会論文誌}, {\Bbf 52}  (12), \mbox{\BPGS\ 3408--3422}.

\bibitem[\protect\BCAY{Shima, Kanayama, Lee, Lin, Mitamura, Miyao, Shi, \BBA\
  Takeda}{Shima et~al.}{2011}]{ShimaRITE12011}
Shima, H., Kanayama, H., Lee, C.-W., Lin, C.-J., Mitamura, T., Miyao, Y., Shi,
  S., \BBA\ Takeda, K. \BBOP 2011\BBCP.
\newblock \BBOQ Overview of NTCIR-9 RITE: Recognizing Inference in TExt.\BBCQ\
\newblock In {\Bem Proceedings of the 9th NTCIR Workshop}, \mbox{\BPGS\
  291--301}.

\bibitem[\protect\BCAY{Speriosu, Sudan, Upadhyay, \BBA\ Baldridge}{Speriosu
  et~al.}{2011}]{Speriosu_2011}
Speriosu, M., Sudan, N., Upadhyay, S., \BBA\ Baldridge, J. \BBOP 2011\BBCP.
\newblock \BBOQ Twitter polarity classification with label propagation over
  lexical links and the follower graph.\BBCQ\
\newblock In {\Bem Proceedings of the 1st Workshop on Unsupervised Learning in
  NLP}, \mbox{\BPGS\ 53--63}.

\bibitem[\protect\BCAY{Turney}{Turney}{2002}]{Turney_2002}
Turney, P.~D. \BBOP 2002\BBCP.
\newblock \BBOQ Thumbs up or thumbs down?: Semantic orientation applied to
  unsupervised classification of reviews.\BBCQ\
\newblock In {\Bem Proceedins of the 40th Annual Meeting on Association for
  Computational Linguistics(ACL2002)}, \mbox{\BPGS\ 417--424}.

\bibitem[\protect\BCAY{Watanabe, Asahara, \BBA\ Matsumoto}{Watanabe
  et~al.}{2010}]{watanabe10jointwsdsrl}
Watanabe, Y., Asahara, M., \BBA\ Matsumoto, Y. \BBOP 2010\BBCP.
\newblock \BBOQ A Structured Model for Joint Learning of Argument Roles and
  Predicate Senses.\BBCQ\
\newblock In {\Bem Proceedings of the 48th Annual Meeting of the Association of
  Computational Linguistics (ACL2010)}, \mbox{\BPGS\ 98--102}.

\end{thebibliography}

\clearpage
\appendix

\ref{subsec:class_method}節で示した反論表現辞書について,
収録した反論表現の一覧を以下に載せる.

\begin{table}[b]
\vspace{-2\Cvs}
\caption{反論表現辞書}
\input{16table11.txt}


\vspace{2\Cvs}
\begin{biography}
\bioauthor{大和田裕亮}{
2011年東京大学工学部電子情報工学科卒業.
2013年同大学院情報理工学系研究科・電子情報学専攻修士課程修了.
同年よりヤフー株式会社.
}

\bioauthor{水野 淳太}{
2012年奈良先端科学技術大学院大学情報科学研究科博士課程修了.同年より東北
大学大学院情報科学研究科研究員.2013年より独立行政法人情報通信研究機構
耐災害ICT研究センター研究員.博士(工学).自然言語処理,耐災害情報通信
の研究に従事.情報処理学会,人工知能学会各会員.}

\bioauthor{岡崎 直観}{
2007年東京大学大学院情報理工学系研究科・電子情報学専攻博士課程修了.同大
学院情報理工学系研究科・特別研究員を経て,2011年より東北大学大学院情報
科学研究科准教授.自然言語処理,テキストマイニングの研究に従事.情報理
工学博士.情報処理学会,人工知能学会,ACL 各会員.
}

\bioauthor{乾 健太郎}{
1995年東京工業大学大学院情報理工工学研究科博士課程修了.同研
究科助手,九州工業大学助教授,奈良先端科学技術大学院大学助教授を経て,
2010年より東北大学大学情報科学研究科教授,現在に至る.博士(工学).自然
言語処理の研究に従事.情報処理学会,人工知能学会,ACL,AAAI各会員.
}

\bioauthor{石塚  満}{
1971年東京大学工学部電子卒.1976年同大学院博士課程修了.NTT横須賀研究所,
東大生産研究所・助教授,同大学工学部電子情報・教授,同情報理工学系研究科創造情報/電子情報
・教授を歴任し,2013年東大を退職.専門は人工知能,Web知能化,意味計算,マルチモーダルメディア.
人工知能学会・元会長.
}

\end{biography}


\biodate

\end{document}
