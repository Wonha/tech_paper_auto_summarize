



\documentstyle[jnlpbbl,punc,epsbox]{jnlp_j_b5}

\newcommand{\mltclm}[2]{}

\def\figcap#1#2{}

\setcounter{page}{3}
\setcounter{巻数}{3}
\setcounter{号数}{3}
\setcounter{年}{1996}
\setcounter{月}{7}
\受付{1995}{6}{19}
\再受付{1995}{8}{11}
\採録{1996}{2}{20}


\title{日本語の述部階層構造に基づく形態論的な文法規則の記述法}
\author{佐野 洋\affiref{TOSIBA} \and 福本 文代\affiref{YAMANASI}}

\headauthor{佐野 洋・福本 文代}
\headtitle{日本語の述部階層構造に基づく形態論的な文法規則の記述法}

\affilabel{TOSIBA}{東京外国語大学 外国語学部}
{Faculty of Foreign Studies, Tokyo University of Foreign Studies}
\affilabel{YAMANASI}{山梨大学 工学部}
{Faculty of Engineering, Yamanashi University}

\jabstract{
自然言語処理システムに求められている分析性能が向上するにつれ
て,そのシステムで用いる文法規則や辞書データといった言語知識
ベースも複雑化,巨大化してきた.一方,自然言語処理システムを
用いる応用分野がますます多様化することが予想され,応用分野ご
とに新たな分析性能が要求される.言語知識ベースにおいても追加
と修正の作業が発生する.しかし,現状では,その開発には多数の
人員と多くの時間を必要とするため,言語知識ベースの再構築は困
難な作業である.応用分野に適合するシステムを,より効率的に開
発する手段が必要である.そのためには,融通性を持ち容易に修正
できる文法規則や辞書データの作成技法と,作成された言語知識ベー
スの保守性の向上を図る必要がある.この課題は,応用分野の多様
化に伴う需要と規模が増大する中でますます重要となっている.
この稿では,この技術課題に対して,言語知識ベースのうち,文法
規則の系統的な記述の方法を提案し,その方法に従って作成した機
械処理を指向した文法規則について述べる.まず,形態素と表層形
態の概念区分をした上で,日本語の持つ階層構造に注目した.形態
素の述部階層位置との関係から,表層での形態の現れ方を構文構造
に結び付ける形態構文論的な文法作成のアプローチを採用し,文法
規則の開発手続きを確立した.融通性を持ち容易に修正できること
を例証するため,試作した文法規則を新聞の論説文の分析に適用し,
分析の出来なかった言語現象を検討した.そして,その言語現象を
取り上げて,これを新たな分析性能を満たす要求仕様と見なし,同
じ手続きを用いて文法規則を拡張した.この結果,拡張した文法規
則の分析性能が漸増していることを確認した.
系統的な記述の手続きに従うことによって,文法規則記述の一貫性
を維持しながら,その分析性能を向上させることが可能となった.
このため工学上,文法規則の開発作業手順に一般性が生じ,開発時
間を短縮することができる.試作した文法規則は実際に計算機上に
実装している.本稿は機械処理を指向した文法規則記述のノウハウ
を体系化する試みとして位置づけられる.
}

\jkeywords{日本語文法規則,文法規則の設計知識}

\etitle{A Method for describing Japanese Grammar Rules\\
  \hspace*{8mm}based on Hierarchical Structure of Predicate}
\eauthor{SANO Hiroshi\affiref{TOSIBA} \and FUKUMOTO Fumiyo\affiref{YAMANASI}}

\eabstract{
As the analysis performance of the Natural Language
Processing (NLP) systems has improved, grammar rules and
lexicons to be used with NLP systems become complicated and
large than before.  On the other hand, it must be expected
that new application filed using a NLP system is enlarging.
The underlying task is to develop and modify the NLP systems
with reconstruction of the grammar rules and lexi- cons that
has the desired performance.
Since the work needs many staffs and much time for the
development, reconstruction of grammar rules and lexicons is
hard work and also expensive.  Methods to develop more
efficiently the desired system which can be applied to new
application filed are required.
In this paper, a new method of a systematic description of
grammar rules is proposed for the purpose of creating
grammar rules and modify them. The way of abstracting
grammar rules from linguistic phenomena has been studied
with special reference to the formation of morph-syntactic
structure of Japanese. Based on the generalization of the
way, the new method was made for further improvement for
grammar writing.
The paper also describes the grammar rules created by using
the proposed method to illustrate the effectiveness of the
method. The grammar rules was applied to analysis of a set
of newspaper editorial sentences.
First, the linguistic phenomenon that the analysis was not
completed were specified and extracted. We considered that
the phenomenon were a demand specification of meeting a new
analysis performance. According to the specification, the
grammar rules were extended using the same procedure of the
method. Next, we checked that the analysis performance of
the extended grammar rules was increasing.
The result shows that it becomes possible to raise the
analysis performance, maintain consistency of grammar rule
description by following the method through the experiment. 
Generality raises for development work procedure of a
grammar rule on engineering. Development time can also be
shortened.
The grammar rules described as an experiment runs on
computer and are published from ICOT(Institute for New
Generation Computer Technology). The proposed method in the
paper is considered to be a trial which systematizes
know-how of description of grammar rules as a knowledge
base.
}

\ekeywords{Japanese grammar rules, Design knowledge of grammar writing}

\begin{document}
\maketitle


\section{はじめに}


自然言語処理技術は,単一文の解析等に関しては一定の水準に到達
し,文の生成技術を統合して幾つかの機械翻訳システムが商用化さ
れて久しい.このような段階に達した現在においては,従来,問題
とされてきた形態素解析や構文解析とは異なる以下のような課題が
現れてきている.


自然言語処理システムは求められる分析性能が向上するにつれて,
そのシステムで用いる言語知識ベース(文法規則や辞書データ)も次
第に複雑化,巨大化してきた.ひとたび実働したシステムも,利用
者が使い込むことによって既存の分析性能では扱えない言語現象へ
の対応に迫られる.利用者が増えるに従って新たな分析性能が要求
される.

一方,自然言語処理システムを用いる応用分野はますます多様化す
ることが予想され,応用分野ごとにも新たな分析性能が要求される.
言語知識ベースにおいても機能の更新が求められ,追加と修正の作
業が発生する.しかし,一般に言語知識ベースの開発には多数の人
員と多くの時間を必要とするため,その再構築にも手間を要する.
応用分野に適合するシステムを効率的に開発するためには,融通性
を持ち容易に修正できる文法規則や辞書データの作成技法と,作成
された言語知識ベースの保守性の向上を図る必要がある.この課題
は,応用分野の多様化に伴う需要と規模が増大する中でますます重
要となっている.


言語知識をコンピュータへ実装する過程での技術的な課題を論じた
研究~\cite{吉村,神岡,奥}がある.しかし,文法
規則の記述の方法やノウハウの開示が見られない.どのようにして
規則が見つけだされたのかという言語知識の構成過程の研究は少な
かった.前述のように,適用分野の多様化に応じて,文法規則の追
加や修正を整然と実現するには,文法規則の開発手続きを整理する
ことから取り組むべきである.具体的には個々の文法規則がどのよ
うな言語現象に着目して作成されたのか,そして,その記述の手段,
すなわちどのような手続きで規則化されたのかのノウハウを方法論
的に明らかにすることである.

本稿では,この課題への一解決策として,文法規則の系統だった記
述の方法を提案する.さらに,我々が提案した方法に従って作成し
た文法規則について説明する.まず,形態素と表層形態の概念区分
をした上で,日本語の持つ階層構造に注目した.形態素の述部階層
位置との関係から,表層での形態の現れ方を構文構造に結び付ける
形態構文論的な文法作成のアプローチを採用し,文法規則の開発手
続きを確立した.この文法規則は機械処理に適合した文法体系の一
つとなっている.その特徴は,(1) 系統だった記述法に則り作成さ
れたものであること,(2) そのため,工学上,文法規則の開発作業
手順に一般性が備わり,誰がどのように文法規則を作成するにせよ,
ある条件を満たすだけの言語の分析能力を持った文法規則を記述す
ることができる.

なお,もう一方の言語知識である辞書データについても,その知識
構成過程の把握が必要であるが,本稿では,特に文法規則について
のみ着目する.


以下の第\ref{文法規則の体系だった記述法} 章では,文法体系と文
法規則の具体化の方法について述べ,文法規則を体系的に記述して
ゆくための記述指針を提案する.第\ref{文法規則の記述の手順} 章
では,提案した手続きに従って記述した文法規則例を示す.新聞テキ
ストを用いた分析実験を通して,文法規則の記述の手続きの一貫性
を評価した.第\ref{記述手続きの評価} 章では,その詳細を報告す
る.

\section{文法規則の体系だった記述法}
\label{文法規則の体系だった記述法}

\subsection{文について}

文法規則は,言葉に内在すると見られる表現と解釈のためのきまり
である.現実に,我々が日々接する言語事実は,多種多様で加えて
複雑であるので,言葉のきまりを包括的に且つ網羅的に説明する文
法論は現在のところ存在しない.そこでまず取り扱う文の範囲を設
定する.{\dg 文}は,記述する叙述内容を表す部分とその叙述内容
に対する話し手の判断部分が表層の表現形式に現れている言語表現
とする.従って,「うそ.」,「文法規則の体系だった記述法」,
「そんなこと!」といった例に見るように叙述部分がなく,述語を
含まない言語表現は,本稿では文の範中\\に含めない.「彼,そこ行っ
た!(``彼がそこへ行ったのだ''の意味)」,「改革の流れ徐々に.
(``改革の流れが徐々に(Φ)''Φがどのような述語をとるのかは文
脈からしか判断できない)」のように不完全な叙述表現であるもの
も除外する
\footnote{文章表現は,一般に,描写文,物語文,説明文,説得文
に分類することができる.文章内容に基づいた分類である.文の叙
述の仕方の見方に立つと,文の形式の整え方で区分できる.この区
分によると本稿の扱う{\dg 文}以外の種類には,メモ,伝言,掲示,
広告,宣伝文などがある.論文などの専門的説明文に限ると,タイ
トル部分,図表などへの注釈,章立てのための表現,参考文献の記
載部分を除いた部分は文字数にして,その9割は本稿で扱う文であ
る.}.

\vspace*{-0.5mm}
\subsection{文法記述のアプローチ}

\vspace*{-0.2mm}
文法規則の記述には,背景となる文法論(文法の考え方)が必要であ
る.文法論を,文の構成要素が持つ外形とその結び付きの有様を探
究するものであるとした時,その構成法には,二つのアプローチが
あるといわれている\cite{森岡2}.一つには,文法論で扱う構文的
な機能を単語の語形にまで言及し体系を立てる方法(方法 1),もう
一つは,単語の語形とはそれほど密な関係を持ち込まずに体系を立
てる方法(方法2)である.こうしたアプローチの違いと語のとらえ
方の視点によって,その体系がどのような品詞を認め,どのような
単語を認定するのかに違いが生じる.品詞の種類や単語の認定基準
は文法体系の単語観に依存している.文法規則には,このようなア
プローチの違いに関する知識が関与しており,暗黙の知識として働
いている.


構成法の違いは文法規則の記述法の違いになって現れる.例えば,
外国人向けの日本語教育の文法(例えば\cite{吉岡})は,方法 1 に
沿って作られたものである.日本語を母国語とする初学者向けの文
法(例えば\cite {渡辺1})は,概ね方法 2 に沿っている例として挙
げることができる.


外国人向けの日本語教育の文法\cite{吉岡}では,日常会話の手段
として日本語の構造を説明することを主な目的とする.こうした文
法体系の特徴は,話し手の意志伝達や応答の仕方に注目しているこ
とである.すなわち文体\footnote{叙述の中心となる述語が文法カ
テゴリーに応じてその形を変えること.例えば,「だ」から「です」
「ます」に見られるような通常態から丁寧態への変化,あるいは「〜
する」から「〜しない」へのような肯定から否定への変化を指す.}
に応じて,どのように文の構成要素が外形と結びつくのか(文体に
応じた構文的機能)を重視する.例えば,表\ref{文体による活用形
の組織化の例} のような動詞の活用形設定が可能となり,「ます」
あるいは「た」といった語は助動詞ではなく動詞の活用形の一部と
なる.表現上の機能対立が「〜る」「〜た」のような表層の形の現
れ方にまで及んでいる.このように日常会話としての機能対立が顕
在化している文型を網羅的に調べてゆくことで,文法規則を体系的
に記述してゆくことが可能である.

\vspace*{-0.3mm}
\btb{文体による活用形の組織化の例}\small
\bt{|l|l|l|} \hline
\mltclm{1}{中核単位} & \mltclm{1}{活用形} & \mltclm{1}{日常会話上の機能} \\ \hline
食べ & (食べ)\,--\,る     & 主体の現在の意思表示の通常体   \\ \cline{2-3}
     & (食べ)\,--\,ます   & 主体の現在の意思表示の丁寧体   \\ \cline{2-3}
     & (食べ)\,--\,た     & 主体が確認した意思表示の通常体 \\ \cline{2-3}
     & (食べ)\,--\,ました & 主体が確認した意思表示の丁寧体 \\ \hline
\et
\etb
\vspace*{-0.3mm}

日本語を母国語とする初学者に対する文法\cite{渡辺1}では,表層
の形には比較的無関心に,抽象化した構文単位(例えば文節)を設定
して,その要素間の性質に基づいて文としての結び付きを調べる.
活用形の組織化(表\ref{助動詞との接続の基準による活用形の組織
化の例} を参照)を見ると,語形変化と構文的な機能との関連があま
りない.

\btb{助動詞との接続の基準による活用形の組織化の例}\small
\bt{|l|l|l|} \hline
\mltclm{1}{中核単位} & \mltclm{1}{活用形} & \mltclm{1}{助動詞との接続のし方} \\ \hline
食      & (食)\,--\,べ  & 「ない」「う」「よう」に連なる \\
\cline{2-3}     & (食)\,--\,べ  & 「ます」「た」に連なる \\
\cline{2-3}     & (食)\,--\,べる        & 言い切るかたち \\
\cline{2-3}     & (食)\,--\,べる        & 体言に連なるかたち 
\\ \hline
\et
\etb

このように,個々の文法論がそれぞれの立場を持ち,その立場の見
方によって言葉を分析することから,分析の対象となる言語を一つ
に限っても幾つかの文法論が存在する.すなわち文法規則の構成過
程は文法論の構成法に依存していると考えられる.

機械処理を考えた時,高度な推論機構や語彙の意味に,できるだけ
依存しないように文法体系を構成することが望ましい.表層の形の
違いができるだけ構文に則する方法を採る.日本語では,(1)殊に
述部にあって形態素の序列関係と文法属性に関連があること,(2)
いわゆる学校文法でいう活用の活用語尾に,「う」「よう」「まい」
などの無活用の助動詞を組み入れて再構成すれば,述部末尾の語形
変化を構文的な機能に結び付けることができることから,本稿では
方法1の作成法を採る.

次節では,方法1に従って機械処理に適した文法規則記述のアプロー
チについて述べる.

\subsection{文法規則の形態構文論的な作成法}
\label{文法規則の形態構文論的な作成法}


次に挙げる点に適うよう文法体系を構成する.

\smallskip

\begin{itemize}\baselineskip1.2em
\item 文の内容記述に関連する要素を外形に現れた形態素でとらえる
\item 叙述の時空間的な位置関係を外形に現れた形態素の中に見つける
\item 書き手の叙述の意図を形態素でとらえる
\end{itemize}

\smallskip

{\noindent 形態素はそれ自身で意味を担うことのできる最小の単
位\cite{森岡1}のことであり,語を構成する基本単位となっている.}
我々は,この形態素の表層での現れ方を重視し,単語の語形にまで
文法規則を関与させる立場で文法体系を構成する.本稿では,この
構成法を{\dg 形態構文論}と呼ぶ.文の意味を近似する上記三点の
特徴が,構文構造に関してどのように具体化されているのかに着目
するのである.

一般に,文の一部分が,着目する表現内容を維持しながら別の文形
に変わる場合,その変化した部分が,文法規則化の対象となる文型
特徴を担っている.従って,変化形態に対する形態素と構文構造の
直接的な関係の発見と,文型特徴である文法上の働きを抽象化する
過程が文法規則の作成過程となる.

この章では,以下に,構文構造に関与する形態素について述べたの
ち,文法記述の手続きを整理するための作成法の詳細を述べる.

\subsubsection{形態素の分類}
\label{形態素}

まず構文的意味を有する単位としての形態素を分類する必要がある.
森岡\cite{森岡1}は,語の構成単位を形態素として子細にその語構
成法を観察している.我々は,森岡の形態素分類に従いながら,文
の意味を近似する上述の三点の特徴について構文構造と形態素の関
係を調べた.そして森岡\cite{森岡1}の基本の分類に基づき,鈴木
\cite{鈴木}を参考にして言語現象の抽象化に機能的に働く形態素
をまとめた.付録の図\ref{形態素の分類一覧} に形態素の基本分類
を示し,付録の表\ref{言語現象の抽象化の手段に対応する形態素} に
言語現象の抽象化の手段に対応する形態素を示す.


\subsubsection{文構造の階層性とその利用}
\label{階層}

構文構造と形態素の直接的な関係から文法規則の構成上の情報を得
ることを基本とする.しかし,文の構成を担う手がかりを,表層の
形態特徴ばかりに求められないことが下記の例からわかる.


\smallskip

\begin{description}\baselineskip 1.2em
\item [(1)] 「学生は英語教育を求めていない」
\item [(2)] 「学生は英語教育を求めている」
\item [(3)] 「学生は英語教育を求められている」
\item [(4)] 「学生は英語教育を求めている」
\end{description}

\smallskip


{\noindent 否定文 (1) に対応する肯定文 (2) と,受動文 (3)に
対応する能動文 (4) のうち(2)と(4)は同じ文形である.}肯定文と
能動文の形態上の弁別ができない.

次に,(1),(2),(4) 例の「学生は」の「は」は格表示機能を有し,
主格の語を現している.一方,(3) 例の「学生は」では,

「(Aが)学生{\dg に}英語教育を求めている」

{\noindent という意味で「学生」が対格を示すために,「は」
が利用されている.}このように構文的機能を有する形態素は文法
機能に関し一対多の対応関係を持つことで多価値となっている.

我々は,文法体系の作成に意味知識や推論機構を前提しないことを
既に述べている.「求める」「英語教育」「学生」といった語彙毎
に意味知識を持ち込まずに,「は」を伴う語が主格になるか,もし
くは対格になるのかを表現し分けなければならない.多価値の形態
素を,言語現象の抽象化に応じて構文機能に正しく結び付けるには,
何らかの表示デバイスが必要になる.その働きは構文構造と形態素
の直接的{\dg でない}関係を補完することにある.

その対策として,我々は,現代日本語文法の研究成果~\cite{日本
語1,日本,渡辺,芳賀,寺村1,山口,南,佐伯} から得られている文の
段階性(階層)を利用する.文の段階(層)構造を本稿での文の定義に
従って変更した.図\ref{述部の階層構造(分析に用いる構造)} に文
の段階(層)構造を示す.

\bfg
\vspace*{0mm}
\epsfile{file=kaisou.eps}
\vspace*{-0.1mm}
\CAPLA{述部の階層構造(分析に用いる構造)} \vspace*{3mm}
\vspace*{-0.3mm}
\parbox{100mm}{\small 日本語の述部には,図に示す層状構造があ
るとされる.格の階層は,文の叙述に論理的な関係を設定した際に,
その論理関係の関与する要素が含まれる階層である.例えば,「A
ガ B ヲ食べ(ル)」という述語があるとする.この場合,「食べる」
という述部が中心語となり,「Aが」と「Bを」という2つの格要
素が認められる.これらの要素は中心語に依存する.ヴォイスの層
は,論理関係として関与する格要素が,どのような見方によって叙
述されているかを示す.上の例では,Bに焦点を当てることで「B
が(Aに)食べ{\dg られ}る」という外形の特徴が現れる.依存関係
は格関係と変わりはないが,述部の形が形態素によって変わるので
特徴抽出が可能となる.アスペクトの層は,叙述の時間の捉え方に
関わる部分である.叙述全体を記述するのか,あるいは叙述の時間
的な変化のある部分を捉えて記述するのかを表現仕分ける層である.
ムードの層は,叙述の時間表現に関わる部分であって,その叙述が
完了したことなのか,あるいは未完了の出来事かを表現仕分ける.
モダリティの階層は話し手の叙述の意図が示される.}
\end{figure}

\vspace*{-1mm}
\subsubsection{形態素と形態}
\label{形態素と形態}

\vspace*{-2mm}
図\ref{述部の階層構造(分析に用いる構造)} に示す階層で不整合な
く文法機能の働きが形態との対応で明示できるように,我々は,層
内にあって構文的機能を有する形態素の働きを決めることとした.
テンスとかアスペクトなどの構文機能は,抽象的なものであるから
常に表層文の形が対応するとは限らない.そこで,構文機能を有す
る形態素が表層に現れない場合には,形態素のインスタンスとして
働くことのできる形態という単位を導入する.形態を持った,表層
には直接現れない形態素を認識的な形態素と称する.形態を介して
形態素と構文機能を直接的に結び付ける.我々は,必ずしも形態素
が表層上に現れなくとも文を形態素連鎖として扱うことができると
仮定している.文の意味を近似する構文的意味に結びついて,階層
内で配置される位置に応じた文法機能があるものと考え,同じ形態
素が複数の構文機能を担うことはないものとした.

例文(2)は階層構造表示をすると次の構造となる.

\vspace*{-1.5mm}
\begin{center}
  
 \epsfile{file=rei.eps}
  \vspace*{-2.5mm}
  \figcap{例文の階層構造}{例文の階層構造}
\end{center}

図2 において,(る)は,アスペクト表現の(ル:
未完了)とムード表現の(ル:叙述)のふたつの構文機能を担って
いる.表層形態の「る」は,本稿の分析では,一般にいわれる形態
素ではなく形態である.多価値とされる従来の形態素は,構文機能
を担う認識的な形態素(この例では,アスペクト形態素とムード形
態素)が構文機能を実現するために生じた表層の現れ,つまり形態
\footnote{例えば,{\em go} に対してテンス形態素が機能的に働
くとその表層の形態は,{\em went} となる.形態素解析とは表層
の文字列を単語に区切るだけはない.構文機能を実現する形態素を
見つけることにあり,それが認識的なものであった場合には,形態
を見つけることにある.}であるとみる.

構文機能が同じであるにもかかわらず形態が違っている場合がある.
このように違う形態が,同じ文の階層に属する時,それらは異形態
\footnote{形は違うが同じ意味を持つ形態.「食べた」と「読んだ」
における「た」と「だ」は異形態と呼ばれる.形は違っていても,
動詞で示される叙述内容が同じ過去・完了であるという意味を表し
ている.文の表層の文字の並びは単なる形態素の連鎖ではなく,形
態素と形態,もしくは,周りの言語環境によって形態が変動した異
形態の並びからなる.}であるという見方をとる.こうした分析に
より従来,多価値とされた形態素の文内での働きを正しく捉えるこ
とができる.この形態構文論的な作成法により,言語現象の抽象化
に機能的に働く形態素に曖昧性がなくなり,構文構造と形態素の間
に直接的な関係を設定することが可能となる.この直接的な関係の
一覧が文法枠組に対応する.線状に並ぶ形態素の,形の違いと相互
連鎖の仕方にだけ注目すると曖昧となる言語現象も,文に階層構造
を仮定することで,それを分析するための文法規則を作成できる.


本章では,構文構造と形態素の直接的な関係を抽象化する構文形態
論的な文法の考え方について述べた.次に,この枠組に基づいた文
法規則の記述の方法について述べる.

\section{文法規則の記述の手順}
\label{文法規則の記述の手順}


一般に文法規則の記述では,文中で意味を担う形態素に結びついて
繰り返し現れる機能形態素を利用する.形態構文論的な作成法では,
文法規則を記述する作業は,文に内在するとみられる階層のそれぞ
れの位置に,言語現象を抽象化するために文に繰り返し現れる形態
素を配置することである.我々は,この手続きを整理することで体
系的な文法記述の手順を得た.

\subsubsection*{構文的機能を有する形態素の認定 (P1)}

語彙には実質的な内容面と文法的な機能面が備わっている.意味が
似通っている語彙の語形の変化の様子を調べる.その様子から内容
面が変化しても繰り返し起こる形に着眼する.その形を,集めた語
彙に共通する文法面の機能を担う形態素とする.

\subsubsection*{文の階層性の利用 (P2)}

共通する文法面の機能が複数(形態素が多義)の場合には,文の階層
構造を利用する.出現する階層の違いに文法機能の違いを対応させ
ることによって同じ形態に異なる文法規則を割り当てる\footnote
{例えば,動詞の終止形である「る」は,テンスを表現するとも,
もしくはアスペクトを示す機能を有しているともいわれている.P2 
の手続きはこのような現象に対処するものである.この例では,テ
ンスを示す形態素とアスペクトを担う形態素がそれぞれあり,表層
上,不可避的に同じ形態を共有しているとみなす.構文的には両者
は違う階層で機能するものとして文法規則を作成する.}.

\subsubsection*{認識的な構文的機能を有する形態素の認定 (P3)}

文法機能上,外形や階層に共通する形態特徴が現れない場合には,
文の表層での現れが{\dg 見えない}形態素を設ける.次に,この形
態素に構文機能を割り当てる.この構文機能は言語現象の抽象化に
対応する必要がある.形態素と表層の形態との対応をみいだす.周
りの言語環境によって形態が変化した異形態があればそれを見つけ
る.

\subsubsection*{依存関係を定める (P4)}

依存関係とは,文の構成要素が文階層のどのレベルで,語彙の実質
的な内容面と結びついているかによって表現する.基本的に,修飾
要素は,被修飾要素に依存する場合,修飾要素の最も外郭の階層と
同じ階層位置に依存する.係り受け関係に相当する.例えば,「の
で」「のに」などの接続助詞は,アスペクト層までを含む述語の語
形変化の中で繰り返し現れる形の形態素である.連用修飾句として,
主節に依存する場合,主節のアスペクト層部分に依存する.これを
模式的に表したものが図\ref{依存関係の模式図1} $\sim$
\ref{依存関係の模式図2} である.

\bfg
\epsfile{file=izon1.eps}
\vspace{0.5mm}
\efg{依存関係の模式図1}

述語は,それが表現する動きを成り立たせる上で,構文上選択的に
必要としている要素がある\cite{仁田}.図\ref{依存関係の模式図
1} に示す例では「Aガ B ヲ求め(ル)」である.図\ref{依存関係の
模式図1} (a)の名詞句「英語教育を」は,機能形態素「を」でマー
クされているから,述部の階層構造(図\ref{述部の階層構造(分析
に用いる構造)})の格要素を含む階層と依存関係を構成する(図
\ref{依存関係の模式図1} の(b)).

こうした用言に内在する論理的な関係構成に関わる要素は,叙述に
用いる語彙の性質によって決まる.同時にその要素が語形としてど
のような形態素を取り得るのかも語彙に依存して決まる.この語彙
性質は,予め辞書に記載しておかなければならない.

\bfg
\epsfile{file=izon2.eps}
\vspace{0.3mm}
\vspace*{-0.5mm}
\efg{依存関係の模式図2}


図\ref{依存関係の模式図2} は用言に後接する「い(る)」の依存先
を示している.図\ref{依存関係の模式図2}(a)において,「英語教
育を求めている」の機能形態素「い(る)」は,用言で示される出来
事の時間的な継続状態を示す(表\ref{言語現象の抽象化の手段に対
応する形態素}).この形態素は,動詞が示す動作の一局面を表現し,
アスペクトという文法機能を表現する形態素である.従って,述部
の階層構造(図\ref{述部の階層構造(分析に用いる構造)})のアスペ
クト部分に依存し,図\ref{依存関係の模式図2} の(b)に示す依存関
係が成立する.依存関係は機能形態素に前接する形態素(語基)と文
階層の性質から決まる.

図\ref{文法規則の記述の流れ} は,上記 P1 $\sim$ P4 の手順の,
適用する順序と条件を,記述手続きの流れとして示したものである.

次節では文法規則の作成例を挙げて記述手順を具体的に説明する.


\bfg
\vspace{0.3mm}
\epsfile{file=flow.eps}
\vspace{0.7mm}
\vspace{1mm}
\efg{文法規則の記述の流れ}

\subsection{文法規則の組み立て}
\label{文法規則の組み立て}

本節では,前節で示した文法規則の記述方法に従って,実際に文法規
則を作成した例を示す.

\vspace*{-4mm}

\subsubsection{文の分析のための規則}

\vspace*{-1mm}

{\dg 文}分析のための規則を作成する.

日本語は,主要素が文末に置かれる性質があるので,文末の形態素
もしくは形態に着目してみる.

\begin{enumerate}
\item 構文と直接関係を有する形態素を探す(P1)

話し手の判断にあたる部分を,文法機能として話し手の意図とする.
ヴォイス・アスペクト・ムードなどの文法機能と同様に扱い,意図
は文の記述の内容に無関係に,様相の階層で働く構文機能とする.
次に様相に対する形態素を設定する.

「与える」を言語資料とし,語形変化の様子を調べるため,文末で
の終止の形を例に挙げる\footnote{上段は比較的話し手の強い意図
表現が示される語形で,下段はそれ以外の語形を集めた.}.

\vspace{-2mm}
\btb{用言の文末終止の型の例}\small
\bt{ll|l|l}
 & (1) & (2) & (3) \\
 & 与える       & 与えるのだ            & 与えてほしい \\
(上) & 与えた   & 与えたのだ            & 与えてほしかった \\
 & 与えるだろう & 与えるはずだ          & 与えるべきだ \\
 & 与えただろう & 与えたはずだ          & 与えておくべきだ\\ \hline
 & 与えよう     & 与えますか            & 与えなさい \\
(下) & 与えまい & 与えましたか          & 与えねばならない \\
 & 与えろ       & 与えるのか            & 与えようじゃないか \\
 & 与えるな     & 与えたのか            & 与えるだろうねぇ \\
\et
\vspace{-0.5mm}
\etb


様相という構文機能に対応する形態素は,話し手の判断であるので,
肯定や否定,過去の認定や推量といった中立的な判断の意味を表す
と考える.願望とか疑問,ある根拠に基づくことを示唆する命令調
の判断や意志といった意味を示す形態素は,様相に応じないので文
の終止の形として認められない.

表\ref{用言の文末終止の型の例} で,中立的な判断の意味を表すの
は (1)列上段であり,「〜る」「〜るだろう」のような話し手の意
図が比較的中立かあるいは根拠の曖昧な推量の表現である.こうし
た意図表示の希薄な表現は基本の文型としてよいだろう.(2) 列は,
文内容の叙述に対する話し手の態度が形態に如実に現れていること
が分かる.「のだ」「はずだ」といった話し手の強い認定態度を表
現した推量と「ますか」「のか」のような意図表示が強固な疑念の
表出となっている.(3)列は,直接的な話し手の意志や願望が示さ
れている.

(1)の下段の表現は意図の表現と依頼の表現である.その表す意味
を見ると記述事態の時間的な把握,意図表現や依頼表現が対で並ん
でいる.同じ意味を表す他の言い替え表現がないことから,(1)の
下段は意図表示の表現の基本の形である.結局,(1)列を基本的な
文終止の文型であるとする.

\item 形態素の働きは?

規則記述の手順を示す図\ref{文法規則の記述の流れ} の P1 の手続
きによって様相という文法機能に対する形態素を仮定した.表の
(1) 列に挙がる一連の形態は文の終止機能(構文機能との直接関係)
を有するものの,「与え」を除いては共通する形態素がみつからな
い.手順 P3 に進む.

\item 認識的な形態素を仮定する(P3)

様相を示す形態素を$\varphi$様相(**) \footnote{$\varphi$ は``
見えない''形態素であることを示し,``**'' は,形態素に対する
構文機能が未定であることを示す.例えば,テンスが{\dg 過去}と
か{\dg 現在}といった具体的な構文機能を持っているように,構文
と直接関係を有する形態素には機能に見合った値を与えることがで
きる.}として,表層文字列との対応をとる(図\ref{様相の分析手
順}(a)).

\bfg
\epsfile{file=izon3.eps}
\CAPLA{様相の分析手順}
\vspace*{3mm} \parbox{100mm}{\small (a)図は,認識的な形態素を
仮定した段階であり,構文機能は未定である.形態との結びつきが
明確でない.(b) 図は,構文機能を定め,表層の形態との対応をとっ
た状態を示している.}
\end{figure}

\item 構文と直接関係を有する形態素を探す(P1)

(1)列で,[与え](述語自身が表現する叙述内容)を除くと,「る
(た)」,「るだろう(ただろう)」,「よう(まい)」,「ろ(るな)」
が話し手の判断を示す形態である.そこで様相に具体的な構文機能
を与える.様相を示す形態素に,それぞれ$\varphi$様相(平叙),
$\varphi$様相(推量),$\varphi$様相(意志),$\varphi$様相(依頼)
という文法機能を示す値を割り当てる.「た」は「る」の異形態と
し,「ただろう」,「まい」,「るな」についても同様とする.

\item 形態素の働きは?

文終止の様相に対する構文機能と形態素を結び付けることができた
(図\ref{様相の分析手順}(b)).


\item 依存関係を求める(P4)

階層関係を基に依存構造を求める.様相は階層構造では外郭にある
から文という最上位の階層との直接依存関係を構成することになる.

\end{enumerate}

図\ref{文法規則の記述の流れ} の P1--P3--P1--P4 の手順で文法規
則を作成することができる
\footnote{簡便のため,様相に関する文法規則の作成事例を挙げた.
話し手の意図に関する$\varphi$様相(平叙) 形態素は,「る」と
「た」という形態に対応している.このいずれの形態も文を終止す
る要件は満たしているものの,「る」,「た」はテンスを示してい
るともされ,様相以外の文法機能をこの形態が表している.そこで,
文の階層構造である図
\ref{述部の階層構造(分析に用いる構造)} を当てはめてみる.その
結果,「る」,「た」の形態はムードの文法機能を担って形態とし
て現れていることが分かる.また,ほかの形態,例えば,依頼を示
す形態「ろ」,「るな」は,意志否定(認め方)の文法機能も同時に
担っている.最終的には,P1 -- P3 -- P1 -- P2 -- P4の手順に沿っ
て文法規則を作成することになる.その結果を次の図に示す.

\begin{center}
\epsfile{file=izon4.eps,height=60mm,width=90mm}
\end{center}
}.

\begin{figure}[p]
\input{fig2.tex}
\bigskip
\caption{文法体系構図}
\label{文法体系構図}
\end{figure}

\vspace*{-0.5mm}
\subsubsection{文法規則}
\vspace*{-0.2mm}

紙面の都合上,一例を挙げるに止めるが,我々はすでに中規模の文
法規則を作成している.この文法体系の全体構図を図\ref{文法体
系構図} に示す.個々の規則を逐一挙げることも紙面の関係から不
可能なので,文の階層図を用いて示している.文の階層のそれぞれ
の位置にどのような構文要素が依存するのかを示した.前章までに
示した手続きによって作成された規則は,DCG\cite{fernando}を用
いて記述されており,約700余りある.付録の表\ref {文法規則の
種類と数(1)},表\ref{文法規則の種類と数(2)} には,文法規則の
種類とその数の一覧を示す.実規則の一例として,例で挙げた文分
析の規則を付録の図\ref{文法規則例} に示す.

文を分析すると構文構造が得られるが,この構造は述部の階層構造
を基本として,文の構成要素が文階層のどの位置に属するのか
(依存構造)を示す.形態素によって決まる構文機能を依存関係とし
て利用することで係り受け関係と見なすこともできる.構文的な性
質のうち述語自身が選択的に要求する要素に関する(格の階層に属
する)情報は,語彙ごとに違うので予め辞書に記載しておく必要が
ある.用言を中心とした規則の他に,体言を中心とした連体句,副
詞を扱う連用句に関する規則がある.


\section{記述手続きの評価}
\label{記述手続きの評価}

本章では,文法規則記述の手続きの一貫性を評価する実験とその結
果について述べる.


\subsection{方法}

まず,前章において作成した文法規則を第一版とし,その分析能力
について実際の新聞の論説文を用いて分析を行なう\footnote{分析
対象の文章は,平成 4年 11 月 10 日から 21 日までの朝日新聞社
説,ならびにコラム「窓」からそれぞれ 8 編ずつを選んだ.}.分
析できなかった事例を収集し言語現象ごとに分類を試みる.その中
からあるカテゴリーを選び,これを仮に応用分野で求められる分析
性能の向上要求と定める.次に,この要求を満たすよう,本稿で提
案した記述の手続きを用いて文法規則を追加した.これを第二版と
し,再び同じテキストを用いて分析を行い規則作成手順の有効性を
検討した.

\subsection{第一版の文法規則による分析}

分析において次の条件を与えた.(1) 辞書項目は全て与えられてい
るものとし,(2) 曖昧性を考慮せず句点や記号等を含む全文字列を
文の構成要素とする,(3) 複数の名詞連続からなる複合語はないも
のとし一つの単語とみなすことで,複合語の分析を文法規則の守備
範囲から外した.表\ref{実験結果(1)} の分析実験の結果を得た.

\btb{実験結果(1)}\small
\bt{|r||r|r|r|r|} \hline
  & \mltclm{2}{コラム「窓」} & \mltclm{2}{社説} \\ \hline
文章  & \multicolumn{1}{c|}{文数} & 解析率(\%) &
        \multicolumn{1}{c|}{文数} & 解析率(\%) \\ \hline
1 & 16(17) & 6.3          & 29(29) & 24.1 \\
2 & 17(19) & 41.2         & 39(40) & 43.6 \\
3 & 17(22) & 41.2         & 26(28) & 26.9 \\
4 & 11(15) & 45.5         & 31(32) & 29.0 \\
5 & 19(21) & 47.4         & 14(18) & 14.3 \\
6 & 10(13) & 10.0         & 30(31) & 30.0 \\
7 & 19(27) & 57.9         & 34(34) & 32.4 \\
8 & 13(14) & 15.3         & 24(27) & 29.2 \\ \hline
平均 & 15.3(18.5) & 33.1  & 28.4(29.9)   & 28.7 \\ \hline
\et
\etb

表\ref{実験結果(1)} は,社説とコラムについて,文章中に含まれ
る文数と解析率を一覧にしたものである.解析率は,曖昧性を考慮
しないで解析に成功した文の全文章に対する比率である.

なお,我々が定義した文の範囲外にある言語現象を含む文は予め分
析対象から外した(表\ref{分析対象外言語現象}).文数を示す欄の
括弧内が元の文数である.除外した言語現象として,名詞や記号が
連接する説明文章特有の記述,記号の組合せによる慣例的な表現,
あるいは簡易表現による指示表示がある.コラム文章では体言止め,
副詞句止めなどの修辞用法が使われている.

\btb{分析対象外言語現象}\small
\bt{|l|r|l|} \hline
\mltclm{1}{言語現象} & \mltclm{1}{事例数} & \mltclm{1}{用例} \\ \hline
形態連接               & 25 & 「$\cdots$」「$\cdots$」,「$\cdots$」($\cdots$) \\
体言止め        & 11 & ``教科書倉庫.'',\\
副詞句止め       & 1 & ``温泉旅行に.'' \\
その他         & 4 & ``$\cdot\cdot\cdot$.'' \\ \hline
\et
\etb

8つのコラム文章について解析できない用例を表\ref{第一段階で分
析できない言語現象} に示す.事例数は 8 つの文章を対象とした分
析不可の原因数である.一般には一文の中に複数の原因が存在する.

引用が高い頻度で現れている.鍵括弧で囲まれた文字列全体(複数
文の場合もある)が引用されている例もある.「の」による名詞化
現象では,名詞句全体が助詞を伴う場合が 25例(「の」による名詞
化1)と多い.形式名詞による補文化の例は 4件(形式名詞を使う名
詞化1)と少ない.逆に形式名詞で名詞化されたものがモダリティ機
能を兼ねて働くこと(形式名詞を使う名詞化2)が多いことが特徴と
なっている.連用化は「ように」「ために」「ほど」などの形式名
詞が従属節部分で機能する言語現象である.第一版の文法規則では
接続詞を扱っておらず,この規則を欠くことによる解析率の低下も
大きい.

\btb{第一段階で分析できない言語現象}\small
\bt{|l|r|l|} \hline
\mltclm{1}{言語現象} & \mltclm{1}{事例数} & \mltclm{1}{用例} \\ \hline
文の引用               & 40 & 告訴すると,「$\cdots$」などと \\
名詞句の引用           & 5 & ``肝心,という'', \\
「の」による名詞化1    & 25 & 開いたのが,実現させたのを \\
「の」による名詞化2    & 6 & ``$\cdots$のだろう'',\\
形式名詞を使う名詞化1  & 4 & 認めさせたことも,存在するかもしれぬことを \\
形式名詞を使う名詞化2  & 20 & 薄くなるばかりだ,いうほかない \\
連用化                 & 20 & 指摘したように,住んだ挙げ句 \\
文末のモダリティ       & 11 & してはならない,``$\cdots$ではいられない'' \\
接続詞                 & 13 & だが,それにしても,しかし \\
活用変形               & 7 & あわず,飽きたらず \\
述部内派生現象         & 1 & して\underline{もら}う \\[1mm] \hline
\et
\etb

\subsection{適用分野への模擬的拡張}

前節で示した幾つかの言語現象を取り上げ,その言語現象を,ある
アプリケーションが要求する拡張仕様と見なす.その仕様を満たす
ように文法規則を拡張する.第\ref{文法規則の記述の手順} 章で提
案した手順に従って文法規則化を進める.

表\ref{第一段階で分析できない言語現象} を参考にすると,「の」
による名詞化1と形式名詞の名詞化2,ならびに文末のモダリティに
関する文法規則を新たに作成することで分析範囲が広がることが予
想される.具体的には,(a)書き手の意図表現と,(b)「の」によ
る名詞化された表現に対応できるように拡張する\footnote{失敗の
事例数の点からは「文の引用」の解析規則を追加することで,効率
良く解析率を向上させることができる.本節では,本稿が主眼とす
る文法規則の作成手順(図\ref{文法規則の記述の流れ})の説明の点
から,幾分意図的であるが上記2例を取り上げた.}.

\subsubsection{文法規則の拡張 1}
\label{文法規則の拡張1}

(a) について文法規則の作成手順(図\ref{文法規則の記述の流れ})
を適用してみる.「してはならない」「である」「わけだ」のよう
に,モダリティ\footnote{この場合,叙述内容に対する書き手の判
断様相.様相と同じモダリティの階層に属する.}を示す形態が雑
多である.そこで認識的な構文機能を有する形態素($\varphi$判断
様相(**)
\footnote{(**)は判断様相という形態素に対応する具体的な文法機
能の名前である.例えば,「である」だと`断定'である.}とする)
を仮定する.

「わけ」「ばかり」「の」などの形態が,文末のモダリティ表現の
みならず,名詞化にも関わっていることから,形態素の多価値の問
題を解消する必要がある.そこで文の階層構造を利用し,形態素
($\varphi$判断様相)が属する階層をモダリティの階層に設定する.
図\ref{文法規則の記述の流れ} の P1 -- P3 -- P1 -- P2 -- P4 の
流れに従うことで,$\varphi$判断様相(**)という形態素の出現位
置ならびに,文に現れた異形態と構文機能を特定する文法規則が出
来上がる.追加される文法規則は,扱おうとする文末の判断様相に
関わるモダリティ表現の数に等しい.ここでは,表\ref {第一段階
で分析できない言語現象} の文末のモダリティの項目に現れた形式
的な名詞の分析をカバーするだけの規則数を追加する(規則数13).

\subsubsection{文法規則の拡張 2}
\label{文法規則の拡張2}

(b) については,「の」を形式的な体言に所属する助辞と考え,先
に挙げた図\ref{文法体系構図} の中の述部からなる連体修飾句に関
する文法規則を応用することで拡張が可能である.図\ref{文法規
則の記述の流れ} における,(3)の場合に相当する.名詞化という
文法機能は「の」だけでなく「こと」によっても実現されることか
ら,名詞化を担う単一の形態素が見あたらない.$\varphi$名詞化
(**)という認識的な構文機能に応じる形態素
\footnote{この形態素は階層図ではアスペクトの層に属する.}を設
定し,表層で具現した形態を「の」とする.そしてその異形態を
「こと」とする.図
\ref{文法規則の記述の流れ} において P1 -- P3 -- P1 -- P4 の手
順で規則を作ることができる.文法規則は,述部からなる連体修飾
句と$\varphi$名詞化(**)の依存関係,実質的には「の」,「こと」
との依存関係から構成できる(規則数11).

\subsection{第二版の文法規則による分析}


\subsubsection{分析の結果}

上述の手続きで作成した文法規則を加えた拡張版を第二版の文法規
則とし,この文法規則を使って,再び同じ資料に対して分析を行っ
た.その結果を表\ref{実験結果(2)} に示す.表\ref{第一段階で分
析できない言語現象} で示した項目の「の」による名詞化1と形式
名詞の名詞化2,ならびに文末のモダリティ表現に関わる言語現象
の分析が可能になった.

それぞれの分析対象の資料について,「拡張後の解析率」が示すよ
うに解析率が向上している.第二版の文法規則の分析能力が向上し
ていることを確認した.一連の実験から適用分野の要求仕様に応じ
て,図\ref{文法規則の記述の流れ} で示した文法規則作成の手続
きが繰り返し適用可能であることを確認した.

\btb{実験結果(2)}\small
\bt{|r||r|r|r|r|} \hline
  & \mltclm{2}{コラム「窓」} & \mltclm{2}{社説} \\ \hline
文章 & \mltclm{1}{拡張後の} & 解析率(\%) & \mltclm{1}{拡張後の} & 解析率(\%) \\
     & 解析率(\%) & & 解析率(\%) & \\ \hline
1 & 18.8       & 6.3        & 38.0       & 24.1 \\
2 & 52.9       & 41.2       & 56.4       & 43.6 \\
3 & 58.8       & 41.2       & 46.2       & 26.9 \\
4 & 54.5       & 45.5       & 48.4       & 29.0 \\
5 & 68.4       & 47.4       & 42.9       & 14.3 \\
6 & 20.0       & 10.0       & 53.3       & 30.0 \\
7 & 57.9       & 57.9       & 55.9       & 32.4 \\
8 & 38.5       & 15.3       & 45.8       & 29.2 \\ \hline
平均 & 46.2    & 33.1       & 48.4       & 28.7 \\ \hline
\et
\etb

\vspace*{-2mm}

形式名詞を使って叙述内容を書き手の判断の様相で締めくくる文章
スタイルが論説文の特徴であることから,社説を対象とした資料で
解析率の向上が著しいことがわかる.同機能の文法規則を追加して
も,文章の性質によって解析率の向上に違いがみられた.このこと
は分析対象となる文章に対して,その表現上の性質の違いに応じて
文法規則を選択的に適用することが,効率的な分析の実現につなが
ることを示唆している.

文法規則の分析能力は背景となる文法論に依存している.取り分け
拡張の可能性については,分析対象とする文を,文法論がどのよう
に定義するのかにかかわる.また,文法規則を拡張する時,その分
析能力の漸増性が問題になる.

一般に,文の一部分が,着目する表現内容を維持しながら別の文形
に変わる場合,その変化した部分が,文法規則化の対象となる文型
特徴を担っている.たとえば,例文(5),(6)は,「それを食べる」
という表現内容を維持しつつ,それぞれ違った意味を担っている.
「わけ」に導かれて文形が変化している.

\begin{itemize}\baselineskip1.2em
\item [(5)] 「君がそれを食べるわけなのだ」
\item [(6)] 「君がそれを食べるわけがわかった」
\end{itemize}

ここで,仮に「わけ」を形態素として,構文構造と形態素の直接的
な関係に基づいた抽象化を行い文法規則を作成する.

\smallskip

\begin{description}\baselineskip1.2em
\item [規則] 連体修飾構造の主名詞部分に「わけ」は位置する.
この場合,「わけ」は,・「…という次第」,・「理由・事情」の
意味がある.そして連体修飾構造を構成する述語部分と依存関係を
持つ.
\end{description}

\smallskip

{\noindent 上記文法規則によれば,形態素「わけ」に対応する構
文機能が2つ(それぞれ,・叙述に対する判断,・「わけ」による
名詞化)あるので,例文(5)(あるいは例文(6))を解析すると2つの
解析候補が得られる.}この曖昧さは,「わけ」によって表される
言語現象の抽象化が不十分なために生じたものである.一般に曖昧
さの解消は意味解析に委ねられることになるが,構文解析の段階で
精度を上げようとして,一方の言語現象に適合させて分析能力を調
整すれば,上記規則は基本的に2つの言語現象を分析対象にするか
ら,必然的に他方の言語現象の分析に調整の影響が及ぶ.この意味
で他と干渉する文法規則となっている.

我々が提案する形態構文論的な作成法では次に示す手続きで規則化
する.例文(5)の「わけ」が持つ意味「という次第」が示す構文機
能は書き手の意図であることから,前節(\ref{文法規則の拡張1}節)
で示した文法規則の拡張にあるように,$\varphi$判断様相という
形態素を認め,この形態素と構文機能に直接の関係を持たせる.こ
の言語現象の抽象化に機能的に働いている形態素が具体化した表層
の形態は「わけ」となり,その異形態として「ばかり」「はず」が
ある.

形態素($\varphi$判断様相)はモダリティの階層に属するから,例
文(5)の「わけ」に前接する「それが食べた」は様相表現までの文
法要素を含まなければならない(図\ref{連体修飾構造を構成する文
法要素} の(a)).これに対して,例文(6)の「わけ」が持つ意味
「理由・事情」が示す構文機能は名詞化であって,
\ref{文法規則の拡張2} 節で示した規則と同様である.この場合
の認識的な形態素($\varphi$名詞化)は,アスペクトの階層にある
ので,例文(6)の「わけ」に前接する「それで食べた」はアスペク
ト表現までの文法要素を含む(図\ref{連体修飾構造を構成する文法
要素} の(b)).それぞれの形態素は所属する階層が違い,従って,
形態素が機能する時の周囲の文法環境に違いが生じる.この違いが
その形態素と直接的な関係を持つ構文構造を解析する規則の適用制
限となって,例えば,例文(5)を解析する文法規則は,例文(6)の解
析には失敗する.逆の場合も同様である.

本作成法では,認識的な形態素の選定の妥当性を支持する表示デバ
イスに述部の階層構造を利用することで,互いに相反したり矛盾す
ることのない文法規則を作る手続きを確立した.それは構文機能に
結び付く形態素が正しく判断できていることが条件である.手続き
では,形態素の選択が正しい判断のもとに行なわれたか否かの指針
を与えることはできていない.図\ref{文法規則の記述の流れ} の破
線の四角で示す例外処理があるのはこのためである.

\bfg
\vspace{0.5mm}
\epsfile{file=kisoku.eps}
\vspace{0.5mm}
\CAPLA{連体修飾構造を構成する文法要素}
\vspace*{3mm} \parbox{100mm}{\small (a)図は,「君がそれを食べ
るわけなの(だ)」に対応する階層構造図である.破線は連体修飾句が含
む文法要素に関連する階層を示している.(b) 図は,同様に「君がそれ
を食べるわけが(分かった)に対応する階層構造図である.}
\end{figure}

線状に並ぶ形態素の,形の違いと相互連鎖の仕方にだけ注目すると
曖昧となる言語現象も,文に階層構造を仮定することで,それを分
析するための文法規則を作成できる.但し,形態素の曖昧性がこれ
だけに尽きるのではなく,修飾-被修飾の関係や音調の違い等をも
基礎にして曖昧さを解消しているようである.個々の言語現象の曖
昧性がそれぞれ何に起因しているのかを隅無く押えてゆくことが必
要だろう.









\section{おわりに}

この稿では,文法規則の体系的な記述方法を提示した.まず,形態
素と表層形態の概念区分をした上で,日本語の持つ階層構造に注目
した.形態素の述部階層位置との関係から,表層での形態の現れ方
を構文構造に結び付ける形態構文論的な文法作成のアプローチを採
用し,文法規則の開発手続きを確立した.融通性を持ち容易に修正
できることを例証するため,試作した文法規則を新聞の論説文の分
析に適用し,分析の出来なかった言語現象を検討した.そして,そ
の言語現象を取り上げて,これを新たな分析性能を満たす要求仕様
と見なし,同じ手続きを用いて文法規則を拡張した.この結果,拡
張した文法規則の分析性能が漸増していることを確認した.

これまでにも何らかの設計の指針を使って文法規則の開発は行われ
てきた.しかし,それは,基本的な文法の枠組みがあるとしても,
実際に文法規則を書くものの経験に基づく勘であったり,あるいは,
言語現象ごとに規則を演繹する場合も,この言語現象についてはこ
のような文法規則の書き方,ある言語現象についてはこの規則に類
似させる,というような体系性に欠けるものであった.

こうした経験的な方法や,言語現象に依存する方法は,手順が明確
でなくとも文法規則を記述してゆくことができるという意味で役に
立つが,新しい言語現象に対応する文法を記述してゆく一般的な方
法とはいい難い.文法規則を記述する際に,経験的な方法や言語現
象に依存する方法を使って,適用分野の変化に応じてその都度文法
規則を開発してゆくことは,コスト的にも,加えて文法規則の分析
能力の不安定さの点からも避けることが望ましい.



言葉は,分析対象が認識的なものであるために分析のために客観的
な方法論が適用されにくく,アプリケーションの多様化に対応する
客観的な文法規則の記述の手続きを求めることは困難な課題である.
本稿では一アプローチとして\,(1)\,文法規則の開発手続きを手順
化することによって展望を見いだそうとした.さらに,\,(2)\,そ
の手順に従った文法規則の作成の試み,\,(3)\,計算機上への文法
規則の実装による動作確認と文法規則の適用実験によって有効であ
ることを確認した.

これまで言語データは大学や企業内において収集が進められ,蓄積
も進んでいる.国家的なプロジェクトとしてデータの蓄積を進める
試みもある\cite{EDR}.しかしながら,そうした資料の資源保全に
ついての取り組みは具体例をみない.ニーズの多様化に伴い,他の
分野での言語データベースの有効活用を進めるためには,言語デー
タベースを再利用する技術の開発を推進する必要がある.

最後に,本稿で試作した中規模の文法規則は,そのすべてが公開さ
れている.個々の文法規則は DCG\cite{fernando} 記述のため,
Prolog の実行メカニズムをパージングの処理過程とすることがで
き,機械の種類に依存することがなく,Prolog の動作するいかな
る計算機においても利用が可能である.DCG についても LangLAB
\cite{徳永}ならびにSAX \cite{松本}といった無償公開ソフトウェ
アを利用することができる.そのためにパーサーを作る必要はない.


\acknowledgment

本稿に対してコメントをいただいた査読者に感謝する.


\bibliographystyle{jnlpbbl}

\newcounter{ichi}
\newcounter{ni}
\setcounter{ichi}{1}
\setcounter{ni}{2}


\begin{thebibliography}{[1]}

\bibitem[\protect\BCAY{吉村,武内,津田,首藤}{吉村\Jetal }{1989}]{吉村}
吉村賢治,武内美津乃,津田健蔵,首藤公昭 \BBOP 1989\BBCP.
\newblock \JBOQ 未登録語を含む日本語文の形態素解析\JBCQ\
\newblock \Jem{情報処理学会論文誌}, {\Bbf 30} (3).

\bibitem[\protect\BCAY{神岡,土屋,安西}{神岡\Jetal }{1989}]{神岡}
神岡太郎,土屋孝文,安西祐一郎 \BBOP 1989\BBCP.
\newblock \JBOQ 述語複合体の生成と表現\JBCQ\
\newblock \Jem{情報処理学会論文誌}, {\Bbf 30} (4).

\bibitem[\protect\BCAY{奥}{奥}{1990}]{奥}
奥雅博 \BBOP 1990\BBCP.
\newblock \JBOQ 日本文解析における述語相当の慣用表現の扱い\JBCQ\
\newblock \Jem{情報処理学会論文誌}, {\Bbf 31} (12).

\bibitem[\protect\BCAY{森岡}{森岡}{1984}]{森岡2}
森岡健二 \BBOP 1984\BBCP.
\newblock \JBOQ 文法論の構想\JBCQ\
\newblock 「国語学」, 136集.

\bibitem[\protect\BCAY{吉岡}{吉岡}{1989}]{吉岡}
吉岡武時 \BBOP 1989\BBCP.
\newblock 日本語文法入門. アルク.

\bibitem[\protect\BCAY{渡辺}{渡辺}{1983}]{渡辺1}
渡辺正数 \BBOP 1983\BBCP.
\newblock 教師のための口語文法. 右文書院.

\bibitem[\protect\BCAY{森岡}{森岡}{1987}]{森岡1}
森岡健二 \BBOP 1987\BBCP.
\newblock 語彙の形成. 明治書院.

\bibitem[\protect\BCAY{鈴木}{鈴木}{1972}]{鈴木}
鈴木重幸 \BBOP 1972\BBCP.
\newblock 日本語文法・形態論. むぎ書房.

\bibitem[\protect\BCAY{宮地}{宮地}{1983}]{日本語1}
宮地裕(編) \BBOP 1983\BBCP.
\newblock \JBOQ 特集\ 意味と構文\JBCQ\
\newblock 日本語学, 12月号, VOL.2, 明治書院.

\bibitem[\protect\BCAY{北原}{北原}{1981}]{日本}
北原保雄(編) \BBOP 1981\BBCP.
\newblock 日本文法事典. 有精堂出版株式会社.

\bibitem[\protect\BCAY{渡辺}{渡辺}{1974}]{渡辺}
渡辺実 \BBOP 1974\BBCP.
\newblock 国語文法論. 笠間書店.

\bibitem[\protect\BCAY{芳賀}{芳賀}{1979}]{芳賀}
芳賀やすし \BBOP 1979\BBCP.
\newblock 日本文法教室. 教育出版, 東京.

\bibitem[\protect\BCAY{寺村}{寺村}{1984}]{寺村1}
寺村秀夫 \BBOP 1984\BBCP.
\newblock 日本語のシンタクスと意味 第 \Roman{ichi} 巻.
\newblock pp.202--321, くろしお出版, 東京.

\bibitem[\protect\BCAY{山口}{山口}{1987}]{山口}
山口明穂 編集 \BBOP 1987\BBCP.
\newblock 国文法講座\ 6\ \ 時代と文法--現代語. 明治書院.

\bibitem[\protect\BCAY{南}{南}{1974}]{南}
南不二男 \BBOP 1974\BBCP.
\newblock 現代日本語の構造. 大修館書店.

\bibitem[\protect\BCAY{佐伯}{佐伯}{1983}]{佐伯}
佐伯哲 \BBOP 1983\BBCP.
\newblock \JBOQ 語順と意味\JBCQ\
\newblock 日本語学, 12月号, VOL 2.












\bibitem[\protect\BCAY{仁田}{仁田}{1988}]{仁田}
仁田義雄 \BBOP 1988\BBCP.
\newblock \JBOQ 「文の構造」\JBCQ\
\newblock 講座\ 日本語と日本語教育\ 第4巻, pp.25--52, 明治書院.

\bibitem[\protect\BCAY{EDR}{EDR}{1993}]{EDR}
EDR 電子化辞書仕様説明書 \BBOP 1993\BBCP.
\newblock (株) 日本電子化辞書研究所.

\bibitem[\protect\BCAY{Fernando, Pereira, David, Warren}{Fernando
  et~al.}{1980}]{fernando}
Fernando, C., Pereira, N., David H., and\ Warren, D. \BBOP 1980 \BBCP.
\newblock \JBOQ Definite clause Grammars for Language Analysis --
  A Survey of the Formalism and a Comparison
  with Augmented Transition Networks\JBCQ\
\newblock Artificial Intelligence 13 (3) pp.231--278.

\bibitem[\protect\BCAY{徳永}{徳永}{1988}]{徳永}
徳永健伸 \BBOP 1988\BBCP.
\newblock \JBOQ LangLAB\JBCQ\
\newblock \Jem{情報処理学会論文誌}, {\Bbf 7}, (29).

\bibitem[\protect\BCAY{松本・杉村}{松本\Jetal }{1986}]{松本}
松本・杉村 \BBOP 1986\BBCP.
\newblock \JBOQ 論理型言語に基づく構文解析システム SAX\JBCQ\
\newblock \Jem{コンピュータソフトウェア}, Vol.3, No.4, pp.308--315.

\end{thebibliography}


\setcounter{figure}{0}
\setcounter{table}{0}

\appendix
\small

\subsection*{形態素の分類一覧}
\bfg
\epsfile{file=goki.eps,height=50mm,width=70mm}
\efg{形態素の分類一覧}

\subsection*{言語現象の抽象化の手段に対応する形態素}

左の欄に代表的な形態素を挙げて,右欄には形態素が担う機能に対
する言語現象の説明を簡略に挙げている.他にもいくつかの助辞と
その助辞に対応する言語現象がある.

\btb{言語現象の抽象化の手段に対応する形態素} \footnotesize
\bt{|p{55mm}|p{75mm}|} \hline
\mltclm{1}{形態素(助辞)} & \mltclm{1}{言語現象との対応} \\ \hline
が,を,に & 用言に内在する論理関係の構成に関与する要素を表
示する \\ \hline
へ,と,で,より,から,によって, にとって , $\cdots$ & 用言
が示す事柄を,制限的に修飾する要素を表示する \\ \hline
は & 提題要素を表示し,用言が示す事柄を非制限的に修飾する
要素を表示する \\ \hline
も,こそ,さえ,でも,しか,$\cdots$ & 対比的な要素を表示し,
用言が示す事柄を非制限的に修飾する \\ \hline
ほど,くらい,ばかり,$\cdots$ & 前接する体言を制限的に修飾
する \\ \hline
る,た,るだろう,ただろう,よう, まい , $\cdots$ & 叙述する表
現内容が現在,未来の出来事であるのか,過去,完了の出来事であ
るのかを区分する.話し手の叙述の内容の確からしさを示す度合い
を表示する \\ \hline
て,れば,たら,と,ても,たって, だって, $\cdots$ & 用言で示される
出来事間の因果関係を表示する \\ \hline
れ(る),させ(る) & 用言に内在する論理関係の構
成に関わる要素を替える \\ \hline
い(る),あ(る),つつあ(る),てやが(る),$\cdots$ &
用言で示される出来事の時間的な継続状態を示す \\ \hline
はじめ(る),おわ(る), つづけ(る), $\cdots$ & 用言で示
される出来事の時間的な変化状態を示す \\ \hline
い(く),く(る),み(る),$\cdots$ & 用言で示される出来
事への話し手の関与状態を示す \\ \hline
い,かった,いだろう, かっただろう , $\cdots$ &  叙述する表
現内容が現在,未来に認められる属性であるのか,過去,完了に認
められる属性であるのかを区分する.話し手の叙述の内容の確から
しさを示す度合いを表示する \\ \hline
く,しければ,いかったら,と, くても , $\cdots$ & 用言で示さ
れる属性間の関係を表示する \\ \hline
だ,だった,だろう, だっただろう , $\cdots$ &  叙述する表
現内容が現在,未来に認められる状態であるのか,過去,完了に認
められる状態であるのかを区分する.話し手の叙述の内容の確から
しさを示す度合いを表示する \\ \hline
な(い),ず,ん & 否定的な叙述を構成する \\ \hline
ま(す),で(す),くださる, なさる , $\cdots$ & 叙述の表現
の文体を変える\\ \hline
る,た,い,かった,だった,な,$\cdots$ & 後接する体言を制
限的に修飾する \\ \hline
\et
\etb

\subsection*{文法規則の種類とその数}

\btb{文法規則の種類と数(1)} \footnotesize
\bt{|p{100mm}|p{10mm}|} \hline
\mltclm{1}{文法規則のタイプ} & \mltclm{1}{規則数} \\ \hline
文の認可規則                            & 1 \\
遂行形式の認可規則 - 1 -                & 4 \\
「提題 -- 叙述」構造の分析規則 - 1 -    & 108 \\
文境界規則                              & 2 \\
文法機能の規定値付与規則                & 41 \\
事態構造の認可規則                      & 2 \\
述部構造の構成規則                      & 3 \\
語基から語への構成規則                  & 1 \\
「提題--叙述」構造の分析規則 - 2 -      & 4 \\
「提題--叙述」構造の分析規則 - 3 -      & 4 \\
ムード値付与規則                        & 29 \\
文境界規則                              & 6 \\
基本否定辞分析規則                      & 17 \\
基本アスペクト辞分析規則                & 2 \\
二次アスペクト辞分析規則                & 10 \\
基本アスペクトの二次相の分析規則        & 33 \\
「提題--叙述」構造の分析規則 - 3 -      & 11 \\
「提題 -- 叙述」構造の分析規則 - 4 -    & 4 \\
文体変化分析規則 - 1 -                  & 8 \\
意志否定辞分析規則                      & 6 \\
文体変化分析規則 - 2 -                  & 8 \\
文体変化派生規則                        & 33 \\
動作性の体言からの用言構成規則          & 6 \\
「提題--叙述」構造の分析規則 - 5 -      & 8 \\
一般体言句の構成規則                    & 3 \\
「格補語--述部」依存構造の分析規則      & 5 \\
「提題句--述部」依存構造の分析規則      & 3 \\
体言からの述部構成規則                  & 3 \\
態変化分析規則                          & 22 \\
状態変化述語「にする」「になる」の構成  & 10 \\
接尾辞の付与による体言句構成規則        & 3 \\ \hline
\et
\etb

\btb{文法規則の種類と数(2)} \footnotesize
\bt{|p{100mm}|p{10mm}|} \hline
\mltclm{1}{文法規則のタイプ} & \mltclm{1}{規則数} \\ \hline
連体修飾句の依存構造の分析規則          & 7 \\
連体句の構成規則                        & 53\\
連体句の依存関係                        & 7 \\
「一般補語--述部」依存構造の分析規則    & 3 \\
副詞句の述部への依存構造の分析規則      & 4 \\
連用修飾句の構成規則                    & 10 \\
付帯状況を示す連用句の述部への依存構造の分析規則 & 65 \\
体言からの連用句の構成規則              & 1 \\
「取り立て句 -- 叙述」依存構造の分析規則        & 3 \\
「並列句 -- 述部」依存構造の分析規則    & 6 \\
「従属句 -- 述部」依存構造の分析規則    & 36 \\
「接続助辞でマークされる従属句 -- 述部」の依存構造の分析規則 & 6 \\
「条件形の従属句 -- 述部」依存構造の分析規則    & 49 \\
「仮定形の従属句 -- 述部」依存構造の分析規則    & 43 \\
「並立形の従属句 -- 述部」依存構造の分析規則    & 24 \\
「並列形(属性)従属句 -- 述部」依存構造の分析規則 & 7 \\
「接続句 -- 述部」依存構造の分析規則    & 3 \\
遂行形式の認可規則 - 2 -                & 6 \\ \hline
\et
\etb

\bfg

{\footnotesize \begin{verbatim}
(1) sentence(..,[態度(X,P)|REL],..)-->用言_5(..,[態度(X,P)|REL],..).


(2) 用言_5(..,[態度(X,中立),様相(X,平叙)|REL],..)-->
    用言_5(..,[様相(X,平叙)|REL],..).
(3) 用言_5(..,[態度(X,中立),様相(X,推量)|REL],..)-->
    用言_5(..,[様相(X,推量)|REL],..).
(4) 用言_5(..,[態度(X,表明),様相(X,意志)|REL],..)-->
    用言_5(..,[様相(X,意志)|REL],..).
(5) 用言_5(..,[態度(X,表明),様相(X,依頼)|REL],..)-->
    用言_5(..,[様相(X,依頼)|REL],..).


(6) 用言_5(..,[様相(Y,平叙),認め方(Y,肯定),
      content(..,[ムード(X,未完了)|REL],..),..)-->
    用言_4(..,[ムード(X,未完了)|REL],..).
(7) 用言_5(..,[様相(Y,平叙),認め方(Y,肯定),content(..,[ムード(X,完了)|REL],..),..)-->
    用言_4(..,[ムード(X,完了)|REL],..).
(8) 用言_5(..,[様相(Y,推量),認め方(Y,肯定),
      content(..,[ムード(X,未完了)|REL],..),..)-->
    用言_4(..,[ムード(X,未完了推量)|REL],..).
(9) 用言_5(..,[様相(Y,推量),認め方(Y,肯定),content(..,[ムード(X,完了)|REL],..),..)-->
    用言_4(..,[ムード(X,完了推量)|REL],..).

(``様相''は,文法機能を示し,``平叙'',``推量''は認識的な形態素を示す
 ``content''は文の叙述部分の構文情報と依存構造が含まれる.``REL'' は
 いわゆる Prolog 変数で,情報が単一化されることを表している.)

(10) 用言_4(..,[ムード(X,未完了)|REL],..)-->用言_4(..,[ムード(X,-)|REL],..),[る].
(11) 用言_4(..,[ムード(X,完了)|REL],..)-->用言_4(..,[ムード(X,-)|REL],..),[た].
(12) 用言_4(..,[ムード(X,未完了推量)|REL],..)-->
     用言_4(..,[ムード(X,-)|REL],..),[るだろう].
(13) 用言_4(..,[ムード(X,完了推量)|REL],F,PRO)-->
     用言_4(..,[ムード(X,-)|REL],..),[ただろう].

(14) 用言_5(..,[様相(Y,意志),認め方(Y,肯定),content(..,[ムード(X,未完了)|REL],..),..)-->
     用言_4(..,[ムード(X,-)|REL],..),[よう].
(15) 用言_5(..,[様相(Y,意志),認め方(Y,否定),content(..,[ムード(X,未完了)|REL],..),..)-->
     用言_4(..,[ムード(X,-)|REL],..),[まい].
(16) 用言_5(..,[様相(Y,依頼),認め方(Y,肯定),content(..,[ムード(X,未完了)|REL],..),..)-->
     用言_4(..,[ムード(X,-)|REL],..),[ろ].
(17) 用言_5(..,[様相(Y,依頼),認め方(Y,否定),content(..,[ムード(X,未完了)|REL],..),..)-->
     用言_4(..,[ムード(X,-)|REL],..),[るな].

\end{verbatim} }

\CAPLA{文法規則例}
\vspace*{3mm} \parbox{100mm}{\small この文法規則の例は,文の
分析に対応する.説明のために規則の番号を書き入れている.さら
に説明に関係しない部分は省略(``..'')してある.様相形態素を認
識する規則は,(6) $\sim$ (9)である.(10) $\sim$ (13) は,
「る」「た」「るだろう」「ただろう」の形態を処理する規則で,
ムード形態素が示す文法特徴を反映している.(14) $\sim$ (17) 
は「よう」 $\sim$ 「るな」の形態に対する分析規則である.(1) 
は文の認可規則で,話し手の判断が現れている述部を文として認可
している.なお,規則名(用言
\_5,用言\_4など)に見られる添え字は,階層位置を数字で示してい
る.}
\end{figure}

\normalsize
\begin{figure}[tb]

\begin{biography}
\biotitle{略歴}
\bioauthor{佐野 洋}{
1985年 豊橋技術科学大学大学院情報工学専攻修了. 同年(株)東芝入社. 総合
研究所に勤務. 1988年6月より(財)新世代コンピュータ技術開発機構へ出向. 
1992年10月より(株)東芝関西研究所に勤務. 1996年4月より東京外国語大学
外国語学部人文系講師, 現在に至る. 自然言語処理の研究に従事. 
情報処理学会会員.}
\bioauthor{福本 文代}{
1986年 学習院大学理学部数学科卒業. 同年沖電気工業(株)入社. 総合システム
研究所に勤務. 1988年10月より(財)新世代コンピュータ技術開発機構へ出向. 
1992年よりマンチェスタ工科大学計算言語学部修士課程入学, 翌年終了. 同大
学客員研究員を経て, 1994年4月より山梨大学工学部電子情報工学科助手, 現
在に至る. 自然言語処理の研究に従事. 情報処理学会, ACL各会員.}

\bioreceived{受付}
\biorevised{再受付}
\bioaccepted{採録}

\end{biography}

\vspace*{130mm}
\end{figure}

\end{document}
