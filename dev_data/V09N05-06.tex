\documentstyle[epsf,jnlpbbl]{jnlp_j_b5}

\setcounter{page}{111}
\setcounter{巻数}{9}
\setcounter{号数}{5}
\setcounter{年}{2002} 
\setcounter{月}{10}
\受付{2002}{5}{5}
\再受付{2002}{6}{6}
\採録{2002}{7}{17}

\setcounter{secnumdepth}{4}
\renewcommand{\theenumi}{}

\newcommand{\maru}[1]{}
\newcommand{\underlines}[1]{}
\newcommand{\kanji}[1]{}


\title{日中機械翻訳におけるとりたて表現の翻訳について \\ --「も」,「さえ」,「でも」--}

\author{卜 朝暉 \affiref{gifu} \and 謝 軍 \affiref{gifu} \and 池田 尚志 \affiref{gifu}}

\headauthor{卜,謝,池田}
\headtitle{日中機械翻訳におけるとりたて表現の翻訳について}

\affilabel{gifu}{岐阜大学工学研究科}{Graduate school of Engineering, Gifu University}

\jabstract{
「も,さえ,でも$\cdots$」などのとりたて詞による表現は日本語の機能語の中でも特有な一族である.その意味上と構文上の多様さのために,更に中国語との対応関係の複雑さのために,日中機械翻訳において,曖昧さを引き起こしやすい.現在の日中市販翻訳ソフトでは,とりたて表現に起因する誤訳(訳語選択,語順)が多く見られる.本論文では,とりたて詞により取り立てられる部分と述語部の統語的,意味的な特徴,更に中国語側での取り立てられる部分の統語的意味的な特徴によって,とりたて詞の意味の曖昧さを解消する手順を提案した.また,とりたて詞に対応する中訳語の位置について,訳語の文法上の位置に対する約束と,取り立てられる部分の中国語側での成分などから特定する手順を提案した.またこれらの手順を,「も,さえ,でも」の三つのとりたて詞をそれぞれ含む100文に対して手作業で検証した.正訳率はすべて80\,\%以上となり,本手法の有効性が示された.
}

\jkeywords{とりたて詞,日中機械翻訳,意味曖昧さ解消,意味解析,中国語語順生成}

\etitle{A Translation Method of Expressions Containing \\ the ``Toritate'' Words ``mo, sae, demo'' \\ in Japanese-Chinese Machine Translation}
\eauthor{Bu ZHaohui \affiref{gifu} \and Xie Jun \affiref{gifu} \and Ikeda Takashi \affiref{gifu}}

\eabstract{
Words such as ``mo'', ``sae'' and ``demo'' are particular function words in Japanese, and are known as “toritate” words. They have a variety of syntactic and semantic uses, and complicated corresponding relations to Chinese leads to ambiguities in Japanese-Chinese machine translation. Thus the use of current commercially available machine translation software results in numerous mistranslations of these words, in terms of vocabulary selected and word order determination. \\
In this paper, we propose a method for disambiguating the meaning of expressions containing the ``toritate'' words ``mo'', ``sae'' and ``demo'' by referring to the following syntactic and semantic features: (1) the features of the scope of the ``toritate'' word (it may be NP or VP), (2) the features of the predicate that are related to the ``toritate'' word, and (3) the features of the corresponding Chinese word for the Japanese scope of the ``toritate'' word. The positions of these ``toritate'' words in Chinese are determined according to their syntactic rules as well as the grammatical role of the scope of the ``toritate'' words in Chinese. \\
We evaluated our translation algorithm manually using 100 example sentences each for ``mo'', ``sae'' and ``demo''. The translation accuracy for each of these words was over 80\,\%, indicating that our method provides a more accurate Japanese-Chinese translation than currently commercially available translation software. 
}

\ekeywords{``Toritate'' words, Japanese-Chinese machine translation, word sense disambiguation, semantics analysis, Chinese generation}

\begin{document}
\maketitle
\thispagestyle{empty}


\section{はじめに}
「も,さえ,でも$\cdots$」などのとりたて詞による表現は日本語の機能語の中でも特有な一族である.言語学の角度から,この種類の品詞の意味,構文の特徴について,~\cite{teramura91,kinsui00,okutsu86,miyajima95}などの全般的な分析がある.また,日中両言語の対照の角度から,文献~\cite{wu87,ohkouchi77,yamanaka85}のような,個別のとりたて詞に関する分析もある.しかしながら,日中機械翻訳の角度からは,格助詞を対象とする研究はあるが~\cite{ren91a},とりたて詞に関する研究は,見当たらない.

とりたて詞は,その意味上と構文上の多様さのために,更には中国語との対応関係の複雑さのために,日中機械翻訳において,曖昧さを引き起こしやすい.現在の日中市販翻訳ソフトでは,取立て表現に起因する誤訳(訳語選択,語順)が多く見られる.

本論文は,言語学の側の文献を参考にしながらとりたて詞に関する日中機械翻訳の方法について考察したものである.すなわち,とりたて詞により取り立てられる部分と述語部の統語的,意味的な特徴によってとりたて詞の意味の曖昧さを解消する方法を示し,さらに同じ意味的な用法でも,対応する中訳語が状況により異なる可能性があることを考慮し,中国語側で取り立てられる部分の統語的,意味的な特徴及び関係する構文特徴によって,訳語を特定するための意味解析を行った.また,とりたて詞に対応する中訳語の位置を,その訳語の文法上の位置の約束と,取り立てられる部分の構文上の成分などから特定する規則を提案した.また,これらの翻訳規則を手作業により評価した.

なお,本論文では,とりたて詞として,文献~\cite{kinsui00}が挙げている「も,でも,すら,さえ,まで,だって,だけ,のみ,ばかり,しか,こそ,など,なんか,なんて,なんぞ,くらい,は」の17個のうちの「も」,「さえ」,「でも」の三つを検討の対象とした.

論文の構成は次の通りである.
第2章では とりたて表現の特徴と中国語との対応関係を述べ,第3章ではとりたて表現の中国語への翻訳方式とその方式の構成の主要な内容---意味解析と語順規則を説明する.第4章では,「さえ」,「も」,「でも」の翻訳の手順を例文を用いて示す.第5章では,手作業による翻訳の評価実験と問題点の分析について述べ,第6章では論文のまとめを述べる.


\section{とりたて表現の特徴と中国語との対応関係}
\subsection{日本語のとりたて詞ととりたて表現}
意味から見ると,とりたて詞は「文中のさまざまな要素を取り立て,これとこれに対する他者との関係を示す」作用を持つ助詞である.その「他者」は常に暗示されて,文の表に出てこない.文脈や,社会知識により,とりたて詞に取り立てられる部分(自者)を通して,暗示されている「他者」を推測でき,更にこの「両者」の関係を理解するのである.この「両者の関係」は場合によって,「他者への肯定,全面的な肯定(否定),強調,制限,譲歩,意外性,例示」などのさまざまな意味が現れてくる.例えば:

\begin{enumerates}
 \item 先生\underlines{も}今日のパーティに出席した.
 \item 戦争中,16歳の子供\underlines{も}(\underlines{でも}\,)徴兵された.
 \item ご飯\underlines{さえ}食べられれば満足だ.
 \item 誰\underlines{も}私の話を聞いてくれない.
 \item ここから駅まで30分\underlines{も}あれば足りる.
\end{enumerates}

上記の例文の下線部がとりたて詞である.意味から見ると:(1)では,「先生」はパーティに出席したが,「先生」以外の人もパーティに出席したと暗示されている.(2)では,16歳の子供が徴兵されるのは普通の世界では道理に合わないため,そのことに対する「意外」の意味を表す.(3)番では,「ご飯」を食べることは満足の十分な条件で,ほかのものはいらないという「最低限の十分条件」を示す.

とりたて詞のこのような意味上の多様性は,「が,を」のような格助詞の「格関係的意味」とは異質のものであり,また名詞や動詞が表現する「明示的なこと的意味」とも異なる.文の表に表現されている意味より,とりたて詞によって暗示されている意味の方が主目的であると考えられる.

とりたて詞が全体として多様な意味を持っているばかりでなく,1つのとりたて詞も一般に複数の意味用法を持っている.例えば,「も」には,(1)の「他者への肯定」,(2)の「意外性」,(4)の「全面的な否定」,(5)の「数量に対する評価」など多くの意味がある.

とりたて詞の意味解釈にはなおもう一つ特徴がある.それは話し手がどのように状況を認識するか,社会常識ではどのように理解されるのかによって,文の意味解釈が定まる場合があるということである.これがいわゆるとりたて詞の語用論的意味あるいは性質である.たとえば,例(2)の「も」の「意外性の意味」を理解するには,「16歳の子供を兵士にするのは常理には合わない」という社会常識が必要である.

構文上の特徴から見ると,とりたて詞は,名詞ないし形式名詞とのみ共起する格助詞と違って,さまざまの品詞と共起できる.下記の例では,(6) (7)では述部の真中に,(8)では格助詞の直後に,(9)では副詞の直後に現れている.

\begin{enumerates}
 \setcounter{enumi}{5}
 \item 親子の子は私の言うことを疑って\underlines{でも}いるようだ.
 \item 彼女はその手紙を見\underlines{も}しなかった.
 \item あの若い女優は外国に\underlines{まで}有名になった.
 \item もっとゆっくり\underlines{でも}間に合う.
\end{enumerates}

更に,とりたて詞によっては,同じ成分においても,名詞の直後か,名詞+格助詞の後か,或は名詞と格助詞の真中か,動詞の連用形か「て形」か,などまちまちである.

\subsection{取立て表現の中国語との対応関係}
孤立語とされる中国語には,意味上から見ると,たいていとりたて詞と同じ意味を持つ表現があるが,直接的に対応する品詞は無く,副詞,介詞,連詞など様々の品詞を用いた表現に翻訳することになる.日本語側で1つのとりたて詞は複数の意味をもつ場合があるが,中国語では,意味が違えば異なる訳語に対応する場合もあるし,また同様な意味も異なる訳語で表す場合もある.また日本語側で同じ意味を異なるとりたて詞で表現できる場合があるが,中国語ではそれらが1つの訳語に対応できる場合もある.また,訳語が訳文の構造に依存して決まる場合もある.訳語の訳文での位置は関連する統語上の成分に依存して決まるが,日本語側の位置とは必ずしも対応しない場合がある.

以下に「さえ」と「も」を例として,対応する中国語の訳語,訳語の品詞及び位置などを示す.

\vspace{1em}
\noindent
さえ:
\begin{enumerates}
 \setcounter{enumi}{9}
 \item 戦争の時期,お粥\underlines{さえ}食べられなかった.
       \begin{flushright}
	(意外性の意味,目的語の後)
       \end{flushright}
       訳文:\kanji{a}争\kanji{b}期,\underlines{\kanji{c}}粥\underlines{都}吃不上.
       \begin{flushright}
	(介詞+副詞,前置賓語の前)
       \end{flushright}
 \item 彼は授業中に寝て\underlines{さえ}いた.
       \begin{flushright}
	(意外性の意味,述語と助動詞いるの間)
       \end{flushright}
       訳文:他在\kanji{d}堂上\underlines{甚至}睡\kanji{f}.
       \begin{flushright}
	(副詞;述語の前)
       \end{flushright}
 \item 水\underlines{さえ}あればこの花は何週間も枯れない.
       \begin{flushright}
	(最低限の条件;主語の後)
       \end{flushright}
       訳文:\underlines{只要}有水,\kanji{g}支花\underlines{就}几周也不枯.
       \begin{flushright}
	(連詞;「只要」は主語の後,「就」は状語の前)
       \end{flushright}
\end{enumerates}

\vspace{1em}
\noindent
も:
\begin{enumerates}
 \setcounter{enumi}{12}
 \item 彼は自分の名前\underlines{も}書けない.
       \begin{flushright}
	(意外性の意味,目的語の後)
       \end{flushright}
       訳文:他\underlines{\kanji{c}}自己的名字\underlines{都}不会\kanji{h}.
       \begin{flushright}
	(介詞+副詞;「\kanji{c}」は前置賓語の前,「都」は述語の前)
       \end{flushright}
\newpage
 \item 私\underlines{も}中華料理が好きだ.
       \begin{flushright}
	(他者への肯定の意味,主語の後)
       \end{flushright}
       訳文:我\underlines{也}喜\kanji{i}中国菜.
       \begin{flushright}
	(副詞;主語の後か述部の前)
       \end{flushright}
\end{enumerates}

このように,日本語のとりたて詞の意味用法が非常に多様であるのに加え,中国語との意味的,位置的な対応も複雑多岐であることが,とりたて詞の日中機械翻訳を難しくしている要因である.

\section{取立て表現の中国語への翻訳}
\subsection{翻訳方式の構成}
上記の取立て表現の日中両言語における対応関係を考慮して,とりたて詞を含む文は,図~\ref{housiki}で示す翻訳方式で翻訳するものと考えた.全体は二つの部分に分けて進行する.

\renewcommand{\theenumi}{}
\newcommand{\labelenumii}{}
\renewcommand{\theenumii}{}
\begin{enumerates}
 \item とりたて詞を含む日本語文を解析し,とりたて詞以外の部分(以下骨格文という)に対して翻訳を行い,中国語文の骨格構造を得る.
 \item とりたて詞の翻訳
       \begin{enumeratess}
	\item 訳語を特定するための意味解析を行う.
	\item 訳語の中国語文での語順を特定する.
       \end{enumeratess}
\end{enumerates}
\renewcommand{\theenumi}{}

この方式においては,「とりたて詞の意味解析規則」と「語順規則」の二つの規則が必要である.語順を特定する際,中国語文の骨格文の構造と関連して決めるため,骨格文があらかじめ翻訳されていることが前提となる.

本論文では,とりたて詞の翻訳の部分(図~\ref{housiki}の点線の部分)を議論する.

\begin{figure}[hbtp]
\begin{center}
  \epsfile{file=figures/Fig1.eps,scale=0.6}
\caption{翻訳の方式}
\label{housiki}
\end{center}
\end{figure}


\subsection{とりたて詞の意味解析}
とりたて詞の翻訳を行うために,取立て表現の意味的用法をまず明らかにする必要がある.以下の論述の便利のため,とりたて詞をT,係り部分(取り立てられる部分)をX,結び部分(述部)をP,またXの中訳語をX$^{\prime}$,Tの訳語をT$^{\prime}$とする.(Xには単語,句,文などの可能性がある.また連体修飾語があればそれを含んでXとする.Tによって取り立てられる部分が二つ以上あれば,それをX1,X2$\cdots$とし,複文ならば,Tを含む従属節の述部をP1,主節の述部をP2とする.)

\subsubsection{とりたて詞の意味分類}
ここで,まず対象としている三つのとりたて詞「さえ/も/でも」の意味を中国語との対応付けも考慮して分類を行った(表~\ref{imibunrui}).意味分類は文献~\cite{teramura91,kinsui00,okutsu86,matsumura68,gu97}などを参考にして纏めた(同形の他品詞と,慣用句にある同形の表現は含まれていない).

\begin{table}[hbtp]
\begin{center}
 \caption{とりたて詞「も」,「さえ」,「でも」の意味分類}
  \epsfile{file=figures/Table1.eps,scale=0.82}
\label{imibunrui}
\end{center}
\end{table}


\subsubsection{とりたて詞の意味解析法}
自然言語処理では,「多義語の文中での意味を,文中の他の単語との意味的整合性から決定することが意味解析の一つの目的である」~\cite{nagao96}.多様な意味を持つ取り立詞の意味を特定するには,その関連する要素から分析しなければならない.とりたて詞の意味については,~\cite{teramura91,kinsui00,okutsu86,miyajima95}など多くの分析があるが,関連要素から系統的に分析したのは文献~\cite{teramura91}の寺村氏である.氏は「とりたて詞の意味は,係部分と,結び部分のそれぞれの文法的,意味的特徴と相関している」と指摘し分析している.係部分と,結び部分は,とりたて詞に取り立てられる部分と述語部(そのモダリティや,テンスなどを含める)である.我々もこの観点にたって実際にとりたて詞を含む多数の例文を観察した.その結果,とりたて詞の意味はその取りたてられる部分と述語部の各種の属性から基本的に特定できることを確認した.

      \hspace*{4.0zw}\begin{picture}(115,20)
	\put(75,0){\line(0,1){10}}
	\put(75,10){\line(1,0){40}}
	\put(115,10){\vector(0,-1){10}}
      \end{picture}
\begin{enumerates}
 \setcounter{enumi}{14}
 \item 家から駅まで\hspace{1zw}\fbox{\underlines{三時間}\hspace{1zw}\underlines{も}}\hspace{1zw}\underlines{かかる}\,. \\
       \hspace*{8.3zw}X\hspace{1.9zw}T\hspace{2.4zw}P \\
       \begin{picture}(110,20)
	\put(70,0){\line(0,1){10}}
	\put(70,10){\line(1,0){40}}
	\put(110,10){\vector(0,-1){10}}
       \end{picture}
 \item 私の肩は\hspace{1zw}\fbox{\underlines{痛く}\hspace{1zw}\underlines{も}}\hspace{1zw}\underlines{なった}\,. \\
       \hspace*{5.9zw}X\hspace{1.3zw}T\hspace{2.4zw}P
\end{enumerates}

取り立てられるもの(X)は(15)のように通常名詞あるいは名詞+格助詞であるが,(16)のように用言を取り立てる場合もある.その場合には,(16)のように「形容詞連用形+T+なる」や,「走ってもいない」のように「動詞て形+T+いる」などの形をとる.この場合Xは用言の連用形あるいは「テ」形であり(「も」には,基本形,タ形もあり得る.例えば,「聞くも涙話すも涙」),Pは「なる/いる/する/ある」などの補助動詞となる.この場合のPは中国語の述語動詞に対応する場合と単にテンスやアスペクト補助字に対応する場合がある.

(15)の例は,\{X=数量詞\hspace{1zw}P=肯定式という条件が成立する場合,「も」の意味は「時間が意外に多いことを強調する意味」\}という意味解析規則で解釈できる;

(16)の例は,\{X=形容詞の連用形\hspace{1zw}P=肯定式という条件が成立する場合,「も」は「他者への肯定の意味」\}という意味解析規則で解釈できる.

以上のように,XとPに対する条件によってTの意味を特定できる場合が多いが,「X1もX2も$\cdots$」のようにXが複数のとりたて部分を持つパターンであるかどうかも意味判定条件にかかわる場合がある.たとえば表~\ref{imibunrui}の「も」の1番と5番の意味は,Xの属性は同じであるが,1の意味では複数のXを含めるパターン(X1もX2も$\cdots$),5の意味では単一のXを含めるパターンという条件が必要である.

また日本語側の条件のみでは,意味を区別できても中訳語T$^{\prime}$を特定できない場合がある.例えば,「も」は,日本語側の条件として,P=肯定式,X=普通数量詞/疑問数量詞であれば,表~\ref{imibunrui}の「も」の2のaの意味であると解釈できるが,中訳語としては,X$^{\prime}$が普通数量詞の主語か賓語か,或は疑問数量詞かによって,それぞれ「竟然」,「竟然有」,「好」となる.

本論文ではとりたて詞の部分を除いた骨格構造の翻訳を用いる2段構造の翻訳方式を前提としているが,中国語側の構文構造を必要に応じて参照し,中訳語を特定することができる.

このほか,表~\ref{imibunrui}の「さえ」の2番目の意味は,「只要$\cdots$就$\cdots$」に訳すが,中国語文の主節が反問文あるいは主節の述部が「是$\cdots$」の構造ならば,また日本語側で主節が省略されれば「只要$\cdots$」のみに訳す,というように,日本語文と中国語文の構文特徴が条件になる場合もある.

上記をまとめると,本論で提案している中訳語を特定するためのとりたて詞の意味解析は,表~\ref{ruibetsu}に示す5種類の関係要素の属性,特徴によって行うことになる.

関係要素の属性についての条件は,品詞,活用形,肯(否)定式,ムード,成分及び文の構造特徴などのさまざまの角度から記述する.記述の細かさは,その記述によって他の意味用法との区別及び中国語への訳し分けが可能な程度に分解されればよい.例えば「X=人を表す名詞/数量詞/述語\hspace{1zw}P=仮定形/否定式/勧誘のモダリティ.X$^{\prime}$=普通数量詞の主語」など.

\begin{table}[htbp]
\begin{center}
\caption{確意味の判定条件の類別}
\label{ruibetsu}
 \begin{tabular}{|l|l|} \hline
  1 & Xの属性 \\ \hline
  2 & Pの属性 \\ \hline
  3 & 日本語側のパターン特徴 \\ \hline
  4 & X$^{\prime}$の属性 \\ \hline
  5 & 日本語,中国語文の構文特徴 \\ \hline
 \end{tabular}
\end{center}
\end{table}

以上の分析に従い,「も,さえ,でも」の三つのとりたて詞の意味解析を行った.表~\ref{kaisekikisoku}にその解析の規則を示す.(表~\ref{kaisekikisoku}の意味分類欄の数字は表~\ref{imibunrui}の意味分類の数字と対応する.)

\begin{table}[pt]
\begin{center}
\vspace*{-2em}
\caption{とりたて詞「さえ」,「も」,「でも」の意味解析規則}
\label{kaisekikisoku}
  \epsfile{file=figures/Table3.eps,scale=0.82}
\end{center}
{\footnotesize
[表~\ref{kaisekikisoku}に関する注]
\renewcommand{\labelenumi}{}
\begin{enumerate}
 \setlength{\labelwidth}{15pt}
 \item 表~\ref{kaisekikisoku}中のPは「する,なる,ある」などの補助動詞以外の述語である.
 \item ※が付いている処は完全に区別できない場合がある.それについての詳細な分析は5.2節の問題点考察に譲る.
 \item 「も」の2のaのX$^{\prime}$の属性は,「普通数量詞の賓語」は実際に「普通数量詞の賓語」と「普通数量詞の補語」の二つの場合があるが,ここで統一に数量賓語とする.
\end{enumerate}
}
\end{table}
\renewcommand{\labelenumi}{}

\subsection{とりたて詞に対応する訳語の語順処理}
格標識がない中国語には,語順は構文上で重要な役割を果たしている.とりたて詞に対応する中訳語T$^{\prime}$は副詞,介詞,連詞など様々であり,文中での位置も様々であるが,基本的には取り立てられる対象の中訳語X$^{\prime}$に関係して位置が決まる場合とX$^{\prime}$とは無関係に位置が決まる場合の2つの場合がある.

例えば,「さえ」の「最低限の条件」の意味に対応する訳語は「只要$\cdots$就$\cdots$」である.「只要」の語順は,文法上の約束により,条件節の主語の前あるいは直後に置き,「就」は主節の謂語部の前に置く.つまり,「只要」の位置はX$^{\prime}$とは無関係に位置が決まる.一方では,「も」,「さえ」の「意外性」の意味に対応する訳語の「\kanji{c}$\cdots$也$\cdots$」の場合,「\kanji{c}」はX$^{\prime}$が成分として主語,謂語,賓語,状語などのいずれであっても,X$^{\prime}$の前に置く(「也」謂語部の前に置く).すなわち,「\kanji{c}$\cdots$也$\cdots$」の位置がX$^{\prime}$に関係して決まる.

文献~\cite{lu80,yu98}を参考にして,3つのとりたて詞の中訳語T$^{\prime}$の位置に関して調査分析し,語順処理規則を作成した(表~\ref{tposition}).


\begin{table}[p]
\begin{center}
\vspace*{-2em}
\caption{T$^{\prime}$の位置を決める規則}
\label{tposition}
  \epsfile{file=figures/Table4.eps,scale=0.74}
\end{center}
{\scriptsize
[表~\ref{tposition}に関する注]
\renewcommand{\labelenumi}{}
\begin{enumerate}
 \setlength{\labelwidth}{15pt}
 \item 「$\cdots$和$\cdots$都$\cdots$」に関して,並列謂語の間に置くのは,その謂語は双音節詞及びその前後に共通的な連体成分があるという条件が必要である.また表~\ref{kaisekikisoku}では,この訳語に,X1$^{\prime}$とX2$^{\prime}$=主語/状語/賓語の条件であるため,X1$^{\prime}$とX2$^{\prime}$=謂語の場合を無視してもよい.
 \item 表~\ref{tposition}にある訳語は上記の品詞以外,他の品詞を兼ねる可能性もあるが,ここでは表~\ref{kaisekikisoku}の意味に対応する時の品詞のみを扱っている.
 \item 主語,賓語の前に定語があれば,T$^{\prime}$はその定語の前に置く.
 \item 一つの文で大主語,小主語が同時にあり,X$^{\prime}$=小主語であれば,「連」「即使」は小主語の前に置き,X$^{\prime}$=大主語であれば,「連」「即使」は大主語の前に置く.
 \item 文は複文で,いくつかの謂語がある場合,ここの語順規則で言っている謂語は,指定するもの以外に,X$^{\prime}$にもっとも近い謂語である. しかし,実際に適応しない例もあった.
\end{enumerate}
}
\end{table}
\renewcommand{\labelenumi}{}

\subsection{他品詞との区別の処理}
表層ではとりたて詞と同じ表現であるが,品詞が異なる場合がある.たとえば「でも」には,とりたて詞「でも」の他に,

\renewcommand{\theenumi}{}
\begin{enumerates}
 \item 接続助詞の「でも」
 \item 場所/道具格助詞「で」+とりたて詞「も」
 \item 助動詞「だ」の連用形「で」+とりたて詞「も」
 \item 逆接の接続詞「でも」
\end{enumerates}
などの場合がある.

言語学的には,例えば~\cite{kinsui00}で,

\renewcommand{\theenumi}{}
\begin{enumerates}
 \item 分布の自由性
 \item 任意性
 \item 連体文内性
 \item 非名詞性
\end{enumerates}
という四つの統語的特徴を同時に持つ場合がとりたて詞であると分析している.しかし,このような基準はコンピュータで判断することは不可能である.

この問題の解決は困難であるが,部分的には関係要素の属性を用いて解決できる場合もあり,また曖昧性を保留したままで中訳語を決めることができる場合もある.これについての詳細な分析は第5.2節の問題点と第5.1節の表~\ref{demodemo}に譲る.
 
\subsection{とりたて詞が慣用句の中に現れる場合の処理}
上記のとりたて詞と同形の他品詞の用法以外,他の言葉と接続し,固定的な用法を構成する同形の表現も多い(例えば,「言うまでもなく」,「それでも」).このような慣用句の中に含まれるとりたて詞と同形の表現は,上記の処理と別に,慣用句として対訳辞書を作ることにより解決すべきであると考える.


\section{3つのとりたて詞の翻訳手順}
\subsection{とりたて詞の全体の翻訳手順}
上記の過程を総合すると,とりたて詞を含む文の翻訳アルゴリズムは下記のようになる.

\renewcommand{\labelenumi}{}
\renewcommand{\theenumi}{}
\renewcommand{\labelenumii}{}
\renewcommand{\theenumii}{}
\begin{enumeratess}
 \item とりたて詞Tを含む日本語文を構文解析する.
 \item とりたて詞以外の部分の中国語への翻訳を行い,中国語骨格文を得る.
 \item とりたて詞の翻訳は下記のように行う.
       \begin{enumeratess}
	\item 日本語側の取り立てられる部分X,述語P,Xの中訳語X$^{\prime}$や構成したパターンの特徴,日,中文の構文特徴などを表~\ref{kaisekikisoku}の条件と照合し,中訳語T$^{\prime}$を決める.
	\item T$^{\prime}$X$^{\prime}$等の属性を表~\ref{tposition}と照合し,T$^{\prime}$の位置を特定する.
       \end{enumeratess}
 \item 2の中国語骨格構造と3で得た結果を総合して生成し,T$^{\prime}$を含む中国語文を得る.
\end{enumeratess}


\subsection{翻訳手順の例}
以下に「さえ」,「も」,「でも」を含む例文をそれぞれ一つ挙げ,その翻訳のアルゴリズムを述べる.

\subsubsection{「さえ」を含む例文の翻訳}
\renewcommand{\labelenumi}{}
\renewcommand{\labelenumii}{}
\begin{enumerates}
 \setcounter{enumi}{16}
 \item あなた\underlines{さえ}同意すれば,私が反対するわけはないのだ.
       \begin{enumerate}
	\item 構文解析し,下記の各情報を得る.\\
	      \begin{tabular}{lll}
	       とりたて詞 & (T)= さえ & \\
	       取り立てられる部分 &(X)= あなた & (X$^{\prime}$)= \kanji{j} \\
	       述語 & (P1)= 同意する & (P1$^{\prime}$)= 同意 \\
	       \multicolumn{3}{l}{日本語文主節=私が反対するわけがないのだ} \\
	       \multicolumn{3}{l}{中国語文主節=我\underlines{是}不会\kanji{k}\kanji{l}的} \\
	       \multicolumn{3}{l}{日本語文の構造「XさえP1$\cdots$P2」} \\
	       \multicolumn{3}{l}{中国語文の主節は「是$\cdots$」の構造である}
	      \end{tabular}
	\item stept1の結果と表~\ref{kaisekikisoku}を照合する.\\
       パターン=「XさえP1$\cdots$P2」, P1=仮定形,中国語文の主節は「是$\cdots$」の構造という条件で,表~\ref{kaisekikisoku}の「さえ」の意味分類1のB状況の訳語と照合して,訳語「只要」を選択する.
	\item 「只要」をもって,表~\ref{tposition}の語順処理規則と照合し,『「只要」を条件節の主語の直前に置く.』という語順を得る.この場合,中国語骨格文において条件節の主語は「\kanji{j}自己」であることが分っているので,「只要」を「\kanji{j}自己」の直前に置く.
	\item stept3の結果と他の骨格部分と共同に生成し,下記の訳文を得る.\\
	      訳文:只要\kanji{j}自己同意,我是不会\kanji{k}\kanji{l}的.
       \end{enumerate}
\end{enumerates}

\newpage

\subsubsection{「も」を含む例文の翻訳}
\begin{enumerates}
 \setcounter{enumi}{17}
 \item 誰\underlines{も}が私の友達だ.
       \begin{enumerate}
	\item 構文解析し,下記の各情報を得る.\\
	      \begin{tabular}{lll}
	       とりたて詞 & (T)= も \\
	       取り立てられる部分 & (X)= だれ(疑問詞) & (X$^{\prime}$)= 誰 \\
	       述語 & (P)= 私の友達だ. \\
	       & \multicolumn{2}{l}{P$^{\prime}$= 是我的朋友.且P$^{\prime}$は肯定式} \\
	       \multicolumn{3}{l}{日本語文の構造は「XもP」} \\
	      \end{tabular}
	\item stept1の結果と表~\ref{kaisekikisoku}と照合する.\\
	      パターン=「XもP」,X=疑問詞,P=肯定式という条件で,表~\ref{kaisekikisoku}の「も」の意味分類3のaと照合して,訳語「都」を選択する.
	
	\item 「都」をもって,表~\ref{tposition}の語順処理規則と照合し,『X$^{\prime}$=主語/状語/補語ならば,「都」をそれらの成分の直後に置く.』という語順を得る.この場合,中国語骨格において,X$^{\prime}$「誰」は主語であることが分かっているので,「都」を「誰」の直後に置く.
	\item stept3の結果と他の骨格部分と共同に生成し,下記の訳文を得る.\\
	      訳文:\kanji{m}都是我的好朋友.
       \end{enumerate}
\end{enumerates}

\subsubsection{「でも」を含む例文の翻訳}
\begin{enumerates}
 \setcounter{enumi}{18}
 \item 一円\underlines{でも}浪費したくない.
       \begin{enumerate}
	\item 構文解析し,下記の各情報を得る. \\
	      \begin{tabular}{lll}
	       とりたて詞 & (T)= でも \\
	       取り立てられる部分 & (X)= 一円=最小数量詞 & X$^{\prime}$= 一日元 \\
	       述語 & (P)=「浪費したくない」 \\
	        & \multicolumn{2}{l}{P$^{\prime}$=「不想浪\kanji{n}」且Pは確言式(ムード)} \\
	       \multicolumn{3}{l}{日本語文の構造は「XもP」} \\
	      \end{tabular}
	\item stept1の結果と表~\ref{kaisekikisoku}と照合する. \\
	      パターン=「XでもP」,X=最小数量詞,P=確言のムードという条件で,表~\ref{kaisekikisoku}の「でも」の意味分類2の一つの状況と照合して,訳語「即使$\cdots$也$\cdots$」を選択する.
	
	\item 「即使$\cdots$也$\cdots$」をもって、表~\ref{tposition}の語順処理規則と照合し,『「即使」をX$^{\prime}$の前に置き,「也」を状語/謂語の前に置く(状語と謂語があれば状語を優先とする).』という語順を得る.この場合,X$^{\prime}$=一日元,中国語骨格において,謂語は「不想浪\kanji{n}」であることが分かっているので,「即使」を「一日元」の前に置き、「也」を「不想浪\kanji{n}」の前に置く.
	\item stept3の結果と他の骨格部分と共同に生成し,下記の訳文を得る.\\
	      訳文:即使一日元也不想浪\kanji{n}.
       \end{enumerate}
\end{enumerates}


\section{翻訳規則の評価と問題}
\subsection{評価}
以上の翻訳手順を「も」,「さえ」,「でも」の各100例文について手作業により評価した.評価データは『朝日新聞』の「天声人語」の約6万文から抽出した.この約6万文を我々の研究室で開発した文節解析システムIbukiで解析したところ,「も」,「さえ」,「でも」を持つ文はそれぞれ7897,168,1039文あった.それらを先頭から順次チェックし,同形の他品詞や慣用句を構成する表現を除外しながら各100文を抽出した(表~\ref{100bun}).この各100文に対して,我々の翻訳手順を手作業で適用し,その結果を人手で判断し,評価した(表~\ref{hyouka},表~\ref{algo}).表~\ref{hyouka}は我々の意味解析による訳語の妥当性,表~\ref{algo}は語順まで含めた妥当性評価である.またある市販翻訳ソフトとの比較も示した.

表~\ref{hyouka},表~\ref{algo}はとりたて詞であることが分かっている場合に対しての評価であるが,表~\ref{100bun}には,各100文を抽出するために必要な文数と除外した状況を示した.表~\ref{demodemo}に「でも」の例文100文を抽出する際に必要であった総文数141とその内容,及びそれらを含めて評価した場合の訳語の正訳率を示した.

これらの結果から見ると,全体として80%以上の正訳率であり,市販の翻訳ソフトの現状と比較すると,我々の方法は十分な有効性が期待できると考えている.

  \begin{table}[htbp]
 \caption{意味解析規則の評価 (A: 我々の訳 B: 市販ソフトの訳)}
 \label{hyouka}
 \begin{center}
 \begin{tabular}{|l|c|c|c|c|c|} \hline
  \multicolumn{1}{|c|}{とりたて詞} & 例文数 & \multicolumn{2}{|c|}{正訳数} & \multicolumn{2}{|c|}{正訳率} \\ \cline{3-6}
  & & A & B & A & B \\ \hline
  も & 100 & 97 & 77 & 97\,\% & 77\,\% \\ \hline
  さえ & 100 & 99 & 20 & 99\,\% & 20\,\% \\ \hline
  でも & 100 & 95 & 45 & 95\,\% & 45\,\% \\ \hline
 \end{tabular}
 \end{center}
 \end{table}

  \begin{table}[htbp]
 \begin{center}
 \caption{語順を含む翻訳アルゴリズムの評価}
 \label{algo}
  \begin{tabular}{|l|c|c|c|c|c|} \hline
  \multicolumn{1}{|c|}{とりたて詞} & 例文数 & \multicolumn{2}{|c|}{正訳数} & \multicolumn{2}{|c|}{正訳率} \\ \cline{3-6}
  & & A & B & A & B \\ \hline
  も & 100 & 93 & 61 & 93\,\% & 61\,\% \\ \hline
  さえ & 100 & 94 & 11 & 94\,\% & 11\,\% \\ \hline
  でも & 100 & 82 & 24 & 82\,\% & 24\,\% \\ \hline
 \end{tabular}
 \end{center}
 \end{table}

\begin{table}[htbp]
\begin{center}
\caption{100文のとりたて詞を抽出するための必要文数と除外状況}
\label{100bun}
\begin{tabular}{|c|c|l|l|c|} \hline
 とりたて詞 & 除外文数 & \multicolumn{1}{|c|}{除外原因} & \multicolumn{1}{|c|}{例} & 必要総文数 \\ \hline
 も & 2 & \maru{1}解析の誤り(1文) & \maru{1}着もの & 102 \\
 & & \maru{2}慣用句用法(1文) & \maru{2}途方もなく & \\ \hline
 さえ & 1 & \multicolumn{1}{|c|}{解析の誤り} & \multicolumn{1}{|c|}{さえずる} & 101 \\ \hline
 でも & 41 & \maru{1}{他品詞}(31文) & \maru{1}地下街でも & 141 \\
 & & \maru{2}慣用句用法(10文) & \maru{2}それでも & \\ \hline
\end{tabular}
\end{center}
\end{table}

\begin{table}[htbp]
\begin{center}
\caption{とりたて詞「でも」と同形の他品詞「でも」を含めた場合の正訳率統計}
\label{demodemo}
 \begin{tabular}{|c|c|c|c|c|c|c|} \hline
  総文数 & T & (b) & 両方 & 慣用句 & (c) & 正訳率 \\ \hline
  141 & 100 & 30 & 16 & 10 & 1 & 79\,\% \\ \hline
 \end{tabular}
\end{center}
{\small
[表~\ref{demodemo}に関する注] 

「両方」は(a)と5.2節の(24)例のような「意外性」のとりたて詞にも(b)にも解釈できる場合. (b) (c)は,各々『場所/道具格助詞「で」+とりたて詞「も」』と,『助動詞「だ」の連用形「で」+とりたて詞「も」』の意味用法しか取れない場合.

評価基準:「慣用句」と「両方」に属すものは正訳とし,(b), (c) ととりたて詞の「でも」との区別の条件は,X=用言ならば,とりたて詞とし,X=場所/道具を表す名詞ならば,場所/道具格助詞「で」+とりたて詞「も」とし,「でも」の後に「ある/ない」が付いていれば,助動詞「だ」の連用形「で」+とりたて詞「も」とした.また,この評価では訳語のみで,語順を考慮していない.
}
\end{table}


\subsection{問題点と考察}
現在の規則では,下記のような問題を残している.

\subsubsection{}
長い単文あるいは複雑な複文では,語順規則が正しくない場合がある.例えば,
\begin{enumerates}
 \setcounter{enumi}{19}
 \item そして,どんなに異様な植物\underlines{も},きれいな水といい土で育てれば,可愛いい花が咲き,毒も出さないことを知る. \\
       訳文:然后知道\underlines{不管}什\kanji{o}\kanji{p}的植物,只要有干\kanji{q}的水和肥沃的土,\underlines{都}会\kanji{r}可\kanji{s}的花,不\kanji{t}出\kanji{u}素.
 \item 今は,どんな国\underlines{でも}発展しようとするなら,鎖国政策をとるわけにはいかない. \\
       訳文:\kanji{v}在(\underlines{不管}\,)什\kanji{o}\kanji{p}的国家如果要\kanji{t}展\underlines{都}不能采取\kanji{w}国政策.
\end{enumerates}

分析:(20)と(21)二例とも複雑な複文である.表~\ref{kaisekikisoku}によると,(20)の「も」も(21)の「でも」も「都」に訳す.表~\ref{tposition}によると,『X$^{\prime}$=主語/状語ならば,「都」を主語/状語の直後に置く.』ということになるが,実際は(20)では,主語の「植物」の直後ではなく,主節の謂語の「花が咲き(\kanji{r}花)」の直前に置く.(21)も主節の謂語「不能采取」の前に置く.訳語も「不管」を入れた方が自然である.これは,主語と謂語の間に,また条件節や,仮定節が入ると関係があると思われるが,どのような条件で語順を判断すればよいことか判然としない.

\subsubsection{とりたて詞と同形の他品詞との区別の問題}
とりたて詞と同形の他品詞との区別がはっきりとしない場合がある.

「でも」には,次のような同形の他品詞がある.

\renewcommand{\theenumi}{}
\begin{enumerates}
 \item 接続助詞の「でも」
 \item 場所/道具格助詞「で」+とりたて詞「も」
 \item 助動詞「だ」の連用形「で」+とりたて詞「も」
\end{enumerates}
\vspace{1em}
\renewcommand{\theenumi}{}
\begin{enumerates}
 \setcounter{enumi}{21}
 \item 日曜日\underlines{でも}出勤しなければならない.
 \item そんなことは子供\underlines{でも}知っている.
 \item レーガン総統の構想については,米国内\underlines{でも}強い批判がある.
 \item 宴会の席\underlines{でも},いい考えが浮かぶと急に立ち上がって消えてしまうことがあった.
 \item 彼はダンサーであると同時に俳優\underlines{でも}ある.
\end{enumerates}

(22)と(23)は(a)と関わる問題である.どれが接続助詞かどれがとりたて詞か言語上でも必ずしも明確でないが,機械翻訳では,意味も中訳語も同様であるため,区別して処理する必要がない.

(24)と(25)は(b)と関わる問題である.表現の形のみから見ると,2例とも『「意外性」のとりたて詞の「でも」』と,『場所格「で」+「も」』の解釈があり得る.人間なら,社会知識から(25)を「意外性」のとりたて詞の意味と取れるが,機械には困難である.(24)は人間にも二つの解釈があり得る.

(26)は(c)の場合である.『助動詞「で」+とりたて詞「も」』の用法は,常に「ある/ない」が後続しているため,「意外性」のとりたて詞「でも」との区別が明白である.ただし「ある/ない」が後続しても,従属節の中に包まれていれば,「提案」の意味のとりたて詞の「でも」にもなり得るため,また曖昧性が残っている.


\subsubsection{}
文脈や状況,社会知識などに頼らなければ,とりたて詞の間の意味用法を区別できない場合がある.

\begin{enumerates}
 \setcounter{enumi}{26}
 \item その国では,7歳の子供\underlines{も}入学できる.
 \item その国では,4歳の子供\underlines{も}入学できる.
\end{enumerates}

人間は社会知識によって,(28)の「意外性」を感じ,(27)の「他者への肯定」の意味と区別して理解できる.コンピーターには,構造も前後起語の属性も同じである(27)と(28)を区別するのは困難である.

4節の規則では,頻度が遥かに高い「他者への肯定」の意味という解釈をとるように扱っている.5節の表~\ref{hyouka}の評価で誤った3例は,ともにこの区分に関するものである.

ただし,X=動詞の場合には,Pが否定の形なら(Xもしない)「意外性」の用法,Pが肯定の形なら(Xもする),「他者への肯定」の意味であると推定できる.また,中国語でこの意味の訳語の「也」は,ある場合には「意外性」の意味も体現できるため,そのような場合には訳語の曖昧性が保留できる.

また「も」の4番と6番の意味の区別も困難であり,我々の規則ではやはり頻度が大きい4番の解釈をとるようにしている.ただし6番の中訳語の一部は4番と同じの「也」になっていることもあって,評価では,誤った例は出現しなかった.

「でも」の場合,1と2の意味を誤ったのが(29)の1件あった.

\begin{enumerates}
 \setcounter{enumi}{28}
 \item やはりトンボの短い一生\underlines{でも}楽しいことはあるのでしょう.
 \item 彼はビール\underlines{でも}飲んだのでしょう.
\end{enumerates}

現在の規則で,Pは「の+だろう」のような概言のムードならば,「提案」の意味とする.この規則では(30)は正しいが,(29)の「でも」は「意外性」の意味である.

また,Pが疑問文,命令のムードの場合も,1と2の意味ともあり得るため,曖昧さが残る.本規則では2番の意味を取るようにしている.


\subsubsection{その他}

\paragraph{}
省略された表現の場合,判定条件が適用できない場合がある.
\begin{enumerates}
 \setcounter{enumi}{30}
 \item 回答は「美味しいサンドイッチ\underlines{でも}」だったが,それを上回る接待ぶりだった
\end{enumerates}

表~\ref{hyouka}の評価で,誤訳した「でも」の3例は(31)の例と同じく述語が省略されたあるいは述部に省略があるものであった.

\paragraph{}
「提案」の意味の「でも」は,中国語で不定代詞「什\kanji{o}的」,不定数量詞の「一点児」及び「$\phi$」などに訳しうる.しかしながら,どのような条件で訳し分ければよいか特定しにくい.

\begin{enumerates}
 \setcounter{enumi}{31}
 \item こんな暑いときに,アイスクリーム\underlines{でも}食べられればいいなあ. \\
       現在訳:\kanji{g}\kanji{p}\kanji{x}的\kanji{y}候,能吃上冰淇淋就好了. \\
       完全訳:\kanji{g}\kanji{p}\kanji{x}的\kanji{y}候,能吃上冰淇淋\underlines{什\kanji{o}的}就好了.
 \item 仕方がない,弟と\underlines{でも}行くとするか. \\
       訳文:没\kanji{z}法,\kanji{z1}是和弟弟去\kanji{z2}.
 \item まるで野中の一つ家に\underlines{でも}住んでいるような,隣近所に少しの遠慮もない,ぱりぱりした叫び方で$\cdots\cdots$. \\
       訳文:好象住在荒野似的,一点也不用\kanji{z3}\kanji{z4}周\kanji{z5}的人,用清脆的叫声$\cdots\cdots$
\end{enumerates}

現在の処理ではすべて「$\phi$」に訳しており,(32)の完全訳のようなニュアンスは表せない.しかし,(33)や(34)の例では,「$\phi$」に訳すのは適当である.なぜならば,(32)は提案されている「アイスクリーム」と関係の他要素は思いつくものであり,(33)はその他要素は実に存在しないためである.(訳語の「什\kanji{o}的」の使用により,「アイスクリーム」と同様の「西瓜,冷ジュース」などの冷たいものを暗示されている.)

(34)のように,「でも」を含む節の述語が「ような,みたい」などの比況助動詞がついている場合は,「$\phi$」に訳すのも正確である.

\paragraph{}
ある中訳語は前後の文脈により,部分的に変化する場合がある.

\begin{enumerates}
 \setcounter{enumi}{34}
 \item どんな話題\underlines{でも}こなすにはどういう手があるか. \\
       訳文:\underlines{不管}什\kanji{o}\kanji{p}的\kanji{z6}\kanji{z7},要\kanji{z8}理好需要什\kanji{o}\kanji{p}的方法\kanji{z9}?
 \item 金\underlines{さえ}あれば仙女でも買える.\\
       訳文:\underlines{只要}有\kanji{za}即使(就是)仙女也\kanji{zb}的起.
\end{enumerates}

(35)は表~\ref{imibunrui}の「でも」の意味分類の3のaに属する.(文末は「か」の疑問式であるが,この分類は「あるか」を肯定式とし,「ないか」を否定式とする.)表3の規則によると,「都」に訳すが,この例ではその代わりに「話題」の前に「不管」を付き,「都」は訳さないほうが自然である.原因として,「こなすには」という目的状語と関係あるようであるが,確定できていない.

(36)は,「最低限の条件」を表す「さえ」であり,規則によると,「只要$\cdots$就$\cdots$」に訳すが,ここでは「只要$\cdots$」のみに翻訳する.しかし構文の特徴は表~\ref{kaisekikisoku}の1の意味のbの状況に属していない.ここの「就$\cdots$」の省略は後続する「譲歩」の「でも」と関係があると考える.即ち,「条件」文と「逆接」の文を一緒になると(他の「が,のに」も),条件を表す連詞「只要$\cdots$就$\cdots$」も「只要$\cdots$」に変化すると推測しているが,更に多くの例で証明することが必要である.

上記の分析から,とりたて詞の意味は文脈知識,状況知識,世界知識のような言語外知識に依存する場合が多くあるとわかった.これらの知識は現在の機械翻訳の領域では,未だ処理することはできない.一方,対応する中訳語の弾力性により,曖昧性を保留したまま翻訳できる場合があることも分かった.

\renewcommand{\labelenumi}{}
\begin{enumeratess}
 \item 「も」は,4,5,6番の意味とも,4番の訳語の「也」で翻訳できる場合がある.現在使用頻度の高い4番の「也」を選択しているため,5,6番の意味と区別できなくても,その二つの意味に属する一部の訳は正確である.
 \item 「でも」の場合,2番の意味では,「意外性」と「譲歩」の二つのニュアンスがある.単純な意味から見ると,区別が存在する.しかし,中国語では,「即使$\cdots$也$\cdots$」という言葉は両方の意味とも表せるため,機械翻訳から,上記の二つの意味を一つとして処理することができる.
\end{enumeratess}
\renewcommand{\labelenumi}{}

\section{おわりに}
本論文では,とりたて詞ととりたて詞に関連する文の構成要素との整合性に着目し,日中機械翻訳における取り立て表現の翻訳手法を提案した.本手法で,とりたて詞と,その取り立てられる部分と述部をパターンに構成し,その二つの要素に対する多方面の統語的,意味的な属性及びパターンの特徴,ないし中国語側の関係要素の属性などによって,取り立て表現の種々な意味用法を区別し,訳語を特定した.このような両言語から,関係要素の多方面の特徴からの条件制限は曖昧さに富むとりたて詞には適当だと考えられる.意味解析の時,目的言語の主導性を重視し,訳語の選択や訳語の曖昧さの保留などを十分に考慮した.また,とりたて詞の訳語の語順に対して,その文法上の約束以外に,中国語側の取り立てられる部分の成分や,それと訳語の位置関係などを分析し,語順特定規則を作成した.手作業によって,正訳率は各詞とも80\,\%以上であり,市販の翻訳ソフトと比較して,本手法の有効性を確認した.本論の提案は意味用法が複雑で,文法の用法と語彙的な意味を両方兼ねて持つとりたて詞の機械翻訳に,一つの手がかりを示したと考える.

しかしながら,とりたて詞の意味上と構文上の活発な性質のため,同一詞の用法や,他品詞の用法と完全に区別が出来ない場合もあるし,訳文の訳語の語順が不自然の場合もある.複雑な複文における取立ての範疇や,文脈との関連などに対する分析は本論では取り上げていない.とりたて表現の翻訳は更に様々な角度からアプローチする必要がある課題である.残る12個のとりたて詞に対する考察と,これらの問題に対するさらなる分析,及び翻訳システムに組み込むことが,今後の課題である.





\nocite{*}
\bibliographystyle{jnlpbbl}
\bibliography{391}


\begin{biography}
\biotitle{略歴}
  \bioauthor{卜 朝暉}{1991年中国広西大学外国語学部日本語科卒.2001年岐阜大学教育学研究科国語科教育専修修了.教育学修士.現在同大学工学研究科電子情報システム工学専攻博士後期課程に在学中.日中機械翻訳に興味を持つ.言語処理学会、情報処理学会各学生会員.}
  
  \bioauthor{謝 軍}{2000年岐阜大学大学院修士課程電子情報工学専攻修了.工学修士.現在、同大学院博士課程在学.機械翻訳,中国語処理の研究に従事.情報処理学会学生会員.}
  
  \bioauthor{池田 尚志(正会員)}{1968年東大・教養・基礎科学科卒.同年工業技術院電子技術総合研究所入所.制御部情報制御研究室,知能情報部自然言語研究室に所属.1991年岐阜大学工学部電子情報工学科教授.現在,同応用情報学科教授.工博.人工知能,自然言語処理の研究に従事.情報処理学会,電子情報通信学会,人工知能学会,言語処理学会,ACL,各会員.}
  
\bioreceived{受付}
\biorevised{再受付}
\bioaccepted{採録}
\end{biography}

\end{document}
