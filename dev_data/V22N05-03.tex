    \documentclass[japanese]{jnlp_1.4}
\usepackage{jnlpbbl_1.3}
\usepackage[dvipdfm]{graphicx}
\usepackage{amsmath}
\usepackage{hangcaption_jnlp}
\usepackage{udline}
\setulminsep{1.2ex}{0.2ex}
\let\underline

\usepackage{lingmacros}
\usepackage{multirow}
\usepackage[dvipdfm]{colortbl}
\usepackage{algorithm}
\usepackage{algorithmic}
\def\event#1{}


\Volume{22}
\Number{5}
\Month{December}
\Year{2015}

\received{2015}{5}{22}
\revised{2015}{8}{6}
\accepted{2015}{8}{28}

\setcounter{page}{397}


\jtitle{誤り分析に基づく日本語事実性解析の課題抽出}
\jauthor{成田 和弥\affiref{Author_1} \and 水野 淳太\affiref{Author_2} \and 上岡 裕大\affiref{Author_1} \and 菅野 美和\affiref{Author_1} \and 乾 健太郎\affiref{Author_1}}
\jabstract{
事実性は,文中の事象の成否について,著者や登場人物の判断を表す情報である.
事実性解析には,機能表現や,文節境界を越えて事実性に影
響を与える語とそのスコープなどの4種類の問題が含まれており,性能の向上が容易ではない.
本研究では,事実性解析の課題分析を行うために,機能表現のみを用いたルールベース
の事実性解析器を構築し,1,533文に含まれる3,734事象に適用した結果の誤りを分析した.
このとき全ての事象表現について,付随する機能表現に対して人手で意味ラベルを付与した.
その結果,主事象の事実性解析については,機能表現の意味ラベル
が正しく解析できれば,現在の意味ラベルの体系と本研究で用いた
単純な規則だけでも,90\%に近い正解率が得られることがわかった.
従属事象の事実性解析では,後続する述語やスコープといった従属事象特有の誤りが多く見られた.
それらの要素についてさらなる分析を行い,今後の事実性解析の指針を示した.
}
\jkeywords{事実性,モダリティ,機能表現,スコープ}

\etitle{Understanding the Technical Issues \\
	in Japanese Factuality Analysis Through Error Analysis}
\eauthor{Kazuya Narita\affiref{Author_1} \and Junta Mizuno\affiref{Author_2} \and Yudai Kamioka\affiref{Author_1} \and \\ Miwa Kanno\affiref{Author_1} \and Kentaro Inui\affiref{Author_1}} 
\eabstract{
Event factuality is information pertaining to whether events mentioned in the natural language correspond to either actual events that have occurred in the real world or events that are of uncertain interpretation. In factuality analysis, sufficient performance is yet to be achieved because of the complexity of issues such as functional expression and linguistic scope. This paper discusses the issues involved in factuality analysis by analyzing errors when applying a rule-based system to 3,734 events in 1,533 sentences. We annotate functional expression labels for all events. In the main events, the factuality analyzer, consisting of simple functional expression rules, achieves approximately 90\% accuracy if correct functional expression labels are provided. In subordinate events, we found many errors specific to subordinate events, such as errors caused by predicates and linguistic scopes. We provide guidelines for factuality analysis through additional discussion regarding predicates and linguistic scope.
}
\ekeywords{Event Factuality, Modality, Functional Expressions, Linguistic Scope}

\headauthor{成田,水野,上岡,菅野,乾}
\headtitle{誤り分析に基づく日本語事実性解析の課題抽出}

\affilabel{Author_1}{東北大学}{Tohoku University}
\affilabel{Author_2}{情報通信研究機構}{National Institute of Information and Communication Technology (NICT)}




\begin{document}
\maketitle


\section{はじめに}
\label{sec_intro}

近年,ブログ等の個人が自由に情報を発信できる環境の爆発的な普及に伴い,膨大なテキスト情報がWeb上に加速度的に蓄積され,
利用できるようになってきている.
これらの情報を整理し,そこから有益な情報を得るためには,
「誰が」「いつ」「どこで」「何を」といった情報を認識するだけでなく,
文に記述されている事象が,実際に起こったことなのかそうでないことなのかと
いう情報を解析する必要がある.
我々はこのような,文中の事象に対する,著者や文中の登場人物による成否の判断を表す情報を事実性と呼ぶ.
\eenumsentence{
\item[a.] \underline{\mbox{商品Aを使い}}始めた。
\item[b.] \underline{\mbox{商品Aを使う}}のは簡単ではなかった。
\item[c.] \underline{\mbox{商品Aを使っ}}てみたい。
\item[d.] \underline{\mbox{商品Aを使っ}}ているわけではない。
\item[e.] \underline{\mbox{商品Aを使っ}}ているはずだ。
} \label{ex_ie}
(\ref{ex_ie})に示す例は,いずれも「商品Aを使う」という事象が含まれるが,その事実性は異なる.
(\ref{ex_ie}a)と(\ref{ex_ie}b)は,事象が成立していると解釈できる一方で,(\ref{ex_ie}c)と(\ref{ex_ie}d)は,事象は成立していないと解釈できる.
さらに(\ref{ex_ie}e)は,事象の成立を推量していると解釈できる.
評判分析などの文脈で,商品Aを使っているユーザの情報のみを抽出したい場合,
(\ref{ex_ie})に示した全ての文に対して,「商品Aを使う」と照合するだけでは,
(\ref{ex_ie}c)や(\ref{ex_ie}d)といった,商品Aを実際には使っていないユーザの情報まで抽出されてしまう.
そこで事実性解析を用いると,(\ref{ex_ie}a)や(\ref{ex_ie}b)が実際に商品Aを使っており,(\ref{ex_ie}c)や(\ref{ex_ie}d)が使っていない,
(\ref{ex_ie}e)は使っていない可能性がある,ということを区別することができる.
事実性解析は,評判分析だけでなく,含意関係認識や知識獲得といった課題に対しても重要な技術である
~\cite{Karttunen2005,Sauri2007,Hickl2008}.

事実性解析は,事象が実際に起こったかを解析する技術ではあるが,真に起こっ
たかどうかを与えられた文のみから判断することは不可能である.例えば,「太
郎は先に帰ったはずです。」という文に対して,「太郎は帰った」という事象が
真に事実か否かは,「太郎」にしか分からない.
そこで本研究では,事実性を,文中の事象の成否について,著者の判断を表す情報と定義する.
ただし,実際には著者の判断も真にはわからないため,著者の判断を読者がどう解釈できるかによって事実性を表す.
前述の例では,著者は事象「太郎は帰った」の成立を推量していると読者は解釈できる.

事実性の付与対象となる事象は,\citeA{Matsuyoshi2010}と同様に,行為,出来
事,状態の総称であると定義する.
\eenumsentence{
\item[a.] 雨が\event{降っ}$_{\mathrm{出来事}}$たら、バスで\event{行き}$_{\mathrm{行為}}$ます。
\item[b.] \event{混雑}$_{\mathrm{状態}}$していたら、別のところに
\event{行き}$_{\mathrm{行為}}$ます。
} \label{ex:event}
(\ref{ex:event})に示す例では,「(雨が)降る」,「(バスで)行く」,「混
雑する」,「(別のところに)行く」が全て事象である.\event{}で囲まれた述
語は,それぞれの事象の中心となる語であり,事象参照表現(あるいは単に事象
表現)と呼ぶ.アノテーションや解析において,事実性のラベルは事象表現に付
与する.

先行研究では,事実性だけでなく,時制などの関連情報についても,付与基準
が議論されるとともに,コーパス構築が進められてき
た~\cite{Sauri2009,Matsuyoshi2010,Kawazoe2011,Kawazoe2011_report}.日本語
を対象とした事実性解析の研究は少なく,述部(本研究の事象表現に相当)に続
く表現形式によるルールベースの解析~\cite{Umezawa2008SAGE}や,機械学習に基
づく解析器~\cite{Eguchi2010_nlp}がある.前者はその性能は報告されていない
が,後者の解析性能は,9種類の事実性ラベルの分類性能がマクロF値で48\%であり,実用上
十分とはいえない.

事実性解析の性能向上が困難である理由の一つは,事象表現に続く機能表現の
多様性にある.詳しくは\ref{sec_factvalue}節で述べるが,例えば「\event{使わ}
\underline{ない}」「\event{使う}\underline{わけない}」「\event{使わ}\underline{ねぇ}」
「\event{使う}\underline{もんか}」のように,事象が成立しないことを示す機能表現(下線部)
が多々ある.機能表現以外に,「\event{使う}のを\underline{やめ
た}」のように,文節境界を越えて事象の不成立を示唆する述語(下線部)の存
在もあり,さらにこれらの要素の組み合わせが,事実性解析の性能向上を阻んでいる.

本研究では,事実性解析の課題分析を行うために,機能表現のみを用いたルール
ベースの事実性解析器を構築し,1,533文に含まれる3,734事象に適用した結果の
誤りを分析する.このとき全ての事象表現に続く機能表現に対して意味ラベルを
人手で付与する.要素の組み合わせを解きほぐすために,3,734事象
を,最も文末に近い主事象1,533事象と,それ以外の従属事象2,201事象に分割し,
それぞれについて誤り分析を行う.

誤り分析の結果,主事象の事実性解析については,機能表現の意味ラベルが正
しく解析できれば,現在の意味ラベルの体系と本研究で用いた単純な規則だけで
も,90\%に近い正解率が得られることがわかった.また,機能表現解析の問題を
除けば,誤りの半数は副詞に起因するものであった.一方で,従属事象の事実性解析は,
主事象に比べて考慮すべき要素が多いため,性能も低いことがわかった.
従属事象でのみ考慮すべき要素は大きく二つあり,文節境界を越えて事実性に影響を与える述語と,従属事象に直接付随しない機能表現の影響である.
前者は,既存の辞書のカバレッジを調査した結果,これを利用することで誤りの一部を解消できるものの,さらなる拡充が必要であることが分かった.
後者は,問題となるケースは多様ではないことと,隣接する事象の機能表現が及ぼす範囲(スコープ)を精緻に判定することで概ね解決できることを確認した.




\section{関連研究}
\label{sec_related}

事実性に大きく関連する概念として,態度表明者の主観的な態度(モダリティ),および,肯定/否定があげられる.
本研究における事実性は,事象の真偽に対する書き手の確信度を表した「真偽判断のモダリティ」\cite{Masuoka2007}と,
肯定/否定の組み合わせに相当している.

\subsection{タグ体系およびコーパス構築に関する研究}
\label{subsec_corpus}

事実性およびその周辺情報を付与するためのタグ体系およびコーパス構築の関連研究として,
\citeA{Sauri2008_Guidelines,Sauri2009}によるFactBankや,
\citeA{Matsuyoshi2010,Matsuyoshi2011}による拡張モダリティタグ付与コーパス
などがある.

\citeA{Sauri2008_Guidelines,Sauri2009}は,
事象を対象とし,以下の2つ組のタグによって事実性を定義した.
\begin{description}
  \item[modality] 事実らしさに対する態度表明者の確信度.CT (Certain),PR (Probable),PS (Possible),U (Underspecified)の4種類で表す
  \item[polarity] 事象に対する確信の方向.+ (positive),$-$ (negative),u (underspecified)の3種類で表す
\end{description}
例えば,事象が実際に起こったことである,ということをCT+と表す.
そして,事象とその時間情報や,事象間の時間的順序関係が付与されたTimeML~\cite{Sauri2006}の上に,
確信度と肯否極性を態度表明者 (source) ごとに付与する枠組みを提案し,FactBankと呼ばれるコーパスを構築した.
\citeA{Marneffe2012}は,PR+とPS$-$,PS+とPR$-$をそれぞれ論理的に等価なラベルとして扱っており,それらのラベルを統合した,5種類のラベル体系による評価を行っている.

\citeA{Matsuyoshi2010,Matsuyoshi2011}は,
〈態度表明者〉,〈相対時〉,〈仮想〉,〈態度〉,〈真偽判断〉,〈価値判断〉の6項目からなる拡張モダリティタグ体系を設計し,
それを現代日本語書き言葉均衡コーパス(BCCWJ)\footnote{http://www.ninjal.ac.jp/corpus\_center/bccwj/}の各事象に付与したコーパスを構築した.
彼らのタグ体系の内,〈真偽判断〉は\citeA{Sauri2009}の事実性に相当している.
拡張モダリティタグ体系を用いた解析では,項目間の依存関係を考慮することが可能であるが,それ故に処理が複雑化してしまうという問題がある.

本研究では,\citeA{Marneffe2012}の枠組みに基づいて事実性のラベルを定義する.
この枠組みを利用することで,事実性を確信度と肯否極性の2軸に分けることができるため,問題の分析がしやすくなると考えた.
誤り分析には,\citeA{Matsuyoshi2010,Matsuyoshi2011}の拡張モダリティタグ付与コーパスを用いる.
事実性解析における課題分析をする上で十分な量であり,一般に利用可能なコーパスは他にないため,
拡張モダリティタグのうち〈真偽判断〉を事実性の正解として利用し,事実性解析の誤り分析を行う.


\subsection{解析および課題分析に関する研究}
\label{subsec_method}

事実性は,機械学習に基づく手法や,人手で構築した語彙的・統語的な知識を利用したパターンベースの手法
などを用いて解析が行われているが,その性能および課題分析は十分でない.

\citeA{Hara2008}は,事象の事実性情報を,時間情報と話者態度で表現し,
SVMを学習器に用いた解析手法を提案した.
\citeA{Inui2008}は,\citeA{Hara2008}の提案するタグ体系を整理統合し,条件付き確率場を学習器として用いた解析手法を提案した.
SVMを用いるよりも,項目間の依存関係を考慮できる条件付き確率場を用いたほうが,高い精度を示すことが報告されている.
\citeA{Eguchi2010_nlp}は,項目間,および事象間の依存関係を考慮できるFactorial CRF~\cite{Sutton2007}を用いた拡張モダリティ解析システムを構築した.
事実性に関連の深い〈真偽判断〉には9種類のラベルが存在するが,そのマクロF値で48\%の性能を示している.
さまざまな枠組みによって事実性の解析が行われているが,いまだ十分な性能は達成できておらず,その課題を分析する余地が多分に残されている.

モダリティ解析における課題分析としては,\citeA{Matsuyoshi2011}が最大エントロピーモデルを用いた拡張モダリティ解析システム分析を試作し,
その中の1つの項目である〈態度〉に着目した誤り分析を行っている.
彼らは語義曖昧性解消や連体節内の述語に及ぼす影響の解明,節間の意味的関係の認識などが,
〈態度〉に関するモダリティ解析の精度向上に向けた課題であることを述べている.
事実性に,より直接的に関連する〈真偽判断〉の誤り分析でも同様の結果が得られるかどうかは明らかではない.

英語においては,\citeA{Sauri2012}や\citeA{Marneffe2012}が事実性の解析に取り組んでいる.
\citeA{Sauri2012}は,事象の成立に影響を与える手がかり表現を利用し,
態度表明者ごとに,確信度と肯否極性で表される事実性を,依存構造木の根から伝搬させて解析する,パターンベースの決定的アルゴリズムを提案した.
例えば{\it not}があれば肯否極性を反転させる,{\it may}があれば確信度を下げる,といったルールに基づいて解析を行い,
F値でマクロ平均70\%,マイクロ平均80\%の性能を実現している.
誤り分析の結果,ルールのカバレッジや表現の曖昧性が大きな問題であることを報告している.
\citeA{Sauri2012}のアノテーション基準では,事実性は可能な限り客観的に判断される.
一方で,\citeA{Marneffe2012}は,主観的な判断の自動推定に取り組んだ.
主観的な判断とは,例えば態度表明者の社会的な信頼性によるものである.
信頼に足る組織が表明した事象の事実性は,CT$+$にバイアスがかかるが,表明者が不明な場合は事象の事実性は,CT$-$にバイアスがかかる.
彼女らは,FactBank中の各事象に対して,10名ずつアノテーションを行い,その分布を最大エントロピーモデルによって推定した.
解析性能は,多数がアノテートしたラベルを正解とした場合に,F値でマクロ平均70\%,マイクロ平均83\%の性能をあげられている.

本研究では,日本語事実性解析の課題に関して議論するために,機能表現に基づき,決定的に事実性を解析するルールベースのモデルを構築し,誤り分析を行う.
ここでルールベースモデルを用いる理由としては,機械学習に基づく手法と比べ,出力結果がどのような要素に基づいて選択されたかがわかりやすく,
本研究の目的とする,日本語事実性解析における課題の分析に対して適当であると判断したためである.
また,事実性に影響を与える要素はさまざま存在しており,いろいろな要素を複合的に加味したモデルが提案されてきている.
しかしながら,どの要素がどの程度事実性に影響を与えるのか,という分析は十分に行われていない.
そこで,事実性に影響を与える要素を切り分けることにより,事実性解析における各要素の重要性を議論し,課題の分析を行う.


\subsection{事実性に影響を与える要素に関する研究}
\label{subsec_mergin}

事実性に影響を与える要素としては,機能表現や後続する述語,および,それらの作用する範囲(スコープ)などがある.
\eenumsentence{
\item[a.] もう\event{遅い}から、彼は先に\event{帰っ}ている\underline{だろう}。
\item[b.] 問題が\event{発生する}のを\underline{防いだ}。
} \label{ex_effect}
例えば,(\ref{ex_effect}a)の事象「帰る」の事実性は,「だろう」という機能表現に影響を受け,
(\ref{ex_effect}b)の事象「発生する」の事実性は,「防いだ」という述語に影響を受けている.
また,(\ref{ex_effect}a)では,「だろう」という機能表現は「帰る」のみに影響を与え,先行する事象「遅い」には影響しない,というように,
機能表現や後続する述語の作用する範囲,即ちスコープを特定することも,事実性解析において重要な要素だと考えられる.

事実性に影響を与える表現として,「〜ない」「〜だろう」などの機能表現があり,
このような日本語機能表現の意味に関連した研究が多く進められている.
例えば,機能表現を網羅的に集めた辞書として,日本語機能表現辞書『つつじ』\cite{Matsuyoshi2007}が利用されている.
この辞書は,日本語の機能表現の表層形約17,000種に対して,そのID,意味,文法的機能,音韻的変化などを網羅的に収録した辞書であり,
機能表現の意味として,「対象」や「目的」,「名詞化」など,89種類のラベルが定義されている.
その中には「推量」や「否定」,「疑問」など,事実性に影響を与えるラベルも多数含まれている.
また,機能表現の中には表層を見ただけでは判別が難しいものも存在する.
\enumsentence{
パソコンが\event{壊れ}\underline{てしまったかも知れない}。
} \label{ex_funcchunk}
(\ref{ex_funcchunk})では,事象「壊れる」に対して,「てしまっ」「た」「かも知れない」という機能表現が付随している.
「知る」という表現は,機能表現の一部として用いられるだけでなく,述語としても用いられるため,
(\ref{ex_funcchunk})では「かも知れない」で 1 つの機能表現として用いられている,ということを判別する必要がある.
このような曖昧性を解消するため,どの部分が機能表現なのかを特定し,その意味を同定する研究も行われている\cite{Suzuki2011,Imamura2011,Kamioka2015}.

事実性に影響を与える述語に関する研究としては,\citeA{Eguchi2010_nlp}が構築した,モダリティ解析手がかり表現辞書がある.
彼らは,「防いだ」のような,拡張モダリティに影響する動詞,形容詞が存在していることに着目した.
こうした動詞,形容詞が直前の事象に与える影響を記述した,モダリティ解析のための手がかりを集めた表現辞書を作成し,
機械学習による拡張モダリティ解析を行う上で,素性として利用している.
このような表現を集めた利用可能なリソースは他に存在しておらず,
この辞書を利用することでどの程度事実性解析の性能改善につながるのか,この辞書でどの程度の述語がカバーできているのか,といったことを調査する必要がある.

事実性を決定する上で,否定や推量などのスコープを決定することは重要だと考えられる.
否定表現および推量表現のスコープを同定する研究は,近年盛んに行われている.
例えばBioScope~\cite{Szarvas2008}は,医学・生物学ドメインのテキストを対象に,否定表現,様相表現,そして,それらのスコープをアノテーションしたコーパスであり,
このコーパスを用いてShared Task~\cite{CoNLL2010,SEM2012}が開催されるなど,スコープを特定する研究に広く利用されている.
日本語においては,\citeA{Kawazoe2011,Kawazoe2011_report}が,テキストに現れる事実とそれ以外の情報との区別,
また推量や仮定などの間に見られる確実性の差を自動的に識別するため,
様相表現,否定表現といった「確実性」に影響を与える言語表現を分析・分類し,
それに従ってそれらの言語表現およびそのスコープをアノテーションしたコーパスを構築しているが,それらの定量的な分析を行うまでには至っていない.
\citeA{Matsuyoshi2014}は,否定の焦点検出システムを構築するための基盤として,
日本語における否定の焦点をテキストにアノテーションする枠組みを提案し,否定の焦点コーパスを構築している.
否定の焦点は,否定のスコープの中で特に否定される部分であるため,焦点の検出はスコープの特定と密接に関連している.

このように,事実性に影響を与える要素がいくつか存在しており,これらの要素を複合的に考慮することで事実性を決定できると考えられる.
しかしながら,どの要素がどの程度事実性に影響を与えるのか,ということは明確ではない.
本研究では,これらの事実性に影響を与える要素を切り分け,事実性解析における各要素の重要性を議論することにより,課題の分析を行う.



\section{誤り分析を通した課題分析の方針}
\label{sec_factvalue}

本節では,事実性解析に関連する種々の要素(機能表現や副詞など)について整
理し,その関連性を述べる(\ref{sec:fact_features}節).事実性は,これら
の要素が複合的に組み合わさって決定されるため,各要素の重要性を見極めるた
めには,これらを切り分けて分析することが肝要である.
\ref{sec:fact_approach}節では,どのようにしてそれらを切り分けるのかを述べ
る.
3.3節では,課題分析に用いるコーパスについて述べ,
3.4節では,分析に用いるルールベースの事実性解析モデルについて述べる.


\subsection{事実性解析に関わる言語要素}
\label{sec:fact_features}

先行研究によって,事実性解析に関連する言語要素は文内では大きく4つに分け
られることが分かっている.事象に含まれる機能表現,疑問詞を含む副詞,文節
境界を越えて事実性に影響を与える語とそのスコープ,その他の4種類である.図
\ref{fig:sent_structure}に,文内での各要素の関係を示す.ここで,事象の中
心的な述語を\textbf{事象表現}と呼び,事象の事実性は事象表現に割り当てら
れると定義する.以降の例では,\event{}で囲むことで表す.また,文中での出
現位置によって事象表現を二種類に分類する.各文につき,最も文末に近い事象
表現を\textbf{主事象}と呼び,それ以外の事象表現を\textbf{従属事象}と呼ぶ.
図中の矢印は,その言語要素が事象表現の事実性に影響することを示している.
以下では,その他以外の3つの要素について,事象表現の事実性にどのように関
連するかを述べる.なお,文をまたいだ言語要素による否定や推量も存在するが,
本研究では文内の現象のみを取り扱う.

\begin{figure}[b]
\begin{center}
 \includegraphics{22-5ia3f1.eps}
\end{center}
 \hangcaption{事実性に関わる言語要素の構造}
\centering\small 矢印は,要素が事象表現の事実性に影響することを示す.
 \label{fig:sent_structure}
\end{figure}


\subsubsection{機能表現}

事象表現に直接後続する機能表現(図\ref{fig:sent_structure}では,述語Aに
対する機能表現A,述語Bに対する機能表現B)の問題は,多義性と表現の多様性の二つに大き
く分けられる.
\eenumsentence{
\item[a.] 太郎は\event{走っ}たんでした\underline{よね}$_{\mathrm{態度}}$
\item[b.] 太郎は\event{走る}んです\underline{よね}$_{\mathrm{疑問}}$
} \label{ex:func_sem}
まず,多義性について,(\ref{ex:func_sem})に示す2つの例には,いずれも機能
表現「よね」が出現しているが,それが示す意味は異なる.(\ref{ex:func_sem}a)は
「太郎が走る」ことを推量しているが,(\ref{ex:func_sem}b)は「太郎が走る」ことを確認しているこ
とから事象は成立していないことを示している.このような文を解析するために
は,機能表現の多義性の解消は必須の技術である.

\eenumsentence{
\item[a.] 太郎は\event{走ら}\underline{ない}$_{\mathrm{否定}}$。
\item[b.] 太郎は\event{走る}\underline{わけない}$_{\mathrm{否定}}$。
\item[c.] 太郎が\event{走ら}\underline{ねぇ}$_{\mathrm{否定}}$。
\item[d.] 太郎が\event{走れる}\underline{もんか}$_{\mathrm{否定}}$。
} \label{ex:func_vari}
次に,表現の多様性について,(\ref{ex:func_vari})に示す4つの例は,いずれ
も「太郎が走る」という事象が成立していないということを,異なる機能表現に
よって記述している.そのため,否定を認識するためには,典型的な否定の機
能語である「ぬ,ない」だけでなく,「ねぇ,もんか」といった砕けた表現もとらえる必要がある.

これらに加えて,複数の表現の組み合わせの問題がある.
\eenumsentence{
\item[a.] 太郎は\event{走れ}\underline{なくなる}$_{\mathrm{否定}}$\underline{よ
うだ}$_{\mathrm{推量}}$。
\item[b.] 太郎が\event{走る}\underline{かもしれない}$_{\mathrm{推量}}$。
} \label{ex:func_combi}
機能表現の組み合わせは,複数の表現の意味が組み合わさって事実性を表す場合
(\ref{ex:func_combi}a)と,単語自体では意味を持たず複数の語が組み合わさっ
てはじめて意味を持つ場合(\ref{ex:func_combi}b 複合辞と呼ばれる)がある.
(\ref{ex:func_combi}a)は,否定の機能表現「なくなる」と,推量の機能表現
「よう」が組み合わさることで,「太郎が走る」という事象が成立しないことを
推測していることを示している.この事例を正しく認識するためには,機能語単
位での意味ラベルだけでなく,その組み合わせに従って事実性を演算することが
必要となる.一方,(\ref{ex:func_combi}b)は複合辞の事例であるが,「走る」に後続する3つの単語
「かも,しれ,ない」は,ひとまとまりで推量の機能表現を構成している.
このとき,「ない」は否定の意味を持っておらず,機能表現を解釈するには,特定の単語列を複合辞
としてまとめた上でその意味を認識する必要がある.


\subsubsection{述語周辺の副詞}

事実性は,事象に後続する機能表現だけでなく,周辺の副詞(図
\ref{fig:sent_structure}では,述語Aに対する副詞A,述語Bに対する副詞B)に
よって決定される場合がある.
\eenumsentence{
\item[a.] \underline{確か}太郎は\event{走っ}た。
\item[b.] 太郎は\underline{果たして}\event{走る}のだろうか。
\item[c.] \underline{どうしたら}太郎は\event{走る}だろう。
} \label{ex:func_fukushi}
(\ref{ex:func_fukushi})に示す例は,いずれも下線部の副詞が事象「太郎が走
る」の事実性に影響する.(\ref{ex:func_fukushi}a)では,副詞がなければ事象は成立しているが,副詞
「確か」が付加されることによって確信度が下がる.(\ref{ex:func_fukushi}b)は,同様に「果たして」
が付加されることにより,事象成立の確信度は大きく下がり,どちらかといえば
事象は成立しないと読み取れる.(\ref{ex:func_fukushi}c)は,下線部の副詞がなければ,推量を意味す
る機能表現「だろう」により,事象の成立を推量していると読み取れる.しかし
方法を問う副詞「どうしたら」が付加されることにより,事象は成立していない
と読み取れる.また,このとき「だろう」は推量の意味を持たない.

前述の例のように,事象表現の周辺に副詞が存在する場合,事実性に大きな影響を与える場合がある.
副詞は,用法がまとめられた辞書はあるものの~\cite{Hida1994Fukushi},事実
性に及ぼす影響についての研究は進められていない.よって,副詞が事実性に影
響を与える事例を収集するところから着手する必要がある.


\subsubsection{文節境界を越えて事実性に影響を与える語とそのスコープ}

事象表現が含まれる文節よりも文末側に現れる語(図
\ref{fig:sent_structure}では,述語Aに対する述語Bおよび機能表現B)によっ
て,事実性が決定される場合がある.
\eenumsentence{
\item[a.] 太郎は\event{走る}ことを\underline{拒否した}。
\item[b.] 太郎は\event{走る}と言っていたが、\underline{やめた}。
\item[c.] 太郎は\event{走り}も歩きもし\underline{なかった}$_{\mathrm{否
定}}$。
\item[d.] 太郎は\event{走った}が、楽しく\underline{なかっ
た}$_{\mathrm{否定}}$\underline{らしい}$_{\mathrm{伝聞}}$。
} \label{ex:func_followings}
(\ref{ex:func_followings})に「太郎が走る」という事象が,文節境界を越えた
後続の述語や機能表現によって,否定あるいは推量されている事例を示す.(\ref{ex:func_followings}a)およ
び(\ref{ex:func_followings}b)は,下線部の述語によって事象の成立が否定されている.このような述語は,
他の事象表現の事実性に及ぼす影響を決定すること,その及ぼす範囲を決定する
ことが重要である.

(\ref{ex:func_followings}c)は,後続の述語「歩く」に付随する否定の機能表現「なかった」が,
事象「太郎が走る」にも影響して,その事実性が否定であることが示唆される.
一方で,(\ref{ex:func_followings}d)は,後続の述語「楽しい」に否定の機能表現「なかった」と,伝聞の機能
表現「らしい」が付随するが,事象「太郎が走る」の事実性には影響せず,この
事象が成立することが読み取れる.このように,後続の述語に付随する機能
表現が,文節境界を越えて事実性に影響する場合があり,その範囲の同定は,
否定/推量のスコープの問題として知られている.


\subsection{課題分析の方針}
\label{sec:fact_approach}

事実性は,\ref{sec:fact_features}節で述べた各言語要素が単体で影響するだ
けでなく,その組み合わせによって決定される.
\eenumsentence{
\item[a.] 太郎が\event{走ら}\underline{ない}$_{\mathrm{否定}}$というのは
\underline{間違っていた}。(機能表現と後続する述語の組み合わせ)
\item[b.] \underline{たぶん}太郎は\event{走ら}\underline{ない}$_{否定}$。 (副詞と機能表
現の組み合わせ)
} \label{ex:fact_combi}
例えば(\ref{ex:fact_combi}a)は,事象
表現「走る」の直後にある否定の機能表現「ない」と,後続する述語「間違って
いた」が組み合わさって,事象「太郎が走る」が成立することを示している.(\ref{ex:fact_combi}b)
は,副詞と機能表現の組み合わせによって,事象が成立しないことが推量されて
いる.

課題分析においては,複合的に影響する要素は可能な限り切り分けることが重要
である.そこで,事実性が機能表現のみで決定可能であるかという点と,文節境
界を越えて後続する述語や機能表現の影響を受けるかという点の2つに着目して,
課題を切り分ける.機能表現は,\ref{sec:fact_features}節で述べた3種類の要
素の中では,記述的研究に基づいて体系化が進められている領域であ
り~\cite{Morita1989,Endo2003},辞書も整備されている~\cite{Matsuyoshi2007}ため,切り
分けが容易であると考えた.
文節境界を越えて後続する述語や機能表現の影響を
受ける場合とそうでない場合とを切り分けるために,文の主節(日本語の場合は最も文末に近い述語節)
に着目する.主節に位置する事象表現(主事象)は,文節境界を越えて後続する
述語や機能表現を持たないため,その影響を受けることはない.それ以外の事象
(従属事象)は,後続の述語や機能表現の影響を受ける可能性がある.まとめる
と,事実性解析課題の切り分けは図\ref{fig:approach}のようになる.

\begin{figure}[b]
\begin{center}
 \includegraphics{22-5ia3f2.eps}
\end{center}
 \caption{事実性解析課題の切り分け}
 \label{fig:approach}
\end{figure}

課題の切り分けを効率的に行うために,機能表現に基づくルールベースの事実性
解析器を構築する(詳細は\ref{sec:fact_model}節で述べる).本解析器は,各
事象表現について,直接後続する機能表現の意味ラベルのみに基づいて事実性を
決定する.事実性解析において,機能表現のみで事実性が決定可能な事例は少な
くないと考えられる.そのため,偶然正解する可能性の低い解析器を用いること
で,難度の低い事例を分析対象から除外し,課題分析に注力することができる.

本解析器を主事象に対して適用すると,正解事例は機能表現のみで決定可能な事
例であると判断できる.一方で,誤り事例は以下の3種類に分類できる.
\begin{enumerate}
 \item 機能表現の意味ラベルや解析器のルールがナイーブであることが原因で
       誤ったが,機能表現のみで決定可能な事例
 \item 副詞の影響を受けるため機能表現のみでは決定できない事例
 \item その他
\end{enumerate}
次に,従属事象に対して適用すると,正解事例は,主事象と同様に,機能表現の
みで決定可能な事例である.一方で,誤り事例は以下の4種類に分類できる.
\begin{enumerate}
 \item 文節境界を越えて影響を及ぼす述語を持つ事例
 \item 文節境界を越えて影響を及ぼす機能表現を持つ事例
 \item 文節境界を越えて文末側に位置する述語や機能表現の影響を受けず,
       機能表現の意味ラベルや解析器のルールがナイーブであることが原因で
       誤ったが,機能表現または副詞によって決定可能な事例
 \item その他
\end{enumerate}
このような誤り分析によって,事実性解析の性能を向上させるには,どの言語要
素に注力することが重要か,また各要素の部分課題にどの程度の解析性能が要求
されるのかが明らかとなる.


\subsection{課題分析のためのコーパス構築}

本研究では,課題分析のために拡張モダリティタグ付与コーパス
\cite{Matsuyoshi2010,Matsuyoshi2011}を利用する.拡張モダリティタグ付与コー
パスは,『現代日本語書き言葉均衡コーパス』(BCCWJ)\footnote{http://www.ninjal.ac.jp/corpus\_center/bccwj/}を付与対象
としており,そのうちのYahoo!知恵袋データを利用して誤り分析を行う.拡張モダリティタグ
付与コーパスを用いるのは,事実性に関する情報が付与されており,課題分析に
十分な事例数が確保できるとともに,同様の規模の利用可能なコーパスが他にな
いためである.
拡張モダリティタグ付与コーパスには,Yahoo!知恵袋データの他にも,新聞,書籍,白書を対象としたデータも含まれているが,
本研究ではYahoo!知恵袋データを対象に分析を行い,その他のドメインに対する分析は今後の課題とする.
その理由としては以下の2点がある.
\begin{enumerate}
  \renewcommand{\labelenumi}{}
  \item アノテーションガイドライン\footnote{http://cl.cs.yamanashi.ac.jp/nldata/modality/}の付録Bに「OC(Yahoo!知恵袋)の14,089 事象に対しては、実装した解析システムの解析結果をフィードバックさせ、それを参照しながらのタグ見直し作業を数回行い、タグの質を向上させている。」と記載がある.できるだけ信頼性の高いタグで誤り分析を行うため,Yahoo!知恵袋データを利用した
      \item 言論マップ (水野,渡邉,ニコルズ,村上,乾,松本 2011)\nocite{Mizuno2011}や
    対災害SNS情報分析システム (後藤,大竹,De Saeger,橋本,Kloetzer,川田,鳥澤 2013)\nocite{Goto2013} などの,
Webデータを用いたアプリケーションに対する利用を想定している
\end{enumerate}
拡張モダリティタグ付与コーパスには6,362文が収録されている.
6,362文のうち,主事象に機能表現を一つ以上含む文は5,198文ある.
そのうちの約30\%である1,533文をランダムに選択し,課題分析の対象とした.
この中には,主事象が1,533事象,従属事象が2,201事象含まれている.

事実性ラベルは,\citeA{Sauri2009}の体系を一部簡素にした
\citeA{Marneffe2012}によるラベル体系を採用する.本体系は,3種類の確信度
と,3種類の肯否極性の2軸に分けて事実性を定義しており,それぞれの軸で評価
できることが,課題分析に有効であると考えた.拡張モダリティタグ付与コーパ
スの真偽判断タグと,事実性ラベルとの対応を表\ref{tab_def}に示す.表
\ref{tab:fact:distribution}に,本実験の解析対象である1,533文における,主事象と従属事象の事実性の分布を示す.

\begin{table}[b]
\caption{確信度と肯否極性の組み合わせによる事実性のラベル}
\label{tab_def}
\input{03table01.txt}
\vspace{4pt}\small 下段は拡張モダリティタグ付与コーパスの真偽判断タグとの対応
\end{table}
\begin{table}[b]
\caption{コーパス中の事実性の分布}
\label{tab:fact:distribution}
\input{03table02.txt}
\end{table}


事実性は,それを判断したのが著者なのか,あるいは文中の登場人物なのかによっ
て変化する.\citeA{Sauri2009}は,著者以外の文中の登場人物から見た事実性
も考慮してアノテーションを行っている.
以下に\citeA{Sauri2009}によるFactBankのアノテーション例を示す.
\enumsentence{
He does not \textbf{\underline{think}}$_{e_0}$ she \textbf{\underline{followed}}$_{e_1}$ the rules.\\
f($e_0$, {\tt author}) = CT$-$\\
f($e_1$, {\tt author}) = Uu\\
f($e_1$, {\tt he\_author}) = PR$-$
} \label{ex_factbank}
f($e$, {\tt s})は態度表明者{\tt s}から見た事象$e$の事実性を示している.
この文では,著者から見た$e_0$~({\bf think})の事実性がCT$-$,
著者から見た$e_1$~({\bf followed})の事実性がUuであることが付与されるとともに,
文中の登場人物{\it he}から見た$e_1$~({\bf followed})の事実性を著者はPR$-$と判断している,ということが付与されている.
一方で,\citeA{Marneffe2012}は著者から見た事実性のみに焦点を当てている.
日本語においては,\citeA{Matsuyoshi2011}が登場人物ごとの事実性判断をアノテーションしている.
しかしながら,彼らの構築したコーパ
スでは,著者が事実性を判断している事象が,いずれのドメインにおいても9割
前後という大きな割合を占め,著者以外の事実性判断を認識すべき事象は少ない.
これらの事実を背景として,本研究では著者の事実性のみを解析対象とする.著
者以外から見た事実性に関しては,今後の課題とする.


機能表現に関連する問題であるかを切り分けるために,
1,533文に含まれる3,734個の事象について,それに続く機能表現に意味ラベルを
付与した.機能表現の意味ラベルは,記述的研究\cite{Morita1989,Endo2003}に
基づいて体系化された『つつじ』~\cite{Matsuyoshi2007}がある.しかし,
\citeA{Imamura2011}によると,『つつじ』に掲載されていない機能表現および
意味ラベルが存在する.本研究では,『つつじ』に不足する機能表現や意味ラベ
ルを拡充しながら,事象表現に付随する機能表現の意味ラベルを付与した.『つつじ』では89種類の
意味ラベルが定義されているが,1,533文からなる本コーパスに付与された意味
ラベルは66種類であった.『つつじ』には名詞に続く格助詞なども掲載されている
が,それらは本研究では付与対象外であるため,付与されたラベルの種類は少な
くなる.構築したコーパスは,BCCWJとの差分データとして,アノテーション仕
様と合わせて公開している\footnote{http://tinyurl.com/ja-fe-corpus}.


\subsection{誤り分析に用いる事実性解析モデル}
\label{sec:fact_model}

日本語事実性解析における課題分析のため,挙動が明確なルールベースモデルを用いて事実性解析を行う.
事実性解析器の入力は,解析対象となる事象表現,文全体の形態素情報,および事象表現に付随する機能表現の意味ラベル列であり,
入力された事象表現に対して事実性ラベルを付与して出力する.
事実性を付与すべき事象表現の同定に関しては,あらかじめ正解を与える.
形態素情報は,UniDic体系で与えられているが,機能表現はIPA辞書体系であるため,自動でマッピングを行っている.

本研究では,事実性の解析に,各事象表現よりも後ろにある機能表現の意味ラベルを利用する.
例えば,$\langle\text{否定}\rangle$の機能表現が付随している場合には肯否極性を
反転する,といった事実性更新ルールを適用する.
主事象の事実性は,文末から主事象の間に存在する,すべての機能表現の意味ラベル列に基づいて更新ルールを適用することで決定される
\footnote{疑問符も事実性に影響を与える要素として考えられるが,疑問符があっても,事実性がCT+である事象も少なくないため,本研究では採用していない.}.
従属事象の事実性は,従属事象から次の内容語までの間に連なる機能表現の意味ラベル列に基づいて更新ルールを適用することで決定される.
更新ルールは以下の3種類を用いる.
\begin{enumerate}
  \renewcommand{\labelenumi}{}
  \item 機能表現の意味ラベルが$\langle\text{否定}\rangle \langle\text{否定意志}\rangle \langle\text{否定推量}\rangle \langle\text{無意味}\rangle \langle\text{不明確}\rangle \langle\text{不可能}\rangle \langle\text{回避}\rangle 
\linebreak
\langle\text{不必要}\rangle \langle\text{放置}\rangle \langle\text{困難}\rangle$のいずれかの場合,肯否極性を反転する
  \item 機能表現の意味ラベルが$\langle\text{推量-不確実}\rangle$$\langle\text{推量-高確実性}\rangle$$\langle\text{否定推量}\rangle$$\langle\text{意志}\rangle$$\langle\text{否定意志}\rangle$
\linebreak
$\langle\text{伝聞}\rangle$$\langle\text{様態}\rangle$$\langle\text{容易}\rangle$$\langle\text{困難}\rangle$のいずれかの場合,確信度を下げる
  \item 機能表現の意味ラベルが$\langle\text{疑問}\rangle$$\langle\text{勧誘}\rangle$$\langle\text{勧め}\rangle$$\langle\text{願望}\rangle$$\langle\text{依頼}\rangle$のいずれかの場合,事実性をUuにする
\end{enumerate}
更新ルールは,機能表現の意味ラベルの定義およびその表現例にもとづき,人手で設計した.
それぞれの意味ラベルにおける表現例と,コーパス中にその意味ラベルをもつ機能表現が出現した延べ数を表\ref{tab:fe}に示す.
延べ数が0の意味ラベルは,分析対象のコーパスに一度も出現していない意味ラベルである.

\begin{table}[b]
\caption{更新ルールと意味ラベルの対応}
\label{tab:fe}
\input{03table03.txt}
\end{table}

\begin{algorithm}[t]
  \caption{ルールベースの事実性解析モデル}
  \label{alg:model}
\input{03algo01.txt}
\end{algorithm}

ルールベースモデルのアルゴリズムをAlgorithm~\ref{alg:model}に示す.
本モデルは,事象に付随する機能表現に基づく更新ルールを順次適用することで,事象の事実性を決定するモデルとなっている.
以下にこのアルゴリズムによる解析例を示す.
\enumsentence{
小さい方がいい場合も\event{ある}\underline{らしい}$_{伝聞}$\underline{ので}$_{理由}$一概にそうとも\event{言え}\underline{ない}$_{否定}$\\ \underline{みたい}$_{推量\mathchar`-不確実}$\underline{です}$_{判断}$。
} \label{ex:factuality:model}
主事象「言う」の事実性を決定する場合には,付随している3つの機能表現「ない」「みたい」「です」の意味ラベル列である$\langle\text{否定}\rangle$$\langle\text{推量-不確実}\rangle$$\langle\text{判断}\rangle$に基づいて解析を行う.
Algorithm~\ref{alg:model}中の$C$,$P$は,それぞれ確信度,肯否極性の値をもつ変数であり,最終的にこれらの組み合わせで事実性の値を表す.
初期値として,$C$にCT,$P$に+を割り当てる (line 3).
次に,文末側から順に,機能表現の意味ラベルに対応した更新ルールを適用していく (line 4--20).
まず,「です」は$\langle\text{判断}\rangle$の機能表現であり,更新ルール1--3のいずれにも該当しないため,$C$,$P$は更新しない.
次に,「みたい」は$\langle\text{推量-不確実}\rangle$の機能表現であり,更新ルール2に該当するため,$C$をPRに更新し,$P$は更新しない (line 12--16).
最後に,「ない」は$\langle\text{否定}\rangle$の機能表現であり,更新ルール1に該当するため,$C$は更新せず,$P$を$-$に更新する (line 5--11).
結果的に,$C=PR$,$P=-$となり,主事象「言う」の事実性としてPR$-$が得られる (line 21).
従属事象「ある」の場合は,直後に連なる機能表現列である「らしい」「ので」の意味ラベル列$\langle\text{伝聞}\rangle$$\langle\text{理由}\rangle$に基づいて,更新ルール2のみを適用する (line 12--16).
その結果,従属事象「ある」の事実性はPR+となる.

\begin{table}[b]
\caption{機械学習モデルで用いた素性一覧および (\ref{ex:factuality:model}) における素性抽出例}
\label{tab:zunda}
\input{03table04.txt}
\end{table}

本モデルは機能表現の意味ラベルのみを用いたシンプルなモデルであるため,必
要以上に多く誤解析してしまう恐れがある.そこで,既存の素
性~\cite{Eguchi2010_nlp}を,オープンソースのモダリティ解析器
Zunda~\cite{Mizuno2013}\footnote{https://code.google.com/p/zunda/}
に実装することで,リファレンスとなる解析性能を得る.Zundaは,拡張モダリ
ティタグ体系に基づいて,タグごとに線形分類器による多クラス分類を行う.ま
ず,真偽判断タグのラベルを表~\ref{tab_def}に基づいて本研究の事実性ラベル
に置き換える.他の5種類のタグについては,拡張モダリティタグをそのまま採
用する.次に,素性は,\citeA{Eguchi2010_nlp}で利用されている素性の
うち,リソースが利用可能なものを利用する.表~\ref{tab:zunda}に,利用した
素性の一覧と(\ref{ex:factuality:model})から抽出される素性の例を示す.
「事象選択述語が示唆する事実性」は,5.1節で詳述するが,解析対象の素性と
して述語が含まれる文節の係り先文節に含まれる述語が示唆する事実性である.
例えば「たばこを / \event{吸う}のを / \event{やめる}。」について,「やめ
る」は係り元文節中の「吸う」がCT$-$であることを示唆する述語であることから,
「吸う」を解析するとき,その事実性がCT$-$であることが示唆されるという素性
を抽出する.最後に,分類器について,\citeA{Eguchi2010_nlp}は事象間
の依存構造が考慮できるFactorial CRF~\cite{Sutton2007}を利用していたが,
Zundaは
LIBLINEAR~\cite{REF08a}\footnote{http://www.csie.ntu.edu.tw/{\textasciitilde}cjlin/liblinear/ の1.80を利用した.}を利用している.
事象間の依存関係を考慮するため,解析対
象の事象より文末側にあり,かつ最も近傍にある事象の拡張モダリティタグのう
ち,真偽判断と態度の2つについて,その解析結果を素性として利用する.
例えば(\ref{ex:factuality:model})では,解析対象が「ある」のときに,素性として「言う」の解析結果を利用する.
LIBLINEARの学習アルゴリズムは,L2正則化ロジスティック回帰を利用し,パラ
メータはweightを0に設定した以外はデフォルトの値を利用した (epsilon$=0.1$,
cost$=1$, bias$=-1$).評価は10分割交差検定によって行う.文単位で分割するこ
とによって,同一文中の複数の事象が学習データとテストデータに属することは
ない.交差検定の段階では,主事象と従属事象は区別せずに学習させるが,精度
と再現率を算出する段階では,主事象と従属事象を区別する.

ルールベースによる解析モデルを主事象に適用し,誤り分析を行うことで,機能表現のみで事実性
が決定可能な事例の割合を明らかにするとともに,副詞の影響を受ける事例がど
の程度存在するのか,また,その他の要素はどのようなものがあるのかを明らか
にする.
次に,ルールベースによる解析モデルを従属事象に適用し,誤り分析を行うことで,
機能表現以外の事実性を決定するための要素に関して,その重要性を定量的に分析し,事実性解析の今後の方針を議論する.



\section{主事象に対する事実性解析}
\label{sec_matrix}

\begin{table}[b]
\caption{主事象に対する事実性解析の評価}
\label{tab:eval:mat:fact}
\input{03table05.txt}
\end{table}

\ref{sec:fact_model}節で構築した事実性解析器を主事象に対して適用し,誤り
分析を行うことで,機能表現のみで決定可能な事象,副詞の影響を受ける事象,
その他の3種類に分類する.対象となる事象は1,533事象あり,その解析結果を表
\ref{tab:eval:mat:fact},\ref{tab:eval:mat:axis}に示す.表
\ref{tab:eval:mat:fact}には,確信度と肯否極性を組み合わせた事実性の各ラ
ベルにおける精度,再現率,F値,およびそれらのマイクロ平均,マクロ平均を
示した.表\ref{tab:eval:mat:axis}には,確信度と肯否極性の二軸それぞれに
おける精度,再現率,F値,およびそれらのマクロ平均を示した.これらの結果
から,機能表現のみを利用したシンプルなルールベースモデルであっても肯否極
性は高い精度,再現率で判定可能であることが分かる.一方で,確信度について
は,PRの分類性能は高くない.
また,機械学習ベースのモデルでは,機能表現などの素性も導入されているものの,十分な性能があげられていない.
これは,事例数の偏りや,機能表現の多様性などの要因により,事実性解析が簡単な課題ではないことを示している.
ルールベースモデルと機械学習ベースのモデルとを比較すると,
全体の事例数が少なく,大きな偏りもあるため,機械学習ベースのモデルの方が若干不利ではあることを考慮しても,
ルールベースモデルは,機械学習ベースのモデルと遜色ない性能を示している.
このことから,本ルールベースのモデルの性能は極端に低いわけではなく,このモデルを用いた誤りを分析することで,事実性解析の課題分析を行うのは妥当であるといえる.

\begin{table}[b]
\caption{主事象に対する事実性解析の各軸ごとの評価}
\label{tab:eval:mat:axis}
\input{03table06.txt}
\end{table}
\begin{table}[b]
\caption{誤りの種類の分布}
\label{tab:error:mat}
\input{03table07.txt}
\vspace{4pt}\centering\small カッコ内は,事実性のアノテーション誤りを除いた部分での誤りの割合
\end{table}

前述の実験で正解した事象は,機能表現のみで決定可能な事象であると判断する
ことができる.残る240個の誤り事例を分析することによって,機能表現の意味
ラベルあるいは事実性解析モデルが原因による誤り事例,副詞の影響を受ける事
例,その他の事例に分類する.誤り分析の結果を表\ref{tab:error:mat}に示す.

機能表現のみで決定可能な事例が4割以上と,まだ多く残されている.
本実験で用いたルールは人手で構築しているため,ルールの改善の余地が残されている.
このようなルールが改善でき,機能表現をしっかり捉えることが出来るようになると,すでに正解できている1,293事例と合わせて,
機能表現のみで90.5\% (1,387/1,533) の正解率をあげられることがわかった.
\eenumsentence{
\item[a.] ビタミンは、野菜や海草から\event{補給する}\underline{べき}$_{当為}$\underline{です}$_{判断}$。\\ (正解ラベルに基づく解析:CT+,正解:Uu)
\item[b.] 入札前に\event{確認す}\underline{べき}$_{当為}$\underline{でし}$_{判断}$\underline{た}$_{完了}$\underline{ね}$_{態度}$。\\(正解ラベルに基づく解析:CT+,正解:CT$-$)
\item[c.] 大至急オーストラリアへ書類を\event{送ら}\underline{なくてはなりません}$_{当為}$。\\(正解ラベルに基づく解析:CT+,正解:PR+)
} \label{ex:factuality:addrule}
(\ref{ex:factuality:addrule})は,機能表現だけで事実性を決定できるものの,現在の更新ルールが不足しているために誤った事例である.
このようなルール不足に起因する誤りとしては,$\langle\text{当為}\rangle$や$\langle\text{不許可}\rangle$のように,
更新ルールを割り当てるべき意味ラベルを追加することで改善が期待できる事例が見られた.
(\ref{ex:factuality:addrule}a)の機能表現「べき」「です」はそれぞれ$\langle\text{当為}\rangle$$\langle\text{判断}\rangle$の意味をもっているが,
これは,$\langle\text{当為}\rangle$に関する更新ルールが不足していたことによる誤りである.
(\ref{ex:factuality:addrule}a)は$\langle\text{当為}\rangle$を更新ルール3の適用対象に加えれば解決する問題である.
今回の分析対象のコーパス中に,$\langle\text{当為}\rangle$が付随している主事象は,8事例見られた.
そのうち,更新ルール3を変更することによって,もともと正解できていた事例が2事例,誤りだった事例が正解できるようになる事例が4事例,
もともと誤っており,更新ルール3を変更しても正解できない事例が2事例あり,正解できていた事例が誤りとなるような事例は見られなかった.
更新ルール3を変更しても正しく解析できない事例を
(\ref{ex:factuality:addrule}b),(\ref{ex:factuality:addrule}c)に示す.(\ref{ex:factuality:addrule}b)では,「確認する」に付随する機能表現列は,$\langle$当為
$\rangle$$\langle\text{判断}\rangle$$\langle\text{完了}\rangle$$\langle$態度
$\rangle$であるから,その事実性はCT$+$となるが,正解はCT$-$である.この
場合,更新ルール1の適用対象として$\langle\text{当為}\rangle$と$\langle$完了
$\rangle$の組み合わせを追加し,更新ルール3は適用しないように変更すること
で正しく解析することができる.$\langle\text{当為}\rangle$と$\langle$完了
$\rangle$の両方が付随する事例は,コーパス中に1事例のみであることから,こ
の変更による悪影響はない.本分析では,1,533文を分析したが,このように一
度しか現れない機能表現のパターンがある.従って,更新ルールを洗練するには,
規則ベース,学習ベースのいずれのアプローチをとるにせよ,機能表現の意味ラ
ベルのアノテーションを拡充していく必要があるであろう.
(\ref{ex:factuality:addrule}c)は,「大至急」という副詞があることからPR+と判断されている.
この事例は$\langle\text{当為}\rangle$を考慮するだけでは不十分で,「大至急」という副詞を考慮しなければならない.


一方で,副詞の影響を加味する必要がある事例は半分近くにのぼった.
\eenumsentence{
\item[a.] \underline{やはり}この御時世、\event{きつい}のではない\underline{でしょうか}$_{疑問}$?\\ (正解ラベルに基づく解析:Uu,正解:PR+)
\item[b.] \underline{どうやって}\event{判別し}てる\underline{んでしょうか}$_{疑問}$?\\(正解ラベルに基づく解析:Uu,正解:CT+)
} \label{ex:factuality:discussion}
(\ref{ex:factuality:discussion})は,機能表現だけは不十分であり,副詞と機能表現とを組み合わせる必要がある事例である.
いずれの事例も,$\langle\text{疑問}\rangle$の機能表現が後続しているため,主事象の事実性はUuと解析された.
しかしながら,
(\ref{ex:factuality:discussion}a)は,問いかけではあるものの,推量の意味合いが強いため,正解はPR+となっている.
(\ref{ex:factuality:discussion}b)は,前提として起こった事象である「判別する」の方法を問う文であるため,CT+が正解である.
このような事象の事実性を決定するためには,$\langle\text{疑問}\rangle$の機能表現を利用するだけでは不十分であり,
「やはり」や「どうやって」のような副詞を手がかりとし,それらを組み合わせて解析する必要がある.


\enumsentence{
\underline{おそらく}ただの\event{見栄っ張り}です。\\ (正解ラベルに基づく解析:CT+,正解:PR+)
} \label{ex:factuality:adverb}
(\ref{ex:factuality:adverb})では,事実性に影響を与えるような機能表現は付随していないが,その代わりに副詞「おそらく」によって事実性が決定されている.
どのような副詞が事実性に影響を与えるかを分類し,手がかりとして捉える必要がある.

また,機能表現や副詞のみでは決定できない,その他の誤りとして,機能表現の省略による誤り事例が見られた.
\enumsentence{
とれないので\event{注意}!!\\ (正解ラベルに基づく解析:CT+,正解:Uu)
} \label{ex:notfunc}
例えば(\ref{ex:notfunc})では,事象「注意」で文が終わっており,機能表現が存在していないが,依頼の意味をもつ文であることが解釈できる.
しかしながら,機能表現のみに基づいた解析では,機能表現が存在していないために依頼の意味を捉えられず,正しく解析することができない.
そこで,文末の感嘆符など,機能表現以外の要素を利用して解析を行わなければならない.
また,「注意」で文が終わる場合には依頼文であることが多いと予測できるため,事象自身の情報を利用することで,
「注意」で終わる場合には依頼であると判定する,といったことが考えられる.

形態素解析やアノテーションの誤りについて,正しい情報が与えられた場合についても検証した.まず,形態素解析および機能表現のアノテーション誤りが解消された場合,事実性も正しく解析可能であることが分かった.また,事実性のアノテーション誤りについては,システムが出力したラベルの方が正しいことが分かった.

以上より,主事象の事実性解析については,機能表現の意味ラベルが正しく解析できれば,現在の意味ラベルの体系と本研究で用いた単純な規則だけでも,90\%に近い正解率が得られることがわかった.3節で述べたように,現在の機能表現の意味ラベルは,既存の記述的研究に基づいた体系になっているが,これが事実性解析に最適な体系になっているかを評価することは容易ではない.しかし,現在の体系でも90\%に迫る正解率が得られる余地があることは,この体系に基づく機能表現の解析モデルを研究開発することに一定の支持を与えるものと考える.今後は,\citeA{Kamioka2015}のような機能表現解析の研究に注力したい.

もう一つの大きな課題は副詞の扱いである.今回得られた誤りの半数近くは副詞に起因するものであった.意味解析における副詞の扱いは先行研究も乏しく,辞書の整備を初め,やるべき課題は多い.まずは事実性解析という切り口で,それに関連する情報に焦点を当ててリソースを設計・開発していく予定である.


\section{従属事象における事実性解析}
\label{sec_subord}

表\ref{tab:eval:sub:fact}, \ref{tab:eval:sub:axis}に,2,201
の従属事象に対して事実性解析器を適用した結果を示す.
主事象の場合と比較すると,全体の性能は下がっており,従属事象の方が解析が難しいことがわかる.
機械学習ベースのモデルと比較すると,主事象の場合と同様に,
ルールベースモデルが,機能表現等が素性に入った機械学習ベースのモデルと遜色ない性能を示している.
従属事象の場合においても,本ルールベースのモデルの性能は極端に低いわけではなく,
このモデルを用いた誤りを分析することで,事実性解析の課題分析を行うのは妥当であるといえる.

\begin{table}[b]
\caption{従属事象に対する事実性解析の評価}
\label{tab:eval:sub:fact}
\input{03table08.txt}
\end{table}
\begin{table}[b]
\caption{従属事象に対する事実性解析の各軸ごとの評価}
\label{tab:eval:sub:axis}
\input{03table09.txt}
\end{table}

従属事象において,機能表現以外に考慮すべき要素として,どのような要素が重要なのかを定量的に分析するために,正解ラベルを用いた場合の誤り分析を行う.
主事象においては機能表現が付随している事象のみ扱ったが,機能表現が付随した主事象を含む文における従属事象を対象としているため,
必ずしもすべての従属事象に機能表現が付随しているとは限らない.
そこで,機能表現が付随している事象であるか否かをまず分類する.
また,名詞述語なのか,動詞や形容詞といった述語なのかが,事実性解析の難易度に影響していると考えられるため,
名詞述語なのか否かに従属事象を分類する
\footnote{単純に事実性が付与された名詞をすべて名詞述語と分類するわけではないため,人手による分類を行っている.
例えば「判別する」「判別できる」など,「〜する」「〜できる」が付随する場合には,まとめて動詞述語として扱うが,
「判別をする」といった場合には「判別」を名詞述語,「する」を動詞述語として扱い,「判別」が「する」の項になっていると判断する.}.
510の誤り事例の中から,200事例をランダムにサンプリングし,誤り分析を行った.

従属事象特有の誤りとしては,後続する述語の影響が事実性を決めている場合と,さらにその後ろの機能表現が事実性を決めている場合の2種類が考えられる.
\eenumsentence{
\item[a.] 安いものだと\event{防水}は怪しいです。\\
(正解:PR$-$,システム:CT+)
\item[b.] 物語を\event{楽しみ}つつ、\event{冒険}を堪能してください。\\
(「楽しむ」正解:Uu,システム:CT+,「冒険する」正解:Uu,システム:CT+)} \label{ex:suberror}
(\ref{ex:suberror})は,機能表現の正解ラベルを用いても正解できなかった従属事象の例である.
まず(\ref{ex:suberror}a)の従属事象「防水する」は,事実性の正解がPR$-$であるが,
「防水する」自身に付随している機能表現だけでこれが決まっているわけではなく,後続する述語であり,疑いをもっていることを示す表現である「怪しい」の影響が大きい.
このように,従属事象に後続する述語が事実性を決めている場合を「後続する述語の影響」による誤りと分類した.
このような事象の事実性を解くためには,どの述語が事実性にどういった影響を与えるかを分類する必要がある.
次に(\ref{ex:suberror}b)の「楽しむ」および「冒険する」は,正解がUuであるが,(\ref{ex:suberror}a)と同様に,それぞれに付随している機能表現だけでは事実性は決定できない.
しかしながら,後続する述語である「堪能する」自身にそういった影響があるとはいえない.
これは,「堪能する」を抽象的な述語「する」に置き換えた場合でも,「楽しむ」および「冒険する」の事実性がUuと判断できることから明らかである.
ではどういった要素が「楽しむ」および「冒険する」の事実性をUuにしているかというと,
「堪能する」に付随する機能表現「ください」が影響を与えているということが考えられる.
このように,後続する述語ではなく,さらにその後ろに見られる機能表現が広く影響を与えているために,
事実性をうまく解析できない事例を「後方の機能表現が影響する範囲」による誤りと分類した.
このような事象の事実性を解くために,機能表現がどの事象までその影響を与えるのかを解析する必要がある.
「後続する述語の影響」と「後方の機能表現が影響する範囲」とを区別する基準としては,
後続する述語を「する」などの抽象的な述語に置き換えた場合に,事実性が変化するかどうかを考える.
例えば(\ref{ex:suberror}a)において「怪しいです」を「しています」と置き換えた場合,事実性が全く異なってしまう.
このような述語の置き換えを一つの判断基準として,誤りの分類を行った.

誤り分析の結果を表\ref{tab:error:sub}に示す.
主事象の場合と同様の誤りも見られたが,従属事象特有の誤りが誤り全体の6割を占めた.
特に,機能表現が付随していない名詞述語においては,機能表現が影響する範囲を考慮すべき誤りが大半を占めていることから,機能表現が影響する範囲を捉えることの重要性を示している.
以降の節では,後続する述語や機能表現が影響する範囲のような従属事象特有の問題を解決するために,既存のコーパス中における現象を分析することで,今後の方針について議論する.

\begin{table}[t]
\makeatletter
\def\@cline#1-#2\@nil{
  \noalign{\vskip-\arrayrulewidth}
  \omit
  \@multicnt#1
  \advance\@multispan\m@ne
  \ifnum\@multicnt=\@ne\@firstofone{&\omit}\fi
  \@multicnt#2
  \advance\@multicnt-#1
  \advance\@multispan\@ne
  {\CT@arc@\leaders\hrule\@height\arrayrulewidth\hfill}
  \cr}
\makeatother
\caption{誤りの種類の分布}
\label{tab:error:sub}
\input{03table10.txt}
\par\vspace{4pt}\small カッコ内は,事実性のアノテーション誤りを除いた部分での誤りの割合
\par\vspace{-0.5\Cvs}
\end{table}



\subsection{事象参照表現に後続する述語に関する分析}

「あり得る」のような,事実性に影響を与える述語(以降,事象選択述語と呼ぶ)については,\citeA{Eguchi2010_nlp}が構築した辞書がある.
\citeA{Eguchi2010_nlp}は,拡張モダリティを解析する手がかりとして利用するために,拡張モダリティに影響を与える表現を収録した辞書(以降,事象選択述語辞書と呼ぶ)を構築した.
事象選択述語辞書は,行為・出来事を表す事象を必須格にとり得る述語を対象に,分類語彙表\cite{BunruiGoihyo}に収録されている述語の中から,
拡張モダリティに影響を与える8,580述語を収録している.
事象選択述語辞書の項目の例を表\ref{tab_mkd}に示す.
この辞書は,各述語が格にとる事象に与える影響を,直前の事象の時制
および,述語の肯否極性ごとに収録している\footnote{この辞書は http://bit.ly/ja-esp-dicより入手可能である.}.
この辞書のうち,真偽判断の項目が,事実性解析に利用できると考えられる.
\eenumsentence{
\item[a.] 問題が\event{発生する}のを防いだ。
\item[b.] 問題が\event{発生する}のを防がなかった。
} \label{ex_pred}
例えば(\ref{ex_pred})の「防ぐ」という述語は,
\pagebreak
(\ref{ex_pred}a)のような肯定環境下では不成立,(\ref{ex_pred}b)のような否定環境下では成立というように,事象「発生する」の肯否極性に影響を与える.
(\ref{ex_mkd})の「忘れる」は,直前の事象の時制を考慮した例である.
\eenumsentence{
	\item[a.] 彼は\event{発言し}{たのを忘れ}ている。
	\item[b.] 彼は\event{発言する}{のを忘れ}ている。
} \label{ex_mkd}
事象「発言する」に対して,(\ref{ex_mkd}a)では過去に成立している事象であるが,
(\ref{ex_mkd}b)では「発言する」ことが実際には起こっておらず,不成立である.

\begin{table}[t]
\caption{事象選択述語辞書の記述例}
\label{tab_mkd}
\input{03table11.txt}
\end{table}

事象選択述語に関する問題は,このような既存の辞書を手がかりとして解決できると考えられる.
現在の辞書のカバレッジを見積もるため,表\ref{tab:error:sub}において,後続する述語の影響が原因とされた誤りである25事例を対象に,事象選択述語辞書がカバーできているかどうか,を人手で分類した.
例えば(\ref{ex:suberror}a)の「怪しい」といった述語が辞書中に登録されているかを判断する.
そして,「怪しい」が辞書中に登録されている場合,登録されている情報を利用すれば正しく事実性ラベルを選択できるのか,
即ち(\ref{ex:suberror}a)では,「直前の事象の時制が未来」であり,「述語自身の肯否極性が成立」である場合に,辞書に「真偽判断が低確率」と登録されているかどうか,を人手で判定した.
このとき,直前の事象の時制や述語自身の肯否極性も人手で判定を行った.
その結果,25事例のうち20事例については,事象選択述語が辞書に収録されており,辞書の情報を利用すれば正しく事実性ラベルを選択できることがわかった.現在の辞書でも事実性解析の精度向上に貢献できることを示している.
残りの5事例についても,現在の辞書には収録されていないものの,適切な情報が辞書に収録されていれば,辞書情報を用いて正しく事実性ラベルを選択することができる.
現在の辞書でカバーできていた述語とカバーできていなかった述語を表\ref{tab:esp}に示す.「気がある」「関係ある」などの複合表現が現在の辞書でカバーできていない傾向が見られ,こうした多様な表現の獲得が今後重要な課題として浮かび上がった.

\begin{table}[t]
\caption{誤り事例における事象選択述語}
\label{tab:esp}
\input{03table12.txt}
\vspace{-1\Cvs}
\end{table}


\subsection{事象間の接続表現に基づくスコープに関する分析}

(\ref{ex:suberror}b)のように,機能表現が直接付随する事象だけでなく,従属事象にまで影響を与えることによって解析に失敗した事例が,従属事象における誤りの4割以上を占めた.
このような後方の機能表現が影響する範囲による誤りを解消するために,どのような情報が利用できるのかを分析する.
本研究では,機能表現が影響する範囲を決定する問題を,機能表現のスコープを認識する問題として扱う.
スコープとは「否定などの作用が及ぶ範囲」\cite{Grammar3}であり,
(\ref{ex:sc})では,角括弧で囲まれた範囲が否定を表す機能表現のスコープとなる.
\eenumsentence{
\item[a.] [仕事で\event{行っ}た] の\underline{ではない}$_{否定}$。
\item[b.] \event{残念}なことに、[鈴木さんは\event{来}] \underline{なかっ}$_{否定}$た。
} \label{ex:sc}
(\ref{ex:sc}a)では,「仕事で行った」という事象が否定されている.
(\ref{ex:sc}b)では,「鈴木さんは来た」という事象が否定されており,「残念である」という事象は否定されていない.
一方で,(\ref{ex:suberror}b)では,主事象に付随する「ください」の影響が,従属事象である「楽しむ」「冒険する」にも影響を与える.
現在のモデルでは,スコープを機能表現の直前の事象のみとして解析を行うため,(\ref{ex:suberror}b)は正解できなかった.
そこで,スコープを必要に応じて広げ,機能表現の影響をスコープ内の事例に与えることで,後方の機能表現が影響する範囲による誤りを解消することができる.
機能表現のスコープを広げるべき場合とそうでない場合とを認識するために,どのような情報が利用できるのか,それらの事例の割合はどの程度なのかを分析する.

\citeA{Minami1974}は,従属節内の要素の表れ方に基づき,従属節を接続助詞で分類している.
\eenumsentence{
\item[a.] [タバコを飲むが]ガンのことは心配していない。
\item[b.] [タバコを飲みながら]おしゃべりしている。
} \label{ex:minami}
例えば(\ref{ex:minami}a)では,従属節の述語的部分「飲むが」には,「飲まないが」「飲んだが」「飲みますが」「飲むだろうが」などのさまざまな要素を入れられる.
一方(\ref{ex:minami}b)では,「*飲まないながら」「*飲んだながら」「*飲みましながら」「*飲むだろうながら」などを用いることは出来ず,表れる要素が制限されている.
これは,「〜ながら」を伴う従属事象では,主事象に付随する機能表現が否定やモダリティなどを表しており,
接続表現「〜ながら」によってスコープが広がっていることを示唆している.
また,\citeA{Arita2007}は日本語の時制節性に着目することで,\citeA{Minami1974}の分類がさらに分類できることを示している.
\citeA{Takubo2010}は,\citeA{Minami1974}の分類を一部修正し,その分類をもとに疑問の焦点やスコープに関して議論している.
このように,主事象と従属事象をつなぐ接続表現の差によって,スコープの判断に接続表現を利用することが考えられる.
そこで,実際にコーパス中に含まれる文を対象に,機能表現のスコープが従属事象にまで及んでいるかどうかを接続表現ごとに分類することで,
スコープを考えるべき事例がどの程度存在するのか,接続表現がスコープ解析ならびに事実性解析に利用できるのか,を明らかにする.

\begin{table}[b]
\vspace{-0.5\Cvs}
\caption{誤り事例における接続表現の分類}
\label{tab:scope}
\input{03table13.txt}
\end{table}

我々が分析の対象とした従属事象の誤り200件のうち,後方の機能表現を考慮しなかったことによる誤りは80件あった(表\ref{tab:error:sub}の「後方の機能表現が影響する範囲」).これらは,(\ref{ex:suberror}b)の従属事象「楽しむ」の事実性のように,文節境界を越えた後方の機能表現(この例では「ください」)を事実性推定に考慮していないことによる誤りである.これらの従属事象の事実性は後方の機能表現の影響を受けるので,それぞれの従属事象は後方の機能表現のスコープの中に入っていることになる.上の80件の従属事象がそれぞれ後方の機能表現にどのように繋がっているかのパターンを調べると表\ref{tab:scope}のような分布が得られた.主なパターンは次のとおりである.
\begin{description}
\item[直接の項] 従属事象(「冒険」)が上位事象(「堪能し」)の項になっており,上位事象に付随する機能表現(「ください」)の影響を受けるパターン
\enumsentence{物語を楽しみつつ、\event{冒険}を\event{堪能し}て\underline{ください}$_{依頼}$。}
\item[テ形接続] 従属事象(「活かし」)がテ形接続で後続事象(「働く」)に係っており,その後続事象の機能表現(「なかっ」)の影響を受けるパターン
\enumsentence{うまく\event{活かし}て\event{働く}ことができ\underline{なかっ}$_{否定}$た。}
\item[項を修飾] 従属事象(「難しい」)が後続の事象表現(「ある」)の項になっている名詞(「試験」)を修飾しているパターン
\enumsentence{そんなに\event{難しい}試験が\event{ある}\underline{のでしょうか}$_{疑問}$?}
\item[名詞述語を修飾] 従属事象(「質問し」が名詞述語(「子かな」)の名詞を修飾しており,その名詞述語の機能表現(「かな」)の影響を受けるパターン
\enumsentence{昨日楽譜何がいいって\event{質問し}た\event{子}\underline{かな}$_{疑問}$。}
\end{description}

これらのパターンについては,事実性解析時に後続の機能表現の影響を考慮する必要があるが,そのためには当該の従属事象が後続の機能表現のスコープ内にあるかどうかを正確に判別する必要がある.そこで,こうした機能表現のスコープの分布についてさらにデータを拡充して調査を行った.

\begin{table}[b]
\caption{ランダムに抽出した140文中の従属事象の分布}
\label{tab:subdist}
\input{03table14.txt}
\end{table}

拡張モダリティタグ付与コーパスのうち,2個以上事象が含まれており,かつ,主事象の事実性がCT+ではない文を140文ランダムに抽出した.
主事象の事実性がCT+でない文では,主事象の後ろに何らかの機能表現が付随している場合が多いため,今回の分析目的にかなうと考えられる.
140文中には事象表現が全部で440個含まれ,そのうち主事象が140個,従属事象が300個であった.この300個の従属事象を対象に,主事象に付随する機能表現のスコープ内に従属事象が入っているか,主事象と従属事象の間にどのような接続パターンが見られるかを人手で調査した.ただし,当該の従属事象が主事象から表層的に離れている場合は,隣接する場合にくらべて主事象に付随する機能表現のスコープ内には入りにくいと予測されるので,表\ref{tab:subdist}では,上記300個の従属事象をさらに主事象に隣接する事例140個とそれ以外の160個に場合分けして集計した.
ここでいう「隣接」とは,係り受け関係にある事象の中で最も表層上近いものを指す.
係り受けは,CaboCha~\cite{CaboCha}による自動解析結果を利用した.

まず,主事象から離れた従属事象160個について,従属事象が主事象と同じスコープ内に入っているかどうかを調べた.表\ref{tab:subdist}に示すように,スコープ内に入っている従属事象が11個,スコープ外にある従属事象が147個,後方の機能表現ではなく事象選択述語の影響を加味すべき事象が2個であり,「スコープ外」への偏りが極めて大きいことがわかった.すなわち,主事象から離れた従属事象が主事象の機能表現の影響を受けることは極めてまれで,その可能性を事実性解析プロセスの中で考慮しても精度のゲインはほとんど期待できない.

\begin{table}[b]
\vspace{-0.5\Cvs}
\caption{主事象と最も近い従属事象との間の接続表現の分類}
\label{tab:scope:add}
\input{03table15.txt}
\end{table}

つぎに,当該従属事象が主事象に隣接している事例140個の分布を表\ref{tab:subdist}に示す.上段の「スコープ内が見られた表現」には,従属事象が主事象に付随する機能表現のスコープ内に入っている場合が一度でも観察された接続パターンを並べた.「〜てから」のように表\ref{tab:scope}に入っているが,上記140個の事例の中には出現しなかったものも含めてある.表\ref{tab:scope}と表\ref{tab:scope:add}を合わせると興味深い知見が得られる.表\ref{tab:scope}の誤りを解消するためには,主として「直接の項」「テ形接続」「項を修飾」「名詞述語を修飾」などの接続パターンのスコープを決定する必要があるが,このうち「直接の項」をのぞく 3 つのパターンはいずれもスコープ内外の選択が高度に曖昧であり
(例えば「テ形接続」は「スコープ内」が5件,「スコープ外」が9件),これらのパターンのスコープを決定する課題に注力することに一定の効用があることがわかる.
(\ref{ex:scopeamb})に「テ形接続」でスコープ内外が異なる例を示す.
\eenumsentence{
\item[a.] うまく\event{活かし}て\event{働く}ことができ\underline{なかっ}$_{否定}$た。(スコープ内)
\item[b.] 諸事情が\event{あっ}て\event{離婚する}ことができ\underline{なかっ}$_{否定}$た。(スコープ外)
} \label{ex:scopeamb}
一方,「直接の項」については,つねにスコープ内であると判断してもよい.また,「〜が」「〜ので」「〜たら」などの接続パターンは「スコープ外」への偏りが大きく,決定的に「スコープ外」と決めても大きなリスクにはならない可能性がある.
その他の接続パターンに関しても,ある程度の偏りが見られ,規則ベースで決めても問題はないと考えられる.離れた事象と比較して,隣接する事象のほうがスコープ内に入る場合が多いことから,事実性解析プロセスの中で隣接事象のスコープを考慮することによって,ある程度のゲインが期待できる.

隣接事象のスコープを考慮することが,事実性解析の性能改善に繋がるのかを検証するために,隣接事象のスコープを付与し,それを考慮した解析モデルを適用して,誤り分析を行う.まず,隣接事象対に対して同じスコープ内に入るかを人手で付与する.例えば (\ref{ex:scopeamb}a) では,「活かす」と「働く」は同じスコープ内に入ると付与し, (\ref{ex:scopeamb}b) では,「ある」と「離婚する」は同じスコープに入らないと付与する.次に,解析モデルを,スコープを考慮したものに拡張する.同じスコープに入ると付与された事象対について,前件の事象(文頭側の事象)については,自身に付随する機能表現の意味ラベル列に加えて,後件の事象(文末側の事象)に付随する機能表現の意味ラベル列についても考慮して,事実性の更新ルールを適用する.
例えば (\ref{ex:scopeamb}a) では「活かす」と「働く」が同じスコープ内であり,
 (\ref{ex:scopeamb}b) では「ある」と「離婚する」が同じスコープ内にはない,というアノテーションを行う.
このアノテーションを利用し,3.4節で述べた解析モデルを拡張することで,事実性の解析を行う.
具体的には,「同じスコープ内である」と付与された事象対のうち,前件の事象については,前件の事象自身に付随する機能表現の意味ラベル列に加えて,後件の事象に付随する機能表現の意味ラベル列に基づいた更新ルールを適用することで,事実性を決定する.
例えば (\ref{ex:scopeamb}a) では「活かす」と「働く」が同じスコープ内であるため,「活かす」の事実性を決定する際には,「活かす」自身の機能表現がもつ更新ルールを適用する(今回は更新ルールをもつ機能表現は付随していない)だけでなく,「働く」に付随する機能表現である「なかっ」がもつ更新ルール1も適用する.

\begin{table}[b]
\vspace{-0.5\Cvs}
\caption{スコープのアノテーションによる事実性解析性能}
\label{tab:scope:result}
\input{03table16.txt}
\end{table}
\begin{table}[b]
\caption{スコープのアノテーションによる事実性解析結果の変化}
\label{tab:scope:change}
\input{03table17.txt}
\end{table}

1,533文のうち,2個以上事象が含まれており,かつ,主事象の事実性がCT+ではない441文を抽出し,
その中で係り受け関係にある900事象対に対してスコープのアノテーションを行った.
その結果,同じスコープ内に入ると判断されたのは120事象対であった.
これらの事象対のうち,後件の事象の事実性はスコープに関わらず変化しないが,前件の事象の事実性はスコープを利用することによって,
後件の事象に付随する機能表現の影響を受けて変化する.
前件の事象120事象における事実性解析の性能を表\ref{tab:scope:result},事実性解析性能の変化を表\ref{tab:scope:change}に示す.
事例数の変化を見ると,改善事例が多く,36事例見られたものの,スコープを考慮しても誤る事例も51事例見られた.
しかしながら,その誤り原因を確認すると,51事例のうち32事例は事実性のアノテーション誤りであり,システムは正しく事実性を解析することができていた.
それ以外の事例において,スコープを考慮しても正解できなかったものとしては,以下の事例がある.
\enumsentence{
あなた自身が\event{貯金する}くせを\event{つけ}\underline{ないと}$_{当為}$。\\
(「つける」正解:Uu,スコープ無:CT+,スコープ有:CT+)\\
(「貯金する」正解:Uu,スコープ無:CT+,スコープ有:CT+)
} \label{ex:inscope:error}
(\ref{ex:inscope:error})では,「貯金する」と「つける」が同じスコープ内にあると判断された事例であるが,
従属事象「貯金する」だけでなく,主事象「つける」も誤りとなっている.
主事象「つける」が誤った原因は,主事象における誤り分析で述べたように,
主事象に付随する機能表現である$\langle\text{当為}\rangle$のルールが不足していることである.
このルールが追加されれば,主事象の改善とともに,同一スコープ内の事象である「貯金する」も同時に正しく判定できるようになる.
このように,主事象で見られた誤りを改善することで,スコープ内と判断された従属事象の性能改善にもつながる事例が19事例見られた.
スコープを考慮した事実性解析を行うことで,CT+以外の性能,特に再現率を向上させることができるため,マクロ平均はスコープを考慮した方が大きく上回る性能となった.
このことから,隣接事象対のスコープ判定を精緻に行うことが,事実性解析の性能向上に貢献することを確認することができた.

以上の観察を合わせると,次のことが言える.
\begin{itemize}
\item 調査した接続パターンのうち,誤りの半分近く (36/80) にあたる「直接の項」は我々の分析データを見る限り,全ての場合においてスコープ内に来るので,述語項構造解析の結果に基づいてスコープを広げることにより,事実性解析の性能を向上させることができる.
\item 誤りのうち4割以上 (33/80) にあたる「テ形接続」「項を修飾」「名詞述語を修飾」等の接続パターンの場合には,スコープ内外の選択が高度に曖昧であり,これらのパターンのスコープを決定する課題に注力することに一定の効用があることがわかる.
\item スコープを人手で付与し,事実性解析に取り入れることで,CT+以外の性能,特に再現率を向上させることができるため,マクロ平均はスコープを考慮した方が大きく上回る性能を得られる.このことから,隣接事象対のスコープ判定を精緻に行うことが,事実性解析の性能向上に貢献することを確認できた.
\end{itemize}





\section{おわりに}
\label{sec_conc}

事実性解析には,事象に含まれる機能表現,疑問詞を含む副詞,文節境界を越え
て事実性に影響を与える語とそのスコープ,その他の4種類の問題が含まれてい
る.それぞれは単独でも一つの研究課題になるほどに,容易な問題ではないが,
事実性解析ではさらにその組み合わせがあるため,性能の向上が難しい.
本研究では,事実性解析の課題分析を行うために,機能表現のみを用いたルール
ベースの事実性解析器を構築し,1,533文に含まれる3,734事象に適用した結果の
誤りを分析した.このとき全ての事象表現について,述語に続く機能表現に対
して意味ラベルを付与した.

主事象の事実性解析については,機能表現の意味ラベルが正しく解析できれば,
現在の意味ラベルの体系と本研究で用いた単純な規則だけでも,90\%に近い正解
率が得られることがわかった.
本研究で用いた規則は人手で構築したものであるため,その整備は必要ではあるものの,
それよりもむしろ,現在の機能表現の意味ラベル体系に
基づいて機能表現解析モデルの研究開発を行うことに一定の支持を与えるものと
考える.また,機能表現解析の問題を除けば,誤りの半数は副詞に起因するもの
であった.したがって,事実性解析は副詞の意味解析の研究を動機付ける良い課
題となりうる.

従属事象の事実性解析は,主事象に比べて考慮すべき要素が多く,性能も低い.
従属事象でのみ考慮すべき要素は大きく二つあり,文節境界を越えて事実性に影響を与える述語と,従属事象に直接付随しない機能表現の影響である.
文節境界を越えて事実性に影響を与える述語については,既存の事象選択述語
辞書が一定のカバレッジを持っており,これを利用することで誤りの多くを解消
できる可能性がある.しかし,複合語のカバレッジに問題があるなど,こうした
リソースの整備が今後の課題であることがわかった.

従属事象に直接付随しない機能表現については,直接の親の事象に付随する機能表現の影響を受ける可能性が
あるが,その他の事象表現に付随する機能表現の影響はほとんど無視できること
も明らかになった.前者の場合については,誤りの半分近く (36/80) にあたる「直接
の項」は我々の分析データを見る限り,全ての場合においてスコープ内に来るの
で,述語項構造解析の結果に基づいてスコープを広げることにより,事実性解析
の性能を向上させることができる.一方で,誤りのうち4割以上 (33/80) 「テ形
接続」「項を修飾」「名詞述語を修飾」等の接続パターンの場合には,スコープ
内外の選択が高度に曖昧であり,これらのパターンのスコープを決定する課題に
注力することに一定の効用があることかがわかる.それ以外の主要な接続パターン
はスコープの範囲を規則ベースで決めても大きな問題は生じそうにない.
また,離れた事象対と比較して,隣接事象対のスコープを特定する方が,事実性
解析に対して大きなゲインが期待できる.
実際にスコープを人手で付与し,事実性解析に取り入れることで,CT+以外の性能,
特に再現率を向上させることができた.このことから,隣接事象対のスコープ判定
を精緻に行うことが,事実性解析の性能向上に貢献することを確認できた.

本研究で報告した誤り分析・課題分析は「Yahoo!知恵袋」のコーパスを用いてお
り,他のドメインやスタイルの文章で同様の傾向が得られるかは明らかでない.
今後は調査の範囲を広げ,問題の性質の一般化を図る.
また,本研究では,機能表現の意味ラベルに関して正解を与えることで,
人手で構築した規則ベースのモデルでも,主事象においては90\%近くの正解率となり,ある程度の性能をあげられることを示した.
機能表現の意味ラベルを人手で与えることで,更新ルールや辞書の改善によって得られるゲインよりもかなり大きなゲインが得られていると考えられる.
このことから,更新ルールや辞書の整備も必要な課題ではあるものの,今後は,本研究では正解を与えた,
機能表現の意味ラベルを自動で解析する課題に注力することが重要であると考える.
従属事象においては,主事象同様に機能表現解析も重要な要素となるが,特に隣接する事象が同一スコープに入るか否かを自動で解析することが,事実性解析の性能向上に寄与することが明らかになっている.
人手で与えていたスコープを自動で解析することが重要な課題であると考える.






\acknowledgment

本研究は,文部科学省科研費 (15H01702),
JST戦略的創造研究推進事業CREST,および,
文部科学省「ビッグデータ利活用のためのシステム研究等」委託事業 「実社会ビッグデータ利活用のためのデータ統合・解析技術の研究開発」
の一環として行われた.


\bibliographystyle{jnlpbbl_1.5}
\addtolength{\baselineskip}{-0.75pt}
\begin{thebibliography}{}

\bibitem[\protect\BCAY{有田}{有田}{2007}]{Arita2007}
有田節子 \BBOP 2007\BBCP.
\newblock \Jem{日本語条件文と時制節性}.
\newblock くろしお出版.

\bibitem[\protect\BCAY{de~Marneffe, Manning, \BBA\ Potts}{de~Marneffe
  et~al.}{2012}]{Marneffe2012}
de~Marneffe, M.-C., Manning, C.~D., \BBA\ Potts, C. \BBOP 2012\BBCP.
\newblock \BBOQ Did It Happen? The Pragmatic Complexity of Veridicality
  Assessment.\BBCQ\
\newblock {\Bem Computational Linguistics}, {\Bbf 38}  (2), \mbox{\BPGS\
  301--333}.

\bibitem[\protect\BCAY{江口\JBA 松吉\JBA 佐尾\JBA 乾\JBA 松本}{江口 \Jetal
  }{2010}]{Eguchi2010_nlp}
江口萌\JBA 松吉俊\JBA 佐尾ちとせ\JBA 乾健太郎\JBA 松本裕治 \BBOP 2010\BBCP.
\newblock モダリティ,真偽情報,価値情報を統合した拡張モダリティ解析.\
\newblock \Jem{言語処理学会第16回年次大会発表論文集}, \mbox{\BPGS\ 852--855}.

\bibitem[\protect\BCAY{遠藤\JBA 小林\JBA 三井\JBA 村木\JBA 吉沢}{遠藤 \Jetal
  }{2003}]{Endo2003}
遠藤織枝\JBA 小林賢次\JBA 三井昭子\JBA 村木新次郎\JBA 吉沢靖\JEDS\ \BBOP
  2003\BBCP.
\newblock \Jem{使い方の分かる類語例解辞典 新装版}.
\newblock 小学館.

\bibitem[\protect\BCAY{Fan, Chang, Hsieh, Wang, \BBA\ Lin}{Fan
  et~al.}{2008}]{REF08a}
Fan, R.-E., Chang, K.-W., Hsieh, C.-J., Wang, X.-R., \BBA\ Lin, C.-J. \BBOP
  2008\BBCP.
\newblock \BBOQ LIBLINEAR: A Library for Large Linear Classification.\BBCQ\
\newblock {\Bem Journal of Machine Learning Research}, {\Bbf 9}, \mbox{\BPGS\
  1871--1874}.

\bibitem[\protect\BCAY{Farkas, Vincze, M\'{o}ra, Csirik, \BBA\ Szarvas}{Farkas
  et~al.}{2010}]{CoNLL2010}
Farkas, R., Vincze, V., M\'{o}ra, G., Csirik, J., \BBA\ Szarvas, G. \BBOP
  2010\BBCP.
\newblock \BBOQ The CoNLL-2010 Shared Task: Learning to Detect Hedges and Their
  Scope in Natural Language Text.\BBCQ\
\newblock In {\Bem Proceedings of the 14th Conference on Computational Natural
  Language Learning --- Shared Task}, \mbox{\BPGS\ 1--12}.

\bibitem[\protect\BCAY{後藤\JBA 大竹\JBA {Stijn De~Saeger}\JBA 橋本\JBA {Julien
  Kloetzer}\JBA 川田\JBA 鳥澤}{後藤 \Jetal }{2013}]{Goto2013}
後藤淳\JBA 大竹清敬\JBA {Stijn De~Saeger}\JBA 橋本力\JBA {Julien Kloetzer}\JBA
  川田拓也\JBA 鳥澤健太郎 \BBOP 2013\BBCP.
\newblock 質問応答に基づく対災害情報分析システム.\
\newblock \Jem{自然言語処理}, {\Bbf 20}  (3), \mbox{\BPGS\ 367--404}.

\bibitem[\protect\BCAY{原\JBA 乾}{原\JBA 乾}{2008}]{Hara2008}
原一夫\JBA 乾健太郎 \BBOP 2008\BBCP.
\newblock 事態抽出のための事実性解析.\
\newblock \Jem{情報処理学会研究報告, 2008-FI-89, 2008-NL-183}, \mbox{\BPGS\
  75--80}.

\bibitem[\protect\BCAY{Hickl}{Hickl}{2008}]{Hickl2008}
Hickl, A. \BBOP 2008\BBCP.
\newblock \BBOQ Using Discourse Commitments to Recognize Textual
  Entailment.\BBCQ\
\newblock In {\Bem Proceedings of the 22nd International Conference on
  Computational Linguistics}, \lowercase{\BVOL}~1, \mbox{\BPGS\ 337--344}.

\bibitem[\protect\BCAY{飛田\JBA 浅田}{飛田\JBA 浅田}{1994}]{Hida1994Fukushi}
飛田良文\JBA 浅田秀子 \BBOP 1994\BBCP.
\newblock \Jem{現代副詞用法辞典}.
\newblock 東京堂出版.

\bibitem[\protect\BCAY{今村\JBA 泉\JBA 菊井\JBA 佐藤}{今村 \Jetal
  }{2011}]{Imamura2011}
今村賢治\JBA 泉朋子\JBA 菊井玄一郎\JBA 佐藤理史 \BBOP 2011\BBCP.
\newblock 述部機能表現の意味ラベルタガー.\
\newblock \Jem{言語処理学会第17回年次大会発表論文集}, \mbox{\BPGS\ 308--311}.

\bibitem[\protect\BCAY{Inui, Abe, Hara, Morita, Sao, Eguchi, Sumida, Murakami,
  \BBA\ Matsuyoshi}{Inui et~al.}{2008}]{Inui2008}
Inui, K., Abe, S., Hara, K., Morita, H., Sao, C., Eguchi, M., Sumida, A.,
  Murakami, K., \BBA\ Matsuyoshi, S. \BBOP 2008\BBCP.
\newblock \BBOQ Experience Mining: Building a Large-Scale Database of Personal
  Experiences and Opinions from Web Documents.\BBCQ\
\newblock In {\Bem Proceedings of the 2008 IEEE/WIC/ACM International
  Conference on Web Intelligence and Intelligent Agent Technology-Volume 01},
  \mbox{\BPGS\ 314--321}.

\bibitem[\protect\BCAY{Kamioka, Narita, Mizuno, Kanno, \BBA\ Inui}{Kamioka
  et~al.}{2015}]{Kamioka2015}
Kamioka, Y., Narita, K., Mizuno, J., Kanno, M., \BBA\ Inui, K. \BBOP 2015\BBCP.
\newblock \BBOQ Semantic Annotation of Japanese Functional Expressions and its
  Impact on Factuality Analysis.\BBCQ\
\newblock In {\Bem Proceedings of The 9th Linguistic Annotation Workshop},
  \mbox{\BPGS\ 52--61}.

\bibitem[\protect\BCAY{Karttunen \BBA\ Zaenen}{Karttunen \BBA\
  Zaenen}{2005}]{Karttunen2005}
Karttunen, L.\BBACOMMA\ \BBA\ Zaenen, A. \BBOP 2005\BBCP.
\newblock \BBOQ Veridicity.\BBCQ\
\newblock In Katz, G., Pustejovsky, J., \BBA\ Schilder, F.\BEDS, {\Bem
  Annotating, Extracting and Reasoning about Time and Events},
  \lowercase{\BNUM}\ 05151 in Dagstuhl Seminar Proceedings.

\bibitem[\protect\BCAY{川添\JBA 齊藤\JBA 片岡\JBA 崔\JBA 戸次}{川添 \Jetal
  }{2011a}]{Kawazoe2011}
川添愛\JBA 齊藤学\JBA 片岡喜代子\JBA 崔栄殊\JBA 戸次大介 \BBOP 2011a\BBCP.
\newblock 確実性判断のためのアノテーション済みコーパスの構築.\
\newblock \Jem{言語処理学会第17回年次大会発表論文集}, \mbox{\BPGS\ 143--146}.

\bibitem[\protect\BCAY{川添\JBA 齊藤\JBA 片岡\JBA 崔\JBA 戸次}{川添 \Jetal
  }{2011b}]{Kawazoe2011_report}
川添愛\JBA 齊藤学\JBA 片岡喜代子\JBA 崔栄殊\JBA 戸次大介 \BBOP 2011b\BBCP.
\newblock
  言語情報の確実性に影響する表現およびそのスコープのためのアノテーションガイドラインVer.2.4.\
\newblock \JTR, Department of Information Science, Ochanomizu University,
  OCHA-IS 10-4.

\bibitem[\protect\BCAY{国立国語研究所}{国立国語研究所}{2004}]{BunruiGoihyo}
国立国語研究所 \BBOP 2004\BBCP.
\newblock \Jem{分類語彙表}.
\newblock 大日本図書.

\bibitem[\protect\BCAY{工藤\JBA 松本}{工藤\JBA 松本}{2002}]{CaboCha}
工藤拓\JBA 松本裕治 \BBOP 2002\BBCP.
\newblock チャンキングの段階適用による日本語係り受け解析.\
\newblock \Jem{情報処理学会論文誌}, {\Bbf 43}  (6), \mbox{\BPGS\ 1834--1842}.

\bibitem[\protect\BCAY{益岡}{益岡}{2007}]{Masuoka2007}
益岡隆志 \BBOP 2007\BBCP.
\newblock \Jem{日本語モダリティ探求}.
\newblock くろしお出版.

\bibitem[\protect\BCAY{松吉}{松吉}{2014}]{Matsuyoshi2014}
松吉俊 \BBOP 2014\BBCP.
\newblock 否定の焦点情報アノテーション.\
\newblock \Jem{自然言語処理}, {\Bbf 21}  (2), \mbox{\BPGS\ 249--270}.

\bibitem[\protect\BCAY{松吉\JBA 佐尾\JBA 乾\JBA 松本}{松吉 \Jetal
  }{2011}]{Matsuyoshi2011}
松吉俊\JBA 佐尾ちとせ\JBA 乾健太郎\JBA 松本裕治 \BBOP 2011\BBCP.
\newblock 拡張モダリティタグ付与コーパスの設計と構築.\
\newblock \Jem{言語処理学会第17回年次大会発表論文集}, \mbox{\BPGS\ 147--150}.

\bibitem[\protect\BCAY{{松吉}\JBA 江口\JBA 佐尾\JBA 村上\JBA 乾\JBA
  松本}{{松吉} \Jetal }{2010}]{Matsuyoshi2010}
{松吉}俊\JBA 江口萌\JBA 佐尾ちとせ\JBA 村上浩司\JBA 乾健太郎\JBA 松本裕治 \BBOP
  2010\BBCP.
\newblock テキスト情報分析のための判断情報アノテーション.\
\newblock \Jem{電子情報通信学会論文誌D}, {\Bbf J93-D}  (6), \mbox{\BPGS\
  705--713}.

\bibitem[\protect\BCAY{松吉\JBA 佐藤\JBA 宇津呂}{松吉 \Jetal
  }{2007}]{Matsuyoshi2007}
松吉俊\JBA 佐藤理史\JBA 宇津呂武仁 \BBOP 2007\BBCP.
\newblock 日本語機能表現辞書の編纂.\
\newblock \Jem{自然言語処理}, {\Bbf 14}  (5), \mbox{\BPGS\ 123--146}.

\bibitem[\protect\BCAY{南}{南}{1974}]{Minami1974}
南不二男 \BBOP 1974\BBCP.
\newblock \Jem{現代日本語の構造}.
\newblock 大修館書店.

\bibitem[\protect\BCAY{水野\JBA 成田\JBA 乾\JBA 大竹\JBA 鳥澤}{水野 \Jetal
  }{2013}]{Mizuno2013}
水野淳太\JBA 成田和弥\JBA 乾健太郎\JBA 大竹清敬\JBA 鳥澤健太郎 \BBOP 2013\BBCP.
\newblock 拡張モダリティ解析器の試作と課題分析.\
\newblock \Jem{ALAGIN \& NLP若手の会合同シンポジウム}.

\bibitem[\protect\BCAY{水野\JBA 渡邉\JBA {エリック ニコルズ}\JBA 村上\JBA
  乾\JBA 松本}{水野 \Jetal }{2011}]{Mizuno2011}
水野淳太\JBA 渡邉陽太郎\JBA {エリック ニコルズ}\JBA 村上浩司\JBA 乾健太郎\JBA
  松本裕治 \BBOP 2011\BBCP.
\newblock 文間関係認識に基づく賛成・反対意見の俯瞰.\
\newblock \Jem{情報処理学会論文誌}, {\Bbf 52}  (12), \mbox{\BPGS\ 3408--3422}.

\bibitem[\protect\BCAY{Morante \BBA\ Blanco}{Morante \BBA\
  Blanco}{2012}]{SEM2012}
Morante, R.\BBACOMMA\ \BBA\ Blanco, E. \BBOP 2012\BBCP.
\newblock \BBOQ *SEM 2012 Shared Task: Resolving the Scope and Focus of
  Negation.\BBCQ\
\newblock In {\Bem *SEM 2012: The 1st Joint Conference on Lexical and
  Computational Semantics - Volume 1: Proceedings of the Main Conference and
  the Shared Task}, \mbox{\BPGS\ 265--274}.

\bibitem[\protect\BCAY{森田\JBA 松木}{森田\JBA 松木}{1989}]{Morita1989}
森田良行\JBA 松木正恵 \BBOP 1989\BBCP.
\newblock \Jem{日本語表現文型用例中心・複合辞の意味と用法}.
\newblock アルク.

\bibitem[\protect\BCAY{日本語記述文法研究会(編)}{日本語記述文法研究会(編)}{2003}]{Grammar3}
日本語記述文法研究会(編) \BBOP 2003\BBCP.
\newblock \Jem{現代日本語文法 3}.
\newblock くろしお出版.

\bibitem[\protect\BCAY{Saur\'{\i}}{Saur\'{\i}}{2008}]{Sauri2008_Guidelines}
Saur\'{\i}, R. \BBOP 2008\BBCP.
\newblock \BBOQ FactBank 1.0 Annotation Guidelines.\BBCQ\ {\ttfamily
  http://\linebreak[2]www.cs.brandeis.edu/\linebreak[2]{\textasciitilde}roser/\linebreak[2]pubs/\linebreak[2]fb\_annotGuidelines.pdf}.

\bibitem[\protect\BCAY{Saur\'{\i}, Littman, Gaizauskas, Setzer, \BBA\
  Pustejovsky}{Saur\'{\i} et~al.}{2006}]{Sauri2006}
Saur\'{\i}, R., Littman, J., Gaizauskas, R., Setzer, A., \BBA\ Pustejovsky, J.
  \BBOP 2006\BBCP.
\newblock \BBOQ TimeML Annotation Guidelines, Version 1.2.1.\BBCQ\ {\ttfamily
  http://\linebreak[2]www.timeml.org/\linebreak[2]site/\linebreak[2]publications/\linebreak[2]timeMLdocs/\linebreak[2]annguide\_1.2.1.pdf}.

\bibitem[\protect\BCAY{Saur\'{\i} \BBA\ Pustejovsky}{Saur\'{\i} \BBA\
  Pustejovsky}{2007}]{Sauri2007}
Saur\'{\i}, R.\BBACOMMA\ \BBA\ Pustejovsky, J. \BBOP 2007\BBCP.
\newblock \BBOQ Determining Modality and Factuality for Text Entailment.\BBCQ\
\newblock In {\Bem Proceedings of the International Conference on Semantic
  Computing}, \mbox{\BPGS\ 509--516}.

\bibitem[\protect\BCAY{Saur\'{\i} \BBA\ Pustejovsky}{Saur\'{\i} \BBA\
  Pustejovsky}{2009}]{Sauri2009}
Saur\'{\i}, R.\BBACOMMA\ \BBA\ Pustejovsky, J. \BBOP 2009\BBCP.
\newblock \BBOQ FactBank: A Corpus Annotated with Event Factuality.\BBCQ\
\newblock {\Bem Language Resources and Evaluation}, {\Bbf 43}  (3),
  \mbox{\BPGS\ 227--268}.

\bibitem[\protect\BCAY{Saur\'{\i} \BBA\ Pustejovsky}{Saur\'{\i} \BBA\
  Pustejovsky}{2012}]{Sauri2012}
Saur\'{\i}, R.\BBACOMMA\ \BBA\ Pustejovsky, J. \BBOP 2012\BBCP.
\newblock \BBOQ Are You Sure that This Happened? Assessing the Factuality
  Degree of Events in Text.\BBCQ\
\newblock {\Bem Computational Linguistics}, {\Bbf 38}  (2), \mbox{\BPGS\
  261--299}.

\bibitem[\protect\BCAY{Sutton, McCallum, \BBA\ Rohanimanesh}{Sutton
  et~al.}{2007}]{Sutton2007}
Sutton, C., McCallum, A., \BBA\ Rohanimanesh, K. \BBOP 2007\BBCP.
\newblock \BBOQ Dynamic Conditional Random Fields: Factorized Probabilistic
  Models for Labeling and Segmenting Sequence Data.\BBCQ\
\newblock {\Bem The Journal of Machine Learning Research}, {\Bbf 8},
  \mbox{\BPGS\ 693--723}.

\bibitem[\protect\BCAY{鈴木\JBA 阿部\JBA 宇津呂\JBA 松吉\JBA 土屋}{鈴木 \Jetal
  }{2011}]{Suzuki2011}
鈴木敬文\JBA 阿部佑亮\JBA 宇津呂武仁\JBA 松吉俊\JBA 土屋雅稔 \BBOP 2011\BBCP.
\newblock 大規模階層辞書と用例を用いた日本語機能表現の解析.\
\newblock \Jem{『現代日本語書き言葉均衡コーパス』完成記念講演会予稿集},
  \mbox{\BPGS\ 105--110}.

\bibitem[\protect\BCAY{Szarvas, Vincze, Farkas, \BBA\ Csirik}{Szarvas
  et~al.}{2008}]{Szarvas2008}
Szarvas, G., Vincze, V., Farkas, R., \BBA\ Csirik, J. \BBOP 2008\BBCP.
\newblock \BBOQ The BioScope Corpus: Annotation for Negation, Uncertainty and
  Their Scope in Biomedical Texts.\BBCQ\
\newblock In {\Bem Proceedings of the Workshop on Current Trends in Biomedical
  Natural Language Processing}, \mbox{\BPGS\ 38--45}.

\bibitem[\protect\BCAY{田窪}{田窪}{2010}]{Takubo2010}
田窪行則 \BBOP 2010\BBCP.
\newblock \Jem{日本語の構造 推論と知識管理}.
\newblock くろしお出版.

\bibitem[\protect\BCAY{梅澤\JBA 西尾\JBA 松田\JBA 原田}{梅澤 \Jetal
  }{2008}]{Umezawa2008SAGE}
梅澤俊之\JBA 西尾華織\JBA 松田源立\JBA 原田実 \BBOP 2008\BBCP.
\newblock 意味解析システム SAGE
  の精度向上とモダリティの付与と辞書更新支援系の開発.\
\newblock \Jem{言語処理学会第14回年次大会発表論文集}, \mbox{\BPGS\ 548--551}.

\end{thebibliography}

\begin{biography}
\bioauthor{成田 和弥}{
2015年東北大学大学院情報科学研究科博士後期課程科目修了退学.同年より同研究科研究員,現在に至る.自然言語処理の研究に従事.
}
\bioauthor{水野 淳太}{
2012年奈良先端科学技術大学院大学情報科学研究科博士課程修了.同年より東北大学大学院情報科学研究科研究員.2013年より独立行政法人情報通信研究機構耐災害ICT研究センター研究員,現在に至る.博士(工学).自然言語処理,耐災害情報通信の研究に従事.情報処理学会,人工知能学会各会員.
}
\bioauthor{上岡 裕大}{
2014年東北大学工学部情報知能システム総合学科卒業.同年,同大学大学院情報科学研究科博士前期課程に進学,現在に至る.自然言語処理の研究に従事.情報処理学会学生会員.
}
\bioauthor{菅野 美和}{
2013年より東北大学大学院情報科学研究科技術補佐員,現在に至る. 自然言語処理研究データのアノテーションに従事.
}
\bioauthor{乾 健太郎}{
1995年東京工業大学大学院情報理工学研究科博士課程修了.同研究科助手,九州工業大学助教授.奈良先端科学技術大学院大学助教授を経て,2010年より東北大学大学院情報科学研究科教授,現在に至る.博士(工学).自然言語処理の研究に従事.情報処理学会,人工知能学会,ACL,AAAI各会員.
}
\end{biography}


\biodate




\end{document}
