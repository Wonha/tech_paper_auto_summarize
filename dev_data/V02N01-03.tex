\documentstyle[jtheapa]{jnlp_j_b5_old}

\makeatletter
\def\@captype{}
\newtheorem{定義}{}
\def\@begintheorem#1#2{}
\newcounter{ex}[section]
\def\sample{}
\let\endsample
\def\p@ex{}
\makeatother
\newcommand{\G}{}
\newcommand{\eos}{}

\setcounter{page}{39}
\setcounter{巻数}{2}
\setcounter{号数}{1}
\受付{1994}{6}{1}
\再受付{1994}{8}{12}
\採録{1994}{10}{20}

\setcounter{secnumdepth}{2}

\title{文章内構造を複合的に利用した論説文要約システムGREEN}
\author{山本 和英\affiref{TUTKIE}
   \and 増山 繁\affiref{TUTKIE}
   \and 内藤 昭三\affiref{NTT} }

\headauthor{山本 和英・増山 繁・内藤 昭三}
\headtitle{文章内構造を複合的に利用した論説文要約システム GREEN}


\affilabel{TUTKIE}{豊橋技術科学大学 知識情報工学系}
{Department of Knowledge-based Information Engineering,\\
 Toyohashi University of Technology.
(E-mail: yamamoto@smlab.tutkie.tut.ac.jp)}
\affilabel{NTT}{NTT ソフトウェア研究所}
{NTT Software Laboratories.}

\jabstract{
日本語文章要約システム\G について報告する. 一般に, 質の良い文章要約を
行うためには, ある一つの言語現象だけをとらえた談話解析だけでは不十分
である. なぜなら, 談話に関わる言語現象は相互に関連しているからである. 
本研究ではこの観点から, 日本語での様々な表層的特徴をできるだけ多く利用
して, 日本語文章の要約を試みる. 本稿では実際に計算機上で試作した論説文
要約システム\G に関して, これで用いられている論説文要約の手法の紹介と, 
これによって出力された文章の評価を行う. }

\jkeywords{談話解析, 要約作成, 論説文, 修飾語省略}


\etitle{GREEN: An Experimental System Generating \\
Summary of Japanese Editorials by Combining \\
Multiple Discourse Characteristics}
\eauthor{Kazuhide Yamamoto\affiref{TUTKIE}
    \and Shigeru Masuyama\affiref{TUTKIE}
    \and Shozo Naito\affiref{NTT} }

\eabstract{
This paper describes an experimental system \G \ for summarizing
Japanese texts. Analyzing several aspects of discourse phenomena is in
general indispensable to make a summary of good quality because each
linguistic phenomenon mutually affects in a complex manner. From this
point of view, we have developed a system \G, which is capable of
summarizing Japanese editorial newspaper articles. This system uses
several surface linguistic characteristics appeared in Japanese texts.
We evaluate the performance of the system using a method of
questionnaire where human subjects evaluate the quality of the
summaries generated by the system.}

\ekeywords{Discourse analysis, Making summary, 
           Editorial article, Deleting modifiers.}

\begin{document}
\maketitle


\section{はじめに}

本論文では日本語の論説文を対象にした要約文章作成実験システム\G
\footnote{\sl Generator of REcapiturations of Editorials and Notices
 の略. } (以下 \G と呼ぶ)について述べる. 

一般に, 質の良い文章要約を行うためには, 照応, 省略, 接続的語句, 語彙
による結束性, 主題・話題, 焦点など多くの談話現象の処理が必要であり, こ
れらの談話現象は互いに複雑に影響しあっているので, これらの談話現象の一
部のみの処理を行って要約を試みても, 質の高い要約が得られる可能性は低
い. 本研究の目的は, 以上の見地から現状で解析可能な談話要素をできるだけ
多く取り込み, 実際に計算機上で動作する実験的な要約作成システムを試作し
てその効果を検討することである. 

文章要約については, 日本語学あるいは日本語教育の分野でも, 現状では定義
や手法が確立していない\cite{要約本}. 本論文では, 文章要約とは, 重要度
が相対的に低い部分を削除することであるとみなす. 一般には, 文章中のある
部分の「重要度」は文章の種類によって異なるので, 要約の方法は, 文章の種
類によって異なったアプローチを取る必要があると考えられる. 本研究では, 
新聞社説などの, 筆者が読者に対して何らかの主張や見解を示す文章(以下, 
論説文章と呼ぶ)を要約の対象にする. 

田村ら\cite{田村}は, 文章の構造および話題の連鎖を表現する修辞構造ネッ
トワークおよび話題構造を作成することによる要約方式を提案しているが, 思
考実験に留まっており, その実現には, 一般的知識に関するシソーラスの構築
や, 修辞構造ネットワークの自動作成手法などの困難な問題が残されている.
 また間瀬らは, 「重要文に比較的よく出現する表層的特徴を多種類含んでい
る文が真の重要文である」という仮定に基づき, 題名語, 高頻度名詞, 主題
(助詞「は」)などのパラメータを総和することによって重要語を決定し, 要約
文を選択するという統計的手法に基づく要約法を提案している\cite{杉江}.

本研究では, 要約中で原文章の文をそのまま使用するのではなく, 文内で比較
的重要度が低いと考えられる連体修飾要素の削減も行った. 一方, 本手法は
文章内の談話構造の利用による文章要約を試みたものであり,\cite{杉江}など
の従来の抄録作成に使用されてきた語の頻度に関する情報は, 使用しなかった.
 また, 前述の両論文でも使用している文章のタイトル(題名)の情報も, タイ
トルはそもそも文章の「究極的な要約」であるという立場から, 要約処理への
使用は循環論的であると考えられるので, 本手法では利用しなかった.

以下, \ref{システム}節で, \G のシステム構成を述べる. \ref{要約文選択}
節から\ref{段落分け}節で, 要約文の選択, 一文内で修飾語を削減することに
よる文長の短縮法, 要約文章の段落分け, のシステム各部の詳細を述べる.
\ref{評価}節では, アンケート調査に基づき\G を評価する. \ref{議論}節で
は, 大量の要約文生成で明らかになった問題点や, 得られた知見を紹介する. 
本論文では要約実験対象として日本経済新聞の社説を用いた. 論文中の例文, 
要約例は, [例文\ref{作例から}]〜[例文\ref{作例まで}]を除いてすべて1990
年9月と1990年11月の同社説から引用したものである. 



\section{システム構成}\label{システム}

要約システム \G は Sun SPARC Station I 上で, Perl 言語を使用して作成さ
れている. システムは, 以下の五つの部分からなる. 以下では要約文として使
用することを「採用する」と表現する. 

\begin{description}
\item[形態素解析部]
解析方法は, 文節数最小法を基本として, いくつかのヒューリスティックスを
取り入れている. 文献\cite{手がかり語}で使用したLisp 版の形態素解
析部を移植して, 改良を加えたものである. 

辞書は角川類語新辞典\cite{角川類語}に, 固有名詞及び機能語を追加したも
のである. 形態素解析を行うと同時に, 各語について同辞典の分類番号を調
べておく. 

\item[要約文選択部]
すでに採用した文の文長の合計と, あらかじめ設定した目標要約率とを比較し
ながら, 新たに採用する要約文を選択していく. 要約文として採用された文に
対して, 手がかり語による結束処理, 及び省略処理を行い, 条件に該当する
場合, その前文も要約文として採用する. 

\item[文要約解析部]
要約文として採用された文に対して, 修飾句を削減することにより文レベルの
圧縮を行う. 

\item[段落分け解析部]
文献\cite{手がかり語}で述べた方法を用いて, 原文の意味段落にそって段落分けを
行い, 要約文章を生成する. 

\end{description}

\section{要約文選択}\label{要約文選択}

\subsection{見解文と現象文}
論説文章中の文は, 著者の主張, 意見, 希望などを述べた文と, その主張, 意
見, 希望などを述べるために必要な出来事, 事実, 現象を述べた文の二種類に
大別することができる. 例えば, 以下の[例文\ref{one}]は現在の状況を事実
として述べた文, [例文\ref{two}]は筆者の意見を述べた文である. 

\begin{sample}
\item 地球温暖化の防止を目指してジュネーブで開かれていた第二回
地球気候会議が終わった。[13/Nov/1990]\label{one}
\item 今回の会議は、来年二月から始まる温暖化防止のための条約作りの
基礎になるだけに、目標が不明確のままに終わったのは残念である。
[13/Nov/1990]\label{two}
\end{sample}

以下では, 著者の主張, 意見, 希望などを述べた文を「見解文」, 出来事, 事
実, 現象を述べた文を「現象文」と呼ぶ. 文章中のすべての文は, 見解文か現
象文のどちらかに属すると仮定する. 

\subsection{見解文の抽出}
日本語の文章から見解文を抽出するためには, 文末表現に注目することが有効
である. 例えば, 「〜が必要である」「〜すべきである」などは見解文に特徴
的な文末表現である. \G では, あらかじめ作成した見解文の文末の典型的パ
ターンとのマッチングを行うことにより, 近似的に見解文を抽出する. 

\begin{center}
\begin{tabular}{|l|l|}
\hline
文末表現
 & 「〜求められる」「言うまでもない」\\
 & 「〜と思われる」「〜だろう(か)」 \\
 & 「〜と言える」「〜のではないか」\\
 & 「〜したい」「ほしい」「〜と考える」 \\
 & 「なければならない」「〜ずにはおかない」\\
 & 「気になる」\\
\hline
単語
 & 「大切」「必要」「期待」「残念」「はず」 \\
 & 「注目」「べき」「歓迎」「課題」「危険」 \\
\hline
\end{tabular}
\caption{見解文の文末パターンの例}
\end{center}

\subsection{冒頭文と最終文}
文章の冒頭文は, 前提を全く持っていない読者(聴者)に, はじめて著者(話者)
の持つ情報を伝達して, 話題に関する情報を共有するという重要な役割を果た
す文であり, 文章要約においても原文章の冒頭文は重要な役割を果たすと考え
られる. また文章の最終文は, 著者が話を締めくくる必要があるため, 文章の
他の部分と異なった, 何らかの特別な意図を持って書かれた文と考えることが
できる. このため, 冒頭文と同様に重要であり, 要約作成時にも欠かせない重
要な文と考えられる. 

文献\cite{要約本}にも, 「要約文の作成では, 一般的に原文の冒頭文と最終
文の重要性が高く, その中間文において思いきった圧縮が行われることが多
い」(p.138)と述べられており, 本研究での考察と一致する. 本論文の要約シ
ステムでは, 文章の冒頭文および最終文を重要視する. 


\subsection{文章の総括方式と見解文の位置}

一般に文章の統括形式は, 以下の4つに分類できる\cite{市川}. 

\begin{enumerate}
\item 冒頭で統括するもの(頭括式)
\item 結尾で統括するもの(尾括式)
\item 冒頭と結尾で統括するもの(双括式)
\item 中ほどで統括するもの(中括式)
\end{enumerate}

実際の新聞社説の観察では, 中括式や純粋な頭括式はまれであることから, 文
章の最終部分では, 文章の主要な結論が述べられていることが多いといえる.

これらの考察に基づき, 冒頭, および最終の見解文を除いた中間の部分に存在
する見解文では, 主要な結論が述べられる最終の見解文と距離的に近いところ
にある, より文章の後半部に出現する見解文の方が相対的に重要であると仮定
し, 要約文選択のヒューリスティックスとして採用した.

\subsection{段落内構造}
段落内構造に関しても, 基本的には文章の構造と同じモデルを考える. つまり, 
各段落の冒頭文は, 新しい主題に関する前提を持っていない読者に, 著者の持
つ情報を初めて伝達する役割を持つので, その段落を代表する文である場合が
多いと考えられる. 


\subsection{結束性解析}\label{結束性解析}
 \G では, 以前に行った研究\cite{手がかり語}での考察などに基づき, 近
似的な結束性の処理を行い, 要約文選択に用いている. 


\subsubsection{手がかり語による結束性}\label{anaphora}
文献\cite{手がかり語}では, 文頭に出現する指示語(これ, その, など)や接
続詞(そして, しかし, など)などの語句を「手がかり語」と定義し, これらの
語が文頭に出現した場合に, 前文との強い結束性を示す場合が多いことを示し
た. このことから, 文頭にこれらの語が出現した文は, 要約文中には単独で出
現せず, 必ず前文を伴って出現すると考え, 要約文として選択する際にはこれ
ら二文のどちらも採用することにする. また, 前文の文頭にも「手がかり語」
を含んでいる場合には再帰的にさらに前文を要約文として採用する. 

\begin{sample}
\item 夏場以降は一転して中東情勢に関心が移っている。しかし、構造協議
で示された課題はイラク問題の発生で後退したわけではない。[9/Sep/1990]
\item 課徴金の引き上げについては具体化を約し、独禁法改正法案を次期通
常国会に提出することになっている。その措置が形式的なものに終らぬように、
議論を広めるべきだ。[9/Sep/1990]
\end{sample}

\subsubsection{省略による結束性}
文の要素の省略は, 照応よりも扱いが難しい. そこで, \G では広義の主語(主
格を表す格助詞, またはとりたて詞(例えば, は, こそ, さえ)が後接している
句)が省略されている文は, その文単独で要約文中に採用されても意味の把握
が難しいと判断し, 前文も採用する\footnote{照応の場合と同様に, 前文にも
主語が存在しない場合は再帰的にさらにその前文を採用するが, 実際にはこの
ような場合はほとんどない. }. 以下の2例文では, それぞれ「政府, 与野党が」
(主格), 「約二十六億円を投じて地方公共団体に電気自動車を普及させようと
いう政策は」(とりたて詞の後接する句)が省略されているので, 2文目が採用
される時には, 1文目も採用される. 

\begin{sample}
\item 放置しておけば大きなツケが残るのは目に見えている。一刻も早い対
応を望む。[20/Sep/1990]
\item 日本全国で走っている自動車の数を考えると、大気汚染を防止する直
接的な効果は皆無である。この点、割箸と非常に似ている。[8/Sep/1990]
\end{sample}

\subsection{要約文選択アルゴリズム}
以上の要因を組み込んだ\G での要約文選択アルゴリズムを以下に示す. また, 
主要語, 要約率は以下のように定義する. 

\vspace{5mm plus 5mm}
\begin{定義}
「主要語」とは, 角川類語新辞典\cite{角川類語}に掲載されている語のうち, 
大分類の番号が\{0,5,7,8,9\}である語, 及び固有名詞である\footnote{実際
にはさらに一部の小分類に属する語, 及び一部の語を除いてある. }. 複数の
意味分類に含まれる多義性のある語については, その一つが前述の大分類に含
まれていれば, 主要語とする. 
\end{定義}

\begin{equation}
\mbox{要約率(\%)} = \frac{\mbox{要約文長}}{\mbox{原文章の(原文のままの)文字長}} \times 100
\end{equation}
\begin{equation}
\mbox{要約文長} =  \sum_{\mbox{採用した文}}\mbox{単文要約処理後の文字長}
\end{equation}

\vspace{10mm plus 10mm}

\noindent
\underline{要約文選択のアルゴリズム:}
\begin{description}

\item[Step 0.]
文章の冒頭文, 及び最終文を採用する. 

\item[Step 1.1.]
各段落冒頭文に対して手がかり語の検査・省略の検査を行い(詳細は\ref{段落
分け}節を参照), 原文の段落を意味段落に再構成する. 

\item[Step 1.2.]
第一意味段落の全文と各段落の冒頭文の中で, 文章冒頭文に含まれる主要語が
主語になっている文を採用する. 

\item[Step 1.3.]
最終意味段落の全文と各段落の冒頭文の中で, 文章最終文に含まれる主要語が
主語になっている文を採用する. 

\item[Step 2.]
Step 1が終了した時点で, 要約率が $(\alpha + \delta)\%$ 以上ならば, 
冒頭文, 最終文以外の採用\\した文すべてを未採用にする. ここで, $\alpha$は
要約率の目標, $\delta$は許容範囲パラメータである. 

\item[Step 3.1.]
まだ採用されてない見解文のうち, 文章の最も後ろにある文を採用する. 

\item[Step 3.2.]
手がかり語による結束性の解析を行い, 要約文の文頭に手がかり語が出現す
る場合は, その前文を要約文に採用する. 

\item[Step 3.3.]
要約文に対する省略解析を行い, 広義の主語(主格を表す格助詞, またはと
りたて詞が後接している語)が省略されている要約文に対しては, その前文も
要約文として採用する. 

\item[Step 4.]
要約率が$\alpha \%$未満ならば, Step3.1へ. $\alpha \%$ 以上ならば, 採用
した文を意味段落に沿って出力, 終了. \eos

\vspace*{-0.2mm}
\end{description}




\section{文要約解析}\label{文要約}

文中の修飾句を削減することにより, 一文内での要約を行う. 文の中心内容
は, 文中の修飾句の削減による影響を受けない. \vspace*{-0.2mm}本研究では, (1) 二重修飾・
多重修飾, (2) 固有名詞への修飾, (3) 例示の三通りの
場合に, 連体修飾句を削除する. \vspace*{-0.2mm}


\subsection{二重修飾・多重修飾}
ある名詞を修飾する連体修飾句には, 表\ref{kindof}に示す種類が考えられ
る\cite{国研文法下}\footnote{ただし, 文献とは一部の品詞名を変更してい
る. }. そこで, 実際の文によく見られる「二重修飾」及びそれを一般化した
「多重修飾」を, 以下のように定義する. 


\begin{table}[htbp]
\begin{center}
\begin{tabular}{|l|l|}
\hline
 文法的性質 & 例 \\
\hline
こそあど詞連体形                & この話 \\
連体詞                          & ある話 \\
形容詞の現在形・過去形          & おもしろい話 \\
形容動詞の連体形・過去形        & 変な話 \\
名詞+連体助詞「の」            & 昔の話 \\
名詞+格助詞+「の」            & 昔からの話 \\
名詞+取り立て詞+「の」        & ここだけの話 \\
副詞+「の」                    & 突然の話 \\
節(被修飾名詞が修飾節の格要素)  & 私が聞いた話 \\
節(被修飾名詞が修飾節の格要素でない) & 子狸が少年と仲良くなる話 \\
\hline
\end{tabular}
\vspace{3mm}
\caption{連体修飾の種類}\label{kindof}
\end{center}
\end{table}

\begin{定義}
一つの名詞に対して, 上の例に示したような要素が二つ以上修飾している状態
を「多重修飾」と呼ぶ. 特に, 二つの要素が修飾している状態を「二重修飾」
と呼ぶ. 複合名詞(地域紛争など)は, 名詞が名詞を修飾しているとみなす. 
\end{定義}

本研究では, 名詞の二重修飾があった場合, 前方の修飾要素を省略しても意味
は大きく変化しないことが多いと考え, 要約文生成の際には前方の修飾要素を
省略する. 同様に, 多重修飾の場合には, 最終の修飾要素を残して, 残りを省
略する.

例として, 「おもしろい昔の話」は「昔の話」に, 「突然のうれしい話」は
「うれしい話」に, 「私が聞いた変なおもしろい話」は「おもしろい話」のよ
うに省略を行う. ただし, 後半の修飾要素が, 名詞(+格助詞または取り立て
詞)+「の」である場合には, 意味的なあいまいさが生じる. 

\begin{sample}
\item 私が聞いた作家の話$\cdots$\label{作例から}
\end{sample}
\begin{sample}
\item 私がインタビューした作家の話$\cdots$\label{作例まで}
\end{sample}

上の二つの例では, 形態上は同一であるが, 前者は「私が聞いた」が「話」に, 
後者は「私がインタビューした」が「作家」に, それぞれかかる. しかし, 以
上のことを形態情報だけで判断することは不可能である. 

ここで, 被修飾名詞「話」を基本に考えると, 前者の例は「私が聞いた」「作
家の」の二つが「話」にかかり, 後者の例は, 「作家の」だけが「話」にかかっ
て「私がインタビューした」は「作家」にかかる. ここで, 両者に共通するの
は, 「作家の」は「話」にかかる, という点である. 

以上より, 例えば「私が$V$した作家の話」(「$V$した」は任意の動詞)とい
う表現を短縮する場合, 

\begin{enumerate}
\item 修飾要素をすべて取り除いた「話」だけでは漠然としすぎて一般に意
味がつかめない. 
\item 「私が$V$した」は「作家」と「話」のどちらにかかるか不明. 
\item 「作家の」は必ず「話」にかかる. 
\end{enumerate}

\noindent
という理由から, 「私が$V$した」を削除し, 「作家の話」としている. また
一般に, 修飾節のほう\\が名詞+「の」よりも文字数が多く, このため修飾節を
削除することによる要約の効果が高いことも, 修飾節削除の理由となっている. 

実際に出現した例文を以下に示す. ただし, 文中の[$\cdots$]の部分は計算機
が削除可能と判断した修飾要素である. 

\begin{sample}
\item
要綱素案で目につくのは、[海部首相の要請にこたえて去る四月末、選挙制度
審議会がまとめた]衆院選改革の答申と大きく異なることだ。[14/Nov/1990]
\item
統一を実現するうえで最大の難関は、米英仏とともに[[統一問題やベルリンの
地位変更に関する]国際法上の権限を留保している]ソ連の承認をどう取りつけ
るかだった。[14/Sep/1990]\label{double}
\end{sample}

[例文\ref{double}]のように, 該当する削除可能候補が複数ある場合は, 
削除文字列の長い方を採用する. 



\subsection{固有名詞への修飾}
修飾の用法は, 一般に限定修飾と, 非限定的修飾に分類することができる
\cite{寺村}. 個々の修飾語句がこのどちらで使用されているかの判断は, 必
ずしも容易ではないが, 本研究では, 非限定に用法が限られる固有名詞への修
飾を扱う. ここでいう固有名詞とは, 一般的な意味の固有名詞の他, 固有物を
示す一般名詞(例えば, わが国)も含めて考える.  \G では, 固有名詞にかかる
連体修飾節は限定機能を果たさないので削減可能と判断し, 単文要約処理の際, 
削除する. 

\begin{sample}
\item{ [絵の具に油を混ぜ表面をニスで覆う]西欧絵画に比べ、[ニカワが下
に沈んで絵の具がむき出しになる]日本の絵は、長期的に照明下に置くことは
できないのである。[16/Sep/1990]}\label{proper1}
\item{ [消費と設備投資をけん引車とする内需と、堅調な輸出に支えられて
順調な拡大を続けてきた]日本経済に、二つの警戒信号がついた。
[3/Sep/1990]}\label{proper2}
\item{[党より人で投票する傾向の強い]日本の選挙の実情からすると、候補
者への投票がそのまま政党への投票とみなされる一票制による小選挙区比例代
表制は、[地域密着型の選挙を得意とする]自民党候補に有利と考えられる。
[14/Nov/1990]}\label{proper3}
\end{sample}

[例文\ref{proper1}]〜[例文\ref{proper3}]に示す例は, いずれも固有名詞を
直接修飾していない. すなわち, 固有名詞の前方にある修飾要素(例えば, [例
文\ref{proper2}]では「消費と$\cdots$続けてきた」)が, 固有名詞(日本)を修飾し
ているのではなく, その後方の複合名詞(日本経済)を修飾している. \G 
では, その後方の複合名詞(日本経済)に固有名詞が含まれていることを利用して, 
その名詞(日本経済)も固有名詞に準ずる取り扱いを行い, その前方の修飾要
素の削除を行う. 

\subsection{例示}
「$\cdots$などの」「$\cdots$といった」のような例示も, 広い意味の修飾語
と考えることができる. 修飾としての例示は, そのほとんどが非限定的修飾用
法と考えられ, 削除しても意味的に変化が生じないと近似的に仮定し, これら
の語句が文内に出現した場合に, その例示部分を削除する. 

\begin{sample}
\item 警視庁では、[[最高時には三万七千人という]空前の警備体制を決める
など、]過激派の動きに対応してきた。[3/Nov/1990]\label{重複}
\item 百二十ヶ国の政府代表が閣僚宣言を採択したが、[焦点の二酸化炭素な
ど]温室効果ガスの排出量を規制する具体的な目標値を設定することはできな
かった。[13/Nov/1990]
\end{sample}

[例文\ref{重複}]では, 二重修飾の削除対象(「最高時$\cdots$という」)と,
 例示の削除対象(「最高時$\cdots$など, 」)が重複している. この場合, よ
り広い範囲を対象とした例示部分を削除対象とする.

現在のシステムで, 例示の対象としている語句は「など」「などの」「といっ
た」「のような」「のように」の五つである.


\section{段落分け解析}\label{段落分け}
 \G では, 原文の段落をそのまま意味的な段落, つまり「一つの主題を持つ文
の集合」と考えるのではなく, 前述した結束性に関する処理を行うことによっ
て意味段落を再構成している. 意味段落は, 以下の手順によって原文章の一つ
以上の段落の集まりで構成される. すなわち, 原文の段落の冒頭文について, 
手がかり語の検査(冒頭に「手がかり語」を含むかどうか)及び省略の検査(主
語が省略されていないかどうか)を行い, 少なくとも一方が該当する場合, 
その段落はその前の段落とつながっている(一つの意味段落を構成している)と
する. この処理を第2段落以降の全段落の冒頭について行い, 最終的にでき
た意味段落を対象にして, その後の様々な処理を行っている. 以下の例では, 
原文中の手がかり語(「こう(した)」)を検知し, 原文にある直前の改行を削除
して, 要約文章を作成する. 

\begin{sample}
\item $\cdots$一般市民にも危険が迫ったことをうかがわせる。(改行)こう
した過激派のゲリラ活動は、かねて十分予想されていたことであった。
$\cdots$ [3/Nov/1990]
\end{sample}


\section{評価}\label{評価}
要約システム\G の有効性を評価するために, 被験者18名に対して要約文章の
評価に関するアンケート調査を行った. 

調査はそれぞれの被験者に対し, 各5編の社説と \G による要約結果(ただし, 
目標要約率$\alpha=25 \%$, および$\delta=5 \%$に設定して出力した. 調査
対象要約文章を末尾に示す. )を提示し,\\以下に示す3項目について, それぞれ
0 〜 5 までの数字(整数に限らず, 小数を含んだ数も許す)で回答してもらう, 
という形式で行った. 


\begin{enumerate}
\item (社説の原文を読まずに)各要約文章のみを独立した文章として読んだ
時に, 自然かどうか
\item 社説本文, 及びそのタイトルと比較した時に, 原文で重要と考えられ
る部分を抽出しているか
\item 文内で修飾語を省略している部分について, それが適切な省略かどうか
\end{enumerate}

以下では, この3つの質問とその回答結果について述べる. 

\subsection{要約文の自然さ}
要約文章は文章全体としてまとまりのあるものでなければならない. そこで最
初の質問では, 要約文章を独立した文章と考えた時の自然さを被験者に判断し
てもらった. なお, 被験者が評価する際の判断基準を以下のように設定し, 被
験者に提示した. 

\begin{description}
\item[5点]ほぼ自然である. 
つまり, このような文章を書く人間もいると考えられる. 
\item[0点]非常に不自然である. 
つまり, 文章全体としてのまとまりがなく, 文章として何を意味するのか
ほとんど理解できない. 
\end{description}

表\ref{表:自然さ}に調査結果の平均と, 満点(5点)の評価をつけた人数を示す. 
全体の平均点は, 3.78 という比較的良好な評価が得られている. 特に文章Aに
ついて高い評価点が得られた. 

質問項目とは別に被験者に感想を求めたところ, 今回の要約率が25 \%となっ
ていることに関連して, 要約文の中に多くの内容を盛り込み過ぎて, その結果
として内容にまとまりを欠いている点を指摘する声があった. また, 要約され
た文章の段落間, 及び文間に接続詞がないために読みやすさに欠けるという意
見が多かった. この事実から, 日本語では文章の結束性を維持するために接続
詞が重要な役割をしていることを再確認するとともに, \vspace*{-0.1mm}要約文章は相対的に文
章の情報量が減少しているため, その質を高めるためには, \vspace*{-0.1mm}接続詞などを加え
ることで文章全体の結束性を補う必要のあることを示唆しているものと考えら
れる. \vspace*{-0.1mm}

\begin{center}
\begin{tabular}{|c||c|c|c|c|c||c|}
\hline
文章 & A & B & C & D & E & 平均 \\
\hline
評価 & 4.05 & 3.45 & 3.96 & 3.81 & 3.62 & 3.78 \\
満点 & 5 & 0 & 2 & 3 & 0 & \\
\hline
\end{tabular}
\caption{要約文章の自然さの評価}
\label{表:自然さ}
\vspace*{-0.2mm}
\end{center}


\subsection{要約内容の適切さ}
要約文章は, それ自身にまとまりがなければならないと同時に, 原文で重要と
考えられる情報を適切に抽出しなければならない. そこで, 被験者を対象に, 
原文で重要と考えられる部分を\G で抽出しているかどうかを尋ねた. 評価基
準及び調査結果を以下に示す. \vspace*{-0.1mm}

\begin{description}
\item[5点]このような抽出を行う人間もいると考えられる. \vspace*{-0.1mm}
\item[0点]全く at random に抽出したものとそう大きな差はない. \vspace*{-0.1mm}
\end{description}

\begin{center}
\begin{tabular}{|c||c|c|c|c|c||c|}
\hline
文章 & A & B & C & D & E & 平均 \\
\hline
評価 & 3.81 & 3.39 & 3.89 & 3.61 & 3.78 & 3.70 \\
満点 & 2 & 0 & 2 & 2 & 1 & \\
\hline
\end{tabular}
\caption{要約内容の適切さの評価}
\vspace*{-0.2mm}
\end{center}

この調査では, 他と比較して文章Bの評価値が低かったが, 全文書の平均では
評価値 3.70 という比較的良好な値が得られた. \G の行った文書Bの要約で
は, その第3段落で日本の行動計画を全文引用しているが, これについて, 要
約に全文引用することの必要性に疑問を持った被験者がいた. また, 要約では
第2段落に米ソの話題を採用しているが, これよりも, 目標値の設定がなぜ重
要なのかという理由を採用すべきであるという指摘など, 目標値に関する話題
を採用すべきだという意見が多かった(要約例参照). 

また, \G の要約では原文の後半に要約の重点がおかれ過ぎている傾向がある
という意見もいくつか聞かれた. このことから, \G では見解文を必要以上に
抽出した場合に, 近似的に文章の後半部分の見解文を採用するようにしている
が, この近似方法をさらに検討する必要性が明らかになった. 

\subsection{修飾句省略}
 \G では, 単文単位での修飾句の省略も試みている. この点についても, 被験
者にその適切さを判断してもらった. 判断基準は以下の通りとした. 

\begin{description}
\item[5点]ほぼ適当である. 
つまり, 省略された部分は, 原文の中では重要性の低い部分であり, このよう
な省略を行うことは妥当だと考えられる. 
\item[0点]ほぼ不適当である. 
つまり, 全く at random に省略したものと大差はない. 
\end{description}

\begin{center}
\begin{tabular}{|c||c|c|c|c||c|}
\hline
文章 & B & C & D & E & 平均 \\
\hline
評価 & 3.28 & 4.02 & 3.02 & 3.72 & 3.64 \\
満点 & 0 & 4 & 1 & 1 & \\
\hline
\end{tabular}
\caption{修飾語省略の適切さの評価}
\end{center}

被験者の判断した4文書\footnote{文書Aでは修飾句省略の処理が行われなかっ
たため, 評価の対象から外している. }のうち, 特に文書Cで高い評価が得られ
た. ただ, \G で使った修飾語省略のヒューリスティックスの適切さに関して
は, 今後の課題として検討すべきであるとの意見がいくつか出された. 

文中で, 構文解析を行わずに修飾・非修飾の関係, あるいは主述関係を完全
に特定することは不可能である. 現在のシステムはこの処理を近似的に行っ
ているため, 以下の例のような連体修飾節の認定に問題が生じる場合がある. 

\begin{sample}
\item
だが、各国の意見が割れたまま会議を開くのは[危険だとする]スペインの主張
はうなずける。[11/Sep/1990]
\end{sample}

また, 文法的には正しいが, 連体修飾節が限定用法を示す場合, 削除すると意
味的におかしくなる. 

\begin{sample}
\item インドシナ難民の受け入れが十二万人を超し米、加に次ぎ、[人口比で
みた]日本語学習者数が韓国に次いで多いという事実はそうした方向の反映だ
ろう。[21/Sep/1990]
\end{sample}


\subsection{主語のない文}
文章では, さまざまな談話メカニズムにより結束性が保たれている. その結果
主
語などの省略は自然に行われているが, [例文\ref{主語なし}]のような主語
のない文を要約文章として抽出してしまうと, 要約文章の結束性は原文のそれ
よりも弱いために人間は主語のない文にとまどいを感じ, また場合によっては
文の意味が理解できなくなってしまう. 今回のアンケート調査でも, 主語のな
い文に不自然さを感じた被験者が何人かいた. 要約文章作成では, 接続詞など
を補うことと同様に, 主語などの省略語を補うことも検討する必要がある. 

\begin{sample}
\item 経済のメカニズムを働かせる試みで、うまく機能するかどうか注目し
たい。[16/Nov/1990]\label{主語なし}
\end{sample}


\section{議論}\label{議論}

ここでは, 機械処理した大量の要約結果の考察から得られた知見や明らかになっ
た問題点などを述べる. 

\begin{itemize}
\item 
本研究では, 原文の結束性のまとまりをくずすことなく, そのまま要約文にも
反映させることで, 要約文の読みやすさの維持に努めた. その結果, 文を芋蔓
式に採用してしまい, 文数の短縮にならない例が見られた. 
\item
例えば文章の冒頭文に, より抽象的で, その結果文章中に多用される語(例え
ば「経済」「事件」)が含まれている場合, 有効な要約文抽出が出来ないこと
があった. またこのことは, 高頻度の語が(文章の分野特定には有効でも)必ず
しも要約で重要な語にはならないことも示している. 
\item
文章の冒頭文が例えば「文化の日. ([3/Nov/1990])」のように, 極端に短い場
合, あるいは, 文章冒頭文が極端に長い場合, 何らかの比喩から文章が始まっ
ている場合などは, 本手法が有効に機能しない. 
\item
比較的短い文で構成されている文章に対して, 本手法が特に有効であることが
観察された. この理由としては, 文が短いと, 要約率の微調整が容易になるこ
と, 重要文の抽出の精度が上がることが考えられる. このことより, 要約の前
処理として, 文の短文への分割, あるいは文のパラフレーズを行うことが有効
であると考えられる. 
\item
論説文中にある現象文は, 以降の記述内容の特定のために周知の事実を述べる
場合と, 未知の事実そのものを紹介するために事実を述べる場合とに分けられ
る. このうち, 前者の要約文としての重要性は低いが後者のそれは高い. これ
の判別は, 構文や修辞関係の情報を使用しただけでは不可能なことから, 現実
世界の一般的な知識が必要と考えられる. 
\end{itemize}



\section{おわりに}\label{まとめ}

本論文では日本語の論説文を対象にした要約文章作成実験システム \G を紹介
した. \G は現在論説文だけを対象としているが, 見解文抽出に関連する処理
以外の処理は, 他の種類の文章の要約にも十分に利用できるものである. 

\G の要約文の品質の評価をアンケート調査により行ったが, アンケート結
果の中には, 出力された文章に最小限の後編集をすることによって, 人間が行
った要約と変わらない程度の質の高い文章となる要約文が多い, という意見
があった. このことから, \G では論説文からの重要な情報の抽出は比較的う
まくいっているが, より「まとまりのある文章」, つまり, 結束性と首尾一貫
性をより強く持った文章にするための編集機能を強化することが\G の今後の
課題であると考えられる. ただしこのために必要な処理である, 接続詞や主語
の補完の実現には, より高度な文章の解析が必要であり, 困難な問題が多く存
在する. 


\section*{謝辞}
本研究で, シソーラスに使用した「角川類語新辞典」\cite{角川類語}を機械
可読辞書の形で提供いただき, その使用許可をいただいた(株)角川書店に深謝
する. 



\bibliographystyle{jtheapa}
\bibliography{yamamoto}

\newpage

\section*{付録:要約例}
以下に, \G で出力した要約のうち, 本文のアンケート調査に使用した要約例
を示す. 要約目標は 25 %とした. なお, 文章中で[$\cdots$]があるものは, 
文レベルの圧縮処理で削除された部分を示し, その部分は出力された要約には
含まれないが, 便宜的にその部分も示す. 

\subsection*{文章 A  {\rm ([8/Nov/1990], 要約率:30.6 %)}}
\centerline{\large 「再検討が必要な政令恩赦」}

政府は天皇陛下の即位の礼に伴い、恩赦(政令恩赦と特別恩赦)を実施する
方針だ。

救済の対象はおおむね狭まっており、戦前のように殺人犯まで釈放してしま
うようなことはなくなった。その代わりに復権という形で救済措置が取られる
ようになり、その結果として選挙違反者の公民権回復が目立つようになった。

道路交通法違反者の免許取得などの権利を回復する件数が数百万に上るよう
だが、だからといって選挙違反者を含めていいという説明にはならない。こう
したことが繰り返されるようでは、政治への信頼は損なわれるばかりだ。

政府部内には「ずっと続いてきたものを、今やめることも難しい」との意見
もある。しかし、国民に影響を与えることを惰性で続けるのはおかしい。広く
意見を聞きながら、恩赦を改めて考え直すべき時期に来ている。


\subsection*{文章 B {\rm ([13/Nov/1990], 要約率:27.9 %)}}
\centerline{\large 「環境保全は目標値が必要」}

地球温暖化の防止を目指してジュネーブで開かれていた第二回世界気候会議
が終わった。

[米ソの緊張緩和が進み、世界に対する]米ソ両国の影響力が減少したといわれ
る。両国共に厳しい国内事情があるとはいえ、地球環境問題でもその指導的役
割を自覚して欲しいものである。

わが国が先に決めた「国民一人あたりの二酸化炭素排出量を二〇〇〇年に一
九九〇年の水準で安定化する、日本の総排出量を二〇〇〇年以降一九九〇年の
水準で安定化するよう努力する」という地球温暖化防止行動計画は、会議の席
上でも高く評価され、日本の指導力にも期待が寄せられている。日本は、この
行動計画に向けて最大限の努力を払うと共に、条約交渉に当たって米ソ両国に
対し目標を設定すべく強く働きかける必要がある。


\subsection*{文章 C {\rm ([17/Nov/1990], 要約率:27.0 %)}}
\centerline{\large 「全欧安保会議に期待する」}

[東西冷戦後の欧州新秩序を討議する]全欧安保協力会議(CSCE)の首脳会
議が十九日、パリで開幕する。CSCEは来るべき統合欧州の「屋根」として
の役割を担い、世界各地域の安全保障体制のあり方にも示唆を与える。

東側が西側の価値観を受け入れ、協調しながら欧州統合を目指そうとしてい
るのが今の局面と見ていいだろう。

ECの果たす役割は一段と大きくなるが、その中軸である統一ドイツとソ連
の接近で、欧州の重心は東の方に移動するだろう。

欧州にとって当面の最大の課題は、どうすればソ連・東欧の市場経済化に成功
するか、にある。経済改革には十年単位の時間が必要だ。欧州の統合が進めば、
米国のプレゼンスは確実に小さくなっていく。米側の対欧州戦略にも注目した
い。


\subsection*{文章 D {\rm ([23/Nov/1990], 要約率:25.9 %)}}
\centerline{\large 「突然終わった『サッチャーの時代』」}

サッチャー英首相は二十二日辞意を表明、一九七九年五月以来の長期政権に
終止符を打った。

[レーガン前大統領との親密な関係とこれを背景にした]欧州外交の展開には見
るべきものがあった。サッチャー首相は、閣内の統一に厳しく、批判閣僚を容
赦なく解任した。だが、政権誕生以来首相の股肱の臣だったハウ副首相の辞任
は、さすがに首相にこたえたようである。ハウ氏は辞任のあいさつで「過去へ
の感傷にとらわれてはならない」と批判した。

サッチャー退陣は、英国の国内政策ばかりでなく、対外政策に大きな影響を
及ぼそう。特に、サッチャー首相はブッシュ米大統領の強力な支持者であった
だけに、湾岸危機への西側の対応で変化が生じかねない。また、ゴルバチョフ
大統領は良き理解者を失うことになり、ソ連の欧州政策にも微妙な影響が出よ
う。


\subsection*{文章 E {\rm ([25/Nov/1990] 要約率:25.8 %)}}
\centerline{\large 「国連の武力行使決議案と日本の選択」}

国連安全保障理事会は[イラクのクウェート撤退を求めた]国連諸決議の実効を
あげるため、対イラク武力行使を認める新たな決議案の協議に入る。

米国や英国政府内部には、[国連憲章で集団的自衛権が認められており、安保
理決議がなくともクウェート、サウジなど]紛争関係国の要請と合意に基づく
対イラク武力行使はできるという見解がある。

国連の重要メンバーであることを自認してはいるが、[いま常任理事国でも理
事国でもない]わが国として、どうこの国連外交の重大局面に対処するのか。

軍事力行使による解決策に反対するのならば、代替案を明確に表明しなけれ
ばならないし、そのための外交努力を一段と強化する必要がある。

例えば[現実問題として、イラクに撤退決意を促すためにも、米軍など]多国籍
軍の増強が必要とみられるが、[戦わず存在し続けるだけでも意義のある]「国
際警察軍」の維持費をだれが、どう負担すべきかについて明確にしなければな
らない。そうした対応を欠けば、湾岸危機は欧米のニッポンただ乗り論の火に
油を注ぐ結果になりかねない。


\begin{biography}
\biotitle{略歴}
\bioauthor{山本 和英}{
1969年生. 
1991年豊橋技術科学大学知識情報工学課程卒業. 
1993年同大学院修士課程修了. 
現在同大学院博士後期課程システム情報工学専攻在学中. 
自然言語処理, 特に談話処理の研究に従事. 
情報処理学会, ACL各会員. }

\bioauthor{増山 繁}{
1952年生. 
1977年京都大学数理工学科卒業. 
1983年同博士後期課程修了. 
1982年日本学術振興会奨励研究員. 
1984年京都大学工学部助手. 
1989年豊橋技術科学大学知識情報工学系講師. 
1990年同助教授, 現在に至る. 
グラフ・ネットワークのアルゴリズム, 組合せ最適化, 
並列アルゴリズム, 自然言語処理等の研究に従事. 
工学博士. 
日本オペレーションズ・リサーチ学会, 電子情報通信学会, 
情報処理学会, ACL等会員. }

\bioauthor{内藤 昭三}{
1955年生. 
1979年京都大学工学部数理工学専攻修士課程修了. 
同年NTT入社. 
現在NTTソフトウェア研究所広域コンピューティング研究部所属. 
自然言語処理, 要求仕様獲得の研究開発に従事. 
電子情報通信学会, 情報処理学会, 人工知能学会, ACL, 
計量国語学会各会員. }

\bioreceived{受付}
\biorevised{再受付}
\bioaccepted{採録}

\end{biography}

\end{document}
