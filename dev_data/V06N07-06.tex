



\documentstyle[jnlpbbl]{jnlp_j_b5}

\setcounter{page}{109}
\setcounter{巻数}{6}
\setcounter{号数}{7}
\setcounter{年}{1999}
\setcounter{月}{10}
\受付{1999}{4}{2}
\再受付{1999}{6}{21}
\採録{1999}{6}{25}

\setcounter{secnumdepth}{2}

\title{
  連体形形容詞に先行する格助詞「が」「の」格の\\係りに関する体系的分析}
\etitle{
  Analysis on Dependency of GA/NO-particles\\ preceding adjective 
  in adnominal form}


\author{菊池 浩三\affiref{DENSHIKAKEN} \and 伊東 幸宏\affiref{JOHO}}
\eauthor{Kozo Kikuchi\affiref{DENSHIKAKEN} \and Yukihiro Itoh\affiref{JOHO}}


\headauthor{
  菊池, 伊東}
\headtitle{
  連体形形容詞に先行する格助詞「が」「の」格の係りに関する体系的分析}


\affilabel{DENSHIKAKEN}
  {〒432-8011~~静岡大学大学院電子科学研究科}
  {Graduate School of Electronic Science and Technology, Shizuoka University,
    Hamamatsu 432-8011,Japan}
\affilabel{JOHO}
  {〒432-8011~~静岡大学情報学部}
  {Faculty of Information, Shizuoka University, Hamamatsu 432-8011,Japan}


\jkeywords{係り受け解析,形容詞,統計的解析}
\ekeywords{Dependency Analysis, adjective, statistical analysis}


\jabstract{
長文の係り解析の精度向上は,自然言語処理において重要な課題の一つである.我々はすで
に,連体形形容詞周りの「が」「の」格に関して,以下の3つのパターンに分類される7つ
の係りを規定するルールを見つけだした.
\begin{itemize}
\item 前後の名詞のみで係りが決まる.
\item 前後の名詞と形容詞の関係で係りが決まる.
\item 形容詞そのものの属性で係りが決まる.
\end{itemize}
本論文では,形容詞を網羅的に分析できるようにするために,国立国語研究所での形容詞の
体系的分類に従い分析対象形容詞を選択しその係りを調べた.それらの形容詞に対し7つの
ルールの妥当性を検証し拡張する.また,対象形容詞を増やすことにより2つの新たなルー
ルを検出することができた.これら,分類を網羅するように選択した形容詞に対しても約9
5%の精度で係りを決めることができた.}

\eabstract{
  Improving the accuracy of dependency analysis in long Japanese sentences is a big 
problem in the natural language analysis.  Concerning the form of 
``$<$noun-1$>$ + $<$adjective$>$ + $<$noun-2$>$",
we found the seven effective dependency rules which are 
classified into three following patterns in the previous paper. 
\begin{itemize}
\item determined only from preceding or following nouns.
\item determined from the relation between preceding or following nouns
  and adjective.
\item  determined from the characteristic of adjective itself.
\end{itemize}
In this paper, we research more systematically the dependency of adnominal which 
contains I-adjective or NA-adjective, based on classification of adjectives.  We explain 
(1)these seven rules are effectively applied to new adjectives by some extension, (2)some 
other rules are effective to improve the accuracy of dependency.  Finally we obtained an 
accuracy of about 95\% by applying our method to determine the dependency of 
adnominals which included an adjective.}


\begin{document}
\maketitle

\section{はじめに}

機械翻訳等の自然言語処理システムでの品質向上におけるボトルネックとして構文解析の
問題があり,解析する文が長くなると係り受け処理で解析を誤る場合がある.このため,長
文を意識した構文解析の品質向上に向け各種研究が行われているが,依然として未解決のま
ま残されている課題がある.そのような課題の一つに連体形形容詞に関する係りがある.こ
の課題に対し,我々は,連体形形容詞周りの「が」格,「の」格の係り決定ルールを提案し,
技術文でよく利用される形容詞に対して約97%の精度で係りを特定できることを示した
(菊池,伊東~1999).しかし,そこで対象とした形容詞は技術文での出現頻度を考慮して
選択したので,抽出したルールが形容詞全般に対しても有効かどうか,また,同様な考え方
が形容詞全般に対しても成り立つのかどうかについては検証できていなかった.そこで,本
論文では,分析対象を広げ,抽出済みルールが形容詞全般に対して妥当なものであるかどう
かを検証し,必要に応じてルールの拡張を行う.

用語のスパース性のため形容詞全般にルールが適用可能かどうかを調べることは\mbox{困難であ
る.}そのため,分析対象語のカバー範囲を明確にする必要がある.そこで,国立国語研究所
で行われた分析体系(西尾~1972)に基づいて,形容詞を分類し,その分類体系を網羅するよ
うに各形容詞を選び,その係りの振る舞いを調べることとした.このような分析を通し,若
干のルール拡張を行い,最終的に今回拡張した形容詞群に対しても,約95%という高い精
度で係りを特定できることを示す.

第2章では,我々がこれまでに提案した係りに関するルールを概説し,その問題点を\mbox{整理
する.}第3章では,国立国語研究所での研究に基づき形容詞全体を分類整理する尺度を定め,
多様なタイプの形容詞を分析対象として抽出可能とする.また,本論文で利用するコーパス
についても説明する.第4章では形容詞無依存ルールと形容詞依存ルールに分けて検証し,
その精度とルールの拡張について述べる.また,今回までに,7000件を越えるデータが
蓄積されたので,直感的に決定していた形容詞無依存ルールのルール間の適用順位について
も検証する.第5章では,対象語の拡張に伴い,新たに検出できたルールについて説明を行
い,全ルールを適用した後に得られる各形容詞の係りのDefault属性について説明する.
第6章では,それらを適用した結果の係り解釈の精度と現行システムとの比較を行う.


\section{提案済ルールの概説とその問題点}

我々の研究は,連体形形容詞に先行する格助詞「が」「の」格の係り特性に関するもので
あり,大量コーパスを利用し,言語現象を分析して係りの特徴をルール化し,それに,統計
的振る舞いを加味して係りを精度よく決めようとするものである.分析方法は,先行研究(田
中,荻野~1980,荻野~1987)とは異なり,形容詞を分類のキーとして,形容詞と前後の名詞
の関係から係りを決定するものである.対象とする構文は,連体形形容詞を含む「名詞1+
<*の|*が>+[副詞]+<イ形容詞|ナ形容詞>+名詞2」の形式である.菊池,伊東
(1999)の分析では,技術文においてある程度以上の頻度で利用されている形容詞を対象と
した.その結果抽出した係り決定のためのルールは,おおむね次のようなものである.

ルールは形容詞無依存のルールと形容詞依存のルールに大別され,以下のような\mbox{ものから
構成}される.なお,ルールはルール番号の小さい順に適用される.
\begin{description}
\item[\underline{形容詞の種類に無依存のルール}]~\\
  ルール1:名詞1が以下のものである場合は,名詞1は形容詞には係らない.\\
       数詞/単位/まとめる語/時詞/位置/区切り記号がある場合\\
  ルール2:複合助詞(での,からの,への等)及び「等の,等が」は形容詞には係らない.\\
  ルール3:形式名詞は係りを強く制御する.\\
    ルール3−1:名詞2が形式名詞の場合,名詞1は形容詞に係る.\\
    ルール3−2:名詞1が形式名詞の場合,名詞1は形容詞に係らない.\\
  ルール4:「が」格は名詞2以降にある係助詞「は」を越えて係ることはない.

\item[\underline{形容詞に依存するルール}]~\\
  ルール5:動きや相互関係に関連する形容詞はサ変名詞や転成名詞と結びつく.\\
  ルール6:「人」,「組織」,「国」などの属性を持つ名詞と特定の形容詞には係りに相関が\\     ある.\\
    ルール6−1:「人」,「組織」,「国」などの属性を持つ名詞と結びつかない形容詞が\\         ある.\\
    ルール6−2:名詞2が「人」の属性を持つ場合,その語と強く結びつく形容詞(若\\         い)がある.\\
  ルール7 接尾語と結びつきの強い形容詞は接尾語を持つ語と結びつく.

\item[\underline{係りのDefault属性}]~\\
  上記のルールに該当しない場合は,各形容詞の統計的な振る舞いから決定される係り
  の特性(Default属性)に従って係りが決まる.
\end{description}

しかし,これらのルールの抽出は技術文のコーパスから行い,一定の回数以上\mbox{該当する構
文で使}用されていた形容詞を対象とした.そのため,(1)対象とした形容詞に偏りがある恐れ
がある,(2)対象外の形容詞についての知見が得にくい等の問題が残る.この問題を解消する
ためには,形容詞の網羅的な分析が必要となる.そのため,まず次章で形容詞の体系的分類
について説明する. 

\section{形容詞選定と利用したコーパス}

\subsection{形容詞の分類体系}

国立国語研究所での研究(西尾~1972)によれば,形容詞は以下のように体系化される.\mbox{た
だし,}国立国語研究所の分類でもまだ分類項目が不足すると思われるものがあったため,分
類項目を追加した.追加した分類項目は【】で示したものである.また,文献
(菊池,伊東~1999)で分析対象とした用語を多く含む分類項目を《》で示した.
これにより,どの分類項目の形容詞が分析できているかを示した.
\begin{center}
\framebox{
\begin{minipage}{.9\textwidth}
  感情形容詞\\
   (1)感情,(2)感覚\\
  属性形容詞
  \begin{description}
  \item[]1.広範なものごとの属性\\
    (1)《存在》,(2)異同・関係,(3)《普通でないこと》,(4)危険・害の有無,\\
    (5)【《評価・状態・様態》】
  \item[]2.ものに関する属性\\
    (1)《空間的な量》,(2)【《数》】,(3)色,(4)音,(5)味,(6)におい,
    (7)【熱】,\\(8)【《動き・変化》】,(9)【《評価・状態・様態》】
  \item[]3.人に関する属性
  \item[]4.ことに関する属性\\
    (1)必然的な事態,(2)程度
  \end{description}
\end{minipage}}
\end{center}

上記分類に属する形容詞については,付録にその詳細を示した.上記分類からも分かる\mbox{通
り,文}献(菊池,伊東~1999)で分析対象とした形容詞は,ほとんどが「属性形容詞」
の「広範なものごとの属性」と「ものに関する属性」に属するものであった.技術文書での
使用頻度から対象語を選んだこともあり,分析対象とした形容詞に偏りがあることが分かっ
た.これは,技術文は事実の説明が主であり,人間の感情や感覚にまつわる説明はほとんど
なく,人の五感に関係する様な形容詞があまり出現していないためと思われる.今回の報告
では,形容詞全般を扱えるようにするため,上記分類を網羅するよう用語を選定した.

\subsection{利用したコーパスと分析対象とした形容詞}

本論文の主目的は,技術文ではあまり使用されない形容詞に対して,既抽出のルールの検
証と拡張を行うものである.そのため対象とするコーパスは新聞文に限定した.使用したコ
ーパスは,毎日新聞 95年度版 1月,10月,12月の記事(計約24Mバイト)である.
ただし,ルールの適用順位の検証には,抽出済の全例文を使用した.評価が主目的のため,
コーパスはすべて評価用と位置づけ,分析用と評価用という区分けはしなかった.

係りを決定するためには,一定頻度以上の形容詞を対象とする必要がある.そのため本論
文でも,6件以上出現した形容詞のみを分析対象とした.その結果,183語を調べたが,
出現頻度が5件以下の語が121語あり,最終的な分析対象語は表1の62語となり,該当
する分析対象文総数は1273件であった.

\begin{table}[h]
\caption{分析対象語}
\begin{center}
  \begin{tabular}{|l|l|} \hline
    分類 & 分析対象とした用語 \\ \hline
    イ形容詞 & 青い,赤い,明るい,浅い,温かい,厚い,熱い,甘い,\\
    & 痛い,著しい,〜色い,薄い,美しい,美味しい,遅い,\\
    & 重い,かたい,軽い,厳しい,暗い,黒い,苦しい,濃い,\\
    & 寂しい,白い,すごい,鋭い,狭い,楽しい,近い,\\
    & 冷たい,つらい,遠い,苦い,鈍い,速い,早い,古い,\\
    & 欲しい,細い,珍しい,やさしい,弱い\\ \hline
    ナ形容詞 & 安全な,緩やかな,同じ,勝手な,〜急な,危険な,\\
    & 貴重な,静かな,自由な,新鮮な,好きな,速やかな,\\
    & 大胆な,大変な,得意な,苦手な,熱心な,不安な,\\
    & 不自由な\\ \hline
  \end{tabular}
\end{center}
\end{table}


\section{抽出済みルールの検証}

\subsection{形容詞無依存ルール}
\subsubsection{ルールの検証}

形容詞無依存の4つの既抽出ルールを,体系的に抽出した全ての分析対象文に適用し,そ
の正当性を調べた.その結果を表2に示す.
\begin{table}[h]
\caption{形容詞無依存ルールの評価結果}
\begin{center}
  \begin{tabular}{|l|c|c|r|} \hline
    適用ルール名 & 該当件数 & 正解件数 & 正解率\\ \hline
    ルール1 & 176 & 176 & 100\% \\ \hline
    ルール2 & 76  & 76  & 100\% \\ \hline
    ルール3−1 & 86 & 83 & 96.5\% \\ \hline
    ルール3−2 & 7 & 7 & 100\% \\ \hline
    ルール4 & 2 & 2 & 100\% \\ \hline
  \end{tabular}
\end{center}
\end{table}
この結果,ルール3−1を除いて,形容詞無依存ルールの適用は問題がないことが分かった.
ルール3−1は,「こと」に関する例であり,これらは更なるルールの拡張が必要なことを
示している.これは,判定詞「だ」を伴って助動詞化するものの範疇をどのようなものとす
るかというものである.以下に,解釈の範囲の拡張について述べる.

\subsubsection{ルールの拡張}

形式名詞は,形容詞に関係なく直前の用言の振る舞いを規定する.そのため,形式名詞「こ
と」に関して一般的と考えられる以下のルールを追加する.
{\bf \begin{description}
\item[\underline{ルールの拡張}]: 呼応関係にある語については,その間に挟まれる形式名詞「こと」は,
呼応の前の部分を閉じる働きはしない.このような用法には以下のようなものがある.\\
        いかに 〜 ことか.\\
        どんなに 〜 ことか.\\
        何と 〜 ことか.
\end{description}}

但し,各々の品詞は「いかに」が副詞,「どんなに」が連体詞,「なんと」が連語とされ
るがここでは,呼応関係にある副詞的語として,副詞として扱う.\\
   例1)夫人はマスコミに曲解されることがいかに苦しいことかを分かっている.\\
   例2)瞬時の異変がもたらした結果の,なんとすさまじく壊滅的なことか.

例2は,句読点を含む場合(ルール1)のケースであるが,それは呼応関係を中に含むこ
とをより鮮明にしている例文である.なお,これらは,判定詞を伴って助動詞化するケース
(「ことであるか」)の短縮形(「ことか」)と解釈すれば,既存のルールの範疇に納まる
ことになる.

\subsubsection{ルールの適用順位の決定}

今回までの研究により,ルールの適用順位を検討できる程度の用例が蓄積された(\mbox{約70
00}件).そこで,形容詞無依存ルールに対して適用順位の評価を行った.ルール間で競合
が起きるのは,名詞1に関するルール(ルール1,ルール2,ルール3—2)と名詞2の関
するルール(ルール3—1)の間である.評価した結果は以下のようになった.\\
   評価結果:   ルール1>ルール3−1>ルール3−2>ルール2

以下に,ルール3—1との関係で順位検証に利用した例と出現件数を示す.
\begin{description}
\item[ルール1(名詞1が単位)>ルール3−1(名詞2が形式名詞)](出現件数2件)\\
        例)野生では母子で群れに戻るまでのこの数時間が最も危険な時である.\\
        この例では,「数時間が」は「危険な」には係らず,「である」に係る.
\item[ルール1(名詞1が時)>ルール3−1(名詞2が形式名詞)](出現件数2件)\\
        例)雪の降る日は囲いをし,夏の暑い時は水やりに気をつけ,・・・.\\
        この例では,時を表す名詞句「夏の」は,「暑い」に係らず「暑い時」に係る.
\item[ルール3−1(名詞2が形式名詞)>ルール3−2(名詞1が形式名詞)](出現件数3件)\\
        例)それを承知で親や生徒も入学する場合が多いため,一概には非難できない.\\
        この例では,「ため」が形式名詞であるため,「場合」も形式名詞であるが「多い」に係っている.
\item[ルール3−1(名詞2が形式名詞)>ルール2(複合助詞)](出現件数2件)\\
        例)前足や鼻筋,首などが白いため「なぜこの犬が黒ベエなのか」と瀬戸雄三社長が指  摘.\\
        この例では,「ため」が形式名詞であるため,「〜などが」は複合助詞相当であるが「白い」に係っている.
\end{description}

最後に示した例が,ルール2とルール3−1の適用順位の決定例である.これは,文献(菊
池,伊東~1999)と順位が異なっている.但し,評価に使った件数が約7000件であ
るので,出現件数(2件)からみても,この順位の食い違いによる影響は非常に小さいもの
である.しかし,この例文により適用順位の変更が必要となった.

\subsection{形容詞依存ルール}

自然言語では,利用される用語は多岐にわたる.そのため用語を文字列として分類するの
ではスパース性のために,適用範囲に制約が生じる.スパース性を解消するためにはシソー
ラス情報を利用する等の上位カテゴリでの分類が必須である.しかし,一般にシソーラス情
報は意味的考察の上でなされており,それをそのまま係りの分類に利用できるかどうかは明
確でない.そのため,ここで既存の意味的分類が,係りを規定するための有効情報となり得
るか検証する.その検証手段の一つとして,類似語や反意語に対して既抽出ルールを適用し,
その正当性を評価する.もしこれらが正当であれば,意味的分類がそのまま係りを決める分
類として利用できることとなる.ここでは,以下に示す類似語と反意語に対して分析する.\\
\begin{tabular}{l@{ →→ }l}
(分析済み語)激しい & (類似語)早い,速い,速やかな,遅い,〜急な\\
(分析済み語)深い   & (反意語)浅い\\
(分析済み語)強い   & (反意語)弱い\\
(分析済み語)広い   & (反意語)狭い
\end{tabular}\\
各々に対して,抽出済みルールと,それの適用結果を以下で説明する.

\subsubsection{「激しい」の類似語でのルールの検証}

「激しい」に適用されるルールは,次の2つである.\\
   ルール5:動きや相互関係に関連する語はサ変名詞や転成名詞と結びつく.\\
   ルール6−1:「人」,「組織」,「国」などの属性を持つ名詞と結びつかない.

これを動作を表す語に適用した結果を表3に示す.この表の意味は,「早い」では,\mbox{ルー
ル5に}該当する表現が6件出現し,そのうち5件は正しく解釈できたが,1件はルール5を
適用すると誤った解釈をする.ルール6−1は該当するものが3件見つかり,3件とも正し
く解釈されたというものである.
\begin{table}[ht]
\caption{類似語へのルール適用の確からしさ}
\begin{center}
  \begin{tabular}{|l|l|l|} \hline
    該当語 & ルール5 & ルール6−1\\ \hline
    早い & 6件出現 5件正解 & 3件出現 3件正解\\ \hline
    速い & 19件出現 18件正解 & 4件出現 4件正解\\ \hline
    速やかな & 8件出現 8件正解 & 該当例なし\\ \hline
    遅い & 1件出現 1件正解 & 1件出現 1件正解\\ \hline
    〜急な & 13件出現 13件正解 & 1件出現 1件正解\\ \hline
  \end{tabular}
\end{center}
\end{table}

「早い」で誤った解釈をした例は以下の通り.\\
   例)朝の早い仕事やから,五時すぎ電車に乗って大阪へ向かってた.

この例では,「早い」は「仕事」と結びつくのでなく,「朝」に係る.このようにルールを
適用すると解釈を誤る場合もあるが,ほとんどは正解であり,動作を示す形容詞には,ルー
ル5とルール6−1が適用可能との結果を得た.なお,「朝」と「早い」は,早い朝=早朝
のように,共起性の強い語とすべきものと思われるが,ここでは共起の扱いはしなかった.

\subsubsection{反意語でのルールの検証}

\noindent
(1)「強い」の反意語「弱い」でのルールの検証

「強い」に適用されるルールは,次の2つである.\\
   ルール6−1:「人」,「組織」,「国」などの属性を持つ名詞と結びつかない.\\
   ルール7:接尾語(色,性,力)と結び付く.

これを反意語「弱い」に適用した結果を表4に示す.
\begin{table}[h]
\caption{反意語「弱い」への適用精度}
\begin{center}
  \begin{tabular}{|l|l|l|} \hline
    該当語 & ルール6−1 & ルール7 \\ \hline
    弱い & 25件出現 25件正解 & 6件出現 6件正解 \\ \hline
  \end{tabular}
\end{center}
\end{table}
表4より,反意語へのルールの適用可能性は十分にあることが分かった.ただし,「弱い」
の場合,接尾語として出現したものは,「性」,「力」のみであった.コーパスの量が増え
ると「色」も出現するものと思われる.
\vspace{.5\baselineskip}

\noindent
(2)「深い」の反意語「浅い」でのルールの検証

「深い」に適用されるルールは,次の2つである.\\
   ルール5:動きや相互関係に関連する語はサ変名詞や転成名詞と結びつく.\\
   ルール6−1:「人」,「組織」,「国」などの属性を持つ名詞と結びつかない.

これを反意語「浅い」に適用する.
\begin{table}[h]
\caption{反意語「浅い」への適用精度}
\begin{center}
  \begin{tabular}{|l|l|l|} \hline
    該当語 & ルール5 & ルール6−1\\ \hline
    浅い & 1件出現 1件正解 & 3件出現 2件正解\\ \hline
  \end{tabular}
\end{center}
\end{table}

ルール6—1の誤り例は,「横一線の浅い4人のDFラインを敷き,〜」というものであ
る.この文では,「浅い」は「DFライン」に係るが,「4人」に係るか「DFライン」に
係るかを決定することは容易ではない.我々は,直後の名詞を選択することにしているので,
「4人」を「人」の属性を持つものと判断し,ルール6—1により「横一線の」は「浅い」
に係ると判断した.「4人のDFライン」が正しく解析され,この語の主要語がDFライン
と判別できれば,ルールの適用誤りは解消される.すなわち,反意語へのルールの適用も十
分可能であることが分かった.
\vspace{.5\baselineskip}

\noindent
(3)「広い」の反意語「狭い」でのルールの検証

「広い」に適用されるルールは,次のものである.\\
   ルール6−1:「人」,「組織」,「国」などの属性を持つ名詞と結びつかない.\\
しかし,調査したコーパス中には該当する「狭い」の例がなく,このルールの正否の判断は
できなかった.

以上の例からも分かるように,既抽出のルールは類似語及び反意語に対して,適用可能と
判断できると思われる.しかし,形容詞は各語が多面的な意味を持つので,検証なしに無条
件で類似語や反意語に同じルールを適用するのは注意を要するものと思われる.
\vspace{.5\baselineskip}

\noindent
(4) 反意語と係りのDefault属性の相関

「新しい」という語は形容詞依存のルールは持たないが,先行する「が」「の」格が形容
詞に係らないという意味でイ形容詞としては特徴的であった.この反意語「古い」について
も同様の傾向が存在するかを分析した.結果は,共起(歴史の古い〜)を除いて,48件中
47件は,先行する「が」「の」格が形容詞「古い」に係らなかった.しかし,このような,
係りのDefault属性は,すべての反意語に適用できる訳ではない.例えば,「自由な」−「不
自由な」のペアでは,「の」格に対して,まったく係り方が異なる.「遠い」−「近い」で
は,「遠い」は先行する「が」「の」格が係らない傾向があるのに対し,「近い」は先行す
る「が」「の」格が係る傾向がある.そのため,係りのDefault属性に関しては,各語ごと
に決定しなければならないことが分かった.

\subsubsection{ルールの適用範囲の拡張(適用可能な語の拡張)}

形容詞に依存するルールを今回抽出した語にも適用可能とするために,\mbox{ルール適用用語の
拡張}をした.対象となるルールはルール5,6,7である.
\vspace{.5\baselineskip}

\noindent
(1)「ルール5:動きや相互関係に関連する語はサ変名詞や転成名詞と結びつく」の対象となる用語の拡張

サ変名詞と結びつく語には以下の様なものがある.\\
   軽い,早い, 速い,かたい,著しい\\
ただし,次の語にもこのルールは適用可能であるが,特にこのルールを適用しなくても
係り決定の精度には変化がなかったため,ルール適用の対象語とはしない.\\
   速やかな,遅い,〜急な
\vspace{.5\baselineskip}

\noindent
(2)「ルール6−1:「人」,「組織」,「国」などの属性を持つ名詞と結びつかない形容詞」の対象となる用語の拡張

人と結び付かない語には以下の様なものがある\\
   \begin{tabular}{l@{ → }l}
        重さを示す語 & 軽い,重い\\
        時を示す語 & 古い\\
        色を示す語 & 黒い,白い\\
        \multicolumn{2}{l}{人の状態を示すことのある語}\\
        & やさしい,速い,温かい,寒い,鋭い,厚い,甘い,可愛い,\\
        \multicolumn{1}{l@{\protect\phantom{ → }}}{ } & 弱い,苦しい,美しい,著しい,濃い
      \end{tabular}
\vspace{.5\baselineskip}

\noindent
(3)「ルール7 接尾語と結び付きの強い形容詞は接尾語のある語に結び付く」の対象となる用語の拡張

接尾語と強い結びつきのある語には次のようなものがある.ここで,「色」は同等の意
味合いを持つ「色彩」「色合い」にも拡張する.\\
   \begin{tabular}{l@{ $\cdots$ }l}
        濃い & 色(色彩,色合い),度\\
        弱い & 力,性
      \end{tabular}
      

\section{新規ルールの追加}

分析対象となる形容詞を拡張した結果,係りにおいて特徴的な振る舞いを\mbox{するものが検出
でき}た.そこでそれを,新たにルールとして設定することとした.このようなものには形容
詞無依存ルールと,形容詞依存ルールで各々1つ存在する.

\subsection{形容詞無依存ルールの追加}

網羅的に抽出した形容詞を分析した結果,次のルールも有効であることが分かった.
{\bf \begin{description}
\item[\underline{追加ルール1}]:「名詞1」の直前に最上級,もしくは最上級相当を示す副詞
  (副詞相当語)が存在する場合,形容詞に先行する「が」「の」格は形容詞に係る.
\end{description}}

このような副詞には以下のものがある.\\
   一番,最も\\
これらは4例存在し,4例ともこの規則にしたがった.\\
   例)破防法で最も規制の厳しい解散指定を請求する.\\
   例)ソスコベツ第一副首相は昨年,最も批判色の濃い独立テレビの放送免許取り\\
     消しの可能性に言及.\\
   例)世界で一番森の美しい国,亜熱帯から亜寒帯までのさまざまな樹種に恵まれ\\
     た国が,この日本だ.\\
   例)一行のなかでも最も体重の重い人間を乗せて先に送ってあった.

このルールは,形容詞無依存のルールである.用例数はあまり多くなくはないが,\mbox{これは
係りの}非交差の原則に則るものでもあるのでルール化した.ルール適用順位は,形容詞無依
存ルールとしては一番弱い順位(5番目のルール)に設定する.
    
\subsection{形容詞依存ルール追加}

「感情形容詞」の中には,目的格「を」をとると分類される特殊なものがある(西尾~1972).
このような用語は係りの面からも特徴ある振る舞いをしている.そのため次のルールを追加
する.
{\bf \begin{description}
\item[\underline{追加ルール2}]:「を」格を取る以下の感情形容詞
  (苦手な,好きな,得意な,欲しい)には,先行する「が」「の」格が係る.
\end{description}}
これらの語の係りの分析結果を表6に示す.表からも分かる通り,先行する「が」「の」\mbox{格
はこ}れらの形容詞に係る確率が非常に高い.
\begin{table}[h]
\caption{「を」格をとる感情形容詞の係り特性}
\begin{center}
  \begin{tabular}{|l|c|c|c|c|} \hline
    該当語 & \multicolumn{2}{c|}{先行する「が」格が}
    & \multicolumn{2}{c|}{先行する「の」格が}\\ \cline{2-5}
    &  係る  & 係らない &  係る  & 係らない \\ \hline
    苦手 & 4 & 0 & 2 & 0 \\ \hline
    好きな & 15 & 3 & 36 & 3 \\ \hline
    得意な & 2 & 1 & 4 & 0 \\ \hline
    欲しい & 6 & 0 & 6 & 0 \\ \hline
  \end{tabular}
\end{center}
\end{table}

これらの形容詞が出現した場合で,「が」「の」格が係らないケースは,(1)動作主が\mbox{省略
されて}いるためか,(2)複数箇所に係るため係りの強度がより強い方を係りの正解としたこと
により発生したものである.たとえば,次の例では,係り先が複数個所となり,片方に対し
て日本語特有の省略が発生しているために「が」格が係らないと解釈したものである.\\
   例)ファワズさんが不得意な英語で事情聴取を受けた.\\
     → ファワズさんが(彼女が)不得意な英語で事情聴取を受けた.\\
しかし,この文は「ファワズさんは,(彼女が)不得意な英語で事情聴取を受けた.」と解
釈することが可能である.これは,ゼロ代名詞の省略の問題として解析されるべきものであ
る.そうすると,全てのケースで,先行する「が」「の」格が係るという結果になる.

ただし,意味の問題に立ち入ると,これらの語は動詞と非常に似通った振る舞いを\mbox{してい
る.例}えば,「私は本が欲しい」という文について考察する.この文の意味は「私が本を欲
する」である.すなわち,「欲しい」は「を」格を対象格としてとることを示している.た
だしこの対象格は動詞の場合とは異なり,連体形形容詞との関係では,格助詞「が」「の」
が表層格として使用される.この文から連体修飾を含む文を作ると「私の欲しい本を・・・」
「本の欲しい私は・・・」のようになる.どちらの場合も「欲しい」には先行する「の」格
が係る.しかし同じ「の」格であっても,「私の」の「の」は動作主を表し,「本の」の「の」
は対象を表している.そのため,解釈を正確に行うためには,係りを正しく処理するだけで
なく,各用語の意味属性を意識する必要がある.

\subsection{分析対象語の係りのDefault属性の決定}

以上のルールを適用することにより,係りの揺らぎに相当する部分が取り除かれる.ルー
ル適用後の係りは各形容詞の固有の係り特性(係りのDefault属性)となる.そのようにし
て調べた各形容詞の係りのDefault属性を次の分類に従い表7にまとめる.\\
   分類1:「が」「の」格は形容詞に係る.\\
   分類2:「が」格は形容詞に係るが,「の」格は形容詞に係らない.\\
   分類3:「が」格は形容詞に係らないが,「の」格は形容詞に係る.\\
   分類4:「が」「の」格は形容詞に係らない.
\begin{table}[h]
\caption{形容詞の係りのDefault属性}
\begin{center}
  \begin{tabular}{|l|l|} \hline
    分類 & 分析対象語 \\ \hline
    分類1 & 浅い,著しい,薄い,遅い,濃い,好きな,近い,得意な, \\
    & 早い,速い,不自由な,欲しい,苦手な,鈍い,弱い\\ \hline
    分類2 & 〜色い,美しい,重い,軽い,黒い,苦しい,寂しい,自由な,\\
    & 狭い,楽しい\\ \hline
    分類3 & 該当なし \\ \hline
    分類4 & 青い,赤い,明るい,温かい,熱い,厚い,甘い,安全な,痛い,\\
    & 美味しい,同じ,かたい,勝手な,危険な,貴重な,厳しい,\\
    & 〜急な,暗い,白い,静かな,新鮮な,すごい,速やかな,鋭い,\\
    & 大胆な,大変な,冷たい,つらい,遠い,苦い,熱心な,不安な,\\
    & 古い,細い,珍しい,やさしい,緩やかな\\ \hline
  \end{tabular}
\end{center}
\end{table}


\section{適用後の係り解釈精度と現行システムとの比較}

以上の全てのルールと係りのDefault属性を適用した結果,分析対象文総数\mbox{1273件の
うち,}係りを正しく解釈したものは1217件,誤って解釈したものは56件であり,正解
率は95.6%であった.すなわち,本方式を利用すれば,約95%の精度で「が」「の」
格が形容詞に係るかどうかを判定することが可能となることが分かった.

本方式の精度がどの程度であるかを判断する目的で,ある市販のソフトを利用して\mbox{解釈精
度の}比較評価を行った.市販のソフトの場合一般に,解析の中間段階にある係りを図示する
ことはない.そのため,出力された英文を頼りに係りの精度を検証した.しかし係りの曖昧
性は,単語の持つ曖昧性に起因する場合も多々存在する.そこで入力となる文に次の様な処
置を施して評価した.
\begin{enumerate}
\item 文は本来の意味を損ねないように考慮しながらできるだけ短くした.
\item 未登録語になりそうな語(人名等)は明らかにその属性であると分かる語にした.
\end{enumerate}
このような処置をほどこして,係りを検証すると,市販システムでの連体形形容詞の係りの
解釈精度は80%程度であった.

以上の結果から,本報告で説明したルールの組み込みは非常に有効であると言える.


\section{まとめ}

本論文では,体系的分析に従い形容詞を網羅的に分析し,反意語や類似語には同種のルー
ルが適用できる可能性が高いことを示した.また追加分析した用語から新たに2つの追加ル
ールを見出すことができた.その結果,約95%の精度で連体形形容詞に関わる係りを特定
できた.今回は分析対象とする形容詞を体系的に選んだので,係りに関してはこれらのルー
ルで形容詞全体をカバーできるものと考えている.また,現行システムの係り解釈の精度と
比較し,本方式が有効であることを示した.しかし,実際に現行システムへの組み込みには
費用対効果比に基づく詳細なフィージビリティ分析が必要となる.既存のシステムの改良は
大変費用の掛かるものとなるが,本方式では段階的な組み込みが可能であるため,以下のよ
うな手順での組み込みが可能と思われる.
\begin{enumerate}
\item 顕著な係りのDefault属性の組み込み
\item 形容詞単位でのルールの組み込み
\item 顕著ではない係りのDefault属性の組み込み
\end{enumerate}
また,本論文で提案した方式は,形容詞の「を」格や「に」格の係り分析にも利用可能であ
る.今後は,連体形形容詞と「が」「の」格以外の格の関係について同様に係り関係を明確
にしていく予定である.


\section*{謝辞}

本研究にあたり,多くの支援をしていただきました富士通静岡エンジニアリング\mbox{社長田口尚
三}氏に心より感謝いたします.また,コーパスの利用を許諾していただきました毎日新聞社
に感謝いたします.


\newpage
\section*{付録 形容詞の意味的分類(調査対象の用語とその分類)}

{\small
\begin{center}
\begin{tabular}{|c|l|l|l|} \hline
大分類 & 中分類 & 小分類 & 具体的用語 \\ \hline
感情   & 感情   & 感情   &
	愛しい,嬉しい,気味悪い,悔しい,憎い,恥ずかしい,\\
形容詞 & & &
	安心な,嫌な,可笑しい,つらい,可愛い,懐かしい,楽しい,\\
& & & 	苦しい,嫌いな,好きな,寂しい,心配な,悲しい,不安な,\\
& & & 	怖い,満足な,面白い,愉快な,可哀相な,気の毒な,汚い,\\
& & & 	恐ろしい,苦手な,得意な,美しい,有り難い,欲しい\\ \cline{2-4}
& 感覚 & 感覚 &
	痛い,痒い,眠い\\ \hline

属性   & ものごと& 存在 &
	《無い》 \\ \cline{3-4}
形容詞 & の属性  & 異同・ &
	あべこべな,同じ,逆な,さかさまな,反対な,等しい,\\
& & 関係 & 	そっくりな\\ \cline{3-4}
& & 普通で &
	《特殊な》《特別な》,異様な,おかしい,変な,妙な\\
& & ない & \\ \cline{3-4}
& & 危険・害 &
	危ない,安全な,危険な,有害な\\ \cline{3-4}
& & 評価・ &
	《よい》《良い》《悪い》《完全な》《さまざまな》《詳細な》\\
& & 状態・ &
	《複雑な》《正確な》《正しい》《難しい》《新たな》《適切な》\\
& & 様態 & 	
	《重要な》《可能な》《不可能な》《容易な》《困難な》\\
& & & 	《簡単な》《有効な》《容易な》《大幅な》《主な》《主要な》\\
& & & 	《〜的な》,緩やかな,自由な,不自由な,きつい\\ \cline{2-4}
& ものに関 & 空間的な &
	《大きい》《大きな》《高い》《小さい》《小さな》《長い》\\
& する属性 & 量 &
	《低い》《広い》《深い》《短い》《〜大な》《〜規模な》,\\
& & & 	かたい,鋭い,丸い,狭い,固い,細い,四角い,鈍い,太い,\\
& & & 	浅い,厚い,粗い,薄い,濃い,近い,遠い\\ \cline{3-4}
& & 数 &
	《多い》《少ない》《豊富な》\\ \cline{3-4}
& & 色 &
	青い,赤い,黒い,白い,〜色い\\ \cline{3-4}
& & 音 &
	うるさい,けたたましい,騒がしい,静かな,静寂な,静粛な,\\
& & & 	騒々しい,やかましい\\ \cline{3-4}
& & 味 &
	甘い,旨い,美味しい,からい,しつこい,渋い,淡白な,\\
& & & 	苦い,美味な,まずい\\ \cline{3-4}
& & におい &
	かぐわしい,臭い,こうばしい\\ \cline{3-4}
& & 熱 &
	涼しい,ぬるい,暖かい,温かい,寒い,暑い,熱い,冷たい\\ \cline{3-4}
& & 動き・ &
	《激しい》《〜速な》〜急な,早い,速い,速やかな,遅い,\\
& & 変化 & 荒い,緩い,穏やかな\\ \cline{3-4}
& & 評価・ &
	《詳しい》《新しい》《強い》《不要な》《必要な》,古風な,\\
& & 状態・ &
	新鮮な,古い,古臭い,貴い,貴重な,珍しい,軽い,重い,\\
& & 様態 &
 	明るい,暗い,眩しい,弱い,淡い\\ \cline{2-4}
& 人に関す & 人に関す &
	《若い》,あいくるしい,あどけない,うまい,内気な,\\
& る属性   & る属性 &
	おおげさな,おだやかな,男らしい,おとなしい,快活な,\\
& & & 	かしこい,勝気な,勝手な,か弱い,頑固な,几帳面な,\\
& & & 	きびしい,器用な,気楽な,軽率な,軽薄な,元気な,健康な,\\
& & & 	懸命な,強情な,正直な,上手な,丈夫な,真剣な,慎重な,\\
& & & 	親切な,丹念な,貞淑な,丁寧な,柔和な,熱心な,のんきな,\\
& & & 	本気な,敏捷な,無愛想な,真面目な,まめな,無口な,\\
& & & 	むじゃきな,やさしい,雄弁な,利発な,冷酷な,冷淡な,\\
& & &   腕白な\\ \cline{2-4}
& ことに関 & 必然的な &
	当然な,当たり前な\\ 
& & 事態 & \\ \cline{3-4}
& する属性 & 程度 &
	著しい,すごい,甚だしい,顕著な,大変な\\ \hline
\end{tabular}
\begin{tabular}{p{.8\textwidth}}
《》は文献(菊池,伊東~1999)で分析した用語を示す.\\
文脈によっては複数のカテゴリに属する用語もあるが,表層上での用語の
洗い出しを主目的としたため,用語は一個所のみに現れるよう集約した.
\end{tabular}
\end{center}}
\newpage

\nocite{Kikuchi1999i,Nishio1972,Tanaka1980,Ogino1987}
\bibliographystyle{jnlpbbl}
\bibliography{v06n7_06}


\begin{biography}
\biotitle{略歴}
\bioauthor{菊池 浩三(正会員)}{
1970年大阪大学理学部物理学科卒業.
1972年大阪大学大学院修士課程修了.
1973年富士通株式会社入社.
1983年(株)富士通静岡エンジニアリング出向.
1999年静岡大学大学院電子科学研究科博士課程修了,工学博士.
自然言語処理の研究に従事.
言語処理学会,情報処理学会各会員.}

\bioauthor{伊東幸宏(正会員)}{
1980年早稲田大学理工学部電子通信学科卒業.
1987年同大学院博士後期課程修了.
同年,早稲田大学理工学部電子通信学科助手.
1990年静岡大学工学部情報知識工学科助教授.
現在静岡大学情報科学科助教授,工学博士.
自然言語理解,知的教育システムなどに興味をもつ.
言語処理学会,電子情報通信学会,情報処理学会,
人工知能学会,日本認知学会,教育情報システム学会各会員.}

\bioreceived{受付}
\biorevised{再受付}
\bioaccepted{採録}

\end{biography}

\end{document}
