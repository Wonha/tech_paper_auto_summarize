    \documentclass[japanese]{jnlp_1.4}
\usepackage{jnlpbbl_1.3}
\usepackage[dvips]{graphicx}
\usepackage{amsmath}
\usepackage{hangcaption_jnlp}
\usepackage{udline}
\setulminsep{1.2ex}{0.2ex}
\let\underline


\usepackage{color}

\Volume{19}
\Number{3}
\Month{September}
\Year{2012}

\received{2012}{1}{5}
\revised{2012}{3}{8}
\accepted{2012}{4}{6}

\setcounter{page}{193}

\jtitle{地方自治体の例規比較に用いる条文対応表の作成支援}
\jauthor{竹中 要一\affiref{Osaka} \and 若尾 岳志\affiref{Dokkyou}}
\jabstract{
地方自治体が制定する条例(規則も含め,以下例規という)は,章節/条項号という階層を有する,
基本的に構造化された文書である.各自治体はそれぞれ別個に各議会等でこの例規を制定するため,
複数の自治体が同一の事柄に関する規定(例えば「淫行処罰規定」など)を有している事が多い.
この同一の事柄に関する規定の自治体間における異同を明らかにするための比較は,
法学教育や法学研究,地方自治体法務,企業法務において実施されている.
実務における法の比較では,対応する条項を対とし,それらの条文を左右または上下に
並べた条文対応表の作成が主体となっている.
これまで条文対応表は手作業で作成されてきたが,
対象とする例規の条数や文字数が多い場合の表作成には3時間以上も必要としていた.
そのため計算機による条文対応表の作成支援が強く求められているが,本件に関する研究はこれまでに行われていない.
そこで我々の研究は,条文対応表を計算機で自動作成することによる条文対応表の作成支援を目的とする.
この目的を達成するため,我々は条文対応表を,各条をノードとする二部グラフとしてモデル化し,
このモデルに基づき条文対応表を自動作成するために有効な手法の検討を行った.
二文書間の類似度を定義する多くの研究がこれまでに報告されている.
これらの類似度比較手法より
本研究ではベクトル空間モデル,最長共通部分列,及び文字列アライメント(編集コスト可変のレーベンシュタイン距離)
に基づく96個の類似尺度の性能を比較した.
評価には愛媛県の11の条例とそれに対応する香川県の11の条例を用い,
法学者が作成した条文対応表に基づき正解率を求めた.
その結果,名詞,副詞,形容詞,動詞,連体詞を対象としたベクトル空間モデルに基づく類似尺度の
正解率が85\%と最も高かった.また,文字列アライメントに基づく類似尺度の正解率は最高で81\%,最長共通部分列は
最高で75\%であった.
本研究は条文対応表の作成支援であるため,
推定された対応関係の信頼度,あるいは尤もらしさを提示する事が望ましい.
そこで
各比較手法で最も正解率の高かったパラメータを用いた合計3つの類似尺度に対して
受信者操作特性曲線による評価を行ったが,曲線下面積がいずれも狭くて
信頼度の尺度として適さない.
そこで,推定された対応関係の類似度を二番目に高い類似度を持つ対応関係の値で
割る事による正規化を行ったところ,
最長共通部分列の曲線下面積が0.80と最も高く,
ベクトル空間モデルの面積は0.79と良好であった.
以上の評価結果より,条文対応表の作成支援では
条見出しに対して最長共通部分文字列を,
条文に対してベクトル空間モデルをそれぞれ適用した類似尺度を
併用する事が,
そして得られた条文対応関係の信頼度を評価する尺度としては
二番目に高い類似度で割った値を用いるとよい事を明らかにした.
}
\jkeywords{法,条例,比較法,ベクトル空間モデル,文字列アライメント,\pagebreak 地方自治体}

\etitle{Automatic Generation of Article Correspondence Tables for the Comparison of Local Government Statutes}
\eauthor{Yoichi Takenaka\affiref{Osaka} \and Takeshi Wakao\affiref{Dokkyou}} 
\eabstract{
Local governments establish ordinances and regulations (hereinafter collectively referred to as ``statutes''). They are structured documents that
possess a chapter $>$article $>$paragraph $>$item hierarchy. Since each local government establishes statutes in its councils independently, similar statutes
on the same matter are often found in separate local governments (e.g. punishment for obscene habits). In legal education, legal research and legal
works at local government and business enterprise, comparisons are made to clarify the differences between similar statutes. In the comparison of laws
for practical purposes, article correspondence tables are normally created with pairs of corresponding articles aligned horizontally or vertically.
The objective of our research is to use a computer to automatically generate the article correspondence tables that are currently created
manually. In order to accomplish this objective, we have focused on the relationships between articles in article correspondence tables, which were
modeled with directed bipartite graphs that used each article as a node. 96 methods based on the vector space model, longest common subsequence and
sequence alignment were examined in order to clarify effective methods for searching for corresponding articles. In the course of the research, we
automatically generated article correspondence tables  of 22 statutes in total (11 statutes of Ehime and Kagawa Prefectures, respectively). Their
accuracy rates were calculated based on article correspondence tables created by legal scholars. Consequently, the vector space model-based method
proved the highest accuracy rate at 85\%. Its targets were nouns, adverbs, adjectives, verbs and attributives. The sequence alignment-based method
showed up to 81\% of accuracy rate, while the rate with the longest common subsequence method was 75\%.
As the results of the computer-generated article correspondence tables are checked by legal scholars on a practical level,
it is required to posess the degree of reliability for each relationships between articles.
To meet the requirement, we examined two measurements for the three methods by receiver operating charasteristic curve.
The results shows the ratio of the selected relations and the runner-up gives 0.80 AUC for longest common subsequences.
In this research, the problem was defined by focusing on the correspondence relationship between articles in the article correspondence tables.
For practical purposes, there is a need to focus not just on the correspondence relationship between articles, but also on the clarification of
different words used in corresponding articles. Since  vector space model cannot be used to clarify such differences, sequence alignment---with which
it is feasible to clarify differing texts---is necessary. Composite methods that combine those two will therefore be required in the future.
}
\ekeywords{Law, Local Government Statue, Comparative Law, Vector Space Model, \linebreak Sequence Alignment, Local Government}

\headauthor{竹中,若尾}
\headtitle{地方自治体の例規比較に用いる条文対応表の自動生成}


\affilabel{Osaka}{大阪大学大学院情報科学研究科}{Graduate School of Information Science and Technology, Osaka University}
\affilabel{Dokkyou}{獨協大学法学部}{Faculty of Law, Dokkyo University}



\begin{document}
\maketitle

\section{はじめに}\label{sec:hajimeni}


法は章節/条項号という階層を有する,基本的に構造化された文書であり,
国(国会)の制定する法律,地方自治体(議会)が制定する条例の二つがある.
前者に規則を加え法規,後者に規則を加え例規と総称される.
日本国内で法律を制定する主体は国家のみだが,条例を制定する地方自治体は多数存在する.
そのため,同一の事柄について規定する多数の例規が地方自治体ごとに存在することになる.
例えば,各県の象徴であり,旗に用いられる県章を定めた条例は全都道府県で制定されており,
青少年の保護育成を目的とする条例は,長野県を除く46都道府県で制定されている.
これら同一事項に関する条例は相互に類似しているものの,
地方自治体の置かれた状況が異なるため,随所に相違点が存在している.
一例として,青少年の保護育成を目的とした条例では,青少年の深夜外出を制限しているが,
その制限される時間が異なっている事が挙げられる.
東京都や愛媛県では午後11時から午前4時を深夜と定義している一方,高知県では午後10時から午前4時を
深夜としている.
また,大阪府では外出を制限する時間帯を年齢によって変えており,
16歳未満の場合は午後8時から午前4時まで外出を制限される.
このような違いを明確化するため例規比較が行われる.

例規比較は,自治体間の違いを明らかにする教育・研究活動以外にも,企業法務や自治体法務においても
発生する業務である.自治体法務における例としては,例規を制定・改正する際の参考資料作成,
さらには自治体合併時に全例規を擦り合せて一つに纏めるための準備作業が挙げられる.
特に自治体合併時には,対象となる全自治体の全例規に対する例規比較を短時日に
行う必要がある仕事量の多い法務となっている\cite{加藤幸嗣:2006-05,伊佐美浩一:2005-05,伊佐美浩一:2005-08,藤井真知子:2007-07-31}.
現在,この例規比較は専門家が手作業で実施しているため,計算機を利用した作業の省力化が望まれている.
そこで本研究では,条文対応表の作成支援を目的とし,
与えられた2つの例規の条文対応表を計算機で作成する手法の検討及び,
得られた条文の対応関係の尤もらしさについての評価を行う.

法を計算機で扱う研究は,
法律の専門家を模倣するエキスパートシステムに関する研究として,
人工知能研究の派生領域として発達してきた.
本分野初期の国際会議として,1987年より隔年開催されているInternational Conference on Artificial Intelligence and Law \cite{ICAIL}
と,1988年より毎年開催されているInternational Conference on Legal Knowledge and Information System \cite{JURIX}がある.
日本では
平成5年度から9年度の文部省化学研究費重点領域研究
「法律エキスパートシステムの開発研究」において促進された\cite{吉野一}.
この期間を通してインターネット上における法律の閲覧が可能となり,
特に判例を計算機で利用する知的システムに関する多数の研究が実施された.
法律や例規以外の法関係の文書に対する情報科学との融合研究としては,
特許における公開特許公報中の請求項と発明の詳細な説明文との
対応付けを行う研究が行われている\cite{ronbun2-4,ronbun2-2}.
また,法律用語のオントロジー構築に対する研究も行われた\cite{山口高平:1998-03-01}.
そして,日本においても2007年より人工知能学会全国大会の併設ワークショップとして
International Workshop on Juris-informatics (JURISIN:JURIS INformatics)が
毎年開催されている.
自治体の情報化を支援する企業も多数存在し,例規のインターネット上での公開支援にとどまらず,
例規改正の編集過程に基づき,改正前後の差異を表現した新旧対照表を自動作成する事も可能となっている\cite{kakuda}.
現在では,官報を基に法務省行政管理局が整備した法令データ提供システムが日本の法令を
提供している\cite{eGov}.
また,多くの法律の英対訳も名古屋大学の日本法令外国語訳データベースシステムを通じて提供されている\cite{JaLII}.
現在では法律だけでなく,多くの自治体が例規をインターネット上に公開するようになった.



しかし例規を対象とした情報科学との融合研究は少なく,これまでに例規を分類する研究\cite{原田隆史2009}が
存在するに留まっている.
そのため,例規の条文対応表の自動作成に関する研究は本論文が嚆矢である.


米国の連邦法と州法とで整合性の取れていない条文の発見を目的とした,
法律体系の中から関連する条文を網羅的に抽出する研究がある\cite{ronbun3-1}.
この研究は類似する条文を抽出する点で,例規の条文対応を推定する本研究と類似している.
しかしながら,彼らの研究は米国における領域知識の利用を前提としている事及び,
不整合性検出のために数値や単位に特化した処理を追加している点で
日本の例規を対象とした条文対応表への適用は困難である.
条文対応表は,条文を一般の文書と見なした場合,類似文書を探す研究と見なしうる.
類似文書の探索に関する研究としては,
英語で記された複数のコーパス間の類似する文を抜き出す研究や\cite{ronbun1-1}や,
コーパス内に存在する類似文のクラスタを抜きだす研究がある\cite{ronbun1-3}.
また日本語を対象とした研究も挙げられる\cite{ronbun2-1,ronbun2-3}.
これらの論文では同一事象に対して記述された記事の抽出及び,
記事の要約をその目的としている.
これらはよく整備されたコーパスや類義語辞典を用いたり,豊富に収集された事例に基づく
機械学習によりその性能向上を図っている.
そのため研究事例のない例規を対象とした本研究に直接利用する事は困難である.


\section{モデル化と問題定義}
\label{sec:definition}

\subsection{例規の構造}

本研究で対象とする例規とは,法特有の階層構造を有する文章である.
典型的な法では,例規名を表す「表題」,効力を発する日を記した「発令」,公布を宣言する「公布文」,
例規の内容を記した「本則」,そして「制定附則」及び「改正附則」が第一番目の階層を構成する.
このうち本則は,「章」「節」/「条」「項」「号」の階層を有している.
図\ref{reikiStructure}に例規に共通する主要な階層構造を示す.
ただし,実際の例規では章が存在しない場合も多く,特に制定時期が古い例規ではこの階層構造に従わない
場合もある事を付記する.
章と一部の条にはその内容を記載した見出しがついている.
図\ref{reikiExample}に愛媛県青少年保護条例から抜粋した本則の一部を記す.
図\ref{reikiExample}において章と条の右横に括弧で記載した文字列が章見出しおよび条見出しである.

\begin{figure}[t]
\noindent\begin{minipage}[b]{171.5pt}
\begin{center}
\includegraphics{19-3ia947f1.eps}
\end{center}
\caption{例規の主要な共通階層構造}
\label{reikiStructure}
\end{minipage}
\begin{minipage}[b]{248.5pt}
\begin{center}
\includegraphics{19-3ia947f2.eps}
\caption{本則の階層構造の例(抜粋)}
\label{reikiExample}
\end{center}
\end{minipage}
\end{figure}


\subsection{条文対応表}\label{sec:taiouhyou}

例規比較を行う際には,対応する条文の関係を記した比較対照表が作成される.この表を
条文対応表と呼ぶ.
表\ref{reikiHikakuTable}に,愛媛県青少年保護条例と香川県青少年保護育成条例の
条文対応表の典型例を記す.
表では,例規構造より表題,発令,公布文がまず並ぶ.
その後,二つの条例において対応する条が左右に並ぶという形をとっている.
条番号の一行上に書かれている括弧書きの文字列は条見出しである.
表より以下のような事がわかる.
\begin{itemize}
\item 愛媛県の第1条が香川県の第1条に,愛媛県の1条が香川県の4条に
対応している.
\item 香川県の第3条に対応する愛媛県の条文が存在しない.
\item 香川県の第15条は,愛媛県の第12条と第13条の2つに対応している.
\end{itemize}
この表を利用する事により「青少年」の定義や,夜間の時間帯といった二つの条例の違いを
網羅的に比較する事が行われている.

\begin{table}[p]
\caption{愛媛県と香川県の青少年保護に関する条例の条文比較表(一部抜粋)}
\label{reikiHikakuTable}
\input{04table01.txt}
\end{table}

条文比較表において対応する条文は類似する事が多い.これを示すため,
表中で対応する両県の第一条の共通部分に下線を引いた.

\begin{quote}
愛媛県:\underline{この条例は、青少年の}健全な育成を\underline{阻害する恐れのある行為}
から青少年を保護し、
もって青少年の\underline{健全な育成をはかる事を目的とする。}

香川県:\underline{この条例は、青少年の}福祉を\underline{阻害するおそれのある行為}を禁止し、
その\underline{健全な保護育成を図る事を目的とする。}
\end{quote}

この場合,両方の条が大変よく一致している事がわかる.
次に対応する条の文字数に差がある包含関係にある例として,愛媛県の第13条の3と第10条の3を挙げる.
\begin{quote}
愛媛県:\underline{何人も、青少年に}対し、ツーショットダイヤル等
\underline{利用カード}(ツーショットダイヤル
等営業に関して提供する役務の数量に応ずる対価 を得ることを目的として発行する文書その他の物 品をいう。以下同じ。)を\underline{販売}し、配布し、贈与し、 又は貸し付けては\underline{ならない。}

香川県:\underline{何人も、青少年に利用カード}の\underline{販売}等をしては
\underline{ならない。}
\end{quote}

この例は「利用カード」の定義が条内で行われているか否により共通部分に偏りが出ている事を示している.
そのため,香川県側は24文字中18文字 (75\%) が一致しているが,愛媛県側は110文字中18文字 (16\%) が一致しているにすぎない.
ただし,共通部分が多い場合でも必ずしも対応する条であるとは限らない.
例として愛媛県の第5条の8と香川県の第10条の2を示す.

\begin{quote}
愛媛県:自動販売機等業者は、次に掲げる施設の敷地の周囲から200メートル以内の区域に、(中略)設置しないように努めなければならない。
\\(1) \underline{学校教育法(\mbox{昭和22年法律第26号})\mbox{第1条}に規定する学校(大学を除く。)}
\\(2) 児童福祉法(昭和22年法律第164号)第7条第1項に規定する児童福祉施設
\\(3) \underline{図書館法\mbox{(昭和25年法律第118号)第2条第1項}に規定する図書館}
\\(後略)

香川県:卑わいな姿態等を被写体とした写真又は描写した絵を掲載した広告文書等は(中略)「有害広告文書等」(中略)とする。
\\2 何人も、次に掲げる行為をしてはならない。
(中略)
\\(3) 次に掲げる施設の敷地内において有害広告文書等の配布をすること。
\\ア \underline{学校教育法\mbox{(昭和22年法律第26号)第1条}に規定する学校(大学を除く。)}
\\イ \underline{図書館法\mbox{(昭和25年法律第118号)第2条第1項}に規定する図書館}
\\(後略)
\end{quote}
このように,共通部分が多い場合でも必ずしも対応するとは限らない事がわかる.







\subsection{条文対応表のモデル化と推定問題}
\label{sec:jidouseisei}

本研究で目的とする計算機支援を達成するためには,与えられた2条例より条文比較表を自動生成する必要がある.
そのためには,計算機で解決可能な形に問題をモデル化する必要がある.
一般に条文対応表は,その名の通り表として表現されているが,
条の対応関係は多対多であり,また対応する条がない場合も存在する.
当然にして同一例規内の条の間には対応関係は存在しない.
また,例規の主要構造のうち対応関係を決める必要があるのは
本則に属する条のみであり,制定附則や改正附則に属する条の対応関係を決める必要はない.
そこで,我々は条文比較表より本則の部分に着目し,
各例規の本則に属する条を頂点とする2部グラフとしてモデル化する.
2部グラフを構成する2つの頂点集合は例規ごとに構成され,
対応する条の間に辺が引かれる.

ここで2つの例規A, Bの条文対応表のモデルを以下のように定義する.
\begin{description}
\item 二部グラフ$G=(V_A,V_B,E)$, ただし,
\begin{itemize}
\item 頂点$v_a \in V_A$がAの本則に属する条に,$v_b\in V_B$がBの本則に属する条に対応し,
\item 辺$e=(v1\in V_A,v2\in V_B)$は,$v1$と$v2$が条文比較表において対応する事を表す.
\end{itemize}
\end{description}
図\ref{taiouhyou}に,愛媛県と香川県の青少年保護に関する条例の条文対応表を示す.
図中左側の四角が愛媛県側の,右側の四角が香川県側の各条を表す頂点である.
左右の頂点間の辺が,両県の条例における条の対応関係を表現している.

\begin{figure}[t]
\begin{center}
\includegraphics{19-3ia947f3.eps}
\end{center}
\caption{二部グラフによる条文対応表のモデル}
\label{taiouhyou}
\end{figure}

条文対応表の推定問題とは,2つの例規$A,\ B$を入力とし,
法学者が作成する条文対応表と一致または類似した
二部グラフ$G=(V_A,V_B,E)$を出力する問題と定義する.



\subsection{条文対応表生成アルゴリズム}
\label{algorithm}

条文対応表の推定問題は,
入力として与えられる二例規の各条間の類似度を計算し,
その類似度に基づき条文が対応するか否かを判定
する事で解く事ができる.
本研究では,類似度の定義,すなわち類似尺度が与えられたとき以下のアルゴリズムによって
条文の対応関係を推定する方法を提案する.

\begin{description}
\item[入力] 2つの例規A, B
\item[出力] 二部グラフ $G(V_A,V_B,E)$
\item[step1] 2つの例規の本則に属する条に対応する頂点集合$V_A,V_B$を生成する.
\item[step2] 2つの頂点$v_a\in V_A,v_b\in V_B$間の辺重みを条$v_a,v_b$間の類似度とする
重み付き完全二部グラフ $G_p=\{V_A,V_B,E_p\}$を生成する.
\item[step3] $V_A$に属する各頂点$v_a$を端点とする辺集合$E_{v_a}=\{(v_a,v_b| v_b \in V_B)\}$のうち,
最も重みの大きい辺を二部グラフ$G$の辺集合$E$に加える.
\item[step4] $V_B$に属する各頂点$v_b$を端点とする辺集合$E_{v_b}=\{(v_a,v_b| v_a \in V_A)\}$のうち,
最も重みの大きい辺を二部グラフ$G$の辺集合$E$に加える.
\item[step5] 二部グラフ$G$を出力する.
\end{description}

条文対応表では,図\ref{taiouhyou}に示したように一つの条に対して複数の条が対応する事例がある.
提案アルゴリズムでは両方の例規からではなく,Step3 と 4において一方の例規の条からみて
類似度の最も高い条に対応関係があると判定した辺をグラフに追加している.
Step3とStep4は独立して辺,すなわち1つの条と対応関係にある条を選択しているため,
一つの頂点が複数の辺の端点となりうる.
結果的にStep5において多対多の二部グラフ,すなわち多対多の関係を含む条文対応表が出力される.



\section{条間の類似尺度}
\label{sec:method}

前節のアルゴリズム内では条間の類似度を定義していない.
この類似度は,法学者が二条間に対応関係が存在する場合に大きい値を,
対応関係が存在しない場合に小さい値をとる事が求められる.
しかしながら,法学者の暗黙知を適切に表現する類似尺度が明らかとなっていないため,
本研究では既存の3種類の文書比較,文字列比較法に基づく類似尺度を用い,
それらの比較評価を行った.



\subsection{ベクトル空間モデル}

与えられた文章を,単語の出現頻度を表現したベクトルとしてモデル化する方法を
ベクトル空間モデルと呼ぶ\cite{salton1975vector}.
2つの文章に対応するベクトル間の距離を計算する事により,文章間の関連度を求める方法であり,
情報抽出や情報フィルタリング等に用いられる.
距離尺度としては,コサイン,内積,マンハッタン距離やユークリッド距離等が用いられる.

要素数$n$個の単語集合$W=\{w_1,w2_,\cdots,w_n\}$が与えられたとき,
ベクトル空間モデルによりある文章$T$は長さ$n$のベクトル$V_T = (v_1,v_2,\cdots,v_n)$ で
表現される.ここで,$v_i$は,文章$T$中における単語$w_i$の出現回数である.
このベクトルはしばしばtf-idf (Term Frequenc, Inverse Document Frequency) に基づく重みが
加味される.
この重みにより,多くの文章に出現する単語の重要度を下げ,特定の文章にしか出現しない単語の重要度を上げる
事が可能となる.

本研究では,
条文対応表作成問題における各条文をこのベクトル空間モデルでベクトル化し,
2つのベクトルの距離によって対応関係の強さを数値化する.
なお,距離尺度としてコサインを用い,
利用する単語を頻出順に10,50,100個選んだもの,及び全例規に出現する全単語の4種類を比較した.
ベクトルの重み付けについては,定数重み及びtf-idf重みの2種類を比較した.
また,利用する単語の品詞として,全品詞の利用,全品詞の原形を利用,
名詞のみ利用,名詞,副詞,形容詞,動詞,連体詞の5種類を利用,の4種類を適用した.

例規の条文はその長さ,すなわち単語数や文字数の分散値が大きい.
これが本手法で計算される条文間の類似度は条文の長さの違いが影響を及ぼす可能性がある.
そこで類似度を2つの条文のうち短い方の条文の文字数で割る事で正規化した値も類似尺度とする.
以降,正規化前を「絶対スコア」,正規化後を「相対スコア」と呼ぶこととする.
以上によりベクトル空間モデルに基づく類似尺度数は,
$(\text{ベクトル長}) \times (\text{対象品詞}) \times (\text{重み}) \times (\text{絶対}|\text{相対スコア}) = 4 \times 4 \times 2 \times 2 = 64$ 個となる.



\subsection{最長共通部分列}
\label{subsec:LCS}

最長共通部分列(Longest Common Subsequence)とは,
入力として与えられた2つの文字列における最長の共通部分文字列をいう\cite{Maier:1978:CPS:322063.322075}.
共通部分文字列とは,もとの文字列から文字を出現順序をかえずに取り出したものとなる.
今,二本の文字列 $X=(\text{アイウエオ})$, $Y=(\text{アイクエオ})$ が与えられたとする.
このとき最長共通部分列は,(アイエオ)となり,その長さは4である.

最長共通部分列は動的計画法により計算する事が可能である.
入力として2つの文字列$X=(x_1x_2\cdots x_n)$と
$Y=(y_1y_2\cdots y_m)$が与えられたとき,最長共通部分列長を求めるアルゴリズムは以下の通りである.

\begin{description}
\item[Step1] $n+1$行,$m+1$列の行列$M$を準備する
\item[Step2] 行列$M$の値を以下の漸化式によって計算していく
\begin{equation}
M(i,j) = \begin{cases}
0 & ( i=0\text{または} j=0 )\\
M(i-1,j-1) +1  & ( i,j>0\text{かつ} x_i=y_j )  \\
\text{max}(M(i-1,j), M(i,j-1)) & ( i,j>0 \text{かつ} x_i \neq y_j ) \nonumber
\end{cases}
\end{equation}
\item[Step3]行列 $M$ の要素より最大値を出力する
\end{description}

本手法による条文の対応関係の強さは,最長共通部分列の長さとして定義する.
比較する単位としては,文字単位と単語単位の比較を行った.
文字単位では,ひらがなカタカナを含む全ての文字を対象とした場合と漢字のみを対象とした場合を,
単語単位では,全品詞,全品詞の原形,名詞のみ,名詞,副詞,形容詞,動詞,連体詞の5種類の4つの場合の
比較を行った.
また,本手法の性能比較では,入力として条見出しを用いた場合の評価も行った.
以上により最長共通部分列に基づく類似尺度数は,
定数重みを用いたものが $(\text{条題}|\text{条文}) \times (\text{対象文字}) \times  (\text{絶対}|\text{相対スコア}) = 2\times 2\times 2= 8$ 個,
tf-idf 重みを用いたものが $(\text{対象品詞}) \times (\text{絶対}|\text{相対スコア}) = 4 \times 2= 8$ 個の
合計16個となる.



\subsection{文字列アライメント}

文字列アライメントは,入力として与えられた文字列に存在する類似した領域を特定できるよう,
文字列を整列させる事をいう.
この整列に必要とする文字・単語の挿入や削除,置換コストの合計値によって文字列間の類似度が定義される.
挿入・置換のコストを1,置換コストを2とした場合,レーベンシュタイン距離あるいは編集距離と呼ばれる
類似尺度になる.
挿入等のコストの異なる例としては,生物情報学におけるアミノ酸配列(タンパク質)に適用される手法がある\cite{smithwaterman,needleman}.この手法におけるコストは生物の進化において変化が発生する確率に基づいて
決定されている.

文字列アライメントは,文字列を整列させるため,一致しない事を表す文字として「—」を用いる.
今,二本の文字列 $X=(\text{アイウエオ})$, $Y=(\text{アイクエオ})$ が与えられたとすると,
(アイ—エオ)がアライメントである.
アライメントは,例えば一致した文字に2点を加点し,一致しない場合(「—」の場合)に2点を減点する,といった
基準を設定する事により,類似度の数値が行われる.
上記のアライメントは6点となる.
与えられた基準において最も高い類似度を持つアライメントの計算には動的計画法が用いられる.
入力として2つの文字列$X=(x_1x_2\cdots x_n)$と
$Y=(y_1y_2\cdots y_m)$が与えられたとき,最も高い類似度を求めるアルゴリズムは以下の通りである.

\begin{description}
\item[Step1] $n+1$行,$m+1$列の行列$M$を準備する
\item[Step2] $0$行及び,$0$列の値を$0$にする

\item[Step3] 行列$M$の値を以下の漸化式によって計算していく
\[
M(i,j) = \text{max}
\begin{pmatrix}
0, &\\
M(i-1,j-1) +s(x_i,y_j),  &  \\
M(i-1,j)-g, &\\
M(i,j-1)-g& 
\end{pmatrix}
\]

\item[Step4] 行列Mの要素より最大値を出力する
\end{description}

ここで関数$s(x,y)$は,文字$x$と$y$をアライメントさせた場合の点数であり,上記の例の場合$2点$である.
また,定数$g$は,一致しない場合,すなわち「—」の時の点数であり,上記の例では$-2$点である.


一般に最長共通部分列や文字列アライメントでは,共通部分文字列の順序は保存される.
しかし例規比較において対応させるべき条文は,必ずしも順序が保存されているとはいいがたい.
例として,愛媛県と香川県の青少年の保護に関する条例を挙げる.
愛媛県の5条の2及び香川県の8条の2はともに,有害ながん具類等の販売等の制限や禁止について規定している.
愛媛県では,青少年に対する有害がん具の所持制限,有害がん具の定義の順で記述しているのに対し,
香川県では,有害がん具の定義,所持制限の順で記述されている.
表\ref{alignOrderReal}に該当部分の抜粋を記載する.
そこで,本研究では,アライメントアルゴリズムを再帰的に適用する事により
順序関係が保存されていない条文へ対応した手法も用いた.
再帰的な適用法は,以下の通りである.

\begin{description}
\item[入力]長さ$l$の文字列Aと長さ$m$の文字列B
\item[Step1] 文字列にアライメントを行う.その結果,文字列Aの$l_s$〜$l_e$までの部分文字列と
文字列Bの$m_s$〜$m_e$までの文字列のアライメントが得られたとする.
\item[Step2] 文字列Aと文字列Bの整列していない部分文字列の組合せ4種類に対してそれぞれアライメントを行う.
4種類の組合せは以下の通りである.
\begin{itemize}
\item[(a)] 文字列Aの$1$〜$l_s-1$文字と文字列Bの$1$〜$m_s-1$文字
\item[(b)] 文字列Aの$1$〜$l_s-1$文字と文字列Bの$m_e+1$〜$m$文字
\item[(c)]文字列Aの$l_e+1$〜$l$文字と文字列Bの$1$〜$m_s-1$文字
\item[(d)]文字列Aの$l_e+1$〜$l$文字と文字列Bの$m_e+1$〜$m$文字
\end{itemize}
\item[Step3]  対角線に位置する(a)と(d)の類似度の和と(b)と(c)の類似度の和のうち大きい方のアライメント結果と
Step1でえられたアライメント結果を出力する.
\end{description}


\begin{table}[b]
\caption{青少年保護に関する条例における記述順序が異なっている箇所}
\label{alignOrderReal}
\input{04table02.txt}
\end{table}


文字列アライメントによる条文の対応関係の強さは,再帰的に得られた各アライメントの類似度の値の和とした.
アライメントの単位としては,\ref{subsec:LCS}節に記した最長共通部分列と同様に文字単位と単語単位の比較を行った.
関数$s(x,y)$の値としては,文字単位の場合は漢字が一致した場合に2点,漢字以外が一致した場合1点とし,
単語単位の場合は各単語のtf-idfスコアを用いた.
以上により文字列アライメントに基づく類似尺度数は,
定数重みを用いたものが $(\text{順序の保存関係}) \times (\text{対象文字}) \times (\text{絶対}|\text{相対スコア}) = 2 \times 2 \times 2 = 8$ 個,
tf-idf重みを用いたものが $(\text{対象品詞}) \times (\text{絶対}|\text{相対スコア}) = 4 \times 2 =8$ 個の
合計16個となる.


\section{評価実験}
\label{sec:solution}

\subsection{実験条件と評価項目}
\label{sec:experiments}

\begin{table}[b]
\caption{条文対応表を作成した例規の組合せ一覧}
\label{reikiIchiran}
\input{04table03.txt}
\end{table}
\begin{table}[b]
\caption{条文対応表の作成時間(分)}
\label{reikiTime}
\input{04table04.txt}
\end{table}

法学者の暗黙知を適切に表現し,条文対応表の作成に適した類似尺度を明らかにするため,
愛媛県と香川県の22条例を対象とした性能評価を行った.
対象とした条例を表\ref{reikiIchiran}に記す.
また,若尾の監督下において3名の法学部生を被験者とした条文対応表の作成に要した時間を
表\ref{reikiTime}に記す.なお被験者が条文対応表を作成するにあたり
青少年保護に関する条例を対象とした講義を行ったため,
表\ref{reikiTime}において条例対応ID3の青少年保護に関する条例の
作成時間が欠損している.
ベクトル空間モデルにおける距離にはコサインを用い,条文の品詞分解にはMecab \cite{mecab}を用いた.

三種類の手法で計算される条文間の類似度は条文の文字数に依存する.
そこで類似度を短い方の条文の文字数で割る事で正規化した値も類似尺度として比較した.
自動生成した条文対応表の評価を行うため,若尾が作成した条文対応表を正解例として正解率の比較を行った.
ここで正解率とは,正解となる条文対応表を表現する二部グラフにおける辺の数を母数とし,
各類似尺度を用いて得られる二部グラフと一致する辺数を母数で割った値である.



\subsection{類似尺度の正解率}
\label{sec:results}

表\ref{res:VS}, \ref{res:LCS}, \ref{res:alignment}に
最長共通部分列,アライメント,ベクトル空間モデルに基づく手法の
正解率を示す.
表中の「五詞」は名詞,副詞,形容詞,動詞,連体詞を表している.
正解率が上位5位の手法にはその順位を併記した.

\begin{table}[b]
\caption{ベクトル空間モデルの正解率}\label{res:VS}
\input{04table05.txt}
\end{table}

\begin{table}[t]
\caption{最長共通部分列の正解率}\label{res:LCS}
\input{04table06.txt}
\end{table}
\begin{table}[t]
\caption{文字列アライメントの正解率}
\label{res:alignment}
\input{04table07.txt}
\end{table}

各アルゴリズムにおいて最も高い正解率は,
ベクトル空間モデルが85\%,最長共通部分列が75\%,文字列アライメントが81\%となっている.
全ての場合において絶対スコアが相対スコアの正解率を上回っている事,
定数重みがtf-idf重みの正解率を上回っている事がわかる.

ベクトル空間モデルでは,ベクトル長が長いほど正解率が高くなっており,
用いる品詞の影響はさほど大きくない.
最長共通部分列では,条見出しのみを用いた場合でも最高で71\%の正解率となっている.
全ての条には条見出しが存在しない事を考慮すると大変高い正解率であると考えられる.
最長共通部分列と文字列アライメントでは,単語単位よりも文字単位の方が正解率が高かった.
また,文字列アライメントにおける条文内容の記述順序を考慮する事による正解率の向上はみられなかった.





\subsection{受信者操作特性曲線}
\label{sec:roc}

前節の結果が示す通り,本手法で得られた条文の対応関係の正解率は100\%でなく,
必ずしも正しいとは限らない.すなわち,この結果を用いて条文対応を作成する法務関係者は,条文の対応関係が正しいか否かを
判断する必要がある.
そのため,本手法で得られる対応関係の信頼度,すなわち得られた対応関係が正解である尤もらしさを
提示する事が望ましい.
そこで,対応関係を得るために用いた類似尺度を信頼度として利用した場合の評価を
受信者操作特性曲線(ROC曲線:Receiver Operating Characteristic curve)を用いて行った.
受信者操作特性曲線とは,正解と判定する類似度の閾値を変化させた場合の敏感度と偽陽性率の変化を表現したものである.
敏感度とは正解を正しく正解として捕捉する率であり,偽陽性率とは,不正解を誤って正解と判定する率である.
なお,提案手法によって得られた条文の対応関係の類似度が閾値よりも大きい場合に正解と判定する.

ベクトル空間モデル及び文字列アライメントにおいて最も正解率の高い類似尺度及び,
条見出しを対象とする最長共通部分列において最も正解率の高い類似尺度の
受信者操作特性曲線(ROC曲線)を図\ref{roc_score}に示す.
図の縦軸は敏感度を,横軸は偽陽性率を表す.また,各類似尺度の受信者操作特性曲線下面積を表\ref{AUC}に記す.
受信者操作特性曲線では,グラフの形状が敏感度1,偽陽性率0である左上点に近い凸形状を示し,
曲線下面積が1に近づくほど良い指標である事を示す.
図及び表より,文字列アライメントに基づく類似尺度の曲線下面積は0.5を下回っており,信頼度
として利用ができない事を示している.
また,ベクトル空間モデル及び最長共通部分文字列についても0.5を少し上回っている程度であり,
信頼度として利用するには低いものとなっている.

\begin{figure}[t]
\begin{minipage}[b]{.5\textwidth}
\begin{center}
\includegraphics{19-3ia947f4.eps}
\end{center}
\caption{類似度を指標としたROC曲線}\label{roc_score}
\end{minipage}
\begin{minipage}[b]{.5\textwidth}
\begin{center}
\includegraphics{19-3ia947f5.eps}
\end{center}
\caption{二位との比率を用いた場合のROC曲線}\label{roc_ratio}
\end{minipage}
\end{figure}

\begin{table}[t]
\caption{受信者操作特性曲線下面積(AUC: Area Under Curve)}
\label{AUC}
\input{04table08.txt}
\end{table}

上の結果は,条文や条文の見出しの長さが異なる事が原因であると考えられる.
そこで,
\ref{algorithm}節に示すアルゴリズムにおいて対応条を決定する{\bf step3}および{\bf step4}において,
頂点$v_a$または$v_b$と接続する辺集合ごとにその類似度を正規化する必要があると考える.
そこで,信頼度を表す新たな尺度として{\bf step3}および{\bf step4}で選択される辺の類似度を
二番目に大きい類似度の値で割った値(以降,二位との比率と呼ぶ)の評価を行う.
この信頼度に対する受信者操作特性曲線を図\ref{roc_ratio}に,曲線下面積を表\ref{AUC}に示す.
図\ref{roc_score}と図\ref{roc_ratio}を比較すると,
二位との比率を用いる事により
信頼度を表す評価尺度の性能が大幅に向上していることがわかる.
また表\ref{AUC}に示す曲線下面積は最長共通部分文字列が0.80と
最大の値となっており,ベクトル空間モデルが次いで0.79となっており,
二位との比率の優位性を示している.



\subsection{結果の考察}

\ref{sec:results}節の結果より,
本研究の条文対応表作成では
全単語に基づくベクトル空間モデルを用いたtf-idf重みを使わない類似尺度が最も有効である事が
わかった.
文字列アライメントは公開特許公報における請求項と「発明の詳細な説明」との対応付けのような,
2つの文章内で言及される事柄の出現順序が同じ場合には有効であるが\cite{ronbun2-4},
事柄や単語の出現順に対応できないために一般の文書での利用は不適切であると言われている\cite{ronbun1-1}.
本研究においてベクトル空間モデルが文字列アライメントの結果よりも良かったのは,
条文間で言及される事柄の語順が,公開特許公報ほどには保存されておらず,
一般の文書に近かったためであると考えている.

次に,ベクトル空間モデルの軸を構成する単語数が多い方が,そしてtf-idf重みを
用いない類似尺度の方が,推定精度が高かった理由を考察する.
例規で用いられる単語を調べると,県名や地名等の固有名詞の出現頻度が高く,
県名,地名等に加え,甲乙,委員,委員会,規定といった単語のtf-idf値が高かった.
本研究におけるベクトル空間モデルの軸は単語の出現頻度に基づいて選択した.
そのため,ベクトル空間モデルの軸を構成する単語数が少ない場合,
都道府県名,たとえば「愛媛」のように都道府県固有の単語が占める率が高くなる.
都道府県固有の単語は,すなわち他の都道府県には出現しない単語であるため,
軸を構成する単語としては不適切である.
これがベクトル空間を構成する単語数が多い方が推定精度のよかった一因だと推察する.
また,都道府県固有の単語に加え,甲乙や委員会といった一般的とは言えないものの,
法律用語としてはありふれている単語のtf-idf値が高い事が,
対応する条文を特定する精度を低下させたと考えている.


本論文で掲げた全ての手法はいずれも100\%の正解率を得る事はできなかった.
また,今後の研究により正解率の向上は期待できるものの
法学者の暗黙知を表す数式が明らかとなり正解率が100\%となる事は期待しがたい.
そのため実用上は,計算機で作成した条文対応表を基づき専門家が最終的な
条文対応表を作成する,という計算機支援システムの形にならざるを得ない.
この場合計算機が提示する解の信頼度や尤もらしさを提示できる事が望ましい.
そこで受信者操作特性曲線を用いた評価を\ref{sec:roc}節で行った.
その結果,対応する条を決定するのに用いた類似度そのものではなく,
二位との比率を信頼度を評価する尺度として用いる事により
最長共通部分文字列及びベクトル空間モデルにおいて
受信者操作特性曲線下面積が約0.8と高い値を示す事がわかった.
以上の結果により,条文対応表の生成及び作成支援のためには
条見出しに対して最長共通部分文字列を,
条文に対してベクトル空間モデルを適用して得られる結果を併用することがよい事がわかった.

実務で利用される条文対応表では,条文の対応関係だけでなく,対応する2つの条文の
差異が明示されている.
そのため条文対応表の作成支援の今後としては,
条文の対応関係を明らかにするだけではなく,
対応する2つの条文の差異を明確化する事が求められると考えている.
この目的を達成する方法としてベクトル空間モデルは適切ではない.
なぜならベクトル空間モデルは対応する条文を決定するのに留まり,
専門家が最終的な条文対応表を作成するための手がかりとなる情報を提示する事はできないからである.
一方,文字列アライメントを用いた場合には一致する文字列群とその出現順序,
そして一致する文字列に挟まれた不一致の文字列といった情報を提示する事が可能である.
そのため,文字列アライメントの正解率はベクトル空間モデルよりも低いが,
条文対応表を作成するために条文の差異を提示する支援システムとしての利用価値は高いと考えている.
支援システムとして考えるならば,
ベクトル空間モデルと最長共通部分文字列により条文の対応関係をその信頼度と共に提示し,
文字列アライメントにより条文の差異を示すという形が望ましいと考えている.



\section{まとめ}

地方自治体の法である例規を比較する条文対応表の作成支援のための
枠組みを提案した.条文対応表を二部グラフとして表現することで条文対応表の自動生成問題を定義した.
情報科学的手法を適用するため,法学者の暗黙知に類似した類似尺度を探すため,
3つの計算手法にもとづく96個の類似尺度の評価を行った.
愛媛県と香川県の22条例を対象として行った比較により,
ベクトル空間モデルに基づく手法が最も高い正解率である事を明らかにした.
また提案手法で推定した条文の対応関係の信頼度を示す尺度としては,
最も高い類似度を二番目に高い類似度で割った値を
利用する事で高い操作特性曲線下面積が得られる事を明らかにした.
特に条見出しを対象に最長共通部分文字列を適用した結果が良い事を示した.




\acknowledgment
本研究の一部は科研費 JSPS (21500253) の助成を受けたものである.

\bibliographystyle{jnlpbbl_1.5}
\begin{thebibliography}{}

\bibitem[\protect\BCAY{Barzilay \BBA\ Elhadad}{Barzilay \BBA\
  Elhadad}{2003}]{ronbun1-1}
Barzilay, R.\BBACOMMA\ \BBA\ Elhadad, N. \BBOP 2003\BBCP.
\newblock \BBOQ Sentence alignment for monolingual comparable corpora.\BBCQ\
\newblock In {\Bem Proceedings of the 2003 conference on Empirical methods in
  natural language processing}, \mbox{\BPGS\ 25--32}, Morristown, NJ, USA.
  Association for Computational Linguistics.

\bibitem[\protect\BCAY{藤井}{藤井}{2007}]{藤井真知子:2007-07-31}
藤井真知子 \BBOP 2007\BBCP.
\newblock 市町村合併における自治体法務の現状と課題 :
  甲賀市の条例整備を手がかりとして.\
\newblock \Jem{龍谷大学大学院法学研究}, {\Bbf 9}, \mbox{\BPGS\ 181--214}.

\bibitem[\protect\BCAY{原田\JBA 青木\JBA 真島}{原田 \Jetal
  }{2009}]{原田隆史2009}
原田隆史\JBA 青木淳一\JBA 真島由里香 \BBOP 2009\BBCP.
\newblock クラスタリング手法に基づく条例の自動分類.\
\newblock \Jem{情報ネットワーク法学会第9回研究大会予稿集}, \mbox{\BPGS\
  65--68}.

\bibitem[\protect\BCAY{Hatzivassiloglou, Klavans, Holcombe, Barzilay, yen Kan,
  \BBA\ McKeown}{Hatzivassiloglou et~al.}{2001}]{ronbun1-3}
Hatzivassiloglou, V., Klavans, J.~L., Holcombe, M.~L., Barzilay, R., yen Kan,
  M., \BBA\ McKeown, K.~R. \BBOP 2001\BBCP.
\newblock \BBOQ SIMFINDER: A Flexible Clustering Tool for Summarization.\BBCQ\
\newblock In {\Bem In Proceedings of the NAACL Workshop on Automatic
  Summarization}, \mbox{\BPGS\ 41--49}.

\bibitem[\protect\BCAY{平尾\JBA 鈴木\JBA 磯崎\JBA 前田}{平尾 \Jetal }{2005 10
  15}]{ronbun2-3}
平尾努\JBA 鈴木潤\JBA 磯崎秀樹\JBA 前田英作 \BBOP 2005-10-15\BBCP.
\newblock 単一言語コーパスにおける文の自動対応付け手法(自然言語).\
\newblock \Jem{情報処理学会論文誌}, {\Bbf 46}  (10), \mbox{\BPGS\ 2533--2545}.

\bibitem[\protect\BCAY{法情報研究センター}{法情報研究センター}{}]{JaLII}
法情報研究センター.
\newblock 日本法令外国語訳データベースシステム.\
\newblock \Turl{http://japaneselawtranslation.\linebreak[2]go.jp}.

\bibitem[\protect\BCAY{ICAIL}{ICAIL}{}]{ICAIL}
ICAIL.
\newblock \BBOQ International Conference on Artificial Intelligence and Law
  (ICAIL).\BBCQ\
\newblock \Turl{http://www.iaail.org/}.

\bibitem[\protect\BCAY{伊佐美}{伊佐美}{2005a}]{伊佐美浩一:2005-05}
伊佐美浩一 \BBOP 2005a\BBCP.
\newblock 市町村合併調整のポイント(1)合併に関する法的問題(1)条例・規則の調整
  西東京市.\
\newblock \Jem{自治体法務研究}, {\Bbf 1}, \mbox{\BPGS\ 108--114}.

\bibitem[\protect\BCAY{伊佐美}{伊佐美}{2005b}]{伊佐美浩一:2005-08}
伊佐美浩一 \BBOP 2005b\BBCP.
\newblock
  市町村合併調整のポイント(2)合併に関する法的問題(2)合併関連法令の問題点
  西東京市.\
\newblock \Jem{自治体法務研究}, {\Bbf 2}, \mbox{\BPGS\ 108--113}.

\bibitem[\protect\BCAY{JURIX}{JURIX}{}]{JURIX}
JURIX.
\newblock \BBOQ International Conference on Legal Knowledge and Information
  Systems (JURIX).\BBCQ\
\newblock \Turl{http://www.jurix.nl/}.

\bibitem[\protect\BCAY{角田}{角田}{2010}]{kakuda}
角田篤泰 \BBOP 2010\BBCP.
\newblock
  \Jem{ソフトウェア工学との類似性に着目した立法支援方法(三){\kern-0.5zw}},
  237\JVOL, 第二節 \JCH, \mbox{\BPGS\ 191--252}.
\newblock 名古屋大學法學部.

\bibitem[\protect\BCAY{加藤}{加藤}{2006}]{加藤幸嗣:2006-05}
加藤幸嗣 \BBOP 2006\BBCP.
\newblock 比較分析 市町村合併と条例制定--福知山市の公の施設条例等を題材として
  (自治体情報 条例 制定の動向).\
\newblock \Jem{法令解説資料総覧}, {\Bbf 292}, \mbox{\BPGS\ 76--78}.

\bibitem[\protect\BCAY{Kudo, Yamamoto, \BBA\ Matsumoto}{Kudo
  et~al.}{2004}]{mecab}
Kudo, T., Yamamoto, K., \BBA\ Matsumoto, Y. \BBOP 2004\BBCP.
\newblock \BBOQ Applying conditional random fields to Japanese morphological
  analysis.\BBCQ\
\newblock In {\Bem Proc. of EMNLP}, \mbox{\BPGS\ 230--237}.

\bibitem[\protect\BCAY{Lau, Law, \BBA\ Wiederhold}{Lau
  et~al.}{2006}]{ronbun3-1}
Lau, G.~T., Law, K.~H., \BBA\ Wiederhold, G. \BBOP 2006\BBCP.
\newblock \BBOQ A relatedness analysis of government regulations using domain
  knowledge and structural organization.\BBCQ\
\newblock {\Bem Inf. Retr.}, {\Bbf 9}, \mbox{\BPGS\ 657--680}.

\bibitem[\protect\BCAY{Maier}{Maier}{1978}]{Maier:1978:CPS:322063.322075}
Maier, D. \BBOP 1978\BBCP.
\newblock \BBOQ The Complexity of Some Problems on Subsequences and
  Supersequences.\BBCQ\
\newblock {\Bem J. ACM}, {\Bbf 25}, \mbox{\BPGS\ 322--336}.

\bibitem[\protect\BCAY{丸川\JBA 岩山\JBA 奥村\JBA 新森}{丸川 \Jetal
  }{2002}]{ronbun2-4}
丸川雄三\JBA 岩山真\JBA 奥村学\JBA 新森昭宏 \BBOP 2002\BBCP.
\newblock ローカルアラインメントを用いたテキスト間の柔軟な対応付け.\
\newblock \Jem{情報処理学会研究報告. 情報学基礎研究会報告}, {\Bbf 2002}  (87),
  \mbox{\BPGS\ 23--28}.

\bibitem[\protect\BCAY{宮部\JBA 高村\JBA 奥村}{宮部 \Jetal }{2006}]{ronbun2-1}
宮部泰成\JBA 高村大也\JBA 奥村学 \BBOP 2006\BBCP.
\newblock 文書横断文間関係の特定.\
\newblock \Jem{言語処理学会第12回年次大会}, \mbox{\BPGS\ 496--499}.

\bibitem[\protect\BCAY{Needleman \BBA\ Wunsch}{Needleman \BBA\
  Wunsch}{1970}]{needleman}
Needleman, S.\BBACOMMA\ \BBA\ Wunsch, C. \BBOP 1970\BBCP.
\newblock \BBOQ A general method applicable to the search for similarities in
  the amino acid sequence of two proteins.\BBCQ\
\newblock {\Bem Journal of Molecular Biology}, {\Bbf 48}, \mbox{\BPGS\
  443--453}.

\bibitem[\protect\BCAY{新森\JBA 奥村}{新森\JBA 奥村}{2005}]{ronbun2-2}
新森昭宏\JBA 奥村学 \BBOP 2005\BBCP.
\newblock 特許請求項読解支援のための「発明の詳細な説明」 との自動対応付け.\
\newblock \Jem{自然言語処理}, {\Bbf 12}  (3), \mbox{\BPGS\ 111--128}.

\bibitem[\protect\BCAY{Salton, Wong, \BBA\ Yang}{Salton
  et~al.}{1975}]{salton1975vector}
Salton, G., Wong, A., \BBA\ Yang, C.-S. \BBOP 1975\BBCP.
\newblock \BBOQ A vector space model for automatic indexing.\BBCQ\
\newblock {\Bem Communications of the ACM}, {\Bbf 18}  (11), \mbox{\BPGS\
  613--620}.

\bibitem[\protect\BCAY{Smith \BBA\ Waterman}{Smith \BBA\
  Waterman}{1981}]{smithwaterman}
Smith, T.\BBACOMMA\ \BBA\ Waterman, M. \BBOP 1981\BBCP.
\newblock \BBOQ Identification of common molecular subsequences.\BBCQ\
\newblock {\Bem Journal of Molecular Biology}, {\Bbf 147}, \mbox{\BPGS\
  195--197}.

\bibitem[\protect\BCAY{総務省行政管理局}{総務省行政管理局}{}]{eGov}
総務省行政管理局.
\newblock 法令データ提供システム.\
\newblock \Turl{http://e-gov.go.jp/}.

\bibitem[\protect\BCAY{山口\JBA 槫松}{山口\JBA
  槫松}{1998}]{山口高平:1998-03-01}
山口高平\JBA 槫松理樹 \BBOP 1998\BBCP.
\newblock 法律オントロジー (小特集 法律と人工知能).\
\newblock \Jem{人工知能学会誌}, {\Bbf 13}  (2), \mbox{\BPGS\ 189--196}.

\bibitem[\protect\BCAY{吉野}{吉野}{2000}]{吉野一}
吉野一\JED\ \BBOP 2000\BBCP.
\newblock \Jem{法律人工知能 ー法的知識の解明と法的推論の実現}.
\newblock 創成社.

\end{thebibliography}


\begin{biography}

\bioauthor{竹中 要一}{
1995年大阪大学基礎工学部情報工学科飛び級中退.
1997年同大基礎工学研究科博士前期課程修了.
2000年同大基礎工学研究科博士後期課程修了.
同年同大基礎工学研究科助手.
2002年同大情報科学研究科助教授.
2007年同大情報科学研究科准教授.現在に至る.
博士(工学).
生物情報学,法情報科学の研究に従事.
}

\bioauthor{若尾 岳志}{
1996年早稲田大学法学部卒.
1999年早稲田大学大学院法学研究科修士課程修了.
2004年早稲田大学大学院法学研究科博士後期課程満期退学.
2004年大阪学院大学法学部専任講師.
2008年大阪学院大学法学部准教授.
2009年年獨協大学法学部総合政策学科准教授.現在に至る.
刑事法学の研究に従事.
}

\end{biography}


\biodate


\end{document}
