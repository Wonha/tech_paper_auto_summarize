



\documentstyle[epsf,jnlpbbl]{jnlp_j_b5}

\setcounter{page}{31}
\setcounter{巻数}{3}
\setcounter{号数}{4}
\setcounter{年}{1996}
\setcounter{月}{10}
\受付{1995}{5}{22}
\再受付{1996}{4}{15}
\採録{1996}{6}{14}

\setcounter{secnumdepth}{2}

\title{表層表現を手がかりとした日本語名詞句の指示性と数の推定}
\author{村田 真樹\affiref{KUEE} \and 黒橋 禎夫\affiref{KUEE} 
\and 長尾 真\affiref{KUEE}}

\headauthor{村田 真樹・黒橋 禎夫・長尾 真}
\headtitle{表層表現を手がかりとした日本語名詞句の指示性と数の推定}

\affilabel{KUEE}{京都大学工学部 電子通信工学教室}
{Department of Electronics and Communication, Kyoto University}

\jabstract{
日本語を英語に翻訳する時には,
日本語にはないが英語では必要な冠詞や数の問題に直面する.
この難しい問題を解決するために,
われわれは文章における名詞句の指示性と数をそれぞれ三種類に分類した.
指示性には総称名詞句,定名詞句,不定名詞句を設け,
数には単数,複数,不可算を設けた.
この論文では,名詞句の指示性と数が
その名詞句の現れる文中の言葉によりかなりの程度推定できることを示した.
その推定のための規則は
確信度付きのエキスパートシステムの書き換え規則に類する形で,
文法書などから得られる知識をもとに経験的に作成した.
この方法は,確信度を用いて推定するので,
指示性や数のような曖昧な問題には適した方法である.
規則を作るのに利用したテキストでの正解率は,
指示性で85.5\%,数で89.0\%であった.
規則を作るのに利用していないいくつかのテキストでの正解率は
平均して指示性で68.9\%,数で85.6\%という結果となった.
この指示性と数は冠詞の決定に利用されるのみならず,
照応処理,談話解析にも利用されていくと考えられる.
}

\jkeywords{表層表現,指示性,数,総称名詞句,定名詞句,不定名詞句}

\etitle{An Estimate of Referential Property and Number\\ 
of Japanese Noun Phrases from  Surface Expressions}
\eauthor{Masaki Murata \affiref{KUEE} \and Sadao Kurohashi 
\affiref{KUEE} \and Makoto Nagao\affiref{KUEE}} 

\eabstract{
When we are translating Japanese nouns into English, we face the problem of
articles and numbers which the Japanese language does not have, 
but which are necessary for the English composition.  
To solve this difficult problem we classified 
the referential property and the number of noun phrases 
into three types respectively: 
generic, definite and indefinite for 
the referential property of noun phrases, 
singular, plural, and uncountable for 
the number of noun phrases. 
This paper shows that the referential property and the number of noun phrases
can be estimated fairly reliably by the words in the sentence.
Many rules for the estimation were written
in forms similar to rewriting rules in expert systems with scores.
Since this method uses scores, 
it is good to deal with 
vague problems like referential properties and numbers.
We obtained the correct recognition scores of 85.5\% and 89.0\% 
in the estimation of referential property and number 
respectively for the sentences
which were used for the construction of our rules.
We tested these rules for some other texts,
and obtained the scores of 68.9\% and 85.6\% respectively.
These referential properties and numbers of noun phrases will be used 
not only for determination of articles and numbers 
but also for anaphora resolution and discourse analysis.
}

\ekeywords{Surface expression, Referential property, Number, Generic noun phrase, Definite noun phrase, Indefinite noun phrase}

\begin{document}
\maketitle


\section{はじめに}\label{intro}

機械翻訳システムには,少し微妙だが重要な問題として冠詞の問題がある.
例えば,
\vspace*{5mm}
\begin{equation}
\mbox{\underline{本}\.と\.い\.う\.の\.は人間の成長に欠かせません.}
  \label{eqn:book_hito}
\end{equation}
の「本」は総称的な使われ方で,英語では``a book''にも``books''にも
``the book''にも訳される.
これに対して,

\begin{equation}
\mbox{\.昨\.日\.僕\.が\.貸\.し\.た\underline{本}は読みましたか.}
  \label{eqn:book_boku}
\end{equation}
の「本」は英語では ``the book''と訳される.


冠詞の問題は,多くの場合,
名詞句の{\bf 指示性}と
{\bf 数}を明らかにすることによって解決できる.
文(\ref{eqn:book_hito})の「本」は総称名詞句で数は未定であり,
``a book''にも``books''にも``the book''にも訳される.
また,文(\ref{eqn:book_boku})の「本」は定名詞句で
ほとんどの場合単数と解釈してよい.
よって,英語では``the book''となる.


名詞句の指示性と数は日本語の表層表現から得られることが多い.
例えば,文(\ref{eqn:book_hito})では
「\.と\.い\.う\.の\.は」という表現から
「本」が総称名詞句
とわかる.
文(\ref{eqn:book_boku})では
修飾節「昨日僕が貸した」が限定していることから
「本」が定名詞句とわかる.
そこで,本研究では名詞句の指示性と数を日本語文中にある
このような
表層表現を手がかりとして
推定することを試みた.

名詞句の指示性と数の推定は文脈依存性の高い問題であり,
本来文脈処理などを行なって解決すべき問題である.
しかし,現時点での自然言語処理の技術では
文脈処理を他の解析に役立てるところまでは来ていない.
また,近年コーパスベースの研究が盛んであるが,
指示性と数の正解の情報が付与されているコーパスがなく,
タグなしコーパスから指示性と数の問題を解決することはほとんど不可能であるので,
コーパスベースでこの問題を解決することはできない.
そういう状況の中で,本論文は
表層の手がかりを利用するだけでも指示性や数の問題を
かなりの程度解決することができることを示すものである.

本論文は文献\cite{Murata1993B}を詳しくしたものである.
近年,本研究は,文献\cite{Bond1994,Murata1995}
などにおいて引用され,具体的に重要性が明らかになりつつある.
\cite{Bond1994}においては,
日本語から英語への翻訳における数の決定に利用され,
また,\cite{Murata1995}においては,
同一名詞の指示対象の推定に利用されている.
そこで,本論文は本研究を論文としてまとめることにしたものである.
以前の文献ではあげられなかった規則も若干付け加えている.


\section{名詞句の指示性と数の分類}\label{sec:riron}

\subsection{名詞句の指示性の分類}

名詞句の指示性とは名詞句の対象への指示の仕方である.
まず名詞句を,その名詞句の類の成員すべてか
類自体を指示対象とする
{\bf 総称名詞句}と,
類の成員の一部を指示対象とする
{\bf 非総称名詞句}に分ける.
次に,非総称名詞句を指示対象が確定しているか否かで,
{\bf 定名詞句}と{\bf 不定名詞句}に分ける
(図\ref{fig:sijisei_bunrui})
\footnote{
この分類は文献\cite{Inoue1985}を参考にして行なった.
日本語の名詞句に対して,この分類と同じような分類をしているものに
文献\cite{Kinsui1986}が挙げられる.
しかし,
そこでは総称名詞句,定名詞句,不定名詞句の他に
指示対象を持たない名詞句が考えられている.
例えば,「私は大学教師です」の「大学教師」は指示対象を持たないと
してあった.
それに対し,
本研究では「大学教師」は
大学教師という類のある成員と考え「不定名詞句」と考える.
また,「定」「不定」の区別は聞き手の知識による分類となっており,
ここでの分類とは異なる.
}
.
この分類は英語の名詞句についての分類を念頭において行なったが,
日本語の名詞句についてもかなりの程度役に立つと考えている
\footnote{
\ref{sec:junbi}節の表\ref{fig:sousyou}で述べるように
    指示性の判断が難しい名詞句も多く,
    新たな分類を設けなければならなくなることも考えられ,
    本論文の分類はまだ完全なものではない.
    しかし,第一近似としては有用なものであると考える.
}.

\begin{figure}[t]
\small
\begin{center}
\fbox{
  \begin{minipage}[c]{220pt}
\begin{center}
{
\tiny

\[
\mbox{\normalsize 名詞句}
\left\{
\begin{array}[h]{cc}
\mbox{\normalsize 総称名詞句} & \\
 & \\
\mbox{\normalsize 非総称名詞句} &
\left\{
\begin{array}[h]{c}
\mbox{\normalsize 定名詞句}\\
 \\
\mbox{\normalsize 不定名詞句}
\end{array}
\right.
\end{array}
\right.
\]
}
\end{center}
  \vspace*{1mm}
  \end{minipage} }
  \caption{名詞句の指示性の分類}
  \label{fig:sijisei_bunrui}
  \end{center}
\end{figure}


\paragraph{総称名詞句}

総称名詞句は,その名詞句が意味する類に属する任意の成員
(単数でも,複数でも,不可算のものでもよい)のすべて,
もしくはその名詞句が意味する類それ自身を指示する.
例えば,次の文(\ref{eqn:doguse})の「犬」は総称名詞句である.
\begin{equation}
\underline{犬}は役に立つ動物です.
  \label{eqn:doguse}
\end{equation}
ここでの「犬」は
「犬」という類に属する成員のすべてを指示対象としている.

\paragraph{定名詞句}

定名詞句は,その名詞句が意味する類に属する文脈上唯一の成員(単数でも複数でも不可算のものでもよい)を指示する.
例えば,次の文(\ref{eqn:thedoguse})の「その犬」は定名詞句である.
\begin{equation}
\underline{その犬}は役に立ちます.
  \label{eqn:thedoguse}
\end{equation}
ここでの「その犬」は,
「犬」という類に属する文脈上唯一の成員を指示対象としている.
このことは,指示詞「その」によって表わされており,
聞き手は「その犬」なるものを確定できる.

\paragraph{不定名詞句}

不定名詞句は,その名詞句が意味する類に属するある不特定の成員
(単数でも複数でも不可算のものでもよい)を指示する.
不特定の成員を指示するというのは,現時点での聞き手の情報ではその名詞句が
成員のどれを指し示すのか確定していないという意味である.
また,現時点での聞き手の情報では,
その名詞句が成員のどれを
指し示しているとしても,その文の解釈として間違っていないということでもある.
不定名詞句は総称名詞句とは異なり,その名詞句の意味する類の成員のすべてを
指示するのではなくて,その名詞句の意味する類の成員の一部を指示する.
次の文の「犬」は不定名詞句である.
\begin{equation}
\underline{犬}が三匹います.
  \label{eqn:dog3}
\end{equation}
ここでの「犬」は
犬という類に属する任意の三匹の成員を指示対象として持ちえる.
これは
どんな犬でも三匹いればこの文が使えるということである.

\begin{figure}[t]
\small
\begin{center}
\fbox{
  \begin{minipage}[c]{220pt}
\begin{center}
{
\tiny

\[
\mbox{\normalsize 名詞句} 
\hspace{1mm}
\left\{
\begin{array}[h]{cc}
\mbox{\normalsize 可算名詞句}  &
\left\{
\begin{array}[h]{c}
\mbox{\normalsize 単数名詞句} \\
 \\
\mbox{\normalsize 複数名詞句}
\end{array}
\right. \\
  & \\
\mbox{\normalsize 不可算名詞句} &
\end{array}
\right.
\]

}
\end{center}
  \vspace*{1mm}
  \end{minipage} }
  \caption{名詞句の数の分類}
  \label{fig:suu_bunrui}
  \end{center}
\end{figure}

\subsection{名詞句の数の分類}
名詞句の数とはその名詞句が指示する対象の数のことである.
名詞句をその指示対象が
数え上げられるか数え上げられないかに応じて,
{\bf 可算名詞句},
{\bf 不可算名詞句}に分ける.
次に,可算名詞句をその指示する対象が一個か複数個かに応じて
{\bf 単数名詞句},{\bf 複数名詞句}に分ける
(図\ref{fig:suu_bunrui})
.
この分類は名詞句の指示性と同様に
英語の名詞句の分類を念頭において行なった
\footnote{
    この分類は日本語から英語への翻訳を念頭において行なったものであるが,
    以下の例の school(学校)のように
    英語では無冠詞で表現されて不可算を思わせるものであっても
    意味的には単数であるものは単数として考える.
    \begin{quote}
    私はたいてい八時に\underline{学校}へ行きます.\\
    (I usually go to \underline{school} at 8:00.)
    \end{quote}
}.



\paragraph{単数}
名詞句の指し示す対象が,
話者
の頭の中で
一個のものとして他のものと区別して捉えることができる場合,
その名詞句の数は単数となる.
例えば,
次の文の「ケーキ」は単数である.
\begin{equation}
彼女は\underline{ケーキ}を一個持って行きました.
\end{equation}
ここでの「ケーキ」は個々に区別して捉えることができ,
一個である.


\paragraph{複数}
名詞句の指し示す対象が,話者の頭の中で
個々に区別できるものとして複数個ある場合,
その名詞句の数は複数となる.
例えば,
次の文の「たくさんのケーキ」は複数である.
\begin{equation}
この店には\underline{たくさんのケーキ}があります.
  \label{eqn:cake_mise}
\end{equation}
ここでの「たくさんのケーキ」は個々に区別して捉えることができ,
またたくさんあるので複数個ある.


\paragraph{不可算}
名詞句の指し示す対象が,話者の頭の中で
個々に区別できないものである場合,
名詞句の数は不可算となる.
例えば,
次の文の「銅」は不可算である.
\begin{equation}
\underline{銅}はよく熱を伝導します.
  \label{eqn:cake_kinou}
\end{equation}
ここでの「銅」は個々に区別して捉えることができない.
「銅」は「銅」という物質として使われており,
不可算である. 
\vspace*{-2mm}


\section{名詞句の指示性と数の推定方法}\label{sec:decide}

\subsection{「可能性」と「得点」}\label{sec:point}

名詞句の指示性と数の推定は,表層の言語表現を手がかりにした規則を
異なった種類の表現に応じて必要なだけ作り,
入力文に対してそれらを適用することによって行なう.

ある表層表現を手がかりにして,
そこにあらわれる名詞句がある分類に属さないことがわかる場合がある.
例えば,「\underline{\.あ\.る犬}」
は連体詞「ある」がついていることから,「不定名詞句」であって
「総称名詞句」「定名詞句」になる可能性はないことがわかる.
これを表現するために,{\bf 「可能性」}という評価値を導入する.
「可能性」が1のときその分類に属する可能性があることを意味し,
「可能性」が0のときその分類の可能性がないことを意味する.
「可能性」が0となる規則が適用されれば,
名詞句がその分類に属する可能性はなくなることになる.

ある名詞句がある分類に属するかどうかを一つの表層表現から推定するのではなく,
複数の表層表現を手がかりに推定を行なえば分類の精度はよくなると考えられる.
そのため各規則に重要性を表わす{\bf 「得点」}という評価値を導入し,
ある名詞句に対して適用された規則の「得点」を合計することで
その名詞句がある分類に属する場合の評価値とする.
「可能性」の情報だけでは
どの分類に属するのか一意に推定できない場合に,
この評価値は
いずれの分類が最も適当であるか推定する基準になる.

\begin{figure}[t]
\small
  \begin{center}
\fbox{
  \begin{minipage}[c]{220pt}
\baselineskip=12pt
\hspace*{1.0cm}\protect\verb+(規則の適用条件)+\\
\hspace*{2.0cm}\protect\verb++\{\verb+不定   (可能性 得点)+\\
\hspace*{2.18cm}\protect\verb+定    (可能性 得点)+\\
\hspace*{2.18cm}\protect\verb+総称   (可能性 得点)+\}

  \end{minipage} }
  \caption{名詞句の指示性を推定する規則}
  \label{fig:rule_kouzou_sijisei}
  \end{center}
\end{figure}


\begin{figure}[t]
\small
  \begin{center}
\fbox{
  \begin{minipage}[c]{220pt}
\baselineskip=12pt
\hspace*{1.0cm}\protect\verb+(規則の適用条件)+\\
\hspace*{2.0cm}\protect\verb++\{\verb+単数   (可能性 得点)+\\
\hspace*{2.18cm}\protect\verb+複数   (可能性 得点)+\\
\hspace*{2.18cm}\protect\verb+不可算  (可能性 得点)+\}

  \end{minipage} }
  \caption{名詞句の数を推定する規則}
  \label{fig:rule_kouzou_suu}
  \end{center}
\end{figure}

規則は
指示性の場合に図\ref{fig:rule_kouzou_sijisei},
数の場合に図\ref{fig:rule_kouzou_suu}
の構造をしている.
図の「規則の適用条件」には,
その規則が適用されるかどうかの条件として,
後で説明する
依存構造の表現
の形で文中の手がかりとなる表現を記述する.
各分類には「可能性」と「得点」を一つずつ与えている.
「可能性」は1か0のみであり,
「得点」は0から10の間の整数である.

「可能性」が1の分類がただ一つ求まった場合は,その分類を推定の結果とする.
「可能性」が1の分類が複数ある場合は,
その中で「得点」が最も大きい分類を推定の結果とする.
同点の場合は同点の分類すべてを推定の結果とする.


\begin{figure}[t]
\small
  \begin{center}

\fbox{
\begin{minipage}[c]{300pt}
\baselineskip=16pt
\hspace*{0cm}\protect\verb+        依存構造の表現 ::= ( 文節  依存構造の表現  依存構造の表現 ... )+\\
\hspace*{0cm}\protect\verb+             文節 ::= < 単語  単語 ... >+\\
\hspace*{0cm}\protect\verb+             単語 ::= [ 品詞 品詞細分類 活用型 活用形 基本形 変化形 ]+
  \end{minipage} }
  \caption{依存構造の表現の基本要素の形式}
  \label{fig:s_eps}
  \end{center}
\end{figure}

\begin{figure}[t]
\small
  \begin{center}

\fbox{
  \begin{minipage}[c]{340pt}
\begin{center}
  \begin{minipage}[c]{260pt}

\baselineskip=0pt
{
\hspace*{0.000cm}\protect\verb+               彼は──┐ +\\
\hspace*{0.000cm}\protect\verb+      その──┐        │ +\\
\hspace*{0.000cm}\protect\verb+         弁護士の──┐   │ +\\
\hspace*{0.000cm}\protect\verb+              息子の──┤ +\\
\hspace*{0.000cm}\protect\verb+                 一人です.+
}
\center{(a):依存構造}
 \end{minipage}
\end{center}

\vspace{3mm}

\begin{center}
 \begin{minipage}[c]{260pt}
\baselineskip=12pt
\hspace*{0cm}\protect\verb+( <[名詞 普通名詞 +\_\verb+ +\_\verb+ 一人 一人]+\\
\hspace*{0.33cm}\protect\verb+   [判定詞 +\_\verb+ 判定詞 デス列基本形 だ です]+\\
\hspace*{0.33cm}\protect\verb+   [特殊 句点 +\_\verb+ +\_\verb+ . .]> +\\
\hspace*{0.33cm}\protect\verb+   ( <[名詞 普通名詞 +\_\verb+ +\_\verb+ 息子 息子]+\\
\hspace*{0.84cm}\protect\verb+      [助詞 名詞接続助詞 +\_\verb+ +\_\verb+ の の]> +\\
\hspace*{0.84cm}\protect\verb+      ( <[名詞 普通名詞 +\_\verb+ +\_\verb+ 弁護士 弁護士]+\\
\hspace*{1.36cm}\protect\verb+         [助詞 名詞接続助詞 +\_\verb+ +\_\verb+ の の]> +\\
\hspace*{1.36cm}\protect\verb+         ( <[指示詞 +\_\verb+ +\_\verb+ +\_\verb+ その その]> )))+\\
\hspace*{0.33cm}\protect\verb+   ( <[名詞 普通名詞 +\_\verb+ +\_\verb+ 彼 彼]+\\
\hspace*{0.84cm}\protect\verb+      [助詞 副助詞 +\_\verb+ +\_\verb+ は は]+\\
\hspace*{0.84cm}\protect\verb+      [特殊 読点 +\_\verb+ +\_\verb+ , ,]> ))+
\center{(b):依存構造の表現}
  \end{minipage}
\end{center}

{

\hspace*{2cm}
(a)に示す依存構造は
(b)に示す形に表現される.
}



  \end{minipage} }
  \caption{入力文「彼はその弁護士の息子の一人です.」を表わす表現}
  \label{fig:弁護士_csan}
  \end{center}
\end{figure}
\subsection{システムの動作}\label{sec:system}

文中の名詞句の指示性と数の推定は次のようなステップで行なわれる.
\begin{itemize}
\item[(1)]
  与えられた文の形態素解析,構文解析
  \footnote{
    形態素解析,構文解析は参考文献
    \cite{Matsumoto1992,Kurohashi1992}
    のものを用いた.
    }
  が行なわれ,依存構造の表現に変換される.
  この依存構造の表現は図\ref{fig:s_eps} のような形式のものであり,
  文節間の係り受けの情報を含んだ表現である.
  その例を図\ref{fig:弁護士_csan} に示す.

\item[(2)]
  依存構造の表現に変換された文の名詞句を文頭から順に推定する
\footnote{
このため,既に推定された指示性と数は後に出てくる名詞句の解析の時に
手がかりとして使用できる(例:\ref{subsec:abs_rule}節の具体例(c)(d)).
}.
  各名詞句に対しては指示性を先に数を後に推定する
\footnote{
このため,
数の推定には指示性の解析結果を用いることができる
(例:\ref{subsec:num_rule}節の規則の例の3).
}.
  指示性の推定は,指示性の規則をすべて用いて
  各分類の「可能性」と「得点」を計算する.
  この「可能性」と「得点」から\ref{sec:point}節で述べたように
  指示性を推定する.
  数の推定も同様である.
  規則の適用条件は依存構造の表現に似た形で表す.
  例えば,
  「その」がかかる名詞句を表現する場合は,
  図\ref{fig:その} のような構造となる.
  図中の``\verb+-+''は任意の依存構造の表現の部分を表す.
  このような適用条件の表現と入力文の依存構造の表現とを比較して
  規則が適用されるか否かを決定する.
  規則の適用条件の部分には,
  正規表現,論理和,論理積,否定などを書くことができる.
  また,比較部分を指定することによって
  文章中の任意の部分と比較することができる.

\item[(3)]
  上記(2)で得られる推定の結果を図\ref{fig:弁護士_noun} に
  示す形で出力する.
\end{itemize}

\begin{figure}[t]
\small
  \begin{center}
\fbox{
 \begin{minipage}[c]{300pt}

\baselineskip=12pt
\hspace*{1.10cm}\protect\verb+      ( <[名詞 -]>+\\
\hspace*{1.6cm}\protect\verb+         ( <[指示詞 +\_\verb+ +\_\verb+ +\_\verb+ その その]> ) - )+

  \end{minipage} }
  \caption{「その」がかかる名詞句を表す規則の適用条件の表現}
  \label{fig:その}
  \end{center}
\end{figure}


\begin{figure}[t]
\small
  \begin{center}
\fbox{
 \begin{minipage}[c]{300pt}

\baselineskip=12pt


\begin{center}
 \begin{minipage}[c]{220pt}
\baselineskip=12pt
\hspace*{0cm}\protect\verb+( <[名詞 普通名詞 +\_\verb+ +\_\verb+ 一人 一人 不定 単数]+\\
\hspace*{0.33cm}\protect\verb+   [判定詞 +\_\verb+ 判定詞 デス列基本形 だ です]+\\
\hspace*{0.33cm}\protect\verb+   [特殊 句点 +\_\verb+ +\_\verb+ . .]> +\\
\hspace*{0.33cm}\protect\verb+   ( <[名詞 普通名詞 +\_\verb+ +\_\verb+ 息子 息子 定 複数]+\\
\hspace*{0.84cm}\protect\verb+      [助詞 名詞接続助詞 +\_\verb+ +\_\verb+ の の]> +\\
\hspace*{0.84cm}\protect\verb+      ( <[名詞 普通名詞 +\_\verb+ +\_\verb+ 弁護士 弁護士 定 単数]+\\
\hspace*{1.36cm}\protect\verb+         [助詞 名詞接続助詞 +\_\verb+ +\_\verb+ の の]> +\\
\hspace*{1.36cm}\protect\verb+         ( <[指示詞 +\_\verb+ +\_\verb+ +\_\verb+ その その]> )))+\\
\hspace*{0.33cm}\protect\verb+   ( <[名詞 普通名詞 +\_\verb+ +\_\verb+ 彼 彼 定 単数]+\\
\hspace*{0.84cm}\protect\verb+      [助詞 副助詞 +\_\verb+ +\_\verb+ は は]+\\
\hspace*{0.84cm}\protect\verb+      [特殊 読点 +\_\verb+ +\_\verb+ , ,]> ))+
 \end{minipage}
\end{center}


  \end{minipage} }
  \caption{図6の文に対する指示性と数の判定結果の表現}
  \label{fig:弁護士_noun}
  \end{center}
\end{figure}


システムでは文章ごとに解析しており,
文章全体の表層表現を利用できるようにしている.
これは同一名詞が既出のとき適用される規則に用いられる.


\subsection{解析の対象から除外した名詞句}
時間を表わす名詞句,
「中」\hspace*{-.5em}「上」\hspace*{-.5em}「左」\hspace*{-.5em}「右」\hspace*{-.5em}「下」\hspace*{-.5em}「後」\hspace*{-.5em}「前」\hspace*{-.5em}「近く」\hspace*{-.5em}「遠く」\hspace*{-.5em}
「別」\hspace*{-.5em}「他」
\hspace*{-.5em}
のよ\\うな名詞を主要部に持つ名詞句,
「本当の」「普通の」「役に立つ」
などの連体詞の一部とみなせる名詞を主要部に持つ名詞句は対象から除外した.
「\underline{〜のまま}」「\underline{〜する程}」,
「\underline{〜する訳}でない」,「\underline{〜する度}」
の下線部に当たる名詞句なども除外した.
\hspace*{-2mm}「大分」\hspace*{-2mm}「全員」\hspace*{-2mm}「\underline{一緒}に」\hspace*{-2mm}
などの副詞とみなせるものも除外した.
以上の名詞句以外は,
対訳の英語文では名詞に訳されていない場合でも
すべて解析の対象とした.


\section{推定に用いる規則}\label{sec:rule}

規則は
日本語,英語の文法書\cite{Kokuritukokugokenkyusho1978,Kumayama1985,Ikeuchi1985}を
参考として作ったが,
実験対象テキストを見て独自に考えて作ったものもある.
実験中に新たな規則を随時追加していったが,
現時点ですべてを網羅できているとはいえない.
現在の規則の数は,
指示性が86個で,
数が48個である
\footnote{
すべての規則は文献\cite{Murata1993A}にある.
}.
次に規則の例をあげる.

\subsection{指示性の規則}\label{subsec:abs_rule}

\begin{enumerate}
\item 指示詞(「この」や「その」など)によって修飾される時,\\
\{
\mbox{不定名詞句}  (0 0) \,
\mbox{定名詞句}   (1 2)  \,
\mbox{総称名詞句}  (0 0)
\} 
\footnote{
各分類の「可能性」と「得点」を表わす.図\ref{fig:rule_kouzou_sijisei} 参照.
}\\
(例文)\underline{\.こ\.の本}はおもしろい. \\
(訳文) \underline{This book} is interesting.
\item 名詞句につく助詞が「は」で
      述語が過去形の時,\\
\{
\mbox{不定名詞句} (1 0) \,
\mbox{定名詞句}   (1 3) \,
\mbox{総称名詞句} (1 1)
\} \\
(例文)\underline{犬}\.は向うに\.行\.き\.ま\.し\.た.\\
(訳文) \underline{The dog} went away.
\item 名詞句につく助詞が「は」で
      述語が現在形の時,\\
\{
\mbox{不定名詞句} (1 0) \,
\mbox{定名詞句}   (1 2) \,
\mbox{総称名詞句} (1 3)
\}\\
(例文)\underline{犬}\.は役に立つ動物\.で\.す.\\
(訳文) \underline{Dogs}
\footnote{
主語が総称名詞句になる場合であるので
``a dog''でも``the dog''でもよい.
}
 are useful animals.
\item 名詞句につく助詞が「へ」「まで」「から」の時,\\
\{
\mbox{不定名詞句} (1 0) \,
\mbox{定名詞句}   (1 2) \,
\mbox{総称名詞句} (1 0)
\} \\
(例文)彼を\underline{空港}\.ま\.で迎えに行きましょう.\\
(訳文) Let us go to meet him at \underline{the airport}.
\item 名詞句につく助詞が「の」で体言にかかる時
\footnote{
名詞句につく助詞が「の」で体言にかかる場合,
いつでも総称名詞句であるとは限らない.
しかし,「の」は旧情報と結び付きやすい性質を持っており,
ほとんど定名詞句と総称名詞句のいずれかである.
定名詞句の場合は他の情報により推定可能になると考え,
総称名詞句により高い得点を与えている.}
,\\
\{
\mbox{不定名詞句} (1 0) \,
\mbox{定名詞句}   (1 2) \,
\mbox{総称名詞句} (1 3)
\} \\
(例文) 彼は\underline{教育}\.の\.価\.値を認識していません.\\
(訳文) He doesn't realize the value of \underline{education}.
\end{enumerate}

他にも,
(i)「地球」「宇宙」のような名詞句自身から定名詞句と推定する規則
\footnote{
\label{foot:tikyuu}
これは本来的には語の意味として取り扱うのが適切だろうが,
    これまで取り扱ってきた場合の特殊な場合と位置付けて
    規則の形で処理することにしている.
},
(ii)名詞句に数詞がかかることから総称名詞句以外と推定する規則,
(iii)同一名詞の既出により定名詞句と推定する規則,
(iv)「いつも」「昔は」「〜では」のような副詞が動詞にかかることから
総称名詞句と推定する規則,
(v)「〜が好き」「〜を楽しむ」のような動詞から
総称名詞句と推定する規則,
(vi)「用」「向き」のような接尾辞から
総称名詞句と推定する規則などがある.
手がかりとなる語がない時は不定名詞句と推定するようにしている
\footnote{
      ここであげた規則の他に,
      「息子」「お腹」などの親族呼称,体の一部を意味する名詞句は
      定名詞句である割合が高いので,
      定名詞句であると推定する規則を追加した方が良いと思われる.
      ただし,この規則は5節で述べる
      テストサンプルの実験の後に作成したものであるので,
      5節での実験では用いていない.
      しかし,この規則の有効性を確かめるため
      この規則を追加して実験したところ,
      5節で述べる指示性の精度に比べ学習サンプルでは0.4\%下がり,
      テストサンプルでは3\%上がるという結果となった.
      これは学習サンプルでは親族呼称,体の一部を意味する名詞句が
      定名詞句以外で使われる例が意外に多かったためで,
      一般のテキストでは
      親族呼称,体の一部の規則を利用した方がよいと思われる.
      このとき,
      親族呼称,体の一部を意味する名詞の判定には,
      分類語彙表\cite{Kokuritukokugokenkyusho1964}を用いている.
      分類語彙表の分類番号が121ではじまるものを親族呼称とし,
      157ではじまるものを体の一部とした.
}.

\vspace{3mm}

例として,次の文の中に現れる名詞句「我々が昨日摘みとった果物」に注目し,
これにどのような規則が適用され得点がどのようになるか,
具体的に説明する.

\newpage
\noindent
\underline{我々が昨日摘みとった果物}は味がいいです.\\
\underline{The fruit that we picked yesterday} tastes delicious.

\bigskip
\vspace*{-.5mm}
以下のように七つの規則が適用され,
この「果物」は定名詞句と推定された.

\begin{itemize}
\item[(a)] 名詞句につく助詞が「は」で述語が現在形の時,\\(果物\.は味が\.い\.い\.で\.す.)
\footnote{規則が適用される手がかりとなる表現.}
\\
\{
\mbox{不定名詞句}  (1 0) \,
\mbox{定名詞句}  (1 2)   \,
\mbox{総称名詞句}  (1 3)
\}

\item[(b)] 述部が過去形の節が係る時,\\(摘み\.と\.っ\.た)\\
\{
\mbox{不定名詞句}  (1  0) \,
\mbox{定名詞句}    (1  1) \, 
\mbox{総称名詞句}  (1  0)
\}

\item[(c)] 「は」か「が」がついた定名詞句を含む節が係る時,\\(\.我\.々\.が)\\
\{
\mbox{不定名詞句}  (1  0) \,
\mbox{定名詞句}  (1  1) \,
\mbox{総称名詞句}  (1  0)
\}

\item[(d)] 助詞がついた定名詞句を含む節が係る時,\\(\.我\.々が)\\
\{
\mbox{不定名詞句}  (1  0) \,
\mbox{定名詞句}  (1  1) \,
\mbox{総称名詞句}  (1  0)
\}

\item[(e)] 代名詞を含む節が係る時,\\(\.我\.々が)\\
\{
\mbox{不定名詞句}  (1  0) \,
\mbox{定名詞句}    (1  1) \,
\mbox{総称名詞句}  (1  0)
\}

\item[(f)] 名詞句につく助詞が「は」で述語が形容詞の時,\\(果物\.は味が\.い\.い\.で\.す.)\\
\{
\mbox{不定名詞句}  (1  0) \,
\mbox{定名詞句}    (1  3) \,
\mbox{総称名詞句}  (1  4)
\}

\item[(g)] 主要部の名詞が普通名詞の時,\\(果物)\\
\{
\mbox{不定名詞句}  (1  1) \,
\mbox{定名詞句}    (1  0) \,
\mbox{総称名詞句}  (1  0)
\}

\end{itemize}
\smallskip
これらすべての規則の適用の結果として
「果物」の最終の「可能性」と「得点」は,\\
\hspace*{9mm}
\{
\mbox{不定名詞句}  (1  1) \,
\mbox{定名詞句}    (1  9) \,
\mbox{総称名詞句}  (1  7)
\}\\
となり,定名詞句と推定された.

\subsection{数の規則}\label{subsec:num_rule}
\begin{enumerate}
\item 「その」「この」「あの」によって修飾される時,\\
\{
\mbox{単数}  (1  3) \,
\mbox{複数}    (1  0) \,
\mbox{不可算}  (1  1)
\}\\
(例文)\underline{\.あ\.の本}をください.\\
(訳文) Give me \underline{that book}.
\item 
  名詞句につく助詞が「は」「が」「も」「を」で,
  述部に数詞が係る場合,\\
数詞の数が単数の時,\\
\{
\mbox{単数}  (1  2) \,
\mbox{複数}    (1 0) \,
\mbox{不可算}  (1  0)
\} \\
数詞の数が複数の時,\\
\{
\mbox{単数}  (1  0) \,
\mbox{複数}    (1  2) \,
\mbox{不可算}  (1  0)
\} \\
(例文)\underline{りんご}\.を\.二\.個食べる.\\
(訳文) I eat two \underline{apples}.
\item
  名詞句の指示性が総称名詞句と推定されており,
  係り先の動詞が「が好きです」「を楽しむ」などのように,
  総称の名詞句を格にとる場合,\\
\{
\mbox{単数}  (1  0) \,
\mbox{複数}    (1  2) \,
\mbox{不可算}  (1  0)
\}\\
(例文)私は,\underline{りんご}\.が\.好\.き\.で\.す.\\
(訳文) I like \underline{apples}.
\end{enumerate}

他にも,(i)「空気」「水」のような名詞句自身から不可算と推定する規則,
(ii)「達」「ら」のような接尾辞から複数と推定する規則,
(iii)数詞が係ることにより推定する規則,
(iv)名詞述語文における主語と述語の数が一致することにより推定する規則,
(v)「集める」「溢れる」などの動詞から推定する規則,
(vi)「いくらでも」「何度でも」のような副詞が動詞にかかることから
推定する規則などがある.
手がかりとなる語がない時は単数と推定するようにしている.

\section{実験と考察}\label{sec:jikken}


\subsection{実験の準備}
\label{sec:junbi}

\begin{table}[t]
\small

\caption{総称名詞句とした名詞句の例(下線部の名詞を主要部に持つ名詞句)}\label{fig:sousyou}

{
  \begin{center}

\begin{tabular}{|l|} \hline

(1)\underline{ラクダ}は\underline{水}を飲まなくても長い間歩くことができます.\\ 

(2)ワシントンスクールから一クラスの学生たちが,昨日,\underline{見学}にいきました.\\ 

(3)多くの若い\underline{男}の\underline{人たち}は\underline{陸軍}に兵役します.\\ 

(4)\underline{紳士}は普通\underline{淑女}のために\underline{ドア}を開けます.\\ 

(5)有名なシャ−ロックホ−ムズ探偵の物語は大抵ロンドン地域を\underline{背景}にしたものです.\\ 

(6)彼はクリスマスの\underline{贈り物}に本を買いました.\\ 

(7)ワールドカップ大会の決勝戦は,\underline{タンゴ}のアルゼンチンと\underline{行進曲}の西ドイツとの勝負だ.\\[0.1cm] \hline
\end{tabular}
    
  \end{center}
}

\end{table}


実験に用いたテキストは
名詞句の指示性と数の正解の分類が
わかりやすいように日英の対訳がある文章に限った.
実験対象のテキストの各名詞句に対してあらかじめ正解の分類を
人手で決定した.
正解の決定の際には対訳の英語文を見て行なったが,
必ずしも冠詞にとらわれることなく
\ref{sec:riron} 節で説明した分類の定義によって正解を決定した.
指示性の分類については,
総称名詞句の判定は極めて困難であり,
表~\ref{fig:sousyou} のようなものを
総称名詞句としたが,
正解が間違っている可能性があ
る.
以下,正解とはこの人手による分類のことをいう.


数の分類については,対訳の英語文で名詞に訳されているものは
冠詞に合わせて正解を設定した.
明らかに複数とか不可算とわかるものはそのように設定し,
それ以外は単数とした.
総称名詞句が主語に来る場合は
冠詞が何になるのかわからないので,
数の分類は単数でも複数でも不可算でもいいことにして
正解は「未定」
\footnote{
各分類の得点が同点の時のみ正解とする分類.
}
とした.

\begin{table}[t]
\small

\caption{学習サンプル}\label{tab:kanshi_d}

\begin{center}


{

\begin{tabular}[c]{|l|r|r|r|r|r|r|r|r|r|r|} \hline
 & \multicolumn{5}{c|}{指示性}  &  \multicolumn{5}{c|}{数}  \\ \hline 
\multicolumn{1}{|c|}{評価}  &  不定  &  定 &  総称 &  その他 & 総数 &  単数   &  複数    &  不可算 &  その他 &   総数  \\ \hline
\multicolumn{11}{|c|}{英語冠詞用法辞典(140文, 380名詞句)} \\ \hline 
   正解   &      96  &     184  &      58  &       1  &     339  &    274 &    32 &      18  &      25  &  349   \\ 
正解を含む&       0  &       3  &       1  &       0  &       4  &       1  &       1  &       1  &       0  &       3  \\ 
  部分解  &       0  &       0  &       0  &       0  &       0  &       0  &       0  &       0  &      11  &      11  \\ 
  不正解  &       4  &      25  &       7  &       1  &      37  &       3  &      10  &       0  &       4  &      17   \\ 
\hline 
 正解率&    96.0  &    86.8  &    87.9  &    50.0  &    89.2  &    98.6  &    74.4  &    94.7  &    62.5  &    91.8    \\ \hline 
\multicolumn{11}{|c|}{こぶとりじいさん(104文, 267名詞句)} \\ \hline 
   正解   &      73  &     140 &       6  &       1  &    222  &    205  &     24  &       5  &       0  &   234   \\ 
正解を含む&       3  &       4  &       0  &       0  &       7 &       2  &       0  &       0  &       0  &       2    \\ 
  部分解  &       0  &       0  &       0  &       0  &       0   &       0  &       0  &       0  &       7  &       7  \\ 
  不正解  &      11  &      23  &       4  &       0  &      38    &       1  &      22  &       1  &       0  &      24   \\ 
\hline
正解率&   83.9  &   84.0  &   60.0  &  100.0  &   83.2  &   98.7  &   52.2  &   83.3  &    0.0  &   87.6    \\ \hline 
\multicolumn{11}{|c|}{天声人語(23文, 98名詞句)} \\ \hline 
   正解   &      25  &     35 &      16  &       0  &    76  &      64  &   13 &       0  &       3  &   80    \\ 
正解を含む&       0  &       4  &       2  &       0  &       6  &       2  &       1  &       0  &       0  &       3   \\ 
  部分解  &       0  &       0  &       0  &       0  &       0   &       0  &       0  &       0  &       6  &       6  \\ 
  不正解  &       5  &      10  &       1  &       0  &      16   &       1  &       6  &       1  &       1  &       9  \\ 
\hline
正解率&    83.3  &    71.4  &    84.2  &    -----  &    77.6   &    95.5  &    65.0  &     0.0  &    30.0  &    81.6   \\ \hline
全体での出現率 &  29.1  &  57.7   &  12.8   &  0.4  &  100.0 &  74.2  &  14.6   &  3.5  &  7.7   & 100.0  \\

全体での正解率 &  89.4  &  84.0   &  84.2   &  66.7   &  85.5  &  98.2  &  63.3   &  88.5  &  49.1   &  89.0   \\ \hline 
\end{tabular}
}
\end{center}
\end{table}

\subsection{実験}

\ref{sec:system} 節で述べたように
指示性と数の推定の前に
形態素・構文解析を行なうが,
そこでの誤りは人手で修正した.
まず,三つの資料\{英語冠詞用法辞典\cite{Kumayama1985}
の典型的な用法の例文(140文,解析した名詞句380個),
物語の「こぶとりじいさん」\cite{Nakao1985}全文(104文,解析した名詞句267個)
,86年7月1日の天声人語(23文,解析した名詞句98個)\}
を学習サンプルとして実験を行なった.
システムには正解を入力して自動的に正解率を出力できるようにして,
正解率が向上するように規則を変更,追加した.
最も正解率が良くなった規則の時の正解率を
表\ref{tab:kanshi_d} に示す.
このとき,三つの資料に対してはすべて同じ規則で実験を行なった.

    規則の変更,追加は,学習サンプルでのすべての誤り箇所に対して
    以下のことを行なうことによって実現する.
    \begin{enumerate}
    \item 
      \label{enum:error_mod}
      誤り箇所を見て規則の変更・追加を行なう.
      具体的には,誤り箇所の周辺の表層表現を眺め,
      新たに規則を作成できないかを考える.
      また,このとき適用されている規則の条件部や得点を変更することで
      この誤り箇所を直すことができないかを調べる.

    \item 
      \ref{enum:error_mod}のルールの変更・追加を行なった後に
      実験を行ない全体の正解率が上がるか下がるかを調べる.
      正解率が上がれば\ref{enum:error_mod}で行なった変更・追加
      を正式に採用する.
      正解率が下がった場合は,\ref{enum:error_mod}で行なった変更・追加
      は行なわず,
      \ref{enum:error_mod}の検討を何回か行なう.
    \end{enumerate}
    
    このとき,大雑把に誤り例を調べ,
    同じ理由で誤ったもので,規則を追加することで
    それらを改善できる場合はその規則を追加するということも行なう.
    また,ある規則を追加するべきかどうかが問題となったときに,
    その規則が適用される箇所をすべて出力し,
    それらを総合的に眺めた上で判断する場合もある.

\begin{table}[t]

  \caption{テストサンプル}\label{tab:turu_d}

\begin{center}


{

\begin{tabular}[c]{|l|r|r|r|r|r|r|r|r|r|r|} \hline
 & \multicolumn{5}{|c|}{指示性}  &  \multicolumn{5}{c|}{数}  \\ \hline 
\multicolumn{1}{|c|}{評価}  &  不定  &  定    &  総称 &  その他 &   総数  &  単数  &  複数    &  不可算 &  その他 &   総数    \\ \hline 
\multicolumn{11}{|c|}{つるのおんがえし(263文, 699名詞句)} \\ \hline 
   正解   &     109  &    363  &      13  &      10  &   495   &     610  &      13  &       1  &       1  &     625  \\ 
正解を含む&       6  &      25  &       0  &       0  &      31  &      12  &       2  &       0  &       0  &      14   \\ 
  部分解  &       0  &       0  &       0  &       0  &       0  &       0  &       0  &       0  &       1  &       1   \\ 
  不正解  &      32  &     135  &       6  &       0  &     173   &       2  &      20  &      37  &       0  &      59  \\ 
\hline
  正解率   &    74.2  &    69.4  &    68.4  &   100.0  &    70.8   &    97.8  &    37.1  &     2.6  &    50.0  &    89.4    \\ \hline 
\multicolumn{11}{|c|}{天声人語(75文, 283名詞句)} \\ \hline 
   正解   &      75  &    81  &      16  &       0  &   172  &     197  &      13  &       2  &       3  &     215   \\ 
正解を含む&       8  &       9  &       1  &       0  &      18 &       3  &       1  &       0  &       0  &       4    \\ 
  部分解  &       0  &       0  &       0  &       0  &       0   &       0  &       0  &       0  &       3  &       3     \\ 
  不正解  &      33  &      51  &       9  &       0  &      93  &       3  &      55  &       3  &       0  &      61    \\ 
\hline
  正解率   &    64.7  &    57.5  &    61.5  &    ----- &    60.8   &    97.0  &    18.8  &    40.0  &    50.0  &    76.0    \\ \hline 
\multicolumn{11}{|c|}{冷戦後世界と太平洋アジア(22文, 192名詞句)} \\ \hline 
   正解   &      21  &    108  &      11  &       2  &  142 &     157  &       6  &       1  &       1  &     165   \\ 
正解を含む&       6  &       7  &       0  &       0  &      13 &       3  &       0  &       0  &       0  &       3   \\ 
  部分解  &       0  &       0  &       0  &       0  &       0   &       0  &       0  &       0  &       0  &       0   \\ 
  不正解  &      11  &      24  &       2  &       0  &      37  &       3  &      20  &       1  &       0  &      24    \\ 
\hline
  正解率   &    55.3  &    77.7  &    84.6  &   100.0  &    74.0   &    96.3  &    23.1  &    50.0  &   100.0  &    85.9   \\ \hline
全体での出現率 &  25.6  &  68.4  &  4.9   &  1.0   &  100.0 &  84.3  &  11.1   &  3.8   &  0.8   &  100.0   \\
全体での正解率 &  68.1   &  68.7 &  69.0 &  100.0  &  68.9  & 97.4 &  24.6  & 8.9  &  55.6  & 85.6  \\ \hline
\end{tabular}
}
\end{center}
\end{table}

以上の学習サンプルでの実験では新しい文での正解率がわからない.
そこで,以上のようにして作った規則を固定して
,新たな三つの資料\{
物語の「つるのおんがえし」\cite{Nakao1985}全文(263文,解析した名詞句699個)
,
86年7月8,9,15日の天声人語の三回分(75文,解析した名詞句283個),
冷戦後世界と太平洋アジア
$\langle$国際文化会館会報 Vol.3 No.2 1992年 4月号$\rangle$
(22文,解析した名詞句192個)\}
をテストサンプルとして実験を行なった.
これらの正解率を表\ref{tab:turu_d} に示す.

表中の「正解」は推定の結果が
正解と一致した場合である.
「正解を含む」は
推定の結果の中に正解の分類がある場合である.
例えば,正解が定名詞句で推定の結果が定名詞句と不定名詞句が同点で得られた
場合,「正解を含む」となる.
「部分解」は
推定の結果が正解に含
まれる場合である.
「不正解」は
以上の評価以外のものである.
「正解率」は「正解」の個数を総数で割ったものである.
「全正解率」は三つの資料全てにおける正解率である.
「出現率」は各分類の個数を総数で割ったものである.
「その他」は,
単数と複数と不可算の得点が同点の時のみ正解とする「未定」のように,
複数個の分類が正解になるものの個数である.



\subsection{考察}

\subsubsection{指示性の実験に対する考察}

正解率が上がるようにテキストの表現に対して
規則を変更して実験した学習サンプルの
テキスト全体での正解率は85.5\%であった.
また,各分類に対する正解率も極端に悪いものはない.
このことから
表層表現を手がかりとした
我々の方法で
極めて多くの名詞句の指示性が推定できることがわかった.

規則を固定して実験した
テストサンプルの
テキスト全体での正解
率は,68.9\%であった.
また,各分類に対する正解率もほぼ均等に良くすべて50\%以上である.
つるのおんがえしの実験では,
定名詞句の出現率が74.8\%であったので
すべての解析結果を定名詞句にする規則を作ると
正解率が74.8\%になり実験の正解率の70.8\%より高くなるが,
不定名詞句と総称名詞句の正解率が0\%になるので意味がない.
われわれはそれぞれの分類が均等に良い正解率が得られることに
価値があると考えている.

テストサンプルのテキストでの
正解率はあまり良くないが,
規則を修正すれば容易に上がると考えられる.
しかし,新しい文章に対する正解率を上げようとすると
どこまでも規則を増やしていかねばならないという
危険性がある.


規則を変更しても解析が失敗する例として,
表\ref{fig:false_tei},
表\ref{fig:false_sousyou}
がある.
表\ref{fig:false_tei} は定名詞句であるのに,
定名詞となる手がかりがなく
解析を失敗したものである.
これを解決するには
文脈や発話状況などの
表層表現以外の情報が必要である.

表\ref{fig:false_sousyou} は総称名詞句であるのに,
解析を失敗したものである.
それぞれの例に対して失敗した理由を付けている.
総称名詞句は判断が難しく,
成功はしているが正解があっているのか不確かな例もある.
\begin{equation}
\underline{ラクダ}は\underline{水}を飲まなくても長い間歩けます.
  \label{eqn:rakuda}
\end{equation}
\vspace*{-.3mm}
という文の「ラクダ」は明らかに総称名詞句であるが,
「水」は総称名詞句でいいのだろうか.
総称名詞句は他と明確に区別して
考えにくく種々の性質のものがありそう
なので,
新たに分類を考え直さなくてはならないだろう.



\setcounter{bottomnumber}{2}

\begin{table}[t]
\small

\caption{解析を失敗した定名詞句の例(下線部の名詞を主要部に持つ名詞句)}\label{fig:false_tei}


{
  \begin{center}
\begin{tabular}{|p{300pt}|} \hline


(1)彼は\underline{社長}の兄さんです.\\ 
  

(2)ジョンは\underline{クラス}の中で一番背が高い.\\ 
  

(3)彼女は\underline{テーブル}のほこりを取り除くためにふきんを使いました.\\ 


(4)\underline{仕事}で難しい所がありましたが,克服しました.\\ 
  

(5)私は\underline{先生}と同じ本を持っています.\\ 


(6)車は\underline{道}の\underline{わき}に駐車してあります.\\ 


(7)ジョンソン教授は\underline{学会}で\underline{論文}を読みました.\\[0.1cm] \hline
\end{tabular}
    
  \end{center}
}

\end{table}


\begin{table}[t]
\small

\caption{解析を失敗した総称名詞句の例(下線部の名詞句)}\label{fig:false_sousyou}

{



  \begin{center}

\begin{tabular}{|l|} \hline


 (1)修飾節で限定され定名詞句になる例 \\

\underline{それ自体を守ろうとしない文化}は滅びます. \\[0.1cm] \hline


(2)述語が過去形のために定名詞句になる例 \\

\underline{中国人}は独自の文字を発明しました. \\[0.1cm] \hline



(3)判定詞「だ」がつくために不定名詞句になる例 \\

日本の社会では父親は\underline{家長}です. \\[0.1cm] \hline



(4)手がかりがなく不定名詞句になる例 \\

\underline{食物}がおいしければおいしいほど,たくさん食べます. \\

普通\underline{肺炎}にかかると入院しなければなりません. \\[0.1cm] \hline
\end{tabular}

    
  \end{center}
}

\end{table}



これ以外の成功した例でも深く考察すると
表層表現で推定できるのか疑問なものがある.
\begin{equation}
\mbox{これは\underline{\.私\.が\.彼\.か\.ら\.借\.り\.た辞書}です.}
  \label{eqn:kari_jisyo}
\end{equation}
この文の「私が彼から借りた辞書」は修飾節により限定されていて
定名詞句と解析される.
しかし,この「私が彼から借りた辞書」は,
「私」が「彼」から複数の辞書を借りており,そのうちの一つを指す場合には
不定名詞句となる.
つまり,ある程度の知識がなければ定名詞句か不定名詞句かの判断ができない.

\subsubsection{数の実験に対する考察}

正解率が上がるようにテキストの表現に対して
規則を変更して実験した
学習サンプルとしての
テキスト全体での正解率は,89.0\%であった.
しかし,「複数」の正解率が全体的に悪い.
また,「不可算」は不可算名詞句としてあらかじめ
登録してあるので見かけ上正解率は高いが,問題がないわけではない.


規則を固定して
実験したテストサンプルとしてのテキスト全体での正解率は,85.6\%であった.
しかし,「複数」「不可算」の正解率は悪く,
「単数」の出現率と正解率が高いおかげで高い正解率が出ているにすぎない.
また,その「単数」の正解率が高いのも,
手がかりのないときは「単数」と推定させているからにすぎない.

正解が「複数」である名詞句で,
解析が失敗した例に以下の下線部の名詞句がある.
\begin{equation}
\underline{あなたが注文した建築材料}がきました.
  \label{eqn:kentiku}
\end{equation}
手がかりがなく,解析結果は「単数」となってしまう. 
これを「複数」と判定できるようにするには,
「建築材料」という名詞自身から
「複数」と判定できるようにしなければならない.
しかし,いつでも「建築材料」が「複数」とは限らない.


数量表現以外から「複数」とわかった例として
以下の下線部の名詞句がある.\hspace*{-2mm}
\begin{equation}
\mbox{その事故が発生してから\underline{野次馬}\.が\.集\.ま\.っ\.てきました.}
  \label{eqn:jiko}
\end{equation}
\vspace*{-.3mm}
「が集まる」から「野次馬」が「複数」と解析された.
このような規則を作っていけば,
数が明\\示されていなくても
名詞句の数がわかる場合がある.
しかし,以下の例のように間違う場合もある.
\begin{equation}
\hspace*{-0.3cm}\mbox{\underline{シュート}\.を\.浴\.び\.たゴールキーパーが,
右のてのひらを裂いた.}
  \label{eqn:shoot}
\end{equation}
\vspace*{-.3mm}
「を浴びる」の規則は
実験では「複数」と「不可算」が強いと
していたが,
一発のシュートで手\\を裂いたので,
上のシュートは「単数」である.
動詞からの推定もそう安易なものではない.

\begin{table}[t]
\small
  \leavevmode
    \caption{数を推定するのに利用できそうな動詞の例}
  \begin{center}
    \label{tab:num_verb}
\begin{tabular}{|p{13cm}|}\hline
浴びる,吹きかける,まぶす,わきでる,
そろえる,たてこむ,うもれる,流れる,
吸い出す,しみる,もれる,そそぐ,
もぐる,こぼれる,散らばる,
群がる,押し寄せる,並べる,連ねる,
増える,溢れる,折り重ねる,
数える,飲む,どよめく\\\hline
\end{tabular}
\end{center}
\end{table}

学習サンプル,テストサンプルでの実験の後,
「集まる」「並べる」「浴びる」などの
動詞から数を推定する規則について考察した.
分類語彙表の動詞の部分をながめ,
数の推定に利用できそうな動詞を約300個人手で抽出した.
その例を表\ref{tab:num_verb} にあげる.
また,これらの動詞から数を推定できる名詞句がどれくらいの割合で出現するかを
調べた.
85年2月の天声人語一カ月分のうち構文解析が成功した文(526文,名詞2680個)に対して
動詞から数を推定できる名詞句の数を数えたところ21個存在した.
この数は少ないがそれでも数を解析できる名詞句が増えるので,
動詞から数を推定する規則も利用する必要があると考える.

\section{おわりに}\label{sec:end}

学習サンプルでの正解率は,
指示性で85.5\%,
数で89.0\%であり,
テストサンプルでの正解率は,
指示性で68.9\%,
数で85.6\%であった.

指示性の推定における課題としては,次の二つのことが残っている.
一つは,人間が見ると状況から定名詞句
であることが
明らかであるのに推定できていない場合である.
状況の情報をうまく使えるようになれば,
推定できるようになる.
しかし,このときも知識だけでなく
表層表現と知識の連携が必要であろう.
もう一つは,総称名詞句に関することである.
総称名詞句は他の分類とはっきりと区別して定義することが
難しいという性質をもっており,
まだまだ考えてゆかなければならない問題である.
しかし,現時点で総称名詞句としているものは
表層表現を手がかりとして,
ある程度取り出すことができるので,
分類がどう変化しても
本研究で用いた規則の適用条件の部分はそのまま使えると期待できる.

数の方は数量詞のような表層表現があれば容易に推定できるが
いつでも数量詞があるとは限らないので数の推定はそう容易ではない.
しかし,数量詞のような表層表現がなくても,
動詞「集める」や副詞「いくらでも」などの表層表現によって
数を推定できる場合がある.
このような規則によって推定できるものは少しではあるが,
それでも解析できる名詞句の数が少しでも増えることになるので,
このような規則も利用するべきであると考える.


\bibliographystyle{jnlpbbl}
\bibliography{jpaper}

\begin{biography}
\biotitle{略歴}
\bioauthor{村田 真樹}{
1993年京都大学工学部電気工学第二学科卒業.
1995年同大学院修士課程修了.
同年,同大学院博士課程進学,現在に至る.
自然言語処理,機械翻訳の研究に従事.}
\bioauthor{黒橋 禎夫}{
1989年京都大学工学部電気工学第二学科卒業.
1994年同大学院博士課程修了.
同年,京都大学工学部助手,現在に至る.
自然言語処理,知識情報処理の研究に従事.
1994年4月より1年間Pennsylvania大学客員研究員.}
\bioauthor{長尾 真}{
1959年京都大学工学部電子工学科卒業.工学博士.京都大学工学部助手,
助教授を経て,1973年より京都大学工学部教授.国立
民族学博物館教授を兼任(1976.2 -- 1994.3).
京都大学大型計算機センター長(1986.4 -- 1990.3),
日本認知科学会会長(1989.1 -- 1990.12),パターン認識
国際学会副会長(1982 -- 1984),日本機械翻訳協会初代会長(1991.3 -- 1996.6),
機械翻訳国際連盟初代会長(1991.7 -- 1993.7).
電子情報通信学会副会長(1993.5 -- 1995.4).
情報処理学会副会長(1994.5 -- 1996.4).
京都大学附属図書館長(1995 -- ).
パターン認識,画像処理,機械翻訳,自然言語処理等の分野を並行して研究.}

\bioreceived{受付}
\biorevised{再受付}
\bioaccepted{採録}

\end{biography}

\end{document}

