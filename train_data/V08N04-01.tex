\documentstyle[jnlpbbl,epsf]{jnlp_j}

\setcounter{page}{3}
\setcounter{巻数}{8}
\setcounter{号数}{4}
\setcounter{年}{2001}
\setcounter{月}{10}
\受付{2000}{9}{20}
\再受付{2000}{11}{22}
\再々受付{2001}{4}{4}
\採録{2001}{6}{29}

\setcounter{secnumdepth}{2}

\title{拡張言語行為論による了解の分析\\
-- あいづち「はい」による了解の程度と過程 --}
\author{土井 晃一\affiref{CLS} \and 大森 晃\affiref{RIKADAI}}
\headauthor{土井,大森}
\headtitle{拡張言語行為論による了解の分析\\
-- あいづち「はい」による了解の程度と過程 --}
\affilabel{CLS}{(株)セレスター・レキシコ・サイエンシズ}{Celestar Lxico Sciences Ltd.}
\affilabel{RIKADAI}{東京理科大学 工学部 経営工学科}
{Dept. of Management Science, Faculty of Engineering, Science University of Tokyo}

\jabstract{
{\bf 了解}という言語現象が言語行為の分析にとって重要であることが,
Austinによって指摘された.しかし了解に関しては,これまで十分な分析が行
なわれてこなかった.本論文では,{\bf 了解}の語用論的な分析を行った.語
用論的な分析をするためにAustinとSearleによる言語行為論の拡張を行い,拡
張言語行為論の枠組みを提案した.その枠組みには以下のような特徴がある.
\begin{itemize}
\item 新たに二つの概念要素({\bf 隠蔽された命題行為}と{\bf 意図})を既
存の言語行為論に取り入れている.
\item 既存の言語行為論における発語媒介行為と発語媒介的効果を,それぞれ,
二種類の行為および四種類の効果に分割している.
\end{itemize}
その結果,拡張言語行為論の枠組みは13の概念要素からなることになった.提
案した枠組みに基づいて,了解の代表的表現のひとつである「はい」の意味の
多様性を,了解の過程・程度を軸にして語用論的に分析した.分析の結果,
{\bf 了解の程度}には八つの段階,{\bf 了解の過程}には七つの段階があるこ
とが明らかになった.}

\jkeywords{了解,言語行為論,あいづち,語用論的推論}

\etitle{Analyzing the Uptake \\
with an Extended Speech Act Theory \\
-- The Degree and the Process of the Uptake \\
through Chiming-in ``Hai'' --}

\eauthor{Kouichi DOI\affiref{CLS} \and Akira OHMORI\affiref{RIKADAI}}

\eabstract{
It was pointed out by Austin that the linguistic phenomenon of the
``uptake'' was important for an analysis of the speech act in
conversation. The ``uptake'', however, has not been analyzed
sufficiently. This paper has analyzed the ``uptake'' in terms of
pragmatics. We have extended the speech act theory by Austin and
Searle in order to analyze the ``uptake'' in terms of pragmatics, and
then proposed a framework of the extended speech act theory. It has
the following features:
\begin{itemize}
\item It newly incorporates two conceptual elements (a {\em hidden
prepositional act} and an {\em intention}) into the existing speech
act theory.
\item The perlocutionary act and the perlocutionary effect in the
existing speech act theory are divided into two kinds of act and into
four kinds of effect, respectively.
\end{itemize}
As a result, the framework of the extended speech act theory has 13
conceptual elements in total.
Based on the framework proposed, a diversity of meanings of ``hai'' in
Japanese, which is a typical representation of the ``uptake'', has
been investigated in terms of pragmatics, in the context of the degree
and process of the ``uptake''. We have found eight levels in the {\em
degree} of the ``uptake'', and seven stages in the {\em process} of
the ``uptake''.}

\ekeywords{uptake, speech act theory, chiming-in, pragmatic inference}

\begin{document}

\thispagestyle{plain}
\maketitle
\vspace*{1em}
\section{まえがき}

本論文では,{\bf 了解}の語用論的な分析を行う.語用論的な分析を可能にす
るために言語行為論の拡張を行い,それに基づいて{\bf 了解}の分析を行う.

了解の類義語として理解・納得などがある.理解は比較的浅い了解,納得は比
較的深い了解を指すものであり,これらは了解の一形態である.本論文では,

\begin{enumerate}
\item 一般に使われている了解
\item 理解
\item 納得
\end{enumerate}

\noindent のすべてを包含する用語として,{\bf 了解}を用いることとする.

了解は,様々な形態で顕現しうる.我々は,了解の顕現形態を図
\ref{response}のように分類・定義する.すなわち,主として言語一文節によ
る了解の顕現形態(例えば「はい」)を「あいづち」と呼び,「あいづち」およ
び,「あいづち」以外の言語による了解の顕現形態(例えば「私もそう思いま
す」)の双方を総括して「了解応答言語表現」と呼び,「了解応答言語表現」
および言語によらない了解の顕現形態(例えば,うなずき)の双方を総括して
「了解応答」と呼ぶ.図\ref{response}における実線矢印は包含関係を,破線
矢印は例をそれぞれ示している.

\begin{figure}[htbp]
\begin{center}
\atari(92,67)
\caption{了解の顕現形態(Figure \ref{response} The Representation of the ``Uptake'')}
\label{response}
\end{center}
\end{figure}

なお,あいづちの具体例としては,「はい」以外にも以下のものがある.

\begin{quote}
はーい,ええ,はあ,はー,そう,そうですね,そうですよね,そうそう,そ
うだね,そうよねー,なるほどね,うん,うーん,ふん,ふーん,ああ
\end{quote}

\noindent これらは,実際の会話で具体的に観察されたものであり,頻繁に出
現したものである.

島津ら\cite{shimazu} は,会話における了解の顕現形態として「はい」を典
型とする「間投詞的応答表現」を挙げている.彼らの研究では,非対面的会話
を対象にしており,
了解の顕現形態を図\ref{response}の「あいづち」(彼らの言うところの「間
投詞的応答表現」)に限定している.しかし,対面的会話を対象にすると,了
解の顕現形態は「間投詞的応答表現」を含む図\ref{response}のようになる.

本論文では,了解応答の分析を通じて,了解の程度と過程を明らかにすること
を目的とする.その際,分析対象とする了解応答は,あいづちである「はい」
に限定する.
従来,あいづちの分析では,国語学的
あるいは文法的な分析が行われていた(例えば島津ら\cite{shimazu} による).
本論文では,拡張言語行為論を用いて語用論的な分析を行う.ここでいう拡張
言語行為論は,Searle\cite{searle} の言語行為論にいくつかの概念要素を追
加し,既存の概念要素のいくつかを詳細化したものである.また,語用論の分
野で周知の間接発話行為を詳細化したものでもある.


まず第2節では,関連研究の概要を述べる.第3節ではSearleの言語行為論を
概説し,第4節では拡張言語行為論の枠組みを与える.第5節では,拡張言語
行為論の枠組みを用いて,あいづち「はい」による了解応答を分析し,さらに
「はい」による了解の程度と過程を明らかにする.第6節では,本論文のまと
めと発展的研究の可能性について述べる.

\section{関連研究}


まず,言語学の分野では,Austin\cite{austin} ,Searle\cite{searle} によ
る言語行為の研究が挙げられる.彼らは言語行為を分析するために,主として
単独の発話だけを対象にして言語行為論\footnote{言語学の分野では発話行為
論という呼称が主流となっているが,ここではSearle\cite{searle}における訳語を
採用した.}を展開した.そこでは,意図・状
況・文脈を特定しやすい典型的な発話を分析している.彼らの言語行為論は,単独
の発話ではあるが,言語行為を分析するための基本的な枠組みを与えた.また
Austinは,会話の中の言語現象である了解が言語行為の分析にとって重要であ
ることを指摘した\cite{austin} .しかし,Austin,Searleともに了解の分析
を行ったわけではない.


一方,言語行為論に代表されるトップダウン的なやり方では限界があるので,
ボトムアップ的なやり方として会話分析が用いられている.会話を書き起こし,
ビデオを観察しながら知見を得ようとするこの分野では,近接ペア
\cite{levinson83a} ・三組のリスト\cite{levinson83a} などの成果が得られ
たが,残念ながら方法論が経験的であり,得られた規則も抽象的すぎて他の分
野への応用は困難である.近年,この流れを汲んで,小磯らは音声的特徴との
関連で了解の一顕現形態であるあいづちの研究を進めている\cite{koiso}.し
かし,了解に関する語用論的な知見は得られていない.

首尾よく欠陥なく遂行された会話では了解の過程が示され,意図・状況・文脈
が確定するので,単独の発話だけに比べて言語行為の分析が容易になり,発話
という複雑な現象の分析が容易になる.この点に着目して,共有知識(相互信
念)の問題については人工知能の分野でプラン認識として研究が試みられてい
る\cite{kato} .会話に登場する各人物のプランをその会話を観察する人が認
識する問題は「鍵穴認識」\cite{suchman} と呼ばれている問題で,話し手の
感情・知識・信念などを直接把握できないために,これは非常に困難な問題と
されている.この分野の研究では,「鍵穴認識」の性質についていろいろわかっ
てきたものの,発話の了解の過程を十分に分析していない.しかも聞き手がど
のように,どの程度了解したかということが十分に分析されていない.

音声認識の分野では,発話の発声スタイルによる分類として,「読み上げ音声」
(read: 文章を読み上げる),「自然な発話」(spontaneous: 認識装置を意識す
るが思ったことをそのまま話す),「会話音声」(conversational: 認識装置を意
識しないでそのまま話す)の三つの分類が知られている\cite{kawahara}.音声
認識技術の進展により,研究対象が「読み上げ音声」から「自然な発話」に移っ
てきている\cite{okada}.
まず自然な発話を客観的に記述する方法を開拓し,研究が進んでいる
\cite{shimazu92a} .了解に関しても前述のような研究が進んでいる
\cite{shimazu}.また,平沢らはユーザインタフェースにあいづちを用いる研
究を進め\cite{hirasawa99},さらに人間側の理解について心理学的に研究し
ている\cite{hirasawa00}.

上記のように,了解に関連する研究はこれまで,言語学・心理学・音声学など
の視点から試みられてきた.しかし,了解の語用論的分析は行われていない.


\section{言語行為論}

言語行為論はAustin\cite{austin} に始まり,その後Searle\cite{searle} に
よってさらに発展していった.両者の言語行為論は大枠としては変わらないが,
Searleの言語行為論の方が概念要素が洗練されているので,本節ではSearleの
ものについて概説する.

Searleの言語行為論では,人間の発話を,話し手の言語行為と聞き手側の効果
を軸にして表\ref{speechact}のように分類している.発話も世界に変化をも
たらす行為として捉え,発話に伴う諸種の言語行為を導入し,独自の語用論を
展開している点が言語行為論の特徴である.

表\ref{speechact}に列挙されている言語行為と効果はおおむね次のような意
味を持つ.発話行為とは,人間の行為の中で現実に音声を発してものを言う,
つまり「発話」する行為である.命題行為とは,意味論的推論で算出可能な指
示および述定,あるいはそのどちらかを行う行為である.話し手の言語行為で,
陳述,質疑,命令,約束などのような,発話行為,命題行為と同時に遂行され
る行為を発語内行為と呼ぶ.発語内的効果とは発話行為・命題行為・発語内行
為の三種類の言語行為に対する聞き手の単なる認知または単なる理解である.
聞き手側における信念や反応の形成ではない.つまり話し手の発話による単な
る効果である.話し手が意図する意図しないに関わらず,説得する,納得させ
るなどのような,発語内行為の帰結または結果として聞き手の行動,思考,信
念などに影響を及ぼす行為を発語媒介行為と呼ぶ.聞き手の側に実際に起こる
行為ないしは効果を発語媒介的効果と呼ぶ.

\begin{table}[htbp]
\begin{center}
\begin{tabular}{c|c}
(言語行為) & 聞き手 (効果)\\	\hline
\fbox{発話行為} &\\
「発話」 &\\
\fbox{命題行為} &\\
意味論的推論で &\\
算出可能な指示と述定 &\\
\fbox{発語内行為} & \fbox{発語内的効果}\\
陳述,疑問,命令など & 聞き手の理解\\
\fbox{発語媒介行為} & \fbox{発語媒介的効果}\\
聞き手の行動,思考,信念など & 聞き手の側に実際に起こる\\
に影響を及ぼす行為 & 行為ないしは効果\\
\end{tabular}
\caption{言語行為論(Table \ref{speechact} The Speech Act Theory)}
\label{speechact}
\end{center}
\end{table}

\section{拡張言語行為論}

言語行為論は,言語行為の語用論的分析をも視野に入れており,また,了解を
分析するための枠組みへの拡張性が高いという利点も持つ.しかしながら,了
解の詳細な分析では,発話内に直接現れない命題と意図を扱う必要があり,
言語行為論に含まれる概念要素だけでは分析力が十分でない.また未分化な概
念要素もあり,さらに分析力を高めるためにはそれらの詳細化が必要である.

本節ではこうした問題点を念頭に置いて言語行為論を拡張し,拡張言語行為論
の枠組みを与える.なお本枠組みは,Searle\cite{searle}の正常入出力条件が成立してい
ることを前提としている.正常入出力条件は,話し手・聞き手ともに普通の対
話をしていることを指す.具体的には,双方とも当該言語の使いこなしができ,
双方とも対話に集中していて,双方とも発話に関する身体的欠陥はなく,劇中
の役を演じているわけではなく,冗談を言っているのではない場合などが挙げ
られる.

\subsection{隠蔽された命題行為}

Austinは,意味論的に推論できるものに限定して,発話に付随する感情,考え,
意図,帰結,含意,前提(存在前提と叙実前提)を例示している.しかし,発話
には,語用論的推論\footnote{通常,狭い意味では「語用論的推論」とは言語
内に閉じた推論を指し,言語を離れた推論は「語用論的推論」とは呼ばない.
しかし,狭義の「語用論的推論」と言語を離れた推論は区別しにくい.本論文
では,狭義の「語用論的推論」と言語を離れた推論を一括して「語用論的推論」
と呼ぶことにする.}を必要とする感情,考え,意図,帰結,含意,前提(存在
前提と叙実前提)も付随する.

一方Searleの命題行為では,発話で明示される(意味論的推論\footnote{本論
文では,発話の字句から直接行い得る推論を「意味論的推論」と呼び,発話の
字句からは直接行い得ないが,発話の字句を超えれば行い得る推論を「語用論
的推論」として区別した.}により算出可能な)「指示と述定」(命題)のみを対
象としており\cite{searle} ,発話には明示的に現れない(つまり,語用論的
推論を必要とする)命題を記述することはできない.したがって,語用論的推
論を必要とする感情・考えなど,言い換えれば,発話では明示的ではない命題
は,AustinとSearleの言語行為論の枠組みではうまく扱うことはできない.
そこで,発話では明示的ではない語用論的推論を必要とする命題を隠蔽された
命題と呼び,隠蔽された命題を示唆する行為を{\bf 隠蔽された命題行為}と呼
び,後者を新たな概念要素として導入する.





\subsection{意図}

了解は話し手の言語行為だけに対する反応ではない.話し手の意図に対する了
解もありうる.そのため{\bf 意図}という概念要素も導入する必要がある.

話し手の意図には{\bf 発話自体の意図}(意味論的推論で算出可能な意図)と,
{\bf 意図の意図}(語用論的推論を必要とする意図)が存在する.これらは了解
の対象として次元の異なるものであり,明確に区別すべきものである.したがっ
て,{\bf 意図}という概念要素は,これらの二つに分割した形で導入する.

\subsection{詳細化}

前節と同様に意味論的推論と語用論的推論という点からすると,発語媒介行為
には発話から直接わかる意味論的推論で算出可能な行為と,直接にはわからず
語用論的推論を必要とする行為がある.我々は,前者を{\bf 直接発語媒介行
為},後者を{\bf 間接発語媒介行為}と呼ぶ.これまでの言語行為論では,発
語媒介行為についてこのような推論形態による区別がされていない.
このような分化によって分析は詳細になる.

一方,発語媒介的効果には,{\bf 心の状態の変化},{\bf 言語行為},{\bf 
その場の行為},{\bf その後の行為}の四種類が考えられる.{\bf 心の状態の
変化}は,聞き手の聞き手なりの理解による信念の変化である.{\bf 言語行為}
は,一連の発話の場における話し手の発話に対する聞き手の発話である.典型
的には,話し手に対する返答である.{\bf その場の行為}は,一連の発話の場
でなされた言語行為以外の聞き手による行為である.その行為の後にも引き続
き発話の場が続く.それに対して,{\bf その後の行為}は,一連の発話の場を
離れた後に聞き手が行う行為(言語行為を含む)である.

これらは現象面からみた分類であり,発語媒介行為の分化と同様に,分析が詳
細になるという利点を持つ.

\subsection{言語行為論と拡張言語行為論の差異}

以上の概念要素を用いて従来の言語行為論を拡張し,構築した拡張言語行為論
の枠組みを表\ref{fwos}に示す.表中,太字の概念要素は,基本的な概念要素
を表す.なお,拡張言語行為論で新たに追加ないしは細分化された概念要素に
は*をつけてある.概説すると,{\bf 隠蔽された命題行為}と{\bf 意図}とい
う二つの概念要素を新たに加えるとともに,発語媒介行為を推論形態によって
直接発語媒介行為と間接発語媒介行為に区分し,発語媒介的効果を四種類に分
割して詳細化を試みた.

後にSearleが提唱した「間接発話行為」は,大まかに言うと,従来の発語媒介
行為にあたる.我々はその発語媒介行為を直接発語媒介行為と,間接発語媒介
行為に分けた.ここで,直接発語媒介行為は,発語内行為の帰結または結果と
してなされ,意味論的推論によって算出可能な行為である.一方,間接発語媒
介行為は,語用論的推論によって算出可能な行為であり,これがいわゆる間接
発話行為に相当する.



\section{拡張言語行為論の枠組みによる分析法}

表\ref{fwos}に現れる概念要素を{\bf 表形式}にし,それぞれの概念要素の内
容を述語によって書き下すことを,{\bf 拡張言語行為論の枠組みによる分析}
と呼ぶことにする.表中,{\bf 発語媒介行為},{\bf 発語媒介的効果},{\bf 
意図}以外の概念要素が分析に実際に用いられる概念要素である.

\begin{table}[htbp]
\begin{center}
\begin{tabular}{c|c|c}
\multicolumn{2}{c|}{話し手} & 聞き手\\	\hline
意図 & 言語行為 & 効果\\	\hline
& \fbox{{\bf 発話行為}} &\\
& 「発話」 &\\
& \fbox{{\bf 命題行為}} &\\
& 意味論的推論で算出可能な &\\
& 指示と述定 &\\
& \fbox{{\bf 隠蔽された命題行為}}* &\\
& 隠蔽された(語用論的推論を &\\
& 必要とする)命題を示唆する行為 &\\
& \fbox{{\bf 発語内行為}} & \fbox{\bf 発語内的効果}\\
& 陳述,疑問,命令など & 聞き手の理解\\
& \fbox{\bf 発語媒介行為} & \fbox{\bf 発語媒介的効果}\\
& 聞き手の行動,思考,信念 & 聞き手の側に実際に起こる\\
& などに影響を及ぼす行為 & 行為ないしは効果\\
& \fbox{直接発語媒介行為}* & \fbox{心の状態の変化}*\\
& 発話から直接わかる発語媒介行為 & \fbox{言語行為}*\\
& \fbox{間接発語媒介行為}* & \fbox{その場の行為}*\\
& 発話から直接わからない発語媒介行為 & \fbox{その後の行為}*\\
\fbox{\bf 意図}* &&\\
\fbox{発話自体の意図}* &&\\
\fbox{意図の意図}* &&\\
\end{tabular}
\caption{拡張言語行為論の枠組み
(Table \ref{fwos} The Framework of an Extended Speech Act Theory)}
\label{fwos}
\end{center}
\end{table}

以下では,簡単な発話を取り上げ,拡張言語行為論の枠組みによる分析例を示
す(表\ref{fwsaex},\ref{fwsaex2}).これによって,本枠組みが言語行為の分
析にどのように用いられるかを示すとともに,本枠組みの具体的なイメージを
明らかにする.なお,ここで示すのは{\bf 分析の一例}であり,他の分析も成
立しうる.拡張言語行為論の枠組みは本論文の主題ではないが,次節であいづ
ち「はい」による了解応答を語用論的に分析するための枠組みとして,重要な
位置付けにある.

\begin{table}[htbp]
\begin{center}
{\footnotesize
\begin{tabular}{l|l|l|l}
項番 & 1 & 2 & 3 \\  \hline
発話行為 & 窓を開けて! & 窓を開けてくれますか? & ここは暑い.\\
& (Open the window!) & (Will you open the window?) & (It's hot in here. )\\	\hline
命題行為 & $p_{1}$ = open(H, Window) & $p_{2}$ = open(H, Window) & $p_{3}$ = be(Here, Hot) \\  \hline
隠蔽された命題行為 & なし & なし & $p'_{3}$ = do(H, something) \\     \hline
発語内行為 & !($p_{1}$) & ?($p_{2}$) & $\vdash$($p_{3}$) \\      \hline
直接発語媒介行為 & let(S, H, $p_{1}$) & let(S, H, answer(H, $p_{2}$)) & let(S, H, know(H, $p_{3}$)) \\  
   \hline
間接発語媒介行為 & なし & let(S, H, $p_{2}$) & let(S, H, A)(A = $p'_{3}$) \\      \hline
発話自体の意図 & want(S, H, $p_{1}$) & want(S, H, know(H, $p_{2}$)) & want(S, H, know(H, $p_{3}$)) \\    \hline
意図の意図 & なし & want(S, H, $p_{2}$) & want(S, H, A)(A = $p'_{3}$)  \\   \hline
発語内的効果 & know(H, !($p_{1}$)) & know(H, ?($p_{2}$)) & know(H, $\vdash$($p_{3}$))\\  \hline
心の状態の変化 & think(H, $p_{1}$) & think(H, $p_{2}$) & think(H, $p'_{3}$)\\    \hline
言語行為 & なし & なし & なし \\        \hline
その場の行為 & $p_{1}$ & $p_{2}$ & $p'_{3}$ \\    \hline
その後の行為 & なし & なし & なし\\
\end{tabular}
}
\caption{拡張言語行為論の枠組みによる分析例-その1\\
(Table \ref{fwsaex}Some Examples of Analysis \\
with the Framework of an Extended Speech Act
Theory - Part 1)}
\label{fwsaex}
\end{center}
\end{table}

\begin{table}[htbp]
\begin{center}
{\footnotesize
\begin{tabular}{l|l|l|l}
項番 & 4 & 5 & 6 \\  \hline
発話行為 & これから銀行に行きます. & 5時ですよ. & じゃあ,明日行きます.\\
& (I'll go to the bank.) & (It's five o'clock. ) & (Then I'll go tomorrow.)\\ \hline
命題行為 & $p_{4}$ = go\_to(S, Bank) & $p_{5}$ = be(Time, five) &
$p_{6}$ = go\_to(S, Bank); \\  
& & & Time=Tomorrow \\ \hline
隠蔽された命題行為 & なし & $p'_{5}$ = be(Bank, Closed)  & なし \\     \hline
発語内行為 & $\vdash$($p_{4}$) & $\vdash$($p_{5}$) & $\vdash$($p_{6}$) \\      \hline
直接発語媒介行為 & let(S, H, know(H, $p_{4}$)) & let(H, S, know(S, $p_{5}$)) & let(S, H, know(H, $p_{6}$)) \\	\hline
間接発語媒介行為 & なし & let(H, S, know(S, $p'_{5}$)) & なし\\      \hline
発話自体の意図 & want(S, H, know(H, $p_{4}$)) & want(H, S, know(S, $p_{5}$)) & want(S, H, know(H, $p_{6}$)) \\    \hline
意図の意図 & なし & want(H, S, know(S, $p'_{5}$)) & なし\\   \hline
発語内的効果 & know(H, $\vdash$($p_{4}$)) & know(S, $\vdash$($p_{5}$)) & know(H, $\vdash$($p_{6}$))\\  \hline
心の状態の変化 & think(H, $p_{4}$) & think(S, $p_{5}$), think(S, $p'_{5}$) & think(H, $p_{6}$)\\    \hline
言語行為 & It's five o'clock. & Then I'll go tomorrow. & なし \\        \hline
その場の行為 & なし & なし & なし\\    \hline
その後の行為 & なし & go\_to(S, Bank); & なし\\
& & Time=Tomorrow & \\
\end{tabular}
}
\caption{拡張言語行為論の枠組みによる分析例-その2\\
(Table \ref{fwsaex2}Some Examples of Analysis \\
with the Framework of an Extended Speech Act
Theory - Part 2)}
\label{fwsaex2}
\end{center}
\end{table}
表\ref{fwsaex},\ref{fwsaex2}の中で,命題行為 $p_{i}$ は行為 (例えば 
open(H, Window)) か状態 (例えば be(Here, Hot)) を表す.また,S は話し
手,H は聞き手,A は何らかの行為を表す.さらに発語内行為の ''!'' は依
頼を,''?'' は疑問を,''$\vdash$'' は陳述を表す.また,表4の命題行為,
その後の行為に現れる'';'' の後の記述は,直前の述語が遂行される時刻を付
記するものである.以下,表\ref{fwsaex},\ref{fwsaex2}の内容について順次
説明する.ここで,分析対象はあくまで日本語であるが,述語表現(述語名お
よび引数)は慣例として英語表記が用いられることから,ここでも英語表記を
用いている.また,英語による述語表現を分かりやすくするために,発話行為
に妥当な英訳を付記してある.

1はいわゆる命令文である.話し手が{\bf 発話行為}として 「窓を開けて!」(Open the
window!)と言ったことについて分析している.{\bf 命題行為}は$p_{1}$ =
open(H, Window)となる.{\bf 隠蔽された命題行為}はない.{\bf 発語内行為}
は $p_{1}$ を依頼したこと,すなわち,!($p_{1}$)となる.{\bf 直接発語媒
介行為}はSがHに$p_{1}$をさせる,すなわち,let(S, H, $p_{1}$)となる.
{\bf 間接発語媒介行為}はない.{\bf 発話自体の意図}はSがHに$p_{1}$をし
て欲しい,すなわち,want(S, H, $p_{1}$)となる.この場合,{\bf 意図の意
図}はない.また,聞き手側の効果は,{\bf 発語内的効果}としては,Hが
$p_{1}$を依頼されたことがわかったこと,すなわち,know(H, !($p_{1}$))と
なる.発語媒介的効果の{\bf 心の状態の変化}はHが$p_{1}$をしようと思った
こと,すなわち,think(H, $p_{1}$)となる.聞き手が何も返事をしなかった
場合をここでは想定しているので,{\bf 言語行為}はない.{\bf その場の行
為}は$p_{1}$になる.最後に{\bf その後の行為}はない.ここで,述語
knowは外界からの情報を単に認知あるいは理解したことを,述語thinkは主体
内部の情報だけからの推論によって主体の信念が変化したことを表す.

2は疑問文の形をとった依頼を表す.さらにこの場合はコミュニケーションが
首尾よく,欠陥なく進行した場合を想定する.話し手が{\bf 発話行為}として 
「窓を開けてくれますか?」(Will you open the window?)と言ったことについて分析している.{\bf 
命題行為}は $p_{2}$ = open(H, Window)となる.{\bf 隠蔽された命題行為}
はない.{\bf 発語内行為}は$p_{2}$を質問したこと,すなわち,?
($p_{2}$)となる.{\bf 直接発語媒介行為}はSがHに$p_{2}$と答えさせる,す
なわち,let(S, H, answer(H, $p_{2}$))となる.{\bf 間接発語媒介行為}はS
がHに$p_{2}$をさせる,すなわち,let(S, H, $p_{2}$)となる.{\bf 発話自
体の意図}はSがHに$p_{2}$を知って欲しい,すなわち,want(S, H, know(H,
$p_{2}$))となる.{\bf 意図の意図}は,SがHに$p_{2}$して欲しい,すなわち,
want(S, H, $p_{2}$)となる.また,聞き手側の効果は,{\bf 発語内的効果}
としては,Hが$p_{2}$を質問されたことがわかったこと,すなわち,know(H, ?
($p_{2}$))となる.発語媒介的効果の{\bf 心の状態の変化}はHが$p_{2}$をし
ようと思ったこと,すなわち,think(H, $p_{2}$)となる.聞き手が何も返事
をしなかった場合をここでは想定しているので,{\bf 言語行為}はない.{\bf 
その場の行為}は$p_{2}$になる.最後に{\bf その後の行為}はない.

上記の分析例で,我々は,質問を依頼であるとする推論を意味論的推論ではな
く,語用論的推論と考えた.従来,意味論的推論と語用論的推論の境界線は曖
昧であった,我々は,この種の推論を語用論的推論の方へ分類することにした.

3は「ここは暑い」と言うことによって,窓を開けて欲しいなど,何かをして
欲しいことを知らせる場合である.話し手が{\bf 発話行為}として「ここは暑
い」(It's hot in here.)と言ったことについて分析している.{\bf 命題行
為}は $p_{3}$ = be(Here, Hot)となる.{\bf 隠蔽された命題行為}はHが何か
をすること,すなわち,$p'_{3}$ = do(H, something)となる.{\bf 発語内行
為}は $p_{3}$ を陳述したこと,すなわち,$\vdash$($p_{3}$)となる.{\bf 
直接発語媒介行為}はSがHに$p_{3}$であることを知らせる,すなわち,let(S,
H, know(H, $p_{3}$))となる.{\bf 間接発語媒介行為}はSがHに何かAをさせ
る,すなわち,let(S, H, A)(A=$p'_{3}$)となる.{\bf 発話自体の意図}はS
がHに$p_{3}$であることを知って欲しい,すなわち,want(S, H, know(H,
$p_{3}$))となる.{\bf 意図の意図}は,SがHに何かAをして欲しい,すなわち,
want(S, H, A)(A=$p'_{3}$)となる.また,聞き手側の効果は,{\bf 発語内的
効果}としては,Hが$p_{3}$を陳述したことがわかったこと,すなわち,
know(H, $\vdash$($p_{3}$))となる.発語媒介的効果の{\bf 心の状態の変化}
はHが$p'_{3}$をしようと思ったこと,すなわち,think(H, $p'_{3}$)となる.
聞き手が何も返事をしなかった場合をここでは想定しているので,{\bf 言語
行為}はない.{\bf その場の行為}は$p'_{3}$になる.最後に{\bf その後の行
為}はない.

4,5,6は一連の会話の例である.まず,話し手(S)が「これから銀行に行き
ます」と言ったのに対して,聞き手(H)が「5時ですよ」と指摘して,銀行が既
に閉まっていることを示唆し,結局,最初の話し手(S)は「じゃあ,明日行きます」と
言った場合である.

4は話し手が{\bf 発話行為}として「これから銀行に行きます.」(I'll go to the bank.)と言ったことに
ついて分析している.{\bf 命題行為}は$p_{4}$ = go\_to(S, Bank)となる.
{\bf 隠蔽された命題行為}はない.{\bf 発語内行為}は$p_{4}$を陳述したこ
と,すなわち,$\vdash$($p_{4}$)となる.{\bf 直接発語媒介行為}はSがHに
$p_{4}$を知らせる,すなわち,let(S, H, know(H, $p_{4}$))となる.{\bf 
間接発語媒介行為}はない.{\bf 発話自体の意図}はSがHに$p_{4}$を知って欲
しい,すなわち,want(S, H, know(H, $p_{4}$))となる.{\bf 意図の意図}は
ない.また,聞き手側の効果は,{\bf 発語内的効果}としては,Hが$p_{4}$を
陳述したことがわかったこと,すなわち,know(H, $\vdash$($p_{4}$))となる.
発語媒介的効果の{\bf 心の状態の変化}はHが$p_{4}$だと思ったこと,すなわ
ち,think(H, $p_{4}$)となる.{\bf 言語行為}は引き続きなされた発話,す
なわち,「5時ですよ.」(It's five o'clock.)となる.{\bf その場の行為}と{\bf その後の行
為}はない.

5では,{\bf 発話行為}として「5時ですよ.」(It's five o'clock.)と言ったことについて分析
している.{\bf 命題行為}は$p_{5}$ = be(Time, five)となる.{\bf 隠蔽された
命題行為}は銀行が既に閉まっていること,すなわち,$p'_{5}$ = be(Bank,
Closed)となる.{\bf 発語内行為}は$p_{5}$ を陳述したこと,すなわち,
$\vdash$($p_{5}$)となる.{\bf 直接発語媒介行為}はHがSに$p_{5}$を知らせ
る,すなわち,let(H, S, know(S, $p_{5}$))となる.{\bf 間接発語媒介行為}
は既に銀行が閉まっていることを知らせる,すなわち,let(H, S, know(S,
$p'_{5}$))となる.{\bf 発話自体の意図}はHがSに$p_{5}$であることを知っ
て欲しい,すなわち,want(H, S, know(S, $p_{5}$))となる.{\bf 意図の意
図}は,HがSに$p'_{5}$を知って欲しい,つまり,want(H, S, know(S,
$p'_{5}$))となる.また,聞き手側の効果は,{\bf 発語内的効果}としては,
Sが$p_{5}$を陳述したことがわかったこと,すなわち,know(S,
$\vdash$($p_{5}$))となる.発語媒介的効果の{\bf 心の状態の変化}はSが
$p_{5}$と$p'_{5}$であると思ったこと,すなわち,think(S, $p_{5}$)と
think(S, $p'_{5}$)になる.{\bf 言語行為}は引き続きなされた発話,すなわ
ち,「じゃあ,明日行きます.」(Then I'll go tomorrow.)となる.{\bf その場の行為}はない.{\bf 
その後の行為}は明日,銀行へ行くこと,すなわち,go\_to(S,
Bank);Time=Tomorrowとなる.

6では,{\bf 発話行為}として「じゃあ,明日行きます.」(Then I'll go tomorrow.)と言ったことにつ
いて分析している.{\bf 命題行為}は$p_{6}$ = go\_to(S,
Bank);Time=Tomorrowとなる.{\bf 隠蔽された命題行為}はない.{\bf 発語内
行為}は$p_{6}$ を陳述したこと,すなわち,$\vdash$($p_{6}$)となる.{\bf 
直接発語媒介行為}はSがHに$p_{6}$を知らせる,すなわち,let(S, H,
know(H, $p_{6}$))となる.{\bf 間接発語媒介行為}はない.{\bf 発話自体の
意図}はSがHに$p_{6}$を知って欲しい,すなわち,want(S, H, know(H,
$p_{6}$))となる.{\bf 意図の意図}はない.また,聞き手側の効果は,{\bf 
発語内的効果}としては,Hが$p_{6}$を陳述したことがわかったこと,すなわ
ち,know(H, $\vdash$($p_{6}$))となる.発語媒介的効果の{\bf 心の状態の
変化}はHが$p_{6}$だと思ったこと,すなわち,think(H,$p_{6}$)となる.
{\bf 言語行為}と{\bf その場の行為}と{\bf その後の行為}はない.

\section{「はい」による了解応答の分析}

主としてここでは,拡張言語行為論の枠組みを用いて,了解応答としてのあい
づち「はい」について一般的な分析を行うとともに,その分析を踏まえて{\bf 
「はい」による了解の程度と過程}の分析を行い,これらの分析から得られた
知見を述べる.

\subsection{一般的な分析}
\label{yesspeechact}

まず,S(最初の発話者)の発話に対するH(最初の発話者Sの聞き手)のあい
づちとしての「はい」を,了解応答とみなして,前節の拡張言語行為論の枠組
みを用いて分析してみる(表\ref{yesspeech})\footnote{表中に含まれていな
い概念要素に関する分析結果は「なし」であり,それらは省略してある.}.H
の{\bf 発話行為}として「はい」がなされたとき,{\bf Hの隠蔽された命題行
為}$p_{H}$は,HがSのいずれかの言語行為あるいは意図をわかったことを指す.
すなわち,$p_{H}$=know(H, $p'_{S}$)($p'_{S}$はSのいずれかの言語行為あ
るいは意図)ということになる.Hの{\bf 発語内行為}はSの{\bf 命題行為}
$p_{S}$に対するあいづちであるので,chimeIn(H, $p_{S}$)となる.{\bf 間
接発語媒介行為}は,HがSに$p_{H}$を知らせる,すなわちlet(H, S, know(S,
$p_{H}$))となる.最後にHの{\bf 発話自体の意図}は,Hが$p'_{S}$を知った
こと(つまり,$p_{H}$)をSに知って欲しい,すなわちwant(H, S, know(S,
$p_{H}$))となる.人間はこれらのことを一瞬にしてやってのける(あるいは指
摘されればわかる)が,分析してみると上記のようになる.

\begin{table}[htbp]
\begin{center}
\begin{tabular}{r|l}
言語行為/意図 & 分析内容\\	\hline\hline
Hの発話行為 & 「はい」\\	\hline
Hの隠蔽された命題行為 & $p_{H}$=know(H,$p'_{S}$)\\
& $p'_{S}$はSのいずれかの言語行為あるいは意図\\	\hline
Hの発語内行為 & chimeIn(H, $p_{S}$)\\	\hline
Hの間接発語媒介行為 & let(H, S, know(S, $p_{H}$))\\	\hline
Hの発話自体の意図 & want(H, S, know(S, $p_{H}$))\\
\end{tabular}
\caption{拡張言語行為論の枠組みを用いた「はい」による了解応答の一般
的分析 \\
(Table \ref{yesspeech} A General Analysis of the ``Uptake''
through ``Hai'' \\
in Japanese with the Framework of an Extended Speech
Act Theory)}
\label{yesspeech}
\end{center}
\end{table}

「はい」はこれまで概念的に未分化な状態にあった.しかし上述のような分析
によって,「はい」を言語行為という観点から概念的にある程度分化させるこ
とができた.また,「はい」によって行われる各言語行為の内容と,言語行為
間の関係とが分かった.これによって,ある程度まで鍵穴認識(つまり第三者
による観察)が可能になった.しかし一方で,Hの{\bf 隠蔽された命題行為}における
$p'_{S}$の指示対象が具体的に決定困難であることが,鍵穴認識の難しさの原
因になっていることが分かった.これまで鍵穴認識の難しさについては抽象的
な議論しか行われてこなかったが,具体的な問題点を明らかにしたことになる.


\subsection{了解の程度と過程}

表\ref{yesspeech}においてHの{\bf 隠蔽された命題行為}における $p'_{S}$ の指示
対象が何であるかによって,H の{\bf 隠蔽された命題行為} $p_{H}$ =know(H,
$p'_{S}$) は以下のように多様に解釈できる.

\begin{description}
\item [(1) $p'_{S}$がSの発話行為] いわゆるあいづちであり,聞いていると
いうことを意味する.
\item [(2) $p'_{S}$がSの命題行為] Sの言っている命題内容がわかったこと
を意味する.
\item [(3) $p'_{S}$がSの発語内行為] Sの発話が,約束,依頼/命令,陳述,
疑問,感謝,助言,警告,あいさつなどであること(Sの発語内行為)がわかっ
たことを意味する.
\item [(4)$p'_{S}$がSの直接発語媒介行為] Sの直接発語媒介行為がわかっ
たことを意味する.
\item [(5)$p'_{S}$がSの隠蔽された命題行為] Sの隠蔽され
た命題行為がわかったことを意味する.
\item [(6) $p'_{S}$がSの間接発語媒介行為] Sの間接発語媒介行為がわかっ
たことを意味する.
\item [(7) $p'_{S}$がSの発話自体の意図] Sの発話自体の意図がわかったこ
とを意味する.
\item [(8) $p'_{S}$がSの意図の意図] Sの意図の意図がわかったことを意味
する.
\end{description}



以上のように,拡張言語行為論の枠組みを用いた「はい」による了解応答の語
用論的分析に限れば,$p_{H}$の多様性は上記の八種類に限定できる.このよ
うな多様性がHの了解の {\bf 程度}に対応するものと考えられる.したがって,
了解の程度には八つの段階があるということになる.$p'_{S}$の指示対象に依
存して了解の程度が変わり,上述の(1)から(8)に向けて了解の程度が高くなる
ことは明らかであろう.ここで$p'_{S}$は,H側が了解した内容に相当すると
いう意味で,{\bf 了解内容}と呼ぶ.

これまで了解の程度については詳細な議論がなされていなかったが,これによっ
て了解の程度を考察する枠組みを明らかにすることができた.しかし,前節で
も述べたように$p'_{S}$の指示対象が具体的に決定困難であることから,了解
の程度がどの段階にあるかを客観的に特定するのは一般に困難であるという問
題は残されている.

一方,了解の{\bf 過程}には発話の確認,命題行為の理解,発語内行為の理解,
意図の理解,納得,言語行為,行為と七段階が考えられる.
ここで,意図の理解とは単に発話者の意図(発話自体の意図あるいは意図の意
図)を理解したものであり,納得とは発話者の意図(発話自体の意図あるいは意
図の意図)に同意することである.さらに,納得を前提として,発話に対する
答え(言語行為)がなされ,さらに具体的な行為がなされる.

これらの各段階は,典型的には,前段階の通過を前提とするものである.した
がって,各段階が観察可能であれば(実際,発話者間ではお互いがどの段階に
あるかを確認することは可能である),上述した了解の過程はさらなる了解に
向けての次なる段階を示唆するものであり,会話において了解を促進させる手がかりとなり
うる.

またこれまでは,了解について明確なモデルがないまま鍵穴認識の研究が行わ
れていた.本論文では,拡張言語行為論という枠組みを用いて,了解の程度と
過程のモデルを構築した.これは鍵穴認識の領域におけるひとつの前進といえ
る.

\section{むすび}

本論文では,
了解の程
度と過程を分析するために拡張言語行為論の枠組みを与えた.これによって,
了解の語用論的分析を可能にした.

本枠組みは,Searleの言語行為論の枠組みを拡張したものであり,{\bf 隠蔽
された命題行為}と{\bf 意図}という概念を新たに導入している.また大まか
に述べれば,{\bf 意図}を二種類に,{\bf 発語媒介行為}を二種類に,{\bf 
発語媒介的効果}を四種類にそれぞれ分類し,言語行為論を詳細化している.
さらに本枠組みは,了解の語用論的分析を可能にしたという点で,分析力を高
めたものである.

また,拡張言語行為論の枠組みを利用してあいづち「はい」による了解の程度
と過程について分析し,いくつか有用な知見を得た.得られた知見を要約する
と以下の通りである.了解の程度には八つの段階が,了解の過程には七つの段
階が識別できる.これまでは了解の程度については詳細な議論がなされていな
かったが,本分析によって了解の程度を考察する枠組みを与えることができた.
しかし一般には,第\ref{yesspeechact}節でも述べたように$p'_{S}$(発話者S
のいずれかの言語行為あるいは意図)の指示対象が具体的に決定困難であるこ
とから,了解の程度がどの段階にあるかを客観的に特定するのは困難であると
いう問題は残されている.次に,これまでは,了解の程度と過程が不明確なが
らも鍵穴認識が行われていた.了解の程度と過程を明らかにしたことは,鍵穴
認識の領域におけるひとつの前進といえる.

拡張言語行為論の枠組みは,それによる分析例(表\ref{fwsaex2})で示した
ように,あいづち「はい」による了解だけではなく,他の発話による了解の分
析にも使える.そのため,例えば「鍵穴認識」で何が困難なのかが判明する可
能性がある.


最後に,了解の程度に関するモデルを利用した発展的研究の可能性について述
べる.Austinは,発語内行為を判定宣言型,権限行使型,行為拘束型,態度表
明型,言明解説型の五つの型に分けている\cite{austin}.それぞれの型に対
応する発話と,その発話に対するあいづち「はい」とを対にした対話例をいく
つか用意する.対話例は,紙に書かれたもの,音声のみで記録されたもの,ビ
デオテープに記録されたものが考えられる.また,各対話例について,了解の
程度に対応して了解内容を明記した一覧表を用意する.そして,各対話例とそ
れに対応する了解内容を複数の被験者に示し,発話「はい」がどの了解内容を
指しているように見えるかを回答させる.ある発語内行為の型に対して回答の
ばらつきが大きいならば,その型は,発話「はい」による了解内容が多様で曖
昧になると判断できる.こうした発展的研究は,以下のようにユーザインタフェー
スの改善に貢献し得る.

人間が計算機に対して話しかけ,それに応じて計算機が「はい」と答える状況
を考える.こうした状況では,計算機の「はい」による了解内容が曖昧になる
ことがある.言い換えれば,「はい」が単なるあいづちであるのか,それより
も上位の了解を意味するのかが,人間には分からない場合がある.その場合に
は,計算機側に「はい」以外の,人間に了解内容が伝わるような発話を必要と
する.計算機の「はい」による了解内容が曖昧になる場合を上述したような発
展的研究を通じてあらかじめ知っておくことにより,会話の正確さと円滑さを
保つことができる.

了解の程度と過程のモデルを利用したこのような研究を行うことが,今後の研
究課題である.



\bibliographystyle{jnlpbbl}
\bibliography{final}

\begin{biography}
\biotitle{略歴}
\bioauthor{土井 晃一}{1961年生.
1991年東京大学工学部情報工学専攻博士課程修了.
工学博士.
同年富士通研究所国際情報社会科学研究所入社(現富士通研究所コンピュータシス
テム研究所).
自然言語理解,人工知能,ソフトウエア工学,情報検索などの研究に従事.
1998年9月より1999年10月まで文部省学術情報センター客員助教授併任.
現在は,(株)セレスター・レキシコ・サイエンシズに勤務.
日本認知科学会,情報処理学会,人工知能学会,ソフトウエア科学会,言語処理学
会各会員.}
\bioauthor{大森 晃}{1954年生.
1985年広島大学大学院工学研究科博士課程後期修了(システム工学専攻).
工学博士.
1982年9月より1年間ケースウェスタンリザーブ大学客員研究員.
1985年4月より富士通国際情報社会科学研究所に勤務.
1993年10月より東京理科大学工学部第二部経営工学科助教授.
ソフトウエア工学,品質管理,ユーザ・インタフェース,自然言語理解などの研究に従事.
IEEE ComputerSociety,ACM,日本品質管理学会,計測自動制御学会,情報処理学会各会員.}
\bioreceived{受付}
\biorevised{再受付}
\biorerevised{再々受付}
\bioaccepted{採録}
\end{biography}

\end{document}
