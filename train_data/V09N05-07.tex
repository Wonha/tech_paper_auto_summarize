\documentstyle[epsf,jnlpbbl]{jnlp_j_b5}

\setcounter{page}{131}
\setcounter{巻数}{9}
\setcounter{号数}{5}
\setcounter{年}{2002}
\setcounter{月}{10}
\受付{2001}{11}{6}
\再受付{2002}{6}{11}
\採録{2002}{7}{17}

\setcounter{secnumdepth}{2}

\title{自然言語処理技術を用いた大会プログラム作成支援について}
\author{小作 浩美\affiref{CRL}\affiref{NAIST} \and 内山 将夫\affiref{CRL}
\and 村田 真樹\affiref{CRL} \and 内元 清貴\affiref{CRL} \and 井佐原 均\affiref{CRL}}

\headauthor{小作,内山,村田,内元,井佐原}
\headtitle{自然言語処理技術を用いた大会プログラム作成支援について}

\affilabel{CRL}{通信総合研究所}
{Communications Research Laboratory}
\affilabel{NAIST}{奈良先端科学技術大学院大学}
{Nara Institute of Science and Technology}

\jabstract{
2000年言語処理学会第\ 6回年次大会プログラムの作成において,
言語処理技術を適用し,大会プログラムを自動作成することを
試みた.
本稿では,第\ 5回大会のデータを利用して,大会プログラム作成のために
行なった一連の実験について説明する.その結果に基づき,実際に第\ 6回の
大会プログラムを作成した手続きについて報告する.大会プログラム作成に
キーワード抽出および文書分類の言語処理技術は十分に利用でき,事務手続き
の効率化に貢献できることを報告する.また,大会終了後のアンケート
調査の結果を示し,参加者からの評価についても報告する.
}

\jkeywords{キーワード抽出,自動分類}

\etitle{Supporting Conference Program Production \\
Using Natural Language Processing Technologies}
\eauthor{Hiromi itoh Ozaku\affiref{CRL}\affiref{NAIST} 
\and Masao Utiyama\affiref{CRL} \and Masaki Murata\affiref{CRL} 
\and Kiyotaka Uchimoto\affiref{CRL} \and Hitoshi Isahara\affiref{CRL}}


\eabstract{
We applied natural language processing technologies
to automatically produce a program for the sixth annual meeting of the
Association for Natural Language Processing. In this paper, we 
describe experiments used to automatically generate 
the program using the fifth annual meeting data. 
We produce the sixth annual meeting program on the basis of the experiments. 
We report the process of making the sixth annual meeting program 
in practice and show to what extent the natural language processing 
technologies are efficient for this task. Furthermore, we show the results 
of a questionnaire targeting the participants of the sixth annual meeting.
}

\ekeywords{keyword extraction, automatic classification}

\begin{document}

\maketitle
\thispagestyle{empty}
\section{はじめに}
インターネットの普及により,電子化されたテキストの入手が容易に
なってきた.それらのテキストをより効率的かつ効果的に利用
するために,多くの言語処理技術が研究,提案されてきている.
それに伴い,言語処理の研究分野は注目を浴び,言語処理学会でも
年々会員が増加し,事務作業が増加する傾向にある.
このような増加傾向から考えると,今の言語処理学会の状況では,事務処理
の負担が処理能力を越えてしまい,その結果,事務作業が滞ることが
予想される.もし,事務作業が滞れば,学会の活気や人気に水をさすことに
なってしまう可能性があり,その結果,学会の将来に悪影響を与えると
考えられる.そのため,事務処理の効率化は必須である.

学会の差別化,効率化を図るため,電子化された投稿論文の
査読者への割り当てを行なう際に言語処理技術を利用した
報告が出てきている.例えば,投稿論文を最適な査読者に
割り当てることを試みたもの\cite{Susan1992,Yarowsky1999}
などである.ただし,これらの論文は適切な査読者を決定することを
目的としているだけであり,事務処理の効率化については論じられていない.

そのような中で,2000年言語処理学会第\ 6回年次大会プログラムを
作成する機会を得た.大会プログラム作成において作業効率向上に
寄与する言語処理技術を明確にすることを目的として,いくつかの言語処理
技術を用いて第\ 5回大会の講演参加申込データに対して,大会プログラム
自動作成実験を行い,それらの技術の有効性を比較した.そして,
その実験結果を基に第\ 6回年次大会プログラム原案を作成した.

大会プログラムを作成するには,講演参加申込を適当なセッションに
分割し,セッション名を決める作業が必要である.それには,講演参
加申込の内容(タイトルとアブストラクト)をすべて確認してから
作成作業をするのが一般的であるが,講演数が増加している現在,講演
申込内容をすべて確認し,大会プログラムを手動で作成するのは大変
な作業である.この作業を省力化するために,アブストラクトは
読まずに,タイトルだけを利用し,大会プログラムを作成することも
可能であると考えられるが,タイトルだけを利用した場合,たとえ
講演参加申込に記述されている講演分野を利用したとしても
適切なセッションに割り当てられない場合が存在すると考えられる.
また,タイトルだけでは,適切なセッション名を決めることも困難である.
我々はそのような作業を支援し,効率化する方法を本稿で提案する.我々の手法を
利用すれば,大会の発表傾向にあったセッション名を
決定できるだけでなく,適切なセッションに講演申込を割り振ることも
可能となる.そのため,事務作業の負担を軽減することが可能と
なるだけでなく,講演者の興味にあうセッションを作成できる.

以下,\ref{yaya}章でその一連の実験について報告する.\ref{gogo}章で
第\ 6回年次大会プログラム作成の詳細について説明する.そして,
\ref{haha}章で大会後に行なったアンケート調査の結果を報告し,
\ref{mumu}章で今後の大会プログラム作成の自動化および事務処理の
効率化に向けた考察を行なう.

\section{大会プログラム作成実験}
\label{yaya}
大会プログラムとは,大会参加講演の集合を,類似した内容の講演のグループ
(セッション)に分類し,そのグループを時間と会場の条件に合わせて
割り振った講演発表一覧である.

大会プログラムを作成するには,以下のような手続きが必要である.

\begin{itemize}
\item[A] 講演参加申込を収集しデータベース化する.
\item[B] 講演参加申込を類似性に基づいて分割する.
\item[C] 各類似講演グループにセッション名をつける.
\item[D] 講演グループの講演数や会場数や時間配分を調整する.
\end{itemize}

上記の手続きの作業順は決定していない.例えば,講演会場の情報が前もって
判っていれば,会場数からセッション数(講演グループ数)あるいは,1セッション
あたりの講演数が決定可能であるし,会場情報がない場合はセッション名を
決定してから講演分類を行ない,後で講演数の調整を行なうことも可能である.
作業順はどうであれ,大会プログラムを作成するには,講演参加申込を適当な
グループに分類し,各グループに適切なセッション名
を付けることが必要である.

本稿では,上記作業の\ B,Cに特に注目し,大会プログラムを作成する
課題を講演参加申込(文書)の自動分類および自動ラベリング課題と
捉え,既存の文書分類技術がどれほど利用できるかを明らかにするための実験を
行なった.
この課題においては,以下の\ 2つのアプローチのいずれかを取ることができる.
\begin{itemize}
\item[1] 講演参加申込を分類し,分類された講演グループからセッション名を決める.
\item[2] セッション名を決定し,セッション名に合うように講演参加申込を分類する.
\end{itemize}

どちらのアプローチにおいても,1セッションあたりの講演参加申込数が
ほぼ同数に分類され,分類されたグループの講演タイトルなどからセッション名を
連想できることが必要である.なぜなら,適当な分類ができても,そこ
からセッション名を
連想できなければ良いプログラムができたとは言えないし,講演タイトルから
セッション名が連想できたとしても,各セッションの講演数がバラバラであれ
ば大会運営からして良いプログラムを作成できたとは言い難いからである.

我々はそれぞれのアプローチについて以下のような実験を行なった.
本稿では,それらの実験の結果について述べる.

\begin{itemize}
\item[1] 講演参加申込を分類後,セッション名を決める手続き;
\begin{itemize}
\item クラスタリング手法を用いた分類
\end{itemize}
\item[2] セッション名決定後,講演参加申込を分類する手続き;
\begin{itemize}
\item 学習アルゴリズムを用いた分類
\item キーワード抽出・類似検索法を用いた分類
\end{itemize}
\end{itemize}

\subsection{クラスタリング手法を用いた分類}
\label{class}
まず,クラスタリング手法で,ある程度の講演参加申込のグルーピングができ,各
グループに含まれる講演参加申込数がそれほど大きく違わず,なおかつ,その
グループに適当なセッション名をつけることができれば,そのグ
ループに基づいて大会プログラム原案を簡単に作成できるのではないかと考えた.

クラスタリング手法としては,トップダウンの手法\cite{Tanaka1997}と
ボトムアップの手法\cite{BakerAndMcCallum1998}を用いた.どちらも
第\ 5回の講演参加申込データを利用し分類実験を行なった.
この方法では,セッション名を前もって決めることをせずに分類し,それ
ぞれ分類されたグループからセッション名を抽出してプログラムを作成する.

トップダウンクラスタリングでは,論文集合を分割していくというこ
とを,論文集合の大きさが\ 1になるまで再帰的に繰り返す.このとき,
分割の際に,出現頻度の分散が最大の単語に着目し,その単語の出現
頻度がある一定以上のものと一定以下のものに集合を分割する
\cite{Tanaka1997}.

一方,ボトムアップクラスタリングにおいては,各クラスタ間の距離を比べ
て,最も近いもの同士をくっつけていくということをクラスタ数が1に
なるまで繰り返す.このときの距離として,ダイバージェンス(K-L情
報量)を利用する\cite{BakerAndMcCallum1998}.

2つのクラスタリング手法では,どちらも何らかの意味概念を持ったグループに
分類されたが,各グループに含まれる申込数に大きなバラツキがあった.また,
第\ 5回のプログラムのセッションと分類されたグループとを
比較すると,図\ref{top}\footnote{この図は,トップダウン手法で得られた
文書集合が\ 1になった結果の一部分を示している.数字は申込番号を示し,
英文字は申込者が記入した講演分野である.また,[\ ]内は第\ 5回で割り当て
られていたセッション名であり,この実験での正解として扱っている
セッション名である.理想的な解であれば,文書集合が\ 1に
なった時の近隣のタイトルは第\ 5回大会の同名のセッションに分類されて
いるはずである.ここでは結果として,別のセッションに分けられた
講演タイトルが近隣に来ており,近隣をまとめてグループとした際にそのグループ
にセッション名としてのグループ名を与えることが困難であった.なお,
この実験の際には,セッション中の順序については考慮していない.}
で示すように,大きく異なっていた.さらに,グル
ープの内容からセッション名を決定することも難しかった.
そのため,これらのクラスタリング手法は採用しなかった.

\begin{figure}
\begin{center}
 \epsfile{file=topdown3b.epsf}
\end{center}
\caption[図]{\label{top}トップダウン手法の結果例}
\end{figure}

\subsection{学習アルゴリズムを用いた分類}
\label{saidai}
もし,第\ 6回大会が第\ 5回大会と同様な話題の講演傾向で
あれば,セッション名は変化していないと考えられる.
そのような場合,学習アルゴリズムを用いて,第\ 5回大会のセッション名と
それへの講演参加申込の割りつけの傾向を学習し,その学習結果を利用することで
大会プログラムを作成できる.ここ
では,学習アルゴリズムとして最大エントロピー法を利用した.

最大エントロピー法を用いた分類とは,各講演参加申込が各セッションに
割り当てられる確率を最大エントロピー法により学習し,その確率が
最大になるセッションに講演参加申込を割り振るというものである.
この方法を用いた文書の自動分類の研究には
文献\cite{Inui1998,Nigam1999}がある.

本稿では,第\ 5回の大会プログラムと講演参加申込データより,
セッション名と講演参加申込との関係を学習し,セッション名は
第\ 5回で利用されたものを用いて,第\ 6回の講演参
加申込データを確率が最大となるセッションに分類した.

\begin{table*}
\caption[表]{\label{meout} 分類確率の高い結果}
\begin{center}
\scriptsize
\begin{tabular}{|c|c|c|c|l|}\hline
セッション名& 確率 & 登録番号 & 分野 & \multicolumn{1}{|c|}{タイトル}\\\hline
機械翻訳 & 0.989 & 90 & d & 語彙化されたツリーオートマトンに基づく会話文翻訳システム \\\hline
機械翻訳 & 0.972 & 37 & d & 英日・日英機械翻訳の実力 \\\hline 
機械翻訳 & 0.887 & 50 & d & 日英機械翻訳における名詞の訳語選択\\\hline
検索 & 0.839 & 38 & d & 情報検索における絞り込み語提示による検索者支援の試み\\\hline
検索 & 0.830 & 97 & d & 科学論文における要旨—本文間のハイパーリンク自動生成 \\\hline
検索 & 0.735 & 9 & d & 用例利用型翻訳のための類似用例検索手法 \\\hline
\end{tabular}
\end{center}
\end{table*}

\begin{table*}
\caption[表]{\label{memondai} 分類確率の低い結果}
\vspace*{-3mm}
\begin{center}
\scriptsize
\begin{tabular}{|c|c|l|c|}\hline
 登録番号 & 分野 & \multicolumn{1}{|c|}{タイトル} & セッション名(確率) \\\hline
25 & b & 辞書定義文を用いた複合語分割 & 分類・他 (0.163) タグ付け(0.152) 辞書(0.126) \\\hline
18 & c,d & GDAタグを利用した複数文書の要約 & 分類・他(0.237) 言語モデ(0.179) 抽出(0.156) \\\hline
\end{tabular}
\end{center}
\end{table*}

最大エントロピー法では確率を学習する際にデータを表現するための
素性が必要である.学習に
用いる素性としては,分類すべき講演参加申込のタイトルとアブストラクトを
JUMAN\cite{KurohashiAndNagao} で形態素解析して形態素列に
分解し,その形態素のうち名詞のみを取り出して用いた.
また,タイトルに現れたキーワードは特に重要と考え,上記の
素性とは別にタイトルから得られた形態素は別個の素性として用いた.
さらに,申込書に記述された講演分野も素性の一つとして学習した.
講演分野とは講演申込時に申込者が関係する分野として
選択するものである.講演分野として選択肢はaからeまで存在し,aは音韻論,
形態論,構文論など,bは計算辞書学,ターミノロジー,テキストデータ
ベースなど,cは言語処理アルゴリズム,解析・生成システム,対話理解など,
dはワードプロセッサ,機械翻訳,情報検索,対話システムなど,eはその他
となっている\footnote{これら講演分野について
は,第\ 7回大会の講演参加申込までは特別セッション以外,選択肢に
変更がなかったが,第\ 8回大会の申込から,より
詳細な区分に変更されている.}.
なお,実際に利用した素性の種類の数は1,818個であった.

この方法により,第\ 5回のデータで学習し第\ 6回のデータを分類して得た結果
の一部を表\ref{meout}と表\ref{memondai}に示す.表\ref{meout}に
示すように,例えば「機械翻訳」や「検索」のセッションに割り振られる
確率の高いものは,比較的良い結果となっている.
しかしながら表\ref{memondai}に示すようにセッションに割り振られる
確率が低いものについては,確率最大のものと
それに続く確率のものとの間に目立った差が存在せず,どのセッションに
含まれるべきか判断に困るような結果であったり,分類
自体全く見当外れであったりした.

これは,アブストラクトのような文字制限があるところで,講演内容を
幅広く記述したため曖昧になっている場合や新たな分野への研究である場合に低い
確率になると考えられる.また,講演分野の記述の無い申込や複数の分野を
指定されたものがあったが,これらの場合には申込の講演分野は学習素性と
して信頼性のない素性となる\footnote{この実験で用いた講演分野は,第\ 7回
大会まで数年にわたって利用され変更されていない.そのため,実際の講演内容
とあう講演分野が無かった可能性も高い.}.
利用できる情報が少ない状態で,信頼性のない素性を利用することは
結果に悪い影響を与える.利用できる情報が少ない場合は,利用できる
情報の信頼性ができる限り高い必要がある.

さらに,この手法では,申込の分類先を学習データと一致させる必要があり,分
類先が固定化される問題がある.この問題は,最大エントロピー法に
限ったものではなく,教師あり機械学習手法で分類を行なうときには
必ず生じる問題である\footnote{この問題については,次のような改善方法が
あり得る.まず,いずれかのセッションに割り振られる確率の高い講演と
どのセッションにも割り振られる確率の低い講演に分割する.
割り振られる確率の高いものは,言
語処理学会の定番的な研究の内容の講演であると考え,それらを学習で
利用したセッションへ分類する.割り振られる確率の低い講演については,
今回\ref{hoho}節で提案するような手法により,新たなセッションを
生成し,それらに分類することで分類先が固定化されることなく,大会
プログラムを作成することが可能である.ただし,定番的な研究内容と
新たなものを分離するための具体的な閾値を決定する方法などを
検討する必要がある.}.

上記の理由により,この手法を採用しなかった.

\subsection{キーワード抽出・類似検索法を用いた分類}
\label{hoho}
本節で述べる手法では,講演参加申込全体から出現頻度が高く,
重要性が高いと思われるキーワードを
抽出し,そのキーワードをセッション名とする.そして,それらセッション
名と講演参加申込の類似度を計算し,類似度の高いセッション名に講演参加
申込を分類することにより,大会プログラムを作成する.

本節の実験では,第\ 5回の講演参加申込データより,セッション名の候補となる
キーワードを抽出し,次に,それに対して類似している講演参加申込を
検索し,その結果を分類結果とした.
この手法を我々はキーワード抽出・類似検索法と呼ぶ\cite{Ozaku1998}.
この実験結果が,クラスタリング法,学習アルゴリズムを用いた
分類結果より良かったので,第\ 6回の大会プログラム原案は
キーワード抽出・類似検索法を利用して作成している.

\subsubsection{セッション名の抽出}
\label{keyda}
まず,セッション名となるキーワードを選択する際に,以下の方法を比較した.

\begin{itemize}
\item 出現頻度のみを利用した方法
\item スコアリング手法
\end{itemize}

出現頻度のみを利用した方法は,各講演参加申込のタイトルとアブストラクトから
平仮名,句読点,記号を除いて
得られる漢字やカタカナなどの文字列をキーワードとして
抽出\cite{Miyamoto1993}し,すべての講演参加申込から抽出したキーワードの
出現頻度をカウントし,出現頻度の高いキーワードをセッション名の候補として
選択する方法である.

スコアリング手法は,上記と同様の方法で各講演参加申込から抽出した
キーワードに出現頻度,出現位置,そのキーワードの後に手がかり
語(は,が,を,について)がある場合に応じ
てスコアを加算\footnote{出現位置によるスコア加算は,キーワードが
タイトルにある場合は\ 5ポイント,手がかり語のあるものには\ 3ポイント
を経験的に加算している.}し,そのスコアを手がかりにセッション名の
候補を抽出する手法である.ここでは,「構文解析」のような
複合語が出現している場合は,「構文」と「解析」の出現頻度へ複合語の
出現頻度分を加算し\footnote{複合語の出現頻度を
加算する場合には,複合語(例:意味的構文解析)の出現頻度を
短い単語(例:意味,構文,解析)のスコアへ加算する場合と,短い単語
の頻度を複合語のスコアへ加算する場合とがある.タイトルには
より長い複合語が出現する傾向がある.短い単語の出現頻度のスコアを
複合語へ加算してしまうと,タイトルへ出現した時の出現位置に
よる加算もされているため,より長い複合語のスコアが高くなり
すぎてしまう.そのため,ここでは複合語の出現頻度を
短い単語のスコアへ加算した.},
各講演参加申込からスコアの高いキーワード(上位\ 5位まで)を
抽出する\footnote{
スコア\ =\ 出現頻度\ +\ 位置スコア\ +\ 複合語出現頻度\ +\ 手がかり語スコア 
}.そして,抽出されたキーワードの出現した講演参加申込数を集計し,
出現数の上位から適当な数を大会プログラムのセッション名候補とする.

例えば,図\ref{aaa}の講演申込からは「新聞記事」「構文」「情報抽出」
「定型的」「抽出」などがキーワードとして抽出される.
それぞれのスコアが,「新聞記事」の場合,タイトルに含まれるため,
タイトルポイント\ 5と出現頻度\ 3回,さらに手がかり語(この場合は「が」)
の前に現れるため\ 3ポイントを加算して,総スコアは11ポイントと
なる.「構文」はタイトルスコア\ 5と出現頻度\ 1回から\ 6ポイント,
「情報抽出」は出現頻度\ 3回と手がかり語スコア\ 6ポイントから合計\ 9ポイント,
「定型的」は出現頻度\ 1回のみから\ 1ポイント,
「抽出」は出現頻度\ 2回と複合語ポイント(「情報抽出」の出現頻度\ 3
回)から\ 5ポイントとそれぞれの抽出されたキーワードにスコアを計算する.
そして,スコアの上位\ 5位までのキーワード(ここでは,
「新聞記事」と「情報」がスコア\ 11ポイント,「情報抽出」が\ 9ポイント,
「構文」が\ 6ポイント,「従属節」と「抽出」が\ 5ポイントとなり,
上位\ 5位までの\ 6つのキーワードが抽出される.)がこの講演参加申込
から抽出されたセッション名候補となる.そして,各講演参加申込から
抽出されたセッション名候補の出現した講演参加申込数を集計し,講演参
加申込数の多いものから大会プログラムのセッション名候補を選択した.

出現頻度の高いキーワードには「システム」「本稿」「我々」「利用」
「提案」「手法」などのものが並び,セッション名には
適当ではないものが多かった.一方,後者のスコアリング手法では
「対話」「構文解析」「統語」など第\ 5回大会でのセッション名として
利用されているキーワードが得られた.そのため,セッション名には
スコアリング手法で得られたキーワードを利用することとした.

スコアリング手法によるキーワード抽出で,講演参加申込数の上位10位
(13個)のセッション名候補(システム,
対話,統語,情報検索,翻訳,モデル,解析,構文解析,抽出,
辞書,生成,分類,手法)を自動抽出し,その候補から
セッション名として妥当と思われる\ 9個(対話,情報検索,
翻訳,モデル,解析,抽出,辞書,生成,分類)を人手で選択した.
それらを第\ 5回大会の実験用セッション名とした.

\begin{figure}
\footnotesize
$\langle$ TITLE $\rangle$ 新聞記事における書き出し文の構文
$\langle$ /TITLE $\rangle$ \\
$\langle$ ABSTRACT $\rangle$ \\
 情報抽出では定型的な文章である新聞記事が対象となることが多
い。しかし、一般に、いつ・誰が・どこで、何を、どうしたといった5W1
H型の固定的な表現で記述される書き出し文であっても、連体修飾節
や連用節などの従属節を含む複雑な構造をとることがある。現在、こ
れらの文から情報抽出を行うのは難しいと考えられている。一方で、
どのような情報を抽出すべきかについても十分検討されていない。本
研究では、このような複雑な文から抽出できる情報は何か、またどの
ような観点に着目して情報抽出を行うべきか明らかにする。そのため、
新聞記事の書き出し文を対象にして、主節と従属節の関係に着目し
分析を行う。
$\langle$ /ABSTRACT $\rangle$ \\
\vspace*{-4mm}
\caption[図]{\label{aaa}セッション名候補抽出の例}
\end{figure}

\subsubsection{講演申込のセッションへの分類}
\label{bunbun}
次に,各セッションへ講演参加申込を,分類する方法について述べる.
この分類では,キーワードベクトルを利用している.
キーワードベクトルとは,タイトルやアブストラクトの中に存在している
特定のキーワードを要素とするベクトルである.
キーワードベクトルには,セッション名候補のキーワードと
その関係語からなるセッションベクトル,および各講演参加申込のキーワード群
からなる講演ベクトルがある.
まず,前節で抽出し選択したセッション名候補と各講演参加申込の中に
おいて共起関係の高いキーワードをセッション関係語として抽出する.
セッション関係語は,セッション名として選ばれたキーワードと
同じ講演参加申込の中に同時に現れる出現頻度の高い上位\ 2個の
キーワードを共起関係の高いキーワードとして抽出したものである.
ただし,上位\ 2個の同時出現頻度が
低い(出現頻度\ 5回未満)場合は,セッション関係語は
ないものとした.同時出現頻度が同数で頻度の高いキーワードが複数ある場合は,
セッション関係語として,それらのキーワードをすべて利用し
た\footnote{セッションベクトルには,最大\ 4個のキーワードを利用した.}.
このようにして抽出されるセッション関係語とセッション名からなるセッションの
キーワードベクトルをセッションベクトルと呼ぶ.また,
各講演参加申込について,それぞれから抽出したすべてのキーワードの
ベクトルを講演ベクトルと呼ぶ.

各々の講演ベクトルと最も類似したセッションベクトルを算出し
セッションベクトル中のセッション名候補キーワードをその講演参加申込の
セッション名として選択した.ただし,類似度は\ 2つのベクトルの
内積である.これは,講演を質問としてセッションを検索しているようなも
のなので,これを類似検索法と呼ぶ.

類似度の計算時にも各キーワードのスコア(キーワードベクトルの
要素の値)として,出現頻度のみ
を利用した場合と,スコアリング法を利用した場合とを比較した.
出現頻度のみをスコアとして利用し,類似度を求めた場合,
同じスコアで複数のクラスに現れてしまう申込が存在した.
一方,出現位置による加点をほどこしたスコアリング法のスコアを
利用して類似度を求め分類したところ,
表\ref{99nlp}にその一部を示すように良い結果が得られた.よってスコアリング
法で計算したスコアをキーワードベクトルのスコアとして利用し,類似度を
求める方法を利用することとした.

\begin{table*}
\begin{center}
\caption[表]{\label{99nlp}自動分類した結果の例}
\footnotesize 
\begin{tabular}{|c|c|l|}\hline 
\multicolumn{3}{|c|}{\bf セッション名: 検索} \\\hline
登録番号& 分野 & \multicolumn{1}{|c|}{タイトル}\\\hline
  3 & d  & 係り受け情報や語の意味情報を利用した日本語テキスト検索システム \\\hline
  42 & d  & 要素の順序関係から見た類似文最適照合検索 \\\hline
  55 & d  & 分類標数の相互参照に基づく多言語書誌データ検索システム \\\hline
  70 & d  & コンプリメントタームを用いた情報検索 \\\hline
  81 & d  & 情報検索の類似尺度を用いた検索要求文の単語分析 \\\hline
  88 & c  & ニュース音声データベースの検索システムの試作 \\\hline 
\end{tabular}
\end{center}
\end{table*}

\section{第\ 6回大会プログラム作成} 
\label{gogo}
\ref{yaya}章の実験に基づき,スコアリングに基づいたキーワード
抽出・類似検索法を用いて,第\ 6回の大会プログラムの原案を作成した.

2000年\ 3月に開催された言語処理学会第\ 6回年次大会の
講演参加申込をすべて図\ref{datada}に示すようなデータへ変換した
\footnote{自動化,効率化のためには,入力形式が統一
されている必要があると考える.ここではテスト的に\ XML形式を採用した.}.
\begin{figure}
{\footnotesize {\bf
\begin{verbatim}
<APPLICATION>
<MAIL-ID>130</MAIL-ID>
<TYPE>講演発表</TYPE>
<TITLE> 大会プログラム自動生成に向けての一考察 
</TITLE>
<AUTHOR id=1>
<KANJI>○小作 浩美</KANJI>
<KANA>オザク ヒロミ</KANA>
<AFFILIATION> 通信総合研究所
</AFFILIATION>
<NUMBER>123-456-7890</NUMBER>
</AUTHOR>
<AUTHOR id=2>
...
</AUTHOR>
<CATEGORY>d,e</CATEGORY>
<APPLIANCE>OHP</APPLIANCE>
<ABSTRACT>
  いくかの言語処理技術を利用して言語処理学会の
大会プログラムを自動作成することを試みた.
その結果と自動生成するにあたり明らかになった
問題点,改良点について報告する.
</ABSTRACT>
<ADDRESS>
住所:   〒651-2492 神戸市西区岩岡町岩岡588-2
所属:   通信総合研究所
氏名:   小作 浩美
...
</ADDRESS>
</APPLICATION>
\end{verbatim}
}}
\vspace*{-5mm}
\caption[図]{\label{datada}申込書から作成したデータ}
\vspace*{-5mm}
\end{figure}
\ref{keyda}節で述べた方法により自動抽出したセッション名
候補の\ 20キーワードを決定した.
続いて,人手で\ 9キーワ
ード(対話,要約,辞書,コーパス,検索,抽出,解析,生成,翻訳)を
セッション名として選択した.
次に,\ref{bunbun}節で述べた方法を利用しセッションベクトルと
各講演参加申込の講演ベクトルの類似度に基づき自動分類した.
なお,どのセッション名とも類似しない講演参加申込も存在する.その
申込については「その他」として取り扱うこととした.

\begin{table*}
\begin{center}
\caption[表]{\label{prog}自動分類した結果の例}
\footnotesize 
\begin{tabular}{|c|c|l|}\hline 
\multicolumn{3}{|c|}{\bf セッション名: 対話} \\\hline
登録番号 & 分野 & \multicolumn{1}{|c|}{タイトル}\\\hline
  40 & c  & 混合主導対話における音声認識誤りに対処するための対話管理 \\\hline
59 & c,d  & 制限知識下における効率的対話制御 \\\hline
  85 & c  & 道案内WOZシステムとの対話における言い淀み表現の分析 \\\hline
 102 & c  & 係り受け関係を用いた即時発話理解 −音声対話メールシステムにおける手法−\\\hline
 124 & c  & 多重文脈に即応的な対話インターフェース:半可通 \\\hline
\multicolumn{3}{|c|}{\bf セッション名: 辞書} \\\hline
登録番号 & 分野 & \multicolumn{1}{|c|}{タイトル}\\\hline
  15 & a  & 概念体系における反対語の検討 \\\hline 
  25 & b  & 辞書定義文を用いた複合語分割 \\\hline
32 & b,d  & 翻訳システム用の辞書ツール \\ \hline
 110 & a  & 既知形態素からなる未知複合語概念推定とその辞書登録 \\\hline
 116 & b  & 不要語リストを用いたRFC英和辞書作成過程における課題 \\ \hline
 117 & b  & ソフトウェア開発工程における用語構造と翻訳辞書作成過程における課題 \\ \hline 
\end{tabular}
\end{center}
\end{table*}

ここまでの処理では,大会プログラムとして
想定すべき条件(例えば会場数や時間の制約など)については
考慮しなかった.そのため,表\ref{prog}のように自動作成された結果を原案とし,後
から時間的な制約を考慮した上で,\ 1つのセッションでの発表件数を調整
した.

\begin{table*}
\begin{center}
\caption[表]{\label{idou}手動で移動したタイトル}
\scriptsize
\begin{tabular}{|c|c|l|}\hline 
登録番号& 分野 & \multicolumn{1}{|c|}{タイトル}\\\hline
 11 & c & 語の重要度を考慮した談話構造表現の抽出\\\hline
 13 & d & 直接引用表現を利用した要約知識の自動抽出の試み\\\hline
 12 & c & 単語ラティス形式の音声認識結果を対象とした発話意図の認識\\\hline
 19 & c & Generating coherent text from finely classified semantic network\\\hline
 39 & c & 日本語における口語体言語モデル\\\hline
 65 & d & 意味的共起関係を用いた動詞と名詞の同音意義語の仮名漢字変換\\\hline
 77 & a & 韓日語の副詞節の階層性に関する対照言語学的研究-南(1974)の階層性モデルの観点から-\\\hline
 83 & a & FB-LTAGからHPSGへの文法変換 \\\hline
 91 & b & 文節解析のための長単位機能語辞書\\\hline
 97 & d & 科学論文における要旨-本文間のハイパーリンク自動生成\\\hline
 98 & c & 確率付き項構造による曖昧性解消\\\hline
118 & a & コンピュータ西暦2000年対応の標準化におけるデータ,用語,処理,試験\\\hline
119 & a & 日本語待遇表現の評価実験による誤用とその認知について\\\hline
121 & d & 図書館の自動リファレンス・サービス・システムの構築\\\hline
128 & c & SGLR-plusによる話者の対象認識構造を抽出する英語文パーザの試作\\\hline
130 & d,e & 大会プログラム自動生成に向けての一考察\\\hline 
131 & b & 簡単なフィルターを用いた二言語シソーラスの自動構築\\\hline
\end{tabular}
\end{center}
\end{table*}

\begin{table}
\begin{center}
\caption[表]{\label{henko}自動分類結果と変更先}
\footnotesize 
\begin{tabular}{|c|c|l|l|}\hline
登録番号 & 分野 & 自動分類結果 & 変更先 \\\hline
 11 & c & 抽出 & 解析 \\\hline
 13 & d & 要約 & 抽出 \\\hline
 12 & c & 翻訳 & 理論 \\\hline
 19 & c & その他 & 生成 \\\hline
 39 & c & コーパス & 理論 \\\hline
 65 & d & 意味 & システム \\\hline
 77 & a & その他 & 理論 \\\hline
 83 & a & 生成 & 解析 \\\hline
 91 & b & 解析 & 辞書 \\\hline
 97 & d & 生成 & システム \\\hline
 98 & c & システム & 解析 \\\hline
118 & a & その他 & 理論 \\\hline
119 & a & その他 & 理論 \\\hline
121 & d & システム & 検索 \\\hline
128 & c & 抽出 & 解析 \\\hline
130 & d,e & 抽出 & 理論 \\\hline
131 & b & 抽出 & コーパス\\\hline
\end{tabular}
\end{center}
\end{table}

調整においては,手作業で一部の変更を行なった.この処理には
数時間程度かかっている.
変更のあった発表は17件(表\ref{idou})である.また,その中でどの講演とも
類似度が低く,「その他」となっていた発表は
英文タイトルのものを含め\ 4件存在した.表\ref{henko}に自動分類結果と人手
で変更したセッションを示す.また,\ 1つの
セッション名に複数のセッション分の講演参加申込が分類された
場合は,その分類された申込中の文書からセッション名として
利用するための新たなキーワードを抽出して再度分類すること
をくりかえした\footnote{この処理で「システム」と
「意味」のキーワードがセッション名として加わっている.}.しかし
ながら,「解析」のセッションではセッション名として利用するのに良い
キーワードが抽出できず,3つのセッションに跨ることとなってしまった.
これは「解析」に分類された申込から,次のセッション名となるような
スコアの高いキーワードが抽出できなかったためである.
抽出出来なかった理由は,概念的には近い意味を持つ別の
キーワードでそれぞれ記述されていたり,あるいは「解析」には
変わりがないが研究対象の違いにより,出てくるキーワードが若干異なった
ためである.これは頻度や位置情報だけで意味的情報を利用せずに
セッション名を選択する場合の限界と考えられる.

最後に,「その他」のセッションと分類された発表に概念的に合う
セッション名「理論」を人手により付加し,大会プログラムの案とした.その
後,講演参加申込の際に講演の日付希望や,講演順の要望もあり,それらを
考慮した修正と,同じ所属の講演者の発表が別のセッションにおいて
重ならないように考慮して,発表順に変更を加え,最終の大会
プログラムとした\footnote{自動抽出によるキーワードから
人手で選択したセッション名は,最終的に「対話,要約,辞書,
コーパス,検索,抽出,解析,生成,翻訳,意味,システム」となった.}.
そして,採否連絡と共に講演番号通知とした
\footnote{なお,ポスター発表に
ついては,ここでの方法を用いずに,申込者からの講演分野
の申請を利用した.これは,第5回大会のポスター発表の取り扱い
に倣ったものである.会場数に合わせて
講演分野\ a,bの組と\ c,d,eの組の\ 2つに分け,番号をつけた.
なお,ポスター発表以外の講演申込は104件あり,3件がキャン
セルされ,1件がプログラム作成後に登録されたもので,最終的な
講演数は102件であった.}.

\section{アンケート結果} 
\label{haha}
第\ 6回大会中および大会終了後,大会プログラムに対するアンケート調査を
聴講者と講演者に対して実施した.
聴講者に対しては大会会場においてアンケート用紙を配布し,参加したセッショ
ンにおいて他のセッションで発表すべきであると感じた講演について意見を求めた.
発表者に対しては,大会終了後,ポスター発表以外の講演者にメイルを
送り,発表したセッションであっていたかどうか,また発表したセッションに
興味のある発表があったかどうかの調査を行なった.

発表者に対する調査においては,講演発表を行なった\ 102名にメイルし,
79名の発表者から回答を得ることができた.

発表したセッションが発表内容とあっているかの問いについて,表\ref{awanai}に
示すように10名からマッチしていないとの回答があった.
そのうち,\ 4名は自分の発表したセッションは「はみ出した講演を
集めたセッションのようだ」と回答している.実際,この\ 4件の講演は
どの講演とも類似度が低く,「その他」と分類された講演を含むセッションであった.
さらにマッチしていないと回答があったもののうち,\ 5件はタイトルに
セッション名にあたるキーワードが存在しており,タイトルに現れたキーワー
ドへの加点の影響が悪い方に出てしまったものと考えられる.
また,\ 3件はセッション名にあたるキーワードがタイトルには
存在しておらず,アブストラクトも含めた自動分類結果で複数のセッションに
低いスコアで,かつ同じスコアで分類されているものであった.セッション名に
あたるキーワードを全く含まないタイトルは表\ref{nashi}に示すように
全体で22件存在し,そのうち,低いスコアで,かつ同じスコアで,複数の
セッションに分類されたものは\ 7件存在した.この場合,プログラムの出力した順の
一番上位のセッションに分類してしまったが,その中の\ 3件が
結果的にマッチしていないと回答されたものであった.
これら10名以外の,69名は発表したセッションはマッチしていたと回答し,その
うち,自分の興味のある発表が同じセッションに無かったとの
回答は\ 2名からであった.全回答者でも自分の発表したセッションに興味のある
発表が無かったと回答されたのが\ 4名だけであった.以上のことにより,比較的
良いプログラムが作成できたと言える.


\begin{table}
\begin{center}
\caption[表]{\label{awanai}セッションと合わない回答(10回答)}
\footnotesize 
\begin{tabular}{|c|c|c|c|}\hline
回答者 & タイトル影響有り & はみ出し感 & 低・同スコア\\\hline
 1 & ○ & & \\\hline
 2 & ○ & & \\\hline
 3 & ○ & & \\\hline
 4 & ○ & ○ & \\\hline
 5 & ○ & ○ & \\\hline
 6 &  & ○ & \\\hline
 7 &  & ○ & ○ \\\hline
 8 &  &  & ○ \\\hline
 9 &  &  & ○ \\\hline
 10 &  &  & \\\hline
 合計 & 5 & 4 & 3 \\\hline
\end{tabular}
\end{center}
\end{table}
\begin{table*}
\begin{center}
\caption[表]{\label{nashi}セッション名を含まないタイトル}
\scriptsize
\begin{tabular}{|c|c|l|}\hline 
登録番号 & 分野 &  \multicolumn{1}{|c|}{タイトル}\\\hline
  3 & d & かぎ括弧で囲まれた表現の種類の自動判別\\\hline
  8 & b & 概念体系における反義概念の検討\\\hline
 12 & c & 単語ラティス形式の音声認識結果を対象とした発話意図の認識\\\hline
 19 & c & Generating coherent text from finely classified semantic network\\\hline
 28 & d & 文分割による連体修飾節の言い換え\\\hline
 35 & c & HPSGの複数の文脈自由文法へのコンパイル\\\hline
 39 & c & 日本語における口語体言語モデル\\\hline
 41 & d & ワールドワイドウェブを利用した住所探索\\\hline
 42 & c & LR表への複数の接続制約の組み込みによる一般化LR法の拡張\\\hline
 47 & d & n-gramモデルとIDFを利用した統計的日本語文短縮\\\hline
 57 & c & グルーピング法によるGLRパーザの効率的な実装\\\hline
 58 & c & キーワードの活性度の変かを用いたテキスト中の単語と話題の対応付け\\\hline
 65 & d & 意味的共起関係を用いた動詞と名詞の同音意義語の仮名漢字変換\\\hline
 77 & a & 韓日語の副詞節の階層性に関する対照言語学的研究-南(1974)の階層性モデルの観点から-\\\hline
 83 & a & FB-LTAGからHPSGへの文法変換 \\\hline
 98 & c & 確率付き項構造による曖昧性解消\\\hline
101 & d & ニュース速報記事の前文情報との照合に基づく見出し文の言い替え\\\hline
118 & a & コンピュータ西暦2000年対応の標準化におけるデータ,用語,処理,試験\\\hline
119 & a & 日本語待遇表現の評価実験による誤用とその認知について\\\hline
125 & a,c & 人間の文処理は左隅型である:埋め込み構造と記憶負荷\\\hline
129 & a & 形容詞的ふるまいをする「名詞+の」について \\\hline
131 & b & 簡単なフィルターを用いた二言語シソーラスの自動構築\\\hline 
\end{tabular}
\end{center}
\end{table*}

一方,聴講者からの回答は残念ながら\ 3名からしか回収できなかった.アン
ケート項目等を検討し回答しやすいものにする必要があったと思われる.ま
た,今回の実験が,発表者や聴講者にどのような利益をもたらすか
宣伝し,協力を得る方法を考える必要性も感じた.

\section{考察} 
\label{mumu}
今回のプログラム作成の実験において,
セッション名の傾向,プログラム作成に
必要と思われる制約条件,プログラム作成の効率化について考察する.

\subsection{セッション名の傾向について}
我々が利用したキーワード抽出・類似検索法では,
\begin{itemize}
\item[1] プログラム作成に重要と思われるキーワードを選出
\item[2] スコアリング手法でキーワードのスコアを算出
\item[3] 類似検索実行
\end{itemize}
という手続きをとる.

スコアリング手法では出現位置を考慮しているため,キーワードが
タイトルに含まれるならば高いスコアになっている.
それは,一般に著者が自分の論文を的確に示すようなタイトル
を,なるべく一般性を持つ形で提示しようとするときには,
自分の論文が所属する分野を示すキーワードと共に,自分の論文の特
徴を示すキーワードをタイトルにいれる傾向があると考えられることによる.
そのため,論文タイトルには分野を示すキーワードが含まれやすく,
このスコアリングによるキーワード抽出法は,その点に着目したものである.

第\ 6回大会プログラムは,タイトルにセッション名を含んだもの
がほとんどである.具体的には,タイトルの中に,その講演が割り振られた
セッション名にあたるキーワードを含んでいないものは全タイトル
102件中\ 29件で,そのうち,22件はセッション名となっている
キーワードをタイトルに全く含まないものであった.
つまり,\ 8割のタイトルにはセッション名にあたるキーワードが
含まれており,タイトルからキーワードを選択し
それのみから講演ベクトルを作り,それに基づいて講演を
分類するだけでも,かなり良い結果が得られると考えられる.実際,第\ 5回
大会のプログラムを眺めてみても,セッション名にあたるキーワードを
含むタイトルは多い\footnote{ポスター発表,特別セッション,英文タイト
ルを除く,107件のタイトルのうち,セッション名にあたるキーワードを含む
タイトルは79件である.}.上記のように,分野を含む
キーワードをタイトルに含み易いことは明らかである.
もちろん,タイトルにセッション名を含んでいても,アブストラクトの
情報を考慮すると他のセッションに割り振られるべきものも存在する.本手法では,
タイトルにセッション名にあたるキーワードを持ちながら,そのキーワード
とは別の適切なセッションに割り振ることが出来ている.
タイトル中にその講演が割り振られたセッション名にあたるキーワードを
含んでいない\ 29件中,タイトルにセッション名となっているキーワードを
全く含まない\ 22件を除いた残りの\ 7件が該当する.
今回,アンケート調査において,セッション名の妥当性について確認を取って
いないため,即断はできないが,比較的良いプログラムが作成できたと考える.

\subsection{プログラム作成における制約条件について}
我々は大会プログラム作成を文書分類課題とみなして,実験を
行なった.しかし,実際のプログラム作成においては,時間的な
制約,場所的な制約も考慮する必要がある.それは,会場数に
よる制約だけではなく,例えば,所属が
同じ講演者の講演が時間的に重なることを避ける
ような制約も考えられ,今回のプログラム作成にはこの制約も
考慮している.これは,同じ所属の講演者同士は,お互いの発表を
聞きたいであろう,あるいは,発表時に使う機材を共有する可能性もあると
判断したことによる.しかし,同じ所属の講演については
職場内ですでに聞いている,あるいは,機材を共有する必要がない
ため,同じ所属の講演者の発表時間が重なっていても問題がない可能性も
ある.このような条件や制約については,
今回は我々が思いつく制約を独断で設けたが,もっと別の重要な観点が
ある可能性もあり,制約の妥当性も調査する必要があると思われる.


今後,これらの制約条件が整理されれば,
それら制約条件を蓄え,その適応結果も合わせて収集することで
制約条件に関するヒューリスティックな知識が獲得でき,
その知識を利用することで大会プログラム作成にかかる
時間をさらに短縮できるものと考える.

\subsection{プログラム作成の効率化について}
今回の大会プログラム作成作業で一番時間を要した処理は,データ作成
の部分である.
それは,講演参加申込のフォーマットが統一されていないかったために,手作業での
データ修正が必要であったことと,締め切り後の講演参加申込の対処を
行なったためである.本手法のように自動的に何かの処理をする
ような場合,入力は統一されたフォーマットでシステムに渡される
必要があると考える.本研究での,このデータ化の処理は大会プログラムを
作成する際には避けて通れない作業であるため,正しく効率的にデータ
ベース化するための手続きを考える必要がある.
さらに,当初,プログラムの組織表記を講演参加申込の表記から自動生成したため,
同じ組織でありながら記述の違いがあり,プログラム上の統一感が損なわれ
ていた.そのため,プログラム作成後,プログラムの組織表記を手作業で統一
する作業が必要であった.


この問題については,申込時にWWW(World Wide Web)を利用して,申込の
フォーマット化,簡略化が行なわれれば,処理時間を
短縮できると考える.組織の表記についても,会員番号などから自動的に
組織名が入力されるようにするなど,学会事務局側で対処することが可能であると
考える.それにより,参加者の意識も変化し,事務処理や大会参加の手続きが
よりスムーズにできると考える.なお,同様な考察が\cite{IEE}にもある.

\section{おわりに}
本研究は,実際の大会プログラムの作成作業で,現在の様々な言語処理
技術がどの程度利用でき作業効率をあげられるのかを試したも
のである.本研究の実験の結果,大会プログラム作成において,キーワード
抽出および文書分類の言語処理技術が十分に役立ち,これらの技術により
作業効率を向上できることがわかった.さらに,データ作成に時間が一番かかって
いることから,申込時にWWWなどを利用して講演参加申込データの
統一化を行なうことで事務効率がさらに向上できる可能性があることが
わかった.

今回の実験では,セッション名をタイトルとアブストラクトから抽出する
ことで,その大会の発表傾向に沿ったセッションを作成できた.
そのことは,講演者の興味にあった発表がセッション中に
含まれていたかどうかのアンケート調査結果が良好であったことからも
示されている.さらに,データ作成以外のプログラム作成手続きにおいての
人手による作業が数時間程度で済んだことから,作業の効率化を
実現できたといえる.

今回の実験では,1\ つのセッション名に複数セッション分の講演参加申込
が分類された問題と,セッション名をタイトルに含んだため
に誤って分類された問題が残されている.この\ 2つの問題を解決するためには
短いアブストラクトからでも,より的確なキーワードを抽出する
技術が必要であると考える.
また,日本語以外の発表への対応や,大会プログラムの完全
自動作成に向けての手続き処理方法の提案,考慮すべき条件
の明確化,セッション名の妥当性調査などを行う必要もあると考える.
そして,参加者,講演者,実行者共に快適に大会に参加でき,運営
できるような点を明確にし,学会の事務処理の簡素化,学会の活性化などに
貢献できるツールの実現や研究につながればと考えている.

\section*{謝辞}
北陸先端科学技術大学院大学の島津明教授,望月源氏に1999年
第\ 5回大会のデータの利用に際しご協力いただいた.ここに感謝する.

\bibliographystyle{jnlpbbl}
\bibliography{jgijutu}

\begin{biography}
\biotitle{略歴}

\bioauthor{小作 浩美}{
1985年郵政省電波研究所入所.現在,独立行政法人通信総合研究所研究員.
奈良先端科学技術大学院大学博士課程在学中.
自然言語処理,情報検索の研究に従事.
言語処理学会,情報処理学会,電子情報通信学会,ACM,各会員.}

\bioauthor{内山 将夫}{
1992年筑波大学第三学群情報学類卒業.
1997年同大学院博士課程修了,博士(工学).
同年,信州大学工学部助手.
1999年郵政省通信総合研究所非常勤職員.
現在,独立行政法人通信総合研究所任期付き研究員.
言語処理学会,情報処理学会,人工知能学会,日本音響学会,ACL,各会員.}

\bioauthor{村田 真樹}{
1993年京都大学工学部卒業.1995年同大学院修士課程修了.
1997年同大学院博士課程修了,博士(工学).
同年,京都大学にて日本学術振興会リサーチ・アソシエイト.
1998年郵政省通信総合研究所入所.現在,独立行政法人
通信総合研究所研究員.
自然言語処理,機械翻訳,情報検索の研究に従事.
言語処理学会,情報処理学会,人工知能学会,電子情報通信学会,ACL,各会員.}

\bioauthor{内元 清貴}{
1994年京都大学工学部卒業.1996年同大学院修士課程修了.
同年,郵政省通信総合研究所入所.現在,独立行政法人
通信総合研究所研究員.
自然言語処理の研究に従事.
言語処理学会,情報処理学会,ACL,各会員.}

\bioauthor{井佐原 均}{
1978年京都大学工学部卒業.1980年同大学院修士課程修了.博士(工学).
同年,通商産業省電子技術総合研究所入所.
1995年郵政省通信総合研究所入所.現在,独立行政法人
通信総合研究所けいはんな情報通信融合研究センター自然言語グループリーダー.
自然言語処理,機械翻訳の研究に従事.
言語処理学会,情報処理学会,人工知能学会,日本認知科学会,ACL,各会員.}

\bioreceived{受付}
\biorevised{再受付}
\bioaccepted{採録}
\end{biography}
\end{document}
