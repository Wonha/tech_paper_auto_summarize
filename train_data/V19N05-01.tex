    \documentclass[japanese]{jnlp_1.4}
\usepackage{jnlpbbl_1.3}
\usepackage[dvips]{graphicx}
\usepackage{amsmath}
\usepackage{hangcaption_jnlp}


\Volume{19}
\Number{5}
\Month{Decemter}
\Year{2012}

\received{2012}{3}{22}
\revised{2012}{7}{24}
\accepted{2012}{9}{2}

\setcounter{page}{367}

\usepackage{udline}
\usepackage{color}
\newcommand{\addtext}[1]{}
\newcommand{\inzero}{}
\newcommand{\zero}{}
\newcommand{\indc}{}
    \newcommand{\dc}{}


\jtitle{モーラ系列と音象徴ベクトルによるオノマトペの印象推定法}
\jauthor{土屋 誠司\affiref{Author_1} \and 鈴木 基之\affiref{Author_2} \and 任  福継\affiref{Author_2} \and 渡部 広一\affiref{Author_1}}
\jabstract{
オノマトペとは,擬音語や擬態語の総称である.文章で物事を表現する際に,より印象深く,豊かで臨場感のあるものにするために利用される.このようなオノマトペによる表現は,その言語を\addtext{母語}としている人であれば非常に容易に理解することができるため,国語辞書などにあえて記載されることは稀なケースである.また,記載があったとしても,使用されているオノマトペをすべて網羅して記載していることはない.そのため,その言語を\addtext{母語}としない人にとっては学習し難い言語表現である.そこで本稿では,オノマトペが表現する印象を推定する手法を提案する.日本語を対象に,オノマトペを構成する文字の種類やパターン,音的な特徴などを手がかりに,そのオノマトペが表現している印象を自動推定する.これにより,日本語を\addtext{母語}としない人に対して,日本語で表現されたオノマトペの理解の支援に繋がると考えられる.結果として,オノマトペの表記内のモーラ系列間の類似度とオノマトペの表記全体の音象徴ベクトルによる類似度を用いた手法が最も良い推定結果となり,\addtext{参考値である人間同士の一致率の8割程度にまで近づくことができた.}
}
\jkeywords{オノマトペ,印象推定,モーラ系列,音象徴ベクトル}

\etitle{\protect\setlength{\baselineskip}{18pt}A Novel Estimation Method of \\
	Onomatopoeic Word's Feeling based on \\
	Mora Sequence Patterns and Feeling Vectors}
\eauthor{Seiji Tsuchiya\affiref{Author_1} \and Motoyuki Suzuki\affiref{Author_2} \and Fuji Ren\affiref{Author_2} \and Hirokazu Watabe\affiref{Author_1}} 
\eabstract{
Onomatopoeic words are frequently used for expression of rich presence.
These words can be understood easily for native speakers. Therefore
most of onomatopoetic words are not written in a national language
dictionary, or only a part of meaning is described.
On the other hand, it is hard to understand a meaning of onomatopoetic
words for non-native speakers. They can neither feel a meaning of an
onomatopoetic word nor look it up in a dictionary.
In this paper, an estimation method of feeling of an onomatopoeic
word has been proposed. The feeling of the onomatopoeic word is inferred
by using several features, such as morae sequence pattern of a
onomatopoeic word, feeling of each mora, and so on.
From the experimental results, the estimation performance of the
proposed method was 0.345 (F-value). It was approximately
80\% of the estimation performance given by human (F-value was 0.427).
It can be said that the proposed method is useful for supporting
learners of onomatopoeic words.
}
\ekeywords{Onomatopoeic word, Estimation of feeling, Mora Sequence Patterns, Feeling Vectors}

\headauthor{土屋,鈴木,任,渡部}
\headtitle{モーラ系列と音象徴ベクトルによるオノマトペの印象推定法}

\affilabel{Author_1}{同志社大学理工学部}{Faculty of Science and Engineering, Doshisha University}
\affilabel{Author_2}{徳島大学大学院}{Institute of Technology and Science, The University of Tokushima}



\begin{document}
\maketitle


\section{はじめに}

オノマトペとは,「ハラハラ」,「ハキハキ」のような擬音語や擬態語の総称である.文章で物事を表現する際に,より印象深く,豊かで臨場感のあるものにするために利用される.日本語特有の表現方法ではなく,様々な言語で同じような表現方法が存在している\addtext{{\cite{Book_03}}}.

このようなオノマトペによる表現は,その言語を\addtext{母語}としている人であれば非常に容易に理解することができる.また,オノマトペは音的な情報から印象を伝えるため,ある程度固定した表現もあるが,音の組み合わせにより様々なオノマトペを作ることも可能であり,実際様々なオノマトペが日々創出されている\addtext{{\cite{Book_05,Book_06}}}.そのため,国語辞書などにあえて記載されることは稀なケースであり,また,記載があったとしても,使用されているオノマトペをすべて網羅して記載していることはない\addtext{{\cite{Book_04}}}.そのため,その言語を\addtext{母語}としない人にとっては学習し難い言語表現である.

特に,オノマトペを構成する文字が少し異なるだけでまったく異なる印象を与えることも学習・理解の難しさを助長していると考えられる.例えば先の例の「ハラハラ」という危惧を感じる様子を表現するオノマトペの場合,「ハ」を濁音にすると「バラバラ」となり,統一体が部分に分解される様子を表現し,また,半濁音の「パ」にすると「パラパラ」となり,少量しか存在しない様子を表現する.さらに,「ハラハラ」の「ラ」を「キ」にした「ハキハキ」では,物の言い方が明快である様子を表現するオノマトペになる.これらのオノマトペの特徴は,人が学習するときだけでなく,コンピュータで扱う際にも困難を生じさせる.

そこで本稿では,オノマトペが表現する印象を推定する手法を提案する.日本語を対象に,オノマトペを構成する文字の種類やパターン,音的な特徴などを手がかりに,そのオノマトペが表現している印象を自動推定する.
\addtext{例えば,「チラチラ」というオノマトペの印象を知りたい場合,本手法を用いたシステムに入力すると「少ない」や「軽い」などという形容詞でその印象を表現し出力することができる.}これにより,日本語を\addtext{{母語}}としない人に対して,日本語で表現されたオノマトペの理解の支援に繋がると考えられる.また,機械翻訳や情報検索・推薦の分野でも活用することができると考えられる.


\section{関連研究と本研究の位置づけ}

\addtext{オノマトペは,感覚と強く関連する言葉であることから,心理学や認知科学など幅広い分野で研究対象とされている.例えば,映像などの視覚や音などの聴覚から感じる印象をオノマトペを用いて調査し,人が感じる印象とオノマトペとの関係性を抽出したり}
\addtext{{\cite{Article_10,Article_11,Article_12,Article_13}},味覚の官能評価においてその評価項目としてオノマトペが利用され,オノマトペと食品の硬さを示}
\addtext{す応力との関連性を評価したりしている{\cite{Article_14,Article_15}}.}

本稿で対象とする言語処理の分野におけるオノマトペに関する関連研究として,オノマトペ辞書の構築\cite{Article_01,Article_02}やオノマトペの自動分類手法\cite{Article_03}などが提案されている.

前者では,大規模なWebの情報を利用し,主に用例や動詞による同義表現などが調べられるオノマトペ辞書を自動で構築している.高い精度で用例を抽出できているが,新たなオノマトペが日々創出され続け,用法も変化していくため,辞書も構築し続けなければならないという問題はある.また,本稿で対象にしているオノマトペが表現する印象を扱うことはできない.

後者では,10種類の意味を動詞で表現し,292語のオノマトペがどの意味であるかを自動分類している.Webの情報から算出した共起頻度と子音と母音の出現頻度を用い,クラスタリングすることで自動分類を実現している.しかし,オノマトペ中のモーラの並びなどオノマトペの構造に深く着目した処理とはなっていない.また,定義している意味や分類するオノマトペの数が少ないという問題がある.

本稿では,未知のオノマトペが表現する印象は,類似したオノマトペが表現する印象と似ているという考えに基づき,類似度計算により類似したオノマトペを特定し,そのオノマトペが表現する印象を推定する.しかし,前述したように,オノマトペを構成する文字が少し異なるだけでまったく異なる印象を与えることもオノマトペの特徴である.そこで,既に出版されているオノマトペ辞書\cite{Book_01}に掲載されているオノマトペ1,058語を基に,オノマトペを構成する文字列とその構造を解析し,オノマトペが表現している印象を自動推定できる手法を提案する.具体的には,オノマトペ中のモーラの並びとモーラを構成する各音素が表現する印象をベクトル化した音象徴という2種類の特徴を利用することにより,先の問題を解決できる類似度計算を提案する.これにより,大規模な辞書を作成・利用することなく,新たに創出される未知のオノマトペの印象も推定することができる.また,推定する印象は48種類の形容詞で定義し,より詳細にオノマトペを理解できるようにしている.



\section{オノマトペの印象の推定方法}

本論文では,既に出版されているオノマトペ辞書に掲載されているオノマトペにそれらが表現している印象を事前に人手で付与したデータを用い,類似したオノマトペに付与されている印象を推定結果として出力する.ここで重要になるのが,「類似度の計算方法」と「印象語の出力方法」である.以下,順に説明する.


\subsection{類似度の計算方法}

オノマトペから抱く印象は,主にオノマトペの表記に使用されているモーラ自体やモーラの並び方に影響すると考えられる\cite{Article_05,Article_04}.そこで以下に,4種類の類似度算出手法を提案する.


\subsubsection{オノマトペ中のモーラの並びに基づく類似度}


オノマトペには,「フワフワ」,「チラチラ」のようにモーラの並び方にいくつかのパターンがあり,そのパターンによって表現する印象が変化する\cite{Article_04}.そこで,モーラの並びのパターンに着目した類似度を以下に2種類定義する.

\begin{enumerate}
\item \textbf{抽象化した型表現間の類似度}

 オノマトペの表記をモーラの並びを表す型表現\cite{Article_04}に変換し,その型表現同士のレーベンシュタイン距離を元に類似度を計算する.型表現への変換は,オノマトペ内の1モーラを1つの記号へと変換する.特定のモーラに``X'',``Y'' 等の記号を割り振るが,一部の特徴的なモーラには特別な記号を付与する.オノマトペの型表現に用いられる記号のリストを表\ref{type-symbol}に,オノマトペの型表現への変換例を表\ref{type-example}に示す.

 ここで,``t'',``r'',``n''については,オノマトペの表記中で基本とみられる表現(例えば「パチ」)に対して付与されるモーラ(「パチッ」,「パチリ」,「パチン」)に対してのみ用いられ,基本とみられる表現内のモーラ(「キリキリ」内の「リ」等)には用いられない(この場合,通常のモーラと同様,``X''等が用いられる).例えば「フワフワ」「チラチラ」はいずれも``XYXY''と変換される.

\begin{table}[b]
\hfill
\begin{minipage}{189pt}
\caption{オノマトペの型表現に用いられる記号}
\label{type-symbol}
\input{01table01.txt}
\end{minipage}
\hfill
\begin{minipage}{162pt}
\caption{オノマトペの型表現への変換例}
\label{type-example}
\input{01table02.txt}
\end{minipage}
\hfill
\end{table}


 このように変換された型表現同士のレーベンシュタイン距離を算出し,更に類似度に変換する.本論文では,記号の置換・挿入・脱落をそれぞれ距離1としてレーベンシュタイン距離の計算を行うため,系列長がそれぞれ$l_{x}$,$l_{y}$である2つの系列$x$,$y$のレーベンシュタイン距離の最大値$\hat{d}(x, y)$は,
\begin{equation}
\hat{d}(x, y) = \max(l_{x}, l_{y})
\end{equation}
となる.そこで,2つの系列$x$,$y$のレーベンシュタイン距離を$d(x, y)$と
して,2つのオノマトペ間の型表現による類似度$T(x, y)$を
\begin{equation}
T(x, y) = \frac{\hat{d}(x, y) - d(x, y)}{\hat{d}(x, y)}\label{eq:leven}
\end{equation}
と定義する.この類似度は,全く同じ系列同士の時に最大値1を,全く異なる系列同士の時に最小値0をとる.



\item \textbf{モーラ系列間の類似度}

 前項ではオノマトペの表記を型表現に変換し,その系列間の類似度を計算していたが,ここでは,より直接的にオノマトペ内のモーラ系列間の類似度を計算する.

 オノマトペをモーラに分解し,モーラ系列間のレーベンシュタイン距離を算出する.その後,式 (\ref{eq:leven})を用いて類似度を計算する.なお,この計算方法は,前項の型表現に使用した記号の代わりにモーラを直接用いる点を除き,その他はまったく同じ計算である.ここで得られる類似度を以降,$H(x, y)$と表記する.

 前項の型表現間の類似度では「フワフワ」と「チラチラ」はどちらも``XYXY''に変換されるため,類似度は1となるが,本項のモーラ系列間の類似度では0となる.そのため,モーラの並びが持つ印象の違いに加えて,モーラ自体が持つ印象の違いも考慮した類似度であると言える.

\end{enumerate}


\subsubsection{音象徴に基づく類似度}


オノマトペに使用されている様々なモーラを構成する各音素には,その音自体が印象を持っていることが知られている.\addtext{この各音素が表現する印象をベクトルとして表現したものに音象徴{\cite{Article_05}}がある.そこで,音象徴を基に類似度を計算する.なお,音象徴}
\addtext{ベクトルは,「強さ」,「硬さ」,「湿度」,「滑らかさ」,「丸さ」,「弾性」,「速さ」,「温かさ」の8次元の属性を有し,各属性に$-2$から2までの5段階の数値を与えることで音素が表現する印象を定義している.}





まず,音素ごとに定義された音象徴ベクトルを用い,あるモーラの音象徴ベクトルをそのモーラを構成するすべての音素に対応する音象徴ベクトルの総和として定義する.この時,モーラが表現する印象として,母音と比較して子音の方がより強く影響することが知られている\cite{Article_05}ことから,子音の音象徴ベクトルを$w_{c}$倍してから和をとる.

このようにして得られたモーラの音象徴ベクトル$\vec{a}$,$\vec{b}$間の類似度は,以下の式で計算される正規化されたコサイン類似度$c(\vec{a}, \vec{b})$により計算する.
\begin{equation}
c(\vec{a}, \vec{b}) = \frac{1}{2}\frac{\vec{a}\cdot\vec{b}}{|\vec{a}||\vec{b}|} + \frac{1}{2}\label{eq:cosine}
\end{equation}
この式では,値域が0〜1になるように通常のコサイン類似度に対して正規化を行っている.

本論文では,音象徴ベクトルに基づく類似度として,以下の2種類の類似度算出手法を提案する.


\begin{enumerate}
\item \textbf{モーラの\addtext{並び順と長さ}を考慮した類似度}

\addtext{ オノマトペの表記中のモーラの並び順と長さを考慮するため,2つのオノマトペのモーラ系列間の類似度を動的計画法 (DP) を用いたDTW (Dynamic Time Warping) で計算する.DTWは2つのシンボル系列において,各シンボル間に定義された類似度をもとに,系列同士の類似度を求める方法である.2つのシンボル系列{$A=a_{1}a_{2}\cdots a_{I}$}と{$B=b_{1}b_{2}\cdots b_{J}$}({$I, J$}は各系列の長さ)があった時,あらかじめ各シンボル間の類似度{$d(a_{i}, b_{j})$}を定義しておき,シンボルの順序を保存する(シンボル{$a_{i}$}が{$b_{j}$}と対応したとすると,{$a_{i+k}$} ({$k>0$})は必ず{$b_{j+m}$} ({$m\geq 0$}) と対応づけがなされる)という制約のもとで,対応するシンボル間の類似度の総和が最大になるような対応づけを求める.この時,{$a_{1}$}は{$b_{1}$}と,{$a_{I}$}は{$b_{J}$}と対応づけを行うこととする.最適な対応づけの探索は動的計画法を用いて効率よく計算され,結果として非線形に伸縮する系列間の類似度を計算することが可能となる.}

\addtext{ ここでは,各オノマトペをモーラの系列と捉え,2つのモーラ系列間の類似度をDTWで計算する.この時,各モーラ間の類似度は式 ({\ref{eq:cosine}}) で計算される正規化されたコサイン類似度を用いる.こうして得られた類似度の最大値を,{$D(x, y)$}と表記する.}


\item \textbf{全体の音象徴ベクトルによる類似度}

 オノマトペの表記中のモーラの並びを考慮せず,各モーラの音象徴ベクトルをすべて加算することでオノマトペの音象徴ベクトルを計算し,その正規化コサイン類似度を式 (\ref{eq:cosine}) を用いて計算する.ここで得られた類似度を,$M(x, y)$と表記する.

\end{enumerate}



\subsubsection{オノマトペ同士の類似度}

前節までに定義した4種類の類似度を用い,オノマトペ同士の類似度を計算する.ここでは,4種類の類似度の重み付き和を計算することで,2つのオノマトペ$x$,$y$間の類似度$S(x, y)$を算出する.
\begin{equation}
S(x, y) = w_{T}T(x, y) + w_{H}H(x, y) +w_{D}D(x, y) +w_{M}M(x, y)
\end{equation}
ここで,$w_{T}$,$w_{H}$,$w_{D}$,$w_{M}$はそれぞれの類似度に対する重みであり,モーラの音象徴ベクトルを計算する際に子音に与える重み$w_{c}$と合わせて調整可能なパラメータである.\addtext{各類似度はすべて,全く同じもの同士の時が1,全く異なるもの同士の時に0となるように正規化されている.しかし,各類似度がオノマトペの印象を推定する上でどの程度重要であるかはわからないため,各類似度の重要度を変化させるためのパラメータとして重みを用い,実験的に最適な値を探索することとする.}




\subsection{印象語の出力}

本論文での提案手法の基本的な考え方は,入力されたオノマトペと高い類似度を持つオノマトペを辞書内から検索し,それに付与されている印象語を出力するというものである.しかし,常に最も類似したオノマトペに付与された印象語だけを出力するだけではなく,2位以下のオノマトペに付与されている印象語ついても,その類似度に比例した重みで考慮する必要があると思われる.

そこで,以下の手順で検索を行い,最終的な印象語を出力する.
\begin{enumerate}
\item 辞書内のすべてのオノマトペと類似度計算を行い,上位$n$個のオノマトペを抽出する.
\item 抽出された各オノマトペに付与されているすべての印象語について,類似度に比例した得点を与える.具体的には,最も類似度の高いオノマトペに付与されているすべての印象語に対して,類似度と同じ値の得点をそれぞれ与える.また,2位のオノマトペに付与されているすべての印象語についても同様に得点を与える.この時,1位と2位のオノマトペ双方に同じ印象語が付与されていれば,その印象語には1位の類似度と2位の類似度が加算された得点が与えられることになる.以下同様に,$n$位のオノマトペに付与された印象語まで得点を与えていく.
\item 得られた印象語を得点の高い順に並べかえ,最も高い得点を$s$とした時,$s\times r$以上の得点を持つすべての印象語を出力する.
\end{enumerate}
ここで,$r$は0〜1の値であり,$n$とあわせて実験的に決定するパラメータである.



\section{評価と考察}

評価データには,オノマトペ辞書\cite{Book_01}に掲載されているオノマトペ1,058語を用いた.この辞書には,見出し語と共に用例や解説などが記載されている.しかし,本論文で焦点を当てるオノマトペが持っている印象については,明確に定義されていない.そこで,形状,質感,量・感覚的概念,心理的状態を表現する形容詞を24対,計48種類選出し,それらを推定する印象と定義した.選出した形容詞対を表\ref{table01}に示す.\addtext{なお,推定する印象として使用する形容詞は,複数の文献{\cite{Book_02,Article_06}}}
\addtext{{\cite{Article_07,Article_08}}}
\addtext{{\cite{Article_09}}を参考に,討議の上決定した.}オノマトペが表現していると思われる印象を選出した形容詞対から独自に判断し,すべてのオノマトペに対して人手で印象を付与した.なお,オノマトペは多義であることが多いため,複数の印象を付与することを許容している.\addtext{例えば,「チラチラ」というオノマトペの印象は,「速い」,「小さい」,「少ない」,「軽い」という形容詞の集合で表現される.}


\begin{table}[b]
\caption{オノマトペの印象を表現する形容詞対}
\label{table01}
\input{01table03.txt}
\end{table}



\subsection{人間同士の印象推定結果}\label{re}

オノマトペから抱く印象は人によってある程度の範囲でゆれが生じると考えられる.そこで,印象の付与は2名の評価者がそれぞれの感覚で独立に作業を行った.\addtext{なお,表{\ref{table01}}の左上から右下の順に横方向に5対ずつ並べてディスプレイに表示させた形容詞対とオノマトペ辞書の見出し語のみを参照しながら印象の付与作業を行った.}それぞれの評価者が作成した印象ラベル間の一致率を検証するため,一方の評価者が作成した評価データを正解とし,他方の評価者が作成した評価データの適合率と再現率ならびにF値を算出した.結果を表\ref{table02}に示す.なお,印象の平均付与数は評価者Aが3.40語,評価者Bが2.45語となった.\addtext{また,各形容詞ごとに「その形容詞が選択されたか否か」という2カテゴリ選択問題として人間同士の回答のカッパ係数を求めたところ,最大で0.602,最小で$-0.006$であり,48形容詞の平均で0.332であった.}


\begin{table}[b]
\caption{評価者間におけるオノマトペから抱く印象のゆれの検証結果}
\label{table02}
\input{01table04.txt}
\end{table}

この結果から,オノマトペから抱く人の印象にゆれが生じることが確認された.つまり,人間であっても100\%印象を特定することはできないと言える.\addtext{この結果は2名の評価者だけからの結果であるため,あくまでも参考程度の値ではあるが,本論文では,人間同士の一致率にあたるF値0.427を提案方法の性能評価を行う際の目安として用いることとする.}



\subsection{提案手法の印象推定結果}

前述のオノマトペ1,058語に対し,1語をテストデータ,残りの1,057語を辞書データとして印象語を出力する実験をテストデータを変更しながら行った.各種パラメータについては,$w_{c}$は「1,2,3」の3種類,$w_{T}$,$w_{H}$,$w_{D}$,$w_{M}$は「0,1,2,5,10,20」の6種類,$n$は「1,3,5,10,20」の5種類,$r$は「1,0.8,0.6,0.4,0.2,0」の6種類を設定し,最適なパラメータを探索した.

各類似度の有効性を検証するため,それぞれの類似度を用いる場合と用いない場合のそれぞれについて,すべてのパターンで性能を比較した.\addtext{各評価者が付与した印象語を正解とし,それぞれに対するF値を平均した値が最も高くなるように}パラメータを事後的に設定した時の結果を表\ref{results}に示す.ここで,「類似度に対する重み」が``\inzero''であるものは,その類似度を使用しないために強制的に重みを0としたことを,また ``0''であるものは,その類似度に対する重みを変化させながら実験した結果,最終的に重みが0(その類似度は使用しない)の時が最も性能が良かったことを示す.なお,各種パラメータについては,最も性能が良かったもののみをその値とともに示している.``\indc''は,そのパラメータが使われなかったことを示す.


\begin{table}[t]
\caption{提案手法によるオノマトペの印象推定結果}
\label{results}
\input{01table05.txt}
\end{table}



\subsection{考察}

2名の評価者に対する結果から最適にパラメータ調整すると提案手法のオノマトペの印象推定性能は,F値0.345であった.\addtext{この数値は,{\ref{re}}節で述べた本論文の参考値である人間同士の一致率にあたるF値0.427の8割程度にあたり,有効な手法であるといえる.}

詳細に分析すると,モーラ系列間のレーベンシュタイン距離 ($H(x,y)$) と音象徴ベクトル ($M(x,y)$) を用いる場合が最も性能が良く,この場合,子音に重みを与える必要があった.この手法では,モーラ系列間のレーベンシュタイン距離でオノマトペの構造パターンを音象徴ベクトルで音的な特徴をうまく捉えて扱うことができていると思われる.

次に良い性能であったのは,モーラ系列間のレーベンシュタイン距離とモーラの並び ($D(x,y)$) を用いる手法であった.この手法の場合,印象語を出力する際に上位10位までのオノマトペに付与された印象語を対象にする必要がある.また,3番目に良い性能であった手法は,モーラ系列間のレーベンシュタイン距離のみを用いたもので,この場合,上位20位までの印象語を対象にする必要がある.これは,多くの印象語の候補を扱うことで音的な特徴から推定される印象語を補完しようと働いていると考えられる.

\begin{table}[b]
\caption{類似度の使用の有無による印象推定性能の違い}
\label{karnaugh}
\input{01table06.txt}
\end{table}

さらに,表\ref{results}からF値だけを抜き出し,4種類の類似度を使用するか否かという条件により分類したものを表\ref{karnaugh}に示す.これより,以下のことが分かった.

\begin{itemize}
\item $H(x, y)$が最も重要である

 モーラ系列間のレーベンシュタイン距離に``○''が付いている列は``×''の列より全体的にF値が高くなっていることから,最も重要な考え方であることが分かる.

\item $M(x, y)$は$H(x, y)$と組みあわせると性能が向上する

 音象徴ベクトルは,モーラ系列間のレーベンシュタイン距離と組み合わせた時のみ性能が向上しており,良い補完関係になっているといえる.この2種類を組み合わせた手法が最も良い組み合わせである.

\item $D(x, y)$は$H(x, y)$と組みあわせると性能が少しだけ向上する
\item $D(x, y)$は$M(x, y)$と組みあわせると性能が向上するが,$H(x, y)$がある場合は不要である

 モーラの並びは,モーラ系列間のレーベンシュタイン距離または音象徴ベクトルと組み合わせると少し性能向上に寄与する.しかし,モーラの並びとモーラ系列間のレーベンシュタイン距離は同じような特徴を捉えているため,より性能の良いモーラ系列間のレーベンシュタイン距離を用いる方が効率的である.

\item $T(x, y)$は不要である

 オノマトペを抽象化した型表現は,用いたとしてもすべての組み合わせで性能が向上しないことから不用であるといえる.型表現では,オノマトペの特徴をうまく捉えることができないためであると思われる.

\end{itemize}



\addtext{本稿では,日本語を対象にオノマトペの印象推定を行ったが,オノマトペの表記内のモーラ系列間の類似度とオノマトペの表記全体の音象徴ベクトルによる類似度を用いた手法が最も良い推定結果となったことから,他の言語への対応も可能であると考えられる.他の言語では日本語程に多くのオノマトペを頻繁に利用するわけではないが{\cite{Book_03}},例えば,中国語では日本語と良く似た構造のオノマトペが利用されており,また,英語では日本語で多く見られる反復する形ではないオノマトペが利用されているが,実際に聞こえる音をアルファベットの発音に照らし合わせてオノマトペとして表現するため,日本語と同様の関連性を見出すことができると思われる.}




\section{おわりに}

本稿では,日本語のオノマトペ辞書を基に,文字の種類やパターン,音的な特徴などを手がかりに,そのオノマトペが表現している印象を自動推定できる手法を提案した.これにより,大規模な辞書を作成・利用することなく,新たに創出される未知のオノマトペの印象も推定することができる.また,日本語を\addtext{母語}としない人に対して,日本語で表現されたオノマトペの理解の支援に繋がると考えられる.さらには,機械翻訳や情報検索・推薦の分野でも活用するなどの展開が考えられる.

計4種類の類似度計算手法を提案し,7つのパラメータを実験的に調整し最適値を探索した.結果として,オノマトペの表記内のモーラ系列間の類似度とオノマトペの表記全体の音象徴ベクトルによる類似度を用いた手法が最も良い推定結果となり,\addtext{参考値である人間同士の一致率の8割程度にまで近づくことができた.}

本稿では,評価者が2名と少数であったことから,今後さらに評価者を増やして辞書構築を行い,ゆれの少ない辞書を構築する必要があると思われる.

\acknowledgment

本研究の一部は,科学研究費補助金(若手研究 (B) 24700215)の補助を受けて行った.



\bibliographystyle{jnlpbbl_1.5}
\begin{thebibliography}{}

\bibitem[\protect\BCAY{浅賀\JBA 渡辺}{浅賀\JBA 渡辺}{2007}]{Article_02}
浅賀千里\JBA 渡辺知恵美 \BBOP 2007\BBCP.
\newblock Webコーパスを用いたオノマトペ用例辞典の開発.\
\newblock \Jem{電子情報通信学会 第18回データ工学ワークショップ}, {\Bbf D9}
  (2).

\bibitem[\protect\BCAY{池田\JBA 早川\JBA 神山}{池田 \Jetal }{2006}]{Article_15}
池田岳郎\JBA 早川文代\JBA 神山かおる \BBOP 2006\BBCP.
\newblock テクスチャを表現する擬音語・擬態語を用いた食感性解析.\
\newblock \Jem{日本食品工学会誌}, {\Bbf 7}  (2), \mbox{\BPGS\ 119--128}.

\bibitem[\protect\BCAY{市岡\JBA 福本}{市岡\JBA 福本}{2009}]{Article_03}
市岡健一\JBA 福本文代 \BBOP 2009\BBCP.
\newblock Web上から取得した共起頻度と音象徴によるオノマトペの自動分類.\
\newblock \Jem{電子情報通信学会論文誌}, {\Bbf J92-D}  (3), \mbox{\BPGS\
  428--438}.

\bibitem[\protect\BCAY{芋阪}{芋阪}{1999}]{Book_06}
芋阪直行 \BBOP 1999\BBCP.
\newblock \Jem{感性の言葉を研究する—擬音語・擬態語に読む心のありか}.
\newblock 新曜社.

\bibitem[\protect\BCAY{奥村\JBA 齋藤\JBA 奥村}{奥村 \Jetal }{2003}]{Article_01}
奥村敦史\JBA 齋藤豪\JBA 奥村学 \BBOP 2003\BBCP.
\newblock Web上のテキストコーパスを利用したオノマトペ概念辞書の自動構築.\
\newblock \Jem{情報処理学会研究報告 自然言語処理研究会報告}, {\Bbf 23},
  \mbox{\BPGS\ 63--70}.

\bibitem[\protect\BCAY{加藤\JBA 青山\JBA 福田}{加藤 \Jetal }{2005}]{Article_09}
加藤雅士\JBA 青山憲之\JBA 福田忠彦 \BBOP 2005\BBCP.
\newblock 映像視聴時における感情と生体信号の関係の分析.\
\newblock \Jem{ヒューマンインタフェースシンポジウム2005}, {\Bbf 1}  (3334),
  \mbox{\BPGS\ 885--888}.

\bibitem[\protect\BCAY{小松\JBA 秋山}{小松\JBA 秋山}{2008}]{Article_05}
小松孝徳\JBA 秋山広美 \BBOP 2008\BBCP.
\newblock ユーザの直感的表現を支援するオノマトペ意図 理解システム.\
\newblock {\Bem Human-Agent Interaction Symposium 2008}, {\Bbf 2A}  (4).

\bibitem[\protect\BCAY{米谷\JBA 渡部\JBA 河岡}{米谷 \Jetal }{2003}]{Article_07}
米谷彩\JBA 渡部広一\JBA 河岡司 \BBOP 2003\BBCP.
\newblock 常識的知覚判断システムの構築.\
\newblock \Jem{人工知能学会全国大会}, {\Bbf 3C1}  (7).

\bibitem[\protect\BCAY{丹野}{丹野}{2004}]{Article_04}
丹野眞智俊 \BBOP 2004\BBCP.
\newblock Onomatopoeia(擬音語・擬態語)に関する音韻分類.\
\newblock \Jem{神戸親和女子大学児童教育学研究}, {\Bbf 23}  (1), \mbox{\BPGS\
  11--26}.

\bibitem[\protect\BCAY{土田}{土田}{2005}]{Article_10}
土田昌司 \BBOP 2005\BBCP.
\newblock オノマトペによる映像の感性評価—感性検索への応用可能性—.\
\newblock \Jem{感性工学研究論文集}, {\Bbf 5}  (4), \mbox{\BPGS\ 93--98}.

\bibitem[\protect\BCAY{得猪}{得猪}{2007}]{Book_03}
得猪外明 \BBOP 2007\BBCP.
\newblock \Jem{へんな言葉の通になる—豊かな日本語,オノマトペの世界}.
\newblock 祥伝社.

\bibitem[\protect\BCAY{早川}{早川}{2000}]{Article_14}
早川文代 \BBOP 2000\BBCP.
\newblock 性別・年齢別にみた食感覚の擬音語・擬態語.\
\newblock \Jem{日本家政学会誌}, {\Bbf 53}  (5), \mbox{\BPGS\ 437--446}.

\bibitem[\protect\BCAY{飛田\JBA 浅田}{飛田\JBA 浅田}{2001}]{Book_02}
飛田良文\JBA 浅田秀子 \BBOP 2001\BBCP.
\newblock \Jem{現代形容詞用法辞典}.
\newblock 東京堂出版.

\bibitem[\protect\BCAY{飛田\JBA 浅田}{飛田\JBA 浅田}{2002}]{Book_01}
飛田良文\JBA 浅田秀子 \BBOP 2002\BBCP.
\newblock \Jem{現代擬音語擬態語用法辞典}.
\newblock 東京堂出版.

\bibitem[\protect\BCAY{前川\JBA 吉田}{前川\JBA 吉田}{1998}]{Article_08}
前川純孝\JBA 吉田光雄 \BBOP 1998\BBCP.
\newblock SD 法による流行歌の聴取印象評価—探索的・検証的因子分析—.\
\newblock \Jem{国際関係学部紀要}, {\Bbf 1}  (23), \mbox{\BPGS\ 93--108}.

\bibitem[\protect\BCAY{村上}{村上}{1980}]{Article_12}
村上宣寛 \BBOP 1980\BBCP.
\newblock
  音象徴仮説の検討:音素,SD法,名詞及び動詞の連想語による成分の抽出と,それら
のクラスター化による擬音語・擬態語の分析.\
\newblock \Jem{教育心理学研究}, {\Bbf 28}  (3), \mbox{\BPGS\ 183--191}.

\bibitem[\protect\BCAY{山内\JBA 岩宮}{山内\JBA 岩宮}{2005}]{Article_13}
山内勝也\JBA 岩宮眞一郎 \BBOP 2005\BBCP.
\newblock 周波数変調音の擬音語表現とサイン音としての機能イメージ.\
\newblock \Jem{日本生理人類学会誌}, {\Bbf 10}  (3), \mbox{\BPGS\ 115--122}.

\bibitem[\protect\BCAY{山口}{山口}{2003}]{Book_04}
山口仲美 \BBOP 2003\BBCP.
\newblock \Jem{暮らしのことば—擬音・擬態語辞典}.
\newblock 講談社.

\bibitem[\protect\BCAY{湯澤\JBA 松崎}{湯澤\JBA 松崎}{2004}]{Book_05}
湯澤質幸\JBA 松崎寛 \BBOP 2004\BBCP.
\newblock \Jem{音声・音韻探求法}.
\newblock 朝倉書店.

\bibitem[\protect\BCAY{吉村\JBA 関口}{吉村\JBA 関口}{2006}]{Article_11}
吉村浩一\JBA 関口洋美 \BBOP 2006\BBCP.
\newblock オノマトペで捉える逆さめがねの世界.\
\newblock \Jem{法政大学文学部紀要}, {\Bbf 54}  (1), \mbox{\BPGS\ 67--76}.

\bibitem[\protect\BCAY{渡部\JBA 堀口\JBA 河岡}{渡部 \Jetal }{2004}]{Article_06}
渡部広一\JBA 堀口敦史\JBA 河岡司 \BBOP 2004\BBCP.
\newblock 常識的感覚判断システムにおける名詞からの感覚想起手法 理解システム.\
\newblock \Jem{人工知能学会論文誌}, {\Bbf 19}  (2), \mbox{\BPGS\ 73--82}.

\end{thebibliography}

\begin{biography}
\bioauthor{土屋 誠司}{
2000年同志社大学工学部知識工学科卒業.
2002年同大学院工学研究科知識工学専攻博士前期課程修了.
同年,三洋電機株式会社入社.
2007年同志社大学大学院工学研究科知識工学専攻博士後期課程修了.
同年,徳島大学大学院ソシオテクノサイエンス研究部助教.博士(工学).
2009年同志社大学理工学部インテリジェント情報工学科助教.
2011年同准教授.
主に,知識処理,概念処理,意味解釈の研究に従事.
言語処理学会,人工知能学会,情報処理学会,日本認知科学会,電子情報通信学会各会員.
}
\bioauthor{鈴木 基之}{
1993年東北大学工学部情報工学科卒業.
1996年同大大学院博士後期課程を退学し,同大大型計算機センター助手.
2006年〜2007年英国エジンバラ大学客員研究員.
2008年徳島大学大学院ソシオテクノサイエンス研究部准教授,
現在に至る.博士(工学).
音声認識・理解,音楽情報処理,自然言語処理,感性情報処理等の研究に従事.
電子情報通信学会,情報処理学会,日本音響学会,ISCA各会員.
}
\bioauthor{任  福継}{
1982 年北京郵電大学電信工程学部卒業.
1985 年同大学大学院計算機応用専攻修士課程修了.
1991 年北海道大学大学院工学研究科博士後期課程修了.
博士(工学).
広島市立大学助教授を経て,2001 年より徳島大学工学部教授.
現在に至る.
自然言語処理,感性情報処理,人工知能の研究に従事.
情報処理学会,人工知能学会,言語処理学会,電気学会,AAMT,IEEE 各
会員.日本工学会フェロー.
}
\bioauthor{渡部 広一}{
1983年北海道大学工学部精密工学科卒業.
1985年同大学院工学研究科情報工学専攻修士課程修了.        
1987年同精密工学専攻博士後期課程中途退学.
同年,京都大学工学部助手.
1994年同志社大学工学部専任講師.
1998年同助教授.
2006年同教授.工学博士.
主に,進化的計算法,コンピュータビジョン,概念処理などの研究に従事.
言語処理学会,人工知能学会,情報処理学会,電子情報通信学会,システム制御情報学会,精密工学会各会員.
}
\end{biography}


\biodate




\end{document}
