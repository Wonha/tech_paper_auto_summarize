


\documentstyle[epsf,jnlpbbl]{jnlp_j_b5}




\setcounter{page}{139}
\setcounter{巻数}{10}
\setcounter{号数}{5}
\setcounter{年}{2003} 
\setcounter{月}{10}
\受付{2003}{5}{26}
\採録{2003}{7}{14}

\title{連想概念辞書の距離情報を用いた重要文の抽出}
\author{岡本 潤\affiref{SFC} \and 石崎 俊\affiref{SFC}}

\headauthor{岡本,石崎}

\headtitle{連想概念辞書の距離情報を用いた重要文の抽出}

\affilabel{SFC}{慶應義塾大学政策・メディア研究科}{Graduate School of Media and Governance, Keio University}


\jabstract{
大量の文書情報の中から必要な部分を入手するために,自動要約技術などによって文書の量を制御し,短い時間で適確に内容を把握する必要性が高くなってきている.自動要約を行なうには文書中のどの箇所が重要なのかを判断する必要があり,従来の重要文の抽出方法には単語の出現頻度にもとづいた重要語の計算方法などがある.本論では連想概念辞書における,上位/下位概念,属性概念,動作概念などの連想関係を用いて文書中の単語の重要度を計算し重要文を抽出する手法を提案して有効性を示す.連想概念辞書は,小学校の学習基本語彙を刺激語とし大量の連想語を収集して構造化すると同時に,その連想語との距離が定量化されている.また既存の重要語抽出法と本手法での抽出結果とを,人間が行なった要約結果と比較することによって評価した.従来の手法に比べて連想関係を計算に含めることによって要約精度が人間の要約に近く,本手法によって改良されることがわかった.
}


\jkeywords{連想概念辞書,連想関係,概念間の距離,単語の重要度,重要文抽出}

\etitle{Evaluating a Method \\ of Extracting Important Sentences \\ using
Distance between Entries \\ in an Associative Concept Dictionary
}
\eauthor{Jun Okamoto\affiref{SFC} \and Shun Ishizaki\affiref{SFC}}


\eabstract{
In this paper, we propose a method for calculating scores of
importance for sentences for text summarization purposes. In this
method, scores for sentences are calculated based on quantitative
distance information in an associative concept dictionary, which
includes about 160,000 associated concepts. Eight articles are used
to evaluate our method.The articles are chosen form Japanese elementary school textbooks because the dictionary was constructed using basic nouns in the textbooks. In order to evaluate the quality of the importance score ranking resulting from our method and other conventional methods using term frequency (tfidf), we carry out experiments where 40 human subjects chose the five most important sentences from each of the eight articles. The evaluation results show that sentences chosen by our method using association relationships is more comparable to those chosen by human subjects. The results show that summarization accuracy can be improved by applying our method.
}



\ekeywords{Associative Concept Dictionary, Semantic Relation, Distance between Concepts, Importance Score of Words, Important Sentence Extraction}

\begin{document}

\maketitle
\thispagestyle{empty}


\section{はじめに}

大量の文書情報の中から必要な部分を抽出するために,自動要約技術などによって文書の量を制御し,短い時間で適確に内容を把握する必要性が高くなってきている.自動要約には,文書中の文を単位とし,なんらかの情報をもとに重要語を定義して各文の重要度を計算する方法がある.たとえば,文書中の出現頻度が高い単語は重要語になる可能性が高いという仮定のもとに,単語の重要度を計算する方法({\it tfidf}法)\cite{salton1989},自立語の個数を考慮して単語の重要度を計算する方法\cite{robertson1997},語彙的連鎖を用い重要度を計算する方法\cite{mochizuki2000}がある.新聞など文書の構造上の特徴から重要文を抽出する方法や,主張,結論,評価などの特別な語を含む文を重要文とする方法など,文書の重要な記述部分を示す語を含む文や,その文に含まれる単語の重要度を他の単語より上げる方法がある\cite{watanabe1996}.その他にも,接続詞,照応関係などから文間・単語間のつながりを解析し要約する方法,文書を意味ネットワーク化して,その上でコネクショニスト・モデルを用いて,接点の活性値の収束値を重要度として計算する要約方法\cite{Hasida1987,Nagao1998}などがある.

要約文の表示には文書中の文単位で重要度を計算し,文書中での出現順にあわせて重要な文を提示していくという方法をとるものや,重要な語句・文・パラグラフなどの単位で抽出・表示するものが多い.複数の文を接続詞などでつなげてまったく新しい要約文書を生成するものは少ないが,文脈,文の重要部分,または構造を考慮して,重要文をさらに小さい単位で表示するシステムも出てきている\cite{nomura1999}.

システムが抽出した要約文を評価する方法には,人間の被験者の要約と{\it tfidf}法などでシステムが抽出した要約とを再現率/適合率によって比較する方法\cite{Zechner1996},様々な手法で抽出された要約文を利用して,ある種のタスクを行ないその達成率で間接的に評価を行なう方法\cite{mochizukiLREC2000},要約は読み手の観点によって変化することに着目して複数の正解に基づいて評価する方法\cite{ishikawa2002}などがある.

本論では,連想概念辞書をもとに,単語と単語の連想関係とその距離情報を使って文書中の単語の重要度を計算し,各文ごとの重要度を求め重要文の抽出を行なう.連想概念辞書は,小学校の学習基本語彙の名詞を刺激語とし,「上位概念」「下位概念」「部分・材料概念」「属性概念」「類義概念」「動作概念」「環境概念」の7つの課題に関して,大量の連想語を収集して構造化すると同時に,刺激語と連想語との距離が定量化されている\cite{Okamoto2001}.連想概念辞書の規模は見出し語が約660語,連想語が延べで約16万語である.単語の重要度の計算は,その単語の連想語もしくはその単語を連想する刺激語が文書中にあれば,二つの単語の距離から得られる値を使用して重要度を計算する.たとえば,「ガラパゴスには巨大な\underline{ゾウガメ}がいる.この\underline{カメ}は,島の中を悠然と歩いている.」のように「ゾウガメ」の上位概念である「カメ」を用いて言い換えている場合,「カメ」「ゾウガメ」の重要度を二つの単語間の距離に基づいて計算する.これによって,表層的に文書中の単語の出現頻度をもとにした重要度の計算では別の単語として処理されるが,本手法では上位/下位概念や部分・材料概念,属性概念,動作概念,環境概念などの連想関係も用いているので関連する単語の重要度を精密に計量することができる.

次に人間を被験者として重要文を抽出する実験を行なう.被験者の観点によって抽出される重要文が違ってくる場合があるが,40人の被験者で実験を実施し,多くの被験者が上位に抽出している順番を重視して重要度を決定した.本論文では,既存の重要語抽出法と本手法での抽出結果とを,被験者による実験結果との一致度を比較することによって評価した.



\section{連想概念辞書の使用}

本論文では,人間を被験者として大規模な連想実験を行ない,実験より得られたデータをもとに構築した連想概念辞書\cite{Okamoto1998,Okamoto1999,Okamoto2001}を使用した.

\subsection{連想実験}

連想実験は自由連想ではなく,被験者に名詞を刺激語として呈示して,「上位概念」「下位概念」「部分・材料概念」「属性概念」「類義概念」「動作概念」「環境概念」の7つ課題に関して連想させ,任意の個数の連想語をキーボード入力させる.刺激語として小学校の国語の教科書から学習基本語彙の名詞を選択し,さらに学習基本語彙以外の実験に用いる名詞の計約660語を刺激語セットとした.被験者数は1刺激語に対し50人である.

刺激語と連想語との概念間の距離$D$は連想実験から得られる連想頻度$F$,連想順位$S$のパラメータによる線形結合で表現し,線形計画法を用いて(1)式のように最適解が求められている\cite{Okamoto2001}.ここではパラメータをもとに境界条件を距離$D$の値が最大で10.0程度,最小で1.0程度になるように定め,シンプレックス法で計算している.

\begin{equation}
\label{D2}
D=0.81 \times F + 0.27 \times S ,
\end{equation}

\begin{center}
\begin{tabular}{l}
$F = \frac{N}{n+\delta}$,\\

$\delta = \frac{N}{10} - 1 ~~~~ (N \geq 10)$,\\

$S = \frac{1}{n}\sum^{n}_{i=1} s_{i}$,\\

\end{tabular}
\end{center}


ここで刺激語をA,連想語をBとした時,$F$は連想語Bを連想した被験者の割合,$S$は連想語Bが連想された順位の平均した値,$n$は連想人数 (n$\geq$1),$N$は被験者数,$s_{i}$は被験者$i$が連想した語の順位
である.(\ref{D2})式では連想頻度$F$の係数が連想順位$S$の係数より大きく,連想人数が概念間の距離に与える影響は大きい.多くの被験者が同一の語を連想している場合は,その連想語は刺激語にとって連想しやすい語であると考えられ,概念間の距離も短くなる.


\subsection{連想概念辞書の記述形式}

\begin{figure}[htb]
	\begin{center}
		\epsfile{file=okamoto_graph2.eps,width=138mm,height=69mm}
		\caption{刺激語「いす」に関する連想概念辞書の記述例(連想語は一部のみ表示)}
		\label{isu}
	\end{center}
\end{figure}

図 \ref{isu} は刺激語「いす」についての連想概念辞書の記述例である.「いす」の上位概念として,まず「家具」が連想されており,続く右側の4つの数字は順に頻度(連想者数を被験者数で割った値),連想順位,正規化された連想時間,「いす」と「家具」の概念間の距離である.「上位概念」の他に「下位概念」「部分・材料概念」「属性概念」「類義概念」「動作概念」「環境概念」「関連語」の課題も同じ形式で記述してある.「関連語」は7つの課題に分類できない連想語がある場合にもうけた課題である.たとえば刺激語「犬」に対しての連想語「猫」などは「関連語」とする.used-inとは刺激語が他の刺激語の連想語となっていた場合,逆引き情報として元の刺激語と課題を記述するものである.概念間の距離は,連想順位$S$の値にもよるが,おおよそ$1$〜$10$の間にある\cite{Okamoto2001}.



\section{単語の重要度による重要文抽出}

\subsection{単語の出現頻度に基づく主な重要文抽出手法}

文書中で出現頻度が高い単語はその文書を特徴づける重要な単語であるという仮定により,単語の出現頻度や出現文書の数をもとに各単語の重要度を計算\cite{salton1989}し,重要文を抽出する方法などがある.また,タイトルも文書の内容に大きく関連しているとし,{\it tfidf}法にタイトルに含まれる単語に重み$\alpha_{i}$を加えて重要度を計算する方法も考えられる.ここでは本論文の抽出手法と比較するために{\it tfidf}法と,タイトルも考慮に入れた${\it tfidf}+\alpha_{i}$を用いて単語の重要度に基づく重要文抽出を行なう.
\begin{itemize} 
\item[{\bf (1)}] {\bf {\it tfidf}法を用いた単語の重要度の計算}
\end{itemize}

タイトルを考慮しない場合で単語の重要度$P_{ijk}$を以下の式で計算する.

\begin{equation}
\label{tfidf}
P_{ijk} = \sum_{k=1}^{L_{ij}} \sum_{j=1}^{M_{i}} F_{ijk}*log\frac{N}{n_{ijk}}
\end{equation}

\begin{center}
	\begin{tabular}{l}
		$i = 1, 2, \cdots N$, \hspace*{1em}$N$は文書の数.\\

		$j = 1, 2, \cdots M_{i}$, \hspace*{1em}$M_{i}$は文書$i$中の文の数.\\

		$k = 1, 2, \cdots L_{ij}$, \hspace*{1em}$L_{ij}$は文書$i$中の文$j$の単語数.\\
	\end{tabular}
\end{center}

単語$w_{ijk}$を,文書$i$中の$j$番目の文中の$k$番目の単語として,$P_{ijk}$は単語$w_{ijk}$の重要度,$F_{ijk}$は,ある単語$w_{ijk}$がその文書$i$中に現れる頻度,$n_{ijk}$は単語$w_{ijk}$が一つでも現れる文書の数である.

\begin{itemize} 
\item[{\bf (2)}] {\bf タイトルを考慮に入れ,{\it tfidf}法を用いた単語の重要度の計算}
\end{itemize}

文書のタイトルを考慮に入れて,{\it tfidf}に重み$\alpha_{i}$を加えて単語の重要度$P_{ijk}$を計算する.重み$\alpha_{i}$は,タイトルに含まれる単語のうち一番高い重要度の数値とした.

\begin{equation}
\label{tfidf2}
P_{ijk} =	\left\{ \begin{array}{ll}
\sum_{k=1}^{L_{ij}} \sum_{j=1}^{M} F_{ijk}*log\frac{N}{n_{ijk}} & タイトルに含まれない単語の場合 \\
\sum_{k=1}^{L_{ij}} \sum_{j=1}^{M} [ F_{ijk}*log\frac{N}{n_{ijk}} +
\alpha_{i} ] & タイトルに含まれる単語の場合\\
\end{array}
\right .
\end{equation}

\begin{center}
\begin{tabular}{l}
$i = 1, 2, \cdots N$, \hspace*{1em}$N$は文書の数.\\

$j = 1, 2, \cdots M_{i}$, \hspace*{1em}$M_{i}$は文書$i$中の文の数.\\

$k = 1, 2, \cdots L_{ij}$, \hspace*{1em}$L_{ij}$は文書$i$中の文$j$の単語数.\\
\end{tabular}
\end{center}

$\alpha_{i}$は文書$i$におけるタイトルを考慮する場合の重要度の付加部分である.

次に,(\ref{tfidf})または(\ref{tfidf2})式により得られた重要度の総和を次式で計算し,重要度の高い文の抽出を行なう.


\begin{equation}
T_{ij} = \sum_{k=1}^{L_{ij}} P_{ijk} 
\end{equation}

\begin{center}
\begin{tabular}{l}
$i = 1, 2, \cdots N$, \hspace*{1em}$N$は文書の数.\\

$j = 1, 2, \cdots M_{i}$, \hspace*{1em}$M_{i}$は文書$i$中の文の数.\\

$k = 1, 2, \cdots L_{ij}$, \hspace*{1em}$L_{ij}$は文書$i$中の文$j$の単語数.\\
\end{tabular}
\end{center}

$T_{ij}$は,文書$i$における文$j$の重要度,$P_{ijk}$は,単語$w_{ijk}$の重要度,$L_{ij}$は文書$i$中の文$j$の単語数である.


\subsection{単語の連想関係とその距離情報を用いた重要文抽出法}

本論文では,単語の頻度などによって単語の重要度を計算する手法に加えて連想概念辞書の刺激語と連想語の関係を使用してそれらの間の距離情報を用いて単語の重要度を計算する\cite{Okamoto2002}.具体的には,以下のような方針で単語の重要度を計算し,重要文を抽出する.

\begin{enumerate}

\item {茶筅\cite{chasen1999}を用いて文書の形態素解析を行ない修正を加え,単語ごとに基本形,品詞などの情報を得る.}

\item {文書中の単語が名詞,形容詞,副詞,動詞ならば,その頻度を単語の重要度$w_{ijk}$として計算する.ただし名詞のうち茶筅の出力結果で名詞-代名詞,名詞-接尾,名詞-非自立,名詞-特殊(茶筅の品詞名による)とあるものは除く.}
\item {文書中の単語が名詞で,連想概念辞書の刺激語の場合.ただし名詞のうち茶筅の出力結果で上記のものは同様に除く.次に,単語を刺激語として,その連想語が文書中にあれば,刺激語−連想語間の距離の逆数を単語の重要度として加算する.}

\item {文書中の単語が名詞,形容詞,副詞,動詞で,連想概念辞書の連想語の場合.ただし名詞のうち茶筅の出力結果で上記のものは同様に除く.次に,単語を連想する刺激語が文書中にあれば,刺激語−連想語間の距離の逆数を単語の重要度として加算する.}

\item {タイトルに含まれる単語の場合は重み$\alpha_{i}$を加算する.重み$\alpha_{i}$は,タイトルに含まれる単語のうち一番高い重要度の数値とした.}

\begin{equation}
P_{ijk} = \left\{ \begin{array}{l}
			\sum_{k=1}^{L_{ij}} \sum_{j=1}^{M_{i}} [
			F_{ijk} + 1/dist(w_{ijk},a_{ijk}) +
			1/dist(s_{ijk},w_{ijk}) ] \\ \hspace*{5cm} \cdots タイトルに含まれない単語の場合\\
			\sum_{k=1}^{L_{ij}} \sum_{j=1}^{M_{i}} [
			F_{ijk} + 1/dist(w_{ijk},a_{ijk}) +
			1/dist(s_{ijk},w_{ijk}) + \alpha_{i} ] \\ \hspace*{5cm} \cdots タイトルに含まれる単語の場合\\
			\end{array}
		  \right .
\end{equation}
\begin{center}

\begin{tabular}{l}
$i = 1, 2, \cdots N$, \hspace*{1em}$N$は文書の数.\\

$j = 1, 2, \cdots M_{i}$, \hspace*{1em}$M_{i}$は文書$i$中の文の数.\\

$k = 1, 2, \cdots L_{ij}$, \hspace*{1em}$L_{ij}$は文書$i$中の文$j$の単語数.\\
\end{tabular}
\end{center}

ここで,$F_{ijk}$は文書中の単語$w_{ijk}$の頻度,$dist(w_{ijk},
a_{ijk})$は単語$w_{ijk}$を刺激語として,その連想語$a_{ijk}$が文書中
にある場合の刺激語−連想語間の距離,$dist(s_{ijk}, w_{ijk})$は単語$w_{ijk}$を連想する刺激語$s_{ijk}$が文書中にある場合の刺激語−連想語間の距離である.

\newpage

\item {各文での単語$w_{ijk}$の重要度$P_{ijk}$の合計をその単語を含む文の単語数$L_{ij}$で割ることで,文の重要度$T_{ij}$を計算する.}
\end{enumerate}


\begin{equation}
T_{ij} = \frac{\sum_{i}P_{ijk}}{L_{ij}}
\end{equation}
\begin{center}

\begin{tabular}{l}
$i = 1, 2, \cdots N$, \hspace*{1em}$N$は文書の数.\\

$j = 1, 2, \cdots M_{i}$, \hspace*{1em}$M_{i}$は文書$i$中の文の数.\\

$k = 1, 2, \cdots L_{ij}$, \hspace*{1em}$L_{ij}$は文書$i$中の文$j$の単語数.\\
\end{tabular}
\end{center}


たとえば「ガラパゴスには,巨大な\underline{ゾウガメ}がいる.この\underline{カメ}は,島の中を悠然と歩いている.」のように「ゾウガメ」の上位概念である「カメ」で言い換えている場合,「カメ」「ゾウガメ」に二つの単語間の距離に基づいた値を使用して単語の重要度を計算する.これによって,表層的に文書中の単語の出現頻度をもとにした重要度の計算では別の単語として処理されるが,本手法では連想関係も用いているので関連する単語の重要度を上げることができる.同様に「ゾウガメ」の環境概念である「ガラパゴス」や「ゾウガメ」「カメ」の動作概念である「歩く」などの連想関係も使用して重要度を計算する.連想概念辞書は連想実験を用いて抽出した人間の知識をもとに構築しているので,規模は小さいが人間の直感に近い単語間の連想関係が抽出できていると考えている.


\section{本手法の評価}


連想概念辞書は小学校の基本語彙をもとに構築している.そこで重要文抽出の対象として小学校の国語の教科書から自然科学に関する説明文として扱われている8文書を選択した.使用した8文書はタイトルを含み本文は平均17文からなり,最短9文,最長22文の文書である.また,8文書中の約7割の名詞は連想概念辞書に刺激語として含まれている語である.残りの約3割は名詞-代名詞,名詞-接尾,名詞-非自立,名詞-特殊,名詞-固有名詞(茶筅の品詞名による)などである.

本章では,単語の連想関係と距離情報を使い重要度を計算する本手法での抽出結果と既存の重要語抽出法とを,被験者による要約の実験結果と比較することによって評価した.

\subsection{人間の被験者による重要文抽出実験}\label{human_subject_extraction}


重要文抽出の対象とする8文書について,40人の被験者を用いて重要文抽出実験を行なった.被験者には各文書と回答欄を一目で見渡せるようにした回答用紙を配布し,「タイトル」以外の本文のうち,最も重要な文だとする順に5つの文を選択させ,文番号を記述させた.被験者によって抽出される重要文は異なると予想されるが,文書ごとに各被験者が最も重要だとした文に5点,順に2番目には4点とし,5番目に重要だとした文に1点を与え,各文ごとの合計点数が高い順に重要文の順位を決定する.また,ケンドールの一致係数を用いて,各文書において被験間の順位付けの一致度を計算した.ケンドールの一致係数はある対象に対して順位づけがされているときに順位の一致度を表す指標である.値が高いほど順位が一致していることを示す.

\begin{table}[htb]
\begin{center}
\caption{各文書における被験者間の順位づけに対するケンドールの一致係数}
\label{kendole}
\vspace*{1ex}

\begin{tabular}{|r||r|r|r|r|r|r|r|r|} \hline
文書番号&D1&D2&D3&D4&D5&D6&D7&D8\\ \hline 
文の数& 15&	18& 22& 21& 9& 9& 19& 18 \\ \hline
一致係数&0.44&0.45&0.35&0.38&0.57&0.54&0.37&0.48\\ \hline
有意水準$1$\,$\%$&***&***&***&***&***&***&***&***\\ \hline
\end{tabular}
\end{center}
\end{table}


表\ref{kendole}は40人の被験者による順位づけに関するのケンドールの一
致係数と本文の文の文書ごとに総数を示したものである.被験者が一人も選
ばなかった文は省き,1〜5番以外の順位に関しては加重平均し同順位として,
一致係数を計算した.D3, D4, D7の3文書は他の5文書より被験者間のケンドー
ルの一致係数が比較的低い.これはD1, D2, D5, D6, D8の5文書の重要文の
順位付けより,D3, D4, D7の3文書の順位付けの方が,被験者によって比較的ばらついた傾向にあることを示す.しかし,文書ごとの一致係数に関して$\chi^2$検定を行なうと$1$\,$\%$有意水準ですべての文書で有意な結果が得られ,被験者間の順位付けの一致度は比較的高いことがわかった.


\subsection{人間による重要文抽出結果を用いた既存の抽出法と本手法との比較}

連想概念辞書の距離情報を用いる本手法と,{\it tfidf},${\it tfidf}+\alpha_{i}$の2つの手法を用いて各文書で上位5番目までに抽出された文について被験者による結果との一致度を(\ref{c})式を用いることによって比較した.

\begin{equation}
\label{c}
C=\sum^{5}_{i=1} [ R_{i}(hs, m) - \Delta r_{i}(hs, m) ] 
\end{equation}

$R_{i}(hs, m)$は,各々の手法で上位5文に選択された文と被験者が選択し
た上位5文が一致する程度を表す関数で点数で表現される.被験者による実
験で1番目に重要された文を10点,2番目以下順に1点減らしていき,5番目に
重要された文を6点とする.$\Delta r_{i}(hs, m)$は被験者が選んだ文の順
位と,各手法で選ばれた文の順位との差である.$C$は被験者による実験結
果と各々の手法との一致度を点数で表したもので被験者と全く同じ結果の場
合40点に,被験者が選択した上位5文と同一の文を抽出しなかった場合は0点
となる.たとえば,被験者による文番号の抽出結果が(15, 10, 2, 6, 8)に,
ある手法による文番号の抽出結果が(15, 8, 10, 14, 5)となった時は,(\ref{c})式の計算から表\ref{c-example}のような結果が得られる.


\begin{table}[htb]
\begin{center}
\caption{重要文の抽出結果と被験者による抽出結果に対する一致点数の計算例}
\label{c-example}

\vspace*{1ex}

\begin{tabular}{|r||r|r|r|r|r||r|} \hline
順位&1&2&3&4&5&一致点数($C$)\\ \hline
被験者による結果&\underline{15}&\underline{10}&2&6&\underline{8}&-\\ \hline
抽出手法による結果&\underline{15}&\underline{8}&\underline{10}&14&5&21\\ \hline
\end{tabular}

\begin{tabular}{l}
$C = 10+(9-1)+(6-3) = 21$ となる.\\
一致する文数は3となる.\\
\end{tabular}
\end{center}
\end{table}

\begin{table}[htb]
\begin{center}
\caption{文書ごとの被験者による抽出結果に対する一致点数と文の数}
\label{result}
\vspace*{1ex}
\label{c-all}
\begin{tabular}{|r|r||r|r|r|r|r|r|r|r||r|} \hline
&文書番号&D1&D2&D3&D4&D5&D6&D7&D8&平均\\ \hline 
{\small 本手法による結果}&C&{\bf30}&22&{\bf15}&{\bf18}&{\bf32}&27&{\bf21}&{\bf24}&{\bf23.6}\\ \cline{2-11}
&一致する文数& {\bf4}& 3& {\bf2}& {\bf3}& {\bf5}& {\bf4}& {\bf3}& {\bf3}&{\bf3.6}\\ \hline 
{\small ${\it tfidf}+\alpha_{i}$による結果}&C&29&{\bf30}&{\bf15}&13&28&{\bf29}&13&17&21.8\\ \cline{2-11}
&一致する文数&{\bf4}& {\bf4}&{\bf2}& 2& 4& {\bf4}& 2& 2&3.0\\ \hline 
{\small {\it tfidf}による結果}&C&  29&24&7&13&28&{\bf29}&10&17&20.5\\ \cline{2-11}
&一致する文数&{\bf4}& 3&1&2& 4& {\bf4}&2& 2&2.8\\ \hline
\end{tabular}
\end{center}
\end{table}

表\ref{c-all}は各手法ごとに(\ref{c})式で計算した一致点数と,被験者が抽出した上位5文に一致する文の数である.8文書で一致点数($C$)と抽出率を平均すると,タイトルを考慮した${\it tfidf}+\alpha_{i}$法やタイトルを考慮しない{\it tfidf}法よりもよい結果を得た.個々に見るとD2を除いた文書については本手法は他の手法と同等か,もしくはより良い結果が得られた.D2では「自然」「生物」といった抽象度の高い語や固有名詞の出現頻度が高かった.連想概念辞書では固有名詞を刺激語として扱っていないため,固有名詞が本文やタイトルに含まれていても,高い重要度を得られない.D6には「地球」といった自然物の環境概念と
被験者による重要文抽出実験においてケンドールの一致係数が比較的低かっ
たD3, D4, D7(第\ref{human_subject_extraction}節)はタイトルを考慮した${\it tfidf}+\alpha_{i}$法やタイトルを考慮しない{\it tfidf}法での被験者の抽出結果に対する一致点数や一致する文の数が他の文書と比べて低かったが,本手法では同等か,もしくはより良い結果が得られている.単語の出現頻度だけで重要度を計算するより,単語の連想関係を考慮に入れて計算する本手法の方が一致点数でも,被験者が抽出した上位5文に一致する文の数でも良い結果を得た.


図\ref{document}は,重要文抽出の対象とした文書の例である.文番号1の「動物の赤ちゃん」は文書のタイトルである.
この文書に出現する単語は比較的平易で連想しやすい語が多く,また刺激語とそ
の連想語が文書中に含まれる可能性が高い.そこで単語の連想関係と距離情報
を用いる本手法の特徴を反映していると考えられる.

図\ref{document}の文書での被験者の抽出結果に対する各手法の一致点数と一致する文の数は,表\ref{result}のD7の文書にあたる.
図\ref{human}は図\ref{document}の文書において,40人の被験者による重要文抽出実験から得られた上位5文である.
また,図\ref{new-method}は本手法での重要文抽出結果,図\ref{tfidf+a}はタイトルを考慮に入れた{\it tfidf}法
での重要文抽出結果である.
図\ref{human}と図\ref{new-method}から式\ref{c}によって被験者の抽出結果に対する本手法の一致点数と一致する文の数を計算する(表\ref{result}中のD7の「本手法による結果」).
同様に,図\ref{human}と図\ref{tfidf+a}からタイトルを考慮に入れた{\it tfidf}法の一致点数と一致する文の数を計算する(表\ref{result}中のD7の「${\it tfidf}+\alpha_{i}$による結果」).
タイトルを考慮に入れた{\it tfidf}法では,「お母さん」「赤ちゃん」など文書中に数多く出現する語を含み比較的長い文を重要文とする傾向にある.
本手法では,「赤ちゃん」から連想される動作概念である「生まれる」という語の重要度が高くなっていた.



\begin{figure}[htb]

\begin{small}

\begin{tabular}{|r|p{350pt}|}\hline
文番号&	本文 \\ \hline
     1&	動物の赤ちゃん。\\
     2&	動物の赤ちゃんは生まれたばかりのときは、どんな様子をしているのでしょう。 \\
     3&	そして、どのようにして大きくなっていくのでしょう。\\
     4&	ライオンの赤ちゃんは、生まれたときには、子猫ぐらいの大きさです。\\
     5&	目や、耳は、閉じたままです。\\
     6&	ライオンは動物の王様、といわれます。\\
     7&	けれども、赤ちゃんは、弱々しくて、お母さんにあまり似ていません。\\
     8&	ライオンの赤ちゃんは自分で歩くことができません。\\
     9&	よそへ行くときは、お母さんに、口にくわえて運んでもらうのです。\\
    10&	ライオンの赤ちゃんは、生まれて二ヶ月くらいは、お乳だけ飲んでいますが、やがてお母さんのとった獲物を食べ始めます。\\
    11&	一年くらい経つと、お母さんがするのを見て、獲物のとり方を覚えます。\\
    12&	そして、自分で捕まえて食べるようになります。\\
    13&	シマウマの赤ちゃんは、生まれたときに、もうヤギくらいの大きさがあります。\\
    14&	目は開いていて、耳のピンと立っています。\\
    15&	縞の模様もついていて、お母さんにそっくりです。\\
    16&	シマウマの赤ちゃんは、生まれて、三十分も経たないうちに、自分で立ち上がります。\\
    17&	そして、次の日には走るようになります。\\
    18&	だから、強い動物に襲われても、お母さんと一緒に逃げることができるのです。\\
    19&	シマウマの赤ちゃんが、お母さんのお乳だけ飲んでいるのはたったの七日ぐらいの間です。\\
    20&	その後は、お乳も飲みますが、自分で草も食べるようになります。\\ \hline
\end{tabular}
\end{small}

\caption{重要文抽出の対象とした文書の例}\label{document}


\end{figure}
※文番号1の「動物の赤ちゃん」は本文書のタイトル

※参考:38 光村 124 国語 小学校国語科用,光村図書出版株式会社,平成8
年.


\begin{figure}[htb]

\begin{small}
\begin{tabular}{|r|p{350pt}|} \hline
文番号&	重要な文の順に表示 \\ \hline
     2&	動物の赤ちゃんは生まれたばかりのときは、どんな様子をしているのでしょう。 \\
     4&	ライオンの赤ちゃんは、生まれたときには、子猫ぐらいの大きさです。\\
     3&	そして、どのようにして大きくなっていくのでしょう。\\     
    13&	シマウマの赤ちゃんは、生まれたときに、もうヤギくらいの大きさがあります。\\
    16&	シマウマの赤ちゃんは、生まれて、三十分も経たないうちに、自分で立ち上がります。\\ \hline  
\end{tabular}
\end{small}

\caption{人間による重要文抽出結果}\label{human}

\end{figure}     

\begin{figure}[htb]

\begin{small}
\begin{tabular}{|r|p{350pt}|} \hline
文番号&	重要な文の順に表示 \\ \hline
     6&	ライオンは動物の王様、といわれます。\\
     8&	ライオンの赤ちゃんは自分で歩くことができません。\\
     2&	動物の赤ちゃんは生まれたばかりのときは、どんな様子をしているのでしょう。\\
    13&	シマウマの赤ちゃんは、生まれたときに、もうヤギくらいの大きさがあります。 \\
     4&	ライオンの赤ちゃんは、生まれたときには、子猫ぐらいの大きさです。\\ \hline
\end{tabular}
\end{small}

\caption{本手法での重要文抽出結果}\label{new-method}

\end{figure}

\begin{figure}[htb]

\begin{small}
\begin{tabular}{|r|p{350pt}|} \hline
文番号&	重要な文の順に表示\\ \hline
	10& 	ライオンの赤ちゃんは、生まれて二ヶ月くらいは、お乳だけ飲んでいますが、やがてお母さんのとった獲物を食べ始めます。\\
	19& 	シマウマの赤ちゃんが、お母さんのお乳だけ飲んでいるのはたったの七日ぐらいの間です。\\
	2& 	動物の赤ちゃんは生まれたばかりのときは、どんな様子をしているのでしょう。 \\
	16& 	シマウマの赤ちゃんは、生まれて、三十分も経たないうちに、自分で立ち上がります。\\
	7&	けれども、赤ちゃんは、弱々しくて、お母さんにあまり似ていません。\\ \hline
\end{tabular}
\end{small}

\caption{タイトルを考慮に入れた{\it tfidf}法での重要文抽出結果}\label{tfidf+a}

\end{figure}


\section{今後の課題}

文書中で得られた二つの単語の連想関係は「上位概念」「下位概念」「部分・材料概念」「環境概念」が多く,「類義語」「属性概念」「動作概念」「関連語」は少なかった.「属性概念」「動作概念」は文,パラグラフをまたがって様々な刺激語(名詞)と関連づけてよいとは限らない.文や文書の構造も考慮した上で重要度の計算をする必要があると考えられる.たとえば「ガラパゴスは火山で出来た島である.大きなゾウガメがその島には住んでいる.」という文では,「大きい」という属性概念が「ゾウガメ」だけでなく「火山」でも連想されている.また,連想関係使った単語の重要度の計算方法では,8つ連想関係で同じ計算式を使うのではなく,各々でその特徴にあった計算式を考える必要がある.

連想概念辞書は刺激語−連想語の一対の対応関係ではなく連想語(名詞のみ)も刺激語として実験している場合もあるので,刺激語・連想語をノードとした大きな意味ネットワークの構造になっている\cite{Okamoto2001}.本手法のように刺激語−連想語の関係だけでなく,ネットワークの経路を辿って連想関係と概念間の距離を導き出す方法も考慮に入れる必要性があると考えている.

一般に単語や重要文だけを並べるのでは理解しにくい.たとえば,重要文に照応詞が含まれていた場合,それが何を指しているのかわからない場合がある.その時は照応詞が指している内容に置き換える作業が必要になってくる.また,文単位の要約ではなく,単語や句レベルで要約する場合は,文生成の技術に関するさまざまな課題が残っている.

\vspace*{1.8cm}

\section{おわりに}

本研究では,連想概念辞書の連想関係や距離情報を用い,文書中の単語の重要度を計算することによって,重要文の抽出を行なった.小学校の教科書のテキストに対して既存の重要語抽出法と本手法での抽出結果とを,人間が行なった抽出結果と比較することによって評価した.単語の出現頻度のみよりも連想関係を計算に含めることによって改良されることがわかった.



\acknowledgment


本研究を進めるにあたって,連想実験の被験者の皆様に感謝いたします.適切な
支援と実験を手伝ってくださった慶應義塾大学石崎研究室の皆様に,また実験デー
タの修正を手伝ってくださった研究室の概念辞書班のメンバーに感謝いたします.



\bibliographystyle{jnlpbbl}
\bibliography{jpaper}

\begin{biography}
\biotitle{略歴}
\bioauthor{岡本 潤}{
1997年慶應義塾大学環境情報学部卒.1999年慶應義塾大学大学院政策メディ
ア研究科修士課程修了.同研究科博士課程在学中.
}
\bioauthor{石崎 俊}{1970年東京大学工学部計数工学科卒,同助手を経て1972年
通産省工業技術院電子技術総合研究所勤務,1985年推論シ
ステム研究室室長,自然言語研究室長を経て1992年から慶
應義塾大学環境情報学部教授,1994年から政策メディア研
究科教授兼任.自然言語処理,音声情報処理,認知科学な
どに興味を持つ.}

\bioreceived{受付}
\bioaccepted{採録}

\end{biography}

\end{document}
