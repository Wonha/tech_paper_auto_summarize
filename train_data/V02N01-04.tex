\documentstyle[epsf,jtheapa]{jnlp_j_b5_old}
\setcounter{page}{57}
\setcounter{巻数}{2}
\setcounter{号数}{1}
\受付{1994}{7}{5}
\再受付{1994}{8}{26}
\採録{1994}{11}{7}
\setcounter{secnumdepth}{2}


\title{動詞の長さの分布に基づいた文章の分類と和語\\及び合成語の比率}
\author{金 明哲\affiref{AA}}

\headauthor{金 明哲}
\headtitle{動詞の長さの分布に基づいた文章の分類と和語及び合成語の比率}

\affilabel{AA}{国立国語研究所 特別研究員}
{National Language Institute, The JSPS Postdoctoral Fellowship}

\jabstract{
文章(テキスト)の執筆者の推定問題などに対して,文章の内容や成立に関す
る歴史的事実の考証とは別に,文章から著者の文体の計量的な特徴を抽出し,
その統計分析によって問題解決を試みる研究が多くの人々の注目をあつめつ
つある.文章に関するどのような要素に著者の特徴が現れるかについて,欧米
文に関してはいくつかの研究の例があるが,それは言語によって異なるとも
考えられるため,欧米文に関する研究成果が日本文の場合にもあてはまるか
について実証的な研究が必要である.また,各言語はその言語における著者の
特徴を表す独特な要素があることも考えられる.本論文では,今まで明らかに
されていない,日本文における動詞の長さの分布に著者の特徴が現れること
と,その結果が動詞中の漢語・和語,合成語・非合成語の使用率の影響では
ないことを著者3人の計21の文章を用いて明らかにした.計量分析の手法とし
ては,同一著者の文章における動詞の長さの分布間の距離の平均値と,異な
る著者の文章における動詞の長さの分布の距離の平均値との差,および距離
マトリックスを用いて主成分分析を行うという方法を用いて数量・視覚的文
章の分類を試みた.
}

\jkeywords{文章の分類, 動詞の長さの分布, 和語・漢語, 合成語・非合成語, 相関係数, 主成分分析}
\etitle{Classification of Sentences \\by Word-length Distribution of
Verbs\\ with Proportions of Japanese Words\\ and those of Compound Words}
\eauthor{Jin Ming-Zhe}

\eabstract{
For the identification of a unknown author of a certain literature,
it is significant to find the main elements of writing styles which 
characterize authors or distinguish them.  It was previously  known that the 
authors' writing styles are characterized in  distributions fo lengths of words  are categorized by parts of speech. This paper presents a quantitative analysis of the classification of sentences by word-length distribution of verbs, and the   
relationships between writing styles in word-length distribution  of verbs and the proportions of Japanese words and Chinese words  or those of compound words and non-compound words in verbs. This analysis shows that word-length distribution of verbs characterizes writing styles even where there are no differences in the proportions of Japanese words and Chinese words or those of compound words and non-compound words.
}

\ekeywords{classification of sentences, verb-length distribution, Japanese words and Chinese words, compound words and non-compound words, correlation coefficient, principal component analysis}

\begin{document}
\maketitle




\section{まえがき}
文章(文献)の執筆者の推定問題(authorship problem), あるいは執筆順序の推定や執筆時期の推定などの問題(Chronology)に対して, 文章の内容や成立に関する歴史的事実の考証とは別に, 文章から著者の文体の計量的な特徴を抽出し, その統計分析によって問題の解決を試みる研究が多くの人々に注目をあつめつつある. 

統計分析の手法を用いた文章の著者の推定や執筆の時期の推定などの研究は今世紀の初頭から行なわれていたが, 本格的な研究が現れたのは今世紀の中ごろである. 研究の全体像を把握するため今世紀の主な研究を表\ref{rri}に示した. 


\begin{table}[htb]
\caption{{\dg 著者の推定などの研究のリスト}\label{rri}}
\begin{center}
\renewcommand{\arraystretch}{}
\footnotesize{
\begin{tabular}{llll} \hline
 分析の対象となった文章      &用いた情報    &用いた情報, 手法   &研究者  \\ \hline
Shakespeare, Bacon    &単語の長さ    &モード               & Mendenhall, T.C.(1887) \\ 
The Imitation of Christ &文の長さ      &平均値, 中央値など   & Yule, G.U.(1939)       \\
The Imitation of Christ &語彙量        &K特性値     & Yule, G.U.(1944)       \\
Shakespeare et al.         &単語の音節数  &Shannon エントロピー      &Fucks, W.(1952)       \\
Shakespeare et al.         &音節数の接続関係& 分散共分散の固有値, &                  \\
                        &              &Shannon エントロピー& Fucks, W.(1954)        \\
Shakespeare et al.      &単語の長さの分布& 平均値など       & Williams, C.B.(1956)    \\
プラトンの第七書簡      &文の長さ      &平均値, 中央値など   & Wake, W.C(1957)      \\
Work of Plato           &文末の単語のタイプ&判別分析         &Cox, D.R. et al. (1958) \\
Quintus Curtius Snodgrass&              &                    &                       \\ 
letter                  &語の長さの分布 &$\chi ^2$の検定, $t$検定& Brinegar, C.S.(1963) \\
新約聖書の中のパウロの書簡 
                        &語の使用頻度  & $\chi ^2$検定          &Morton, A.Q.(1965)      \\
源氏物語の宇治十帖      &頁数, 和歌数など&U検定, $\chi ^2$検定  &安本 美典(1960)        \\
Federalist paper         &単語の使用頻度  &線形判別分析, 確率比  &Mosteller, F. et al. (1963) \\
由良物語                &単語の使用頻度  &線形判別分析, 確率比  &韮沢 正(1965)     \\
Shakespeare and Bacon   &単語の長さの分布  &分布の比較          &Williams, C.B.(1975) \\
Shakespeare             &語彙量          &ポアソン分布          &Thisted, R. et al. (1976)\\
源氏物語                &頁数, 和歌数等  &因子分析              &安本 美典(1977)     \\ 
Shakespeare             &単語の出現頻度  &ポアソン分布, 検定    &Thisted, R. et al. (1987) \\
紅楼夢                  &虚詞の使用頻度  &主成分,               &  \\
                        &                &クラスターリング など &Li, X.P.(1987, 1989) \\ 
日蓮遺文                &品詞の使用率など&$t$検定, 主成分,      &                     \\
                        &                & クラスターリング           &村上 征勝 他(1992, 1994) \\ 
\hline
\end{tabular}
}
\end{center}
\end{table}

文章の著者の推定や文章の分類などを行なう際, 文章に関するどのような著者の特徴を表す情報(特徴情報)を用いるかが問題解決の鍵である.
今までの文章の著者の推定や文体の研究では著者の特徴を表す情報としては, 単語の長さ, 単語の使用頻度, 文の長さなどがよく用いられている. 

日本文に関して, 少し詳細に見ると, 安本は直喩, 声喩, 色彩語, 文の長さ, 会話文, 句読点, 人格語などの項目を用いて100人の作家の100編の文章を体言型—用言型, 修飾型—非修飾型, 会話型—文章型に分類することを試み(安本1981, 1994), 
また長編度(頁数), 和歌の使用度, 直喩の使用度, 声喩の使用度, 心理描写の数, 文の長さ, 色彩語の使用度, 名詞の使用度, 用言の使用度, 助詞の使用度, 助動詞の使用度, 品詞数の12項目の情報を用いて源氏物語の宇治十帖の著者の推定を試みた(安本1958). 韮沢は, 「にて」, 「へ」, 「して」, 「ど」, 「ばかり」, 「しも」, 「のみ」, 「ころ」, 「なむ」, 「じ」, 「ざる」, 「つ」, 「む」, 「あるは」, 「されど」, 「しかれども」, 「いと」, 「いかに」などの単語の使用率を用いて, 「由良物語」の著者の判定(韮沢1965, 1973)を行い, 
村上らは品詞の接続関係, 接尾語などを用いて日蓮遺文の真偽について計量分析を行なっている(村上1985, 1988, 1994). 
このように, 日本文に関して, 文章の著者の推定を試みる研究はいくつかあるが, 著者の推定などのための文章に関するどのような情報が有効となるかに関する基礎的な研究はほとんどない状況である. 

文章に関するどのような要素に著者の特徴が現れるかに関して, 外国での研究ではいくつかあるが, それは言語によって異なると考えられるため, 外国語での研究成果が日本語の場合もあてはまるのか, もしあてはまらないとすれば日本語の文章ではどのような要素に著者の特徴が現れるかというようなことが文体研究の重要な課題である.

筆者は日本語の文章の著者の推定あるいは著者別に文章を分類する基礎的
な研究として, 文章の中のどのような要素が著者の文体の特徴になるかについて研究を進めている. 
コンピュータのハードウエアとソフトウエアの発展に伴い, コンピュータを利用することによって文章の中から膨大な情報が抽出できるようになった. しかし, 今度はそのような膨大な情報の中からどの情報を用いるべきかという新しい問題が生じた.
筆者らは文章の中に使用された読点について計量分析を行ない, 読点の前の文字に関する情報で
文章を著者別に分類する方法を提案し, この方法は文学作品だけではなく研究論文に
ついても有効であることを実証した(金1993a, b, 1994c, d, e). このような日本文に適応した著者の文体の特徴情報の抽出に関する研究は始まったばかりで決して十分とはいえない.

ところで, コンピュータで著者の文体の特徴を抽出するためには計算機処理可能な文章のデータベースが必要であるが, そのようなデータベースが入手できなかったため, 作成することにした. データベース化したのは井上靖, 三島由紀夫, 中島敦の短篇小説である. 分析に用いた情報の安定性の考察及び用いた短い文章とのバランスをとるため, 比較的長い文章はいくつかに分割して用いた. 例えば, 井上の「恋と死と波と」は二つに, 中島の「弟子」は三つに, 「李陵」は四つに分割して用いることにした. 表\ref{list}に, 用いた文章と発表年などを示した. 

\begin{table}[htb]
\caption{{\dg 分析に用いた文章のリスト}\label{list}}
\begin{center}
\small{
\begin{tabular}{l l c c c c c }\hline
著者         & 文章名           & 記号&単語数 & 出版社 &発表の年\\ \hline 
井上靖       & 結婚記念日       & I1  &  4749 &角川文庫& 1951 \\
             & 石庭             & I2  &  4796 & 同上   & 1950 \\
             &死と恋と波と(前半)& I3  &  4683 & 同上   & 1950 \\
             &死と恋と波と(後半)& I4  &  4386 & 同上   & 同上\\
             & 帽子             & I5  &  3724 &新潮文庫& 1973 \\
             & 魔法壜           & I6  &  3624 & 同上   & 同上 \\
             & 滝へ降りる道     & I7  &  3727 & 同上   & 1952 \\
             & 晩夏             & I8  &  4269 & 同上   & 同上 \\ 
三島由紀夫   & 遠乗会           & M1  &  4984 &新潮文庫& 1951 \\
             & 卵               & M2  &  4004 & 同上   & 1955 \\
             & 詩を書く少年     & M3  &  4502 & 同上   & 1955 \\
             & 海と夕焼         & M4  &  3359 & 同上   & 1955 \\
中島敦       &山月記            & L1  &  3226 &新潮文庫& 1942 \\
             &名人伝            & L2  &  3202 & 同上   & 1942 \\
             &弟子(前の1/3)     & L3  &  4078 & 同上   & 1943 \\
             &弟子(中の1/3)     & L4  &  4092 & 同上   & 同上 \\
             &弟子(後の1/3)     & L5  &  3727 & 同上   & 同上 \\
             &李陵(前の1/4)     & L6  &  4563 & 同上   & 1944 \\
             &李陵(中の1/4)     & L7  &  4561 & 同上   & 同上\\
             &李陵(中の1/4)     & L8  &  4638 & 同上   & 同上  \\
             &李陵(後の1/4)     & L9  &  4458 & 同上  & 同上\\  \hline
\end{tabular}
}
\end{center}
\end{table}

この3人を選んだのは, OCR(光学読み取り装置)で文章を入力する場合に漢字の認識率が問題になるため, 現代文の中で漢字の使用率がわりに高い中島の文章を用いてOCRでの入力テストを行なったのがきっかけであった. 中島と同時期の作家として井上, 三島を選んだ. 

データベースは分析に用いる文章をOCRで入力し, 読み取りの誤りを訂正し, 品詞コードなどを入力して作成した. 表\ref{datas}に作成したデータベースの一部分を示した. 単語の認定は「広辞苑」に従った. ただし, 広辞苑にない複合動詞については複合された全体を1語とした.

\begin{table}[htb]
\caption{\dg データベースの例}\label{datas}
\begin{center}\footnotesize
\renewcommand{\arraystretch}{}
\begin{tabular}{l} \hline
\\
(2)/父(M)は(J)(27)/軍医(M)で(Z), (4)/当時(M)(5)/聯隊(M)の(J)\\
(6)/ある(R)(6)/地方(M)の(J)(9)/小都市(M)を(J)(9)/転々と(F)\\
(10)/し(D)て(J)(27)/おり(D), (11)/子供(M)を(J)(13)/自分(M)の(J)\\
(14)/手許(M)に(J)(27)/置く(D)と(J), (16)/何回(M)も(J)\\
(17)/転校させ(D)なけれ(Z)ば(J)(23)/なら(D)なかっ(Z)た(Z)ので(J), \\
(19)/そう(F)(20)/し(D)た(Z)(23)/こと(M)から(J)(23)/私(M)を(J)\\
(23)/郷里(M)に(J)(24)/置く(D)(25)/気(M)に(J)(26)/なっ(D)た(Z)\\
(27)/もの(M)らしかっ(Z)た(Z). \\
\\
(3)/たとえ(F)(3)/田舎(M)の(J)(11)/小学校(M)でも(J), (7)/まだ(F)\\
(6)/同じ(R)(7)/小学校(M)に(J)(14)/落着い(D)て(J)(9)/通わ(D)せ(Z)\\
た(Z)(11)/方(M)が(J)(11)/教育上(M)(11)/いい(K)と(J)(13)/考え(D)\\
た(Z)の(J)で(Z)(14)/ある(D). \\

\\ \hline
\end{tabular}
\end{center}
\hspace*{0.8cm}{\footnotesize 記号/は文節の境界線で, (数字)は(数字)の直後の文節が係る文節の番号で,(ローマ字)は品詞コードである. }
\end{table}

\section{単語の長さの分布に基づいた文章の分類}

文章の中の著者の特徴情報を用いて, 文章の著者を推定するためには, 用いた情報に基づいて文章を分類する際, 文章が著者別に分類できることが望まれる. 

どのような単語を好んで用いるのか, どのような長さの単語を好んで用いるのかは著者の文章の一つの特徴であると考える. 前者を用いる場合は, 著者の好みの単語が何であるかを見つけ出すのはかなり厄介なことである. 前者にくらべ後者はわりに簡単である. 

欧米では, 著者の推定などの研究には単語の長さの情報がよく用いられている. しかし, 日本文の分析においては, 単語の長さに関する情報はあまり用いておらず, また, 単語の長さと著者の関係に関する基礎的な研究はない. その主な原因としては, 
\begin{enumerate}
\item 日本文は「分ち書き」されていないため単語の認定が難しいこと, 
\item 日本語の機械処理技術が遅れたこと
\end{enumerate}
などがあげられる(村上1989a, b). これらの問題は計算機科学の発展により次第に解決されてきている.  
今日では, コンピュータは日本語を自由に処理できるようになってきているし, 近い将来かなり精度の良い日本文の単語分割システムも開発されると予測される(中野1991).  

筆者は実際に書かれた文字を単位とした単語の長さと著者との関係を明らかにするため, 単語の長さの分布に著者の特徴が現れるかどうかについて計量分析を行なってきた. 
分析はすべての単語を用いた場合と単語を名詞, 動詞, 形容詞, 形容動詞, 助詞, 助動詞, 副詞に分けた場合の品詞別の単語の長さの分布について行なった. その結果, 日本語の単語の長さの分布を用いて著者別に文章を分類する際, 単語を品詞別に分けていない場合には, 著者の特徴が出にくい単語や文章の内容に依存性が高い単語などが含まれてしまうため, 分類はうまくいかないが, 単語を品詞別に分けることによって分類の精度をあげることが可能であり, 特に動詞の単語の長さの分布に著者の特徴が出やすいと言うことが分かった(金1994a, b). 

本文では, 動詞の長さの分布に基づいた文章の分類及び動詞の単語の長さの分布に現れる著者の特徴と動詞の中の和語・漢語(漢語と和語が合成された動詞を指す. 例えば, 勉強する. 叙述の便利のため漢語と呼ぶことにする.)、合成語・非合成語の比率との関係についての計量分析について述べる. まず, 動詞の長さの分布を用いた文章の分類について述べておく.

表\ref{wlf}に, 表\ref{list}の21編の文章における実際に書かれた文字を単位とした1文字から5文字までの長さ別の動詞の使用頻度を示した. 


\begin{table}[htb]
\caption{\dg 長さ別の動詞の使用頻度}\label{wlf}
\begin{center}
\small{
\begin{tabular}{c c r  r r r r }\hline
著者名  & 文章の記号& 1 & 2 & 3 & 4 & 5 \\ \hline
井上    & I1  &   114&  515&   88&   77&   29  \\
        & I2  &   138&  518&   94&   60&   12 \\ 
        & I3  &   117&  492&  119&   73&   17\\ 
        & I4  &   119&  476&   95&   54&   17 \\ 
        & I5  &   101&  398&   89&   52&    7\\ 
        & I6  &   114&  438&   77&   45&    7\\ 
        & I7  &    98&  408&   77&   70&   12\\ 
        & I8  &   110&  429&   87&   85&   17\\ 
三島    & M1  &    74&  509&  171&   77&   24  \\
        & M2  &   111&  353&  120&   87&   21\\
        & M3  &   100&  419&  136&   63&   12\\
        & M4  &    68&  379&   92&   58&    8\\
中島    & N1  &    57&  357&   73&   36&    8  \\
        & N2  &    60&  340&  104&   43&    7\\
        & N3  &    87&  474&   77&   37&    5\\
        & N4  &    73&  480&   86&   42&   11 \\
        & N5  &    69&  448&   80&   35&    3\\
        & N6  &    74&  503&  124&   51&   17\\
        & N7  &   101&  491&  112&   39&   16 \\
        & N8  &    90&  490&  116&   55&   13\\
        & N9  &    98&  499&   98&   39&    7\\ \hline
\end{tabular}
}
\end{center}
\end{table}

いま, \hspace{-0.15mm}文章$i$の長さ\hspace{-0.1mm}$j$の単語の使用頻度を$x_{ij}$と表すと, \hspace{-0.15mm}$I$編の文章における長さ\hspace{-0.1mm}$J$までの単語\\の使用頻度と使用率のマトリックスはそれぞれ

\[X_{I\times J}=\left[x_{ij} \right] \]   
\[ P_{I\times J}=\left[p_{ij} \right ] \]
\[ p_{ij}=\frac{x_{ij}}{\sum_{v=1}^{J}x_{iv}},\hspace{0.5cm}
     \sum_{j=1}^{J}p_{ij}=1 \]
で表示できる. 

本文では, $P_{I\times J}$の一行を一つの分布とみなし, 分布の間の距離を用いて分析を行なった. 叙述の便利のため, 分布の間の距離を文章の間の距離と呼ぶことにする. 

さて, 文章$i$の単語の長さの分布と文章$l$の単語の長さの分布の間の距離$d_{il}$を次のように定義する. 

\[ d_{il}=\frac{1}{2}\sum_{j=1}^{J}{(p_{ij}log\frac{2p_{ij}}{p_{ij}+p_{lj}}+p_{lj}log\frac{2p_{lj}}{p_{ij}+p_{lj}})} \]
\vspace*{-0.2mm}ただし\vspace*{-0.2mm}\begin{center} 
$p_{ij}=0$ \hspace{0.5cm} なら \hspace{0.5cm} $p_{ij}log{\frac{2p_{ij}}{p_{ij}+p_{lj}}}=0$\\
$p_{lj}=0$ \hspace{0.5cm} なら \hspace{0.5cm} $p_{lj}log{\frac{2p_{lj}}{p_{ij}+p_{lj}}}=0 $\\ 
\end{center}
\vspace*{-0.2mm}
とする. 

また上式で求められた分布の間の距離のマトリクスを
\[ D_{I\times I}=\left[ 
\begin{array}{cc}
0  &d_{ji} \\
d_{ij}& 0  \\
\end{array} \right] \]
で表記する. 

著者別に文章を分類する観点からは, 同一の著者(今後は群内と呼ぶ)の任意の二つの分布の間の距離が, 異なる著者間(今後は群間と呼ぶ)の任意の二つの分布の間の距離より小さいことが望まれる. しかし, 本研究ではこのような望ましい結果は得られなかった. 分類を行なう際, 文章が著者別に分類されたとしても, グループの間の距離が十分大きくない場合は, 同一の著者の文章の間の距離が, 異なる著者の文章の間の距離より大きい可能性も十分あり得る. そこで,群内の距離の平均値と群間の距離の平均値を用いて分析を進めることにする. 群内の距離の平均値が群間の距離の平均値より小さいと, 用いた分布には著者の特徴があり, 群内の距離が群間の距離より小さければ小さいほど著者の特徴が明確である(分類がよい)と考えられる.

著者\hspace{-0.1mm}$k$の\hspace{-0.1mm}$k_{n}$編の文章における任意の二つの分布の間の距離の平均(今後は群内の距離と呼ぶ)\\を
\[ \overline{d(k)}=\frac{2\sum_{k_{i}=k_1}^{k_{n}-1} \sum_{k_{j}=k_{i}+1}^{k_{n}}
d_{k_{i}k_{j}}}{(k_{n}-1)k_{n}} \times 100 \]
著者\hspace{-0.1mm}$k$と著者\hspace{-0.1mm}$h$の, それぞれの$k_{n}, h_{m}$編の文章における任意の二つの分布の間の距離の平均(今\\後は群間の距離と呼ぶ)を

\[\overline{d(k,h)}=\frac{\sum_{k_{i}=k_1}^{k_{n}} \sum_{h_{j}=h_1}^{h_{m}}
d_{k_{i}h_{j}}}{k_{n}h_{m}}\times 100 \]
で求めた.

表\ref{dld}に3人の21編の文章について, 動詞の長さの分布を用いた場合の群内, 群間の距離を示した. 
\begin{table}[htbp]
\caption{\dg 動詞における距離}\label{dld}
\begin{center}
\small{
\begin{tabular}{ c  c c c c c } \hline
       &   群内   &   群     &          &   間     &最小の群間\\      
著者名 &          & 井上     & 三島     & 中島     &          \\ \hline
井上   &   0.2317 &          &  0.7101  &  0.5612  &0.5612    \\ 
三島   &   0.5342 &  0.7101  &          &  0.8513  &0.7101  \\
中島   &   0.2630 &  0.5612  &  0.8513  &          &0.5612 \\ \hline
\end{tabular}
}
\end{center}
\end{table}
3人の群内の距離はいずれの群間の距離より小さいため, 動詞の長さの分布には著者の特徴があると判断する. 

井上, 中島の群内の距離が三島の群内の距離に比べてかなり小さいが, これは長い文章をいくつかに分割して用いた影響ではないかと考えられる. このことを明かにするため, 長い文章を分割しなかった場合の各著者の群内の距離の平均値を求めてみた. 井上, 中島の群内の距離の平均値はそれぞれ0.2394, 0.2389で, 長い文章を分割して用いた場合と比べて大きな差が見られない. したがって, 井上, 中島の群内の距離が三島の群内の距離よりかなり小さいのは長い文章を分割して用いた影響ではないと判断する. 三島の群内の距離が大きい原因としては\begin{enumerate}
\item 三島の文章における動詞の使用法の分散が大きい.
\item 分析に用いた単語の長さの分布が十分安定していない.
\end{enumerate}
などが考えられる. 

長い文章をいくつかに分割した場合, 必ずしも同一の小説の文章の間の距離が, 同一の著者の異なる小説の間の距離より小さいという結果は得られなかった. この結果については, 以下のような原因が考えられる. 
\begin{enumerate}
\item 動詞の単語の長さの分布は文章の内容に関して, 依存性が低い.
\item 文章における動詞の単語の長さの分布が十分安定していない.

\end{enumerate}
残念ながらこの原因については用いたデータベースでは実証することが困難であるため, 今後の研究課題にせざる得ない.

動詞の単語の長さの分布を用いた場合の分類の結果を視覚化してみる. 分類を視覚化するクラスタ分析の方法は, いくつかの方法があるが, ここでは主成分分析を用いた. 分類を行なう際の主成分分析としては, 二つの方法が考えられる. 一つは, 分類に用いるデータ
\[X_{I \times J}=\left [ x_{ij} \right ] \]
\[ P_{I\times J}=\left[ \begin{array}{c} p_{ij} \end{array} \right] \]
\[p_{ij}=\frac{x_{ij}}{\sum_{j=1}^J{x_{ij}}},\hspace{0.5cm}
\sum_{j=1}^J{p_{ij}}=1\]
を用いて主成分分析を行なう方法で, もう一つは, 上記のデータから個体間の距離(あるいは類似度)を求め, 距離(あるいは類似度)のデータを用いて主成分分析を行なう方法である. 本研究では後者を用いた. 
主成分分析は, 求められた距離のマトリックス
\[ D_{I\times I}=\left[ \begin{array}{cc}
0  & d_{ji} \\
d_{ij}& 0 \\
\end{array} \right] \]
を以下のように標準化し, 
\[ \widehat{d}_{ij}=\frac{d_{ij}}{\sum_{v=1}^I d_{iv}},\]
 \vspace*{-0.2mm}
\[ \widehat{D}_{I \times I}=\left[ \widehat{d}_{ij} \right], \]
$\widehat{D}_{I \times I}$の分散共分散の行列を用いて行なった. 

動詞の単語の長さの分布を用いて求められた$\widehat{D}_{I \times I}$の分散共分散行列の主成分分析では, 第1\\主成分, 第2主成分, 第3主成分の寄与率はそれぞれ57.04\%, 34.04\%, 3.94\%で, 第2主成分までの累積寄与率は91.08\%であった. 図\ref{dlcp}に第2主成分までの主成分得点のプロットをに示した. 横軸は第1主成分で, 縦軸は第2主成分である. 主成分分析の結果から動詞の単語の長さの分布を用いた場合, 著者別に文章が分類できるといえよう. 見やすくするため, 文章を著者別に滑らかの曲線(境界線)で囲んだ. もちろん, このような境界線はどの文章がどの著者のものであるかの情報を用いて引いている. このような方法を用いて著者不明の文章の著者の判別などを行なう場合は, まずトレーニングーデータを用いて境界線を引き, 著者不明の文章がどのグループに属するかの判別(判定)を行なうべきである.
{\unitlength=1mm
\begin{figure}
\begin{center}
    
    
    
  \epsfile{file=fig1.eps,width=118mm}
\end{center}
\caption{動詞の長さの分布に基づいた$\widehat{D}_{21 \times 21}$の主成分得点のプロット}\label{dlcp}
\end{figure}
}

\section{動詞における和語・漢語, 合成語・非合成語の比率}
前節では, 動詞の単語の長さの分布を用いた場合の文章の分類について述べた. 動詞の長さの分布に著者の特徴が現れる現象の原因の一つとしては和語・漢語, 合成語・非合成語の比率が考えられる.  
もし, 動詞の長さに著者の特徴が現れるという結果が, 和語・漢語, 合成語・非合成語の比率の影響によるものであるとすると, 和語・漢語, 合成語・非合成語の比率に差がない著者に対しては動詞の長さの分布は著者の特徴情報になれない.
したがって, 動詞の単語の長さの分布に著者の特徴が現れるという現象と和語・漢語, 合成語・非合成語の比率との関係について明かにする必要がある. 

\subsection{和語・漢語の比率}
本節では動詞は, 和語と漢語(和語と漢語より合成された動詞)より構成されたと見なし, 動詞の単語の長さの分布に現れる著者の特徴と和語・漢語の比率との関係について計量分析を行なう.
 和語の比率は漢語の比率より高いため, 和語の比率について分析を行なうことにする. 表\ref{wagop}に3人の作家の21編の文章における動詞の中の和語の比率(動詞の中の和語の数/動詞の総単語数)を示した. 長さ1, 4, 5の動詞では, 和語の比率はほぼ同じで, 1である. 長さ2, 3の動詞には漢語が多少含まれている. 和語の比率に著者の特徴が現れるかどうかを調べてみる. 好都合なことに動詞の中の和語の比率に著者の特徴がありそうなのは長さ2, 3の動詞だけであるため,  動詞の長さ2, 3における和語の比率を図\ref{wagopf}にプロットした. 横軸は長さ2の和語の比率で, 縦軸は長さ3の和語の比率である. 図\ref{wagopf}でわかるように中島は単独に一つのクループを作っているが, 井上と三島は著者別にクループを作っていない. つまり, 動詞の中の和語の比率には著者の特著が現れているが, 文章を著者別に分類できるほどではない. 
\begin{table}[htb]
\caption{{\dg 動詞の中の和語の比率}\label{wagop}}
\begin{center}
\small{
\begin{tabular}{c    c c c c c} \hline
   & 1&    2     &     3    &     4    &    5   \\ \hline
I1 & 1&  0.9825& 0.9667& 1.0000&  0.9600 \\
I2 & 1&  0.9750& 1.0000& 1.0000&  1.0000\\
I3 & 1&  0.9939& 0.9919& 1.0000&  1.0000\\
I4 & 1&  0.9937& 0.9677& 1.0000&  1.0000 \\
I5 & 1&  0.9824& 0.9778& 1.0000&  1.0000 \\
I6 & 1&  0.9932& 0.9740& 1.0000&  1.0000\\
I7 & 1&  0.9902& 0.9750& 0.9851&  1.0000 \\
I8 & 1&  0.9860& 0.9886& 1.0000&  1.0000 \\
M1 & 1&  0.9666& 0.9765& 1.0000&  1.0000 \\
M2 & 1&  0.9914& 0.9748& 1.0000&  1.0000 \\
M3 & 1&  0.9691& 0.9209& 1.0000&  1.0000 \\
M4 & 1&  0.9841& 0.9892& 1.0000&  1.0000 \\
N1 & 1&  0.9469& 0.8472& 1.0000&  1.0000 \\
N2 & 1&  0.9676& 0.9429& 1.0000&  1.0000 \\
N3 & 1&  0.9346& 0.8816& 0.9730&  1.0000 \\
N4 & 1&  0.9439& 0.8721& 1.0000&  1.0000 \\
N5 & 1&  0.9509& 0.9620& 1.0000&  1.0000 \\
N6 & 1&  0.9382& 0.9302& 1.0000&  1.0000 \\
N7 & 1&  0.9532& 0.8654& 1.0000&  1.0000 \\
N8 & 1&  0.9470& 0.9130& 1.0000&  1.0000 \\
N9 & 1&  0.9579& 0.9200& 1.0000&  1.0000 \\ \hline
\end{tabular}
}
\end{center}
\end{table}

{\unitlength=1mm
\begin{figure}
\begin{center}
    
    
    
\epsfile{file=fig2.eps,width=120mm}
\end{center}
\caption{長さ2, 3の動詞の中の和語の比率のプロット}\label{wagopf}
\end{figure}


\begin{figure}
\begin{center}
    
    
    
\epsfile{file=fig3.eps,width=120mm}
\end{center}
\caption{和語の動詞の長さの分布に基づいた$\widehat{D}_{21 \times 21}$の主成分得点のプロット}\label{wagopf2}
\vspace*{14cm}
\end{figure}
}


漢語を含んだ場合の長さの分布と漢語を除いた場合の長さの分布との関係を調べるため, 漢語を含んだ場合($AV$で表記する)と漢語を除いた場合($JV$で表記する)との動詞の使用頻度について相関係数を求めてみた. 相関係数が最も小さいのは0.9843で, 
$COR(AV, JV)$の対角要素の平均は0.9940である. $COR(AV, JV)$から漢語を含んだ場合と漢語を除いた場合の動詞の長さの分布は強い相関を持っていることが分かる. 

\[ COR(AV, JV)=\left[ \begin{array}{ccccc}
1    &       &       &       &  \Large{r_{ij}}\\
     &0.9864 &       &       &        \\
     &       &0.9843 &       &        \\
     &       &       &0.9998 &         \\
\Large{r_{ji}}     &      &       &       &0.9995  \\
\end{array} \right] \]

漢語を含んだ動詞の長さの分布を用いた場合の分類と漢語を除いた動詞の長さ
の分布を用いた場合の分類の結果を比較してみる. 

第2章と同様な方法で, 漢語を除いた動詞の長さの分布の間の距離を用いて求めた$\widehat{D}_{21\times 21}$に\\ついて主成分分析を行なった. 第1, 2主成分得点の寄与率はそ
れぞれ59.24\%,  30.96\%で, 第2主成分までの累積寄与率は90.20\%である. 
図\ref{wagopf2}に第2主成分までの主成分得点のプロットを示した. 主成分得
点のプロットから漢語を含んだ動詞の長さの分布を用いた場合と漢語を除いた
動詞の長さの分布を用いた場合とには大きな差がないことがわかる(図
\ref{dlcp}を参照). 以上の分析結果から, 動詞の長さの分布に現れる著者の
特徴は漢語・和語の使用率の影響ではないことが分かった.  

\subsection{合成語・非合成語の比率}
本節では動詞は合成語と非合成語により構成されていると見なし, 合成語・非合成語の比率が動詞の長さの分布に現れる著者の特徴に与える影響について分析を行なう. 非合成語の比率は合成語の比率より高いため, 表\ref{goseigop}に, 動詞の中の非合成語の比率(動詞の中の非合成語の数/動詞の単語総数)を示した. 表\ref{goseigop}から分かるように, 長さ1, 2の動詞には合成語がない. また長さ5における非合成語の比率は安定していないことが目立つ. これは各文章における長さ5の動詞が少なかったためであると考える(情報が不安定する). 表\ref{goseigop}のデータを見る限り, 著者の特徴が現れそうなのは長さ3と長さ4の非合成語の比率である. 非合成語の比率に著者の特徴が現れるかどうかを知るため, 長さ3と長さ4の非合成語の比率を図\ref{goseigopf}にプロットした. 横軸は長さ3の非合成語の比率で, 縦軸は長さ4の非合成語の比率である. 図\ref{goseigopf}では21作品が著者別にグループを作っていないことから, 非合成語の比率には著者の特徴が見られないと判断する. 

\begin{table}[htb]
\caption{{\dg 動詞の中の非合成語の比率}\label{goseigop}}
\begin{center}
\small{
\begin{tabular}{cc c c c c} \hline
   & 1 &    2     &     3    &     4    &5    \\ \hline
I1& 1& 1.000& 0.7889& 0.8571& 0.9200\\
I2& 1& 1.000& 0.6264& 0.8167& 0.5455\\
I3& 1& 1.000& 0.8211& 0.7286& 0.8333\\
I4& 1& 1.000& 0.8602& 0.8909& 0.8667\\
I5& 1& 1.000& 0.6444& 0.8235& 0.6000\\
I6& 1& 1.000& 0.7922& 0.9111& 1.0000\\
I7& 1& 1.000& 0.8875& 0.7761& 0.8333\\
I8& 1& 1.000& 0.7273& 0.8765& 1.0000\\
M1& 1& 1.000& 0.7765& 0.8571& 0.9524\\
M2& 1& 1.000& 0.7227& 0.9080& 1.0000\\
M3& 1& 1.000& 0.8201& 0.7759& 0.9167\\
M4& 1& 1.000& 0.8387& 0.9655& 1.0000\\
N1& 1& 1.000& 0.8056& 0.8333& 1.0000\\
N2& 1& 1.000& 0.9143& 0.8810& 1.0000\\
N3& 1& 1.000& 0.8684& 0.6757& 0.6000\\
N4& 1& 1.000& 0.7442& 0.6667& 0.8182\\
N5& 1& 1.000& 0.6709& 0.8000& 1.0000\\
N6& 1& 0.998& 0.7287& 0.7708& 0.6429\\
N7& 1& 1.000& 0.7404& 0.7568& 0.1875\\
N8& 1& 1.000& 0.8348& 0.7255& 0.9231\\
N9& 1& 1.000& 0.7400& 0.7368& 0.8333\\ \hline
\end{tabular}
}
\end{center}
\end{table}

合成語を含んだ動詞の単語の長さの分布($AV$で表記する)と, 合成語を除いた動詞の単語の長さの分布($\overline{CV}$で表記する)との相関係数を求めてみた. 相関係数の中で最も小さいのは長さが5の場合で, 0.8794である. 
 相関係数の対角要素の平均は0.9568で, 高い相関を持っていると言えよう. 

\[ COR(AV,\overline{CV})= \left[ \begin{array}{ccccc}
1          &          &         &          &\Large{r_{ij}}     \\
           &0.9999    &         &          &       \\
           &          & 0.9346  &          &      \\
           &          &         & 0.9699   &        \\
\Large{r_{ji}}    &          &         &          &0.8794  \\
\end{array}     \right ] \]

合成語を含んだ場合と合成語を除いた場合との動詞の単語の長さの分布を用い
た分類の結果を比較してみた. 第2章で用いた方法と同じく, 非合成語におけ
る動詞の長さの分布の間の距離を用いて求めた$\widehat{D}_{21 \times 21}$
について主成分分析を行なった. 第1, 2主成分得点の寄与率はそれぞれ
61.77\%, 30.74\%で, 第2主成分までの累積寄与率は92.51\%である. 図
\ref{goseigopf2}に主成分得点のプロットを示した. 主成分得点のプロットか
ら合成語を含んだ場合と合成語を除いた場合では大きな差がないことがわかる
(図{\ref{dlcp}}を参照).  
長さ5における合成語を含んだ場合と合成語を除いた場合の相関係数が0.8794
であるにも関わらず分類のプロットにはあまり差が見られなかったのは, 長さ
5の動詞の使用頻度は少なく, 分類に対する寄与も小さいためであると考える
(金1994a). 以上の分析結果から, 動詞の長さの分布に現れる著者の特徴は合
成語・非合成語の比率の影響ではないことがわかった.  

{\unitlength=1mm
\begin{figure}
\begin{center}
    
    
    
\epsfile{file=fig4.eps,width=118mm}
\end{center}
\caption{長さ3, 4の動詞の中の非合成語の比率のプロット}\label{goseigopf}
\end{figure}

\begin{figure}
\begin{center}
    
    
    
\epsfile{file=fig5.eps,width=122mm}
\end{center}
\caption{非合成語の動詞の長さの分布に基づいた$\widehat{D}_{21 \times 21}$の主成分得点のプロット}\label{goseigopf2}
\end{figure}
}

\section{むすび}
本研究では, 3人の作家の21編の文章における動詞の長さの分布に基づいた文
章の分類及び動詞の単語の長さの分布に現れる著者の文体の特徴と和語・漢語, 
合成語・非合成語の比率との関係について計量分析を行なった. その結果, 動
詞の長さの分布は,  和語・漢語, 合成語・非合成語の比率に著者毎の差が見
られない場合でも, 著者の文体の特徴になる可能性があることを明かにした.


動詞の長さの分布を用いて分類を行なう場合の有効性の考察を行なうため, 従
来よく用いられている文の長さの分布, 品詞の使用率, 漢字・仮名の使用率, 
筆者が提案した読点の前の文字の分布を用いた場合と比較も行なった. その結
果, 最も分類がよいのは読点の前の文字の分布を用いた場合であり, その次が
動詞の長さの分布で, 従来よく用いられている文の長さの分布, 品詞の使用率, 
漢字・仮名の使用率より著者の特徴が明確に現れることが分かった(金1994a,
b, c, d, e).  

動詞の単語の長さの分布だけを用いて文章を分類する場合, 分類の結果が十分
に満足できるとはいえない. しかし, 著者不明の文章の著者の推定(判別)など
を行なう場合は, 無相関である複数の文体の特徴情報を組み合わせて用いて分
析を行なわなければならないため, このような文体の特徴情報に関する研究は
重要であると考える. 

単語の長さを計る単位を実際に書かれた文字を単位とした場合, 単語の長さは
表記文字の種類の影響を受けることも考えられる. しかし, 本計研究に用いた
動詞には平仮名やローマ字によるものはない. 表記文字の種類が単語の長さに
与える影響について考察を行なうため, 漢字・仮名の使用率を調べてみた. そ
の結果, 漢字・仮名の使用率には著者の特徴が明確に現れなかった(金1994a). 
単語の長さを計る単位として音節, ローマ字を用いることも考えられるが, 今
後の課題にする. 

単語の長さの分布に関する情報を抽出するためには, 品詞情報が付加されてい
るデータベースが必要である. このようなデータベースが入手できなっかたこ
とと, 大量の作品のデータベースを作成する経費がないため本研究では3人の
21編文章だけを用いた分析に留まった. 今後より多くの作家の作品や異なるジャ
ンルについて実証的な研究が必要であろう. また, 文章の量と動詞の長さの分
布の安定性との関係に関しても興味深い課題である.


\acknowledgment

本研究に用いたデータベースは文部省統計数理研究所及び総合研究大学院大学
の村上征勝教授の研究費で作成しました, 本研究をご支援及び御指導下さった
村上 征勝教授, 言語学及び文体学の観点からご指導及びご助言下さった神戸
学院大学の樺島 忠夫教授, 論文の仕上げにご協力くださっ統計数理研究所の
吉野諒三助教授, 有益なコメントを下さった査読者に深く感謝します. 本論文
の最終の修正は学術振興会の特別研究員(The JSPS Postdoctoral Fellowship)
として国立国語研究所にいる期間中に行ないました. 特別研究員として採用し
て下さった学術振興会, ご支援下さった国語研究所の江川 清, 中野 洋両部長, 
米田 正人室長に心より感謝します. 


\bibliographystyle{jtheapa}
\bibliography{jpaper}



\begin{biography}
\biotitle{略歴}
\bioauthor{金 明哲}{
1978年中国吉林師範大学(現東北師範大学)数学系卒業. 
同年中国長春郵電学院(大学)に就職. 
1988年4月来日, 宇都宮大学工学研究科, 神戸大学工学部の外国人研究員を経て, 1991年10月に総合研究大学院大学数物科学統計科学専攻の博士後期課程に入学, 1994年9月博士後期課程終了. 
博士(学術). 現在学術振興会の特別研究員(The JSPS Postdoctoral Fellowship)として国立国語研究所で研究. 統計分析, パターン認識・分類, 自然言語に関する研究に従事.}

\bioreceived{受付}
\biorevised{再受付}
\bioaccepted{採録}

\end{biography}

\end{document}
