\documentstyle[epsf,jnlpbbl]{jnlp_j_b5_re3vised}

\setcounter{page}{3}
\setcounter{巻数}{3}
\setcounter{号数}{2}
\setcounter{年}{1996}
\setcounter{月}{4}
\受付{1994}{5}{13}
\再受付{1994}{9}{26}
\再々受付{1995}{4}{10}
\再三受付{1995}{8}{30}
\採録{1995}{11}{28}

\setcounter{secnumdepth}{2}


\title{日本語の終助詞の機能 \\ --- 「よ」「ね」「な」を中心として ---}
\author{中川 裕志\affiref{YNU} \and 小野 晋\affiref{YNU}}

\headauthor{中川 裕志・小野 晋}
\headtitle{日本語の終助詞の機能 ---「よ」「ね」「な」を中心として---}

\affilabel{YNU}{横浜国立大学 工学部 電子情報工学科}
{Division of Electrical and Computer Engineering, Faculty of Engineering, Yokohama National University}

\jabstract{
終助詞「よ」「ね」「な」は,書き言葉の文には殆んど用いられないが,日常
会話において頻繁に使われており,文全体の解釈に及ぼす影響が大きい.  そ
のため,機械による会話理解には,終助詞の機能の研究は不可欠である.
本論文では,代表的な終助詞「よ」「ね」「な」について,階層的記憶モデ
ルによる終助詞の機能を提案する.まず,終助詞「よ」の機能は,文の表す命題が発
話以前に記憶中のある階層に存在することを表すことである.次に,終助詞「ね」「な」
の機能は,文の表す命題を記憶中に保存する処理をモニターすることである.
本稿で提案する機能は,従来の終助詞の機能が説明してきた終助詞
「よ」「ね」「な」の用法を全て説明できるだけでなく,従来のものでは
説明できなかった終助詞の用法を説明できる.
}

\jkeywords{日本語, 終助詞, 階層的記憶モデル, 確信, 信念, 意味論, 語用論}

\etitle{Function of Japanese Sentence Final Particles \\ 
         ---about `YO' `NE' and `NA'---}
\eauthor{Hiroshi Nakagawa\affiref{YNU} \and Susumu Ono\affiref{YNU}}

\eabstract{
Japanese \hspace{-0.2mm}sentence \hspace{-0.2mm}final \hspace{-0.2mm}particles(JSFPs henceforth) \hspace{-0.2mm}are \hspace{-0.2mm}used \hspace{-0.2mm}extremely \hspace{-0.2mm}frequently \hspace{-0.2mm}in \hspace{-0.2mm}utterances.
We propose functions of Japanese Sentence Final Particles YO NE and NA
that are based on a hierarchal memory model which consists of Long Term
Memory, Episodic Memory and Discourse Memory. 
The proposed functions of JSFPs are basically monitoring functions of
the mental process being done in utterance.  
YO shows that the propositional content of the utterance that ends with YO was al-ready in the speaker's Episodic Memory or Long Term Memory, while
NE and NA show that the speaker is processing the propositional contents with the contents of speaker's memory.
The proposed functions succeed in accounting for the phenomena yet to be
explained in the previous works.
}

\ekeywords{Japanese, Sentence Final Particles, Hierarchal memory model, 
Semantics, Pragmatics}

\makeatletter
\newcounter{enums}
\def\enumsentence{}
\long\def\@enumsentence[#1]#2{}
\makeatother

\newcommand{\res}[1]{}
\newcommand{\rep}[1]{}
\newcommand{\red}[1]{}

\newtheorem{obserb}{}
\newcommand{\reobs}[1]{}

\begin{document}
\maketitle
\section{はじめに} \label{haji}
終助詞は,日本語の会話文において頻繁に用いられるが,新聞のような書き言
葉の文には殆んど用いられない要素である.  日本語文を構造的に見ると,終
助詞は文の終りに位置し,その前にある全ての部分を従要素として支配し,そ
の有り方を規定している.  そして,例えば「学生だ」「学生だよ」「学生だ
ね」という三つの文が伝える情報が直観的に全く異なることから分かるよう
に,文の持つ情報に与える終助詞の影響は大きい.  そのため,会話文を扱う
自然言語処理システムの構築には,終助詞の機能の研究は不可欠である.
そこで,本稿では,終助詞の機能について考える.

\subsection{終助詞の「よ」「ね」「な」の用法}

まずは,終助詞「よ」「ね」「な」の用法を把握しておく必要がある.  終助詞
「よ」「ね」については,\cite{kinsui93-3}で述べられている.それによると,
まず,終助詞「よ」には以下の二つの用法がある.
\begin{description}
\item[教示用法] 聞き手が知らないと思われる情報を聞き手に告げ知らせる用法
\item[注意用法] 聞き手は知っているとしても目下の状況に関与的であると
                気付いていないと思われる情報について,
                聞き手の注意を喚起する用法
\end{description}
\res{teach}の終助詞「よ」は教示用法,\rep{remind}のそれは注意用法である.
\enumsentence{あ,ハンカチが落ちました{\dg よ}.} \label{teach}
\enumsentence{お前は受験生だ{\dg よ}.テレビを消して,勉強しなさい.} \label{remind}
以上が\cite{kinsui93-3}に述べられている終助詞「よ」の用法であるが,漫
画の中で用いられている終助詞を含む文を集めて検討した結果,さらに,以下
のような,聞き手を想定しない用法があった.
\enumsentence{「あーあまた放浪だ{\dg よ}」\cite{themegami}一巻 P.50}\label{hitori1}
\enumsentence{「先輩もいい趣味してる{\dg よ}」\cite{themegami}一巻 P.114}\label{hitori2}
本稿ではこの用法を「{\dg 独り言用法}」と呼び,終助詞「よ」には,「教示」
「注意」「独り言」の三用法がある,とする.  次に,終助詞「ね」につい
て,\cite{kinsui93-3}には以下の三種類の用法が述べられている.
\begin{description}
\item[確認用法]     話し手にとって不確かな情報を聞き手に確かめる用法
\item[同意要求用法] 話し手・聞き手ともに共有されていると目される情報について,
                    聞き手に同意を求める用法
\item[自己確認用法] 話し手の発話が正しいかどうか
                    自分で確かめていることを表す用法
\end{description}
\rep{confirm}の終助詞「ね」は確認用法,
\rep{agree}Aのそれは同意要求用法,
\rep{selfconfirm}Bのそれは自己確認用法である.
\enumsentence{\label{confirm}
\begin{tabular}[t]{ll}
\multicolumn{2}{l}{(面接会場で)} \\
面接官: & 鈴木太郎君です{\dg ね}.\\
応募者: & はい,そうです.
\end{tabular}
}
\enumsentence{\label{agree}
\begin{tabular}[t]{ll}
A: & 今日はいい天気です{\dg ね}.\\
B: & ええ.
\end{tabular}
}
\enumsentence{\label{selfconfirm}
\begin{tabular}[t]{ll}
A: & 今何時ですか.\\
B: & (腕時計を見ながら)ええと,3時です{\dg ね}.
\end{tabular}
}
以上が,\cite{kinsui93-3}で述べられている終助詞「ね」の用法であるが,
本稿でもこれに従う.
\rep{confirm},\rep{agree}A,\rep{selfconfirm}Bの終助詞の「ね」を「な」に代えても
ほぼ同じような文意がとれるので,終助詞「な」は,終助詞「ね」と同じ三
つの用法を持っている,と考える.  ところで,発話には,聞き手を想定する
発話と,聞き手を想定しない発話があるが,自己確認用法としての終助詞「ね」
は主に聞き手を想定する発話で,自己確認用法としての終助詞「な」は主に
聞き手を想定しない発話である.  さらに,\res{megane}のような,終助詞
「よ」と「ね/な」を組み合わせた「よね/よな」という形式があるが,こ
れらにも,終助詞「ね」「な」と同様に,確認,同意要求,自己確認用法がある.
\enumsentence{(眼鏡を探しながら)私,眼鏡ここに置いた{\dg よね}/{\dg よな}.}
\label{megane}
\subsection{従来の終助詞の機能の研究}
さて,以上のような用法の一部を説明する,計算言語学的な終助詞の機能の研究は,
過去に,人称的分析によるもの\cite{kawamori91,kamio90},談話管理理論に
よるもの\cite{kinsui93,kinsui93-3}, Dialogue coordinationの観点から
捉えるもの\cite{katagiri93},の三種類が提案されている.
以下に,これらを説明する.ところで,\cite{kawamori91}では終助詞の表す情
報を「意味」と呼び,これに関する主張を「意味論」と呼んでいる.
\cite{kinsui93,kinsui93-3}では,それぞれ,「(手続き)意味」「(手続き)意味
論」と呼んでいる.\cite{katagiri93}では,終助詞はなにがしかの情報を表す
「機能(function)」があるという言い方をしている.本論文では,
\cite{katagiri93}と同様に,「意味」という言葉は用いずに,終助詞の「機能」
を主張するという形を取る.ただし,
\cite{kawamori91},\cite{kinsui93,kinsui93-3}の主張を引用する時は,原典に
従い,「意味」「意味論」という言葉を用いることもある.
\begin{flushleft}
{\dg 人称的分析による意味論}\cite{kawamori91,kamio90}
\end{flushleft}
この意味論では,終助詞「よ」「ね」の意味は,「従要素の内容について,終
助詞『よ』は話し手は知っているが聞き手は知らなそうなことを表し,終助
詞『ね』は話し手は知らないが聞き手は知っていそうなことを表す」となる.
この意味論では,終助詞「よ」の三用法(教示,注意,独り言)のうち教示用法
のみ,終助詞「ね」の三用法(確認,同意要求,自己確認)のうち確認用法のみ
説明できる.終助詞「よ」と「ね」の意味が同時に当てはまる「従要素の内
容」はあり得ないので,「よね」という形式があることを説明出来ない.
また,聞き手が終助詞の意味の中に存在するため,聞き手を想定しない終助詞「よ」
「ね」の用法を説明できない.この二つの問題点(とその原因となる特徴)は,
後で述べる\cite{katagiri93}の主張する終助詞の機能でも同様に存在する.
\begin{flushleft}
{\dg 談話管理理論による意味論}\cite{kinsui93,kinsui93-3}
\end{flushleft}
この意味論では,「日本語会話文は,『命題+モダリティ』という形で分析
され,この構造は『データ部+データ管理部』と読み替えることが出来る」,
という前提の元に,以下のように主張している.終助詞は,データ管理部の
要素で,当該データに対する話し手の心的データベース内における処理をモ
ニターする機能を持っている.この意味論は,一応,前述した全用法を説明
しているが,終助詞「よ」に関して,後に\ref{semyo}節で述べるような問
題点がある.終助詞「ね」「な」に関しても,「終助詞『ね』と『な』の意
味は同じ」と主張していて,これらの終助詞の性質の差を説明し
ていない点が問題点である.
\begin{flushleft}
{\bf Dialogue coordination}{\dg の観点から捉えた終助詞の機能}\cite{katagiri93}
\end{flushleft}
\cite{katagiri93}では,以下のように主張している.  終助詞「よ」「ね」は,話し
手の聞き手に対する共有信念の形成の提案を表し,さらに,終助詞「よ」は話
し手が従要素の内容を既に信念としてアクセプトしていることを,終助詞
「ね」は話し手が従要素の内容をまだ信念としてアクセプトしていないこと
を,表す.  これらの終助詞の機能は,終助詞「よ」の三用法(教示,注意,独り言)のう
ち独り言用法以外,終助詞「ね」の三用法(確認,同意要求,自己確認)のうち,
自己確認用法以外を説明できる.  この終助詞の機能の問題点は,
\cite{kawamori91,kamio90}の意味論の説明の終りで述べた通りである.
\subsection{本論文で提案する終助詞の機能の概要}
本論文では,日本語会話文の命題がデータ部に対応しモダリティがデータ管
理部に対応するという\cite{kinsui93-3}の意味論と同様の枠組を用いて,
以下のように終助詞の機能を提案する.ただし,文のデータ部の表すデータを,簡
単に,「文のデータ」と呼ぶことにする.
終助詞「よ」は,データ管理部の構成要素で,「文のデータは,発話直前に判
断したことではなく,発話
時より前から記憶にあった」という,文のデータの由来を表す.終助詞「ね」
「な」
も,データ管理部の構成要素で,発話時における話し手による,文のデータを
長期的に保存するかどうか,するとしたらどう保存するかを検討する処理を
モニターする.

さて,本稿では,終助詞を含む文の,発話全体の表す情報と終助詞の表す情報
を明確に区
別する.つまり,終助詞を含む文によって伝えられる情報に,文のデータと話
し手との関係があるが,それは,終助詞で表されるものと語用論的制約で表さ
れるものに分けることができる.そこで,どこまでが終助詞で表されるものか
を明確にする.

ただし,本稿では,活用形が基本形(終止形)または過去形の語で終る平叙文を
従要素とする用法の終助詞を対象とし,名詞や動詞のテ形に直接付加する終
助詞については,扱わない(活用形の呼び方については\cite{katsuyou}に従っ
ている).  また,上向きイントネーションのような,特殊なイントネーション
の文も扱わない.  さらに,終助詞「な」は,辞書的には,命令の「な」,禁止
の「な」,感動の「な」があるが,本稿では,これらはそれぞれ別な語と考え,
感動の「な」だけ扱う.

以下,本論文では,\ref{bconcept}節で,我々の提案する終助詞の機能を表
現するための認知主体の記憶モデルを示し,これを用いて\ref{sem}節で終
助詞の機能を提案し,終助詞の各用法を説明する.\ref{conclusion}節は結論
である.
\section{認知主体の記憶のモデル} \label{bconcept}
\subsection{階層的記憶モデル} \label{class}
終助詞「よ」の表す文のデータの由来や,終助詞「ね」「な」の表す文
のデータの保存のための処理を,表現するためには,認知主体の記憶をモデ
ル化する必要がある.本稿では,階層的記憶モデルを用いてこれを行なう.
これは,認知心理学の分野で作られたモデルである
\cite{koyazu85}\nocite{tanaka92}.  本稿では,\cite{kinsui92}の三階層
の階層的記憶モデルの二階層のそれぞれを確信・信念の二領域に分割したモデルを仮定
する.
 
階層的記憶モデルは,談話記憶領域,出来事記憶領域,長期記憶領域から
なる.  以下,談話記憶領域をDMA(Discourse Memory Area),出来事記憶領域
をEMA(Episodic Memory Area),長期記憶領域をLMA(Long term Memory Area)
と呼ぶことにする.  また,EMAとLMAを合わせて「EMA以下」と言うことにす
る.  DMAでは,音声情報や文字情報,言語情報,概念情報などの処理がなされ
る.  EMAには,会話の過程で参照する情報及び生成した概念情報が蓄えられ
る.  会話している場面の情報もここに蓄えられる.  ここにある情報は,DMA
にある情報ほどではないが,認知主体自身から注意を払われている.  LMAに
は,語彙情報,文法情報,常識などが蓄えられている.  ここにある情報は,注
意を払われていない.  これらの記憶階層において,DMAは,最も浅い階層に
あり,ここにある情報は認知主体が直接参照できる.  EMAは,LMAとDMAの中間
の階層にあり,ここにある情報は,直接参照されることはなく,DMAを介して参
照される.  LMAは,最も深い階層にあり,ここにある情報も,直接参照される
ことはなく, EMAとDMAを介して参照される.  言語器官から入力された情報
はDMAに置かれ, EMAとLMAの情報を参照することによってDMA上で処理され,
それにより結果的に導かれた情報はEMAに伝達される.  EMAに伝達された情
報の一部は,無意識的にLMAに伝達される.

さて,情報はただ来た順番に記憶領域に積み重ねているのではない.  むしろ,
この記憶システムは高度なデータベースシステムとなっている.  特に,DMA
にある命題をキーとして, EMA以下(主に,LMA)にあるその命題の真偽に関係
する情報を短時間のうちに検索してまとめてEMAに持ってくることが出来る.
ここで,「確認する」という言葉を,以下のように定義する.
\enumsentence{{\dg 「確認する」の定義}\\ \label{dif-confirm}
「ある認知主体Aが,ある命題PをA自身のDMAに置いて, Pの真偽に関係する情
報を, A自身のEMA以下(主にLMA)から検索してまとめてA自身のEMAに持って
きて,それらの情報とPの間に矛盾が生じないことを確かめる」ことを「認知
主体Aが命題Pを確認する」と言う.  
}
この定義は,後で,終助詞「ね」の機能の説明に用いる.
\subsection{「確信」「信念」} \label{kandb}
ある認知主体が信じている情報には,信じる強さに違いがある.
例えば,ある認知主体が「月が地球の衛星であ
ること」を信じているとする.  この命題を,このことを知っていそうもない
五,六歳の子供に発話する時は,殆んどこのことを「確信」として発話するだ
ろう.  一方,天文学者を前にした時は「確信でない信念」として発話する場合もありえ
よう.  このように,ある命題を伝える側の認知主体にとって,その命題を確
信とするか否かは,その命題を伝えられる側の認知主体との相対的
関係に依存する.  このことは,以下の\rep{kb}(※)に反映している.
本稿ではこのような観点から「確信」「信念」の定義を行なう.

本稿では,「信じる強さ」の一次近似として,強く信じている場合
を「確信している」,信じてはいるが確信していない場合を「信じている」とする.  また,
確信していることを「確信」,信じていることを「信念」,と呼ぶことにする.
そして,「\(a\)にとって\(p\)が確信である」「\(a\)にとって\(p\)が信念
である」を以下のように定義する.
\enumsentence{\label{kb}
\(a\)を認知主体とする.\\
「\(a\)にとって\(p\)が確信である」とは,
以下の二つのどちらかが成立することである.
\begin{enumerate}
\item \(p\)は,aにとって直接経験により得た情報である
\item \(p\)は,aが直接経験によって得た情報\(q\)と,
      \(q \rightarrow p\)という\(a\)にとっての確信を使って導いたものである
      \(^{\mbox{{\tiny ※}}}\)
\end{enumerate}
一方,
「\(a\)にとって\(p\)が信念である」
とは,
以下のようなことである.
\begin{enumerate}
\item \(p\)は,aが直接経験によって得た情報\(q\)と,
      \(q \rightarrow p\)という\(a\)にとって確信ではない信念を
      使って導いたものである
      \(^{\mbox{{\tiny ※}}}\)
\end{enumerate}
(※)  \(a\)にとって\(q \rightarrow p\)が確信か否かは,
\(a\)を取り巻く状況に依存している.
}
良く知られていることだが,日本語では,
ある認知主体は他の認知主体の主観的経験を,確信として発話出来ない
ことに注意されたい.
\ref{sem}節では,このことを利用して終助詞の機能を考える.

なお,ある情報に対する確信・信念の区別は,
その情報を記憶するときになされる,とする.
つまり,
DMAにある情報をEMA以下に伝達する時にDMA上でなされる.
だから,EMAとLMAは確信と信念の領域に分割されるが,
DMAはそうはならない,と考える.
図1は,以上の説明を踏まえた認知主体の記憶のモデルである.
\begin{center}
\begin{tabular}{|c||l|l|} \hline
\begin{tabular}[t]{c}
談話記憶領域 \\
(DMA)          
\end{tabular}
& \multicolumn{2}{c|}{a} \\ \hline
\begin{tabular}[t]{c}
出来事記憶領域 \\
(EMA)          
\end{tabular}
& 
\begin{tabular}[t]{l}
b \\
\multicolumn{1}{c}{確信}
\end{tabular}
& 
\begin{tabular}[t]{l}
c \\
\multicolumn{1}{c}{信念}
\end{tabular}  \\ \hline
\begin{tabular}[t]{c}
長期記憶領域 \\
(LMA)          
\end{tabular}
& 
\begin{tabular}[t]{l}
d \\
\multicolumn{1}{c}{確信}
\end{tabular} 
&
\begin{tabular}[t]{l}
e \\
\multicolumn{1}{c}{信念}
\end{tabular} \\ \hline
\end{tabular}\\[3mm]
図1  認知主体の記憶のモデル \\
Fig.1 Model of a cognitive agent's memory
\end{center}
aは談話記憶領域,
bとcは出来事記憶領域,
dとeは長期記憶領域である.
また,
bとdは確信のある領域,
cとeは信念の領域である

以後,
記憶モデルは,特に断らない限り,発話時の話し手のものとする.
当然,
DMA,EMA,LMAも,発話時の話し手のもので,
このモデル上での操作も発話時の話し手によるものである.
そして,
確信,信念も,発話時の話し手にとってのものとする.
また,
確信/信念について,
EMA以下にある情報だけではなく,DMA上にある情報についても,
発話後にEMAの確信/信念の部分への伝達が決まっているものは,
確信/信念と呼ぶことにする.

\section{終助詞「よ/ね/な」の機能} \label{sem}
\subsection{発話に伴う認知的処理} \label{managedata}
認知主体が発話を行なう場合,その前後に,自身の記憶に対して必ず行なう
処理がある.例えば,文のデータとなる情報を最も強く意識する,つまり,
DMA上に存在させる必要がある.また,文のデータとなったDMA上にある情報
を中長期的に保存する必要もある.そのような,発話に伴う処理を
\rep{ninchi}に示す.
\enumsentence{{\dg 発話に伴う処理}\label{ninchi}
\begin{enumerate}
\item DMA上の概念が,
  文のデータとなる(日本語の)文を作り,さらに,音声の形式に変換する \label{ctos}
\item 文のデータ部を発話する
\item EMA以下にデータを保存する必要があるかどうか,
      また必要ならその保存方法を検討する \label{store}
      (文に「データ管理部」の要素がある場合,この発話と並行して行なう) 
\item 必要ならEMA以下に保存する \label{last}
\end{enumerate}
}
\rep{ninchi}\ref{store},\red{last}において,
EMA以下にデータを保存する必要がある場合とは,例えば,EMA以下には存在
しないデータだった場合である.また,EMA以下から持ってきた情報でも,デー
タがDMA上で変化した場合,例えば,このデータを発話したことがあるとい
うマークをデータ自身に付加する場合は,再保存することになる.DMAの情
報と全く同じ信頼性と内容のデータがEMA以下にある場合は,保存する必要
はなく,例えば,DMAに持って来られたが,文解析に使われただけで内容の
変化しなかった語彙情報は,再保存されずに消去される.
\rep{ninchi}\red{store}のように,
発話と並行して行なう処理もある.文のデータは,発話の最中は DMA にあ
り続けることに注意されたい.

ところで,
\rep{ninchi}\red{ctos}のDMA上の概念には,
その由来により,以下の二種類に分けられる.
\enumsentence{{\dg 文のデータの由来}\label{ddata}
\begin{enumerate}
\item DMA上で他のデータからの変換,推論などで導かれたばかりのもの \label{immid}
\item 以前DMAにおいて導かれた後EMA以下に保存されていて,
      再びDMAに持ってこられたもの \label{fromE}
\end{enumerate}
}
文のデータの由来が\rep{ddata}\red{immid}の場合,そのデータが導かれた
時刻を「発話直前」と言うことにする.また,文のデータの由来が\rep{ddata}\red{fromE}の
場合,そのデータが導かれた時刻を「発話時より前」と言うことにする.

\subsection{発話に伴う語用論的制約}
\subsubsection{文のデータの制約}
「太郎は学生だ」という文のデータは,普通,話し手の確信である.  しかし,
{\dg 文の中に,文のデータが確信であるかどうかを示す要素は無い}.  この
ことから,逆に,文のデータが確信でないことを示す表現がなければ文の
データを確信とみなすべきである,という制約があることになる.文のデー
タが確信でないことを示すものとしては,例えば,上向きイントネーション
で発話すれば,いわゆる疑問文になることがある.本稿では,\ref{haji}節で
述べたように,特殊なイントネーションの文は扱っていない.また,
\cite{kamio90}などが示すように終助詞「ね」「な」も,文のデー
タが確信でない可能性を示す表現である.まとめると,以下の制約となる.
\enumsentence{{\dg 文のデータの制約}\\ \label{datakn}
文のデータは,後の部分(データ評価部とデータ管理部)で話し手にとって確
信でない可能性が示されなければ,確信である.  文のデータが確信でない可
能性を示す形式には,データ評価部に現れる要素の「だろう」,データ管理部
に現れる要素の「ね」「な」がある.  
}
一般に終止形,過去形で終る文が話し手の確信を表すのは,この制約のため
である.\cite{kawamori91}の意味論では,文のデータが話し
手の確信であることを,終助詞「よ」の表す情報に含めてしまっていたため,
「よね/よな」を説明できなくなっていた.本稿で提案する終助詞「よ」の表す情報には,文のデータが話し手の確信であることを
含めていないので,「よね/よな」を説明でき,しかも,発話「\(\Phi\)--
よ」で\(\phi\)の内容が確信であることをも説明できる.
\subsubsection{会話の目的の制約}
\ref{haji}節の終りで述べた本稿で扱う範囲の文が会話で用いられる場合,
それには,\rep{cobj}のような通常の会話参加者が同調している語用論的制
約がある.
\enumsentence{\label{cobj}{\dg 会話の目的}\\
  会話の目的は,
  話し手および聞き手の内的世界が,話題に関して同一の状態になることである.\cite{kamio90}}

これは,\cite{katagiri93}が提案する終助詞「よ」「ね」の機能に含まれ
ていた,「共有信念の提案」に相当する.本稿で提案する終助詞「よ」「ね」の機能には,
共有信念の提案は含めない.  聞き手を含む終助詞の用法は,
制約\rep{cobj}を用いて説明することになる.

\subsection{終助詞「よ」の機能と発話「\(\Phi\)--よ」が伝える情報} \label{semyo}
終助詞「よ」の機能を考える上で,以下の二つの観察に注目する
\begin{obserb}\label{yohyp1}
火のついているストーブにうっかり手を触れてしまった場面で,その瞬間に,
思わず「熱い」と言うことがあっても,「熱い{\dg よ}」と言うことはあり
得ない.  「熱い」と言った後に「\verb+(+このストーブ\verb+)+熱い{\dg 
よ}」と言うことは可能である.
\end{obserb}
\begin{obserb}\label{yohyp2}
「私は眠い{\dg よ}」という文のデータは話し手の確信だが,「君,今,眠い
{\dg よね/よな}」という文のデータは,聞き手の主観的経験なので,話し
手の確信ではない.
\end{obserb}
\reobs{yohyp1},\ref{yohyp2}から,
終助詞「よ」の機能を以下のように提案する.
\enumsentence{{\dg 終助詞「よ」の機能} \\ \label{yo}
DMA中の文のデータが発話直前にEMA以下より持ってきたものであることを,表す.
}
さらに,\rep{yo}とGriceの会話の公理の中の量の公理により,終助詞「よ」
がないことに関して以下のことが導かれる.
\enumsentence{\label{noyokinou}
{\dg 終助詞「よ」が無いことが表す情報}\\ 
DMA中の文のデータは,発話直前に他
の情報から(変換,推論などして)DMA上に導いた情報である.  
}
以下,\(\phi\)\hspace{-0.1mm}は文のデータとし,\(\Phi\)\hspace{-0.2mm}を\hspace{-0.2mm}\(\phi\)に対応する文の表層上の表現とする.する
と,\(\Phi\)\hspace{-0.1mm}は終助詞を含\\まないので,終助詞「よ」や「ね」が無いことが何かを表す
とするなら,発話「\(\Phi\)--よ」についても,「ね」「な」が無いことが何を表すのかを考慮する必要があ
る.実際,後で述べるように,終助詞「ね」も「な」も無いことが表す情報として
\rep{nonenakinou}があるので,発話「\(\Phi\)--よ」が表す情報の一部として
\rep{syo}\red{syo3}が得られる.これと,終助詞「よ」の表す情報\rep{yo}と,
文のデータの制約
\rep{datakn}により,発話「\(\Phi\)--よ」の表す情報は以下のようになる.
\enumsentence{\label{syo}{\dg 発話「\(\Phi\)--よ」が表す情報} 
\begin{enumerate}
\item \(\phi\)は,発話直前にEMA以下から持ってきたものであり,かつ, \label{syo1}
\item \(\phi\)は,確信であり,かつ, \label{syo2}
\item \(\phi\)は,発話後直ちにEMA以下に確信として保存される \label{syo3}
\end{enumerate}
}
終助詞「よ」の三用法(教示,注意,独り言)の差は,意味論ではなく語用論の
レベルのものである.  話し手が,「聞き手は\(\phi\)を確信していない」と予想し,
\rep{syo}\ref{syo1},\red{syo2}で話し手が\(\phi\)を既に確信\\してい
ることを聞き手に示すことで,会話の目的の制約\rep{cobj}により
聞き手にも\(\phi\)を確信させようとする場合は,教示用法である.
教示用法の例として\res{teach}を再掲する.
\enumsentence{あ,ハンカチが落ちました{\dg よ}.} \label{reteach}
\res{reteach}では,話し手が,
「ハンカチが落ちました」という事実を確信してなさそうな聞き手に対して発話
時以前からこれが話し手の確信になっていることを
\rep{syo}\ref{syo1},\red{syo2}で示すことで,
会話の目的の制約\rep{cobj}により,この事実が確信であることに
関して話し手と聞き手が同一の状態になることを意図しているので,この
「よ」は教示用法になる.
話し手が,「聞き手は\(\phi\)を確信している\\が注意を払っていない」と予想し,\(\phi\)
をEMA以下に確信として保存することを\rep{syo}\red{syo3}で示すとき,こ
の「EMA以下」がLMAではない部分,即ちEMAで,\rep{cobj}により\(\phi\)がEMAにあること(つまり,\(\phi\)に注意
を払っていること)に関して話し手と聞き手が同一の状態になることを意図す
れば,注意用法になる.注意用法の例として\res{remind}を再掲する.
\enumsentence{お前は受験生だ{\dg よ}.テレビを消して,勉強しなさい.} \label{reremind}
この場面で,「聞き手が受験生である」ことを,話し手も聞き手も確信してい
る.話し手は聞き手がこれに注意を払っていないと予想し,話し手がこれに
注意を払っていることを\rep{syo}\red{syo3}で示し,\rep{cobj}によりこ
れに注意を払っていることに関して話し手と聞き手が同一の状態になること
を意図しているので,聞き手の注意を喚起する用法になる.
\rep{syo}にあてはまる\(\phi\)がDMAに現れたときに,思わず,聞き手を想定せず
に発話すれば,独り言用法になる.

\cite{kinsui93,kinsui93-3}の主張する終助詞「よ」の機能と,
本稿で提案した終助詞「よ」の機能は,大きく異なっている.  前者の終助
詞「よ」の機能は,本稿の記憶モデルで表すと,以下のようになる.  まず,
EMA以下は幾つかの領域に分割されている.  各領域は,何らかの共通点を持っ
た幾つかの情報が保存されている.  この領域の一つ一つが
\cite{kinsui93,kinsui93-3}の意味論における「文脈」である.  領域のう
ちの一つはDMAと,直接,情報を授受できるようになっている.  この領域を
仮に「焦点領域」と呼ぶ.  EMA以下にあるどの領域も焦点領域になり得る.
会話が始まると,EMA以下の領域のうち最もふさわしいものが,話し手であ
る認知主体によって,選ばれ,会話の間ずっと存在し続ける.  焦点領域は,
終助詞の無い文の発話では,変更されない.  そして,終助詞「よ」の機能は
\rep{yodmt}の処理を表すことである.
\enumsentence{{\dg \cite{kinsui93,kinsui93-3}の意味論で主張されている
                   終助詞「よ」の表す処理}\\ \label{yodmt}
EMA以下の適当な領域を焦点領域に選択してDMA中の文のデータを転送する
}
終助詞が無い文では,\rep{yodmt}の処理ではなく,\rep{noyodmt}の処理を行
なう.
\enumsentence{{\dg 終助詞「よ」が無い場合に行なわれる処理}\\\label{noyodmt}
現在の焦点領域にDMA中の文のデータを伝える} 
\rep{yodmt}と\rep{noyodmt}は,
焦点領域を選択するか,現在の焦点領域を用いるかで,異なる.現状の焦点
領域を維持するなら\rep{yodmt}の「焦点領域に選択する」必要は無いので,
\rep{yodmt}を行なう場合,普通,現在のとは異なる焦点領域を選択する.
つまり,焦点領域を変更する.
以上が,\cite{kinsui93,kinsui93-3}における終助詞「よ」の機能を本稿の記憶モ
デルで再解釈したものである.これに対し,本稿で提案した終助詞「よ」
の機能は,データ部の由来を表すことである.以下に改めて終助詞「よ」の
機能\rep{yo}を再掲する.
\enumsentence{{\dg 終助詞「よ」の機能}\\ 
DMA中の文のデータが発話直前にEMA以下より持ってきたものであることを,表す.
}
\cite{kinsui93,kinsui93-3}の終助詞「よ」は,
文のデータを発話時にDMAからEMA以下の焦点領域の適合する部分に移す
処理を表すのに対し,本稿で提案した終助詞「よ」は,EMA以下のどの
部分かは問わずに,単に,文のデータが発話直前にEMA以下からDMAに来たことを
表す.
さて,以上のように全く異なる,終助詞「よ」に関する,
\cite{kinsui93,kinsui93-3}で述べる機能と本稿で提案した機能を比較するために,
終助詞「よ」に関する\reobs{atui}に注目する.
\begin{obserb}  \label{atui}
話し手が,一生懸命,政治に関する話をしているときに,火のついているス
トーブにうっかり手を触れてしまった場面で,その瞬間に,思わず「熱い」
と言うことがあっても,「熱い{\dg よ}」と言うことはあり得ない.  「熱い」
と言った後に「\verb+(+このストーブ\verb+)+熱い{\dg よ}」と言うことは
可能である.
\end{obserb}
\cite{kinsui93,kinsui93-3}の意味論では,
まず,発話直前の焦点領域は政治に関するものである.そして,「熱い」こ
とは,政治の話とは無関係であるから,焦点領域を変更すべきである.だか
ら,ストーブに触った瞬間「熱い{\dg よ}」と言うことになっ
てしまい,観察と矛盾する.仮に,「このような緊急の場面では,人間の情
報処理能力では焦点領域を選ぶ時間的余裕がない」と説明するとしても,これ
は,人間の情報処理能力という性質がよく分からない外的な要因を用いた説
明で,この能力の異常に速い認知主体には,このような場面でも「熱い{\dg 
よ}」と言えることになってしまう.これに対し,本稿で提案した終助詞の機能では,
「『(触る前の)ストーブが熱い』という情報は,発話時より前,つまり,ス
トーブに触る以前のEMA以下にはあり得ないから,ストーブに触った瞬間に
『熱い{\dg よ}』とは言えない」と終助詞「よ」の本質的な機能に基づいて
説明できる.

次に,
\cite{kinsui93,kinsui93-3}に述べられている,以下の現象に注目する.
\begin{obserb} \label{1a1}
「\(1 + 1 \)は\verb+?+」という問いに「\(2\) です」と答えずに
「\(2\) です{\dg よ}」と答えることで,
「何でそんなことを聞くのか」という回答者の``いぶかしみ''が表現される
\end{obserb}
\cite{kinsui93,kinsui93-3}では,これを以下のように説明している.
焦点領域を変更しない場合,
「\(2\) です{\dg よ}」と答えることで,
焦点領域を,
「何でそんなことを聞くのか」というAの発話の意図の推測までを含んだ
発話状況に関するものに変更することが,表されるので,
その結果として\reobs{1a1}の``いぶかしみ''が表現される.
本稿で提案した終助詞の機能では,次のように説明する.  \(1 + 1\) が \(2\) 
であることは,その場で計算するまでもなく誰でも知っている.  つまり,
誰にとってもDMAで計算して導くまでもなく,質問される以前から記憶(この
場合,EMA以下の確信の部分)にあることである.  そのため,「\(2\) です{\dg 
よ}」という発話は,終助詞「よ」により,\(1 + 1\) が\(2\) であることが,発話時より
前から記憶にあることが示され,「そんなこと,私が知らない筈ないでは
ないか.なのに何で聞くんだ」という具合に,\reobs{1a1}の``いぶかしみ''
が表現される.  つまり,我々の理論は「何で聞くのか」という発話意図を
より明確に導くことが出来る.

\subsection{「ね」の機能}\label{semnena}
\cite{katagiri93}で,終助詞「ね」の機能の一部について,
「終助詞『ね』は話し手が文の従要素の内容をまだ信念としてアクセプトしていないこと
を表す」と主張していた.
これは,本稿の記憶モデルで再解釈すると,以下のようになる.
\enumsentence{\label{nenahyp1}
終助詞「ね」「な」は,
文のデータを,EMAに伝達せずに,DMA上に保っていることを表す.
}
次に,「ね」「な」について以下の観察がある.
\begin{obserb}\label{nenahyp2}
「私は眠い{\dg ね}/{\dg な}」という文のデータは話し手の確信だが,
「君,今,眠い{\dg ね}/{\dg な}.」という文のデータは,
聞き手の主観的経験なので,話し手の確信ではない.
\end{obserb}
さらに,終助詞「ね」と「な」の機能が
同じではないことを示す以下のような現象がある.
\begin{obserb}\label{nenahyp3}
「眠い{\dg な}」は眠い人を話し手{\dg のみ}と解釈しやすいが,
「眠い{\dg ね}」では必ずしもそうではない
\verb+(+少なくとも筆者の第一の読みは,話し手と聞き手の両方である\verb+)+.
\end{obserb}
このことは,終助詞「ね」が確認作業を表すが「な」は表さない,と考えれば
説明できる.  つまり,「話し手{\dg のみ}が眠い」かどうかは話し手にとっ
て確認するまでもないことなので,「眠い{\dg ね}」の眠い人は話し手
{\dg のみ}にはならない.  「確認」の定義は\rep{dif-confirm}で既に述べ
た.  我々の利用している記憶の階層モデルによれば,
\rep{nenahyp1}および\reobs{nenahyp2},\ref{nenahyp3}から,
終助詞「ね」の機能は以下のようになる.
\enumsentence{\label{ne}
{\dg 終助詞「ね」の機能} \\
文のデータを確認中であることを,表す.
}
\rep{ne}の確認の間は,文のデータは DMA上にあり続けることに注意されたい.
次に,終助詞「な」の機能は以下のようになる.
\enumsentence{\label{na}
{\dg 終助詞「な」の機能} \\
文のデータになんらかの処理をしている最中であることを,表す
}
こちらも,終助詞「ね」の場合と同様,文のデータは,(EMAに伝達せずに)
DMA上にあり続けることになる.

終助詞「ね」を含む文では,終助詞「ね」の機能\rep{ne}により,文のデータ
が確かめる必要のないものだと,不自然になる. 例えば,山田太郎という名
前の男の「?私は山田太郎だ{\dg ね}」という発話は不自然である. 
同様に,終助詞「な」を含む文は,文のデータが何らかの処理をする必然性の無いものの場合,
終助詞「な」の機能\rep{na}により,不自然になる. 例えば,次郎という名の男による
「?私は次郎だ{\dg な}」という発話は不自然である. 

\rep{ne},\rep{na}とGriceの会話の公理の中の量の公理により,
終助詞「ね」も「な」も無いことは,
\rep{ninchi}\red{store}の処理において終助詞「ね」の機能
\rep{ne},「な」の機能\rep{na}で表すような処理がなされていないことを表
す. 
\enumsentence{\label{nonenakinou}
{\dg 終助詞「ね」も「な」も無いことが表す情報}\\
必要ならば,
文のデータを直ちにEMAの適当な部分(確信の部分か信念の部分)に伝達することを表す.
}
ただし,伝達の必要性については\rep{ninchi}\ref{store},\red{last}に関
連して述べた.

発話「\(\Phi\)--ね」「\(\Phi\)--な」が表す情報は,終助詞「よ」が無いことが表す情報
\rep{noyokinou}と,終助詞「ね」の機能\rep{ne}あるいは「な」の機能
\rep{na}と,終助詞「ね」「な」が\rep{datakn}の「確信でない可能性を示
す」要素であることから,以下のようになる.
\enumsentence{\label{sne}{\dg 発話「\(\Phi\)--ね」が表す情報}
\begin{enumerate}
\item \(\phi\)は,発話直前に他の情報から(変換,推論などして)DMA上に導いた情報である
\item \(\phi\)を,確認中である
\end{enumerate}
}
\enumsentence{\label{sna}{\dg 発話「\(\Phi\)--な」が表す情報}
\begin{enumerate}
\item \(\phi\)は,発話直前に他の情報から(変換,推論などして)DMA上に導いた情報である
\item \(\phi\)を,処理中である
\end{enumerate}
}
終助詞「ね」「な」の,三用法(確認,同意要求,自己確認)の差も,
意味論ではなく語用論のレベルのものである.
話し手が\rep{sne}2.または\rep{sna}2.で\(\phi\)をまだEMA以下に保存していない
ことを伝えて,聞き手にその手伝い,つまり,確信とすべきか否かを判断するための協力,を求める場合は,確認用法
となる.確認用法の例として
\rep{confirm}を再掲する.
\enumsentence{\label{reconfirm}
\begin{tabular}[t]{ll}
\multicolumn{2}{l}{(面接会場で)} \\
面接官: & 鈴木太郎君です{\dg ね}.\\
応募者: & はい,そうです.
\end{tabular}
}
\rep{reconfirm}の面接官は,
面接時の状況(手元の履歴書写真と聞き手の顔が似ている,など)から聞き手
が鈴木太郎であることを確認中であることを\rep{sne}2.で示し,同時に聞き
手に確認のための協力を求めることで,確認用法となっている.
\rep{sne}2.または\rep{sna}2.で,
話し手が文のデータを確認中(処理中)であることを聞き手に示すと同時に,文
のデータを確認(処理)していることに関して,会話の目的\rep{cobj}により話
し手が聞き手と同一の状態になることを意図すると,同意要求用法とな
る.同意要求用法の例として\rep{agree}を再掲する.
\enumsentence{\label{reagree}
\begin{tabular}[t]{ll}
A: & 今日はいい天気です{\dg ね}.\\
B: & ええ.
\end{tabular}
}
\rep{reagree}では,
話し手は,今日がいい天気であることを確認中であることを
\rep{sne}2.で示すと同時に,\rep{cobj}により
聞き手にもこれを確認することを求めている.
\(\phi\)の導出に手間どったり\(\phi\)に自信が持\\てないために,聞き手を意識せずに,
\rep{sne}2.または\rep{sna}2.の処理を行なう場合は
自己確認用法となる.
自己確認用法の例として,\rep{selfconfirm}を再掲する.
\enumsentence{\label{reselfconfirm}
\begin{tabular}[t]{ll}
A: & 今何時ですか.\\
B: & (腕時計を見ながら)ええと,3時です{\dg ね}.
\end{tabular}
}
\rep{reselfconfirm}Bの発話は,
\rep{sne}2.で,(時間が)3時であることを確認していることを表すが,
聞き手への確認や同意を求めているわけではなく,自己確認用法になっている.
確認,同意要求,自己確認のどの用法についても
\(\phi\)の由来を表す\rep{sne}1,\rep{sna}2.と無関係であることに注意されたい.


本稿の終助詞「ね」の機能は,\cite{kinsui93,kinsui93-3}の主張する終助詞「ね」の機能と,同じである.
終助詞「な」については,
\cite{kawamori91,kamio90},\cite{katagiri93}では述べられておらず,
\cite{kinsui93,kinsui93-3}では,「終助詞『ね』と同じ意味」とされている.
しかし,本論文では,
終助詞「ね」と「な」の性質に差にもとづいて,異なる機能を提案している.

発話「\(\Phi\)--よね」「\(\Phi\)--よな」が表す情報は,
終助詞「よ」の機能\rep{yo}と「ね」の機能\rep{ne}と「な」の機能\rep{na}により,
以下のようになる.
\enumsentence{\label{syone}{\dg 発話「\(\Phi\)--よね」が表す情報} 
\begin{enumerate}
\item \(\phi\)は,発話直前にEMA以下から持ってきたものであり,かつ, \label{syone1}
\item \(\phi\)を,確認中である \label{syone2}
\end{enumerate}
}
\enumsentence{\label{syona}{\dg 発話「\(\Phi\)--よな」が表す情報} 
\begin{enumerate}
\item \(\phi\)は,発話直前にEMA以下から持ってきたものであり,かつ,\label{syona1}
\item \(\phi\)を,処理中である \label{syona2}
\end{enumerate}
}
これらをそれぞれ,発話「\(\Phi\)--ね」が表す情報\rep{sne},発話「\(\Phi\)--な」が表す情報
\rep{sna}と比較すると,\(\phi\)の由来を表す部分(それぞれ,\rep{syone}1.と
\rep{sne}1,\rep{syona}1.と\rep{sna}1.)は異なるが,\rep{syone}2.と
\rep{sne}2.が同じで,\rep{syona}2.と\rep{sna}2.が同じである.終助詞「ね」
「な」の用法は,\rep{sne}2,\rep{sna}2.によるもので,\(\phi\)の由来を表す
部分とは無関係であった.だから,終助詞の複合形「よね」「よな」にも終助詞
「ね」「な」と同様の用法があることを説明できる.
「よね」の例として,\res{megane}を再掲する.
\enumsentence{(眼鏡を探しながら)私,眼鏡ここに置いた{\dg よね}/{\dg よな}.}
これは,話し手が,
「話し手が眼鏡をここに置いた」ことを確認中であることを
\rep{syone}\red{syone2}で表すだけでなく,
\rep{syone}\red{syone1}で,このことが過去に判断し,
記憶しておいたことであることを表す.

\ref{haji}節で述べた,過去に提案された終助詞の意味(機能)のうち,
「よね」を説明できたのは\cite{kinsui93,kinsui93-3}だけであるが,
本稿でも,以上のように説明することが出来た.


\section{おわりに} \label{conclusion} 
本稿では,終助詞「よ」「ね」「な」と複合形「よね」「よな」について,
終助詞が無いことが何を表すのかを考慮に入れて,階層的記憶モデルにより,
機能を提案した.そして,従来提案されてきた終助詞の全用法と,従来の終助詞の研究
で説明できなかった現象を説明できた.他の終助詞については特に扱わなかっ
たが,本稿で取り上げた終助詞と同様に,機能を考えることが出来る.例えば,
終助詞「ぞ」「ぜ」については,機会を改めて報告したい.
\nocite{kamio90}

\acknowledgment

本論文をまとめるに当たって議論に参加して頂き,有益なコメントを頂いた
本学科森辰則講師に感謝いたします.
また,初期の原稿に有益なコメントを頂いた査読者の方にも感謝いたします.


\bibliographystyle{jnlpbbl}
\bibliography{jpaper}

\renewcommand{\refname}{}
\begin{thebibliography}{}

\bibitem[\protect\BCAY{藤島}{藤島}{1989〜1993}]{themegami}
藤島康介 \BBOP 1989〜1993\BBCP.
\newblock \Jem{ああっ女神様1〜8}.
\newblock 講談社, 東京.

\end{thebibliography}

\begin{biography}
\biotitle{略歴}
\bioauthor{中川裕志}{
  1953年生.
  1975年東京大学工学部卒業.
  1980年東京大学大学院博士課程修了.
  工学博士.
  現在,横浜国立大学工学部電子情報工学科教授,現在の主たる研究テーマは
  自然言語処理.日本認知科学会,人工知能学会などの会員.
}
\bioauthor{小野 晋}{
  1967年生.
  1991年横浜国立大学工学部卒業.
  1993年横浜国立大学大学院工学研究科博士過程前期修了.
  現在,横浜国立大学大学院工学研究科博士過程後期に在学中.
  現在の主たる研究テーマは自然言語処理および日本語の語用論.
  情報処理学会の正会員,日本認知科学会の学生会員.
}

\bioreceived{受付}
\biorevised{再受付}
\biorerevised{再受付}
\biore3vised{再受付}
\bioaccepted{採録}

\end{biography}
\end{document}

