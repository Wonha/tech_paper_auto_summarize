    \documentclass[japanese]{jnlp_1.4}
\usepackage{jnlpbbl_1.2}
\usepackage[dvips]{graphicx}
\usepackage{amsmath}
\usepackage{hangcaption_jnlp}
\usepackage{udline}
\setulminsep{1.2ex}{0.2ex}
\let\underline


\usepackage{lingmacros}
\usepackage{tabularx}

\def\sec#1{}
\def\ssec#1{}
\def\sssec#1{}
\def\eq#1{}
\def\fig#1{}
\def\tab#1{}
\def\bs#1{}
\def\brace#1{}



\Volume{17}
\Number{2}
\Month{April}
\Year{2010}


\received{2009}{7}{17}
\revised{2009}{10}{2}
\accepted{2009}{11}{1}

\setcounter{page}{25}

\jtitle{述語項構造と照応関係のアノテーション:\\
	NAISTテキストコーパス構築の経験から}
\jauthor{飯田  龍\affiref{Author_1} \and 小町  守\affiref{Author_2} \and 
	井之上直也\affiref{Author_2} \and 乾 健太郎\affiref{Author_2} \and 
	松本 裕治\affiref{Author_2}}
\jabstract{
本論文では,日本語書き言葉を対象とした述語項構造と照応関係のタグ付与に
ついて議論する.
述語項構造解析や照応解析は形態素・構文解析などの基盤技術と自然言語処理
の応用分野とを繋ぐ重要な技術であり,これらの解析のための主要な手法はタ
グ付与コーパスを用いた学習に基づく手法である.
この手法を実現するためには大規模な訓練データが必要となるが,これまでに
日本語を対象にした大規模なタグ付きコーパスは存在しなかった.また,既存
のコーパス作成に関する研究で導入されているタグ付与の基準は,言語の違い
や最終的に出力したい解析結果の粒度が異なるため,そのまま利用することが
できない.
そこで,我々は既存のいくつかのタグ付与の仕様を吟味し,述語項構造と共参
照関係のアノテーションを行うためにタグ付与の基準がどうあるべきかについ
て検討した.本論文ではその結果について報告する.また,京都コーパス
第3.0版の記事を対象にタグ付与作業を行った結果とその際に問題となった点に
ついて報告する.さらにタグ付与の仕様の改善案を示し,その案にしたがい作
業をやり直した結果についても報告する.
}
\jkeywords{アノテーション,述語項構造,照応,共参照,コーパス}

\etitle{Annotating Predicate-Argument Relations and Anaphoric Relations: 
	Findings from the Building of \\
	the NAIST Text Corpus}
\eauthor{Ryu Iida\affiref{Author_1} \and Mamoru Komachi\affiref{Author_2} \and 
	Naoya Inoue\affiref{Author_2} \and \\
	Inui Kentaro\affiref{Author_2} \and 
	Yuji Matsumoto\affiref{Author_2}} 
\eabstract{
  This paper addresses how to annotate predicate-argument and
  anaphoric relations in Japanese written text.  Predicate-argument
  structure analysis and anaphora resolution are important problems
  because they bridge the gap between basic techniques in NLP such as
  morpho-syntactic analysis and NLP applications.  To solve these
  problems, researchers have generally made use of annotated corpora
  for machine learning-based approaches.  Although we need large
  corpora where predicate-argument relations and anaphoric relations
  are annotated to examine their occurrence in Japanese text,
  there have been no such resources so far.  In addition, existing
  specifications for annotating predicate-argument and anaphoric
  relations are not directly applicable due to the difference of
  languages and different problem settings.  For these reasons, we
  explore how to annotate these two kinds of relations and then define
  an adequate specification of each annotation task.  In this article, we
  report the results of annotation, taking the Kyoto Corpus 3.0
  as a starting point.  Furthermore, we refine our annotating
  specification to adapt actual phenomena existing in our corpus and
  then report the results of the annotation work according to the new
  specification.  
}
\ekeywords{annotation, predicate-argument structure, anaphora, coreference, corpus}

\headauthor{飯田,小町,井之上,乾,松本}
\headtitle{述語項構造と照応関係のアノテーション}

\affilabel{Author_1}{東京工業大学大学院情報理工学研究科}{Graduate School of Information Science and Engineering, Tokyo Institute of Technology}
\affilabel{Author_2}{奈良先端科学技術大学院大学情報科学研究科}{Graduate School of Information Science, Nara Institute of Science and Technology}



\begin{document}
\maketitle

\input{macro}


\section{はじめに}
\label{sec:first}

情報抽出や機械翻訳などのNLPの応用処理への需要が高まる中で,その技術を実
現するための中核的な要素技術となる照応・共参照と述語項構造の解析に関し
て多くの研究者が解析技術を向上させてきた.それらの技術の多くは各情報が
付与されたコーパス(以後,タグ付与コーパス)を訓練用データとして教師あ
り手法を用いるやり方が一般的であり,解析の対象となるコーパス作成の方法
論についても議論がなされてき
た\cite{Hirschman:97,Kingsbury:02,Doddington:04}.

照応・共参照解析については,主に英語を対象にいくつかのタグ付与のスキー
マが提案されており,実際にそのスキーマに従ったコーパスが作成されてい
る\cite{Hirschman:97,Kawahara:02,Hasida:05,Poesio:04,Doddington:04}.
例えば,Message Understanding Conference (MUC) のCoreference (CO) タスク
\cite{Hirschman:97}や,その後継にあたるAutomatic Content Extraction
(ACE) program のEntity Detection and Tracking (EDT)タスクでは,数年に
渡って主に英語を対象に詳細な仕様が設計されてきた.
また,述語項構造解析に関しては,CoNLLのshared task\footnote{
    http://www.lsi.upc.edu/\~{}srlconll/}で評価データとして利用されて
いるPropBank~\cite{Palmer:05}を対象に仕様が模索されてきた.

日本語を対象に述語項構造と照応・共参照の研究をするにあたり,分析,学習,
評価のための大規模なタグ付きコーパスが必要となるが,現状で利用可能
なGlobal Document Annotation (GDA)~\cite{Hasida:05}タグ付与コーパス(以
後,GDA コーパス)や京都テキストコーパス第4.0版(以後,京都コーパス4.0)
は,述語項構造や共参照の解析のための十分な規模の評価データとはいえな
い.

日本語を対象に述語項構造を照応・共参照の研究を進めるためには,英語の場
合と同様にタグ付きコーパスを構築する必要があるが,日本語では述語の格要
素が省略される\textbf{ゼロ照応}の現象が頻出するため,後述するように述語
項構造の記述の中で照応現象も同時に扱う必要がある.そのため,英語では独
立に扱われている述語項構造と(ゼロ)照応の関係の両方のタグ付与の仕様を
把握し,2つの関係横断的にどのようにタグ付与の仕様を設計するかについて考
える.
タグ付与の仕様は最初から完成したものを目指すのではなく,作業仕様を経験
的に定め,人手によるタグ付与の作業を行い,作業結果を検討することで洗練
していくことを想定している.本論文ではこれまでに行った仕様に関する比較
検討の内容と現在採用している我々の作業仕様について説明する.
この際,MUCやACEの英語を対象に設計されたタグ付与の仕様に加え,日
本語を対象に作成された既存の共参照・述語項構造のタグ付きコーパスであ
るGlobal Document Annotation (GDA)~\cite{Hasida:05}タグ付与コーパス(以
後,GDA コーパス)や京都テキストコーパス第4.0版(以後,京都コーパ
ス4.0)\footnote{
	http://www-lab25.kuee.kyoto-u.ac.jp/nl-resource/corpus.html}との比較
も行う.

本論文ではまず\sec{second}で照応と共参照の関係について確認
し,\sec{third}では述語項構造と照応・共参照のタグ付与に関する先行研究を
紹介する.次に,\sec{fourth}で先行研究を踏まえた上の我々のタグ付与の基
準を示し,その基準に従った作業結果についても報告する.さら
に,\sec{fifth}で今回作業を行った際に問題となった点について説明
し,\sec{fifth}でその改善案とその案にしたがって作業をやり直した結果につ
いて報告し,最後に\sec{seventh}でまとめる.

また,今回の作業の結果作成された述語項構造と照応・共参照タグ付与コーパ
スをNAISTテキストコーパスとして公開している.詳細は{http://cl.naist.jp/nldata/corpus/}を参照されたい.







\section{照応と共参照}
\label{sec:second}

\begin{figure}[b]
\begin{center}
\includegraphics{17-2ia3f1.eps}
\end{center}
\caption{文内ゼロ照応}
\label{fig:zero}
\end{figure}

\textbf{照応}とはある表現が同一文章内の他の表現を指す機能をいい,指す
側の表現を\textbf{照応詞},指される側の表現を\textbf{先行詞}という.
日本語の場合は述語の格要素の位置に出現している照応詞が頻繁に省略される.
この省略された格要素を\textbf{ゼロ代名詞}といい,ゼロ代名詞と照応関係と
なる場合を\textbf{ゼロ照応}と呼ぶ.
本研究では便宜上ゼロ照応の関係をゼロ代名詞とその先行詞の出現位置で3種類
に分類する.ゼロ代名詞と先行詞が同一文内に出現している場合を\textbf{文
内ゼロ照応}と呼び,また先行詞がゼロ代名詞と同一文章内の異なる文に出現
している場合を\textbf{文間ゼロ照応}と呼ぶ.例えば,\fig{zero}で``行く''の
ガ格が省略されており,その項が述語と係り受け関係にないため,``行く''の
ガ格のゼロ代名詞とその項``太郎''は文内ゼロ照応の関係にあると考え
る\footnote{この例は並列表現はゼロ照応と区別すべきという議論もあるが,
  項が係り受け関係にない場合は統一的にゼロ照応とみなすほうが機械処理を
  行う際には見通しがよいと考えている.}.最後に,ゼロ代名詞の先行詞が文
章内に出現しない場合を\textbf{外界照応}と呼ぶ.



一方,二つ(もしくはそれ以上)の表現が現実世界(もしく
は仮想世界)において同一の実体を指している場合には\textbf{共参照}(もし
くは\textbf {同一指示})の関係にあるという.先行詞となる表現が固有表現
になる場合など,多くの場合は照応関係かつ共参照の関係が成り立つ.
例えば,文章\NUM{ex1}では,代名詞``彼$_i$''が``横尾$_i$''を指しており,
かつ同一の人物を指しているため,照応関係かつ共参照関係である.
\EX{ex1}{
  \ul{横尾}$_i$は画家でもないし、デザイナーでもない。\\
  ~要するに、そんなことは\ul{彼}$_i$にとってはどうでもよいことなのだ。
} 
これに対し,文章\NUM{ex2}では,2文目の``それ$_i$''は1文目
の``iPod$_i$''を指しているため照応関係となるが,同じ実体を指していない
ため共参照関係とはならない.
\EX{ex2}{
  太郎は\ul{iPod}$_i$を買った。\\
~次郎も\ul{それ}$_i$を買った。}
このように照応関係にある場合でも,同一の実体を指している場合とそれ以外
の場合が存在する.文献~\cite{Mitkov:02}では,前者のような共参照かつ
照応関係となる関係をidentity-of-reference anaphora (IRA), 後者を
identity-of-sense anaphora (ISA) と呼び区別している.


照応と共参照は異なる概念であるにもかかわらず,IRAが両方の性質を兼ねる
ため,既存研究ではそれぞれの概念が混同して扱われてきた.\sec{third}で
述べるタグ付与コーパス作成の先行研究でも同様にいくつかの異なる解釈で仕様が
設計されている.




\section{先行研究}
\label{sec:third}

この節では,共参照と述語項構造のタグ付与に関する主な先行研究を説明する.


\subsection{照応・共参照のタグ付与}
\label{ssec:pre_coref}

情報抽出の主要な会議であるMessage Understanding Confernce (MUC)では,
第6回と第7回の会議(以後,MUC-6とMUC-7)において,情報抽出の部分問題と
して共参照解析の問題を扱っている\footnote{
    http://www-nlpir.nist.gov/related\_projects/muc/proceedings/co\_task.html}
.MUC-6,MUC-7の共参照関係タグ付与コーパスでは名詞句間の共参照関係がタ
グ付与され(ただし,動名詞 (gerund) は除く),Soonら\cite{Soon:01}やNgら\cite{Ng:02}などさまざまな機械学習に基づ
く共参照解析手法のgold standardデータとして利用されてきた.しかし,この
コーパスの仕様では,一般に共参照関係とはみなされないような量化表現
(every,mostなど)を伴う場合や同格表現 (\ul{Julius Caesar}$_i$,
\ul{\mbox{the/a well-known} \mbox{emperor}}$_i$,...) も共参照関係とみなしてタグ付与さ
れているという問題を含んでいる\footnote{詳細は文献\cite{Deemter:99}を参照されたい.}.

MUCの共参照解析タスクの後継に相当するAutomatic Content Extraction
(ACE)~\cite{Doddington:04}のEntity Detection and Tracking (EDT)では,こ
の過剰な共参照関係の認定を回避するため
に,mention(\textbf{言及})とentity(\textbf{実体})という2つの概念を
導入しタスクを設定している.言及とは文章中に出現する表現のことで,情報
抽出で解析の対象となる特定の種類の固有表現を含む.これに対し,実体と
は\sec{second}で述べた意味での実体,つまり現実世界(もしくは仮想世界)
で指せるモノを意味する.例えば,\fig{ace}の例については,``ジョ
ン''と``彼''はそれぞれクラスがnamesとpronounsの言及であり,その2つの表
現が実体のレベルではクラスがspecific\_referenceである同一の実体を指して
いるというタグ付与を行う.

\begin{figure}[t]
\begin{center}
  \includegraphics{17-2ia3f2.eps}
\end{center}
  \caption{言及 (mention) と実体 (entity)}
  \label{fig:ace}
\end{figure}

EDTのタグ付与\footnote{
    http://projects.ldc.upenn.edu/ace/annotation/} では,言及の型が人名
や組織名など特定の種類の固有表現に該当する場合で,かつ総称的 (generic) 
でない場合にのみ共参照関係のタグ付与を行う.このため,ACEのデータセット
では文章内に出現する共参照関係に網羅的にタグが付与されず,文章内のすべ
ての実体を対象として解析を行うには不十分なデータとなっている.


日本語に関しても,京都コーパス4.0 \cite{Kawahara:02}やGDAコーパス
\cite{Hasida:05}などのタグ付きコーパスに共参照相当のタグが付与されてい
る.
京都コーパス4.0には,係り受けの情報に加え,毎日新聞95年度版の一部(555
記事,5,127文)に114,729もの共参照タグが付与されている.ただし,このコー
パスではACEで導入されている実体と実体の間の共参照関係に加え,実体と属性
の間にも共参照関係のタグを付与している.例えば,文\NUM{kyc}では,実
体``村山富市$_i$''とその属性``\nobreak 首相$_i$''の間に共参照の関係が付与さ
れることになる.
\EX{kyc}{ 
\ul{村山富市}$_i$ \ul{首相}$_i$の年頭記者会見の要旨は次の通り。
}

一方,GDAコーパスでは,表現が実体を指している場合と総称的な表現の場合を
区別せずに共参照の関係を認定している.下記の\NUM{gda_ex}はGDAコーパスか
ら抜粋したものだが,この文章では総称名詞である2つの``フロン$_i$''に対し
て共参照タグが付与されている.このような例が多数見られたため,GDAの共参
照タグはIRAとISAの両方の関係で付与されていると考えられる.
\EX{gda_ex}{
  \ul{フロン}$_i$対策急げ...\ul{フロン}$_i$による環境破壊対策は...
}



\subsection{述語項構造のタグ付与}
\label{ssec:pre_pred}

述語とその項\footnote{本稿で用いる用語``項''はcomplementとadjunctの両
  方を指す.} のタグ付与に関しては,表層格レベルから深層格レベルまでさ
まざまなレベルでのタグ付与についての議論がある.例えば,英語を対象とし
たPropBank~\cite{Palmer:05}では,述語の項のラベルとして,基本的には
agentやthemeなどの意味役割に相当するARG0, ARG1, $\dots$, ARG5, AA, AM,
AM-ADVなど,35種類のタグを用いて文章にタグ付与を行っている.一例をあげ
ると,文\NUM{prop}に出現している動詞``earned''に対し,``the
refiner''をagent相当であるARG0,``\$66 million, or \$1.19 a share''を
themeに相当するARG1としてタグが付与されている.
\EX{prop}{
[$_{{\rm ARGM-TMP}}$ \textit{A year earlier}], [$_{{\rm ARG0}}$ \textit{the refiner}] [$_{{\rm rel}}$ \textit{earned}] [$_{{\rm ARG1}}$ \textit{\$66 million, or \$1.19 a share}]. 
}
ただし,PropBankでは項が付与される範囲は同一文内に限定されている.

一方,日本語を対象に述語項構造を考える場合は必須格が省略されるゼロ照応
の現象が頻繁に起きるため,文を越えて出現している表現や,もしくは文章外
の要素も考慮してタグ付与を行う必要がある.京都コーパス4.0では文間ゼロ照応,
外界照応となる項に関してもタグが付与されている.共参照タグ付与の対象と
なった555記事を対象にガ/ヲ/ニ/カラ/ヘ/ト/ヨリ/マデなどの格助詞相当の表
層格に加え,ニツイテのような連語も一つの表層格として述語と項の関係が付
与されている.
例えば,文\NUM{zero_ex}に出現している述語``答え(る)''では前方に出現し
ている``状態$_i$''をニツイテという表層格でタグ付与している.
\EX{zero_ex}{
体の\textbf{状態}$_i$について健康と\ul{答えた}$_{ニツイテ:i}$人は八七・八%で、体力に自信を持っているとの回答は八一・一%だった。
}

また,GDAコーパスではゼロ照応に関してagent,themeなどの粒度で意味役割の
タグが付与されているが,我々が確認した限りでは,述語と係り関係にある場
合や,ゼロ照応の関係として認定できる場合であっても先行詞が同一文内に出
現している場合にはタグが付与されておらず,述語項構造解析の訓練事例とし
て利用するには網羅性の点で問題がある.



\subsection{事態性名詞のタグ付与}
\label{ssec:event_noun}

動詞や形容詞などの述語への項構造の付与に加え,動詞派生名詞やサ変名詞な
どの名詞(以後,\textbf{事態性名詞})
についても述語と同様に,項同定の問題が設計され
\cite{Meyers:04,Hasida:05,Kawahara:02},実際にそれらの問題への
取り組みも報告されている\cite{Jiang:06,Komachi:07,Liu:07}.

例えば,Meyersらが作成したNomBank~\cite{Meyers:04}では,Penn
Treebank~\cite{Marcus:93}を対象に名詞とその項構造のタグ付与を行ってい
る.このコーパスでは英語における動詞の名詞化に着目し
て,PropBank~\cite{Palmer:05}で用いられている意味役割相当の項のラベルを
句の中に項が出現している場合に限ってタグ付与している.
例えば,文\NUM{nom}中の名詞``complaints''について,``customer''がagent相当の表現であり,また
``about that issue''がtheme相当の表現であるといったタグ付与を行っている.
\EX{nom}{
There have been [$_{\rm ARGM-NEG}$ \textit{no}] [$_{\rm ARG0}$ \textit{customer}] [$_{\rm rel}$ \textit{complaints}] [$_{\rm ARG1}$ \textit{about that issue}].
}

日本語に関しても,京都コーパス4.0では事態性名詞とその項に対して表層
格でタグが付与されている.例えば,文\NUM{event_ex}に示すように``及ぼ
す''の格要素となっている事態``影響''に対して,``離党$_i$''が``影響する''ことの
ガ格として付与されている.
\EX{event_ex}{ 
  村山富市首相は年頭にあたり首相官邸で内閣記者会と二十八日会見し、社会
  党の新民主連合所属議員の\textbf{離党}$_i$問題について「政権に\ul{影響}$_{ガ:i}$
  を及ぼすことにはならない。離党者がいても、その範囲にと
  どまると思う」と述べ、大量離党には至らないとの見通しを示した。
}
名詞の項構造については,名詞と項の関係が「\textbf{候補}$_i$\ul{擁
  立}$_{ヲ:i}$」や「\textbf{兵士}$_j$の\ul{脱走}$_{ガ:j}$」のように,複
合名詞句の中や``AノB''などに縮退される場合もあり,このような場合どこま
でをタグ付与の対象とするかを決定する必要がある.



\section{NAISTテキストコーパスで採用するタグ付与の仕様}
\label{sec:fourth}

\sec{third}で述べた先行研究のタグ付与仕様の背景にある考え方,およびまた
その仕様を採用した場合に生じる問題点を考慮した結果,我々は以下の3つの基
準を基本方針として採用するに至った.
\begin{enumerate}
\item \textbf{述語項構造については,述語の基本形にその項となる表現を表層格(ガ格,ヲ格,ニ格)レベルでタグ付与する.}
\item \textbf{事態性名詞についても,述語と同様に表層格レベルで項を付与する.}
\item \textbf{共参照関係については,IRAの関係のみを対象として共参照の関係を認定する.}
\end{enumerate}
以下で,それぞれの詳細を説明する.


\subsection{述語と項のタグ付与}
\label{ssec:spec_pred}

述語そのものの認定に関しては,品詞体系としてIPADIC~\cite{Asahara:03}を
採用し,動詞,形容詞,名詞+``だ(助動詞)''の3種類をタグ付けの対象とな
る述語とみなし,作業を行う.

述語の格要素については,京都コーパス4.0が採用しているような表層
格,GDAで採用されているagent,theme のような意味役割,またPropBankで付
与されているARG0やARG1といった意味役割相当のラベルなど,さまざまなタグ
付与のレベルが考えられる.この中で我々は「誰が何を何に対してどうする」
という情報抽出的な観点でタグを付与することが自然だと考え,述語の原形の
必須格に対して項のタグを付与するという仕様を採用した.
この仕様に従った場合,さらにガ格などの表層格レベルでのタグ付与を行うか,
もしくはagentもしくはARG0といった意味役割レベルでタグを付与するかとい
う2つの選択肢があるが,ここでは
表層レベルからなんらかの情報を捨象して意味のレベルを考えることが応用処
理横断的に有益かどうかが現状では判断できないという理由で,格交替の情報
のみを捨象した表層格でタグ付与を行った.
例えば,京都コーパス4.0では文\NUM{pred_ex1}の述語``食べさせる''に対して
``私$_i$'',``彼$_j$'',``リンゴ$_k$''をそれぞれガ,ヲ,ニ格でタグ付与
するのに対し,我々の仕様では述語の原形``食べる''に対して``彼$_j$ガリン
ゴ$_k$ヲ食べる''というタグを付与する.
ただし,述語の原形に対してタグを付与する場合には使役者に相当する``私
$_i$''と述語``食べる''の間の関係にタグが付与されないことになる.これを
回避するため,格要素を増やす助動詞に対してタグ``追加ガ(ニ)格''を付与
した.例えば,文\NUM{pred_ex1}では,助動詞``させる''に対し``私$_i$''を
追加ガ格でタグ付与し,文\NUM{pred_ex2}では助動詞``やる''に対し``彼
$_j$''を追加ニ格でタグ付与する.
\EXS{pred_ex1}{
\small
\item \textbf{私}$_i$は\textbf{彼}$_j$に\textbf{リンゴ}$_k$を\ul{食べさせる}$_{ガ:i,ヲ:k,ニ:j}$
\item \textbf{私}$_i$は\textbf{彼}$_j$に\textbf{リンゴ}$_k$を\ul{食べ}$_{ガ:j,ヲ:k}$\ul{させる}$_{追加ガ格:i}$
}
\EX{pred_ex2}{
\small
\textbf{私}$_i$は\textbf{彼}$_j$に\textbf{本}$_k$を\ul{読ん}$_{ガ:i,ヲ:k}$で\ul{やる}$_{追加ニ格:j}$
}
京都コーパス4.0では表層格を網羅する形をとっているが,
今回の作業では,頻出するガ/ヲ/ニ格のみを対象にタグ付与を行う.

当面このような基準で作業を進めることで,今後深層格の情報をタグとして付
与する必要がでてきた場合にも,述語の基本形の必須格に付与した情報
は,agent,themeのような意味役割を付与する場合や語彙概念構造 (LCS) \cite{Jackendoff:90}の意味述語の情報を付与する際にも役立つと考え
られる.

また,ゼロ照応の関係として項を付与する場合には,前方文脈に先行詞と認め
得る名詞句が複数個出現している場合がある.例えば,文
章\NUM{multi_ants}の2文目で動詞``接する''のガ格は省略されており,このゼ
ロ代名詞に対して1文目の``村山富市首相'',もしくは2文目の``首相''の両方
が補完可能である.
\EX{multi_ants}{
就任後初めて地元の大分県へ里帰りしていた\textbf{村山富市首相}$_i$は三十一日夕、三泊四日の日程を終えて日航機で羽田空港に到着した。\\
~\textbf{首相}$_i$は記者団に対し、「突然大分に帰ったが、温かい歓迎に\ul{接し}$_{ガ:i}$『地元はいいなあ』という感謝の気持ちでいっぱい。
期待に応えてしっかり頑張らないといかんという気持ちを一層強く持った」と感想を述べた。
}
このような状況の場合,ゼロ照応の正解データとしては両方の表現が先行詞と
してタグ付与されているべきであるが,すべての先行詞となり得る表現を作業
者が把握しタグ付与していくことは作業効率と作業品質の両方から見て得策と
はいえない.このような問題に対し,我々が行った作業では共参照の
関係も同時にタグ付与することで,共参照関係にある表現の集合のうちどれか
一つが先行詞としてタグ付与されるだけで,共参照関係にある他の表現にもタ
グが付与されたこと同じ結果として考えることができる.文
章\NUM{multi_ants}の例に戻ると,``村山富市首相''と``首相''が共参照の関
係であるとタグ付与されていることで,``接する''のガ格ゼロ代名詞の先行詞
はいずれかの表現にタグ付与するだけでよくなる.


最後に,\tab{comp_pred}に述語に関する我々の仕様と他のコーパスの仕様の比較をまとめる.


\begin{table}[t]
  \caption{述語と項のタグ付与の比較}
  \label{tab:comp_pred}
\input{03table01.txt}
\end{table}



\subsection{事態性名詞と項のタグ付与}
\label{ssec:spec_event}

動詞や形容詞などの述語に加え,事態性名詞に対して述語と同様に必須格とな
るガ/ヲ/ニ格を付与する.
作業者は与えられた名詞(主にサ変名詞)が事態を表しているか否かを判定し,
事態性名詞と判断した名詞(句)に対して必須格を付与する.例えば,
文\NUM{ex_event}で出現している二つの``電話''という名詞のう
ち,``電話$_i$''が「電話する」というコトを表しているのに対
し,\mbox{``電話$_j$''}は「(携帯)電話」というモノを表している.この状況で作
業者は``電話$_i$'' のみを事態性名詞と認定し,これに対し
て``\nobreak 彼\nobreak $_a$''をガ格,``私$_b$''をニ格として付与しなけ
ればならない.
\EX{ex_event}{
  \textbf{彼}$_a$からの\ul{電話}$_{i_{(ガ:a,ニ:b)}}$によると、\textbf
  {私}$_b$は彼の家に\ul{電話}$_j$を忘れたらしい。
}

また,タグ付与の対象が複合語の場合はその構成素を構成的に分解した上でそ
れぞれの構成素に対して事態性判別の作業を行う.例えば,「紛争仲裁」は構
成素``紛争''と``仲裁''のそれぞれの意味を構成的に組み合せてできた複合語
だとみなし,``仲裁''を事態性名詞と判断する.


\subsection{名詞句間の共参照関係のタグ付与}
\label{ssec:spec_coref}

共参照のタグ付与では,\sec{second}で述べたIRAに加えISAの関係も含め
てタグを付与するか否かの選択肢があるが,ISAの関係まで含めてしまうと,
総称名詞間の包含関係のような複雑な関係を考慮して作業を行う必要がある.
例えば,文章\NUM{book}では,``食べ物''と``食料''が同一の概念であるため,
共参照の関係とするか否かの判断が必要となるが,厳密にはこの二つの表現
は\mbox{``兵}庫県内で不足している食べ物''と``被災地(兵庫県)を離れた場所にあ
る食料''という異なる概念を指すためISAの関係として認定すべきではない.こ
のように,概念の同一性を判断するためにはその表現が出現する文脈との関連
性を捉え,その表現が指す範囲を考えた上で同じ概念を指しているかを考える
必要があり,共参照の認定が非常に困難な作業となる.
\EX{book}{
兵庫県内の暗やみの中で、人々が水と\ul{食べ物}の不足に苦しんでいる同じ夜、隣接した大阪の繁華街ではネオンが光り、飲食店はにぎわっている。\\
~水も\ul{食料}も、被災地を離れるとふんだんにある。
}
そこで,述語や事態性名詞がISAも含めた関係にタグ付与しているのに対
し,共参照に関してはIRAの関係にのみタグを付与する.
ただし,EDTの仕様のように,実体が組織名や場所名など数種の固有表現に限定
して共参照関係のタグを付与することは,さまざまな応用分野で必要となる共
参照の表現を網羅できないため望ましくない.そこで,今回の作業では,以下
の3つの基準に基づいて表現のクラスを限定せずに共参照関係のタグ付与を行い,
どのような問題が生じるのかを調査した.
\begin{enumerate}
\item \textbf{照応詞は文節の主辞(最右の名詞自立語)のみに限定する}.
\item \textbf{談話内に出現した名詞句のみを先行詞とする}.
\item \textbf{総称名詞は照応詞,先行詞とみなさない}.
\end{enumerate}
既存の共参照関係のタグ付与の研究と比較すると\tab{diff_coref}のようになる.

\begin{table}[t]
  \caption{共参照タグ付与の差異}
  \label{tab:diff_coref}
\input{03table02.txt}
\end{table}



\subsection{統計}

\ssec{spec_pred},\ssec{spec_event},\ssec{spec_coref}の仕様に従い,京
都コーパス3.0の全記事(2,929記事,38,384文)\footnote{タグ付与の対象と
  して本研究では京都コーパス3.0を採用しているが,その後新たに公開された
  京都コーパス4.0は京都コーパス3.0と比較して2記事を削除してあるだけであ
  り,こちらのコーパスにタグ付与された係り受け情報と照合する際に
  もNAISTコーパスの述語項構造・共参照のタグの情報は問題なく照合できると
  考えられる.}を対象に,2人の作業者が述語項構造と共参照の関係について
タグ付与作業を行った.
述語/事態性名詞とその項に付与されたタグの個数を\tab{statics}にまとめる.
ただし,項の出現位置によって,同一文節内\footnote{「明らかになる」のよ
  うな表現がコーパス中では一文節であるのに対し,作業者が「明らかに」と
  「なる」を分けて付与した場合などを含む.},係り関係にある場
合\footnote{「サンマを焼く男」の「男」が「焼く」のガ格となるような,連
  体修飾の関係も含む.},文内のゼロ照応関係,文間のゼロ照応関係,外界照
応の5つに分類して頻度を求めた\footnote{今回タグ付与される情報には係り受
  け関係は含まないが,京都コーパスの係り受け情報と統合することによりゼ
  ロ照応か否かの判別が可能である.}.
\tab{statics}より,述語の項目ではヲ格,ニ格の項のほとんどは係り関係に
あるのに対し,ガ格の約6割はゼロ照応の関係にあることがわかる.
これに対して,事態性名詞のヲ格,ニ格は同一文節内,つまり複合語の構成素
として項が出現している割合が高く,ガ格に関しては約8割がゼロ照応の関係
にあり,述語の場合と比較して項の出現箇所がおおきく異なっていることがわかる.

\begin{table}[t]
  \caption{述語と事態性名詞のタグの統計(NAISTテキストコーパス全体)}
  \label{tab:statics}
\input{03table03.txt}
\end{table}


共参照関係のタグについては,タグ付与された実体の総数が10,531,最初に出
現した表現を先行詞,その他を照応詞とみなしたときの照応詞の個数が25,357
であった.京都コーパス4.0より圧倒的に個数が少ないが,これは厳密にISAの
関係と判断できる場合のみタグ付与を行ったためだと考えられる.また,共参
照に関与する代名詞の個数は622と,京都コーパスと比較してタグ付与した記事
に対する割合が小さい.これは作業者が(1)節照応の付与を行わずに,(2)共参
照タグ付与に関して厳密な実体の一致を強いたために完全に一致するとみなせ
ない場合にはタグが付与されなかったためだと考えられる.代名詞に関しては
現状の仕様のようにIRAの関係でタグを付与する立場と,実体の一致を問題とし
ないISAの関係で付与するという立場の二つを考えることができ,どちらが良い
かは応用分野によって異なる.そのため,代名詞については追加的にISA関係の
タグを分けて付与することも今後検討したい.


次に,実際に作業を行っている2人の作業者間のタグ付与の一致率を調査する
ため,ランダムに選択した報道30記事を対象に作業を行った.評価は一方の作
業者のタグ付与の結果を正解,他方の作業結果をシステムの出力とみなし再現
率と精度で評価する.ただし,それぞれのタグの一致率は各タグの終了位置の
一致で評価した.また,述語と事態性名詞の項の一致率については,2人の作
業者の述語(事態性名詞)が一致した箇所のみを対象に評価した.また,共参
照の一致率についてはMUC score \cite{Vilain:95}を用いて再現率と精度を求めた.
これらの基準で評価した結果を\tab{agree}に示す.
\tab{agree}よりわかるように,それぞれのタグ付与は多くの場合8割を越え
る品質で作業ができているが,改善の余地は大きい.\sec{fifth}では,各タ
グ付与において,問題となった主要な点を説明し,その問題を解決するための
今後の方向性について議論する.


\begin{table}[t]
  \caption{タグの一致率(報道30記事)}
  \label{tab:agree}
\input{03table04.txt}
\end{table}



\section{タグ付与の問題点と今後の展望}
\label{sec:fifth}

この節では述語,事態性名詞,共参照のそれぞれのタグ付与作業中に生じた問
題を説明し,それに対する今後の対応などをまとめる.


\subsection{述語のタグ付与の問題点}

まず,述語そのもののタグ付与に関してだが,タグ付与対象となる述語が「〜
と\ul{し}て」のような機能語相当表現と表現上では同一の場合に述語認定に
揺れが生じることがわかった.
例えば「会社Aが会社Bを子会社と\ul{し}て」では「として」が``ある一つの側
面からの価値付け・意味付け''という意味の機能語相当表現なのか,それとも
「会社Aが会社Bを子会社と\ul{する}」と解釈すべきなのかを判断することが難
しい.
この機能語相当表現の問題については,土屋ら\cite{Tuchiya:06}がタグ付与作
業に関して作業者間の高い一致率を得ており,彼らの作業方針を参考に仕様を
洗練していく予定である.


\subsection{事態性名詞タグ付与の問題点}
\label{ssec:problem_eventnoun}

事態性名詞の認定に関して,今回は「対象となる名詞(句)が出現文脈でモノと
コトのどちらを表しているかを判断し,コトの場合のみ項を付与する」という
仕様を採用したが,タグ付与作業の結果,モノとコトの2値に分類することが困
難な事例が多数出現し,それが作業の揺れの主な原因となった.例えば,
例\NUM{report}の名詞``報告''は``文化庁ガ報告スル''というコトを表してい
ると同時に報告された結果(内容物)というモノを表していることになり,こ
の例からもわかるように,事態性名詞の中にはモノとコトのどちらとも解釈で
きるものがある.
\EX{report}{ 
\textbf{文化庁}の2005年の\ul{報告}によると、各宗教団体の
  報告による信者数は合計2億1100万人である。
}
つまり,項となり得る表現(例\NUM{report}では``文化庁'')が近傍に出現し
ているか否かがコトを指すか否かの判定におおきく影響し,上のような場合で
も項となる表現が近くに出現していない場合はモノと判断されるなど,一貫し
た作業結果が得られていない.



\subsection{項のタグ付与の問題点}

項のタグ付与に関しては述語が取り得る格パタンが複数存在するために
作業者間で揺れが生じることがわかった.この問題の典型的な例が自動詞と他
動詞の交替である.例えば,述語\mbox{``実現する''}は同じ語義に対して表層格レベ
ルで``agentガthemeヲ実現する''と``themeガ実現する''の2つの格パタンが存
在するため,文章中ですべての格要素が省略されている場合は,作業者はどち
らの解釈でもタグ付与が可能になってしまう.
自他交替の問題と類似して,制度などの表現に動作主性 (agentivity) を認め
るか否かで解釈が異なるために揺れが生じる場合もある.
例えば,文\NUM{alter}において,述語``しばる''は,直前に出現している
``規制''の動作主性を認め,``規制ガthemeヲしばる''と``agentガ規制デ
themeヲしばる''の2つの解釈が存在する.
\EX{alter}{
我々の生活が知らず知らずにどれだけ規制で\ul{しばら}れているか、規制緩
和によって豊かさが変わっていくのかを考えてみた。
}
このような交替を伴う場合の揺れに関しては,どちらかのパタンを優先すると
いう規則をあらかじめ決めておき作業することで対応できると考えられる.

動作主性の問題に関連して,組織のような実体にどのくらい動作主性を認める
かが作業者間で異なるために揺れが生じる場合も頻繁に起こった.
例えば,文\NUM{ex_role}では,組織``与野党''もしくはその組織の``党首''が
事態``協力(する)''のガ格として解釈可能である.
\EX{ex_role}{
...自民、さきがけ、新進各党の\textbf{与野党}$_a$の\textbf{党首}$_b$会
談を呼び掛けて\ul{協力}を求めるべきだ。
}
このような組織とその組織の関係者のような対立や,また文\NUM{ex_part}の
``北朝鮮''と``同指導部''のような組織とその部署の関係など,ある名詞が他
の名詞と関連しているために,複数の解釈が可能な場合は``(与野党ノ)党首''
のように詳細化されているほうにタグを付与することによって作業者間の揺れ
を少なくすることができると考えられる.このように扱うことで,もし
$\langle$所属(党首, 与野党)$\rangle$のような名詞間の関係解析が実現でき
れば,他方の名詞(\NUM{ex_role}では``与野党'')と対象となる述語を関連
付けて扱うことができる.
\EX{ex_part}{
\textbf{北朝鮮}$_a$における新年の辞は、\textbf{同指導部}$_b$の施政方針
\ul{発表}に当たる重要行事である。
}

また\fig{lack}(a)のように,項としてタグ付与されるべき名詞句がIRAの関係
で他の名詞句と関連付けられている場合は,共参照関係にある名詞句のいずれ
かを項として同定する問題とみなすことができるが,一方\fig{lack}(b)の
``子供''と``児童''のようなISAの関係で出現している名詞句については,ラ
ベルが付与されていない``子供''は述語の項としてタグが付与されないという問題
が起こる.

\begin{figure}[b]
\begin{center}
  \includegraphics[width=.7\columnwidth,keepaspectratio]{17-2ia3f3.eps}
\end{center}
  \caption{先行詞が総称名詞の場合のタグ付与の漏れ}
  \label{fig:lack}
\end{figure}


また,本研究では必須格となるガ/ヲ/ニを対象に述語項構造のタグ付与につい
て議論したが,それ以外の格(カラ/ヘ/ト/ヨリ/マデ/デ)についても付与する
ことが可能であるかを調査する必要がある.そこでコーパスの一部136記事を対
象にこれらのタグを試験的に付与し,どのような結果となるかを調査した.
ただし,項の出現箇所に制限を加えずに作業を行った場合,作業者は各述語に
対し文章全体を対象に格要素を探す必要があるため,すべての項に網羅的にタ
グ付与できるかどうかがわからない.そのため,今回は項を同定する範囲を述
語と同一文内に限定し,その中で網羅的に項のタグ付与を行った.
\tab{adjunct}に作業者2人が付与したタグの個数を表層格ごとにまとめる.
\tab{adjunct}より,ガ/ヲ/ニ以外の項についてもある程度の個数が付与可能
なように見えるが,このうち文~ \NUM{adjunct1}や文~\NUM{adjunct2}のよう
な,複数の述語が同一表現を項として持つ並列や文内ゼロ照応など,明示的に
タグを付与すべき現象がどのくらい出現しているかを人手で調査したところ,
作業者1と2でそれぞれ16回と31回であった.つまり,項のタグ付与の対象を同
一文内に限定した場合,ほとんどの項は係り受け関係にあり,かつ明示的にタ
グ付与対象となる格助詞を伴い出現するため,今回人手でタグ付与作業を行っ
た結果のほとんどは機械的に処理できる問題となる.
\EX{adjunct1}{
\textbf{台北}$_i$では、屋外のスタジアムも満員に\ul{なり}$_{デ:i}$、失神者が\ul{出た}$_{デ:i}$ほど。
}
\EX{adjunct2}{ 
...「新民主連合」は六、九の両日に\textbf{総会}$_i$を\ul{開き}$_{ヲ:i}$、
離党問題などの対応を\ul{話し合う}$_{デ:i}$ことにしており、党内調整は大き
なヤマ場を迎える。
}
このため,文を越えて述語と任意格の関係を付与することを考慮する必要があ
るが,どのような基準でその作業に取り組めばよいのかは自明ではなく,今後
さらに検討する必要があると考えている.

\begin{table}[t]
  \caption{ガ/ヲ/ニ格以外のタグ付与結果(新聞136記事)}
  \label{tab:adjunct}
\input{03table05.txt}
\end{table}



\subsection{共参照タグ付与の問題点}
\label{ssec:problem_coref}


共参照関係のタグ付与に関して作業を行った結果,共参照関係を厳密な同一実
体を参照している場合に限定したことで,多くの関係は固有名の間で認定され,
照応の現象と関連する代名詞や指示連体詞を伴う名詞句のような表現について
はほとんどには共参照の関係が付与されなかった.これは,新聞記事の場合は
読み手が理解できる箇所では代名詞のような表現よりゼロ代名詞を優先的に利
用している,また,代名詞が名詞句に加え,直前の節を指す場合などが比較的
多く,仕様で定義した共参照関係として認定できない,などの理由がある.た
だし,代名詞,指示連体詞などの指示表現については照応関係を研究するため
の良い題材になると考えられるので,これらの指示表現に特化したタグ付与作
業を行った.作業結果については\ssec{revise_coref}でまとめる.


また,IRAのみを対象に共参照のタグを付与する作業に関してもいくつかの問題
が残る.まず1つ目の問題を文章\NUM{gpe_ex}を例に説明しよう.
\EX{gpe_ex}{
グロズヌイからの報道によると三日、大統領官邸の北西一・五キロの鉄道駅付
近で\ul{ロシア軍部隊}$_i$とチェチェン側部隊が衝突したが、\ul{ロシア側}
$_i$は中心部への進撃を阻まれて苦戦。...  \ul{ロシア政府}$_j$は三日、戦
況に関する声明を発表し、大統領官邸を含む首都中心部は依然としてロシア側
が支配していると強調した。しかし現地からのテレビ映像では、官邸はじめ中
心部は依然としてドゥダエフ政権部隊の兵士が警戒に当たっており、\ul{ロシ
ア側}$_j$の発表と食い違いを見せている。
}
この例で,最初に出現する``ロシア側$_i$''が``ロシア軍部隊$_i$''の換喩に
相当するのに対し,次に出現する``ロシア側$_j$''は``ロシア政府$_j$''の換
喩として解釈できる.現状の仕様では``ロシア軍部隊$_i$''と``ロシア側
$_i$'',``ロシア政府$_j$''と``ロシア側$_j$''それぞれに共参照関係のタグ
を付与することになるが,換喩の解釈しながらタグ付与作業を行なうことが困
難であることに加え,実際の自動解析する際にも非常に困難な問題設定となる.
この問題を回避するために,換喩の解釈で共参照のタグを付与するのではなく,
文章に出現している4つの``ロシア''を同一の実体としてタグ付与するなどの
方法が考えられるが,どのように仕様を決めた方がよいかは明らかでないため
今後検討していく必要がある.


また,IRAの認定に関しても,実体が具体名詞である場合は2つの言及が同一の
実体を指すか否かの認定が容易であるが,抽象名詞の場合は同じものを指して
いるかの判定が困難である.\ssec{spec_coref}で共参照関係のタグ付与には
あらかじめ名詞のクラスを指定して作業を行うことは望ましくないと述べたが,
抽象名詞に関してはいくつかの意味クラスに限定して作業を行い,どのくらい
揺れなく作業できるかを調査したい.



\section{タグの仕様の改善}
\label{sec:sixth}

\sec{fifth}で見たように,今回採用したタグ付与の基準には解決しなければな
らないいくつかの問題点が含まれている.この節ではその中で事態性名詞と名
詞句の照応関係についてさらに仕様を洗練し,作業を行った結果について報告する.


\subsection{事態性名詞}

\ssec{problem_eventnoun}に示したように,事態性名詞に関する典型的な作業
の揺れは,事態性名詞が文脈によってモノとコトの両方の解釈がある場合に
「あらかじめ明示的にコトかモノかを判別し,コトへのみ項を付与する」とい
う仕様と矛盾するために起こる.この矛盾を回避するために,以下に示すの2つ
の項目をタグ付与の仕様として採用した.

\paragraph{修正点1: モノを指す表現へも項を付与する}
~ モノとコトの境界を項付与できるか否かで弁別することは困難であり,
モノとして解釈できる場合にも項を持つ場合がある.そこで,今回の仕様では
モノである場合でも項を持つと判断できた場合には,モノ/コトの判別とは独立
に項を付与する.

\paragraph{修正点2: モノとコトを指す表現を区別するためモノと判断した根拠
  もタグ付与する}
~ 提案1で述べた仕様を採用すると,モノの場合も項を付与するため項を付与し
たことがコトを指すという情報と等価ではなくなる.しかし,文章中の事態の
みを抽出したい応用分野も存在するため,作業結果には事態性名詞がコトを指
すという情報もできる限り残しておくことが望ましい.
そこで,まず我々はあらかじめ揺れが起こった事態性名詞を人手分析し,モノ
とコトの解釈で曖昧性が生じる名詞を
\textbf{結果物/内容\nobreak},\textbf{モノ(具体
  物)},\textbf{役割},\textbf{述語と事態性名詞との語義のずれ}の
4種のクラスに分類した.事態性名詞をモノとして解釈できる場合には,これ
ら4つのうちいずれかをタグ付与することでモノとしての証拠を残す.逆にこ
れらのタグが付与されないことで純粋に事態を表す事態性名詞を表現する.
また,名詞クラスのタグを用意することで,項付与が困難な場合に作業者が無
理に項を付与しようとする事態を回避することができる.

上述の2つの提案を採用することにより,これまでモノに分類するか項を取る
かどちらか一方の情報しか付与できなかった例\NUM{report}の``報告''につい
ても,``\nobreak モノ(具体物)''としての解釈と``文化庁ガ報告スル''と
いう項構造の両方を情報を付与できるようになる.
以下で,今回付与する4種の名詞クラスについて説明する.

\begin{itemize}

\item \textbf{結果物/内容}: 
典型的には例\NUM{content}のような内容節をとる(トノ,トイウを伴って出現
する)場合,``意見''のような名詞は内容節が表す内容と同格であり,``意見
スル''というコトを指すとは考えにくい\footnote{もちろん``意見''という表
  現だからといって必ず\brace{結果物/内容}タグを付与するわけではなく,文
  脈から``XガYト意見スル''という事態と判断できる場合は項を付与
  し,\brace{結果物/内容}タグはしない.}.
\EX{content}{
  党内には「社会党会派の離脱者は従来通り除名すべきだ」との\ul{意見}が
  根強く...
}
この類例としては``提案'',``決定'',``報告''などがある.


また,``連合シタ''結果,``連合''という実体が存在するという解釈に基づき,
例\NUM{union}のような実体を指すのみで事態を表すとは考えにくい``連合''
についても\brace{結果物/内容}のタグを付与し,項は付与しない.この類例
としては``組織''などの表現がある.
\EX{union}{
  十日夜には、自由\ul{連合}の新年会に自民党から森喜朗幹事長、島村宜伸国
  対委員長らが出席した。
}

同様に,例\NUM{regulation}の``規制''も``規制スル''事態よりも規制の内容
そのものを指すと判断した場合は,項は付与せず\brace{結果物/内容}タグの
みを付与する.
\EX{regulation}{
  また、経済問題については日本経済の構造変革のため\ul{規制}緩和に積極
  的に取り組むと訴える。
}

\item \textbf{モノ(具体物)}:
事態性名詞が文脈中でモノ(具体物)を指しているかを判定する.前述の例
\NUM{ex_event}のモノとしての``電話$_j$''や場所としての``施設'',道具と
しての``装備''などの表現がこれに相当する.
例えば,ある文脈で``携帯''という表現が``携帯電話''というモノを指す場合
は\brace {\nobreak モノ}タグを付与する.

\item \textbf{役割}:
``課長\ul{補佐}'',``松本\ul{教授}'',``オシム\ul{監督}''などの表現は,
名詞句全体で個人を指示しており,名詞句内の事態性名詞がコトを指すとは考
えにくく,この場合には\brace{\nobreak 役割}タグを付与することで項の付与を回避
する.このような表現は典型的に主辞の位置に出現している場合が多い.

\item \textbf{述語と事態性名詞との語義のずれ}:
事態性名詞が派生前の動詞の意味と異なる場合には,項を付与することができ
ない.例えば,例\NUM{zure}のサ変名詞``\nobreak 一定''は動詞``一定スル''
と異なった意味で用いられており,このような場合には\brace{ずれ}タグを付
与し,項は付与しない.
\EX{zure}{
  \ul{一定}の得票で議席を占めた後に今回と同様「除名」などの騒動が起きれば、...
}

\end{itemize}


上述の作業方法を採用することで人手でのタグ付与品質にどのような影響が出
るかの調査を行った.具体的には,作業者2人が新聞報道50 記事中のサ変名
詞665箇所に対し,その名詞が項を持つか否かの判定と項を持つ場合は項の付与
を行った.この作業とは独立に\sec{third}に示した名詞クラスの付与を行っ
た.
今回の作業では頻出するサ変名詞を対象に作業し,和語動詞派生の名詞は対象
外とした.作業者2名の作業結果とその一致率を\tab{result}に示す.
\tab{result}より,665件のサ変名詞のうちどちらの作業者も550を越えるサ変
名詞に対して項を持つと判定しており,文章中のほとんどのサ変名詞は項付与
対象となっていることがわかる.また,項を持つか否かの作業者間の一致率は
それぞれの作業者について見た場合0.95と0.91と以前の作業品質の調査
\cite{Iida:07}
(一致率は0.905と0.810)と比較して一致率が向上しており,
今回の作業方針が品質向上に有効であったことがわかる.また,項を持つか否
かKappa値で評価したところ0.522という結果を得た.
名詞クラスの一致率については良いとは言い難いが,これは作業者間で
\brace{結果物/内容}と\brace{ずれ}にそれぞれ付与するなど,クラス間の揺
れが生じたためであり,またそもそもスル接続で表現する頻度が低い``確証''
などの事態性名詞に関する解釈の異なりも作業の揺れの原因となった.
また,項を取るか否かのタグ付与が不一致だった78事例を調べたところ,44事
例は項を付与するか否かに作業者間で解釈が異なる事例であり,残りは付与の
誤りとみなせる事例であった.

\begin{table}[b]
  \caption{名詞クラスのタグ付与の作業結果(報道50記事,サ変名詞665箇所)}
  \label{tab:result}
\input{03table06.txt}
\end{table}


次に,2人の作業者が項を持つと判断した531事例について,項(ガ/ヲ/ニ格)
がどのくらい一致するかを評価した結果を\tab{agree_arg}に示す.
項を取ると判断された事態性名詞のうち付与された項が一致しなかった265事
例を人手で分析,揺れの原因を調査した結果を\tab{inconsistence}にまとめ
る\footnote{1事例を複数の誤りの原因に割り割り振ったため,合計は265事例
  より多くなる.}.
作業の揺れはおおきく2つの問題に起因している.一つは,各事態性名詞を述
語化して考えた際に作業者間で異なった格パタンを想起したためである.特に,
ある事態性名詞に対して,一方の作業者は必須格としてヲ格を取ると判断した
が,他方はそれを取らないと判断した場合が揺れの大部分を占めていることが
わかった.
この問題に関しては,語彙概念構造\cite{Jackendoff:90}を考慮して作成され
ている動詞辞書\cite{Takeuchi:06}のような情報を作業の際に提示することで,
作業者が想起できない格パタンを網羅的に把握することができ,揺れが少なく
なると考えられる.このような作業者支援については,野口らの作成してい
るアノーテションツール\cite{Noguchi:08}でどのように情報を提示するかという問題と同時に考
えていきたい.

\begin{table}[b] 
  \caption{事態性名詞の項タグ付与の一致率(サ変名詞531箇所)}
  \label{tab:agree_arg}
\input{03table07.txt}
\end{table}
\begin{table}[b]
  \caption{タグ付与不一致の原因分析の結果(サ変名詞265箇所)}
  \label{tab:inconsistence}
\input{03table08.txt}
\end{table}


また,もう一つの主要な揺れの原因は,同定すべき項の粒度に関するものであ
る.例えば,例\NUM{exo}で項を持つと判断された``整備''には,前方文脈に
出現している``\nobreak 日本鉄道建設公団''という組織が``整備スル''とい
う解釈と,文章中に出現しない``特定の誰か(もしくは集団)''が``整備スル''
という2つの解釈が存在する.
\EX{exo}{
  \textbf{日本鉄道建設公団}は十一日、整備新幹線の北海道新幹線について、
  ルート公表に向けた函館市と小樽市付近の調査に一月下旬から着手すると発表した。\\
  調査は、\ul{整備}新幹線建設費とは別枠の、建設推進準備事業費三十億円
  の中で行われる。
}
この問題は「できるだけ文章内から項を選択する」という基準を用いた場合でも,
作業結果は作業者の解釈に左右されるため,解決はできず,述語の場合も同様に
問題となる.
この問題と関連して,これまでのタグ付きコーパス構築の方法論はできるだけ
揺れを無くすことが前提であり,解析はその厳密な設定のもと問題を解くとい
う立場で研究が進められてきたが,今後はその代替案として作業者一人もしく
は複数人の揺れを許容するような学習・分類の枠組みを検討すべきかもしれな
い.


\subsection{名詞句の照応関係}
\label{ssec:revise_coref}

\ssec{problem_coref}で示したように,\ssec{spec_coref}で示した名詞句の共
参照関係の仕様に従って作業を進めた場合,厳密な共参照関係であることを作
業者に強いてタグを付与させるため,例えば表現が代名詞であっても,照応関
係が付与されないことになる.どの粒度での照応関係が応用処理に必要となる
かは応用処理それぞれの問題に依存するため,同一の実体を指していることが
保証できない場合でも,照応関係を認定してタグを付与したい.しかし,この
対象を名詞句全体に広げた場合,\ssec{spec_coref}で述べた概念間の包含関係
など,複雑な関係を把握した上で照応関係を付与するという作業を作業者に強
いることになる.ここではそのような照応関係を捉える第一歩として,``こ
の''や``その''といった指示連体詞を伴う名詞句を作業対象とすることで,照
応関係についてどのような作業を進めていくべきかを考える.以下で示すよう
に,指示連体詞を伴う名詞句を対象にする場合,共参照の関係を含
む\textbf{指定指示(限定指示)}やbridging reference \cite{Clark:77}など
の間接照応の関係に相当する\textbf{代行指示}といった複数のの関係を考慮す
る必要があり,名詞句全体の照応現象を考える上での良い縮図となっていると
考えられる.

\paragraph{指定指示(限定指示)}: ``指示連体詞+\brace{名詞(句)}''が文章中の他の
表現と照応関係になる場合にその関係を指定指示(限定指示)という.例えば,
例\NUM{direct_ana}において,``このデータ$_i$''は先行詞``資料$_i$''を指す.
\EX{direct_ana}{ 
図書館で\ul{資料}$_i$を手に入れた。\ul{このデータ}$_i$ は機械的に処理される。
}

\paragraph{代行指示}: 指示連体詞単体が前方文脈の表現と照応関係になる場
合にその関係を{代行指示}という.例えば,例\NUM{indirect_ana}におい
て,``こ(の)''は前文の``水質調査''と照応関係にある.
\EX{indirect_ana}{ 
5年間、\ul{水質調査}$_i$を行った。\ul{この}$_i$データは機械的に処理される。
}

この2種類の関係を分けて付与するため,指定指示の場合は``指示連体
詞+\brace{名詞(句)}''全体を照応詞とし,代行指示の場合は指示連体詞のみに
照応詞のタグを付与する.また作業の確認のため,各に指示代名詞にはどちら
の関係で出現しているかを明示的にタグ付与を行う.また,``この日''などの
表現は文章中に必ずしも先行詞が存在するとは限らず,このような場合に
は\textbf{外界照応}のタグを付与し,指定指示や代行指示と区別する.この結
果,指示連体詞には指定指示,代行指示,外界照応のいずれかのタグが付与さ
れることになる.
共参照関係を付与した際には,談話要素間の厳密な共参照関係を強いたため先
行詞は名詞(句)に限られたが,指示連体詞を伴う場合には,先行詞の品詞に
は制約を加えずに作業を行った.このため,照応詞の指し先が節となる場合も
含まれる.このような場合は節の主辞(多くの場合は述語)をその節を代表し
て先行詞としてタグ付与した.例えば,例\NUM{ant_pred}では,``その前計
算''は``前述のシステムは前もって値を計算する''ことを指しており,この場
合はこの節を代表して,``計算する''を先行詞としてタグ付与する.
\EX{ant_pred}{ 
システムは前もって値を\ul{計算する}$_i$。\ul{その前計算}$_i$はシステム
の性能を大幅に向上させている.
}
また,先行詞の表現によっては指示関係が指定指示なのか代行指示なのか曖昧
な例が存在する.例えば,例\NUM{rel_ambiguous}では,``その土地''が``アメ
リカ''を指す指定指示の関係なのか,``その''が``アメリカ''を指し,``アメ
リカの土地''として解釈すべきかの判断が困難である.
\EX{rel_ambiguous}{ 
彼は\ul{アメリカ}$_{ij}$へ向かった。\ul{\mbox{その$_{i}$土地}}$_{j}$で彼は新しい仕事をみつけるつもりだ。
} 
このような曖昧性のある指示関係については指示関係が曖昧であることをタグ付与し,明かにわかる指定指示や代行指示の関係とは区別する.

上述の作業内容にしたがい,一人の作業者が指示関係とその先行詞のタグ付与
作業を行った.作業対象はすでにタグ付与を行ったNAISTテキストコーパスの中
から指示連体詞を含む記事をあらかじめ抽出し,その記事を対象に作業を行う.
茶筌\footnote{http://chasen.naist.jp/hiki/ChaSen/}で形態素解析し
た結果を利用し,品詞が``連体詞''として解析された形態素を含む記事1,463記
事(報道883記事,社説580記事)を対象に作業を行った.この結果4,089の指示
連体詞にタグが付与され,このうち指定指示は30.9\%(1,264/4,089)代行指示
は57.4\%(2,345/4,089),外界照応は11.5\%(470/4,089)であっ
た.\tab{ana_table}に記事の種類や先行詞の種類ごとの出現数をまとめる.

\begin{table}[b]
  \caption{照応関係のタグ付与の統計(新聞1,463記事,4,089の指示連体詞)}
  \label{tab:ana_table}
\input{03table09.txt}
\end{table}

次に,作業の信頼度を調査するために,別の作業者がすでにタグ付与した記事
の一部(418の指示連体詞)を対象に新たにタグ付与作業を行い,作業の一致率
を見る.まず,指示関係の判別についてKappa値で評価したところ0.73と高い数
値を得た.また,先行詞の一致率については,評価に使った418の指示連体詞の
うち,先行詞を持つ322の指示連体詞についてどの程度同じ先行詞にタグ付与さ
れたかの一致率を見た.その結果,指定指示については80.7\% (88/109)が一致
するという高い一致率が見られたが,代行指示については,62.9\%
(134/213)と指定指示に比べ低い一致率となった.これは,指定指示についての
作業は照応詞に対し先行詞が同じ意味カテゴリに入る候補のみを探す比較的容
易な作業であるのに対し,代行指示に関してはさまざまな意味的な関係を考慮
して先行詞を探す必要があり,作業がより困難であることに起因すると考えら
れる.この品質の向上のためにある意味カテゴリを先行詞として持つ可能性の
ある表現については,あらかじめその情報を提示した状況で作業を行うなどが
考えられる.その点を含め,どのような情報をどのような状況に提示するかに
ついては今後さらに検討していく予定である.



\section{おわりに}
\label{sec:seventh}

本稿では,日本語を対象とした述語項構造・共参照タグ付与コーパスに関して,
我々が今回採用したタグ付与の基準について報告した.\sec{third}の議論に
基づき,述語項構造のタグに関してはISAとIRAの関係両方で,共参照関係は
IRAの関係でタグ付与作業を行い,京都コーパス3.0を対象にこれまでにない大
規模な述語項構造・共参照タグ付きコーパスを作成した.
また,特に一致率の悪かった事態性名詞のタグ付与に着目し,作業仕様の洗練
を行った.具体的にはモノのタグ付与と項付与を独立に扱うことで,作業品質が
向上するという結果を得た.
さらに,その他のタグ付与作業に関しても,作業の過程で起こった問題につい
て考察し,作業の詳細化のための項目を述べた.

今回作業では述語と事態性名詞の表層ガ/ヲ/ニ格と共参照関係のタグ付与を行っ
たが,情報抽出などの応用分野を想定した場合,今回作業したラベルに加え,
以下に示す内容に取り組む必要があると考えている.

まず,今回の作業では名詞間の関係として共参照関係のみを作業対象としたが,
上位下位や部分全体,所属関係など,さまざまな関係の解析も述語項構造・
共参照解析と同様に応用処理のための重要な構成素となる.
この名詞間の関係について,京都コーパス4.0で採用されている``AノB''の粒
度でタグを付与した場合,この``ノ''で付与した結果には上位下位関係や部分
全体関係などさまざまな関係を含んでしまうため,関係抽出の粒度としては不十分
である.
また,この名詞句間の関係解析は,bridging reference \cite{Clark:77}や間
接照応\cite{Yamanashi:92}などの用語で表現される場合もあるが,bridging
referenceは一般に英語の定情報 (definite) の存在が仮定された上で述べら
れることが多い.つまり,``the''を伴った名詞句があるにもかかわらず,参
照する先行詞が文章中に出現していない場合にどう解釈すればよいかという点が
議論の中心となっている.
一方,日本語などの冠詞のが利用できない言語の場合,``the''のような手が
かりがないために,どの名詞句の対に対して間接照応の関係を付与するかとい
う課題設計そのものが困難になると考えられる.ACEのRelation Detection
and Characterization (RDC)タスクでは,\ssec{pre_coref}で述べた実体の間
の関係にのみ抽出対象となる関係を定義しているが,実体のクラスをオープン
にした場合に揺れなく作業できるかについても今後調査したい.


さらに,今回の作業では新聞記事を対象に作業を行ったが,例えば代名詞の出
現が少ないなど,このコーパス内の用例だけを学習手法の訓練事例として利用
すると,blogなどの照応解析,述語項構造解析を適用したい記事との異なりの
ために適切に解析できない恐れがあり,今後はタグ付与作業をいくつかの領域
に拡張して進める必要がある.

また,タグ付与に関する仕様書に関して,それぞれ個別の仕様について,外延
的に例を示すだけで仕様をまとめるのではなく,それぞれのタグがどのような
性質を持っているために付与されているかという内包的な仕様も明示的に記述
することで,実際に解析に利用した研究者が問題の性質を分析するのに役立つ
仕様書を作成することが重要だと考えており,今後の作業内容については順次
Webページ\footnote{
    http://cl.naist.jp/\~{}ryu-i/coreference\_tag.html}にまとめていく
予定である.




\acknowledgment

本研究は科研費特定領域研究「代表制を有する大規模日本語書き言葉コーパス
の構築」,ツール班「書き言葉コーパスの自動アノテーションの研究」(研究
代表者: 松本裕治)の支援を受けた.記して謝意を表する.


\bibliographystyle{jnlpbbl_1.4}
\begin{thebibliography}{}

\bibitem[\protect\BCAY{浅原\JBA 松本}{浅原\JBA 松本}{2003}]{Asahara:03}
浅原正幸\JBA 松本裕治 \BBOP 2003\BBCP.
\newblock ipadic version 2.6.3 ユーザーズマニュアル.

\bibitem[\protect\BCAY{Clark}{Clark}{1977}]{Clark:77}
Clark, H.~H. \BBOP 1977\BBCP.
\newblock \BBOQ Bridging.\BBCQ\
\newblock In Johnson-Laird, P.~N.\BBACOMMA\ \BBA\ Wason, P.\BEDS, {\Bem
  Thinking: Readings in Cognitive Science}. Cambridge University Press.

\bibitem[\protect\BCAY{Doddington, Mitchell, Przybocki, Ramshaw, Strassel,
  \BBA\ Weischedel}{Doddington et~al.}{2004}]{Doddington:04}
Doddington, G., Mitchell, A., Przybocki, M., Ramshaw, L., Strassel, S., \BBA\
  Weischedel, R. \BBOP 2004\BBCP.
\newblock \BBOQ Automatic Content Extraction (ACE) program---task definitions
  and performance measures.\BBCQ\
\newblock In {\Bem Proceedings of the 4rd International Conference on Language
  Resources and Evaluation {\rm (}LREC-2004{\rm )}}, \mbox{\BPGS\ 837--840}.

\bibitem[\protect\BCAY{Hasida}{Hasida}{2005}]{Hasida:05}
Hasida, K. \BBOP 2005\BBCP.
\newblock GDA日本語アノテーションマニュアル 草稿 第 0.74 版.\
  http://i-content.org/gda/tagman.html.

\bibitem[\protect\BCAY{Hirschman}{Hirschman}{1997}]{Hirschman:97}
Hirschman, L. \BBOP 1997\BBCP.
\newblock \BBOQ \textit{MUC-7 coreference task definition}. {\rm Version
  3.0}.\BBCQ.

\bibitem[\protect\BCAY{飯田\JBA 小町\JBA 乾\JBA 松本}{飯田 \Jetal
  }{2007}]{Iida:07}
飯田龍\JBA 小町守\JBA 乾健太郎\JBA 松本裕治 \BBOP 2007\BBCP.
\newblock NAISTテキストコーパス: 述語項構造と共参照関係のアノテーション.\
\newblock \Jem{情報処理学会研究報告(自然言語処理研究会) NL-177-10},
  \mbox{\BPGS\ 71--78}.

\bibitem[\protect\BCAY{Jackendoff}{Jackendoff}{1990}]{Jackendoff:90}
Jackendoff, R. \BBOP 1990\BBCP.
\newblock {\Bem Semantic Structures}.
\newblock Current Studies in Linguistics 18. The MIT Press.

\bibitem[\protect\BCAY{Jiang \BBA\ Ng}{Jiang \BBA\ Ng}{2006}]{Jiang:06}
Jiang, Z.~P.\BBACOMMA\ \BBA\ Ng, H.~T. \BBOP 2006\BBCP.
\newblock \BBOQ Semantic Role Labeling of NomBank: A Maximum Entropy
  Approach.\BBCQ\
\newblock In {\Bem Proceedings of the 2006 Conference on Empirical Methods in
  Natural Language Processing (EMNLP 2006)}, \mbox{\BPGS\ 138--145}.

\bibitem[\protect\BCAY{河原\JBA 黒橋\JBA 橋田}{河原 \Jetal
  }{2002}]{Kawahara:02}
河原大輔\JBA 黒橋禎夫\JBA 橋田浩一 \BBOP 2002\BBCP.
\newblock 「関係」タグ付きコーパスの作成.\
\newblock \Jem{言語処理学会第8回年次大会発表論文集}, \mbox{\BPGS\ 495--498}.

\bibitem[\protect\BCAY{Kingsbury \BBA\ Palmer}{Kingsbury \BBA\
  Palmer}{2002}]{Kingsbury:02}
Kingsbury, P.\BBACOMMA\ \BBA\ Palmer, M. \BBOP 2002\BBCP.
\newblock \BBOQ From TreeBank to PropBank.\BBCQ\
\newblock In {\Bem Proceedings of the 3rd International Conference on Language
  Resources and Evaluation (LREC-2002)}, \mbox{\BPGS\ 1989--1993}.

\bibitem[\protect\BCAY{Komachi, Iida, Inui, \BBA\ Matsumoto}{Komachi
  et~al.}{2007}]{Komachi:07}
Komachi, M., Iida, R., Inui, K., \BBA\ Matsumoto, Y. \BBOP 2007\BBCP.
\newblock \BBOQ Learning Based Argument Structure Analysis of Event-nouns in
  Japanese.\BBCQ\
\newblock In {\Bem Proceedings of the Conference of the Pacific Association for
  Computational Linguistics (PACLING)}, \mbox{\BPGS\ 120--128}.

\bibitem[\protect\BCAY{Liu \BBA\ Ng}{Liu \BBA\ Ng}{2007}]{Liu:07}
Liu, C.\BBACOMMA\ \BBA\ Ng, H.~T. \BBOP 2007\BBCP.
\newblock \BBOQ Learning Predictive Structures for Semantic Role Labeling of
  NomBank.\BBCQ\
\newblock In {\Bem Proceedings of the 45th Annual Meeting of the Association of
  Computational Linguistics}, \mbox{\BPGS\ 208--215}.

\bibitem[\protect\BCAY{Marcus, Santorini, \BBA\ Marcinkiewicz}{Marcus
  et~al.}{1993}]{Marcus:93}
Marcus, M.~P., Santorini, B., \BBA\ Marcinkiewicz, M.~A. \BBOP 1993\BBCP.
\newblock \BBOQ Building a Large Annotated Corpus of English:The Penn
  Treebank.\BBCQ\
\newblock In {\Bem Computational Linguistics}, \mbox{\BPGS\ 313--330}.

\bibitem[\protect\BCAY{Meyers, Reeves, Macleod, Szekely, Zielinska, Young,
  \BBA\ Grishman}{Meyers et~al.}{2004}]{Meyers:04}
Meyers, A., Reeves, R., Macleod, C., Szekely, R., Zielinska, V., Young, B.,
  \BBA\ Grishman, R. \BBOP 2004\BBCP.
\newblock \BBOQ The NomBank Project: An InterimReport.\BBCQ\
\newblock In {\Bem Proceedings of the HLT-NAACL Workshop on Frontiers in Corpus
  Annotation}.

\bibitem[\protect\BCAY{Mitkov}{Mitkov}{2002}]{Mitkov:02}
Mitkov, R.\BED\ \BBOP 2002\BBCP.
\newblock {\Bem Anaphora Resolution}.
\newblock Studies in Language and Linguistics. Pearson Education.

\bibitem[\protect\BCAY{Ng \BBA\ Cardie}{Ng \BBA\ Cardie}{2002a}]{Ng:02}
Ng, V.\BBACOMMA\ \BBA\ Cardie, C. \BBOP 2002a\BBCP.
\newblock \BBOQ Improving Machine Learning Approaches to Coreference
  Resolution.\BBCQ\
\newblock In {\Bem Proceedings of the 40th ACL}, \mbox{\BPGS\ 104--111}.

\bibitem[\protect\BCAY{野口\JBA 三好\JBA 徳永\JBA 飯田\JBA 小町\JBA 乾}{野口
  \Jetal }{2008}]{Noguchi:08}
野口正樹\JBA 三好健太\JBA 徳永健伸\JBA 飯田龍\JBA 小町守\JBA 乾健太郎 \BBOP
  2008\BBCP.
\newblock 汎用アノテーションツールSLAT.\
\newblock \Jem{言語処理学会第14回年次大会発表論文集}.

\bibitem[\protect\BCAY{Palmer, Gildea, \BBA\ Kingsbury}{Palmer
  et~al.}{2005}]{Palmer:05}
Palmer, M., Gildea, D., \BBA\ Kingsbury, P. \BBOP 2005\BBCP.
\newblock \BBOQ The Proposition Bank: An Annotated Corpus of Semantic
  Roles.\BBCQ\
\newblock {\Bem Computational Linguistics}, {\Bbf 31}  (1), \mbox{\BPGS\
  71--106}.

\bibitem[\protect\BCAY{Poesio}{Poesio}{2004}]{Poesio:04}
Poesio, M. \BBOP 2004\BBCP.
\newblock \BBOQ Discourse Annotation and Semantic Annotation in the GNOME
  Corpus.\BBCQ\
\newblock In {\Bem Proceedings of the ACL 2004 Workshop on Discourse
  Annotation}, \mbox{\BPGS\ 72--79}.

\bibitem[\protect\BCAY{Soon, Ng, \BBA\ Lim}{Soon et~al.}{2001}]{Soon:01}
Soon, W.~M., Ng, H.~T., \BBA\ Lim, D. C.~Y. \BBOP 2001\BBCP.
\newblock \BBOQ A Machine Learning Approach to Coreference Resolution of Noun
  Phrases.\BBCQ\
\newblock {\Bem Computational Linguistics}, {\Bbf 27}  (4), \mbox{\BPGS\
  521--544}.

\bibitem[\protect\BCAY{竹内\JBA 乾\JBA 藤田}{竹内 \Jetal }{2006}]{Takeuchi:06}
竹内孔一\JBA 乾健太郎\JBA 藤田篤 \BBOP 2006\BBCP.
\newblock 語彙概念構造に基づく日本語動詞の統語・意味特性の記述.\
\newblock 影山太郎\JED, \Jem{レキシコンフォーラム}, 2\JNUM, \mbox{\BPGS\
  85--120}. ひつじ書房.

\bibitem[\protect\BCAY{土屋\JBA 宇津呂\JBA 松吉\JBA 佐藤\JBA 中川}{土屋 \Jetal
  }{2006}]{Tuchiya:06}
土屋雅稔\JBA 宇津呂武仁\JBA 松吉俊\JBA 佐藤理史\JBA 中川聖一 \BBOP 2006\BBCP.
\newblock 日本語複合辞用例データベースの作成と分析.\
\newblock \Jem{情報処理学会論文誌}, 47\JVOL, \mbox{\BPGS\ 1728--1741}.

\bibitem[\protect\BCAY{van Deemter \BBA\ Kibble}{van Deemter \BBA\
  Kibble}{1999}]{Deemter:99}
van Deemter, K.\BBACOMMA\ \BBA\ Kibble, R. \BBOP 1999\BBCP.
\newblock \BBOQ What is coreference, and what should coreference annotation
  be?\BBCQ\
\newblock In {\Bem Proceedings of the ACL '99 Workshop on Coreference and its
  applications}, \mbox{\BPGS\ 90--96}.

\bibitem[\protect\BCAY{Vilain, Burger, Aberdeen, Connolly, \BBA\
  Hirschman}{Vilain et~al.}{1995}]{Vilain:95}
Vilain, M., Burger, J., Aberdeen, J., Connolly, D., \BBA\ Hirschman, L. \BBOP
  1995\BBCP.
\newblock \BBOQ A Model-Theoretic Coreference Scoring Scheme.\BBCQ\
\newblock In {\Bem Proceedings of the 6th Message Understanding Conference
  (MUC-6)}, \mbox{\BPGS\ 45--52}.

\bibitem[\protect\BCAY{山梨}{山梨}{1992}]{Yamanashi:92}
山梨正明 \BBOP 1992\BBCP.
\newblock \Jem{推論と照応}.
\newblock くろしお出版.

\end{thebibliography}

\begin{biography}
\bioauthor{飯田  龍}{
1980年生.
2007年奈良先端科学技術大学院大学情報科学研究科博士後期課程終了.
同年より奈良先端科学技術大学院大学情報科学研究科特任助教.
2008年12月より東京工業大学大学院情報理工学研究科助教.
現在に至る.博士(工学).自然言語処理の研究に従事.
情報処理学会員.
}
\bioauthor{小町  守}{
2005年東京大学教養学部基礎科学科科学史・科学哲学分科卒.
2007年奈良先端科学技術大学院大学情報科学研究科博士前期課程修了.
同年同大学博士後期課程に進学.修士(工学).
日本学術振興会特別研究員.
大規模なコーパスを用いた意味解析に関心がある.
言語処理学会第14回年次大会最優秀発表賞,
情報処理学会第191回自然言語処理研究会学生奨励賞受賞.
ACL・人工知能学会・情報処理学会各会員.
}
\bioauthor{井之上直也}{
1985年生.2008年武蔵大学経済学部経済学科卒業.同年より奈良
先端科学技術大学院大学情報科学研究科博士前期課程.現在に至る.
自然言語処理の研究に従事.
}
\bioauthor{乾 健太郎}{
1967年生.
1995年東京工業大学大学院情報理工学研究科博士課程修了.
同年より同研究科助手.
1998年より九州工業大学情報工学部助教授.
1998年〜2001年科学技術振興事業団さきがけ研究21研究員を兼任.
2001年より奈良先端科学技術大学院大学情報科学研究科准教授.
現在に至る.
博士(工学).
自然言語処理の研究に従事.
情報処理学会,人工知能学会,電子情報通信学会,ソフト科学会各会員.
}
\bioauthor{松本 裕治}{
1955年生.
1977年京都大学工学部情報工学科卒.
1979年同大学大学院工学研究科修士課程情報工学専攻修了.
同年電子技術総合研究所入所.
1984〜85年英国インペリアルカレッジ客員研究員.
1985〜87年(財)新世代コンピュータ技術開発機構に出向.
京都大学助教授を経て,1993年より奈良先端科学技術大学院大学教授.
現在に至る.
工学博士.
専門は自然言語処理.
人工知能学会,日本ソフトウェア科学会,情報処理学会,認知科学会,AAAI,
ACL, ACM各会員.
}

\end{biography}


\biodate





\end{document}
