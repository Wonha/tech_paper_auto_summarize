\documentstyle[epsbox,jnlpbbl]{jnlp_j}

\setcounter{page}{19}
\setcounter{巻数}{8}
\setcounter{号数}{3}
\setcounter{年}{2001}
\setcounter{月}{7}
\受付{2000}{9}{25}
\採録{2001}{4}{13}

\setcounter{secnumdepth}{2}
\makeatletter
\def\le{}
\def\ge{}
\def\gl@align#1#2{}
\def\dfrac#1#2{}
\def\cfrac#1#2{}
\makeatother


\title{手指動作記述文間の類似性に基づく手話単語の分類方法}

\author{安達 久博\affiref{UIS}}
\headauthor{安達}
\headtitle{手指動作記述文間の類似性に基づく手話単語の分類方法}

\affilabel{UIS}{宇都宮大学 工学部 情報工学科}
{Department of Information Science, Utsunomiya University}

\jabstract{
手話は視覚言語としての側面を持つため,手話単語の語構成(造語法)に
おける特徴の一つとして「写像性」が挙げられる.
例えば,日本手話の日本語単語見出し「家」に対する手話表現は,
屋根の形を視覚的に写像して
いる.すなわち,手話表現が概念特徴の一部を視覚的に模倣している点である.
一般に,概念特徴は定義的特徴と性格的特徴に分類される.
ここで,定義的特徴とは,ある概念の定義に不可欠な特徴素の集合であり,
性格的特徴とは概念を間接的に特徴付ける特徴素の集まりを指す.
例えば,「家」に対する手話表現は,定義的特徴としての特徴素からの写像と
捉えることができる.
一方,「破産」に対する手話表現は,比喩的な表現「家が潰れる」という
概念の間接的な記述,すなわち,性格的特徴を視覚的に写像し
「家」の手話表現を提示した後に,両手を付け合わせる表現で定義されている.
すなわち,一義的には、双方の単語間に概念の類似性はみられないものの,
手指動作特徴の類似性という観点からみると「家」の派生語と捉える
ことができる.
また,日本語との言語接触により,日本語の単語見出しの構成要素を借用した
複合表現(例えば,「青森」は「青い」と「森」から成る.)で構成される単語が
少なくない.この借用も広義の写像性と捉えることができる.
このように,
手指動作特徴の類似性により手話単語を分類することは,
手指動作特徴が担う概念特徴と造語法との関係を明らかにする重要な手がかり
の一つを提供できると考える.
また,手話単語を対象とする電子化辞書システムなどにおいては,
手指動作特徴を検索キー
とする類似検索機構を実現する上での有益な知識データと捉えることができる.
本論文では,与えられた手話単語の有限集合を手指動作特徴間の類似性に基づき
分類する方法として,市販の手話辞典に記述されている手指動作記述文間の類似
性に着目した手法を提案する.本手法の特徴は,手指動作記述文間の類似度を
求め,集合の要素間の同値関係により単語集合を同値類に分割する点にある.
実験により,提案手法の妥当性を示す結果が得られた.
}
\jkeywords{手話単語,分類,同値関係,手指動作記述文}

\etitle{A Clustering Method of Signs Based on\\
Similarity between Manual Motion\\
Descriptions}
\eauthor{Hisahiro Adachi \affiref{UIS}}

\eabstract{
Since sign language is a kind of visual language,
there is ``iconicity'' as salient visual characteristics of
the word formation.
That is, iconicity in sign language refers to a visual
resemblance between signs and the things they stand
for (i.e. the meanings). 
The property of the meaning can be divided into the definition
and characteristic features.
For example, a sign for ``house'' provides a direct representation
that both hands outline the shape of the roof of a house; 
there is a direct relation between the meaning of sign and
a visual characteristic of what it presents as the definition features.
However, a sign for ``bankruptcy'' provides an indirect representation
that both hands touch each other after the `house', which
is derived from the causal relationship such that the house is
destroyed by bankruptcy as the characteristic features.  
Although their words don't resemble in the meanings,
there is similarity between their manual motion properties, that is,
it can be considered that the `bankruptcy' is a derivation of the `house'. 
By being in contact with Japanese, furthermore, signs are often
formed by borrowing from a part of the elements of word formation.
For example, a sign ``{\it Ao-mori}'' is a compound of the
signs ``blue'' and ``forest''. 
Borrowing also can be considered as symbolic iconicity in a broad sense.
By clustering signs with similar manual motion properties, therefore,
an important clue can be provided to explicate the relationship between
the meaning of manual motion properties and the word formation. 
Furthermore, in an electronic sign dictionary system, 
it can be considered that the result of clustering play the significant
role as knowledged database in the retrieval mechanism.
This paper proposes a method for grouping signs into disjoint
clusters with similar manual motion properties.
The method is based on the similarity between manual motion
descriptions(MMDs)
appeared in the ordinary sign dictionary.
By computing the similarity between the MMDs and
translating them into the equivalence relation, the equivalence
classes formed by the relation can be considered as clustering
signs that are similar to each other. 
The results of evaluation experiments show the applicability of
the proposed method.
}

\ekeywords{Sign, Clustring, Equivalence relation,
Manual motion description}

\begin{document}
\thispagestyle{plain}
\maketitle


\section {はじめに}\label{hajime}

一般に,手話言語は視覚言語としての側面を持つ.
この視覚言語としての特性の一つは,音声言語が単語を線条的に配列し.文を構成する
のに対して,単語を空間的かつ同時的に配列することで文を構成できる点である
\cite{Baker1980}.
また,単語の語構成においても,
例えば,右手で「男」を示し,左手で「女」を
同時的に空間に配置し,両手を左右から近付けることで「結婚」を,逆に「結婚」
の手話表現を示し,両手を左右に引き離すことで「離婚」を表現している.
すなわち,
音声言語に比べて,単語を造語する際の{\gt 写像性}({\it iconicity})が高い
言語であると捉えることができる.

また,手話単語の造語法の特徴には,この事物,事象の仕草(ジェスチャ)という
写像性を持つと同時に,ある手話単語の構成要素(手の形,手の位置,手の動き)の
パラメータの一部を変更したり,他の手話単語との複合表現により,
別の意味を担う単語見出しに対応できる点が挙げられる\cite{Ichida1994}.
例えば,日本語の単語見出し「破産」に対する日本手話の手話表現は,
破産との因果関係「家が潰れる」を比喩的に表象し,「家」の手話表現,
すなわち,屋根の形を構成する両手を中央で付け合わせる仕草で表現している.
また,「家族」は左手で「家」の手話を構成しながら,右手で「人々」の手話を同時に提示する
ことで表現される.さらに,「学校」は,「教える」と「家」の複合語表現として
定義されている\cite{Honna1994}.

このように,手話単語を構成する手指動作特徴の各パラメータは,
手話単語の構造を記述する表記法として重要である\cite{Yonekawa1984}と同時に,
単語の表す概念の一部を写像的に表現していると捉えることができる.
これは,単語間の手指動作特徴の類似性を調べることで,その類似の特徴パラメータ
が示す概念特徴とは何か,すなわち,概念特徴が表現するどの部分を特徴素として
抽出しているのかを解明する一つの手がかりとなると考える.

さて,一般に,単語見出しは単語が担う複数の概念を表す総称的なラベルの一つである.
また,意味特徴モデル\cite{Smith1974}では,概念は幾つかの特徴素の集合
として表現されるとしている.この概念の特徴素には二つの種類があり,
その一つは,ある概念を定義し,かつ不可欠な要素を列挙する
{\gt 定義的特徴}であり,他方は{\gt 性格的特徴}である.
例えば,日本語の単語見出し「ウグイス」の定義的特徴としては,
``翼がある,飛べる,ホーホケキョと鳴く''などである.
これに対して,性格的特徴は,``早春に飛来する,梅に止まる''など
である.
このように,性格的特徴は,ウグイスらしさを記述しているが,概念の定義として
不可欠な特徴素ではない\cite{Ohsima1986}.
ここで,先に示した「家」の手話表現は建物としての概念の定義的特徴を視覚的に
写像しているのに対して,「破産」は,性格的特徴による表現と捉えることができる.

本研究では,
市販の辞書に収録されている日本手話の手話単語を
対象に,複数の手話単語間に存在するであろう手指動作特徴の類似性と,
その類似の手指動作特徴を含む単語間に共有される概念の特徴素とは何かを
明らかにするため,手指動作特徴間の類似性による単語の部分集合(クラスタ)
を求める方法について検討を行った.

この類似の動作特徴を含む手話単語のクラスタの獲得は,
言語学分野における,手話単語の構造や造語法を解明する手がかりとして,
重要であるばかりでなく,手話言語を対象とする計算機処理にも有益な知識データ
の一つとなると考える.
例えば,日本語と手話の橋渡しとなる手話通訳システムや電子化辞書システムでは,
単語の登録や検索が重要な要素技術の一つであり,手指動作特徴からの日本語単語
見出しの効率の良い検索方法の実現は重要である.このように,
手指動作特徴の類似性に基づく分類方法は,検索辞書の構築に
有効利用できると考える.
例えば,ニュース原稿を手話通訳する現場から,新たに手話単語を造語する必要性が
報告\cite{Shigaki1991}されており,
造語する場合の観点として,
ある動作特徴の果たしている意味は何か,あるいは,類似の動作特徴を含む
他の単語との整合性があるか(既に定義されている単語との競合はないか)
が重要であり,これらを効率よく調べる手段を提供できる可能性がある.

このような背景から,
本論文では,与えられた手話単語の有限集合を手指動作特徴間の類似性
に基づき,単語のクラスタ(部分集合)を求めるための一つの分類方法を提案し,
その有効性を検証するために行った実験結果について述べる.
本提案手法の特徴は,市販の手話辞典に記述されている日本語の手指動作記述文を
手指動作パターンの特徴系列と捉え,手指動作記述文間の類似関係から同値関係を
導出し,与えられた単語集合を同値類に分割する点にある.

なお,
関連する研究として,従来,手話単語の構造を記述する表記法に焦点を当てた
研究が言語学と工学の分野から幾つか報告されている.例えば,
\cite{Stokoe1976}は,ASL(American sign language)の手話単語を対象に
手の形,手の位置,手の動きを手指動作特徴の特徴素とする表記法を提案し,
\cite{Kanda1984,Kanda1985}は日本手話の表記法についての検討結果を報告している.
また,手話の画像処理\cite{Kamata1991}や画像通信\cite{JunXU1993}の
観点からの表記法も提案されている.これらの表記法は,手話の表現を
厳密に再現することを目的としているため,\cite{Naitou1996}が指摘している
ように,複雑なコード体系を用いている.

一方,
\cite{Adachi2000}は複雑なコード体系
により記号化された表現
ではなく,市販の辞書中に記述されており,初学者にも親しみやすい(扱いやすい)
自然言語文として表現されている
手指動作記述文間の類似関係を手話単語間の類似関係とみなし,
手指動作記述文間の類似度を計算することで,
類似の動作特徴を含む手話単語対の抽出方法を提案している.
この手法の利点の一つは,データ収集の容易さと同時に
対象単語数の大規模化が容易に行える可能性がある点である.
本研究では,同様に単語間の類似性を手指動作記述文間の類似性とみ
なす考え方を採り入れ,さらに,「単語と単語」との直接的な類似関係による
単語間の関係に,推移律を満たす関係式を新たに導入することで,
集合の同値関係を規定し,間接的な類似関係をも考慮した
「単語対と単語対」との類似関係に焦点をあて,与えられた単語集合から同値類を抽出し分類することを特徴としている.

以下,2章で,手指動作記述文間の類似度の計算方法を概説し,
3章で,類似関係を表す類似行列の推移行列への変換手続きによる分類方法について
述べ,4章で,本提案手法の妥当性を検証するために行った実験結果を示し,
5章で考察を行う.

\section {手指動作記述文間の類似度}

\subsection {手話単語間の類似度の考え方}\label{idea}

一般に,パターン認識においては,構造を持つオブジェクト間の関係を計る尺度として,
距離や類似度を定義する必要がある\cite{Tanaka1990}.
本論文では,手話単語が$n$個の手指動作特徴を持つとし,$n$次元空間上
の点で表現する.
この空間上での$n$次元の特徴ベクトル
のなす角を用いて,手話単語間の類似度を近似する.
ここで,手話単語とそれに対応する手指動作特徴を自然言語文に写像した
手指動作記述文に1対1の対応関係があるとすると、手話単語間の類似度問題は,
手指動作記述文間の類似度問題と捉えることができる.

\subsection {手指動作記述文間の類似度の計算方法}

\ref{idea}節で示した類似度の考え方から,
手話単語間の類似度を,対応する手指動作記述文間の類似度とみなす.
ここでは,
二つの手話単語 $A, B$ に対する手指動作記述文の文字列を
$A = a_1 a_2 \cdots a_m, B = b_1 b_2 \cdots b_n$  とし,
両者の最長共有部分列の長さを$LCS$と表記するとき,
次式で示した手話単語$A,B$ の類似係数$S(A, B)$を
$A,B$間の類似度とみなす\cite{Adachi1993a}.

\begin{equation}
\label{sim}
S(A, B) = \frac{LCS(A,B)^2}{mn} = \frac{LCS(A,B)}{m}\frac{LCS(A,B)}{n}
\end{equation}

ここで,$LCS(A,B)$は,動的計画法を利用して次式で計算できることが知られている
\cite{Thomas1990}.
また,$LCS(A_i,B_j)$は部分列$A_i$と$B_j$の最長共有部分列の長さを示し,
$LCS(A_i,0)=LCS(0,B_j)=0\  (1\le i \le m, 1\le j \le n)$とする.
なお,$LCS$は複数の最長共有部分列を導出する可能性があるが,その長さは
一意に決定できる.

\begin{equation}
LCS ( A, B ) = LCS ( A_m, A_n )
\end{equation}

\[
LCS(A_i,B_j) = \left\{
\begin{array}{ll}
LCS(A_{i-1},B_{j-1})+1 & a_i=b_j \\
\max \{LCS(A_i,B_{j-1}),LCS(A_{i-1},B_j) \} & \mbox{otherwise}
\end{array}
\right.
\]

例えば,$A=``右手を右に倒す'', B=``右手を左に倒す''$ とした場合,
表\ref{lcs}に示すように
$LCS(A,B)=LCS(A_{6}, B_{6})=6$ となり,$S(A,B)=0.73469388$となる.
なお,表中の括弧で示した部分は両者の文字が一致する箇所($a_i=b_j$)を示す.

\begin{table}[htb]
\caption{$LCS(A,B)$ の計算例}
\label{lcs}
\tabcolsep=3pt\footnotesize
\begin{center}
\begin{tabular}{c|ccccccc}
       & 右& 手& を& 左& に& 倒& す\\ \hline
     右&(1)&  1&  1&  1&  1& 1 &  1\\
     手&  1&(2)&  2&  2&  2& 2 &  2\\
     を&  1&  2&(3)&  3&  3& 3 &  3\\
     右&(1)&  2&  3&  3&  3& 3 &  3\\
     に&  1&  2&  3&  3&(4)& 4 &  4\\
     倒&  1&  2&  3&  3&  4&(5)&  5\\
     す&  1&  2&  3&  3&  4& 5 &(6)\\
\end{tabular}
\end{center}
\end{table}


\section {手話単語の分類方法}

\subsection {有限集合の同値関係}

一般に,ある有限集合$X$の直積$X\times X$における
二項関係を$R(x,y)$と表記する.
ここで,反射律と対称律を満たす$R(x,y)$を類似関係と呼び,
その関係を行列で表現したものを類似行列と呼ぶ.
\begin{eqnarray*}
 R(x,x)&=&1,  \forall x \in X\\
 R(x,y)& =& R(y,x)
\end{eqnarray*}


さらに,類似関係が推移律を満たす場合,$R(x,y)$を同値関係と呼び,
行列で表現したものを推移行列(あるいは同値関係行列)と呼ぶ.
例えば,次式は推移関係を示している\cite{Ito1986}.

\[
 R(x,z) \ge \max_{y}\min\{R(x,y),R(y,z)\}
\]
なお,この同値関係により,与えられた有限集合の要素を同値類に分割できることが
知られている\cite{Klir1988}.

本論文で提案する分類方法は,与えられた手話単語の有限集合を手指動作記述文間の
類似性に基づく同値関係により,集合要素を同値類へ分割するものである.
前章で定義した類似度$S(x,y)$は以下に示すように,
反射律と対称律の二つの条件を満たしていることは明らかである.

\begin{center}
\begin{tabular}{ll}
反射律 & $S(x,x) = 1$\\
対称律 & $S(x,y) = S(y,x)$\\
\end{tabular}
\end{center}

そこで,以下の関係式を導入し,推移律を満たす
同値関係を導出する.

\begin{equation}
\label{eq:suii}
S(x,z) \ge \max_{y}\min\{S(x,y),S(y,z)\}
\end{equation}

すなわち,推移律は,類似度 $S(x,y)$と$S(y,z)$から得られる{\gt 間接}
の関係と,類似度 $S(x,z)$から得られる{\gt 直接}の関係により一義的に定義される.
次節では,具体的な例を用いて,同値関係による分類方法について詳細に述べる.

\subsection {分類方法}\label{tejun}

$X = \{a,b,c,d,e \}$を与えられた手話単語の有限集合とし,
集合$X$の要素間の類似関係 $S(x,y)$ は,
以下に示す類似行列 $S$ で表現されているとする.
ここで,対称律により,例えば,$S(a,b)=S(b,a)=0.2$であり,
反射律により,対角線成分はすべて1となる.

\[ S = 
\begin{array}{r@{}l}
& \begin{array}{ccccc}
\makebox[2.5em]{a}& b& \makebox[2.0em]{c}& d& \makebox[2.0em]{e}
\end{array}\\
\begin{array}{l}
a\\ b\\ c\\ d\\ e
\end{array} &
\left(
\begin{array}{ccccc}
1 & 0.2 & 0.5 & 0.3 & 0.8 \\
0.2 & 1 & 0.3 & 0.5 & 0.3 \\
0.5 & 0.3 & 1 & 0.2 & 0.7 \\
0.3 & 0.5 & 0.2 & 1 & 0.2 \\
0.8 & 0.3 & 0.7 & 0.2 & 1
\end{array}
\right)
\end{array}
\]

次に,
手話単語 $a$ と $b$ の関係を例として,推移関係を満たす
類似度を求める手続きについて述べる.
まず,$a$と$b$との間接の関係は,

(1)$S(a,c)=0.5$, $S(c,b)=0.3$,

(2)$S(a,d)=0.3$, $S(d,b)=0.5$,

(3)$S(a,e)=0.8$, $S(e,b)=0.3$ 

\noindent であり,
それぞれの組の中で最小の類似度の集合を$S_{min}$と表記すると,
$S_{min} = \{0.3,0.3,0.3 \}$となり,その最大値は$0.3$となる.
一方, 手話単語 $a$と$b$の直接の関係は,$S(a,b) = 0.2$である.
この直接と間接の関係にある類似度を比較して,大きい方の値を推移関係における
$S(a,b)$の類似度とする.この例では間接の類似度の方が大きく,
$S(a,b)=0.3$ となる.

ここで,類似行列$S$における類似度$S(x,z)$と区別するため,推移行列を
$T$と表記し,$T$における類似度を$T(x,z)$と表記すると,
式 (\ref{eq:suii}) は次式で表現でき,
推移行列 $T$ は以下のように表現される.

\begin{equation}\label{eq:trans}
 T(x,z)=\max \Bigl( S(x,z),\max_{y}\min\{S(x,y),S(y,z)\} \Bigr)
\end{equation}


\[ T =
\begin{array}{r@{}l}
& \begin{array}{ccccc}
\makebox[2.5em]{a}& b& \makebox[2.0em]{c}& d& \makebox[2.0em]{e}
\end{array}\\
\begin{array}{l}
a\\ b\\ c\\ d\\ e
\end{array} &
\left(
\begin{array}{ccccc}
1 & 0.3 & 0.7 & 0.3 & 0.8 \\
0.3 & 1 & 0.3 & 0.5 & 0.3 \\
0.7 & 0.3 & 1 & 0.3 & 0.7 \\
0.3 & 0.5 & 0.3 & 1 & 0.3 \\
0.8 & 0.3 & 0.7 & 0.3 & 1
\end{array}
\right)
\end{array}
\]

さらに,推移行列 $T$ を対角線成分に近い程,行列成分(類似度の値)が大きくなる
ように行列の要素間の交換を行うと,以下の推移行列 $T_{sort}$
が得られる.

\[ T_{sort} = 
\begin{array}{r@{}l}
& \begin{array}{ccccc}
\makebox[2.5em]{a}& e& \makebox[2.0em]{c}& d& \makebox[2.0em]{b}
\end{array}\\
\begin{array}{l}
a\\ e\\ c\\ d\\ b
\end{array} &
\left(
 \begin{array}{ccc|cc}
1   & \multicolumn{1}{c|}{0.8} & 0.7 & 0.3 & 0.3 \\
0.8 & \multicolumn{1}{c|}{1} & 0.7 & 0.3 & 0.3 \\ \cline{1-2}
0.7 & 0.7 & 1 & 0.3 & 0.3 \\ \hline
0.3 & 0.3 & 0.3 & 1 & 0.5 \\
0.3 & 0.3 & 0.3 & 0.5 & 1
\end{array}
\right)
\end{array}
\]

これにより,与えられた手話単語の有限集合$X$は,ある適切な閾値$\alpha$を設定する
ことで,$\alpha$における同値関係$T_{\alpha}$により,互いに素な部分集合に直和分割される.
ここで,集合$X$の閾値$\alpha$による分割を$X/T_{\alpha}$ と表記し,
閾値( $1 \ge \alpha \ge 0$ )を段階的に変化させることにより,
以下に示すように,$\alpha$による階層構造を構成することができる.

\begin{eqnarray*}
X/T_{1.0} &=& \{a, b, c, d, e \} \\
X/T_{0.8} &=& \{ \ (a, e), \ c, d, b \} \\
X/T_{0.7} &=& \{ \ (a, e, c), \ d, b \} \\
X/T_{0.5} &=& \{ \ (a, e, c), \ (d, b) \} \\
X/T_{0.3} &=& X/T_{0.0} = \{ \ (a, e, c, d, b)\  \}
\end{eqnarray*}

同様に,行列$T_{sort}$は以下に示すように,一般に,
デンドログラム(樹系図)と呼ばれる階層的なクラスタリング結果を内包していると
捉えることができる.
なお,本論文では便宜上,以下に示す連分数の表現形式を用いて階層関係を表現する
ことにする.

\[
\cfrac{0.3}{\dfrac{0.5}{b,d} + \cfrac{0.7}{\cfrac{0.8}{a,e} + c}}
\]

以上の手続きで得られた推移行列を用いると,単語対と単語対との類似度
が一義的に決定される.


\section {実験と結果}

ここでは,本手法の妥当性を検証するため,与えられた手話単語の集合を同値類に
分割する実験を行った結果を示し,
得られた
同値類に含まれる手話単語を分析し,どのような手指動作特徴の類似性により
手話単語が結束しているのかを明らかにし,手指動作特徴と単語の意味との関係,
すなわち,手話単語の造語法を解明する上での手がかりや手話単語の電子化辞書を構築する上での有用な情報が得られたか否かで評価を行う.

\subsection {実験データ}\label{pre}

議論を明確にするため,本論文では顔,特に``口''の部分を
手指動作特徴の要素(手の位置)として用いる手話単語の有限集合を
実験対象とし,以下の手順で実験データを準備した.
最初に,手話辞典\cite{MaruyamaKoji1984}からキーワードとして,
``口''または``唇''を含む手指動作記述文(以下,記述文と略記する.)を抽出し,
人手により計算機に入力した.

次に,表\ref{marge}に示すように,同一の
記述文($S(x,x) = 1$に相当)をマージし,最終的に,101記述文とその単語見出し
をペアとする構造の実験データを準備した\footnote{記述文中の「唇」は,「口」と同一視し,文字の置換処理により「口」に統一した.}.
なお,表\ref{marge}中の単語見出しの添字の意味は,
数字が複合語を構成する記述文の出現する配列順序を示す.
一方,英字は同一の単語見出しに対して,異なる手話表現が辞書に
定義されていることを意味する.
例えば,表\ref{marge}中の「恥ずかしい.A.1」と「恥ずかしい.B.1」は,
単語見出し「恥ずかしい」に対して,
二つの手話表現A,Bが辞書に定義されており,A,Bともに同一の記述文を
複合語表現の最初に用いていることを示す.

\begin{table}[htb]
\caption{同一の手指動作記述文を含む単語群}
\label{marge}
\tabcolsep=3pt\footnotesize
\begin{center}
\begin{tabular}{|l|}\hline
右手の人差指を下唇にあてて右に引く\\ \hline
遺伝.1,火事.1,赤十字.1,速達.2,ソ連.1,日曜日.B.1,
日赤(日本赤十字社の略称).1,\\
はしか.1,恥ずかしい.A.1,恥ずかしい.B.1,はにかむ.1,
火.1,貧血.1,もみじ.1,りんご.1,\\
火曜日.1,血液.1,錆.1,出血.1,信号.1\\ \hline
人差指を立てて唇にあてる\\ \hline
家出.1,隠す.1,スパイ.1,亡命.1\\ \hline
人差指と親指を伸ばしてそのつけ根を口の前におき二指を開閉する\\ \hline
鳥取.1,鳥.1\\ \hline
人差指で口のところに小さく円を描く\\ \hline
読話.1,口話.1\\ \hline
小指を下唇にあてる\\ \hline
海.1,しょうゆ.1\\ \hline
五指を折り曲げた右手を口の前でまわす\\ \hline
辛い,カレーライス.1\\ \hline
右手の親指を口の前で右から左に往復させる\\ \hline
通訳,紹介\\ \hline
\end{tabular}
\end{center}
\end{table}


\subsection {実験方法と結果}

以下では,\ref{tejun}節で述べた分類手順に従い,その過程で得られた結果を
段階的に示しながら実験方法の説明を行う.

まず最初に,\ref{pre}節で得られた記述文間の類似度を式(\ref{sim})で求める.
表\ref{sim_kekka}は類似度 $0.6$ 以上の単語対として抽出された25組を示す.
その結果,図\ref{s_matrix1}に示すように $31\times 31$ の類似行列が得られる.
ここで,行列は対角線成分($S(x,x)=1$)と
表\ref{sim_kekka}の単語対に対応する要素成分($S(x,y)\ge 0.6$)を記号``$\ast$''で示す.
すなわち,図\ref{s_matrix1}に示した類似行列は,
閾値$\alpha$を$0.6$に設定し,閾値$\alpha$以上の成分を1とみなし,
$\alpha$未満の成分は0とした閾値行列と捉えることができる.


\begin{table}[htb]
\caption{類似度 $0.6$ 以上の手話単語ペア}
\label{sim_kekka}
\tabcolsep=3pt\footnotesize
\begin{center}
\begin{tabular}{c|l||c|l}\hline
類似度 & 手話単語ペア & 類似度 & 手話単語ペア \\ \hline\hline
0.97 &(梅干し.1, 梅.1) & 0.68 & (言い訳.2, 打ち消す.1) \\ \hline
0.94 &(言う, 言い訳.2) & 0.66 & (アドバイス.2, 言葉.1) \\ \hline
0.89 &(遺伝.1, 苺.1)   & 0.66 & (お世辞.1, 打ち消す.1) \\ \hline
0.88 &(遺伝.1, 日曜日.A.1) & 0.65 & (渋い, 唐辛子.1) \\ \hline
0.85 &(餅.2, そば(蕎麦)) & 0.64 & (言う, 取り寄せる.1) \\ \hline
0.85 &(読話.1, 口実.1) & 0.63 & (辛い, こしょう.1) \\ \hline
0.77 &(日曜日.A.1, 苺.1) & 0.63 & (言う, お世辞.1) \\
0.75 &(ニュース.2, 発言) & 0.62 & (発言, 白状) \\ \hline
0.75 &(こしょう.1, 唐辛子.1)& 0.62 & (風邪.1, 咳) \\ \hline
0.74 &(言う, 打ち消す.1) & 0.62 & (辛い, 渋い) \\ \hline
0.69 &(ソース.1, こしょう.1) & 0.62 & (赤字.1, 紅茶.1) \\ \hline
0.69 &(ラーメン.2, 餅.2) & 0.61 & (取り寄せる.1, 言い訳.2) \\ \hline
0.69 &(恥ずかしい.B.2, はにかむ.2) & & \\ \hline
\end{tabular}
\end{center}
\end{table}

\begin{figure}[htb]
\begin{center}
\atari(71,121)
\end{center}
\caption{類似行列}
\label{s_matrix1}
\end{figure}

次に,この類似行列を式(\ref{eq:trans})を用いて推移関係を満たす類似度
を求め,図\ref{tr_matrix2}に示す推移行列が得られる.
ここで,例えば.図\ref{s_matrix1}に示した類似行列の単語ラベル「辛い」に
注目すると,
直接的な類似関係として単語対(辛い,渋い)と(辛い,こしょう.1)の二つがある
ことが分かる.
一方,図\ref{tr_matrix2}に示した推移行列では,間接的な類似関係
(辛い−こしょう.1−ソース.1),(辛い−こしょう.1−唐辛子.1)により,
新たに,単語対(辛い,ソース.1)と(辛い,唐辛子.1)の類似関係が,
類似度$0.6$以上の二項関係として導出されていることが分かる.

\begin{figure}[htb]
\begin{center}
\atari(72,120)
\end{center}
\caption{推移行列}
\label{tr_matrix2}
\end{figure}

さらに,図\ref{tr_matrix2}に示した推移行列を対角線成分に近いほど類似度の値が
大きくなるように成分間の交換を行い,
図\ref{tr_matrix3}に示した推移行列(閾値行列)が最終的に得られ,
31単語見出しは11個の同値類に結束されたことが分かる.
同様に,類似度 $0.5$ 以上では,
図\ref{tr_matrix5}に示すように 41単語見出しが 16個の同値類に結束された.

\begin{figure}[htb]
\begin{center}
\atari(72,120)
\end{center}
\caption{類似度の閾値を0.6に設定した場合の分割例}
\label{tr_matrix3}
\end{figure}

以下では,同値類に結束された手話単語間の類似の手指動作特徴は何かを明らかに
するため,単語間の記述文を比較し,
類似の手指動作特徴が示す概念特徴と手話単語の造語法との関係について分析を行う.
また,閾値$\alpha=0.6$と$0.5$の場合の分類結果を
比較し,単語間の階層性についても議論する.

\begin{figure}[htb]
\begin{center}
\atari(82,146)
\end{center}
\caption{類似度の閾値を0.5に設定した場言の分割例}
\label{tr_matrix5}
\end{figure}

分析の結果,
$\alpha=0.6$で結束された「遺伝.1」を含む同値類の手話単語は,表\ref{marge}に
示した同義の手話単語見出しも含め,
単語見出し「赤」に対する手話表現を複合語の構成要素とする部分集合
を構成していることが分かった.また,手指動作表現は「唇の色」あるいは「口紅を引く(塗る)仕草」を表現し,``赤い''という属性概念に対応していると捉えることができる.
同様な例として,単語見出し「黒」,「白」に対応する手話表現は
それぞれ「掌で頭(髪の毛)をこする」,「人差指で歯を示す」というように,「黒い髪」,「白い歯」を強調的に示すことで色に関する属性概念を表現している.

一方,同値類(赤字.1, 紅茶.1)は,右手で「赤」に対する手指動作を表現し,
左手でそれぞれ,「帳簿」,「カップ」を示す手指動作表現を行っている.
このように,左手の手指動作表現に対する記述文の差異が類似度に反映され,
結果として異なる同値類を構成している.

\begin{list}{}{\setlength{\topsep}{3pt}}
\item [{\bf 赤字.1}]
掌を上に向けた左手を胸の前におき右手の人差指を下唇にあてて軽く右に引く
\item [{\bf 紅茶.1}]
わん曲させた左手を胸の前におき右手の人差指を下唇にあてて軽く右に引く
\end{list}

\noindent
なお,図\ref{tr_matrix5}に示すように,$\alpha=0.5$ では,
この二つの同値類は併合され,``赤''の属性概念を示す単語
集合となり,以下に示す階層関係を構成している.この場合には,
片手手話と両手手話という差異を示していると捉えることができる.
また,1章で述べた写像性という観点でみると,空間的かつ同時的に
配列する手話の特徴を示しており,手指動作記述文では左手の特徴を記述した
後に右手の特徴を記述する傾向がみられる.

\[
\cfrac{0.5}{\cfrac{0.6}{赤字,紅茶} + \cfrac{0.6}{遺伝,苺,日曜日,etc}}
\]


同様に,同値類(辛い, ソース.1, こしょう1, 唐辛子.1, 渋い)は
「五指を折り曲げた右手を口の前で平面的に動かす」という手指動作特徴
により結束され,
特に,(辛い,ソース.1,こしょう.1,唐辛子.1)は回転動作を共有し,
「辛い」という味覚に関する属性概念を示している.一方,「渋い」は
上下の動作であり,味覚に関する別の属性値を担っている.
また,$\alpha=0.5$ では,この同値類に「苦い」が結束され,
左右の動作を示している.
この結果,この同値類に結束された単語集合は(辛い、苦い、渋い)という味覚概念を
示していると同時に,手の形に共通性がある.
一方,同様に味覚に関する概念を表している「甘い」は,
「甘やかす.1」と結束され,「辛い」と回転動作の共通性がみられるが,手の形が
五指を広げたものであり,この手の形の差異が別の同値類を構成している
要因と考えられる.このように,手指動作特徴の要素である「手の形」が
「甘い」と,「辛い」を代表とする``甘くない''概念を担う単語集合との対立観点と
捉えることができる.以下に記述文間の類似性による階層関係を示す.


\[
\cfrac{0.5}{苦い + \cfrac{0.6}{辛い,渋い}}+ \cfrac{0.5}{甘い}
\]

なお,手話表現は手指動作表現だけでなく,顔の表情や口形
なども重要な単語の構成要素であるが,本論文では,手指動作特徴に焦点をあて分析を
行った.
他の同値類においても,例えば,同値類(ラーメン.2,餅.2,蕎麦)は,``箸で口に運ぶ仕草''を
表現した手指動作特徴を共有する単語集合であり,「食べる」に関する概念を
示す,また,「言う」を含む同値類は,``口から出ていく仕草''を表象し,
「発言」とラベル付けが可能な概念を共有する単語集合と捉えることができ,
類似の手指動作特徴により同値類を構成し,手指動作特徴が示す概念との対応関係が確認された.

\begin{list}{}{\setlength{\topsep}{3pt}}
\item [{\bf ラーメン}] = 指文字ラ + 箸で食べる仕草
\item [{\bf 蕎麦}]     = 箸で食べる仕草
\item [{\bf 餅}]      = 餅をつく仕草 + 箸で食べる仕草 
\end{list}

\noindent
このように,本実験により得られた同値類を分析した結果,
「口」を手指動作特徴の要素(手の位置)とする単語集合は,
例えば,
「赤」,「発言」,「味覚」,「食べる」などとラベル付けが可能な概念特徴を共有
する部分集合に分類できることが分かった.
このように,「口」を手の位置とする手指動作特徴を持つ単語集合の分類実験から,
類似の動作特徴を含む手話単語を結束し,手指動作特徴の表す概念との関係など
手話単語の造語法を明らかにする一つの手がかりを示している.
また,分類結果は電子化辞書システムなどの構築に有用な
知識データと捉えることができ,
本提案手法の有効性を示す結果が得られたと考える.

\section {考察}

実験により,本提案手法を用いて手話単語の造語法の特徴を示す幾つかの同値類を抽出
し,手指動作特徴と概念との対応関係を示す重要な手がかりの一部を提供することが
できたと考える.以下では,明らかになった問題点を整理し,今後の課題と利用法に
ついて考察を行う.

\subsection {問題点と今後の課題}

「味覚を表す概念」とラベル付けが可能な同値類として,例えば,類似度の閾値を
$0.5$とした場合,(辛い,渋い,苦い)と(甘い)が異なる同値類として結束
された.この両者の対立観点は,「手の形」に関する手指動作特徴と捉えることが
できる.すなわち,「右手の五指を折り曲げる」という手の形と
「右手の五指を伸ばした」という手の形
の対立である.また,(辛い,甘い)の単語対については,「回転」動作を表す
手指動作特徴を共有している.一方,(辛い,渋い,苦い)の単語は,それぞれ,
(回転,上下,左右)の動作を表す手指動作特徴の差異が認められる.

\begin{list}{}{\setlength{\topsep}{3pt}}
\item [{\bf 辛い}]
五指を折り曲げた右手を口の前でまわす
\item [{\bf 渋い}]
五指を折り曲げた右手を口の前で二度ほど上下する
\item [{\bf 苦い}]
五指を折り曲げた右手を口の前におき二度ほど左右に動かす
\item [{\bf 甘い}]
掌を口の前にあてて二度ほど回転させる
\end{list}

\noindent
ここで,記述文間の差異が類似度の値にどのように影響したかについて分析を
行うと,(甘い)の手形を「掌」で表現し,(辛い)に代表される手形を
「五指を折り曲げた」で表現しており,この文字列の差異が類似度に反映されている
.また,回転動作に対して,(甘い)は「まわす」であり,(辛い)は「回転させる
」と表現されている.本来は共通の動作を示すべき表現の差異が類似度の計算に
反映されている.

一方,「甘い」以外の味覚表現の単語群は,手の形に関する記述文
に共通性(五指を折り曲げる)がある.
また,「辛い」に対する手話表現を用いる単語見出し
(ソース.1,唐辛子.1,こしょう.1)は,
表\ref{marge}に示した単語見出し「カレーライス.1」と同様に,
事前に同一の手話表現としてマージされるべきものであるが,
``表現(表記)のゆれ''により,機械的な文字列照合での一致ができなかった.
「赤」や「言う」を用いる単語についても同様である.

\begin{list}{}{\setlength{\topsep}{3pt}}
\item [{\bf ソース.1}]
五指を折り曲げた右手を口の前におきぐるぐる回転させる
\item [{\bf 唐辛子.1}]
五指を折り曲げた右手を口の前で二度ほど回転させる
\item [{\bf こしょう.1}]
五指を折り曲げた右手を口の前で回転させる
\end{list}

\noindent
このように,本論文の実験では,市販の手話辞典に記載の記述文をそのまま利用
したが,記述文の記述形式を正規化する方法を今後検討したい.その際に,手指動作
特徴(手の形,手の位置,手の動き)の各要素に対する記述文中での占める割合を
重み付けした類似度を検討する必要があると考える.

また,「赤」という属性値を持つ単語集合として結束された同値類の手話単語は,
単語間の弁別要素として複合語を構成する他の手話表現を利用しているが,
複合語として比較した結果,(遺伝,血液),(火,火曜日),(赤十字,日赤),
(恥ずかしい,はにかむ)の単語対は同一の手話表現であった.
なお,(火,火曜日)と火事は,弁別要素となる動作表現として,
片手と両手で表現する差異が認められる.
このように,複合語として表現される手話単語を全体として,
その手指動作特徴の類似性を計算する手法についても今後の課題とする.

\begin{list}{}{\setlength{\topsep}{3pt}}
\item [{\bf 火.2}] 
掌を上にしてわん曲させた右手を上にひねりながら上にあげていく
\item [{\bf 火事.2}]
掌を向かい合わせてわん曲させた両手を炎のように動かしながら上にあげていく
\end{list}

\subsection {利用法について}

ニュース原稿を手話通訳する現場サイドから,市販の辞書に収録
されていない(未定義語としての)手話単語を新たに造語する必要性
が指摘されている\cite{Shigaki1991}.
その中で,「政党」と「政治団体」は明確に区別して報道する必要がある.
そのため,(団体,サークル,集団)を表す手話表現の手指動作特徴の要素である
手の形だけを指文字の「と」に変更して手話単語「党(とう)」を造語した事例
が報告されている.

また,\cite{Tokuda1998}は手話通訳システムに
おける問題点の一つである日本語の単語見出しと手話単語の日本語ラベル
とのギャップを解消する手話単語辞書の補完方法として,
日本語辞書の概念説明文や語釈文から
手話辞書に未定義の日本語単語見出しに対応する手話表現を造語(類推)する場合の
問題点を指摘している.
これは,\ref{hajime}章で述べたように,概念特徴として語釈文などの定義的特徴よりも
性格的特徴から,視覚的な「写像」が容易な特徴素を抽出し,手話表現に
利用される傾向があることを示している.
例えば,「速達」や「日曜日」の手話表現に「赤」の手話表現を用いて
いる.すなわち,``手紙に押される赤いスタンプ''や``カレンダー上で赤い数字で示される''というような性格的特徴に位置付けられる概念特徴に基づき手話単語を造語
している.

このように,手話単語を手指動作特徴の類似性により分類することは,
既に定義されている単語の語構成を明らかにし,未定義語を
類推する場合に有効利用できると考える.
\cite{Honna1994}が報告しているように,新しく定義された
単語と既に定義されている単語との不整合を解消するため,既存の手話単語を変更する
必要性が生じる場合がある.その際に,手話単語の造語成分となる
手指動作特徴の担う概念の整理と分類は今後,ますます重要になると考える.

例えば,本実験に使用した手話辞典では,「赤字」は定義されているが,「黒字」が未定義語である.このような場合に,
類義語として「赤字」が検索され,その語構成「赤+記帳する仕草」から
「赤」を「黒」に置換することで妥当な手話表現「黒字」\cite{Ito1982}を
類推することができると考える.

同様な考えから,\cite{Adachi1993}は,辞書に収録(定義)されている
手話単語の複合語の造語成分と
構成順序に着目した分類を用いて,類義語の造語成分の一部を置換することで
未定義語の手話表現を類推する方法を提案している.
ここで,造語成分を分析,整理する手がかりとして本手法を有効に利用できると考える.
さらに,本手法により得られた推移行列を検索辞書と捉えれば,自然言語で表現された記述文を入力とし,類似の動作特徴を含む単語集合(同値類)
を提示する類似検索機構に有効利用できると考える.

ここで,類似の動作特徴を含む単語を
提示する類似検索の機能は,手話単語を弁別する要素と単語の造語法を理解できるなど,手話の学習効果を高める効果が期待できると考える.この検索方法の検討と実現は今後の重要な検討課題としたい.

このように,市販の手話辞典の手指動作記述文間の類似性を
計る尺度となる類似度の計算方法の改良や手指動作記述文の正規化など残された
課題もあるが,本提案手法により与えられた単語集合を同値類に分割することで,
手話単語の造語法の解明と計算機処理に有用な手がかりを比較的容易に
抽出、収集することができると考える.
また,対象データが市販の手話辞典から収集できることは,データ量の確保が容易
であり,複雑なコード体系による表現でなく,自然言語文として表現される点は,
人手による編集や分析作業を容易にする可能性が高いと考える.

\section {む す び}

本論文では,市販の手話辞典に定義されている手話単語の
手指動作記述文間の類似性に着目した手話単語の分類方法を提案した.
本手法の特徴は,
手指動作記述文間の類似関係を手指動作特徴間の関係とみなし,
同値関係に基づき手話単語を同値類に分割する点にある.
具体的には,
手指動作記述文間の類似度を計算し,
手話単語間の類似行列を求める.次に,推移関係式により推移行列
を導出し,与えられた手話単語の有限集合を
同値関係による同値類に分割する.
また,類似度の閾値を段階的に変化させることで,同値関係に基づく単語集合の
階層分類が可能であることを示した.
実験の結果,
類似の手指動作特徴により結束された同値類を抽出し,
手指動作特徴の類似性と手話単語の造語法との関係を解明する手がかりを
示す結果が得られた.
手話を対象とした自然言語処理システムにおいて,手指動作特徴に基づく
手話単語の検索機能の実現は不可欠な要素技術の一つであり,
同値関係に基づく推移行列を検索辞書と捉えた,手話単語の類似検索方法
の検討が今後の重要な課題である.

\acknowledgment

本研究を進めるにあたり,有益なご示唆,ご討論を頂いた宇都宮大学鎌田一雄教授,
熊谷毅助教授に心より感謝する.
また,データ整理,実験等に協力頂いた研究室の学生諸氏に感謝する.
なお,本研究の一部は文部省科研費,厚生省科研費,
実吉奨学会,電気通信普及財団,放送文化基金,
トヨタ自動車,栢森情報科学振興財団,大川情報通信基金の援助によった.



\bibliographystyle{jnlpbbl}
\bibliography{B}

\begin{biography}
\biotitle{略歴}
\bioauthor{安達 久博}{
1981年宇都宮大学工学部情報工学科卒業.
1983年同大学院工学研究科修士課程修了.
同年,東京芝浦電気株式会社(現.(株)東芝)入社.
同社総合研究所情報システム研究所に所属.
この間,(株)日本電子化辞書研究所(EDR)に4年間出向.
1992年より宇都宮大学工学部助手.
現在,聴覚障害者の情報獲得を支援する手話通訳システムに関する研究に従事.
言語処理学会,情報処理学会,電子情報通信学会,人工知能学会,
日本認知科学会,計量国語学会,各会員.
}

\bioreceived{受付}
\bioaccepted{採録}

\end{biography}

\end{document}

