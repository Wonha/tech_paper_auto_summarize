    \documentclass[japanese]{jnlp_1.4}
\usepackage{jnlpbbl_1.3}
\usepackage[dvips]{graphicx}
\usepackage{amsmath}
\usepackage{udline}
\setulminsep{1.2ex}{0.2ex}

\usepackage{amssymb}
\usepackage{cm-lingmacros}


\Volume{21}
\Number{2}
\Month{April}
\Year{2014}

\received{2013}{9}{20}
\revised{2013}{12}{6}
\accepted{2014}{1}{17}

\setcounter{page}{333}

\jtitle{日本語文章に対する述語項構造アノテーション仕様の考察}
\jauthor{
松林優一郎\affiref{Author---1} \and
飯田  龍\affiref{Author---2} \and
笹野 遼平\affiref{Author---2} \and
横野  光\affiref{Author---3} \and
松吉  俊\affiref{Author---4} \and \\
藤田  篤\affiref{Author---5} \and
宮尾 祐介\affiref{Author---3} \and
乾 健太郎\affiref{Author---1}
}
\jabstract{
日本語の述語項構造アノテーションコーパスは,これまでにいくつかの研究によって整備され,
その結果,日本語の述語項構造解析の研究は飛躍的にその成果を伸ばした.
一方で,既存のコーパスのアノテーション作業者間一致率やアノテーション結果の定性的な分析をふまえると,
ラベル付与に用いる作業用のガイドラインには未だ改善の余地が大きいと言える.
本論文では,より洗練された述語項構造アノテーションのガイドラインを作成することを目的とし,
NAISTテキストコーパス(NTC), 京都大学テキストコーパス(KTC)のアノテーションガイドラインと実際のラベル付与例を参考に,
これらのコーパスの仕様策定,仕様準拠のアノテーションに関わった研究者・アノテータ,
仕様の改善に関心のある研究者らの考察をもとにガイドライン策定上の論点をまとめ,
現状の問題点や,それらに対する改善策について議論・整理した結果を報告する.
また,アノテーションガイドラインを継続的に改善可能とするための
方法論についても議論する.
}
\jkeywords{述語項構造,コーパスアノテーション,アノテーションガイドライン,格解析,意味解析,意味役割付与}

\etitle{Issues on Annotation Guidelines for Japanese Predicate-Argument Structures}
\eauthor{
Yuichiroh Matsubayashi\affiref{Author---1} \and
Ryu Iida\affiref{Author---2} \and
Ryohei Sasano\affiref{Author---2} \and
Hikaru Yokono\affiref{Author---3} \and
Suguru Matsuyoshi\affiref{Author---4} \and
Atsushi Fujita\affiref{Author---5} \and \\
Yusuke Miyao\affiref{Author---3} \and
Kentaro Inui\affiref{Author---1}
} 
\eabstract{
Japanese corpora annotated with predicate-argument structure (PAS) have been constructed as part of several research projects and these annotated corpora have significantly advanced the field of PAS analysis. 
However, according to an inter-annotator agreement study and qualitative analysis of the existing corpora, there is still a strong need for further improvement of the annotation guidelines of the corpora. 
To improve the quality of PAS annotation guidelines, we have collected and summarized the practical knowledge and a list of problematic issues concerning the task of the PAS annotation through discussions with researchers actively engaged in the construction of NAIST Text Corpus (NTC) and Kyoto Text Corpus (KTC), researchers concerned with existing PAS annotation guidelines, and an annotator who is working on the annotation task, using NTC and KTC guidelines. 
This paper reports the problems and suggestions that we collected and possible solutions to those problems on the basis of results of the discussions. 
Finally, we suggest a method for continuously improving annotation guidelines.
}
\ekeywords{Predicate-Argument Structure, Corpus Annotation, Annotation Guideline, \linebreak Semantic Role Labeling, Semantic Parsing}

\headauthor{松林,飯田,笹野,横野,松吉,藤田,宮尾,乾}
\headtitle{日本語文章に対する述語項構造アノテーション仕様の考察}

\affilabel{Author---1}{東北大学}{Tohoku University}
\affilabel{Author---2}{東京工業大学}{Tokyo Institute of Technology}
\affilabel{Author---3}{国立情報学研究所}{National Institute of Informatics}
\affilabel{Author---4}{山梨大学}{University of Yamanashi}
\affilabel{Author---5}{公立はこだて未来大学}{Future University Hakodate}



\begin{document}
\maketitle


\section{はじめに}

述語項構造は,文章内に存在する述語と,その述語が表現する概念の構成要素となる複数の項との間の構造である.
例えば次の文,
\enumsentence{
[太郎]は[手紙]を\underline{書い}た.
}
では,述語「書く」に対して,「太郎」と「手紙」がこの述語の項であるとされる.また,述語が表現する「書く」という概念の上で
それぞれの項の役割は区別される.役割を表すためのラベルは用途に応じて様々であるが,例えば,ここでの「太郎」には
「ガ格」「動作主」「書き手」などのラベル,「手紙」には「ヲ格」「主題」「書かれる物」などのラベルが与えられる.
このように,述語に関わる構成要素を構造的に整理する事によって,
複雑な文構造・文章構造を持った文章において「誰が,何を,どうした」のような文章理解にとって重要な
情報を抽出することができる.このため,述語項構造の解析は,機械翻訳,情報抽出,言い換え,含意関係理解などの
複雑な文構造を取り扱う必要のある言語処理において有効に利用されている\cite{shen2007using,liu2010semantic}.

述語項構造解析においても,近年,形態素解析や構文解析などで行われている方法と同様に,
人手で作成した正解解析例をもとに統計的学習手法によって解析モデルを作成する方法が主流となっている\cite{marquez2008srl}.
述語項構造を付与したコーパスとしては,日本語を対象にしたものでは,
京都大学テキストコーパス(KTC) \cite{KUROHASHISadao:1997-06-24}の一部に付けられた
格情報\cite{kawahara2002construction,河原大輔2002関係}や
NAISTテキストコーパス(NTC) \cite{iida2007annotating,飯田龍2010述語項構造}, GDAコーパス\cite{hashida05}, 
解析済みブログコーパス(Kyoto University and NTT Blog Corpus: KNBC) \cite{橋本力2009}, 
    NTCの基準に従ってBCCWJコーパス (国立国語研究所)\nocite{bccwj}に述語項構造情報を付与したデータ(BCCWJ-PAS) \cite{komachi2011}などがあり,
英語を対象にしたものでは,PropBank~\cite{palmer2005pba}, FrameNet~\cite{Johnson2003}, 
NomBank~\cite{meyers2004nombank}, OntoNotes~\cite{hovy2006ontonotes}などが主要なコーパスとして挙げられる.
過去十年間の述語項構造解析技術の開発は,まさにこれらのデータによって支えられてきたといって過言ではない.


しかしながら,日本語の述語項構造コーパスは,その設計において未だ改善の余地を残す状況にあると言える.
第一に,比較的高品質な述語項構造がアノテートされた英語のコーパスに比べて,
日本語を対象とした述語項構造のアノテーションは,省略や格交替,二重主語構文などの現象の取り扱いのほか,
対象述語に対してアノテートすべき項を列挙した格フレームと呼ばれる情報の不足などにより,
作業者間のアノテーション作業の一致率に関して満足のいく結果が得られていない.
例えば,現在ほとんどの研究で開発・評価に利用されているNTCに関して,飯田らは,
作業者間一致率や作業結果の定性的な分析を踏まえれば,アノテーションガイドラインに少なからず改善の余地があるとしている\cite{飯田龍2010述語項構造}.
また,
我々は,述語項構造アノテーションの経験のない日本語母語話者一名を新たに作業者とし,
KTC,NTCのアノテーションガイドラインを熟読の上で新たな日本語記事に対して
述語項構造アノテーションを行ったが,KTC, NTCのどちらのガイドラインにおいても
付与する位置やラベルを一意に決めることの出来ないケースが散見された.
述語項構造のようにその他応用解析の基盤となる構造情報については,
これに求められる一貫性の要求も高い.したがって,今後,述語項構造の分析や解析器の開発が高水準になるにつれて,
既存のコーパスを対象とした学習・分析では十分な結果が得られなくなる可能性がある.
そのような問題を防ぐためには,
現状のアノテーションガイドラインにおいて判断の揺れとなる原因を洗い出し,
ガイドラインを改善しつつ,アノテーションの一貫性を高めることで,
学習・分析データとしての妥当性を高い水準で確保していく必要がある.


第二に,
より質の高いアノテーションを目指してガイドラインを改善することを考えた場合,
それぞれの基準をどういった観点で採用したかが明確に見てとれるような,論理的で一貫した
ガイドラインが必要となるが,
KTC, NTCなどの既存のアノテーションガイドライン\cite{ntcguideline,ktcguideline}や
関連論文\cite{kawahara2002construction,河原大輔2002関係,iida2007annotating,飯田龍2010述語項構造}を参照しても,
個々の判断基準の根拠が必ずしも明確には書かれていない.
典型的に,アノテーションガイドラインの策定時に議論される内容は
コーパス作成者の中で閉じた情報となることが多く,その方法論や根拠が明示的に示された論文は少ない.
このため,付与すべき内容の詳細をどのように考えるかという,
アノテーションそのものの研究が発展する機会が失われているという
現状がある.
また,KNBCやBCCWJ-PASのように既存のガイドラインに追従して作られるコーパスの場合,
新規ドメインに合わせるなど一部仕様が再考されるものの,
アノテーションの研究は一度おおまかにその方向性が決まってしまうと,
再考するための情報の不足もあり,本質的に考えなければならない点が据え置かれ,
さらに詳細が議論されることは稀である\footnote{公開されているガイドラインを確認する限りでは,KNBC作成時には格関係に関するガイドラインは再考されていない.
BCCWJ-PASの仕様は,機能語相当表現の判別に辞書を用いる点と,ラベル付与の際に既存の格フレームを参照する点をのぞいて,NTCの仕様とおよそ同等である.}.

そこで,本研究では,この二つの問題を解消するために,
既存のコーパスのガイドラインにおける相違点や曖昧性の残る部分を洗い出し,
どのような部分に,どのような理由で基準を設けなければならないかを議論し,その着眼点を明示的に示すことを試みた.
具体的には,
(i)既存のガイドラインに従って新たな文章群へあらためてアノテーションを行った結果に基づいて
議論を行い,論点を整理したほか,
(ii)新規アノテーションの作業者,既存の述語項構造コーパスの開発者,
また既存の仕様に問題意識を持つ研究者を集め,
それぞれの研究者・作業者が経験的に理解している知見を集約した.
(iii)これらをふまえ,述語項構造に関するアノテーションをどう改善するべきか,
どの点を吟味すべきかという各論とともに,
アノテーション仕様を決める際の着眼点としてどのようなことを考えるべきかという議論も行った.
本論文ではこれらの内容について,それぞれ報告する.

次節以降では,まず,\ref{sec:related_work}~節で述語項構造アノテーションに関する先行研究を概観し,
\ref{sec:ntc}節で今回特に
比較対象としたNAISTテキストコーパスの述語項構造に関するアノテーション
ガイドラインを紹介する.
\ref{sec:how-to-discuss}節で研究者・作業者が集まった際の人手分析の方法を説明し,
\ref{sec:individual}~節で分析した事例を種類ごとに紹介する.さらに,
\ref{sec:framework}~節で,述語項構造アノテーションを通じて考察した
,アノテーションガイドライン策定時に考慮される設計の基本方針について報告
し,\ref{sec:individual}~節で議論する内容との対応関係を示す.
最後に\ref{sec:conclusion}~節でまとめと今後の課題を述べる.


以降,本論文で用いる用語の意味を以下のように定義する.
\begin{itemize}
\item アノテーション仕様:
どのような対象に,どのような場合に,どのような情報を付与するかについての詳細な取り決め.
\item アノテーションスキーマ:
アノテーションに利用するラベルセット,ラベルの属性値,及びラベル間の構造を規定した体系.
アノテーション仕様の一部.
\item アノテーションフレームワーク:
アノテーションにおいて管理される文章やデータベースの全体像,及び
アノテーション全体をどのように管理するか,どのような手順で作業を行うかなどの運用上の取り決め.
\item アノテーションガイドライン:作業の手順や具体的なアノテーション例などを含み,
実際のアノテーションの際に仕様の意図に従ったアノテーションをどのようにして実現するかを細かく指示する指南書.
\item アノテーション方式:特定のコーパスで採用される仕様,スキーマ,フレームワークのいずれか,もしくはその全体.
\item アノテーション基準:あるラベルやその属性値を付与,あるいは選択する際の判断基準.
\item アノテーション規則:アノテーション基準を守るべき規則として仕様やガイドラインの中に定めたもの.
\end{itemize}


\section{関連研究}
\label{sec:related_work}

述語項構造を解析したコーパスとしては,
日本語文章に対するものに,
京都大学テキストコーパス(KTC), 
NAISTテキストコーパス(NTC), 
GDAタグ付与コーパス(GDA), 
KTC準拠のアノテーションをブログ記事に対して行った解析済みブログコーパス (Kyoto University and NTT Blog Corpus: KNBC), 
日本語書き言葉均衡コーパス(BCCWJ)に対してNTC準拠のアノテーションを
行ったコーパス(BCCWJ-PAS)などがある.
英語を対象としたコーパスとしては,FrameNet, PropBank, NomBank, OntoNotesなどが主要なものとして挙げられる.
特に,NTC, FrameNet, PropBank, NomBankなどは,比較的多くの文章事例を含むことから,
これまでに,様々な解析器の学習データとして用いられてきた\cite{marquez2008srl,Yoshikawa2011,Iida2011,taira2008japanese}.

\begin{table}[b]
\caption{述語項構造コーパスの比較}
\label{tbl:corpora}
\input{ca12table01.txt}
\end{table}

表~\ref{tbl:corpora}に,各コーパスの特徴を示した.
コーパス間の主な仕様の差としては,文書ドメイン,述語-項関係を表すラベル,格フレーム辞書の有無,
文外の項に関する取り扱いの有無などが挙げられる.

コーパスの文書ドメインは,従来,新聞記事を中心に整備されてきたが,
係り受け解析等のその他の技術同様,教師あり学習によって開発された
述語項構造解析器の精度が学習データの文書ドメインに依存するという結果\cite{Carreras:2005:ICS:1706543.1706571}から,
近年は複数文書ドメインへのアノテーションが進みつつある(BCCWJ-PAS, KNBC, OntoNotesなど).

述語-項関係ラベルとしては,文中の統語的なマーカーを関係ラベルに利用した表層格,
項のより意味的な側面を取り扱った意味役割ラベル等のバリエーションがある
\footnote{
表~\ref{tbl:corpora}の表記では,述語-項の意味的な関係を規定するラベル全般を「意味役割」ラベルと表記したが,
中でも格文法理論から派生し,少数のラベルを用いて述語横断的な項の統語/意味的な性質を
表現する主題役割(thematic roles)については,特に区別して「主題役割」と表記した.}.
既存のコーパスでは,英語のコーパスが意味役割を中心としたアノテーションを行った
のに対して,日本語では表層格を中心としたアノテーションが一般的である.
この違いが現れた理由としては,言語による性質の違いと,
それまでに作成された他のコーパスとの情報の差分の違い,という2点が挙げられる.
日本語においては,項の省略が頻繁に起こるという性質のほか,
副助詞「は」「も」等が使用されている場合や,連体修飾の関係にある場合など,KTCの文節単位の係り受け情報だけからでは表層的な格関係
自体が自明でない場合があるため,
述語とその項となる句の位置関係や表層的な格関係を明らかにすることが第一の目標とされた.
一方で,英語の場合,項の省略のほとんどはto不定詞や関係節,疑問詞などの統語的な性質に基づいた
移動によって説明でき,また,この移動には,
句構造にもとづいて統語的アノテーションを行ったPenn Treebank
コーパス~\cite{marcus1993building,marcus1994penn}
においてtraceというラベルを用いて述語項構造相当のアノテーションがなされており,
実際の項の位置や,移動前における統語関係が既に明らかにされていたことから,
述語が表現する概念におけるそれぞれの項の意味的な役割を表現するラベルを
アノテートすることが次の段階の目標となったと考えられる.

日本語の述語項構造アノテーションの主要なコーパスであるKTCとNTCでは,
日本語の統語上の格関係マーカーである格助詞を関係ラベルとして利用している.
KTCでは,述語が現れた時,その述語が伴っている助動詞・補助動詞等を含めた形(述部出現形)に対して
用いる格助詞を利用して,項にラベルを付与する.
\eenumsentence{
\item $[太郎_{ガ}$]が[本$_{ヲ}$]を\underline{買う}。
\item $[この本_{ヲ}$]は[太郎$_{ニ}$]に\underline{買ってほしい}。
}
上の例では,下線部が述語表現,[ ]括弧で囲まれた部分が項,その内部の下付き文字が格関係ラベルを表す.
以降,特に断りのない限りは,例文での項構造はこのように表す.
一方で,NTCでは,述語の原形に対して用いる格助詞を使って
ラベルを付与する.
\eenumsentence{
\item $[太郎_{ガ}$]が[本$_{ヲ}$]を\underline{買う}。
\item $[この本_{ヲ}$]は[太郎$_{ガ}$]に\underline{買っ}てほしい。
}
この方法は,使役・受身・願望など,格の交替が起こる表現の間で格のラベルを正規化することで,
表層格に主題役割のようなより意味機能的な側面を持たせることを試みたものと捉えることができる.
ただし,\ref{sec:ntc-case-ktc-case}節でも述べる通り,この二つについては
他方には含まれない情報をそれぞれ持っており,
どちらの方式がより適切かはアプリケーションによっても異なるため,一概に優劣を決めることは出来ない.
述部出現形アノテーションにおける格交替の情報を補う研究として,
自動的に収集された出現形の格フレームの間で格ラベルの交替がどのように起こるかを自動的に対応付ける研究\cite{sasano2013}も試みられている.

英語に対する主要なコーパスでは,述語と項の間のより詳細な意味関係をとらえる
「意味役割ラベル」が用いられる.これは,例えば,
同じ意味機能を持った項が異なる統語関係として表れる統語的交替と呼ばれる
現象に対して,それぞれの項に一貫した意味的役割を割り当てたり,
Agent, Theme, Goalなどの主題役割(thematic roles)のように,
項の述語横断的な意味機能を扱いたい場合に有用である.
また,日本語でのアノテーションではあまり取り扱いのない,必須格と周辺格の区別についても扱っている.

ただし,意味役割によるアノテーションスキーマでは
項の役割を表すラベルの数が数十から数千という規模になり,
意味の類似するラベルも多種存在するのが一般的であることから,
ガイドラインにおいて類似ラベルの取り扱いを明確に区別したり,
あるいは述語の語義ごとに格フレーム情報をあらかじめ作成し,
各語義で項として取り得るラベルの選択肢を厳密に定めることによって
曖昧な選択肢が生じないように工夫を行う必要がある.

既に構築が完了している
日本語のコーパスでは,唯一GDAが主題役割を取り扱っているが,アノテーション対象が文外のゼロ照応関係にある項に絞られており,
述語項構造に見られる現象を網羅しているとは言い難い.
NTCの表層格アノテーションでは,
述語-項関係を述語が原形の場合の格関係に正規化するため,格助詞と述語とその語義の三つ組を考えれば,
この三つ組は各述語の各語義に固有の意味役割を考えるPropBankやFrameNetとおよそ同等の意味表現となる.
ただし,主題役割のような述語横断的な意味機能については考慮できない.

一方で,近年では,日本語に対する新たな意味役割アノテーションの試みも進みつつある.
現状では一致率や規模の問題から言語処理研究への実用レベルには至っていないものの,
小規模な日本語文章への主題役割の試験的な付与例として,
林部ら~\cite{hayashibe2012}やMatsubayashi et al.~\cite{MATSUBAYASHI12.941}の研究が挙げられる.
林部らの研究では,作業者間一致率がF値で67\%前後と低く,実用に至っていない.
Matsubayashi et al.の研究では,あらかじめ述語ごと,語義ごとの格フレームを用意するため
必須格に対する一致率は$91\%$と高いが\footnote{Matsubayashi et al.の研究では,文外の項に対するアノテーションを行っていない点に注意されたい.},アノテーションに必要となるフレーム辞書のサイズが未だ小さく,規模を拡充する必要がある.
開発過程にある意味役割付与コーパスとして,BCCWJに対し,
動詞項構造シソーラスを用いた意味役割アノテーションを行う研究~\cite{takeuchi2013}や,
同じくBCCWJに対し,FrameNetと同様の理論的枠組を利用して意味フレームの
アノテーションを行う研究\cite{ohara2013}などが進んでいる.

英語のコーパスでは,それぞれの述語が取り得る格を列挙した格フレーム辞書と呼ばれる資源を構築するのが一般的な手法である.
格フレーム辞書は,大規模な生コーパスの観察によりアノテーション作業に先立って構築される.
アノテータは格フレーム辞書を参照しながら項構造の付与
を行うことによりアノテーションの揺れを抑えることができるため,
高い作業者間一致率を得ることができる.
日本語の場合,英語に比べて項の省略が多く,また,英語のコーパスでは行っていない文をまたいだ項のアノテーションを行っているなど,
アノテータが確認しなければならない領域が相対的に広いため,英語の場合との一致率の単純な比較は出来ないが,
PropBankの項アノテーションに関する一致率は周辺格を含める場合でkappa値で0.91,
含めない場合で0.93と極めて高い\cite{palmer2005pba}.
また,含意関係認識タスクのためにFrameNet準拠のコーパスアノテーションを行った研究では,意味役割の
付与に関する
一致率が$91\%$であったとしている\cite{burchardt2008fate}.
これに対して,明示的な格フレーム辞書を持たないNTCでは,一致率が$83\%$前後と相対的に低い.
KTCでは,ガイドラインを安定化させた段階での格関係アノテーションの作業者間一致率を$85\%$と報告している\cite{河原大輔2002関係}.
NTCの仕様に準拠する形でBCCWJに対するアノテーションを行った研究では,アノテータが既存の格フレーム辞書を参照しながら
作業を行うことによって作業者間一致率に一定の改善を得ることが出来たとしている\cite{komachi2011}.

日本語コーパスの初期のアノテーションにおいて,英語コーパスであらかじめ整備された格フレームが
用いられなかった理由としては次の2点が挙げられる.
第一に,英語のコーパスで行われた意味役割を用いたアノテーションでは,
項のラベルとして統語機能的なラベルを用いず,
純粋に項の持つ意味そのものを表現するラベルを用いたため,
それぞれの述語が取る項の数やその意味役割を明示的に記述する必要があったのに対し,
日本語の場合は格助詞を関係ラベルとして採用することで,ラベルセットが少数のラベルで規定されるので,
明示的に述語ごとのラベルセットを列挙する必然性がなかったことが挙げられる.
このため,初期のアノテーション作業として,格フレームを記述するためのコストとのバランスを考慮して,
格フレームを用意せずに作業が進められたことはきわめて自然なことであった.
第二に,日本語では項の省略
が頻繁に起こるため,
統語的な文構造の制約が強い英語の場合に比べて格フレームの分析が難解となっていることが挙げられる.
日本語の述語に対して表層格の格フレーム情報を与える既存の言語資源としては
NTT語彙大系・構文体系の辞書\cite{nttlexicon}や計算機用日本語基本動詞辞書IPAL \cite{ipal}, 
竹内らの動詞語彙概念辞書\cite{takeuchi2005}, 
京都大学格フレーム\cite{kawahara2006case}などがあるが,いずれも異なった格フレームを
与えており,またNTC開発における実際のアノテーション作業時には
既存の格フレーム辞書では被覆されない格が出現するなどの問題があった.
このため,日本語においては精緻な格フレーム辞書を構築する手段についても研究課題の一つとなっている.


英語を対象としたコーパスにおいては,一般に,文をまたいだ項についての取り扱いがない.
これは,日本語が項の省略を頻繁に伴うのに対して,英語における項の省略が
比較的少ないことに由来する.しかし,英語の文章においても,イベント間の照応関係や推論的解釈により,
同一文中には現れないが暗黙的に定まっている項があると解釈される場合もあるため,
近年は,この問題を解消するための試みも研究されている\cite{laparra2013impar,frankpredicate,Silberer:2012:CIR:2387636.2387638}.

また,多くのコーパスでは,名詞についてもその項構造が考慮されている.
NTCでは,名詞のうち一般の述語で表されているような状態やイベントを表現するもの
~\cite{noun2008}(本論文中では,これをイベント性名詞と呼ぶ)について,他の述語と同様に項構造を割り当てている.
KTCやNomBankでは,イベント性名詞に加えて,ある名詞の意味解釈をするにあたって
その名詞の意味の中に取り込まれていない
別の何らかの概念との関係が必須であるもの,いわゆる非飽和名詞~\cite{nishiyama2003}についての項(\ref{enum:ktc-no-a})や,所有の関係,修飾の関係など,二つの名詞間に何らかの関係が成り立つ場合もラベル付与を行っている(\ref{enum:ktc-no-b})(\ref{enum:ktc-no-c}).
\eenumsentence{
\item $[米国_{ノ}$]の\underline{大統領}(KTC.「大統領」は非飽和名詞.「ノ格」のラベル付与)\label{enum:ktc-no-a}
\item $[花子_{ノ?}$]の\underline{眼鏡}(KTC.非飽和名詞以外の関係.「ノ?格」のラベル付与)\label{enum:ktc-no-b}
\item the [vice$_{ARG3}$] [\underline{president}$_{ARG0}$] of [North America operations$_{ARG2}$] (NomBank)\label{enum:ktc-no-c}
}


\section{NAISTテキストコーパス}
\label{sec:ntc}

我々は,可能な限り多くの現象を網羅した分析を行うという観点から,
これまでに最も多くの文数にアノテーションが行われてきたNTCの仕様をベースとし,
適時KTCとの対比を行いながら議論を進める方針とした.
本節では,NTCのアノテーションガイドラインについて,本論文の理解に必要な範囲の内容を簡単に説明する.
また,\ref{sec:ntc-iaa}~節では,NTCの作業者間一致率について,我々があらためて詳細に分析した結果を述べる.

一般に日本語述語項構造アノテーションを行うにあたって同時に含まれる照応・共参照情報については,
それ自体が難解な問題を多く含んでおり,
加えて,種々の問題を包括的に考慮して議論を進めなければ解決は難しいと判断した.
このため,照応・共参照アノテーションに対する考察・理論化は
一つの大きな研究テーマに相当するものであると考え今後の課題とし,議論の対象外とした.


\subsection{アノテーションガイドライン}
\label{sec:ntc-guideline}

ここでは,NTCのガイドラインについて,
公開されているWebサイト\cite{ntcguideline}の情報を抜粋・再編集する形で概要を説明する.
照応・共参照や名詞間関係に関わる部分については本論文での議論の対象外とするため説明を省略するが
\footnote{
NTCにおいて名詞間関係を表す「ノ」や「外の関係」はNTC 1.5版に含まれておらず,
付与事例が検証できなかったため,議論の対象外としている.},
ガイドラインの全容についてはWebサイトを参照されたい.
また,より詳細な内容については,必要に応じて\ref{sec:individual}~節での個別の議論の際に付け加える.
ただし,
同Webサイトの内容は,ガイドライン開発過程の情報が入り混じっており,
必ずしも公開版データ\footnote{NTC 1.5版をさす.}の作業時の規定を反映していな
かったため,
文書化されたガイドラインと公開版のデータに相違が見られる点については,
NTCの開発者に確認し,実際の作業がどのようなものであったかを説明に追加した.
また,表~\ref{tbl:difference}にNTC 1.5版とWeb上のガイドラインにおける差異を対応表としてまとめた.

\begin{table}[b]
\caption{NTC 1.5版とWeb上のガイドラインにおける仕様の差異(共参照・照応・名詞間関係を除く)}
\label{tbl:difference}
\input{ca12table02.txt}
\end{table}

\ref{sec:problem-collection}~節の論点収集のプロセスでは,
Web上のガイドライン及びここで示す実際のNTC 1.5版との差異において,
明文化された規定のない項目については曖昧な取り決めであるという
立場をとり,このうち簡潔な規則を定めることで問題を解決できなかった部分についてを
\ref{sec:individual}~節で議論する.

NTCでは,(i)動詞,形容詞,名詞句+助動詞「だ」,ならびに(ii)
サ変動詞や『名詞句+助動詞「だ」』の体言止め,
(iii)ナ形容詞の語幹で文や節が終わる場合
を述語とみなし,対象表現に述語ラベルを付与する.
さらに各述語について,
その項構造を述語原形に対する表層格ラベルを用いて付与する.
また,イベント性の名詞についても述語同様の項構造を考えアノテーションを行う.
\eenumsentence{
\item $[太郎_{ガ}$]が[花子$_{ニ}$]に[リンゴ$_{ヲ}$]を\underline{あげ}た。
\item 本日未明に$[竜巻_{ガ}$]が\underline{発生}、(サ変動詞の体言止め)
\item $[県_{ガ}$]の現在の一般事務[職$_{ヲ}$]の\underline{採用}は日本国籍が要件。(イベント性名詞)
}

項は,必須格\footnote{ただし,NTCガイドラインWeb版では必須格と周辺格の区別の方法を示してはいない.}であるもののうちガ・ヲ・ニ格に相当するもののみを扱う.項や述語
の領域は,IPADIC~\cite{ipadic}で定められる形態素分割における一形態素とする.
項のスコープが句や節の場合は,最も後ろの形態素を
項の範囲とする\footnote{NTCでは,項のスコープとして句や節を考慮に入れるが,
句や節の範囲は文構造から特定出来るものとして,最も後ろの形態素のみをラベルの範囲とした.}
.
述語が「名詞+する」のサ変動詞の場合や名詞句+「だ」の場合は
複数形態素から構成される述語と解釈するが,ラベルを付与する箇所は一形態素とし,
サ変動詞の場合は「名詞+する」の「する」に,
名詞句+「だ」の場合は名詞句の最も後ろの形態素に述語ラベルを割り当てる.

\eenumsentence{
\item 彼が来たかどう[か$_{ヲ}$] \underline{知り}たい。
\item $[A社_{ガ}$]は[新型交換器$_{ヲ}$]を導入\underline{する}。
\item 彼とお茶\underline{する}。
\item $[太郎_{ガ}$]は九州\underline{男児}だ。
}

機能語相当表現
については述語とはみなさない.
同様に,形容詞の副詞的用法,固有表現内の述語も述語とみなさない
\footnote{NTCガイドラインWeb版には
転成名詞(ガイドライン上では「動名詞」との記述)についても述語とみなさない
とあるが,実際にはNTC 1.5版ではイベント性名詞としてアノテーションがなされている.また,
機能動詞についても
アノテーション対象とみなさないとあるが,実際には機能動詞の認定が難しいとの判断から,
通常の述語と同様のラベル付与が行われた.}
(下線部は述語ラベルを付与しない箇所).
\eenumsentence{
\item 彼の話に\underline{よる}と、(機能語相当表現)
\item 本を買って\underline{しまう}。(機能語相当表現)
\item 彼にリンゴを食べて\underline{ほしい}。(機能語相当表現\footnote{ただし,格が新たに追加される補助動詞に該当.})
\item 点の取り方を\underline{よく}知っている。(形容詞の副詞用法)
\item 野鳥を\underline{守る}会。(固有表現)
}

受身,使役などの場合は述語原形の格を付与する.ただし,これらの格交替によって原形の場合は取らなかった格が
新たにガ・ニ格として増えている場合は,述語に付随する助動詞や補助動詞を仮想的な述語とみなし,
そこに追加ガ/ニ格などの格を割り当てる\footnote{ただし,追加ガ/ニ格はNTC 1.5版には含まれていない.}.
\eenumsentence{
\item $[私_{追加ガ@B}$]は[父$_{ガ@A}$]に\underline{死な$_{A}$} \underline{れ$_{B}$}た。
\item $[私_{追加ガ@B}$]は[彼$_{ガ@A}$]に[リンゴ$_{ヲ@A}$]を\underline{食べ$_{A}$} \underline{させる$_{B}$}。
}

項が省略されている場合は,文章中から対象の項を探しラベルを付与
する.文章中に候補となる句や節が存在しないが
何らかの項が埋まっていると認識できる場合は,
外界(一人称),外界(二人称),外界(一般)という三つの特別な記号を用意し,そこに項
のラベルを割り当てる.照応先が一人称単数の場合は「外界(一人称)」,二人称単数の場合は「外界(二人称)」,
それ以外の場合は全て「外界(一般)」の記号にラベルを割り当てる.

\eenumsentence{
\item $[牡蠣_{ヲ}$]を\underline{食べる}ため、[太郎$_{ガ}$]は広島へ行った。(項の省略)
\item $[\phi_{外界(一人称)ガ}$] そろそろ\underline{帰ろ}うと思う。(外界照応)
}
二重に主語を取る構文においては,
「N1はN2がV」を「N1のN2がV」として置き換えることが可能な場合は「ノ格」で付与,
それ以外の場合は「ハ」と「ガ格」を用いて付与する\footnote{ただし,ハ,ノ格は
NTC 1.5版には含まれていない.}.
\eenumsentence{
\item $[広島_{ノ}$]は[牡蠣$_{ガ}$]が\underline{うまい}。
\item $[太郎_{ハ}$]が[花子$_{ガ}$]が\underline{好き}だ。
\item $[彼_{ハ}$]が[英語$_{ガ}$]が\underline{読める}。(可能動詞)
}

項が並列構造を取る場合には,以下の例文のとおり 1 形態素に限ってラベルを付与する\footnote{Web版のガイドラインでは,並列構造内に現れるそれぞれのエンティティを個別に扱い,複数の項としてアノテートするとしているが,NTC 1.5版ではアノテーション対象を述語に最寄りの項のみに限っている.}.
(\ref{enum:coordination-a})の場合は,「太郎と次郎」という名詞句が項であるとみなし,
前述の「項のスコープが句や節の場合は,最も後ろの形態素をラベルの範囲とする」という規則に従い,「次郎」をラベルの範囲とする.
(\ref{enum:coordination-b})の場合は,「中学校」と「高校」がそれぞれガ格とみなせるが,述語に最も近い項のみラベルを付与する
((\ref{enum:coordination-c})も同様).(\ref{enum:coordination-d})のように「と」が「が」よりも後に出現する場合,及び他の項をはさみ離れて出現する場合は
並列構造とは区別し,ガ格とはみなさない.
\eenumsentence{
\item 太郎と[次郎$_{ガ}$]が\underline{遊ん}でいた。\label{enum:coordination-a}
\item 中学校は四割、[高校$_{ガ}$]も三割\underline{あっ}た。\label{enum:coordination-b}
\item 太郎はリンゴを、[次郎$_{ガ}$]は[オレンジ$_{ヲ}$]を\underline{食べ}たい。\label{enum:coordination-c}
\item $[太郎_{ガ}$]が花子と結婚\underline{し}た。\label{enum:coordination-d}
}

述語が連体修飾をする場合において,被連体修飾句と述語との関係を格助詞を用いて表現できない場合は
「外の関係」のラベルを付与する\footnote{ただし,「外の関係」はNTC 1.5版には含まれない.}.
\enumsentence{
$[サンマ_{ヲ}$]を\underline{焼く} [けむり$_{外の関係}$]
}


\subsection{作業者間一致率}
\label{sec:ntc-iaa}

ガイドラインの分析に先立ち,我々は,飯田ら\cite{飯田龍2010述語項構造}が用いたものと同一のデータを用いて,
NTCの作業者間一致率を更に詳しく分析した.その結果を表\ref{tbl:ntc_iaa}に示す.
一致率は,二名の作業者が30記事にアノテートした結果について,一名の結果を正解,もう一名の結果をシステムの推定と仮定した場合の
適合率,再現率,F値として算出した.
このとき,推定されたトークンが正解データにおいて項となる共参照クラスタの中のいずれかのトークンと一致すれば
正解とした\footnote{各作業者が
アノテートした共参照クラスタが異なるため,表\ref{tbl:ntc_iaa}は作業者二名のうちどちらを正解と見なすかによって
僅かに結果が異なるが,どちら側からもおよそ同じような結果となったため,片側だけを記載した.}.
ただし,我々の評価方法では,飯田らの方法と異なり,述語やイベント性名詞の位置が不一致の場合は,
それらに付与された全ての項を不正解とした.

分析は,格ごとに,係り受け関係の有無,述語・イベント性名詞の別に分けて行った.
係り受け関係がない場合とは,すなわち,本来統語的な関係として規定されるはずの項が省略される
ゼロ照応と呼ばれる現象が現れていることを指す.
結果として,格ごと,または係り受け関係の有無によって一致率にかなりのばらつきがあることが分かった.
特に,ゼロ照応を伴う事例では,格の種類横断的に一致率が低い.
顕著に低い値を示すのはゼロ照応のヲ格・ニ格,及びイベント性名詞に関するニ格であるが,
これらは事例数自体が少ないため,この結果がガイドラインの不備によるものかどうかを確かめるには
あらためて事例を収集し検証する必要がある.

\begin{table}[t]
\caption{NAIST テキストコーパスの作業者間一致率}
\label{tbl:ntc_iaa}
\input{ca12table03.txt}
\end{table}



\section{論点の収集方法}
\label{sec:how-to-discuss}
\label{sec:problem-collection}

本節では,既存コーパスのガイドラインにおける問題点を洗い出すために我々が取った方法を説明する.
ガイドラインの問題点を収集するための具体的な方法論は
確立されていないため,今回は(i)既存のガイドラインを利用して新規アノテーションを行い,曖昧な箇所を探るという方法と,
(ii) NTC・KTCの仕様策定,NTC, KTCを用いた応用処理に関わった研究者,
述語項構造アノテーションの仕様に対して問題意識を持つ研究者が経験的に持つ知見を集約するという方法の二つの方法を取った.

前述のとおり,本論文で取りまとめる考察はNTCのアノテーションガイドラインを基準に行う.
ただし,議論上関連のある項目についてはKTCのガイドラインとの対比を取り,より広範囲に考察を加えられるよう努めた.
また,NTCやKTCのガイドラインにおいては,アノテートする文書ドメインが限定されていることにより
認知されなかった問題がある可能性も否定出来ないため,
今回の論点収集の過程では新聞ドメイン外の文に新たにアノテーションを行うことを試みた.
議論の対象となる題材は,述語項構造アノテーションの経験がない一般人の日本語母語話者1名,及び
NTC・KTCの仕様策定関係者3名と
述語項構造アノテーションの仕様に対して問題意識を持つ言語処理研究者5名
(著者ら8名)の計9名によって具体的に以下の手順で収集した.

\begin{enumerate}
\item 述語項構造アノテーションの経験がない日本語母語話者1名を新規アノテーションの作業者とする.
作業者にはNTCのアノテーションガイドラインを熟読してもらい,その後,基本的なアノテーション方法について指導を行う.
\item Wikipedia, BCCWJよりサンプリングした文書に対して,NTCのガイドラインに従い,作業者が述語項構造を付与する.
判断に迷いが出た事例は取りまとめて著者らに報告する.
\item 報告された事例について,著者らがNTC・KTCのガイドライン及びNTCデータ内の実際のアノテーション例と照らし,
簡潔に解決可能かどうか確かめる.解決可能な場合,ガイドラインを更新し,解決案の説明と具体例を加える.
解決不可能なものは議論対象の分類表に加える.このとき,NTCとKTCの間での取り決めの対比も行う.
\item 作業者は新しいガイドラインと未解決問題の分類表を持ち,作業済みのデータを修正する.$1,000$文程度になるまで新しい文章セットを受け取り(2)に戻る.
\item NTC・KTCの仕様策定に関わった研究者,既存の仕様に問題意識を持つ研究者ら計8名(著者ら)の意見を集約し,
研究者が経験的に理解している仕様上の改善点を,(1)〜(4)の工程で出来た議論対象の分類表に追加する.
また,新たに用意したBCCWJ上の記事20記事程度\footnote{コアデータ内の,
書籍,雑誌,白書,Yahoo!知恵袋,Yahoo!ブログをドメインとする記事の冒頭10文程度を利用した.
OC01\_00006, OC01\_00472, OC01\_00485, OC01\_01765, OC01\_02071, OW6X\_00007, OW6X\_00009, OW6X\_00016, OY03\_04233, OY03\_04343, OY04\_01354, OY14\_02901, PB13\_00021, PB14\_00016, PB14\_00057, PB19\_00011, PM11\_00031, PM11\_00207,
PM11\_00223, PM11\_00226}
に対して,上記(1)〜(4)の工程で改善したガイドラインを見ながら実際にアノテーションを行い,
問題となった点を議論対象の分類表に加える.
\end{enumerate}

以上の方法で収集・整理した4種15項目の論点(\ref{sec:individual}~節,表~\ref{tbl:topics}を参照)について,
著者らが議論を交わし,結果として得られた知見をまとめ上げた.


\section{個別の論点}
\label{sec:individual}

本節には,\ref{sec:problem-collection}~節の方法によって収集された
ガイドライン策定上の論点に関して,研究者間で議論した結果をまとめる.
まず,我々は収集された問題をおおまかな種類ごとに分別し,結果,4種15項目の論点を得た.
表~\ref{tbl:topics}にその一覧を示す.内容としては,述語の認定基準,格の取り扱い,
格や格フレームの曖昧性の問題といった既存のコーパスに本質的に潜んでいた問題のほか,
新聞ドメイン以外で新たに見られた現象もある.
以下では,それぞれの論点について議論の詳細を記す.

\begin{table}[b]
\caption{述語項構造アノテーションのガイドライン設計に関わる論点}
\label{tbl:topics}
\input{ca12table04.txt}
\end{table}

各論点に対する議論は,著者らが
種々のアノテーションタスクの設計を通して知る経験的な知見にもとづいて行われる.
我々の目的の一つは,これら設計時の基本的な理念とガイドライン上の取り決めの対応関係を
集約することであるので,議論の過程で現れたガイドライン策定上の基本原則については
\ref{sec:framework}~節にあらためて取りまとめる.


\subsection{アノテートすべき述語の認定基準}
\label{sec:pred-dicision}

\subsubsection{述語項構造を重要視すべき述語とそうでない述語}
\label{sec:important-pred}

文章中の述語は,その全ての述語項構造が等しく重要性を持つわけではなく,一部の述語に関しては,
その述語項構造を解析する重要性が低いものもある.
例えば,以下の文,
\enumsentence{
\underline{驚い}ては\underline{い}られない.
}
において,「驚く」は文の内容上その項構造の解析が重要になるが,一方の「いる」のほうは,
より機能的な述語であり,項構造を捉えるというよりはむしろ「てはいられない」という 1 フレーズを
機能的な表現とみなす方が自然と考えられる.
述語項構造そのものを解析する重要度の低い述語に関しては,アノテーションコストの観点からも,
解析器の評価をより重要度の高い項構造だけで適切に行えるようにするという点からも,区別して取り扱いたい.

述語項構造の重要度に関する問題として,本論文では,
\begin{enumerate}
\item[(a)] 複合語
\item[(b)] 機能語相当表現
\item[(c)] 機能動詞構文・格交替を伴う機能表現
\end{enumerate}
を取り上げる.これらは文章中にありふれた事象のため,アノテーションコストに対する影響も大きい.
以下では,それぞれの項目について,どのように取り扱うべきかについての議論結果をまとめる.

\noindent
{\bf (a) 複合語}

以下のように,述語となりうる語の後ろに項が追従する形からなる複合語を考える.
この場合,項自体がその複合語の主辞であるため,これら語の内部に現れる述語と項の意味関係は
そのまま項の意味を修飾する構造となっている.この形では一般に項の部分が単体で持つ語の意味は
それほど重要ではなく,複合語全体のかたまりの意味となって初めて実用的な意味を持つ場合が多いため,
内部構造を分解して解析することの重要度はその他の項構造と比べて低いと考えられる.
\eenumsentence{
\item \underline{作業} [者$_ガ$]
\item \underline{書き} [手$_ガ$]
\item \underline{輸入} [品$_ヲ$]
\item \underline{提案} [手法$_ヲ$]
}
NTCやKTCでは,これらの複合語に関しては,全て内部の項構造をアノテートしているが,
このような表現は出現頻度も高く,アノテーションコストに対して占める割合も高い.
従って,もし応用処理の観点から見て重要度の低い関係とするならば,実際にこのような情報が必要なアプリケーション
からのニーズを待って,後発的にアノテーションを始めるのでも良い.

一方で,次の例文のように,述語部分が主辞となる場合や二つ以上の項を伴う複合語,複合語の外側にも項を取る場合などは,
一般に項が内容語となるため,分解して項構造を考えることに通常と同様の価値があると取れる.
\eenumsentence{
\item $[計算機_{ヲ}]$ \underline{使用} \label{enum:multiwd-outarg-a}
\item $[計算機_{ヲ}]$ \underline{使用} [者$_ガ$] \label{enum:multiwd-outarg-b}
\item $[計算機_{ヲ}]$の\underline{使用} [者$_ガ$] \label{enum:multiwd-outarg-c}
}

ただし,接尾辞などのひときわ判断が容易なものを除いては,
どの複合語の内部の項構造については価値が薄いかを判断することは容易ではないため,
個別に判断することは現状では難しい.
例えば,その代わりに,作業コストを下げ一貫性を保つための工夫として,
複合語内部の項構造がほとんどの場合に一意に定まる事に着目し,
複合語内部の格関係を辞書的に管理しておくことなどが考えられる.
こうすることで,文章中の事例ごとにアノテーションを行う必要がなく作業コストが低下する上に,
アノテーション結果の一貫性も保たれる.
この方法をとった場合,
アノテータは複合語の外側に項が出現する場合のみに対処すればよいことになる.


\noindent
{\bf (b) 機能語相当表現(モダリティ等)}

次の例文の下線部の述語は,助詞相当表現やモダリティ表現の一部と
考えるのが自然である.
\eenumsentence{
\item 彼の話に\underline{よる}と、その店はとても有名らしい。
(格助詞相当表現)\label{enum:fe-yoru}
\item 夏休みの課題で蝉に\underline{つい}て調べた。 
(格助詞相当表現)
\item 気温が上がるに\underline{したがっ}て、だんだんと汗がでてきた。
(接続助詞相当表現)
\item 見つけたと\underline{いっ}ても、これはかなり小さいものです。
(接続助詞相当表現)
\item 驚いては\underline{い}られない。
(モダリティ表現)
\item すぐに食べなければ\underline{なら}ない。
(モダリティ表現)
\item ジムに通うように\underline{なっ}た。
(モダリティ表現)
}
これについて,NTCでは,例えば「通うようになる」の「なる」に対して「機能語相当」のラベルを付けることで
区別し,
述語項構造をアノテートしないとし
ている.
ただし,網羅性を保証できないとの観点から配布版(1.5版時点)には
「機能語相当」ラベルの情報は含まれていない.
KTCでも,複合辞,モダリティ表現は述語認定の対象外としている.

助詞相当表現やモダリティ表現は,内容語の慣用表現(\ref{sec:idioms}節)と同様に,
句として強く結びつくことで非構成的な意味を形成している.
たとえば,(\ref{enum:fe-yoru}) に見られる「によると」は,
このひとかたまりで情報の出所や判断の拠り所を表現する機能を持つ\cite{Morita1989}.
「によると」は文において一つの格助詞のように振る舞うので,
この中の「よる」のガ格が何であるのかを考えるのは不自然である.

上の例文からは,それぞれ,
下線部の述語を含む次のような機能表現を抽出することができる.
\begin{quote}
に\underline{よる}と、
に\underline{つい}て、
に\underline{したがっ}て、
と\underline{いっ}ても、
ては\underline{い}られない、
なければ\underline{なら}ない、
ように\underline{なる}
\end{quote}
機能表現を例外扱いするにあたり問題となるのは,どのような基準で機能表現と
そうでないものを弁別するかということであるが,
これらの機能表現は言語学や言語教育の分野で研究されており,
\cite{Morita1989}や\cite{Jamasi1998}などの辞書が出版されている.
自然言語処理の分野で電子的に利用可能な辞書として,
松吉らが編纂した機能表現辞書\cite{Matsuyoshi2007}などが存在する.
アノテーション作業前に,
これらの辞書を用いてあらかじめ機能表現に印を付け,
ほぼ自動的\footnote{
一部の機能表現に対しては,
機能表現かどうかの曖昧性を解消する必要がある.
例えば,「コンビニに\underline{よる}と、ついお菓子をたくさん買ってしまう。」
の「よる」は,述語項構造解析の対象とすべき述語である.}
に「アノテートすべきでない述語」と認定することにより
作業コストを下げることができる.
辞書には載っていないが機能表現と考えるべき表現を見つけた場合,
作業時にその表現を辞書に追加するなど,
既存の機能表現リストから漏れている表現を拡充することも必要であると考える.


\noindent
{\bf (c) 機能動詞構文・格交替を伴う機能表現}

次の例文に見られるような
機能動詞構文(\ref{enum:functional-a})や授受表現(\ref{enum:functional-b})における下線部bの述語は,
直前の述語aに対して,アスペクトや態,ムード等の意味を付加する機能的な働きをするものと考えられている\cite{matsumoto1996syntactic,村木新次郎1991日本語動詞の諸相}.
\eenumsentence{
\label{enum:functional}
\item 事件が社会に\underline{混乱$_{A}$}を\underline{与える$_{B}$} \label{enum:functional-a}
\item 私が彼にサインを\underline{書い$_{A}$}て\underline{もらう$_{B}$} \label{enum:functional-b}
}
このような述語に対して,下線部AとBの双方の述語項構造を付与することは,構造の重複となり,作業の価値が低い.
また,述語Bに関しては,機能的な振る舞いをするものであるから,述語項構造として取り扱う必要性も低い.
したがって,より内容的意味を持つ述語Aの方を基準の構造とし,Bで追加される意味情報を態・アスペクト・ムードのマーカー
と解釈する方法も考えられる.
これに関し,既存コーパスのガイドラインでは,
NTCでは,機能動詞については通常の述語と同様にラベルを付与し,
一方,「てもらう」などの表現には述語ラベルをアノテートしない,としている.
KTCでは,機能動詞についてはNTCと同様に扱われ,
「てもらう」「てほしい」などの表現は述語の一部としてアノテートされる
(「サインを \underline{書いてもらう}」など).

機能動詞や授受表現を特別に扱う際の問題点は,
機能語相当表現の場合と同様,その表現と取り扱いの方法が網羅的に
列挙できるかという点にある.
機能動詞に関するリストとしては,\cite{izumi2009}などがあるが,現象を網羅するわけではない.
従って,具体的な作業方法の一案としては,上記のようなリストを出発点として,
予め,あるいは作業時に段階的に機能動詞・授受動詞等に関する述語のリストを
作っていき,コーパス中の事例を自動チェックするような仕組みを用いることで,作業を簡素化・半自動化する方法が考えられる.

(\ref{enum:functional})の例でも見られる通り,
これらの表現が使役・受身相当の機能表現の場合は述語Aが本来持つ格に
加えて使役格などの新たな格が追加される場合もある.この場合の取り扱いについては,
\ref{sec:additional-cases}~節と同様の議論となる.


\subsubsection{名詞のイベント性認定}
\label{sec:event-noun-dicision}

サ変名詞,転成名詞に対して,対応する動詞と同等
の項構造をアノテートすることを考える場合には,
その名詞が実際に何かしらの状態やイベントを表しているかどうかが問題となる.
例えば,次のフレーズにおける,「施設」という語について考えてみる.
\enumsentence{
研究\underline{施設}
}
この,「施設」という語はサ変名詞であり,「施設する」という動詞が作れるが,
ここで「研究施設」は施設した結果物であり,イベントではない.このような語にも
便宜的に述語項構造を割り当てることはできるが,文脈上イベントとして解釈できない語に関して,
イベントとしての項構造を付与
することは本質的ではない.むしろ,イベントとして解釈される
「施設」と,そうでない「施設」を区別することのほうが重要といえる.

NTCでは,イベント性名詞ともなりうるタイプの名詞
に関して,イベント性を持たないことを表示するための
ラベル(結果物/内容,もの,役割,ズレ)
を用意しているが,
明瞭な判断基準が存在せず,イベント性の判定は内省に頼っているのが実情である.
ここでの論点は,どのような基準を設ければ
名詞のイベント性をより明確な方法で判別できるかということである.
あるいは,明確な基準を設けることが不可能であっても,閉じたデータ内においては一貫性を保つような
方法を模索する必要がある.
この問題については,既存研究で詳細な分析がなされており,
アノテーションスキーマの改善も実施されているものの\cite{飯田龍2010述語項構造},
ガイドラインとしての整備が行われていないため,再度事例を収集し,問題を整理する必要がある.

我々の議論の中では,複合語と同じように,このような語が出現する度に
それぞれの語が結果物/内容,もの,役割,ズレのいずれのラベルと共に出現したかを記す
チェックリストに追加しておき,
アノテーション時に自動的に注意をうながす仕組みを用意することで一貫性を高めるという方法が挙がった.
また,このチェックリストを利用し,コーパス中の事例を収集してガイドライン策定の
検討材料とすることも考えられる.


\subsubsection{述語が複合語である場合の分解}
\label{sec:complex-word-decomposition}

\begin{table}[b]
\caption{IPA辞書,JUMAN辞書,UniDicによる形態素分割の違い}
\label{tbl:morpheme}
\input{ca12table05.txt}
\end{table}

NTCでは,述語は基本的に一形態素
の範囲に対してラベルを付与
するとしているが,
形態素の分割基準は既存の形態素辞書を拠り所にするため,
どのような辞書を使うかによって述語単位の取り扱いが大きく異なってくる.
表\ref{tbl:morpheme}には,いくつかの複合語についてIPA, JUMAN, UniDic辞書
に基づく形態素分割の差を示したが,辞書によって,あるいは単語によって
分割の位置は異なる.

このような語の扱いに関しては,次の二点が問題となる.
(1) どのような形態素分割基準を基準とするのが述語項構造を考える上で最も適切か,
(2) ある形態素分割基準に基づいて複合語が二形態素以上に分割されたとき,
複合語内部の述語はその全てがアノテーション対象として適切かである.
しかし,どちらの問題も現状で合理的結論を出すことは簡単ではない上,
\ref{sec:criteria}~節に述べるように,言語処理アプリケーションによっては,
どの単位を述語として扱うのがよいか,また,どの程度複合語内部の項構造が必要となるかに異なりがある.
例えば,含意関係認識タスクにおいては,表\ref{tbl:morpheme}の「立ち読み」や「消し 忘れ」が
どのような理論に基づいて分割されているかにかかわらず,「私が、立って、本を、読む」ことや
「私が、ライトを、消そうとして、消すのを、忘れる」ことを理解する必要がある.

したがって,現状で完全な解決策を提示することは難しい
が,部分的な対処案として,複合語の辞書的なアノテーション管理
を考えることができる.
例えば,まずは
ある特定の形態素分割辞書に依存して述語範囲の認定を行い,その上で
\ref{sec:important-pred}~節の複合語の項目で述べたような複合語内部の項構造を辞書的に管理するのと同様の
方法を必要に応じて一形態素と認識されている語に対しても適用することで,
どのような形態素分割基準を用いた場合でも想定するアプリケーションの要求に対応できる
柔軟な構造を取るという方法が考えられる.
なお,複合動詞に関する述語項構造の具体的な分析例として,複合動詞用例データベース\cite{yamaguchi2013}が分析の出発点として参考にできる.


\subsection{格の取り扱い}
\label{sec:case}

\subsubsection{ニ格の「必須格」性}
\label{sec:ni-imperativeness}

述語のそれぞれの項を,主題役割のような意味役割のレベルで考えると,
「が」「を」に比べて,助詞「に」を伴って出現する述語-項の関係には
様々なものがある\cite{contemporaryJapanese2,muraki:84}.
このうち,初期段階の述語項構造アノテーションとして特別重要度が高いのは,
アノテーション対象の述語そのものの概念を説明するために必須となる項目(必須格)である.
一般に,助詞「に」を伴って出現する述語の項のうち必須のニ格とみなされるのは,
動作による移動の着点や結果状態を表すものなどである.
一方,状態やイベントが起こる時間,動作や変化の様態などを表す「に」は述語横断的に利用可能な付加的修飾要素であるため,周辺格などと呼ばれる.

しかし,「が」「を」に比べて,ニ格では必須格性の判断が容易ではないケースも多い.
本論文では,特に,
\begin{enumerate}
\item[(a)] 必須格と周辺格の境界
\item[(b)] ニ格の任意性
\end{enumerate}
の二つについて取り上げる.

\noindent
{\bf (a) 必須格と周辺格の境界}

例えば,次の例,
\eenumsentence{
\item 二つに 割る \label{enum:ni-a}
\item こなごなに 割る \label{enum:ni-b}
\item めちゃくちゃに 割る \label{enum:ni-c}
}
を見ると,(\ref{enum:ni-a})では,ニ格は動作の結果状態を表しているように見えるが,
(\ref{enum:ni-b})や(\ref{enum:ni-c})のような表現になると,それが結果状態を指すのか,
動作(あるいは変化)の様態を指すのかは極めて曖昧になり,判断が難しくなる.
必須格と周辺格の区別については,明確な基準を持って分けられる事例もあれば,
上記のようにどちらに属するとも言えない曖昧な事例も存在する.
アノテーションを行う際に本質的に問題にしなければならないことは
(i) 理論上どのようにアノテートするのが合理的かということと,
(ii)揺れなく,明確にアノテーションや評価が行える基準を設けなければならないということである.
(i)の観点から言えば,もし上記のように必須格と周辺格の間の境界が本質的に曖昧なのであれば,
曖昧な状態を取り扱うことのできる表現にしておけば良い.
一方で,アノテーションや評価を行う場合は不確かなものは問題となる.
少なくとも,どの事例に関しては明確に区別可能であり,
どの事例が本質的な曖昧さを含むのかを明らかにしておかなければ,
作業者間一致率や解析システムの評価時に,
アノテーションやシステムの誤りであるのか,本質的な曖昧性のために揺れているのかを区別できない.

この問題を解消するための方法として,
ラベルの定義の問題でアノテータがいずれか一つのラベルを明確に選べない事例に対しては,ラベルの解釈に
迷ったことを示すマーカーを
用意し,対立候補と共にチェックをしてもらうことで明確な事例と曖昧な事例を区別しておく
方法が考えられる.
そうすることで,評価用データとして用いる際も,該当する事例を除外するなどしてより厳密な評価を行うことができるようになる.
また,学習に用いる際には付与されたラベルの一貫性を担保したい場合が
あると考えられるが,曖昧な事例があらかじめチェックされていれば,その部分はアノテータの判断に
かかわらず機械的に一方のラベルに修正したうえで学習するなどの処理を行うことができる.

\noindent
{\bf (b) ニ格の任意性}

第二に,文章中に存在しないニ格を補う場合の問題がある.
ある格が必須格だと判断した場合,それはすなわち,
仮にその格を埋める項が文章中に存在しない場合でも,
概念上は項が存在しているとみなすということである.
しかし,必須格と周辺格を一般によく知られている意味機能的な役割で分類しようとすると,
動作の結果状態のように,一般的には周辺格ではないと認識されている役割であっても,
述語によっては項が埋められている必要がある(暗に省略されている)と感じにくいケースもある.
\eenumsentence{
\item 信号が ($\phi$ニ) 変わったので、停車した。\label{enum:optional-ni-a}
\item 花瓶を ($\phi$ニ?) 割った。 \label{enum:optional-ni-b}
\item ボールが ($\phi$ニ?) 落下する。\label{enum:optional-ni-c}
}
例えば,(\ref{enum:optional-ni-a})では,信号が変わった結果の状態について,
文脈から何かしら明確な項を仮定する(赤に変わった,と仮定する)のが普通であるが,
(\ref{enum:optional-ni-b})については,特定の具体的な結果が指定されていなくとも,
「割る」の一般的な結果状態は「割る」という語の語義の中に初めから含まれているため意味は解釈できる.
(\ref{enum:optional-ni-c})の「落下する」という動詞では,ニ格で移動の着点を指定することはできるが,
必ずしも落下の結果どこかに到達している必要はないので,ニ格が必須の項であるとは言い難い.
このようなニ格の任意性は,述語,あるいは文脈ごとにそれぞれ判断が必要である.
どのような基準でニ格の任意性を認めるかについては現状では明確な基準は用意されていない.

また,仮に,ある述語についてニ格の任意性が判定できたとしても,
実際の文中の事例で,
任意であるニ格が明示的に格助詞「に」を伴って出現していなかった場合,
それが未定義なのか,概念上存在しているのか,
あるいは同一記事中の別の箇所に出現しているかどうかの判断も困難を極める.
例えば,次の文
\enumsentence{
衛星は\underline{落下し}始めた。2時間後、太平洋で発見された。
}
の「落下する」のニ格は,未定義なのか,文章中に存在しない「地球」なのか,それとも「太平洋」なのかは,
文脈をどのように解釈するかに依存する.
したがって,このような文脈や事前知識に深く依存する問題については
述語項構造アノテーションの範疇外としておき,
それ以降の,例えば推論モデル等で取り扱う問題と規定する考え方もありうる.
仮にそうした場合は,明示的に格助詞と共に表れる場合や,文脈上自明な場合を除いては未定義とするのが
妥当である.


\subsubsection{可能形・願望・二重ガ格構文・持主受身}
\label{sec:potential}

可能動詞や可能形,願望,及び,いわゆる二重ガ格構文においては,異なる意味機能を持った二つの格助詞「が」を伴うことがある.
\eenumsentence{
\item 太郎は(が)英語が/を 読める。(可能動詞)\label{enum:possible-a}
\item 太郎は(が)ブロッコリーが/を 食べられない。(可能形)
\item 太郎は(が)ビールが/を 飲みたい。(願望)
\item 太郎は(が)足が 長い。(二重ガ格構文)
}
この問題について,NTCでは,可能形の場合は原形に戻してラベルを付与し,
「AはBがV」を「AのBがV」として置き換えることが可能な場合は「ノ格」で付与,
それ以外の場合は「ハ」と「ガ格」を用いてラベルを付与するとしている.
\eenumsentence{
\label{enum:double-ga}
\item $[太郎_{ハ}]$は [英語$_{ガ}$]が/を \underline{読める}。 \label{enum:double-ga-a}
\item $[太郎_{ガ}]$は [ブロッコリー$_{ヲ}$]が/を \underline{食べ}られない。 \label{enum:double-ga-b}
\item $[太郎_{ハ}]$は [ビール$_{ガ}$]が/を \underline{飲み} たい。 \label{enum:double-ga-c}
\item $[太郎_{ノ}]$は [足$_{ガ}$]が \underline{長い}。 \label{enum:double-ga-d}
}
しかし,この方法を取る場合,
次のようなガ格あるいはヲ格の選択肢の範囲を限定する「は」の用法が現れたときに,
ラベルを付与すべき
対象が複数現れてしまい,場合によっては二重の「ハ」となってしまう.
\eenumsentence{
\item $[ワイン_{ハ?}]$は [ロゼ$_{ガ}$]が \underline{美味しい}。\label{enum:double-ga-e}
\item $[私_{ハ}]$は [ワイン$_{ハ?}$]は [ロゼ$_{ガ}$]が \underline{好き}だ。
\item $[本_{ハ?}]$は 英語の[もの$_{ヲ}$]を \underline{読む}。\label{enum:double-ga-f}
\item $[私_{ガ}]$は $[本_{ハ?}]$は 英語の[もの$_{ヲ}$]を \underline{読む}。
}
上記のような例を考えると,項の選択範囲を限定する「は」は述語横断的に利用できる周辺的な格と類推できる.
したがって,
必須格と周辺格を付け分ける現行の仕様上では
(\ref{enum:double-ga})における「ハ」と,(\ref{enum:double-ga-e})における「ハ」の用法は
明確に区別したい.

経験的に,格のラベルと文中の実際の助詞が見た目上一致すると,アノテータは
こうした混同を起こしやすい.
したがって,これを避けるために「ハ」の名称を二つに分けるという方法が有効な可能性がある.
ここでは,例えば便宜的に(\ref{enum:double-ga})のハの場合を「属性所有のガ」,
(\ref{enum:double-ga-e})(\ref{enum:double-ga-f})の場合を「限定ハ」と決めるような方法である.
ラベルの名称を機能によって細分化するという方法は,格助詞を直接格関係のラベルに用いる
日本語の述語項構造アノテーションにおいては,同じ助詞によって表される必須格と周辺格を区別する際に
有効な手段であると考えられる.

一方,KTCの場合,動作主体や経験者といった意味役割的な観念を用いて,
『二重のガとなるもののうち,「は」「が」が動作主体や経験者である場合は,
用言からみて遠い方のガ格をガ2格とする』とすることで,必須格と周辺格の混同を避けている.
また,NTCでノ格に対応する「太郎は足が長い」などの表現は,
「は」を「が」に言い換えると不自然だとして,
ガ・ヲ・ニなどの格助詞では言い表せない「外の関係」として定義している.
\eenumsentence{
\item $[太郎_{ガ2}]$は [英語$_{ガ}$]が \underline{読める}。 
\item $[太郎_{ガ2}]$は [ブロッコリー$_{ガ}$]が \underline{食べられない}。 
\item $[太郎_{ガ2}]$は [ビール$_{ガ}$]が \underline{飲みたい}。 
\item $[太郎_{外の関係}]$は [足$_{ガ}$]が \underline{長い}。 
}

これとは別に,(\ref{enum:possible-a})に見られる可能動詞では,NTC方式の
アノテーションを行う際に格ラベルの組み合わせに曖昧性が出る
という問題がある.具体例として,(\ref{enum:possible-a})の例文では,ラベル付与の方法に
(\ref{enum:kanou-ambiguity-a})(\ref{enum:kanou-ambiguity-b})(\ref{enum:kanou-ambiguity-c})の三通りの曖昧性が発生する.
\eenumsentence{
\label{enum:kanou-ambiguity}
\item $[太郎_{ハ}]$は(が)[英語$_{ガ}$]が/を \underline{読める}。(NTC方式)\label{enum:kanou-ambiguity-a}
\item $[太郎_{ハ}]$は(が)[英語$_{ヲ}$]が/を \underline{読める}。(NTC方式)\label{enum:kanou-ambiguity-b}
\item $[太郎_{ガ}]$は(が)[英語$_{ヲ}$]が/を \underline{読める}。(NTC方式)\label{enum:kanou-ambiguity-c}
}
この問題は,格フレームの曖昧性の問題として\ref{sec:frame-ambiguity}~節で詳しく議論する.
KTCのガイドラインでは,同様の場面で「基準として,可能形の動詞の対象(目的語)の格はヲ格,動作主体の格はガ格とするが,もっとも自然な格を選択する.
目的語の表層格がガ格になっている場合などには,その格を別の格に変えることはしない.ガ格
がすでに使われている場合の動作主体の格はガ2格とする」と,厳密な優先規則を規定することで
曖昧性を回避している.


\subsubsection{使役・受身・ムード・授受表現・機能動詞で追加される格}
\label{sec:additional-cases}

NTCは,述語と項の間の格関係を,述語原形に対する表層格によって記述する.
このような方法を取る場合,述語が使役・受身などの形を取った場合に,
原形では対応のない格が出現する問題があるため,これに対処する必要がある.
\eenumsentence{
\item $[私_{追加ガ@B}$]が[太郎$_{ガ@A}$]に勉強\underline{さ$_{A}$} \underline{せる$_{B}$}。(使役)
\item $[彼_{追加ガ@B}$]が[父$_{ガ@A}$]に\underline{死な$_{A}$} \underline{れ$_{B}$} た。(迷惑受身)
\item $[私_{追加ガ@B}$]が[彼$_{ガ@A}$]に[ゲーム$_{ヲ@A}$]を\underline{壊さ$_{A}$} \underline{れ$_{B}$} た。(持主受身)
\item $[両親_{追加ガ@B}$]が[太郎$_{ガ@A}$]に勉強\underline{し$_{A}$}て\underline{ほし$_{B}$}がっている。(願望)
\item $[私_{追加ガ@B}$]が[彼$_{ガ@A}$]に[本$_{ヲ@A}$]を\underline{書い$_{A}$}て\underline{もらう$_{B}$}。(授受表現)
}
この問題に関して,NTCでは,上記のように助動詞や補助動詞を新たにマークし,
追加ガ/ニ格を割り当てるとしている\footnote{NTC 1.5版時点で,
公開版にはこのラベル情報は含まれていない.}.
NTCのガイドラインでは,少数の助動詞・補助動詞に関して,具体的な事例を用いてアノテーション方法を指示しているが,
これに加えて,機能動詞構文について\ref{sec:pred-dicision}~節で取り上げたような取り扱いをする場合は,
機能動詞構文によって追加される格についても
取り扱う必要がある.
また,述語によっては,機能表現によって格が追加されたと
見なすべきか,受益格のような周辺格と見なすべきか明確でないケースも存在する.
\eenumsentence{
\item $[事件_{追加ガ@B}$]が[社会$_{ガ@A}$]に\underline{混乱$_{A}$}を\underline{与える$_{B:使役}$}(機能動詞構文)
\item $[彼_{追加ニ@B?/受益格(周辺格)@A?}$]に [ジャム$_{ヲ@A}$]を \underline{取っ$_{A}$}て\underline{やる/あげる$_{B}$}
}
特に,機能動詞や補助動詞については,表現の種類が多岐にわたるため,
追加されている項が省略されている場合の見落としなどを抑制して
作業の一貫性を高めるためには,
これらの現象に関わる表現について,網羅的にかつ統一的な扱いをする必要がある.
これには,追加の格が存在する表現を一覧化し,自動的に確認を促す
仕組みを設けるなどの方法が考えられる.


\subsubsection{慣用表現}
\label{sec:idioms}

次の例のように,見た目上は述語とその項が個別に現れているようにも取れるが,
実際にはこれらが句として強く結びつくことで,一つの新たな意味を形成している慣用表現がある.
\eenumsentence{
\item 私が/の 気が 滅入る \label{enum:idiom-a}
\item 私の チームに 手に 入れ たい \label{enum:idiom-b}
\item 確認作業に 骨を 折る \label{enum:idiom-c}
\item 彼の 耳に 入る \label{enum:idiom-d}
}


NTCでは,どのような表現までが慣用表現と言えるのかの境界が厳密には規定できない
だろうという前提から,
慣用表現かどうかを区別せずに見た目上の述語に対してアノテーションを行っている.
KTCも同様に,慣用表現かどうかは区別せずにアノテーションを行っている.

これらの表現に対して,述語項構造アノテーションのガイドラインが取り得る戦略としては,
(i) NTCやKTCと同様に,慣用表現内部の述語項構造も全て分解してアノテートする,もしくは
(ii)慣用表現は複数形態素にまたがる述語表現として特別扱いする,ということが
考えられる.ただ,どちらの場合に関しても議論の余地がある.

(i)の場合は,
まず,\ref{sec:complex-word-decomposition}~節の複合語の議論の時と同様,慣用表現内部の項構造は,出現事例ごとに
異なるということはほとんどないため,同じ構造を何度もアノテートする無駄が生じる
可能性がある.また,慣用表現の表す意味は,比喩的な派生の結果,
元の語句から構成的に組み上げられる意味と一致しないため,
分解して項構造をアノテートする意味自体が薄い
\footnote{ただし,一部の慣用表現では,
「手に入れる」と「手に入る」,「骨を折る」「骨が折れる」などのように,慣用表現内部の
述語項構造に依存した自他交替などがありうる.}.
さらには,(\ref{enum:idiom-a})(\ref{enum:idiom-b})にも見られる通り,慣用表現によっては格の重複が起こり,
どちらが内容的に見て重要な格で,どちらが慣用表現内の「意味的重要度の低い」
格かの区別が難しくなる.(\ref{enum:idiom-c})に見られるように,
元々の述語(この場合,「折る」)に存在しなかった格(ニ格)が増える場合もあり,
アノテーションに際して格フレーム辞書を用意した場合などには,分解された語のみの格フレームでラベルを付与しようとすると扱いが難解になる.

(ii)の場合は,ある句をどのような基準で慣用表現とみなすかが問題となる.
慣用表現を整理した既存の研究としては,佐藤の基本慣用句五種対照表\cite{佐藤理史:2007-03-28}や
橋本らのOpenMWE:日本語慣用句コーパス\cite{hashimoto2008construction}などが挙げられるが,
佐藤の研究では「慣用句の定義はいまだに決定的なものがない」としている.
また,慣用表現全体を述語と見なすこととした場合には,(\ref{enum:idiom-d})のように
慣用表現内の項の一部を修飾する情報をどのように扱うかも問題となる.この例の「彼の」は,
もし慣用表現を分解して考えた場合にはニ格相当の句の一部となっているため,
この関係にも何らかのラベルを用意するのが望ましいと考えられる.

この問題に対しては,まずはコーパス内の慣用表現と思われる事例を集め,
慣用表現を述語項構造という観点で見た場合にどのような現象が起こりうるのかを網羅的に収集する必要がある.
そのため,初期のアノテーションでは慣用表現内を分解した状態でアノテーションを行い,その上で慣用表現の取り扱いを決めるといった
段階的なアノテーションを行うこともコーパス構築上の戦略として考えられる.
また,実際に慣用表現をひとまとめにしたアノテーションを
行う際は,機能表現や機能動詞での議論と同様,対象表現を辞書的に管理するのが望ましい
と考えられる.


\subsubsection{格交替と表層格ラベルの種類(KTC方式とNTC方式)}
\label{sec:ntc-case-ktc-case}

\ref{sec:related_work}~節で紹介したとおり,KTCは述部の出現形に対する
格関係を付与し,NTCは述語原形に対する格関係を付与する.
このため,格交替をともなって述語が出現する場合には,これら二つの基準では異なったアノテーションが行われる.
出現形アノテーションと原形アノテーションは,互いに他方には含まれない情報を持っており,
どちらの方式がより適切かはアプリケーションによって異なる.

例えば,含意関係認識のような命題間の同一性を扱いたいタスクでは,
(\ref{enum:entail})の文aと文bが同じ内容を表していることを捉えたい.
そのため,このような場合は,格交替を吸収するNTC方式が有用である.
\eenumsentence{
\label{enum:entail}
\item $[次郎_{追加ガ@B}$]は[太郎$_{ガ@A}$]に[ご飯$_{ヲ@A}$]を\underline{食べ$_{A}$} \underline{られ$_{B}$} た。(NTC方式)
\item $[太郎_{ガ}$]が[ご飯$_{ヲ}$]を\underline{食べ}た。(NTC方式)
}
一方で,機械翻訳や文書要約などの
表層的な形式をそのまま扱うことが可能なアプリケーションでは,
受身や使役などはそのまま翻訳・要約すれば良いため,
必ずしも述語原形の格に戻す必要性はない.
項の省略がある場合も述部出現形の格助詞を用いて補完すればよい.
このような場合にも述語を原形に戻そうとした結果,
原形に対する格パタンを選択する際に処理を誤る可能性もあるため,無理に原形に戻す処理を行うことはリスクをともなう.
したがって,このような場合には出現形でアノテーションを行うKTC方式を採用するほうが望ましい.
\eenumsentence{
\item $[太郎_{ニ}$]が来た。[りんご$_{ヲ}$]を\underline{食べられた}。(KTC方式)
\item $[太郎_{ガ}$]が来た。[りんご$_{ヲ}$]を\underline{食べ} られた。(NTC方式:受身のまま「太郎」を補う場合に,ニ格で補われるべきという情報を得られない)
}

これらに関連して,格交替前と格交替後の
格の対応関係を獲得したい場合には,
KTC方式でアノテートした
コーパスからはこの対応関係を直接学習出来ないため,
対応関係を獲得するための新たな資源が必要となる.
NTC方式の場合,コーパス上にこの対応関係をアノテートしていることになるので
見た目上はそのような対応関係表は必要ないが,
実際にはコーパス中に
格交替をともなって出現する事例は全事例の1割程度であるため,
異なる格交替の振る舞いをするそれぞれの述語に対して
対応関係の学習に十分な量の交替事例が得られるとは限らない.
出現形表層格における格交替関係については,$10$億文規模の大規模なコーパスから
自動獲得する方法も研究されているため\cite{sasano2013},
格交替の扱いについては,今後どちらの方針でアノテートすることが効果的かを検証する必要がある.


この検証を行うためのデータ作成の方法として,
KTC方式,NTC方式の双方で同一文章にアノテーションを行う方法が考えられる.
この場合のコストは,格交替が起こらない場合などの重複する作業は省略できるため,単純に倍というわけではない.
ただし,効果的に対応関係を取るためのアノテーションの方法については今後検討する必要がある.

もう一つの方法は,仮にいくつかのデータがアプリケーションによる要請などによって
異なるラベルセットを用いてアノテートされたとしても,
それぞれのスキーマによるアノテーションの結果を自然に統合し,
互いにラベルセットを交換可能とする仕組みを考えることである.
KTCとNTCの場合は,各述語に対する語義別の格フレーム辞書と,
各語義に関する格交替の性質を網羅的に記述した辞書を用いてこの仕組みが設計可能である.
この方法を取れば,将来,主題役割などのラベルを導入する場合にも,既存のアノテーションの結果をマッピングすることで,
最小限のアノテーション作業によって新たな結果を得ることが
できると期待できる.
ただし,このようなスキーマ間のラベルの対応を得るのは容易ではない.
アノテーション作業の重複を避けるためには,異なるスキーマ間のラベルが事例ベースで一対一対応する必要があるが,
各事例で適切な対応関係を得るためには,それぞれのスキーマが,
お互いのラベルがエンコードしている情報の差を明確に意識し,その差が追加情報によって
将来的に埋められるよう綿密に設計されたスキーマでなければならない.
また,格フレームや語義等も,共通の基盤データに基づいておく必要がある.
さもなければ,それぞれのスキーマの理論上のずれや格フレームのカバレッジ,語義の粒度のずれによる影響で,
ラベル間の対応が一対多,多対多の曖昧な関係となり,結局,
コーパス全体にわたってほとんど網羅的な確認作業を行わざるを得ないことになる.
実際に,英語圏では,異なる述語項構造コーパス間にアノテートされた異なる情報を
有効に活用しようと,資源間でのラベルのマッピングを試みた研究があるが,それぞれ異なる理念で設計されたコーパスであったため,
格フレームやラベルの対応関係は多対多となり再アノテーションを必要とした\cite{loper2007combining,semlink}.
したがって,仮に,アプリケーションからの要請や,段階的に情報を付加していく設計などによって,
異なるアノテーションスキーマを使い分ける場合にも,将来の統合性をはっきりと意識した設計をしておくことが
重要となる.例えば,\ref{sec:ni-imperativeness}~節で述べたような必須格と周辺格の区別などは
現状のガイドラインでは明確に取り扱われていないが,意味役割との親和性を考えれば重要な事項である.


\subsubsection{項としての形容詞(ニ格相当)}
\label{sec:adjective-ni}

次の二つの例文は,非常に似通った意味を表している.
\eenumsentence{
\item $[服_{ガ}$]が[赤$_{ニ}$]に\underline{染まる}。
\item $[服_{ガ}$]が 赤く\underline{染まる}。
}
どちらの文からも,我々は「服が赤くなった」という同一の結果状態を想像することができる.
しかし,現状の表層格を用いたアノテーションでは,「赤く」という形容詞を用いた表現は
項として認識されず,これら二文の間の項構造は異なるものになる.
この違和感は,特に項の省略を伴う次のような例文に対して,どのような表現まで項として補うかという
判断を行うときに大きくなる.
\eenumsentence{
\item $[真っ赤_{ニ?}$]なペンキで、[服$_{ガ}$]が\underline{染まっ}てしまった。
\item $[赤い_?$]ペンキで、[服$_{ガ}$]が\underline{染まっ}てしまった。
}

この問題は,我々が,表層格というラベルを用いて,述語とそれを取り巻く要素の間のどのような
関係を取り扱おうとしているかを考える際の良い題材である.
現状のスキーマでは,格助詞の表層的な違いとして認識できる粒度の意味関係しか取り扱っておらず,
果たしてどのような意味機能をもったものならばガ格・ヲ格・ニ格との意味的対応関係が取れる
ものなのかについて,網羅的な結論を即座に出すことは難しい.
しかし,もし,述語項構造を,述語と項の間の意味的関係の同一性を示すための表現として用いようとするならば,
「名詞+格助詞」や「形容詞」といった統語上の区分にかかわらず,同一の意味機能を持つものには
同一の関係ラベルを与えるのがよいかもしれない.
これは将来発展的に,主題役割のような,より意味機能的なラベルを
用いてアノテーションスキーマを設計しようとする際には十分検討されるべき課題である.


\subsection{格及び格フレームの曖昧性解消・必須項の見落とし}
\label{sec:frame-ambiguity-argument-loss}

\subsubsection{AのB, 連体節,ゼロ照応等における格フレームの曖昧性}
\label{sec:frame-ambiguity}

ある述語が複数の格フレーム候補を持つとき,その述語が,AのB・連体節・項のゼロ照応などの形を
取った場合,アノテーション時にどの格フレームを選択すべきかについて曖昧性が生じる.
\eenumsentence{
\item 自他交替: パソコンの起動 $\rightarrow$ パソコンが起動する/パソコンを起動する
\item 道具格交替: ドアを開けた鍵 $\rightarrow$ (誰かが)鍵で開ける/鍵が開ける
\item 他動詞/自動詞+使役:政府による経済再生 $\rightarrow$ [政府$_{ガ}$]が[経済$_{ヲ}$]を再生する/[政府$_{追加ガ}$]が[経済$_{ガ}$]を再生させる
\item その他: 私が教える生徒 $\rightarrow$ 生徒を教える/(何かを)生徒に教える
}
また,述語によっては,同一の意味機能を持つ項に対して複数の格助詞が代替可能である場合がある.
\eenumsentence{
\item 私が/から 話す
\item 太郎に/から もらう
\item 風に/で 揺れる 花びら
\item 土台に/と くっつける
}
この例では,ガ・ヲ・ニの間で代替可能なものはないが,仮に今後付与対象の格助詞を拡充することを考える際には,
このような曖昧性を生み出す要素に対してどのように一貫したアノテーションを行うかを考慮する必要がある.


ラベルの選択に本質的な曖昧性が出る場合には,ある基準にもとづいて(例えば,出現頻度順や,アノテーションコストが低くなるように,などで)
決めた規則に従って,
付与するラベルが一意に定まるようにするのが一般的である.
しかし,前者の格フレームの曖昧性については,文脈によってはどちらか一方の格フレームの方が他方での解釈よりも自然な場合があり,そのような場合は適切な解釈となる格フレームを選ぶのが好ましいと考えられる.
一方で,文脈の曖昧さによってはアノテータ間の意見が一致しない場合もありうるし,
当然,本質的にどちらに解釈しても自然な場合もある.そのような事例に対しては,
自動解析器の学習や評価時に適切な取り扱いができるように配慮しなければならない(どちらの解釈でも正解と
して学習・評価するなど).

このような場合に総合的に配慮した対策を検討してみる.
例えば
前者の格フレーム間の曖昧性については,
(i)事前に述語ごとの格フレーム辞書を用意しておくか,代表的な格フレーム交替についての名称を列挙しておき,
(ii)アノテーション時に,複数の格フレームで判断に迷うものや本質的に曖昧なものについては,その交替の候補を列挙し,
(iii)自己の判断,もしくは規則に従った判断で選んだ格フレームで格関係をアノテートする,という方法
で,本質的に曖昧な事例と,規則や主観にもとづいた上での不一致を弁別することができる.

次に,後者の代替可能な格に関するアノテーションを検討するため,
今,仮に,ラベルの数を拡張し,ガ・ヲ・ニ・カラ・デ・トの6つの表層格を使ってアノテーションを
行っている場合を考える.
この6つのラベルに対して,$ガ・ヲ・ニ > カラ > ト > デ$などの半順序を与えることで
規則的にこれを解消することもできるが,文中で,
\enumsentence{
彼には 私から 話しておく。
}
と格助詞「から」を伴って出てきている事例に対して,これを「ガ格」として正規化するためには,
アノテータは事例
毎に対象の述語に関する格フレーム辞書を想起し,格の交替関係を確認せねばならず,
アノテーションのコストが大きい.したがって,作業コストの観点からすれば,少なくとも,
述語が原形で使用されているもので,項が格助詞を伴って現れる場合には出現形の格でアノテートし,
受身・使役などで格交替しているものや,
ゼロ照応などで元の格助詞が不明なものに関しては,
上記の半順序規則を適用するなどといった方法が好ましいと考えられる.
一方で,解析器の学習や評価を行うときの観点からすれば,ラベル付与
時に格を正規化しない場合に,格に意味的な一貫性を持たせて取り扱う,もしくは曖昧な格のうちいずれの格でも本質的に正しいという
取り扱いをするためには,別途格フレーム辞書等に述語毎の格の交替情報を記述しておくなどする必要がある.


\subsubsection{格フレーム辞書とアノテーションの一貫性}
\label{sec:frame-dictionary}

NTCやKTCのアノテーションでは,被連体修飾詞やゼロ代名詞として出現する項など,
明示的に格助詞を伴わなかったり,
対象の述語と何らかの統語的関係を伴わない項に関しても
格関係の付与を行う.
\pagebreak
ただし,開発作業時点では,ある述語の取り得る格(格フレーム)について参照できる
辞書等が存在しなかったため\footnote{
NTCでは,NTT語彙大系・構文体系の辞書\cite{nttlexicon}や竹内らの動詞語彙概念辞書\cite{takeuchi2005}などを参照して作業したものの,辞書に記述されている格パタンと新聞に出現する格パタンが必ずしも一致せず,逆に作業者が混乱したために,使用が中止された.},アノテータは内省に頼りながら,
文章中からその述語に足りない項を補う作業を必要とした.
しかし,一つ一つの述語の格フレームの定義をアノテータの内省に頼る方法には限界があり,
その影響は\ref{sec:ntc-iaa}~節表~\ref{tbl:ntc_iaa}のNTCの作業者間一致率においても,
ゼロ照応項の不一致という形で顕著に見られる.

述語項構造アノテーションの一貫性を今以上に向上させるためには,予め各述語に対して
正確な格フレーム辞書を定義しておくなどして,全てのアノテータが共有する共通の語彙知識ベースを
整備する必要がある.実際に,NTCガイドライン準拠のアノテーションをBCCWJに対して行った研究\cite{komachi2011}では,
既存の格フレーム辞書を一部参照することによって,作業者間一致率に一定の向上が得られたとしている.

作業の上で参照する格フレーム情報はできるだけ精緻なものが求められるが,
一方で,大規模な文章に対する述語項構造アノテーションを行うにあたって,
ある述語の様々な言語現象を網羅した実用に耐えうる頑健な格フレーム辞書を人手で用意するには膨大なコストを必要とする.
整備コストを抑えた方法として,大規模な文書データから自動的に格フレームを獲得する研究が存在するが\cite{kawahara2006case},
獲得したフレームにはノイズも存在するため,アノテーション作業での運用には工夫が必要である.

アノテーションを行う全てのデータができるだけ正確となるよう運用するのが最も望ましいが,
現実的な面で言えば,例えば,初期の段階では,全体からサンプルした一部のデータに出現する述語のみ,
あるいは主要語のみに絞るなどして,一部の述語に対してのみ
精緻な格フレーム辞書を作ってアノテーションを行う方法が考えられる.
この場合,精密なアノテーションデータは評価用のデータとして整備し,
残りの部分は自動獲得した格フレーム等を参考にしながら大規模にアノテートするなど,
質と量の双方を兼ね備えるコーパスを設計する方法が好ましいと考えられる.
このような方法論はBCCWJのコアデータとデータ全体の間の関係などにも見られる.


\subsubsection{非文へのアノテーション}
\label{sec:ungramatical-sentence}

アノテーション対象のデータには,場合によっては一部非文(らしき文)も含まれる.このような文に対して
どのようなアノテーションの方針を取るのかについても考慮の余地がある.
\eenumsentence{
\item 服を乾燥する (受容の余地あり)\label{nonsentence-a}
\item ガラスを壊れる (マークされている意味機能的に受容出来ない)	\label{nonsentence-b}
}
例えば,一般には「乾燥する」は自動詞だとされているが,(\ref{nonsentence-a})のような用例はWeb上には多数見られる.
一方で,(\ref{nonsentence-b})の「壊れる」のように,形態論上は自動詞の形を取っているにもかかわらず格助詞「を」を
取るような構造の文は相対的に受容しがたい.このような文に対して,(i)アノテートするかということと,
(ii)どのような文を非文とみなすかということが問題となる.
(i)に関しては,現実にデータ上に存在する事例であり,応用事例によっては特によく使われている過ちは
頑健に解析したいという場合もあるため,書かれたままの表層格を基にアノテートする方法が望ましいと思われる.
(ii)に関しては,我々の知る限り,現在までに非文というものの明確な定義は存在せず,個人の内省にもとづいて判断されるもののため,
例えば,非文かどうかの判断はアノテータに任せ,代わりに非文と判断された事例を記録しておくことで,
必要に応じてデータを区分できるようにしておくような方法が有用と考えられる.


\subsection{新聞ドメイン以外で見られた現象}
\label{sec:other-domains}

本節では,新聞記事以外のドメインに対する試験的なアノテーション作業において現れた,
既存のガイドラインで対象としていない項目についてまとめる.このことについて
我々が分析の対象としたデータはWikipedia及びBCCWJコアデータ\footnote{書籍,雑誌,白書,Yahoo!知恵袋,Yahoo!ブログの5つのドメインを含む.}
より収集した1,000〜1,200文程度であるため,
各項目の事例を網羅的に
収集するに至ったとは言い難い.従って,ここではそれぞれの項目についての現象の説明をするにとどめ,
具体的な考察については今後の課題とすることにした.

また,今回の分析に用いた新ドメインの文章量は,
新聞ドメイン以外で新たに必要となる基準を
多岐にわたって示すには十分ではないが,一般には
ドメインごとに少なからず特定の言語現象の分布に偏りがあるなど,
各ドメインは特有の性質を持つ場合が多い.このため,
アノテーションガイドラインを新ドメインに対応させるためには,
それぞれのドメインにおける十分な量の個別事例を収集し分析するとともに,
同ドメインにおけるアプリケーションからの要請等も検討しながら
適切な仕様を策定していく必要がある.


\subsubsection{述語の省略}
\label{sec:pred-omission}

口語的な文においては,文末の述語が省略され,項のみが残されるというケースがよく見られる.
\enumsentence{
タモリさんから、「これは誰から?」と聞かれた。(「貰ったの」の省略?)
}
このような例で省略されている述語が文脈上容易に想像できる場合,何かしらの述語を補うか,
あるいは「述語-非出現」などのラベルを用意して対応する格をアノテートするか,そもそも項構造を解析しないか,ということが
議論の対象となる.述語省略の究極的なケースとしては
\eenumsentence{
\item これは…。
\item それはちょっと…。
}
などがあり,このような場合は,述語が何であるかのみならず,残された格が何格であるかすら推定が難しい場合があるため,
どこまでが
アノテーションを行って有用な情報となるかの判断は難しい.


\subsubsection{疑問文の照応}
\label{sec:question}

対話文においては,疑問文とその回答の間での照応も存在する.

\eenumsentence{
\item 「あれは誰?」「彼は山田太郎だよ」
\item 「誰からもらったの?」「太郎からだよ」
}
NTC・KTCにおいては,現在のところ疑問文に対する照応関係の取り扱いはない.
共参照・照応については本論文での議論の範疇外としたが,
対話文の多いドメインに対して照応・述語項構造を付与する場合は,
疑問文とその回答に対するアノテーション仕様も考慮する必要が出てくる.


\subsubsection{音象徴語}
\label{sec:sound-symbol}

次の例のように,音象徴語がサ変名詞のように振る舞い,述語として現れる場合がある.
そのような場合,音象徴語にも述語項構造をアノテートすることが考えられるが,事例によっては,
副詞的振る舞いとサ変名詞的振る舞いのどちらと取るか判断に迷う場合があった.
\eenumsentence{
\item $[胸_{ガ}$]がドキドキ\underline{する}
\item $[胸_{ガ}$]が\underline{ドキドキ}
\item $[胸_{ガ}$]の\underline{ドキドキ}(イベント性名詞)
\item $[胸_{ガ}$]がドキドキと\underline{高鳴る}音(副詞用法)
}

サ変名詞的振る舞いをする場合,副詞的振る舞いをする場合の他,
その他の音象徴語の統語的振る舞いについて,述語として認定するための明確な
ガイドラインを整備する必要がある.


\section{見通しの良いフレームワークの設計}
\label{sec:framework}

より質の高いアノテーションを目指してガイドラインを改善していくことを考えた場合,
対象のガイドラインは,その中で示されるそれぞれの基準が
どのような視点で採用されたのかが明確に分かるものでなければならない.
また,仕様策定時の理念をコーパス作成者の中で閉じた情報とせず,広く研究者間で共有できる形に
整理することにより,継続的な議論が可能になると考える.

\begin{figure}[t]
\begin{center}
\includegraphics{21-2iaCA12f1.eps}
\end{center}
\caption{設計の基本方針と各論点との対応関係}
\label{fig:principles}
\end{figure}

このような背景から,本節では,述語項構造アノテーションを題材とすることで集約した,
アノテーション仕様及びガイドラインの策定時に配慮されるべき基本方針を述べる.
これらは,議論に関わった研究者らが種々のアノテーションタスクの設計を通して
経験的に理解している事柄を集約したものであり,
複雑でアノテーションコストが高く,
また,現象の網羅のために大規模なアノテーションを行う必要がある同様のタスクに対しても
有用なものである.
\ref{sec:criteria}~節では,\ref{sec:individual}~節の議論から集約したガイドライン
策定上の着眼点を,各論点との対応関係を示しながら述べる.
図~\ref{fig:principles}には,\ref{sec:criteria}~節で説明する設計の基本方針と
\ref{sec:individual}~節で示した各議論との対応関係をあらかじめまとめた.
\ref{sec:example-framework}~節では,議論全体を俯瞰する目的で,
\ref{sec:individual}~節で議論した内容にもとづいた述語項構造アノテーションのフレームワークの具体的な一例を
示す.


\subsection{大規模アノテーションタスクに関するガイドラインの設計時に考慮すべきこと}
\label{sec:criteria}


\noindent
\textbf{A. データ内の現象に関する取り扱いの網羅性}
(\ref{sec:important-pred}節,\ref{sec:event-noun-dicision}節,\ref{sec:complex-word-decomposition}節,
\ref{sec:additional-cases}節,\ref{sec:idioms}節):
大規模なデータに対してアノテーションを施す場合,
そのデータ内で起こりうる,判断に特別のガイダンスを必要とする現象に対して,
現状のガイドラインがその現象のそれぞれの事例を十分に被覆できるかどうかについて十分な配慮が必要である.
特に,アノテーションの判断の決め手が,単語の意味や,慣用表現など,
言語の生産的な部分に関与している場合には,予め判断基準が列挙しつくせない場合もある.その場合,
アノテーション作業中に逐次的にガイドラインの一部を更新・反映する仕組みについても考慮する必要がある.

綿密に判断基準を決めるべき特定の現象があるとき,そのバリエーションが,
事前に少量の努力もしくは既知の知識で列挙可能かどうか,
非生産的で有限個なのか,あるいは生産的なのかについて考察し,
もし,事前列挙不可能な場合は,作業中に新たなバリエーションを発見した際に
他の基準に極力影響しない形でガイドラインを更新する方法についても
フレームワークの設計時点で考慮する必要がある.


\noindent
\textbf{B. 利用目的とアノテーション仕様の関係}
(\ref{sec:important-pred}節,\ref{sec:complex-word-decomposition}節,\ref{sec:ntc-case-ktc-case}節):
テキストに対するアノテーションを考えたとき,一般には,利用目的が異なれば,それに応じて必要になるラベル情報も異なる.
必要な情報のみを表現するラベルセットを作成し,不必要な情報の付与は避けるのが自然なスキーマ設計の方法である.
述語項構造のような基本的構造のアノテーションにおいても,同様の考え方は必要である.
例えば,次に挙げる応用処理においては,述語項構造の
どの部分の情報を用いたいか,どのような目的で用いるかによって,ラベル付与の基本単位や
前提となる意味論の精密度が異なる.

\begin{itemize}
\item 機械翻訳:フレーズベースで翻訳する場合には,複合語内部の項構造情報は比較的必要性が低い.
また,項のラベルについては,基本的には表層格のレベルで十分な場合が多い.
日英の場合,省略(ゼロ照応)解析は重要課題とされる.
\item 含意関係・言い換え認識:複合語内部も分解して解析する必要がある.場合によっては,項構造が意味役割のレベルで表現されるのが望ましい.
\end{itemize}

述語項構造のような基本的な構造をアノテートする際は,様々な応用の可能性を想定しなければならない.
また,このようなそれぞれの応用処理からの異なる要求に対し,柔軟にアノテーションの方法を提供でき,かつ,
仮にアノテーションがタスク志向で行われた場合にも,最終的にタスク個別のアノテーションデータを
統合できるようなフレームワークであることが好ましい.


\noindent
\textbf{C. 段階的に質と情報密度を向上できるフレームワーク}
(\ref{sec:idioms}~節,\ref{sec:ntc-case-ktc-case}~節,\ref{sec:adjective-ni}~節):
述語項構造や語義のアノテーションのように,アノテーションコストの非常に高いタスクでは,
人的資源の制約から,一部の現象のみに対象を絞って初期のアノテーションが行われる場合もある\cite{pradhan2007semeval}.
また,解析のための理論自体が未解決なタスクの場合,全ての問題点が解決するのを待たずに,
判断の境界が明確な部分についてのみアノテーションを進めるという方法が取られる場合もある.

こうしたアノテーションタスクにおいては,アノテーションスキーマを順次拡充・変更し,
段階的に新しいタイプの情報を付加していける設計のほうが好ましい.しかし,そのような設計の実現のためには,
実際には仕様設計の初期段階においても,将来追加される情報をある程度想定し,
表現の親和性や,データの自然な統合方法について配慮しておく必要がある.
仮に問題を全て解決せずとも,タスク内で起こりうる現象について把握し,
可能な限りスキーマ拡張のためのインターフェースを用意しておくのがよい.


\noindent
\textbf{D. 本質的に曖昧な選択肢に対する作業の一貫性・評価}
(\ref{sec:ni-imperativeness}節,\ref{sec:frame-ambiguity}節,\ref{sec:ungramatical-sentence}~節):
ある現象に対してアノテーションを行うとき,複数の選択肢に関して,
理論的な要請も特になく,応用処理の観点からもどちらの選択肢を取っても問題ない場合がある.
このような場合にどちらの選択肢を選ぶかの基準を設けなければ,
曖昧性のために見かけ上の作業者間一致率が低下する.
また,明確に判断可能な事例と本質的に曖昧性のある事例が混在することで,
解析システムの評価時にも,各事例において,解析誤りのために精度が低いのか,
タスク自体が持つ曖昧性のために精度が低いのかの判断が難しくなる.

したがって,このような事例に対して,作業の一貫性を与える,もしくは
適切な評価手法を与える配慮が必要である.
一般に,ラベルの選択に曖昧性がある場合には,
選択に優先順位を定義するなど,
ラベルが一意に定まるような規則を決める方法が取られる.
結果,一致率が向上し,応用処理に用いる際も一貫した利用方法を考えることができる.
もう一つの解決策は,本質的に曖昧な事例が出現した際に,その事例が曖昧であることを示しておくことである.
そうすることで,曖昧な事例に関しては評価に含めない,あるいはどちらでも正解とみなすなどして,
一致率,精度の評価がより適切に行えるようになる.


\noindent
\textbf{E. 作業上のコストや作業者が直面する選択肢数をできるだけ減らすフレームワーク}
(\ref{sec:important-pred}節,\ref{sec:complex-word-decomposition}~節,\ref{sec:frame-ambiguity}~節): 
複雑で作業コストが高く,また,表現のバリエーションや頻度分布を観測
するために大規模な事例数が必要なタスクにおいては,
限られた資源を用いてより多くのデータを作成できるよう,
いかにその作業上のコストを下げるかを検討することも重要な課題の一つである.
加えて,作業者に複数の選択肢から一つを選ぶような判断を迫る場面においては,できるだけ選択肢を事前に絞り込み,不要な迷いを避ける工夫が,
作業効率の面だけでなく作業結果の一貫性を向上させる意味でも必要不可欠である\cite{Bayerl:2011:DIA:2077692.2077696}.


\noindent
\textbf{F. データ量と質のコントロール}
(\ref{sec:frame-dictionary}~節): 
前述のとおり,大規模なアノテーションを行うためには,一つ一つの事例に対して
大きな作業コストのかかる方法を気軽に採用することは難しい.
一方で,述語項構造のような基礎的な構造の分析に関しては,
解析システムの正確な評価のために,できるだけ高品質なデータが必要とされることも確かである.
このような,データ量と品質のトレードオフをどのような方針で管理するかについても考慮が必要である.


\subsection{述語項構造アノテーションフレームワークの一案}
\label{sec:example-framework}
本節では,考察結果全体を俯瞰する目的で,
\ref{sec:individual}~節での個別の論点への考察を通して導かれた
述語項構造アノテーションフレームワーク全体の具体的な設計の一案について述べる.
ただし,このフレームワークは
議論の中で出された複数の選択肢のうちの一つを組み合わせたものであり,
議論の唯一解を示すものではないことに注意されたい.

図~\ref{fig:framework}には,その全体像を示した.
これは,これまでの議論をふまえて精査し・修正したガイドラインと共に,
複数の述語-項関係ラベルセット,異なるラベルセット間のラベルの対応関係,複数の質の異なる格フレーム辞書,
機能表現・慣用表現等に関する辞書などを保持し,アプリケーションで必要な情報に応じて
柔軟に運用できるよう配慮された設計となっている.

\begin{figure}[t]
\begin{center}
\includegraphics{21-2iaCA12f2.eps}
\end{center}
\caption{述語項構造アノテーションフレームワークの一案}
\label{fig:framework}
\end{figure}

図~\ref{fig:framework}(a)の部分では,
機能表現,複合語内項構造,慣用表現など,個々の事例に判断を要し,かつ
事例のバリエーションが豊富なものに対して,
既存の言語資源をベースとするなどして予め取り扱いの指示を定めた辞書を用意しておく.
そうすることで,テキスト内で該当する可能性がある箇所を自動チェックし,作業漏れの抑制,
アノテーションの半自動化を行うことができる.
この辞書は,アノテータ間に共通の判断を強制する効果があり,
その結果,各事例について一貫した作業結果を得ることができる.
複合語内部の述語項構造など,ほとんど全ての事例に同一の関係しか認められないものについては,
その項構造を辞書的に保持することで,アノテータが事例毎に自明な
アノテーション作業
を行うことを避けることもできる.
アノテータは辞書の規則に反する一部の例外のみを作業するだけでよい.
また,辞書に未収録の事例が出現した時点で辞書エントリを追加し,コーパスを再チェックする仕組みを作成しておく.

図~\ref{fig:framework}(b)の部分では,
格の情報に関して二つのことを管理する.
一つ目は,格フレーム辞書の管理である.必須格の見落としなどの作業の揺れを防ぎ,作業者間一致率を向上させるため,
各述語の語義ごとに取り得る格をあらかじめ列挙した辞書を作っておく.
この辞書は,図~\ref{fig:framework}(d)の部分で述べる格の曖昧性の管理にも有効である.

二つ目は,
アプリケーションの用途に応じて異なるラベルセットで行ったアノテーションについて,
データをマージしたり,新たに追加で別のラベルセットを用いてアノテートする際,
最小限の追加作業でアノテーションを行えるよう,
異なるラベルを持った格フレーム辞書間での格の対応関係を管理する.
格の対応関係表には,各述語のフレームに対する異なるラベルセット間での格ラベルの対応関係が記述される.
例えば,述部出現形の表層格(KTC形式)と述語原形の表層格(NTC形式)の対応関係であれば,表~\ref{tbl:kuwaeru}
のような情報である.

\begin{table}[b]
\caption{述語「加える」に関する述部出現形表層格と述語原形表層格の対応関係}
\label{tbl:kuwaeru}
\input{ca12table06.txt}
\end{table}

この情報は,人手,または自動的な方法のいずれかで構築する.
格ラベルの対応表と,変換に必要な付加情報(語義・態・アスペクト・ムードなど)が用意できれば,
いずれのラベルセットを使ってアノテートしておいても用途に応じた適切な粒度のラベルに変換して取り出すことができる
ようになる.ただし,それぞれの辞書が精緻に作成されていない場合,
\ref{sec:ntc-case-ktc-case}節で述べるような変換上の問題も存在する.

このような対応関係表の作成が現実的に難しい場合でも,代替的な方法として,
例えば,述部出現形の表層格と述語原形の表層格の場合には,
格の交替が起こりえない状況を列挙しておくなどすることで,
同一の文章に異なるラベルセットでアノテーションを追加する際に一意にラベルの対応が取れる箇所の
アノテーションを省略することができる.

図~\ref{fig:framework}(c)の部分では,コーパスの量と質のバランスを管理する.
全データ中の$n\%$,あるいは主要な$m$語への述語項構造といった方法でコーパスを区分し,
一定量のデータに対しては大規模コーパスの調査などから人手で構築した精緻な格フレーム辞書を用意する.
これを参照して作業することで
作業者間一致率を上げ,また,アノテーション結果の多重チェックを行うなどして精密な分析・評価用データとして確保する.
その他のデータは,従来通り格フレームと関連付けない,もしくは人手や自動獲得によって作られた既存の
格フレーム辞書を参考情報としてアノテートするなどして,質と量の双方をバランスよく確保する.

図~\ref{fig:framework}(d)の部分では,
ラベル付与の本質的な曖昧性を管理する.
曖昧な事例といっても,事例によっては本質的に完全に曖昧な場合もあれば,文脈上いずれかの選択肢が優勢と判断できる場合や,
その中間のようなあやふやな場合もある.したがって,これらを区別するために次の三つの付与方法を用意する.
\begin{enumerate}
\item アノテータが文脈に応じて疑いなく一つの選択肢を選ぶ場合,曖昧性を示唆するマーカーを付けない.
\item アノテータがいずれかの選択肢が優勢と感じたものの,はっきりと判断出来ない場合は,優位なラベルを1stとし,その他の候補をothers欄に列挙する.
\item アノテータが文脈上完全に曖昧だと判断した場合は,予め決めておいた順列に従って付与するラベルを選ぶ.その際は,
曖昧であることを示すマーカーを付け,他の候補も列挙しておく.
\end{enumerate}
こうしておくことで,どの事例が曖昧で,どれがそうでないのか明確に区別できる上,
本質的に曖昧な事例については統一的にラベルを振ることで,解析器の学習を行う際や解析器の出力を応用処理に用いる際も一貫した利用方法を考えることができる.
ラベルの順列を決める場合は,極力簡潔な方を選ぶ
(例えば,項の数がより少ない格フレームを選ぶ,使役・受身より原形,他動詞より自動詞,自動詞+使役より他動詞を選ぶ)
ように規則を決めておくことにより,アノテータの判断時の負荷を下げる工夫をする.

格フレーム辞書として精緻なフレームを用意している場合は,
格の曖昧性を格フレーム側で管理する.例えば,図~\ref{fig:framework}(d)の 2 文目では
「起動」に自動詞と他動詞の解釈があるが,アノテータが文脈上曖昧と判断した場合はこれに「自動詞・曖昧」とマークし,
「パソコン」には,自動詞時の解釈であるガ格を割り当てる.
格フレームには,自他交替など曖昧な格についての交替関係を記述しておくことで,
仮に他動詞と判断した場合には
「パソコン」がヲ格となることを知ることができるため,解析システムの評価時にも公平な評価が行える.

以上の設計の他に,\ref{sec:individual}~節の議論の結果から,
段階的な質・情報密度の向上を行う際の問題の切り分け方と作業の優先順位を設定する.
\begin{enumerate}
\item 動詞・形容詞・コピュラ・サ変の体言止めの項構造アノテーション
(慣用句と思う事例はチェックしておく)(複合語は分解せず,語の外側に出現する項のみアノテーション)
\item 複合語の分解(辞書的処理)
\item イベント性名詞の項構造アノテーション
(転成名詞・サ変名詞)
\item 慣用句の収集・整理・述語化
\item 照応・共参照情報に関わる整備(本論文の範疇外)
\item ニ格相当の形容詞
\item 必須格と周辺格の区別
\item 意味役割によるアノテーション
\end{enumerate}
このように,ラベル付与の判断がより明確な部分から段階を踏んでアノテーションを行うことによって,
より複雑な現象についてコーパス内の事例を収集し,問題を分析しつつ設計を進めることができる.

\ref{sec:criteria}~節で示したとおり,これらの個々の取り決めの一例は,
ガイドライン設計時の指針や個別に行った議論と明確に結びついている.
このような形で設計の理念・問題の議論・対応する規定の間の関係を明文化して示すことで,
継続的・建設的に仕様やガイドラインを改善するための議論を重ねることが可能となる.


\section{まとめ}
\label{sec:conclusion}

本論文では,より洗練された述語項構造アノテーションのガイドラインを作成する目的で,NTC・KTCの仕様策定,仕様準拠のアノテーション,
応用処理に関わった研究者,アノテータらの考察を基に,議論の対象となる点を整理した.
具体的には,既存のガイドラインを用いた新規アノテーションによる考察と,
研究者・アノテータが経験的に持つ知見を集約するという方法の二つの方法で,
既存のガイドラインからは簡潔に解決出来ない問題として$4$種$15$項目の論点を洗い出し,
それぞれの論点について現状の問題点やそれに対する改善策を議論し報告した.
議論結果を整理するにあたっては,ガイドライン策定の基準となる着眼点を示し,
議論内容や,結論との対応関係を示すことで,将来のガイドライン改善に向けて
建設的な知見となることを目指した.
本論文で示すアノテーションガイドライン改善のための論点の洗い出し方法は,
現行のアノテーションガイドラインにもとづいてラベル付与を行った際の
一致率を問題視して行った手法であったため,
既存のガイドライン,もしくはその簡単な修正版によって
明確にアノテーション規則が定まるものに関しては議論の対象としてあまり取り上げていないが,
\ref{sec:criteria}~節で示した「利用目的とアノテーション仕様の関係」を
仕様改善の指針として想定すれば,仕様の改善点は
必ずしも作業の一致率という観点のみで推し量られるべきものではなく,
コーパスの利用目的調査などに基づく
仕様の改善や
新たな付加情報の
列挙も試みられるべきである.したがって,
ここに記した問題が残る問題の全てとは言えないが,
こうした建設的な考察の積み重ねによって,実用に耐えうる一貫性を持った
アノテーション方針が作られるとともに,
統一的かつ頑健な言語解析理論の基礎が積み上がるものと信じるものである.


我々の考察の手順や結果を例に取ると,問題点の洗い出しの方法論や,ガイドライン作成時の理念など,
アノテーションに関わる科学は,未だ経験的知見によるところが大きい.しかし,近年では,
アノテーションタスクの複雑度や,一致率に影響する因子などに客観的指標を与えようと
試みる研究も見られる\cite{Bayerl:2011:DIA:2077692.2077696,fort2012modeling}.
\ref{sec:criteria}~節において,我々が経験的知見によりガイドライン設計の指針としている事柄についても,
広く一般的に成り立つ指針として,客観的指標で評価できるような仕組みを生み出していくことも
今後の課題である.





\bibliographystyle{jnlpbbl_1.5}
\begin{thebibliography}{}

\bibitem[\protect\BCAY{浅原\JBA 松本}{浅原\JBA 松本}{2003}]{ipadic}
浅原正幸\JBA 松本裕治.
\newblock ipadic version 2.6.3 ユーザーズマニュアル.\
\newblock
  \Turl{http://\linebreak[2]chasen.\linebreak[2]naist.\linebreak[2]jp/\linebreak[2]stable/\linebreak[2]doc/\linebreak[2]ipadic-2.6.3-j.pdf}.
\newblock Accessed: 2014-02-19.

\bibitem[\protect\BCAY{Bayerl \BBA\ Paul}{Bayerl \BBA\
  Paul}{2011}]{Bayerl:2011:DIA:2077692.2077696}
Bayerl, P.~S.\BBACOMMA\ \BBA\ Paul, K.~I. \BBOP 2011\BBCP.
\newblock \BBOQ What Determines Inter-Coder Agreement in Manual Annotations? A
  Meta-Analytic Investigation.\BBCQ\
\newblock {\Bem Computational Linguistics}, {\Bbf 37}  (4), \mbox{\BPGS\
  699--725}.

\bibitem[\protect\BCAY{Burchardt \BBA\ Pennacchiotti}{Burchardt \BBA\
  Pennacchiotti}{2008}]{burchardt2008fate}
Burchardt, A.\BBACOMMA\ \BBA\ Pennacchiotti, M. \BBOP 2008\BBCP.
\newblock \BBOQ FATE: a FrameNet-Annotated Corpus for Textual Entailment.\BBCQ\
\newblock In {\Bem Proceedings of the 6th Edition of the Language Resources and
  Evaluation Conference (LREC 2008)}, \mbox{\BPGS\ 539--546}.

\bibitem[\protect\BCAY{Carreras \BBA\ M\`{a}rquez}{Carreras \BBA\
  M\`{a}rquez}{2005}]{Carreras:2005:ICS:1706543.1706571}
Carreras, X.\BBACOMMA\ \BBA\ M\`{a}rquez, L. \BBOP 2005\BBCP.
\newblock \BBOQ Introduction to the CoNLL-2005 Shared Task: Semantic Role
  Labeling.\BBCQ\
\newblock In {\Bem Proceedings of the 9th Conference on Computational Natural
  Language Learning (CoNLL 2005)}, \mbox{\BPGS\ 152--164}.

\bibitem[\protect\BCAY{Fort, Nazarenko, \BBA\ Rosset}{Fort
  et~al.}{2012}]{fort2012modeling}
Fort, K., Nazarenko, A., \BBA\ Rosset, S. \BBOP 2012\BBCP.
\newblock \BBOQ Modeling the Complexity of Manual Annotation Tasks: A Grid of
  Analysis.\BBCQ\
\newblock In {\Bem Proceedings of the 24th International Conference on
  Computational Linguistics (COLING 2012): Technical Papers}, \mbox{\BPGS\
  895--910}.

\bibitem[\protect\BCAY{グループ・ジャマシイ}{グループ・ジャマシイ}{1998}]{Jamasi1998}
グループ・ジャマシイ\JED\ \BBOP 1998\BBCP.
\newblock \Jem{教師と学習者のための日本語文型辞典}.
\newblock くろしお出版.

\bibitem[\protect\BCAY{橋田}{橋田}{2005}]{hashida05}
橋田浩一.
\newblock GDA 日本語アノテーションマニュアル草稿第 0.74 版.\
\newblock
  \Turl{http://\linebreak[2]i-content.\linebreak[2]org/\linebreak[2]gda/\linebreak[2]tagman.html}.
\newblock Accessed: 2014-02-19.

\bibitem[\protect\BCAY{橋本\JBA 黒橋\JBA 河原\JBA 新里\JBA 永田}{橋本 \Jetal
  }{2009}]{橋本力2009}
橋本力\JBA 黒橋禎夫\JBA 河原大輔\JBA 新里圭司\JBA 永田昌明 \BBOP 2009\BBCP.
\newblock 構文・照応・評判情報つきブログコーパスの構築.\
\newblock \Jem{言語処理学会第 15 回年次大会論文集}, \mbox{\BPGS\ 614--617}.

\bibitem[\protect\BCAY{Hashimoto \BBA\ Kawahara}{Hashimoto \BBA\
  Kawahara}{2008}]{hashimoto2008construction}
Hashimoto, C.\BBACOMMA\ \BBA\ Kawahara, D. \BBOP 2008\BBCP.
\newblock \BBOQ Construction of an Idiom Corpus and its Application to Idiom
  Identification based on WSD Incorporating Idiom-Specific Features.\BBCQ\
\newblock In {\Bem Proceedings of the 2008 Conference on Empirical Methods in
  Natural Language Processing (EMNLP 2008)}, \mbox{\BPGS\ 992--1001}.

\bibitem[\protect\BCAY{林部\JBA 小町\JBA 松本\JBA 隅田}{林部 \Jetal
  }{2012}]{hayashibe2012}
林部祐太\JBA 小町守\JBA 松本裕治\JBA 隅田飛鳥 \BBOP 2012\BBCP.
\newblock 日本語テキストに対する述語語義と意味役割のアノテーション.\
\newblock \Jem{言語処理学会第 18 回年次大会論文集}, \mbox{\BPGS\ 397--400}.

\bibitem[\protect\BCAY{Hovy, Marcus, Palmer, Ramshaw, \BBA\ Weischedel}{Hovy
  et~al.}{2006}]{hovy2006ontonotes}
Hovy, E., Marcus, M., Palmer, M., Ramshaw, L., \BBA\ Weischedel, R. \BBOP
  2006\BBCP.
\newblock \BBOQ OntoNotes: the 90\% Solution.\BBCQ\
\newblock In {\Bem Proceedings of the Human Language Technology Conference of
  the North American Chapter of the ACL (HLT-NAACL 2006)}, \mbox{\BPGS\
  57--60}.

\bibitem[\protect\BCAY{飯田\JBA 小町\JBA 井之上\JBA 乾\JBA 松本}{飯田 \Jetal
  }{2005}]{ntcguideline}
飯田龍\JBA 小町守\JBA 井之上直也\JBA 乾健太郎\JBA 松本裕治.
\newblock 照応関係タグ付けマニュアル第 0.02.1 版.\
\newblock
  \Turl{https://\linebreak[2]www.\linebreak[2]cl.\linebreak[2]cs.\linebreak[2]titech.\linebreak[2]ac.jp/\linebreak[2]\~{}ryu-i/\linebreak[2]coreference\_tag.html}.
\newblock Accessed: 2014-02-19.

\bibitem[\protect\BCAY{飯田\JBA 小町\JBA 乾\JBA 松本}{飯田 \Jetal
  }{2008}]{noun2008}
飯田龍\JBA 小町守\JBA 乾健太郎\JBA 松本裕治 \BBOP 2008\BBCP.
\newblock 名詞化された事態表現への意味的注釈付け.\
\newblock \Jem{言語処理学会第 14 回年次大会論文集}, \mbox{\BPGS\ 277--280}.

\bibitem[\protect\BCAY{飯田\JBA 小町\JBA 井之上\JBA 乾\JBA 松本}{飯田 \Jetal
  }{2010}]{飯田龍2010述語項構造}
飯田龍\JBA 小町守\JBA 井之上直也\JBA 乾健太郎\JBA 松本裕治 \BBOP 2010\BBCP.
\newblock 述語項構造と照応関係のアノテーション:NAIST
  テキストコーパス構築の経験から.\
\newblock \Jem{自然言語処理}, {\Bbf 17}  (2), \mbox{\BPGS\ 25--50}.

\bibitem[\protect\BCAY{Iida, Komachi, Inui, \BBA\ Matsumoto}{Iida
  et~al.}{2007}]{iida2007annotating}
Iida, R., Komachi, M., Inui, K., \BBA\ Matsumoto, Y. \BBOP 2007\BBCP.
\newblock \BBOQ Annotating a Japanese Text Corpus with Predicate-Argument and
  Coreference Relations.\BBCQ\
\newblock In {\Bem Proceedings of the Linguistic Annotation Workshop (LAW
  '07)}, \mbox{\BPGS\ 132--139}.

\bibitem[\protect\BCAY{Iida \BBA\ Poesio}{Iida \BBA\ Poesio}{2011}]{Iida2011}
Iida, R.\BBACOMMA\ \BBA\ Poesio, M. \BBOP 2011\BBCP.
\newblock \BBOQ {A Cross-Lingual ILP Solution to Zero Anaphora
  Resolution}.\BBCQ\
\newblock In {\Bem Proceedings of the 49th Annual Meeting of the Association
  for Computational Linguistics (ACL 2011)}, \mbox{\BPGS\ 804--813}.

\bibitem[\protect\BCAY{池原\JBA 宮崎\JBA 白井\JBA 横尾\JBA 中岩\JBA 小倉\JBA
  大山\JBA 林}{池原 \Jetal }{1997}]{nttlexicon}
池原悟\JBA 宮崎正弘\JBA 白井諭\JBA 横尾昭男\JBA 中岩浩巳\JBA 小倉健太郎\JBA
  大山芳史\JBA 林良彦\JEDS\ \BBOP 1997\BBCP.
\newblock \Jem{日本語語彙大系}.
\newblock 岩波書店.

\bibitem[\protect\BCAY{泉\JBA 今村\JBA 菊井\JBA 藤田\JBA 佐藤}{泉 \Jetal
  }{2009}]{izumi2009}
泉朋子\JBA 今村賢治\JBA 菊井玄一郎\JBA 藤田篤\JBA 佐藤理史 \BBOP 2009\BBCP.
\newblock 正規化を指向した機能動詞表現の述部言い換え.\
\newblock \Jem{言語処理学会第 15 回年次大会論文集}, \mbox{\BPGS\ 264--267}.

\bibitem[\protect\BCAY{情報処理振興事業協会技術センター}{情報処理振興事業協会技術センター}{1987}]{ipal}
情報処理振興事業協会技術センター \BBOP 1987\BBCP.
\newblock \Jem{計算機用日本語基本動詞辞書 IPAL (basic verbs): 辞書編}.
\newblock 情報処理振興事業協会技術センター.

\bibitem[\protect\BCAY{河原\JBA 黒橋\JBA 橋田}{河原 \Jetal
  }{2002}]{河原大輔2002関係}
河原大輔\JBA 黒橋禎夫\JBA 橋田浩一 \BBOP 2002\BBCP.
\newblock 「関係」 タグ付きコーパスの作成.\
\newblock \Jem{言語処理学会第 8 回年次大会論文集}, \mbox{\BPGS\ 495--498}.

\bibitem[\protect\BCAY{河原\JBA 笹野\JBA 黒橋\JBA 橋田}{河原 \Jetal
  }{2005}]{ktcguideline}
河原大輔\JBA 笹野遼平\JBA 黒橋禎夫\JBA 橋田浩一.
\newblock 格・省略・共参照タグ付けの基準.\
\newblock
  \Turl{http://nlp.\linebreak[2]ist.\linebreak[2]i.\linebreak[2]kyoto-u.\linebreak[2]ac.\linebreak[2]jp/\linebreak[2]nl-resource/\linebreak[2]corpus/KyotoCorpus4.0/\linebreak[2]doc/\linebreak[2]rel\_guideline.pdf}.
\newblock Accessed: 2014-02-19.

\bibitem[\protect\BCAY{Kawahara \BBA\ Kurohashi}{Kawahara \BBA\
  Kurohashi}{2006}]{kawahara2006case}
Kawahara, D.\BBACOMMA\ \BBA\ Kurohashi, S. \BBOP 2006\BBCP.
\newblock \BBOQ Case Frame Compilation from the Web using High-Performance
  Computing.\BBCQ\
\newblock In {\Bem Proceedings of the 5th International Conference on Language
  Resources and Evaluation (LREC 2006)}, \mbox{\BPGS\ 1344--1347}.

\bibitem[\protect\BCAY{Kawahara, Kurohashi, \BBA\ Hasida}{Kawahara
  et~al.}{2002}]{kawahara2002construction}
Kawahara, D., Kurohashi, S., \BBA\ Hasida, K. \BBOP 2002\BBCP.
\newblock \BBOQ Construction of a Japanese Relevance-tagged Corpus.\BBCQ\
\newblock In {\Bem Proceedings of the 3rd International Conference on Language
  Resources and Evaluation (LREC 2002)}, \mbox{\BPGS\ 2008--2013}.

\bibitem[\protect\BCAY{国立国語研究所}{国立国語研究所}{}]{bccwj}
国立国語研究所.
\newblock 現代日本語書き言葉均衡コーパス(BCCWJ).\
\newblock
  \Turl{http://\linebreak[2]www.\linebreak[2]ninjal.\linebreak[2]ac.\linebreak[2]jp/\linebreak[2]corpus\_\linebreak[2]center/\linebreak[2]bccwj/}.
\newblock Accessed: 2014-02-19.

\bibitem[\protect\BCAY{小町\JBA 飯田}{小町\JBA 飯田}{2011}]{komachi2011}
小町守\JBA 飯田龍 \BBOP 2011\BBCP.
\newblock BCCWJ に対する述語項構造と照応関係のアノテーション.\
\newblock \Jem{日本語コーパス平成 22 年度公開ワークショップ}, \mbox{\BPGS\
  325--330}.

\bibitem[\protect\BCAY{黒橋\JBA 長尾}{黒橋\JBA
  長尾}{1997}]{KUROHASHISadao:1997-06-24}
黒橋禎夫\JBA 長尾眞 \BBOP 1997\BBCP.
\newblock 京都大学テキストコーパス・プロジェクト.\
\newblock \Jem{言語処理学会第 3 回年次大会論文集}, \mbox{\BPGS\ 115--118}.

\bibitem[\protect\BCAY{Laparra \BBA\ Rigau}{Laparra \BBA\
  Rigau}{2013}]{laparra2013impar}
Laparra, E.\BBACOMMA\ \BBA\ Rigau, G. \BBOP 2013\BBCP.
\newblock \BBOQ ImpAr: A Deterministic Algorithm for Implicit Semantic Role
  Labelling.\BBCQ\
\newblock In {\Bem Proceedings of the 51st Annual Meeting of the Association
  for Computational Linguistics (ACL 2013)}, \mbox{\BPGS\ 1180--1189}.

\bibitem[\protect\BCAY{Liu \BBA\ Gildea}{Liu \BBA\
  Gildea}{2010}]{liu2010semantic}
Liu, D.\BBACOMMA\ \BBA\ Gildea, D. \BBOP 2010\BBCP.
\newblock \BBOQ Semantic Role Features for Machine Translation.\BBCQ\
\newblock In {\Bem Proceedings of the 23rd International Conference on
  Computational Linguistics (Coling 2010)}, \mbox{\BPGS\ 716--724}.

\bibitem[\protect\BCAY{Loper, Yi, \BBA\ Palmer}{Loper
  et~al.}{2007a}]{loper2007combining}
Loper, E., Yi, S.-T., \BBA\ Palmer, M. \BBOP 2007a\BBCP.
\newblock \BBOQ Combining Lexical Resources: Mapping between PropBank and
  VerbNet.\BBCQ\
\newblock In {\Bem Proceedings of the 7th International Workshop on
  Computational Semantics (IWCS-7)}, \mbox{\BPGS\ 1--12}.

\bibitem[\protect\BCAY{Loper, Yi, \BBA\ Palmer}{Loper et~al.}{2007b}]{semlink}
Loper, E., Yi, S.-T., \BBA\ Palmer, M.
\newblock \BBOQ SemLink 1.1.\BBCQ\
\newblock
  \Turl{http://\linebreak[2]verbs.\linebreak[2]colorado.\linebreak[2]edu/\linebreak[2]semlink/}.
\newblock Accessed: 2014-02-19.

\bibitem[\protect\BCAY{Marcus, Kim, Marcinkiewicz, MacIntyre, Bies, Ferguson,
  Katz, \BBA\ Schasberger}{Marcus et~al.}{1994}]{marcus1994penn}
Marcus, M., Kim, G., Marcinkiewicz, M.~A., MacIntyre, R., Bies, A., Ferguson,
  M., Katz, K., \BBA\ Schasberger, B. \BBOP 1994\BBCP.
\newblock \BBOQ The Penn Treebank: Annotating Predicate Argument
  Structure.\BBCQ\
\newblock In {\Bem Proceedings of the Workshop on Human Language Technology},
  \mbox{\BPGS\ 114--119}.

\bibitem[\protect\BCAY{Marcus, Marcinkiewicz, \BBA\ Santorini}{Marcus
  et~al.}{1993}]{marcus1993building}
Marcus, M.~P., Marcinkiewicz, M.~A., \BBA\ Santorini, B. \BBOP 1993\BBCP.
\newblock \BBOQ Building a Large Annotated Corpus of English: The Penn
  Treebank.\BBCQ\
\newblock {\Bem Computational Linguistics}, {\Bbf 19}  (2), \mbox{\BPGS\
  313--330}.

\bibitem[\protect\BCAY{M\`{a}rquez, Carreras, Litkowski, \BBA\
  Stevenson}{M\`{a}rquez et~al.}{2008}]{marquez2008srl}
M\`{a}rquez, L., Carreras, X., Litkowski, K.~C., \BBA\ Stevenson, S. \BBOP
  2008\BBCP.
\newblock \BBOQ {Semantic Role Labeling: an Introduction to the Special
  Issue}.\BBCQ\
\newblock {\Bem Computational Linguistics}, {\Bbf 34}  (2), \mbox{\BPGS\
  145--159}.

\bibitem[\protect\BCAY{Matsubayashi, Miyao, \BBA\ Aizawa}{Matsubayashi
  et~al.}{2012}]{MATSUBAYASHI12.941}
Matsubayashi, Y., Miyao, Y., \BBA\ Aizawa, A. \BBOP 2012\BBCP.
\newblock \BBOQ Building Japanese Predicate-argument Structure Corpus using
  Lexical Conceptual Structure.\BBCQ\
\newblock In {\Bem Proceedings of the 8th International Conference on Language
  Resources and Evaluation (LREC 2012)}, \mbox{\BPGS\ 1554--1558}.

\bibitem[\protect\BCAY{Matsumoto}{Matsumoto}{1996}]{matsumoto1996syntactic}
Matsumoto, Y. \BBOP 1996\BBCP.
\newblock \BBOQ A Syntactic Account of Light Verb Phenomena in Japanese.\BBCQ\
\newblock {\Bem Journal of East Asian Linguistics}, {\Bbf 5}  (2), \mbox{\BPGS\
  107--149}.

\bibitem[\protect\BCAY{松吉\JBA 佐藤\JBA 宇津呂}{松吉 \Jetal
  }{2007}]{Matsuyoshi2007}
松吉俊\JBA 佐藤理史\JBA 宇津呂武仁 \BBOP 2007\BBCP.
\newblock 日本語機能表現辞書の編纂.\
\newblock \Jem{自然言語処理}, {\Bbf 14}  (5), \mbox{\BPGS\ 123--146}.

\bibitem[\protect\BCAY{Meyers, Reeves, Macleod, Szekely, Zielinska, Young,
  \BBA\ Grishman}{Meyers et~al.}{2004}]{meyers2004nombank}
Meyers, A., Reeves, R., Macleod, C., Szekely, R., Zielinska, V., Young, B.,
  \BBA\ Grishman, R. \BBOP 2004\BBCP.
\newblock \BBOQ The NomBank Project: An Interim Report.\BBCQ\
\newblock In {\Bem HLT-NAACL 2004 Workshop: Frontiers in Corpus Annotation},
  \mbox{\BPGS\ 24--31}.

\bibitem[\protect\BCAY{Moor, Roth, \BBA\ Frank}{Moor
  et~al.}{2013}]{frankpredicate}
Moor, T., Roth, M., \BBA\ Frank, A. \BBOP 2013\BBCP.
\newblock \BBOQ Predicate-specific Annotations for Implicit Role Binding:
  Corpus Annotation, Data Analysis and Evaluation Experiments.\BBCQ\
\newblock In {\Bem Proceedings of the 10th International Conference on
  Computational Semantics (IWCS 2013): Short Papers}, \mbox{\BPGS\ 60--65}.

\bibitem[\protect\BCAY{森田\JBA 松木}{森田\JBA 松木}{1989}]{Morita1989}
森田良行\JBA 松木正恵 \BBOP 1989\BBCP.
\newblock \Jem{日本語表現文型—用例中心・複合辞の意味と用法}.
\newblock アルク.

\bibitem[\protect\BCAY{村木}{村木}{1991}]{村木新次郎1991日本語動詞の諸相}
村木新次郎 \BBOP 1991\BBCP.
\newblock \Jem{日本語動詞の諸相}.
\newblock ひつじ書房.

\bibitem[\protect\BCAY{村木\JBA 青山\JBA 六条\JBA 村田}{村木 \Jetal
  }{1984}]{muraki:84}
村木新次郎\JBA 青山文啓\JBA 六条範俊\JBA 村田賢一 \BBOP 1984\BBCP.
\newblock 辞書における格情報の記述.\
\newblock \Jem{情報処理学会研究報告. 自然言語処理研究会報告}, {\Bbf 1984}
  (46), \mbox{\BPGS\ 1--8}.

\bibitem[\protect\BCAY{日本語記述文法研究会}{日本語記述文法研究会}{2009}]{contemporaryJapanese2}
日本語記述文法研究会\JED\ \BBOP 2009\BBCP.
\newblock \Jem{現代日本語文法}, 2\JVOL.
\newblock くろしお出版.

\bibitem[\protect\BCAY{西山}{西山}{2003}]{nishiyama2003}
西山佑司 \BBOP 2003\BBCP.
\newblock \Jem{日本語名詞句の意味論と語用論:指示的名詞句の非指示的名詞句}.
\newblock ひつじ書房.

\bibitem[\protect\BCAY{小原}{小原}{2013}]{ohara2013}
小原京子 \BBOP 2013\BBCP.
\newblock 日本語フレームネット:文意理解のためのコーパスアノテーション.\
\newblock \Jem{言語処理学会第19回年次大会論文集}, \mbox{\BPGS\ 166--169}.

\bibitem[\protect\BCAY{Palmer, Kingsbury, \BBA\ Gildea}{Palmer
  et~al.}{2005}]{palmer2005pba}
Palmer, M., Kingsbury, P., \BBA\ Gildea, D. \BBOP 2005\BBCP.
\newblock \BBOQ {The Proposition Bank: An Annotated Corpus of Semantic
  Roles}.\BBCQ\
\newblock {\Bem Computational Linguistics}, {\Bbf 31}  (1), \mbox{\BPGS\
  71--106}.

\bibitem[\protect\BCAY{Pradhan, Loper, Dligach, \BBA\ Palmer}{Pradhan
  et~al.}{2007}]{pradhan2007semeval}
Pradhan, S.~S., Loper, E., Dligach, D., \BBA\ Palmer, M. \BBOP 2007\BBCP.
\newblock \BBOQ SemEval-2007 Task 17: English Lexical Sample, SRL and All
  Words.\BBCQ\
\newblock In {\Bem Proceedings of the 4th International Workshop on Semantic
  Evaluations (SemEval 2007)}, \mbox{\BPGS\ 87--92}.

\bibitem[\protect\BCAY{Ruppenhofer, Ellsworth, Petruck, Johnson, \BBA\
  Scheffczyk}{Ruppenhofer et~al.}{2010}]{Johnson2003}
Ruppenhofer, J., Ellsworth, M., Petruck, M.~R.~L., Johnson, C.~R., \BBA\
  Scheffczyk, J. \BBOP 2010\BBCP.
\newblock {\Bem {FrameNet II: Extended Theory and Practice}}.
\newblock Berkeley FrameNet Release.

\bibitem[\protect\BCAY{Sasano, Kawahara, Kurohashi, \BBA\ Okumura}{Sasano
  et~al.}{2013}]{sasano2013}
Sasano, R., Kawahara, D., Kurohashi, S., \BBA\ Okumura, M. \BBOP 2013\BBCP.
\newblock \BBOQ Automatic Knowledge Acquisition for Case Alternation between
  the Passive and Active Voices in Japanese.\BBCQ\
\newblock In {\Bem Proceedings of the 2013 Conference on Empirical Methods in
  Natural Language Processing (EMNLP 2013)}, \mbox{\BPGS\ 1213--1223}.

\bibitem[\protect\BCAY{佐藤}{佐藤}{2007}]{佐藤理史:2007-03-28}
佐藤理史 \BBOP 2007\BBCP.
\newblock 基本慣用句五種対照表の作成.\
\newblock \Jem{情報処理学会研究報告. 自然言語処理研究会報告}, {\Bbf 2007}
  (35), \mbox{\BPGS\ 1--6}.

\bibitem[\protect\BCAY{Shen \BBA\ Lapata}{Shen \BBA\
  Lapata}{2007}]{shen2007using}
Shen, D.\BBACOMMA\ \BBA\ Lapata, M. \BBOP 2007\BBCP.
\newblock \BBOQ Using Semantic Roles to Improve Question Answering.\BBCQ\
\newblock In {\Bem the 2007 Joint Conference on Empirical Methods in Natural
  Language Processing and Computational Natural Language Learning (EMNLP-CoNLL
  2007)}, \mbox{\BPGS\ 12--21}.

\bibitem[\protect\BCAY{Silberer \BBA\ Frank}{Silberer \BBA\
  Frank}{2012}]{Silberer:2012:CIR:2387636.2387638}
Silberer, C.\BBACOMMA\ \BBA\ Frank, A. \BBOP 2012\BBCP.
\newblock \BBOQ Casting Implicit Role Linking as an Anaphora Resolution
  Task.\BBCQ\
\newblock In {\Bem Proceedings of the 1st Joint Conference on Lexical and
  Computational Semantics (*SEM)}, \mbox{\BPGS\ 1--10}.

\bibitem[\protect\BCAY{Taira, Fujita, \BBA\ Nagata}{Taira
  et~al.}{2008}]{taira2008japanese}
Taira, H., Fujita, S., \BBA\ Nagata, M. \BBOP 2008\BBCP.
\newblock \BBOQ A Japanese Predicate Argument Structure Analysis using Decision
  Lists.\BBCQ\
\newblock In {\Bem Proceedings of the 2008 Conference on Empirical Methods in
  Natural Language Processing (EMNLP 2008)}, \mbox{\BPGS\ 523--532}.

\bibitem[\protect\BCAY{竹内\JBA 乾\JBA 藤田\JBA 竹内\JBA 阿部}{竹内 \Jetal
  }{2005}]{takeuchi2005}
竹内孔一\JBA 乾健太郎\JBA 藤田篤\JBA 竹内奈央\JBA 阿部修也 \BBOP 2005\BBCP.
\newblock 分類の根拠を明示した動詞語彙概念構造辞書の構築.\
\newblock \Jem{情報処理学会研究報告. 自然言語処理研究会報告}, {\Bbf 2005}
  (94), \mbox{\BPGS\ 123--130}.

\bibitem[\protect\BCAY{竹内\JBA 上野}{竹内\JBA 上野}{2013}]{takeuchi2013}
竹内孔一\JBA 上野真幸 \BBOP 2013\BBCP.
\newblock
  日本語コーパスに対する動詞項構造シソーラスの概念と意味役割のアノテーション.\
\newblock \Jem{言語処理学会第19回年次大会論文集}, \mbox{\BPGS\ 162--165}.

\bibitem[\protect\BCAY{山口}{山口}{2013}]{yamaguchi2013}
山口昌也 \BBOP 2013\BBCP.
\newblock 複合動詞用例データベースの構築と活用.\
\newblock \Jem{国語研プロジェクトレビュー}, {\Bbf 4}  (1), \mbox{\BPGS\
  61--69}.

\bibitem[\protect\BCAY{Yoshikawa, Asahara, \BBA\ Matsumoto}{Yoshikawa
  et~al.}{2011}]{Yoshikawa2011}
Yoshikawa, K., Asahara, M., \BBA\ Matsumoto, Y. \BBOP 2011\BBCP.
\newblock \BBOQ {Jointly Extracting Japanese Predicate-Argument Relation with
  Markov Logic}.\BBCQ\
\newblock In {\Bem Proceedings of the 5th International Joint Conference on
  Natural Language Processing (IJCNLP 2011)}, \mbox{\BPGS\ 1125--1133}.

\end{thebibliography}

\clearpage
\begin{biography}
\bioauthor{松林優一郎}{
1981年生.2010年東京大学大学院情報理工学系研究科・コンピュータ科学専攻博士課程修了.
情報理工学博士.同年より国立情報学研究所・特任研究員.2012年より東北大学
大学院情報科学研究科・研究特任助教.意味解析の研究に従事.人工知能学会,ACL各会員.
}
\bioauthor{飯田  龍}{
1980年生.2007年奈良先端科学技術大学院大学情報科学研究科博士期課程修了.
同年より奈良先端科学技術大学院大学情報科学研究科特任助教.
2008年12月より東京工業大学大学院情報理工学研究科助教.現在にいたる.
博士(工学).自然言語処理の研究に従事.情報処理学会,日本教育工学会各会員.
}
\bioauthor{笹野 遼平}{
2009年東京大学大学院情報理工学系研究科博士課程修了.博士(情報理工学).2010年より東京工業大学精密工学研究所助教.照応解析,述語項構造解析の研究に従事.情報処理学会,人工知能学会,ACL会員.
}
\bioauthor{横野  光}{
2003年岡山大学工学部情報工学科卒業.2008年同大大学院自然科学研究科産業創成
工学専攻単位取得退学.同年東京工業大学精密工学研究所研究員,2011年国立情
報学研究所特任研究員,現在に至る.博士(工学).自然言語処理の研究に従事.
情報処理学会,人工知能学会各会員.
}
\bioauthor{松吉  俊}{
1980 年生.2008 年京都大学大学院情報学研究科博士後期課程修了.同年より奈良先端科学技術大学院大学情報科学研究科特任助教.2011 年より山梨大学大学院医学工学総合研究部助教.現在にいたる.博士(情報学).自然言語処理の研究に従事.情報処理学会会員.
}
\bioauthor{藤田  篤}{
1977年生.
2005年奈良先端科学技術大学院大学情報科学研究科博士後期課程修了.
博士(工学).
京都大学大学院情報学研究科産学官連携研究員,名古屋大学大学院工学研究科助手,同助教を経て,
2009年より公立はこだて未来大学システム情報科学部准教授.
現在にいたる.
自然言語処理,主に言い換え表現の生成と認識,機械翻訳の研究に従事.
人工知能学会,情報処理学会,ACL 各会員.
}
\bioauthor{宮尾 祐介}{
1998年東京大学理学部情報科学科卒業.2006年同大学大学院にて博士号(情報理工学)取得.2001年より同大学にて助手,
のち助教.2010年より国立情報学研究所准教授.構文解析とその応用の研究に従事.人工知能学会,情報処理学会,ACL 各会員.
}
\bioauthor{乾 健太郎}{
1995年東京工業大学大学院情報理工学研究科博士課程修了.同研究科助手,九州工業大学助教授,奈良先端科学技術大学院大学助教授を経て,2010年より東北大学大学情報科学研究科教授,現在に至る.博士(工学).自然言語処理の研究に従事.情報処理学会,人工知能学会,ACL, AAAI各会員.
}
\end{biography}


\biodate



\end{document}
