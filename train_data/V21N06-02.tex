    \documentclass[japanese]{jnlp_1.4}

\usepackage{jnlpbbl_1.3}
\usepackage[dvipdfm]{graphicx}
\usepackage{amsmath}
\usepackage{pifont}
\usepackage{okumacro}
\usepackage{plext}
\usepackage[dvipdfm]{colortbl}
\usepackage{udline}
\setulminsep{1.2ex}{0.2ex}
\newcommand{\RaiseCell}[2]{}
\newcommand{\MaruOne}{}
\newcommand{\MaruTwo}{}


\Volume{21}
\Number{6}
\Month{December}
\Year{2014}

\received{2014}{1}{22}
\revised{2014}{6}{24}
\accepted{2014}{8}{5}

\setcounter{page}{1133}

\jtitle{「コーパスベース国語辞典」構築のための\\
	「古風な語」の分析と記述}
\jauthor{柏野和佳子\affiref{Author_1}\affiref{Author_2} \and 奥村  学\affiref{Author_3}}
\jabstract{
従来の紙版の国語辞典はコンパクトにまとめることが優先され,用例の記述は厳選され,必要最小限にとどめられていた.しかし,電子化編集が容易になり,電子化された国語辞典データや種々のコーパスが活用できるようになった今,豊富な用例を増補した電子化版国語辞典の構築が可能になった.そうした電子化版国語辞典は,人にも計算機にも有用性の高いものと期待される.著者らはその用例記述の際に見出し語のもつ文体的特徴を明記する方法を提案し,より利用価値の高い,電子化版の「コーパスベース国語辞典」の構築を目指している.文体的特徴の記述は,語の理解を助け,文章作成時にはその語を用いる判断の指標になり得るため,作文指導や日本語教育,日本語生成処理といった観点からの期待も高い.本論文では,古さを帯びながらも現代語として用いられる「古風な語」を取り上げる.これに注目する理由は,三点ある.一点目は,現代語の中で用いられる「古風な語」は少なくないにも関わらず,「古語」にまぎれ辞書記述に取り上げ損なってしまう危険性のあるものであること.二点目は,その「古風な語」には,文語の活用形をもつなど,その文法的な扱いに注意の必要なものがあること.三点目は,「古さ」という文体的特徴を的確かつ,効果的に用いることができるよう,十分な用法説明が必要な語であるということ,である.そこで,本論文では,これら三点に留意して「古風な語」の用法をその使用実態に即して分析し,その辞書記述を提案する.はじめに,現行国語辞典5種における「古風な語」の扱いを概観する.次に,「古風な語」の使用実態を『現代日本語書き言葉均衡コーパス』に収録される図書館サブコーパスを用いて分析し,「古風な語」の使用を,(1) 古典の引用,(2) 明治期から戦前まで,(3) 時代・歴史小説,(4) 現代文脈,に4分類する.そして,その 4 分類に基づく「コーパスベース国語辞典」の辞書記述方法を提案する.このような辞書記述は例えば,作文指導や日本語教育,日本語生成処理の際の語選択の参考になるものと期待される.
}
\jkeywords{国語辞典,辞書,古風な語,文体,コーパス}

\etitle{Usage Analysis of Old-Fashioned Words based on the Balanced Corpus of Contemporary Written Japanese}
\eauthor{Wakako Kashino\affiref{Author_1}\affiref{Author_2} \and Manabu Okumura\affiref{Author_3}} 
\eabstract{
\pretolerance=10000
A fundamental issue in compiling a Japanese dictionary is selecting lexical 
entries and describing the meaning(s) and use cases for the selected 
entries. Because some old-fashioned Japanese words continue to be used even 
now, modern Japanese dictionaries usually include certain old-fashioned 
words. However, up to now, no systematic study has investigated the 
selection and usage description of old-fashioned words. Therefore, here we 
first review five already-published modern Japanese dictionaries and clarify 
the characteristics and variations among them. Subsequently, we propose four 
categories of old-fashioned Japanese words in terms of the nature and 
chronological features of the text where those words appear. According to 
the categorization, we analyze the use cases of the old-fashioned words in 
the ``Balanced Corpus of Contemporary Written Japanese.'' Finally, we 
discuss a systematic methodology of lexical description for such entries, 
with a typical example.
}
\ekeywords{Japanese Dictionary, Compiling Dictionary, Old-fashioned Word, Style, \linebreak Corpus}

\headauthor{柏野,奥村}
\headtitle{「コーパスベース国語辞典」構築のための「古風な語」の分析と記述}

\affilabel{Author_1}{国立国語研究所}{National Institute for Japanese Language and Linguistics}
\affilabel{Author_2}{東京工業大学大学院総合理工学研究科}{Tokyo Institute of Technology}
\affilabel{Author_3}{東京工業大学精密工学研究所}{Precision and Intelligence Laboratory, Tokyo Institute of Technology}



\begin{document}
\maketitle

\section{はじめに}

従来の紙版の国語辞典\footnote{国語辞典は,対象や規模により多種類のものが存在する.著者らが研究対象としているものは,小型国語辞典(6〜9万語収録)と呼ばれ,「現代生活に必要な語,使用頻度の高い語」の収録と記述とに重きがおかれているものである(柏野2009).} は紙幅の制約などから,用例の記述は必要最小限に厳選されていた.しかし,電子化編集が容易になり,国語辞典データ\footnote{『岩波国語辞典』(岩波書店)はCD-ROM版が市販され,さらに,電子化データ(岩波国語辞典第5版タグ付きコーパス2004)が研究用に公開されている(http://www.gsk.or.jp/catalog.html).} や種々のコーパスが活用できるようになった今,新たな「コーパスベース国語辞典」の構築が可能になった.ここで,「コーパスベース国語辞典」とは,従来の紙版の国語辞典の記述に加え,コーパス分析から得られる豊富な用例,そのほか言語のさまざまな辞書的情報を詳細に記述する,電子テキスト版の国語辞典のことである.紙幅によって制約されていた記述量の制限をなくし,辞書記述の充実をはかることがねらいである.そうした「コーパスベース国語辞典」は,人にも計算機にも有用性の高いものと期待される.しかし,単に情報を増やせばよいというものではなく,有用な情報を的確に整理して記述することが不可欠である.著者らはそのような観点から,その用例記述の際に見出し語のもつ文体的特徴を明記することにより,より利用価値の高い「コーパスベース国語辞典」を構築することを目指している.文体的特徴の記述は,語の理解を助け,文章作成時にはその語を用いる判断の指標になり得るため,作文指導や日本語教育,日本語生成処理といった観点からの期待も高い.

従来の国語辞典では,文体的特徴として,「古語,古語的,古風,雅語,雅語的,文語,文語的,文章語,口語,俗語」などのように,位相と呼ばれる注記情報が付与されてきた\footnote{そのほか,使用域についてその語が用いられる専門分野を示すことが試みられている.}.本論文では,そのような注記が付与されるような語のうち,「古さ」を帯びながら現代語として用いられている語に着目する.本論文ではそのような語を「古風な語」と呼び,次の二点を満たすものと定義する.

\begin{itemize}
\item[(a)]
「時代・歴史小説」を含めて現代で使用が見られる.
\item[(b)]
明治期以前,あるいは,戦前までの使用が見られる.
\end{itemize}

(a) は,現代ではほとんど使われなくなっている古語と区別するものである.(b)は「古風な語」の「古さ」の範囲を定めるものである.本論文では,現代語と古語との境と一般にされている明治期以前までを一つの区切りにする.また,戦前と戦後とで文体変化が大きいと考えられるため,明治期から戦前までという区切りも設ける.しかしながら,一般には,戦前までさかのぼらずとも,事物の入れ替わりや,流行の入れ替わりにより,減っていったもの,なくなっていったものに「古さ」を感じることは多い.例えば,「ポケベル」「黒電話」「ワープロ」「こたつ」などである.こういった,近年急速に古さを感じるようになっている一連の語の分析も辞書記述の一つの課題と考えるが,本論文で取り上げる「古風な語」は,戦前までさかのぼって「古さ」を捉えることとし,それ以外とは区別する.

「古風な語」に注目する理由は,三点ある.一点目は,現代語の中で用いられる「古風な語」は少なくないにも関わらず,「古語」にまぎれ辞書記述に取り上げ損なってしまう危険性のあるものであること.二点目は,その「古風な語」には,文語の活用形をもつなど,その文法的な扱いに注意の必要なものがあること.三点目は,「古風」という文体的特徴を的確かつ,効果的に用いることができるよう,十分な用法説明が必要な語であるということ,である.

「古風な語」には,例えば,「さ【然】」がある.これは,「状態・様子がそうだという意を表す語。」(『岩波国語辞典』第7版,岩波書店)であり,現代では,「さほど」「さまで」「さばかり」「さしも」「さも」…のように結合して用いられる.その一つ,「さもありなん」(そうなるのがもっともだ)は,「さも」+「あり」+文語助動詞「ぬ」の未然形「な」+文語助動詞「む」である「ん」,から成る連語である.枕草子(128段)に,「大口また、長さよりは口ひろければさもありなむ」と使われている.一方,国立国語研究所『現代日本語書き言葉均衡コーパス』(Balanced Corpus of Contemporary Written Japanese; 以下,BCCWJ と記す\footnote{BCCWJ の詳細は,山崎 (2009, 2011),前川 (2008, 2013)を参照.})には,全体で34件の用例があり,いずれも,現代文脈での使用である.「まさか和久さんが指導員として復帰してるなんて思わなかったから。でも、\textbf{さもありなん}、という気もする。」(君塚良一(1950年代生まれ)/丹後達臣,『踊る大捜査線スペシャル』扶桑社,1998年)などである.同じように,「なきにしもあらず」「いわずもがな」「推して知るべし」…など,現代文脈で用いられる文語調の表現は他にもあり,BCCWJの現代文脈でそれぞれの用例を得ることができる.

「古風な語」は,これまでにも現代日本語における特徴的な語として着目されてきた.実際,多くの国語辞典では,現代文脈で使われる古さを帯びている語については,「古語」とはせず,「古語的」「古風」「雅語」「文語」「文語的」といった注記が付されている.しかし,これらの注記を横断的に俯瞰することや,「古風な語」の使用実態とその辞書記述との関連を検討する試みは,これまで行われていなかった.

以上の問題を解決するために,本論文では,まずは「古風な語」の調査語として,電子化版が市販されている『CD-ROM 岩波 日本語表現辞典—国語・漢字・類語—』(2002年)収録の『岩国』(第6版)に「古語的」「古風」と注記されている語を用い,現在刊行されている国語辞典で「古風な語」がどのように取り上げられているかを横断的に俯瞰する.次に,現代語のコーパスであるBCCWJに収録されている約3,000万語分の書籍テキストを用いて,その使用実態を分析し(柏野,奥村2010,2011),それに基づき,文脈の特徴や用例を『コーパスベース国語辞典』に記述する方法を提案し,その有用性を論じる.


\section{「古風な語」の現行辞典での扱い}

\subsection{現行の国語辞典}

多くの国語辞典で,「古風な語」に関し,「古風」あるいは,「古語的」「雅語」「文語」「文語的」といった注記を付す記述が試みられている.積極的に「古風な語」の収録と記述が試みられているものは『三省堂国語辞典』(第6版,三省堂,以下『三国』)である.巻頭にある「この辞書の使い方」(p.~15)には,次の注意書きが記されている.

\vspace{1\Cvs}
\begin{quotation}
\noindent
注意 古語ではないかと思われることばが、ずいぶん〔文〕や〔雅〕として出ていますが、これらは評論文〔新聞のコラムをふくむ〕・小説、新聞・雑誌の短歌欄や俳句欄などで実際に使われている\bou{現代語}です.
\end{quotation}
\vspace{1\Cvs}

そして,〔文〕とは文章語〔現代語のうち、文章などに使われる、話しことばとの差の大きいことば〕であり,〔雅〕とは雅語の略であり,「短歌・俳句・歌詞などで、現代でも使われるみやびやかなことばや詩的なおもむきのあることば」であると説明されている.さらに,「年配者の話しことば、小説のせりふ、落語・時代劇などに出てくる古めかしいことばは、〔古風〕として示しました.」とも記されている.つまり,『三国』では,現代語として扱うべき「古風な語」を「文章語,雅語,古風」と分類して取り上げ,記述しているということである.

一方,『新選国語辞典』(第8版,小学館,以下『新選』)のように,「古語」を現代文脈で使われる語だけに収録対象を絞っていないものもある.巻頭にある「この辞典を使う人のために」(p.~6)には,「中学校・高等学校の国語科学習に必要な基本的な古語」は収録したと説明されている.

それら以外の国語辞典においては,「古風な語」について,何らかの注記は付されているものの,その扱いについての明確な定義や説明はない.宮島 (1977) が「「文章語」「文語」「雅語」など,それぞれの辞書で呼んでいるものの内容が,はたして一致しているのかどうかは,凡例にかいていないのでわからない」と指摘したのをはじめ,注記の用語や定義が明確でないことはしばしば指摘されてきた (遠藤1988,後藤2001,前坊2009).

そこで,本論文では,「古風な語」を定義し,そのうえで「古風な語」の実態を調査・分析し,辞書記述案を提案する.


\subsection{現行の英語辞典}

英語辞典では,1970年代頃より,さまざまな文体や使用域をどうしたら辞書で最もよく示し得るかという考察がはじまったと言われている(ハートマン 1984).カウイー (2003)によると,文体や使用域に関する情報が「ラベル」(label) として英語辞典の記述に初めて導入されたのは『The 
Oxford Advanced Learner's Dictionary of Current English』(Oxford Univ Press; 3版, 1974年)であり,体系的に定義されたラベルが導入されたのは『Longman 
Dictionary of Contemporary English』(Longman; 初版,1978年)であるという.以降,多くの英語辞典において,さまざまなラベルが活用されている.本論文で焦点をあてている「古風」に近いラベルは,「Time」に関わるものである.H\"{u}nig(2003)\footnote{調査対象は次の4つの辞書である.Collins Cobuild English Dictionary (1995, 2nd),Longman Dictionary of Contemporary English (1995, 3rd),Oxford Advanced Learners' Dictionary of Current English (1995, 5th),Cambridge International Dictionary of English (1995, 1st).} によると,「Time」に関わるラベルには,「dated」,「old-fashioned」,「old use」,「archaic」の4つがある.これらは,「比較的最近まで使用され,現在高齢者に用いられている」「近現代英語では使われない」「過去の世紀に使用された」ことを示すものとして用いられている.しかしながら,それら用語と定義,付与される語は辞典間に差異のあることが指摘されている (Fedorova, 2004,H\"{u}nig, 2003,Ptaszynski, 2010,Sakwa, 2011).

英語辞典の編集は早くからコーパスベースになっていると言われているが(5.1節),コーパスベースでこれらのラベルを体系的に付与することを論じた文献は見当たらない\footnote{Michael Rundell (2012) ``Labels in Dictionaries'' (http://trac.sketchengine.co.uk/wiki/AK/CourseNotes{\#})}.


\section{「古風な語」の従来の国語辞典における記述}

\subsection{対象の国語辞典}

現行の国語辞典の「古風な語」に対する付与情報の記述の現状を把握するため,次の5種の国語辞典を取り上げ,「古風な語」の見出し語としての採否,および,ラベル・注記の有無とその内容とを比較した.まず,表1に,対象とした辞典名と,各辞典において「古風な語」に該当すると思われる語に付されていたラベル・注記情報の一覧を示す.表1をみると,ラベルや注記情報は似てはいるがそれぞれに特徴のあることがわかる.例えば,『新選』にはほかで「雅語(的)」とされるものに相当するものは設けられていない.

\begin{table}[t]
\caption{調査対象の国語辞典と「古風な語」への注記一覧}
\input{02table01.txt}
\end{table}


\subsection{調査対象語の選定}

調査対象とした国語辞典のうち,唯一電子化版が市販されているのが『岩国』である.そこで,『CD-ROM 岩波 日本語表現辞典—国語・漢字・類語—』(2002年)収録の『岩国』(第6
\linebreak
版)\footnote{紙版は第 7 版新版(2011年)が最新である.} を利用して,付された注記を手掛かりに調査対象語を選定し,それらの語について,ほかの4辞書の記載を確認することとした.

先の表1に示したとおり,『岩国』では,「古語的」「古風」「雅語的」「文語(的)」の4つの注記が付されている.本論文では,これらのうち,「古い」ということだけを表す指標と思われる「古語的」「古風」に着目した.「雅語」には風雅な趣があることや,和歌などに用いられる語という側面があり,「文語」「文章語」には主に文章に用いられる語という側面があるため,今回の調査には含めないこととした.

「古語的」や「古風」という注記は,例えば次の『岩国』の記述例に示すように,下位区分された特定の語義にだけ付されるもの(例:「いたい」)もあれば,語に付されるもの(例:「あいやく」「ころおい」)もある(引用中の太字表示は本論文著者による)\footnote{「いでたち」と「かまえて」の注記は,語につくのか\MaruTwo のみにつくのかがわからないが,これらの場合は,いずれの語も\MaruOne も\MaruTwo も「古語」としてすでにその用法があったものなので,語につく注記と考える.}.また,「古風」の場合は,実際は,「古風」「既に古風」「やや古風」や,「古風な言い方」といったように注記の仕方に差異が見られる(例:「いでたち」「あいやく」「ころおい」「かまえて」).本論文では,これらを区別せず「古風」とあるものをすべてひとくくりにして調査候補語とする.

\textbf{『岩国』(第6版)より:}

{\setlength{\leftskip}{2zw}\setlength{\parindent}{0pt}
\textbf{いた‐い}

一【痛い】神経に耐えがたいほど強い刺激を受けた感じだ.

\MaruOne 刃物で手を切る、虫歯がうずく等、外力・病気で肉体や精神が苦しい.(略)

\MaruTwo しまったと思うほど手ひどい打撃を受けたり、弱点を鋭く突かれたりして、つらい.(略)

二《「—・く」の形で》はなはだしく.ひどく.「自分の不明を—・く恥じる」「—・く感心した」一とは別語源か.\textbf{古語的}.

\textbf{いでたち【出(で)立(ち)】}

\MaruOne (外出する時の)身なり.装い.「たいそうな—だ」

\MaruTwo 旅立ち.しゅったつ.▽\textbf{古風}.

\textbf{あいやく【相役】}

同じ役(についている者).▽\textbf{既に古風}.

\textbf{ころおい【頃おい】}

その折.「晩秋の—」.ころあい.「—を見て訪ねる」 ▽\textbf{やや古風}.

\textbf{かまえて【構えて】}

《副詞的に》\MaruOne 待ちうけて.用意して.心にかけて.

\MaruTwo 決して.「—油断するな」 ▽\textbf{古風な言い方}.
\par}

具体的には,『岩国』には,「古語的」と注記される語は,「いたく【痛く】(〜する)」「いとど」など16語あり,「古風」と注記される語は,「あいやく【相役】」,「あとげつ【後月】」など,151語あった.それらの語について最新版の第 7 版の記述を確認したところ,「古語的」と注記のあった「いやちこ」,「古風」と注記のあった「かえり【回り】」「じする【治する】」「みやばら【宮腹】」の合計4語は,第7版未収録語となっていた.現代語辞書の見出し語として取り上げずともよいという判断のもと収録から外されたものと考えられる.そこで,以上の語を除いた,「古語的」15語(付録:表5の「見出し」を参照),「古風」147語(付録:表6の「見出し」を参照)を本論文の調査対象語として選定した.


\subsection{採否と注記付与状況の比較}

調査対象語の辞典における見出し語としての採否,および,調査対象語へのラベル・注記の有無とその内容とを比較した.表2に『岩国』で「古語的」の注記のあるものより3例の,表3に『岩国』で「古風」の注記のあるものより5例の調査結果を示す.

\begin{table}[t]
\caption{採否と注記付与状況の例(古語的)}
\input{02table02.txt}
\end{table}

\begin{table}[p]
\caption{採否と注記付与状況の例(古風)}
\input{02table03.txt}
\end{table}

表2と表3には,5種の辞典に見出し語として採用されている数,そのうち,古さについてのラベルや注記が付与されている数,ラベルや注記の内容,語釈を示した.調査対象語に対して,辞典によっての見出し語としての採否や注記付与に差異が見られる.例えば,表2の「まがまがしい【禍々・枉々・曲々・凶々しい】 
」(災いをもたらしそうだ。いまわしい。不吉だ。『岩国』)と表3の「よしなに【良しなに】」(よろしく。いいぐあいになるように。『岩国』)は,5種の辞典とも見出し語に採用としている点は一致しているが,注記が付与されているのは2種の辞典のみである.

調査対象語すべてについての採用数と注記付与数の調査結果は,付録の表5,表6に示した.そして,付録の表5,表6には,半数以上の辞典が見出し語として採用し,かつ,注記をつけていた語の数も示した.『岩国』で「古語的」と扱っている15語については6語であり,「古風」と扱っている語148語については62語であった.辞典により編集方針が異なるため,採否や注記の付与に違いが生じるのは当然のことではある.しかしながら,著者らが構築を目指す「コーパスベース国語辞典」は,辞書編集者の主観的洞察の重要性を認識しつつも,見出し語の採否や注記の付与をはじめとする辞書記述全般において,コーパスを用いた使用実態の分析に基づく客観的指標を導入することを主眼とする.そこで,次章でBCCWJを用いた「古風な語」の使用実態の分析を行う.


\section{「古風な語」のコーパス分析}

\subsection{「古風な語」の使用頻度}

「古風な語」の現代書き言葉における使用実態を把握するために,BCCWJ\footnote{BCCWJを本調査で用いた主な理由は次の2点である.(1)現代日本語書き言葉の均衡がとられたコーパスであり,1億語の規模があるため,現代語を対象とする国語辞典の記述でおさえるべき用例,用法を調べることができる.(2)著作権処理済であるため,国語辞典の例文作成時の参考にできる.} の使用頻度を調査した.全文検索システム『ひまわり』(http://www2.ninjal.ac.jp/lrc/)を用いて,「BCCWJ 領域内公開データ2009」の「図書館サブコーパス」\footnote{1986 年から2005 年までの20 年間に発行された書籍のうち,東京都内の 13 自治体以上の公共図書館で共通に所蔵されていた書籍が母集団とされ,そこから抽出したサンプルから成るサブコーパスである.}(約3,000万語)における調査対象語の使用頻度を調査した (柏野,奥村2010,2011).3.2節では,「古語的」15語,「古風」147語を調査対象語としたが,ここでは,別語や別の意味用法の用例と紛れ,該当用例の判別が困難であった次の語は,検索の対象外とした.

{\setlength{\leftskip}{1zw}\setlength{\parindent}{0pt}
●「古語的」15語より対象外とした語

こうべ【首・頭】,つつみ【堤】,ぶにん【補任】

●「古風」147語より対象外とした語(23語)

いっそ,う,うつ【打つ】,くにびと【国人】,さと【里】,じきげ【直下】,しも【下】,じゃ(ぢや),しょせい【書生】,ぜんぶ【全部】,そち【其方】,それ【其(れ)】,たいじん【大人】,たいぜい【大勢】,つ【唾】,つかさ【司・官】,であう【出合う・出会う】,とうじ【当時】,とも,の,むやく【無益】,やうち【家内】,やくだい【薬代】
\par}

つまり,使用頻度調査の対象語は,「古語的」12語と,「古風」124語である\footnote{念のため第 7 版の未収録を理由に調査対象外とした4語についても使用頻度を調査したところ,「かえり【回り】」(回数・度数を表す,古風な助数詞.回かい.たび.)のみ,「蕎麦の三かえり」(藤村和夫,1930年代生まれ,『蕎麦屋のしきたり』日本放送出版協会,2001年)という用例があった.ほかは,使用頻度0であった.}.

これらのBCCWJにおける使用頻度を求めた結果を付録の表5,表6に示した.「使用有」の欄に「—」のある語は,上記で非調査対象とした語である.「○」のある語が使用頻度の得られたものであり,その数は「使用頻度」の欄に示した.使用頻度が得られたのは,付録の表5,表6の最終行に示した通り,「古語的」12語のうちでは7語,「古風」124語のうちでは76語であった.約3,000万語という規模のコーパスで「古風な語」を検索した場合,頻度が100を超えた語は6語であった.また,頻度が0だった語は53語あった.


\subsection{「古風な語」の用法の分類}

「古風な語」の用法を分析するためには,コーパスから得られる用例がどういった文脈の用例であるのかを区別する必要がある.BCCWJ「図書館サブコーパス」より得られる用例は,その文脈により,執筆及び,記述対象の年代に着目して,大きく次の4つに分類することができる.

\begin{enumerate}
\item 
古典(江戸時代以前の文章)の引用での使用

\item 
明治期から戦前までの使用

\item 
時代・歴史小説での使用

\item 
現代文脈での使用
\end{enumerate}

以下,4分類の詳細を述べる.


\subsubsection{古典(江戸時代以前の文章)の引用での使用}

BCCWJは現代語コーパスであるが,収録テキストに「非現代語」(BCCWJでは,明治元年より前に書かれた日本語と定義)が若干混在している\footnote{BCCWJ に収録するテキストの抽出基準についての詳細は,柏野・稲益・田中・秋元(2009)を参照.}.まとまった「非現代語」は収録テキスト対象外要素として収録しないのだが,一文単位でのテキストの完全収録を保証するために,インライン中に引用されているような「非現代語」は排除せず,そのまま収録している.本論文ではこれを「古典の引用」と呼ぶ.そのような引用中に出現する「古風な語」は「古語」としての使用例である.例えば,次に古典の引用中に現われる「あんずるに【案ずるに・按ずるに】」の例を示す(用例中に用いる,太字,下線及び,括弧内注記は本論文著者による).

\begin{itemize}
\item
仏御前は「つくづく物を\textbf{案ずるに}、娑婆の栄華は夢のうちの夢、たのしみさかえてもなにかせん」(「百二十句本」巻一〈義王出家〉)と言い、つづいて「一旦のたのしみにほこりて、後生を知らざらんことのかなしさに、今朝まぎれ出でて、かくなりてこそ参りたれ」(同前)と言って、かぶっていた衣をのけると、仏御前は、すでに尼姿になっていたのである。(中石孝,1920年代生まれ,『平家れくいえむ紀行』新潮社,1999年)
\end{itemize}

\subsubsection{明治期から戦前までの使用}

BCCWJでは,明治期以降に執筆されたテキストは現代語のテキストであるとされ,明治期以降のテキストが収録されている.しかしながら,明治期から戦前までに執筆されたものは,例えば,旧仮名遣いを用いるなど,現代とは異なる印象を受ける文体のものが多い.また,「古風な語」が「古風」という意識なしで用いられている可能性があると考える.よって,この時期に執筆されているテキストはBCCWJにそう多くは収録されていないが,現代文脈とは区別することとする.例えば次のようなテキストである.前者からは「いずれ【何れ】」の,後者からは「はたまた【将又】」の用例が得られる.

\begin{itemize}
\item
遠野物語の中にも書いてある話は、同郡松崎村の寒戸といふ処の民家で、若い娘が梨の樹の下に草履を脱いで置いたまゝ、行方知れずになつたことがあつた。三十何年を過ぎて或時親類知音の者が其家に集まつて居るところへ、極めて老いさらぼうて其女が戻つて来た。どうして帰つて来たのかと尋ねると、あまりみんなに逢ひたかつたから一寸来た。それでは又行くと言つて、忽ち\textbf{何れ}へか走り去つてしまつた。(柳田國男,1870年代生まれ,『柳田國男全集 
第3巻』筑摩書房,1997年)

\item
然らば私の希ふ真の自由解放とは何だらう。云ふ迄もなく潜める天才を、偉大なる潜在能力を十二分に発揮させることに外ならぬ。それには発展の妨害となるものゝ総てをまず取除かねばならぬ。それは外的の圧迫だらうか、\textbf{はたまた}智識の不足だらうか、否、それらも全くなくはあるまい、併し其主たるものは矢張り我そのもの、天才の所有者、天才の宿れる宮なる我そのものである。(平塚雷鳥,1880年代生まれ,『元始、女性は太陽であった 
平塚らいてう自伝1』大月書店,1992年)
\end{itemize}

(※いずれも明治期の文章.)


\subsubsection{時代・歴史小説での使用}

いわゆる「時代小説」「歴史小説」などと呼ばれる,江戸時代以前を舞台とする文芸作品(国内,国外を問わず)のテキストに「古風な語」が多く現れる.「時代小説」とは「古い時代の事件や人物に題材をとった通俗小説。」であり,「歴史小説」とは「過去の時代を舞台にとり、その時代の様相と人間とを描こうとする小説。(中略)単に過去の時代を背景にする時代小説とは異なる。」(以上,『広辞苑第』第 6 版,岩波書店)と,両者には異なる定義がされているが,本論文では特に両者を区別することはせず,まとめて,「時代・歴史小説」とひとくくりで扱うこととする.

石井(1986)は,例えば「おぬし,…でござるか」などは,「歴史小説なり時代小説なりに現はれるからと言つて,その小説の扱ふ時代の古代語と考へるのは,早計である.非現代語すなはち古代的言語を用ゐた作品においては,作家が古代的言語を創造し,読者がそれを享受する,といふ図式が想定できる.」と述べ,そういった享受と創造による「非現代語」が『源氏物語若紫』現代語訳や,日本文芸家協会『歴史ロマン傑作選』の会話文に多く現れることを調査分析し,報告している.本論文では,まさにこれも「古風な語」であると捉える.BCCWJには「時代小説」「歴史小説」などのテキストが多数収録されており,そういった文脈で用いられる「古風な語」の用例が多く得られる.次に,「あんずるに【案ずるに・按ずるに】」と,「にょにん【女人】」の例を示す.

\begin{itemize}
\item
成之の天才的な兵站事務の噂をききつけてのことであった。\textbf{按ずるに}、成之常に加賀藩の事務に従ひしも、其理財に老けたるの名、夙に朝廷に聞へしを以て、終に此事ありし也。(磯田道史,1970年代生まれ,『武士の家計簿 「加賀藩御算用者」の幕末維新』新潮社,2003年)\\
(※江戸時代を舞台にした時代・歴史小説.)

\item
廐戸には見せなかったが、河勝はいずれ山背に戻る運命にあることを覚悟していたようだった。\\
「そうか、河勝も里心がついたか、それもそうだ、河勝は山背に戻れば葛野の王者だ、屋形も大きく、そちにかしずく\textbf{女人}も多い、いつまでも吾に仕えてくれる、と思っていた吾が甘かった」(黒岩重吾,1920年代生まれ,『聖徳太子 日と影の王子1』文芸春秋,1990年)\\
(※飛鳥時代を舞台にした時代・歴史小説.)
\end{itemize}


\subsubsection{現代文脈での使用}

執筆時期が戦後であり,上記(3)に該当しない現代語の文章(口語体)を,本論文では「現代文脈」と呼んでいる.一部の見出し語に関しては,その現代文脈に出現する「古風な語」の用例が数多く得られている.例えば次のようなものである.

\begin{itemize}
\item
ここで泣いては\textbf{いかん}、と咽喉の塊を懸命にのみ下しながら、(宮尾登美子,1920年代生まれ,『朱夏』新潮社,1998年)

\item
不審な気持ちが消え\textbf{失せ}て、とにかく言葉が交わしたかった。(浅倉卓弥,1960年代生まれ,『雪の夜話』中央公論新社,2005年)

\item
コミック誌から飛び出してきたような\textbf{いでたち}の男だ。(佐々木譲,1950年代生まれ,『新宿のありふれた夜』角川書店,1997年)

\item
セブン‐イレブンの店にほしいものがなければ、\textbf{いと}も単に、買いにこなくなります。(鈴木敏文\textbar 
述;緒方知行\textbar 編,『商売の創造』講談社,2003年)

\item
恋なのか、忠義なのか、\textbf{はたまた}親子の恩愛なのか。(古井戸秀夫,1950年代生まれ,『歌舞伎』新潮社,1992年)
\end{itemize}

例えば,「ものども【者共】」には,分類(1)(3)(4)に該当する次のような用例がある.

{\setlength{\leftskip}{1zw}\setlength{\parindent}{0pt}
\textbf{・(1)古典の引用:}壇の浦の合戦では、源平両軍は三十余町をへだててあい対し、いよいよ戦闘開始ということになったが、早い潮に流された平家の舟を梶原景時の舟が熊手でひっかけて、敵の首を数多く取る功名第一の働きから始まった。そして両軍あわせて鬨をつくり、それが静まると、平知度が大音声に、\ul{「天竺震旦にも、日本わが朝にも、雙びなき名将勇士といへども、運命尽きぬれば力及ばず、されども名こそ惜しけれ、東国の{\bfseries 者ども}に弱気見すな、何時の為にか命をば惜しむべき、軍ようせよ、{\bfseries 者ども}、只これのみぞ思ふ事よ」}と全軍に宣した(同上・巻十一)。(阿部猛,1920年代生まれ,『鎌倉武士の世界』東京堂出版,1994年)

(※下線が古典の引用部分.上記「同上」とは『平家物語』.)

\textbf{・(3)時代・歴史小説での使用: 
}久幸ははじめて深い笑みを見せ、「この臆病者が必ずお守り申す。ご安心あれ」といった。そして、「\textbf{者ども}続け!」と大声をあげると、私兵五百人を率いて、まっしぐらに大内勢に向かって行った。(童門冬二,1920年代生まれ,『小説毛利元就』PHP研究所,2002年)

(※時代小説.この例は室町時代から戦国時代の時代設定.)

\textbf{・(4)現代文脈での使用: 
}「なるほど、これは無用のことを申しました\textellipsis 
さあ、\textbf{ものども}、引っ立てい!」ドラホマノフの命令を受けて、兵士たちがユリアスとパッシェンダールに縄をかける。さすがに、この人数差では、パッシェンダールといえども抵抗のすべがなかった。(赤城毅,1960年代生まれ,『滅びの星の皇子』中央公論新社,2001年)
\par}

つまり,「ものども【者共】」は,BCCWJにおいて,古語としての用例が現われる語であり,時代小説や歴史小説では,その時代設定にあう語として使われる語であり,かつ,現代文脈においても使われる語であることがわかる.「古風な語」としてひとくくりする中には,現代語書き言葉においてこのように幅広い用法で出現する語のあることが,本調査により確認できた.


\subsection{「古風な語」の用法の分類結果}

BCCWJが現代語書き言葉を収集したコーパスであるため,古典の引用の用例数はそもそも少ない.また,先に述べたとおり,明治期から戦前までに執筆されたテキストも少ないため,その用例数も少ない.一方,時代小説や歴史小説のテキストはBCCWJに多く収録されていることから,それらの用例数は多い.また,現代文脈から得られる用例も一定数以上得られている.

\begin{table}[b]
\caption{「古風」使用頻度上位9語の用例分類結果}
\input{02table04.txt}
\end{table}

使用頻度が50以上あった,上位9語(「古語的」1語(いたく),「古風」8語(いかん,ほか),付録の表5,表6を参照)を取り上げ,それらの用例を分類した結果を表4に示す.上から5語は,時代・歴史小説の用例の割合が多い順,次の3語は現代文脈の用例の割合が多い順である.そして,最後の1語が両者の用例の割合が拮抗していた語である.割合の高いところに色をつけて示している.

このように,BCCWJにおいて高頻度である「古風な語」は,時代・歴史小説の用例の割合が多いものと,現代文脈の用例の割合が多いものとがあった.同じ「古風な語」とくくるには,用法の傾向は大きく異なっている.つまり,時代・歴史小説の使用が多い,現代文脈の使用が多い,といったその語が使用されている文脈の特徴を辞書に明記し,その用例を具体的に記述することが,「古風な語」とひとくくりに説明するよりも,それぞれの語の文体的特徴を説明することができるようになるということである.


\subsection{辞典間の記載に差異のある語の分析}

3.3節では,5種の辞典間に,見出し語として採用されている数や,古さについてのラベルや注記が付与されている数が異なる語があることを述べた.表2の「まがまがしい【禍々・枉々・曲々・凶々しい】」と,表3の「よしなに【良しなに】」の2語は,見出し語として採用しているのは5辞書であるが,注記は2辞書ずつという語であった.また,表2の「やわか」は,見出しとして採用し,注記を付与しているのが2辞書のみ,という語であった.

付録の表5,表6に示す通り,BCCWJには,「まがまがしい」に33例,「よしなに」に7例の使用がある.そのうち,2例ずつ引用する.

\begin{itemize}
\item
それは途方もなく暗鬱な感じの建物で、\textbf{まがまがしい}気配にみち、見物するのは愉快な体験ではなかつた。(丸谷才一,1920年代生まれ,『日本の名随筆』作品社,1988年)

\item
「お気をおつけ下さい」 ヒロはいう。「今夜は、\textbf{まがまがしい}気配が満ちております。その心を感じるのです。ザック様のことですから何がおころうときっと大丈夫でしょうが\textellipsis 
心配でなりません」(眉村卓,1930年代生まれ,『迷宮物語』角川書店,1986年)

\item
「よう冷えるなあ。こらたまらんワ。泰平堂、まあ\textbf{よしなに}調べてくれ」 
声をかけて信濃は去っていった。(阿部牧郎,1930年代生まれ,『出合茶屋』講談社,2003年)

\item
「お世話をかけます、なにとぞ\textbf{よしなに}」 懐しさを顔一ぱいにみせてお茂の方は、頭を下げた。(竹内勇太郎,1920年代生まれ,『甲府勤番帖』光風社出版,1992年)
\end{itemize}

「まがまがしい」は33例中,5例が時代・歴史小説の用例であったが,残り28例は上記2例のような現代文脈の使用であった.「よしなに」は,7例中のすべての例が,上記2例のような時代・歴史小説の用例であった.どちらにも時代・歴史小説の用例のあることから,「古風な語」という注記を付与すべき語であることがまずはわかる.それに加え,「まがまがしい」は現代文脈での使用が多く,「よしなに」は時代・歴史小説での使用が多い語であるということもわかる.

続けて,2辞書のみ見出し語に採用していた「やわか」の使用例をみてみる.BCCWJから得られる用例3例中より,2例を示す.

\begin{itemize}
\item
「この胴も手足も、蓬山から出た鉄を百年も磨き抜いてこしらえたものよ。項羽の豪刀をもってしても傷ひとつつかなんだ。\textbf{やわか}、おまえの妖糸ごときに引けは取らぬぞ」 
「そりゃ、どうも」(菊地秀行,1940年代生まれ,『夜叉姫伝』祥伝社,1991年)

\item
「いかなる御用とて、われらにおきかせあれ! 
拙者、\textbf{やわか}島田虎之助の働きに劣りましょうや」と、斎藤弥九郎が、いかにも体調の悪いらしい島田虎之助をあごでさして、そのあごをお耀の方にぐいとつき出す。(山田風太郎,1920年代生まれ,『武蔵野水滸伝』富士見書房,1993年)
\end{itemize}

こういった用例に触れて国語辞典をひく読者がいることを想定すると,「やわか」は「古風な語」として現代語辞書に取り上げるとよい語であると言える.

このようにコーパスで使用例を確認することは,辞書によって扱いに差異のある語の辞書記述の判断の参考になることを示した.


\section{コーパス分析を活かした辞書記述}

前章で,「古風な語」の使用実態についてコーパスの分析結果を示した.本章にて,そのコーパス分析に基づく辞書記述を提案する.


\subsection{コーパス分析に基づく辞書記述の利点}

従来,辞書記述はもっぱら編集者の知識や内省によって行われていたが,これを補うものとして徐々にコーパスの利用が試みられるようになった.1964年にアメリカで 
Brown Corpus が公開されて以降,各種大規模コーパスの構築,公開に伴い,欧米における辞書記述のコーパス活用は一気に加速していく (Sinclair 1991,石川2004,井上2005).例えば,英国のCollins,Longman,Oxford,Cambridge各社の辞書である.それらの影響を受け,日本の英和辞書にもコーパスの活用は広がっている.コーパスベースの辞書記述を取り入れた英和・和英辞書には,『ウィズダム英和辞典』(三省堂,2002年),『ウィズダム和英辞典』(三省堂,2007年)や『ユース・プログレッシブ英和辞典』(小学館,2004年)がある.

コーパスベースを謳う国語辞典や日本語の辞書は今現在ないが,コーパス分析を日本語の辞書記述に活かそうとする議論は徐々に活発になってきている (加藤1998,後藤2001,田野村2009,荻野2010,石川2011,カルヴェッティ2011など).

辞書記述がコーパスベースになる利点は,以下の3点である (柏野2011).

\begin{itemize}
\item[\textbf{(1)}]
\textbf{見出し語の選定や,語義の選定・配列の客観性}:見出し語単独の頻度の情報が得られる.辞書に記載すべき見出し語の選定や,語義の選定・配列を,客観的に行える.

\item[\textbf{(2)}]
\textbf{用例記述の網羅性}:見出し語とほかの語との組み合わせの頻度の情報が得られる.見出し語とその前後の連なり,見出し語と共起する語(コロケーションと呼ばれる),見出し語の現れる構文などの特徴的なパターンを発見しやすい.用例を網羅的に辞書に記述できる.

\item[\textbf{(3)}]
\textbf{見出し語の使用域記述の具体性}:資料の幅が広がり,見出し語の多様性を捉えることができる.ジャンルあるいはレジスタなどと呼ばれる,言葉の使用域が異なる場合,それを具体的に辞書記述に反映できる(例えば,話し言葉的か,書き言葉的か,フォーマルか,インフォーマルか).
\end{itemize}


\subsection{コーパス分析を「古風な語」の辞書記述に活かす方法}

5.1節で述べたコーパスベースによる辞書記述に期待できる3点に照らし合わせながら,コーパス分析を「古風な語」の辞書記述に活かす方法を検討する.


\subsubsection{見出し語の選定や,語義の選定・配列の客観性}

各種国語辞典やコーパスを利用して得た「古風な語」の頻度情報をコーパスで得る.頻度情報は,「コーパスベース国語辞典」の見出し語選定の参考になる.

4.1節,および,付録の表5,表6にて,『岩国』から抽出した「古風な語」の使用頻度を示した.頻度の高い語はどの辞典でも見出し語として採用されていることが確認できる.が,頻度が低い語の場合,例えば,「あんずるに【案ずるに・按ずるに】」「て」「しんずる【進ずる】」「のう」の4語は,見出し語として採用していない辞典のあることがわかる.現行辞典の採用状況から見出し語選定をしようとする場合には,これらが採否のボーダー上になってくるだろう.しかし,筆者らは,BCCWJで低頻度でも用例の得られる語は現代語として目にする機会のある語であると考え,これら4語いずれについても,見出し語として採用すべき語と考える.

一方,今回の調査で頻度0であった語を,ただちに見出し語として採録しないと結論づけるのは難しい.現代語の辞典で取り扱う必要性の低い「古語」であるのか,あるいは,たまたま例がとれなかった「現代語」であるのかがわからないからである.BCCWJにおいて頻度0であることの意味は,今後,調査範囲とするコーパスを増やしたさらなる検討が必要であるだろう.

次に,既存の国語辞典等によらず,コーパスから自動的に「古風な語」の抽出が可能であるかを考える.BCCWJ「図書館サブコーパス」の形態素解析結果より,「活用型:文語」となっている語の抽出を行った.BCCWJは,短単位(意味を持つ最小の単位を結合させる,または結合させないことによって認定)と長単位(文節を規定に基づいて分割する,または分割しないことによって認定)とで解析されている (小椋,小磯,冨士池,宮内,小西,原2011,小椋,冨士池2011).短単位の抽出結果では,その使用頻度の上位は「べし:13,889,なり:5,421,たり:3,357,ごとし:2,048,り:1,888」であった.高頻度の助詞・助動詞の用例が多数得られることがまずは確認できた.また,「有り:618,然り:525,来たる:284,恐る:262」など,文語の活用形で用いられる語の抽出ができることも確認できた.しかしながら,長単位の抽出結果とあわせてみても,文語形を含む見出し語そのものの形での抽出は自動では簡単ではない.例えば,「止む無し」は一つの長単位として抽出できているが,そういう語は少ない.多くは,「なきにしもあらず」が「なき+に+しも+あら+ず」となっているように,長単位も短単位同様に複数に切れている.また,「さもありなん」のように誤解析されている場合もある(「なん」が「名詞【何】」と誤解析.正しくは文語助動詞「な+む」).いずれの場合も,自動抽出のためには形態素解析の辞書整備が必要であるが,そのためには,結局,既存の国語辞典等にある見出し形をあらかじめ先に参照しなければならない,ということになる.

語義の選定・配列についてのコーパス情報の利用は,5.3節で具体例をもって示す.


\subsubsection{用例記述の網羅性}

BCCWJにおいて一定数以上の用例が得られれば,ある程度の網羅的な記述は可能である.ただし,「古風な語」を扱う場合は,BCCWJよりも前の時代のコーパス分析も欠かせないだろう.現時点では,少し前の時代のコーパスとして,著作権の消滅した作品が中心に集められ,明治〜昭和初期の小説が多く収録されている『青空文庫』\footnote{http://www.aozora.gr.jp/} や,明治後期〜大正期の総合雑誌『太陽』から5年分が抽出されている『太陽コーパス』\footnote{http://www.ninjal.ac.jp/corpus\textunderscore center/cmj/taiyou/}(国立国語研究所)などが利用可能である.


\subsubsection{見出し語の使用域記述の具体性}

「古風な語」の用法は,(1)古典の引用での使用,(2)明治期から戦前までの使用,(3)時代・歴史小説での使用,(4)現代文脈での使用,と分類でき,かつ,その使用傾向が語の用法把握につながることを,4.2節と4.3節とで示した.その際に,特に,分類(3)と(4)に相当する用法の具体的記述が重要であることを述べた.(3)の,主に時代小説や歴史小説での使用は,現代においてそれら小説を読む際の理解に欠かせないものである.現代語ではないと排除することなく,辞書にとりあげ,詳細に記述すべきものであろう.さらに,(4)の,現代文脈での使用は,一部の「古風な語」における顕著な特徴である.よって,「古風な語」の現代文脈における使用例がある場合は,それを具体的に辞書に記述すべきである.

これらの考察をふまえ,次節では具体例を用いて,コーパス分析に基づく「古風な語」の辞書記述案を示す.


\subsection{コーパス分析に基づく「古風な語」の辞書記述}

本論文では,次のとおり,コーパス分析に基づく「古風な語」の「コーパスベース国語辞典」記述方法を提案する.

\vspace{1\Cvs}
\vbox{
{\hrule height 0.25pt depth 0pt width 420pt}
\hbox to420pt{{\vrule width 0.25pt}\hfill\begin{minipage}{410pt}
\kern0.5\Cvs
\textbf{「古風な語」の「コーパスベース国語辞典」記述方法}
\begin{itemize}
\item[1.]現行の国語辞典類や,BCCWJ,『青空文庫』,『太陽』等のコーパスを利用し,次の条件を満たす,「古風な語」を選定する.
\begin{itemize}
\item[(a)]「時代・歴史小説」を含めて現代で使用が見られる.
\item[(b)]明治期以前,あるいは,戦前までの使用が見られる.
\end{itemize}
\item[2.]BCCWJ,『青空文庫』,『太陽』等のコーパスから,「古風な語」の使用頻度,用例を得る.
\end{itemize}
\end{minipage}\hfill{\vrule width 0.25pt}}
}

\clearpage
\hbox to420pt{{\vrule width 0.25pt}\hfill\vbox spread0.5\Cvs{\noindent\begin{minipage}{410pt}
\begin{itemize}
\item[3.]多義語の場合,意味分析を行い,意味別の使用頻度を得る.

\item[4.]得られた用例を,(1)古典の引用での使用,(2)明治期から戦前までの使用,(3)時代・歴史小説での使用,(4)現代文脈での使用,に分類し,各頻度を得る.

\item[5.]分類別の使用頻度を参考に,中心的となる語義から順に配列する.

\item[6.]用例は,用例の(1)〜(4)の分類傾向や,具体的な使用域がわかるよう明記する.

\item[7.]そのほか,表記情報や,使用者の性別・年代など,コーパスから抽出できた情報を明記する.
\end{itemize}
\end{minipage}}\hfill{\vrule width 0.25pt}}
{\hrule height 0.25pt depth 0pt width 420pt}
\vspace{1\Cvs}

「そなた【其方】」を例に,現行の国語辞典の記述(『岩国』)に対する,上記,「コーパスベース国語辞典」記述方法に即した記述案を次に示す.「そなた」は,上記,(a),(b)の条件を満たす.

\textbf{例:そなた【其方】}

\textbf{[『岩国』(第 7 版)]}

\MaruOne そちらのほう。そちら側の所。

\MaruTwo 目下の相手を指す語。おまえ。なんじ。古風な言い方。

\textbf{[コーパス分析]}

\begin{itemize}
\item
「そなた【其方】」はBCCWJで434の頻度があり,「古風な語」の中では高頻度である語である.\\
→\textbf{見出し語の選定:}十分な頻度が得られるため見出し語として採録すべき語と判断.

\item
『岩国』\MaruOne の該当用例はBCCWJでは頻度0.『青空文庫』には3例見つかる.BCCWJで得られる434の用例は,すべて『岩国』\MaruTwo の用法の該当例である.\\
→\textbf{語義の選定・配列:}\MaruOne と\MaruTwo を語義と認定.ただし,\MaruOne の用法はBCCWJで0であり,現代語では,\MaruTwo の用法がより中心的な語義と認定できるため,語義の配列順を入れ替える.

\item
『岩国』\MaruTwo の用例分類は4.3節の表4の通り,ほとんどが「(3)時代・歴史小説での使用」である.国内の時代小説のほか,時代設定の古い翻訳小説の用例もある.現代文脈での使用は少ないが,ある.また,BCCWJで得られた用例はすべて発話部分における使用.\\
→\textbf{用例記述・見出し語の使用域:}(3)時代・歴史小説での使用例を1番に.続けて,翻訳小説,現代文脈の使用例を記述.「発話で多用」を明記.発話については話者間の関係を明記。

\item
漢字表記「其方」の使用例は見つけられない.検索される「其方」は「そちら」「そのほう」の漢字表記の使用例と思われる.\\
→表記情報を注記.
\end{itemize}

\textbf{[「コーパスベース国語辞典」記述案]}

\begin{itemize}
\item[\rlap{\MaruOne}]
多くは目下の相手をさす語。おまえ。なんじ。時代・歴史小説的。同等の相手をさすこともある。

\begin{itemize}
\setlength{\fboxsep}{1pt}
\item[\textbullet]
時代・歴史小説等の発話で多用。\\
「\colorbox[gray]{.9}{そなた}、勘六どのを見舞って来やれ。さっきの飴の甘味はきつすぎます。よけい食べさせてはなりませぬ。急いでゆきゃれ」(山岡荘八『徳川家康』)\\
※後に家康を産む於大の方(目上)から,召使いの百合(目下)への発話。\\
「ほう、\colorbox[gray]{.9}{そなた}もさようなことを考えておったのか?」豪族は大きく頷いた。(山田智彦『木曽義仲』)\\
※信濃国の各地から集まってきた豪族たち同士(同等)の対話。

\item[\textbullet]
時代設定の古い小説(特に翻訳小説)における発話で使用。\\
「\colorbox[gray]{.9}{そなた}は誰なのか、教えてくれ」。(井村君江『アーサー王物語』)\\
※アーサー王(目上)から,騎士であるトリストラム卿(目下)への発話。\\
「ねえ乳母や、\colorbox[gray]{.9}{そなた}の申し条、もっともなことです。ただ怒りのあまり分別を失っていたのです」と言いました。(池田修『アラビアン・ナイト』)\\
※王女である姫(目上)から,老女の乳母(目下)への発話。

\item[\textbullet]
現代文脈での使用は多くはないが,威厳や威圧があるよう造形された人物(多くは目上)からの発話を表すものとして使用。\\
「厚志よ」\\
舞が呼びかけると、厚志はビクリと身を震わせた。\\
「え、なに?」\\
「何を驚く? \colorbox[gray]{.9}{そなた}の名を呼んだだけだぞ」\\
(榊涼介『ガンパレード・マーチ5121小隊九州撤退戦』)\\
※小隊司令官である舞(目上)から,その下にいる厚志(目下)への発話。
\end{itemize}

\item[\rlap{\MaruTwo}]そちらのほう。そちら側の所。▽明治期頃まで。

\begin{itemize}
\setlength{\fboxsep}{1pt}
\item[\textbullet]
使用はまれ。\\
\ruby{五月雨}{さみだれ}に四尺伸びたる\ruby{女竹}{めだけ}の、\ruby{手水鉢}{ちょうずばち}の上に\ruby{蔽}{おお}い重なりて、余れる一二本は高く軒に\ruby{逼}{せま}れば、風誘うたびに戸袋をすって\ruby{椽}{えん}の上にもはらはらと所\ruby{択}{えら}ばず緑りを\ruby{滴}{したた}らす。「あすこに画がある」と葉巻の煙をぷっと\colorbox[gray]{.9}{そなた}へ吹きやる。(夏目漱石『一夜』)\\
\ruby{恋}{こひ}の\ruby{淵}{ふち}・峯の薬師・百済の\ruby{千塚}{ちづか}など、通ひなれては、\colorbox[gray]{.9}{そなた}へ足むくるもうとましきに、折しも秋なかば、汗にじむまで晴れわたりたる日を、たゞ一人、小さき麦稈帽子うち傾けて、家を出でつ。(折口信夫『筬の音』)
\end{itemize}
※漢字表記「其方」の使用例は見つけにくい.該当表記例があっても,別見出し「そちら」「そのほう」との区別がつけがたい。
\end{itemize}

以上の記述案により,「そなた【其方】」の用法は,時代・歴史小説の発話文での使用が中心的な語であることが明確になる.多くの用例を示したことにより,現代の文章生成時の語選択においては,時代がかったセリフとなる効果のある語であることに留意すべき語であることがわかる.さらに,漢字表記の【其方】はほとんど使用例がなく,かつ,「そちら」「そのほう」とまぎれる可能性があるので,ひらがな表記を選択すべきこともわかる.このように,「コーパスベース国語辞典」に文体的特徴や用例を記述することは,語の理解はもとより,語の選択時の情報量を増やし,その利用価値を高めると期待される.

なお,中世・近世のコーパスがない状況においては,国語史研究の知見を参照することが有効である.「そなた」は中世より,主に上位の話し手が,下位の話し相手に対して使用する例の多い人称の一つと位置づけられている.中世から近世にかけての使用実態については,山崎 (2004),小島 (1998) に詳細な分析があり,上位から下位に用いられ,時には対等の間柄でも用いられるものであると報告されている.「上位から下位」の例,「対等」の例がともにコーパスから得られたため,辞書記述案ではそれら先行研究の知見を活かし,「主に上位から下位へ」「時には対等」で用いられることがわかるよう注記することも有用である.

国立国語研究所では,将来的に上代から近代の作品をカバーする「日本語歴史コーパス」の構築が進められている (http://www.ninjal.ac.jp/corpus\textunderscore 
center/chj/).それらコーパスの整備に伴い,コーパス分析の可能性が広がることが期待される.


\section{おわりに}

本論文では,「古風な語」に着目し,『岩波国語辞典』に「古語的」と注記がついている15語と,「古風」と注記がついている145語を調査対象語に選定し,現行の国語辞典5種の記述調査及び,現代語コーパスの使用頻度調査と用例分析とを行った.そして,コーパス分析に基づく「コーパスベース国語辞典」における記述方法を提案した.(1)古典の引用での使用,(2)明治期から戦前までの使用,(3)時代・歴史小説での使用,(4)現代文脈での使用,という4分類に基づく辞書記述方法が,従来の「古風な語」をひとくくりにする辞書記述よりも,語の文体的特徴をより豊富に記述できることを示した.その一例として「そなた」の記述例を示した.

なお,「コーパスベース国語辞典」記述方法(5.3節)の3に挙げた意味判別に関しては,例えば,Pulkit・白井(2012)など,すでに自動化の研究が進んでいる.筆者らは,4に挙げた,当該語の用いられる文脈が,分類(1)〜(4)のいずれであるかの判別についても,自動化が可能であると考えている.コーパスから自動的に抽出できる辞書情報を活用していくことにより,従来の主に人手による国語辞典の編集とは異なる,一貫性のある「コーパスベース国語辞典」の構築が可能になると考える.


\acknowledgment

本研究は,文部科学省科学研究費補助金基盤研究(C)「辞書用例の記述仕様標準化のための実証研究」(課題番号:20520428),並びに,文部科学省科学研究費補助金基盤研究(C)「コーパス分析に基づく辞書の位相情報の精緻化」(課題番号:23520572)の助成を受けたものです。


\bibliographystyle{jnlpbbl_1.5}

\begin{thebibliography}{}
\item
カウイー, A. P. (赤須薫, 浦田和幸訳)(2003).学習英英辞書の歴史:パーマー,ホーンビーからコーパスの時代まで.研究社.(Cowie, A. P. (2002) \textit{English Dictionaries for Foreign Learners. A History.} Oxford: Oxford University Press).

\item
カルヴェッティ・パオロ(2011). 
イタリア人向けの和伊辞典編纂におけるBCCWJの貢献.「現代日本語書き言葉均衡コーパス」完成記念講演会予稿集,pp. 217--226.

\item
遠藤織枝(1988). 
話しことばと書きことば—その使い分けの基準を考える—.日本語学, \textbf{7}(3), pp. 27--42.

\item
Fedorova, I. V. (2004). ``Style and Usage Labels in Learners' Dictionaries: 
Ways of Optimization.'' In \textit{Proceedings of the 11th EURALEX International Congress, July 6--10.} pp. 265--272.

\item
後藤斉(2001). 
日本語コーパス言語学と語の文体レベルに関する予備的考察.東北大学文学研究科研究年報, 
\textbf{50}, pp. 201--214.

\item
ハートマン, R. R. K. (編)(1984). 辞書学:その原理と実際.三省堂.木原研三, 
加藤知己(翻訳監修),Hartmann, R. R. K. (ed.) (1983). \textit{Lexicography: Principles and Practice.} London: Academic 
Press.

\item
H\"{u}nig, W. H. (2003). ``Style labels in monolingual English learners' 
dictionaries.'' In H. Cuyckens et al. (eds), \textit{Motivation in Language: Studies in Honor of G\"{u}nter Radden.} Amsterdam: John Benjamins, pp. 
367--389.

\item
井上永幸(2005). 
コーパスに基づく辞書編集.齊藤俊雄,中村順作,赤野一郎(編).英語コーパス言語学—基礎と実践—改定新版.研究社,pp. 207--228.

\item
石井久雄(1986). 
古代的言語の享受と創造.文部省昭和60年度科学研究費補助金による一般研究(C)研究報告書.

\item
石川慎一郎(2004). Corpus, Dictionary, and Education:近刊EFL辞書に見る辞書編集の潮流.Kobe English Language Teaching, \textbf{19}, pp. 61--79.

\item
石川慎一郎(2011). 
現代日本語書き言葉均衡コーパス (BCCWJ) における複合動詞〜出すの量的分析.統計数理研究所研究レポート, 
\textbf{238}, pp. 15--34.

\item
柏野和佳子(2009). 1.3 辞書. 言語処理学会(編).言語処理学事典,pp. 80--85. 共立出版.

\item
柏野和佳子・稲益佐知子・田中弥生・秋元祐哉(2009). 第4章 対象外要素の排除指定. 
特定領域研究「日本語コーパス」平成20年度研究成果報告書 「現代日本語書き言葉均衡コーパス」における収録テキストの抽出手順と事例,pp. 66--88.

\item
柏野和佳子・奥村学(2010). 
国語辞典に「古い」と注記される語の現代書き言葉における使用傾向の調査.情報処理学会 
人文科学とコンピュータ研究会報告, \textbf{88},pp. 59--70.

\item
柏野和佳子・奥村学(2011). 
国語辞典に「古風」と注記される語の使用実態調査—「現代日本語書き言葉均衡コーパス」を用いて—.言語処理学会第17回年次大会発表論文集, 
pp. 444--447.

\item
柏野和佳子(2011). 
コーパスに基づく辞書づくり—これからの国語辞典はこう変わる—.日本知能情報ファジイ学会誌, 
\textbf{23}(5), pp. 705--713.

\item
加藤安彦(1998). 辞典とコーパス.日本語学, \textbf{17}(12), pp. 37--44.

\item
小島俊夫(1998). 日本敬語史研究—後期中世以降.笠間書院.

\item
小椋秀樹,小磯花絵,冨士池優美,宮内佐夜香,小西光,原裕(2011). 
「現代日本語書き言葉均衡コーパス」形態論情報規程集 
第4版(上)(下).国立国語研究所内部報告書,LR-CCG-10-05-01, 02.

\item
小椋秀樹,冨士池優美(2011). 第4章 
形態論情報.国立国語研究所「現代日本語書き言葉均衡コーパス」利用の手引 
第1.0版,4-1--4-35.

\item
前坊香菜子(2009). 
語の文体的特徴に関する情報についての一考察—国語辞典と類語辞典の調査から—.一橋日本語教育研究報告, 
\textbf{3}, pp. 50--60.

\item
前川喜久雄(2008). 
KONONOHA「現代日本語書き言葉均衡コーパス」の開発.日本語の研究, \textbf{4} 
(1), pp. 82--95.

\item
前川喜久雄(編)(2013). 講座日本語コーパス1 コーパス入門.朝倉書店.

\item
宮島達夫(1977). 単語の文体的特徴.村松明教授還暦記念国語学と国語史, pp. 871--903. 明治書院.

\item
荻野綱男(編)(2010). 
特定領域研究日本語コーパス平成21年度研究成果報告書,コーパスを利用した国語辞典編集法の研究.

\item
Ptaszynski, M. O. (2010) ``Theoretical Considerations for the Improvement of 
Usage Labelling in Dictionaries: A Combined Formal-Functional 
Approach.''\textit{International Journal of Lexicography, }\textbf{23}(4), pp.411--442.

\item
Pulkit, K., 白井清昭(2012). 
パラレルコーパスから自動獲得した用例に基づく語義曖昧性解消.情報処理学会研究報告,自然言語処理研究会報告, 
\textbf{2012-NL-207}(3),pp. 1--8.

\item
Sakwa, L. N. (2011). ``Problems of Usage Labelling in English 
Lexicography.'' \textit{LEXIKOS}, \textbf{21}(1), pp. 305--315.

\item
Sinclair, J. (1991). \textit{Corpus Concordance Collocation.} Oxford: Oxford University Press.

\item
田野村忠温(2009). 
コーパスを用いた日本語研究の精密化と新しい研究領域・手法の開発.人工知能学会誌,\textbf{24}(5),pp. 647--655.

\item
山崎久之(2004). 増補補訂版 国語待遇表現体系の研究 近世編.武蔵野書院,

\item
山崎誠(2009). 代表性を有する現代日本語書籍コーパスの構築.人工知能学会誌, 
\textbf{24}(5), pp.~623--631.

\item
山崎誠(2011). 
第2章「現代日本語書き言葉均衡コーパス」の設計.国立国語研究所「現代日本語書き言葉均衡コーパス」利用の手引 
第1.0版, 2-1--2-8.

\end{thebibliography}

\appendix

(1) 表5「古語的」15語の調査結果

(2) 表6「古風」147語の調査結果

※表中,「多義」の欄に「*」を付してある語は,「古語的」あるいは「古風」の注記が多義のうちの一つの語義についていた語であることを示す.

※表中,「使用有」の欄に「—」のある語は,使用頻度調査の対象外の語であることを示す.

※表中,使用頻度が50を超えている語には,使用頻度の欄に色をつけて強調している.

\begin{table}[h]
\caption{「古語的」15語の調査結果}
\input{02table05.txt}
\end{table}

\clearpage

\begin{table}[p]
\caption{「古風」147語の調査結果}
\input{02table06-1.txt}
\end{table}

\addtocounter{table}{-1}
\begin{table}[p]
\caption{(続き)}
\input{02table06-2.txt}
\end{table}

\addtocounter{table}{-1}
\begin{table}[p]
\caption{(続き)}
\input{02table06-3.txt}
\end{table}

\addtocounter{table}{-1}
\begin{table}[t]
\caption{(続き)}
\input{02table06-4.txt}
\end{table}

\clearpage

\begin{biography}
\bioauthor{柏野和佳子}{
東京女子大学文理学部日本文学科卒業.現在,大学共同利用機関法人人間文化研究機構国立国語研究所言語資源研究系准教授.辞書,語彙研究に従事.これまで『計算機用日本語基本辞書(IPAL)』(情報処理振興事業協会),『国立国語研究所資料集14 
分類語彙表』(大日本図書),『岩波国語辞典』(岩波書店)の編集作業に携わる.情報処理学会,計量国語学会,日本語学会,人工知能学会各会員.
waka@ninjal.ac.jp
}

\bioauthor{奥村  学}{
1962年生.1984年東京工業大学工学部情報工学科卒業.1989年同大学院博士課程修了.同年,東京工業大学工学部情報工学科助手.1992年北陸先端科学技術大学院大学情報科学研究科助教授,2000年東京工業大学精密工学研究所助教授,2009年同教授,現在に至る.工学博士.自然言語処理,知的情報提示技術,語学学習支援,テキスト評価分析,テキストマイニングに関する研究に従事.情報処理学会,電子情報通信学会,人工知能学会, 
AAAI,ACL, 認知科学会,計量国語学会各会員.
oku@pi.titech.ac.jp, http://oku-gw.pi.titech.ac.jp/\textasciitilde oku/.
}
\end{biography}


\biodate


\end{document}

