
\documentstyle[epsf,jnlpbbl]{jnlp_j_b5}
\setcounter{page}{93}
\setcounter{巻数}{6}
\setcounter{号数}{5}
\setcounter{年}{1999}
\setcounter{月}{7}
\受付{1999}{2}{4}
\採録{1999}{4}{23}

\setcounter{secnumdepth}{2}

\title{放送ニュース文を対象とした効果的類似用例検索法}
\author{田中 英輝\affiref{ATR} \and 熊野 正\affiref{NHK} \and 浦谷 則
好\affiref{NHK} \and 江原 暉将\affiref{NHK}}

\headauthor{田中, 熊野, 浦谷, 江原}
\headtitle{放送ニュースを対象とした効果的類似表現検索法}

\affilabel{ATR}{ATR 音声翻訳通信研究所}
{ATR Interpreting Telecommunications Research Laboratories}
\affilabel{NHK}{NHK 放送技術研究所}
{NHK Science and Technical Research Laboratories}

\jabstract{著者らは用例提示型の日英翻訳支援システムを開発している.こ
の中には利用者が入力する日本語の表現に類似する表現を検索して,検索結果を含ん
だ日本語文とその対訳を表示する機能がある.著者らの日本語データベースの
文は平均長が 88.9 文字と長い.このように長い用例を対象に類似検索を行
う場合,キーワードによる AND 検索は適切ではない.
なぜなら用例が長いため 1 文中に同一キーワードが複数回出現する場合があり,
これが原因で不適切な用例を検索しやすくなるからである.
これに対して著者らは入力キーワードの語順とその出現位置の間隔を考慮した
検索手法を提案する.これによって構文解析を行うことなく構文情報を反映した検索を
行うことができる.本稿では従来の AND 検索と提案手法を使った評定者に
よる主観評価実験を報告する.この中で,提案手法の有効性が統計的に有
意となったことを示す.また,検索時間の増加は 1.3 倍であった.
}

\jkeywords{用例提示型翻訳支援システム, 類似性, 語順, AND 検索}

\etitle{An Efficient Way of Gauging Similarity\\
between Long Japanese News Expressions
}
\eauthor{Hideki Tanaka \affiref{ATR} \and Tadashi Kumano\affiref{NHK}
\and Noriyoshi Uratani\affiref{NHK} \and Terumasa Ehara\affiref{NHK}} 

\eabstract{
We are developing a Japanese--to--English Translation Aid system for
news translators. The system consists of a voluminous bilingual news
database whose sentences are properly aligned across languages
beforehand, and a similar expression search engine. A user can find
past translation examples of input Japanese with the system.
Similar expression search engines like the one in this paper have
usually employed an AND retrieval technique that uses keywords in the
input expression, to measure the similarity between the input and the
target by the number of shared keywords.
In many cases of applying such search engines to our database,
however, a number of spurious search results have been produced as a
consequence: the sentences have been quite long (88.9 Japanese
characters on average) and a single sentence has often contained
identical keywords many times.
In this paper, we propose adding two constraints to the AND retrieval
technique: the order and positions (deviations) of keywords. We
enhance AND retrieval allowing it to be able to reflect some syntactic
similarity by this inexpensive modification.
We will show, through a set of experiments, that the proposed method
significantly improves the level of user satisfaction in search results in
a statistical sense, with only a 1.3--fold increase in the search time.
}

\ekeywords{Example--Based Translation Aid System,
Similarity, Word Order, AND Retrieval}

\begin{document}
\maketitle


\section{はじめに}
\label{sec:hajimeni}
人間の翻訳作業を支援するシステムは,電子単語辞書から機械翻訳システムま
でいろいろ提案されており関連する研究も多い\cite{MT97}.
著者らはこの中の用例提示型の翻訳支援システムの研究を行っている.このシ
ステムは一般的に巨大な対訳用例データベースと検索システムから構成される.
このシステムに対して利用者は「翻訳がわからない」と思う表現を入力する.す
るとシステムは入力に一致した表現,あるいは類似した表現をデータベース中で
検索してその翻訳例を提示する.利用者は提示された翻訳例を参考に翻訳を作
成する.

機械翻訳システムと違ってこの場合の翻訳の主体は利用者にあり,システムは
利用者に参考となる情報を提示するだけである.
このように利用者主体の翻訳作業を支援する考え方は Kay\cite{Kay97}によっ
て 1980 年に提案されている.この文献では電子化辞書を使った支援を提案し
ているが,対訳用例を使う翻訳支援もこの考えを基本的に踏襲したものである.
また実際に対訳用例を使って日英翻訳支援システムを作成した例としては
\cite{Naka89,Sumi91}等の先駆的なシステムがある.さらに最近では商
用システムもいくつか販売されている.

著者らは上記一連の研究と同一の考えに基づいて,日本語ニュースの英訳支援
のためのシステムを開発している.このシステムには二つの特徴がある.一つ
は利用する日英用例の対応付けの粒度である.
従来の研究では,表現の対応を集めた日英表現翻訳辞書や文間の対応付けを行っ
たデータベースなど詳細な単位で対応のとれたデータベースを利用することが多
かった.

これらに対して著者らのデータベースは記事と
いう大きな単位での対応はとれているが,それより細かな対応はとれていない.
これは日本語ニュース記事を英訳する場合に,英語視聴者の背景知
識や興味に合わせて大きく意訳することがあるためである.極端な場合は日本語ニュー
スを参考にして英文ニュースを新たに作成する場合もある.

このため入力の
検索結果に対応する翻訳部分を提示するには,日英表現の自動的な照合が必要になる.
そしてこの場合に表現が照合しないことも前提にしなくてはならない.

第二点は「意訳の支援」である.
従来,用例提示型のシステムはマニュアル翻訳のような定型
的な翻訳に応用する場合が多かった.たしかにニュース翻訳の場合でも「株価」
「天気予報」「新車販売台数月例報告」などの項目はほぼ定型的な文から成り
立っており,これらを有効に支援できると思われる.

しかし著者らは本システムで意訳を積極的に支援
したいと考えている.なぜなら意訳こそニュース翻訳の難しい部分であり,ま
た用例によって有効に支援できると考えるからである.
例えば日本語の短い言い回し「いかがなものか」は本稿のデータベース中だけ
でも過去 10 通り程度に訳されている.同様に同じ単語や似たような文が文脈
によってどのように意訳されているかを観察すれば,意訳のための知識を効果
的に学ぶことができると考える.

意訳であろうと定型的な翻訳を支援する場合であろうと,表現を検索する部分
には同じ手法を利用できる.しかし,結果の表示には異なった配慮が必要であ
る.定型的な翻訳であれば入力に対応する翻訳例を一つ示せば十分である.
しかし意訳を支援するにはできるだけたくさんの翻訳例を文脈付きで利用者に
提示する必要がある.このため著者らのシステムは検索速度を重視している.また
どのような長さの入力であっても出力は日本語と英語の記事を提示した上で,
対応個所を強調して表示している.


本稿は上記のシステム中の検索部分を対象としている.著者らは一文字から
一記事までの範囲を入力として類似検索ができるシステムを研究して
いる.これは意訳が単語や短い表現から文や記事までの広い範囲で行われるた
めである.

実際には一文字から一文までを対象にした検索システムと
記事を対象にした検索システムの二つを作成した.本稿はこのうちの一文までの表現を対象として類似用例を検索す
る手法について報告する.

著者らはこの検索を頑健で柔軟かつ高速に行うためキーワードの AND 検索を基本
的な手法として採用した.すなわち,入力を形態素解析してあらかじめ指定し
ている品詞のキーワードを抽出して AND 検索を行う手法である.しかし単純
な AND 検索を行うと不適切な結果を多数表示することが判明した.

そこで著者らは AND 検索に語順と「変位」と呼ぶ制限を加えることを提案す
る.これは表層的な情報を利用して AND 検索に構文的な情報を反映させようという
試みである.この手法は構文解析を利用していないため速度と頑健性に優れている.

以下,本稿の構成を示す.
まず~\ref{sec:gaiyou}~章で著者らの用例提示型翻訳支援システムの概要
を説明して,この中の類似用例検索部分の設計方針を示す.\ref{sec:mondai}~章で
は類似用例検索にキーワードによる AND 検索を利用した場合に起こる問題を
示す.続く~\ref{sec:algo}~章では AND 検索に語順と変位を使う手法を提案す
る.またこの手法を使った検索手順をアルゴリズムの形で示す.そして~
\ref{sec:jikken}~章で約 160 
万用例からなるデータベースを使った検索実験を報告する.ここでは検索時間
と,検索結果の主観的な満足度などを報告し,提案手法は AND 検索にくらべてわずか
に検索時間が増加するものの(約 1.3 倍)利用者の満足度は統計的に有意に優れて
いたことを示す.次に~\ref{sec:kanren}~章では関連研究を紹介して本研究との比較
を行い最後に~\ref{sec:ketsuron}~章で本稿のまとめを行う.

\section{用例提示システムの概要}
\label{sec:gaiyou}
\subsection{構成}
用例提示システムは下記の部分から構成されている.
\begin{itemize}
 \item 日本語と英語の 2 言語ニュース記事データベース
 \item 日本語類似用例検索システム
 \item 検索結果の表示システム
\end{itemize}

このシステムは次のような形で利用する.まず英訳したい日本語ニュース記
事がある.この中で翻訳を調査したい表現があればこれをシステムに入力する.この
とき利用者は表現を編集せず,カットアンドペーストで入力することを想定している.
システムは入力に一致する表現,あるいは最も近い表現を日本語記事データベー
スで検索する.結果は日本語,英語とも記事を表示単位として,日本語の検索
結果を含む文とそれに対応する英文を強調して表示する.記事を単位として表
示するのは,文脈を利用者に提供することが重要だと考えるからである.ユー
ザは提示された過去の翻訳例を参照して自分の翻訳を作成する.もし結果に満
足できなければ次の検索結果をシステムに要求する.
\vspace{-3mm}
\subsection{データベース}
\label{sec:database}
\vspace{-1mm}
2 言語記事データベースは NHK の日本語ニュース記事とその人手による英訳
を 1995 年 3 月から 1997 年 2 月までの期間蓄積して作成したものである.
表~\ref{tab:database}~に本稿のシステムで使った日本語記事データベースの大きさを示す
\footnote{表~\ref{tab:database}~中の文の数は,記事内容を表す文と,記事
の作成者,タイトルといった付加情報を表す文を合わせた数である.検索ではこれら
すべてを対象としている.記事内容を表す文は 446,444 件(76 MB)である.
}.また,英語の部分もほぼ同じ規模である.

英語のニュース記事は日本語の記事全体を元にして作成しており,日本語文の単純
な直訳を集めたものではない.これは英語視聴者の視点に立った分かりやすい
ニュースを作成するために意訳が求められるからである.このため文間の対応関係
を単純に求めることは難しい.そこで\cite{Kuma97}で提案された手法を使って
対応付けを実施した.

さらに任意の日本語文字列を高速に検索できるように日本語データベースに対
してポインタ表現の部分列インデックスを作成した\cite{Naga96}.このイン
デックスにより,任意長の入力文字列の出現位置を漏らさず高速に求めること
ができる.この時,どの記事のどの文に出現したかもわかるようにインデック
スを作成している.

\begin{table}
 \begin{center}
  \caption{日本語部分のデータベースの規模}
  \begin{tabular}{l|r}\hline\hline
   記事数 & 94,830 件\\
   文の数 & 1,615,119 件\\
   バイト数 & 104 MB\\\hline
  \end{tabular}
  \label{tab:database}
 \end{center}
\end{table}

\subsection{検索部分の設計方針}
\label{sec:houshin}
\ref{sec:hajimeni}~章で述べたように著者らはこのシステムを使って定型
的な翻訳だけでなく意訳を支援したいと考えている.
この目標はシステムの想定利用者であるニュース翻訳者への面接調査を行って
設定した.
翻訳者は日本語,英語とも基本的に堪能である.しかし経験によってはニュー
ス翻訳の知識が十分でない場合がある.
面接調査によると,「固有名詞」や「複合語」などの主に定型的翻訳を行う表
現と,単語,表現,文,記事のさまざまな段階で必要になる意訳を支援してほ
しいという要求があった.

定型的翻訳,意訳とも多種類の表現が対象になる.そこで著者らのシステムで
はさまざまな長さの入力に対して検索できるようにした.具体的には一文字か
ら一文までを入力対象にした検索システムと記事を入力対象にした検索システ
ムの二つを作成した.日本語ニュース記事は 5 文程度からなっており段落が
ない.そこで実用上はこの分類で十分と考えたからである.
本稿はこのうちの一文までの文字列を入力として類似用例を検索す
る手法について報告する
\footnote{
記事を入力とした検索および閲覧システムに付いては\cite{Tan97a,Tan97b,Tan99a}を参照されたい.
}
.なお本稿ではこの検索システムを表現検索システムと呼ぶ.

表現検索システムには次の 2 種類の検索機能がある.
\begin{itemize}
\vspace{-0.25mm}
\item  完全一致検索\\
\vspace{-0.25mm}\hspace*{-3pt}
一文までの範囲の文字列を入力して
これに完全に一致する表現の出現位置を~\ref{sec:database}~節のインデック
スを参照して漏らさず求める.この結果,これを
含む記事と文とその中の位置を特定することができる.長い入力に対して
は結果が得られない可能性が高いが,慣用的な表現を検索するのに有効である
\vspace{-0.25mm}
\item  類似検索\\
\vspace{-0.25mm}\hspace*{-3pt}
入力文字列を形態素解析して自立語を抽出しこれをキーワードとする
\footnote{このとき活用する自立語は活用形に展開し,さらに間違ったキーワードを検索しないように接続し得る機能語を付与した展開を行う.\ref{sec:jissou}~節参照.}.
データベース中\break
の日本語の各文を対象にキーワードをなるべく多く含む文を検索す
る.すなわち入力表現
と用例文の類似性は共有するキーワードの数で評価する.キーワードの出現位
置は上記の完全一致検索を利用することで高速かつ完全に求めることができる.
キーワードの組み合わせによる検索もあとで述べるように高速に実現できる.
なお類似検索は一文を検索対象とするため,以後,一文と用例を同じ意味で使
用する
\end{itemize}
類似検索は次のような手順で実行する.また具体例は図~\ref{fig:nagare} に
示す.
\begin{enumerate}
\item 最初はすべてのキーワードを含んだ用例を検索する.成功すればそれら
を表示する.
\item もし,検索に失敗するか,成功しても利用者がさらに検索を要求
した場合にはキーワード数を一つ減らして検索を続ける.この時一度表示し
た用例は検索の対象としない.なぜなら同じ用例を提示しても利用者は新たな
情報を得られないからである.
\end{enumerate}
このようにキーワード数の条件を利用者の指示で徐々に緩和して検索を実行す
る.条件を緩和する場合にはキーワードすべてが同じ重要性を持つと仮定して,
任意のキーワードが一つなくなった条件で検索を行う.

ここで提案した検索は特殊な処理を想定していないため頑健である.またキーワー
ドの選択方法,キーワードの緩和方法を変えることでさまざまな検索を実現で
きるため柔軟性も高い.このため将来の拡張も比較的簡単である.
また高速なため,満足な解が得られない場合は何度でも検索できる.
さらに類似検索の結果を提示する場合も根拠としてキーワードを提示できるた
め直感的な理解が容易になる.
なお,完全一致検索はポインタ表現の部分列インデックスを参照することでそのまま
実現できるので以降では類似検索部分のみ議論する.

\begin{figure}
 \begin{center}
  \epsfile{file=98.eps,height=10cm}
  \caption{類似検索概念図}
  \label{fig:nagare}
 \end{center}
\end{figure}
\section{AND 検索の問題点}
\label{sec:mondai}
\ref{sec:gaiyou}~章で述べた類似検索を実現するにあたり著者らは最初キー
ワードの AND 検索を採用した\cite{Salt83}.
すなわち入力キーワードと検索対象中のキーワードの語順の一致
を考慮しない手法である.

AND 検索を採用した理由の 1 つは高速性である.データベース中での各キーワードの
出現位置さえわかれば,これらを AND 条件で含む記事を特定するのは容易で高
速である.
もう 1 つの理由は
\ref{sec:kanren}~章で述べるように AND 検索を採用した用例検索システムが
多く提案されており効果的であると報告されていたからである.

しかし日本語ニュース原稿を対象に AND 検索を使うと問題が発生することが
明らかになった.
問題の例を示そう.
例えば「政府の作業」の類似用例を検索するのに \{政府 {\bf and} 作業\} で検索する
と下記の文をすべて出力する.尚,用例中の照合キーワードを太字で強調している.
\smallskip
\begin{quote}
例 1

外務省の橋本外務報道官も,きのうの記者会見で,「保証人委員会は一生懸命
{\gt 作業}をしているが,ペルー{\gt 政府}と武装グループが,保証人委員会
の努力を受け入れる所まで事態は進んでいない」と述べました.

例 2

この問題に関する自民党の対外経済協力特別委員会が今日午後開かれ,{\gt 
政府}側は,「中国は去年七月に核実験を行なった後,今後の核実験を凍結す
ると表明しており,無償資金協力の再開に向けた準備{\gt 作業}を進めていき
たい.」と述べました.

例 3

また池田外務大臣は,「日本{\gt 政府}とペルー{\gt 政府}との間は信頼関係
が出来ている」と述べ,両国{\gt 政府}の間で緊密に連絡を取っていることを
明らかにするとともに,今後の日本の役割について「関係国の間で,バラバラ
の対応にならないよう,国際社会が一致してペルー{\gt 政府}の進め方を支え
ていくことが重要だ.日本{\gt 政府}は,事件の解決に向けたペルー\underline{{\gt 政府}の{\gt 作業}}がうまく運ぶよう,条件を整える努力をしてきており,今後
はこうした努力が一層大切になる」と述べました.
\end{quote}

\smallskip

例 1 には「政府」「作業」というキーワードが一文中に出現している.しかしこの順
序が逆転しており,またその間に関連がなく類似用例とは考えられない.
例 2 では 2 つのキーワードが出現しており語順も入力と同じである.しかし
両者に係り受け関係はないため類似用例とは考えられない.
例 3 では「政府」が 6 個所,「作業」が 1 個所出現している.この中で,
下線部が入力表現と一致しており用例 3 は類似用例と判断でき
る.しかしこの用例には「政府」が 6 個所
も出現しているため下線がなければ該当個所を見いだすのは容易ではない.

ここで使った入力表現「政府の作業」は短いため,この中に重複するキー
ワードはない.
しかし長い入力表現では同じキーワードが出現する可能性がある.こ
の場合,照合部分を把握するのはさらに困難になる.

まとめると AND 検索の問題は例 1 と 例 2 で示したような不正解文
を拾いやすいこと,正解であっても例 3 のように該当個所を確認しにくいこ
とである.このような問題が発生する主な原因は日本語ニュースの文の平均長が 
88.9 文字\cite{Kuma96}と長いことにある.短い用例を使ったシステムではこ
のような問題は発生しにくいであろう.

これらの問題を解決する
には構文解析を利用する手法が考えられる.入力のキー
ワード間の係り受け関係を認定して,同様の係り受け関係を持つ用例を検索す
る手法である\cite{Hyou94}.
しかし現時点では構文解析器の精度が十分でないためこの手法は採用しにくい.そこで
著者らはこれらの問題を構文解析せずに \ref{sec:algo}~章で提案する近似的な手
法で解決することにした.
\section{提案手法}
\label{sec:algo}
単純な AND 検索手法には\ref{sec:mondai}~章で述べた問題がある.またこ
れらを解決するのに構文解析を使うことは困難である.そこで
著者らは AND 検索に語順と変位とよぶ制約を加えた検索手法を考案した.尚,
以下ではこの手法を AND+W+D(AND + Word order + Deviation)検索とよぶ.
この手法は構文解析をせずに,表層の単語のならびと位置情報を使って近似
的に構文的な情報を捉えたものである.
本章では AND 検索, AND 検索に語順を加えた検索(AND+W 検索)について説
明し,その上で提案手法(AND+W+D 検索)を説明する.次にその実装アル
ゴリズムを説明する.

以下では
次の入力例を用いて説明を行う.
\begin{center}
\begin{tabular}{ll}
入力表現 & **A*B**A*C*
\end{tabular}
\end{center}
ここで ``A, B, C'' はキーワード, ``*'' はそれ以外の単語とする.また簡
単のためキーワードや単語はすべて一文字とする.
\subsection{AND 検索}
AND 検索では次の 4 つの用例をすべて出力する.ここでは用例中の照合
したキーワードを強調表示している.
\begin{center}
\begin{tabular}{ll}
用例 1 & *\underline{\bf A}**\underline{\bf B}*\underline{\bf C}**\underline{\bf A}**\\
用例 2 & *\underline{\bf AA}*\underline{\bf B}***\underline{\bf C}\\
用例 3 & *\underline{\bf A}**\underline{\bf A}**\underline{\bf B}*\underline{\bf A}**\underline{\bf C}\\
用例 4 & *\underline{\bf A}**\underline{\bf A}*\underline{\bf B}**\underline{\bf A}*\underline{\bf C}
\end{tabular}
\end{center}

これらの用例は順序が違ってもキーワード ``A, B, C'' を含んでいるので条
件を満たす.また 4 つの用例の間に優先順位はない.

用例 3 と用例 4 に
は ``A'' が 3 つあるが,この中のどの 2 つと照合したかを決めることができな
い
\footnote{入力に合わせて 2 つ ``A'' を選択するのであれば任意に選択する
しかなくあいまいである.}.
以上の問題は先に \ref{sec:mondai}~章の例 3 で具体例で説明した問題と同一であ
る.

\subsection{語順を考慮した AND 検索}
入力と語順が同じ表現はそうでない表現より近いであろう.なぜなら語順はある程
度構文の情報を担うからである.そこで AND 検索に語順の制約を付加するこ
とで類似性の低い不適切な検索結果を減らせると期待できる.
例えば上記の例でキー
ワードの語順を考慮して検索すると用例 3 と用例 4 だけが出力される.
\ref{sec:mondai}~章の例で言うと,「政府の作業」に対して例 2 と例 3 だ
けに解を絞ったことに相当する.
\begin{center}
\begin{tabular}{ll}
用例 3 & *\underline{\bf A}**A**\underline{\bf B}*\underline{\bf A}**\underline{\bf C}\\
用例 4 & *\underline{\bf A}**A*\underline{\bf B}**\underline{\bf A}*\underline{\bf C}
\end{tabular}
\end{center}
しかし,語順だけでは不十分な点がある.
まず用例 3 と 4 には最初に A が 2 つあるがどちらが照合キーワードなのか決め
ることができず照合個所を特定できない.
また用例の間に優先順位をつけることができない.キーワードの数と語順が同
じ用例が検索されたときにその提示の優先順位を決められない問題
である.これは大規模なデータベースを対象にした場合に結果を絞り込めない
問題につながる.
\subsection{語順と変位を考慮した AND 検索}
\label{sec:teian}
著者らは上記の問題を解決するために以下で説明するキーワードの「変位」を
使った手法を利用した.

まず入力中のキーワード $x_i$ の出現位置を与える関数を $org(x_i)$ とす
る.この値は任意のキーワードについて一意に決めることができる.
これに対してキーワード $x_i$ の用例内での出現位置を与える関数を 
$pos(x_i)$ とする\footnote{$org(x_i), pos(x_i)$ は差をとるためデータベース中での絶対位置であっても文毎の相対位置であってもかまわない.以下では文毎の相対位置とする.}.
もし $pos(x_i)$ の値が決まればキーワード $x_i$ を入力と用例で照合で
きることになる.しかし,現在の例のように用例に同一キーワードが複数出現す
る場合には一意に照合できない.

ここで入力中の $x_i$ の右隣のキーワードが $x_{i+1}$ であるとする.また用例中にも $x_i$ と $x_{i+1}$ と同じ 2 つのキーワードが出現しているとする.ただし用
例にはこの 2 つのキーワードが複数出現しておりキーワードの対応があいま
いだとする.この時,次式で定義するキーワード対の変位 
$dev(x_i,x_{i+1})$ が最小になるように $pos(x_i),pos(x_{i+1})$ を決める
ことにする.この基準を使えば変位が同じ場合を除いて一意に照合することが
できる.

\begin{equation}
dev(x_i,x_{i+1}) = |(org(x_{i+1})-org(x_i))  - ( pos(x_{i+1})-pos(x_i))|
\label{for:dev}
\end{equation}

(\ref{for:dev})~式は入力のキーワード対の間隔と用例のキーワード対の間
隔の差である.これが最も小さくなるように照合するのは,キーワードの間隔
が似ている場合には係り受け関係も近い可能性があると考えたからである.
例えばこの経験則で~\ref{sec:mondai}~章の例 3 の「政府の作業」の照合を
正しく行うことができる.

一般に入力のキーワードが $n(n \ge 2)$ 個ある場合には隣接キーワード対の
変位を利用して,その合計が最小になるように照合する\footnote{隣接キーワー
ド一組では必ずしも構文の近さを反映しない場合がある.「政府の作業」と
「政府に作業」は係り受けは違うが変位は 0 である.キーワードが増えて制
限が強くなるほど構文的近さの良い近似になる傾向がある.}.
\begin{equation}
 \sum_{i=1}^{n-1} dev(x_i, x_{i+1})
\end{equation}

現在の例で入力には $\{A,B,A,C\}$ というキーワードがある.そこ
で $dev(A, B)+dev(B,A)+dev(A,C)$ が最小となるようキーワードを対応させ
る.また,この値の小さな順に用例を提示する.この結果は次の通りである.

\begin{center}
\begin{tabular}{lll}
用例 4 & *A**\underline{\bf A}*\underline{\bf B}**\underline{\bf A}*\underline{\bf C} & 変位合計 0\\
用例 3 & *A**\underline{\bf A}**\underline{\bf B}*\underline{\bf A}**\underline{\bf C} & 変位合計 3
\end{tabular}
\end{center}

AND+W 検索と同じ用例を検索しているが,照合したキーワードを特定できて
おり,また検索用例に順位がついていることに注意されたい.

ここで AND+W+D 検索の特徴をまとめる.
\begin{itemize}
\item キーワードのあいまい性の解消\\
AND+W+D 検索はキーワード照合にあいまい性がある場合にそれを解消する能力
がある.\ref{sec:mondai}~章の例 3 の場合では「政府」の照合個所を下線部分に特
定できる.この性質は結果を表示する場合に有用である.
\item 用例の順位付けが可能\\
AND+W 検索 と AND+W+D 検索が同じ入力キーワード群で出力する用例集合は上
記の例のように常に一致する.違いの一つは用例に優先順位がつく点である.
例えば,\ref{sec:mondai}~章の例の例 2 と例 3 のキーワード数は 2 で同じ
である.しかし例 3 の変位合計は 0 であるため 1 位となり例 2 は変位が大
きいので 2 位の解となる.
\item 完全一致検索に近い\\
名詞複合語を検索する場合は構成要素の名詞が連続した用例が正解である.AND+W+D 検索ではもし入力と同一の複合語があればその変位合計は 0 となっ
て第 1 位で出力される.すな
わち完全一致検索の機能も包含した検索手法となっている.一方,AND+W 検索
ではこのような保証はない.この性質は特に名詞複合語の検索が多くなる場
合に有利である.
\end{itemize}

一般的に AND 検索 は同じ入力キーワード
群に対して語順を考慮した AND+W 検索 と AND+W+D 検索より多くの文を検索する傾
向がある.
ただし,キーワード数を 1 まで減らして検索できる文の集合はいずれの手法
も同じである.すなわちデータベース中の類似用例の正解がどう定義されていて
も最大限に条件を緩和すれば 3 手法の再現率は同じことになる.
\subsection{アルゴリズムの概要}
\label{sec:jissou}
類似検索全体のアルゴリズムは~\ref{sec:houshin}~節に示した手順に
従っている.すなわち利用者の要求に従ってキーワードの数を一つずつ減らして検索を
行う.このとき利用者は途中で検索を打ち切ることが可能である
\footnote{実際,キーワード数が一つになるまで条件を緩和することは考えにくい.}
.
また,用例はキーワードを最大個数含む段階で表示するものとし,それ以後の
キーワードを削減した段階では表示しない.
ここでは上記を考慮した AND+W+D 検索アルゴリズムの概要を説明する.
処理の大まかな流れは以下の通りである.
\begin{itemize}
 \item 入力表現の形態素解析を行ってキーワードを求める
 \item 用例集合中でのキーワードの出現位置を完全一致検索で求める
 \item キーワードが出現している用例についてはノードテーブルを作成する.
ノードテーブルは検索に使うデータ構造である
 \item 検索,表示処理のループ.ユーザの要求によって繰り返す
 \begin{itemize}
  \item 検索処理
  \item 表示処理
 \end{itemize}
\end{itemize}

入力表現は形態素解析されて自立語がキーワードとして抽出される.
キーワードのうち活用語は活用語尾や「な(い)」「つつ」など接続し得る機能語をすべて付加して展開する.本システムではキーワードの出現位置を
文字列検索(完全一致検索)によって求めている.このため活用するキーワー
ドは可能な出現形
で検索する必要がある.このとき活用語尾を付加しただけでは間違った品詞のキーワー
ドを検索する場合がある.例えば一段動詞の未然形や連用形の「衰え」を検索
すると名詞の「衰え」を検索する恐れがある.このため機能語も付加して検索
することで誤検索を防いでいる.
ただし,例えば否定の「ない」は「なかっ」「なかろ」「なく」「ない」「な
けれ」と活用するが「な」だけを付加する.つまり誤検索を防ぐのに必要十
分な機能語部分文字列を付加する戦略を取っている.

入力には同じ表層形のキーワードが複数出現する場合があるため,出現順に付
番してすべてを区別する.この番号をキーワード id と呼ぶ.ただし展開で得られ
る派生キーワードは同一のキーワード id とする.

全用例集合を対象に各キーワード表層形を完全一致検索によって検索し,そ
れぞれが出現した用例とその中での位置を求める.キーワー
ドが一つ以上出現している用例については,ノードテーブルと呼ぶデータ構造
を作成する.

前節で使った入力と 4 つの用例に対応するノードテーブルの例を図~\ref{fig:dousa}~に示す.
\begin{figure}
\begin{center}
\epsfile{file=104.eps,height=10cm}
\caption{ノードテーブルと検索結果}
\label{fig:dousa}
\vspace*{-1mm}
\end{center}
\end{figure}

この図の最上段は入力のキーワードの出現位置を示している.下部の $g_1$ 
から $g_4$ に示したノード群が用例 1 から 4 に対応するノードテーブルで
ある.このテーブルは入力のキーワードを出現順,つまりキーワード id 順に
並べ,各 id のキーワード表層形の出現位置をノードとして記述して
いる.
各ノードは対応するキーワード id で管理されている\footnote{活用する語の場合には同一キーワード id に複数の表層形があ
り,そのすべての出現位置を同一キーワード id 下のノードとする.}.

ここで入力中にキーワードが $M$ 個あるとする.
このノードテーブルには次の性質がある.
\begin{itemize}
  \item [(性質 1)] 用例のノードテーブルに対して任意のノードから
右方向でかつ出現位置が増加するような経路を作成したとする.この経路上のノード
集合は語順の条件を満たすキーワード集合である.
図~\ref{fig:dousa}~の矢印が経路の例である.以後,経路とはこの
条件を満たす経路であるとする.

任意のノードからの経路を求める場合に,ノードをなるべく多く含んでかつ変
位合計が最小となる経路はグラフの最短経路問題として定式化できるので従来のアルゴ
リズムで高速に求めることができる
\footnote{ノード間の辺のスコアは式~(\ref{for:dev})~で定義する変
位とする.}
.すなわち各ノードを始点とする最適な検索結果を求めることができる.
  \item [(性質 2)] $N$ 個のキーワードを含む経路はキーワード id  
が $M-N+1$ 以下(左)であるノードを開始点とする経路上にしかない.
例えば図~\ref{fig:dousa}~においてキーワード数 3 の経路は 
キーワード id が 1  のノードと 2 のノードを開始点とする場合しかない.

別な見方をするとキーワード id が $M-N+1$ に属するノードを開始点として
経路を求めると,$N$ 以下のキーワードを含む場合の経路を求めることができ
る.
\end{itemize}
以上の性質を利用して図~\ref{fig:shousai}~に示す AND+W+D 検索を実現した.
\begin{figure}
\begin{quote}
\baselineskip=12pt
\sfcode`;=3000
\def\q{}
a1) \q 入力のキーワードを抽出,展開する;\\
a2) \q 用例集合でのキーワードの出現位置(文と位置)をすべて求める;\\
a3) \q $M$  キーワード数;\\
a4) \q $N = M$;\\[3mm]
b1) \q  $G \leftarrow \emptyset$;\\
b2) \q {\bf foreach} \{ $i \mid s_i \in S$\} \{   全用例に対して\\
b3) \q \q {\bf if} (用例 $s_i$ がキーワードを一つ以上含むならば) \{\\
b4) \q \q \q 用例 $s_i$ のノードテーブル $g_i$ を作成する;\\
b5) \q \q \q $G \leftarrow G \cup \{i\}$;  検索対象用例リストを作成
する\\ 
b6) \q \q \}\\
b7) \q \}\\[3mm]
c1) \q {\bf while} ( $N > 0$) \{\\
c2) \q \q $startId = M-N+1$;  開始点のキーワード id の設定\\
c3) \q \q {\bf foreach} \{$i \mid i \in G$\}\{  検索対象用例 $i$ に対して \\
c4) \q \q \q {\bf foreach} \{$n \mid g_i$ の $startId$ に属する各ノード
\}\{\\
c5-1) \q \q \q \q $g_i$ の ノード $n$ から始まる最適経路を求める\\
c5-2) \q \q \q \q 最適経路上のノード数を $num$\\
c5-3) \q \q \q \q 変位を $newdev$\\
c5-4) \q \q \q \q キーワード出現位置リストを $list$ とする\\
c6) \q \q \q \q {\bf if} ($R(i).dev > newdev$) \{\\
c7-1) \q \q \q \q \q $R(i).dev  \leftarrow newdev;$\\
c7-2) \q \q \q \q \q $R(i).kwd\_num \leftarrow num;$\\
c7-3) \q \q \q \q \q $R(i).kwd\_list \leftarrow list;$\\
c8) \q \q \q \q \}\\
c9) \q\q \q \}\\
c10) \q \q \}\\[3mm]
d1) \q \q {\bf foreach} \{$i \mid i \in \{R(i).kwd\_num = N\}\}$\{\\
d2) \q \q \q {\it dev} の小さい順に用例 $i$ とその照合キーワードを表示する;\\
d3) \q \q \q $G \leftarrow G - \{i\}$; 表示した集合を対象用例から削除
する\\
d4) \q \q \}\\
d5) \q \q {\bf if} ( 利用者が終了を指示 {\bf or} $G = \emptyset$ ) \{ 終了;\}\\
d6) \q \q {\bf else} \{ $N \leftarrow N - 1$;\}\\
d7) \q \}
\end{quote}
\caption{AND+W+D 検索の基本アルゴリズム}
\label{fig:shousai}
\end{figure}

アルゴリズム中の変数 $S$ は全用例集合を示す$S = \{s_1, s_2,\ldots s_i, \ldots\}$.$G$ は検索対象の用例の番号を記録するリスト変数である(b5).
一度表示した用例はこのリストから削除することで以降の処理を行わないように
する(d3). 

$R(i)$ は用例 $s_i$ に関する情報を格納する構造体の配列である.この変数
には用例 $s_i$ の最大キーワード含有数 
$R(i).kwd\_num$,その変位合計 $R(i).dev$,キーワード群の出現位置 
$R(i).kwd\_list$ を記録する(c7).

検索処理の中心部分は(c1--c10)である.キーワード数 $N$ の用例を検索す
るために開始キーワード id を(c2)で設定している.このあと各用例 $i$ 
の中でこのキーワード id に属するノードから最適経路を探索している(c5).
検索開始キーワード id はキー
ワード数の緩和(d6)に伴って 1 から順に増加するように設定されている(c2).こ
のため(性質 2) から用例 $i$ でキーワード数 $N$ の経路が見つかった場
合に,すでに同数の解が以前の開始キーワード id ($\{1, \ldots, M-N\}$)
での探索で発見されている可能性がある.そこでこのような場合には変位合計
の小さい解だけを残す処理を行っている(c6--c7).

図~\ref{fig:dousa}~の矢印で示す経路と変位は,キーワード id 1 のノード
を開始点として最適経路を求めた結果である.用例 1 と 2 についてはキーワー
ド数 3 の解が,用例 3 と 4 についてはキーワード数 4 の解が求められてい
る.

以上のように本手法はキーワードを最大限含む順に解を求めている.多くの場
合利用者は条件緩和の途中で検索を打ち切るので,この順序で解を求めている.
ただしこの手法では用例のノード集合を最初に全部保持するためデータベース
の大きさによってはメモリの消費が問題になる可能性がある.この場合には各
用例の最適解を最初に求めるなど変更する余地はある.いずれにせよ動的計画
法を利用すれば実用的な速度で解を求めることができる.
\section{検索実験}
\label{sec:jikken}
\subsection{実行時間}
\label{sec:jikan}
AND 検索,AND+W 検索,AND+W+D 検索を対象に検索時間を評価した.
AND+W 検索は AND+W+D 検索アルゴリズムをほぼそのまま利用して作成した.
AND 検索はキーワードのノード集合を作成する過程で出現キーワード数を計数す
ることで実現した.

検索対象データベースは 1995 年 3 月から 1997 年 2 月までに収集した日本語ニュースである(表~\ref{tab:database}).
入力したのは 1997 年 
3 月のニュース記事からランダムに選んだ 500 記事の各先頭行 500 行である.これらの記事は検索対
象データ
ベースに含まれていない.また入力の平均文字数は 92.7 文字と長い.このた
めキーワードが完全一致する用例はほとんどない.

実験手順は以下のとおりである.
\begin{table}
 \begin{center}
  \caption{検索時間の比較(秒)}
  \label{tab:jikan}
  \begin{tabular}{l|rrr}\hline\hline
   手法 & \multicolumn{1}{c}{AND} & \multicolumn{1}{c}{AND+W} &
\multicolumn{1}{c}{AND+W+D}\\\cline{2-4}
   総時間 & 25,573.1 & 33,201.6 & 33,426.9\\
   一回の緩和の平均時間 & 2.33 & 3.02 & 3.04 \\\hline 
  \end{tabular}
 \end{center}
\end{table}
入力の各文を形態素解析して自立語キーワードを抽出する.そして各文でキー
ワード数 1 になるまで条件を自動的に緩和しながら検索
を行ってその累積時間を計測した
\footnote{利用したワークステーションの記憶容量は 256MB であり,処理速度は {\it SPECint92} $=202.9$,{\it SPECfp92} $=259.5$ である.
 }
.
結果を表~\ref{tab:jikan} に示す.緩和の合計回数は 3 手法で等しく 
10,989 回である.

この表から語順を考慮した AND+W 手法と AND+W+D 手法 でも AND 検索にくら
べて約 1.3 倍の時間増だったことがわかる.また AND+W と AND+W+D を比
較すると,時間の差はほとんどなく変位の有無の影響はほとんどなかったこと
がわかる.

1 回の緩和,すなわち利用者が 1 つキーワードを減少するよう指示した時に
要した平均検索時間(総時間$/10,989$)を 2 行目に記した.これによれば AND 
検索で 2 秒,AND+W,AND+W+D で 3 秒程度である.ただし総時間に
はキーワードの出現位置をディスクからメモリに転送する初期処理の時間を含めて
おり,この時間がかなりの部分を占めている.
初期処理が終了した後の 1 回の緩和に要する時間は語順と変位を考慮
しても 1 ないし 2 秒であり実用上満足できる速度であった.
以上の実験より語順と変位を考慮しても AND 検索なみに十分高速に検索でき
ることを確認した.
\subsection{検索文数の絞り込み効果}
\label{sec:siborikomi}
\ref{sec:teian}~節の終わりに述べたようにキーワード数を 1 まで緩和すれ
ば,AND,AND+W,AND+W+D 手法で検索できる用例集合は同じである.
しかし実際にはキーワード
数の大きなところで検索を打ち切るので利用者の見る用例数は語順の制約の有
無で違ってくる.
この違いを前節と同じ入力を使って評価した.

語順を考慮した手法 AND+W
\footnote{AND+W と AND+W+D は同じキーワード集合に対して同じ用例集合を検
索するので,ここでは AND+W で代表する.}と
AND 検索それぞれについて,各キーワード数での検索用例数を計測した.結果を
図~\ref{fig:kazu} に示す.検索用例数は入力 500 文で合計したものである.横軸
はキーワード数で縦軸は対数を取った検索数である.また,このグラフの一部
の検索用例数を表~\ref{tab:kazu} に示す.

\begin{figure}
 \begin{center}
\epsfile{file=108.eps,height=10cm}
  \caption{各キーワード数での検索用例数(500 入力での合計)}
  \label{fig:kazu}
 \end{center}
\end{figure}
\begin{table}
 \begin{center}
  \caption{各キーワード数での検索用例数(500 入力での合計)}
  \label{tab:kazu}
   \begin{tabular}{r|r|r}\hline\hline
    \multicolumn{1}{c|}{キーワード数} & \multicolumn{1}{c|}{AND} & \multicolumn{1}{c}{AND+W(+D)}\\\cline{1-3}
    1 & 54,374,495 & 59,258,536\\
    2 & 13,923,270 & 11,338,161\\
    3 & 3,470,894 & 2,088,640\\
    $\vdots$ & $\vdots$ & $\vdots$\\\hline
    
    
    
    
    
    
    
    
    
    
    
    
    
    
    
    
    
    
    
    
    
    
    
    
    
    
    
    
    
  \end{tabular}
 \end{center}
\end{table}
アルゴリズムの説明で述べたように,文は含有キーワード数が最大の時に一度
表示するだけである.このため図~\ref{fig:kazu} の 2 つのグラフの面積は
等しくなる.表~\ref{tab:kazu} から明らかなようにキーワード数 1 の部分
で AND+W 検索の検索用例数が AND 検索の結果を大きく超えている.この結果キーワー
ド数 が 1 以外の部分では AND+W 検索の検索数は AND より小さく押さえられ
ている.
すなわち語順制約を使うことでキーワード数が 1 より大きい部分で検索結果
数を絞り込むことができたことが確認できる.

しかし,語順による絞り込みの効果が十分であるとは言えない.キーワード数が小さ
い部分での絶対的な検索数が大きいのが問題である.例えば 3 語のキーワー
ドで検索する用例数は平均で $4,177$($= 2,088,640/500$) に達した.このような
用例も検索する可能性があるため絞り込み効果は十分でない.
逆に 50 程度の検索用例数を許容すると仮定すればキーワードの語順を考え
ない場合でも 10 語以上含む入力であれば許容範囲の検索数となる.語順を使っ
た場合は 8 語以上で許容範囲となる.
\vspace*{-4mm}
\subsection{検索結果の満足度}
\vspace*{-0.5mm}
\label{sec:manzoku}
語順に変位の制約を加えると検索結果に優先順位をつけることができる.この
順位が妥当であれば検索用例数の多さの問題は解決できる.そこで著者らはこ
の順位の妥当性を検証するため 3 手法の検索結果の満足度の高さを主観的に
評価して比較した.
実験要領は以下の通りである.
\begin{itemize}
\item 被験者\\
翻訳者 3 名.このうち 1 名はニュース翻訳の経験が豊富である.残りの 2 名はニュー
ス翻訳を直接担当しているわけではないが翻訳の経験は豊富である.いずれの被験者
も検索手法については知らされていない.
\item 入力表現\\
現実の入力を想定して 58 の日本語表現を作成した.具体的には先の実験で
使用した 500 文から文をランダムに抽出し,それらの一部を切り出して作成した.
すなわちカットアンドペーストで入力することを想定した.
長さの平均は 22.6 文字,最大の入力は 62 文字,最小の入力は 14 文字である.
\item 出力\\
一つの入力に対して各 3 手法の上位 5 個の日本語検索結果を印刷して提示し
た.この時,各手法の検索結果の提示順序を 1 つの入力ごとに変更して,検
索手法と結果の対応がわからなくなるようにした.
また照合したキーワードを $<>$ で囲んで表示した.被験者は最初のキーワードか
ら最後のキーワードが出現した区間を中心に評価した.これはキーワードで検
索した部分以外に偶然類似した部分があった場合にこれを評価しないためである.
\item 評価手法\\
 検索した部分の正しい英訳があると想定したときの満足度を次の 2 種類で評
価した.
 \begin{itemize}
  \item 相対評価\\
   3 手法すべての結果を見て,最も満足度の高い結果を 5 点として,以下この
結果との満足度の差を下記の要領で評価した.
   \begin{center}
   \begin{tabular}{llr}
        全く差がない &  & 5\\
        わずかに差がある & (わずかに劣る) & 4\\
        差がある & (劣る) & 3\\
        かなり差がある & (かなり劣る) & 2\\
        非常に差がある & (非常に劣る) & 1\\
   \end{tabular}
   \end{center}
  \item 絶対評価\\
  翻訳を行うときに英訳があればどの程度役に立つかを以下の 5 段階で絶対
評価した.
   \begin{center}
   \begin{tabular}{llr}
        非常に役に立つ & (非常に良い) & 5\\
        かなり役に立つ & (良い) & 4\\
        まあ役に立つ & (ふつう) & 3\\
        あまり役に立たない & (悪い) & 2\\
        全然役に立たない & (非常に悪い) & 1\\
   \end{tabular}
   \end{center}
 \end{itemize}
\end{itemize}
評価シートの実例を付録に示す.
3 名の評価結果の平均値を表~\ref{tab:soutai} と~\ref{tab:zettai} に示す.
\begin{table}
 \begin{center}
  \caption{相対評価結果}
  \label{tab:soutai}
  \begin{tabular}{c|ccc}\hline\hline
   被験者  &  AND & AND+W & AND+W+D\\\hline
   A & $2.88$ & $2.94$ & $3.60$\\ 
   B & $2.99$ & $3.07$ & $3.90$\\ 
   C & $3.07$ & $3.21$ & $3.80$\\\hline
   平均 & $2.98$ & $3.08$ & $3.77$\\\hline 
  \end{tabular}
 \end{center}
\end{table}
\begin{table}
 \begin{center}
  \caption{絶対評価結果}
  \label{tab:zettai}
  \begin{tabular}{c|ccc}\hline\hline
   被験者  &  AND & AND+W & AND+W+D\\\hline
   A & $2.78$ & $2.81$ & $3.13$\\ 
   B & $2.84$ & $2.98$ & $3.44$\\ 
   C & $3.02$ & $3.10$ & $3.44$\\\hline
   平均 & $2.88$ & $2.96$ & $3.34$\\\hline 
  \end{tabular}
 \end{center}
\end{table}

また各手法の違いを検定するため平均値の差の $t$--検定を行った.
手法間の $t$ --値を表~\ref{tab:soutaiT} と \ref{tab:zettaiT} に
示す
\footnote{二つの母分散が等しい場合とそうでない場合で計算したところ表中の桁
数で値は変わらなかった.}
.
\begin{table}
\vspace*{-4mm}
 \begin{center}
  \caption{相対評価の $t$--値}
  \label{tab:soutaiT}
  \begin{tabular}{c|rr}\hline\hline
& AND & AND+W \\\hline
AND+W & 1.55 & \\
AND+W+D & 13.28 & 11.77 \\\hline
  \end{tabular}
 \end{center}
\bigskip
 \begin{center}
  \caption{絶対評価の $t$--値}
  \label{tab:zettaiT}
  \begin{tabular}{c|rr}\hline\hline
& AND & AND+W \\\hline
AND+W & 1.44 & \\
AND+W+D & 8.73 & 7.23 \\\hline
  \end{tabular}
 \end{center}
\end{table}
$t_{0.025}(\infty) = 1.960$ であるからどちらの表でも AND+W+D と AND 
,AND+W+D と AND+W の組み合わせでは有意水準 5\% で帰無仮説(平均値に差がない)を棄却できる
\footnote{$t_{0.0005}(\infty) = 3.291$ であり 0.1\% であっても棄却でき
る.}
.
一方, AND と AND+W では 5\% 有意水準の棄却はできない.以上の結果より
次のことが結論できる.
\begin{itemize}
 \item AND 検索と AND 検索に語順制約を加えた AND+W 検索では AND+W 検索
の方が満足度が高くなる傾向は認められたが,統計的に有意な差はなかった.
 \item AND+W+D 検索の満足度は AND 検索と AND+W 検索のいずれの満足度よ
り高くなった.またこれは統計的に有意な差であった.
\end{itemize}
以上より提案手法の精度が最も高かったと結論できる.
\vspace*{-1.5mm}
\subsection{議論}
\vspace*{-0.5mm}
本章の実験結果のまとめを示す.
\begin{itemize}
 \item 検索時間は AND 検索が有利であったが語順と変位を加えても  1.3 倍
程度の時間増であった.
 \item 語順制約を追加することで検索結果を絞り込む効果は確認できた.し
かし,短い入力に対する絞り込み効果は十分ではなかった.
 \item 変位の制約を加えると検索結果に順位付けができる.この効果を計
測したところ有効性を有意に検出できた.
\end{itemize}

今回の実験から,カットアンドペースト方式の入力では AND+W+D 検索と AND 検索を
使うのが良いと著者らは考えている.AND+W+D 検索は最も満足度が高かったからであ
る.
AND 検索は今回の満足度の実験の結果は一番低かった.しかし,速度の面
は最も優れている.また入力の表現が長い部分で
は検索結果の数も問題にならない.そこで長い表現を検索する場合に AND+W+D 
検索を補完する意味で利用する価値があると考える.

一方 AND+W 検索と AND+W+D 検索の両方を使う必要性は小さいと言え
る.両者は基本的に同じ検索結果となるからである.ただし,これは本稿のように入
力表現を原文のカットアンドペーストで作成する場合に限る.
利用者が直接キーワード列を入力する場合には変位の情報を使えない.このため語順だ
けを使った AND+W 検索を使う必要も生じる
\footnote{著者らのシステムにもキーワードを直接
入力する機能があり,この場合には AND と AND+W 検索を利用している.}
.
\section{関連研究}
\label{sec:kanren}
類似用例提示型翻訳支援システムの提案はこれまで多くなされて
いる
\cite{Naka89,Sumi91,Tera92,Sato93,Take94,Hyou94,Kitamu96,Aoya95}
.
ここではこのようなシステムの中で著者らの検索の研究と近い研究について比
較を行う.

中村\cite{Naka89}の研究は著者らの研究の出発点になったものである.この
論文は用例
検索による翻訳支援の考え方と構成を示している.
中村はこの論文で入力表現と用例が共有する自立語の数に基づいて類似性を
計算する手法を提案している.また,検索結果の順位を次の 3 つの条件
で整列している:1)構成語
(本研究のキーワード)がその他の語をはさまない,2)用例中の自立語の個数に
対する構成語の数の比率,3)含んでいる構成語の数.
このシステムを使った小規模な評価実験では長い(複数文節)表現を入力した
場合に検索結果が「あいまい」となって被験者の評価が低くなったと報告してい
る.
これは検索結果の効果的な絞り込みの必要性を示唆した結果である.著者らはこ
れに対して語順と変位を考慮した検索を提案しその有効性を確認した.

隅田ら\cite{Sumi91}は表現辞典の用例文を検索する翻訳支援システムを提案し
ている.著者らとの主要な違いは 2 点ある.1)用例文はニュースの記事に比べて
短く入力も単文に近い用例を想定している点,2)検索では構文的な類似性を
重視している点である.類似検索は入力の自立語を順次無視して最後に付属語
列のパターンまで検索条件を緩和する手法を採用している.ここで語順は考慮
していない.

隅田らは構文情報の把握に助詞を利用している.これは短い用例を対象にした
場合に有効であるが,著者らのように長い用例を扱う場合には不適切である.な
ぜなら長い用例では表層の助詞だけで主要な構文構造を把握することができな
いからである.
また,助詞は極めて多くの場所に出現するためこれを使う処理は遅くなる問題
もある.実際著者らのシステムで助詞をキーワードに含めて検索実験を行ったところ
自立語だけを対象にする場合の約 23 倍の時間がかかることが分かった.

佐藤\cite{Sato93}は文字を連続して多く共有する 2 つの文を近いと考
えた「最適照合検索」を提案している.この論文では 1 文字を照合単位としかつ順
序を考慮した照合手法(CTM1)と 2 文字と 3 文字を照合単位とし,順序を無視
した照合手法(CTM2)を提案している.またどちらの手法も文字列が連続して
出現することを類似性の条件に含めている.

文字と単語の違いを無視すれば,著者
らの語順と変位を考えた手法が CTM1 に,語順を無視した手法が CTM2 に対応
する.またこの二つの手法の検索結果の違いを次のような実験で検討している.
1)100 個の入力をそれぞれ検索し,上位 5 つの類似表現を得る,2)その英
訳の中で最良の英訳の有用性を 4 段階で主観評価する
\footnote{利用者は一つの有用な対訳が得られれば十分という考えによる.一
方,著者らは多様な対訳の検索を重要視している.}
.

実験結果よると二つの手法の有用性は同等もしくは CTM2 の方が若干良かった
となっている.著者らの結果と比べると出現順序を考えない CTM2 の結果が
良いのは意外である.原因は用例データベースの違い,入力表現の違い,評価
法の違いがあるため断定できないが,CTM2 の文字列の連続性の条件が貢献し
ている可能性がある.この条件が著者らの変位と対応したと考えられる.著者
らの実験結果でも語順だけでは効果が薄く変位が有効であった点を考え
ると両者の実験結果に矛盾はない.
\section{おわりに}
\label{sec:ketsuron}
翻訳支援を目的とした類似文検索手法を提案してその有効
性を実験で確認した.提案したのは入力と用例のキーワードの共有数,語順,
その変位を類似性の基準とした検索手法である.
提案手法は入力文の形態素解析以外は表層文字列の一致を使うため,高速かつ
頑健という利点を持つ.

今後の課題を述べる.現在はすべての自立語を同等の重要性を持ったキーワードとして
検索を実行している.しかし,利用者が知りたい表現が特定のキーワードを含
む場合があろう.また,動詞を含んだ表現で条件緩和をする場合に,一般的に動詞は
削除しない方が良いと考えられる.このような特別な条件や経験則を現在
の処理に追加するのは今後の課題である.このような変更は内容が明らかにな
れば現在の枠組みで簡単に実現できると考えている.

現在,本検索部分を含んだ日英翻訳支援システムをニュースの翻訳現場で実際
に使用し始めている.
そこで検索条件の改良は実際の利用者の
意見を取り入れて進めていきたいと考えている.
また今回は日本語の検索部分だけの評価を行ったが,実際の英語の出力を使っ
てシステム全体の評価も行う予定である.これについては別途報告したい.

今回は日本語を対象にした検索システムを報告した.今回の内容は言語に依存
した部分がほとんどないためその他の言語への応用も簡単である.すでに英語
の検索部分を作成しており,その有効性を調査したいと考えている.さらにそ
の他の言語への適用可能性も検討したい.

\vspace{-2mm}
\bigskip\medskip
\noindent
{\Large\bf 付録\quad 評価シートの例}
\bigskip

\vspace{-2mm}
\label{app:sheet}
\baselineskip=0.98\normalbaselineskip
この場合は AND+W,AND+W+D,AND の順に検索結果を表示している.評価文の
最初の括弧付の数字が相対評価,2 番目の数字が絶対評価の値を示す.


++++++++++ 入力文 ++++++++++++++

IN 事故のあった施設の中を調査しました.

++++++++++ キーワード ++++++++++

KW $<$事故$>$の$<$あっ$>$た$<$施設$>$の中を$<$調査し$>$ました.

++++++++++ 検索文 ++++++++++++++

SR 2-1 (1) (2)  この研究グループは,車に携帯電話を備えていて,軽い物損$<$事故$>$を起こ\break
したことの$<$ある$>$ドライバー六百九十九人について,事故のデータと通話記録を$<$調査し$>$ま\break
した. 

SR 2-2 (1) (2)  公海上でおきた今回の$<$事故$>$の原因調査は国際条約で船籍の$<$ある$>$ロシア\break
が行うことになりますが事故の原因を特定するためには船首部分の破断面などを詳しく分析す\break
る必要があるため,運輸省では引き続き外交ルートを通して共同で$<$調査す$>$ることをロシア側\break
に求めていくことにしています. 

SR 2-3 (1) (2)  埼玉県内では小山代表が理事長を務める特別養護老人ホームが$<$ある$>$北本市\break
でもきょう臨時の市議会が開かれ,$<$施設$>$建設に至る経緯や運営について$<$調査す$>$る特別委\break
員会を設けたほか,現在同じ社会福祉グループの特別養護老人ホームの建設が進んでいる上福岡市も,市役所の中に対策委員会を設置して補助金の使い途などについて調査を始めています. 

SR 2-4 (1) (2)  埼玉県内では小山容疑者が理事長を務める特別養護老人ホームが$<$ある$>$北\break
本市でもきょう臨時の市議会が開かれ,$<$施設$>$建設に至る経緯や運営について$<$調査す$>$る特\break
別委員会を設けたほか,現在同じ社会福祉グループの特別養護老人ホームの建設が進んでいる\break
上福岡市も,市役所の中に対策委員会を設置して補助金の使い途などについて調査を始めてい\break
ます. 

SR 2-5 (1) (1)  去年,山形県でジェット機の低空飛行が原因で女性が馬から落ちて怪我をし\break
た$<$事故$>$について在日アメリカ軍はこのジェット機がアメリカ軍機で$<$ある$>$ことを認め被害\break
者に賠償の支払いに応じる意向を防衛$<$施設$>$庁に伝えてきていたことが分かりました. 

++++++++++ 検索文 ++++++++++++++

SR 2-1 (5) (4)  東京電力によりますと$<$事故$>$が$<$あっ$>$た$<$施設$>$は定期点検中でタービン\break
などを分解して組み直し再び発電を始めるための試運転中に事故が起きたということです. 

SR 2-2 (3) (3)  警察庁は今年六月に全国で起きた死者やケガ人のでたおよそ六万二千件の交\break
通事故を対象に携帯電話が原因とみられる$<$事故$>$がどの程度$<$ある$>$のか初めて$<$調査し$>$ま\break
した. 

SR 2-3 (3) (3)  警察庁は今年六月に全国で起きた死者やケガ人のでたおよそ六万二千件の交\break
通事故を対象に携帯電話が原因とみられる$<$事故$>$がどの程度$<$ある$>$のか初めて$<$調査し$>$ま\break
した. 

SR 2-4 (1) (1)  このため,ノースウエスト航空では問題のエンジンをアメリカのミネアポリ\break
スに$<$ある$>$本社の整備$<$施設$>$に運んで詳しく$<$調査す$>$る事にしたもので,きょう,機体か\break
ら問題のエンジンを取り外して新しいエンジンと取り替え,このエンジンをきょうにもミネア\break
ポリスに向け送る事にしています. 

SR 2-5 (1) (1)  それによりますと,去年の六月二十四日,ワシントン州のフェアチャイルド\break
空軍基地の上空で航空ショーのリハーサル飛行をしていたBー五十二戦略爆撃機が墜落し乗員\break
四人全員が死亡した$<$\hspace{-0.2pt}事故\hspace{-0.2pt}$>$で,
墜落地点は基地に$<$\hspace{-0.2pt}ある\hspace{-0.2pt}$>$\mbox{核兵器の貯蔵$<$\hspace{-0.2pt}施設\hspace{-0.2pt}$>$のすぐ近くで,}
距離はわずか十五メートルしか離れていなかったいうことです. 

++++++++++ 検索文 ++++++++++++++

SR 2-1 (1) (1)  このうち,原子力の問題に取り組んでいる原子力資料情報室は,地球規模で\break
環境に影響を及ぼす恐れの$<$ある$>$原子力発電所の$<$事故$>$について,去年十二月の高速増殖炉\break
「もんじゅ」のナトリウム漏れ$<$事故$>$を例に挙げて▼地元への$<$事故$>$の報告を義務付けるこ\break
とや▼原子力$<$施設$>$での$<$事故$>$の原因を$<$調査す$>$る第三者機関を設けること,それに▼民\break
間の調査に対しても情報を公開することなどを盛り込んだ
「原子力$<$施設$>$
$<$事故$>$対応法」の\break
制定を提案しました. 

SR 2-2 (1) (1)  沖縄県に$<$ある$>$アメリカ軍基地の排水管から高い濃度の有害物質PCBが\break
検出された問題できょう基地をかかえる沖縄県内の自治体の代表らが防衛$<$施設$>$庁を訪れ,ア\break
メリカ軍による相次ぐ事件や$<$事故$>$の防止に全力をあげるよう要請しました. 

SR 2-3 (1) (1)  NHKはこの問題の実態をつかむため保健所を持ち処理$<$施設$>$に許認可権\break
が$<$ある$>$各都道府県や市あわせて八十二の自治体を対象にアンケート方式で$<$調査し$>$すべて\break
の自治体から回答を得ました. 

SR 2-4 (1) (1)  NHKはこの問題の実態をつかむため保健所を持ち処理$<$施設$>$に許認可権\break
が$<$ある$>$各都道府県や市あわせて八十二を対象にアンケート方式で$<$調査し$>$すべての自治体\break
から回答を得ました. 

SR 2-5 (1) (1)  これに対して池田外務大臣は「在日アメリカ軍は今回の$<$事故$>$で使用され\break
たのと同様の劣化ウランを含む砲弾を日本国内の一部の$<$施設$>$に所蔵しているがこれは日本が\break
攻撃を受けるなど緊急事態が発生した場合には使用する必要が$<$ある$>$ものでそうした意味で撤\break
去を求めるのは適当ではない」と述べ,アメリカ軍に対して,日本国内に所蔵している同種の\break
砲弾の撤去は求めないという考えを示しました. 
\baselineskip=\normalbaselineskip
\bigskip
\acknowledgment
本研究は第一著者が NHK 放送技術研究所勤務中に行った研究をまとめたもの
です.本論文をまとめる機会を与えていただいた ATR 音声翻訳通信研究所
の山本誠一社長と横尾昭男室長に感謝いたします.また,プログラムの作成と実験
に協力していただいた株式会社 KIS の松田伸洋氏に感謝いたします.


\bibliographystyle{jnlpbbl}
\bibliography{v06n5_05}

\bigskip

\begin{biography}
\biotitle{略歴}
\bioauthor{田中 英輝}{
1982 年九州大学工学部電子工学科卒業.
1984 年同大学院修士課程修了.
同年,日本放送協会に入局.
1987 年同放送技術研究所勤務.
1997 年 ATR 音声翻訳通信研究所勤務.
現在第 4 研究室主任研究員.
機械翻訳,機械学習,情報検索の研究に従事.
工学博士.情報処理学会,人工知能学会各会員.
}
\bioauthor{熊野 正}{
1993 年東京工業大学工学部情報工学科卒業.1995 年同学理工学研究科情報工
学専攻修士課程修了.
同年,日本放送協会に入局.放送技術研究所勤務.自然言語処理,人工知能の研究に従事.
情報処理学会,人工知能学会各会員.
}
\bioauthor{浦谷 則好}{
1975 年東京大学大学院修士課程(電気工学)修了.同年,日本放送協会に入局.
1979 年同放送技術研究所勤務.1991 年より 3 年間 ATR 自動翻訳電話研究所ならびに
音声翻訳通信研究所に勤務.現在,NHK放送技術研究所ヒューマンサイエンス
主任研究員.
情報検索,自然言語処理の研究に従事.工学博士.情報処理学会,電子情報通信学会,
映像情報メディア学会各会員.
}
\bioauthor{江原 暉将}{
1967 年早稲田大学第一理工学部電気通信学科卒業.同年,日本放送協会に
入局.1970 年より放送技術研究所に勤務.現在,ヒューマンサイエンスグ
ループ主任研究員.かな漢字変換,機械翻訳,音声認識などの研究に従事.工
学博士.本会評議委員.情報処理学会,機械翻訳協会,電子情報通信学会,映
像情報メディア学会各会員.
}

\bioreceived{受付}
\bioaccepted{採録}

\end{biography}

\end{document}
