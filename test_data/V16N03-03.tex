    \documentclass[japanese]{jnlp_1.4}
\usepackage{jnlpbbl_1.1}
\usepackage[dvips]{graphicx}
\usepackage{amsmath}
\usepackage{hangcaption_jnlp}
\usepackage{udline}
\setulminsep{1.2ex}{0.2ex}
\let\underline

\newcommand{\KetujiX}[1]{}

\Volume{16}
\Number{3}
\Month{July}
\Year{2009}

\received{2008}{12}{12}
\revised{2009}{1}{28}
\accepted{2009}{2}{24}

\setcounter{page}{51}

    \usepackage{covington_re}


\jtitle{日本語LFGにもとづく助数詞の処理 }
\jauthor{大熊 智子\affiref{FX} \and 梅基  宏\affiref{FX} \and 三浦 康秀\affiref{FX} \and 増市  博\affiref{FX}}
\jabstract{
事物の数量的側面を表現するとき,数詞の後に連接する語を一般に助数詞と呼
ぶ.英語などでは名詞に直接数詞が係って名詞の数が表現されるが,日本語で
は数詞だけでなく助数詞も併せて用いなければならない.名詞と助数詞の関
係を正しく解析するためには,助数詞が本来持つ語彙としての性質と構文中に
現れる際の文法的な性質について考慮する必要がある.本稿では,数詞と助数
詞の構文を解析するためのLexical-Functional Grammar (LFG) の語彙規則と
文法規則を提案し,その規則の妥当性と解析能力について検証した.提案した
規則によって導出される解析結果 (f-structure) と英語,中国語の
f-structureをそれぞれ比較することによって,日本語内での整合性と多言語
間との整合性を有していることが確認できた.また,精度評価実験の結果,従
来のLFG規則に比べて通貨・単位に関する表現では25\%,数量に関する表現で
は5\%,順序に関する表現では21\%のF値の向上が認められた.
}
\jkeywords{助数詞,語彙機能文法,文法規則,構文解析}

\etitle{Treatment of Numerical Classifier Based on Lexical-Functional Grammar}
\eauthor{Tomoko Ohkuma\affiref{FX} \and Hiroshi Umemoto\affiref{FX} \and Yasuhide Miura\affiref{FX} \and Hiroshi Masuichi\affiref{FX}} 
\eabstract{
Japanese numeral classifier (NC) expresses quantificational
expressions by following number nouns. The parser based on
phrase structure grammar like Lexical-Functional Grammar (LFG) or
Head-driven Phrase Structure Grammar (HPSG) has to define the
relationships between noun and NC or number and NC by the grammatical
rules. NC has various relationships syntactically because of its'
various characteristics.  The treatment of NCs in LFG formalism should
take account of their lexical and syntactical characteristics. This
report proposed LFG rules for NCs and examined their validity and
accuracy.  The comparison between Japanese f-structure inducted by the
rules proposed in this report and English f-structure and Chinese
f-structure which correspond to the Japanese f-structures show that
the rules make f-structures parallel between same phrases in Japanese
and in other languages. The experiment of evaluations shows the rules
improve F-score of NCs for currency and unit from 52.10\% to 77.90\%
and F-score of NCs for quantity from 81.92\% to 87.32\% and F-score of
NCs for Order from 59.62\% to 81.26\%.
}
\ekeywords{Numerical Classifier, Lexical-Functional Grammar, grammar rules, syntactic parsing}

\headauthor{大熊,梅基,三浦,増市}
\headtitle{日本語LFGにもとづく助数詞の処理}

\affilabel{FX}{富士ゼロックス(株)研究技術開発本部システム要素技術研究所}{Research and Technology Group, System Technology Laboratory, Fuji Xerox Co., Ltd.}


\begin{document}
\maketitle


\section{はじめに}

\subsection{背景\label{haikei}}

事物の数量的側面を表現するとき,「三人」,「5個」,「八つ」のように,
「人」,「個」,「つ」という付属語を数詞の後に連接する.これらの語を一
般に助数詞と呼ぶ.英語などでは``3 students'',``5 oranges'' のように名詞に直接
数詞が係って名詞の数が表現されるが,日本語では「3人の学生」,「みかん
五個」のように数詞だけでなく助数詞も併せて用いなければならない.

形態的には助数詞はすべて自律的な名詞である数詞に付属する接尾語とされる.
しかし,助数詞の性質は多様であり,一律に扱ってしまうことは統語意味的見
地からも計算機による処理においても問題がある.また構文中の出現位置や統
語構造によって,連接する数詞との関係は異なる.つまり,数詞と助数詞の関
係を正しく解析するためには1)助数詞が本来持つ語彙としての性質,そして2)
構文中に現れる際の文法的な性質について考慮する必要がある.

KNP~\cite{Kurohashi}やcabocha~\cite{cabocha}などを代表とする文節単位の
係り受け解析では,上記のような数詞と助数詞の関係は同じ文節内に含まれる
ため,両者の関係は係り受け解析の対象にならない.ところが,単なる係り受
け以上の解析,例えばLexical Functional Grammar(以下,LFG)やHead-driven Phrase
Structure Grammar(以下,HPSG)のような句構造文法による解析では,主辞の
文法的役割を規程する必要がある.つまり文節よりも細かい単位を対象に解析
を行うため,名詞と助数詞の関係や数詞と助数詞の関係をきちんと定義しなけ
ればならない.

上記のような解析システムだけでなく,解析結果を用いた応用アプリケーショ
ンにおいても助数詞の処理は重要である.\cite{UmemotoNL}で紹介されている
検索システムにおける含意関係の判定では数量,価格,順番などを正しく扱う
ことが必要とされる.






\subsection{\label{mokuteki}本研究の目的}
本稿では,数詞と助数詞によって表現される構文\footnote{但し,「3年」,
「17時」など日付や時間に関する表現は\cite{Bender}と同様にこの対象範囲
から除く.}を解析するLFGの語彙規則と文法規則を提案し,計算機上で実装す
ることによってその規則の妥当性と解析能力について検証する.これらのLFG
規則によって出力された解析結果 (f-structure) の妥当性については,下記
の二つの基準を設ける.

\begin{enumerate}
\item{他表現との整合}\\
統語的に同一の構造を持つ別の表現と比較して,
f-structureが同じ構造になっている.
\item{他言語との整合}\\
他の言語において同じ表現のf-structureが同じ
構造になっている.
\end{enumerate}

\ref{senkou}章では助数詞に関する従来研究を概観し,特に関連のある研究と
本稿の差異について述べる.\ref{rule}章では助数詞のためのLFG語彙規則と
助数詞や数詞を解析するためのLFG文法規則を提案する.\ref{fstr}章では
\ref{rule}章で提案したLFG規則を\cite{Masuichi2003}の日本語LFGシステム
上で実装し,システムによって出力されるf-structureの妥当性を上記の二つ
の基準に照らして検証する.日本語と同様にベトナム語や韓国にも日本語のそ
れとは違う性質をもった固有の数詞と助数詞が存在する\cite{yazaki}.また,
日本語の助数詞は一部の語源が中国語にあるという説もあり,その共通性と差
異が\cite{watanabe}などで論じられている.そこで,Parallel Grammar
Project \cite{Butt02}(以下,ParGram)においてLFG文法を研究開発している
中国語LFG文法\cite{ji}で導出されたf-structureを対象にして,基準2を満
たしているかを確認するために比較を行う.

``3~kg'' の`kg'や,``10 dollars'' の`dollar'など,英語にも数字の後に連接す
る日本語の助数詞相当の語が存在する.また,日本語においても英語のように
助数詞なしに数詞が直接連接して名詞の数量を表現する場合もある.ParGram
において英語は最初に開発されたLFG文法であり,その性能は極めて高い
\cite{Riezler}.ParGramに参加する他の言語は必ず英語のf-structureとの
比較を行いながら研究を進める.以上のことから,中国語だけではなく
\cite{Riezler}の英語LFGシステムで出力されたf-structureとの比較を行う.
\ref{hyouka}章では精度評価実験を行って,解析性能を検証する.数詞と助数
詞によって形成される統語をLFG理論の枠組みで解析し,適切なf-structureを
得ることが本研究の目的である.


\section{\label{senkou}先行研究}
国語学や言語学の分野において,助数詞に関する研究は古くから行われてきた.
また,文法的な性質だけでなく,助数詞や助数詞の数える対象である名詞との
関係に着目して,意味的な分類を行う試みもある.さらに,自然言語処理の分
野では助数詞の特徴を考慮して処理を行う解析手法が提案されている.本章で
は助数詞の文法に関する研究,助数詞の分類に関する研究,助数詞を対象にし
た自然言語処理についてそれぞれ概観する.


\subsection{助数詞の文法に関する研究}

日本語助数詞の代表的な研究に\cite{kenbo}が挙げられる.形態的には「最も
単純な品詞」とされてきた助数詞であるが用法の点では助数詞の振る舞いが複
雑多様であることを指摘した上で,その個別的な用法を用例に即して紹介して
いる.

\cite{kenbo}では世の中には生活の必要に応じていろいろな物の数え方があり,
生活の新しい場面,新しい話題に即して新しい助数詞が生まれると述べられて
いる.その一方で明治の始まる前後までは助数詞として用いられてきたが,現
在はその機能が失われている語の例が指摘されている.このような「助数詞の
用いられ方の変化」に着目し,\cite{ogino}は大規模なアンケートを実施して,助数
詞の実際の用いられ方について調査を行った.さまざまな年代,さまざまな職
種の男女半数づつ計425人に対して行われたアンケート結果から年齢層別の助
数詞の使われ方の差異を明らかにしている.さらに\cite{tanihara}では
\cite{ogino}の調査結果に基づき,5つの助数詞の典型的な用法について考察
している.

助数詞を数詞に連接する付属語ではなく数詞の語尾とする研究もある
\cite{morishige}.しかし,本稿では二つの理由からこの立場は採用しない.
一つは,助数詞に関する様々な現象を説明するのに,語尾という考え方は不適
切であると考えるからである.\cite{kenbo}で指摘されているような多様性や
可変性を説明するのに,数詞と助数詞は独立した語であると考えるのが自然で
ある.もう一つは,LFG規則を実装する日本語LFGシステムにおいて,形態素解
析に茶筌\cite{chaurl}を用いているためである.茶筌では数詞と助数詞
を明確に分けている\footnote{但し,「一つ」,「二つ」のよう
な数詞と助数詞「つ」,及び「一筆」,「四方八方」のような名詞として定着
している数詞と助数詞については,まとめて一つの名詞としている.}.

本稿で扱う対象ではないが,日本語数量詞の大きな研究課題の一つに数量詞遊
離構文の問題がある\cite{okabe}.遊離数量詞とは(a)「3個のりんごを食べた」
(b)「りんごを3個食べた」の(b)のような構文である.この構文をめぐって,
(a)の連体数量詞文との置換の可否や,その決定要因,また(a),(b)の表現の
意味やニュアンスの違いについて様々な先行研究が分析や解釈を行っている.
これらの課題を解決するためには,文法的な関係というよりはむしろ数量表現
とそれが指す名詞や動詞などの意味関係に着目しなければならない.

\subsection{助数詞の分類に関する研究}

\cite{Yo93}はプロタイプ理論に基づいて,認知意味論的な立場から,助数詞
の分類を試みている. \cite{Iida}は,助数詞ならびに助数詞と同じ働きをす
る名詞600語に対し,数えられる対象や同じ対象を数える助数詞群の共通点や
差異について解説を行っている. \cite{Sirai}は\cite{Iida}に記載されてい
る助数詞cとそれが数える名詞nを書き起こしたデータから一つのcが二つ以上
のnとペアになっているもの計9,352組を用いて,自動的に助数詞オントロジー
を構築する手法について提案している.

\subsection{助数詞を対象とした自然言語処理に関する研究}

 \cite{Bond}は翻訳などのアプリケーションに必要となる日本語文生成の際に,
 オントロジーを用いて名詞の意味的な情報を参照し助数詞を選択する手法に
 ついて提案している.英語でも``ten feet long'' の`feet'や``two shelves
 higher'' の`shelves'のように日本語の助数詞に似た数量表現が存在する.
 \cite{Dan}はこのような表現を対象にしてHPSGシステム上で解析を行う手法
 を提案している.

上記と同様にHPSGに関する研究ではあるが,本稿に最も関連のある先行研究は
\cite{Bender}である.\cite{Bender}は日本語HPSG文法において助数詞を解析
し,適切な解析結果,すなわちMinimal Recursion Semantics(以下,MRS)を出
力する手法について提案している.本研究も基本的な目的は同様であるが,次
の点が異なる.まず,\cite{Bender}では網羅的な助数詞の処理を提案してい
ながら,MRSについての考察は主に対象となる名詞を数え上げるための助数詞
に限定されていることである.本稿では,その他の助数詞についても考察を行
う.さらに,助数詞が省略されていてもその省略の有無に関係なく同一の
f-structureを導出する文法や助数詞の性質を変化させる接頭辞や接尾語を考
慮した文法について提案する.また,二つの基準を設定してf-strucrtureの妥
当性を検討することも本稿の特徴である.さらに\cite{Bender}には,提案さ
れている語彙規則(type hierarchy)に助数詞をどのように割り当てるかの具
体的な記述がない.本稿では,LFGの語彙規則(lexical category)に語を割
り当てる方法についても言及する.また,本稿では,今回提案した文法がどの
くらい網羅的に実際のテキストを処理できるかを検証するための精度評価実験
を行う.


\section{\label{rule}助数詞のための日本語LFG規則}

\subsection{\label{rule-intro}LFG規則の表記法について}
提案規則の詳細について述べる前に,この節ではLFG規則の表記方法について
説明する.

LFG規則は語彙規則と文法規則によって定義される.語彙規則では,f-structureの素性に対応するSUBJ(ect)やOBJ(ect)などの引数 (argument) やlexical category,属性と属性値などを定義する.
(\ref{lex-intro})に語彙規則の例を示す.1行目は動詞「読む」にVという
lexical categoryが割り当てられており,PRED(icate)「読む」はargumentに
SUBJとOBJという二つの引数を持つことが記述されている.2行目には
名詞「太郎」のlexical categoryがNでありPREDが「太郎」であることが記述
されている.3行目には名詞「本」のlexical categoryがNでありPREDが「本」
であることが記述されている.4行目には格助詞「が」のlexical categoryが
PPであり,主格を示す属性NOM(inative)とその値 `+'を持つことが記述されて
いる.5行目には格助詞「を」のlexical categoryがPPであり,目的格を示す
属性ACC(usative)とその値`+'を持つことが記述されている.
\begin{example} \label{lex-intro}
\begin{tabular}[t]{lccl}
読む \quad\quad& V & & PRED=`読む$\langle$($\uparrow$SUBJ)($\uparrow$OBJ)$\rangle$'\\ 
太郎 \quad\quad& N & & PRED=`太郎'\\ 
本\quad\quad& N & & PRED=`本'\\ 
が\quad\quad& PP & & ($\uparrow$NOM)=+\\ 
を\quad\quad& PP & & ($\uparrow$ACC)=+\\ 
\end{tabular}
\end{example}
\begin{example}
\label{grammar-intro}
\begin{tabular}[t]{lccc}
S $\longrightarrow$&NP\* &&V\\
&\{($\uparrow$SUBJ)=$\downarrow$\space($\uparrow$NOM)=$_c$ + $|$ ($\uparrow$OBJ)=$\downarrow$\space($\uparrow$ACC)=$_c$ +\}&&$\uparrow$=$\downarrow$ \\[2ex]
NP $\longrightarrow$&N\* &&PP\\
&$\uparrow$=$\downarrow$&&\\[2ex]
\end{tabular}
\end{example}

\begin{figure}[b]
\begin{center}
\includegraphics{16-3ia3f1.eps}
\end{center}
\caption{LFG規則の適用例}
\label{lfg-intro}
\end{figure}

文法規則は,句構造規則とそれに付与される機能的注釈で表現される.注釈に
おいて用いられる`$\uparrow$'という記号は句構造の一つ上の節点に存在する
f-strucutreを,`$\downarrow$'は現在の節点のf-structureを
指す.(\ref{grammar-intro})に(\ref{lex-intro})の語彙規則を含む文法規則
を示す.1行目の句構造規則では,0個以上のNPとVがSになることを定義してい
る.Vに付与されている注釈はVのPREDが一つ上,すなわち文全体のPREDである
ことを定義している.NPに付与されている注釈はNOM属性の値が`+'であれば
NPの値が一つ上のSUBJに,ACC属性が`+'であればOBJになることを定義してい
る.2行目の句構造規則では,NとPPがNPになることを定義している.Nに付与
されている注釈はPREDが一つ上の節点であるNPのPREDであることを定義している.
(\ref{lex-intro})の語彙規則と(\ref{grammar-intro})の文法規則を「太郎が
本を読む」の解析に適用すると,
\pagebreak
図\ref{lfg-intro}に示すように句構造規則
によってc-structureの木構造が形成されると同時に,機能的注釈によって属
性と属性値で表現されるf-structureが生成される.


\subsection{\label{lex}助数詞のための語彙規則}

表\ref{lexical-table}に助数詞の語彙カテゴリの一覧を示す.本稿では助数
詞が数詞に持たせる数の性質(基数/序数)という側面から「数量」,「順番」
という二つの語彙カテゴリを設定する.また,\cite{kokugojiten}では単位名
を表す「メートル」,「円」などを助数詞に含めないとする説があることを紹介し
ているが,文法的には同じ働きをすることが指摘されている.本稿ではこの立
場から,これらの語も助数詞とする.ただし,上述の二つのカテゴリとは別に
「単位」,「通貨」という項目を設け,さらに数詞との位置関係から通貨を細
分類した.それぞれの語彙カテゴリについて以下に述べる.

\begin{table}[b]
\caption{助数詞の語彙カテゴリ一覧}
\label{lexical-table}
\input{03table01.txt}
\end{table}

単位に関する語彙カテゴリは,計測する対象の計測基準量を表す助数詞に割り
当てられる.日本語LFGシステムにおいてこの語彙カテゴリにはEDR概念辞書
\cite{edr}の接尾語のうち,単位に分類されている語から「ヶ月」,「週目」
などの時間に関する語を除いた計659語を採用する.助数詞の中には単独で名
詞として働く語も存在する.「身長を\underline{\mbox{cm}}で,体重は
\underline{\mbox{Kg}}で入力して下さい」や「対\underline{ユーロ}で\underline
{ドル}が下落している」など,単位や通貨も数詞を伴わずに単独で名詞として
用いられる場合がある.従って,このカテゴリの語には単に助数詞としての語
彙ルールだけではなく,名詞としてのカテゴリも割り当てる.さらに,助数詞
として働く場合でも名詞の属性であるNTYPEを付与する.これは,後述する英
語のf-structureと平行にするためでもある.

通貨に関する語彙カテゴリは\cite{Bender}と同様にその語彙的な特徴から二つ
に分ける.まず,数詞の後に連接する後置詞,つまりいわゆる助数詞とほぼ
同じふるまいをする語群である.この語彙カテゴリに属する語はEDR概念辞書
の「計量の単位」(概念ID `30f93d')の下位概念である「金銭の単
位」(概念ID `4446bd')の直下にある語を採用した.ただし,これらの語のう
ち「両」や「文」など通貨の単位ではあるものの現在ほとんど使われておらず,
助数詞の他のカテゴリに分類した方が適切であると判断した語は除いた.一方,
数詞の前に連接する前置詞のカテゴリにはコーパスを観察した結果得られた3
語だけが割り当てられている.これらの語は助数詞というよりは純粋な名詞,
もしくは記号としての性質が強い.しかし,通貨に関する数量表現の整合を考
慮して,これらの語にも語彙カテゴリと文法ルールを割り当てた.通貨も単位
と同様の理由から名詞のカテゴリも割り当て,NTYPE属性も付与する.

数量に関する語彙カテゴリはもっとも典型的な助数詞と言える語に割り当てる
カテゴリである.このカテゴリには,助数詞の表層を示すCLASSIFIER-FORM属
性と助数詞の性質を示すCLASSIFIER-TYPE属性を付与する.このカテゴリに属する語は
IPA辞書\cite{ipadicurl}に登録されている接尾語のうち,数えられるものを
特徴づける語を人手で選別した.

順番に関する語彙カテゴリは物事の順序を示す助数詞である.このカテゴリに
もCLASSIFIER-FORMとCLASSIFIER-TYPE属性を与える.このカテゴリには,英語
の序数を英訳したときに用いられる語である「等」,「位」,「号」,「級」,
「流」及びIPA辞書において品詞が助数詞として定義されて
いる語のうち「目」,「番」,「次」を含む語で,単位を示す「貫目」,「次
元」を除いた語を採用した.







\subsection{\label{grammar}助数詞のための文法規則}

表\ref{grammar-table}に助数詞の語彙カテゴリの一覧を示す.本節では数詞が助数詞もしくは名詞に連接して数量を表現している構文を解析
するための文法について述べる.助数詞の語彙カテゴリは\ref{lex}節
のものを用いる.

\begin{table}[b]
\caption{助数詞を解析するための文法規則一覧}
\label{grammar-table}
\input{03table02.txt}
\end{table}

\subsubsection{単位や通貨に関する文法規則}

単位や通貨に関する文法規則を(\ref{grammar-unit})に示す.
単位や通貨を表す助数詞が数詞と連接する場合には,助数詞をPREDと
し,数詞がそれを修飾する.これは英語の``3~kg'' や``10 dollars'' の
f-structureとの整合性を図ると同時に,\ref{lex}節でも述べたようにこれら
の助数詞が接尾辞というよりもむしろ独立した名詞としての性質が強いと考え
るからである.

単位・通貨は修飾先の名詞の属性について述べるために用いられる.例え
ば「60デニールのストッキング」であれば「60デニール」はストッキング
の数量ではなく,「繊維密度」を表現しているのであり,「300円のストッ
キング」の「300円」はストッキングの「価格」を表現している.つまり
「60デニール(という繊維密度)のストッキング」,「300円(という価
格)のストッキング」と解釈できる.従って,「60」や「300」などの数
詞は直接「ストッキング」を修飾しているのではなく,「デニール」や「円」
を修飾しているとして解析するのが妥当である.

さらに,通貨を示す接頭語が数詞に連接する表現を解析する文法を
(\ref{grammar-currency})に示す.これも上記と同じ理由から接頭語をPREDと
し,数詞がそれを修飾する.

\begin{example}
\label{grammar-unit}
\begin{tabular}[t]{lccc}
Nadj $\longrightarrow$ & (NUMBER) && \{ CL\_unit $|$ CL\_currency \} \\
&($\uparrow$SPEC NUMBER)=$\downarrow$ \hspace*{.3cm} &&($\uparrow$PRED)=$\downarrow$\\[2ex]
\end{tabular}

Nadj: 体言 \quad NUMBER: 数詞 \quad CL\_unit: 単位に関する助数詞 \quad CL\_currency: 通貨に関する助数詞 (): 省略可能 $|$: OR PRED: 主辞 SPEC: 指定部
\end{example}

\begin{example}
\label{grammar-currency}
\begin{tabular}[t]{lccc}
Nadj $\longrightarrow$&CL\_Pref\_currency &&NUMBER\\
&($\uparrow$PRED)=$\downarrow$&& ($\uparrow$SPEC NUMBER)=$\downarrow$\\[2ex]
\end{tabular}

CL\_Pref\_currency: 通貨に関する接頭語
\end{example}


\subsubsection{数量に関する文法規則}

数量を表す助数詞は,数詞の中でも「もの」や「こと」の数量を表現す
るのに用いられる典型的なものである.「3個のりんご」や「3箱のりんご」
の「3個」や「3箱」はいずれもりんごの数や量を表している.単位・通貨と
は異なり,数詞「3」は「りんご」を直接修飾していると考えられる.つまり,
助数詞は「数え方」を表現するために補助的に用いられる接尾語として扱う.
この場合,「個」と「箱」の数え方の違いはf-structureの構造そのものには
反映されないが,助数詞の表層を属性値として表現することで,オントロジな
どの知識源を利用した意味解析や文脈解析の処理を利用してより詳細な数え方
の違いを扱うことを想定している.

数量に関する文法規則は,数詞と助数詞だけではなく,それによって数えられ
る名詞についても記述する必要がある.例えば「りんご3個」と「3個
のりんご」はそれぞれ「3個」が「りんご」の数を表しているため,これらの
解析結果は同じであることが望ましい.

(\ref{grammar-card1})に「3個のりんご」と「りんご3個」を解析するための
文法規則を示す.この規則ではPREDはりんごであり,その数が3であると解析
する.これは英語の``3 apples'' と同じf-structureを持たせることを意図した分
析である.

\begin{example}
\label{grammar-card1}
\begin{tabular}[t]{lccc}
Nadj $\longrightarrow$ & \{ NPnum && Nadj\\
&($\uparrow$ SPEC NUMBER)=$\downarrow$ &&($\uparrow$PRED)=$\downarrow$\\[2ex]
& $|$ Nadj && Nnumeric \}\\
&($\uparrow$PRED)=$\downarrow$&&($\uparrow$ SPEC NUMBER)=$\downarrow$\\[2ex]
\end{tabular}

\begin{tabular}[t]{lccccc}
NPnum $\longrightarrow$ & Nnumeric && PPadnominal\\
&($\uparrow$ PRED)=$\downarrow$\hspace*{.3cm} &&\\[2ex]
\end{tabular}

\begin{tabular}[t]{lccccc}
Nnumeric $\longrightarrow$ & NUMBER && CL\_cardinal\\
&($\uparrow$ PRED)=$\downarrow$\hspace*{.3cm} &&($\uparrow$NUMBER-TYPE)=cardinal&&\\[2ex]
\end{tabular}

CL\_cardinal: 数量に関する助数詞 \quad PPadnominal: 連体化助詞
\end{example}

また「3個の値段」,「3個を食べた」など,数える対象である名詞が省略され,
数詞と助数詞が単独で現れる場合もある.このような現象を解析するための文
法規則を(\ref{grammar-card2})に示す.この文法規則ではPREDには代名詞を
示す記号`PRO'を挿入し,その属性に表層が無いことを示す属性値nullを与
える.この操作によって,名詞の省略の有無によらず同じ構造のf-structure
を出力することができる.

上述(\ref{grammar-card1})と下記の(\ref{grammar-card2})の二つの規則は数
詞と助数詞に同時に適用される.体言には(\ref{Nadj})に示すように連体化助
詞による連体修飾のための規則も定義されていることから,数量に関する表現
が連体化助詞によって連体修飾を行う場合には,常に二つのf-structureが出
力される.

\begin{example}
\label{grammar-card2}
\begin{tabular}[t]{lccc}
Nadj $\longrightarrow$ & Nnumeric && e \\
&($\uparrow$SPEC NUMBER)=$\downarrow$ \hspace*{.3cm} &&($\uparrow$PRED)=`pro'\\
&&&($\uparrow$PRON-TYPE)=null\\[2ex]
\end{tabular}

e: 省略記号
\end{example}
\begin{example}
\label{Nadj}
\begin{tabular}[t]{lccc}
Nadj $\longrightarrow$ & NPadnominal && Nadj\\
&($\uparrow$ ADJUNCT)$\ni$$\downarrow$ \hspace*{.3cm} &&($\uparrow$PRED)=$\downarrow$\\[2ex]
\end{tabular}

\begin{tabular}[t]{lccccc}
NPadnominal $\longrightarrow$ & Nadj && PPadnominal\\
&($\uparrow$PRED)=$\downarrow$\hspace*{.3cm} &&\\[2ex]
\end{tabular}

ADJUNCT: 修飾成分
\end{example}
また,上述とは逆に,「3自衛隊」や「7競技」のように助数詞を伴わずに直
接数詞が名詞に連接してその数を表すことがある.このような現象を解析する
ための文法規則を(\ref{grammar-card3})に示す.この文法規則では助数詞の
表層を表すCLASSIFIER-FORM属性に表層が無いことを示す属性値`null'を与え
ることにより,助数詞の有無によらず同じ構造のf-structureを出力すること
ができる.
\begin{example}
\begin{tabular}[t]{lccc}
Nadj $\longrightarrow$ & NUMBER &e& N\\
&($\uparrow$ SPEC NUMBER)=$\downarrow$ \hspace*{.3cm} &($\uparrow$CLASSIFIER-FORM)=null&($\uparrow$PRED)=$\downarrow$\\
\label{grammar-card3}
\end{tabular}

N: 名詞
\end{example}


\subsubsection{順番に関する文法規則}

数詞が順番を示す表現は主に三つが想定される.まず一つは\ref{lex}節で挙
げた順番に関する助数詞が連接した場合である.このような構文に対しては
(\ref{grammar-card1})の文法がほぼそのまま適用できる.もう一つは接尾辞
や接頭辞の連接によって数詞や助数詞の性質が変化する場合である.これは,
「3人目」,「5回目」のように数量を表す助数詞に接尾辞「目」が連接する場
合と「第3章」,「第1回」など数詞の前に接頭辞「第」が連接する場合がある.
さらに,「第3回目」,「第5次」など,接頭辞は助数詞が順番を示していて
も任意に連接する場合がある.

これらの現象を解析するための文法規則を(\ref{grammar-ord1})に示す.
助数詞の属性値が基数を示すcardinalであっても,数詞の示す数の
性質は基数から序数に変化する.従って,語彙ルールには接尾辞によって数詞
の属性NUMBER-TYPEが序数を示す値ordinalになる可能性について記述しておく
必要がある.(\ref{grammar-ord2})に数量を示す助数詞「回」の語彙規則の記
述例を示す.
\begin{example}
\label{grammar-ord1}
\begin{tabular}[t]{lccc}
Nnumeric $\longrightarrow$ & CL\_Pre\_ord&NUMBER&CL\_cardinal \\
&\{($\uparrow$NUMBER-TYPE)= ordinal&($\uparrow$ PRED)=$\downarrow$&\\
&$|$ ($\uparrow$NUMBER-TYPE)=$_c$ ordinal\}&&CL\_Post\_ord \\
&&&\vtop{\hbox{($\uparrow$NUMBER-TYPE)}\hbox{\quad = ordinal}}
\end{tabular}
CL\_Pre\_ord: 第 CL\_Post\_ord: 目 
\end{example}
\begin{example}
\label{grammar-ord2}
\begin{tabular}[t]{lccc}
回 \quad\quad&  CL\_cardinal &  & \{($\uparrow$NUMBER-TYPE) = cardinal\\
 &  & & $|$ ($\uparrow$NUMBER-TYPE)=$_c$ ordinal\}\\
\end{tabular}

$_c$: 制約条件の相等
\end{example}
さらに接頭辞「第」は「第3の男」,「第6のコース」のように助数詞を伴わな
い単独の数詞に連接して数の性質を序数にする場合がある.このような現
象を解析するための文法規則を(\ref{grammar-ord3})に示す.
\begin{example}
\label{grammar-ord3}
\begin{tabular}[t]{lcccc}
Nadj $\longrightarrow$ & CL\_Pre\_ord&&NUMBER\\
&($\uparrow$NUMBER-TYPE)=ordinal&&($\uparrow$ PRED)=$\downarrow$\\
\end{tabular}
\end{example}

\subsubsection{機能的関係と照応的関係について}

 \cite{Bender}は,可算名詞を修飾する助数詞の処理において,「猫を2匹飼
 う」のように,格助詞「を」に連接するNPの後に単独で現れる数詞と助数詞
 をNPの修飾要素として扱う文法規則を提案している.しかし,本稿の文法規
 則はこのように数詞と助数詞が単独で現れる場合には助数詞の修飾先として
 代名詞を示す記号`PRO'を挿入する.これは,「2匹」と「猫」の関係が統語
 機能的な修飾関係にあるのではなく,照応関係にあると考えるからである.
 \cite{Bresnan01}では,f-structureでは統語機能的関係を表現すべきであり,
 照応的な関係はその対象としないと述べているが,本稿もこの立場から文法
 規則を設計した.

日本語構文では数量表現が指すNPが必ず数量表現の直前に現れるとは限らない.
例えば「猫をマンションで2匹飼っていた」のように,「2匹」の係り先が構文
中の任意の場所に現れる可能性がある.\cite{Bender}の文法規則では修飾先
のNPが直前に現れない場合は数量詞の修飾を特定しない.つまり本稿と同様に
照応的な関係として解析する.しかしその結果,「猫をマンションで2匹飼っ
ていた」と「マンションで猫を2匹飼っていた」の解析結果が異なることにな
る.本稿では,こういった単なる語順の違いにf-structureが影響されないこ
とを配慮している.



\section{\label{fstr}f-structureの妥当性の検討}

\subsection{日本語LFGシステムの構成}

LFG規則の妥当性の検証と解析精度の評価実験を行う処理は
\cite{Masuichi2003}に改良を加えて拡張した日本語LFG解析システムを用いる.
図\ref{kousei}に,本稿で用いる日本語LFGシステムの構成を示す.まず日本
語入力文が日本語taggerに渡される.日本語taggerではChaSen \linebreak
\cite{chaurl}
による形態素解析が行われる.その結果に表層文字列や品詞情報を利用した形
態素区切りの修正ルールを適用し,修正された形態素列に品詞情報を利用した
アノテーションルールによってtagを付与する.

tagが付与された形態素列をXerox Linguistic Environment (XLE) に入力する.
XLEはLFG理論の仕様をほぼ完全に実装した処理系である\cite{Maxwell93}.日
本語システムは大きく分けて3種類のデータで構成されている.一つはメモリ
容量や探索深さ,LFG規則の優先順位などの設定を定義するデータ群である.
二つ目はLFGの文法理論に沿って実装された文法規則群である.\ref{grammar}
章で提案した文法規則群は名詞句に関する文法規則に追加されている.三つ目
は語彙規則群である.\ref{lex}章で提案した助数詞に関する語彙規則はここ
に追加されている.

\begin{figure}[t]
\begin{center}
\includegraphics{16-3ia3f2.eps}
\end{center}
\caption{日本語LFGシステムの構成}
\label{kousei}
\end{figure}
\begin{figure}[t]
\begin{center}
\includegraphics{16-3ia3f3.eps}
\end{center}
\caption{「100ドルを両替した。」と「ドルを両替した。」のf-structure}
\label{currency}
\end{figure}



\subsection{他表現のf-structureとの比較\label{fstr-j}}
本節では\ref{rule}章で提案したLFG規則が導出するf-structureが
\ref{mokuteki}節で設定した基準1を満たしているかを検証するために,同じ
統語意味構造を持ちながらも語順の変化や省略によって表層形が異なった構文
のf-structureを比較する.

一般に通貨や単位は数詞を伴わないで単独の名詞として用いられる例も多い.
そこで通貨の助数詞が数詞を伴う構文「100ドルを両替した。」に対して文
法規則(\ref{grammar-card1})を適応して導出したf-structureと数詞を伴わな
い構文「ドルを両替した。」のf-structureを図\ref{currency}に示す.両者
を比較すると同じ構造を持っていることが分かる.

次に数量に関する表現のf-structureについて検討する.数量を表現する文法
規則は\ref{grammar}節の文法規則(\ref{grammar-card1})と文法規則
(\ref{grammar-card2})があり,これらは同時に適用されるので必ず2通りの
f-structureが出力される.「3箱の煙草」について図\ref{3box-a}に文法規則
(\ref{grammar-card1})によって導出されたf-structureを図\ref{3box-b}に文
法規則(\ref{grammar-card2})によって導出されたf-structureを示す.図
\ref{3box-a}のf-structureで「煙草」のSPECのNUMBER属性中のPREDが「3」
になっているのは,煙草の数量が3であることを表現している.従って,正し
いf-structureは文法規則(\ref{grammar-card1})によって導出された図
\ref{3box-a}のf-structureである.

\begin{figure}[t]
\begin{center}
\includegraphics{16-3ia3f4.eps}
\end{center}
\caption{文法規則(\ref{grammar-card1})によって導出された「3箱の煙草」のf-structure}
\label{3box-a}
\end{figure}
\begin{figure}[t]
\begin{center}
\includegraphics{16-3ia3f5.eps}
\end{center}
\caption{文法規則(\ref{grammar-card2})によって導出された「3箱の煙草」のf-structure}
\label{3box-b}
\end{figure}

さらに,助数詞を伴わず直接数詞が名詞を修飾してその数を表す場合も,文法
規則(\ref{grammar-card3})によって図\ref{3box-a}と同じ構造のf-structure
で表現される.図\ref{3meigara}に「3銘柄」のf-structureを示す.

\begin{figure}[t]
\begin{center}
\includegraphics{16-3ia3f6.eps}
\end{center}
\caption{文法規則(\ref{grammar-card3})によって導出された「3銘柄」のf-structure}
\label{3meigara}
\end{figure}

しかし,数詞の数える対象が係り先の名詞ではない場合がある.例えば,「3
箱の値段」という表現において数詞「3」の数える対象は「値段」ではない.
この名詞句の解釈は「3箱の何か(例えば煙草)の値段」とするのが妥当であ
る.図\ref{3box-c}と図\ref{3box-d}に「3箱の値段」のf-structureを示す.
文法規則(\ref{grammar-card2})によって導出された図\ref{3box-d}の
f-structureでは表層が省略されていることを現すPRON-TYPEの属
性値nullを持つ代名詞`pro'の数量として表現されている.つまり,図\ref{3box-d}の
f-structureが「3箱の値段」の正しい統語意味構造を表している.また,数量
表現が係り先の名詞を伴わずに単独で現れる場合も(\ref{grammar-card2})に
よって適切なf-structureを出力する.例として「彼が選んだ3箱」の
f-structureを図\ref{3box-tandoku}に示す.

\begin{figure}[t]
\begin{center}
\includegraphics{16-3ia3f7.eps}
\end{center}
\caption{文法規則(\ref{grammar-card1})によって導出された「3箱の値段」のf-structure}
\label{3box-c}
\end{figure}
\begin{figure}[t]
\begin{center}
\includegraphics{16-3ia3f8.eps}
\end{center}
\caption{文法規則(\ref{grammar-card2})によって導出された「3箱の値段」のf-structure}
\label{3box-d}
\end{figure}

次に,順序を示す表現について検討する.それぞれの表現の差異は,属性や属
性値によって表現されるが,基本的な構造はすべて等しいことが分かる.

図\ref{3banme}に文法規則(\ref{grammar-ord1})によって導出された「3番目
の男」のf-structureを示す.「番目」は順番を示す助数詞の語彙カテゴリが
割り当てられているため,CLASSIFIER-TYPE属性とNUMBER-TYPE属性の値は共に
序数を示すord(inal)になっている.

図\ref{3ninme}に文法規則(\ref{grammar-ord2})によって導出された「3人目
の男」のf-structureを示す.「人」は数量を示す助数詞の語彙カテゴリが割
り当てられているため,CLASSIFIER-TYPE属性の値には基数を示すcard(inal)
が入る.しかし,接尾辞「目」が順番を表現しているためNUMBER-TYPE属性の値
が序数を示すord(inal)になっている.

図\ref{dai3}に文法規則(\ref{grammar-ord3})によって導出された「第3の男」
のf-structureを示す.この名詞句には助数詞が存在せず,助数詞の省略も想
定され難いためCLASSIFIER-TYPE属性を持たない.しかし,接頭辞「第」が順番
を表現していることをNUMBER-TYPE属性の値がord(inal)になって
いることで示している.

\begin{figure}[t]
\begin{center}
\includegraphics{16-3ia3f9.eps}
\end{center}
\caption{文法規則(\ref{grammar-card2})によって導出された「彼が選んだ3箱」のf-structure}
\label{3box-tandoku}
\end{figure}
\begin{figure}[t]
\begin{center}
\includegraphics{16-3ia3f10.eps}
\end{center}
\caption{文法規則(\ref{grammar-ord1})によって導出された「3番目の男」のf-structure}
\label{3banme}
\end{figure}
\begin{figure}[t]
\begin{center}
\includegraphics{16-3ia3f11.eps}
\end{center}
\caption{文法規則(\ref{grammar-ord2})によって導出された「3人目の男」のf-structure}
\label{3ninme}
\end{figure}




\subsection{他言語のf-structureとの比較\label{fstr-cande}}

本節では\ref{mokuteki}節で設定した基準2について検証するために,日本語
の表現に対応する他言語のf-structureと日本語のf-structureの比較を行う.

\begin{figure}[t]
\begin{center}
\includegraphics{16-3ia3f12.eps}
\end{center}
\caption{文法規則(\ref{grammar-ord3})によって導出された「第3の男」のf-structure}
\label{dai3}
\end{figure}
\begin{figure}[t]
\begin{center}
\includegraphics{16-3ia3f13.eps}
\end{center}
\caption{``I exchanged 100 dollars'' のf-structure}
\label{currency-e}
\end{figure}



まず通貨・数量に関する表現のf-structureについて検討する.図
\ref{currency-e}に「100ドルを両替した.」の英訳``I exchanged 100
dollars.'' を入力として\cite{Riezler}の英語LFGシステムで出力された
f-structureを示す.\ref{fstr-j}節の図\ref{currency}の日本語の
f-structureと比較すると同じ構造を持っていることが分かる.

次に数量に関する表現のf-structureについて中国語のf-structureとの比較を
行う.数量に関する表現は助数詞を伴う場合と助数詞が省略される場合がある.
助数詞を伴う表現については\cite{ji}の中国語LFGシステムで出力された
f-structureと,助数詞を伴わない表現については英語のf-structureと比較する.

図\ref{3bin-1-j}に文法規則(\ref{grammar-card1})が導出した「三本の酒」
のf-structureを,図\ref{3bin-1-c}に「三瓶酒」に対応する中国語の
f-structureを示す.図\ref{3bin-1-c}のf-structureでは助数詞のための属性
であるCLASSIFIERをPREDにしている.これは,中国語の助数詞が連体修飾を受
けることができるほど名詞としての性質が強いためである.このような違いは
あるものの,この表現に対するする日本語と中国語のf-structureの基本的構
造はほぼ同じである.

\begin{figure}[t]
\begin{center}
\includegraphics{16-3ia3f14.eps}
\end{center}
\caption{文法規則(\ref{grammar-card1})によって導出された「三本の酒」のf-structure}
\label{3bin-1-j}
\end{figure}
\begin{figure}[t]
\begin{center}
\includegraphics{16-3ia3f15.eps}
\end{center}
\caption{「三瓶酒」のf-structure}
\label{3bin-1-c}
\end{figure}
\begin{figure}[t]
\begin{center}
\includegraphics{16-3ia3f16.eps}
\end{center}
\caption{文法規則(\ref{grammar-card2})によって導出された「彼女は三本を飲んだ。」のf-structure}
\label{3bin-2-j}
\end{figure}

次に,助数詞が数える対象である名詞が存在しない数量表現について
「彼女が三本を飲んだ」と「\KetujiX{16-3ia3f00.eps}喝了三瓶」の解析結果を比較する.図
\ref{3bin-2-j}に日本語のf-structureを,図\ref{3bin-2-c}に中国語の
f-strcutureを示す.両者とも数える対象に,代名詞を示す`pro'を代入してい
る.これは,数える対象の名詞が存在する場合のf-structureとの整合性を
図るための手当てであるが,それと同時に中国語と日本語の間のf-structre
の整合性を保持することも可能にしている.

助数詞が省略される表現については英語のf-structureと比較を行う.図
\ref{3families}に「3家族」と``3 families'' のf-structureを示す.日本語の
f-structureでは助数詞が存在しないことを明示的に表現するために
CLASSIFIER-FORM属性がnullになっているが,英語では基本的に助数詞を用
いないので,それに対応する属性は存在しない.しかし,基本的な構造は等し
いことが分かる.

\begin{figure}[t]
\begin{center}
\includegraphics{16-3ia3f17.eps}
\end{center}
\caption{「\protect\KetujiX{16-3ia3f00.eps}渇了三瓶」のf-structure}
\label{3bin-2-c}
\end{figure}
\begin{figure}[t]
\begin{center}
\includegraphics{16-3ia3f18.eps}
\end{center}
\caption{「3家族」と「3 families」のf-structure}
\label{3families}
\end{figure}
\begin{figure}[t]
\begin{center}
\includegraphics{16-3ia3f19.eps}
\end{center}
\caption{``third man'' のf-structure}
\label{3rd-man}
\end{figure}


次に順序を示す表現について検討する.図\ref{3rd-man}に\ref{fstr-j}節の
図\ref{3banme}〜\ref{dai3}に対応する英語``third man'' のf-structureを示す.
特に図\ref{dai3}の「第3の男」のf-structureとは同じ構造を持っている.


\section{\label{hyouka}解析結果の精度評価}

\subsection{評価方法}

解析結果の評価はf-structureを\cite{triples}に準拠した形式 (triples) に
変換して行う.この変換の際に文法的な情報ではなく実装の都合上付与されて
いる属性は削除する.図\ref{triples}に「3個のりんご」のf-structureと
triplesの対応関係を示す.

\begin{figure}[b]
\begin{center}
\includegraphics{16-3ia3f20.eps}
\end{center}
\caption{f-structureとtriplesの対応関係}
\label{triples}
\end{figure}

triples形式は``{\it 属性C(ノードA,ノードB)}'' もしくは``{\it 属性C(ノード
A,属性値B)}'' という書式で表現される.いずれも「{\it A}の属性{\it C}は
{\it B}である」と解釈する.属性の要素はノードか値であるが,ノードは必
ずノードIDを持つため,同じ語が同一文中に現れても区別が可能である.図
\ref{triples}中のtriples形式のデータは「りんご」の
NUMBER属性が3であること,「3」のCLASSIFIER-FORM属性の値は「個」であるこ
と,「3」のCLASSIFIER-TYPE属性の値がcardinalであること,「3」のNUMBER-TYPE属
性の値がcardinalであることを表現している.

解析対象はEDRコーパスから数詞を含む文を無作為に選び,単位・通貨に関す
る表現,数量に関する表現,そして順序に関する表現を人手で200表現ずつ抽出
した.

本稿で提案する規則の効果を確認するために,精度の比較を
\cite{Masuichi2003}で採用されているLFG規則の解析結果と行う.今回の提案
規則による効果を正確に計測するために,旧規則の解析結果に単なる仕様変更
によって正解にないNUMBER-TYPEなどの属性が含まれる場合はこの属性を削除した.評価実験にお
けるtriplesに含まれる属性の一覧を表\ref{feature-table}に示す.

今回の実験では,多義性の解消を行っていないため,複数の解析結果が出力される
可能性もあるが,精度の測定はそれらの解析結果すべてに対して行った.例えば,
\ref{fstr-j}節で「3箱の煙草」に対して得られた図\ref{3box-a}と図
\ref{3box-b}の二つの解析結果のうち,図\ref{3box-a}が正解であるならば,
この解析結果の精度は下記の表\ref{tagi-kekka}のように求められる.

\begin{table}[t]
\caption{triplesに含まれる属性一覧}
\label{feature-table}
\input{03table03.txt}
\end{table}
\begin{table}[t]
\caption{「3箱の煙草」の解析結果}
\label{tagi-kekka}
\input{03table04.txt}
\end{table}

今回提案したLFG規則のみを評価するため,解析する範囲を数詞,助数
詞及びそれらが連体連体修飾を行っている場合はその修飾先の名詞句,連体修
飾を受けている場合にはその修飾成分のみに限定した.また「およそ30
個」の「およそ」や「5\%以下」の「以下」など直接連接して助数詞や数詞の
修飾を行う付属語的な名詞も対象とした.下記に解析対象の例を示す.

\begin{itemize}
\item{単位・通貨に関する表現}
\begin{itemize}
\item{\underline{\mbox{400万円のミンクのコート}}、\underline{\mbox{250万円の腕時計}}も頭にちらついた。}
\item{IBMおよびコンパチは\underline{\mbox{パソコン全体の80%以上}}を占めた。}
\end{itemize}
\item{数量に関する表現}
\begin{itemize}

\item{\underline{\mbox{3国}}合わせての総人口が\underline{\mbox{約750万人}}だから \underline{\mbox{5人に1人}}が参加したことになる。}
\end{itemize}
\item{順序に関する表現}
\begin{itemize}
\item{\underline{\mbox{2番目のLAN}}は事務職員のためのLANだ。}
\item{ASTの\underline{\mbox{第1のライバル}}である強化ボードの開発者クォードラムでは、IBMが新しいラインに残したメモリー拡張のギャップに焦点を合わせている。}
\end{itemize}
\end{itemize}




\subsection{結果と考察}


    \subsubsection{単位・通貨に関する表現}

\begin{table}[b]
\caption{単位・通貨に関する表現の解析結果}
\label{unit-kekka}
\input{03table05.txt}
\end{table}
\begin{table}[b]
\caption{単位・通貨に関する表現の属性別の解析結果(旧規則)}
\label{unit-kekka-old}
\input{03table06.txt}
\end{table}
\begin{table}[b]
\caption{単位・通貨に関する表現の属性別の解析結果(提案規則)}
\label{unit-kekka-new}
\input{03table07.txt}
\end{table}

表\ref{unit-kekka}に単位・通貨に関する表現の解析結果を,表
\ref{unit-kekka-old}と表\ref{unit-kekka-new}に属性別の結果を示す.旧規
則では単位・通貨と他の助数詞との区別をしていなかったため,
CLASSIFIER-FORM属性とCLASSIFIER-TYPE属性が解析結果に含まれる.そこで,
今回の比較は,この二つの属性を予め削除して行った.今回提案したLFG規則
により全体的なF値はおよそ25\%向上した.表\ref{unit-kekka-old}の
fragmentは,部分解析結果の属性であり,旧規則では適用される規則が存在せ
ず部分解析しかできなかった表現があることを示している.表
\ref{unit-kekka-new}にこのfragment属性が存在しないことは提案規則ではす
べて全体的な解析に成功していることを示している.また,EDR概念辞書を用
いて語彙エントリを追加したことにより,修飾関係を示す属性adjunctのF値が
向上している.表\ref{unit-kekka-new}でsubj,tense,atype,clause\_type
のprecisionの値が下がっているのはすべての単位・通貨に名詞のlexical
categoryを割り当てた結果,文法規則(\ref{grammar-card3})が誤って適用さ
れてしまう例があったためである.今後は一般的に名詞としても用いられる単
位名とほとんど名詞としては現れない単位の分別が必要である.

併置関係を示す属性conjのF値が両規則ともに0なのは,下記の例にあるような
数詞と記号による表現を提案規則でカバーできなかったためである.特に単位
に関する表現は記号を含む表現が多く見られ,今後はこれらの記号と数詞に関
する規則の精緻化が課題である.

\begin{itemize}
\item{\underline{\mbox{640×400ドットの解像度}}でグラフィックを表示できるという。}
\item{\underline{\mbox{旧システムの30%〜50%}}をリプレースする大規模な計画で、2年がかりで構築する。}
\end{itemize}


    \subsubsection{数量に関する表現}

\begin{table}[b]
\caption{数量に関する表現の解析結果}
\label{card-kekka}
\input{03table08.txt}
\end{table}
\begin{table}[b]
\caption{数量に関する表現の属性別の結果(旧規則)}
\label{card-kekka-old}
\input{03table09.txt}
\end{table}


表\ref{card-kekka}に数量に関する表現の解析結果を,表
\ref{card-kekka-old}と表\ref{card-kekka-new}に属性別の結果を示す.
\pagebreak
全体的なF値はおよそ5%向上している.両者とも連体修飾を示すadjunct属性の値
が低いのは,文法規則(\ref{grammar-card1})と文法規則
(\ref{grammar-card2})によって生じた多義性によるものである.この種の多
義性を解消するためには\cite{Sirai}や\cite{Bond}で提案されているオント
ロジーのような知識資源の利用が有効であると思われる.その一方で,下記の
例の「2人」と「仲間」のように,助数詞と数えられる対象の関係だけでは正
しいf-structureを決定することが困難である場合もあり,文脈解析のような
より深い知識処理が必要である.

\begin{table}[t]
\caption{数量に関する表現の属性別の結果(提案規則)}
\label{card-kekka-new}
\input{03table10.txt}
\end{table}

\begin{itemize} 
\item{3年前には、横浜防衛施設局に\underline{\mbox{仲間2人}}が火炎瓶を投げて逮捕された。}
\item{\underline{\mbox{2人の仲間}}だったシンイチは、いつか姿を消した。}
\end{itemize}


さらに,連体修飾成分と数詞,助数詞の関係を表現するのが難しい場合があっ
た.下記の文の「長男と2人」は「長男と(自分の)2人」を指しており,「長
男」と「2人」の並置と考えるよりは「長男と」が「2人」を修飾すると解釈す
るのが適切であるが,本稿で提案したLFG規則ではこのような表現は扱えない.

\begin{itemize} 
\item{10年前、姉妹を頼って、\underline{\mbox{長男と2人}}で引き揚げてきた。}
\end{itemize}
また,本稿で提案した規則では評価用コーパス中の下記の表現が解析できなかっ
た.「100万票差」の「差」,あるいは「合計」,「積」なども含めて数量
操作に関わる名詞については今後f-structureの表現方法も含めて扱い方を検
討する必要がある.

\begin{itemize}
\item{71年の大統領選挙で朴大統領に\underline{\mbox{約100万票差}}で敗れる。}
\item{\underline{\mbox{11の市民団体}}が実行委員会を結成し、それぞれの企画を
原爆忌の9日まで繰り広げる。}
\item{これで輸入解禁州は全部で \underline{38} になる。}
\end{itemize}

上記の「11の市民団体」の「11」や「全部で38になる」の「38」のよ
うに,助数詞や接頭辞,接尾辞など数詞の性質を表現する機能語が連接せずに
名詞を修飾している場合,その数詞が何を表現しているのか判断するのが難し
い.評価コーパスには数量を表す例しか出現しなかったが,下記の例文の「二百
万のお宝」の「二百万」は「お宝」である「トルエン」の価格を表現しており,
「一の富」の「一」は宝くじの一等を示している.

\begin{itemize}
\item{千本の物が—。\underline{二百万のお宝}が。そのとき、トルエンが爆発した。[大沢在昌「ちきこん」『らんぼう』新潮文庫(1998)]}

\item{ふつうこれは百回ついたものだそうで、百回目を突きどめといって、これがつまり\underline{一の富}、千両もらえるわけである。[横溝正史「鶴の千番」『くらやみ婿』春陽文庫(1984)]}
\end{itemize}


\subsubsection{順序に関する表現}

表\ref{ord-kekka}に順序に関する表現の解析結果を,表\ref{ord-kekka-old}
と表\ref{ord-kekka-new}に属性別の結果を示す.旧文法ではCLASSIFIER-TYPE属性
の序数を表現する値ordinalを定義していなかったため,解析結果を公平に比
較するためにこの属性を削除した.それでも,全体的なF値は21\%以
上向上した.

\begin{table}[b]
\caption{順序に関する表現の解析結果}
\label{ord-kekka}
\input{03table11.txt}
\end{table}
\begin{table}[b]
\caption{順序に関する表現の属性別の結果(旧規則)}
\label{ord-kekka-old}
\input{03table12.txt}
\end{table}

ただし,本稿で提案した規則では扱えない表現も見られた.順序に関する表現
は,通貨・単位や数量に関する表現よりも,助詞を介さずに直接他の名詞と連
結して,複合名詞を形成する場合が多く,下記のように係り先がうまく決定で
きない例が散見された.本稿で用いた日本語LFGシステムでは,処理時間を高
速化するために\cite{modpatent}の手法を用いて,名詞が二つ以上連続すると
きにはすべて最右の名詞を主辞とする複合名詞としてまとめる処理を実施する.
その結果,下記の(a)では「第2次」が複合名詞中で最も右にある「撤兵」を
修飾するため正しいf-structureを得ることができる.しかし,(b)では「2次」
が複合名詞中の「下請け」を修飾しているので本稿のシステムでは誤った
f-structureを導出する.

\begin{table}[t]
\caption{順序に関する表現の属性別の結果(提案規則)}
\label{ord-kekka-new}
\input{03table13.txt}
\end{table}


\begin{itemize}
\item[(a)]{このサマー地区はベトナム軍が\underline{\mbox{第2次}}\underline{カンボジア}\underline{撤兵}の際に通過した地だ。}
\item[(b)]{\underline{\mbox{2次下請け工場}}になると、忙しさだけは1次下請けに負けない。}
\end{itemize}

また,下記の「版」,「条」,「面」のように助数詞が数量を現すも
のであっても実際には順序に関する表現である例があった.これは
\cite{Sogino}で示唆されているように,助数詞には基数と序数の両方を表す
ものがあるためであると思われる.今後は,こういった助数詞の特定も必要で
ある.

\begin{itemize}
\item{この批評は大阪本社発行の \underline{\mbox{13版}}をもとにしています。}
\item{\underline{\mbox{航空法施行規則145条}}に規定された各種の航空計器および航法装置。}
\item{これは先日、離婚請求権が有責者にも認められたという判決が、新聞の \underline{\mbox{1面}}に出た翌日の会話である。}
\end{itemize}


\section{おわりに}

\subsection{まとめ}
本稿では数詞と助数詞によって表現される構文を日本語LFGシステムで解析す
るための語彙規則と文法規則を提案した.さらに提案した規則によって導出さ
れるf-structureを他の表現や他の言語のf-structureと比較して解析結果の妥
当性を検討した.また,精度評価実験を実施して提案規則の解析能力について,
従来のLFG規則との比較を行った.その結果,通貨・単位に関する表現では
25\%,数量に関する表現では5\%,順序に関する表現では21\%のF値の向上が
確認できた.


\subsection{今後の課題}
\ref{rule}章では,助数詞を通貨・単位,数量,順序という三つの語彙カテゴ
リに分類し,それぞれに対して文法規則を提案した.しかし,助数詞の中には
この三つのカテゴリのどれか一つに明確に分類できないものも存在する.例え
ば,本稿で「組」は数を表す助数詞として分類されるが,「ダース」は単位を
表す助数詞として分類されている.このように,単位が数や量を表す場合には
今回の分類を明確に適用することが難しい.また,\ref{hyouka}章の実験結果
にも見られたように,「版」や「条」など数量と順序の両方を表す助数詞も存
在する.このような問題を解決するためには,特殊な性質を持つ助数詞を特定
し,本稿で提案した分類の細分化や複数の語彙カテゴリの割り当てなどを検討
する必要がある.

\ref{fstr-cande}節では本稿で提案したLFG規則が導出するf-structureの妥当
性を検証するために,英語と中国語のf-structureとの比較を行った.これは,
両者とも実用的なLFGシステムが既に構築されており,日本語文の訳文に対し
て適切なf-structureを得ることが可能だったためである.しかし,他の言語
のLFGシステムの研究開発が進めば,他の言語のf-structureとも比較したい.

\ref{hyouka}章の評価実験では,解析精度の比較対象として旧LFG規則の解析
結果を用いたが,より正確に解析能力を検証するためには,他のシステムの出
力結果と比較する必要がある.しかし,\ref{haikei}節でも述べたとおり,現
在公開されている構文解析システムでは本稿が対象にしている助数詞と数詞,
名詞の関係を表現しないため,f-structureと直接比較することができない.
今後,翻訳や質問応答などのシステムの出力を利用して間接的に比較を行うな
ど,評価の工夫が必要である.本稿の評価実験対象は助数詞に関する表現に限
定した.今回の改良によって文全体のカバー率には特に変化が見られなかった
が,個々の解析結果が文全体の解析精度にどのような影響を及ぼしているかは
より詳細に観察する必要がある.

今後は,今回解析できなかった構文を扱うためにLFG規則の改良と拡張
に取り組む.また,\cite{umemoto}や\cite{Crouch}などf-structureを入
力とする意味解析手段,つまりLFG解析の次の段階の処理を用いて,提案した
規則によって生じたf-strcutureの曖昧性解消を実施したい.



\acknowledgment

 ParGramのメンバー,特に助数詞の解析に関する議論を通じて有益なコメント
 をいただいたり,解析結果を提供していただいた Microsoft社のTracy
 Holloway King氏,Martin Forst氏,Palo Alto Research Center Inc.のJi
 Fang氏に感謝いたします.また,日本語LFGについて有益なコメントをいただ
 いた早稲田大学の原田康也教授,
 中野美知子教授に感謝いたします.XLEの開発者であり,日本語システム構築
 時に実装に関する貴重な助言を頂いたPalo Alto Research Center Inc.の
 John Maxwell氏,Microsoft社のRonald Kaplan氏に感謝いたします.


\bibliographystyle{jnlpbbl_1.3}
\input{03refs.bbl}

\begin{biography}
\bioauthor{大熊 智子(正会員)}{
1994年東京女子大学文理学部日本文学科卒業.1996年慶應義塾大学政策・メディ
ア研究科修了.同年,富士ゼロックス(株)総合研究所入社.2006年より慶應義
塾大学政策・メディア研究科後期博士課程在籍.2009年より東京女子大学非常
勤講師.現在,富士ゼロックス(株)研究技術開発本部研究副主任.
}

\bioauthor{梅基  宏(正会員)}{
1995年東京大学大学院工学系研究科機械情報工学修士課程修了.同年,富士
ゼロックス(株)入社.日本語意味解析,情報 検索システムの研究開発に従事.
2004〜2006年 Stanford大学CSLI客員研究員.言語処理学会,情報処理学会各
会員.
}

\bioauthor{三浦 康秀}{
2004年電気通信大学大学院電気通信学研究科電子情報学専攻修了.同年,富士ゼロックス(株)入社.日本語の統計的構文解析,医療テキストを対象とした自然言語処理の研究開発に従事.
}
\bioauthor{増市  博(正会員)}{
1989年京都大学工学部機械工学科卒.1991年京都大学工学研究科精密工学専攻
修士課程修了.同年,富士ゼロックス(株)入社.1998〜2000年米国Stanford大学
CSLI客員研究員およびPalo Alto Research Center Inc.コンサルタント研究員.
現在,富士ゼロックス(株)研究技術開発本部研究主査.博士(工学).
}

\end{biography}

\biodate


\end{document}

