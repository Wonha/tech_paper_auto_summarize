

\documentstyle[jnlpbbl,eclepsf]{jnlp_j} 

\setcounter{page}{1}
\setcounter{巻数}{6}
\setcounter{号数}{5}
\setcounter{年}{2000}
\setcounter{月}{3}
\受付{2000}{1}{1}
\再受付{2000}{1}{1}
\採録{2000}{1}{1}
\setcounter{secnumdepth}{2}


\newcounter{example-number}
\newcommand{\example}[1]{}

\title{機械翻訳を介したコミュニケーションにおける利用者の機械翻訳システム適応の言語依存性}
\author{小倉 健太郎\affiref{NTT} \and 林 良彦\affiref{Osaka} \and 野村 早恵子\affiref{Kyoto}\affiref{UC} \and 石田 亨\affiref{Kyoto}}

\headauthor{小倉,林,野村,石田}
\headtitle{機械翻訳を介したコミュニケーションにおける利用者の機械翻訳システム適応の言語依存性}

\affilabel{NTT}{日本電信電話会社 NTTサイバースペース研究所}
{NTT Cyber Space Laboratories, NTT Corporation}
\affilabel{Osaka}{大阪大学大学院言語文化研究科}
{Graduate School of Language and Culture, Osaka University}
\affilabel{Kyoto}{京都大学大学院情報学研究科}
{Department of Social Informatics, Kyoto University}
\affilabel{UC}{カリフォルニア大学サンディエゴ校認知科学学部}
{Cognitive Science Department, University of California, San Diego}

\jabstract{
本論文では,機械翻訳を介したコミュニケーションにおける利用者の機械翻訳システ
ムへの適応状況を分析し,機械翻訳を介した異言語間コミュニケーション支援の方向
性について論ずる.コミュニケーションの目的が明確で,利用者の機械翻訳への適応
が期待できる状況において,多言語機械翻訳を介したコミュニケーションを行う時,
利用者はどのような適応を行うのか,また,その適応の効果はどの程度のものなのか
を明らかにした.適応のための書き換えの方法は翻訳言語ペアに強く依存することが
分かった.日本語から英語への翻訳の場合,日本語と英語の概念間の食い違いを補う
ための語句の置き換えや言語表現習慣の違いを補う主語の補完などが多く観察された.
また,日本語や韓国語のように類似の言語では,それらの言語における適応の傾向が
似ていることが分かった.日本語から英語への翻訳のための適応は,英訳自体には効
果が大きいが,韓国語訳にはほとんど効果がなく,中国語訳への効果もそれほど大き
くはないことが分かった.
}
\jkeywords{機械翻訳,異文化コラボレーション,適応,コミュニケーション}

\etitle{Language-Dependency in User's Adaptation for MT Systems in MT-mediated Communication}
\eauthor{KENTARO OGURA\affiref{NTT} \and YOSHIHIKO HAYASHI\affiref{Osaka} \and SAEKO NOMURA\affiref{Kyoto}\affiref{UC} \and TORU ISHIDA\affiref{Kyoto}}

\eabstract{
This paper analyzes the impact of user adaptation in MT-mediated communication.
It clarifies how the user adapts to machine translation and how effective the adaptation
is in terms of communication when the purpose of communication is clear. The
most common alterations and their effectiveness strongly depend on the translation
language pairs. In the case of Japanese-to-English translation, we observed two main 
alterations: replacing words or phrases to offset the difference in concepts between
Japanese and English and supplementing subjects to offset the difference in modes
of expression between Japanese and English. Since Korean and Japanese are similar
languages, Korean users exhibited similar adaptation tendencies. The adaptation
performed by Japanese users when referring to the English translation was very effective
in improving the quality of the English translations. However, it was not so
effective for Chinese and even less effective for Korean translations.
}
\ekeywords{machine translation, intercultural collaboration, adaptation, communication}


\begin{document}

\input{title}
\maketitle

\section{はじめに}
\label{sec:intr}

インターネットの世界的な普及により,世界各国に分散したメンバーによるソ
フトウェア開発などが盛んになっている\cite{Jarvenpaa}.特に,アジア太平
洋地域でのインターネットの普及は目覚しく
\footnote{http://cyberatlas.internet.com/big\_picture/geographics/print/\\
  0,,5911\_86148,00.html},
今後,この地域におけるソフトウェアの共同開発などが活発化すると予想され
る.しかし,母国語が異なる国々と共同ソフトウェアの開発などを行う場合,
言葉の壁により円滑にコミュニケーションを行うことは難しい.共通言語とし
て英語を使用することにより,コミュニケーションを行うことも可能であるが,
英語で書くことは負担が大きく,コミュニケーションの沈滞を招く.異文化間
でのコラボレーション参加者は母国語での情報発信を望んでいる.機械翻訳の
利用はこのような異言語間におけるコミュニケーション課題を解決する1つの
手段である.

機械翻訳は異文化コラボレーションを行うためのコミュニケーションの道具と
してどのように役に立つのか?あるいは,役立つようにするためには,どのよ
うな問題を克服する必要があるのか?このような問いに答えることは,コミュ
ニケーションの新しい研究テーマとして有意義であるとともに,機械翻訳シス
テム開発への有益な提言が得られる可能性が高いという意味でも重要である.

また,コンピュータを介したコミュニケーションの研究は最近活発に行われて
いるが\cite{Herring},機械翻訳を介したコミュニケーションの研究
\cite{Miike}は,まだ少なく,二ヶ国語間の機械翻訳で,機械翻訳への適応が
行なわれないコミュニケーションの研究が中心である.さらに,機械翻訳の研
究においても,機械翻訳自体の翻訳品質の評価の研究\cite{Hovy,Papineni}は
活発に行われているが,コミュニケーションという観点からの評価は行われて
いない.

本論文では,機械翻訳を介したコミュニケーションによる母国語が異なる異文
化間での共同ソフトウェア開発のためのコラボレーション実験を行うことによ
り,目的が明確で,かつ,利用者の機械翻訳への適応が期待できる環境におい
て,決して十分な翻訳品質とは言えない機械翻訳に対して利用者がどのように
適応を行ってコミュニケーションを成立させようとするのかを分析する.また,
その適応効果はどの程度のものなのかを明らかにする.適応の翻訳言語ペアに
ついての依存性,英訳を参照した適応の他言語への翻訳への有効性,言語ごと
の適応の違いなどを中心に分析した結果を提示し,機械翻訳を介した異言語間
コミュニケーション支援の方向性について述べる.

\section{ICE2002}
\label{sec:ICE2002}

言葉の壁を克服するコラボレーションを目指して,日本,韓国,中国,マレー
シアなど母国語が異なる異文化間での共同ソフトウェア開発\cite{Othmann:2003HCI}
のコラボレーション実験ICE2002\cite{野村:2003情報処理,Nomura:2003HCI}が
行われた.情報の発信はそれぞれ母国語で行い,他の国からの情報は機械翻訳
を介して,母国語で読むことができる.利用者は母国語だけではなく,英語,
日本語,韓国語,中国語,マレーシア語で読むことも可能であり,大多数の利
用者は,母国語と英語を読む設定を行っていた.
コラボレーションのための手段としては,利用者が母国語で書いたメッセージ
を他の参加者がその人の母国語に翻訳して読むことができる多言語電子会議シ
ステムTransBBS(図\ref{fig:ICE2002})と,HTML形式のソフトウェア開発ド
キュメントを多言語に翻訳して閲覧できるTransWebを
使用した\cite{船越:2004,Funakoshi:2003HCI}.
本稿では,ICE2002においてTransBBS上で行われた利用者が行った機械翻訳シス
テムに対する適応\cite{Ogura:2004}について分析を行った結果について報告する.

\begin{figure}[tb]
 \begin{center}
  \epsfile{file=fig1.eps,scale=0.7}
 \end{center}
 \vspace*{-3mm}
 \caption{TransWEB}
 \label{fig:ICE2002}
\end{figure}

TransBBSでは,メッセージを投稿する前に,翻訳結果を確認できる機能がある.
多くの実験参加者(日本:京都大学,韓国:ソウル国立大学&ハンドン大学,
中国:上海交通大学,マレーシア:マラヤ大学の学生)は英語を理解すること
ができるので,母国語を英語に翻訳し,翻訳結果を確認し,英訳が満足できる
ものでない場合は,原文を修正して翻訳を行い,翻訳が満足できるレベルに達
するか,これ以上翻訳が改善できない時点で,メッセージの発信を行った.ま
た,参加者は母国語に翻訳されたメッセージを読み,必要に応じて,英語や原
文を参照してメッセージの理解を行った.

TransBBSは,開発ソフトウェアの設計に関する意見交換や開発状況の報告など
を行う目的で使用された.TransBBSで用いられた文体は話し言葉であった.ソ
フトウェアの設計フェーズでは,円滑なコラボレーションを行うために,人間
関係を円滑にする呼びかけ,挨拶表現,激励表現が多かった.ソフトウェアの
開発のフェーズでは,質問,依頼,確認の表現が多かった.また,チャットに
見られるような砕けた表現は少なく,丁寧でフォーマルな表現が多かった
\cite{小倉:2003IPSJ}.

日英,日韓,日中間および英語から中国語へは直接翻訳している.しかし,中
韓,韓英間および中国語から英語への翻訳は,いったん別の言語(日本語)に
翻訳してから,その翻訳結果をさらに目的の言語に翻訳する2段翻訳を行って
いるので,翻訳の精度は直接翻訳に比べて低かった.

\section{利用者適応の傾向}
\label{Adaptation-actions}

TransBBS利用者は,メッセージを投稿する前に,自分のメッセージを書き換え
て,自分の翻訳されたメッセージを他の参加者により理解しやすくする.本稿
では,この一つのメッセージを書き換える過程を「書き換え過程」と定義し,
メッセージの1回の書き換えを「書き換え」と呼ぶことにする.メッセージは
書き換え過程後,投稿される.書き換えには,機械翻訳システムの質の不十分
さを補うためのもの(「適応」と呼ぶことにする)とメッセージの内容を充実
させるためのものがある.本論文では,前者のみを分析の対象とした.

一つのメッセージの1回の書き換えで,「主語の補完」,「名詞句の置き換え」
など複数の部分的な書き換えが起こることがある.この部分的な書き換え一つ
を「リペア」と呼ぶことにする.日本語が原文である場合の典型的なリペアの
例を例\ref{ex:common-repairs}に示す.

\vspace*{2mm}
\begin{tabular}{cll}
(\example{ex:common-repairs})
& 書き換え前:   & 翻訳が\underline{復旧する}とき,\\
&                & 私は皆様にそれを伝えます. \\
& 書き換え後:   & 翻訳が\underline{直る}とき,\\
&                & 私は皆様にそれを伝えます.\\
&                &   \underline{Sanny.} \underline{ごめんなさい.}
\end{tabular}
\vspace*{2mm}

「復旧する」という述語を「直る」という同義の述語に書き換えるリペアによ
り,訳語の選択誤りに適応している.また,「Sanny.ごめんなさい.」とい
う文を追加するリペアにより,謝罪の気持ちを表現し,メッセージの内容を充
実することが行われている.

リペアは,「置き換え」,「追加」,「削除」に分類することができる.さら
に,何が書き換えられているのかによって細分することができる.本論文では,
日本語原文のメッセージに関する適応について中心的に分析を行った.韓国語
原文と中国語原文のメッセージに関する適応についても分析を行った.

\subsection{日本語原文における利用者適応の傾向}

表\ref{Table:Repairs_in_Japanese1},表\ref{Table:Repairs_in_Japanese2},
表\ref{Table:Repairs_in_Japanese3}に
日本人のコラボレーション実験参加者が行った機械翻訳への適応についてまと
めた.
表\ref{Table:Repairs_in_Japanese1},表\ref{Table:Repairs_in_Japanese2},
表\ref{Table:Repairs_in_Japanese3}は,
67のメッセージ(メッセージ当たりの平均文数:3.7,メッセージ当たりの平均
文字数:74.6,文当たりの平均文字数:20.4)に関する書き換え過程を調査し,
どのような適応リペアがどれだけ出現したかを示している.
表\ref{Table:Repairs_in_Japanese1}が,「置き換え」によるリペアで
あるが,183件(69.6\%)あった.
表\ref{Table:Repairs_in_Japanese2}が,「追加」によるリペアで,
39件(14.8\%),
表\ref{Table:Repairs_in_Japanese3}が,「削除」によるリペアで,
41件(15.6\%)であった.
リペアは基本的に「置き換え」によって行われるが,必要に応じて「追加」が
行われ,翻訳がうまくできておらず,かつ,その情報がそれほど重要でない場
合は,「削除」が行われていた.

「述語の置き換え」や「名詞句の置き換え」に代表される語句の置き換えは,
英訳で訳語選択に失敗しているような場合に,同義や類義の語句に置き換える
ことにより,適切な翻訳を得られるように適応したものである.36件(13.7\%)
と比較的頻度の多い「文の置き換え」は,うまく翻訳できなかった文を別の言
い方で言い換えてみるリペアである.「文末表現の置き換え」は,うまく翻訳
できない助詞,助動詞,補助動詞などによる複雑な様相表現を,細かいニュア
ンスの伝達は諦めて,例\ref{ex:compact}のように簡潔な表現にすることによ
り,リペアを行っていた.

\vspace*{2mm}
\begin{tabular}{cll}
(\example{ex:compact})
& リペア前:   & タイトルは英語で\underline{書くようにして下さい}. \\
& リペア後:   & タイトルは英語で\underline{書いて下さい}.
\end{tabular}
\vspace*{2mm}

長文は係り受けの曖昧性などが指数的に増えるので,機械翻訳が難しく訳文の
質が悪化する.この問題に対処するため,「文の分割」という適応が起きてい
た.また,「語順の変更」を行うことにより,係り受けを正しく認識させるこ
とにより,正しく翻訳させようとする適応も行われていた.

「追加」で最も多いのは,「主語の補完」であった.日本語では,文脈などか
ら推測可能で省略できる主語が,英語では必要になる場合がある.現在の翻訳
システムでは適切に主語を補って翻訳することが難しいので,良い英訳にする
ためにはこのような補完の適応が必要になる.

「削除」で最も多いのは,「副詞表現の削除」であった.副詞表現はしばしば,
日本語と英語の表現方法が異なったり,句として一まとまりとして扱う必要が
あったり,訳語選択が動詞や名詞に比べて難しかったりすることにより翻訳が
難しい.また,副詞は,動詞や名詞のような文の骨格情報ではないので,うま
く翻訳できない場合は,削除してしまうことがあった.

これらのリペアは,通常,英語の翻訳の良し悪しを見て行われる.しかし,利
用者は機械翻訳がどのような仕組みで翻訳しているか知っているわけではない
ので,適応のプロセスは試行錯誤のプロセスである.試してみてうまく行かな
い場合は,別の適応を行うかそこで適応を諦めることになる.

どのようなリペアが行われているかを詳細に調査することにより,利用者はど
のような翻訳の問題には適応することができ,どのような問題には適応できな
かったかを明らかにすることができた.利用者は,単純な語や句の書き換え,
単純な文の言い換え,文の簡単化を行うことはできるが,機械翻訳が不得意と
する特殊な構文(例えば,「リンゴは赤い\underline{の}が好きだ.」のよう
な形式名詞「の」や「もの」を含む構文)の回避や抽象的な表現(例えば,
「新しい\underline{方}が良い.」のような、抽象的な「方」や「もの」を含
む表現)の回避などは行うことはできなかった.

\begin{table}[htbp]
    \leavevmode
\small
    \caption{日本語におけるリペア(置き換え)}
    \label{Table:Repairs_in_Japanese1}
    \vspace*{-3mm}
\begin {center}
\begin{tabular}{|l|r|r|}
\hline
\multicolumn{1}{|c|}{\bf リペア} & 
\multicolumn{1}{c|}{\bf 頻度} & 
\multicolumn{1}{c|}{\bf 比率} \\
\hline 
名詞句の置き換え             & 33 & 12.5\% \\ 
述語の置き換え               & 21 &  8.0\% \\ 
文末表現の置き換え           & 12 &  4.6\% \\ 
副詞表現の置き換え           &  9 &  3.4\% \\ 
助詞の置き換え               &  8 &  3.0\% \\ 
接辞の置き換え               &  3 &  1.1\% \\
\hline
間投詞の置き換え             &  2 &  0.8\% \\ 
接続表現の置き換え           &  2 &  0.8\% \\ 
\hline
修飾語の置き換え             &  2 &  0.8\% \\ 
補助動詞の置き換え           &  1 &  0.4\% \\ 
\hline
文の置き換え                 & 36 & 13.7\% \\
文の分割                     & 13 &  4.9\% \\
語順の変更                   & 10 &  3.8\% \\
列挙表現の置き換え           &  6 &  2.3\% \\
箇条書きのスタイルを変更     &  6 &  2.3\% \\
\hline
準体助詞を含む表現の置き換え &  4 &  1.5\% \\
文を簡潔化し複数文を統合     &  2 &  0.8\% \\
\hline 
句点の変更                   &  5 &  1.9\% \\  
表現スタイルの変更(括弧類) &  4 &  1.5\% \\ 
半角の句読点を全角に変更     &  2 &  0.8\% \\ 
読点表記の変更               &  2 &  0.8\% \\ 
\hline
{\bf 置き換え小計}           & 183 &  69.6\% \\ 
\hline
\end{tabular}
\end{center}
    \vspace*{-3mm}
\end{table}

\begin{table}[htbp]
    \leavevmode
\small
    \caption{日本語におけるリペア(追加)}
    \label{Table:Repairs_in_Japanese2}
    \vspace*{-3mm}
\begin {center}
\begin{tabular}{|l|r|r|}
\hline
\multicolumn{1}{|c|}{\bf リペア} & 
\multicolumn{1}{c|}{\bf 頻度} & 
\multicolumn{1}{c|}{\bf 比率} \\
\hline 
主語の補完                   & 20 &  7.6\% \\
助詞の補完                   &  5 &  1.9\% \\
修飾語の補完                 &  4 &  1.5\% \\
目的語の補完                 &  2 &  0.8\% \\
副詞表現の追加               &  1 &  0.4\% \\
接辞の追加                   &  1 &  0.4\% \\
\hline
読点の追加                   &  4 &  1.5\% \\
句点の追加                   &  2 &  0.8\% \\
\hline
{\bf 追加小計}               & 39 & 14.8\% \\
\hline
\end{tabular}
\end{center}
    \vspace*{-3mm}
\end{table}

\begin{table}[htbp]
    \leavevmode
\small
    \caption{日本語におけるリペア(削除)}
    \label{Table:Repairs_in_Japanese3}
    \vspace*{-3mm}
\begin {center}
\begin{tabular}{|l|r|r|}
\hline
\multicolumn{1}{|c|}{\bf リペア} & 
\multicolumn{1}{c|}{\bf 頻度} & 
\multicolumn{1}{c|}{\bf 比率} \\
\hline 
副詞表現の削除               &  9 &  3.4\% \\
助詞の削除                   &  5 &  1.9\% \\ 
\hline
非重要情報の削除             &  4 &  1.5\% \\
主語の削除                   &  3 &  1.1\% \\
呼びかけ表現の削除           &  3 &  1.1\% \\
目的語の削除                 &  1 &  0.4\% \\
トピックの削除               &  1 &  0.4\% \\
\hline
文の削除                     &  8 &  3.0\% \\
準体助詞を含む表現の削除     &  1 &  0.4\% \\
\hline 
読点の削除                   &  3 &  1.1\% \\
括弧類の削除                 &  2 &  0.8\% \\
句点の削除                   &  1 &  0.4\% \\
\hline
{\bf 削除小計}               & 41 & 15.6\% \\
\hline
\end{tabular}
\end{center}
    \vspace*{-3mm}
\end{table}

\subsection{言語による利用者適応の違い}

日本人利用者による日本語を原文とする適応以外に,韓国語原文と中国語原文
における適応についても調査した.韓国語は,100メッセージ(メッセージ当
たりの平均文数:3.8,メッセージ当たりの平均文字数:99.6,文当たりの平
均文字数:26.1),中国語は,81メッセージ(メッセージ当たりの平均文数:
2.2,メッセージ当たりの平均文字数:65.6,文当たりの平均文字数:29.4)
について書き換え過程を調査した.
\footnote{リペアと書き換え過程の数は日本語原文における適応に比べて少な
い。日本語原文は67メッセージ中59メッセージ88.1\%の適応があったのに対し,
韓国語原文,中国語原文は,それぞれ,100メッセージ中18メッセージ18.0\%,
81メッセージ中15メッセージ18.5\%の適応しかなかった.これは,この実験が
日本が主催して行われたため,日本人参加者のモチベーションが高く,
翻訳の間違いに対して敏感であったのに対し,韓国人および中国人参加者は,
適応へのモチベーションが日本に比べて低かったためであると思われる.}

表\ref{Table:Repairs_in_Korean}に韓国人のコラボレーション実験参加者が
行った利用者の適応についてまとめた.「主語の補完」がなかったことを除くと,
日本語原文の適応に傾向が似ていた.
これは,日本語と韓国語が言語的に良く似ているためと思われる.

\begin{table}[htbp]
    \leavevmode
\small
    \caption{韓国語におけるリペア}
    \label{Table:Repairs_in_Korean}
    \vspace*{-3mm}
\begin {center}
\begin{tabular}{|l|r|r|}
\hline
\multicolumn{1}{|c|}{\bf リペア} & 
\multicolumn{1}{c|}{\bf 頻度} & 
\multicolumn{1}{c|}{\bf 比率} \\
\hline 
文末表現の置き換え &  4 & 18.1\% \\
述語の置き換え     &  3 & 13.6\% \\
名詞句の置き換え   &  3 & 13.6\% \\
\hline 
文の置き換え       &  2 &  9.1\% \\
文の分割           &  2 &  9.1\% \\
語順の変更         &  1 &  4.5\% \\
\hline 
{\bf 置き換え小計} & 15 & 68.2\% \\
\hline \hline 
文の削除           &  3 & 13.6\% \\
副詞表現の削除     &  2 &  9.1\% \\
句点の削除         &  2 &  9.1\% \\
\hline 
{\bf 削除小計}     &  7 & 16.7\% \\
\hline \hline
\multicolumn{1}{|c|}{\bf 合計} & 22 & 100\% \\
\hline
\end{tabular}
\end{center}
   \footnotesize 
{通常,韓国語では文の終了は句点「.」で表現するが,今回のデータでは
改行により文の終了を表していた例が多数(20例)あった.このため機械翻訳シス
テムは文の終了を認識できず不適切に翻訳された.これを訂正するため多くの「句点
の追加」の適応が行われたが,これは特殊事情と思われるので,上記集計から削除し
た.}
\end{table}

表\ref{Table:Repairs_in_Chinese}に中国人のコラボレーション実験参加者が
行った利用者の適応についてまとめた.韓国語原文の場合と異なり,日本語原
文の適応とはかなり傾向が異なっていた.これは日本語と中国語の言語族の違
いなどが関係しているものと思われる.中国語原文の適応の特徴としては,構
文解析の失敗を解消する適応が多いことが挙げられる.孤立語である中国語は
語順が重要な解析の手がかりとなるが,同じ語が動詞となったり名詞となった
りするので,名詞を動詞として解釈したり,動詞を名詞として解釈する解析誤
りが多い.このような解析誤りへの適応として,助詞「的」を追加したり,介
詞(前置詞)を削除したり,名詞句を置き換えることにより,名詞を動詞とす
る誤りを解消していた.また,「読点の追加」や「読点の削除」により解析を
変更しようとする適応もみられた.

\begin{table}[htbp]
    \leavevmode
\small
    \caption{中国語におけるリペア}
    \label{Table:Repairs_in_Chinese}
    \vspace*{-3mm}
\begin {center}
\begin{tabular}{|l|r|r|}
\hline
\multicolumn{1}{|c|}{\bf リペア} & 
\multicolumn{1}{c|}{\bf 頻度} & 
\multicolumn{1}{c|}{\bf 比率}  \\
\hline 
名詞句の置き換え   &  5 &  16.7\% \\
接続表現の置き換え &  1 &   3.3\% \\
\hline
文の置き換え       &  5 &  16.7\% \\
語順の変更         &  2 &   6.7\% \\
文の分割           &  1 &   3.3\% \\
\hline 
{\bf 置き換え小計} & 14 &  46.7\% \\
\hline \hline
修飾語の補完       &  2 &   6.7\% \\
述語の補完         &  2 &   6.7\% \\
助詞「的」の追加   &  2 &   6.7\% \\
主語の補完         &  1 &   3.3\% \\
読点の追加         &  1 &   3.3\% \\
句点の追加         &  1 &   3.3\% \\
\hline 
{\bf 追加小計}     &  9 &  30.0\% \\
\hline \hline
非重要情報の削除   &  3 &  10.0\% \\
修飾部の削除       &  2 &   6.7\% \\
介詞の削除         &  1 &   3.3\% \\
読点の削除         &  1 &   3.3\% \\
\hline 
{\bf 削除小計}     &  7 &  23.3\% \\
\hline \hline
\multicolumn{1}{|c|}{\bf 合計} & 30 & 100.0\% \\
\hline
\end{tabular}
\end{center}
\end{table}

\section{利用者適応の多言語翻訳への効果}
\label{sec:Effect}

\subsection{日本語原文における利用者適応の効果}

表\ref{Table:Effect_of_adapation_on_MMT}に日本語原文への英語訳のための
適応が,英語,韓国語,中国語の翻訳にどの程度効果があったかを調査した結
果を示す.表\ref{Table:Translation_quality}に言語ペアごとの翻訳品質の
違いを示す.評価は人間による3段階(↑:翻訳の質が良くなった,=:あま
り変わらない,↓:悪くなった)相対評価で,中国語訳,韓国語訳については,
日本語および翻訳言語に堪能な訳語のネイティブ1名と日本人1名による協同
評価,英語訳については英語の堪能な日本人1名による評価結果に基づくもの
である.

\begin{table}[htbp]
    \leavevmode
\small
    \caption{適応の多言語翻訳への効果}
    \label{Table:Effect_of_adapation_on_MMT}
    \vspace*{-3mm}
\begin {center}
\begin{tabular}{|c|r|r|r|}
\hline
\multicolumn{1}{|c|}{\bf 目標言語} & 
\multicolumn{1}{c|}{\bf ↑} & 
\multicolumn{1}{c|}{\bf =} & 
\multicolumn{1}{c|}{\bf ↓}  \\
\hline
英語   & 85.2\%(46) &  11.1\%(6) &   3.7\%(2) \\
\hline
韓国語 & 23.7\%(14) & 54.2\%(32) & 22.0\%(13) \\
\hline
中国語 & 42.4\%(25) & 47.5\%(28) &  10.2\%(6) \\
\hline 
\end{tabular}
\end{center}
    \footnotesize
      {英語は日英翻訳の翻訳システムの切り替え(TransBBSは日英
       翻訳に関して2つの翻訳システムのどちらを使用するかオプションで
       指定することが可能になっている),があった場合やネットワーク
       の混雑などの影響で翻訳結果が得られなかった場合など5件評価できな
       い場合があった.} 
\end{table}

\begin{table}[htbp]
    \leavevmode
\small
    \caption{言語ペアごとの翻訳品質}
    \label{Table:Translation_quality}
    \vspace*{-3mm}
\begin {center}
\begin{tabular}{|c|c|c|c|c|}
\hline
               & \multicolumn{4}{c|}{\bf 目標言語} \\
\cline{2-5}
{\bf 原言語} & {\bf 日本語} & {\bf 英語} & {\bf 韓国語} & {\bf 中国語}  \\
\hline
日本語 & − & ○ & ◎ & ○  \\
\hline
英語   & ○ & − & × & △  \\
\hline
韓国語 & ◎ & × & − & ×  \\
\hline
中国語 & △ & × & × & −  \\
\hline 
\end{tabular}
\end{center}
   \footnotesize  
    {市販翻訳ソフトのWebページの評価結果から.本実験では,評価ソフトも
     しくはそれと同等の性能を持つMTソフトを使用した.}
\end{table}

適応は試行錯誤の過程があるので,1回のリペアの結果を集めても適切な評価
とはならない.本稿では,書き換え過程の最後でどのような効果があったかを
調査した.「↑」は翻訳の質の向上がみられた場合,「=」は翻訳の質が同程
度の場合,「↓」は,翻訳の質が悪化した場合である.調査した67メッセージ
中適応があった59メッセージについて評価した結果である.メッセージの内
88.1\%に対して何らかの適応が行われたことになり,適応率が高かったことが
分かる. 

英語への翻訳に関しては,わずかに悪化する場合もあるが,適応の効果が大き
いことが分かる.韓国語,中国語,マレーシア語を母国語とするコラボレーショ
ン参加者は英語も参照する場合が多いので,適応はコラボレーションのために
有効に働いていたことになる.英訳のための日本語における適応は,韓国語へ
の翻訳に関しては,あまり有効ではなかった.書き換え過程の後の翻訳で品質
の悪化も22.0\%みられた.日韓翻訳は比較的翻訳品質が良く,日英翻訳より翻
訳品質が良い.そのため,英語では翻訳の質が悪い日本語でも,韓国語の訳で
は質が悪くない場合も多い.そのような場合は,適応の効果はない.また,適
応のために不自然な日本語に書き換えたり,情報を削除したりした場合は,訳
質の悪化が起こることがあった.中国語への翻訳については,効果があったが,
訳出が悪化する場合もみられる.すなわち,もともと翻訳の質が高い言語対に
対しては,それより翻訳の質の低い言語対の翻訳結果に基づくシステムへの適
応は効果が薄くなる傾向があることが確認できた.

{\bf 主なリペアの翻訳への効果}

表\ref{Table:Main_repair_Effect}に日本語に対する主なリペアがどの程度翻
訳に効果があったかを示す.これにより,どのようなリペアが原因になって訳
質が上がったり下がったりするか詳細に知ることができる.「述語の置き換え」
や「名詞句の置き換え」は,英訳には効果があるが,韓国語訳や中国語訳には,
効果がなかったことが分かる.これは,日本語と英語では,対応する単語間に
概念の差があり,適切な訳語選択を行うのが難しいのに対して,日本,韓国,
中国は同じアジア文化を持ち,対応する単語間の概念に差があまりなく,訳語
選択があまり問題にならなかった可能性がある.付録Aの「会議」を「ミーティ
ング」に書き換える「名詞句の置き換え」では,英訳は使用された文脈では不
適切な''conference''から''meeting''代わり適応の効果が見られたが,韓国
語訳や中国語訳では,もともと訳に問題が生じておらず,適応の効果は見られ
なかった.

「文末表現の置き換え」では,表現の簡潔化が行われている場合は,中国語訳
に適応の効果がみられた.例えば,複合的な様相表現を含む「有効かもしれま
せん」をより簡潔な「効果的です」に書き換えた場合,中国語訳は向上した.
「文の置き換え」は英語訳でも効果が得られない場合が半数あった.効果的な
「文の置き換え」を行うのは簡単ではないことが分かる.「文の置き換え」は
中国語訳への効果が高い.構文解析が容易な日本語に書き換えられた場合に有
効に働く傾向が見られた.また,「文の書き換え」は,悪化の危険を含むリペ
アであることが分かった.付録Bにリペアが悪影響を及ぼす例を挙げておく.

「文の分割」は,英語訳と中国語訳に効果がみられた.韓国語訳は文を分割し
ても,しなくても訳に問題が生じない場合が多かった.「語順の変更」も英語
訳でも効果が得られない場合が半数あった.付録Bの「語順の変更」の例はそ
のような場合の例となっている.韓国語訳や中国語訳ではほとんど効果が見ら
れなかった.「主語の補完」は英訳には効果があるが,韓国語訳や中国語訳に
はあまり効果がないことが分かった.英語を参照言語として行った適応は,他
言語への翻訳には効果が少なく,英語を参照言語として用いることには限界が
あることが分かった.

\begin{table}[htbp]
    \leavevmode
\small
    \caption{日本語への主なリペアの効果}
    \label{Table:Main_repair_Effect}
    \vspace*{-3mm}
\begin {center}
\begin{tabular}{|c|c|r|r|r|}
\hline
{\bf リペア} & {\bf 目標言語} & {\bf ↑} & {\bf =} & {\bf ↓}  \\
\hline
                   & 英語   & 11 &  6 &  2 \\
\cline{2-5}
述語の置き換え     & 韓国語 &  0 & 20 &  1 \\
\cline{2-5}
                   & 中国語 &  3 & 17 &  1 \\
\hline 
                   & 英語   & 16 & 10 &  1 \\
\cline{2-5}
名詞句の置き換え     & 韓国語 &  3 & 27 &  3 \\
\cline{2-5}
                   & 中国語 &  3 & 28 &  2 \\
\hline 
                   & 英語   &  7 &  2 &  2 \\
\cline{2-5}
文末表現の置き換え & 韓国語 &  1 & 10 &  1 \\
\cline{2-5}
                   & 中国語 &  3 &  8 &  1 \\
\hline 
                   & 英語   & 17 & 17 &  0 \\
\cline{2-5}
文の置き換え       & 韓国語 &  2 & 30 &  4 \\
\cline{2-5}
                   & 中国語 &  9 & 23 &  4 \\
\hline 
                   & 英語   &  9 &  1 &  1 \\
\cline{2-5}
文の分割           & 韓国語 &  2 & 11 &  0 \\
\cline{2-5}
                   & 中国語 &  5 &  7 &  1 \\
\hline 
                   & 英語   &  4 &  4 &  1 \\
\cline{2-5}
語順の変更         & 韓国語 &  0 &  9 &  1 \\
\cline{2-5}
                   & 中国語 &  2 &  6 &  2 \\
\hline 
                   & 英語   & 13 &  5 &  1 \\
\cline{2-5}
主語の補完         & 韓国語 &  2 & 27 &  1 \\
\cline{2-5}
                   & 中国語 &  2 & 16 &  2 \\
\hline 
\end{tabular}
\end{center}
    \footnotesize
      {英語は日英翻訳の翻訳システムの切り替えがあった場合やネットワーク
       の混雑などの影響で翻訳結果が得られなかった場合などがあり評価できな
       い場合があったため,韓国語・中国語に比べ評価数が少なくなっている.} 
\end{table}

\subsection{韓国語と中国語原文における利用者適応の効果}

表\ref{Table:Effect_to_Japanese}に韓国語と中国語における英訳のための適
応の日本語訳への効果を示す.韓国語や中国語から英語への翻訳のための適応
は,日本語訳への効果があった.韓国語における適応は,訳質の悪化もみられ
ず,適応の効果がみられた.日本語訳への効果は,訳質向上83.3\%,訳質悪化
0\%であった.中国語における適応は,効果があることが分かるが,訳質の悪
化もみられた.日本語訳への効果は,訳質向上60.0\%,訳質悪化13.3\%であっ
た.英訳を参照した日本語における適応は,効果が低かったのに対し,英訳を
参照した韓国語や中国語における適応は,効果が高かったことになる.この原
因としては,韓英,中英翻訳は日本語を中間言語とした翻訳であったことが考
えられる.すなわち,この場合は,英語の翻訳の質を上げるためには,日本語
の翻訳の質を上げる必要があったためと思われる.

\begin{table}[htbp]
    \leavevmode
\small
    \caption{韓国語&中国語における適応の日本語訳への効果}
    \label{Table:Effect_to_Japanese}
    \vspace*{-3mm}
\begin {center}
\begin{tabular}{|c|r|r|r|}
\hline
\multicolumn{1}{|c|}{\bf 原言語} & 
\multicolumn{1}{c|}{\bf ↑} & 
\multicolumn{1}{c|}{\bf =} & 
\multicolumn{1}{c|}{\bf ↓}  \\
\hline
韓国語 & 83.3\%(15) & 16.7\%(3) &  0.0\%(0) \\
\hline
中国語 &  60.0\%(9) & 26.7\%(4) & 13.3\%(2) \\
\hline 
\end{tabular}
\end{center}
\end{table}

韓国語における適応では,「文の削除」や「副詞表現の削除」といった「削除」
によるリペアが効果的であった.韓日翻訳では,時々,韓国語の解析失敗が起
こる.この時,日本語および英語への翻訳の質は良くない.解析に失敗する部
分が重要なものでない場合は,その部分を削除するによって,翻訳の質を改善
していた.

中国語における適応では,解析誤りに対応するリペアと「削除」によるリペア
が効果的であった.リペアによる訳質悪化は,「名詞句の置き換え」,「接続
表現の置き換え」,「修飾語の補完」などさまざまであるが,中国語解析の失
敗が訳質悪化の原因であった.

\section{機械翻訳の品質と利用者適応やコミュニケーションとの関係}

本実験では,メッセージの送付前に機械翻訳の結果を確認できる機能により適
応を行なっている.利用者が適応に利用できるのは,利用者の理解できる言語
の翻訳結果であり,今回の実験では,利用者は英語への翻訳結果を確認するこ
とにより適応を行なっていた.そのため,利用者が理解できない言語(例えば,
日本人にとっての韓国語)への翻訳の品質が高くても,それにより適応の回数
が少なくなるということはなかった.各言語から英語への翻訳の品質が向上す
れば,必要となる適応の頻度は減少しより効率良くコミュニケーションを行う
ことが期待できるが,本実験ではそれを確認するための実験は行っていない.
翻訳の品質自体がメッセージの理解に大きく影響することは,実験後の利用者
へのアンケート調査から確認されている\cite{小倉:2003IPSJ}.日本人の韓国
語原文のメッセージ理解度や韓国人の日本語原文のメッセージ理解度が他のメッ
セージ理解度に比べて高かったのは,韓日,日韓翻訳の品質が高かったためと
思われる.また,2段翻訳が利用された場合のメッセージ理解度は低く.現段
階では,2段翻訳はまだまだ実用的な利用は難しいことが明らかになった.

今回分析したデータは,ICE2002の第2トラックの第1フェーズ(ソフトウェ
ア開発フェーズ)に行われたものである.第1トラックでは,適応機能は実装
されていなかった.第1トラックでの経験が生かされているためか,機械翻訳
の難しい長文は少なく,短い文で書かれているものが多かった.また,機械翻
訳が難しい「呼びかけ」表現を含む文や箇条書き表現などは,初期の段階では
現れるが,徐々に,最初から呼びかけ部分を独立の文とすることや箇条書きを
行わないなど,利用者による学習による効率化の現象が見られた.

\section{おわりに}
\label{sec:conclusion}

機械翻訳を介した異言語間コミュニケーションにおいて,利用者がどのような
適応を行い,その適応はどの程度効果があるか示した.適応のための書き換え
は翻訳言語ペアに強く依存することが確認できた.日本語から英語への翻訳の
場合,日本語と英語の概念間の食い違いを補うための語句の置き換えや言語表
現習慣の違いを補う主語の補完などが多く観察された.日本語と韓国語のよう
に類似の言語では,それらの言語における適応の傾向が似ていることが分かっ
た.中国語における適応では,中国語が孤立言語であることに由来する単語の
多品詞から生ずる解析の品詞誤りに適応するものが多く見られた.

日本語から英語への翻訳のための適応は,書き換えの過程後,英語訳自体には
効果が大きい.メッセージ単位の評価で,訳質向上が85\%,訳質悪化が4\%で
ある.しかし,その適応は韓国語訳にはほとんど効果がなく(訳質向上23\%,
訳出悪化22\%),中国語訳への効果もそれほど多くない(訳質向上42\%,訳出
悪化10\%).これは,英語訳に基づくシステムへの適応は,英語への依存性が
高い適応が多く、言語依存性の高い適応は他の言語への翻訳には効果が薄いた
めであると思われる.韓国語や中国語から英語への翻訳のための適応は,日本
語訳への効果があった.これは,英語への翻訳が,日本語を中間言語として翻
訳していたためシステムへの適応が英語依存性の低い,日本語訳へ効果がある
適応になっていたためと考えられる.

機械翻訳を介した異言語間コミュニケーションの支援において,機械翻訳の質
が急激には向上することが期待できない現状を考えると,メッセージの理解を
共有するための相互作用性の向上\cite{石田:2003},すなわち,翻訳の質を向
上させるための翻訳システムへの利用者による適応や相手のメッセージ内容に
ついての確認,自国語翻訳だけでなく他言語(たとえば英語)訳の参照などを
容易に行える支援が重要である.利用者による適応はある程度効果があるが,
利用者に負担もかけている.適応方法に関するガイダンスの表示や,複雑な文
末表現・長文・省略の自動検知・修正候補の表示機能などが有効であると思わ
れる.また,機械翻訳がどのような表現ならば適切に翻訳でき,どのような場
合には翻訳できないかが明確であれば,利用者の適応は容易になる.機械翻訳
システムの透明性の向上も翻訳の質の向上とともに重要であることが確認でき
た.

英語を参照言語として行った適応は,日本語・英語という言語ペアに関する依
存性が高いため,他言語への翻訳には効果が薄く,英語を参照言語として用い
ることには限界があることが分かった.また,この適応方法は,英語が分から
ない人には適応ができないことになってしまう.この問題を解決するためには,
母国語のみでの適応ができ,かつ多言語翻訳をサポートする異言語間コミュニ
ケーションサポートツールが必要である.\citeA{坂本:2004}は母国語のみで
の適応方法について検討している.日本語を英語に翻訳し,その英語をさらに
日本語に翻訳する日日翻訳により,元の日本語と翻訳結果の日本語を比較する
ことにより,適応の効果を確認するという方法である.中間言語を英語以外の
韓国語や中国語にすることにより,言語依存の機械翻訳システムへの適応に対
しても,母国語のみで適応できる可能性がある.今後,このような方法で多言
語翻訳に対応し適応を支援する方法を検討する予定である.


\bibliographystyle{jnlpbbl}

\renewcommand{\baselinestretch}{}\large\normalsize

\begin{thebibliography}{}

\bibitem[\protect\BCAY{船越, 藤代, 野村, 石田亨}{船越\Jetal }{2004}]{船越:2004}
船越要, 藤代祥之, 野村早恵子, 石田亨 \BBOP 2004\BBCP.
\newblock \JBOQ 機械翻訳を用いた協調作業支援ツールへの要求条件 ---
  日中韓馬異文化コラボレーション実験からの知見\JBCQ\
\newblock \Jem{情報処理学会論文誌}, {\Bbf 45}  (1), 112--120.

\bibitem[\protect\BCAY{Funakoshi, Yamamoto, Nomura, \BBA\ Ishida}{Funakoshi
  et~al.}{2003}]{Funakoshi:2003HCI}
Funakoshi, K., Yamamoto, A., Nomura, S., \BBA\ Ishida, T. \BBOP 2003\BBCP.
\newblock \BBOQ Supporting Intercultural Collaboration for Global Virtural
  Teams\BBCQ\
\newblock In {\Bem 10th International Conference on Human-Computer Interaction
  (HCI-03)}, \lowercase{\BVOL}~4, \BPGS\ 1098--1102\ Crete, Greece.

\bibitem[\protect\BCAY{Herring}{Herring}{2004}]{Herring}
Herring, S.~C. \BBOP 2004\BBCP.
\newblock \BBOQ Computer-mediated discourse analysis: An approach to
  researching online behavior\BBCQ\
\newblock In {\Bem Designing for Virtual Communities in the Service of
  Learning}. Cambridge University Press.

\bibitem[\protect\BCAY{Hovy}{Hovy}{1999}]{Hovy}
Hovy, E.~H. \BBOP 1999\BBCP.
\newblock \BBOQ Toward Finely Differentiated Evaluation Metrics for Machine
  Translation\BBCQ\
\newblock In {\Bem Proceedings of the EAGLES Workshop on Standards and
  Evaluation}\ Pisa, Italy.

\bibitem[\protect\BCAY{石田, 林田, 野村}{石田\Jetal }{2003}]{石田:2003}
石田亨, 林田尚子, 野村早恵子 \BBOP 2003\BBCP.
\newblock \JBOQ 異文化コラボレーションに向けて
  —機械翻訳システムの相互作用性—\JBCQ\
\newblock \Jem{電子情報通信学会人工知能と知識処理研究会}, 37--41.
\newblock AI2003-25.

\bibitem[\protect\BCAY{Jarvenpaa \BBA\ Leidner}{Jarvenpaa \BBA\
  Leidner}{1998}]{Jarvenpaa}
Jarvenpaa, S.\BBACOMMA\  \BBA\ Leidner, D.~E. \BBOP 1998\BBCP.
\newblock \BBOQ Communication and Trust in Global Virtual Teams\BBCQ\
\newblock {\Bem Journal of Computer Mediated Communication}, {\Bbf 3}  (4).

\bibitem[\protect\BCAY{Miike, Hasebe, Somers, \BBA\ Amano}{Miike
  et~al.}{1988}]{Miike}
Miike, S., Hasebe, K., Somers, H., \BBA\ Amano, S. \BBOP 1988\BBCP.
\newblock \BBOQ Experiences with an On-line Translating Dialogue System\BBCQ\
\newblock In {\Bem Proceedings of the 26th Annual Meeting of the Association
  for Computational Linguistics}, \BPGS\ 155--162\ Buffalo, NY, USA.

\bibitem[\protect\BCAY{野村, 石田, 船越, 安岡, 山下}{野村\Jetal
  }{2003}]{野村:2003情報処理}
野村早恵子, 石田亨, 船越要, 安岡美佳, 山下直美 \BBOP 2003\BBCP.
\newblock \JBOQ アジアにおける異文化コラボレーション実験2002:
  機械翻訳を介したソフトウェア開発\JBCQ\
\newblock \Jem{情報処理}, {\Bbf 44}  (5), 2--10.

\bibitem[\protect\BCAY{Nomura, Ishida, Yasuoka, \BBA\ Funakoshi}{Nomura
  et~al.}{2003}]{Nomura:2003HCI}
Nomura, S., Ishida, T., Yasuoka, M., \BBA\ Funakoshi, K. \BBOP 2003\BBCP.
\newblock \BBOQ Open Source Software Development with Your Mother Language:
  Intercultural Collaboration Experiment 2002\BBCQ\
\newblock In {\Bem 10th International Conference on Human-Computer Interaction
  (HCI-03)}, \lowercase{\BVOL}~4, \BPGS\ 1163--1167\ Crete, Greece.

\bibitem[\protect\BCAY{小倉, 林, 野村, 石田}{小倉\Jetal }{2003}]{小倉:2003IPSJ}
小倉健太郎, 林良彦, 野村早恵子, 石田亨 \BBOP 2003\BBCP.
\newblock \JBOQ 目的指向の異言語間コミュニケーションにおける機械翻訳の有効性
  の分析 --- 異文化コラボレーションICE2002実証実験から ---\JBCQ\
\newblock \Jem{情報処理学会第65回全国大会}, 5\JVOL, \BPGS\ 5--315--5--318.
\newblock 2T6-4.

\bibitem[\protect\BCAY{Ogura, Hayashi, Nomura, \BBA\ Ishida}{Ogura
  et~al.}{2004}]{Ogura:2004}
Ogura, K., Hayashi, Y., Nomura, S., \BBA\ Ishida, T. \BBOP 2004\BBCP\
\newblock In {\Bem First International Joint Conference on Natural Language
  Processing (IJCNLP-04)}, \BPGS\ 596--601\ Sanya, Hainan Island, China.

\bibitem[\protect\BCAY{Othmann \BBA\ Lakhmichand}{Othmann \BBA\
  Lakhmichand}{2003}]{Othmann:2003HCI}
Othmann, N.\BBACOMMA\  \BBA\ Lakhmichand, B. \BBOP 2003\BBCP.
\newblock \BBOQ TransSMS: A Multi-Lingual SMS Tool\BBCQ\
\newblock In {\Bem 10th International Conference on Human-Computer Interaction
  (HCI-03)}\ Crete, Greece.

\bibitem[\protect\BCAY{Papineni, Roukos, Ward, \BBA\ Zhu}{Papineni
  et~al.}{2002}]{Papineni}
Papineni, K.~A., Roukos, S., Ward, T., \BBA\ Zhu, W.~J. \BBOP 2002\BBCP.
\newblock \BBOQ Bleu: a method for automatic evaluation of machine
  translation\BBCQ\
\newblock In {\Bem Proceedings of the 40th Annual Meeting of the Association
  for Computational Linguistics}, \BPGS\ 311--318\ Philadelphia, PA, USA.

\bibitem[\protect\BCAY{坂本, 野村, 石田, 井佐原, 小倉, 林, 石川開, 小谷克則,
  島津, 介弘, 畠中, 富士秀, 船越要}{坂本\Jetal }{2004}]{坂本:2004}
坂本知子, 野村早恵子, 石田亨, 井佐原均, 小倉健太郎, 林良彦, 石川開, 小谷克則,
  島津美和子, 介弘達哉, 畠中信敏, 富士秀, 船越要 \BBOP 2004\BBCP.
\newblock \JBOQ 機械翻訳システムに対する利用者適応の分析 —
  異文化コラボレーションを目指して —\JBCQ\
\newblock \Jem{電子情報通信学会人工知能と知識処理研究会}, 95--100.
\newblock AI2003-97.

\end{thebibliography}

\appendix
\small

{\bf 付録A 日本語における主なリペアの例}

\vspace*{3mm}
\begin{tabular}{ll}
\multicolumn{2}{l}{\bf <名詞句の置き換え>}   \\
書き換え前: & できるだけTransBBSの上で\underline{会議}を行って下さい.\\
書き換え後: & できるだけTransBBSの上で\underline{ミーティング}を行って下さい.\\
効果: & \underline{conference} ⇒ \underline{meeting} \\
       & ''conferece''より適切な''meeting''と翻訳できるようになった. \\
\end{tabular}

\begin{tabular}{ll}
\multicolumn{2}{l}{\bf <述語の置き換え>}   \\
書き換え前: & あなたの意見を\underline{聞かせて}下さい.\\
書き換え後: & あなたの意見を\underline{教えて}下さい.\\
効果: & Please \underline{let listen to} your opinion. ⇒ \\
       & Please \underline{teach me} your opinion. \\
       & 適切な述語を訳出できるようになった.\\
\end{tabular}

\begin{tabular}{ll}
\multicolumn{2}{l}{\bf <文末表現の置き換え>}   \\
書き換え前: & 従って,タイトルは英語で\underline{書くようにして下さい}.\\
書き換え後: & 従って,タイトルは英語で\underline{書いて下さい}.\\
効果: & Therefore please \underline{written} a title in English. ⇒ \\
       & Therefore please \underline{write} a title in English. \\
       & 
\end{tabular}

\begin{tabular}{ll}
\multicolumn{2}{l}{\bf <副詞表現の置き換え>}   \\
書き換え前: & \underline{思わずそのまま}投稿して \\
書き換え後: & \underline{すぐ}投稿して \\
効果: & \underline{sonomama of - - -} ⇒ {immediately} \\
説明: & 他の部分の影響で「思わずそのまま」の解析に失敗している.\\
       & 解析失敗が起こらない場合も「思わず」を''as it is'' \\
       & 「そのまま」''involuntarily''と二つの副詞として翻訳してしまう.\\
\end{tabular}

\begin{tabular}{ll}
\multicolumn{2}{l}{\bf <助詞の置き換え>}   \\
書き換え前: & 私は日本語の勉強\underline{から}始める必要があります.\\
書き換え後: & 私は日本語の勉強\underline{を}始める必要があります.\\
効果: & I need to begin from the study of Japanese. ⇒ \\
       & I need to begin the study of Japanese. \\ 
\end{tabular}

\begin{tabular}{ll}
\multicolumn{2}{l}{\bf <文の置き換え>}   \\
書き換え前: & \underline{ご心配に感謝します}.\\
書き換え後: & \underline{ありがとう}.\\
効果: & \underline{Worry - appreciates}. ⇒ \\
       & \underline{Thank you}. \\
\end{tabular}

\begin{tabular}{ll}
\multicolumn{2}{l}{\bf <文の分割>}   \\
書き換え前: & Bikeshは日本に来て私を助けてください! \\
書き換え後: & Bikeshは日本に来て\underline{下さい.そして,}私を助けてください! \\
効果: & Bikesh comes to Japan and Bikesh, please help me. ⇒ \\
       & Bikesh, please come to Japan. And please help me. \\
\end{tabular}

\begin{tabular}{ll}
\multicolumn{2}{l}{\bf <語順の変更>}   \\
書き換え前: & \underline{ここに}日本チームに送ったメッセージを転送します. \\
書き換え後: & 日本チームに送ったメッセージを\underline{ここに}転送します. \\
効果: & The message sent to a Japanese team \underline{here} is forwarded. 
         ⇒ \\
       & The message sent to a Japanese team is forwarded \underline{here}. \\
\end{tabular}

\begin{tabular}{ll}
\multicolumn{2}{l}{\bf <列挙表現の置き換え>}   \\
書き換え前: & \underline{一つは}RPGのようなもので, \\
             & \underline{もう一つは}スケジュール管理をしてくれるものです. \\
書き換え後: & \underline{一つ目は}RPGのようなもので, \\
             & \underline{二つ目は}スケジュール管理をしてくれるものです. \\
効果: & The thing which is a thing such as RPG - one and \\ 
       & the thing which schedule administration has \\
       & the kindness to be made one more. ⇒ \\
       & The thing which is a thing such as RPG - one and \\
       & the thing which the second performer has \\
       & the kindness to manage their schedule. \\
       & 列挙表現自体が適切に扱えていないので特に効果なし. \\
\end{tabular}

\begin{tabular}{ll}
\multicolumn{2}{l}{\bf <箇条書きのスタイルを変更>}   \\
書き換え前: & Saeko \& Naomi \underline{:} コミュニケーションコーディネーター.\\
             & Kaname \underline{:} 技術支援者.\\
書き換え後: & Saeko \& Naomi \underline{は,}
               コミュニケーションコーディネーター\underline{です}. \\
             & Kaname\underline{は,}技術支援者\underline{です}.\\
効果: & Saeko\&Naomi - - - - - - - - - - - - - - - - - - - - - - - \\
       & - - - - - - - - - - - - - - \\
       & Kaname - - - - - - - - - - - - - ⇒ \\
       & Saeko\&Naomi is a communication coordinator. \\
       & Kaname is a technologic supporter. \\
\end{tabular}

\begin{tabular}{ll}
\multicolumn{2}{l}{\bf <主語の補完>}   \\
書き換え前: & どこで話し合いをしますか? \\
書き換え後: & \underline{私たちは}どこで話し合いをしますか? \\
効果: & Where do \underline{I} talk? ⇒ \\
       & Where do \underline{we} talk? \\
\end{tabular}

\begin{tabular}{ll}
\multicolumn{2}{l}{\bf <副詞表現の削除>}   \\
書き換え前: & その時に,\underline{再度},今までのメッセージを消去します. \\
書き換え後: & その時に,今までのメッセージを消去します.\\
効果: & They erase the present message \underline{twice} by then. ⇒ \\
       & They erase the present message by then. \\
\end{tabular}

\begin{tabular}{ll}
\multicolumn{2}{l}{\bf <文の削除>}   \\
書き換え前: & 対面議論ではなく,TransBBS上での活発な議論を歓迎します. \\
書き換え後: & \\
効果: & The team welcomes an active argument on TransBBS \\ 
       & that cries for a meeting argument. ⇒ \{\} \\
\end{tabular}
\vspace*{3mm}

{\bf 付録B 日本語へのリペアが韓国語訳や中国語訳に悪影響を与える例}

\vspace*{3mm}
\begin{tabular}{ll}
\multicolumn{2}{l}{\bf <文の置き換え>}   \\
書き換え前: & 翻訳がおかしいです. \\
書き換え後: & 翻訳サービスが間違っています.\\
効果: &  A translation is amusing. \\ 
       &  A translation service is wrong. \\
       & 韓国語訳は書き換え前は正しく翻訳されていたが, \\
       & 書き換え後は「間違っています」に相当する部分が訳語選択誤りを \\
       & 引き起こし訳質が低下した.\\
       & (中国語訳は書き換え前は「おかしいです」に相当する部分が \\
       & 訳語選択誤りを引き起こしていたが,書き換え後は正しく翻訳された.) \\
\end{tabular}

\begin{tabular}{ll}
\multicolumn{2}{l}{\bf <文の置き換え>}   \\
書き換え前: & ごめんなさい. \\
書き換え後: & 私は謝罪します.\\
効果: & They are - - - times. \\ 
       & I apologize. \\
       & 中国語訳は書き換え前は正しく翻訳されていたが, \\
       & 書き換え後は「謝罪」に相当する部分が訳語選択誤りを \\
       & 引き起こし訳質が低下した.\\
       & (韓国語訳は書き換え前,書き換え後ともに正しく翻訳された.) \\
\end{tabular}

\begin{tabular}{ll}
\multicolumn{2}{l}{\bf <語順の変更>}   \\
書き換え前: & TransSMSはなぜSMSの数を制限する必要があるのですか? \\
書き換え後: & TransSMSはSMSの数をなぜ制限する必要があるのですか? \\
効果: & 英訳は書き換え前,書き換え後変化なし. \\ 
       & 中国語訳は「SMSの数(SMS的数)」の位置が変わって訳質が悪化した. \\
       & (韓国語訳も語順の変更が起こったが訳質への影響はなかった.) \\
\end{tabular}


\begin{biography}
\biotitle{略歴}
\bioauthor{小倉 健太郎}{
1978年慶應義塾大学工学部管理工学科卒業.1980年同大学大学院管理工学専攻
修士課程修了.同年,日本電信電話公社(現NTT)入社.1987年〜1990年ATR自動
翻訳電話研究所へ出向.現在,NTTサイバースペース研究所主任研究員.
機械翻訳の研究に従事.1995年人工知能学会論文賞受賞,2002年電気通信普及
財団賞(テレコム・システム技術賞)受賞.情報処理学会,電子情報通信学会,
人工知能学会,言語処理学会,計量国語学会会員.
}
\bioauthor{林 良彦}{
1981年早稲田大学理工学部電気工学科卒業.1983年早稲田大学大学院理工学研
究科博士前期課程修了.同年,日本電信電話公社入社.
2004年日本電信電話株式会社退職(退職時,NTTサイバースペース研究所主幹研
究員・グループリーダ).この間,1994年-1995年スタンフォード大学言語情報
研究センター滞在研究員.2004年より大阪大学大学院言語文化研究科教授.現
在に至る.博士(工学).自然言語処理,知的情報アクセスの研究に従事.情報
処理学会,電子情報通信学会,人工知能学会,言語処理学会,ACL, ACM
SIG-IR各会員.
}
\bioauthor{野村 早恵子}{
2003年京大大学院情報学研社会情報学専攻指導者認定退学. 同年,(独)科学技術
振興機構研究員,アジアの異文化コラボレーション実験に従事. 2004年より
UCSD認知科学部分散認知と HCI研究室ポスドク研究員.エスノグラフィック手法
を用いた,HCI分析,CMC分析に興味を持つ.情報学博士.
}
\bioauthor{石田 亨}{
1976年京都大学工学部情報工学科卒業,1978年同大学院修士課程修了. 
同年日本電信電話公社電気通信研究所入所.ミュンヘン工科大学,パリ第六大学,
メリーランド大学客員教授,NTTリサーチプロフェッサなどを経験. 
工学博士.IEEE Fellow.情報処理学会フェロー.
現在,京都大学大学院情報学研究科社会情報学専攻教授,上海交通大学客員教授,
自律エージェントとマルチエージェント研究に15年以上の経験を持つ.
現在,3次元仮想都市FreeWalk/Qの研究を行い,デジタルシティに適用を試みて
いる. また,日中韓馬の研究者と共に,異文化コラボレーション実験に取り組む.
Kluwer Journal on Autonomous Agents and Multi-AgentSystems の編集委員,
Elsevier Journal on Web Semanticsの共同編集長.
}
\end{biography}


\end{document}
