    \documentclass[japanese]{jnlp_1.4}
\usepackage{jnlpbbl_1.3}
\usepackage[dvips]{graphicx}
\usepackage{amsmath}
\usepackage{hangcaption_jnlp}
\usepackage{udline}
\setulminsep{1.2ex}{0.2ex}
\let\underline
\usepackage{array}
\usepackage{longtable}

 \usepackage{rotating}



\Volume{17}
\Number{5}
\Month{October}
\Year{2010}

\received{2009}{12}{4}
\revised{2010}{2}{26}
\rerevised{2010}{4}{23}
\accepted{2010}{5}{25}

\setcounter{page}{75}


\jtitle{小学生の作文コーパスの収集とその応用の可能性}
\jauthor{坂本 真樹\affiref{Author_1}}
\jabstract{
現在共有されている日本人の子供の書き言葉コーパスは非常に少ないが,子供の書き言葉コーパスは,日本語の使用実態の年齢別推移の分析や,子供の言葉に特徴的に現れる言語形式の分析,国語教育・日本語教育への活用など日本語研究での利用はもちろんのこと,認知発達,社会学など,さまざまな分野での応用の可能性がある.そこで本研究では,全国4,950校の小学校のWebサイトを調査し,公開されている作文について,各テキストが子供の書いたテキストであることや学年などの情報を確認の上,作文データの収集を行った.収集したテキスト総数は10,006,語数は1,234,961である.本研究では,大人よりも子供の言語使用において豊富で多様な使用が観察されると予想されるオノマトペに着目し,その学年別の使用実態の推移について調査した.その結果,オノマトペの出現率は学年が上がるにつれ減少していくことが確認できた.さらに,社会学的応用例として,子供と父母との関係性について調査し,父母とのやりとりとそれに対する子供の反応との関係性が,母親の場合の方が強いことを示し,本コーパスのさまざまな応用の可能性を示した.
}
\jkeywords{小学生の作文,コーパス,言語学,社会学}

\etitle{Corpus of Texts Composed by Japanese Elementary School Children and its Application in Linguistics and Sociology}
\eauthor{Maki Sakamoto\affiref{Author_1}} 
\eabstract{
One of the problems in corpus-based Japanese linguistics is a shortage of shared linguistic corpus written by Japanese children. Written language corpus of Japanese children shared as language resources would enable us to analyze a change of the Japanese use according to age or examine words and grammatical style characteristically used by children. Such corpus would be expected to contribute to a Japanese study or Japanese education as well as related fields such as cognitive development and sociology. Therefore, in this study we collected essays written by Japanese elementary school children shown in Websites of 265 elementary schools. As a result we collected 10,006 texts, about 1.23 million words.   Using date collected through this process, we investigated how children in each school year used mimetic words. Results showed that the amount of mimetic expressions rose to a third grader, the kinds of onomatopoeic expressions increased to a second grader, and then they were dropping. Furthermore, as a sociologically applied study, we investigated what children wrote about their parents and how they reacted to the exchanges with parents. Result showed that the reaction of a child was rich and strong in the case of mother.
}
\ekeywords{essays of elementary school children, corpus, linguistics, sociology}

\headauthor{坂本}
\headtitle{小学生の作文コーパスの収集とその応用の可能性}

\affilabel{Author_1}{電気通信大学大学院情報理工学研究科総合情報学専攻}{Department of Informatics, The University of Electro-Communications}



\begin{document}
\maketitle



\section{まえがき}

本研究では,子供の書き言葉コーパスの収集の取組みとその活用方法の可能性について述べる.

自然言語データに関する情報が詳しくまとめられている奈良先端科学技術大学院大学松本裕治研究室\cite[\texttt{http://cl.aist-nara.ac.jp/index.php}]{Web_NAIST}で情報提供されている公開ツール・データによると,現在共有されている国内の言語資源には,国立国語研究所により作成された分類語彙表,小学校・中学校・高校教科書の語彙調査データ,現代雑誌九十種の用語用字全語彙,日本経済新聞や毎日新聞・朝日新聞などの新聞記事データ,国立国語研究所で作成された現代雑誌九十種の用語用字全語彙,IPALなど各種辞書の文例集,源氏物語・徒然草や青空文庫など著作権の消滅した古い文学作品データなどが挙げられる.全て列挙することはできないものの,いずれも調査対象が教科書や新聞,雑誌,辞書,文学作品などに偏っているコーパスが多い.子供の発話資料を共有する取組みであるCHILDESには日本も参加しているものの,日本語を使う子供のコーパスは非常に少ない.子供の言葉コーパスの現状として,海外には主に

\begin{enumerate}
\item Child Language Data Exchange System (CHILDES)(英語をはじめ29ヶ国語の発話データが収められている大規模コーパス)
\item Vocabulary of First-Grade Children (MOE)(延べ286,108語,異語数6,412語の小学1年生(5歳から8歳)329名の話し言葉のデータ)
\item The Polytechnic of Wales Corpus (PoW)(6歳から12歳の児童120名より収集された約65,000語の話し言葉コーパス)
\item The Bergen Corpus of London Teenager Language (COLT)(ロンドンの13歳から17歳の少年少女の自然な会話を録音した約50万語のコーパス)
\end{enumerate}

\noindent
がある.(1)〜(4)のコーパスはどれも話し言葉コーパスであり,子供の書き言葉コーパスはほとんど存在しない.また子供の発話資料を共有する取組みであるCHILDESには日本も参加しているものの,日本においては,子供の話し言葉コーパス,書き言葉コーパスどちらもほとんど存在しない.

電子コーパスの作成においては,コンピュータに機械的にテキストを収集させる方法が一般的である.特定の年齢で使用される書き言葉の電子コーパスを作成するためには,どの年齢の人が書いたテキストなのか判断する必要があるが,コンピュータではその判断が困難である.そのため手作業によって集めざるをえず,多大な手間と労力を必要とする.これが子供の書き言葉電子コーパスがほとんど存在しない理由のひとつであると考えられる.

また,研究者が収集した子供の書き言葉資料に基づく研究結果を事例研究の域を越えて普遍的なものにするためには,その資料を共有できるようにすること,特に電子化された言語資源として公開することが必要と考えられるが,その際に立ちはだかる問題の一つとして著作権の保護がある.本研究では,Web上に公開されている作文を収集することによって子供の書き言葉コーパスの作成を行った.しかし,Web上で用例を探して見るだけでなく,その元になった文章を自分のPCにダウンロードし,ダウンロードした本人が使用するだけでなく,その資料を研究グループで複製して共有する場合は問題になる.そのため,著作権処理が必要になる.

このように子供の書き言葉コーパスの収集と利用には多大な労力と注意すべき問題があるが,日本の子供の書き言葉コーパスが言語資源として共有されれば,日本語の使用実態の年齢別推移の分析や,子供の言葉に特徴的に現れる言語形式の分析など,国語教育や日本語研究での利用はもちろんのこと,認知発達,社会学など関連分野への貢献など,さまざまな応用の可能性がある.そこで本研究では,子供の書き言葉コーパスとしてWeb上に公開されている小学生の作文データを収集し,書き言葉コーパスとしてまとめたプロセスと結果の報告を行い,そのコーパスの実用例について述べる.


\section{小学生の作文データの収集}
\vspace{-0.5\baselineskip}

\subsection{収集手順}

収集は2004年〜2005年にかけて行った.調査対象となる全国4,950校の小学校のWebサイトをひとつひとつ訪れ,それぞれのテキストが子供の書いたテキストかどうかを目で判断した.Web上でのキーワード(作文,テキストなど)検索や文部科学省のサイトの利用なども検討したが,多くの小学校が登録している点とサイトの内容により,「Yahoo!きっず」(\texttt{http://kids.yahoo.co.jp/}) の小学校カテゴリを調査対象とすることにした.「Yahoo!きっず」とは,株式会社Yahoo! Japanが運営している子供向けポータルサイトである.トップページから小学校カテゴリを選択すると,登録している全国の小学校のWebサイトに進むことができる.これらをひとつひとつ訪れ,小学校が公開している児童の作文を総当りで収集した.各作文が実際に児童の書いたテキストかどうか,何年生が書いたテキストかについては,目で判断した.(1)は収集したテキストの例である(下線は著者によるものである):
\vspace{0.5\baselineskip}

\begin{quote}
(1) 3月6日に、\underline{ぼくたちのために\mbox{「6年生を送る会」}を下級生が開いてくれました。}体育館へ入場する時、「地上の星」の音楽がかかる中、\underline{一〜五年生が拍手をして迎えてくれました。}

ぼくたちが主役なので、かなりドキドキしました。ぼくは、「去年の6年生もこんなにドキドキしよったんやなー。」と思いながら、みんなの前に6年生が並びました。

最初にゲームをしました。ゲームは校内ウォークラリーです。1〜6年生の縦割り班で、校内に隠れている5年生を見つけてゲームをクリアし、シールと言葉の書いたカードをもらいます。カードをならべて文章にするゲームです。ぼくは1回も1位になったことがないので今日こそ1位になるぞと思いました。

ゲームがスタートしました。みんないっきにスタートしました。ぼくは自分の班に「卓球室に行こう。」と言いました。さっそく5年生はカーテンにかくれていました。ここではボーリングをしました。みんな上手に転がして残り一個になりました。最後はぼくの投げる番できんちょうしました。ゆっくり転がすと全部当たって「よっしゃー!」とみんな大喜びしました。シールを一つゲットしてカードをもらいました。他にも次々とクリアしていきました。ゲームの内容は、校内のすな時計と鏡の数を当てるとか、つみ木を十個積み上げるとかです。全部クリアして体育館にいくと、ぼくらが一位でした。僕たちの班みんな大喜びでした。班で記念写真をとってもらいました。

会の最後に在校生から贈る言葉をもらいました。卒業生のそれぞれの良いところを1〜5年生に言ってもらいました。とてもうれしくて、心に残る会になりました。

(URL:\texttt{http://www.kochinet.ed.jp/osaki-e/02/6/0306okurukai/newpage9.htm})\\
\end{quote}

このテキストでは,下線部の文脈のつながりから,6年生が書いた文章であることがわかる.

このような文脈,もしくは「この作文は3年生によって書かれたものです」といったような明確な提示によって判断し,テキストをひとつずつ収集した.様々な観点からの解析の可能性を考え,Webに掲載されているデータにはあえて特別な処理は施さずにコーパスとして登録した.作成したコーパスの情報は以下の通りである.

テキストのファイル形式はプレーンテキスト (\texttt{.txt}),使用文字コードはShift JIS,使用改行コードはCR-LFである.
本文格納情報は表\ref{tbl:corpus_form}の形式によりWebページで確認された項目のみ掲載した:


量的情報は表\ref{tbl:corpus_stats_1}および表\ref{tbl:corpus_stats_2}の通りである:

\begin{table}[p]
\caption{コーパスの本文格納情報}
\label{tbl:corpus_form}
\input{04table01.txt}
\end{table}
\begin{table}[p]
\caption{コーパスの量的情報 (1)}
\label{tbl:corpus_stats_1}
\input{04table02.txt}
\end{table}
\begin{table}[p]
\caption{コーパスの量的情報 (2)}
\label{tbl:corpus_stats_2}
\input{04table03.txt}
\end{table}

なお,形態素数の算出においては全テキストに対して形態素解析を行い,全形態素数を集計した.形態素解析には日本語形態素解析システムである茶筌 (ChaSen) version 2.1 for WindowsであるWinChaを使用した.

収録語数123万語を超える本コーパスは,子供の言葉コーパスとして代表的なCOLTの約50万語と比較しても,教育研究利用価値の高いコーパスと言える.

さらに,収集されたテキストは学年別にフォルダに分類するとともに,県別にもフォルダに分類している.県別のテキスト数は表\ref{tbl:corpus_prf_num}の通りである.

以上より,国語教育・日本語研究との関係で語や文法の学年別使用実態の推移の分析のみなならず,県別データによる方言研究や性別による比較分析を行うことができる.さらに,タイトルごとに作文を分類すれば,子供が家族や未来,環境問題など社会のことについてどのように考えているかといったことについて調査するなど,さまざまな応用的研究にも役立つ可能性がある.

\begin{table}[t]
\caption{コーパスの県別テキスト数}
\label{tbl:corpus_prf_num}
\input{04table04.txt}
\end{table}



\subsection {著作権処理}

収集したデータを上述のような教育研究のために広く役立てることができるようにするために,著作権法上の問題を処理するよう取り組んだ.著者が所属している大学の知的財産部と「文化庁発行の著作権標準テキスト」によれば,収集した作文データの利用法については次の可能性がある.

まず,何も著作権処理を行わない場合,収集した作文を分析することは著作権法上問題なく,作文の引用も一定の要件(文化庁発行の著作権標準テキスト「8.著作物などの例外的な無断利用ができる場合」記載)を満たせば侵害しないとされる.しかし,作文の複製については,授業の教材としての複製などを除いては著作権(複製権)を侵害する,とされる.

本研究で収集したデータはWeb上に公開されている作文であるため,出典を明記すれば本研究論文中に引用することは問題ないと考えられる.しかし,収集した作文データを上述のような教育研究のために広く役立てるには,複製権の問題を解消する必要がある.作文の著作権者は原則作文を書いた児童本人であり,未成年者の場合その保護者に帰属する.しかし,作文をWebサイトに公開する段階で,小学校側が保護者に許可を得て,小学校長に権利が移行している可能性があると考えられた.作文を書いた児童の保護者全員と直接連絡を取ることは困難であるため,小学校長に連絡をとることを試みた.著者が所属している大学の知的財産部門と相談の上作成した文書を,作文をWeb上に公開している小学校の校長宛に郵送し,作文文章をデータベースに採録し公開及び解析に用いることを許可するかどうか,その際学校名の公開をするかどうかについての可否を尋ね,学校長名のサインを頂戴し,一部を学校に保管し一部を返送してもらった.本来作文の著作権は作文を書いた児童ないしは保護者に帰属するため,児童によって書かれた文章をホームページに公開する段階で小学校の方で保護者の同意を得ていない場合には,保護者の許可を得ていただけるよう便宜を図ってもらえるように依頼した.

コーパスに収録した10,006テキストの公開元となっている全265校の小学校長宛に上述の内容の文書を郵送した.その結果,129校から返信があり,74校からデータベースへの掲載と公開の許諾を得ることができた.掲載・公開が許可された作文を格納しているファイル数は3,706であるが,今後協力的な小学校のホームページの作文が掲載されているページを随時閲覧し,更新された作文があれば追加収集し,再度許可を得るなどのプロセスを経てより大きい言語資源としてゆく可能性があると思われる.
以下では,本コーパスを実際に使用して分析を行った実用例について述べる.



\section{言語学的利用:子供のオノマトペ学年別使用実態の推移}

\subsection{日本語のオノマトペ}

オノマトペとは,仏語の ``onomatop\'{e}e'',もしくは英語の``onomatopoeia'' に相当する.
    田守ら(田守, スコウラップ 1999)\nocite{Bk_TamAl}によると,
オノマトペはさまざまに定義されており,その定義は実に多様であるが,それらに共通している考え方は,オノマトペと考えられている語彙の形態と意味の関係が恣意的ではなく,何らかの形で音象徴的に結びついているということである.例えば「がたん」という語は,一般に現実の音を真似たものであると考えられている.「げらげら」「びりっ」「ひらひら」などの語は,それぞれ擬声語・擬音語・擬態語と呼ばれる.田守らが指摘するように,これらの概念が日本語においてオノマトペという言語範疇と実際に対応するかどうかは本来検討が必要であるが,本研究では,擬声語・擬音語・擬態語をまとめて便宜的にオノマトペと呼ぶ.

飛田ら\cite{Bk_HidAl}は,外界の物音や人間・動物の声を表現する方法はいろいろあるが,
\break
具象的な現実から抽象的な言葉に至るまでには,(1) 類似の音・声で対象の音・声を模倣する,(2) 音・声による対象の音・声の表現,(3) 「映像」による対象の音・声の表現,(4) 文字による対象の音・声の表現,(5) 擬音語というような5つの段階を踏んでいるとしている.本研究では,これらの5つのうち,(5)にあたるものを擬音語とする.また,本研究では物音と人間・動物の声を分けずに,実際に音が出ているものの表現という観点でひとまとめにし,音の出ていない表現である擬態語と対をなすものとして,擬声語を区別せずに擬音語として扱うことにした.

飛田らは,擬態語に関しても,音や声の表現とまったく同様に,外界の様子や心情の表現に,(1) 類似の様子で対象の様子を模倣する,(2) 音・声による対象の様子の表現,(3) 「映像」による対象の様子の表現,(4) 文字による対象の様子の表現,(5) 擬態語という5つの段階があるとしている.本研究では,これらの5つのうち(5)にあたるものを擬態語とした.

擬音語は外界の物音や人間・動物の声を表し,擬態語は外界の様子や心情を表すと述べてきたが,その区別はそう簡単ではない.「雨がザーザー降っているよ」という発話において,話者が軒下にいれば,確かにザーザーで表現される現実音は聞こえてくるため,擬音語ということができる.一方,建物の中からガラス窓越しに見るときなどは,ザーザーという音は聞こえなくとも,話者は激しく降る雨の様子が見えれば,「ザーザー降っている」と表現するだろう.この場合,この「ザーザー」は雨が激しく降る様子を表す擬態語ということになる.このように,元は外界の音を表す表現だったものが,その様子をも表す表現になったとき,擬音語と擬態語を区別することは容易でないばかりか,あまり意味のないことにもなる.以上のことを踏まえ,本研究では『現代擬音語擬態語用法辞典』\cite{Bk_HidAl} を参考にし,擬音語,擬態語,また擬音語とも擬態語ともとることができる語,の3つのパターンに分類し,調査・分析を行った.


\subsection{オノマトペ使用の学年別推移調査手順}

作成した小学生の作文コーパスには,各学年のテキスト数に大きな差が見られるため,オノマトペ使用の学年別推移を観察するにあたり,オノマトペの出現絶対数ではなく,出現率による比較を行った.その出現率算出の際の母数には,10,006テキストの形態素の数を用いることにした.そこで,各学年の全テキストに対して形態素解析を行い,形態素に分け,その数を集計した.形態素解析には日本語形態素解析システムである茶筌 (ChaSen) の,version 2.1 for WindowsであるWinChaを使用した.解析例を以下に示す: \\

\begin{quote}
(2) \textbf{原文}

時間がたつにつれてひじきは、ぐんぐんへっていきます。ひじきの太さも細くなっていきます。

(URL:\texttt{http://www4.i-younet.ne.jp/\~{}smihama/hijiki/hijiki10.html})

\textbf{形態素に分けた文}

時間/が/たつ/につれて/ひじき/は/、/ぐんぐん/へっ/て/いき/ます/。/ひじき/の/太/さ/も/細く/なっ/て/いき/ます/。

\textbf{形態素数}:24語 \\
\end{quote}

オノマトペを集計するにあたり,集計の対象となるオノマトペの電子データ化を行った.オノマトペには,『現代擬音語擬態語用法辞典』\cite{Bk_HidAl}の見出し語1,064語を用い,それらを手作業で打ち込んだ.またその際,擬音語をオノマトペ1,擬態語をオノマトペ2,擬音語とも擬態語ともとれる語をオノマトペ3という3つの分類情報も付加し,オノマトペ辞書を作成した.飛田らは,元は外界の音を表す表現だったものがその様子をも表わす表現になったとき,擬音語と擬態語を区別することは容易でないし意味がないとし,このような語は「〜の音や様子を表す」という記述の仕方で統一するとしている.また,音や声のみを表現する擬音語は「〜の音(声)を表す」とし,様子や心情のみを表現する擬態語は「〜の様子を表す」としている.本研究でもこれらの記述を参照し,下例のように各表現を分類した:

\begin{quote}
\textbf{オノマトペ1}:擬音語(きーん,ぱん,ぶーん)

\textbf{オノマトペ2}:擬態語(するする,ころころ,どきどき)

\textbf{オノマトペ3}:擬音語+擬態語(がーん,どんどん,ばしばし)
\end{quote}

作成したオノマトペ辞書を用いて,コーパスからオノマトペを抽出し,出現数と種類数の集計を学年別に行った.オノマトペは,ひらがなとカタカナで表現されるが,その二つの表現方法に違いが見られなかったため,同じオノマトペであれば,ひらがなとカタカナを区別せずに集計を行った.また,集計の方法としては Perlを使用した.集計の際,1文字から4文字のものに関しては,(3)や(4)のように,オノマトペでないものが抽出される可能性があった.そのため,コーパスから抽出されたオノマトペのうち,4文字以下のものについては一つ一つ目視で確認を行い,誤って抽出されたものを除去していった. \\

\begin{quote}
(3) \textbf{誤抽出例}

抽出すべき語:ぱく

私は谷口君の所へ行った時に、梅干作り用の塩がすっごくしょっ\underline{ぱく}て大粒だった。

(URL:\texttt{http://www.agri.gr.jp/kids/sakubun/2000/sakubun17.html}) 
\end{quote}
\vspace{1\baselineskip}

\begin{quote}
(4) \textbf{誤抽出例}

抽出すべき語:ガー

例えば、ぼくが大好きなハンバー\underline{ガー}は、パンの材料の小麦の八九%、牛肉は六四%を輸入している。

(URL:\texttt{http://kids.yahoo.co.jp/docs/event/sakubun2004/contest/sakubun} \\
\texttt{03.html}) 
\end{quote}


\subsection{結果}

\begin{table}[b]
\caption{オノマトペの集計結果}
\label{tbl:ono_count}
\input{04table05.txt}
\end{table}

子供の作文コーパスから抽出されたオノマトペの出現数と種類を表\ref{tbl:ono_count}に示す.ここでオノマトペの出現数,種類数とは,Aというオノマトペが4つ,Bというオノマトペが3つ抽出された場合,出現数は7,種類の数は2となる:

以上の結果から,出現数と種類の数を全形態素数で割り,それぞれ表\ref{tbl:ono_freq}に示されるように出現率を求め,図\ref{fig:ono_freq_change}から図\ref{fig:ono_freq_class}のようにグラフ化した.グラフの横軸は学年を,縦軸は出現率を表している.ただし,種類の数は,学年が上がるに従い上昇する全形態素数と比例しないため,種類の数を用いた出現率については参考程度として記載する.図\ref{fig:ono_freq_change}より,出現数は3年生まで増え,それ以降は減少していることがわかる.また4年生以降は変化が少ない.

\begin{table}[b]
\caption{オノマトペの出現率[\%]}
\label{tbl:ono_freq}
\input{04table06.txt}
\end{table}
\begin{figure}[b]
\begin{center}
\includegraphics{17-5ia4f1.eps}
\caption{オノマトペ全体の学年推移}
\label{fig:ono_freq_change}
\end{center}
\end{figure}

オノマトペの分類別での出現率をグラフ化した図\ref{fig:ono_freq_change}から顕著であるのは,擬音語としての用法(オノマトペ1)は学年を通して少なく,擬態語としての用法(オノマトペ2)が多いということである.


\begin{figure}[t]
\begin{center}
\includegraphics{17-5ia4f2.eps}
\caption{オノマトペの分類別学年推移(出現数のみ)}
\label{fig:ono_freq_class}
\end{center}
\end{figure}


\subsection{考察}

オノマトペ使用の学年別推移を観察していくにあたり,オノマトペの出現率の変化要因として,オノマトペで表されていたことが,意味の類似するオノマトペ以外の表現で表わされるようになることがあるのではないかと考えた.そのため,以下の手順により,相互に意味が類似するオノマトペと形容詞の出現率を学年別に比較することを試みた.

五感を表す形容詞34語(楠見,1995)のうち,意味が類似するオノマトペが存在する「明るい」「うるさい」「臭い」「柔らかい」など20語を抽出対象とする形容詞とした.それぞれの形容詞と意味が類似するオノマトペは,例えば「明るい」を『日本語大シソーラス』の索引で検索し,そのカテゴリ内にあるオノマトペ(「きらきら」「ぎらっ」「ぎらぎら」)とした.各形容詞に対応するオノマトペが数多く見られたため,各形容詞との AND 検索により Google 検索エンジンにかけ,検索件数が多い順にオノマトペを3個選定した.検索は2009年10月12〜20日にかけて行った.その結果,形容詞20個につき3個のオノマトペ,合計60個のオノマトペが抽出対象となった.選定された形容詞とオノマトペについては,それぞれの学年別の出現数とともに付録1に記載する.

これらの形容詞とオノマトペを学年別に分類されたコーパスから抽出し,各々の出現数を集計した.抽出には grep コマンドを使用した.また,表記ゆれ・品詞活用などに対応するために正規表現を用いて抽出した(例えば,/$(明る|あかる)(かろ|かっ|く|い|けれ|かれ)$/).得られた出力結果について目視で確認し,誤抽出・誤検索を除去しつつ集計した.その後,形容詞とオノマトペのそれぞれについて学年別に集計された総出現数を形態素数で割り,学年ごとの出現率を得た.結果は表\ref{tbl:adj_ono_freq}と図\ref{fig:adj_ono_change}に示す.

\begin{table}[b]
\caption{形容詞とオノマトペの出現率[\%]}
\label{tbl:adj_ono_freq}
\input{04table07.txt}
\end{table}
\begin{figure}[b]
\begin{center}
\includegraphics{17-5ia4f3.eps}
\caption{形容詞とオノマトペの出現率推移}
\label{fig:adj_ono_change}
\end{center}
\end{figure}

オノマトペの出現率が3年生を境に減少する原因として,オノマトペで表されていたことが,意味の類似するオノマトペ以外の表現で表わされるようになることがあるのではないかと考え,意味が類似する形容詞の出現率との比較を行った.しかし,形容詞の出現率もオノマトペと同様の推移を示し,予想通りの結果とはならなかった.

そこで次に,オノマトペの出現率の変化要因を探るため,千葉大学大学院安部朋世準教授による小学校の教科書にみられるオノマトペに関する調査報告\cite[\texttt{http://www.u-gakugei.ac.jp/} \break
\texttt{\~{}taiken/houkoku013.pdf}]{Web_Abe}をもとに考察する.

安部は,オノマトペが小学校(及び中学校)の国語教科書にどのように現れるかを調査している.平成16年度の小・中学校国語教科書全て(各5種類)を対象として調査した結果,各学年のオノマトペ総数において,最も多いのが中学1年の1,041で,小学2年は582であることから,小学2年の出現数の多さが特徴的であるとしている.安部によれば,小学2年で各教科書が「音をあらわすことば」や「かたかなで書くことば」を学習することによるものである.平成10年告示の小学校学習指導要領における国語では,第1学年・第2学年の〔言語事項〕「イ文字に関する事項」(ア)に「平仮名及び片仮名を読み、書くこと。また、片仮名で書く語を文や文章の中で使うこと。」とあるとのことである.また,『小学校学習指導要領解説国語編』には,その項目の解説として「指導に当たっては、擬声語や擬態語、外国の地名や人名、外来語などについて、文や文章の中での実際の用例に数多く触れながら、片仮名で書く語が一定の種類の語に限られることに気付くようにする。」とあるということから,各教科書でオノマトペを扱う学習が展開されていると考えられるとしている.本研究による小学生の作文にみられるオノマトペの出現率が2年生で顕著に高いという結果は,この小学生の教科書についての調査結果と整合性がある.さらに,安部は,教科書においては,物事を描写する際に小学校低学年から中学年頃までは「オノマトペを使用して描写するもの」あるいは「オノマトペを使って描写することが望ましい」と認識されており,高学年以降になると,次第に「オノマトペ以外の方法で描写するもの」あるいは「オノマトペ以外の方法で描写するのが望ましいもの」として認識されているとしている.この調査結果についても,小学生の作文においてオノマトペの出現率が高学年で減少しているという本研究の解析結果は整合性がある.本研究による作文にみられるオノマトペの出現率の解析結果は,教科書による学習指導の成果あるいは子供の書き言葉への影響を定量的に示しているといえる.

以上より,本コーパスを用いることにより小学生が使用する言語表現の学年別使用実態の推移を分析することができ,国語の学習指導の効果指標として,言語習得や国語教育へ貢献できることが示された.また,本コーパスを用いてオノマトペと共感覚比喩の一方向性仮説の関係性について言語学的観点からより詳細な分析を行った研究として,
    坂本\cite{Art_Sak}があり,
本コーパスの有効性はすでに示されつつある.


\section{社会学的利用の可能性}

\subsection{調査対象と手順}

作文には小学生が考えていることが反映されていることから,本コーパスは何らかの社会学的テーマについて小学生が考えていることを調査するためにも利用できるのではないかと考え,現代の小学生が家族についてどのような捉え方をしているのかをコーパスから探ってみることにした.本コーパスの幅広い応用の可能性を示唆したい.

2.1節に記載したが,男子が書いた全作文数は2,642,女子が書いた全作文数は3,023,男女の別の記載がない作文数が4,341の合計10,006作文である.性別(男・女・記載なし)に分類されたコーパスに対して家族に関する単語(家族語)を抽出し各々の出現数を集計した.単語の抽出には grep コマンドを使用した.表記ゆれの対応のため検索パターンにそれぞれ表\ref{tbl:search_pat}中の検索パターンとして記載している正規表現を用いた.得られる出力結果には,「母」から「酵母」「外反母趾」,「パパ」から「パパイヤ」,「はは」から「見るのははじめて」や「わははは」など,抽出対象とする家族語と関連のない語が抽出される場合がある.このような例は,目視で確認し出力結果からは除外して集計した.

\begin{table}[t]
\caption{検索パターン}
\label{tbl:search_pat}
\input{04table08.txt}
\end{table}

ただし,このようなテキストの抽出方法では,記述された文章が作文を書いた本人の家族について述べている文なのかどうかが判別できない.実際(5)や(6)の例のように,自分の家族以外の人を指しているものも多く見られたため,目視によりこのような文章を削除した.該当部分に下線を引いた. \\

\begin{quote}
(5) 私は、ミニトマトを植えました。土がぐにゃぐにゃで足がつかなかったけどついたとこもあったよ。\underline{創麻君のお母さん}とか、\underline{勇人君のお母さん}が優しく教えてくれてとっても嬉しかったです。

(URL:\texttt{http://www.agri.gr.jp/kids/sakubun/2000/sakubun13.html}) 
\end{quote}
\vspace{1\baselineskip}

\begin{quote}
(6) 帰るとき、ある家族にあって、ふくろをもっていたので「何につかうんだ?」とわたしのお父さんがききました。答えたのは、\underline{むこうのお母さん}でした。「これにホタルをつかまえてかうんだよ」と言ったので、お父さんは、どなってしまいました。

(URL:\texttt{http://www.nagano-ngn.ed.jp/higashjs/activities.html}) \\
\end{quote}

以上の手続きの結果,作文を書いた本人の家族に関して書かれたと明確に判別できた作文を格納したファイル数(以下,作文ファイル数)および男女別の内訳は,表\ref{tbl:corpus_file_num}の通りとなった.

家族について書かれている全作文中で,父母について言及されている作文とそれ以外の家族について言及されている作文との比率の差の検定を行ったところ1\%水準で有意差がみられた ($ \chi^2 = (1, N=5,665) = 46.650, p<.001 $).このことから子供にとって両親の影響が大きいことわかる.

\begin{table}[t]
\caption{家族に関する作文ファイル数内訳}
\label{tbl:corpus_file_num}
\input{04table09.txt}
\end{table}

男女ともに母親について言及している作文が最も多く,次に父親について言及している作文が多いことがわかる.男子が書いた作文に占める母親について記述した作文と父親について記述した作文の比率の差の検定を行ったところ,1\%水準で有意差がみられた ($ \chi^2 = (1, N=1,374) = 7.166, p<.001 $).同様に,女子が書いた作文に占める母親について記述した作文と父親について記述した作文の比率の差の検定を行ったところ,1\%水準で有意差がみられた ($ \chi^2 = (1, N=1,374) = 18.576, p<.001 $).また,父について記述した作文における男女の比率も,母について記述した作文における男女の比率も比較したが,5\%水準で有意差はなかった.このことから,男子と女子のどちらの方がより母親について記述しやすいかといった傾向はみられなかった.


\subsection{分析}

亀口\cite{Bk_Kam}は,小学生の標準的な家族のイメージを研究するにあたり,兄弟姉妹や祖父母などが同居しているかどうかに関わらず,親子の関係に限定して研究を行っている.前節で述べたように,作文コーパス中でも,父親と母親に関する記述を含む作文ファイル数が顕著に多い.

本研究では,家族のイメージを探る上で,子供が父親や母親とのかかわりをどのように捉えているかという点に着目した.コーパスから抽出された自分の父母について記述している作文を格納しているファイルについて,男女どちらが記述しているものか,父母と何らかのやりとりをしているかどうか,そのやりとりの相手ややりとり自体に対する自分の何らかの感情・評価や行動を記述しているかどうかという観点から分類を行った.

\begin{figure}[b]
\begin{center}
\includegraphics{17-5ia4f4.eps}
\caption{父親との関係性}
\label{fig:rel_father}
\end{center}
\end{figure}

分類したファイルについて,父親と母親それぞれとのやりとりと,子供の感情・評価や行動の共起関係を比較してみた.そして,父親と母親とのやりとりとそれに対する子供の感情・評価や行動についての関係性の視覚化を試みた.手順としては,まず,抽出された母親と父親についての記述からやりとりと子供の感情・評価や行動に関する部分を抜き出し,やりとりと子供の感情・評価や行動の要素の頻度とそれぞれの要素間の共起頻度を集計した.結果は付録2に示す.ここで,やりとりもしくは子供の感情・評価や行動の要素が記述されていなかった場合は0とした.抽出されたやりとりと子供の感情・評価や行動の関係で,一度しか出てこない事例は削除し,頻度2以上のものを視覚化する対象要素とした.ただし,例えば図\ref{fig:rel_father}において,父親が子供を「怒る」というやりとりにおいては,子供の「怖い」という感情・評価との共起頻度と「好き」という感情・評価との共起頻度はそれぞれ1であるが,「怒る」との関係から出ている矢印は2本となるため,このような場合は視覚化している.視覚化においては,まず,父親と母親それぞれとのやりとりを矢印の線でつないだ.また,やりとりは実線の円で表現した.次に,やりとりと共起関係にある子供の感情・評価や行動を同じように矢印でつないだ.子供の感情・評価や行動は点線の円で示した.やりとりに関する記述がなかった子供の感情・評価や行動については,父親や母親と直接矢印でつないだ.矢印と共起頻度の値の関係性は,図中に凡例として示した.

父親との関係性の視覚化結果を図\ref{fig:rel_father},母親との関係性の視覚化結果を図\ref{fig:rel_mother}に示す.父親と母親両者との関係で共通して重要なやりとりは,「言う」であることがわかる.ただし,後でも考察するが,母親の場合の方がより強い結びつきが示されている.

\begin{figure}[t]
\begin{center}
\includegraphics{17-5ia4f5.eps}
\caption{母親との関係性}
\label{fig:rel_mother}
\end{center}
\end{figure}

まず,父親との関係性の視覚化結果を中心に考察する.父親との関係において興味深いこととしては,母親とのやりとりでは挙げられていない「遊ぶ」というやりとりの要素について,「楽しい」,「嬉しい」,「おもしろい」という子供の感情が結びついていることがわかる.

父親と遊ぶことは子供にとって重要なコミュニケーションであることがうかがえる.亀口\cite{Bk_Kam}も,父母ともに子供との絆の形成にはスキンシップが重要であることを述べており,中でも父親においては,やや乱暴に子供を宙に投げて受け止め,肩車をするなど,子供を興奮させて喜ばせる類の筋肉を使う身体接触が大切であるとしている.母親の場合には,「しゃべる」,「聞く」といった言語的なやりとりに対して「楽しい」という感情が結びついているのに対し,父親の場合には,「言う」,「教える」といった言語的なやりとりに対しては「楽しい」という感情が結びついていない.また,「遊ぶ」というやりとりは,「ビデオに撮ってもらう」といった「楽しい」という感情と結びついているその他のやりとりよりも強く「楽しい」という感情と結びついている.このような本研究の分析結果は,亀口の考察を裏付けるものといえる.

次に,図\ref{fig:rel_mother}に示す母親との関係性の視覚化結果をもとに考察する.全体として,母親とのやりとりは多様であり,その多様なやりとりから子供が何らかの反応を示していることがわかる.特に,母親の「言う」という行為は子供の作文に非常に多くみられ,数値的にも,「言う」は子供と母親とのやりとり全体の約25\%を占めている.このことは,子供にとって母親の発言が大きな影響力をもっていることを示唆している.さらに,この「言う」という母親とのやりとりに対して,「うれしい」という子供の感情が非常に強く結びついている.「言う」は他の様々な子供の感情,評価,行動とも関係性があることも確認され,子供にとって母親との会話が重要であることが示された.また,「嬉しい」という子供の感情は,母親とのやりとりの様々な行為と結びついていることもわかる. 

父親との関係性と母親との関係性を比較してみると,やりとりの多様性は母親の場合の方が多く,父親が16種類であるのに対し,母親では25種類ある.また,各やりとりから子供の感情,評価,行動が喚起されている場合も,父親より母親の場合の方が多く,父親では11種類なのに対し,母親では21種類に上り,ほぼ全ての母親とのやりとりに対して何らかの反応が示されている.特に,母親とのやりとりから「うれしい」という感情に結びついている場合が非常に多く,母親が何かを「言う」と「うれしい」という感情が喚起されている.このように視覚化し,全体像を比較しやすくすることにより,各要素の数や広がりが母親の方が大きく,子供との関係性の強さと多様性が大きいことを確認できた.

以上の分析結果から,小学生の作文コーパスを分析することによって,なかなか把握しにくい子供と子供をとりまく様々なもの,社会との関係,子供の心の中を垣間見ることもできることを示した.



\section{むすび}

日本の子供の書き言葉コーパスは非常に少ないという現状に着目し,本研究では,全国4,950校の小学校のWebサイトを訪問し,公開されている作文について,各テキストが子供の書いたテキストであることや学年などの情報を確認の上,作文データの収集を行った.また,2.2節で述べた著作権処理によって,収集したコーパスが教育研究のために広く利用されるようにするための取り組みを行った.

収集したコーパスの利用の可能性は多岐にわたると期待されるが,本研究では大人よりも子供の言語使用において豊富で多様な使用が観察されると予想されるオノマトペに着目し,その学年別の使用実態の推移について調査した.その結果,オノマトペの出現率は2年生が最も高く,その後は学年が上がるにつれ減少していくことが確認できた.本研究による作文にみられるオノマトペの出現率の解析結果は,教科書で用いられるオノマトペに関する先行研究による調査結果と整合性があり,小学校での国語学習指導の成果を定量的に示しているといえる.

さらに,コーパスの社会学的応用の可能性を示唆するため,子供と父母との関係性について調査し,父母とのやりとりとそれに対する子供の反応との関係性が,母親の場合の方が強いことなどを示した.

日本の子供の書き言葉を収集し分析した本研究が,言語学,教育学的研究はもちろんのこと,社会学など関連分野へ幅広く貢献できることを願う.


\acknowledgment

本稿に対して有益なご意見,ご指摘をいただきました査読者の方に感謝いたします.また,本研究の開始時に助言くださった古牧久典氏と,著者の研究室に所属し,本コーパスの収集編纂に協力してくれた児玉公人君(2004年度所属),佐々木大君(2005年度所属),水越寛太君(2007年度所属),小野正理君と清水祐一郎君(2010年度現在所属)に感謝いたします.

なお,本コーパスについてのお問い合わせは,\texttt{http://www.sakamoto-lab.hc.uec.ac.jp/}をご参照下さい.



\bibliographystyle{jnlpbbl_1.5}
\begin{thebibliography}{}

\bibitem[\protect\BCAY{安部}{安部}{\protect\unskip}]{Web_Abe}
安部朋世.
\newblock
  安部朋世「体験活動と言語教育」 第2章 小・中学校国語教科書にみられるオノマ
トペ.\
\newblock
  \Turl{http://www.u-gakugei.ac.jp/{\textasciitilde}taiken/houkoku013.pdf}.

\bibitem[\protect\BCAY{亀口}{亀口}{2003}]{Bk_Kam}
亀口憲治 \BBOP 2003\BBCP.
\newblock \Jem{家族のイメージ}.
\newblock 河出書房新社.

\bibitem[\protect\BCAY{松本}{松本}{}]{Web_NAIST}
松本裕治.
\newblock 自然言語処理学講座 奈良先端科学技術大学院大学松本裕治研究室.\
\newblock \Turl{http://cl.aist-nara.ac.jp/index.php}.

\bibitem[\protect\BCAY{坂本}{坂本}{2009}]{Art_Sak}
坂本真樹 \BBOP 2009\BBCP.
\newblock
  小学生の作文にみられるオノマトペ分析による共感覚比喩一方向性仮説再考.\
\newblock \Jem{日本認知言語学会第10回大会発表論文集}, \mbox{\BPGS\ 155--158}.

\bibitem[\protect\BCAY{田守\JBA ローレンス・スコウラップ}{田守\JBA
  ローレンス・スコウラップ}{1999}]{Bk_TamAl}
田守育啓\JBA ローレンス・スコウラップ \BBOP 1999\BBCP.
\newblock \Jem{オノマトペ—形態と意味—}.
\newblock くろしお出版.

\bibitem[\protect\BCAY{飛田\JBA 浅田}{飛田\JBA 浅田}{2001}]{Bk_HidAl}
飛田良文\JBA 浅田秀子 \BBOP 2001\BBCP.
\newblock \Jem{現代擬音語擬態語用法辞典}.
\newblock 東京堂出版.

\end{thebibliography}


\clearpage
\makeatletter
  \setcounter{section}{0}
  \setcounter{subsection}{0}
  \renewcommand{\thesection}{}
\makeatother


\section{付録}
\subsection{オノマトペの集計結果}

3.4節の考察で用いた,五感形容詞とオノマトペの集計結果を表\ref{tbl:adj_ono_num}に示す.


\subsection{視覚化のためのやりとりと子供の感情,評価,行動の分類結果}

4.2節の分析で図\ref{fig:rel_father}および図\ref{fig:rel_mother}を視覚化するための,父親ないし母親とのやりとりと子供の感情・評価や行動に関する集計結果を表\ref{tbl:rel_father}および表\ref{tbl:rel_mother}に示す.

\vspace{1\baselineskip}
\input{04table10.txt}

\clearpage
\begin{table}[t]
\begin{flushright}
\hfill\begin{minipage}{95pt}
\caption{父親の分類結果}
\label{tbl:rel_father}
\end{minipage}
\par
\scalebox{0.78}{\input{04table11-1.txt}}
\end{flushright}
\vspace*{20pt}

\hfill\begin{minipage}{95pt}
\caption{母親の分類結果}
\label{tbl:rel_mother}
\end{minipage}
\scalebox{0.78}{\input{04table12-1.txt}}
\end{table}
\clearpage


\begin{table}[t]
\vspace{18pt}
\scalebox{0.78}{\input{04table11-2.txt}}
\vspace*{20pt}


\vspace{18pt}
\scalebox{0.78}{\input{04table12-2.txt}}
\end{table}

\clearpage

\begin{biography}
\bioauthor{坂本 真樹(正会員)}{
1998年東京大学大学院総合文化研究科言語情報科学専攻博士課程修了.博士(学術).同年,同専攻助手に着任.2000年電気通信大学電気通信学部講師,2004年同学科助教授を経て,2010年より同大学大学院情報理工学研究科総合情報学専攻准教授.日本認知言語学会,日本認知科学会,電子情報通信学会,Cognitive Science Society等各会員.
}
\end{biography}


\biodate



\end{document}


