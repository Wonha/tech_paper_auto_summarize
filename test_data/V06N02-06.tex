




\documentstyle[epsf,jnlpbbl]{jnlp_j_b5}

\setcounter{page}{97}
\setcounter{巻数}{6}
\setcounter{号数}{2}
\setcounter{年}{1999}
\setcounter{月}{1}
\受付{1998}{4}{3}
\採録{1998}{7}{27}

\setcounter{secnumdepth}{2}

\title{音声認識用言語モデルのための\\ タスク適応化と定型表現の利用}

\author{中川 聖一\affiref{TUT} \and 赤松 裕隆\affiref{TUT} \and 西崎 博光\affiref{TUT}}

\headauthor{中川聖一・赤松裕隆・西崎博光}
\headtitle{音声認識用言語モデルのための\\タスク適応化と定型表現の利用}

\affilabel{TUT}{豊橋技術科学大学工学部情報工学系}
{Dept. of Information and Computer Sciences, Faculty of Eng., Toyohashi University of Technology}

\jabstract{
本研究では大規模コーパスが利用可能な新聞の読み上げ音声の認識のための精度
の良い言語モデルの構築を実験的に検討した.N-gram言語モデルの改善を目指し,
以下\mbox{の3つの点に注目した}.まずN-gram言語モデルはタスクに依存するので,タ
スクに関する大量のデータベースを用いて構築される必要があることに注目し, 
共通の大量データベースによる言語モデルをもとに,同一ジャンルの過去の記事
を用いるタスク\mbox{適応化の方法とその有効性を示す}.次に,新聞記事は話題が経時
的に変化するので,数日間〜数週間の直前の記事内容で言語モデルの適応化を行
\mbox{なう方法とその有効性を}\mbox{示す.最後}に新聞テキストには,使用頻度の高い(特殊)
表現や,固定的な言い回しな\mbox{どの表現(以下},定型表現と呼ぶ)が多いことに注目
し,複数形態素から成る定型表現を抽出し,\mbox{これを1形態素として捉えた上で},
\mbox{N-gram言語モデルを構築する方法を検}討し,有用性を示す.}


\jkeywords{音声認識,言語モデル,N-gram,タスク適応化,定型表現}


\etitle{
A task adaptation method\\
and use of idiomatic expression\\
of stochastic language model \\for speech recognition
}

\eauthor{SEIICHI NAKAGAWA\affiref{TUT} \and HIROTAKA AKAMATSU\affiref{TUT} \and HIROMITSU NISHIZAKI\affiref{TUT}}

\eabstract{
In this paper, we describe a method that constructs language models
using a task-adaptation strategy and idiomatic expressions of news
articles.  To build an effective N-gram based language model, it should
be noted that the training data must be prepared as much as possible.
However, for a given task/topic, it is very \mbox{difficult to} gather much
data.  First, we investigated the effect of a task adaptation \mbox{method of}
N-gram language model using a limited amount of target articles.
Second, we investigated the effect of the language model adaptation
method using the latest articles.  Third, we investigated the effect of
the use of idiomatic expressions as morpheme u\-nits, since some specific
expressions and idiomatic expressions are frequently observed in news
articles.  We show our proposed three methods are effective for
constructing N-gram language models.}

\ekeywords{Speech Recognition, Language Model, N-gram, Task Adaptation, Idiomatic Expression}

\begin{document}

\maketitle


\section{はじめに}
近年の著しい計算機速度の向上,及び,音声処理技術/自然言語処理技術の向
上により,音声ディクテーションシステムやパソコンで動作する連続音声認識の
フリーソフトウェアの公開など,音声認識技術が実用的なアプリケー
ションとして社会に受け入れられる可能性がでてきた\cite{test1,test2}.
我が国では,大量のテキストデータベースや音声データベースの未整備のため
欧米と比べてディクテーションシステムの研究は遅れていたが,
最近になって新聞テキストデータやその読み上げ文のデータが整備され\cite{test3},
ようやく研究基盤が整った状況である.

このような背景を踏まえ,本研究では
大規模コーパスが利用可能な新聞の読み上げ音声の精度の良い言語モデルの構築を
実験的に検討した.
音声認識のためのN-gram言語モデルでは,N=3$\sim$4で十分であると考えられる\hspace{-0.05mm}
\cite{test4,test5,test25}.
しかし,N=3ではパラメータの数が多くなり,音声認識時の負荷が大きい.
そこで,大語彙連続音声認識では,第1パス目はN=2のbigramモデルで複数候補の
認識結果を出力し,N=3のtrigramで後処理を行なう方法が一般的である.
\mbox{本研究では,第2パスのtrigramの改善}ばかりでなく,
第1パス目の\hspace{-0.05mm}bigram\hspace{-0.05mm}言語モデルの改善を目指し,
以下の3つの点に注目した.


まずタスクについて注目する.
言語モデルをN-gram\mbox{ベースで構築する場合(ルールベースで}記述するのとは異
なり),大量の学習データが必要となる.最近では各種データベースが幅広く
構築され,言語モデルの作成に新聞記事などの大規模なデータベー
スを利用した研究が行なわれている \cite{test6}.
しかし N-gram はタスクに依存するので
タスクに関する大量のデータベースを用いて構築される必要がある.
例えば,観光案内対話タスクを想定し,既存の大量の言語データに
特定タスクの言語データを少量混合することによって,
N-gram言語モデルの性能の改善が行なわれている \cite{test7}.
また,複数のトピックに関する言語モデルの線形補間で適応化する方法が
試みられている\cite{test8}.
本研究ではタスクへの適応化のために,同一ジャンルの過去の
記事を用いる方法とその有効性を示す.

次に言語モデルの経時変化について注目する.例えば新聞記事などでは話題が経
時的に変化し,新しい固有名詞が短期的に集中的に出現する場合が多い.
以前の研究では、\mbox{直前の数百単}\mbox{語による言語モデルの適応化(キャッシュ法)が試}
みられ\cite{test20},\mbox{小さいタスクでは}その有効性が示されてはいるが,本論文
では直前の数万〜数十万語に拡大する.つまり
,直前の数日間〜数週間の記事内容で言語モデルを適応化する方法
を検討し,その有効性を示す.


最後に認識単位に注目する.
音声認識において,\mbox{認識単位が短い場合認識誤りを生じやすく,}
付属語において 
その影響は大きいと考えられ,小林らは,付属語列を新たな認識単位とした場\mbox{合
の効果の検証をしている\cite{test9}}.
\mbox{また高木らは,高頻度の付属語連鎖,}関連率の高い複合名詞などを新しい認識単
位とし,\mbox{これらを語彙に加えることによる言語モデ}ルの性能に与える影響を検討
している\cite{test10}.
なお,連続する単語クラスを連結して一つの単語クラスとする方法や句を一つの単
位とする方法は以前から試みられているが,いずれも適用されたデータベースの
規模が小さい\cite{test11,test12}.
同じような効果を狙った方法として,N-gramのNを可変にする方法も試みられて
いる\cite{test8}.なお,定型表現の抽出に関する研究は,テキスト処理分野で
は多くが試みられている(例えば,新納,井佐原 1995; 北,小倉,森本,矢野 1995).

新聞テキストには,使用頻度の高い(特殊)表現や,固定的な言い回しなどの表現
(以下,定型表現と呼ぶ)が非常に多いと思われる.
定型表現は,音声認識用の言語モデルや音声認識結果の誤り訂正のための後処理
に適用できる.
そこでまず,定型表現を抽出した.次に,これらの(複数形態素から成る)定型表
現を1形態素として捉えた上で,N-gram言語モデルを構築する方法を検討する.
評価実験の結果,長さ2および3以下である定型表現を1形態素化してbigram, 
trigram言語モデルを作成することで,bigramに関しては,エントロピーが小さ
くなり,言語モデルとして有効であることを示す.

なお,これらの手法に関しては様々な方法が提案されているが,大規模のテキス
トデータを用いて,タスクの適応化と定型表現の導入の有効性を統一的に評価し
た研究は報告されていない.

\vspace*{-3mm}
\section{言語モデルの評価基準}
\vspace*{-1mm}
\subsection{エントロピーとパープレキシティ}
言語モデルの評価基準として,エントロピーとパープレキシティを
用いる.
エントロピーとパープレキシティは共に,対象とする文集合の複雑さを定量的に 
示す指標で,その文集合が複雑なほど,それぞれの値は大きくなる.

単語列を生成する情報源をモデル化したものを言語モデルと呼ぶ.いま
\mbox{言語Lにおいて,文}(単語列)$W_i = {w_1}\cdots w_{L_i}$の出現確率を$P(W_i)$
とすれば,
文集合$W_1$,$W_2$,$\cdots$,$W_N$のエントロピー\break
は次式で求められる.
\vspace*{-3mm}
\begin{equation}
 H(L) = - \sum^{N}_{i=1} P(W_i) \log P(W_i)
\end{equation}
テキスト文の連接を $W = W_1 W_2 \cdots W_N = w_1 w_2 \cdots w_T$とすれば,
テストセットのエントロピーは
\vspace*{-3mm}
\begin{equation}
  H(L) = - \log P(W)
\end{equation}
で示される.
トライグラムを用いた場合,P(W)は
\vspace*{-3mm}
\begin{eqnarray} 
 P(W) & = & P(W_1) P(W_2) \cdots P(W_N) \nonumber \\
      & = & P(w_1|*\ \#) P(w_2 |\#\ w_1) \nonumber \\
      &   & P(w_3|w_1\ w_2)\cdots P(w_T|w_{T-2}\ w_{T-1})
\end{eqnarray}
となる
(注:\#は文頭を,*は文末を示す.以降の評価実験では句読点を含む).

この時,一単語当たりのエントロピーは
\begin{equation}
 H_0(L) = - \frac{\sum_i \log P(W_i)}{\sum_i L_i}
\end{equation}

また,言語の複雑さ・パープレキシティは
\begin{equation}
 PP = 2^{H_0(L)}
\end{equation}
と定義される.

パープレキシティは,情報理論的にある単語から後続可能な単語の種類数を
表している.この値が大きくなるほど,単語を特定するのが難しくなり,
言語として複雑であるといえる.
また逆に,この値が小さくなるほど,音声認識での後続予測単語を特定するのが
やさしくなるので,認識率が上がる傾向にある\cite{test13}.

日本語の単語の定義は定かでなく,また形態素の定義も異なる.
そこで,本論文では文字単位のエントロピー(パープレキシティ)の指標も用いる.

\vspace*{-1mm}
\subsection{補正パープレキシティ}

本研究で使用したCMU SLM toolkit\cite{test14}
では語彙に含まれないものは全て一つの未知語のカテゴリにまとめられ,
語彙に含まれる形態素と等価に未知語のカテゴリは扱われる.
そのため語彙サイズのセットが小さい程(カバー率が小さい程),
パープレキシティは小さくなるということになり好ましくない.
そこで評価テキスト中に出現した未知語の種類 $m$ と,
未知語の出現回数 $n_u$ を用いてパープレキシティを補正する\cite{test15}.
補正パープレキシティは
\vspace*{-3mm}
\begin{equation}
 APP = (P(w_1...w_n)m^{-n_u})^{-\frac{1}{n}}
\end{equation}
で与えられる.これは,複数の未知語はそれぞれ等確率に生じると仮定して,
補正したものである.勿論,これは評価テキストの大きさに依存する
(テキストが大きくなると未知語の種類が増える)ので,簡易的な補正である.
より厳密には未知語に対しては出現頻度を考慮するか\cite{newapp},
未知語の生成モデルを用いる必要がある\cite{test5}.
なお,一般には,未知語部分はスキップしてパープレキシティを算出する
方法がよく使われている.

\vspace*{-1mm}
\section{言語モデルの適応化}
\vspace*{-1mm}
\subsection{面種別での学習と評価}
\vspace*{-1mm}
 タスク依存の言語モデルを構築する場合,ターゲットとするタスクに関する
 データのみを用いて学習する方がよいと考えられる\cite{test16}.

学習と評価用のコーパスとして毎日新聞の1991年$\sim$1994年の記事を用いた.
形態素解析にはRWCPが提供している毎日新聞形態素解析データを,電総研で作成
された括弧除去ツールで加工し,使用している\cite{test17}.学習には1991年
\mbox{1月から1994年11}月までの記事を用い,評価には 1994年12月の記事を用いた.毎
\mbox{日新聞には全部で13面種に分}類されているが,「社説」,「科学」,「読書」な
どの面種にはデータが少な過ぎるので,面種別の結果は省いた. 登録した形態
素数は 5000 , 20000 の 2 通りで, bigram , trigram の学習と評価に CMU
SLM toolkit を使用した.表\ref{base}に用いたコーパスの諸量をまとめた(学
\mbox{習テキストが}1994年1月〜11月の場合の結果は,文献\cite{test26}を参照された
い).

これらのデータを用いて作成したbigramとtrigramの評価結果を
表\ref{tbl:pp_bi_20k},\ref{tbl:pp_tri_20k}に示す.\mbox{紙数の関係}で,5000形態素
に関する結果は省略した.これらの表では,実験結果をエントロピーではなくパー
プレキシティで表示している.これは音声認識実験を行なうことを踏まえ,情報
理論的にある単語から後続可能な単語の種類数を示すパープキシティという指標
の方が直観的にわかりやすいためである.

またカバー率とは,unigram のヒット率のことである.
これらの結果より以下の事がわかる.

\begin{itemize}
 \item bigramとtrigramを比較すると,
       bigramより,\mbox{trigramで言語モデルを構築した方が,
       ト}レーニングデータとテストデータのどちらのパープレキシティも小さくなる.
 \item テストデータとトレーニングデータを比較すると,
       形態素数 5000 の bigram では,テストデータとトレーニングデータとの
       間にパープレキシティの差はほとんど見られなかった.
       しかし,それ以外の言語モデル(bigram 形態素数 20000, 
       trigram 形態素数 5000, trigram 形態素数 20000)
       では,テストデータとトレーニングデータとの間にパープレキシティの差が
       大きい.これは補正パープレキシティでも同様である.
       特に形態素数 20000 の trigram で差が大きい.これは,9000万形態素では,
       トレーニングデータ量が不足していることを示している.
 \item 全面種で学習した場合と面種別で学習した場合の比較をすると,
       面種別に語彙を設定する方がカバー率は向上する.
       また,テストデータのパープレキシティに関しては,
       形態素数 20000 の bigramでは全面種で学習するより面種別で学習する方が
       パープレキシティが小さくなる.
       trigramでは面種別で学習するより全面種で学習する方が
       パープレキシティが小さくなる.これは,面種別ではトレーニングデータが
       不足することによると考えられる.
       なお,テストデータの補正パープレキシティに関しては,
       形態素数 20000 の trigramでは面種別で学習するより全面種で学習する方が
       補正パープレキシティが小さくなる.
       bigramでは全面種で学習するより面種別で学習する方が
       補正パープレキシティが小さくなる.
       
       また,スポーツ面に関しては全面種で学習するより,面種別(すなわち,
       スポーツ面)で学習する方がパープレキシティが小さくなる傾向が見られる.
       これは,スポーツ面は他の面種と異なった文が多いことによる.
\end{itemize}

表には示さなかったが,形態素数 5000 の bigram に関しては,全面種での学習
では 4 年分の新聞記事で十分な学習が出来ている.
一方,面種別での学習ではトレーニングデータとテストデータのパープレキシティ
の間に差があるのでトレーニングデータの不足が見られる.
しかしトレーニングデータの不足が見られるものの,全面種で学習した言語モデ
ルより面種別で学習した言語モデルの方がテストデータのパープレキシティが小さい.
つまり,全面種で学習した言語モデルより面種別で学習した言語モデルを使用す
る方がよいことになる.
形態素数 5000 の trigram に関しては,面種別学習による効果はパープレキシティ
では見られないが,補正パープレキシティでは効果が見られる.

形態素数 20000 の trigram に関しては,
トレーニングデータとテストデータの(補正)パープレキシティの比較によって,
面種別での学習のみならず全面種での学習でもトレーニングデータ量の不足が起
きていることが分かる.
全面種で学習した言語モデルと面種別で学習した言語モデルをテストデータの(補正)
パープレキシティで比較すると,形態素数 5000 の bigram での比較とは逆に,
面種別で学習した言語モデルより全面種で学習した言語モデルの方が,
(補正)パープレキシティが小さく,面種別で学習した言語モデルを使用するより,
全面種で学習した言語モデルを使用する方がよいという結論が得られた.

以上から,形態素数 5000 の bigram を言語モデルに使用する場合は,
面種別で学習した言語モデルを用いればよいことがわかった.
しかし,最近の大語彙音声認識に用いられる形態素数は20000以上で,
また第2パスに言語モデルとして trigram を使用するのが主流となりつつある.
形態素数 20000 の trigram だと,本研究で用いたトレーニングデータ量程度で
は,面種別で学習した言語モデルを使用するより,全面種で学習した言語モデル
を使用する方がよい.そこで,タスク(新聞では,面種)依存のより精度のよい言
語モデルを構築するために全面種の記事で構築した言語モデルを,ターゲットと
するタスク(面種)に適応化する手法をとる必要がある.

\vspace*{-4mm}
{\small
\input{table_corpus.tex.euc}
\vspace*{-5mm}
}

\newpage

{\small
\input{table_bi_20k.tex.euc}
\vspace{-6mm}
\input{table_tri_20k.tex.euc}
}

\newpage

\subsection{適応化法}
 新聞記事では数日間に渡って関連のある記事が載っていることがある.
 そこで記事の評価時に,過去の数日間の記事で言語モデルを適応化しておけば,
 適応前より精度のよい言語モデルが出来ると考えられる.
 
 ここで,N-gram 言語モデルの適応化には
 MAP推定(最大事後確率推定)\cite{test7,test18,test19,test26}を用いる.
 適応化サンプルを与えた後の推定値は次式で与えられ,
 推定前の条件確率と現在与えたサンプルとの間で,
 サンプル数で重み付けされた線形補間の形になっている.

\begin{center}
\begin{equation}
 prob = \frac{\alpha \cdot N_0 \cdot prob_0 + N_1 \cdot prob_1}
             {\alpha \cdot N_0 + N_1}
\end{equation}

\small\gt
\begin{tabular}{l l}
$\alpha$ & 重み \\
$N_0$ & 標準言語モデルの総数 \\
$N_1$ & 適応化サンプルの総数 \\
$prob$ & MAP推定後の条件確率(N-gram確率) \\
$prob_0$ & 標準言語モデルでの条件確率 \\
$prob_1$ & 適応化サンプルでの条件確率 \\
\end{tabular}
\end{center}

\begin{figure}[htbp]
\begin{center}
\epsfile{file=block.eps,scale=1.1}
\end{center}
\caption{MAP推定のブロック図}
\label{fig:block}
\end{figure}


今回の実験では標準言語モデルと適応化サンプルによる言語モデルの2つを構築
しておき,バックオフを行なってスムージングした2つの条件確率を用いて 
MAP 推定を行なっている.
この過程のブロック図を図\ref{fig:block}に示す.
標準言語モデルでは,\mbox{新聞記事の全面種に対応する学習サン}プルで出現頻度の
高い形態素20000に限定した.
適応化サンプルでは語彙を限定せず,全ての形態素を語彙リストに登録した.
そのため,2つのモデルの語彙リストは独立している.


実験手順としては,{\large$\bigcirc$\hspace{-.73em}}1 \hspace{0.3mm}形態素数 20000 の標
準言語モデル(trigram)を構築し, 
{\large$\bigcirc$\hspace{-.73em}}2 \hspace{0.3mm}\mbox{標準言語モ}デルを事前モデルとして,面種
別の適応化サンプルで ターゲットタスクの言語モデルをMAP推定し,
{\large$\bigcirc$\hspace{-.73em}}3 \hspace{0.3mm}テストデータのパープレキシティを求める.
本実験では,$\alpha$を種々変えてパープレキシティが最小となる場合を求めた.


\subsection{実験結果}実験結果を表
\ref{tbl:MAP_pp_20k},\ref{tbl:MAP_app_20k}に示す.最適な$\alpha$の値は5
日間の適応化データに対してはほとんどの面種で0.01,14日間の適応化データに
対しては0.02$\sim$0.04であり,ほぼデータ量に比例した.これらの表より
 \begin{itemize}
  \item 適応化前より適応化後の方がパープレキシティが小さくなること
  \item 5日より14日間の適応化サンプルの方がパープレキシティが小さくなる
	こと
  \item 6カ月前の数日間より直前の数日間の記事での適応化の方がパープレキ
	\mbox{シティが小さくな}ること
 \end{itemize}
が分かる.通常,直前の数百単語をキャッシュとして用いて適応化する方法が効
果があると言われているが\cite{test20},これよりも大量の直前データを用い
る方が効果があるということである\cite{test8}.特に,スポーツ面において,
直前の記事による適応化の効果が大きい.これは,他の面種記事よりも特定の話
題が短期間継続するためと考えられる.

 国際面とスポーツ面で適応化サンプルの期間を5,14日,1,2,3,6カ月にして
 求めたパープレキシティと補正パープレキシティを図\ref{fig:pp2}に示す.
 \mbox{これを見ると,適応化サンプルの量が多くなるほ}ど,パープレキシティが小さく
 なること,日数が多くなるにつれてパープレキシティが飽和していく
 ことが分かる.

また,直前の適応化データと6カ月前の適応化データを比べると,
後者の場合の方がやや最適な$\alpha$の値が大きくなった.
これは直前の適応化データの方が6カ月前の適応化データよりも
有用であることを示している.

{\small
\input{table_map_20k.tex.euc}
}

\begin{figure}[htbp]
 \centering
 \epsfile{file=MAP.4year.20k.ver2.eps,scale=0.7}
 \caption{MAP推定の日数とパープレキシティの関係 (20000形態素)}
 \label{fig:pp2}
\end{figure}


\subsection{固有名詞の適応化}

前述したように,
新聞記事では数日間に渡って関連のある記事が載っていることが多い.
音声認識では特に固有名詞の扱いが重要となってくるので,固有名詞の登録法に
ついて検討した.
固有名詞はトピックに依存するものが多いので,
数日間に渡って局所的に出現する傾向があると考えられる.
そこで数日間〜数週間中に出現した固有名詞を基本語彙に追加することにより,
評価文の固有名詞をどの程度カバーすることが出来るかを調べた.

実験手順を以下に示す.

\begin{description}
\vspace{-1mm}
 \item[Step.1] 形態素数 5000,20000 の基本語彙を構築する.
\vspace{-1mm}
 \item[Step.2] 基本語彙でテストデータのカバー率を求める.
\vspace{-1mm}
 \item[Step.3] 基本語彙に数日間〜数週間の適応化サンプル中に出現した
固有名詞を高出現頻度順に追加し,固有名詞のカバー率を求める.
\end{description}

実験は,追加登録する形態素を5000に限定した場合と出現したすべてを登録する
場合を行なった.
実験結果を表\ref{tbl:Cover1},\ref{tbl:Cover2}に示す.
\mbox{表中の括弧内の数値は出現した固有名詞をすべて登録した場}合の数を示している.
この結果より次のことが言える.
\begin{itemize}
\item 6ヶ月前の記事より直前の記事に出現する固有名詞を追加する方が
      カバー率が高い.
      これより,新しく出現した固有名詞の多くは直前の数日間に渡って出現していることが分かる.
\item 追加する固有名詞の数を制限しない場合は,
      適応化サンプルが多いほどカバー率が高くなるのは当然だが,
      固有名詞の数を制限した場合でも,
      10日間より30日間の適応化サン\break
      プルを用いた方が,カバー率は少し高くなる.
\item テストデータ全体でのカバー率を見て分かるように,
      固有名詞を追加することによるカバー率の上昇は高々2\%程度である.
      このことは,基本語彙に登録されなかった単語(未知語)において,
      固有名詞の占める割合が低いことを示している
      (5000語彙に対しては約20\%,20000語彙に対しては約25\%).
\end{itemize}

\begin{table}[htbp]
 \caption{固有名詞の追加登録によるテストデータ全体でのカバー率[\%]の変化}\label{tbl:Cover1}
\begin{center}
\vspace*{1ex}
\begin{tabular}{|c|c|c|c|c|c|c|c|c|}
\hline
基本語彙 & 追加語彙 & &
 \multicolumn{3}{c|}{直前の記事で適応} &
 \multicolumn{3}{c|}{6カ月前の記事で適応} \\
\cline{4-9}
サイズ & サイズ &
 適応前 & 10日分 & 20日分 & 30日分 & 10日分 & 20日分 & 30日分 \\
\hline
 5000 & 5000     & 85.2 & 86.6 & 86.8 & 86.8 & 86.4 & 86.5 & 86.6 \\
\cline{2-9}
      & 制限なし & 85.2 & 86.8 & 87.1 & 87.2 & 86.6 & 86.9 & 87.0 \\
      &(追加語彙)&  ---  & (6860)&(11071)&(14380)&(7096) &(11353)&(14677)\\
\hline
 20000& 5000     & 95.2 & 95.7 & 95.7 & 95.8 & 95.5 & 95.5 & 95.6 \\
\cline{2-9}
      & 制限なし & 95.2 & 95.7 & 95.9 & 96.0 & 95.5 & 95.7 & 95.8 \\
      &(追加語彙)&  ---  & (5222)&(9166) &(12383)&(5403) &(9411) &(12655)\\
\hline
\end{tabular}
\end{center}
\end{table}

\begin{table}[htbp]
 \caption{固有名詞の追加登録によるテストデータ中の固有名詞についてのカバー率[\%]の変化}\label{tbl:Cover2}
\begin{center}
\vspace*{1ex}
\begin{tabular}{|c|c|c|c|c|c|c|c|c|}
\hline
基本語彙 & 追加語彙 & &
 \multicolumn{3}{c|}{直前の記事で適応} &
 \multicolumn{3}{c|}{6カ月前の記事で適応} \\
\cline{4-9}
サイズ & サイズ &
 適応前 & 10日分 & 20日分 & 30日分 & 10日分 & 20日分 & 30日分 \\
\hline
 5000 & 5000     & 41.9 & 75.0 & 78.3 & 79.2 & 69.0 & 72.1 & 74.1 \\
\cline{2-9}
      & 制限なし & 41.9 & 79.5 & 85.0 & 87.7 & 73.6 & 80.5 & 84.2 \\
      &(追加語彙)&  ---  & (6860)&(11071)&(14380)&(7096) &(11353)&(14677)\\
\hline
 20000& 5000     & 69.9 & 81.3 & 81.8 & 82.9 & 76.7 & 77.8 & 78.5 \\
\cline{2-9}
      & 制限なし & 69.9 & 81.8 & 85.7 & 87.9 & 77.5 & 81.9 & 84.9 \\
      &(追加語彙)&  ---  & (5222)&(9166) &(12383)&(5403) &(9411) &(12655)\\
\hline
\end{tabular}
\end{center}
\end{table}


なお,固有名詞に限定せずに,出現頻度の多い形態素を登録した場合の結果を
表\ref{tbl:Cover3}に示す.\mbox{表\ref{tbl:Cover3}より},登録する単語を固有名詞
\mbox{に限定しない方がカバー率は大きくなることかがわかる.し}かし,このような新
しい登録単語のbigram,trigramの算出は困難なので固有名詞に限定した方が扱い
やすいと思われる.また表\ref{tbl:Cover3}より,カバー率を98\%にするため
には直前に出現した形態素を中心とした55000形態素程度が必要なことがわかる.
但し,面種別で学習すれば,20000〜30000形態素でも十分である(表~2,3参照).

\vspace*{-1mm}
\begin{table}[htbp]
 \caption{高出現頻度の形態素の追加登録によるテストデータ全体でのカバー率[\%]の変化(品詞の限定なし)}\label{tbl:Cover3}
\begin{center}
\vspace*{-3mm}
\begin{tabular}{|c|c|c|c|c|c|c|c|c|}
\hline
基本語彙 & 追加語彙 & &
 \multicolumn{3}{c|}{直前の記事で適応} &
 \multicolumn{3}{c|}{6カ月前の記事で適応} \\
\cline{4-9}
サイズ & サイズ &
 適応前 & 10日分 & 20日分 & 30日分 & 10日分 & 20日分 & 30日分 \\
\hline
 5000 & 5000     & 85.2 & 90.8 & 91.0 & 91.1 & 90.0 & 90.3 & 90.5 \\
\cline{2-9}
      & 制限なし & 85.2 & 96.8 & 97.9& 98.4 & 96.2& 97.5& 98.1 \\
      &(追加語彙)&  ---  & (34012)&(49539)&(60665)&(34287)&(49457)&(60314)\\
\hline
 20000& 5000     & 95.2 & 96.1 & 96.2& 96.2 & 95.8 & 95.9 & 95.9 \\
\cline{2-9}
      & 制限なし & 95.2 & 97.3 & 98.0 & 98.4 & 96.9 & 97.7 & 98.1 \\
      &(追加語彙)&  ---  &(20906)&(35158)&(45966)&(21014)&(34999)&(45559)\\
\hline
\end{tabular}
\end{center}
\end{table}


\vspace*{-13mm}

\section{定型表現}

新聞テキスト文には,定型表現が多いことに着目し,これらの
高頻出定型表現を1形態素として捉えた上で,
言語モデルを構築すれば,より精度の良いモデルが出来ると
考られる.

今回の実験では,定型表現を抽出するアルゴリズムとして,
池原らの提案した方法\cite{test21}を用いる.
エントロピー基準で連語を抽出する方法も考えられるが\cite{test11,test12,test23},
今回は簡略化のため出現頻度に着目した.
どのような基準で連語を抽出し,言語モデルを構築するかは興味ある課題である
が,手法による実質的な差は少ないと思われる\cite{nakagawa}.
この方法では,最長一致の文字列抽出(ある文字列が抽出されたとき,
その文字列に含まれる部分文字列は統計量を求める際にはこの部分文字列を
定型表現とはカウントしない)を条件とし,
任意の長さ以上,任意の使用頻度以上の表現を,もれなく
自動的に抽出する.文献\cite{test21}では文字列単位で抽出していたが,これを
\mbox{形態素単位で}抽出するように変更した.抽出例を表~\ref{rei}に示す.

\vspace*{-5mm}

\begin{table}[htbp]
\centering
\caption{定型表現抽出例}
\label{rei}
\begin{tabular}{|c|l|} \hline
連語数 & 定型表現(頻度) \\ \hline
       & て/いる (318691) \\
       & は/ない (56333) \\
2      & 東京/都 (23452) \\
       & 大統領/は (14647) \\
       & 国民/の (9909) \\ \hline
      & し/て/いる (106121) \\
       & に/よる/と (24093) \\
3       & に/なっ/て (19718) \\
       & 話し/て/いる (6130) \\
       & 記者/会見/し (4297) \\ \hline
\end{tabular}
\end{table}



\subsection{標準言語モデル}

標準言語モデルは,表~\ref{base}に示した全面種の学習用データから作成した
表~\ref{tbl:pp_bi_20k}のモデルを用いる.
まず,RWC\cite{test22}の毎日新聞形態素解析結果を用いて,
出現頻度が上位20000番目までの形態素を語彙として辞書に登録した.
言語モデルの構築には,CMU SLM Toolkit Ver.1を用いた.
バックオフ・スムージングにはGood-Turing推定を用いた.


\subsection{定型表現を用いた言語モデル}

定型表現を用いた言語モデル構築のための手順を以下に示す.

\subsubsection*{Step.1 定型表現抽出}
RWCの毎日新聞形態素解析結果
に対して,定型表現抽出プログラムを実行し,連結数2または3の
定型表現を抽出する.

\subsubsection*{Step.2 頻度の計算}
定型表現を用いる前のトレーニングデータから,各形態素の頻度リ
ストを求める.上位15000番目くらいの形態素の出現頻度が50回で
あるので,\mbox{定型表現の出現頻度が50回以上のものを新}しい形態素候補として
用いることにする.

\subsubsection*{Step.3 定型表現の連結}
Step.2の定型表現を用い,トレーニングデータ内の定型表現を図~
\ref{renketu}のように
\mbox{1つの単語にま}とめる.
\vspace*{-5mm}
\begin{figure}[htbp]
\centering
\epsfile{file=mai_renketu.eps}
\caption{形態素の連結例}
\label{renketu}
\end{figure}
\vspace*{3mm}

\subsubsection*{Step.4 語彙サイズ20000の辞書作成(1回目)}
トレーニングデータから出現頻度の多い順に20000を求め,語彙サイズ20000の
辞書を作成する.このとき,上位20000の辞書に登録された定型表現は
2連結で9430個,\mbox{3連結で9357個}(このうち2連語が7010,3連語が2347)である.
登録されなかった定型表現が
\mbox{多数あるので,こ}れは未知語の数を増やすだけなので図~\ref{bun}
のようにもとの形態素
に分解しておく.

\begin{figure}[htbp]
\centering
\epsfile{file=mai_bunkai.eps}
\caption{形態素の分解例}
\label{bun}
\end{figure}

\subsubsection*{Step.5 語彙サイズ20000の辞書作成(2回目)}
分解後のトレーニングデータから,もう一度語彙サイズ20000の
辞書を作成する.これは,Step.3で定型表現を分解したことによって形態素の
出現頻度が変わってしまうためである.当\break
然ここでも登録されない定型表
現がでてくる.登録された定型表現は2連結で8944個,3連結\break
で9282個
(このうち2連語が6967,3連語が2315)になった.
ここでも,\mbox{登録されなかった定型}表現はもとの形態素に分解する.

厳密に行なうなら,辞書作成と定型表現の分解といった作業を繰り返し行ない,
辞書に登録される定型表現の数が
収束するまで行わないといけないが,今回は1回だけしか行なっていない.

\subsubsection*{Step.6 言語モデルの構築}
CMU SLM Toolkitを用いてトレーニングデータから,語彙サイズ20000の
辞書を作成し,bigram,trigram言語モデルを構築する.

\subsection{評価実験}

評価を行なう時,注意しなければならないことは,
いずれの比較対象に対しても同じ定義の
1形態素あたりのパープレキシティを求めないといけないということである.
\mbox{通常パープレキシ}ティを求める式は{\bf bigram}の場合で,
\begin{equation}
PP_0 = {}^M\sqrt{\prod_{i=1}^MP(w_i|w_{i-1})^{-1}}
\end{equation}
であるが,これは1連結形態素(定型表現として形態素を連結したもの)あたりの
パープレキシティを求めている.
形態素を連結する前の従来の1形態素あたりのパープレキシティを求めるには,
\begin{equation}
PP_1 = {}^{N}\sqrt{\prod_{i=1}^MP(w_i|w_{i-1})^{-1}}
\end{equation}
を用いなければならない.ここで

$M$:定型表現を1つの形態素としたときの連結形態素と従来の形態素の総数

$N$:定型表現を使わなかったときの従来の形態素の総数

また,定型表現は述語表現に多く現れるため,それらの形態素は
比較的短いものが多い.そのため形態素単位のパープレキシティでは
全体に及ぼす影響が大きいと考えられる.
そこで同時に文字単位のパープレキシティも求めた.

標準言語モデルと,前節に述べた方法で定型表現を用いた言語モデルを
構築し,その評価を行なった.
トレーニングデータには,標準言語モデル作成の場合と同じ,
表~\ref{base}の学習用データを用いている.
テストデータには,標準言語モデル,定型表現を用いた言語モデル
ともに表~\ref{base}の評価用データを使用した.

実験結果を表~\ref{kekka20k}と図~\ref{hyouka}に示す.
まず,bigramモデルでは,トレーニングデータに関しては\break
約半分,
テストデータに関しては約3割,パープレキシティが減少しているのが
わかる.
しかし,trigramモデルではトレーニングデータでは効果があったが,
テストデータに対しては大きな効果が得られなかった.
これは,語彙サイズを一定にしたため,定型表現を登録したために
もとの語彙から省かれた単語が未知語となったのが原因であると考えられる.
実際,定型表現を\break
用いた場合,定型表現を
用いなかった場合と比べて,
未知語の種類数が約8000個増加している.

次に,標準言語モデルを作成した時の語彙サイズ20000の辞書に,2および3連
結の定型表現をそれぞれ高出現頻度順で上位2000個,5000個分を追加した場合
の辞書で言語モデルを構築した.
その評価結果を表~\ref{kekka22k},\ref{kekka25k}に示す.
ここで表~\ref{kekka22k},\ref{kekka25k}の``定型表現の連結なし''は
\mbox{通常の形態素}を22000個および25000個用いた時の結果である.
この言語モデルの作成方法の場合でも,パープレキシティの改善が見られた.
bigramでは定型表現を用いることにより,補正パープレキシティも
大幅に減少している.
また,定型表現5000個追加のものの方が,\mbox{定型表現2000個追加の}ものと比べて,
語彙サイズが大きいのにも関わらず,パープレキシティが減少している.
これより,出来るだけ多くの定型表現を辞書に登録すれば良いということが言える.

以上より,trigramではトレーニングデータに対しては大きな効果があったが,
テストデータに対しては効果がなかった.これはトレーニングデータの不足に
よるものと考えられる.一方,bigramでは大きな効果があった.
同じパラメータ数(bigram)でも\mbox{パープレキシティが小さ}いモデルが
構築できたことは,これを大語彙連続音声認識の第1パスに使用すると
認識率の向上に繋がると考えられる.



{\small

\begin{table}[htbp]
\centering
\caption{定型表現の評価結果(語彙サイズ20000)}
\label{kekka20k}
\vspace*{-1mm}
(括弧内は文字単位のパープレキシティ)\\ 
\vspace*{1mm}
\begin{tabular}{|c|c|@{~}r@{~}r@{~}|@{~}r@{~}r@{~}|@{~}r@{~}r@{~}|@{~}r@{~}r@{~}|@{~}r@{~}r@{~}|@{~}r@{~}r@{~}|} \hline
\multicolumn{2}{|c|@{~}}{データセット} &
\multicolumn{6}{c@{~}|@{~}}{トレーニングデータ} & \multicolumn{6}{c@{~}|}{ テストデータ} \\ \hline
\multicolumn{2}{|c|@{~}}{定型表現}&
\multicolumn{2}{c@{~}|@{~}}{なし} & \multicolumn{2}{c@{~}|@{~}}{2連結} & \multicolumn{2}{c@{~}|@{~}}{3連結} &
\multicolumn{2}{c@{~}|@{~}}{なし} & \multicolumn{2}{c@{~}|@{~}}{2連結} & \multicolumn{2}{c@{~}|}{3連結} \\
\hline\hline 
bigram & PP    &  91.0 &(16.3)&  57.5 &(12.2)&  52.2 &(11.5)& 105.5 &(17.9)&  75.6 &(14.5)&  73.3 &(14.3) \\ \cline{2-14}
       & APP   & 136.7 &(20.9)& 121.3 &(19.4)& 113.1 &(18.6)& 156.0 &(22.8)& 160.8 &(23.2)& 151.6 &(22.4) \\ \hline
trigram& PP    &  29.7 & (8.1)&  20.1 & (6.4)&  13.2 & (4.9)&  61.3 &(12.8)&  65.1 &(13.3)&  55.4 &(12.0) \\ \cline{2-14}
       & APP   &  44.6 &(10.5)&  42.3 &(10.1)&  28.6 & (7.9)&  90.7 &(16.3)& 131.5 &(20.5)& 114.6 &(18.8) \\ \hline
\end{tabular}
\end{table}

\begin{table}[htbp]
\centering
\caption{定型表現の評価結果(語彙サイズ20000+2000)}
\label{kekka22k}
\vspace*{-1mm}
(括弧内は文字単位のパープレキシティ)\\ 
\vspace*{1mm}
\begin{tabular}{|c|c|@{~}r@{~}r@{~}|@{~}r@{~}r@{~}|@{~}r@{~}r@{~}|@{~}r@{~}r@{~}|@{~}r@{~}r@{~}|@{~}r@{~}r@{~}|} \hline
\multicolumn{2}{|c|@{~}}{データセット} &
\multicolumn{6}{c@{~}|@{~}}{トレーニングデータ} & \multicolumn{6}{c@{~}|}{ テストデータ} \\ \hline
\multicolumn{2}{|c|@{~}}{定型表現}&
\multicolumn{2}{c@{~}|@{~}}{なし} & \multicolumn{2}{c@{~}|@{~}}{2連結} & \multicolumn{2}{c@{~}|@{~}}{3連結} &
\multicolumn{2}{c@{~}|@{~}}{なし} & \multicolumn{2}{c@{~}|@{~}}{2連結} & \multicolumn{2}{c@{~}|}{3連結} \\
\hline\hline 
bigram & PP  &  92.7& (16.4)&  73.7& (14.3)&  75.9& (14.5)& 108.5& (18.2)&  93.9& (16.6)&  98.2 &(17.1)  \\ \cline{2-14}
       & APP & 133.4& (20.6)& 110.7& (18.3)& 113.9& (18.7)& 153.7& (22.6)& 138.9& (21.2)& 145.3 &(21.8) \\ \hline
trigram& PP  &  29.7&  (8.1)&  19.5&  (6.3)&  18.8&  (6.1)&  62.8& (13.0)&  62.8& (13.0)&  64.6 &(13.2)  \\ \cline{2-14}
       & APP &  42.7& (10.2)&  29.3&  (8.1)&  28.2&  (7.8)&  89.0& (16.1)&  91.9& (16.4)&  95.5 &(16.8)  \\ \hline
\end{tabular}
\end{table}

\begin{table}[htbp]
\centering
\caption{定型表現の評価結果(語彙サイズ20000+5000)}
\label{kekka25k}
\vspace*{-1mm}
(括弧内は文字単位のパープレキシティ)\\ 
\vspace*{1mm}
\begin{tabular}{|c|c|@{~}r@{~}r@{~}|@{~}r@{~}r@{~}|@{~}r@{~}r@{~}|@{~}r@{~}r@{~}|@{~}r@{~}r@{~}|@{~}r@{~}r@{~}|} \hline
\multicolumn{2}{|c|@{~}}{データセット} &
\multicolumn{6}{c@{~}|@{~}}{トレーニングデータ} & \multicolumn{6}{c@{~}|}{ テストデータ} \\ \hline
\multicolumn{2}{|c|@{~}}{定型表現}&
\multicolumn{2}{c@{~}|@{~}}{なし} & \multicolumn{2}{c@{~}|@{~}}{2連結} & \multicolumn{2}{c@{~}|@{~}}{3連結} &
\multicolumn{2}{c@{~}|@{~}}{なし} & \multicolumn{2}{c@{~}|@{~}}{2連結} & \multicolumn{2}{c@{~}|}{3連結} \\
\hline\hline 
bigram & PP  &  94.8 &(16.7)& 66.0 &(13.3)& 66.5 &(13.4)& 111.9& (18.5)&  89.6& (16.2)&  93.6& (16.6) \\ \cline{2-14}
       & APP & 129.6 &(20.2)& 99.1 &(17.1)& 99.9 &(17.2)& 151.4& (22.4)& 132.6& (20.6)& 138.5& (21.2) \\ \hline
trigram& PP  &  30.2 & (8.2)& 16.6 & (5.7)& 14.9 & (5.3)&  64.6& (13.2)&  63.2& (13.0)&  65.9& (13.4) \\ \cline{2-14}
       & APP &  40.5 & (9.9)& 24.9 & (7.3)& 22.4 & (6.8)&  87.5& (15.9)&  93.5& (16.6)&  97.6& (17.0) \\ \hline
\end{tabular}
\end{table}
}

\begin{figure*}[htbp]
\begin{center}
\epsfile{file=mainichi.eps,scale=0.6}
\caption{定型表現の評価結果 [注:()内の数値は補正パープレキシティを示す]}
\label{hyouka}
\end{center}
\end{figure*}


\newpage



\section{まとめ}

本研究では,
毎日新聞記事データベースを用いた過去の記事による言語モデルのタスクへの
適応化と
抽出した定型表現を用い,N-gram言語モデル
を構築する方法を検討した.

 まず言語モデルのタスクへの適応化については,実験の結果,
 6カ月前の数日間の記事より直前の数日間の記事で適応化した方が
 パープレキシティが小さくなった.
 このことは言語モデルがジャンルだけでなく時間にも依存するものであることを
 示すものである.
 ただ,適応化サンプルの量を多くするほどパープレキシティが小さくなる
 傾向があり,N-gramベースでの言語モデルを少量サンプルで適応化させる
 ことは限界があると考えられる.

次に定型表現を抽出し,これを用いたN-gram言語モデルを構築した.
定型表現を用いた言\mbox{語モデルを}
作成することで,bigramモデルに関しては,テストデータに対し約3割程度
\mbox{パープレ}キシティを低く押えるとこができ,言語モデルの有効性を示すことが
できた.
しかし,trigramではトレーニングデータの量が不十分だったた
め,トレーニングデータでは効果があったがテストデータに対しては
効果が得られなかった.トレーニングデータの量をもっと増やし,
本方法の有効性を調べる必要がある.
また,本研究では言語モデルの有効性をパープレキシティで評価したが,実際の
音声認識で確認する必要がある\cite{akamatsu}.

なお,NHKのニュース原稿に対する経時変化の適応化や定型表現の導入による言語モデルに関しては
文献\cite{nhk,test24}を参照されたい.



\begin{thebibliography}{99}

\bibitem[\protect\BCAY{赤松,中川}{赤松}{1997}]{test26}
    赤松裕隆,中川聖一(1997).
    ``新聞記事のトライグラムによるモデル化と適応化''
    言語処理学会, 第3回年次大会 D5-2,533-536.

\bibitem[\protect\BCAY{赤松,甲斐,中川}{赤松}{1998}]{akamatsu}
    赤松裕隆,甲斐充彦,中川聖一(1998).
    ``新聞・ニュース文の大語彙連続音声認識''
    情報処理学会, 音声言語情報処理 SLP-21-11, 97--104.

\bibitem[\protect\BCAY{Federico}{Federico}{1997}]{test18}
    Federico, M.(1997).
    ``Baysian Estimation Methods for N-gram Language Model Adaptation.''
    Proc.ICSLP-96, 240--243.

\bibitem[\protect\BCAY{Giachin}{Giachin}{1995}]{test11}
    Giachin, E.P.(1995).
    ``Phrase bigrams for continuous speech recognition.''
    Proc.ICASSP, 225--228.

\bibitem[\protect\BCAY{池原,白井,河岡}{池原}{1995}]{test21}
    池原悟,白井諭,河岡司(1995).
    ``大規模日本語コーパスからの連鎖型および離散型の共起表現の自動抽出法.''
    情報処理学会論文誌, Vol.36, No.11, 2584--2596.

\bibitem[\protect\BCAY{井佐原 {\em et al.}}{井佐原 {\em et al.}}{1995}]{test22}
    井佐原均,元吉文男,徳永健伸,橋本三奈子,荻野紫穂,豊浦潤,岡隆一(1995).
    ``RWCにおける品詞情報付きテキストデータベースの作成''
    言語処理学会, 第1回年次大会 B3-1, 181 -- 184.

\bibitem[\protect\BCAY{伊藤,牧野}{伊藤}{1996}]{test7}
    伊藤彰則,好田正紀(1996).
    ``対話音声認識のための事前タスク適応の検討.''
    情報処理学会, 音声言語情報処理 SLP-14-13.

\bibitem[\protect\BCAY{伊藤}{伊藤{\em et al.}}{1997}]{test3}
    伊藤克亘 {\em et al.},(1997). 
    ``大語彙日本語連続音声認識研究基盤の整備---学習・評価テキストコーパスの作成---.''
    情報処理学会, 音声言語情報処理 SLP-18-2,7--12.
 
\bibitem[\protect\BCAY{伊藤,松岡,竹沢,武田,鹿野}{伊藤}{1996}]{test17}
    伊藤克亘,松岡達雄,竹沢寿幸,武田一哉,鹿野清宏(1996).
    ``大語彙連続音声認識研究のためのテキストデータ処理.''
    日本音響学会秋季講演論文集 3-3-10, 105--106.

\bibitem[\protect\BCAY{甲斐,伊藤,山本,中川}{甲斐}{1997}]{test2}
    甲斐充彦,伊藤敏彦,山本一公,中川聖一(1997).
    ``自然な発話を対象としたパソコン/ワークステーション用連続音声認識ソフトウェア.''
日本音響学会秋季講演論文集 2-Q-30, 175--176.


\bibitem[\protect\BCAY{北,小倉,森本,矢野}{北}{1995}]{no18}
    北研二,小倉健太郎,森元逞,矢野米雄(1995).
    ``\mbox{仕事量基準を用いたコーパスからの定型表現の自}動抽出.''
    情報処理学会論文誌, Vol.34, No.9, 1937--1943.

\bibitem[\protect\BCAY{小林,今井,安藤}{小林}{1997}]{nhk}
    小林彰夫,今井亨,安藤彰男(1997).
    ``ニュース音声認識用言語モデルの学習期間の検討.''
    電子情報通信学会, 音声技報 SP97-48, 29--36.

\bibitem[\protect\BCAY{小林,中野,和田,小林}{小林}{1998}]{test9}
    小林紀彦,中野裕一郎,和田陽介,小林哲則(1998).
    ``統計的言語モデルにおける\mbox{高頻度形態素連鎖}の辞書登録に関する一考察.''
    情報処理学会, 音声言語情報処理 SLP-20-5, 33--38.

\bibitem[\protect\BCAY{Kuhn, Mori}{Kuhn}{1990}]{test20}
    Kuhn, R., Mori, R.(1990).
    ``A cache-based natural language model for speech recognition.''
    IEEE Trans Pattern Analysis and Machine Intelligence, Vol.12, No.6, 570--583.

\bibitem[\protect\BCAY{Marlin, Liermann}{Marlin, Liermann}{1997}]{test8}
    Marlin, S.C., Liermann, J.,(1997).
    ``Adaptive topic dependent language modelling using word-based varigrams.''
    Proc.EuroSpeech, 1447--1450.

\bibitem[\protect\BCAY{政瀧,松永,匂坂}{政瀧}{1995}]{test12}
    政瀧浩和,松永昭一,匂坂芳典(1995).
    ``連続音声認識のための可変長連鎖統計言語モデル.''
    電子情報通信学会, 音声技報 SP95-73, 1--6.


\bibitem[\protect\BCAY{政瀧,匂坂,久木,河原}{政瀧}{1997}]{test19}
    政瀧浩和,匂坂芳典,久木和也,河原達也(1997).
    ``MAP推定を用いたN-gram言語モデルのタスク適応.''
    電子情報通信学会, 音声技報 SP96--103, 59--64.

 
\bibitem[\protect\BCAY{松永,山田,鹿野}{松永}{1991}]{test16}
    松永昭一,山田智一,鹿野清宏(1991).
    ``音節連鎖統計情報のタスク適応化.''
    \mbox{情報処理学会第42回}全国大会(2) 6D-5, 114-115.

\bibitem[\protect\BCAY{森,山地}{森}{1997a}]{test5}
    森信介,山地治(1997). 
    ``日本語情報量の上限の推定.''
    情報処理学会論文誌, Vol.38, No.11, 2191--2199.

\bibitem[\protect\BCAY{森,山地,長尾}{森}{1997b}]{test23}
    森信介,山地治,長尾眞(1997).
    ``予測単位の変更によるn-gramモデルの改善.''
    \mbox{情報処理学会, 音}声言語情報処理 SLP-19-14, 87--94.

\bibitem[\protect\BCAY{中川}{中川}{1992}]{test13}
    中川聖一(1992).
    ``情報理論の基礎と応用.'' 
    近代科学社.

\bibitem[\protect\BCAY{中川}{中川}{1998}]{nakagawa}
    中川聖一(1998).
    ``音声認識のための統計的言語モデル.''
    日本音響学会春季講演論文集 1-6-11, 23--26.

\bibitem[\protect\BCAY{中川,赤松}{中川}{1998}]{newapp}
    中川聖一,赤松裕隆(1998).
    ``未知語を含む文集合のパープレ\mbox{キシティの算出法ー新補正パープレ}キシティ.''
    日本音響学会秋季講演論文集 2-1-13, 63--64.

\bibitem[\protect\BCAY{西村,伊藤,山崎,萩野}{西村}{1998}]{test1}
    西村雅史,伊藤伸泰,山崎一孝,萩野紫穂(1998).
    ``単語を認識単位とした日本語の\mbox{大語彙連続音声}認識.''
    情報処理学会, 音声言語情報処理 SLP-20-3,17--24.

\bibitem[\protect\BCAY{西崎,中川}{西崎}{1998}]{test24}
    西崎博光,中川聖一(1998).
    ``音声認識のための定型表現を用いた言語モデルの検討.''
    言語処理学会, 第4回年次大会 C4-3, 520-523.
 
\bibitem[\protect\BCAY{小黒,高木,橋本,尾関}{小黒}{1998}]{test10}
    小黒玲,高木一幸,橋本顕示,尾関和彦(1998).
    ``ニュース音声認識のための\mbox{言語モデルの比較.''
    日}本音響学会春季講演論文集 1-6-22, 47--48.

\bibitem[\protect\BCAY{大附,吉田,松岡,古井}{大附}{1997}]{test4}
    大附克年,吉田航太郎,松岡達雄,古井貞照(1997).
    ``高次n-gramを用いた大語彙連続音声認識の検討.''
     日本音響学会春季講演論文集 2-6-2, 47--48.

\bibitem[\protect\BCAY{大附,森,松岡,古井,白井}{大附}{1995}]{test6}
    大附克年,森岳至,松岡達雄,古井貞照,白井克彦(1995).
    ``新聞記事を用いた大語彙連続音声認識の検討.''
    電子情報通信学会, 音声技報 SP95-90, 63--68.

\bibitem[\protect\BCAY{Rosenfeld}{Rosenfeld}{1995}]{test14}
    Rosenfeld, R.(1995).
    ``The CMU statistical language modeling toolkit and its use in the 1994 ARPA CSR evaluation.''
    Proc. ARPA Spoken Language Systems Technology Workshop, 47--50.


\bibitem[\protect\BCAY{新納,井佐原}{新納}{1995}]{no15}
    新納浩幸,井佐原均(1995).
    ``疑似Nグラムを用いた助詞的定型表現の自動抽出.''
    情報処理学会論文誌, Vol.36, No.1, 32--40.

\bibitem[\protect\BCAY{Ueberla}{Ueberla}{1994}]{test15}
    Ueberla, J.(1994).
    ``Analysing a simple language model - some general conclusion for language models for speech recognition.''
    Computer Speech and Language, Vol.8, No.2, 153--176.

\bibitem[\protect\BCAY{Woodland {\em et al.}}{Woodland}{1997}]{test25}
    Woodland, P.C., Cales, M.J.F., Pye, D., Young, S.J.(1997).
    ``The development of the 1996 HTK broadcast news transcription systems.''
    Proc.Speech Recognition Workshop, 73--78.

\end{thebibliography}
 

\begin{biography}
\biotitle{略歴}
\bioauthor{中川聖一}{
1976年京都大学大学院博士課程修了.同年京都大学情報助手.
1980年豊橋技術科学大学情報工学系講師.
1983年助教授.1990年教授.
1985$\sim$86年カーネギメロン大学客員研究員.
音声情報処理,自然言語処理,人工知能の研究に従事.工博.
1977年電子情報通信学会論文賞,
\mbox{1988年度IETE最優}秀論文賞受賞.
}
\bioauthor{赤松 裕隆}{
1997年豊橋技術科学大学情報工学課程卒業.
現在,同大学研究科修士課程情報工学専攻在学中.
言語モデルに関する研究に従事.}
\bioauthor{西崎 博光}{
1998年豊橋技術科学大学情報工学課程卒業.
現在,同大学研究科修士課程情報工学専攻在学中.
言語モデルに関する研究に従事.}

\bioreceived{受付}
\bioaccepted{採録}

\end{biography}

\end{document}

