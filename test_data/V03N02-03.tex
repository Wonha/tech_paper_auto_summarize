



\documentstyle[epsf,jnlpbbl]{jnlp_j_b5}

\def\atari(#1,#2){}

\setcounter{page}{33}
\setcounter{巻数}{3}
\setcounter{号数}{2}
\setcounter{年}{1996}
\setcounter{月}{4}
\受付{1995}{7}{7}
\再受付{1995}{8}{30}
\再々受付{1995}{10}{5}
\採録{1995}{11}{28}

\setcounter{secnumdepth}{2}

\title{国文学研究とコンピュータ -- 電子本「漱石と倫敦」考を作る}
\author{安永 尚志\affiref{KKS}}

\headauthor{安永 尚志}
\headtitle{国文学研究とコンピュータ−電子本「漱石と倫敦」考を作る}

\affilabel{KKS}{国文学研究資料館 研究情報部}
{Department of Research Information, National Institute of Japanese
 Literature}

\jabstract{
  コンピュータを用いて,文学研究を進めるための検討を行った.
これには,研究過程の中核である研究ファイルの組織化が必要で,そのモデル
を定義した.モデルは研究過程で利用され,生成される様々な情報資源の構造
認識による組織化である.モデルの検証のために,具体的な文学テーマを設定
し,その実装を行い,評価した.その試みとして電子本「漱石と倫敦」考の研
究開発を進めている.
  研究者による評価実験では,概して評判がよい.例えば,漱石の作品倫敦塔
を読む場合に,各種参照情報を随時に利用できること,メモなどを自由に書き
込むことができ,自分の研究環境の整備がコンピュータ上で可能であることが
評価されている.さらに,モデルは実際の文学研究に有効であること,とくに
教育用ツールとして効果的であるとの評価が得られた.}

\jkeywords{国文学データベース, 「漱石と倫敦」考, 電子本, 研究過程モデル}

\etitle{Japanese Literary Research by Computer\\
-- Creating an Electronic Book on the Study of\\
"Soseki in London"}
\eauthor{Hisashi Yasunaga\affiref{KKS}}

\eabstract{
 This paper describes a study on the Japanese classical literary
researches by the computer. A model has been developed with a
definition and description for the structure of the
research process, which is named the FR.(Files for Research) system.
The model is considered as the simulation of the research process and
organization of materials, information, data, etc., which are widely
used in the research processing.
 Then, a prototype system also has set up to the personal
computer, Macintosh, under the FR. model for the feasibility tests. The
system designs as an electronic book named the
"Soseki in London" based on the substantial model for the
Japanese literary work. In the experiments, the book has
been evaluated highly, that is, it is available and effective
for the researching in the literary work and particularly
is useful for educational applications.}

\ekeywords{Japanese Literary Database, Study on Soseki in London, 
Electronic Book, Research Process Model}

\makeatletter
\def\figcap#1#2{}
\def\tblcap#1#2{}
\makeatother

\begin{document}
\maketitle


\section{まえがき}
国文学研究は,わが国の文学全体に渡る文学論,作品論,作家論,文学史な
どを対象とする研究分野である.また,広く書誌学,文献学,国語学などを含
み,歴史学,民俗学,宗教学などに隣接する.研究対象は上代の神話から現代
の作品まで,全ての時代に渡り,地域的にも歴史上のわが国全土を網羅する.

文学は,人の感性の言語による表出であるから,国文学は日本人の心の表現
であり,日本語を育んだ土壌であると言える.すなわち,国文学研究は現代日
本人の考え方と感じ方を育てた土壌を探る学問であると言える.

文学研究の目標は,文学作品を通じて,すなわち文字によるテキストを主体
として,思潮,感性,心理を探求することである.テキストは単なる文字の羅
列ではなく,作者の思考や感情などが文字の形で具象化されたものであるから,
研究者は書かれた文字を「ヨム」ことによって,作者の思考や感情を再構築し
ようとする.換言すれば,文学作品を鑑賞し,評論し,その作品を通しての作
者の考え方を知ることである.なお,「ヨム」こととは,読む,詠む,訓むな
どの意味である.

最近,国文学とコンピュータの関わりに対する関心が高まり,議論が深まっ
てきた\cite{Jinbun1989-1990}.元来,国文学にとってコンピュータは最も縁
遠い存在と見られてきた.国文学者からみれば,コンピュータに文学が分かる
かとか,日本語のコンピュータが無いなどの理由である.一方では,コンピュー
タへ寄せる大きな期待と,現状との落差から来る批判もある
\cite{Kokubun1982,Kokubun1992,Kokubun1989-1994}.

現在,文学研究にコンピュータが役立つかを確かめることが必要となった.
日本語処理可能なパソコンなどの普及により,国文学者の中でも
\cite{DB-West1995},自身でテキストの入力を行い,また処理を始めている.
しかし,未だほんの一部であって,普及にはほど遠く,またワープロ的な利用
が多い.

コンピュータは,単に「ハサミとノリ」の役割\cite{Murakami1989}であると
しても,その使い方によっては,かなり高度な知的生産のツールに成り得る.
また,研究過程で使われる膨大な資料や情報と,それから生成される多様なデー
タや情報の取り扱いには,コンピュータは欠かせないに違いない.具体的にコ
ンピュータの活用を考えるためには,文学の研究過程の構造認識が必要である.
文学研究は個人的と言われるが,この研究過程が普遍化できれば,モデルが導
出できる.すなわち,コンピュータの用途が分かってくる.

本稿では,国文学研究資料館における事例に基づき,国文学とコンピュータ
の課題について考える.国文学研究資料館は,国の内外に散在する国文学資料
を発掘,調査,研究し,収集,整理,保存し,広く研究者の利用に供するため
に,設立された大学共同利用機関である.また,国文学研究上の様々な支援活
動を行っている\cite{Kokubun1982}.

本稿は,以下のような考察を行っている.2章では,国文学の研究態様を分
析し,情報の種類と性質を整理し,研究過程を解明し,モデル化を行っている.
3章では,モデルを詳細に検討し,定義する.また,モデルの役割をまとめ,
コンピュータ活用の意味を考える.4章は,モデルの実装である.研究過程で
利用され,生成される様々な情報資源の組織化と実現を行う.そのために,
「漱石と倫敦」考という具体的な文学テーマに基づき,システムの実装を行い,
モデルの検証を行った.その結果,モデルは実際の文学研究に有効であること,
とくに教育用ツールとして効果的であるとの評価が得られた.

\section{国文学研究におけるコンピュータ}

\subsection{課題の整理}
コンピュータは国文学研究に役立つか,という問いに答えるためには,まず
国文学研究とは何かを知る必要がある.例えば,和歌研究などではよく本歌ど
り\footnotemark \footnotetext{国文学の用語は,まとめて付録で解説してい
る.}の研究などと言われる.これは,ある和歌が過去の和歌の系譜を引いて
詠まれることがよくあり,その関連を探ることである.有名な事例に次のよう
なものがある.

新古今集巻六の冬に,藤原定家朝臣の歌(歌番号671)がある.

\vspace{1em}
「駒とめて\underline{袖打ちはらふかげもなし}
\underline{\underline{佐野のわたり}}の雪の夕暮」

\vspace{1em}
この歌は,万葉集巻三の長忌寸奥麻呂(ナガノイミキノオキマロ)の次の歌
(歌番号265)が,本歌とされている.

\vspace{1em}
「苦しくも降りくる雨か三輪が崎\underline{\underline{佐野のわたり}}
に\underline{家もあらなくに}」

\vspace{1em}
この例では,単純に本歌を見つけるという点では,「佐野のわたり」の文字
列検索でよい.しかし,「佐野のわたり」は他にも例があって,本歌の確定は
これだけでは充分ではない.「袖打ちはらふかげもなし」の意味と,「家もあ
らなくに」が対照されなければならない.定家は雨を雪に変え,家なしという
直接的な表現に新しい描写を与え,寂しさに優美な情感を込めている
\cite{Tanaka1992}.万葉集の素朴さと,新古今集の優雅さを比べる古来より
有名な例である.

和歌の解釈,あるいは鑑賞をコンピュータで行うことは,極めて困難であろ
うが,この例のような本歌を探すこと位は可能であろうか.しかし,本歌取り
と言っても次のような例もある.例えば,散文中で意味や句の切れ端によって
引かれる引歌\setcounter{footnote}{0}\footnotemark ,男女間などでやりと
りされる問答歌\hspace{-0.2mm}\setcounter{footnote}{0}\footnotemark ,
\hspace*{-1.3mm}
また同じ主題で詠
われる連歌\hspace{-0.2mm}\setcounter{footnote}{0}\footnotemark などがある.
\hspace*{-1.2mm}
さらに,
\hspace*{-1.3mm}
折
句\hspace{-0.2mm}\setcounter{footnote}{0}\footnotemark の1つである
沓冠(クツカブリ)\hspace{-1.2mm}
\setcounter{footnote}{0}\footnotemark などの技巧歌な
どもある.これらの研究にコンピュータは使えるであろうか.

このような課題に応えるためには,様々な関連する材料の収集と整理,並びに
組織化が不可欠である.その上で,知的活動を行うための方法と技術が準備さ
れていなければならない.まず,コンピュータの役割はこの辺りにあると考え
られる.すなわち,第1に研究材料の収集と整理であり,第2にそれらを用い
た知的創造への参与である.

国文学の研究は観念的,思弁的であり,かつ主観的であると言われる.しか
し,プロダクトはそうであるにしても,それに至るプロセスは一般化して考え
ることができる.研究過程は普遍的と考えられる.普遍的な文学研究過程を解
明し,その構造を把握し,モデルを構築することができれば,研究過程をコン
ピュータ上に再現できる.さらに,研究過程におけるコンピュータ活用の様相
が分かる.つまり,集積すべき情報やシステムの性格がより明確になるはずで
ある.なお,研究過程を知ることは知的創造のプロセスを知ることにつながる.

要するに,コンピュータの役割は国文学の研究材料の組織化とその高次活用
にあり,並びに研究過程の構造認識にある.そこで,次のようなステップに従っ
て,この問題を考える.
\begin{enumerate}
\item 国文学研究の態様を知る.
\item 取り扱う資料や情報の種類や性質を知る.
\item 資料や情報の整理,分析などの処理の方法を知る.
\item これらについて情報科学的考察を行う.
\end{enumerate}

なお,観点を変えれば,国文学の研究過程のシミュレータの開発とも考えら
れる.

\subsection{国文学の研究態様}
国文学研究のスタイルとも言うべき研究の態様を考える.図\ref{fig:1}に,研究過程
のモデルを示す.このモデルはかなり一般化したもので,国文学研究のみなら
ず,他の多くの分野にも通じるモデルと考えている.研究過程は図に示す順序
で展開するものとする.

\begin{figure}[htb]
\begin{center}
  
\epsfile{file=fig1.eps,width=74mm}
  \caption{国文学の研究過程モデル}\label{fig:1}
\end{center}
\end{figure}

\begin{itemize}
\item[1.] [研究課題]ステップから,研究は開始する.ここでは,関心のある分
野の先人の研究と動向が詳しく調査研究される.各種の関連する資料や文献な
どが収集され,参照され,分析され,やゝ漠とした広い分野から,より具体的
な研究対象が絞り込まれる.すなわち,研究課題の確立が行われる.次に,研
究課題の解のための仮説が立てられる.
\item[2.] [調査]ステップは,仮説の検証のための具体的な実験,観測,調査な
どである.実験を例に取れば,実験計画,方法の準備,その実行とデータ採取
まで,一連の流れを含む.実験の結果はデータとして取得され,集積される.
\item[3.] [展開]ステップは,これらのデータの考察と評価,すなわち検証であ
る.ここでは,データに対する計算,分析,図式化などの処理が行われる.検
証の結果,仮説の構造認識に至り,仮説が実証される.場合によっては,再度
前段階へのフィードバックが行われたり,仮説の修正なども行われる.
\item[4.] [研究成果]ステップで,最後に結果がまとめられる.研究成果は論文
の形で公表され,一応これで完了である.しかし,[研究成果]ステップは,
また新しく次の[研究課題]ステップに戻る.この研究過程はサイクルをなし
ていると考えられる.ただし,回を追う毎に,質的な展開が行われている.
\end{itemize}

図の中心に,研究ファイル(FR:Files for Research)を置く.研究の全過
程において,研究者個人が生成し,かつ参照する資料,データ,情報のファイ
ルである.手書きで採取,記録されたデータやメモ類,あるいは複写物,写真,
原本など,あらゆる研究用の材料も含む.この段階のFR中の各ファイルはたい
へん雑多なもので,それらには相互に何のまとまりも脈絡もない.単に資料類
の集積に過ぎない.研究成果として出版,公表されるもの以外は,FRに個人仕
様に基づき蓄積されている.

さて,このFRのファイル構造の認識は可能であろうか.すなわち,分類,整
理などは可能であろうか.これらのファイルの電子情報化(ここでは,コンピュー
タに入力すること,以下電子化と言う)が可能であれば,FRのデータベース化
が考えられる.この段階のFRの様相を知ることは,コンピュータ利用の可能性
を知ることに繋がる.

なお,現状では利用済みの材料はそのまま死蔵されるか,捨てられることが
多い.他者による再利用を考慮することは,この分野への大きな貢献となる.

\subsection{国文学情報の種類と性質}
国文学研究で取り扱うべき材料の種類と性質を知る必要がある.研究対象で
ある資料は,文献資料\setcounter{footnote}{0}\footnotemark 及びテキスト
資料\setcounter{footnote}{0}\footnotemark である.文献資料には写本や刊
本\setcounter{footnote}{0}\footnotemark などの原本
\setcounter{footnote}{0}\footnotemark や写真資料があり,また翻刻
\setcounter{footnote}{0}\footnotemark され,印刷された活字本もある.
テキスト資料はテキスト,語彙索引,用例索引などがある.また,研究論文の
必要なことは言うまでもない.

表\ref{tbl:1}に,国文学における資料,情報の種類と特徴をまとめる
\cite{Yasunaga1989a}.高次性とは情報の表現形態を表す性質ではなく,取り
扱うべき資料,情報の質的な違いを区分するものである.国文学情報は階層構
造をなすと考えられる\cite{Yasunaga1995b}.情報は各階層においてそれぞれ
独自に記録され,表現されている.文字だけではなく,画像,音声などマルチ
メディアによる表現である.

\begin{table}[htb]
\begin{center}
  \caption{国文学における資料, 情報の種類と特徴}\label{tbl:1}
  
\epsfile{file=tab1.eps,width=82mm}
\end{center}
\end{table}
\vspace{-1mm}
0次情報は,原本そのものに関わる情報である.1つの作品に関する本は,
異本\setcounter{footnote}{0}\footnotemark として複数種類存在する.古典
籍では同じ本はないと言っても過言ではない.主に,画像として取り扱う.さ
らに,動画,音曲,音声も用いる.

1次情報は,原本の翻刻されたテキストに関わる情報(校訂本
\setcounter{footnote}{0}\footnotemark を含む)を対象とする.原本とその
翻刻された本は異なるものである.また,1つの原本に対しても複数種類の翻
刻本がある.主に文字で表すが,画像も取り扱う.

2次情報は,主として目録情報である.目録情報はその伝本
\setcounter{footnote}{0}\footnotemark の書誌や所在
を示すが,目録そのものを研究対象とする場合も多く,1次情報的に取り扱わ
れる場合がある.研究論文などの目録も2次情報である.2次情報はおおむね
文字である.

3次情報は,特定テーマや分野に関する解説,研究動向あるいは目録の目録
などを対象とする.また,高次情報は年間の全論文の総合解説や広範な引用分
析など,より総合的な情報を言う.両者共文字と画像を用いる.音声を用いる
場合もある.

なお,文字は日本語として伝来された文字全てを対象とするため,システム
外字(JIS 規格外字)が多い.外字セットは先験的に分かっているものではな
く,新資料発掘の度に発生する.すなわち,日常的な研究や業務の進行中に頻
出する.恐らく,国文学用文字セットの定義域を規定することは不可能である.

\subsection{従来の情報の処理}
従来からの国文学研究におけるコンピュータの役割は,情報検索である
\cite{Yasunaga1988,Yasunaga1989b}.情報の提供者側から見れば,国文学に
関わる様々な学術情報の組織化を行い,データベースの形成を計り,研究者に
役立つ情報検索システムを作り,サービスを行うことである.

例えば,国文学研究資料館では創設時のコンピュータ導入に当たって,文献
資料の検索,研究論文の検索,主要語彙の検索,及び定本
\setcounter{footnote}{0}\footnotemark の作成が計画され ている.これら
を対象とした情報検索システムの開発が進み,一部はサービスされてきている.
また,表\ref{tbl:2}に示す様なデータベースの研究開発と構築が進んでいる.
これらは,主として大型コンピュータを中心とした大量データの共有と共用に
主眼が置かれたシステムである\cite{Yasunaga1990,Yasunaga1991}.

\begin{table}[htb]
\begin{center}
  \caption{国文学データベースの一覧}\label{tbl:2}
  
\epsfile{file=tab2.eps,width=73.5mm}
\end{center}
\end{table}

一方,情報の利用者側から見れば,コンピュータの役割は情報の探査と取得
である.また,高次に利用することである.研究活動においては単純な情報検
索だけではなく,いわゆる応用プログラマとして,多角的な観点からの柔軟な
活用ができなければならない.

\subsection{データベースの利用}
図\ref{fig:2}に,データベースの利用を前提とした研究過程のモデルの1例
を示す.このモデルは,表\ref{tbl:2}に基づく現在研究中の国文学研究支援
システムに基づいている\cite{Hara1994,Hara1995}.

\begin{figure}[htb]

\begin{center}
  
\epsfile{file=fig2.eps,width=74mm}
  \caption{データベース利用による研究過程のモデル}\label{fig:2}
\end{center}
\end{figure}

論文検索フェーズでは,まず研究テーマを高次情報や3次情報により広く調
査研究する.次いで,その研究テーマを確定するために,過去の研究経緯と成
果を深く調査研究する.これはオンラインデータベース利用によって行う.例
えば,2次情報によって関連資料の所在,有無などを知り,1次情報があれば
直接入手する.

原本検索フェーズでは,研究テーマの研究対象である文献資料を探し,入手
する.すなわち,2次情報によって文献資料の書誌や所在を知り,1次情報あ
るいは0次情報によって直接入手する.

研究推進フェーズでは,研究支援ツールを用いて研究を進める.支援ツール
は一般的なテキスト解析システム,画像処理システムなどである.

図\ref{fig:2}は,図\ref{fig:1}を電子化の側面で,やゝ具体的にとらえ直し
たモデルと考えられる.ただし,情報提供者側からの公用的な共有ファイルの
提供である.ここで,考えるべきは個人の参画である.共有ファイルを個人で
活用し,個人の環境に合わせて改編して行くことが出来ればよい.個人環境へ
の部分集合の切り出しと,そのFRへの転化が必要である.

\section{研究ファイルのモデル}

\subsection{研究ファイル(FR)の基本構成}
再びFRを考える.FRは研究者個人が自分の研究のために,自分で収集,蓄積
し,かつ所有する資料,情報の集合と蓄積である.この蓄積は自分の仮説の検
証のために行うのであって,研究過程の進展において,種類,量,質も変って
行く.ときに,データとの対比により,仮説の再構築も必要となる.

図\ref{fig:3}に,FRの基本構造を示す.5つの基本ファイルから構成している.各基
本ファイルは,それぞれ内部に様々な個別のデータファイルを集積している.
また,基本ファイル間には相互に密接な関連がある.

\begin{figure}[htb]
\begin{center}
  
\epsfile{file=fig3.eps,width=78mm}
  \caption{FRの基本ファイルとその関連}\label{fig:3}
\end{center}
\end{figure}

文献ファイルは,研究動向を知るための研究論文などのデータファイルの集
積である.環境ファイルは作家,登場人物,時代,地域,ジャンルなど,その
作品成立の諸々の環境をデータファイルとして集積する.素材ファイルは研究
対象である作品そのものであって,原本,図書,校訂テキスト,各種索引など
のデータファイルの集積である.ときに原稿も含む.人文科学でよく行われる
用例などの採取されたデータのカードやノート類は,ここに位置づける.参照
ファイルは事例や事項,風俗や習慣,各種制度,行事や式典,及び用字,用語,
用例などの参考事項のデータファイルの集積である.図の中心に,メモファイ
ルを置く.メモファイルは研究途中のデータ分析,計算,実験,シミュレーショ
ン,メモなどの記録のデータファイルである.

各基本ファイルは随時に編集を可能とする.また,個別のデータファイルは
随時に書き込み,切り張りを行う.ときに,関連の付け替えなど構造の改編を
伴う.最大の特徴はメモファイルにある.単なる備忘的記録ばかりではなく,
研究推進中のあらゆる記録が保存され,個人の研究そのものの経緯を持ってい
る.

さて,これらのデータファイルの電子化は可能であろうか.現在の技術環境
からは,データの入力にそれほどの問題はない.問題は入力したデータの蓄積
と取り扱いである.また,相互関連の付与である.関連は,極めて多様なかつ
重層構造を持つであろう.さらに,各個別のデータファイル間,並びに基本ファ
イル間を渡り歩くことを可能としなければならない.

\subsection{FRの成長過程}
研究ファイルの電子化された状態のファイルを,以下FRと言う.とくに,こ
の段階のFRは個人仕様であるので,これを個FR(PFR:Private FR)と言う.

PFR は主観的であり,他者への提供を考慮して初めてデータに客観性が生ま
れる.ただし,この段階のデータはまだ特定の範囲に限定されている.
この段階での客観的なFRを共有FR(SFR :Shared FR )と呼ぶことにする.た
だし,以下の理由で,一次SFR と言う.

研究成果のSFR への投入を考える.このとき,情報構造の再編成が行われる.
研究課題あるいは分野の範囲を明確に規定し,その情報表現と構造を確定する
ことができれば,データファイルとしてのより確かな客観性が生まれる.少な
くとも同じ分野の研究者への提供が可能になる.すなわち,他者への提供,あ
るいはデータファイルの流通である.この段階のSFR を二次SFR と言う.一次
SFR と二次SFR にはあまり差異は無いが,二次SFR には構造の再編があり,よ
り普遍的と考えられる.

ところで,このデータファイルに対して,さらに網羅性と普遍性が保証され
れば,これはデータベースに成長する\cite{Inose1981}.つまり,SFR からデ
ータベースへの転換である.この段階でのデータは研究過程の全てを提供する
のではなく,もっと客観的で普遍的なデータの提供である.一般の利用者が,
誤りなくデータを利用できなければならない.

FRの成長過程を図\ref{fig:4}にまとめる.図\ref{fig:4}では,机上の雑多な
ファイルを整理し,組織化し,電子化をはかる.PFR の形成である.PFR は研
究者間での共有を考慮して,再編をはかればSFR に成長する.すなわち,
一次SFR である.一次SFR は研究成果の投入によって,二次SFR に転換できる.
さらに,データの網羅性を増し,普遍化されることによって,データベースへ
成長する.

\begin{figure}[htb]
\begin{center}
  
\epsfile{file=fig4.eps,width=79mm}
  \caption{FRの成長過程}\label{fig:4}
\end{center}
\end{figure}

\subsection{FRの構築}
FRの実現に当たっての課題は,個人仕様に依存するメモファイルの組織化,
並びに各基本ファイルのマルチメディア対応である.SFR としての実現は前述
したモデルと方法に従って行う.実現の方式にはいくつかの方法が考えられる
が,ここではハイパーテキストによる実装を考える.ハイパーテキストは上記
の課題に応えられ,実装も各種ツールを使えば比較的容易である.

本稿では,SFR の一般的な実装方法については触れていないが,次章で示す
ように具体的な課題を設定し,これを実現し,評価している.この場合,SFR 
そのものの具体的な実現を,一種の本の体裁,形式に合わせて作る.これを電
子本と呼ぶ.電子本としてのイメージはSFR の実現形態の事例として,作る立
場からも,使う立場からも分かり易いことによる.
             
\section{実験的FR「漱石と倫敦」考}

\subsection{実験的FRのテーマ設定}
コンピュータにFRを作り,使うという試みを行った.このねらいはFRを実際
に作ることができるかどうか,またこれを使うことによって文学研究は可能か
どうかということの検証である.さらに,他者が使うことができるかどうかと
いう重要な検証がある.

具体的な文学研究のテーマとして,「漱石と倫敦」考を選んだ.「漱石と倫
敦」考は,漱石が文学創作を始めるに至った彼のロンドン留学の影響を考察し
ようとするものである\cite{Inagaki1988}.19世紀末(ビクトリア期最後)の
大英帝国の首都ロンドンに,夏目漱石(33歳)は文部省給費留学生として,英
語研究のために(本人は英文学研究と認識している)2年間滞在した.この留
学がその後の創作活動に与えた影響は大きい\cite{Deguchi1991}.

そこで,ここでは漱石と共に,この留学を追体験し,漱石研究に資するとい
う研究テーマを考えてみることにする.しかし,このテーマはたいへん大きい
ので,FRの例題としては少し絞らざるを得ない.留学に関わる創作で,最も影
響のある作品の1つに処女作とも言うべき短編「倫敦塔
\setcounter{footnote}{0}\footnotemark 」がある.これをヨムこととする.

すなわち,漱石のロンドン留学を知ることがテーマであるが,その直接的な
影響による創作である倫敦塔を読むことを主題とする.この場合,どのような
文学的テーマがあるであろうか\cite{Mizutani1987}.例えば,倫敦塔は暗く,
漱石的狂気であるなどの研究が主なものであるが,シェークスピアの戯曲の影
響による実験的創作と見ても差し支えない.あるいは,抑圧された漱石の女に
対する憧れの吐露とみても差し支えない.文学は読み手の主観に訴える.

さらに,ここから「ロンドンが漱石に与えた影響」について考え,あるいは
「漱石がロンドンから得た知見」について考えることは可能であろうか.

\subsection{実験的FRの考え方}

\subsubsection{FRの作成}
「漱石と倫敦」考を命題と言おう.この命題を考究し,分析し,文学的なテー
マの解を得るステップ,すなわち文学研究過程を考える.この場合の研究過程
は,図\ref{fig:1}〜図\ref{fig:3}に示す考え方に従って,構造を把握し,モ
デル化し,コンピュータに実装しなければならない.すなわち,FRを実現する
ことが目的である.

なお,当初の開発段階ではPFR の形成で進めたが,開発中にある程度の普遍
化が可能であったため,SFR として実現をはかっている.また,資料などの入
力に当たっては,著作権などを極力考慮している.

FRの作成は以下の手順で進める.

(1)\ \ まず,倫敦塔を読む.また,関連して,漱石の最初の短編集「漾虚集
\setcounter{footnote}{0}\footnotemark 」,
ロンドンでの生活についての小作品や文章(例えば「永日小品」や「自転車日
記」),ロンドン滞在中の日記や書簡なども読む.次に,テキストをコンピュー
タに入力する.現代文であるが,校訂テキストが必要である\cite{Etou1991}.
著作権を考慮し,解説,注記も入れる.

(2)\ \ 命題に関する研究論文を収集する.単行本,学会雑誌などである.現在,
人文科学で利用可能なオンラインデータベースは少ない.冊子体で分散してお
り入手し難いが,国文学では明治期からの研究論文を参照しなければならない
\cite{Sawai1993}.書誌目録程度はFRに入力する.著作権を考慮し,可能な限
りテキストも入力する.

(3)\ \ 可能な限り,文献資料をオンラインで入手する.恐らく写真,絵などの
画像である.著作権を考慮し,FRに入れる.

(4)\ \ 研究論文を読みながら,研究動向を分析し,把握し,研究テーマを確立
する.文学研究のテーマは,例えば,
\begin{quote}
\begin{enumerate}
\item 漱石の倫敦塔におけるx,
\item xの視座から見た倫敦塔,
\item 倫敦塔とx
\end{enumerate}
\end{quote}
などと考えられる.ここで,xは漱石の女性観,狂気,孤独,歴史観などであ
る.ここでは何でも良い.

(5)\ \ 研究を展開する.経緯をFRに取り込む.研究の展開は,まず漱石の留学
に関わる状況,環境,行動などを知ることである.例えば,以下のような課題
について調査研究する.
\begin{quote}
\begin{enumerate}
\item 漱石はどの様な精神状態であったか.
\item 漱石はどの様な生活をしたか.
\item ロンドンはどの様な都市であったか.
\item 時代背景・文化の潮流はどうであったか.
\end{enumerate}
\end{quote}

このためには,参考となる資料,情報を集める必要がある.例えば,彼の日
記や書簡から知る様々な環境情報,ロンドン市街地図,主たる建造物,風景な
どの絵または写真,もしくは動画による疑似体験,彼を取り巻く歴史的環境の
情報,関連人物の情報,漱石がその点景で考えたことなどである.さらに,漱
石は絵画などをよく鑑賞し,作品にはその影響が表れている.これらを全てFR
に入れる.なお,各点景は関連する歴史的事実や解釈などによって,同定され
なければならない.

(6)\ \ これらをデータベースにする.文字,数値,画像,音声などマルチメディ
アである.

\subsubsection{FRの検証}
以上のFRが,データベースとして準備できれば,検証のための利用実験を行
う.利用実験の目的は幾つかある.まず,倫敦塔をヨムことにどれだけ役立っ
たかの評価である.漱石と実際にロンドンを生活し,歩いてみること,あるい
は漱石とロンドン塔に行くことである.このとき,具体的なテーマxが倫敦塔
のどの箇所にどの様に現れているか,データベースより,具体的,網羅的に考
究し,結論が得られるかどうかを検証する.

この過程で,資料,情報の不備が判明し,それらの充足の必要性が分かる.
この繰り返しによって,FRは徐々に完成度に近づく.なお,このとき他者の目
が入ると,SFR が作成可能である.

次の関心事は,新しい研究テーマと研究成果が期待できるかである.これに
ついては後述する.

他の観点として,教育への利用がある.すなわち,教材としての活用が考え
られる.電子本として,完成された本に構成できれば,普遍的知識の提供とい
う点で,利用効果は高いと考えられる.また,新しい出版メディアとしての電
子出版も考えられる.なお,この検証は教育などの専門家によらなければなら
ない.

\subsection{実験的FRの構築}



\subsubsection{FRの基本設計}
図\ref{fig:5}に示すように,FRの基本構造を階層構造で定義する.電子本を
意識しているので,目次建ての構成にしている.横軸は目次の章立てを表す.
縦軸は各章の階層を表すと同時に,その章に含まれるべき各データファイルを
置く.データファイルには,文献ファイルを例にとれば,特定の主題毎の研究
論文リスト,あるいは研究論文のテキストなどがまとまりの単位として,定義
される.

\begin{figure}[htbp]
\begin{center}
  
\epsfile{file=fig5.eps,width=83mm}
  \caption{FRの基本構造}\label{fig:5}
\end{center}
\end{figure}

階層はあまり深くしない.同じ階層に位置づけられるデータファイルは,厳
密な範囲を設けないため,種類と数が多くなる.データファイルは,同じ階層
間並びに異なる階層間で相互に関連を持つ.FRの構造はネットワーク構造な
どが適しているかも知れない.しかし,ここではプロトタイプとしての実現の
容易さを考慮して,階層構造とした.各データファイル間の連絡は,
HTML(Hyper Text Markup Language)などによる実現を想定している.

一方,FRは倫敦塔をヨムことに主眼を置いている.倫敦塔は紀行的作品の形
態をとっている.作者の目と共に,ロンドン塔の点景が移ろい行き,それぞれ
の点景で漱石的幻想空間が広がる.すなわち,漱石が感じたそれぞれの点景が,
時系列として移り行き,作品に盛り込まれている.したがって,具体的な点景
に基づき,そこに必要なあらゆる資料,情報が参照できると良いと考えられる.

このFRの概念モデルを,彼が現実の下宿を出て,市街を歩き,ロンドンを眺
め,ロンドン塔に入り,幻想的体験に浸り,また出て行き,現実の下宿に帰る
という時の流れに従って定義する.図\ref{fig:6}に,時系列に従った構造を
持つ概念モデルを簡潔に表す.この図に従って,電子本,すなわちFRを構成す
るものとする.

\begin{figure}[htbp]
\begin{center}
  
\epsfile{file=fig6.eps,width=82mm}
  \caption{「漱石と倫敦」考の時系列による概念モデル}\label{fig:6}
\end{center}
\end{figure}

なお,他の作品については倫敦塔と同様の構造が定義できなければならない
が,作品によって時系列は使えない場合がある.その場合は,論理的構造,空
間的構造などを考慮する.
 
\subsection{実験的FRの実現}
\subsubsection{電子本を作る}
実装はハイパー構造を考えた.ここでは,ハイパーテキストをベースとして
考えているが,必ずしもこれが最適と言うわけではない.ある知識の固まりが
自由に手軽にポイントされるという点で,実装の便宜から選んだ.Macintosh 
に実装している.

システムはボイジャー社製ツールキットを用いた.ツールキットは,電子本
の作成を意識して開発されたハイパーカードをベースとする作成ツールである.
構造は本の体裁をとる.すなわち,表紙,タイトル,目次,各章,本文,解説,
索引などの目次スタイルの構造を持つ.

図\ref{fig:7}に,「漱石と倫敦」考の電子本の目次と構成例を示す.煩雑さ
を避けるため,第2階層までとし,かつ一部しか示していない.目次から各章
に飛ぶ.もちろん,目次を順に追うこともできる.各章ではその配下のデータ
ファイルに対しては,事項のマウスクリックによるハイパージャンプで参照す
る.データファイル間のジャンプの他,ポインティング用ウインドウが定義で
きる.機能ウインドウも各種定義できる.例えば,テキストに対しての語彙の
探査とコンコーダンスの作成,インデックスの作成などである.大まかな実装
は以下の通りである.

\begin{figure}[htb]
\begin{center}
  
\epsfile{file=fig7.eps,width=80mm}
  \caption{電子本「漱石と倫敦」考のもくじ構成}\label{fig:7}
\end{center}
\end{figure}

1章は,「漱石と倫敦」の関わりである.留学の経緯,ロンドンにおける漱
石を考える.留学に関わるあらゆる情報を集積する.基本ファイルのうち,環
境ファイルと参考ファイルに相当する.

2章は,ロンドンの情景である.環境ファイルである.ロンドンの市街地図
(目下百年前の地図を探している),倫敦塔に関わる市街の建造物,風景の絵
画,挿し絵,写真など,とくにロンドン塔の情景,建物構造図,言及している
建物や文物の写真などを入れる.漱石は2年余の間下宿を5回変わっている.
その図や近郊の情景は作品に関係が深い.

3章は,漱石のロンドンに関わる作品に対する研究者と研究である.文献ファ
イルである.研究論文の索引,あるいは研究論文そのものをテキストで入れる.
研究動向や研究経緯も入力する.研究の現在が辿れ,必要なデータベースなど
へのアクセス法も分かる.

4章は,倫敦塔をヨム.この電子本の中核である.すなわち,素材ファイル
である.校訂テキストと事項解説などを入力する.作品の解題や解説,各種注
記も入力する.このためには様々な参考ファイルが必要である.具体的な構造
は後述の図\ref{fig:8}によっている.

5章は,漱石のロンドン留学に関わる他の作品,例えばカーライル博物館,
自転車日記などである.これも素材ファイルである.構造は4章と同様である.

6章は,史跡の散策である.環境ファイルである.ここでは,現在のロンド
ンの町並みを動画による案内風にまとめた.言わば,観光案内のようなもので
ある.動画と音声による情景描写は,仮想現実感を与え,作品をヨムことの深
まりを与える.

なお,ここではメモファイルは明確に定義していない.7章に準備だけはし
ている.何故ならば,メモは各章の各ページの中で,ツールキットにより自由
に書き込みができるためである.また,逆のような理由で,ロンドンの情景は
全て第2章に集約した.必要なデータファイルから,自由にアクセスできるた
めである.

\begin{figure}[htb]
\begin{center}
  
\epsfile{file=fig8.eps,width=83mm}
  \caption{倫敦塔をヨム}\label{fig:8}
\end{center}
\end{figure}

\subsubsection{倫敦塔をヨム}
4章の実装は図\ref{fig:8}による.倫敦塔のテキストは,ロンドン塔へ実際
に彼と行くことを想定して,時系列に従って,各種点景が参照できるよう配慮
した.いつでも事項によるハイパージャンプによって,対応するデータが参照
可能である.また,テキストには朗読を入れた.これにより,テキストを文字
を追って読むこと以外にも,音声による鑑賞を可能とした.なお,可能ならば,
漱石自筆の原稿が入ると価値が高まる.

図\ref{fig:8}は,また「漱石と倫敦」考の全体構造を示している.実装は図
\ref{fig:7}によっているが,言わば1つのビューとして図\ref{fig:8}が位置
づけられる.この意味では,図\ref{fig:8}を基軸にした実装も考えられる.

電子本として,約40枚の画像,主たるテキストとその朗読,及び各種索引,
解説,注記などを含む.現在,約 200ページの本となっているが,なお追加さ
れつつある.

実装したテキストとその検索例を,図\ref{fig:9}に示す.この例は,倫敦塔
テキスト中から「余」と言う語を探し,そのKWICを表示した例である.画面コ
ピーで示した.余は作品中で主体であり,その視点の移動によって,主題が展
開する.

\begin{figure}[htb]
\begin{center}
  
\epsfile{file=fig9.eps,width=113mm}
  \caption{倫敦塔テキストの検索例}\label{fig:9}
\end{center}
\end{figure}

ハイパーテキストは基本的に階層構造であり,情報のまとまりを関連付ける
ことができる.しかし,それらの複雑な相互関連,とくに論理構造は定義でき
ない.限界はあるが,実験としては十分と考えている.

\subsection{評価実験}

\subsubsection{検証課題(電子本をヨム)}
電子本をヨムことは,漱石と一緒に留学を追体験し,とくに一緒に倫敦塔の
情景に従って,ロンドン塔まで歩いてみることである.日記には,ロンドン着
後4日の10月31日(水)に,「Tower Bridge, London Bridge, Tower,
Monumentヲ見ル」とある.どう歩いたか,何を考えたか想像しながら歩く.実
際に,彼の辿った足取りを追って行く.これは地図,写真,関連する文などに
よって,ある程度可能である.これらの情報は現在入手可能な資料などで作る
が,可能な限り当時の資料を収集する.なお,幸いなことに百年前と今のロン
ドンの町並みはほとんど変わっていない.

漱石のロンドン留学が,漱石の内面にどの様な変化を生じさせたか.彼の文
学論構築に至る精神活動は,彼と共に仮想現実的にロンドン体験を置くことに
より,よりはっきりとするはずである.このシステムを訪れることによって,
何かの発見が生まれると素晴らしい.

次に,作品倫敦塔を読みながら,彼と一緒にロンドン塔内を歩いてみる.作
品で何を言おうとしたかを考える.英国史の断片と,シェークスピアを知らな
ければならない.参照すべき情報が大量に必要である.19世紀末のロンドンを
考え,日本を考えることに役立つ情報の集積がなければならない.これは,あ
る程度図\ref{fig:8}に従って実装されている.

\subsubsection{評価}
作り手以外の他者による利用を,大学院の教育現場における実験として行っ
た.また,研究者による新しい知見獲得の可能性を探る利用実験も試みた.幾
つかの経緯と文学的評価について述べる.

10数名の国文学の学生を対象とした利用実験を行った.結果はおおむね好
評であった.彼らにとっては初めての経験であり,戸惑いも多いようであった
が,数時間の訓練で充分使いこなすことができた.机上と同じように研究を進
めることができ,自分の感想や考えを簡単に入力できる(メモを取る)ことが
評価された.また,題材としては,やや使い尽くされた感があり,必ずしも素
材が完備していないとの批判があるが,プロセスそのものは新しく,かつ応用
が利くことが確認された.

一方,研究者による利用実験では建設的な評価が多く,以下にまとめる.

 漱石はジェーンの処刑の場面を,ドラローシュ
\setcounter{footnote}{0}\footnotemark の絵によっている.この絵
は現在英国ナショナルギャラリにある.実際の絵は畳3枚くらいの大きさの絵
であるが,コンピュータでは画面サイズでしかない.しかし,この程度のイメー
ジでも,単に文を読むだけでは味わえない臨場感を与え,彼の幻想を体験でき
る意義が強調され,評価された.

作品では首切り役人は醜く書かれ,哀れさを助長するのに役立つ風情がある
が,漱石が見たドラローシュの絵の首切り役人は,大した美丈夫に描かれてい
る.果たして,この差には何らかの意味,あるいは意図があるのであろうか,
大きなテーマであり,このシステムによる発見である.すなわち,彼が作品の
中で言及した絵を実際に見ることによって,発見できるテーマである.

ジェーンの処刑の場面は,その幽閉された塔に行って,その題辞を見ること
によって,より深く作品に同化できる.実際に,システムを通じて目にしなく
ては実感が湧かない.処刑の場面に登場する首切り歌などは,そこに恐ろしい
音楽のように響く描写がなされている.これなども,実際に朗読を聴くことで,
より作者と同化する作品鑑賞につながる.

以上は,マルチメディアによる新しい作品鑑賞,並びに研究の道があること
の事例となった.

また,テキストは結構難解な歴史的背景や事項があり,これらの解説など多
くの参考データをハイパージャンプで簡単に参照できる.作品を読むに当たっ
て,たいへん有効な手段であることが確認され,好評であった.さらに,自分
のメモが残せることも有用であった.

これらのことに加え,様々な知識の利用や参照に関して,自由度の高い教材
の提供が可能であろうという点で,一般教育への電子本の活用はかなり有効で
あるとの評価を得た.
                                                            
\section{あとがき}

コンピュータを用いて国文学研究を進めるには,研究ファイルの組織化が必
要なことを述べた.現在,その試みとして電子本「漱石と倫敦」考の研究開発
を進めている.実際に研究者による評価では概して評判がよい.例えば,倫敦
塔と言う作品をヨム場合に,各種関連情報を利用できるメリットが大きい.利
用者はメモを書き込んだり,必要な情報を入れるなど,自分の研究環境の整備
がコンピュータ上でできる.

狙いは,文学的テーマの解が得られるか,あるいは知的生産活動に耐えうる
かということである.この評価が必要である.実験ではマルチメディアで文学
作品をヨムことにより,従来無かった新しい研究テーマも指摘されている.し
かしながら,「漱石と倫敦」考と言うテーマは極めて大きい.しかも倫敦塔に
絞っている.そのため,カーライル博物館や自転車日記など他の関連作品に進
み,トータルな「漱石と倫敦」考を体験できることが望まれている.

なお,電子本は先験的に構造が与えられているので,自由度がなく研究には
向かないとの指摘もある.これは,メモファイルは常時書き込み,修正などが
許され,1部の共有部分を除き,個人研究環境の累積となることで,対処でき
よう.確かに,倫敦塔から他の作品に適用する場合など,時系列を柱とするこ
とはできないかも知れない.この場合は,その作品特有の構造に着目すればよ
い.要するに,本稿で述べた方法は,言わば方式の例示的標本を提供し,そこ
から自由に個人環境を作れるようにすることである.

特筆すべきは,教育用の電子本という点である.大学院レベルの教育用素材
としての価値の他に,大学あるいは高校,一般においても,充分価値のある新
しい形態の本ではなかろうか.著作権などをクリアして上での電子出版が望ま
れている.

また,SFR の共有は例えばグループウェアとして,プロジェクト研究などの
推進に役立つと考えられる.とくに,今後はパソコン環境への実装ばかりでは
なく,例えばWWW (World Wide Web)サーバなどへの登録,あるいはグループ内
のLAN 環境での実装など,用途は広いと考えられる.Mosaicなどによる一般提
供なども考慮する必要がある.現在計画中である.

本稿は,現在死蔵される運命にある研究個人環境の研究過程の素材の活用に
道を開いた.恐らく,今後はコンピュータの活用を前提にすれば,研究過程で
のPFR の活用は,ある程度の研究者の責務と考えるべきかも知れない.

以上のような観点から,コンピュータ応用を考える場を,コンピュータ国文
学と呼んでいる.コンピュータ国文学は,国文学研究にとって従来型の研究を
より一層推進することは当然であるが,一歩進んで従来無かった新しい研究領
域を提供することが期待される.

本論では触れなかった重要な課題が多くある.例えば,データベースの一貫
性制御,典拠コントロール,テキスト処理の実際,並びに著作権の問題である.
とくに,著作権は原著者,校訂者,電子化データ作成者,出版者などの複雑な
関連もあり,今後真剣に考え対処すべき問題である.ここでは,問題点の指摘
に留める.

最後に,本稿は文献\cite{Yasunaga1995a}に基づいている.


\acknowledgment

本研究では,日頃ご指導いただく国文学研究資料館の佐竹昭廣館長,藤原鎮
男教授,立川美彦教授に御礼申し上げる.また,同館松村雄二教授,中村康夫
助教授には,有益な助言と批評などをいただいた.とくに,原正一郎助教授,
情報処理係野村龍氏をはじめ,係員諸氏には,システム開発,実験などの協力
をいただいた.合わせて深謝する.


\bibliographystyle{jnlpbbl}
\bibliography{jpaper}


\section*{付録}

\subsection*{国文学の用語}
主として, 広辞苑(1983, 岩波書店, 第三版)によった, 読みのABC順で示す. 

\begin{description}
\item[異本(イホン):] 同一の書物であるが,文字,語句,順序に異同がある
もの.別本.
\item[折句(オリク):] 短歌,俳句などの各句の上に物名などを一字ずつ置い
たもの.
\item[刊本(カンポン):] 狭義には主として江戸時代の木活字本,銅活字本,
整版本などの称.版本.
\item[沓冠(クツカブリ):] ある語句を各句の始めと終わりに1音ずつ読み込
むもの.折句の1つ.
\item[原本(ゲンポン):] 写し,改訂,翻刻などをする前の元になる本.
\item[校訂本(コウテイボン):]古書などの本文を他の伝本と比べ合わせ,手を
入れて正した本.
\item[定本(テイホン):] 異本を校合して誤謬,脱落などを検討し,校正し類
書中の標準となるように本文を定めた本.
\item[テキスト資料:] 国文学作品の本文(ほんもん),ここでは活字化され
印刷された本.
\item[伝本(デンポン):] ある文献の写本または版本として世に伝存するもの.
\item[ドラローシュ:] Delaroche, Paul(1797-1856).フランスの歴史画家.
漱石の見た絵は,フランスルーブル美術館にある「エドワードの子供達」と,
イギリスナショナルギャラリにある「ジェーンの処刑」.
\item[引歌(ヒキウタ):] 有名な古歌を自分の文章にひき踏まえて表現し,そ
の箇所の情趣を深め広める表現技巧.
\item[文献資料:] 国文学の研究対象となる原本や写真資料.
\item[本歌(ホンカ)どり:] 和歌,連歌などで,意識的に先人の作の用語,語
句などを取り入れて作ること.
\item[翻刻(ホンコク):] 手書き文字,木版文字などを活字に置き換えること.
翻刻本とは,写本,刊本を底本として,木版または活版で刊行した本.
\item[問答歌(モンドウカ):] 一方が歌で問い,他方が歌で答えたものの併称.
\item[漾虚集(ヨウキョシュウ):] 夏目漱石の短編集.倫敦塔を含む初期7編
の短編を収めたもので,明治39年(1906)大月書店から出版された.
\item[連歌(レンガ):] 和歌の上句と下句に相当する長句と短句との唱和を基
本とする詩歌の形態.
\item[倫敦塔(ロンドントウ):] 夏目漱石の短編.明治38年(1905)「帝国文学」
に発表.ロンドン滞在中に見物したロンドン塔の印象を骨子とする.
\end{description}


\begin{biography}
\biotitle{略歴}
\bioauthor{安永 尚志}{
1966年電気通信大学電気通信学部卒業. 同年電気通信大学助手, 東京大学大型
計算機センター助手, 同地震研究所講師, 文部省大学共同利用機関国文学研究
資料館助教授を経て, 1986年より同館教授. 情報通信ネットワークに興味を持っ
ている. 現在人文科学へのコンピュータ応用に従事. とくに, 国文学の情報構
造解析, モデル化, データベースなどに関する研究と応用システム開発を行っ
ている. 最近では, テキストデータベースの開発研究に従事. 電子情報通信学
会, 情報知識学会, 情報処理学会, 言語処理学会, ALLC, ACHなど会員. }

\bioreceived{受付}
\biorevised{再受付}
\biorerevised{再受付}
\bioaccepted{採録}

\end{biography}

\end{document}

