    \documentclass[japanese]{jnlp_1.4}
\usepackage{jnlpbbl_1.3}
\usepackage[dvips]{graphicx}
\usepackage{amsmath}
\usepackage{hangcaption_jnlp}
\usepackage{udline}
\setulminsep{1.2ex}{0.2ex}
\let\underline

\usepackage{graphicx}
\usepackage{multirow}
\usepackage{url}
\usepackage{algorithm}
\usepackage{algorithmic}
\usepackage{ascmac}

\def\sec#1{}
\def\eq#1{}
\def\fig#1{}
\def\tab#1{}
\def\algo#1{}
    \newcommand{\argmax}{}

\Volume{20}
\Number{5}
\Month{December}
\Year{2013}

\received{2013}{5}{9}
\revised{2013}{8}{2}
\accepted{2013}{9}{17}

\setcounter{page}{683}

\jtitle{評価表現と文脈一貫性を利用した\\
	教師データ自動生成によるクレーム検出}
\jauthor{乾  孝司\affiref{Author} \and 梅澤 佑介\affiref{Author} \and 山本 幹雄\affiref{Author}}
\jabstract{
本論文では,レビュー文書からクレームが記述された文を自
動検出する課題に対して,従来から問題となっていた人手負
荷を極力軽減することを指向した次の手続きおよび拡張手法
を提案する:
(1) 評価表現と文脈一貫性に基づく教師データ自動生成の手続き.
(2) 自動生成された教師データの特性を踏まえたナイーブベイズ・モデルの拡張手法.
提案手法では,大量のレビュー生文書の集合と評価表現辞書
が準備できれば,クレーム検出規則の作成・維持・管理,あ
るいは,検出規則を自動学習するために必要となる教師デー
タの作成にかかる人手負荷は全くかからない利点をもつ.
評価実験を通して,提案手法によって検出対象文の文
脈情報を適切に捉えることで,クレーム文の検出精度を向上
させることができること,および,
人手によって十分な教師データが作成できない状況において
は,提案手法によって大量の教師データを自動生成すること
で,人手を介在させる場合と同等あるいはそれ以上のクレー
ム検出精度が達成できることを示した.
}
\jkeywords{クレーム,評価表現,文脈一貫性,教師データ}

\etitle{Complaint Sentence Detection via Automatic Training \\
	Data Generation using Sentiment Lexicons \\
	and Context Coherence}
\eauthor{Takashi Inui\affiref{Author} \and Yusuke Umesawa\affiref{Author} \and Mikio Yamamoto\affiref{Author}} 
\eabstract{
We propose an automatic method for detecting complaint
sentences from review documents. The proposed method
consists of two procedures. One is a data generation
procedure using sentiment lexicons and context
coherence and the other is the expansion of a naive
Bayes classifier based on the characteristics of the
training data.  This method has an advantage of not
requiring human effort for the creation of large-scale
training data and management of rules for complaint
detection. The experimental results indicate that this
method is more effective than the baseline methods.
}
\ekeywords{complaint, sentiment expression, context coherence, labeled data}

\headauthor{乾,梅澤,山本}
\headtitle{評価表現と文脈一貫性を利用した教師データ自動生成によるクレーム検出}

\affilabel{Author}{筑波大学大学院システム情報工学研究科コンピュータサイエンス専攻}{Department of Computer Science, Graduate school of SIE, University of Tsukuba}



\begin{document}
\maketitle


\section{はじめに}
\label{sec:hajimeni}

インターネットの普及により,個人がWeb上で様々な商品を購
入したり,サービスの提供を受けることが可能になった.ま
た,これに伴い,商品やサービスに対する意見や感想が,大
量にWeb上に蓄積されるようになった.
これらの意見や感想は,ユーザが商品やサービスを購入する
際の参考にするだけでなく,企業にとっても商品やサービス
の改善を検討したり,マーケティング活動に活用するなど,
利用価値の高い情報源として広く認識されている.

近年ではさらに,ユーザ参加型の商品開発が注目されるなど,
ユーザと企業とがマイクロブログやレビューサイト等のソー
シャルメディアを通して,手軽に相互にコミュニケーション
を持つことも可能となっている.
そして,このようなコミュニケーションの場においては,い
わゆる「クレーム」と呼ばれる類のユーザの意見に対して企
業側は特に敏感になる必要があり,ユーザが発言したクレー
ムに対しては,適切に対応することが望まれている.しかし
ながら,このようなコミュニケーションの場では,次のよう
な理由からユーザのクレームを見落としてしまう懸念がある.

\begin{figure}[b]
\input{03fig01.txt}
\caption{クレームが含まれたレビューの例 1}
\label{fig:review}
\end{figure}
\begin{figure}[b]
\input{03fig02.txt}
 \caption{クレームが含まれたレビューの例 2}
 \label{fig:review2}
\end{figure}

\begin{itemize}
 \item 
見落とし例 1:特に,マイクロブログ型サービスを通したコ
ミュニケーションでは,多対一型のコミュニケーション,つ
まり,大勢のユーザに対して少数の企業内担当者が同時並行
的にコミュニケーションを持つことが多く,そのため,一部
のユーザが発言したクレームを見落としてしまう可能性がある.

\item 
見落とし例 2:特に,レビューサイトを通したコミュニケー
ションでは,ユーザは様々な意見をひとつのレビュー文書中
に書き込むことが多く,その中に部分的にクレームが埋め込
まれることがある(\fig{review}および\fig{review2}に例を
  示す.下線部がクレームを示す).この場合,レビューの
中からクレームを見つける必要があるが,これらの一部を見
落としてしまう可能性がある.

\end{itemize}

本論文では,上記のうち,2 つ目の見落とし問題に対処すべ
く,レビューからクレームを自動検出する手法について述べ
る.より具体的には,まず,文単位の処理を考え,
\tab{data_detail}のような内容を含む文を「クレーム文」と
定義する.そして,レビューが入力された際に,そのレビュー
中の各文のそれぞれに対して,それらがクレーム文かそうで
ないかを自動判定する手法について検討する.

\begin{table}[b]
 \caption{クレーム文の定義}
 \label{tab:data_detail}
\input{03table01.txt}
\end{table}

これまで,テキストからクレームを検出することを目的とし
た先行研究としては,永井らの研究\cite{nagai1,nagai2}が
ある.
永井らは,単語の出現パタンを考慮した検出規則に基づいた
クレーム検出手法を提案している.しかしながら,彼らの手
法のように人手で網羅的に検出規則を作成するには,作成者
がクレームの記述のされ方に関する幅広い言語的知識を有し
ている必要がある.
また,現実的に検出規則によって運用するには,膨大な量の
規則を人手で作成・維持・管理する必要があり,人的負荷が
高いという問題がある.
この問題に対する解決策のひとつとして,教師あり学習によっ
て規則を自動学習することが考えられるが,その場合でも,
事前に教師データを準備する必要があり,単純には,教師デー
タの作成に労力を要するという別な問題が発生してしまう.

本論文では,上記のような背景を踏まえて,人的な負荷をな
るべく抑えたクレーム検出手法を提案する.
より具体的には,レビュー文書からクレーム文を自動検出す
る際の基本的な設定として,テキスト分類において標準的に
利用されるナイーブベイズ・モデルを適用することを考え,
この設定に対して,極力人手の負荷を軽減させるために,次
の手続きおよび拡張手法を提案する.



\begin{itemize}
\item 
評価表現および文脈一貫性に基づく教師データ自動生成手法
を提案する.従来,学習用の教師データを作成するには負荷
の高い人手作業に頼らざるを得なかったが,本研究では既存
の言語資源と既存の知見に基づくことで,人手作業に頼らず
に教師データを自動生成する手法を提案する.

\item 
次に,上記で生成された教師データに適したモデルとなるよ
うに拡張されたナイーブベイズ・モデルを提案する.上記の
提案手法によって生成された教師データは自動化の代償とし
て人手作成されたデータと比べて質が劣化せざるを得ず,標
準的な分類モデルをそのまま適用するだけでは期待した精度
は得られない.本研究では上記のデータ生成手法で生成され
るデータが持つ特性を踏まえて,ナイーブベイズ・モデルを
拡張する.

\end{itemize}

提案手法では,従来手法で問題となっていた検出規則の作成・
維持・管理,あるいは,規則を自動学習するために必要とな
る教師データの作成にかかる人手負荷は全くかからない利点
をもつ.
本論文では,上記の手続きおよび拡張手法について,実デー
タを用いた評価実験を通して,その有効性を検証する.
本論文の構成は以下の通りである.まず\sec{gen}で教師デー
タの自動生成手法について説明する.その後,\sec{model}で
ナイーブベイズ・モデルの拡張について説明し,\sec{exp}で
評価実験について述べる.\sec{related}で関連研究を整理し
た後,\sec{owarini}で本論文をまとめる.


\section{教師データ自動生成}
\label{sec:gen}

\subsection{教師データ}

まず,生成したい教師データについて整理する.
本研究では,クレーム文を検知するためにナイーブベイズ
分類器\cite{nv}を構築し,文がクレームを表しているか,あ
るいはクレームを表していないかのどちらかに分類したい.
このような分類器の構築に必要となる教師データは,言うま
でもなく,クレームを表している文(以下,クレーム文と呼
  ぶ)の集合と,クレームを表していない文(以下,非クレー
  ム文)の集合となる.
以下では,説明の便宜上,このデータ集合を得る手続きをラ
ベル付けと呼び,【クレーム】および【非クレーム】という
ラベルによって,どちらの集合の要素となるかを区別するこ
ととする.例えば,\fig{review}の各文に対してラベル付け
が実施されたとすると,次のようなラベル付きの教師データ
が得られる.

\begin{itemize}
 \item 【非クレーム】従業員の方は親切でした.
 \item 【非クレーム】最寄りの病院など教えて頂き,とても助かりました.
 \item 【非クレーム】ありがとうございました.
 \item 【クレーム】ただ残念だったのが,シャワーの使い方がよくわからなかったことです.
 \item 【クレーム】使い方の説明をおいて頂きたいです.
\end{itemize}


本論文で提案する教師データ自動生成手法は,評価表現の情
報に基づくラベル付けステップと,ある文の文脈に対する文
脈一貫性の情報に基づくラベル付けステップの 2 ステップで
構成される.
各ステップをそれぞれ核文ラベル付けおよび近接文ラベル付
けと呼ぶことにし,以下で順に説明する.


\subsection{核文ラベル付け}
\label{sec:data_core}

核文ラベル付けは,評価表現の情報に基いて行う.
評価表現とは,「おいしい」や「まずい」等,評価対象に対
する評価を明示的にあらわす言語表現のことである.
一般的には,これら表現に「おいしい/肯定」や「まずい/
  否定」のような肯定・否定の評価極性値を付随させたもの
を集めて評価表現辞書と呼ばれている\cite{inui}.

核文ラベル付けステップでは,評価表現辞書に否定極性とし
て登録されている評価表現に着目し,このような評価表現を
含む文はクレームを表しやすいと仮定する.そして,否定極
性の評価表現を含む文をクレーム文としてラベル付けする.
以降,この手続きで得られる文を次ステップで得られる文と
区別するため核文と呼び,特に,核文がクレーム文である場
合はクレーム核文と呼ぶ.
例えば,「まずい/否定」という単語が評価表現辞書に登録
されている場合,次の例文はクレーム核文としてラベル付け
される.

\begin{itemize}
  \item 【クレーム(核)】 朝食のカレーが\underline{まずい}.
\end{itemize}

もし,ある文が肯定極性をもつ評価表現を含み,かつ「ない」
や「にくい」などの否定辞が評価表現の 3 単語以内に後続し
ていた場合もクレーム核文としてラベル付けする.例えば,
次の例文は「おいしい/肯定」の直後に否定辞「ない」が後
続しているため,クレーム核文としてラベル付けされる.

\begin{itemize}
  \item 【クレーム(核)】朝食のカレーが\underline{おいしく} \unc{ない}.
\end{itemize}

また,評価表現の否定極性と肯定極性を読み替えて上記と同
様の手続きを行った場合に得られる文を非クレーム核文
と呼び,クレーム核文と同じようにラベル付けしておく.

\begin{itemize}
  \item 【非クレーム(核)】朝食のカレーが\underline{おいしい}.
  \item 【非クレーム(核)】ハヤシライスは別に\underline{まずく}は\unc{ない}.
\end{itemize}

さらに,「ほしい」等の要求表現を集めた要求表現辞書が利
用できる場合は,次の例のように要求表現を含む文をクレー
ム核文としてラベル付けする
\footnote{
\setulsep{0pt}
この操作によって,例えば,「このサービスは是非今後も継
  続して\underline{ほしい}」というような肯定的な要求に
ついては誤ってクレーム文として扱ってしまう.しかし,後
述する本研究で使用したデータセットでは,上記のような事
例はごく稀であり,本研究ではこのような肯定的な要求に対
する特別な処理は施していない.}
.ただし,要求表現に注目したラベル付けの場合は,評価表
現の時とは違って,否定辞の有無に関係なくクレーム核文と
してラベル付けする.

\begin{itemize}
  \item 【クレーム(核)】朝食に和食メニューをもっと増やして\underline{ほしい}.
  \item 【クレーム(核)】朝食を洋風なものばかりにして\underline{ほしく} \unc{ない}.
\end{itemize}


以降,クレーム核文と非クレーム核文をあわせた文の集合を
$\mathcal{S}_\mathrm{core}$であらわす.



\subsection{近接文ラベル付け}
\label{sec:data_context}

那須川ら\cite{nasukawa}は,彼らの論文の中で,評価表現の
(文をまたいだ)周辺文脈には以下のような傾向があると述
べており,これを評価表現の文脈一貫性と呼んだ.

{\setlength{\leftskip}{3zw}\noindent
文書中に評価表現が存在すると,その周囲に評価表現の連続
する文脈(以降,評価文脈\footnote{
元論文では「評価部脈」と書かれているが,これは「評価文脈」の書き誤りであると考えられる.})
が形成されることが多く,その中では,明示されない限り,
好不評の極性が一致する傾向がある.
\par}


本研究では,この評価表現の文脈一貫性の考え方に基いて近
接文ラベル付けを行う.
先の核文ラベル付けの際に考慮した評価表現(あるいは要求
  表現)を含む文の周辺文脈について,「評価表現(要求表
    現)の存在に基づいて(非)クレーム文として選ばれた
  文の前後文脈に位置する文は,やはり(非)クレーム文で
  ある」という仮定をおき,この仮定に従って,核文の周辺
文脈に対してラベル付けを行う.この手続きで得られる
文を近接文(より詳細にはクレーム近接文あるいは非クレー
  ム近接文)と呼ぶ.

\begin{algorithm}[b]
 \caption{近接文ラベル付け}
 \label{algo:alg1}
\input{03algo01.txt}
\end{algorithm}

近接文ラベル付けの手続きを\algo{alg1}に示す.この手続き
への入力は,核文ラベル付けを終えたレビュー$d$と,核文に
対する周辺文脈の長さを決定する窓枠長$N$ ($ \ge 0$) であ
り,レビュー$d$ に含まれる核文に対して,レビューの先頭
側に現れる核文から末尾側に現れる核文に向かって順に処理
が進む.
なお,\algo{alg1}において,$s_i$はレビュー$d$内の先頭か
ら$i$番目の文をあらわし,$|d|$は$d$内の文数をあらわす.
処理の大きな流れとしては,line.2--9でラベル付けされる文
が選択され,line.10--16でラベル付けが実施される.
line.17--34の各関数では,付与するラベルの種類(``クレー
  ム''か``非クレーム'')を確定する際に必要な仮のラベル
情報が決められ,その情報が格納される.

\begin{figure}[b]
 \begin{center}
  \includegraphics{20-5ia3f3.eps}
  \end{center}
  \caption{近接文ラベル付けの例(窓枠長$N=2$の場合)}
  \label{fig:context}
\end{figure}


\fig{context}を使って近接文ラベル付けの具体的な実行例を
示す.図の例では,対象となるレビューは 8 つの文から構成
されており,核文ラベル付けによって文$s_1$が非クレーム核
文,文$s_5$ がクレーム核文とラベル付けされた状態であり,
この状態から近接文ラベル付けが開始される.窓枠長は
$N=2$とする.この場合,
まず,核文$s_1$の周辺文脈に対する処理がなされる
(\algo{alg1}のline.3).$s_1$は文書の先頭文であり前方
文脈 (Backward context) は存在しない.そのため,後方文
脈 (Forward context) の$s_2$と$s_3$に対してのみ処理がな
され (line.6--7),それぞれ``非クレーム'' ラベルが配列に
格納される (line.29).
次に核文$s_5$の周辺文脈に対する処理がなされる.$s_5$は
クレーム文であるため,前方文脈では$s_3$に対して 2 つ目の
ラベル``クレーム''が格納され (line.19),また新たに
$s_4$ に対して``クレーム''ラベルが格納される (line.19).
後方文脈では,まず$s_6$に対して``クレーム''ラベルが配列
に格納される (line.28).その一方で,$s_7$ は逆接関係の
接続詞「しかし」の影響があるため,``クレーム''ではなく
``非クレーム''ラベルが格納される (line.31).
最後に,各文に対して格納されたラベル情報をチェックし,
格納されたラベルに不整合がない場合は,そのラベル情報に
従ってラベル付けを行う (line.10--16).不整合が生じ
ている場合はその文に対してどのラベルも付与しない.
以上の操作によって,この例では 2 つの核文から新たに 4 つ
の近接文($s_2$,$s_4$,$s_6$および$s_7$)が得られる.

以降,クレーム近接文と非クレーム近接文をあわせた文の集合
を$\mathcal{S}_{sat}$\footnote{添字$sat$は
  \underline{sat}ellite の略である.}であらわす,また,
必要に応じて,核文の前方文脈から得られた近接文
$\mathcal{S}_{sat}^{B}$と,後方文脈から得られた近接文
$\mathcal{S}_{sat}^{F}$を区別する ($\mathcal{S}_{sat}
  = \mathcal{S}_{sat}^{B} \cup \mathcal{S}_{sat}^{F}$).







\section{ナイーブベイズ・モデルの拡張}
\label{sec:model}

\subsection{ナイーブベイズ・モデル (Na\"{i}ve Bayes model; NB)}
\label{sec:base_model}

前節で述べた手法によって自動生成された教師データは,人
手によって作成された教師データと比べて質が劣化せざるを
得ず,標準的な分類モデルをそのまま適用するだけでは期待
した精度は得られない.
そこで,前節で述べた手法で得られる劣化を含むデータを使
用するという前提をおき,この劣化データがもつ特性を踏ま
えてナイーブベイズ・モデルを拡張することを考える.
以下では,まず,通常のナイーブベイズ・モデル(多項モデ
  ル)について述べ,その後,モデルの拡張について述べる.

ナイーブベイズ分類器では,ある文$s$の分類クラスを判定す
る際に,条件付き確率$P(c|s)$を考え,この確率値が最大と
なるクラス$\hat{c}$を分類結果として出力する.つまり,
\begin{equation}
\hat{c} = \argmax _{c}P(c|s)
\label{eq:eq0}
\end{equation}
である.通常のナイーブベイズ・モデルでは上式
を次のように展開する.
\begin{align}
 \argmax _{c}P(c|s)
	&= \argmax _{c}P(c)P(s|c)\nonumber \\ 
	& = \argmax_{c} \big\{ \log p_{c}+\sum_{w\in{\mathcal{V}}}^{}n_{w}(s)\log q_{w,c} \big\}
\label{eq:eq1}
\end{align}
ここで,$\mathcal{V}$は語彙集合,$n_{w}(s)$は文$s$にお
ける単語$w$の出現回数をあらわす.
また,$q_{w,c}$ ,$p_{c}$は教師データを使ってそれぞれ以
下の式で計算される.本研究ではパラメータを推定する際に
ラプラススムージング\cite{nv}を用いる.
\begin{align}
q_{w,c}&= \frac{\displaystyle n_{w,c}(\mathcal{D})+1}{\displaystyle \sum_{w}^{}n_{w,c}(\mathcal{D})+|\mathcal{V}|}\label{eq:eq2}\\
p_{c}  &= \frac{\displaystyle n_{c}(\mathcal{D})+1}{\displaystyle \sum_{c}^{}n_{c}(\mathcal{D})+|\mathcal{C}|} \label{eq:eq3}
\end{align}
ここで,$\mathcal{D}$は教師データとなる文の集合,
$n_{w,c}(\mathcal{D})$はデータ$\mathcal{D}$においてクラ
ス$c$に属する文に現れる$w$の出現回数,
$n_{c}(\mathcal{D})$はデータ$\mathcal{D}$においてクラス
$c$に属する文の数,
$|\mathcal{V}|$は語彙の種類数,
$|\mathcal{C}|$は分類クラスの種類数である.


以上からもわかるように,通常のモデルでは,分類対象とな
る文内の情報のみを考慮し,分類対象文の周辺文脈の様子は
全く考慮されない.たとえ同一文書内であっても個々の文は
独立に評価・分類する.
また,教師データの利用にあたっても,当然のことながら,
核文であるか近接文であるかといった区別はなく,両タイプ
の文が同等にモデルの構築に利用される.



\subsection{モデル拡張}
\label{sec:pro_model}

前節で述べた教師データ生成過程から得られるデータには,
核文および近接文という 2 種類の文が存在する.
この 2 種類の文のうち,核文は近接文とは独立にラベル付け
される一方で,近接文は核文の情報に基いて間接的にラベル
付けされる.そのため,核文ラベル付けが結果として誤りで
あった事例に関しては近接文もその誤りの影響を直接受ける
ことになる.また,当然ながら,文脈一貫性の仮定が成立し
ない事例もあり得る.このような理由から,近接文は核文に
比べて相対的に信頼性の低いデータとなる可能性が高い.

そこで,このことを考慮し,核文と近接文の情報をモデル内
で区別して扱い,近接文の情報がモデル内で与える影響を下
げるよう,\eq{eq1}の代わりに次のような\eq{eqex}を用いる.
\begin{align}
\argmax _{c}P(c|s)
	&= \argmax _{c}P(c)P(s|c) \nonumber \\
	&= \argmax_{c} \big\{ \log p_{c}+\sum_{w}n_{w}(s)\log q^{tgt}_{w,c} \nonumber \\ 
	& \phantom{= \argmax_{c} \big\{ \log p_{c}}~+ \frac{1}{|ctx(s,N)|}\sum_{w}n_{w}(ctx(s,N))\log q^{ctx}_{w,c} \big\}
\label{eq:eqex}
\end{align}

右辺第 3 項に現れる$ctx(s,N)$は$s$の周辺文脈に位置する前方
および後方のそれぞれ$N$文から構成される文の集合を表して
おり,この項が分類対象の周辺文脈をモデル化している.
この項の係数$1/|ctx(s,N)|$で,周辺文脈の文数に応じてその
影響を調整している.
なお,$n_{w}(ctx(s,N))$は,集合$ctx(s,N)$の要素となる全ての
文における単語$w$の総出現回数をあらわす.
また,右辺第 2 項は通常のモデルと同様に分類対象となる文
をモデル化したものであるが,第 3 項の周辺文脈との区別を
明瞭にするため,$q^{tgt}_{w,c}$という記号を新たに導入し
た
\footnote{\setulminsep{1.2ex}{0.45ex}
上付きの添字$tgt$と$ctx$は,それぞれ
  \underline{t}ar\underline{g}e\underline{t},
\underline{c}on\underline{t}e\underline{x}t をあらわしている.
} .


\eq{eqex}の$q^{tgt}_{w,c}$と$q^{ctx}_{w,c}$,および
$p_{c}$はそれぞれ次式で求める.式中の各記号の意味は
\eq{eq2},\eq{eq3}と同様である.
ここで,$\mathcal{D}_{tgt}$は分類対象文をモデル化するため
の教師データ集合,$\mathcal{D}_{ctx}$は分類対象の周辺文
脈をモデル化するための教師データ集合である.
基本的には,前節で得られる教師データのうち,核文
データを$\mathcal{D}_{tgt}$に割り当て,近接文データを
$\mathcal{D}_{ctx}$に割り当てるが,正確な記述は後述の
\sec{wariate}で与える.
\begin{align}
q^{tgt}_{w,c} & = \frac{\displaystyle n_{w,c}(\mathcal{D}_{tgt})+1}{\displaystyle \sum_{w}n_{w,c}(\mathcal{D}_{tgt})+|\mathcal{V}_{tgt}|}
\label{eq:qtbt}\\
q^{ctx}_{w,c} & = \frac{\displaystyle n_{w,c}(\mathcal{D}_{ctx})+1}{\displaystyle \sum_{w}n_{w,c}(\mathcal{D}_{ctx})+|\mathcal{V}_{ctx}|}
\label{eq:qctx}\\
p_{c}  & = \frac{\displaystyle n_{c}(\mathcal{D}_{tgt})+1}{\displaystyle \sum_{c}n_{c}(\mathcal{D}_{tgt})+|\mathcal{C}|}
\end{align}
以降,便宜的にこの拡張されたモデルを\textbf{NB+ctx}と呼ぶ.

さらに,\eq{eqex}の第 3 項について,周辺文脈を分類対象文
からの相対位置で詳細化した
\begin{equation}
  \frac{1}{|ctx(s,N)|} \big\{ \sum_{w}n_{w}(Bctx(s,N))\log q^{Bctx}_{w,c} 
	+ \sum_{w}n_{w}(Fctx(s,N))\log q^{Fctx}_{w,c} \big\}
\label{eq:eqex2}
\end{equation}
を代わりに利用するモデルも考えられる.
ここで,$Bctx(s,N)$は,$s$の前方文脈に位置する$N$文から構
成される文の集合であり,$Fctx(s,N)$は同様に後方文脈で構成
される文集合である.
また,式中の$q^{Bctx}_{w,c}$および$q^{Fctx}_{w,c}$は次
式で求める.ただし,$\mathcal{D}_{ctx} =
\mathcal{D}^{B}_{ctx} \cup \mathcal{D}^{F}_{ctx}$である.
\begin{align}
q^{Bctx}_{w,c} & = \frac{\displaystyle n_{w,c}(\mathcal{D}^{B}_{ctx})+1}{\displaystyle \sum_{w}n_{w,c}(\mathcal{D}^{B}_{ctx})+|\mathcal{V}_{Bctx}|}\\
q^{Fctx}_{w,c} & = \frac{\displaystyle n_{w,c}(\mathcal{D}^{F}_{ctx})+1}{\displaystyle \sum_{w}n_{w,c}(\mathcal{D}^{F}_{ctx})+|\mathcal{V}_{Fctx}|}
\end{align}
以降,便宜的にこのモデルを\textbf{NB+ctxBF}と呼ぶ.



\subsection{データ割当規則}
\label{sec:wariate}

ここでは,さきほどの説明で保留していた,パラメータ推定
の際に必要となる教師データの与え方について述べる.
ここで,前節で述べた手法によって得られるデータ集合を確
認すると,

\begin{itemize}
 \item $\mathcal{S}_{core}$:クレーム核文と非クレーム核文をあわせた文の集合
 \item $\mathcal{S}_{sat~}$:クレーム近接文と非クレーム近接文をあわせた文の集合
 \item $\mathcal{S}^{B}_{sat~}$:$\mathcal{S}_{sat}$の要素のうち,核文の前方文脈から得られた文で構成される集合
 \item $\mathcal{S}^{F}_{sat~}$:$\mathcal{S}_{sat}$の要素のうち,核文の後方文脈から得られた文で構成される集合
\end{itemize}

であり,$\mathcal{S}_{sat} = \mathcal{S}_{sat}^{B} \cup \mathcal{S}_{sat}^{F}$であった.

これらデータ集合に対して,まず,核文と近接文を区別しな
い単純な割当として,得られた全データをまとめて利用する
ことが考えられる.この場合,拡張モデルNB+ctx においての
割当は,

\begin{itemize}
 \item $\mathcal{D}_{tgt} = \mathcal{S}_{core} \cup \mathcal{S}_{sat} $
 \item $\mathcal{D}_{ctx} = \mathcal{S}_{core} \cup \mathcal{S}_{sat} $
\end{itemize}

となる.これを以降\textbf{NB+ctx(all)}と呼ぶ.また同様に,
拡張モデルNB+ctxBF においての単純な割当は,

\begin{itemize}
 \item $\mathcal{D}_{tgt} = \mathcal{S}_{core} \cup \mathcal{S}_{sat} $
 \item $\mathcal{D}^{B}_{ctx} = \mathcal{S}_{core} \cup \mathcal{S}_{sat} $
 \item $\mathcal{D}^{F}_{ctx} = \mathcal{S}_{core} \cup \mathcal{S}_{sat} $
\end{itemize}

となるが,これは先のNB+ctx(all)と事実上同等となるため
以降の議論では割愛する.

次に,核文と近接文の区別を考慮したデータ割当を考える.
拡張モデルNB+ctx においての割当としては,

\begin{itemize}
 \item $\mathcal{D}_{tgt} = \mathcal{S}_{core}$
 \item $\mathcal{D}_{ctx} = \mathcal{S}_{sat} $
\end{itemize}

が考えられる.これを以降\textbf{NB+ctx(divide)}と呼ぶ.
また同様に,拡張モデルNB+ctxBF においてのデータ割当とし
て,

\begin{itemize}
 \item $\mathcal{D}_{tgt}     = \mathcal{S}_{core}$
 \item $\mathcal{D}^{B}_{ctx} = \mathcal{S}^{B}_{sat} $
 \item $\mathcal{D}^{F}_{ctx} = \mathcal{S}^{F}_{sat} $
\end{itemize}

が考えられる.これを以降\textbf{NB+BFctx(divide)}と呼ぶ.


最後に,通常のナイーブベイズについて考えると,この場合
は,もともとのモデルにデータを区別する枠組みが存在しな
いため,

\begin{itemize}
 \item $\mathcal{D} = \mathcal{S}_{core} \cup \mathcal{S}_{sat}$
\end{itemize}

という割当のみを考えることになる.
なお,$\mathcal{D} = \mathcal{S}_{core}$ という核文のみ
を考慮し,近接文を利用しない割当も考えられるが,これは
$\mathcal{D} = \mathcal{S}_{core} \cup
\mathcal{S}_{sat}$において近接文の窓枠長を$0$とする場合
に等しいため,割当規則として明示的には議論しないが,第
\sec{exp}の評価実験では,近接文の窓枠長が$0$の場合も含
めて議論する.以降,これを\textbf{NB}と呼ぶ.

ここまでの議論を整理すると,前節の手法で自動生成された
教師データを利用するという前提のもとで,通常のナイーブ
ベイズ・モデルも含めて, 4 つのクレーム文検出モデルが与
えられたことになる.次節では,評価実験を通じて,これら
の有効性を検証していく.




\section{評価実験}
\label{sec:exp}


\subsection{検証項目}

評価実験を通して,提案手法の有効性を検証する.具体的に
は以下の 3 項目を検証する.
\begin{enumerate}
 \item 
提案手法の比較:前節までで述べた 4 つのクレーム文検出モ
デルの中で,どのモデルが最良であるかを検証する.

 \item 
他手法との比較:提案手法と他手法との比較実験を行い,
その結果から提案手法の有効性を検証する.

 \item 
学習データ量とクレーム文検出精度の関係について:提案した
データ生成手法は学習データを自動生成できるため,人手に
よる生成に比べて遥かに多くの教師データを準備できる.こ
の利点を実験を通して検証する.

\end{enumerate}



\subsection{実験の設定}
\label{sec:setting}

実験には,楽天データ公開
\footnote{http://rit.rakuten.co.jp/rdr/index.html}
において公開された楽天トラベルの施設データを利用した.
このデータは約35万件(平均4.5文/件)の宿泊施設に関するレ
ビューから構成されており,ここから,無作為に選んだ
1,000レビューに含まれる文を評価用データとして用い,残り
を教師データ生成用に利用した.
評価用データには4,308文が含まれており,その内の24\%にあ
たる1,030文がクレーム文
であった.
つまり,4,308文からクレームを述べている1,030文を過不足
なく検出することがここでの実験課題である.
評価用データの作成では,まず,レビュー文書中の各文が 1
行 1 文となるようにデータを整形し,それを作業者に提示し
た.そして作業者は,与えられたデータの 1 文(1 行)ごと
にクレーム文か否かを判定していった.なお,ある文の判定
時には,同一レビュー内の他の全ての文が参照できる状態に
なっている.2 名の作業者によって上記の作業を独立に並行
に行ったが,このうち 1 名の作業結果を評価用データと
して採用した.2 名の作業者間の一致度を$\kappa$係数の値
によって評価したところ,$\kappa =0.93$であった.この結
果は,作業者間の判断が十分に一致していたことを示してい
る.作業者間で判断が一致しなかった事例としては,文が長
く,ひとつの文で複数の事柄が述べられている場合や,「長
  身で据え置きのものでは短くて…」のように,クレームの
原因が宿泊施設側にあるとは必ずしも言えない場合が多かっ
た.

教師データ生成時に必要となる評価表現辞書には,高村ら
\cite{takamura}の辞書作成手法に基いて作成された辞書を使
用した.ただし,高村らのオリジナルの辞書は自動構築され
たもので,そのままでは誤りが含まれているため,以下の手
続きによって誤り修正を施し,本実験で使用する辞書として
採用した.オリジナルの辞書には各登録語に対して肯定/否定
の強さを示すと解釈できる$[-1,1]$の範囲のスコアが付与さ
れている.このスコアは,値が大きいほど肯定,また,小さ
いほど否定をあらわし,0付近はどちらでもないことを示して
いると解釈できる.そこでまず,このスコアの絶対値の大き
いものから0.9付近までの単語を自動的に選択した.そして,
選択された各単語に対して人手による誤り修正を施し,結果
として肯定表現760件,否定表現862件からなる辞書を作成し,
本実験に用いた.

また,要求表現辞書として,「欲しい」,「ほしい」,
  「べし」からなる辞書を作成して実験に使用した
\footnote{
評価表現辞書と比べて要求表現辞書への登録単語数が少ない
印象を受けるかもしれない.しかし,要求表現は評価表現と
は違い,「〜して\underline{ほしい}」のように自立語に
付随する形態を取りやすく,そのため辞書に登録できる単語もそれほ
ど多くない.}.
周辺文脈の窓枠長$N$の指定は,データ作成時,モデル学習時,
評価用データの分類時のすべての過程で同期させている.ま
た,各データの単語分割は
MeCab\footnote{http://mecab.sourceforge.net/}によって行った.
また,計算の都合上,$N=0$の場合は$1/|ctx(s,N)|=0$とした.

今回のように,分類すべきクラスがクレーム/非クレームと
いう 2 クラスの分類問題の場合,\eq{eq0}による意思決定は,
以下の\eq{deci}の符号が正の場合にクレームと判定すること
になる.
\begin{equation}
P(\text{``クレーム''}|s) - P(\text{``非クレーム''}|s)
\label{eq:deci}
\end{equation}
しかし,本研究では,\eq{deci}に意思決定の閾値$\theta$を
加えた次の条件式を新たに導入し,この条件式が成立する場
合にクレームと判定し,成立しない場合は非クレームと判定
することとした.
\begin{equation}
P(\text{``クレーム''}|s) - P(\text{``非クレーム''}|s) > \theta
\label{eq:deci2}
\end{equation}
\eq{deci2}の左辺は,クレームと判定する際の確信度を示し
ていると考えることができ,閾値$\theta$はこの確信度に応
じて出力を制御する役割りを持つ.閾値を$\theta = 0$ と設
定すると,これは\eq{deci}を用いた通常の意思決定と同じ動
作となる.閾値を$0$から大きくすると,より確信度が高い場
合のみクレームと判定することになる.
実験では,閾値$\theta$を増減させ,以下の式で計算される
適合率および再現率,あるいはその要約である11点平均適合
率\cite{iir}を求め,検出精度を評価した.
11点平均適合率とは再現率が $\{0.0,\ 0.1,\ \ldots ,\ 1.0\}$ となる
11点における適合率の平均値である.
\begin{align}
\mbox{適合率} &=\frac{\mbox{正しくクレーム文として検出できた数}}{\mbox{クレーム文として出力された数}} \\[1ex] 
\mbox{再現率} &=\frac{\mbox{正しくクレーム文として検出できた数}}{\mbox{クレーム文の数}}
\end{align}
データにおけるクレーム文と非クレーム文の割合等に応じて,
検出性能に対して最適な$\theta$ を自動推定することも考え
られるが,これについては今後の課題である.


\subsection{提案手法の比較}
\label{sec:exp_model}

実験結果を\fig{model_length}に示す.
このグラフは,4 つの各検出モデルについて,考慮する周辺
文脈の窓枠長を変化させながら性能変化をプロットしたもの
である.
文脈長$N=0$の場合は,どのモデルも同じになるため,グラフ
上では 1 点に集まっている.

\begin{figure}[b]
 \begin{center}
 \includegraphics{20-5ia3f4.eps}
 \end{center}
 \caption{実験結果(提案手法の性能比較)}
 \label{fig:model_length}
\end{figure}

NBモデルの結果(``◇'')を基準に考えると,
核文と近接文を区別しないNB+ctx(all)では文脈長を$N=0$か
ら$N \ge 1$のどの文脈の長さに変更しても性能が向上しない
一方で,核文と近接文の区別を考慮するNB+ctx(divide)と
NB+BFctx(divide)は文脈長を$N \ge 1$にすることで,一貫し
て性能が向上することがわかる.
このことから,文脈情報を適切にモデルに反映させるために
は,単にモデルを拡張するだけでは効果がなく,データとモ
デルを上手く組合せて,核文と近接文を区別することが重要
であることが確認できる.
性能の向上が見られたNB+ctx(divide)とNB+BFctx(divide)を
比較すると,どちらも $N \ge 1$ の場合は文脈長の変化に対
しては鈍感な傾向を示しているが,近接文の相対位置を考慮
するNB+BFctx(divide)の方が総じて良い結果を示しており,
本論文で述べた 4 つのクレーム文検出モデルの中では,
NB+BFctx(divide) モデルが最良であることがわかる.

次に,教師データとして自動生成されたクレーム近接文に含
まれる単語を確認したところ,\tab{context_word}のような
単語がクレーム核文には現れず,クレーム近接文にのみ現れ
ていた.
このような単語の情報は,周辺文脈の情報を取り込むことで
初めて考慮できるようになった情報であり,定性的にもクレー
ム検出における周辺文脈情報の利用の有効性が確認できる.

\begin{table}[t]
 \caption{クレーム近接文にのみ現れていた単語}
 \label{tab:context_word}
\input{03table02.txt}
\end{table}

また文脈情報を取り込むことで正しく分類ができるようになっ
た事例を以下に示す.下線の引かれた文が分類対象であり,
左端の数字は分類対象文からの相対位置を示す.

\begin{itemize}
\item 【正しくクレーム文であると判定できた例】\\
\begin{tabular}{rl}
$-2$:& 疲れていたので苦情をいうのも面倒で,さっさとチェックアウトしました.\\
$-1$:& 普段だったらクレームを入れるレベルです.\\
$0$:&\underline{もう宿泊はないと思います.}\\
$+1$:&残念です.
\\
\end{tabular}


\item 【正しく非クレーム文であると判定できた例】\\
\begin{tabular}{rp{352pt}}
$-1$:& 昨年寒い1月に宿泊した際,犬を入室させてもらえ助かりました.\\
$0$:& \underline{今回は寒くはないが飼い犬のミニチュアダックスが\mbox{老齢18歳で}泊めて頂け大変助かりました.}\\
$+1$:& また泊めて頂きます.\\
$+2$:& 感謝.
\\
\end{tabular}
\end{itemize}



次に,誤りの傾向を分析したところ,以下のような事例につ
いて,判定誤りが多く見られた.

\begin{itemize}
 \item 【誤ってクレーム文と判定する例】
 \begin{itemize}
  \item[A.] 不満の表明ではあるが,その対象・原因が宿泊者にある場合
  \begin{itemize}
   \item 【例】仕事で到着が遅くなり,ゆっくりできなかったのが残念でした.
  \end{itemize}
  \item[B.] クレームの対象となりやすい事物が文中に多く記述されている場合
  \begin{itemize}
   \item 【例】部屋は\underline{デスク},\underline{姿見},\underline{椅子},\underline{コンセント}があり,…従業員の対応もまずまずでした.
  \end{itemize}
 \end{itemize}

 \item 【誤って非クレーム文と判定する例】
 \begin{itemize}
  \item[C.] 記述の省略を伴う場合
  \begin{itemize}
   \item 【例】バイキングにステーキがあればなぁ.
  \end{itemize}
  \item[D.] 外部的な知識を要する場合
  \begin{itemize}
   \item 【例】全体的な評価としてはE ランクでした.
  \end{itemize}
 \end{itemize}
\end{itemize}

誤ってクレーム文と判定する事例のうち,A. のような事例に
対応するには,意見の対象や原因を特定する等の詳細な自動
解析の実現が望まれる.
手元のデータによると,B. に該当する上述の例のうち,下線
部がクレーム対象となりやすい事物であった.このような事
例については,文中では名詞が多く現れることから,単語の
品詞情報を考慮する等,単語や単語クラス毎にモデル内での
扱いを変更することが考えられる.
また,誤って非クレーム文と判定する事例については,「Eラ
  ンク」を否定極性の単語として扱うなど,ヒューリスティッ
ク規則によるチューニングは可能であるが,総体的には現在
の技術では改善が困難な事例が多い印象である.



\subsection{他手法との比較}
\label{sec:exp_base}

次に,提案手法と他手法との比較実験を行い,その結果
から提案手法の有効性を検証する.他手法としては,以下に
示す 3 手法を検討した.始めの 2 つは,従来から考えられる
ラベル付け方法に基づく手法であり,残りの 1 つは,教師あ
り学習を適用しない,辞書の情報に基づいたルールベースの
手法である.


\begin{itemize}
  \item 人手によって教師データを作成する手法(以下,人手ラベル)

教師データ用のレビュー集合から2,000件のレビューを無作為
に抽出し,そこに含まれる全ての文に対して人手でクレーム/
非クレームのラベル付けを行ったものを教師データとしてモ
デル学習に用いる.この手法で得られるデータでは核文と近
接文の区別がないため,学習には通常のナイーブベイズ・モ
デルを用いる.提案手法と比べると,この手法では量は少量
だが質の高い学習データが利用できる.
このデータ作成作業は,評価実験の正解データ作成と同一の
作業となる.ただし,このデータ作成には正解データの作成
に従事した作業者のうちの 1 名によって執り行なった.作業
時間は約30時間であった.

\item 文書ラベルを教師データ作成に用いる手法(以下,文書ラベル)

本実験で使用しているレビューデータには,本研究でいうク
レームとほぼ同等の概念を示している「苦情」というラベル
がレビュー単位に付与されている.
そこで,ここでは,文よりも粗い文書に対する教師情報を利
用して,文単位の教師データを自動生成することを考える
\cite{nigam2004a}.
具体的には,「苦情」ラベルが付与されたレビューに含まれ
ている全ての文をクレーム文とみなし,逆に,「苦情」ラベ
ルが付与されていないレビューに含まれている全ての文を非
クレーム文とみなすことで教師データを自動生成し,モデル
学習に用いる.
モデルは先と同様の理由で通常のナイーブベイズ・モデルを
用いる.
提案手法と比べると,この手法では相対的に質は低いが,大
量の学習データが利用できる.

\item 辞書による手法

この手法は教師あり学習は行わず,辞書のエントリをルー
ルとみなしたルールベース手法である.評価用データに対し
て\sec{data_core}で述べた核文ラベル付け,および
\sec{data_context}で述べた近接文ラベル付けの手続きを直
接適用してクレーム文を検出する.ただし,ここでの焦点は
データ生成時とは違って,クレーム文を検出できるか否かで
あるため,ラベル付けの結果,クレームとラベル付けされた
文以外は全て非クレームであるとみなして評価した.
なお,辞書は\sec{setting}で述べた辞書を用いる.

\end{itemize}

実験結果を\fig{baseline}に示す.また,\tab{data}に提案
手法のラベル付けと人手ラベル,文書ラベルの各手法による
ラベル付けの特徴をまとめる.

\begin{figure}[b]
 \begin{center}
 \includegraphics{20-5ia3f5.eps}
 \end{center}
 \caption{実験結果(他手法との比較)}
 \label{fig:baseline}
\end{figure}
\begin{table}[b]
 \caption{各ラベル付け手法で生成される教師データの特徴}
 \label{tab:data}
\input{03table03.txt}
\end{table}

\fig{baseline}において,辞書による手法は,ナイーブベイ
ズモデルを用いた分類時に導入した閾値のパラメータが存在
しないため,11 点平均適合率を計算できない.そのため,こ
こでは再現率と適合率によって分類性能を評価している.
なお,図中の``提案ラベル''が提案手法の結果であり,さき
ほどの評価実験で最良であった拡張モデルNB+ctxBF(divide)
で文脈長$N=2$の実験結果を掲載している.
また,``辞書(核)''は,辞書による手法のうち,核文ラベ
ル付けのみを考慮した場合の結果であり,``辞書(核+近
  接)''が,核文ラベル付けと近接文ラベル付けの両方を考
慮した場合の結果である.

\fig{baseline}から,比較したどの手法よりも提案手法が良
い性能を示していることがわかる.
辞書による方法は,辞書に登録されている単語が含まれてい
ない文に対しては適用できないため,再現率が低い.近接文
を考慮することである程度の再現率を確保することは可能で
あるが,当然ながらその代償として適合率が下がる結果となっ
ている.
ここで,固有表現抽出課題がそうであるように,一般に,辞
書に基づいた手法では再現率が低くなるがその一方で適合率
が高くなる傾向がある.近接文の情報を用いない``辞書
(核)'' の結果は特にその傾向を示している.ただし,今回
の実験結果では,再現率を固定させて適合率を見ると,ナイー
ブベイズ・モデルを用いた手法の方が適合率がより高い結果
となっていた.これは,本研究課題では,辞書に登録されて
いる一部の単語の情報だけでは文全体のクラス(クレーム/
  非クレーム)が正しく決定できない場合があり,このよう
な場合には,辞書による方法よりも文内の単語情報を総合的
に考慮できるナイーブベイズ・モデルが適していたためと考
えられる.

次に,文書ラベルを利用する方法は,文書内のすべての文を
教師データとして利用できる.そのため,提案手法と同程度
かそれ以上の教師データが利用できるという特徴がある.し
かし,文書内には一般的にクレームと非クレームが混在する
ことから,文書ラベルと整合していない信頼性の低いデータ
を多く含む結果となり,そのことが性能の低下に繋がってい
ると考えられる.
最後に,人手作成による方法は,もっとも質の高い教師デー
タを準備することができるが,作成負荷の高さから,量を確
保することが難しい.今回は人手で2,000件(8,639 文)のレ
ビューから教師データを作成したが,提案手法を上回ること
はなかった.


\subsection{学習データとクレーム文検出精度の関係について}
\label{sec:exp_size}

先でも述べたように,一般に,人手作成された教師データは
質が高い反面,多くの量を準備することが困難である.一方,
提案手法のように自動生成された教師データは人手作成され
たデータよりも質が落ちるが,ラベルのない生データを準備
するだけで手軽に増量できる.
ここでは,人手によって教師データを作成する場合と第
\sec{gen}の提案手法によって教師データを自動生成する場合
のそれぞれについて,教師データの量と分類性能の関係を調
査する.
なお,両者で教師データ以外の実験条件を合わせるために,
この実験では,モデルには通常のナイーブベイズ・モデルを
用いた.


実験結果を\fig{datasize}に示す.横軸が学習データ量(対
  数スケール)であり,縦軸が11点平均適合率である.
どちらの実験結果についても,まず今回の実験において最大
で利用可能なデータ量(人手ラベルの場合:レビュー2,000件,
  提案ラベルの場合:レビュー約347,000件)から性能測定を
開始し,そこから一部の学習データを無作為に削除すること
で使用できる学習データ量がより少ない環境を設定して,こ
れを繰り返しながらグラフをプロットした.

\begin{figure}[t]
 \begin{center}
 \includegraphics{20-5ia3f6.eps}
 \end{center}
 \caption{教師データ量の影響}
 \label{fig:datasize}
\end{figure}

\fig{datasize}から,まず,どちらの手法においてもデータ
量を増やすことで性能が向上することが確認できる.データ
量が同じ場合は,当然のことながら,人手による方法の方が
良い性能となる.しかし,提案手法によってデータ量を増加
させることで,今回の場合は10,000件までデータ量を増やし
た時点で両者の性能が同等となり,さらにデータ量を増やす
ことで提案手法が人手による手法を上回ることができた.

この実験結果は,あくまでひとつのケース・スタディであり,
具体的な数値自体に意味を求めることは困難であると考えられる.
しかし,この結果は,人手によって十分な教師データが作成
できない状況においては,自動生成手法を適用することで得
られる教師データの量的利点という恩恵を受けられることを
示唆していると言える.


\section{関連研究}
\label{sec:related}

従来から,評判分析に関する研究を中心にして,意見を好評/
不評に分類する研究が多くなされている\cite{sa2} .しかし,
本論文では,応用面を重視した際,主に製品やサービスを提
供する企業にとっては意見の好不評という側面だけでは十分
でないことから,クレームという好不評とは異なる観点を導
入し,意見を含むテキストからクレームという特定の意見を
検出する手法について述べた.
我々と同様に好評/不評以外の意見に着目した研究には,第
\sec{hajimeni}で述べた永井らの研究\cite{nagai1,nagai2}
の他にも幾つか存在する.
例えば,金山ら\cite{kanayama2005a}は,テキストから好評/
不評の評判に加えて要望を抽出する手法を提案している.彼
らの手法は,文に含まれる評判や要望を意図フレームと呼ば
れる独自の形式に自動的に変換しつつ抽出するようになって
おり,この変換・抽出処理において,既存の機械翻訳機構を
再利用している.
彼らの抽出対象である評判,要望の中に本研究におけるクレー
ムも含まれていると考えられるが,彼らの手法は,機械翻訳
機構が内部的に備える各種の言語知識のもとに成立しており,
運用には人手による多大な管理負荷を要すると考えられる.
一方で,本研究では極力人手の負荷を軽減することを指向し
ており,金山らの手法とはアプローチの方向性が異なる.
また,他の関連研究として,自由記述アンケートから要求や
要望を判定することに特化した大塚ら\cite{otsuka2004a} や
山本ら\cite{yamamoto2006a}の研究がある.彼らの論文中に
定義がないため厳密にはよくわからないが,彼らの扱ってい
る要求や要望といった意見の分類クラスは,我々のクレーム
の一部分に該当すると考えられる.
Goldbergら\cite{goldberg2009a}は,新年の願い事が集めら
れたテキストコーパスからWish(願望)を機械学習を用いて
自動抽出する研究を行っている.

本研究では,レビュー中の各文をクレーム/非クレームに分
類する課題に対して,ナイーブベイズ・モデルを採用し,デー
タ特性に合わせて,その拡張を行った.拡張モデルでは,
文間の周辺文脈をモデルに適切に反映させることができる.
ここで,文書中の各文を対象とした分類問題を,文書中の文
系列に対するラベリング問題とみなすことで,条件付確率場
(Conditional Random Fileds; CRF) \cite{lafferty2001a}
のような,より高度なモデルを適用することについて検討す
る.
まず,\sec{gen}で述べたデータ生成過程では,文書内のすべ
ての文に対してラベルを付与するわけではなく,ある特定の
文のみにラベルを付与することで教師データを作成する.そ
のため,CRFのような系列の構成要素についての全てのラベル
を必要とするようなモデルは本研究の設定では直接は適用で
きない.
坪井ら\cite{tsuboi2009a} によって,部分的なアノテーショ
ン情報からCRFの学習を行う手法が提案されており,この手法
を適用することは不可能ではないが,彼らの手法を適切に適
用するにあたり,アノテーションされている部分は人手によ
る信頼性の高い情報であるという暗黙的な仮定が必要である
と考えられ,データの自動生成を前提とする本研究の設定と
は相性が良くないと考えられる.


\section{おわりに}
\label{sec:owarini}

本論文では,レビュー文書からクレームが記述された文を自
動検出する手法として,極力人手の負荷を軽減することを指
向した次の 2 つの手法を提案した.
(1) 評価表現と文脈一貫性に基づく教師データ自動生成手法.
(2) 自動生成された教師データの特性を踏まえたナイーブベイズ・モデルの拡張手法.
そして,評価実験を通して,これらの提案を組合せ,検出対象となる
文の周辺文脈の情報を適切に捉えることで,クレーム文の検
出精度を向上させることができることを示した.
また,人手によって十分な教師データが作成できない状況に
おいては,提案したデータ自動生成手法を適用することで得
られる教師データの量的恩恵を受けられることを示した.

本論文で議論ができなかった今後の課題としては,以下のよ
うな項目があげられる.

\begin{itemize}
\item 分類クラスの事前分布について:
ナイーブベイズ・モデルでは,\eq{eq1}にあるように,分類
クラスの事前分布$P(c)$の情報を考慮する.しかし,本研究
のように教師データを自動生成する際は事前分布$P(c)$はデー
タ自動生成手法に依存しており,本研究の場合では,利用す
る評価表現辞書の特徴に依存することになる.
今後,評価表現辞書および事前分布$P(c)$と検出性能との関
係について考察することが必要である.

\item 各種のパラメータ調整について:
本研究において,幾つかのパラメータは恣意的に指定してい
る.例えば,考慮する周辺文脈の長さについて評価実験では
可変させていたが,それらは,データ生成,モデル学習,分
類の各過程で同期させている.しかし,原理的にはデータ生
成時のみ文脈長を延長するといった設定も可能であり,これ
らの最適な調整は今後の課題である.

\item 学習アルゴリズムについて:
本研究では基本モデルとしてナイーブベイズ・モデルを採用
して議論を進めたが,同様の議論を Support Vector
Machine (SVM) \cite{vapnik1995a} のような別の学習アルゴ
リズムを用いて行うことも興味深い.ただし,SVMにはモデル
の学習速度が遅いという欠点がある.そのため,提案手法の
利点である大規模な教師データを自動生成できるという点を
活かすためにはSVMの高速学習を含めた総合的な検討が必要で
ある.

\item クレームの内容分類について:
本研究はクレーム検出をクレームであるか否かという 2 値分
類問題として扱った.しかし,実利用環境で検出されたクレー
ムを企業内で活かしていくには,クレーム内容も合わせて自
動分類できることが望ましい.例えば,対象が宿泊施設の場
合では,「部屋」や「食事」といったクレームの対象に関す
る分類クラスを別途設定し,これらも同時に考慮した検出モ
デルを検討することも興味深い.

\item 見逃し状況について:
クレームを見逃す状況として,本論文では,レビュー文書内
に部分的に現れるクレームの見逃しについて扱った.しかし,
第\sec{hajimeni}でも述べたように,多対一型のコミュニケー
ションに起因する見逃しへの対処も重要である.今後,多対
一型のコミュニケーションに起因する見逃しに対する提案手
法の適用可能性についても検討したい.

\end{itemize}



\acknowledgment

本研究を遂行するにあたり,楽天株式会社楽天技術研究所の
新里圭司氏,平手勇宇氏,山田薫氏から示唆に富む多くの助
言を頂きました.諸氏に深く感謝いたします.また,実験に
あたり,楽天トラベル株式会社から施設レビューデータを提
供して頂きました.ここに記して感謝の意を表します.

\bibliographystyle{jnlpbbl_1.5}
\begin{thebibliography}{}

\bibitem[\protect\BCAY{Goldberg, Fillmore, Andrzejewski, Xu, Gibson, \BBA\
  Zhu}{Goldberg et~al.}{2009}]{goldberg2009a}
Goldberg, A.~B., Fillmore, N., Andrzejewski, D., Xu, Z., Gibson, B., \BBA\ Zhu,
  X. \BBOP 2009\BBCP.
\newblock \BBOQ May All Your Wishes Come True: A Study of Wishes and How to
  Recognize Them.\BBCQ\
\newblock In {\Bem Proceedings of the Human Language Technology Conference and
  the North American Chapter of the Association for Computational Linguistics},
  \mbox{\BPGS\ 263--271}.

\bibitem[\protect\BCAY{乾\JBA 奥村}{乾\JBA 奥村}{2006}]{inui}
乾孝司\JBA 奥村学 \BBOP 2006\BBCP.
\newblock テキストを対象とした評価情報の分析に関する研究動向.\
\newblock \Jem{自然言語処理}, {\Bbf 13}  (3), \mbox{\BPGS\ 201--241}.

\bibitem[\protect\BCAY{金山\JBA 那須川}{金山\JBA 那須川}{2005}]{kanayama2005a}
金山博\JBA 那須川哲哉 \BBOP 2005\BBCP.
\newblock 要望表現の抽出と整理.\
\newblock \Jem{言語処理学会第 11 回年次大会発表論文集}, \mbox{\BPGS\ 660--663}.

\bibitem[\protect\BCAY{Lafferty, McCallum, \BBA\ Pereira}{Lafferty
  et~al.}{2001}]{lafferty2001a}
Lafferty, J., McCallum, A., \BBA\ Pereira, F. \BBOP 2001\BBCP.
\newblock \BBOQ Conditional random fields: Probabilistic models for segmenting
  and labeling sequence data.\BBCQ\
\newblock In {\Bem Proceedings of the 29th Internatinal Conference on Machine
  Learning}, \mbox{\BPGS\ 282--289}.

\bibitem[\protect\BCAY{Manning \BBA\ Schutze}{Manning \BBA\ Schutze}{1999}]{nv}
Manning, C.~D.\BBACOMMA\ \BBA\ Schutze, H. \BBOP 1999\BBCP.
\newblock {\Bem Foundations of Statistical Natural Language Processing}.
\newblock The MIT Press.

\bibitem[\protect\BCAY{Manning, Raghavan, \BBA\ Schutze}{Manning
  et~al.}{2008}]{iir}
Manning, C.~D., Raghavan, P., \BBA\ Schutze, H. \BBOP 2008\BBCP.
\newblock {\Bem Introduction to Information Retrieval}.
\newblock Cambridge University Press.

\bibitem[\protect\BCAY{永井\JBA 高山\JBA 鈴木}{永井 \Jetal }{2002}]{nagai1}
永井明人\JBA 高山泰博\JBA 鈴木克志 \BBOP 2002\BBCP.
\newblock 単語共起照合に基づくクレーム抽出方式の改良.\
\newblock \Jem{情報科学技術フォーラム}, \mbox{\BPGS\ 113--114}.

\bibitem[\protect\BCAY{永井\JBA 増塩\JBA 高山\JBA 鈴木}{永井 \Jetal
  }{2003}]{nagai2}
永井明人\JBA 増塩智宏\JBA 高山泰博\JBA 鈴木克志 \BBOP 2003\BBCP.
\newblock インターネット情報監視システムの試作.\
\newblock \Jem{情報処理学会研究報告. 自然言語処理研究会報告}, {\Bbf 2003}
  (23), \mbox{\BPGS\ 125--130}.

\bibitem[\protect\BCAY{那須川\JBA 金山}{那須川\JBA 金山}{2004}]{nasukawa}
那須川哲哉\JBA 金山博 \BBOP 2004\BBCP.
\newblock 文脈一貫性を利用した極性付評価表現の語彙獲得.\
\newblock \Jem{情報処理学会研究報告. 自然言語処理研究会報告}, {\Bbf 2004}
  (73), \mbox{\BPGS\ 109--116}.

\bibitem[\protect\BCAY{Nigam \BBA\ Hurst}{Nigam \BBA\ Hurst}{2004}]{nigam2004a}
Nigam, K.\BBACOMMA\ \BBA\ Hurst, M. \BBOP 2004\BBCP.
\newblock \BBOQ Towards a Robust Metric of Opinion.\BBCQ\
\newblock In {\Bem AAAI Spring Symposium on Exploring Attitude and Affect in
  Text: Theories and Applications}, \mbox{\BPGS\ 98--105}.

\bibitem[\protect\BCAY{大塚\JBA 内山\JBA 井佐原}{大塚 \Jetal
  }{2004}]{otsuka2004a}
大塚裕子\JBA 内山将夫\JBA 井佐原均 \BBOP 2004\BBCP.
\newblock 自由回答アンケートにおける要求意図判定基準.\
\newblock \Jem{自然言語処理}, {\Bbf 11}  (2), \mbox{\BPGS\ 21--66}.

\bibitem[\protect\BCAY{Pang \BBA\ Lee}{Pang \BBA\ Lee}{2008}]{sa2}
Pang, B.\BBACOMMA\ \BBA\ Lee, L. \BBOP 2008\BBCP.
\newblock {\Bem Opinion Mining and Sentiment Analysis - Foundations and Trends
  in Information Retrieval Vol.2, Issue 1-2}.
\newblock Now Publishers Inc.

\bibitem[\protect\BCAY{高村\JBA 乾\JBA 奥村}{高村 \Jetal }{2006}]{takamura}
高村大也\JBA 乾孝司\JBA 奥村学 \BBOP 2006\BBCP.
\newblock スピンモデルによる単語の感情極性抽出.\
\newblock \Jem{情報処理学会論文誌}, {\Bbf 47}  (2), \mbox{\BPGS\ 627--637}.

\bibitem[\protect\BCAY{坪井\JBA 森\JBA 鹿島\JBA 小田\JBA 松本}{坪井 \Jetal
  }{2009}]{tsuboi2009a}
坪井祐太\JBA 森信介\JBA 鹿島久嗣\JBA 小田裕樹\JBA 松本裕治 \BBOP 2009\BBCP.
\newblock
  日本語単語分割の分野適応のための部分的アノテーションを用いた条件付確率場の学習.\
\newblock \Jem{情報処理学会論文誌}, {\Bbf 50}  (6), \mbox{\BPGS\ 1622--1635}.

\bibitem[\protect\BCAY{Vapnik}{Vapnik}{1995}]{vapnik1995a}
Vapnik, V.~N. \BBOP 1995\BBCP.
\newblock {\Bem The Nature of Statistical Learning Theory}.
\newblock Springer.

\bibitem[\protect\BCAY{山本\JBA 乾\JBA 高村\JBA 丸元\JBA 大塚\JBA 奥村}{山本
  \Jetal }{2006}]{yamamoto2006a}
山本瑞樹\JBA 乾孝司\JBA 高村大也\JBA 丸元聡子\JBA 大塚裕子\JBA 奥村学 \BBOP
  2006\BBCP.
\newblock 文章構造を考慮した自由回答意見からの要望抽出.\
\newblock \Jem{言語処理学会第 12
  回年次大会併設ワークショップ「感情・評価・態度と言語」}.

\end{thebibliography}


\begin{biography}
\bioauthor{乾  孝司}{
2004年奈良先端科学技術大学院大学情報科学研究科博士課程
修了.日本学術振興会特別研究員,東京工業大学統合研究院
特任助教等を経て,2009年筑波大学大学院システム情報工学研究科
助教.現在に至る.博士(工学).近年はCGMテキストに対する評判分析に興味をもつ.
}
\bioauthor{梅澤 佑介}{
2011 年筑波大学情報学群情報メディア創成学類卒業.2013
年筑波大学大学院システム情報工学研究科コンピュータサイ
エンス専攻修了.同年4 月から株式会社ヤフー.在学中は
自然言語処理の研究に従事.
}
\bioauthor{山本 幹雄}{
1986年豊橋技術科学大学大学院修士課程修了.
同年株式会社沖テクノシステムズラボラトリ研究開発員.
1988年豊橋技術科学大学情報工学系教務職員.1991年同助手.
1995年筑波大学電子・情報工学系講師.1998年同助教授.
2008年筑波大学大学院システム情報工学研究科教授.
博士(工学).自然言語処理の研究に従事.
言語処理学会,人工知能学会,ACL各会員.
}
\end{biography}

\biodate



\end{document}
