\documentstyle[jnlpbbl,multicol]{jnlp_j_b5}

\setcounter{page}{37}
\setcounter{巻数}{2}
\setcounter{号数}{4}
\setcounter{年}{1995}
\setcounter{月}{10}
\受付{1994}{10}{21}
\再受付{1995}{1}{10}
\再々受付{1995}{3}{6}
\採録{1995}{3}{23}

\setcounter{secnumdepth}{2}

\title{日本語形態素解析システムのための形態素文法}
\author{渕 武志\affiref{u-tokyo}\affiref{isl.ntt}
 \and 米澤 明憲\affiref{u-tokyo}}

\headauthor{渕 武志・米澤 明憲}
\headtitle{日本語形態素解析システムのための形態素文法}

\affilabel{u-tokyo}{東京大学理学部 情報科学科}
{Department of Information Science, Faculty of Science, University of Tokyo}

\affilabel{isl.ntt}{現在,NTT情報通信研究所}
{NTT Information and Communication Systems Laboratories}

\jabstract{
本稿では,動詞の語尾変化について体系的な扱いが可能な派生文法に基づいて,
日本語形態素解析システムのための形態素文法を記述した。但し,派生文法に
おける音韻単位での扱いを日本語の文字単位の扱いに変更する方法を示し,よ
り形態素解析処理に適した形で記述した.さらに,これを実働する形態素解
析システムに適用し,EDRコーパスと比較することによって精度を測定した.}

\jkeywords{形態素解析, 日本語, 形態素, 派生文法, 自然言語処理}

\etitle{A Morpheme Grammar\\for Japanese Morphological Analyzers}
\eauthor{Takeshi Fuchi\affiref{u-tokyo}\affiref{isl.ntt}
 \and Akinori Yonezawa\affiref{u-tokyo}} 

\eabstract{
This paper shows a morpheme grammar for Japanese morphological
analyzers. The grammar bases on Derivational Grammar in which the
inflection of Japanese verbs can be managed systematically.  Though
Derivational Grammar uses phonological description, this paper shows a
method to process Japanese character strings di- rectly. A Japanese
morphological analyzer with our grammar is evaluated with EDR corpus.}

\ekeywords{Morphological Analysis, Japanese, Morpheme, Derivational Grammar,
Natural Language Processing}


\begin{document}
\maketitle


\section{まえがき}
形態素解析処理は自然言語処理の基本技術の一つであり,日本語の形態素解析
システムも数多く報告されてい
る\cite{yosimura83}\cite{hisamitu90}\cite{nakamura}\cite{miyazaki}\cite{kitani}\cite{hisamitu94a}\cite{maruyama94}\cite{juman}\cite{nagata}.
しかし,使用している形態素文法について詳しく説明
している文献は少ない.文献\cite{miyazaki} では三浦文法\cite{miura}に
基づいた日本語形態素処理用文法を提案しているが,品詞の体系化と品詞間の
接続ルールの記述形式の提案のみに留まり,具体的な文法記述や実際の解析シ
ステムへの適用にまでは至っていない.公開されている形態素解析シス
テムJUMAN\cite{juman}では,形態素文法は文献\cite{masuoka}に基づくもの
であった.その他の文献は解析のアルゴリズムや,固有名詞や未知語の特定機能
に関する報告で,使用された形態素文法については述べられていない.

言語学の分野で提案されている文法を形態素解析に適用する場合の問題点は,
品詞分類が細か過ぎる点と,ほとんどの場合,動詞の語尾の変化について全て
の体系が与えられていない点である.言語学の分野では文の過剰な受理を避け
るように文法を構築することによって,日本語の詳細な文法体系を解明しよう
とするので,品詞分類が細かくなるのは当然である.しかし,そのために,文
法規則も非常に細かくなり,形態素の統一的な扱いも難しくなる.そこで,本
文法では,「形態素解析上差し支えない」ことを品詞の選定基準とする.つま
り,ある品詞を設定しないが為に,ある文節に関して構文上の性質に曖昧性が
生じる場合に,その品詞を設定する.そして,過剰な受理を許容することと引
き替えに,できる限り形態素を統一的に扱う.

従来の多くの文法では活用という考え方で動詞の語尾変化を説明するが,それ
らの活用形についての規則は,個々の接尾辞について接尾可能な活用形を列挙
するという形になっている.例えばいわゆる学校文法では,「書か」はカ行五
段活用動詞「書く」の未然形であり,否定の接尾辞の「ない」や使役の接尾辞
の「せる」が接尾する等の規則が与えられる.さらに一段活用動詞には「せる」
ではなく「させる」が接尾する等の規則があり,規則が複雑になっている.そ
のため,それらの複雑な規則を吸収するために活用形を拡張し,「書こう」を
意志形としたり,「書いた」を完了形とするような工夫がなされる.しかし,
このように場当たり的に活用形を拡張すると活用形の種類が非常に多くなり,
整合性を保つための労力が大きくなる.

日本語形態素処理における動詞の活用の処理については文
献\cite{hisamitu94b,hisamitu94c}に詳しい.そこでは,音韻論的手法
\cite{bloch,teramura},活用形展開方式,活用語尾分離方式が紹介され,新
たに活用語尾展開方式\footnote{文献\cite{hisamitu94b,hisamitu94c}では,
提案方式と呼ばれている.}が提案されている.音韻論的手法は,子音動詞の
語幹と屈折接辞を音韻単位に分解し,屈折接辞の音韻変化の規則を用いて,活
用を単なる動詞語幹と屈折接辞の接続として捕らえる.
しかし,これまでの音韻論的手
法では,子音動詞についての知見しか得られていなかったために,子音動詞に
接尾する接尾辞と母音動詞に接尾する接尾辞を別々に扱わなければならなかった.
また,音韻単位で処理する必要があると考えられているため,文献
\cite{hisamitu94b,hisamitu94c}でも,処理の効率が落ちるとされている.
一方,活用形展開方式,活用語尾分離方式,活用語尾展開方式は何れも伝統的な学校
文法に基づいている.活用形展開方式は,各動詞についてその活用形を全て展
開して辞書に登録し,それぞれ接尾辞との接続規則を与えるもので,処理速度の
点で有利であるが,登録語数が非常に多くなる上に,接続規則の与え方によっ
ては効率の点でも不利になる可能性がある.活用語尾分離方式は,活用語尾を
別の形態素とし,動詞語幹と活用語尾の接続規則および活用語尾と接尾辞の接続
規則を与えるもので,動詞の屈折形の解析の際に分割数が多くなり,効率の点
で不利である.また,接続規則が非常に複雑になる.活用語尾展開方式は,活
用語尾と接尾辞の組み合わせを形態素とし,これらと動詞語幹との接続規則だけ
を与えるもので,活用語尾分離方式よりも分割数が少なくなり,効率的に有利
であるとされている.しかし,活用形展開方式,活用語尾分離方式,活用語尾
展開方式の共通の問題点は,活用語尾と接尾辞の接続規則が体系的でない点である.
特に活用語尾展開方式では,新しい接尾辞を追加する度に10以上ある動詞の活用
の型それぞれに対する形態素の展開形を追加しなければならない.また,「ら
れ」「させ」といったいわゆる派生的な接尾辞に対してはさらに多くの展開形を
別々の形態素として登録する必要があるはずである\footnote{この点について
は文献中には触れられていない.}.

そこで本文法では,動詞の語尾変化について体系的に扱うことに成功している
派生文法\cite{kiyose}を基にした\footnote{派生文法を基にしたシステムと
しては,文献\cite{nisino}で,何らかの方法で分解した動詞の語尾の構造を
派生文法に基づいて解析するシステムについて報告されているが,形態素解析
システムへの適用は報告されていない.}.派生文法も音韻論的アプローチの
文法であるが,従来のものに対して,連続母音と連続子音の縮退,および内的
連声\footnote{上記の屈折接辞の音韻変化と同じもの}という考え方を用いて,
母音動詞も子音動詞も同様に扱うことができる.しかし,派生文法は音韻論的
手法であるため,形態素解析に適用するには,処理を音韻単位で行う必要があ
るという問題がある.日本語のテキストを処理するような形態素解析システム
では,文字を子音と母音に分けずに日本語の文字でそのまま処理できた方が都
合がよい.本研究では,派生文法における動詞語尾の扱いを日本語の文字単位
で処理できるように変更する方法を見い出すことができた.すると図らずも従
来の活用という考え方に適合する形になることが判明し\footnote{派生文法で
は日本語における活用の考え方を完全に否定している.},これによって,活
用の考えを用いて作られている既存の形態素解析システムに適用することがで
きた.しかも語尾変化についての完全な体系を背後に持つため,新たに認識さ
れた語尾変化に対しても活用形を順次増やす必要がなく,対応する形態素を一
つだけ辞書に登録すれば済むようになった.事実,「食べれる」といったいわ
ゆる「ら抜き表現」や,「書かす」といった口語的な使役表現などもそれぞれ
一つの形態素を追加することで対応できている.このように新しい語尾を簡単
に追加できることから,口語的な語尾の形態素を充実させることができ,口語
的な文章に対しても高い精度で解析できるようになった.また,「食べさせら
れますまい」といった複雑な語尾変化も正確に解析できる.

本研究で開発された形態素解析文法は,文字表記された日本語のテキスト
から言語データを抽出することを主な目的として開発されたものである.従っ
て,日本語の漢字仮名混じりの正しい文\footnote{一般の日本人が許容できる
範囲で正しいという意味で,正式な日本語という意味ではない.}を文節に区
切り,その文節の係り受けの性質を識別することを最優先した解析用の文法と
なっている.
また,形態素の意味的な面を捨象し,過剰な受理を許容することで,形態素の
統一的な扱いをすることに重点を置いている.これはあくまで計算機上へのシ
ステムの構築を容易にするためであり,なんらかの言語学的な主張をする意図
はない.さらに過剰な受理を許容する意味で,この文法は解析用の文法といえ
る.生成等に利用するにはこの過剰な受理が障害になる可能性がある.また,
誤りを含む文の識別に用いるのにも問題がある.本形態素文法はあくまで正し
い文の解析に特化した文法として位置付ける必要がある.

本稿では\ref{system}節で形態素の種類とそれらが満たすべき制約の体系を
説明し,\ref{verb}節で動詞の語尾の扱いについて述べる.\ref{apply}節で
は,それを日本語文字単位の形態素解析向きに変更する方法を示す.さらに,
\ref{detail}節では個別の問題がある語尾について述べ,最後にこの形態素文
法を形態素解析プログラムJUMANに適用した場合の解析性能を評価する.なお,
われわれが作成した形態素文法の形態素解析プログラムJUMAN への適用事例は,
以下のanonymous ftpで入手可能である.但し,評価の際に使用した辞書の一部
について配布に制限のあるものは含まれていない.
\\ camille.is.s.u-tokyo.ac.jp  /pub/member/fuchi/juman-fuchi

\section{形態素の体系} \label{system}
本稿では,形態素文法を品詞間の隣接可能性を示す形で与える.文献
\cite{maruyama94}にあるように,形態素文法も正規文法等で記述した方がよ
り細かい記述ができるが,処理効率や形態素解析システムへの適用の関係でこ
の形に落ちついた.

\begin{table}
\begin{center}
\begin{tabular}{|l|l|}
\hline
属性名 & 属性値 \\
\hline
主属性 & 無$|$動$|$形$|$名$|$数$|$時$|$格$|$尾 \\
係属性 & 無$|$連体$|$連用$|$終止 \\
左隣接属性 & 無$|$動$|$形$|$名$|$数$|$時$|$接続$|$連体$|$連用 \\
右隣接属性 & 無$|$動$|$形$|$名$|$数$|$時$|$接続$|$尾 \\
\hline
\end{tabular}
\end{center}
\caption{形態素の属性}
\label{attribute}
\end{table}

本文法では形態素に対して表\ref{attribute}に示す属性を設定する.逆にこ
のような属性を付加できるような文字の最小の連鎖を形態素と呼ぶ.また,形
態素をその性質によって分類したものを品詞と呼ぶ.但し,システムの解析精
度を上げるために,「かどうか」のように幾つかの形態素から合成され,厳密に
は形態素と言えないものを,一つの形態素として扱う場合がある.形態素
の属性は以下のリストで表記する.\\
[ 主属性, 係属性, 左隣接属性, 右隣接属性 ]\\
主属性は形態素の基本的な性質による分類であり,その形態素を含む文節がど
のような「受け」を構成するかを示す.係属性はその形態素で終わる文節がど
のような「係り」を構成するかを示す.文節の「係り」の種類としては「連体」
と「連用」を設定する.「連体」は主属性が「名」「数」「時」の形態素に対
して係り,「連用」は「述語\footnote{「述語」は名詞+名詞接尾辞,動詞語
幹+動詞語尾,形容詞語幹+形容詞語尾,名詞+句読点,数詞+句読点および
時詞+句読点によって構成される.}」に対して係る.そして,「連用」の
「係り」となる文節の末尾に位置する形態素の係属性の値を「連用」とし,そ
のような形態素を「連用形」と呼ぶ.同様に「連体」の「係り」となる文節の
末尾に位置する形態素の係属性の値を「連体」とし,そのような形態素を「連
体形」と呼ぶ.さらに,「係り」を構成しない文節の末尾に位置する形態素の
係属性の値は「終止」とし,そのような形態素を「終止形」と呼ぶ.

左右の隣接属性は隣接する二つの形態素が満たすべき制約を表している.文を左か
ら右に記述した場合に,隣接している形態素の内,左にある形態素の属性が
[X1,Y1,L1,R1]であり,右にある形態素の属性が[X2,Y2,L2,R2]であったな
らば,以下が成り立つ必要がある.\\
R1 $\in$ X2 $\cup$ Y2$ \cup$ L2 かつ L2 $\in$ X1$ \cup$ Y1 $\cup$ R1.\\
大まかには,左にある形態素の右隣接属性は,その値がすぐ右の形態素の主属性,
係属性,左隣接属性のいずれかの値と同じである場合に,それらの形態素
が隣接可能であることを示す.また,右にある形態素の左隣接属性は,その値
がすぐ左の形態素の主属性,係属性,右隣接属性のいずれかの値と同じである
場合に,それらの形態素が隣接可能であることを示す.また,右隣接属性
が「接続」である形態素を「接続形」と呼ぶ.

これらの属性は,その取りうる値の組合せの内,一部は実在しない.表
\ref{meisi-list}から表\ref{sonota-list}に実在する品詞を示す.以下で,
それぞれの品詞について説明する.なお,それぞれの品詞名は,なるべく統語
的な性質を反映した名前になるように本研究で独自に与えたものである.また,
表中の補足の欄で与える例の内,アルファベットで表記してあるもの
は,\ref{verb}節で説明する連続母音縮退,連則子音の縮退,および内的連声
との関連で,そのままでは日本語文字表記にならないものである.


\subsection{名詞,連用名詞,補助名詞,時詞,数詞}
\begin{table*}
\begin{center}
\begin{tabular}{|l|l|l|l|}
\hline
形態素の属性 & 品詞名 & 略記 & 補足 \\
\hline
[ 名, 無, 無, 無 ]& 名詞 & 名 & 人名, 地名, 物名 \\
			&&	& 動作名 ex. 跳躍 \\
			&&	& 形状名 ex. 静か \\
[ 名, 連用, 無, 無 ]	& 連用名詞 & 名/連用 & ex. 道中, 半分, 反面 \\
[ 名, 連用, 連体, 無 ]& 補助名詞 & 補名 & ex. の, ん, 際 \\
[(名,時), 連用, 無, 無 ]& 時詞   & 時      & 時の名称 ex. 今日, 夏 \\
[(名,数), 無, 無, 無 ]& 数詞     & 数      & 数字 ex. 1, 一, 壱 \\
\hline
\end{tabular}
\end{center}
\caption{名詞,数詞,時詞,補助名詞}
\label{meisi-list}
\end{table*}

表\ref{meisi-list}に名詞関連の形態素を示す.
名詞の属性は[名,無,無,無]である.主属性が「名」である事は,連体
の係りを受けることを意味する.また,隣接属性が左右とも「無」であること
は,名詞が単独で文節を構成可能であることを示す.係属性が「無」であるこ
とは,名詞自身では係りの種類を指定しないことを示す.特に,直後に名詞が
くる場合には,複合して一つの名詞を形成する.

名詞はさらに細かく「動作名詞」や「形状名詞」などへの分類が可能である.
これらの細分類には「する」が接尾可能であるとか,「な」が接尾可能である
などの統語的な振る舞いの違いが見られる.しかし,基本的にはこれらの細分
類は意味的なものを反映している.従って,発話者がある単語にどのような意
味を込めるかによって変動しうるものである.聞き手の立場からは逆にそのよ
うな使われ方から発話者が込めた意味を読みとる必要がある.従って,解析の
場合には,解析の際に不都合がない限り,このような細分類は必要でない
と考える.特に一般的に形容動詞といわれるものも,文献\cite{tokieda}と同
様に,形容的な意味合いが強い名詞として,名詞に含める.

連用名詞は属性が[名,連用,無,無]の品詞で,係りの種類を「連用」に指定
する名詞の一種である.「半分冗談で言った」の例のように,直後に名詞
が来ても複合名詞を形成せず,連用の文節を形成する.但し「冗談半分で言っ
た」のように連用名詞が名詞の後に来る場合には複合名詞を形成する.

補助名詞は述語の連体形の直後のみに現れる\footnote{従って,左隣接属性
が「連体」になる.}という性質以外は連用名詞と同様な振る舞いをする.代
表的な補助名詞は「の」で,属性は[名, 連用, 連体, 無]である.「の」と
「ん」は,実際には述語の連体形の直後にしか現れ得ないなど,さらに細かい
制約があるが,本文法ではそれらの制約を表していない.

時詞は連用名詞の一種であるが,「昨年夏に」などのように時詞が連続した場
合には複合すると考え,別に設定した.属性は[(名,時),連用,無,無]である.
主属性が(名,時)となっているのは,両方の属性を持つことを表す.

数詞は名詞の一種とも見なせるが,数詞のみに接尾する接尾辞が存在し,
これを区別しないと曖昧性が生ずる場合がある.そこで,属性を[(名,数),
無,無,無]として名詞とは別に数詞を設定した.


\subsection{格接尾辞}
\begin{table*}
\begin{center}
\begin{tabular}{|l|l|l|l|}
\hline
形態素の属性 & 品詞名 & 略記 & 補足 \\
\hline
[格,連用,名,無] & 格接尾辞/連用形 & 格尾/連用 & 格助詞 ex. が, を,  に,  で \\
[格,連体,名,無] & 格接尾辞/連体形 & 格尾/連体 & 属格, 助詞 ex. の, や, か, と \\
\hline
\end{tabular}
\end{center}
\caption{格接尾辞}
\label{kakusetsubi-list}
\end{table*}

表\ref{kakusetsubi-list}に格接尾辞を示す.格接尾辞は名詞に接尾して連用
または連体の文節を形成する.格接尾辞が次節の名詞接尾辞と異なる点は,述
語を形成しない点である.終止形は述語を形成してしまうため,この品詞には
終止形が存在しない.文節に区切る目的からは格接尾辞と名詞接尾辞を区別す
る必然性はないが,構文解析での利用を考慮してこのように設定した.しかし
「で」などは格接尾辞と名詞接尾辞の両方に所属すると考えられ,しかも形態
素レベルで区別する方法がない.また「と」に関しては,連用と連体の両方の
用法があると考えられ,これも形態素レベルでは区別ができない.これらにつ
いてはその取り扱いを\ref{detail}節で改めて述べる.

\subsection{名詞接尾辞}
\begin{table*}
\begin{center}
\begin{tabular}{|l|l|l|l|}
\hline
形態素の属性 & 品詞名 & 略記 & 補足 \\
\hline
[尾,終止,名,無] & 名詞接尾辞/終止形 & 名尾/終止 & ex. だ, です, でしょう,  だろ \\
[尾,連用,名,無] & 名詞接尾辞/連用形 & 名尾/連用 & ex. で, でして \\
[尾,連体,名,無] & 名詞接尾辞/連体形 & 名尾/連体 & ex. {\dg な}, だった \\
[尾,無,名,接続] & 名詞接尾辞/接続形 & 名尾/接続 & ex. だ, です, でしょう \\
\hline
\end{tabular}
\end{center}
\caption{名詞接尾辞}
\label{meisisetsubi-list}
\end{table*}

表\ref{meisisetsubi-list}に名詞接尾辞を示す.
名詞接尾辞は名詞に接尾して述語の文節を形成する.
従って「連用」の係りを受ける.

本文法で名詞接尾辞の連体形としている\hspace{-0.3mm}「な」\hspace{-0.3mm}については,\hspace{-1mm}「学生\hspace{-0.5mm}\underline{な}ので」\hspace{-0.5mm}と\hspace{-0.5mm}「健康\hspace{-0.1mm}\underline{な}\hspace{-0.2mm}人」
では意味的に異なるもの
と考えられるが,前者の「の」を補助名詞と考えると,両者とも統語的
には同一に扱える.さらに本文法では「な」は話し手が形容的な意味合いを付
加したあらゆる名詞に接尾可能であると考える.また,「体が健康な人」の例
を考えると「名詞+な」で述語を形成していることが分かる.従って,「な」
は格接尾辞連体形ではなく,名詞接尾辞連体形とする.

接続形の形態素は全て連体形や終止形の形態素と表記が同じであるが,接続形
では必ず接続接尾辞が接尾し,逆に連体形や終止形には接続接尾辞が接尾しな
いので,両者は区別可能である.このことは動詞接尾辞や形容詞接尾辞の接続形に
ついても同じである.

\subsection{動詞語幹,動詞接尾辞}
\begin{table*}
\begin{center}
\begin{tabular}{|l|l|l|l|}
\hline
形態素の属性 & 品詞名 & 略記 & 補足 \\
\hline
[動,無,無,尾]   & 動詞語幹        & 動 & ex. 食べ, 歩k, 走r, 思w \\
\hline
[尾,終止,動,無] & 動詞接尾辞/終止形 & 動尾/終止 & ex. ru, ita, you, ina \\
[尾,連用,動,無] & 動詞接尾辞/連用形 & 動尾/連用 & ex. i, ite, eba \\
[尾,連体,動,無] & 動詞接尾辞/連体形 & 動尾/連体 & ex. ru, ita \\
[尾,無,動,接続] & 動詞接尾辞/接続形 & 動尾/接続 & ex. ru, ita \\
\hline
\end{tabular}
\end{center}
\caption{動詞語幹,動詞接尾辞}
\label{dousi-list}
\end{table*}

表\ref{dousi-list}に動詞関連の形態素を示す.動詞語幹の属性は[動,無,無,尾]
であり,右隣接属性が「尾」なので,「尾」の属性を持つものが接尾しな
ければならない.実際に接尾できるのは動詞接尾辞
もしくは派生接尾辞の一部である.動詞語幹は動詞接尾辞を伴って動詞を形成
する.動詞語幹には子音で終わるものと母音で終わるものがある.例えば「書
く」の語幹は「kak」であり,「食べる」の語幹は「tabe」である.実際に存
在する動詞語幹の末尾の子音は,K, G, S, T, N, B, M, R, Wの九個である.

動詞接尾辞は動詞語幹または「動」の属性値を持つ派生接尾辞に接尾して,述
語を形成する.動詞接尾辞には子音で始まるものと,母音で始まるものがある.
例えば「書く」の動詞接尾辞は後述するように「ru」であると見なせ,「書き
ます」の動詞接尾辞は「imasu」であると見なせる.実際に存在する動詞接尾
辞の先頭は,A, I, U, E, YO, RUである.さらに動詞語幹に接尾する派生接
尾辞にはRA, SA, REで始まるものがある.

これらの語幹に接尾辞が接尾する場合には,連続母音縮退,連続子音縮退,内的
連声という規則的な変換がおこる.その詳細については\ref{verb}節で述べる.

\subsection{形容詞語幹,形容詞接尾辞}
\begin{table*}
\begin{center}
\begin{tabular}{|l|l|l|l|}
\hline
形態素の属性 & 品詞名 & 略記 & 補足 \\
\hline
[形,無,無,尾] & 形容詞語幹 & 形 & ex. 美し, 高 \\
\hline
[尾,終止,形,無] & 形容詞接尾辞/終止形 & 形尾/終止 & ex. い, かった, かれ \\
[尾,連用,形,無] & 形容詞接尾辞/連用形 & 形尾/連用 & ex. く, くて, ければ \\
[尾,連体,形,無] & 形容詞接尾辞/連体形 & 形尾/連体 & ex. い, かった \\
[尾,無,形,接続] & 形容詞接尾辞/接続形 & 形尾/接続 & ex. い, かった \\
\hline
\end{tabular}
\end{center}
\caption{形容詞語幹,形容詞接尾辞}
\label{keiyousi-list}
\end{table*}

表\ref{keiyousi-list}に形容詞関連の形態素を示す.
形容詞語幹の属性は[形,無,無,尾]であり,右隣接属性が「尾」なので,
「尾」の属性を持つものが接尾しなければならない.実際に接尾できるのは
形容詞接尾辞もしくは派生接尾辞の一部である.形容詞
語幹は形容詞接尾辞を伴って形容詞を形成する.

形容詞接尾辞は形容詞語幹または「形」の属性値を持つ派生接尾辞に接尾する.
基本的には形容詞接尾辞は動詞接尾辞のような語形の変化はなく,そのままの
形で形容詞語幹に接尾する.しかし,形動派生接尾辞「ござr」が形容詞語幹
に接尾する場合にのみ内的連声と呼ばれる語形変化があり,表
\ref{adj_onbin}のようになる.例えば,「たか(高)」という形容詞語幹に
「ござる」が接続する場合には「taka $\rightarrow$ takou」と変形され,
「たこうござる」となる.

\begin{table}
\begin{center}
\begin{tabular}{|l|l|} \hline
内的連声 & 具体例 \\ \hline
-a $\rightarrow$ -ou & たこうござる \\
-i $\rightarrow$ -yuu & うつくしゅうござる \\
-u $\rightarrow$ -ou & さもうござる \\
-o $\rightarrow$ -ou & ほそうござる \\
\hline
\end{tabular}
\end{center}
\caption{形容詞の内的連声}
\label{adj_onbin}
\end{table}

\subsection{連体詞,連用詞,連文詞,終止詞}
\begin{table*}
\begin{center}
\begin{tabular}{|l|l|l|l|}
\hline
形態素の属性 & 品詞名 & 略記 & 補足 \\
\hline
[無,連体,無,無] & 連体詞 & 連体 & 指示語 ex. その \\
[名,連用,無,無] & 連用詞 & 連用 & 副詞 ex. ゆっくり, とても \\
[無,連用,無,無] & 連文詞 & 連文 & 接続詞 ex. しかし, ところで \\
[無,終止,無,無] & 終止詞 & 終止 & 感動詞, 感嘆詞 ex. おはよう, おや \\
\hline
\end{tabular}
\end{center}
\caption{連体詞,連用詞,連文詞,終止詞}
\label{rentaisi-list}
\end{table*}

連体詞は属性が[無,連体,無,無]で,それだけで連体形の文節を構成する形
態素である.代表的なものは「その」などの指示を表す形態素である.連体詞
はどのような係りも受けない.

連用詞は一般的には副詞と言われるもので,属性が[名,連用,無,無]で,
それだけで連用形の文節を構成する形態素である.これには「彼ののんびりに
はいらいらさせられる」などに見られる名詞的な用法があるため,主属性を
「名」とした.そのため,属性としては連用名詞と同じであるが,名詞の直後
に来ても複合名詞を形成しない点が連用名詞とは異なる.例えば「車ゆっくり
走らせて下さい」という文では「車ゆっくり」という名詞であるとは受け取
られない.その他に,「とてもゆっくり走らせた」の例では「とても」が「ゆっ
くり」に係ると考えられるが,本文法では「とても」は「走らせた」に係ると
考えることにして,連用詞に係る連用詞を設定していない.これに関しては
\ref{detail}節でも触れる.

連文詞は一般的には接続詞と言われるもので,文と文をつなぐ働きをする.属性は
[無,連用,無,無]で連用詞と似ているが,名詞的な用法がない.また,普通
は文頭に現れる.

終止詞は一般的には感動詞や感嘆詞と言われるもので,単独で文を形成し,係
り受けを形成しない.

\subsection{接頭辞}
\begin{table*}
\begin{center}
\begin{tabular}{|l|l|l|l|}
\hline
形態素の属性 & 品詞名 & 略記 & 補足 \\
\hline
[無,無,無,名] & 名詞接頭辞 & 頭名 & ex. お, ご, 前, 元 \\
[無,無,無,時] & 時詞接頭辞 & 頭時 & ex. 翌, 昨, 来 \\
[無,無,無,数] & 数詞接頭辞 & 頭数 & ex. 第, 約, 計 \\
[無,無,無,動] & 動詞接頭辞 & 頭動 & ex. お, ぶち \\
[無,無,無,形] & 形容詞接頭辞 & 頭形 & ex. お,うすら \\
\hline
\end{tabular}
\end{center}
\caption{接頭辞}
\label{settouji-list}
\end{table*}

表\ref{settouji-list}に接頭辞を示す.
接頭辞は,ある形態素に接頭する形態素で,文節全体の係り受けの性質には影響
を与えない.本文法では,名詞,時詞,数詞,動詞,形容詞に接頭する接頭辞
を設定する.

\subsection{派生接尾辞}
\begin{table*}
\begin{center}
\begin{tabular}{|l|l|l|l|}
\hline
形態素属性 & 品詞 & 略記 & 補足 \\
\hline
[名,無,名,無]         & 名名派生接尾辞 & 名名 & ex. さん, 製, 的 \\
[動,無,名,尾]         & 名動派生接尾辞 & 名動 & ex. する, できr, ぶr \\
[形,無,名,尾]         & 名形派生接尾辞 & 名形 & ex. らし, くさ \\
\hline
[名,連用,時,無]       & 時名派生接尾辞 & 時名 & ex. 前, 中, 下旬 \\
[名,無,数,無]         & 数名派生接尾辞 & 数名 & ex. 人, 個, 姉妹 \\
[時,連用,数,無]       & 数時派生接尾辞 & 数時 & ex. 年, 日, 秒 \\
[数,無,数,無]         & 数数派生接尾辞 & 数数 & ex. 万, 億, 兆 \\
\hline
[(尾,名),連用,動,無]  & 動名派生接尾辞 & 動名 & ex. i手, aなさそう \\
[(尾,動),無,動,尾]    & 動動派生接尾辞 & 動動 & ex. sase, rare, imakur \\
[(尾,形),無,動,尾]    & 動形派生接尾辞 & 動形 & ex. aな, iた, iにく\\
\hline
[(尾,名),連用,形,無]  & 形名派生接尾辞 & 形名 & ex. そう \\
[(尾,動),無,形,尾]    & 形動派生接尾辞 & 形動 & ex. がr \\
[(尾,形),無,形,尾]    & 形形派生接尾辞 & 形形 & ex. かな \\
\hline
[(尾,名),連用,接続,無]& 接名派生接尾辞 & 接名 & ex. か, かどうか  \\
[(尾,動),無,接続,尾]  & 接動派生接尾辞 & 接動 & ex. にすぎ \\
[(尾,形),無,接続,尾]  & 接形派生接尾辞 & 接形 & ex. らし \\
\hline
\end{tabular}
\end{center}
\caption{派生接尾辞}
\label{hasei-list}
\end{table*}

表\ref{hasei-list}に派生接尾辞の一覧を示す.派生接尾辞は,名詞や動詞語
幹,形容詞語幹または接続形に接尾して,品詞を変換し,新たに語幹を形成す
る接尾辞である.これらの派生接尾辞は派生文法\cite{kiyose}を参考に,本研
究で整理,拡充したものである.品詞の名称から個々の派生接尾辞の働きは明
らかなので,以下では幾つか注意を要するものについてのみ説明する.

名動派生接尾辞は,名詞に接尾して動詞語幹を形成する接尾辞である.代表的
なものが「する」で,これは普通は動作を表す名詞に接尾するが,本文法では
発話者が単語に込める意味によって全ての名詞に接尾可能であるとしている.

動名派生接尾辞は,動詞語幹に接尾して名詞を形成する.これは\ref{verb}節
で説明する動詞接尾辞の基本接続規則のみに従い,内的連声には従わない.この点
は動形派生接尾辞や動動派生接尾辞も同様である.

動動派生接尾辞は,動詞語幹に接尾してまた動詞語幹を形成する.代表的なも
のは使役の「sase」や受身・尊敬・自発・可能の「rare」である.この動動派
生接尾辞は動詞の語幹に次々に接尾して動詞語幹を派生する.例えば「書かせ
られますまい」という表現は,後述する連続母音縮退や連続子音縮退に注意すれば,
「書k/sase/rare/imas/umai」である.その他に,口語的な表現では「食べさ
せる」「書かせる」を「食べさす」「書かす」などともいうが,これは「sas」
という形態素で説明できる.さらに,可能の意味での「食べれる」「書ける」
という表現は「re」という形態素で説明できる.このように最近になって新し
く使われるようになったと考えられる表現でも本文法に沿っていることが分かる.

このような派生接尾辞は,個々の形態素の意味に応じて,接尾可能でない場合
がある.特に動動派生接尾辞では,その順番に明らかに制約が存在する.しか
し,本文法では文法の簡潔性を優先し,それらの制約を反映していない.これ
らは文生成においては解明すべき重要な問題である.

\subsection{接続接尾辞}
\begin{table*}
\begin{center}
\begin{tabular}{|l|l|l|l|}
\hline
形態素属性 & 品詞 & 略記 & 補足 \\
\hline
[尾,終止,接続,無] & 接続接尾辞終止形 & 接尾/終止 & ex. ぜ, もん, の, か \\
[尾,連用,接続,無] & 接続接尾辞連用形 & 接尾/連用 & ex. し, が, ので, のに \\
[尾,連体,接続,無] & 接続接尾辞連体形 & 接尾/連体 & ex. だろう, でしょう \\
[尾,無,接続,接続] & 接続接尾辞接続形 & 接尾/接続 & ex. だろう, でしょう \\
\hline
\end{tabular}
\end{center}
\caption{接続接尾辞}
\label{setsuzoku-list}
\end{table*}

表\ref{setsuzoku-list}に接続接尾辞を示す.
接続接尾辞は名詞接尾辞や動詞接尾辞,形容詞接尾辞の接続形に接尾して連用形,
連体形,終止形,接続形を形成する.これには例えば「かのような」のように
幾つかの形態素から成り立っているみなせるものも含まれている.このような
ものは,成り立ちは確かに幾つかの形態素の組合せと考えら
れるが,使用上は一つの接尾辞として振舞うので,まとめて一つの形態素とし
て扱う.

名詞接尾辞の「だ」に関連して,この品詞の隣接規則には例外があり,接続接
尾辞の連用形と終止形の一部のみが「だ」に接尾可能である.さらに「ので」
「のに」に関して個別の例外があり,これは\ref{detail}節で検討する.

\subsection{末尾接尾辞}
\begin{table*}
\begin{center}
\begin{tabular}{|l|l|l|l|}
\hline
形態素属性 & 品詞 & 略記 & 補足 \\
\hline
[尾,終止,終止,無] & 末尾接尾辞 &  尾尾 & ex. よ, ね, さ, か \\
[尾,終止,連用,無] &&&\\
[尾,終止,名,無] &&&\\
\hline
\end{tabular}
\end{center}
\caption{末尾接尾辞}
\label{matsubi-list}
\end{table*}

表\ref{matsubi-list}に末尾接尾辞を示す.
末尾接尾辞は文の末尾に用いられる接尾辞で,基本的には終止形に接尾し,属
性は[尾,終止,終止,無]である.しかし,連用形や名詞に対しても接尾が可
能であり,[尾,終止,連用,無],[尾,終止,名, 無]の属性も持つと考えら
れる.いずれにしても終止形を構成する.末尾接尾辞にさらに末尾接尾辞が接
尾することは可能であるが,全ての組合せが見られるわけではない.これは形
態素の隣接規則とは別の意味的な制約によるものと考えられるが,解析の立場
からは可能な組合せを洗い出す必要はないと考え,全ての末尾接尾辞が隣接可
能であるとしている.また,頻出する末尾接尾辞の連続に対しては,
一つの形態素として辞書に登録している.

\subsection{引用接尾辞}
\begin{table*}
\begin{center}
\begin{tabular}{|l|l|l|l|}
\hline
形態素属性 & 品詞 & 略記 & 補足 \\
\hline
[尾,連用,終止,無] & 引用接尾辞連用形 &  引用/連用 & ex. と \\
[尾,連用,連用,無] &&&\\
[尾,連用,名,無] &&&\\ \hline
[尾,連体,終止,無] & 引用接尾辞連体形 & 引用/連体 & ex. との \\
[尾,連体,連用,無] &&&\\
[尾,連体,名,無] &&&\\
\hline
\end{tabular}
\end{center}
\caption{引用接尾辞}
\label{in'you-list}
\end{table*}

表\ref{in'you-list}に引用接尾辞を示す.
引用接尾辞の基本的な属性は[尾,連用,終止,無]である.これは終止形に接
尾して連用形を構成することを意味する.しかし,連用形や名詞に対しても接
尾が可能であるため,[尾,連用,連用,無],[尾,連用,名,無]という属性も
持つ.また「首相が辞任したとのニュース」の例のように連体形も存在する.

\subsection{連用接尾辞}
表\ref{ren'you-list}に連用接尾辞を示す.
連用接尾辞連用形の属性は[尾,連用,連用,無]であり,その代表的なものは
「は」「も」「すら」「さえ」「だけ」「まで」等である.これらは連用形の
形態素に接尾して再び連用形を形成する接尾辞である.これらはさらに名詞に
も接尾するので,[尾,連用,名, 無]でもある.また隣接規則に例外があり,
接続接尾辞連用形には接尾しない.さらに連用接尾辞の隣接可能な組み合わせ
は全てのものが存在するわけではなく,何らかの意味的な制約によって制限さ
れていると思われるが,本文法では,その制約を追求することはしていない.

連用接尾辞連体形は用いられることは希であるが存在し,属性は[尾,連用,連
用,無],[尾,連用,名,無]である.

\begin{table*}
\begin{center}
\begin{tabular}{|l|l|l|l|}
\hline
形態素属性 & 品詞 & 略記 & 補足 \\
\hline
[尾,連用,連用,無] & 連用接尾辞連用形 & 用尾/連用 & ex. は, も, すら, まで \\
[尾,連用,名,無] &&&\\
\hline
[尾,連体,連用,無] & 連用接尾辞連体形 & 用尾/連体 & ex. すらの, までの \\
[尾,連体,名,無] &&&\\
\hline
\end{tabular}
\end{center}
\caption{連用接尾辞}
\label{ren'you-list}
\end{table*}


\subsection{間投辞,句読点,括弧}
その他の品詞を表\ref{sonota-list}に挙げる.その中に,文節の切れ目にの
み置くことができる間投辞がある.これに属する代表的な形態素は「ね」であ
る.また句読点や括弧なども間投詞と形態素としての性質は同じで,これらの
品詞の属性は全て[無,無,無,無]である.
\begin{table*}
\begin{center}
\begin{tabular}{|l|l|l|l|}
\hline
形態素属性 & 品詞 & 略記 & 補足 \\
\hline
[無,無,無,無] & 間投辞 & 間投 & ex. ね, さ, よ \\
[無,無,無,無] & 読点 & 読点 & ex. ,  \\
[無,無,無,無] & 句点 & 句点 & ex. .?! \\
[無,無,無,無] & 括弧 & 括弧 & ex. 「」{} \\
\hline
\end{tabular}
\end{center}
\caption{間投辞,句読点,括弧}
\label{sonota-list}
\end{table*}

\subsection{隣接規則の例外}
隣接規則の例外は特殊な用法から生まれている.一つは「彼の飛びは最高だっ
た.」のように「動詞語幹+i」を名詞として扱うものである.これを動名詞と
呼ぶ.今一つは,連用形を終止形とみなす用法で,何らかの述語を省略して連
用形で文を終わらせてしまう用法である.この場合,本来終止形に接尾するも
のが連用形に接尾することになる.本研究で使用した形態素解析プログラム
JUMANでは隣接規則が自由に記述できるようになっているので,これらの例外
的隣接規則も,基本的な隣接規則と同様に記述できた.



\section{派生文法による動詞の語尾変化} \label{verb}
本節では,派生文法に則して,動詞語幹と動詞接尾辞もしくは派生接尾辞との接
続規則を記述する.これを日本語の文字単位で処理するのに適する形に変更す
る方法については,次節で述べる.

派生文法では,動詞の語幹と接尾辞との接続規則を連続母音の縮退と連続子音の縮退で
説明する.連続子音の縮退は従来から指摘されていたものであるが,これに連
続母音の縮退という考え方を導入することにより,活用という考え方を用いず
とも体系的に現代日本語の動詞の語尾変化を説明できる.具体的には「kak」
に「ru」が接尾すると「kak/ru」となるが,「k/r」の部分が連続子音となり,
後ろの「r」が縮退し,「kaku」となる.これが連続子音の縮退である.一方
「tabe」に「imasu」が接尾すると「tabe/imasu」となるが,「e/i」の部分が
連続母音となって後ろの「i」が縮退し,「tabemasu」になる.これが連続母
音の縮退である.その他の組み合わせ,「kak」と「imasu」,「tabe」と「ru」
の場合はそれぞれ「kak/imasu」,「tabe/ru」となり,子音も母音も連続しな
いので,縮退せず「kakimasu」,「taberu」になる.以上の接続規則を
基本接続規則と呼ぶことにする.

この基本接続規則に加えて,表\ref{onbin}に示すような内的連声がある.
表中の具体例は完了の接尾辞「ita」との組み合わせで示す.
\begin{table*}
\begin{center}
\begin{tabular}{|l|l|} \hline
音便 & 具体例 \\ \hline
k/it $\rightarrow$ it	&	書く kak/ita $\rightarrow$ kaita 書いた \\
g/it $\rightarrow$ id	&	嗅ぐ kag/ita $\rightarrow$ kaida 嗅いだ \\
t/it $\rightarrow$ tt	&	待つ mat/ita $\rightarrow$ matta 待った \\
n/it $\rightarrow$ n'd	&	死ぬ sin/ita $\rightarrow$ sin'da 死んだ \\
b/it $\rightarrow$ n'd	&	飛ぶ tob/ita $\rightarrow$ ton'da 飛んだ \\
m/it $\rightarrow$ n'd	&	噛む kam/ita $\rightarrow$ kan'da 噛んだ \\
r/it $\rightarrow$ tt	&	掘る hor/ita $\rightarrow$ hotta 掘った \\
w/it $\rightarrow$ tt	&	買う kaw/ita $\rightarrow$ katta 買った \\
(k/it $\rightarrow$ tt) &	(例外) 行く ik/ita $\rightarrow$ itta 行った \\
\hline
\end{tabular}
\end{center}
\centering{(具体例は完了の接尾辞「ita」との組み合わせで示している)}
\caption{内的連声}
\label{onbin}
\end{table*}
例えば「書く」であれば「kak」に「ita」が接尾するとまず「kak/ita」とな
り,この「k/it」の部分が内的連声により「it」となるから,最終的には「kaita」
となる.この内的連声の唯一の例外が「行く」で,「ik/ita」が「iita」とな
らずに「itta」となる.注意すべきはこの内的連声は動詞接尾辞が接尾する場合
にのみ適用されることで,例えば願望を表す動形派生接尾辞の「iた(い)」では連声し
ない.

派生文法における動詞の語形変化の扱いの例外が「する」と「くる」の二つの
動詞と,これらを使って作られる動詞,さらに「感じる」など「する」が濁音
化して変化したと思われる一群の動詞である.「する」「じる」「くる」の語
幹変化を表\ref{henka}に示す.これらには同一の接尾辞に対して複数の語幹
変化があるものがある.

\begin{table}
\begin{minipage}[t]{5cm}
\begin{tabular}{|l|l|} \hline
語幹 & 接尾辞の始まりの部分 \\ \hline
s	&	i-, u-, e-, sa-, ra- \\
si	&	anai, yo- \\ 
se	&	a- \\
sur	&	u-, e-, re-, ru- \\ \hline
\multicolumn{2}{c}{「する」} \\
\end{tabular}
\end{minipage}
\hspace{-3mm}
\begin{minipage}[t]{5cm}
\begin{tabular}{|l|l|} \hline
語幹 & 接尾辞の始まりの部分 \\ \hline
zi	&	a-, i-, u-, e-, yo-, \\
&	sa-, ra-, ru-, re- \\
zur	&	e-, ru-, re- \\ \hline
\multicolumn{2}{c}{「じる」} \\
\end{tabular}
\end{minipage}
\hspace{-3mm}
\begin{minipage}[t]{5cm}
\begin{tabular}{|l|l|} \hline
語幹 & 接尾辞の始まりの部分 \\ \hline
k	&	i-, u-, e- \\
ko	&	a-, yo-, sa-, ra-, ru- \\
kur	&	re- \\ \hline
\multicolumn{2}{c}{「くる」} \\
\end{tabular}
\end{minipage}
\caption{「する」「じる」「くる」の語幹変化}
\label{henka}
\end{table}


また,「おっしゃる」「いらっしゃる」「なさる」「下さる」の四つの動詞は,
基本的には語幹が「r」で終わるものと同じであるが,幾つかの語幹の変化がある.
それを表\ref{nasaru}に示す.注意すべきは,内的連声が適用される場合は,内的連声が
語幹変化よりも優先されることである.

\begin{table}
\begin{center}
\begin{tabular}{|l|l|} \hline
語幹 & 接尾辞 \\ \hline
-r	&	a-, i-, u-, e-, yo-, sa-, ra-, ru-, re- \\
-	&	i, i- \\
\hline
\end{tabular}
\end{center}
\caption{「おっしゃる」「いらっしゃる」「なさる」「下さる」の語幹変化}
\label{nasaru}
\end{table}

最後に命令の動詞接尾辞に例外がある.動詞接尾辞「ro」「yo」は母音で終わる動
詞語幹にのみ接尾し,「e」は子音で終わる動詞語幹にのみ接尾する.また,
語幹が変化する不規則な動詞に関しては命令の形も不規則なものとなる.

動詞の形成における語形の変化は以上の規則で全て説明できるが,このままで
は日本語の文字単位で解析するのには向かない.そこでこれらを基礎にして,
日本語文字単位の解析向きに変更する方法について次の節で述べる.

\section{形態素解析システムへの適用} \label{apply}
前節で説明したような形態素を設定すれば,現代日本語の文を構成する形態素
を説明できるが,このままでは形態素解析には向かない.なぜならば日本語の
文は漢字や平仮名などで記述されているので,特に動詞に関しては一旦文字を
子音と母音に分解しなければ解析できないからである.そこで子音や母音に分
解せず日本語の文字の単位で解析できるように工夫することを考えた.すると,
従来用いられていた活用という考え方に近づき,そのことによって,従来の活
用という考え方に沿って作られた形態素解析プログラムに,本稿で示した形
態素文法を処理させることができるようになった.

\subsection{動詞に関する修正}
まず,動詞語幹に接尾する接尾辞の始まりの部分が数種類しかない.さらに動
詞語幹の末尾の子音もいくつかの種類に限定されている.そこで,これらの組
み合わせを動詞の活用語尾とし,動詞の語幹から末尾の子音を除いた部分を新
たに動詞語幹とする.各接尾辞については先頭の部分を隣接型とし,その部分
を除いた残りの文字列を新たにその形態素の表記とする.そして,それらの新
たな接続規則を設定する.

\begin{table*}
\begin{center}
\begin{tabular}{|rl|ccccccccccc|}
\hline
& \hspace{-5mm}{\tiny 活用形}  & A  & I  & U  & E  & YO & T  & D  & RA & RU & RE & SA \\
{\tiny 活用型} &&&&&&&&&&&&\\
\hline
A && *  & *  & *  & *  & よ & *  & -- & ら & る & れ & さ \\
K && か & き & く & け & こ & い & -- & か & く & け & か \\
G && が & ぎ & ぐ & げ & ご & -- & い & が & ぐ & げ & が \\
S && さ & し & す & せ & そ & し & -- & さ & す & せ & さ \\
T && た & ち & つ & て & と & っ & -- & た & つ & て & た \\
N && な & に & ぬ & ね & の & -- & ん & な & ぬ & ね & な \\
B && ば & び & ぶ & べ & ぼ & -- & ん & ば & ぶ & べ & ば \\
M && ま & み & む & め & も & -- & ん & ま & む & め & ま \\
R && ら & り & る & れ & ろ & っ & -- & ら & る & れ & ら \\
W && わ & い & う & え & お & っ & -- & わ & う & え & わ \\
SX && し & し & する & すれ & しよ & し & --& さ & する & --& さ \\
ZX && じ & じ & じる & じれ & じよ & じ & --& じら & じる & じれ & じさ \\
   &&    &    & ずる & ずれ &      &    & --&      & ずる &      &      \\
KX && こ & き & くる & くれ & こよ & き & --& こら & くる & これ & こさ \\
KKX && 来 & 来 & 来る & 来れ & 来よ & 来 & --& 来ら & 来る & 来れ & 来さ \\
RX && ら & り & る & れ & ろ & っ & -- & ら & る & れ & ら \\
   &&  & い &  &  &  &  &   &  &  &  &  \\
IKU && か & き & く & け & こ & っ & -- & か & く & け & か \\
\hline
\end{tabular}
\end{center}
\centering{(``*''は語尾の表記文字がないことを表し,``--''はその活用形自体がないことを表す)}
\caption{活用型と活用形}
\label{katuyou1}
\end{table*}

具体的に,動詞語幹に接尾する接尾辞の始まりはA, I, U, E, YO,
T, D, RA, RU, RE, SA の11 種類である.ここで,TとDは内的連声を形成する
it-という接尾辞の始まりをi-と区別したもので,さらにTは清音の内的連声に
対応し,Dは濁音の内的連声に対応する.そこで個々の動詞語幹に対して,こ
の11の活用形を設定することになる.この活用形のパターンは動詞語幹の末尾
に応じて決まるが,末尾が母音で終わる場合をAと表記することにすると末尾
の種類はA, K, G, S, T, N, B, M, R, Wの10種類であるから,それに応じた10
種類の活用型があることになる.さらに例外の活用型をSX, ZX, KX, KKX, RX,
IKUで表すことにすると,活用型と活用形の組み合わせは表\ref{katuyou1}のよ
うになる.

個々の動詞に関しては,語幹の末尾の子音を除いた文字列を「語幹」とし,
「食べ」のように語幹が母音で終わる場合にはAを,「書k」のように語幹が子
音で終わるものは子音そのものを活用型とする.例えば,「食べ」の場合は
「食べ」が語幹で活用型がA,「書k」の場合は「書」が語幹で活用型がK,
「嗅g」の場合は「嗅」が語幹で活用型がG,「思w」の場合は語幹が「思」で
活用型がWとなる.

動詞語幹に接尾するのは接尾辞は動名派生接尾辞,動形派生接尾辞,動動派生
接尾辞,そして動詞接尾辞である.これらについて例を挙げると,動名派生接尾
辞の「iそう」は隣接型がI型で表記文字は「そう」となり,動形派生接尾辞の
「aな」は隣接型がA型で表記文字は「な」,動動派生接尾辞の「rare」は隣接
型がRA型,活用型がA型,表記文字が「れ」となる.接尾辞は例えば「iます」は
隣接型がI型で,表記文字が「ます」になる.連声するような接尾辞,例えば
「itarou」では,隣接型がT型で表記文字が「たろう」の形態素と,隣接型がD型
で表記文字が「だろう」の形態素の二つに分ける.

これらの隣接規則は「動詞語幹の活用形名と,接尾辞の隣接型名が
一致するものが隣接可能である」ということになる.例えば「書か」は動詞語
幹「書」の活用形Aの形態であるから,隣接型がAの動形派生接尾辞「な」と隣
接可能である.

\subsection{活用形に対する追加}
上記のような修正を語幹に対して行う場合,連体形,終止形,接続形の動詞接
尾辞「ru」,連用形の動詞接尾辞「i」,さらに可能の動動派生接尾辞「re」は,
動詞語幹の活用形として先頭の文字が吸収されてしまうと形態素としての表記
文字が残らないという問題が起こる.また,命令の動詞接尾辞の「ro」「e」に
は,動詞語幹の末尾が母音か子音かによって接続規則が異なるという問題があ
る.そこで動詞接尾辞の「ru」「i」「ro」「e」に関してはそれぞれ活用形とし
てしまう.そのため,表\ref{katuyou1}に表\ref{katuyou2}を加える.新たに
加わったものは動詞語幹と動詞接尾辞が合成されたものであるので,属性も合成
されたものになる.それを表\ref{gousei}に示す.

可能の動動派生接尾辞「re」は,さらに後ろに動詞接尾辞などが来るため,活用
形として加えられない.そこで,全ての活用型に対して語幹自体を活用
形Xとして設定し,個々の子音との組み合わせによる「re」の変化を別々の形態素
とした.そして,これらについて活用形Xに対する隣接規則をそれぞれ作るこ
とで解決した.

このように,修正された接尾辞の扱いでは,一文字で構成される動詞接尾辞や派生
接尾辞を新たに加えようとすると新たな活用形を作り出さなければならないが,
一文字で構成されるという制約があるため,これ以上追加する必要が生ずる可
能性は低い.


\begin{table*}
\begin{center}
\begin{tabular}{|rl|cccccc|}
\hline
 & \hspace{-5mm}{\tiny 活用形} & X & 連用形  & 連体形  & 終止形  & 接続形  & 命令形 \\
{\tiny 活用型} &&&&&&&\\
\hline
A && * & る & る & る & る & ろ,よ \\
K && * & き & く & く & く & け \\
G && * & ぎ & ぐ & ぐ & ぐ & げ \\
S && * & し & す & す & す & せ \\
T && * & ち & つ & つ & つ & て \\
N && * & に & ぬ & ぬ & ぬ & ね \\
B && * & び & ぶ & ぶ & ぶ & べ \\
M && * & み & む & む & む & め \\
R && * & り & る & る & る & れ \\
W && * & い & う & う & う & え \\
SX && --& し & する & する & する & しろ,せよ \\
ZX && じ & じ & じる & じる & じる & じろ \\
   &&    &    & ずる & ずる & ずる & ぜよ \\
KX && こ & き & くる & くる & くる & こい \\
KKX && 来 & 来 & 来る & 来る & 来る & 来い \\
RX && * & り & る & る & る & れ,い \\
IKU && * & き & く & く & く & け \\
\hline
\end{tabular}
\end{center}
\caption{活用形の追加}
\label{katuyou2}
\end{table*}

\begin{table}
\begin{center}
\begin{tabular}{|l|l|} \hline
活用形 & 属性 \\ \hline
連用形 &[動,連用,無,無] \\
連体形 & [動,連体,無,無]\\
終止形 & [動,終止,無,無]\\
接続形 & [動,無,無,接続]\\
命令形 & [動,終止,無,無] \\
\hline
\end{tabular}
\end{center}
\caption{合成された属性}
\label{gousei}
\end{table}


\subsection{形容詞に関する修正}

前述したように形容詞の語幹は,形動派生接尾辞「ござr」が接尾する場合に
は連声する.しかし,これは非常に限られた現象で,しかもこれを本研究で使
用した形態素解析プログラムの形態素文法に反映させると非常に煩雑になるの
で,実際にはこれを正確に実装はせずに「うござr」「ゅうござr」という形動
派生接尾辞を辞書に登録した.こうすると「高ゅうござる」などを過剰に受理
してしまい,また「たこうござる」のような平仮名表記の場合には解析ができ
ない.過剰な受理に関しては,解析に対してなんらかの悪影響を及ぼさない限
り許容する.実際,今までのところ,解析に関してはこのための悪影響は確認
されていない.また,平仮名表記の場合は解析が不可能であるが,そのような
事例は皆無に近いと考え,対処しないことにした.


\section{問題点の検討} \label{detail}
この節では以下に関する問題点について検討する.
\begin{itemize}
\item 複数の品詞に属する形態素
\item 動名詞
\item 連用詞に係る連用詞
\item 複合名詞
\item 複数解に対する優先度付け
\end{itemize}
複数の品詞に属する形態素の内,幾つかは形態素レベルの情報では識別できな
い.そのような形態素が現れる文は本来複数の解釈が存在し,これを完全に一
つの解釈に決めるためには文脈を参照する必要がある.本研究での形態素解析
システムでは,このような識別は形態素解析システムの範囲を超えるものと見
なしている.以下でそのような形態素のついて述べるが,他の形態素解析シス
テムとの性能の比較を容易にするために,新聞記事1万文中\footnote{形態素
数約20万,文節数約8万5千}の出現頻度についても述べ,正しい品詞を選ぶ確
率が高くなるような規則を付す.

\subsection{「名詞+と」}
\begin{table}
\begin{center}
\begin{tabular}{|l|r|r|r|r|} \hline
               & 格接尾辞連体形 & 格接尾辞連用形 & 引用接尾辞連用形 & 合計 \\
\hline
名詞+と       & 465(37.8\%)    & 211(17.2\%)    & 553(45.0\%)      & 1229(100\%) \\
名詞+と+名詞 & 436(35.4\%)    & 102(8.3\%)     & 8(0.7\%)         & 546(44.4\%) \\
名詞+と+動詞 & 0(0\%)         & 93(7.6\%)      & 519(42.2\%)      & 612(49.8\%) \\
名詞+と+引用性動詞 & 0(0\%)   & 0(0\%)         & 511(41.6\%)      & 511(41.6\%) \\
名詞+と+読点 & 20(1.6\%)      & 4(0.3\%)       & 4(0.4\%)         & 28(2.2\%) \\
\hline
\end{tabular}
\end{center}
\caption{「名詞+と」の用法の分布}
\label{to}
\end{table}

「名詞+と」には,格接尾辞連体形(名詞を並列に並べる用法)と,格接尾辞連
用形(共同作業者を示す用法)と,引用接尾辞連用形の三つがある.この内,
最初の用法は連体であり,その他の用法は連用である.また,引用接尾辞の場
合は述語を形成する.形態素レベルではこれらの用法を識別できない.
表\ref{to}に「名詞+と」の用法の分布を示す.なお,引用性動詞とは「なる」
「する」「いう」「みる」「みなす」「思う」などあらかじめ選ばれた動詞で
ある.この表によると,次の規則により,1060/1229(86.2\%)の場合で正しい
品詞を得られる.
\begin{itemize}
\item 「名詞+と+名詞」の場合は格接尾辞連体形
\item 「名詞+と+引用性動詞」の場合は引用接尾辞連用形
\item 「名詞+と+引用性動詞以外の動詞」の場合は格接尾辞連用形
\item 「名詞+と+読点」の場合は格接尾辞連体形
\end{itemize}

\subsection{「名詞+との」}
「名詞+との」にも,格接尾辞連体形(共同作業者を示す用法)と引用接尾辞連
体形があり,前者は述語を形成しないが,後者は述語を形成する.この場合も,
名詞に「との」が接尾しているものは,識別できない.「名詞+との」は評価
に用いた文中では130箇所に現れ,その内123箇所(94.6\%)が格接尾辞連体形で
あった.引用接尾辞連体形の場合は7箇所で,その係先は「認識」「見方」
「情報」「主張」「理由」「考え」であり,逆にこれらの名詞に係る場合で格
接尾辞であるものはなかった.

\subsection{「名詞+とも」}
\begin{table}
\begin{center}
\begin{tabular}{|l|r|r|r|r|} \hline
                       & 名名派生接尾辞 & 引用接尾辞 & 格接尾辞 & 合計 \\
\hline
名詞+とも             & 46(63.9\%) & 18(25.0\%) & 8(11.1\%)& 72(100\%) \\
名詞+とも+引用性動詞 & 0(0.0\%)   & 18(25.0\%) & 0(0.0\%) & 18(25.0\%) \\
文頭+名詞+とも       & 18(25.0\%) &  1(1.4\%)  & 0(0.0\%) & 19(26.4\%) \\
読点+名詞+とも       & 18(25.0\%) &  0(0.0\%)  & 3(4.2\%) & 21(29.2\%) \\
連用形+名詞+とも     & 7(9.7\%)   &  5(6.9\%)  & 0(0.0\%) & 12(16.7\%) \\
連体形+名詞+とも     & 3(4.2\%)   & 12(16.7\%)  & 5(6.9\%) & 20(27.8\%) \\
\hline
\end{tabular}
\end{center}
\caption{「名詞+とも」の用法の分布}
\label{tomo}
\end{table}

「名詞+とも」は名名派生接尾辞,引用接尾辞+連用接尾辞,格接尾辞+連用
接尾辞の三つの可能性がある.評価文中では名名派生接尾辞が46箇所,引用接
尾辞+連用接尾辞が18箇所,格接尾辞+連用接尾辞が8箇所であった.
表\ref{tomo}に「名詞+とも」の用法の分布を示す.以下の規則により,
66/72(91.7\%)の場合で正しい品詞が得られる.
\begin{itemize}
\item 「名詞+とも+引用性動詞」の場合は,引用接尾辞.
\item 「連体形+名詞+とも+引用性動詞以外」の場合は,格接尾辞.
\item 上記以外は,名名派生接尾辞.
\end{itemize}



\subsection{「名詞+で」}
\begin{table}
\begin{center}
\begin{tabular}{|l|r|r|r|} \hline
& 格接尾辞連用形 & 名詞接尾辞連用形 & 合計 \\ \hline
名詞+で                         & 2124(64.5\%) & 1171(35.3\%) & 3295(100\%) \\
名詞+で+ある/ない              & 0(0.0\%) & 461(14.0\%) & 461(14.0\%) \\
名詞+で+は/も+ある/ない       & 0(0.0\%) & 134(4.1\%) & 134(4.1\%) \\
名詞+で+は/も+(ある/ない)以外 & 476(14.4\%) & 10(0.3\%) & 486(14.7\%) \\
名詞+で+(ある/ない/読点)以外   & 1373(41.7\%) & 347(10.5\%) & 1720(52.2\%) \\
連用形+名詞+で+読点           & 256(7.8\%) & 129(3.9\%) & 385(11.7\%) \\
連用形以外+名詞+で+読点       & 19(0.6\%) & 90(2.7\%) & 109(3.3\%) \\ \hline
\end{tabular}
\end{center}
\caption{「名詞+で」の用法の分布}
\label{de}
\end{table}

「で」には,格接尾辞連用形(場所や道具を示す用法)と名詞接尾辞連用形があ
る.前者は述語を形成せず,後者は述語を形成する.これらは形態素レベルの
情報では識別できない.「名詞+で」の用法の分布を表\ref{de}に示す.これ
によると以下の規則で2790/3295(84.7\%)の場合で正しい品詞が得られる.
\begin{itemize}
\item 「名詞+で+ある/ない」の場合は,名詞接尾辞連用形
\item 「名詞+で+は/も+ある/ない」の場合は,名詞接尾辞連用形
\item 「名詞+で+は/も+(ある/ない)以外」の場合は,格接尾辞連用形
\item 「名詞+で+(ある/ない/読点)以外」の場合は,格接尾辞連用形
\item 「連用形+名詞+で+読点」の場合は,名詞接尾辞連用形
\item 「連用形以外+名詞+で+読点」の場合は,格接尾辞連用形
\end{itemize}

\subsection{「名詞+か」}
「か」には,格接尾辞連体形,名詞接尾辞連用形,接名派生接尾辞,接続接尾
辞終止形がある.この内,「名詞+か」では格接尾辞連体形と名詞接尾辞連用
形が形態素レベルの情報では識別できない.ただし,本研究では例えば「太郎
か次郎か分からない.」という文の場合,両方の「か」は名詞接尾辞連用形で
あると考えている.「名詞+か」は評価の文中の84箇所に現れ,その内,10箇
所(11.9\%)が格接尾辞連体形,74箇所(88.1\%)が名詞接尾辞連用形であった.
また,評価文中では以下の規則で全ての場合を正しく識別できた.
\begin{itemize}
\item 「名詞+か+名詞」の場合は,格接尾辞連体形
\item 「名詞+か+名詞以外」の場合は,名詞接尾辞連用形
\end{itemize}

\subsection{「述語+ので」}
「述語+ので」については,「の(補助名詞)+で(格接尾辞)」,「ので(接続接尾辞連
用形)」の二通りの解釈がある.例えば,「大きいので壊した.」という文で
は,「大きい物で壊した」のか「大きいから壊した」のかの区別ができない.
しかし,前者の解釈は口語的なので,評価に用いた新聞記事では1万文の中
に現れた50箇所全てが後者の用法であった.

\subsection{「述語+のに」}
\begin{table}
\begin{center}
\begin{tabular}{|l|r|r|r|} \hline
           & 補助名詞+格接尾辞 & 接続接尾辞連用形 & 合計 \\ \hline
のに       & 25(53.2\%) & 22(46.8\%) & 47(100\%) \\
のに+読点 & 1(2.1\%)   & 14(29.8\%) & 15(31.9\%)\\
のに+名詞 & 15(31.9\%) & 5(10.6\%)  & 20(42.6\%)\\
のに+述語 & 9(21.3\%)  & 1(2.1\%)   & 10(21.3\%)\\ \hline
\end{tabular}
\end{center}
\caption{「述語+のに」の用法の分布}
\label{noni}
\end{table}
「述語+のに」については,「の(補助名詞)+に(格接尾辞)」,「のに(接続接尾辞連
用形)」の二通りの解釈がある.例えば,「高いのに乗った.」では,「高い
にも関わらず乗った」のか「高いものに乗った」のか識別できない.「述語+
のに」の用法の分布を表\ref{noni}に示す.これによると,以下の規則で
38/47(80.9\%)の場合に正しい品詞を得られる.
\begin{itemize}
\item 「のに+読点」の場合は,接続接尾辞連用形
\item 「のに+名詞」の場合は,補助名詞+格接尾辞
\item 「のに+述語」の場合は,接続接尾辞連用形
\end{itemize}

\subsection{「そう」}
「そう」には,動名派生接尾辞,形名派生接尾辞,名名派生接尾辞,接名派生
接尾辞があり,ほとんどの場合はこれらは形態素レベルの情報で識別できる.
しかし,「動詞語幹(活用型A)+そう」の場合には識別できない二つの解釈があ
る.例えば「食べたそうだ.」という文では「食べ(動詞語幹)た(動形派
生接尾辞[欲求])そう(形名派生接尾辞[様態])だ(名詞接尾辞終止形)」と「食
べ(動詞語幹)た(動詞接尾辞接続形[完了])そう(接名派生接尾辞[伝聞])だ(名詞
接尾辞終止形)」を形態素レベルで識別できない.評価に用いた文中ではこの
ような「そう」は11箇所に現れ,その全てが後者の用法であった.これは評価
に用いた文が新聞記事であるためと考えられる.

\subsection{「動詞語幹+i+に」}
「動詞語幹+i+に」には二つの解釈がある.例えば「話しに花を
添える.」「話しに行く.」では「話し」は前者では「動名詞+格接尾辞``に''」であ
り,後者では「動詞語幹+動詞接尾辞``ini''」である.これらは形態素レベル
の情報では区別できない.これは評価文中では168箇所で現れた.その内150箇
所が動名詞であり,18箇所が動詞であった.動詞の18箇所の内,「動詞+に+行
く」が5箇所であり,「動詞+に+来る」が8箇所,その他,「入る」「通う」
「向かう」「寄る」が直後に来るものがそれぞれ1箇所ずつあった.逆にこれ
らの動詞が直後に来る場合で動名詞であったものはなかった.従って,以下
の規則で167/168(99.4\%)が正しく識別できる.
\begin{itemize}
\item 「来る」「行く」などの特定の動詞が直後に来る場合は「動詞語幹+動
詞接尾辞``ini''」.
\item 上記以外の場合は「動名詞+格接尾辞``に''」.
\end{itemize}


\subsection{「動詞語幹+i」}
「動詞語幹+i」には,動詞の連用形である場合と,動名詞である場合がある.
「動詞語幹+i+名詞接尾辞」「動詞語幹+i+格接尾辞」「動詞語幹+i+連
用接尾辞」の場合は動名詞であると識別することができる.また「動詞語幹+
i+読点」は動詞の連用形と識別できる.

文節に区切る際に最も問題になるのは「動詞語幹+i+名詞」の場合である.
「動詞語幹+i」を動詞の連用形と見なす場合にはそこで文節が区切れるが,
「動詞語幹+i」を動名詞と見なす場合には複合名詞になるので文節が区切れ
ない.このような「動詞語幹+i+名詞」のパターンは評価文中の129箇所に現
れ,動名詞であったのが79箇所であり,動詞であったのが50箇所であった.こ
れらは形態素レベルの情報では区別できない.しかし,評価文中では
表\ref{doumeisi}のような用法の分布があった.従って,下記の規則で
127/129(98.4\%)の場合で正しく識別できる.
\begin{itemize}
\item 「連用接尾辞``は''+動詞語幹+i+名詞」の場合は,動名詞.
\item 「連用接尾辞``は''以外の連用形+動詞語幹+i+名詞」の場合は,動詞.
\item 上記以外は動名詞.
\end{itemize}

\begin{table}
\begin{center}
\begin{tabular}{|l|r|r|r|} \hline
                          & 動名詞     & 動詞       & 合計 \\ \hline
動詞語幹+i+名詞         & 79(60.5\%) & 50(39.5\%) & 129(100\%) \\
読点+動詞語幹+i+名詞   & 13(10.0\%) &  0(0.0\%)  & 13(10.0\%) \\
名詞+動詞語幹+i+名詞   & 25(19.4\%) &  0(0.0\%)  & 25(19.4\%) \\
文頭+動詞語幹+i+名詞   &  6(4.7\%)  &  0(0.0\%)  &  6(4.7\%)  \\
連体形+動詞語幹+i+名詞 & 26(20.2\%) &  0(0.0\%)  & 26(20.2\%) \\
連用形+動詞語幹+i+名詞 &  9(7.0\%)  & 50(39.5\%) & 59(45.7\%) \\
は+動詞語幹+i+名詞     &  7(5.4\%)  &  0(0.0\%)  &  7(5.4\%)  \\
\hline
\end{tabular}
\end{center}
\caption{「動詞語幹+i+名詞」の用法の分布}
\label{doumeisi}
\end{table}


\subsection{「いく」と「いう」}
「いった」「いって」などは「言った」「言って」なのか「行った」「行って」
なのか分からない.評価の文中には77箇所でこのような表現が現れたが,動詞
の連用形の直後に来るものは全て「行く」であり,それ以外はすべて「言う」
であった.これは,評価に用いた文が校正済みの新聞記事であるためと考えら
れる.

\subsection{「ある」}
「ある」には,連体詞と動詞の可能性がある.これは評価文中に260箇所に現
れ,24箇所が連体詞,236箇所が動詞であった.この内,「読点+ある」は8箇
所で,全て連体詞であった.その他の連体詞の「ある」は16箇所全てが「名詞
+の+ある」の形で現れたが,動詞の「ある」が「名詞+の+ある」の形で現
れたのが29箇所であった.従って,以下の規則で,244/260(93.8\%)が正しく
認識される.
\begin{itemize}
\item 「連用形+ある」の場合は,動詞.
\item 「名詞+の+ある」の場合は,動詞.
\item 上記以外の「連体形+ある」の場合は,連体詞.
\item 「読点+ある」の場合は,連体詞.
\end{itemize}

\subsection{連用詞に係る連用詞}
連用詞の中には他の連用詞を修飾していると考えられるものがある.例えば
「非常にゆっくり歩いた.」という文で,「非常に」は「ゆっくり」の様態を
表していると考えられる.一つの解決法は連用詞は他の連用詞に係ることがで
きるとしてしまうことであるが,全ての連用詞が他の連用詞に係るわけではな
いので,連用詞に係ることができる連用詞として別の品詞を設定する必要が出
てくる.別の解決法は,先ほどの例で言えば,「非常に」が「ゆっくり」では
なく「歩いた」に係ると見なすことにしてしまう方法である.その場合は,
「非常に」と「ゆっくり」の関係を「歩いた」を仲介して算出する仕組みを別
に用意しなければならない.しかし,この利点は,「非常に私はゆっくり歩い
た.」という文でも係り受けの非交差の原則が守られていると見なせる点であ
る\footnote{同様な現象は「は」にも見られ,例えば「この料理は私は彼女が
作ったと思う.」という文で「料理は」が「作った」に係るとすると非交差の
原則が破られるが,これも「料理は」は「思う」に係ると見なして,別の仕組
みによって,「料理」と「作った」の関係を算出すると考えれば,非交差の原
則が守られていると見なせる.}.また連用詞の変種を作る必要もないので,
本研究では後者の解決法を取っている.

\subsection{複合名詞}
本研究における形態素解析システムは,その目的から,複合名詞をさらに細
かく区切ることを重要視していない.つまり,文節の区切りの精度を測定する
場合に「名詞+名詞\footnote{「連用名詞+名詞」は複合名詞にならな
い.}」の並びを複合した結果が名詞として正解であればよしとしている.
従って,複合した状態では正解であっても,それをさらに細かく分解した状態
では間違っている場合がある.評価文中では11845箇所に複合名詞の分割が現
れた\footnote{この中には辞書に一つの名詞として登録されている複合名詞は
含まれない.そのような複合名詞は1302箇所に現れた.}.この内,492箇
所(4.2\%)が誤って分割されていた.誤りの内,221箇所(1.9\%) が固有名詞
に起因するものであった.

\subsection{複数解に対する優先度付け}
本稿では述べていないが,実際の形態素解析処理における重要な要素に複数
解に対する優先度付けの問題がある.例えば,「太郎が帰ってきたとき,犬が
吠えた.」という文には,本稿で示した形態素文法だけでは,「とき」の部分に曖昧性が
生じる.一つの解釈は明らかなように「とき(時)」という名詞である.今一つ
の解釈は,「と(引用)」「き(``来る''の連用形)」である.ここでは明らかに
前者の解釈を取らなければならない.その他にも,単語の平仮名表記を含めれ
ば多くの曖昧性がある.そこで,実際の形態素文法の定義では,品詞や品詞の
隣接規則に重み付けをし,優先度の計算を行っている.しかし,この重み付け
はまったくアドホックなものであり,実際,多くの例文を処理させた結果を分
析して,優先度の計算がうまく人間の解釈と適合するように調整する事によっ
て作成した.実際のシステムへの適用にあたってはこの部分が最も時間がかかっ
た部分であり,さらなる精度向上に対する障害の一つである.

\section{性能評価}\label{eval2}
本稿で提案した形態素文法を形態素解析プログラム
JUMAN\cite{juman}\footnote{JUMANは品詞や形態素文法を再定義できる公開さ
れた形態素解析システムである.}に適用し,形態素解析の精度を測定した.
本来のJUMANは接続コストによって枝狩りした解に対して後方最長一致の解を
出力するもの\footnote{オプションによってただ一つの解を出力するように指
定した場合}であるが,本研究ではこれを接続コストが最小になる
解\cite{hisamitu90}を出力するように改造して使用した.利用した辞書は異
なり語数35万程度であり,これらの内,動詞語幹,形容詞語幹,名詞,連用
詞,連体詞,名名派生接尾辞,数名派生接尾辞については日本電子化辞書研究
所の日本語辞書の他,いくつかの仮名漢字変換プログラム用辞書や機械可読な
人間用の辞書から抽出したものを用いた.また,漢字表記の語については,そ
の平仮名表記も辞書に登録し,全体で50万語程度となっている.ただし,単
語の中で漢字の一部だけを平仮名に変えたものは辞書に登録していない.

評価には日本電子化辞書研究所から提供されたコーパス\footnote{このコーパ
スには主に朝日新聞社の記事から収集した文が集められている.}の内,1万文
を使用した.これらの文に対して形態素解析システムに{\dg ただ一つ} の解
を出力させ,これとコーパスに付けられている人手による解析結果とを比較し
た.ただし,日本電子化辞書研究所における品詞の分類と本論文での品詞の分
類が異なっているため,文節単位にまで形態素をまとめたものを比較した.結
果を表\ref{error}に示す.但し,「区切り誤り」は\ref{detail}節で与えた
規則によって品詞が間違う場合の数である.

\begin{table}
\begin{center}
\begin{minipage}[t]{6cm}
\begin{tabular}{|l|r|} \hline
文数 & 10000 \\
文節数 & 84841 \\
形態素数 & 207547 \\ \hline
文節区切り位置の誤り数 & 445 \\ \hline
分割誤り複合名詞数 & 221 \\ \hline
\multicolumn{2}{c}{区切り誤り}
\end{tabular}
\end{minipage}
\begin{minipage}[t]{5cm}
\begin{tabular}{|l|r|r|} \hline
& 頻度 & 誤り数 \\ \hline
名詞+と   & 1229 & 169 \\
名詞+との &  130 & 0 \\
名詞+とも &   72 & 6 \\
名詞+で   & 3295 & 505 \\
名詞+か   &   84 & 0 \\
述語+ので &   50 & 0 \\
述語+のに &   47 & 9 \\
動詞語幹(A)+そう & 11 & 0 \\
動詞語幹+i+に & 168 & 1 \\
動詞語幹+i+名詞 & 129 & 2 \\
ある & 260 & 16 \\ \hline
\multicolumn{3}{c}{品詞付け誤り}
\end{tabular}
\end{minipage}
\end{center}
\caption{誤り数}
\label{error}
\end{table}

文節の区切り位置を誤っていたのは445箇所であった.区切り位置を誤ると,
その前後の文節が共に誤りとなるので,誤りの文節が含まれる率は,全形態素
数に対して,
\[445 \times 2 \div 207547 \times 100 = 0.43\%\]
である.これは全文節数に対しては1.05\%である.また,1文中に複数の区切
り誤りがあったものはなかったので,文節区切りが失敗した文は全文数に対し
て4.45\%である.

文献\cite{maruyama94}では,分割誤りとして複合名詞の分割誤りを含めて,
形態素数に対して分割誤り率は1.25\%と報告されている.本稿のシステムを
同様に評価すると,
\[(445 + 221) \times 2 \div 207547 \times 100 = 0.64\%\]
の分割誤り率である.さらに文献\cite{maruyama94}では品詞誤りを含めた全
体的な誤り率を2.36\%と報告している.文献\cite{maruyama94}では本稿で
与えた形態素文法よりも細かい品詞分類を行っているので,同様に比較できな
いが,表\ref{error}に挙げたものを形態素に対する品詞付けの誤りとすると,
\[((445 + 221) \times 2+169+6+505+9+1+2+16) \div 207547 \times 100 = 0.98\%\]
である.

文節の区切りに関する誤りは,8箇所が複数解の優先度付けの誤りによるもの
であり,残りの437箇所は対応する形態素を辞書に登録することによって解決
できるものであった.従って,辞書を整備することで文節区切りの性能はさら
に向上させることができると期待できる.辞書の整備に関して,語の中の一部
の漢字が平仮名表記されるものについては,自動的に漢字の一部を平仮名に置
き換えたものを登録することが可能である.しかし,その場合,登録語数がほ
ぼ4倍になる.実際にはこれらの内ほとんどのものは用いられない上,解析速
度にも悪影響を与えるので,コーパスの分析結果などから必要な表記のみを登
録するのが望ましい.

複合名詞の区切り誤りについては本形態素文法では対処できない.しかし,複
合名詞としてまとまった形で認識する精度は高い.複数の品詞に属する形態素
に関しては,新聞記事に対しては有効性の高い識別規則を与えたが,これらは
あくまで確率的なものであり,根本的な解決にはならない.

動詞の語尾変化は全て正しく解析され,派生文法における動詞の取り扱い方
法の優秀さが実証された.口語的な表現に対しても,JUNETの生活関連のニュ
ースグループの記事の内,明らかな間違いを除いた500文を解析させたところ,
動詞語の語尾変化に対しては全て正しく解析されていた.


\section{まとめ}
本稿では形態素解析に的を絞った日本語形態素文法を提案した.この形態素
文法における動詞語尾の扱いは,派生文法を拡充整備し,日本語の文字単位で
扱えるように修正したものである.その結果,実存する形態素解析プログラム
JUMANに適用できるようになり,実際に適用して実用的な解析性能を得ること
ができた.辞書を整備することでさらなる精度の向上が期待できる.しかし,
形態素の隣接規則間の優先度を決める重みの決定は,手作業による微妙な調整
によるものであり,何らかの自動的な学習の仕組みが必要である.



\acknowledgment

形態素解析プログラムJUMANを提供して下さった奈良先端科学技術大学院大学
の松本裕治先生,および辞書を提供して下さった日本電子化辞書研究所の方々
に感謝します.


\begin{thebibliography}{}

\bibitem[\protect\BCAY{Bloch}{Bloch}{1946}]{bloch}
Bloch, B. \BBOP 1946\BBCP.
\newblock \BBOQ {S}tudies in {C}olloquial {J}apanese, {P}art {I},
  {I}nflection\BBCQ\
\newblock {\Bem Journal of the American Oriental Society}, {\Bbf 66}.

\bibitem[\protect\BCAY{Hisamitu \BBA\ Nitta}{Hisamitu \BBA\
  Nitta}{1994}]{hisamitu94b}
Hisamitu, T.\BBACOMMA\  \BBA\ Nitta, Y. \BBOP 1994\BBCP.
\newblock \BBOQ {A}n {E}fficient {T}reatment of {J}apanese {V}erb
  {I}nflection for {M}orphological {A}nalysis\BBCQ\
\newblock In {\Bem Coling 94}, \lowercase{\BVOL}~I.

\bibitem[\protect\BCAY{久光\JBA 新田}{久光\JBA 新田}{1990}]{hisamitu90}
久光徹, 新田義彦 \BBOP 1990\BBCP.
\newblock \JBOQ 接続コスト最小法による日本語形態素解析の提案と計算量の評価に
  ついて\JBCQ\
\newblock 言語理解とコミニュケーション\ 90-8, 電子情報通信学会.

\bibitem[\protect\BCAY{久光\JBA 新田}{久光\JBA 新田}{1994a}]{hisamitu94a}
久光徹, 新田義彦 \BBOP 1994a\BBCP.
\newblock \JBOQ ゆう度付き形態素解析用の汎用アルゴリズムとそれを利用したゆう
  度基準の比較.\JBCQ\
\newblock \Jem{電子情報通信学会論文誌 D-II}, {\Bbf J77}  (5).

\bibitem[\protect\BCAY{久光\JBA 新田}{久光\JBA 新田}{1994b}]{hisamitu94c}
久光徹, 新田義彦 \BBOP 1994b\BBCP.
\newblock \JBOQ 日本語形態素解析における効率的な動詞活用処理\JBCQ\
\newblock 自然言語処理研究会\ 103-1, 情報処理学会.

\bibitem[\protect\BCAY{木谷}{木谷}{1992}]{kitani}
木谷強 \BBOP 1992\BBCP.
\newblock \JBOQ 固有名詞の特定機能を有する形態素解析処理\JBCQ\
\newblock 自然言語処理研究会\ 90-10, 情報処理学会.

\bibitem[\protect\BCAY{清瀬}{清瀬}{1989}]{kiyose}
清瀬~義三郎則府 \BBOP 1989\BBCP.
\newblock \Jem{日本語文法新論 --派生文法序説--}.
\newblock 桜楓社.

\bibitem[\protect\BCAY{丸山\JBA 荻野}{丸山\JBA 荻野}{1994}]{maruyama94}
丸山宏, 荻野紫穂 \BBOP 1994\BBCP.
\newblock \JBOQ 正規文法に基づく日本語形態素解析\JBCQ\
\newblock \Jem{情報処理学会論文誌}, {\Bbf 35}  (7).

\bibitem[\protect\BCAY{益岡\JBA 田窪}{益岡\JBA 田窪}{1992}]{masuoka}
益岡隆志, 田窪行則 \BBOP 1992\BBCP.
\newblock \Jem{基礎日本語文法 --改訂版--}.
\newblock くろしお出版.

\bibitem[\protect\BCAY{松本, 黒橋, 宇津呂, 妙木, 長尾}{松本\Jetal}{1994}]{juman}
松本裕治, 黒橋禎夫, 宇津呂武仁, 妙木裕, 長尾真 \BBOP 1994\BBCP.
\newblock \JBOQ 日本語形態素解析システム{JUMAN} 使用説明書 version 2.0\JBCQ\
\newblock \JTR, 奈良先端科学技術大学院大学.

\bibitem[\protect\BCAY{三浦}{三浦}{1975}]{miura}
三浦つとむ \BBOP 1975\BBCP.
\newblock \Jem{日本語の文法}.
\newblock 勁草書房.

\bibitem[\protect\BCAY{宮崎\JBA 高橋}{宮崎\JBA 高橋}{1992}]{miyazaki}
宮崎正弘, 高橋大和 \BBOP 1992\BBCP.
\newblock \JBOQ 三浦文法に基づく日本語形態素処理用文法の構築\JBCQ\
\newblock 自然言語処理研究会\ 90-1, 情報処理学会.

\bibitem[\protect\BCAY{Nagata}{Nagata}{1994}]{nagata}
Nagata, M. \BBOP 1994\BBCP.
\newblock \BBOQ {A} {S}tochastic {J}apanese {M}orphological {A}nalyzer
  {U}sing a {F}orward-{D}{P} {B}ackward-{A}* {N}-{B}est {S}earch
  {A}lgorithm\BBCQ\
\newblock In {\Bem Coling 94}, \lowercase{\BVOL}~I.

\bibitem[\protect\BCAY{中村, 吉田, 今永}{中村\Jetal }{1991}]{nakamura}
中村順一, 吉田将, 今永一弘 \BBOP 1991\BBCP.
\newblock \JBOQ 接続コスト最小法による日本語形態素解析の評価実験\JBCQ\
\newblock 言語理解とコミニュケーション\ 91-1, 電子情報通信学会.

\bibitem[\protect\BCAY{西野, 鷲北, 石井}{西野\Jetal }{1992}]{nisino}
西野博二, 鷲北賢, 石井直子 \BBOP 1992\BBCP.
\newblock \JBOQ 派生文法による日本語構文解析\JBCQ\
\newblock 自然言語処理研究会\ 87-6, 情報処理学会.

\bibitem[\protect\BCAY{寺村}{寺村}{1984}]{teramura}
寺村秀夫 \BBOP 1984\BBCP.
\newblock \Jem{日本語のシンタクスと意味 II}.
\newblock くろしお出版.

\bibitem[\protect\BCAY{時枝}{時枝}{1950}]{tokieda}
時枝誠記 \BBOP 1950\BBCP.
\newblock \Jem{日本語文法 口語篇}.
\newblock 岩波書店.

\bibitem[\protect\BCAY{吉村, 日高, 吉田}{吉村\Jetal }{1983}]{yosimura83}
吉村賢治, 日高達, 吉田将 \BBOP 1983\BBCP.
\newblock \JBOQ 文節数最小法を用いたべた書き日本語の形態素解析\JBCQ\
\newblock \Jem{情報処理学会論文誌}, {\Bbf 24}  (1).

\end{thebibliography}

\begin{biography}
\biotitle{略歴}
\bioauthor{渕 武志}{
1965年生.
1988年東京大学理学部情報科学科卒業.
1991年慶応大学大学院修士課程終了.
1995年東京大学大学院博士課程修了.理学博士.
同年,NTTに入社,現在に至る.
自然言語処理,知識情報処理の研究に従事.}

\bioauthor{米澤明憲}{
1947年生.1977年 Ph.D. in Computer Science (MIT). 
1989年より東京大学理学部情報科学科教授.超並列ソフトウエアアーキテクチャ,
ソフトウエア基礎論,人工知能基礎論などに興味を持つ.著書「算法表現論」,
「モデルと表現」(岩波書店),編著書「ABCL: An Object-Oriented Concurrent 
System」,「 Research Directions in Concurrent Object-Oriented Computing」(MIT 
Press)等.現在IEEE Parallel and Distributed Technology編集委員.1992年より
ドイツ国立情報処理研究所(GMD)科学顧問.}


\bioreceived{受付}
\biorevised{再受付}
\biorerevised{再々受付}
\bioaccepted{採録}

\end{biography}

\end{document}
