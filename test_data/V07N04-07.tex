\documentstyle[epsbox,jnlpbbl,version]{jnlp_j}

\def\atari(#1,#2,#3){}

\setcounter{page}{143}
\setcounter{巻数}{7}
\setcounter{号数}{4}
\setcounter{年}{2000}
\setcounter{月}{10}
\受付{2000}{1}{24}
\再受付{2000}{3}{9}
\採録{2000}{6}{30}

\newcommand{\PP}{}
\newcommand{\Q}{}
\newcommand{\X}{}
\newcommand{\Y}{}
\newcommand{\head}{}
\newcommand{\OBJ}{}
\newcommand{\SUBJ}{}
\newcommand{\SYN}{}
\newcommand{\SEM}{}
\newcommand{\COR}{}

\newenvironment{RULE}[1]{}{}

\newenvironment{DEPLED}{}{}

\newcounter{transcounter}
\newenvironment{TRANS}{}{}

\newcounter{textcounter}
\newenvironment{TEXT}{}{}

\newcounter{transexcounter}
\newenvironment{EX}{}{}

\setcounter{secnumdepth}{2}

\title{日英機械翻訳システムTWINTRANの
\\言語知識と翻訳品質の評価}
\author{吉見 毅彦\affiref{SHARP1} \and
Jiri Jelinek\affiref{NONE} \and
西田 収\affiref{SHARP2} \and
田村 直之\affiref{KOBE} \and
村上 温夫\affiref{ExKOBE}}

\headauthor{吉見, Jelinek, 西田, 田村, 村上}
\headtitle{日英機械翻訳システムTWINTRANの言語知識と翻訳品質の評価}

\affilabel{SHARP1}{シャープ(株) 情報システム事業本部}
{Information Systems Group, SHARP Corporation}
\affilabel{NONE}{所属先なし}
{None}
\affilabel{SHARP2}{シャープ(株) 情報家電開発本部}
{Digital Information Appliances Development Group, SHARP Corporation}
\affilabel{KOBE}{神戸大学 工学部 情報知能工学科}
{Department of Computer and Systems Engineering, 
Faculty of Engineering, Kobe University}
\affilabel{ExKOBE}{神戸大学名誉教授}
{Emeritis Professor of Kobe University}

\jabstract{本稿では,日英機械翻訳システムTWINTRANの言語知識(辞書と規則)
と翻訳品質の評価結果について述べる.
TWINTRANは次のような設計方針に基づくシステムである.
1) 翻訳対象と翻訳方向を日本語テキストから英語テキストへの翻訳に固定し,
日本語の解析知識の記述を,日本語文法だけでなく英語文法も考慮に入れて行な
う.
2) 解釈の曖昧性の解消は,各規則に与えた優先度に基づいて解釈の候補に優
劣を付け,候補の中から最も優先度の高い解釈を選択することによって行なう.
3) 動詞の主語や目的語など日本語では任意的だが英語では義務的である情報を
得るために照応解析を行なう.\\
NTT機械翻訳機能試験文集を対象として行なったウィンドウテストでは,我々の
評価基準で,試験文集の73.1\%の文が合格となり,試験文集全体の平均点も合格
点を上回る結果が得られた.}

\jkeywords{機械翻訳,曖昧性解消,優先度,照応解析,日英対照分析}

\etitle{Linguistic Expertise in the Japanese-to-English\\
MT System TWINTRAN and \\
Evaluation of Translation Results}
\eauthor{Takehiko Yoshimi\affiref{SHARP1} \and
Jiri Jelinek\affiref{NONE} \and
Osamu Nishida\affiref{SHARP2} \and
Naoyuki Tamura\affiref{KOBE} \and
Haruo Murakami\affiref{ExKOBE}} 

\eabstract{In this treatise we set out to describe the linguistic expertise
(dictionary and rules) embodied in the Japanese-to-English MT system 
TWINTRAN, and the evaluation of the translation results. 
TWINTRAN is based on the following design policy:
1) The translation equivalents and the direction of the translation
process are strictly monodirectional, from Japanese to English.
The analysis of the Japanese input is not confined to Japanese grammar
but also anticipates at every step the possible English translation.
2) Disambiguation is based on prioritisation of each rule, where
each rule contains a priority value and the highest aggregate priority
candidate is selected.
3) Verb complements are screened for acceptability not only in the input
Japanese but also in the output English, and anaphora resolution
is used for arriving at the optimum result. \\
In the window test we have carried out, based on 
NTT's functional MT test set,
applying our evaluation procedure, 73.1\% of the corpus was acceptable
and the corpus average was above the point of acceptability.}

\ekeywords{Machine Translation, Disambiguation, Prioritisation, 
Correferentiality Analysis, Japanese-to-English Contrastive Analysis}

\begin{document}
\maketitle

\section{はじめに}
\label{sec:introduction}

これまでに開発されている機械翻訳システムの多くはトランスファ方式に基づい
ており,原言語の性質だけに依存する解析辞書・規則と,原言語と目的言語の両
方の性質に依存する対照辞書・規則が個別に記述されている.
他方,翻訳対象言語対と翻訳方向を固定した上で,解析知識の記述を,原言語の
性質だけでなく目的言語の性質も考慮に入れて行なうという設計方針もある.
このような方針を採ると,ある原言語(例えば日本語)の解析知識を,異なる目的
言語へのシステム(例えば日英システムと日中システム)で共用できるというトラ
ンスファ方式の利点が失われ,別の目的言語へのシステムを開発する場合には新
たな解析知識の記述が必要になる.
しかし,当面,ある特定の原言語から特定の目的言語への翻訳に焦点を絞れば,
以下のような利点が得られる.
\begin{enumerate}
\item 原言語の表現と目的言語の表現を比較的表層のレベルで対応付けることが
できる.
例えば動詞と補足語との結合関係は,翻訳対象
と翻訳方向を日本語テキストから英語テキストへの翻訳に固定した場合,
日本語の助詞と英語の前置詞との対応として記述できる.
従って,結合関係を深層格として抽象化する必要がなくなり,深層格の認定基準
の設定などの困難が避けられる.
\item 
翻訳対象言語対と翻訳方向を固定しない場合,ある目的言語の生成には必要ない
が別の目的言語の生成には必要な情報も解析過程で抽出しておく必要がある.
これに対して,翻訳対象と翻訳方向を固定すると,原言語を解析する知識を記述
する際に目的言語の性質を考慮することが可能になるため,目的言語の生成に必
要な情報のみを抽出すればよくなり,無駄な処理が避けられる.
\end{enumerate}

このようなことから,我々のシステムTWINTRANでは,目的を日英翻訳に限定した
上で,英語の適切なテキストを生成するためには日本語テキストがどのように解
析されていなければならないかという観点から辞書と全規則群を記述している.
辞書では,英語に翻訳する際にこれ以上分解するとその意味が変化してしまう表
現はそれ以上分解せずに一見出しとして登録する(\ref{sec:analysis:dict}\,節).
構文解析規則では,動詞と補足語との結合関係を日本語の助詞とその英訳と
の対応に基づいて区別し,動詞型を日本語の結合価パターンと英語の結合価パ
ターンとの対に基づいて設定する(\ref{sec:analysis:syn}\,節).
また,日本語の連体従属節を英語の関係節に翻訳するための関係詞決定規則を設
ける(\ref{sec:analysis:rel}\,節).
日本語で明示することは希であるが英語では明示しなければならない言語形式上
の必須情報(名詞句の定/不定性の区別や,動詞の主語や目的語になる代名詞など)
を得るために,照応解析規則を陳述縮約パラダイム\cite{Jelinek95}に基づいて
記述する(\ref{sec:analysis:integ:cor}\,節).

TWINTRANと同じように日本語の解析知識を日英対照の観点から記述しているシ
ステムとして,ALT-J/E\cite{Ikehara96,Nakaiwa97}や,
US式翻訳システム\cite{Shibata96a,Shibata98}などがこれまでに報告されている.
これらのシステムでも照応解析が行なわれているが,構造を持たない言語表現の
間で成立する照応すなわち語と語の間の照応の解析に留まっている.
これに対してTWINTRANでは,文や句など構造を持つ言語表現間の照応も扱う
\footnote{より精度の高い解析を実現するためには,構造を持つ表現の意味をそ
の部分から導き出す必要があるが,その実現は今後の課題である.}
.

機械翻訳システムにおける重要な課題の一つは,テキスト解釈の曖昧性を解消
し妥当な解釈を一意に決定することである.
曖昧性解消へのアプローチには,言語知識を絶対的な基準(制約)とみなす立場と,
相対的な比較基準(選好)とみなす立場がある\cite{Nagao92}.
前者では,ある解釈を受理するか棄却するかの判断は,その解釈と他の解釈を比
較せずに行なわれる.
これに対して後者では,ある解釈の選択は他の解釈との比較に基づいて行なわれる
\cite{Wilks78,Tsujii88b,Shimazu89,Hobbs90,Den96}.
TWINTRANでは後者の立場から,各規則に優先度を付与し,それに基づいて解釈
の候補に優劣を付け,候補の中から最も優先度の高い解釈を選択することによっ
て曖昧性の解消を行なう.
テキスト解析では,形態素解析規則から照応解析規則に至るまでいくつかの種類
の規則が利用され,それぞれ異なる観点から一つの解釈の良さが評価される.
このとき,ある種類の規則による解釈の良さと他の種類の規則による解釈の良さ
が競合する可能性があるため,各観点からの評価をどのように調整するかが重要
となる.
TWINTRANでは,構文,共起的意味,照応に関する各規則による優先度の重み付き
総和が最も高い解釈をテキストの最良解釈とする
(\ref{sec:analysis:integ:balance}\,節).

以下,\ref{sec:analysis:dict}\,節ないし\ref{sec:generation}\,節で各処理
過程について説明し,\ref{sec:experiment}\,節で翻訳品質評価実験の結果を示
す.

\section{辞書}
\label{sec:analysis:dict}

TWINTRANには,解析に必要な情報だけでなく変換や生成に必要な情報の大半も記
述した辞書と,屈折形や派生形の生成に必要な情報を記述した辞書が存在する.
以降,前者を単に辞書と呼ぶ.
辞書には現在約13万見出しが登録されており,各見出し表現に与えられている情
報は形態素前接番号,形態素後接番号,同形異義に関する優先度(自然数),語種
(品詞),人称,性,数,意味標識,共起制約,陳述縮約度,対訳情報である.
各情報についての説明は,それらを利用した処理について述べる節で行なう.

辞書は,英語でのひとまとまりの表現に対応する日本語でのひとまとまりの表現
を認識することを目的としている.
このため,日本語文法単独ではひとまとまりの表現ではないが日英翻訳の観点か
らはひとまとまりとみなすべき表現は一つの見出しとして登録している
\footnote{従って,形態素解析でひとまとまりと認識される単位は,通常の単語
あるいは文節とは必ずしも一致しない.}
.
例として,「雨」という語で始まる見出しの一部について語種と対訳情報を表
\ref{tab:dict}\,に示す.

語種``sgnpli''と``ntmo''と``nuc''は名詞類,``a2v''は副詞類,``ve1''は動
詞類,``subj''は主語格標識をそれぞれ意味する.
「雨が降(る)」と「雨模様(だ)」には,一つの語種ではなく``nuc subj ve1''と
いう語種列を与えているが,これによって一つの見出しを構文木上の複数の終
端節点に対応付けている.
この語種列からどのような構文木が構成されるかは\ref{sec:analysis:syn}\,節
で述べる.

実際の辞書の対訳情報には,英単語だけでなく,変換と生成のための特殊記号が
含まれているが,表\ref{tab:dict}\,では簡単のために英単語のみを示した.
実際の処理では,この特殊記号に基づいて動詞の屈折形などを決定する.
\begin{table}[tbhp]
\caption{辞書見出しの一部}
\label{tab:dict}
\begin{center}
\begin{tabular}{|l||l|l|}\hline
\multicolumn{1}{|c||}{見出し表現}&\multicolumn{1}{c|}{語種}&
\multicolumn{1}{c|}{対訳情報}\\\hline\hline
雨                       & sgnpli & rain		       \\
雨の日                   & ntmo   & rainy day                  \\
雨が降ろうと槍が降ろうと & a2v    & come hell or high water    \\
雨後の筍のように乱立     & ve1    & mushroom                   \\
雨が降                   & nuc subj ve1    & it rain           \\
雨模様                   & nuc subj ve1    & it look like rain \\\hline
\end{tabular}
\end{center}
\end{table}

\section{形態素解析}
\label{sec:analysis:morph}

形態素解析では,辞書情報のうち見出し表現,形態素前接番号,形態素後接番号,
同形異義に関する優先度を参照する.
さらに,後接番号と前接番号の接続可能性を記述した接続表を利用する.
接続表には,接続可能な後接番号と前接番号の組と,その接続に関する優先度
($0.1, 0.2, \cdots, 1.0$の十段階)
が記述されている.

形態素解析は,チャート構文解析法\cite{Kay80}の原理に従って行なう
\footnote{もちろん構文解析と異なり,階層的な木構造は生成しない.}.
入力文を構成する各文字の間に番号付きの節点が存在すると考えると,チャート
法による形態素解析は,前述の前接,後接番号と接続表に基づいて,節点の
間を結ぶ弧を順次生成することによって進む.
$\beta_1 \cdots \beta_i$から成る弧$\alpha$と,$\alpha$の右側に隣接する弧
$\beta_{i+1}$を結合して新たな弧$\alpha\prime$を生成するとき,弧$\alpha\prime$
に与える優先度は
1)弧$\alpha$が持つ接続優先度と,$\beta_i$と$\beta_{i+1}$の接続優先度との
積と,
2)弧$\alpha$が持つ同形異義優先度と,$\beta_{i+1}$の同形異義優先度との
和
の二種類とする.

形態素解析での最良解釈の選択は,これら二種類の優先度に基づいて二段階で行
なう.
まず,ある同一節点区間を結ぶ弧の中から,最も高い接続優先度を持
つ弧を選び出し,それ以外の弧は棄却する.
この選択で解釈が唯一に決まれば,それを最良解釈とする.
さもなければ,接続優先度が最も高い弧の中から同形異義優先度が最も高い
弧を選び,それ以外は棄却する.
同形異義優先度が最も高い弧が複数存在する場合,それらすべてを最良解釈とす
る.

形態素解析の後処理として,構文的な構造を作る必要がない語種,すなわち構文
木の一節点とみなす必要がない語種を一つにまとめる.
例えば動詞の語幹を表す語種
と屈折を表す語種
をまとめる処理を行なう.

\section{構文解析}
\label{sec:analysis:syn}

拡張文脈自由文法形式に基づいて,構文解析規則の記述形式を次のように定める.
\begin{equation}
A ::= B_1 \cdots B_i/\head \cdots B_m,\ \ \{Aug\},\ \ Prio,\ \ DL
\label{eq:syn_rule}
\end{equation}
ここで, $B_i/\head$は$B_i$の主辞が$A$の主辞になることを意味し,$\{Aug\}$
は補強項であり,$Prio$はこの規則の適用優先度(自然数),$DL$は構文範疇$A$
の陳述縮約度\cite{Jelinek95}である.
陳述縮約度については\ref{sec:analysis:integ:cor}\,節で説明する.

TWINTRANに実装されている構文解析規則の総数は1734であり,規則記述に用いら
れている終端構文範疇の数(異なり数)と非終端構文範疇の数(異なり数)はそれぞ
れ307と208である.
規則の一部を図\ref{fig:syn_rule_ex}\,に示す.
\begin{figure}[tbhp]
\begin{center}
\fbox{
\vspace*{0.5em}
\begin{tabular}{lllllll}
(a)& TEXT     &::=& SENTENCE/head,    & \{\},               & 0,& 0\\
(b)& TEXT     &::=& SENTENCE TEXT     & \{\},               & 0,& 0\\
(c)& SENTENCE &::=& THM SENT/head end,& \{\},               & 1,& 1\\
(d)& SENTENCE &::=& SENT/head end,    & \{\},               & 1,& 1\\
(e)& THM      &::=& NP/head thm,      & \{\},               & 4,& 6\\
(f)& SENT     &::=& VPAT ve1/head,    & \{check\_vpat(ve1)\},& 0,& 2\\
(g)& SENT     &::=& ve1/head,         & \{\},               & 0,& 2\\
(h)& VPAT     &::=& SUBJ/head,        & \{\},               & 0,& $-$\\
(i)& VPAT     &::=& SUBJ VPAT,        & \{\},               & 0,& $-$\\
(j)& VPAT     &::=& OBJ/head,         & \{\},               & 0,& $-$\\
(k)& VPAT     &::=& OBJ VPAT,         & \{\},               & 0,& $-$\\
(l)& SUBJ     &::=& NP/head subj,     & \{\},               & 0,& 5\\
(m)& NP       &::=& nuc/head,         & \{\},               & 0,& 7\\
\end{tabular}
\vspace*{0.5em}
}
\end{center}
\caption{構文解析規則}
\label{fig:syn_rule_ex}
\end{figure}

図\ref{fig:syn_rule_ex}\,の規則(a)と(b)は,複数の文から成るテキスト全体
を一つの構文木で表現するための規則である.
規則(h)ないし(k)は,動詞の補足語をまとめる規則である.
動詞と補足語の結合関係は,深層格として抽象化せず,日本語の助詞
とその英訳との対応に基づいて表層で区別し,表\ref{tab:case}\,の16種類を設
定している.

表\ref{tab:dict}\,の「雨が降(る)」に与えられている語種列``nuc subj ve1''
からは,規則(m),(l),(h),(f)によって次のような構文木が構成される.
\[	\mbox{SENT(VPAT(SUBJ(NP(nuc) subj)) ve1)} \]

\begin{table}[tbhp]
\caption{動詞と補足語の結合関係の一覧}
\label{tab:case}
\begin{center}
\begin{tabular}{|l||p{0.8\textwidth}|}\hline 
\multicolumn{1}{|c||}{結合関係}&\multicolumn{1}{c|}{説明}
\\\hline\hline
AG & 日本語では助詞「に」や「によって」などで表され,英語では前置詞
``by''で表される.
可能態の動詞とこの関係で結合している補足語は英語では主語になるこ
とがある.
例えば「\underline{私には}その理由が理解できる.」は``{\it I}
can understand the reason.''と英訳される.
\\\hline
AI & 日本語では助詞「で」や「によって」などで表され,英語では前
置詞`by''や``with''で表される.\\\hline
CP & 日本語では助詞「と」などで表され,英語では前置詞``with''で表さ
れる.
ここで``with''は道具や付加ではなく同伴を意味する.\\\hline
IO & 間接目的語.日本語では助詞「に」などで表され,英語では
前置詞``for''や``from''や``to''で表される.\\\hline
LT & 日本語では助詞「まで」などで表され,英語では前置詞``as far as''
や``up to''で表される.\\\hline
ML & 日本語では助詞「までに」などで表され,英語では前置詞``by''で
表される.\\\hline
OBJ & 日本語では助詞「を」や「は」などで表され,英語では直接目的語と
なる.
\\\hline
PAC & 日本語では助詞「で」や「において」などを場所名詞に付加した
形で表され,英語では前置詞``at''や``in''や``on''を場所名詞に付加した形
で表される.\\\hline
PST & 日本語では助詞「に」や「において」などを場所名詞に付加した
形で表され,英語では前置詞``at''や``in''や``on''を場所名詞に付加した形で
表される.\\\hline
PTR & 日本語では助詞「を」などを場所名詞に付加した形で表され,
英語では前置詞``across''や``over''や``through''を場所名詞に付加した形
で表される.\\\hline
QA & 変化を表す動詞によってもたらされる物事や特性を表す.
例えば「彼は\underline{校長に}昇進した.」は``He has been promoted 
{\it to a schoolmaster}.''と英訳される.
\\\hline
QO & 引用.日本語では助詞「と」などで表される.\\\hline
SOBJ & 日本語では助詞「が」などで表され,英語では直接目的語に
なる.
例えば「彼は\underline{林檎が}好きだ.」は``He likes {\it apples}.''と
英訳される.
\\\hline
SUBJ &  日本語では助詞「が」や「は」などで表され,英語では主語となる.
\\\hline
TG & 目標.日本語では助詞「に」や「へ」などで表され,英語では前置詞
``for''や``to''で表される.\\\hline
AUTO & 時間表現や状況表現などと任意の動詞との間の関係を表す.任意の動詞
と結合し得る.\\\hline
\end{tabular}
\end{center}
\end{table}

規則(f)の手続きcheck\_vpat(ve1)は,VPATとしてまとめられた補足語のすべて
が動詞型ve1の動詞と結合できるかどうかを表\ref{tab:verb_type}\,に基づい
て調べる.
表\ref{tab:verb_type}\,は,動詞がどのような補足語と結合できるかという観
点から能動態の動詞を19種類に分類したものの一部である.
この他に,受動態,使役態,間接受動態,可能態と,これらの組合せ(受動使役
態など)について結合可能な補足語の種類を記述している.
\begin{table}[tbhp]
\caption{結合可能な補足語に基づく動詞の分類}
\label{tab:verb_type}
\begin{center}
\begin{tabular}{|c||l|l|}\hline
\multicolumn{1}{|c||}{動詞型}&\multicolumn{1}{c|}{結合可能な補足語}&
\multicolumn{1}{c|}{例}\\\hline\hline
va2  & AI, CP, LT, MT, OBJ, PAC, QO, SUBJ, TG	& 入力する,理解する,打ち明ける\\
vb1  & AI, CP, IO, LT, MT, OBJ, PAC, SUBJ	& 望む,受け継ぐ,借りる\\
vd1  & AI, CP, LT, ML, PAC, QA, SUBJ		& 現れる,減退する,重なる\\
ve1  & AI, CP, LT, ML, PAC, TG, SUBJ            & 引く,揚がる,近づく\\
vg1b & AI, CP, IO, LT, ML, PST, SUBJ            & 存在する,迷う,酔う\\
vh   & AI, CP, LT, MT, PAC, PTR, SUBJ, TG	& 急ぐ,飛び越える,泳ぐ\\\hline
\end{tabular}
\end{center}
\end{table}

構文解析規則の記述では,\ref{sec:analysis:morph}\,節で述べた形態素解析で
の優先度や,\ref{sec:analysis:integ}\,節で述べる共起的意味や照応に関する
優先度と異なり,優先的に適用したい規則ほど小さい数値を与える.
各規則に付与されている優先度の分布を表\ref{tab:syn_pref_distr}\,に示す.
$A$を頂点とする構文木の優先度は,$B_i\ (1 \le i \le m)$を頂点と
する構文木の優先度の和に,規則の優先度$Prio$を加えた値とする.
終端構文範疇を頂点とする構文木の優先度は0とする.
\begin{table}[tbhp]
\caption{構文優先度の分布}
\label{tab:syn_pref_distr}
\begin{center}
\begin{tabular}{|c||r|r|r|r|r|r|r|r|r|r|r|r|}\hline
優先度&
\multicolumn{1}{|c}{0}&\multicolumn{1}{|c}{1}&\multicolumn{1}{|c}{2}&
\multicolumn{1}{|c}{3}&\multicolumn{1}{|c}{4}&\multicolumn{1}{|c}{5}&
\multicolumn{1}{|c}{6}&\multicolumn{1}{|c}{7}&\multicolumn{1}{|c}{8}&
\multicolumn{1}{|c}{12}&\multicolumn{1}{|c}{20}&\multicolumn{1}{|c|}{30}\\\hline
規則数&1545 & 122 & 36 & 1 & 14 & 8 & 2 & 1 & 1 & 2 & 1 & 1 \\\hline
\end{tabular}
\end{center}
\end{table}

構文解析では最良解釈の選択は行なわず,規則で許されるすべての解釈を出力す
る.
これらは圧縮したチャート表現\cite{Tanaka89}で表される.


\section{英語必須補足語の補完と関係節の処理}
\label{sec:analysis:rel}

よく知られているように,日本語動詞の主語や目的語などはそれらが何を指し
ているかが文脈から推測できる場合には明示する必要がないのに対して,英語動
詞の主語や目的語は言語形式上の必須要素であるため代名詞などとして明示する
必要がある.
従って,適切な代名詞を補うことが,品質の高い英語テキストを生成するために
は必要不可欠である.

TWINTRANでは,適切な代名詞を決定する処理を
1)まず,どの補足語を補う必要があるか(例えば主語を補うべきか目的語を補う
べきか)を決定し,
2)次に,補われた補足語を具体的にどのような代名詞(I,you,sheなど)として
英訳するかを決定する
という二段階で行なう.
後者の処理は\ref{sec:analysis:integ}\,節で述べる共起的意味解析と照応
解析で行なう.
前者の処理は図\ref{fig:def_rule_ex}\,に示すような規則に従って実行する.
図\ref{fig:def_rule_ex}\,の規則は,現在着目している節の主動詞が
vg3型の能動態である場合の規則であり,
1)SUBJもAGもその動詞と結合していないならばSUBJを補い,
2)SUBJは結合していないがAGが結合しているならばAGをSUBJとみなす
ことを意味する.
\begin{figure}[tbhp]
\begin{RULE}{0.5\textwidth}
\begin{verbatim}
if(head_verb(vg3)){
  if(~exist(SUBJ)){
    if(~exist(AG)){
      create(SUBJ);
    }
    else{
      relabel(AG, SUBJ);
    }
  }
}
\end{verbatim}
\end{RULE}
\caption{必須補足語補完規則}
\label{fig:def_rule_ex}
\end{figure}

関係節として英訳しなければならない従属節に補足語を補う場合,どの補足語を
補うかの決定は,関係詞の先行詞になる名詞が従属節の主動詞とどのような関係
で結合するかを判定した後,すなわち関係詞を決定した後に行なう.
関係詞を決定する規則は
1)関係詞の先行詞になる主名詞の意味標識,
2)従属節の主動詞の型,
3)従属節の主動詞と結合している補足語の種類
に基づいて記述されている.

関係詞決定規則には適用の優先度を五段階で付与しており,優先度の高い規則か
ら順に適用を試み,ある段階の優先度の規則の適用が成功すれば,それより低い
優先度の規則の適用は行なわない.
この優先度は構文に関する優先度に加算される.
この優先度も\ref{sec:analysis:syn}\,節で述べた構文に関する優先度と同じく,
数値が小さいほど良い.
図\ref{fig:rel_rule_ex}\,に示す第一の規則は,最も優先度が高い規則群に属
する.
この規則は,従属節の主動詞がOBJと結合できる場合の規則であり,
関係詞の先行詞になる主名詞の意味標識がHUM(人間)の下位概念を表し,かつ従
属節の主動詞にOBJが結合していなければ,関係詞``whom''を構文木に補う
ことを意味する.
第二の規則は,他のどの規則も適用できなかった場合に無条件で適用される規
則であり,関係詞``whereby''を生成する.
\begin{figure}[tbhp]
\begin{RULE}{0.9\textwidth}
\begin{verbatim}
if(head_verb(va1)|head_verb(va2)|head_verb(vb1)|head_verb(vb2)|
   head_verb(vc)|head_verb(vd2)|head_verb(vg1a)|head_verb(vg3)){
  if(isa(head_noun, sem_cat:HUM)){
    if(~exist(OBJ)){
      select(whom);
      priority(1);
    }
  }
}

if(true){
  select(whereby);
  priority(5);
}
\end{verbatim}
\end{RULE}
\caption{関係詞決定規則}
\label{fig:rel_rule_ex}
\end{figure}

\section{共起的意味・照応解析}
\label{sec:analysis:integ}

共起的意味解析と照応解析は一つの枠組で統合的に実行する.
共起的意味・照応解析では,AND/ORグラフ上での遅延評価による優先度計算機構
\cite{Tamura91c,Yoshimi97a}によって,テキスト解釈候補の中から(準)最良解
釈を効率的に選び出す.

\subsection{共起的意味解析}
\label{sec:analysis:integ:sem}

動詞とその補足語の間の共起的意味解析は,上位下位関係を記述した意味体系を
参照しながら,動詞の結合価パターンの各スロットに記述されている共起制約と
補足語の人称,性,数,意味標識を照合し,その結果に応じて共起的意味に関す
る優先度を与える.
もし必要ならば,共起制約に違反する解釈も生成する
\footnote{遅延評価による処理を行なっているため,優先度が低い解釈は必要が
ない場合には生成されない.}.
なぜならば,構文,共起的意味,照応の観点からの優先度がそれぞれ最も高い
解釈から,システムが持つ知識全体での最良解釈が構成されるとは限らず,共起
制約に違反する解釈からシステム全体での最良解釈が生成される可能性もあるか
らである.

共起制約に違反する解釈を必要に応じて生成するもう一つの理由は,意味的
に不適格な文に対して頑健な処理を実現するためである.
このような処理が必要になる表現の典型的な例は比喩である.
共起制約に違反する解釈を棄却する立場では比喩に対する解釈を生成するこ
とができない.
これに対して,共起制約に違反する解釈であっても相対的に見て最良解釈で
あるならば,これを受理する本稿のような立場では,比喩に対しても解釈を生成
することができる.
ただし,共起制約に違反する解釈をすべて比喩として受理するのではなく,
比喩とみなせる解釈とそうでない解釈を弁別する処理が必要である
\cite{Wilks78,Ferrari96}が,TWINTRANでは実現されていない.

共起的意味に関する優先度は,共起制約が満たされる場合10とし,満たされない
場合には$-6$としている.

\subsection{照応解析}
\label{sec:analysis:integ:cor}

適格なテキストでは文と文のつながりによって結束が維持されている.
テキストの結束を維持する言語的手段には,照応,代用,省略,接続表現の使用,
語彙的つながりがある\cite{Halliday76}.
照応は,複数の言語表現が一つの事象に言及することによってテキストの結束
を生む手段であり,文や句や語などの様々な言語表現の間で成立する.
例えば次のテキスト\ref{text:gorb}\,では,文「ソ連の国家非常事態委員会は
19日,ゴルバチョフ大統領を解任した」と,名詞句「大統領の解任」と,照応詞
「この」と,空主語「$\phi_{\SUBJ}$」との間で照応が成立している.
\begin{TEXT}
\text ソ連の国家非常事態委員会は19日,ゴルバチョフ大統領を解任したと発
表した.
大統領の解任が西側の対ソ政策に重大な影響を及ぼすことは必至である.
政府は,臨時の閣議を開き,この事態への対応を協議している.
また,$\phi_{\SUBJ}$為替相場へ及ぼす影響も懸念されている.
\label{text:gorb}
\end{TEXT}
テキスト\ref{text:gorb}でこのような照応解釈が成立するのは,陳述縮約パラ
ダイム\cite{Jelinek95}によれば,例えば文「ソ連の国家非常事態委員会は19日,
ゴルバチョフ大統領を解任した」を$X$とし,名詞句「大統領の解任」を$Y$とし
たとき,これらが次の三つの制約を満たすからである.
ここで,解釈$X$と$Y$は構文木上の節点$X$と$Y$にそれぞれ対応するものとする.
\begin{DEPLED}
\item[\bf 構文制約] $Y$はある構文木上で$X$の後方に位置する.
\item[\bf 陳述縮約制約] $Y$は$X$を縮約した言語形式である.
すなわち,$X$と$Y$の陳述縮約度の間で表\ref{tab:depredlevel}\,に示す関
係が成立する.
\item[\bf 意味制約] $Y$の意味は$X$の意味と矛盾しない.
すなわち,$X$の人称,性,数が$Y$の人称,性,数にそれぞれ一致し,かつ,
$X$の意味標識と$Y$の意味標識が上位下位関係にある.
\end{DEPLED}

構文制約は比較的緩い制約であり,文間での前方照応と文内での前方照応を区別
せずに扱える.
このような統一的な扱いを可能にするために,入力が複数の文から成るテキストの
場合でも入力全体を覆う構文木を生成する構文解析規則を
図\ref{fig:syn_rule_ex}\,の(a)と(b)のように記述している.

陳述縮約度は,ある言語表現が他の言語表現をどの程度指しやすいかと,他の言
語表現からどの程度指されやすいかを表す.
完全文の陳述縮約度を1,名詞句の陳述縮約度を7,日本語で表現する必要はない
が英語では表現する必要のある照応詞の陳述縮約度を9のように定める.
表\ref{tab:depredlevel}\,は,例えば節点$X$の陳述縮約度が9であるとき,節
点$Y$の陳述縮約度は8か9でなければならないことを表す.
構文木上の終端節点の陳述縮約度は辞書で与え,非終端節点の陳述縮約度は構文
解析規則で与えている.
\begin{table}[tbhp]
\caption{陳述縮約度に関する制約}
\label{tab:depredlevel}
\begin{center}
\begin{tabular}{|c||r|}\hline
節点$X$の陳述縮約度&\multicolumn{1}{c|}{節点$Y$の陳述縮約度}\\
\hline\hline
0 & 1, 2, 3, 4, 5, 6, 7, 8, 9 \\
1 & 2, 3, 4, 5, 6, 7, 8, 9 \\
2 & 3, 4, 5, 6, 7, 8, 9 \\
3 & 4, 5, 6, 7, 8, 9 \\
4 & 5, 6, 7, 8, 9 \\
5 & 6, 7, 8, 9 \\
6 & 7, 8, 9 \\
7 & 7, 8, 9 \\
8 & 8, 9 \\
9 & 8, 9 \\\hline
\end{tabular}
\end{center}
\end{table}

意味制約が満たされるかどうかの判定では,節点$X$と$Y$の両方が終端節点であ
る場合には,辞書に記述されているそれぞれの意味標識を照合すればよい.
他方,節点$X$と$Y$の両方あるいは一方が非終端節点である場合には,その木構
造とそれを構成する終端節点の意味標識などに基づいて全体の意味を求めなけれ
ばならないが,この処理は実現できていない.
現在のところ,非終端節点の意味標識は,その主辞である終端節点の意味標識
を特化した意味標識であるとしている.
例えば動詞「解任した」の意味標識がAC(action)であるとき,
「解任した」を主辞とする文「ソ連の国家非常事態委員会が
19日,ゴルバチョフ大統領を解任した」全体が持つ意味標識をAC+$\alpha$と表
す.
複合意味標識AC+$\alpha$は,原子意味標識ACに何らかの意味$\alpha$
が加わったACの下位範疇を意味する.
ここで,原子意味標識$A$が原子意味標識$B$の下位範疇であるとき,複合意味標
識$A+\alpha$は$B$の下位範疇であると定める.
また,$A$が複合意味標識$B+\beta$の下位範疇であるかどうかと,$A+\alpha$
が$B+\beta$の下位標識であるかどうかは不明であると定める.
二つの節点$X$と$Y$の意味標識間の上位下位関係が不明である場合も,$X$と
$Y$は意味制約を満たすものとする.

節点$X$と$Y$が構文,意味,陳述縮約に関する制約をすべて満たしているとき,
二つの節点を$Y$から$X$へ向かうリンクで結ぶ.
リンクには次の基準に従って照応に関する優先度を与える.
\begin{enumerate}
\item $X$と$Y$の意味標識の間に上位下位関係にあるかどうかが不明である場
合,優先度を0とする.
\item $X$と$Y$の意味標識の間に上位下位関係にあることがわかってお
り,かつ$Y$の陳述縮約度が7以下である場合,優先度は1とする.
\item $X$と$Y$の意味標識の間に上位下位関係にあることがわかってお
り,かつ
$Y$の陳述縮約度が8以上である場合,優先度を2とする.
\end{enumerate}

テキストの照応解釈の優先度は,その解釈を構成するリンクの優先度の和であ
り,その値が最も高い解釈が最良の照応解釈である.
テキスト\ref{text:gorb}\,の場合,照応関係にある各言語表現は,図
\ref{fig:gorb}\,に示すリンクで結ばれる.
図\ref{fig:gorb}\,では,名詞句「大統領の解任」から文「ソ連の
国家非常事態委員会は19日,ゴルバチョフ大統領を解任した」へのリンクの優先
度が0であり,それ以外のリンクの優先度はすべて2であるので,この照応解釈の
優先度は10となる.
\begin{figure}[tbhp]
\begin{center}
    \begin{epsf}
\fbox{
\vspace*{0.5em}
\epsfile{file=gorb.eps,width=0.7\textwidth}
\vspace*{0.5em}
}
    \end{epsf}
    \begin{draft}
\fbox{
\vspace*{0.5em}
    \atari(284.81084,218.54068, 1pt)
\vspace*{0.5em}
}
    \end{draft}
\end{center}
\caption{テキスト\protect\ref{text:gorb}\,における照応}
\label{fig:gorb}
\end{figure}

照応関係にある各言語表現の間で人称,性,数,意味標識を伝播することによっ
て,これらの曖昧性が解消される.
図\ref{fig:gorb}\,の空主語「$\phi_{\SUBJ}$」は,人称,
性,数,意味標識が三人称(3rd),中性(n),単数(sg),$AC+\alpha$となるので,
代名詞``it''と訳される.

\subsection{構文,共起的意味,照応に関する優先度に基づく総合評価}
\label{sec:analysis:integ:balance}

共起的意味・照応解析系では,さらに,次のような総合評価式に基づいて構文,
共起的意味,照応に関する各優先度を組み合わせた評価を行ない,解析系全体で
の(準)最良解釈を選び出す
\footnote{形態素解析の精度は十分高いため,形態素に関する優先度と他の優先
度との相互作用は考慮しない.}
.
\begin{equation}
S = W_{\SYN} \times S_{\SYN} + W_{\SEM} \times S_{\SEM} + W_{\COR}
\times S_{\COR}
\label{eq:balance}
\end{equation}
$S_{\SYN}$,$S_{\SEM}$,$S_{\COR}$はそれぞれ構文,共起的意味,照応に関す
る優先度であり,
$W_{\SYN}$,$W_{\SEM}$,$W_{\COR}$は各優先度についての相対的重要度であ
る.

理想的なテキストでは,構文,共起的意味,照応の各最良解釈から全体での最良
解釈が構成される可能性が高いと考えられる.
これに対して現実のテキストでは,各優先度に基づく最良解釈の間で競合が生じ,
各最良解釈が相容れないことがある.
このような場合,どの解釈を優先させるかは,主に相対的重要度$W_{\SYN}$,
$W_{\SEM}$,$W_{\COR}$を調整することによって経験的に決定する.
今回の実験では,$W_{\SYN}$,$W_{\SEM}$,$W_{\COR}$を,訓練用テキストの分
析結果に基づいて,それぞれ21,3,1とした.

\section{動詞型に基づく変換}
\label{sec:transfer:sentpat}

動詞型に基づく変換の目的は,補足語に前置詞を付加することと,英文での補足
語の位置を決定することである.
この処理は図\ref{fig:spt_rule_ex}\,に示すような規則に従って行なう.
図\ref{fig:spt_rule_ex}\,の手続きprefix()は,第一引数の補足語に,第二引
数で指定された前置詞を付加する.
手続きmove()は,補足語を動詞の後方に移し,引数で指定された順序に並べ換え
る.
引数に記述されていない補足語は元の位置に残しておく.
例えば「少年が(SUBJ)犬と(CP)川を(PTR)泳いでいる(vh).」という文から生成
される構文木は,図\ref{fig:spt_rule_ex}\,の規則により
``boy swim \{over$|$across$|$through\} river with dog''に対応する木に変
換される.
\begin{figure}[tbhp]
\begin{RULE}{0.5\textwidth}
\begin{verbatim}
if(head_verb(vh)){
  prefix(AI, {from|by|with});
  prefix(CP, with);
  prefix(LT, {as far as|up to});
  prefix(ML, by);
  prefix(PTR, {over|across|through});
  prefix(TG, {for|to});
  move([PTR, CP, TG, PAC, AI, LT, ML]);
}
\end{verbatim}
\end{RULE}
\caption{動詞型に基づく変換規則}
\label{fig:spt_rule_ex}
\end{figure}

\section{個別の変換情報に基づく変換}
\label{sec:transfer:entry_specific}

個別の変換情報による変換は,辞書の対訳情報に基づいて木構造を書き換える.
対訳情報は英単語,中間記号,セパレータから成る.
中間記号は,さらに,ここで扱う変換記号と生成系で扱う生成記号に分けられる.
対訳情報の例を表\ref{tab:translation}\,に挙げる.
表\ref{tab:translation}\,において,英大文字と数字は中間記号を,英小文字
は英単語をそれぞれ意味し,それ以外はセパレータである.
\begin{table}[tbhp]
\caption{対訳情報}
\label{tab:translation}
\begin{center}
\begin{tabular}{|l||l|}\hline
\multicolumn{1}{|c||}{見出し表現}&\multicolumn{1}{c|}{対訳情報}\\\hline\hline
からの    & B\_from\_A				\\
たことがな& have\_never\_DONE			\\
中に      & AV2(while\_SENT)			\\
預け      & entrust\_P(\#TG:0)\_P(\#OBJ:with)	\\
容疑がかか& be\_\symbol{94}\_under\_suspicion	\\
動作      & \{function$|$operate$|$run\}	\\
\hline
\end{tabular}
\end{center}
\end{table}

変換記号は全部で53種類定義されているが,その一部を表\ref{tab:subst}\,に
示す.
簡単な変換の例として「この理論は発話行為の観点からの意図が扱える.」とい
う文に対する処理を図\ref{fig:subst}\,に示す
\footnote{図\ref{fig:subst}\,は実際の木構造を簡単化したものである.}.
変換の前処理として,対訳情報を構造化し,変換記号なども木構造上の一節点と
みなす.
図\ref{fig:subst}\,に現れている変換記号は名詞句の移動に関するAとBと,動
詞の移動に関するDOであり,それぞれの移動を矢印で示すように行なう.
なお記号\symbol{94},AT,T,ZRは生成記号である.
\begin{table}[tbhp]
\caption{変換記号}
\label{tab:subst}
\begin{center}
\begin{tabular}{|l||p{0.55\textwidth}|}\hline
\multicolumn{1}{|c||}{変換記号}&\multicolumn{1}{c|}{説明}
\\\hline\hline
A & 構文木上でこの記号に最も近い前方の名詞句をこの記号の位置に移動する.
\\\hline
B & 最も近い後方名詞句をこの位置に移動する.\\\hline
DO,DID,DOING,DONE &
最も近い前方の動詞をこの位置に移動する.
さらにDID,DOING,DONEの場合はそれぞれ過去形,現在分詞形,過去分詞形を
生成系で生成する.\\\hline
\#AG,\#OBJ,\#TGなど. & それぞれの記号と同一節内
に存在するAG,OBJ,TGなどをこの記号の位置に移動する.
これによって,動詞型に基づく変換で決定された補足語の順序を変更する.
\\\hline
SENT,UOSENT,USSENTなど. &
最も近い前方の埋め込み節をこの位置に移動する.
ただし,UOSENT,USSENTの場合はそれぞれ,SUBJが存
在しない節,OBJが存在しない節でなければならない.\\\hline
P & 補足語に付加すべき前置詞を指定する.
これによって,動詞型に基づく変換で決定された前置詞を変更する.\\\hline
\end{tabular}
\end{center}
\end{table}
\begin{figure}[tbhp]
\begin{center}
    \begin{epsf}
\fbox{
\vspace*{0.5em}
\epsfile{file=subst3.eps,width=0.96\textwidth}
\vspace*{0.5em}
}
    \end{epsf}
    \begin{draft}
\fbox{
\vspace*{0.5em}
\atari(390.60216, 126.89085, 1pt)
\vspace*{0.5em}
}
    \end{draft}
\end{center}
\caption{個別の変換情報に基づく変換の例}
\label{fig:subst}
\end{figure}

\section{生成}
\label{sec:generation}

生成規則には,名詞句へ付加する冠詞を決定する規則,屈折形や派生形を決定す
る規則,副詞の最終的な位置を決定する規則,時制の調整を行なう規則などがある.
これらのうち冠詞の生成に関しては,
辞書で冠詞が一意に指定されている場合と文脈に依存して決まる場合があるが,
後者の場合には照応解析の結果に基づいて冠詞を決定する.
すなわち,ある名詞句が他の事象を指していれば定冠詞を選択し,そうでなけれ
ば不定冠詞あるいはゼロ冠詞を選択する.
また,名詞の屈折形の決定に関しても,照応解析結果に基づいて単数形,複数形
の生成を行なう.

\section{翻訳実験}
\label{sec:experiment}

\subsection{実験方法}

実験には,池原らによって編集された機械翻訳機能試験文集\cite{Ikehara94}
の2868文を用いた.
まず,試験文集に合わせた準備を全く行なわない完全ブラインドテストを行ない,
その評価結果に基づいて辞書と規則を修正した後,ウィンドウテストを行なった.
試験文集だけに合わせた修正は極力避け,一般性のある修正を行なうよ
うに努めた.
翻訳品質の評価,辞書と規則の修正,試験のサイクルは四回繰り返した.
評価および修正はすべて一名で行なった.
完全ブラインドテストの結果の評価から第四回目のウィンドウテストの結果の評
価までに要した期間は,およそ六ヶ月であった.

翻訳品質評価の基準は,訳文が1)文法的か,2)わかりやすいか,3)原文の意味
と一致するか,4)利用者の役に立つかという観点から
表\ref{tab:eval_criterion}\,のように設定した.
この評価基準において,4点以上を合格とし,それ未満を不合格とする.
\begin{table}[tbhp]
\caption{翻訳品質評価の基準}
\label{tab:eval_criterion}
\begin{center}
\begin{tabular}{|r||p{0.2\textwidth}|p{0.2\textwidth}|p{0.15\textwidth}|p{0.2\textwidth}|}\hline
\multicolumn{1}{|c||}{評価点}&
\multicolumn{1}{c|}{文法性}&
\multicolumn{1}{c|}{理解容易性}&
\multicolumn{1}{c|}{意味等価性}&
\multicolumn{1}{c|}{有用性}\\\hline\hline
6   & 文法的である.& 容易に分かる.& 一致する.& 役立つ.\\\hline
5   & 文法的である.& 注意深く読めば分かる.& ある文脈では一致する.&
対象分野や話題に関する知識を持つ利用者には役立つ.\\\hline
4   & 文法的だが不自然である.& 注意深く読めば分かる.& ある文脈では一致する.& 同
上.\\\hline
3   & 非文法的だが原語がすべて訳された.& 注意深く読めば分かる.& ある文
脈では一致する.& 同上.\\\hline
2   & 非文法的で原語が残っている.& 注意深く読めばほぼ分かる.& ある文脈ではほぼ一致する.&
日本語辞書で単語を探す能力を持つ利用者には役立つ.\\\hline
1   & 非文法的で原文を参照する必要がある.& 注意深く読めば部分的に分かる.
& ある文脈では原文と矛盾しない.& 両言語の知識が無いと役に立たない\\\hline
0   & 何も出力されない.& 分からない.& どんな文脈でも原文と一致しない.& 役に立たない.\\\hline
$-1$& 必須情報が欠落している.& 分からない.& どんな文脈でも原文と一致しない.& 役に立
たない.理解してみる努力が無駄になるので原文を読んで理解してみるほうがよい.\\\hline
$-2$& 文法的である.& 分かる.&  どんな文脈でも原文と一致しない.& 危険.
訳文の質が高いほどより危険である.プロの翻訳者でさえ訳文を信じてしまい原文と
照らし合わせない恐れがある.\\\hline
\end{tabular}
\end{center}
\end{table}

\subsection{実験結果}

完全ブラインドテストと四回目のウィンドウテストでの評価点の分布を
表\ref{tab:experiment_result1}\,に示す.
合格した文の数は,完全ブラインドテストでは全体の46.4\%にあたる1332文,ウ
ィンドウテストでは2096文(73.1\%)であった.
評価点の平均は,完全ブラインドテストでは2.7点で合格点に達しなかったが,
ウィンドウテストでは4.2点と合格点を上回った.
各評価点を与えられた文の数が完全ブラインドテストとウィンドウテストでどのよ
うに変化したかを見ると,合格領域の6点ないし4点となる文数はいずれも増加し,
不合格領域の3点ないし$-2$点となる文数はいずれも減少している.
特に,何も出力されないために0点となる文数の減少が著しい.
これは主に,完全ブラインドテストでは入力文全体を覆う構文木が生成できず何
も出力されなかった文に対して,ウィンドウテストでは全体の構文木
が生成できるようになったことによる.

各文の評価値が完全ブラインドテストとウィンドウテストでどのように変化した
かの分布を表\ref{tab:experiment_result2}\,に示す.
表\ref{tab:experiment_result2}\,によれば,
評価点が向上した文数(表の左下隅の領域)は1186文(41.4\%)であり,そのうち不
合格から合格へ改善された文数は835文である.
逆に,評価値が低下した文数(表の右上隅の領域)は204文(7.1\%)であり,そのうち
合格から不合格へ悪化した文数は71文である.
また,両テストで評価点に変化がなかった文数(表の対角線上)は1478文(51.5\%)
である.

この結果から,辞書と規則の修正による悪影響を比較的小さく抑えつつ,翻訳品
質の改善が実現できているといえる.

完全ブラインドテストとウィンドウテストでの翻訳例を付録の
表\ref{tab:trans_example}\,に示す.
\begin{table}[tbhp]
\caption{ブラインドテストとウィンドウテストでの評価点の分布}
\label{tab:experiment_result1}
\begin{center}
\begin{tabular}{|r||r@{}r|r@{}r|r|}\hline
\multicolumn{1}{|c||}{評価点}&
\multicolumn{2}{c|}{ブラインド}&
\multicolumn{2}{c|}{ウィンドウ}&
\multicolumn{1}{c|}{増減文数}\\\hline\hline
 6点   & 645文 & (22.5\%) & 1054文 & (36.7\%) & 409    \\
 5点   & 399文 & (13.9\%) &  616文 & (21.5\%) & 217    \\
 4点   & 288文 & (10.0\%) &  426文 & (14.9\%) & 138    \\
 3点   & 389文 & (13.6\%) &  304文 & (10.6\%) & $-85$  \\
 2点   & 108文 &  (3.8\%) &  106文 &  (3.7\%) & $-2$   \\
 1点   &  41文 &  (1.4\%) &    2文 &  (0.1\%) & $-39$  \\
 0点   & 434文 & (15.1\%) &   95文 &  (3.3\%) & $-339$ \\
$-1$点 & 439文 & (15.3\%) &  221文 &  (7.7\%) & $-218$ \\
$-2$点 & 125文 &  (4.4\%) &   44文 &  (1.5\%) & $-81$  \\\hline
\end{tabular}
\end{center}
\end{table}
\begin{table}[tbhp]
\caption{ブラインドテストとウィンドウテストでの評価点の変化}
\label{tab:experiment_result2}
\begin{center}
\begin{tabular}{|r||r|r|r|r|r|r|r|r|r|}\hline
\multicolumn{1}{|c||}{ブ$\backslash$ウ}&
\multicolumn{1}{c|}{6点}&\multicolumn{1}{c|}{5点}&\multicolumn{1}{c|}{4点}&
\multicolumn{1}{c|}{3点}&\multicolumn{1}{c|}{2点}&\multicolumn{1}{c|}{1点}&
\multicolumn{1}{c|}{0点}&\multicolumn{1}{c|}{$-1$点}&\multicolumn{1}{c|}{$-2$点}\\\hline\hline
6点&595      &26       &9       &4       &3       &0       &4       &3       &1       \\
5点&59       &304      &15      &8       &2       &0       &8       &3       &0       \\
4点&46       &36       &171     &20      &4       &0       &2       &9       &0       \\
3点&85       &65       &65      &146     &12      &0       &5       &11      &0       \\
2点&19       &13       &23      &17      &27      &0       &2       &6       &1       \\
1点&12       &12       &10      &3       &0       &1       &1       &2       &0       \\
0点&98       &94       &81      &50      &17      &0       &59      &33      &2       \\
$-1$点&98       &56       &42      &49      &33      &1       &9       &143     &8       \\
$-2$点&42       &10       &10      &7       &8       &0       &5       &11      &32      \\\hline
\end{tabular}
\end{center}
\end{table}

\section{おわりに}

本稿では,日英機械翻訳システムTWINTRANの辞書と規則について述べ,
NTT機械翻訳機能試験文集を対象として行なった翻訳品質評価実験の結果を示し
た.
ウィンドウテストでは,我々の評価基準で,試験文集の73.1\%の文が合格となり,
試験文集全体の平均点も合格点を上回る結果が得られた.

今回は,複数の文から成るテキストを対象とした評価実験は行なわなかったため,
文間照応の認識によって得られる効果が確認できなかった.
今後,テキストを対象とした実験を行なう必要がある.

\acknowledgment

形態素解析系の設計と開発を行なって頂いたGraham Wilcock氏
(現在UMIST.当時シャープ(株))に感謝します.
ALT-J/Eに関する文献を提供いただいたFrancis Bond氏(NTT)と,
試験文集を作成された関係者の方々に感謝します.
さらに,有益なコメントを頂いた査読者の方に感謝します.

\bibliographystyle{jnlpbbl}
\bibliography{twintran}

\begin{biography}
\biotitle{略歴}
\bioauthor{吉見 毅彦}
{1987年電気通信大学大学院計算機科学専攻修士課程修了.
1987年よりシャープ(株)にて機械翻訳システムの研究開発に従事.
1999年神戸大学大学院自然科学研究科博士課程修了.}

\bioauthor{Jiri Jelinek}
{チェコのプラハのUniversita Karlova卒業(言語学・英語学・日本語学).
1959年以来,日英機械翻訳実験中.
英国Sheffield大学日本研究所専任講師を1995年退職.
1992年より1996年までシャープ専任研究員.}

\bioauthor{西田 収}
{1984年大阪教育大学教育学部中学校課程数学科卒業,
同年より神戸大学工学部応用数学科の教務補佐員として勤務.
1987年シャープ(株)に入社.
現在,同社情報家電開発本部NB第一プロジェクトチームに所属.
情報処理学会会員.}

\bioauthor{田村 直之}
{1985年神戸大学大学院自然科学研究科システム科学専攻
博士課程修了.
学術博士.
同年,日本アイ・ビー・エム(株)に入社し東京基礎研究所に勤務.
1988年神戸大学工学部システム工学科助手.
講師を経て,現在同大学工学部情報知能工学科助教授.
論理型プログラミング言語,線形論理などに興味を持つ.}

\bioauthor{村上 温夫}
{1952年大阪大学理学部数学科卒業.
神戸大学理学部助手,講師,教養部助教授を経て,1968年より工学部教授.
この間,University of Kansas客員助教授,University of New South Wales
客員教授,Nanyang University客員教授を併任.
1992年より1998年まで甲南大学理学部教授.
神戸大学名誉教授.
理学博士(東京大学).
関数解析,偏微分方程式,人工知能,数学教育などに興味を持つ.
著書に``Mathematical Education for Engineering Students''(Cambridge
University Press)など.
日本数学会,日本数学教育学会各会員.}

\bioreceived{受付}
\biorevised{再受付}
\bioaccepted{採録}

\end{biography}

\newpage
\appendix

\begin{table}[hb]
\caption{翻訳例}
\label{tab:trans_example}
\begin{center}
\begin{tabular}{|l|p{0.48\columnwidth}|r|}\hline 
\multicolumn{1}{|c|}{原文}&\multicolumn{1}{c|}{訳文}&\multicolumn{1}{c|}{
評価点}\\\hline\hline
& (ブ) It multiplied the number of connectable devices by incorporating 
the software which it uniquely developed. & 6 \\\cline{2-3}
\raisebox{1.5ex}[0pt]{\begin{minipage}{0.35\columnwidth}
独自に開発したソフトを組み込むことで、接続可能台数を増やした。
\end{minipage}}
& (ウ) It multiplied the number of connectable devices by incorporating 
the software which we uniquely developed. & 6  \\\hline
& (ブ) Although it surveys that problem, it demands time. & $-2$ \\\cline{2-3}
\raisebox{1.5ex}[0pt]{\begin{minipage}{0.35\columnwidth}
その問題は調査するのに、時間を要する。
\end{minipage}} 
& (ウ) We require time to survey that problem. & 6 \\\hline
& (ブ) I want to eat udon from the side. & $-2$ \\\cline{2-3}
\raisebox{1.5ex}[0pt]{\begin{minipage}{0.35\columnwidth}
私はそばよりうどんが食べたい。
\end{minipage}}
& (ウ) I want to eat udon from the side. & $-2$ \\\hline
& (ブ) The room below has not anyone. & 6 \\\cline{2-3}
\raisebox{1.5ex}[0pt]{\begin{minipage}{0.35\columnwidth}
下の部屋は誰もいない。
\end{minipage}} 
& (ウ) There is no room below. & $-2$ \\\hline
& (ブ) Before it went was the rain. & $-2$ \\\cline{2-3}
\raisebox{1.5ex}[0pt]{\begin{minipage}{0.35\columnwidth}
行った先は雨だった。
\end{minipage}}
& (ウ) Before we went was the rain. & $-2$ \\\hline
& (ブ) The number is limitted and only, I participate when a fee is cheap. 
& 3 \\\cline{2-3}
\raisebox{1.5ex}[0pt]{\begin{minipage}{0.35\columnwidth}
人数が制限され、料金が安いときのみ、私は参加する。
\end{minipage}} 
& (ウ) The number is limited and I participate only when a fee is cheap. 
& 5 \\\hline
& (ブ) He ended up increasingly differing as the rival. & 3 \\\cline{2-3}
\raisebox{1.5ex}[0pt]{\begin{minipage}{0.35\columnwidth}
彼はライバルと差が開いてしまった。
\end{minipage}} 
& (ウ) He ended up establishing a lead over a rival. & 6 \\\hline
& (ブ) \#\#\# 構文解析失敗のため出力なし \#\#\# & 0 \\\cline{2-3}
\raisebox{-2.5ex}[0pt]{\begin{minipage}{0.35\columnwidth}
計算機の製造には半導体工場などが必要だが、中国ではこうしたハイテク設備の
建設技術をもつ技術者の絶対数が不足している。
\end{minipage}}
& (ウ) Expert's in technology the semiconductor plant etc is required in 
production and who have building this kind of Hi-Tech equipment technology 
in China of computers absolute number is falling short. & $-1$  \\\hline
& (ブ) \#\#\# 構文解析失敗のため出力なし \#\#\# &0  \\\cline{2-3}
\raisebox{-4.5ex}[0pt]{\begin{minipage}{0.35\columnwidth}
相手の会社が契約の期間までに必要なすべてのデータを揃える事ができなかった
場合、当方に対して納得いく理由を示さない限り、当社はそのデータの受領を拒
否し、損害の保証を求めることができる。
\end{minipage}}
& (ウ) If partner's company could not assemble all data necessary even for 
terms of contracts against this side, as far as it does not indicate the 
sufficient reason, our company denies the acceptance of those data and 
you can demand guaranteeing damage. & 4 \\\hline
\end{tabular}
\end{center}
\end{table}

\end{document}
